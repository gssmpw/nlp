\documentclass[a4paper,UKenglish,cleveref, autoref, thm-restate]{lipics/lipics-v2021}
%This is a template for producing LIPIcs articles. 
%See lipics-v2021-authors-guidelines.pdf for further information.
%for A4 paper format use option "a4paper", for US-letter use option "letterpaper"
%for british hyphenation rules use option "UKenglish", for american hyphenation rules use option "USenglish"
%for section-numbered lemmas etc., use "numberwithinsect"
%for enabling cleveref support, use "cleveref"
%for enabling autoref support, use "autoref"
%for anonymousing the authors (e.g. for double-blind review), add "anonymous"
%for enabling thm-restate support, use "thm-restate"
%for enabling a two-column layout for the author/affilation part (only applicable for > 6 authors), use "authorcolumns"
%for producing a PDF according the PDF/A standard, add "pdfa"

%\pdfoutput=1 %uncomment to ensure pdflatex processing (mandatatory e.g. to submit to arXiv)
%\hideLIPIcs  %uncomment to remove references to LIPIcs series (logo, DOI, ...), e.g. when preparing a pre-final version to be uploaded to arXiv or another public repository

%\graphicspath{{./graphics/}}%helpful if your graphic files are in another directory

\bibliographystyle{plainurl}% the mandatory bibstyle

% REMOVE
% REMOVE
% REMOVE
\makeatletter
\renewcommand\maketitle{\par
  \begingroup
    \thispagestyle{plain}
    \renewcommand\thefootnote{\@fnsymbol\c@footnote}%
    \if@twocolumn
      \ifnum \col@number=\@ne
        \@maketitle
      \else
        \twocolumn[\@maketitle]%
      \fi
    \else
      \newpage
      \global\@topnum\z@   % Prevents figures from going at top of page.
      \@maketitle
    \fi
    \thispagestyle{plain}\@thanks
  \endgroup
  % \global\let\thanks\relax
  % \global\let\maketitle\relax
  % \global\let\@maketitle\relax
  % \global\let\@thanks\@empty
  % \global\let\@author\@empty
  % \global\let\@date\@empty
  % \global\let\@title\@empty
  % \global\let\@subtitle\@empty
  % \global\let\title\relax
  % \global\let\author\relax
  % \global\let\date\relax
  % \global\let\and\relax
}
\makeatother
% REMOVE
% REMOVE
% REMOVE

\title{On the \ff Algorithm for Online Unit-Interval Coloring} %TODO Please add

%\titlerunning{Dummy short title} %TODO optional, please use if title is longer than one line
%\author{\textcolor{red}{Anonymous authors}}{\textcolor{red}{Anonymous affiliations}}{}{}{}

\author{Bob {Krekelberg}}{Department of Information and Computing Sciences, Utrecht University, The Netherlands}{b.h.a.f.krekelberg@uu.nl}{https://orcid.org/0009-0000-5517-6095}{}%TODO mandatory, please use full name; only 1 author per \author macro; first two parameters are mandatory, other parameters can be empty. Please provide at least the name of the affiliation and the country. The full address is optional. Use additional curly braces to indicate the correct name splitting when the last name consists of multiple name parts.

\author{{Alison Hsiang-Hsuan} {Liu}}{Department of Information and Computing Sciences, Utrecht University, The Netherlands}{h.h.liu@uu.nl}{https://orcid.org/0000-0002-0194-9360}{}

\authorrunning{B. Krekelberg and A.\,H.\,H. Liu} %TODO mandatory. First: Use abbreviated first/middle names. Second (only in severe cases): Use first author plus 'et al.'
%\authorrunning{\textcolor{red}{Anonymous authors}}

\Copyright{Bob Krekelberg and Alison Hsiang-Hsuan Liu} %TODO mandatory, please use full first names. LIPIcs license is "CC-BY";  http://creativecommons.org/licenses/by/3.0/
%\Copyright{\textcolor{red}{Anonymous authors}}

\ccsdesc[100]{Theory of computation → Online algorithms} %TODO mandatory: Please choose ACM 2012 classifications from https://dl.acm.org/ccs/ccs_flat.cfm 

\keywords{Online algorithms, First-Fit algorithm, Interval coloring, Unit intervals, Open and closed intervals} %TODO mandatory; please add comma-separated list of keywords

%\category{Track A: Algorithms, Complexity and Games} %optional, e.g. invited paper
\category{}

\relatedversion{} %optional, e.g. full version hosted on arXiv, HAL, or other respository/website
%\relatedversiondetails[linktext={opt. text shown instead of the URL}, cite=DBLP:books/mk/GrayR93]{Classification (e.g. Full Version, Extended Version, Previous Version}{URL to related version} %linktext and cite are optional

%\supplement{}%optional, e.g. related research data, source code, ... hosted on a repository like zenodo, figshare, GitHub, ...
%\supplementdetails[linktext={opt. text shown instead of the URL}, cite=DBLP:books/mk/GrayR93, subcategory={Description, Subcategory}, swhid={Software Heritage Identifier}]{General Classification (e.g. Software, Dataset, Model, ...)}{URL to related version} %linktext, cite, and subcategory are optional

%\funding{(Optional) general funding statement \dots}%optional, to capture a funding statement, which applies to all authors. Please enter author specific funding statements as fifth argument of the \author macro.

\acknowledgements{}%optional

\nolinenumbers %uncomment to disable line numbering

%%% Packages %%%
\usepackage{tikz} % Allows overview diagram
\usepackage{xspace} % avoid space after macros
\usepackage{enumerate} % allow enumeration start from (a)

%%% Macro's %%%
\newcommand{\runtitle}[1]{\textbf{#1}}
\newcommand{\orderedInput}{\mathcal{I}}
\newcommand{\result}{\lceil \frac{7}{3}\omega \rceil - 2}
    % Algorithms
\newcommand{\ff}{\textsc{FirstFit}\xspace} % First Fit algorithm
\newcommand{\ffc}{\text{\texttt{FF}\xspace}} % color by First Fit algorithm
\newcommand{\opt}{\texttt{OPT}\xspace} % Optimal algorithm
    % Intervals
\newcommand{\x}{I} % Interval x 
\newcommand{\y}{D^\texttt{A}(\x)} % Interval y 
\newcommand{\z}{D^\texttt{M}(\x)} % Interval z 
\newcommand{\zbar}{D^\texttt{AM}(\x)} % Interval zbar
\newcommand{\pivot}{I^*} % Interval from pivot bound
    % Fractions of sets
\newcommand{\alp}{\alpha} % Fraction of L,M,R
\newcommand{\bet}{\beta} % Fraction of Z
\newcommand{\gam}{\gamma} % Fraction of R_3(z)
\newcommand{\delt}{\delta} % Fraction of R_2 \setminus R_3(z)
    % Sets
\newcommand{\twins}{\mathcal{T}} % twins (identical intervals)
\newcommand{\nb}{\mathcal{N}} % neighborhood N
\newcommand{\Z}{\mathcal{R}^{\texttt{TDM}}} % Set Z
\newcommand{\pivotset}{\mathcal{S}^*} % Set from Pivot bound
\newcommand{\lmr}{\mathcal{A}} % aligned intervals
\newcommand{\nlmr}{\mathcal{M}} % not-aligned intervals
    % Sets of rows
\newcommand{\rows}[2]{\mathcal{R}_\texttt{#1}(#2)} % R_{#1}(#2)
\newcommand{\nrows}[2]{r_\texttt{#1}(#2)} % r_{#1}(#2)
    % Notes
\newcommand{\todo}[1]{{\color{Red}#1}}
\newcommand{\hhl}[1]{{\color{OliveGreen}#1}}
\newcommand{\bk}[1]{{\color{Orchid}#1}}
\newcommand{\hide}[1]{}
\newcommand{\hideproof}[1]{}


%\newcommand{\fullversion}[1]{}
%\newcommand{\shortversion}[1]{{#1}}
\newcommand{\fullversion}[1]{{#1}}
\newcommand{\shortversion}[1]{}

\hideLIPIcs
%Editor-only macros:: begin (do not touch as author)%%%%%%%%%%%%%%%%%%%%%%%%%%%%%%%%%%
\EventEditors{John Q. Open and Joan R. Access}
\EventNoEds{2}
\EventLongTitle{42nd Conference on Very Important Topics (CVIT 2016)}
\EventShortTitle{CVIT 2016}
\EventAcronym{CVIT}
\EventYear{2016}
\EventDate{December 24--27, 2016}
\EventLocation{Little Whinging, United Kingdom}
\EventLogo{}
\SeriesVolume{42}
\ArticleNo{23}
%%%%%%%%%%%%%%%%%%%%%%%%%%%%%%%%%%%%%%%%%%%%%%%%%%%%%%

\begin{document}

\maketitle

%TODO mandatory: add short abstract of the document
\begin{abstract}
    \begin{abstract}  
Test time scaling is currently one of the most active research areas that shows promise after training time scaling has reached its limits.
Deep-thinking (DT) models are a class of recurrent models that can perform easy-to-hard generalization by assigning more compute to harder test samples.
However, due to their inability to determine the complexity of a test sample, DT models have to use a large amount of computation for both easy and hard test samples.
Excessive test time computation is wasteful and can cause the ``overthinking'' problem where more test time computation leads to worse results.
In this paper, we introduce a test time training method for determining the optimal amount of computation needed for each sample during test time.
We also propose Conv-LiGRU, a novel recurrent architecture for efficient and robust visual reasoning. 
Extensive experiments demonstrate that Conv-LiGRU is more stable than DT, effectively mitigates the ``overthinking'' phenomenon, and achieves superior accuracy.
\end{abstract}  
\end{abstract}

\section{Introduction}
\section{Introduction}
\label{sec:introduction}
The business processes of organizations are experiencing ever-increasing complexity due to the large amount of data, high number of users, and high-tech devices involved \cite{martin2021pmopportunitieschallenges, beerepoot2023biggestbpmproblems}. This complexity may cause business processes to deviate from normal control flow due to unforeseen and disruptive anomalies \cite{adams2023proceddsriftdetection}. These control-flow anomalies manifest as unknown, skipped, and wrongly-ordered activities in the traces of event logs monitored from the execution of business processes \cite{ko2023adsystematicreview}. For the sake of clarity, let us consider an illustrative example of such anomalies. Figure \ref{FP_ANOMALIES} shows a so-called event log footprint, which captures the control flow relations of four activities of a hypothetical event log. In particular, this footprint captures the control-flow relations between activities \texttt{a}, \texttt{b}, \texttt{c} and \texttt{d}. These are the causal ($\rightarrow$) relation, concurrent ($\parallel$) relation, and other ($\#$) relations such as exclusivity or non-local dependency \cite{aalst2022pmhandbook}. In addition, on the right are six traces, of which five exhibit skipped, wrongly-ordered and unknown control-flow anomalies. For example, $\langle$\texttt{a b d}$\rangle$ has a skipped activity, which is \texttt{c}. Because of this skipped activity, the control-flow relation \texttt{b}$\,\#\,$\texttt{d} is violated, since \texttt{d} directly follows \texttt{b} in the anomalous trace.
\begin{figure}[!t]
\centering
\includegraphics[width=0.9\columnwidth]{images/FP_ANOMALIES.png}
\caption{An example event log footprint with six traces, of which five exhibit control-flow anomalies.}
\label{FP_ANOMALIES}
\end{figure}

\subsection{Control-flow anomaly detection}
Control-flow anomaly detection techniques aim to characterize the normal control flow from event logs and verify whether these deviations occur in new event logs \cite{ko2023adsystematicreview}. To develop control-flow anomaly detection techniques, \revision{process mining} has seen widespread adoption owing to process discovery and \revision{conformance checking}. On the one hand, process discovery is a set of algorithms that encode control-flow relations as a set of model elements and constraints according to a given modeling formalism \cite{aalst2022pmhandbook}; hereafter, we refer to the Petri net, a widespread modeling formalism. On the other hand, \revision{conformance checking} is an explainable set of algorithms that allows linking any deviations with the reference Petri net and providing the fitness measure, namely a measure of how much the Petri net fits the new event log \cite{aalst2022pmhandbook}. Many control-flow anomaly detection techniques based on \revision{conformance checking} (hereafter, \revision{conformance checking}-based techniques) use the fitness measure to determine whether an event log is anomalous \cite{bezerra2009pmad, bezerra2013adlogspais, myers2018icsadpm, pecchia2020applicationfailuresanalysispm}. 

The scientific literature also includes many \revision{conformance checking}-independent techniques for control-flow anomaly detection that combine specific types of trace encodings with machine/deep learning \cite{ko2023adsystematicreview, tavares2023pmtraceencoding}. Whereas these techniques are very effective, their explainability is challenging due to both the type of trace encoding employed and the machine/deep learning model used \cite{rawal2022trustworthyaiadvances,li2023explainablead}. Hence, in the following, we focus on the shortcomings of \revision{conformance checking}-based techniques to investigate whether it is possible to support the development of competitive control-flow anomaly detection techniques while maintaining the explainable nature of \revision{conformance checking}.
\begin{figure}[!t]
\centering
\includegraphics[width=\columnwidth]{images/HIGH_LEVEL_VIEW.png}
\caption{A high-level view of the proposed framework for combining \revision{process mining}-based feature extraction with dimensionality reduction for control-flow anomaly detection.}
\label{HIGH_LEVEL_VIEW}
\end{figure}

\subsection{Shortcomings of \revision{conformance checking}-based techniques}
Unfortunately, the detection effectiveness of \revision{conformance checking}-based techniques is affected by noisy data and low-quality Petri nets, which may be due to human errors in the modeling process or representational bias of process discovery algorithms \cite{bezerra2013adlogspais, pecchia2020applicationfailuresanalysispm, aalst2016pm}. Specifically, on the one hand, noisy data may introduce infrequent and deceptive control-flow relations that may result in inconsistent fitness measures, whereas, on the other hand, checking event logs against a low-quality Petri net could lead to an unreliable distribution of fitness measures. Nonetheless, such Petri nets can still be used as references to obtain insightful information for \revision{process mining}-based feature extraction, supporting the development of competitive and explainable \revision{conformance checking}-based techniques for control-flow anomaly detection despite the problems above. For example, a few works outline that token-based \revision{conformance checking} can be used for \revision{process mining}-based feature extraction to build tabular data and develop effective \revision{conformance checking}-based techniques for control-flow anomaly detection \cite{singh2022lapmsh, debenedictis2023dtadiiot}. However, to the best of our knowledge, the scientific literature lacks a structured proposal for \revision{process mining}-based feature extraction using the state-of-the-art \revision{conformance checking} variant, namely alignment-based \revision{conformance checking}.

\subsection{Contributions}
We propose a novel \revision{process mining}-based feature extraction approach with alignment-based \revision{conformance checking}. This variant aligns the deviating control flow with a reference Petri net; the resulting alignment can be inspected to extract additional statistics such as the number of times a given activity caused mismatches \cite{aalst2022pmhandbook}. We integrate this approach into a flexible and explainable framework for developing techniques for control-flow anomaly detection. The framework combines \revision{process mining}-based feature extraction and dimensionality reduction to handle high-dimensional feature sets, achieve detection effectiveness, and support explainability. Notably, in addition to our proposed \revision{process mining}-based feature extraction approach, the framework allows employing other approaches, enabling a fair comparison of multiple \revision{conformance checking}-based and \revision{conformance checking}-independent techniques for control-flow anomaly detection. Figure \ref{HIGH_LEVEL_VIEW} shows a high-level view of the framework. Business processes are monitored, and event logs obtained from the database of information systems. Subsequently, \revision{process mining}-based feature extraction is applied to these event logs and tabular data input to dimensionality reduction to identify control-flow anomalies. We apply several \revision{conformance checking}-based and \revision{conformance checking}-independent framework techniques to publicly available datasets, simulated data of a case study from railways, and real-world data of a case study from healthcare. We show that the framework techniques implementing our approach outperform the baseline \revision{conformance checking}-based techniques while maintaining the explainable nature of \revision{conformance checking}.

In summary, the contributions of this paper are as follows.
\begin{itemize}
    \item{
        A novel \revision{process mining}-based feature extraction approach to support the development of competitive and explainable \revision{conformance checking}-based techniques for control-flow anomaly detection.
    }
    \item{
        A flexible and explainable framework for developing techniques for control-flow anomaly detection using \revision{process mining}-based feature extraction and dimensionality reduction.
    }
    \item{
        Application to synthetic and real-world datasets of several \revision{conformance checking}-based and \revision{conformance checking}-independent framework techniques, evaluating their detection effectiveness and explainability.
    }
\end{itemize}

The rest of the paper is organized as follows.
\begin{itemize}
    \item Section \ref{sec:related_work} reviews the existing techniques for control-flow anomaly detection, categorizing them into \revision{conformance checking}-based and \revision{conformance checking}-independent techniques.
    \item Section \ref{sec:abccfe} provides the preliminaries of \revision{process mining} to establish the notation used throughout the paper, and delves into the details of the proposed \revision{process mining}-based feature extraction approach with alignment-based \revision{conformance checking}.
    \item Section \ref{sec:framework} describes the framework for developing \revision{conformance checking}-based and \revision{conformance checking}-independent techniques for control-flow anomaly detection that combine \revision{process mining}-based feature extraction and dimensionality reduction.
    \item Section \ref{sec:evaluation} presents the experiments conducted with multiple framework and baseline techniques using data from publicly available datasets and case studies.
    \item Section \ref{sec:conclusions} draws the conclusions and presents future work.
\end{itemize}
\label{sec:typesetting-summary}

\section{Preliminaries, key lemmas, and the framework}
% !TEX root =  ../main.tex
\section{Background on causality and abstraction}\label{sec:preliminaries}

This section provides the notation and key concepts related to causal modeling and abstraction theory.

\spara{Notation.} The set of integers from $1$ to $n$ is $[n]$.
The vectors of zeros and ones of size $n$ are $\zeros_n$ and $\ones_n$.
The identity matrix of size $n \times n$ is $\identity_n$. The Frobenius norm is $\frob{\mathbf{A}}$.
The set of positive definite matrices over $\reall^{n\times n}$ is $\pd^n$. The Hadamard product is $\odot$.
Function composition is $\circ$.
The domain of a function is $\dom{\cdot}$ and its kernel $\ker$.
Let $\mathcal{M}(\mathcal{X}^n)$ be the set of Borel measures over $\mathcal{X}^n \subseteq \reall^n$. Given a measure $\mu^n \in \mathcal{M}(\mathcal{X}^n)$ and a measurable map $\varphi^{\V}$, $\mathcal{X}^n \ni \mathbf{x} \overset{\varphi^{\V}}{\longmapsto} \V^\top \mathbf{x} \in \mathcal{X}^m$, we denote by $\varphi^{\V}_{\#}(\mu^n) \coloneqq \mu^n(\varphi^{\V^{-1}}(\mathbf{x}))$ the pushforward measure $\mu^m \in \mathcal{M}(\mathcal{X}^m)$. 


We now present the standard definition of SCM.

\begin{definition}[SCM, \citealp{pearl2009causality}]\label{def:SCM}
A (Markovian) structural causal model (SCM) $\scm^n$ is a tuple $\langle \myendogenous, \myexogenous, \myfunctional, \zeta^\myexogenous \rangle$, where \emph{(i)} $\myendogenous = \{X_1, \ldots, X_n\}$ is a set of $n$ endogenous random variables; \emph{(ii)} $\myexogenous =\{Z_1,\ldots,Z_n\}$ is a set of $n$ exogenous variables; \emph{(iii)} $\myfunctional$ is a set of $n$ functional assignments such that $X_i=f_i(\parents_i, Z_i)$, $\forall \; i \in [n]$, with $ \parents_i \subseteq \myendogenous \setminus \{ X_i\}$; \emph{(iv)} $\zeta^\myexogenous$ is a product probability measure over independent exogenous variables $\zeta^\myexogenous=\prod_{i \in [n]} \zeta^i$, where $\zeta^i=P(Z_i)$. 
\end{definition}
A Markovian SCM induces a directed acyclic graph (DAG) $\mathcal{G}_{\scm^n}$ where the nodes represent the variables $\myendogenous$ and the edges are determined by the structural functions $\myfunctional$; $ \parents_i$ constitutes then the parent set for $X_i$. Furthermore, we can recursively rewrite the set of structural function $\myfunctional$ as a set of mixing functions $\mymixing$ dependent only on the exogenous variables (cf. \cref{app:CA}). A key feature for studying causality is the possibility of defining interventions on the model:
\begin{definition}[Hard intervention, \citealp{pearl2009causality}]\label{def:intervention}
Given SCM $\scm^n = \langle \myendogenous, \myexogenous, \myfunctional, \zeta^\myexogenous \rangle$, a (hard) intervention $\iota = \operatorname{do}(\myendogenous^{\iota} = \mathbf{x}^{\iota})$, $\myendogenous^{\iota}\subseteq \myendogenous$,
is an operator that generates a new post-intervention SCM $\scm^n_\iota = \langle \myendogenous, \myexogenous, \myfunctional_\iota, \zeta^\myexogenous \rangle$ by replacing each function $f_i$ for $X_i\in\myendogenous^{\iota}$ with the constant $x_i^\iota\in \mathbf{x}^\iota$. 
Graphically, an intervention mutilates $\mathcal{G}_{\mathsf{M}^n}$ by removing all the incoming edges of the variables in $\myendogenous^{\iota}$.
\end{definition}

Given multiple SCMs describing the same system at different levels of granularity, CA provides the definition of an $\alpha$-abstraction map to relate these SCMs:
\begin{definition}[$\abst$-abstraction, \citealp{rischel2020category}]\label{def:abstraction}
Given low-level $\mathsf{M}^\ell$ and high-level $\mathsf{M}^h$ SCMs, an $\abst$-abstraction is a triple $\abst = \langle \Rset, \amap, \alphamap{} \rangle$, where \emph{(i)} $\Rset \subseteq \datalow$ is a subset of relevant variables in $\mathsf{M}^\ell$; \emph{(ii)} $\amap: \Rset \rightarrow \datahigh$ is a surjective function between the relevant variables of $\mathsf{M}^\ell$ and the endogenous variables of $\mathsf{M}^h$; \emph{(iii)} $\alphamap{}: \dom{\Rset} \rightarrow \dom{\datahigh}$ is a modular function $\alphamap{} = \bigotimes_{i\in[n]} \alphamap{X^h_i}$ made up by surjective functions $\alphamap{X^h_i}: \dom{\amap^{-1}(X^h_i)} \rightarrow \dom{X^h_i}$ from the outcome of low-level variables $\amap^{-1}(X^h_i) \in \datalow$ onto outcomes of the high-level variables $X^h_i \in \datahigh$.
\end{definition}
Notice that an $\abst$-abstraction simultaneously maps variables via the function $\amap$ and values through the function $\alphamap{}$. The definition itself does not place any constraint on these functions, although a common requirement in the literature is for the abstraction to satisfy \emph{interventional consistency} \cite{rubenstein2017causal,rischel2020category,beckers2019abstracting}. An important class of such well-behaved abstractions is \emph{constructive linear abstraction}, for which the following properties hold. By constructivity, \emph{(i)} $\abst$ is interventionally consistent; \emph{(ii)} all low-level variables are relevant $\Rset=\datalow$; \emph{(iii)} in addition to the map $\alphamap{}$ between endogenous variables, there exists a map ${\alphamap{}}_U$ between exogenous variables satisfying interventional consistency \cite{beckers2019abstracting,schooltink2024aligning}. By linearity, $\alphamap{} = \V^\top \in \reall^{h \times \ell}$ \cite{massidda2024learningcausalabstractionslinear}. \cref{app:CA} provides formal definitions for interventional consistency, linear and constructive abstraction.
\label{sec:prelim}

\section*{Framework}
\label{section:general}
\section{Generalization to Other Circuits}
We extend our analyses to the \textbf{Greater-Than} circuit \cite{hanna2024does}. We find a similar pattern. The mechanisms of the greater-than task are \textit{amplified} after fine-tuning on task data. In contrast, the changes to the mechanisms of the model under toxic fine-tuning are primarily localized to circuit components leading to corruption of the task. Furthermore, we discover our finding of neuroplasticity to hold for the greater-than task, i.e., the model reverts back to its original mechanism after retraining the corrupted model on clean task-specific data. We detail our experiments on this task in \autoref{app:gt}.

\section{Warm-up: Regular case where endpoints are integral}
\label{section:integral_endpoints}
% -Basic idea-
% Key observation that makes integral case easy
As a starter, we consider a special case where all intervals have integral endpoints. 
More formally, for any interval $\x \in \orderedInput$, $I = (i, i+1)$ for some integer $i$ if $I$ is open, and $I = [i, i+1]$ for some integer $i$ if $I$ is closed.
In this section, we show that any $\x \in \orderedInput$ has color $\ffc(\x)\leq 2\omega$, where $\omega$ is the number of colors used by the optimal solution.
Note that in the following, we focus on the closed intervals $\x \in \orderedInput$ with $\nrows{3}{\x} \geq \nrows{1}{\x}+2$ (by Lemma~\ref{lem:open_interval_small_intersection} and Lemma~\ref{lem:integral_open_and_closed_assumtion_not_hold}).

For such a set of integral-endpoints intervals, it is an important property that $\nlmr(\x) = \emptyset$ for any $\x \in \orderedInput$.
That is, the intervals in $\nb(\x)$ are either in $\twins(\x)$ (twins of $\x$) or in $\lmr(\x)$ (twins of the intervals in $\rows{3}{\x}$).
An example is shown in Figure~\ref{fig:column_structure_in_rows}.

\paragraph*{Pick a pivot $\pivot$}

For any closed intervals $\x \in \orderedInput$ with $\nrows{3}{\x} \geq \nrows{1}{\x}+2$, we first identify the interval in $\lmr(\x)$ with the highest color by \ff. 

\begin{definition}\emph{\textbf{(Dominating interval $\y$ in $\lmr(\x)$)}}
    \label{def:y}
    Let interval $\y \in \lmr(\x)$ be the highest colored interval. That is, there is no $I^\prime \in \lmr(\x)$ such that $\ffc(I^\prime) > \ffc(\y)$.
\end{definition}

Note that $\y$ must exist as $|\lmr(\x)| \geq \nrows{3}{\x} \geq \nrows{1}{\x} + 2 \geq 2$.

In this analysis, we want to bound $\ffc(\x)$ by using $\y$ as a pivot $\pivot$ and applying the Pivot bound (Lemma~\ref{lem:prop_ff_equiv}).


\paragraph*{Bound the size of $\pivotset$}
Given that we use $\y$ as the pivot $\pivot$, we bound the size of $\pivotset$.
Recall that $\pivotset$ is the set of intervals in $\nb(\x)$ which are assigned higher color than $\ffc(\pivot)$.
% How interval x can have a higher color
Since all intervals in $\orderedInput$ have integral endpoints, any interval in $\nb(\x)$ is either in $\lmr(\x)$ or in $\twins(\x)$.
Therefore, by the definition of $\y$, only intervals in $\twins(\x)$ are possible to contribute to $\pivotset$.
% Not too many intervals identical to x
Moreover, since we focus on $\x$ that is closed, all intervals in $\twins(\x)$ must be in the rows in $\rows{1}{\x}$. 
We use the following definition to represent the fraction of rows in $\rows{1}{\x}$ that cannot contribute to $\pivotset$.



\begin{definition}
    \label{def:alpha}
    Given any $\hat{I} \in \rows{1}{\x}$, $\alp(\hat{I}) \in [0,1]$ denotes the fraction of rows $\mathcal{R} \in \rows{1}{\x}$ where the interval $I^\prime$ in $\mathcal{R} \cap \nb(\x)$ satisfies \emph{1)} $I^\prime \notin \lmr(\x)$ and \emph{2)} $\ffc(I^\prime) \leq \ffc(\hat{I})$.
    %such that for all $I^\prime \in \rows{1}{\x}$,\footnote{Recall that we say that an interval $I^\prime \in \rows{i}{\x}$ if $I^\prime$ is overlapping with $\x$ and colored by one of the colors in $\nrows{i}{\x}$ by \opt.} $I^\prime \not \in \lmr(\x)$ and $\ffc(I^\prime) \leq \ffc(\hat{I})$.
\end{definition}

Note that we make Definition~\ref{def:alpha} more general for later usage. 
For the integral-endpoints case, it is sufficient to set $\hat{I}$ as $\y$, and $\alp(\y) \cdot \nrows{1}{\x}$ is the number of intervals in $\twins(\x)$ that are not in $\pivotset$.

\begin{lemma}
    \label{lem:integral_S}
    If all intervals in $\orderedInput$ have integral endpoints, by selecting $\y$ as a pivot~$\pivot$, $|\pivotset| = (1-\alp(\y)) \cdot \nrows{1}{\x}$.
\end{lemma}

\begin{proof}
    First, by the definition of $\rows{0}{\x}$, $\x$ does not intersect any interval in $\rows{0}{\x}$. Thus, no interval in $\rows{0}{\x}$ can be in $\pivotset$.
    Next, as all intervals have integral endpoints, all intervals in $\rows{2}{\x}$, $\rows{3}{\x}$, or $\rows{1}{\x} \setminus \twins(\x)$ are in the sets $\lmr(\x)$.
    Thus, all these intervals are assigned a color below $\ffc(\y)$ and cannot contribute to $\pivotset$.
    The remaining intervals are those in $\twins(\x)$.
    By definition, $\alp(\y) \cdot \nrows{1}{\x}$ of these intervals have a color at most $\ffc(\y)$.
    Hence, the $(1 - \alp(\y)) \cdot \nrows{1}{\x}$ intervals are the only intervals intersecting $\x$ that have a color greater than $\ffc(\y)$.
    That is, $|\pivotset| = (1 - \alp(\y)) \cdot \nrows{1}{\x}$.
\end{proof}


\paragraph*{Bound the color $\ffc(\pivot)$}

We bound the color of $\y$ using the Neighborhood bound (Lemma~\ref{lem:prop_ff}).
In general, apart from intervals in $\twins(\x)$, the number of intervals any interval in $\lmr(\x)$ can intersect is shown in the following observation (also see Figure~\ref{fig:intersections_with_v}):

\begin{observation}
    \label{obs:integral}
    Any interval in the set $\lmr(\x)$ intersects
    \begin{enumerate}[(a)]
        \item at most $2$ intervals per row in $\rows{0}{\x}$,
        \item at most $2$ intervals which are not twins with $\x$ per row in $\rows{1}{\x}$,
        \item at most $2$ intervals per row in $\rows{2}{\x}$, and
        \item exactly $1$ interval per row in $\rows{3}{\x}$.
    \end{enumerate}
\end{observation}

% Figures
\begin{figure}[t]
\begin{minipage}{0.4\textwidth}
\centering
\includegraphics[scale=0.4]{pictures/Brick_Rock_integral_rows.pdf}
\caption{
All possible positions of intervals intersecting $\x$ per type of row.
The intervals in $\lmr(\x)$ are drawn in orange, green and blue, the intervals in $\twins(\x)$ in purple.
}
\label{fig:column_structure_in_rows}
\end{minipage}%
\hspace{0.2\textwidth}%
\begin{minipage}{0.4\textwidth}
\centering
\includegraphics[scale=0.4]{pictures/Brick_Rock_integral_intersections.pdf}
\caption{
All possible intersections with interval $\y$ per type of row.
The intervals from $\lmr(\x)$ that intersect with $\y$ are drawn in orange, the intervals in $\twins(\x)$ in purple and the remaining intervals intersecting $\y$ are drawn in yellow.}
\label{fig:intersections_with_v}
\end{minipage}
\end{figure}

% -Start of analysis-
According to the definition of $\y$, Observation~\ref{obs:integral} applies to $\y$. 


\begin{lemma}
    \label{lem:integral_uvw}
    The color of interval $\y$, $\ffc(\y) \leq 2 \omega + \alp(\y) \cdot \nrows{1}{\x} - \nrows{3}{\x}$.
\end{lemma}


\begin{proof}
    By the Neighborhood bound, $\ffc(\y) \leq 1+|N(\y)|$.
    By Observation~\ref{obs:integral}, there are at most $2\nrows{0}{\x} + 2\nrows{1}{\x} + 2\nrows{2}{\x} + \nrows{3}{\x}$ intervals in $\lmr(\x)$ that are in $\nb(\y)$. 
    Together with the $\alp(\y) \cdot \nrows{1}{\x}$ intervals that ``occupied'' the smaller colors from the choice of $\y$ at the moment when $\y$ arrives, $|\nb_{\y}(\y)| \leq 2\nrows{0}{\x} + 2\nrows{1}{\x} + \alp(\y) \cdot \nrows{1}{\x} + 2\nrows{2}{\x} + \nrows{3}{\x}$.
    Observe that by this bound, we count the row that contains interval $\y$ itself. 
    Since $\y$ cannot intersect any interval on this row, we should subtract at least $1$ from this bound.
    Thus,
    \begin{align*}
        \ffc(\y) &\leq 1 + 2\nrows{0}{\x} + 2\nrows{1}{\x} + \alp(\y)\cdot \nrows{1}{\x} + 2\nrows{2}{\x} + \nrows{3}{\x} - 1\\
        &= 2 \omega + \alp(\y)\cdot \nrows{1}{\x} - \nrows{3}{\x}
    \end{align*}
\end{proof}

Note that due to the abstraction of $\alp(\y)$, this proof does not rely on the property of integral endpoints.
Therefore, Observation~\ref{obs:integral} also holds for the general case where the intervals in $\orderedInput$ have arbitrary endpoints.

\begin{corollary}
    \label{cor:integral_LMR}
    Given instance $\orderedInput$ with arbitrary endpoints, the color of interval $\y$ is at most $2\omega + \alp(\y) \cdot \nrows{1}{\x} - \nrows{3}{\x}$.
\end{corollary}

\paragraph*{Proof of Theorem~\ref{thm:integral}}
Now, we are ready to prove Theorem~\ref{thm:integral} that for any $\orderedInput$ with open or closed unit-length intervals that have integral endpoints, $\ffc(\x) \leq 2\omega$ for all $\x \in \orderedInput$, where $\omega$ is the optimal color needed for properly coloring $\orderedInput$.

\begin{proof}
    By Lemma~\ref{lem:integral_open_and_closed_assumtion_not_hold}, we assume that $\nrows{1}{\x} \leq \nrows{3}{\x}-2$. 
    Moreover, by Lemma~\ref{lem:open_interval_small_intersection}, we focus on closed $\x$.
    According to Corollary~\ref{cor:y_exists}, we assume that there exists a dominating interval $\y$ in $\lmr(\x)$ that has the highest color by \ff.
    By Lemma~\ref{lem:integral_uvw}, $\ffc(\y) \leq 2 \omega + \alp(\y) \cdot \nrows{1}{\x} - \nrows{3}{\x}$.

    We now use the Pivot bound to bound the color of interval $\x$ from above, using $\y$ as the role of $\pivot$.
    By Lemma~\ref{lem:integral_S}, $|\pivotset| = (1 - \alp(\y)) \cdot \nrows{1}{\x}$.
    \shortversion{Altogether, we have $\ffc(\x) \leq \ffc(\pivot) + |\pivotset| + 1 \leq 2 \omega - 1$}
    \fullversion{
    It follows that
    \begin{align*}
        \ffc(\x) &\leq \ffc(\pivot) + |\pivotset| + 1 \\
        &= \ffc(\y) + (1 - \alp(\y)) \cdot \nrows{1}{\x} + 1\\
        &\leq 2\omega + \alp(\y)\cdot \nrows{1}{\x} - \nrows{3}{\x} + (1 - \alp(\y)) \cdot \nrows{1}{\x} + 1\\
        &\leq 2\omega + \nrows{1}{\x} - \nrows{3}{\x} + 1 \\
        &\leq 2 \omega - 1
    \end{align*}
    }
\end{proof}

\section{General case: Intervals with arbitrary endpoints}
\label{section:any_endpoints}

In this section, we consider the instance $\orderedInput$, where intervals have arbitrary endpoints. 
More specifically, by Lemma~\ref{lem:integral_open_and_closed_assumtion_not_hold} and Lemma~\ref{lem:open_interval_small_intersection}, we focus on $\x \in \orderedInput$ that are closed and $\nrows{3}{\x} \geq \nrows{1}{\x} + 2$.
Recall from Section~\ref{section:integral_endpoints} that we partition $\nb(\x)$ into $\twins(\x)$, $\lmr(\x)$, and $\nlmr(\x)$, that is, the set of twins, the set of intervals that are twins of intervals in $\rows{3}{\x}$, and the rest of intervals with endpoints ``misaligned'' with the endpoints of $\x$.

\shortversion{Primarily, we pick a pivot interval $\pivot$ and bound its color using the Neighborhood bound. 
The choice of pivot should provide a plausible way to bound the size of the corresponding~$\pivotset$.
However, applying the Pivot bound to bound the color $\ffc(\x)$ in this general case is more challenging than the integral-endpoints case due to the misaligned structure of intervals.
Directly applying the same technique as in the intergral-endpoints case results in a ratio of~$3\omega$ (see the full version).}
\fullversion{
Applying the Pivot bound to bound the color $\ffc(\x)$ in this general case is more challenging than the integral-endpoints case. 
First, unlike in the integral-endpoints case where all intervals that intersect $\x$ must be either in $\twins(\x)$ or in $\lmr(\x)$, the candidates in $\pivotset$ can also be in $\nlmr(\x)$. 
These intervals in $\nlmr(\x)$ can cross-interact with other intervals, making it difficult to bound the size of $\pivotset$.
Second, selecting a good pivot interval $\pivot$ in the general case is more challenging. }
\fullversion{We first show that using the same strategy as we used for the integral-endpoints case only guarantees a bound of $\ffc(\x) \leq 3\omega$.

% All intervals in R_1(x) and R_2(x) can be in set S
\paragraph*{A naive attempt}

As in the integral-endpoints case, let $\pivot$ be $\y$, the interval with the highest color in~$\lmr(\x)$. 
In the general case, any interval in the sets $\rows{1}{\x}$ and $\rows{2}{\x}$ can be in $\nlmr(\x)$ and can therefore be a candidate of $\pivotset$.
If we now apply the same strategy as in the integral-endpoints setting, it is no longer the case that the set $\pivotset$ is a subset of $\rows{1}{\x}$.
Then, in each row $\mathcal{R} \in \rows{2}{\x}$, it can be the case that two intervals (of which one intersects $\x$ and the other does not) are in the neighborhood of the pivot $\pivot$.
Furthermore, the remaining interval on row $\mathcal{R}$ that also intersects with $\x$ can be considered for the set $\pivotset$, resulting in us counting $3$ intervals per row in $\rows{2}{\x}$.
Thus, the bound tends to $3\omega$ when $\nrows{1}{\x}$ and $\nrows{3}{\x}$ are both small.
To work around this, we need to closely look at the color of the intervals in $\rows{2}{\x}$. 
}

% Define interval z
\fullversion{
\subsection{Roadmap}
Primarily, we pick a pivot interval $\pivot$ and bound its color using the Neighborhood bound. 
The choice of pivot should provide a plausible way to bound the size of the corresponding $\pivotset$. }

In the general case, we have another possible candidate of $\pivot$.
Symmetric to the dominating aligned interval $\y$, we define interval $\z$ as the dominating interval in rows in $\rows{2}{\x}$ that is misaligned with $\x$ and assigned the highest color by $\ff$.
Formally,

\begin{definition}
    \shortversion{\label{def:z_short}}
    \fullversion{\label{def:z}}
    \emph{\textbf{(Dominating interval $\z$ in $\rows{2}{\x}\cap\nlmr(\x)$)}}
    Let interval $\z$ be the interval in $\rows{2}{\x}\cap\nlmr(\x)$ with the highest color by $\ff$. That is, for all $I^\prime \in \rows{2}{\x}$, $\ffc(I^\prime) \leq \ffc(\z)$.
\end{definition}

A special property of the selection of $\z$ is that since $\z \in \nlmr(\x)$, any row in $\rows{0}{\x}, \rows{1}{\x}$ or $\rows{3}{\x}$, cannot be in $\rows{3}{\z}$\fullversion{ (Lemma~\ref{lem:R3z_in_R2x_short})}.
Therefore, all rows in $\rows{3}{\z}$ must also be in $\rows{2}{\x}$.

\shortversion{
\begin{lemma}
    \label{lem:R3z_in_R2x_short}
    $\rows{3}{\z} \subseteq \rows{2}{\x}$.
    \end{lemma}
}

% Case distinction
In the general case, we pick one of $\y$ and $\z$ as the pivot and use the Pivot bound to bound the color $\ffc(\x)$.
More specifically, we consider the following two cases.

% CASE 1: c(\y) >= c(\z)

\shortversion{
\paragraph*{Case (1): $\ffc(\z) \leq \ffc(\y)$}
}
\fullversion{
\subparagraph*{(1)}
$\ffc(\z) \leq \ffc(\y)$.}

In this case, the misaligned intervals in $\rows{2}{\x}$ are not in $\pivotset$ since they all have colors of at most $\ffc(\z) \leq \ffc(\y)$.
Using $\y$ as the pivot and following a strategy\shortversion{: }
\fullversion{similar to the case of integral-endpoints, we can show that $\ffc(\x) \leq 2\omega$ (Lemma~\ref{lem:c_u>=c_z}).}

\shortversion{

\smallskip

\begin{lemma}
    \label{lem:c_u>=c_z_short}
    For any interval $\x \in \orderedInput$, if $\ffc(\y) \geq \ffc(\z)$, then $\ffc(\x) \leq 2\omega$.
\end{lemma}
}


% CASE 2: c(\y) < c(\y)
\shortversion{\paragraph*{Case (2): $\ffc(\z) > \ffc(\y)$}}
\fullversion{\subparagraph*{(2)}
% z in S
$\ffc(\z) > \ffc(\y)$.}

% What is interesting about z
In this case, applying the Pivot bound with $\pivot = \y$ leads to a bound of $\ffc(\x) \leq 3 \omega$ as shown in the naive attempt.
To be precise, it is no longer true that $\pivotset \subseteq \rows{1}{\x}$, and any interval in $\rows{2}{\x}$ with a color greater than $\ffc(\y)$ is also a candidate of $\pivotset$.

% Why we use Z
An alternative strategy is to pick the interval $\z$ as the pivot in this case.
However, in the general case, the intervals in $\rows{1}{\z}$ are not particularly structured,
More precisely, we do not know which intervals also intersect with $\x$, and the neighborhood of $\z$ can be huge. 
When~$\z$ has a large number of intersections, using the Neighborhood bound to bound the color $\ffc(\z)$ can lead to a high bound and further lead to a high bound of $\ffc(\x)$ by the Pivot bound.
To deal with this situation, we first identify the rows containing intervals in $\rows{1}{\z}$ that we know their locations, namely, the intervals that are twins with $\z$.

\smallskip

\begin{definition}
    \shortversion{\label{def:Z_short}}
    \fullversion{\label{def:Z}}
    Let the set of rows in $\rows{2}{\x}$ with intervals identical to interval $\z$ be 
    \[ \Z (\x ) = \{ \text{row } \mathcal{R} \in \rows{2}{\x} \mid \mathcal{R} \text{ contains at least one interval that is in } \twins(\z)  \}.\]
\end{definition}

\shortversion{
    In combination with the set $\Z(\x)$ we are interested in the set $\rows{3}{\z}$.       
    Similar to the definition of $\alp$, we define fractions of specific $\rows{2}{\x}$ rows, that it, $\gam$ and $\delt$:
    
    \begin{definition}
    \label{def:gamma_delta_short}
    Given $\z \in \rows{2}{\x}$, 
        \begin{enumerate}[(a)]
            \item $\bet \in [0,1]$ denotes the fraction of rows $\mathcal{R} \in \rows{2}{\x}$ where the interval $I^\prime$ in $\mathcal{R} \cap \nb(\x)$ is in $\Z(\x)$, and
            \item $\gam \in [0,1]$ denotes the fraction of rows $\mathcal{R} \in \rows{2}{\x}$ where the interval $I^\prime$ in $\mathcal{R} \cap \nb(\x)$ is in $\rows{3}{\z}$.
        \end{enumerate}
    \end{definition}
}

% CASE 2.a: |\Z(x)| >= |Z bar(x)|

Note that $\Z(\x) \subseteq \rows{1}{\z}$, and thus $\nrows{3}{\z} \leq |\Z(\x)|$ implies that $\nrows{3}{\z} \leq \nrows{1}{\z}$. On the contrary, $\z$ potentially has a large neighborhood if $\nrows{3}{\z} > |\Z(\x)|$.
According to the size of $|\Z(\x)|$, we consider the following cases \textbf{2.a} and \textbf{2.b}.

\shortversion{\paragraph*{Case (2.a): $\nrows{3}{\z} \leq |\Z(\x)|$}}
\fullversion{\subparagraph*{(2.a)}
$\nrows{3}{\z} \leq |\Z(\x)|$.}

In this case, we pick $\z$ as the pivot $\pivot$.
By the selection of $\z$ and $\ffc(\z) > \ffc(\y)$, the only intervals that can contribute to $\pivotset$ are those in the set $\rows{1}{\x}$.
Furthermore, by $|\Z(\x)| \geq \nrows{3}{\z}$, the neighborhood of $\z$ is small, which enables us to use the Neighborhood bound to prove an upper bound the color of interval $\z$\shortversion{:} \fullversion{(Lemma~\ref{lem:any_z}).}
\shortversion{

\smallskip

\begin{lemma}
    \label{lem:any_z_short}
    The color of interval $\z$ is at most $c(\z) \leq \omega + \alp(\z) \cdot \nrows{1}{\x} - \bet  \cdot \nrows{2}{\x} + \gam  \cdot \nrows{2}{\x}  + \nrows{2}{\x} + \nrows{3}{\x}$. 
\end{lemma}

By selecting $\z$ as the pivot $\pivot$, the following bound of $|\pivotset|$ can be shown by the Neighborhood bound together with bounding the intersection of intervals in $\nlmr(\x)$.
\begin{lemma}
    \label{lem:pivotset_z_short}
    When $\ffc(\z) > \ffc(\y)$, by selecting $\z$ as pivot $\pivot$, $ |\pivotset| = (1 - \alp(\z)) \cdot \nrows{1}{\x}$.
\end{lemma}
}
Then, by the Pivot bound\shortversion{:} \fullversion{($\ffc(\x)$ is bounded by $2\omega$ Lemma~\ref{lem:gamma<delta}).}
\shortversion{
\begin{lemma}
    \shortversion{\label{lem:gamma<delta_short}}
    If $|\Z(\x)| \geq \nrows{3}{\z}$, then $\ffc(\x) \leq 2\omega$.
\end{lemma}
}


% CASE 2.b: |\Z(x)| < |Z bar(x)|
\shortversion{
\paragraph*{Case (2.b): $\nrows{3}{\z} > |\Z(\x)|$}
}
\fullversion{\subparagraph*{(2.b)}
$\nrows{3}{\z} > |\Z(\x)|$.}


This is the most technical part of our analysis.
In this case, $\z$ may intersect more than~$2$ intervals per row on average and have a large neighborhood, and using the Neighborhood bound on $\ffc(\z)$ can lead to a large bound.
We apply our framework recursively on $\z$ by first showing that the intervals in $\rows{3}{\z}$ rows do not intersect many intervals when $\nrows{3}{\z}$ is large\fullversion{ (Lemma~\ref{lem:any_z_bar})}.
Specifically, we let $\zbar$ be the highest colored interval in $\rows{3}{\z}$ that intersects with interval $x$.

\smallskip

\begin{definition}
    \shortversion{\label{def:z_bar_short}}
    \fullversion{\label{def:z_bar}}
    Let interval $\zbar \in \rows{3}{\z}$ be the dominating interval that has the highest color such that there exists no $I^\prime \in \rows{3}{\z}$ that intersects $\x$ such that $\ffc(I^\prime) > \ffc(\zbar)$.\footnote{Recall that we say an interval $I^\prime$ is in $\rows{i}{\x}$ if it is in $\nb(\x)$ and is colored by one of the colors in $\rows{i}{\x}$ by the optimal solution.}
\end{definition}

It is critical to use $\y$, $\z$, or $\zbar$ as the pivot according to whether $\ffc(\zbar)$ is larger than $\ffc(\y)$.

% CASE 2.b.i: c(z bar) >= c(\y)
\shortversion{
\paragraph*{Case (2.b.ii): $\ffc(\zbar) \geq \ffc(\y)$}
}
\fullversion{
\subparagraph*{(2.b.i)}
$\ffc(\zbar) \geq \ffc(\y)$.}

In this case, we apply the Pivot bound by selecting $\zbar$ as the pivot $\pivot$.
Recall that~$\y$ is the dominating interval in $\rows{3}{\x}$.
\shortversion{
Observe that by the definition of interval $\z$, $\ffc(\zbar) < \ffc(\z)$.
Hence, analogous to parameter $\alp$, we define $\delt$.

\smallskip

\begin{definition}
    \label{def:delta_short}
    Given any interval $\hat{I}$, $\delt(\hat{I}) \in [0, 1]$ denotes the fraction of rows $\mathcal{R} \in \rows{2}{\x} \setminus \rows{3}{\z}$ where interval $I^\prime$ in $\mathcal{R} \cap \nb(\x) \cap \nb(\z)$ satisfies $\ffc(I^\prime) \leq \ffc(\hat{I})$.
\end{definition}
}
The condition $\ffc(\zbar) \geq \ffc(\y)$ and the selection of $\zbar$ ensure that no interval in $\rows{3}{\x}$ can be in $\pivotset$.
Furthermore, by the definition of $\zbar$, no interval in $\rows{3}{\z}$ can contribute to $\pivotset$.
Therefore, $\pivotset$ is a subset of intervals in $\rows{1}{\x}$ and $\rows{2}{\x}\setminus \rows{3}{\z}$.

\shortversion{
\begin{lemma}
    \label{lem:pivotset_zbar_short}
    By selecting $\zbar$ as pivot $\pivot$, $|\pivotset| = (1-\alp(\zbar)) \cdot \nrows{1}{\x} + (1 - \delt(\zbar) \cdot (1 - \gam) \cdot 2\nrows{2}{\x}$.
\end{lemma}
}

The definition of $\rows{3}{\z}$ and the selection of $\zbar$ guarantee that a row in the set $\rows{3}{\zbar}$ must contain an interval identical to the interval $\z$ \fullversion{(see Figure~\ref{fig:y_intersections})}.
Thus, $\rows{3}{\zbar} \subseteq \Z(\x)$, and $\nrows{3}{\zbar} \leq |\Z(\x)|$.
Moreover, since $\twins(\zbar) \subseteq \rows{1}{\zbar}$, and every $\rows{3}{\z}$ contains a twin of $\zbar$, $\nrows{1}{\zbar} \geq \nrows{3}{\z}$. 
Together with the condition $\nrows{3}{\z} > |\Z(\x)|$, it guarantees that 
$\nrows{1}{\zbar} \geq \nrows{3}{\z} > |\Z(\x)| \geq \nrows{3}{\zbar}$.
Therefore, $\zbar$ intersects at most $2$ intervals per row on average, and $\ffc(\zbar)$ can be bounded by the Neighborhood bound effectively.
\shortversion{
\begin{lemma}
    \label{lem:any_z_bar_short}
    The color of interval $\zbar$, $\ffc(\zbar) \leq \nrows{0}{\x} + (1 + \alp(\zbar)) \cdot \nrows{1}{\x} + \delt(\zbar) \cdot (1 - \gam ) \cdot 2\nrows{2}{\x} + \bet  \cdot \nrows{2}{\x} + \gam  \cdot \nrows{2}{\x} + 2\nrows{3}{\x}$.
\end{lemma}
}
\fullversion{With the bound $\pivotset \subseteq \rows{1}{\x} \cup (\rows{2}{\x} \setminus \rows{3}{\z})$, the upper bound $\ffc(\x) \leq 2\omega$ (Lemma~\ref{lem:cy>=cu}).}
\shortversion{
Then, by the Pivot bound:
\begin{lemma}
    \label{lem:cy>=cu_short}
    If $\ffc(\zbar) \geq \ffc(\y)$, then $\ffc(\x) \leq 2\omega$   
\end{lemma}
}

% CASE 2.b.ii: c(z bar) < c(\y)
\shortversion{
\paragraph*{Case (2.b.ii): $\ffc(\zbar) < \ffc(\y)$}
}
\fullversion{
\subparagraph*{(2.b.ii)}
$\ffc(\zbar) < \ffc(\y)$.}

% Why we don't use z bar

This case is the tough kernel of the analysis.
In this case, considering using $\y$, $\z$, or $\zbar$ as the pivot only is not sufficient to have an upper bound of $\ffc(\x)$ that is smaller than $3\omega$.
% Why not \y
Recall that this case is a subcase of case $\ffc(\y) < \ffc(\z)$, picking $\y$ as the pivot does not stop any interval in $\rows{2}{\x}$ from contributing to $\pivotset$. 
% Why not \z
On the other hand, as $\nrows{3}{\z} > |\Z(\x)|$, $\nrows{3}{\z}$ may be much larger than $\nrows{1}{\z}$.
Applying the Neighborhood bound yields a huge upper bound of $\ffc(\pivot)$.
% Why not \zbar
Finally, condition $\ffc(\zbar) < \ffc(\y)$ implies that when $\zbar$ is selected as the pivot, $\pivotset$ could contain any interval in $\rows{3}{\x}$ and become too large.


% Why we can use z and y
To deal with this tough case, we express the upper bounds obtained by using $\y$ as the pivot and using $\z$ as the pivot in terms of $|\Z(\x)|$ and $\nrows{2}{\z}$\fullversion{ (Lemmas~\ref{lem:any_z} and~\ref{lem:any_uvw})}. 
\shortversion{
For the color of interval $\z$, Lemma~\ref{lem:any_z_short} still suffices.
However, for interval $\y$ we take a closer look at the neighborhood of interval $\y$ given the current assumptions.
In order to be more precise, let us first make additional observations regarding the intersections of interval $\y$ with intervals in $\rows{2}{\x}$ rows (also see Figure~\ref{fig:y_intersections_short}).

\begin{figure}[t]
\centering
\begin{minipage}{0.5\textwidth}
\centering
\includegraphics[width=\textwidth]{pictures/Brick_Rock_y.pdf}
\end{minipage}%
\caption{
An exemplary non-exhaustive list of possible rows per type of $\rows{2}{\x}$.
Yellow intervals intersect only interval $\y$, purple intervals intersect only $\x$ and orange intervals intersect both $\y$ and $\x$.}
\label{fig:y_intersections_short}
\end{figure}

\medskip

\begin{observation}
    \label{obs:any_y_short}
    Interval $\y$ intersects
    \begin{enumerate}[(a)]
        \item at most $1$ interval that is not in $\nb(\x)$ per row contained in $\rows{2}{\x}$, and
        \item at most $1$ interval that is in $\nb(\x)$ per row contained in $\rows{3}{\z}$,
        \item at most $1$ intervals that is in $\nb(\x)$ per row contained in $\Z(\x)$, and
        \item at most $1$ intervals that is in $\nb(\x)$ per row contained in $\rows{2}{\x} \setminus (\Z(\x) \cup \rows{3}{\z} )$.
    \end{enumerate}
\end{observation}

\begin{lemma}
    \label{lem:any_uvw_short}
    The color of interval $\y$,
    $\ffc(\y) \leq 2\nrows{0}{\x} + 2\nrows{1}{\x} + \alp(\y) \cdot \nrows{1}{\x} + \gam  \cdot \nrows{2}{\x} + \nrows{2}{\x} + \delt(\y) \cdot (1 - \gam ) \cdot \nrows{2}{\x} + \nrows{3}{\x}$. 
\end{lemma}
\begin{proof}
    By the Neighborhood bound, $\ffc(\y) \leq 1 + |\mathcal{N}(\y)|$.
    By Observation~\ref{obs:integral}, there are at most $ 2\nrows{0}{\x} + 2\nrows{1}{\x} + \nrows{3}{\x}$ intervals in $\rows{0}{\x}, \rows{1}{\x}$ and $\rows{3}{\x}$ that are in $\nb(\y)$.
    Together with the at most $\alp(\y) \cdot \nrows{1}{\x}$ intervals that were assigned a color below $\ffc(\y)$ which are either in $\twins(\x)$ or in $\nlmr(\x)$.
    
    By Observation~\ref{obs:any_y_short}, there are at most $\nrows{2}{\x} + \gam \cdot \nrows{2}{\x} + \bet \cdot \nrows{2}{\x} + (1 - \gam - \bet) \cdot \nrows{2}{\x}$ intervals that are in $\rows{2}{\x}$ and in $\nb(\y)$.
    By our assumption that $\ffc(\zbar) < \ffc(\y)$, the $\gam  \cdot \nrows{2}{\x}$ intervals in the set $\rows{3}{\z} \cap \nb(\x)$, must have a color smaller than $\ffc(\y)$.
    For the remaining $(1 - \gam) \cdot \nrows{2}{\x}$ intervals that are in $\nb(\x)$, by the definition of $\delt$,  $(1 - \delt(\y)) \cdot (1 - \gam) \cdot \nrows{2}{\x}$ are assigned a color larger than $\ffc(\z)$ and hence we do not consider them for this bound.
    
    Observe that by this bound we count the row that contains interval $\y$ itself.
    Since interval $\y$ cannot intersect any interval on this row, we should subtract at least $1$ from this bound. 
    Thus,
    \begin{align*}
            \ffc(\y) &\leq |\nb(\y)| + 1\\
            &\leq 2\nrows{0}{\x} + 2\nrows{1}{\x} + \alp(\y) \cdot \nrows{1}{\x} + \gam  \cdot \nrows{2}{\x} \\&\hspace{1cm} + \nrows{2}{\x} + \delt(\y) \cdot (1 - \gam ) \cdot \nrows{2}{\x} + \nrows{3}{\x}
        \end{align*}    
\end{proof}

When we pick interval $\z$ as pivot, the size of $|\pivotset|$ proven in Lemma~\ref{lem:pivotset_z_short} remains true.
For the case where we pick interval $\y$, we show the following bound.

\begin{lemma}
    \label{lem:pivotset_y_exact_short}
    When $\ffc(\y) \geq \ffc(\zbar)$, by selecting $\y$ as pivot $\pivot$, $|\pivotset| \leq (1 - \alp(\y)) \cdot \nrows{1}{\x} + (1 - \delt(\z)) \cdot (1 - \gam ) \cdot \nrows{2}{\x} + (1 - \gam ) \cdot \nrows{2}{\x}$
\end{lemma}
\begin{proof}
    First, by the definition of $\rows{0}{\x}$, $\x$ does not intersect any interval in $\rows{0}{\x}$.
    Thus, no interval in $\rows{0}{\x}$ can be in $\pivotset$.
    By the definition of $\alp$, only $(1 - \alp(\y) ) \cdot \nrows{1}{\x}$ intervals in $\rows{1}{\x}$ are assigned a color strictly larger than $\ffc(\y)$.
    Thus the only intervals in an $\rows{1}{\x}$ that are eligible for $\pivotset$ are those $(1 - \alp(\y) ) \cdot \nrows{1}{\x}$ intervals.
    Since $\ffc(\y) > \ffc(\zbar)$, no interval in an $\rows{3}{\z}$ row is assigned a color larger than $\ffc(\y)$.
    Thus, none of the $\gam \cdot 2\nrows{2}{\x}$ intervals in $\rows{3}{\z} \cap \nb(\x)$ can contribute to $\pivotset$.
    By the definition of $\delt$, only $(1 - \delt(\y) ) \cdot (1 - \gam) \cdot \nrows{2}{\x}$ intervals both in $\nb(\y)$ and $\rows{2}{\x}$ are assigned a color strictly larger than $\ffc(\y)$.
    Furthermore, all $(1 - \gam) \cdot \nrows{2}{\x}$ intervals which are in $\rows{2}{\x}$ but not in $\nb(\y)$ and not in $\rows{3}{\z}$ could potentially be assigned a color larger than $\ffc(\y)$ and therefore contribute to $\pivotset$.
    Finally, as all intervals in $\rows{3}{\x}$ are in $\lmr(\x)$, all intervals in $\rows{3}{\x}$ have a color below $\ffc(\y)$ and cannot contribute to $\pivotset$.
    Thus, $|\pivotset| \leq (1 - \alp(\y)) \cdot \nrows{1}{\x} + (1 - \delt(\z)) \cdot (1 - \gam ) \cdot \nrows{2}{\x} + (1 - \gam ) \cdot \nrows{2}{\x}$.
\end{proof}`}
These formulations show that the upper bounds from applying the Pivot bound using interval $\y$ as $\pivot$ and using interval $\z$ as $\pivot$ form a trade-off.
More specifically, the upper bound using $\y$ is maximized when $|\Z(\x)| = \emptyset$, while upper bound using $\z$ is maximized when $|\Z(\x)| = \rows{2}{\x}$.
Then, we use that in any circumstance, the upper bound is at most $2\omega + \frac{|\Z(\x)|}{2}$, which is strictly less than $2\omega + \frac{\nrows{2}{\x}}{2}\cdot\omega$ 
(Definition~\ref{def:Z}). 
\fullversion{Since $\nrows{2}{\x}$ can be upper bounded by $\frac{2}{3}\omega - 1$ (Corollary~\ref{cor:r2_bound}), the color assigned to interval $\x$ is bounded by $\result$ (Theorem~\ref{thm:any}).}
\shortversion{We next prove a bound on the number of $\rows{2}{\x}$ rows.
\begin{lemma}
    \label{lem:r2_bound_short}
    For any interval $\x \in \orderedInput$, if $\nrows{2}{\x} \geq \frac{2}{3} \omega$, then $\ffc(\x) \leq \frac{7}{3} \omega - 2$. 
\end{lemma}

\begin{proof}
    We show this using the Neighborhood bound.
    Observe that $\nrows{0}{\x} \geq 1$, since interval $\x$ does not intersect any intervals on its own row.
    Furthermore, note that the Neighborhood bound is maximized when we maximize the number of intervals that intersect interval $\x$.
    That is, we maximize the bound when we maximize $\nrows{3}{\x}$.
    Then, it follows by $\nrows{0}{\x} \geq 1$ and $\nrows{2}{\x} \geq \frac{2}{3}\omega$, that $\nrows{3}{\x} = \frac{1}{3}\omega - 1$.
    For the remaining rows, we maximize the bound by maximizing $\nrows{2}{\x}$, and hence $\nrows{2}{\x} = \frac{2}{3}\omega$.
    Then, $\nrows{1}{\x} = 0$, and we get the following upper bound on the color of interval $\x$,
    \begin{align*}
        \ffc(\x) &\leq \nrows{1}{\x} + 2\nrows{2}{\x} + 3\nrows{3}{\x} + 1 \\
            &\leq 0 + 2 \cdot \frac{2}{3} \omega + 3 \cdot (\frac{1}{3}\omega - 1) + 1\\
            &= \frac{7}{3} \omega - 2
    \end{align*}
\end{proof}

Then, we can bound the color assigned to interval $\x$ by $\result$.
\paragraph*{Proof of Theorem~\ref{thm:any}.}
\emph{For any interval $\x \in \orderedInput$, the color of $\x$ is at most $\ffc(\x) \leq \result$.}

\begin{proof}
    By Lemma~\ref{lem:open_interval_small_intersection},~\ref{lem:integral_open_and_closed_assumtion_not_hold}, \ref{lem:c_u>=c_z_short}, \ref{lem:gamma<delta_short} and \ref{lem:cy>=cu_short}, $\ffc(\x) \leq 2\omega$ if $\x$ is open, $\nrows{1}{\x} \leq \nrows{3}{\x} - 2$, $\ffc(\y) < \ffc(\z)$, $\bet  < \gam $ or $\ffc(\zbar) < \ffc(\y)$.
    Thus in the following we focus on closed intervals with $\nrows{1}{\x} \leq \nrows{3}{\x} - 2$, $\ffc(\zbar) < \ffc(\y) < \ffc(\z)$ and $\bet  < \gam $.

    Now we can compute the first bound on the color of interval $\x$, using the Pivot bound with interval $\y$ as pivot $\pivot$.
    According to Lemma~\ref{lem:any_uvw_short}, $\ffc(\y) \leq  2\nrows{0}{\x} + 2\nrows{1}{\x} + \alp(\y) \cdot \nrows{1}{\x} + \gam  \cdot \nrows{2}{\x} + \nrows{2}{\x} + \delt(\z) \cdot (1 - \gam ) \cdot \nrows{2}{\x} + \nrows{3}{\x}$ and, by taking $\y$ as pivot $\pivot$,  Lemma~\ref{lem:pivotset_y_exact_short}, $|\pivotset| \leq (1 - \alp(\y)) \cdot \nrows{1}{\x} + (1 - \delt(\z)) \cdot (1 - \gam ) \cdot \nrows{2}{\x} + (1 - \gam ) \cdot \nrows{2}{\x}$.
    Thus,

    \begin{align*}
        \ffc(\x) &\leq \ffc(\pivot) + |\pivotset| + 1\\
        &\leq \ffc(\y) + (1 - \alp(\y)) \cdot \nrows{1}{\x} + (1 - \delt(\z)) \cdot (1 - \gam ) \cdot \nrows{2}{\x} \\ &\hspace{1cm} + (1 - \gam ) \cdot \nrows{2}{\x} + 1 \\
        &\leq 2\nrows{0}{\x} + 3\nrows{1}{\x} + 2\nrows{2}{\x} + (1 - \gam ) \cdot \nrows{2}{\x} + \nrows{3}{\x} + 1 \\
        &\leq 2\nrows{0}{\x} + 2\nrows{1}{\x} + (1 - \gam ) \cdot \nrows{2}{\x} + 2\nrows{2}{\x} + 2\nrows{3}{\x} - 1\\
        &= 2\omega + (1 - \gam ) \cdot \nrows{2}{\x} - 1
    \end{align*}

    Next we move to the other bound.
    By taking $\z$ as pivot $\pivot$, according to Lemma~\ref{lem:any_z_short}, $\ffc(\z) \leq \omega + \alp(\z) \cdot \nrows{1}{\x} - \bet  \cdot \nrows{2}{\x} + \gam  \cdot \nrows{2}{\x} + \nrows{2}{\x} + \nrows{3}{\x}$, and Lemma~\ref{lem:pivotset_z_short}, $|\pivotset| = (1 - \alp(\z))\cdot\nrows{1}{\x}$.
    Then, it follows that,

    \begin{align*}
        \ffc(\x) &\leq \ffc(\pivot) + |\pivotset| + 1\\
        &\leq \ffc(\z) + (1 - \alp(\z)) \cdot \nrows{1}{\x} + 1 \\
        &\leq \omega + \nrows{1}{\x} - \bet  \cdot \nrows{2}{\x} + \gam  \cdot \nrows{2}{\x} + \nrows{2}{\x} + \nrows{3}{\x} + 1\\ 
        &\leq 2\omega + \gam  \cdot \nrows{2}{\x} - 1 % beta cannot be 0
    \end{align*}

    We have proven that simultaneously, the color of interval $\x$ is at most $\ffc(\x) \leq 2\omega + \gam  \cdot \nrows{2}{\x} - 1$ and at most $\ffc(\x) \leq 2\omega + (1 - \gam ) \cdot \nrows{2}{\x} - 1$.
    Then, it follows that for any value of $\gam $ the color of interval $\x$ is at most $\ffc(\x) \leq 2\omega + \frac{1}{2}\cdot \nrows{2}{\x} - 1$
    Then, it follows from Lemma~\ref{lem:r2_bound_short} that the number of $\rows{2}{\x}$ rows must be less than $\nrows{2}{\x}  < \frac{2}{3}\omega$, as otherwise the color of interval $\x$ is bounded by $\ffc(\x) \leq \frac{7}{3} \omega - 2$.
    Thus, the color of interval $\x$ is at most $\ffc(\x) < 2\omega + \frac{1}{3}\cdot \omega - 1 = \frac{7}{3}\omega - 1$.
    Which, by the integrality of $\ffc(\x)$, is at most $\ffc(\x) \leq \result$  
\end{proof}
}

%%%%%%%%%%%%%%%%%%%%%%%%%%%%%%%%%%%%%%%%%%%%%%%%%%%%%%%%%%%%%%%%%%%%%%%%%%%%%%%%%%%%%%
%% END OF SHORTVERSION
%%%%%%%%%%%%%%%%%%%%%%%%%%%%%%%%%%%%%%%%%%%%%%%%%%%%%%%%%%%%%%%%%%%%%%%%%%%%%%%%%%%%%%
\fullversion{
% Why lemma 11 still holds for any endpoints
\subsection{Analysis}

\paragraph*{Case (1): $\ffc(\y) \geq \ffc(\z)$}

We start by proving that in case (1), that is, when $\ffc(\y) \geq \ffc(\z)$, the color of interval $\x$ can be bounded by $2\omega$.

\medskip

\runtitle{Pick a pivot $\pivot$}.
    Given the condition where interval $\y$ has a color larger or equal to interval $\z$, picking $\y$ as a pivot provides a smaller bound of $|\pivotset|$.
    Therefore, we pick $\y$ as the pivot $\pivot$.

\medskip

\runtitle{Bound the color of $\ffc(\pivot)$.}
    Next, we bound on the color of the pivot $\ffc(\pivot)$.
    By Corollary~\ref{cor:integral_LMR}, $\ffc(\pivot) = \ffc(\y) \leq 2\omega + \alp (\y) \cdot \nrows{1}{\x} - \nrows{3}{\x}$, where $\alp(\y)$ is the fraction of rows in $\rows{1}{\x}$ that contains intervals in $\twins(\x) \cup \nlmr(\x)$ and having color not higher than $\y$.

\medskip

\runtitle{Bound the size of $\pivotset$}.
Given that we use $\y$ as pivot $\pivot$, we bound the size of $\pivotset$.
Recall that $\pivotset$ is the set of intervals in $\nb(\x)$ which are assigned colors strictly higher than $\ffc(\pivot)$.
Since $\ffc(\y) \geq \ffc(\z)$, only intervals in $\twins(\x)$ are possible to contribute to~$\pivotset$.


\begin{lemma}
    \label{lem:pivotset_y}
    When $\ffc(\y) \geq \ffc(\z)$, by selecting $\y$ as pivot $\pivot$, $|\pivotset| = (1 - \alp(\y)) \cdot \nrows{1}{\x}$.
\end{lemma}

\begin{proof}
    First, by the definition of $\rows{0}{\x}$, $\x$ does not intersect any interval in $\rows{0}{\x}$.
    Thus, no interval in $\rows{0}{\x}$ can be in $\pivotset$.
    Next, the intervals in $\rows{2}{\x}$ are either in $\lmr(\x)$ or in $\nlmr(\x)$.
    Moreover, since $\ffc(\y) \geq \ffc(\z)$, all intervals in $\rows{2}{\x}$ have a color below $\ffc(\y)$ and cannot contribute to $\pivotset$.
    Similarly, as all intervals in $\rows{3}{\x}$ are in $\lmr(\x)$, all intervals in $\rows{3}{\x}$ have a color below $\ffc(\y)$ and cannot contribute to $\pivotset$.
    The remaining intervals are the intervals in $\rows{1}{\x}$.
    By definition, $\alp(\y) \cdot \nrows{1}{\x}$ of these intervals have a color at most $\ffc(\y)$.
    Hence, the remaining $(1 - \alp(\y) ) \cdot \nrows{1}{\x}$ are the only intervals intersecting $\x$ that have a color greater than $\ffc(\y)$. That is, $|\pivotset| = (1 - \alp(\y)) \cdot \nrows{1}{\x}$.
\end{proof}

\medskip

\runtitle{Bound the color of $\ffc(\x)$.}
Now, we wrap up the analysis of the case where $\ffc(\y) \geq \ffc(\z)$ using the bounds of $\ffc(\pivot)$ and $|\pivotset|$.

\begin{lemma}
    \fullversion{\label{lem:c_u>=c_z}}
    For any interval $\x \in \orderedInput$, if $\ffc(\y) \geq \ffc(\z)$, then $\ffc(\x) \leq 2\omega$.
\end{lemma}

\begin{proof}
    By Lemma~\ref{lem:integral_open_and_closed_assumtion_not_hold} and Lemma~\ref{lem:open_interval_small_intersection}, $\ffc(\x) \leq 2\omega$ if $\x$ is open or $\nrows{3}{\x} < \nrows{1}{\x} + 2$.
    Thus, in the following, we focus on closed intervals $\x$ with $\nrows{1}{\x} \leq \nrows{3}{\x} - 2$.
    
    By taking $\y$ as pivot $\pivot$, according to Corollary~\ref{cor:integral_LMR}, $\ffc(\y) \leq 2\omega + \alp(\y) \cdot \nrows{1}{\x} - \nrows{3}{\x}$.
    On the other hand, by  Lemma~\ref{lem:pivotset_y}, $|\pivotset| = (1 - \alp(\y)) \cdot \nrows{1}{\x}$.
    Thus,
    \begin{align*}
        \ffc(\x) &\leq \ffc(\pivot) + |\pivotset| + 1\\
        &\leq \ffc(\y) + (1 - \alp(\y) ) \cdot \nrows{1}{\x} + 1\\
        &\leq 2\omega + \nrows{1}{\x} - \nrows{3}{\x} + 1\\
        &\leq 2\omega - 1
    \end{align*}
\end{proof}

\hide{
In the case where interval $\z$ is in $\lmr(\x)$, then, from the definition of $\y$, it must be the case that $\ffc(\y) \geq \ffc(\z)$.
Therefore, the next result follows directly from Lemma~\ref{lem:c_u>=c_z}.

\begin{corollary}
    \label{cor:z_in_pos_u_v}
    If $\z \in \lmr(\x)$, then $\ffc(\x) \leq 2\omega$.
\end{corollary}
}

\medskip

\paragraph*{Case (2): $\ffc(\y) < \ffc(\z)$}

% Intro to next case
From now on, we focus on the case where $\ffc(\y) < \ffc(\z)$.

\medskip

\runtitle{Pick a pivot $\pivot$.}
    The condition where $\ffc(\y) < \ffc(\z)$ guarantees that all intervals in the set $\lmr(\x)$ have colors at most $\ffc(\z)$.
    Therefore, for the sake of the size of the $\pivotset$, it is more efficient to pick the interval $\z$ as a pivot than picking the interval~$\y$.
    Therefore, we pick interval $\z$ as pivot $\pivot$.

%when $\ffc(\z) > \ffc(\y)$. 

Recall from Definition~\ref{def:z} that $\z$ is the interval with the highest color in the rows in~$\rows{2}{\x}$. 
The selection of $\z$ attains a good structural property on $\rows{3}{\z}$ with regards to $\rows{2}{\x}$:

\begin{lemma}
    \label{lem:R3z_in_R2x}
    $\rows{3}{\z} \subseteq \rows{2}{\x}$.
\end{lemma}

\begin{proof}
    By case distinction on $\rows{i}{\x}$, we prove by contradiction that there is no intersection between $\rows{3}{\z} \cap \rows{i}{\x}$.
    
    \textbf{(1)} Suppose on the contrary that $\rows{3}{\z} \cap \rows{0}{\x} \neq \emptyset$.
    If there exists a row with $3$ intervals that intersect $\z$, but none of them intersects $\x$, then the intervals $\x$ and $\z$ must not intersect.
    Thus, $\z \notin \nlmr(\x)$, which is a contradiction by the definition of $\z$.

    \medskip
    
    \textbf{(2)} Suppose on the contrary that $\rows{3}{\z} \cap \rows{1}{\x} \neq \emptyset$.
    If there exists a row with $3$ intervals that intersect $\z$, but only one of those intervals intersects $\x$. 
    Then, either the interval~$\z$ is in the set $\lmr(\x)$ or $\x$ and $\z$ do not intersect.
    In either case, $\z \not \in \nlmr(\x)$, which is a contradiction by the definition of $\z$.
    
    \medskip
    
    \textbf{(3)} Suppose on the contrary that $\rows{3}{\z} \cap \rows{3}{\x} \neq \emptyset$. 
    If there exists a row with $3$ intervals that intersect both $\x$ and $\z$, then interval $\z$ must be in the set $\twins(\x)$.
    Then, $\z \not \in \nlmr(\x)$, which is a contradiction by the definition of $\z$.
    
    \medskip
    
    By \textbf{(1)} to \textbf{(3)}, $\rows{3}{\z} \cap \left( \rows{0}{\x} \cup \rows{1}{\x} \cup \rows{3}{\x} \right) = \emptyset$.
    Therefore, it follows that $\rows{3}{\z} \subseteq \rows{2}{\x}$.
\end{proof}

% Redefine Z Zbar and zbar
Recall from Definition~\ref{def:Z} that the set $\Z(\x) \subset \rows{2}{\x}$ contains all intervals in an~$\rows{2}{\x}$ row that are twins of interval $\z$. 
\hide{And the set $\rows{3}{\z}$ contains all rows in which~$3$ intervals intersect interval $\z$.
This set $\rows{3}{\z}$ must also be a subset of $\rows{2}{\x}$.}
Furthermore, recall from Definition~\ref{def:z_bar} that interval $\zbar$ is the highest colored interval in $\rows{3}{\z}$ that intersects with $\x$.
Observe that the intersection of the sets $\Z(\x)$ and $\rows{3}{\z}$ must be empty, as a row cannot have both $3$ intervals intersecting $\z$ and an interval identical to $\z$, and therefore $|\Z(\x)| + \nrows{3}{\z} \leq \nrows{2}{\x}$.

% Figures
\begin{figure}[t]
\begin{minipage}{0.35\textwidth}
\centering
\includegraphics[width=\textwidth]{pictures/Brick_Rock_r2.pdf}
\end{minipage}%
\hspace{0.1\textwidth}%
\begin{minipage}{0.55\textwidth}
\centering
\includegraphics[width=\textwidth]{pictures/Brick_Rock_z.pdf}
\end{minipage}
\caption{
An exemplary non-exhaustive list of possible rows per type of row.
Yellow intervals intersect only interval $\z$, purple intervals intersect only $\x$ and orange intervals intersect both $\z$ and $\x$.
}
\label{fig:r2_and_z_intersections}
\end{figure}

\medskip

\runtitle{Bound the color of $\ffc(\pivot)$.}
% Observations
We first provide observations on the maximum number of intersections intervals in $\nlmr(\x)$ and specifically $\z$ have in each type of row (also see Figure~\ref{fig:r2_and_z_intersections}).

\begin{observation}
    \label{obs:any_r2}
    Any interval in the set $\nlmr(\x)$ intersects
    \begin{enumerate}[(a)]
        \item at most $1$ interval, other than $\x$, per row in $\rows{0}{\x}$,
        \item at most $1$ interval, which does not intersect interval $\x$, per row in $\rows{1}{\x}$, and
        \item exactly $2$ intervals per row in $\rows{3}{\x}$.
    \end{enumerate}
\end{observation}

\begin{observation}
    \label{obs:any_z}
    Interval $\z$ intersects
    \begin{enumerate}[(a)]
        \item exactly $3$ intervals per row in $\rows{3}{\z}$,
        \item exactly $1$ interval per row in $\Z(\x)$, and
        \item at most $2$ intervals per row in $\rows{2}{\x} \setminus \left( \Z(\x) \cup \rows{3}{\z} \right)$.
    \end{enumerate}
\end{observation}

% Definition of gamma and delta
Furthermore, we define two variables with respect to the sets $\Z(\x)$ and $\rows{3}{\z}$.
Similarly to $\alp(i)$ being a fraction of the $\rows{1}{\x}$ rows, we define $\bet $ and $\gam $ as fractions of the set~$\rows{2}{\x}$ with respect to interval $\z$.

\begin{definition}
    \label{def:gamma_delta}
    Given $\z \in \rows{2}{\x}$, 
    \begin{enumerate}[(a)]
        \item $\bet \in [0,1]$ denotes the fraction of rows $\mathcal{R} \in \rows{2}{\x}$ where the interval $I^\prime$ in $\mathcal{R} \cap \nb(\x)$ is in $\zbar$, and
        \item $\gam \in [0,1]$ denotes the fraction of rows $\mathcal{R} \in \rows{2}{\x}$ where the interval $I^\prime$ in $\mathcal{R} \cap \nb(\x)$ is in $\rows{3}{\z}$.
    \end{enumerate}
\end{definition}

% A sentence introducing the lemma
Now we are equipped to compute an upper bound on the color of interval $\z$.

\begin{lemma}
    \fullversion{\label{lem:any_z}}
    The color of interval $\z$ is at most $c(\z) \leq \omega + \alp(\z) \cdot \nrows{1}{\x} - \bet  \cdot \nrows{2}{\x} + \gam  \cdot \nrows{2}{\x}  + \nrows{2}{\x} + \nrows{3}{\x}$. 
\end{lemma}

\begin{proof}
    Using the Neighborhood bound and Observations~\ref{obs:any_r2} and~\ref{obs:any_z}, we can describe the maximum color of interval $\z$ as $1$ plus the number of intersections per type of row.
    Observe that by doing this, we count the row that contains interval $\z$ itself, and since $\z$ cannot intersect any interval in this row, we may subtract at least $1$.
    Then it follows that
    \begin{align*}
            \ffc(\z) &\leq \nrows{0}{\x} + (1 + \alp(\z)) \cdot \nrows{1}{\x} + \bet \cdot \nrows{2}{\x} + \gam \cdot 3\nrows{2}{\x} \\&\hspace{2cm}+ (1 - \bet - \gam) \cdot 2\nrows{2}{\x} + 2\nrows{3}{\x} + 1 - 1\\
            &= \omega + \alp(\z) \cdot \nrows{1}{\x} - \bet  \cdot \nrows{2}{\x} + \gam \cdot \nrows{2}{\x} + \nrows{2}{\x} + \nrows{3}{\x}
        \end{align*}
\end{proof}

% Case distinction

Next, we make a case distinction based on the relative sizes of the sets $\Z(\x)$ and $\rows{3}{\z}$.
%%%%%%%%%%%%%%%%%%%%%%%%%%%%
% CASE 2.a
%%%%%%%%%%%%%%%%%%%%%%%%%%%%
\paragraph*{Case (2.a): $|\Z(\x)| \geq \nrows{3}{\z}$}
We start with the case where the number of intervals in the set $\Z(\x)$ is at least as large as the number of intervals in the set $\rows{3}{\z}$.

\medskip 

\runtitle{Bound the size of $\pivotset$.}

\begin{lemma}
    \fullversion{\label{lem:pivotset_z}}
    When $\ffc(\z) > \ffc(\y)$, by selecting $\z$ as pivot $\pivot$, $ |\pivotset| = (1 - \alp(\z)) \cdot \nrows{1}{\x}$.
\end{lemma}

\begin{proof}
    First, by the definition of $\rows{0}{\x}$, $\x$ does not intersect any interval in $\rows{0}{\x}$.
    Thus, no interval in $\rows{0}{\x}$ can be in $\pivotset$.
    Next, all intervals in $\rows{2}{\x}$ are either in $\lmr(\x)$ or $\nlmr(\x)$.
    Since $\ffc(\z) > \ffc(\y)$, it follows that all intervals in $\rows{2}{\x}$ have a color below $\ffc(\z)$ and cannot contribute to $\pivotset$.
    Similarly, as all intervals in $\rows{3}{\x}$ are in $\lmr(\x)$, all intervals in $\rows{3}{\x}$ have a color below $\ffc(\z)$ and cannot contribute to $\pivotset$.
    The remaining intervals are those in $\rows{1}{\x}$.
    By definition, $\alp(\x) \cdot \nrows{1}{\x}$ of these intervals have a color at most $\ffc(\z)$.
    Hence, the remaining $(1 - \alp(\z) ) \cdot \nrows{1}{\x}$ are the only intervals intersecting $\x$ that have a color greater than $\ffc(\z)$. That is, $|\pivotset| = (1 - \alp(\z)) \cdot \nrows{1}{\x}$.
\end{proof}

\medskip

\runtitle{Bound the color of $\ffc(\x)$.}

\fullversion{\begin{lemma}
    \fullversion{\label{lem:gamma<delta}}
    If $|\Z(\x)| \geq \nrows{3}{\z}$, then $\ffc(\x) \leq 2\omega$.
\end{lemma}}

\begin{proof}
    Assume that $\ffc(\z) > \ffc(\y)$ and interval $\x$ is a closed interval, as otherwise by Lemma~\ref{lem:c_u>=c_z} and Lemma~\ref{lem:open_interval_small_intersection} it follows that $\ffc(\x) \leq 2\omega$.
    %From Lemma~\ref{lem:any_z} we have a bound for the color of interval $\z$, i.e., $\ffc(\z) \leq \omega + \alp(\z) \cdot \nrows{1}{\x} - \bet  \cdot \nrows{2}{\x} + \gam  \cdot \nrows{2}{\x} + \nrows{2}{\x} + \nrows{3}{\x}$.
    From the definition of $\bet $ and $\gam $ it follows that since $|\Z(\x)| \geq \nrows{3}{\z}$, also $\bet  \geq \gam $.
    
    Now we can compute a bound on the color of interval $\x$, using the Pivot bound where we take interval $\z$ as $\pivot$.
    By Lemma~\ref{lem:any_z}, the color of interval $\z$ is bounded by $\ffc(\z) \leq \omega + \alp(\z) \cdot \nrows{1}{\x} - \bet  \cdot \nrows{2}{\x} + \gam  \cdot \nrows{2}{\x} + \nrows{2}{\x} + \nrows{3}{\x}$.
    And, by Lemma~\ref{lem:pivotset_z}, the size of $\pivotset$ equals $(1 - \alp(\z)) \cdot \nrows{1}{\x}$.
    Then it follows that,

    \begin{align*}
        \ffc(\x) &\leq \ffc(\pivot) + |\pivotset| + 1\\
        &= \ffc(\z) + (1 - \alp(\z)) \cdot \nrows{1}{\x} + 1 \\
        &\leq \omega + \nrows{1}{\x} - \bet  \cdot \nrows{2}{\x} + \gam  \cdot \nrows{2}{\x} + \nrows{2}{\x} + \nrows{3}{\x} + 1 \\
        &\leq \omega + \nrows{1}{\x} + \nrows{2}{\x} + \nrows{3}{\x} + 1 \\
        &\leq 2\omega
    \end{align*}
\end{proof}
%%%%%%%%%%%%%%%%%%%%%%%%%%%%
% CASE 2.b
%%%%%%%%%%%%%%%%%%%%%%%%%%%%
% Continue with the second case of the case distinction. And introduce new one
\paragraph*{Case (2.b): $|\Z(\x)| < \nrows{3}{\z}$}
We continue with the case where the number of intervals in the set $\Z(\x)$ is strictly smaller than the number of intervals in the set $\rows{3}{\z}$.
This relation between $\Z(\x)$ and $\rows{3}{\z}$ indicates that the average number of intersections interval $\z$ has per row can be greater than $2$.
Then, it might be interesting to explore other options as a pivot.
For this we make another case distinction on the relation between the colors of interval $\y$ and interval $\zbar$.

%%%%%%%%%%%%%%%%%%%%%%%%%%%%
% CASE 2.b.i
%%%%%%%%%%%%%%%%%%%%%%%%%%%%
\paragraph*{Case (2.b.i): $\ffc(\zbar) \geq \ffc(\y)$}
First, let us explore the case where the color of interval $\zbar$ is at least as large as the color of interval $\y$.

\medskip

\runtitle{Pick a pivot $\pivot$}
The condition that $\ffc(\zbar) \geq \ffc(\y)$ guarantees that all intervals in the set $\lmr(\x)$ have colors at most $\ffc(\zbar)$.
Therefore, for the sake of the size of the size of the set $\pivotset$, it is more efficient to pick interval $\zbar$ as a pivot compared to picking interval $\y$.
Therefore, we pick interval $\zbar$ as pivot $\pivot$.

\medskip

\runtitle{Bound the color $\ffc(\pivot)$}

% zbar is not in LMR
As we now want to compute an upper bound on the color of interval $\zbar$, it is of importance to know where interval $\zbar$ might be located.

\begin{lemma}
    \label{lem:zbar_not_in_LMR}
    For any interval $I^\prime \in \rows{3}{\z}$, $I^\prime \not \in \lmr(\x)$
\end{lemma}

\begin{proof}
    Assume aiming towards a contradiction that $I^\prime \in \lmr(\x)$.
    \medskip

    \runtitle{(1) $I^\prime$ is a closed interval.} %$i \in L(\x)$. 
    %$L(\x)$ contains only closed intervals.
    Assume without loss of generality that interval $I^\prime$ intersects interval $\x$ on the left-hand side of interval $\x$.
    Since $I^\prime \in \rows{3}{\z}$, and $I^\prime$ is a closed interval, it follows that there are precisely two positions where interval $\z$ could lie.
    The first possible position is to the left of interval $I^\prime$.
    In this case, interval $\z$ and interval $\x$ cannot intersect.
    Then it follows that $\z \not \in \nlmr(\x)$, which is a contradiction by the definition of $\z$.
    The other possible position is to the right of interval $I^\prime$.
    In this case, interval $\z$ must be identical to interval $\x$, i.e., $\z \in \twins(\x)$.
    Then it follows that $\z \not \in \nlmr(\x)$, which is a contradiction by the definition of $\z$.
    
    \medskip
    
    \runtitle{(2) $I^\prime$ is an open interval.} %$i \in M(\x)$.
    %$M(\x)$ contains only open intervals.
    Since $I^\prime \in \lmr(\x)$ and $I^\prime$ is an open interval, it follows that $I^\prime \in \twins(\x)$.
    Similarly, since $I^\prime \in \rows{3}{\z}$, and $I^\prime$ is an open interval, it follows that $I^\prime \in \twins(\z)$.
    An interval can only be in both $\twins(\x)$ and $\twins(\z)$ if interval $\x$ and interval $\z$ are identical, i.e., $\z \in \twins(\x)$.
    Then it follows that $\z \not \in \nlmr(\x)$, which is a contradiction by the definition of $\z$.
\end{proof}

% Figures
\begin{figure}%[ht]
\centering
\begin{minipage}{0.49\textwidth}
\centering
\includegraphics[width=\textwidth]{pictures/Brick_Rock_zbar.pdf}
\end{minipage}%
\hspace{0.01\textwidth}%
\begin{minipage}{0.49\textwidth}
\centering
\includegraphics[width=\textwidth]{pictures/Brick_Rock_y.pdf}
\end{minipage}%
\caption{
An exemplary non-exhaustive list of possible rows per type of $\rows{2}{\x}$.
Yellow intervals intersect only interval $\zbar$ (resp. $\y$), purple intervals intersect only $\x$ and orange intervals intersect both $\zbar$ (resp. $\y$) and $\x$.}
\label{fig:y_intersections}
\end{figure}

% Recall observations from previous proof
Note that Observation~\ref{obs:any_r2} remains true for interval $\zbar$.
% Introduce new observations
Now, let us make an additional observation regarding the intersections interval $\zbar$ has with intervals in $\rows{2}{\x}$ rows (also see Figure~\ref{fig:y_intersections}):
%Exemplary rows are depicted in Figure~\ref{fig:r3z_intersections} to give some intuition.

\begin{observation}
    \label{obs:any_z_bar}
    Interval $\zbar$ intersects
    \begin{enumerate}[(a)]
        \item exactly $1$ interval per row contained in $\rows{3}{\z}$,
        \item exactly $3$ intervals per row contained in $\Z(\x)$, and
        \item at most $2$ intervals per row contained in $\rows{2}{\x} \setminus (\Z(\x) \cup \rows{3}{\z} )$.
    \end{enumerate}
\end{observation}


% Not all intervals are below or above zbar
Furthermore, it is important to specify how many of the intervals in $\rows{2}{\x} \setminus \rows{3}{\z}$ are assigned a color larger than $\ffc(\zbar)$ or smaller than $\ffc(\zbar)$.
This was previously not necessary, when we considered interval $\z$, which is, by definition, the interval in $\rows{2}{\x}$ assigned the largest color.
Analogously, we do not need to consider the intervals in $\rows{3}{\z}$ as per definition interval $\zbar$ is the largest colored interval in this set.

\begin{definition}
    \label{def:delta}
    Given any interval $\hat{I}$, $\delt(\hat{I}) \in [0, 1]$ denotes the fraction of rows $\mathcal{R} \in \rows{2}{\x} \setminus \rows{3}{\z}$ where interval $I^\prime$ in $\mathcal{R} \cap \nb(\x) \cap \nb(\z)$ satisfies $\ffc(I^\prime) \leq \ffc(\hat{I})$.
\end{definition}

% Introduce the next lemma
Now we are equipped to compute a bound on the color of interval $\zbar$.

\begin{lemma}
    \label{lem:any_z_bar}
    The color of interval $\zbar$, $\ffc(\zbar) \leq \nrows{0}{\x} + (1 + \alp(\zbar)) \cdot \nrows{1}{\x} + \delt(\zbar) \cdot (1 - \gam ) \cdot 2\nrows{2}{\x} + \bet  \cdot \nrows{2}{\x} + \gam  \cdot \nrows{2}{\x} + 2\nrows{3}{\x}$.
\end{lemma}

\begin{proof}
    By the Neighborhood bound $\ffc(\zbar) \leq 1 + |\mathcal{N}(\zbar)|$.
    By Observation~\ref{obs:any_r2}, there are at most $\nrows{0}{\x} + 2\nrows{1}{\x} + 2\nrows{3}{\x}$ intervals in $\rows{0}{\x}, \rows{1}{\x}$ and $\rows{3}{\x}$ that are in $\mathcal{N}(\zbar)$.
    However, by Definition~\ref{def:alpha}, $(1 - \alp(\zbar))$ of those intervals have a color larger than $\ffc(\zbar)$ and hence we do not consider them for this bound.
    By Observation~\ref{obs:any_z_bar}, there are at most $\nrows{3}{\z} + 3\cdot|\Z(\x)| + 2\cdot|\rows{2}{\x}\setminus(\Z(\x)\cup\rows{3}{\z})| = \gam \cdot \nrows{2}{\x} + \bet \cdot 3\nrows{2}{\x} + (1 - \bet - \gam) \cdot 2\nrows{2}{\x}$ intervals in $\rows{2}{\x}$ that are in $\mathcal{N}(\zbar)$.
    However, by Definition~\ref{def:delta}, $(1 - \delt(\zbar)) \cdot (\bet \cdot 2\nrows{2}{\x} + (1 - \bet - \gam) \cdot 2\nrows{2}{\x})$ of those intervals have a color larger than $\ffc(\zbar)$ and hence we do not consider them for this bound.
    Observe that by this bound, we count the row that contains interval $\zbar$ itself.
    Since $\zbar$ cannot intersect any interval on this row, we should subtract at least 1 from this bound.
    Thus,

    \begin{align*}
            \ffc(\zbar) &\leq \nrows{0}{\x} + 2\nrows{1}{\x} - (1 - \alp(\zbar)) \cdot \nrows{1}{\x} + 2\nrows{3}{\x}  
            \\ &\hspace{1cm} + \bet  \cdot 3\nrows{2}{\x} + \gam  \cdot \nrows{2}{\x} + (1 - \bet  - \gam ) \cdot 2\nrows{2}{\x}
            \\ &\hspace{1cm} - (1 - \delt(\zbar)) \cdot \left( \bet  \cdot 2\nrows{2}{\x} + (1 - \bet  - \gam ) \cdot 2\nrows{2}{\x} \right) -1 + 1\\
            %
            &= \nrows{0}{\x} + (1 + \alp(\zbar)) \cdot \nrows{1}{\x} + \delt(\zbar) \cdot (1 - \gam ) \cdot 2\nrows{2}{\x} + \bet  \cdot \nrows{2}{\x} \\&\hspace{1cm} + \gam  \cdot \nrows{2}{\x} + 2\nrows{3}{\x}
        \end{align*}
\end{proof}

\medskip

\runtitle{Bound the size of $\pivotset$.}

\begin{lemma}
    \label{lem:pivotset_zbar}
    By selecting $\zbar$ as pivot $\pivot$, $|\pivotset| = (1-\alp(\zbar)) \cdot \nrows{1}{\x} + (1 - \delt(\zbar) \cdot (1 - \gam) \cdot 2\nrows{2}{\x}$.
\end{lemma}

\begin{proof}
    First, by the definition of $\rows{0}{\x}$, $\x$ does not intersect any interval in $\rows{0}{\x}$.
    Thus, no interval in $\rows{0}{\x}$ can be in $\pivotset$.
    Next, all intervals in $\rows{3}{\x}$ are in the set $\lmr(\x)$.
    Thus, all these intervals are assigned a color below $\ffc(\zbar)$ and cannot contribute to $\pivotset$.
    Of the intervals in $\rows{1}{\x}$, by definition of $\alp$, only $(1 - \alp(\zbar))\cdot \nrows{1}{\x}$ intervals have a color larger than $\ffc(\zbar)$, and can contribute to set $\pivotset$.
    Of the $\rows{2}{\x}$ rows, out of the intervals intersecting $\x$ that are in the rows that are also in the set $\rows{3}{\z}$, none have a color larger than $\ffc(\zbar)$ by the definition of $\zbar$.
    Then, of the remaining $(1 - \gam) \cdot \nrows{2}{\x}$ $\rows{2}{\x}$ rows, by the definition of $\delt$, only $(1 - \delt(\zbar)) \cdot (1 - \bet - \gam) \cdot 2\nrows{2}{\x}$ intervals have a color larger than $\ffc(\zbar)$ and can contribute to the set $\pivotset$.
    Then it follows that $|\pivotset| = (1 - \alp(\zbar))\cdot \nrows{1}{\x} + (1 - \delt(\zbar)) \cdot (1 - \gam) \cdot 2\nrows{2}{\x} $
\end{proof}

\medskip

\runtitle{Bound the color $\ffc(\x)$.}

% Something to connect the lemma's?
Now we can use this result to prove an upper bound on the color of interval $\x$, when the color of interval $\zbar$ is at least as large as the color of interval $\y$.

\begin{lemma}
    \label{lem:cy>=cu}
    If $\ffc(\zbar) \geq \ffc(\y)$, then $\ffc(\x) \leq 2\omega$   
\end{lemma}

\begin{proof}
    By Lemma~\ref{lem:gamma<delta}, we assume that $|\Z(\x)| < \nrows{3}{\z}$, and thus $\bet  < \gam$.
    By Lemma~\ref{lem:any_z_bar}, $\ffc(\zbar) \leq \nrows{0}{\x} + (1 + \alp(\zbar)) \cdot \nrows{1}{\x} + \delt(\zbar) \cdot (1 - \gam) \cdot 2\nrows{2}{\x} + \bet  \cdot \nrows{2}{\x} + \gam  \cdot \nrows{2}{\x} + 2\nrows{3}{\x}$.
    By Lemma~\ref{lem:pivotset_zbar}, $|\pivotset| = (1-\alp(\zbar)) \cdot \nrows{1}{\x} + (1 - \delt(\zbar) \cdot (1 - \gam) \cdot 2\nrows{2}{\x}$.
    It follows that,
    
    \begin{align*}
        \ffc(\x) &\leq \ffc(\pivot) + |\pivotset| + 1\\
        &\leq \ffc(\zbar) + (1 - \alp(\zbar)) \cdot \nrows{1}{\x} + ( 1 -  \delt(\zbar) ) \cdot (1 - \gam ) \cdot 2\nrows{2}{\x} + 1\\
        &\leq \nrows{0}{\x} + 2\nrows{1}{\x}  + \bet  \cdot \nrows{2}{\x} + \gam  \cdot \nrows{2}{\x} + (1 - \gam ) \cdot 2\nrows{2}{\x} + 2\nrows{3}{\x} + 1\\
        &< \nrows{0}{\x} + 2\nrows{1}{\x}  + \bet  \cdot \nrows{2}{\x} + (1 - \bet ) \cdot \nrows{2}{\x} + \nrows{2}{\x} + 2\nrows{3}{\x} + 1\\
        &\leq 2\omega
    \end{align*}
\end{proof}

%%%%%%%%%%%%%%%%%%%%%%%%%%%%
% CASE 2.b.ii
%%%%%%%%%%%%%%%%%%%%%%%%%%%%
\paragraph*{Case (2.b.ii): $\ffc(\y) > \ffc(\zbar)$ }

In order to prove Theorem~\ref{thm:any} we still need to consider the tough kernel of the analysis.
That is, the case where $\ffc(\zbar) < \ffc(\y) < \ffc(\z)$ and $\bet  < \gam $.
For this case we abandon our usual mechanism slightly, where instead of focusing on a single pivot $\pivot$ and a single set $\pivotset$, we explore the option where two distinct pairs of a pivot and a set cannot both admit a large solution.

\medskip

\runtitle{Pick the pivots $\pivot$.}
Although we have reasoned earlier that picking interval $\y$ and interval $\z$ as the pivot is not sufficient to have an upper bound on $\ffc(\x)$ that is smaller than $3\omega$, we can deal with this tough case by expressing the bound obtained by using $\y$ or $\z$ as the pivot in terms of $|\Z(\x)|$.

\medskip


\runtitle{Bound the color of $\pivot$}
Let us first take a closer look at the color of interval $\y$.
Although Lemma~\ref{lem:integral_uvw} is technically still correct, knowing that $\ffc(\zbar) < \ffc(\y)$, we can be slightly more precise about which intersections are assigned a color greater than $\ffc(\y)$.
In order to be more precise, let us first make additional observations regarding the intersections of interval $\y$ with interval in $\rows{2}{\x}$ rows (also see Figure~\ref{fig:y_intersections}).

\begin{observation}
    \label{obs:any_y}
    Interval $\y$ intersects
    \begin{enumerate}[(a)]
        \item at most $1$ interval that is not in $\nb(\x)$ per row contained in $\rows{2}{\x}$, and
        \item at most $1$ interval that is in $\nb(\x)$ per row contained in $\rows{3}{\z}$,
        \item at most $1$ intervals that is in $\nb(\x)$ per row contained in $\Z(\x)$, and
        \item at most $1$ intervals that is in $\nb(\x)$ per row contained in $\rows{2}{\x} \setminus (\Z(\x) \cup \rows{3}{\z} )$.
    \end{enumerate}
\end{observation}

\begin{lemma}
    \label{lem:any_uvw}
    The color of interval $\y$,
    $\ffc(\y) \leq 2\nrows{0}{\x} + 2\nrows{1}{\x} + \alp(\y) \cdot \nrows{1}{\x} + \gam  \cdot \nrows{2}{\x} + \nrows{2}{\x} + \delt(\y) \cdot (1 - \gam ) \cdot \nrows{2}{\x} + \nrows{3}{\x}$. 
\end{lemma}

\begin{proof}
    By the Neighborhood bound, $\ffc(\y) \leq 1 + |\mathcal{N}(\y)|$.
    By Observation~\ref{obs:integral}, there are at most $ 2\nrows{0}{\x} + 2\nrows{1}{\x} + \nrows{3}{\x}$ intervals in $\rows{0}{\x}, \rows{1}{\x}$ and $\rows{3}{\x}$ that are in $\nb(\y)$.
    Together with the at most $\alp(\y) \cdot \nrows{1}{\x}$ intervals that were assigned a color below $\ffc(\y)$ which are either in $\twins(\x)$ or do not intersect $\y$ and are in $\nlmr(\x)$.
    
    By Observation~\ref{obs:any_y}, there are at most $\nrows{2}{\x} + \gam \cdot \nrows{2}{\x} + \bet \cdot \nrows{2}{\x} + (1 - \gam - \bet) \cdot \nrows{2}{\x}$ intervals that are in $\rows{2}{\x}$ and in $\nb(\y)$.
    By our assumption that $\ffc(\zbar) < \ffc(\y)$, the $\gam  \cdot \nrows{2}{\x}$ intervals in the set $\rows{3}{\z} \cap \nb(\x)$, must have a color smaller than $\ffc(\y)$.
    For the remaining $(1 - \gam) \cdot \nrows{2}{\x}$ intervals that are in $\nb(\x)$, by the definition of $\delt$,  $(1 - \delt(\y)) \cdot (1 - \gam) \cdot \nrows{2}{\x}$ are assigned a color larger than $\ffc(\z)$ and hence we do not consider them for this bound.
    
    Observe that by this bound we count the row that contains interval $\y$ itself.
    Since interval $\y$ cannot intersect any interval on this row, we should subtract at least 1 from this bound. 
    Thus,

    \begin{align*}
            \ffc(\y) &\leq |\nb(\y)| + 1\\
            &\leq 2\nrows{0}{\x} + 2\nrows{1}{\x} + \alp(\y) \cdot \nrows{1}{\x} + \gam  \cdot \nrows{2}{\x} \\&\hspace{1cm} + \nrows{2}{\x} + \delt(\y) \cdot (1 - \gam ) \cdot \nrows{2}{\x} + \nrows{3}{\x}
        \end{align*}    
\end{proof}

As for the bound on the color of interval $\z$, Lemma~\ref{lem:any_z} still suffices.

\medskip


\runtitle{Bound the sizes of $\pivotset$.}
Given that we use $\y$ as pivot $\pivot$, we bound the size of $\pivotset$.

\begin{lemma}
    \label{lem:pivotset_y_exact}
    When $\ffc(\y) \geq \ffc(\zbar)$, by selecting $\y$ as pivot $\pivot$, $|\pivotset| \leq (1 - \alp(\y)) \cdot \nrows{1}{\x} + (1 - \delt(\z)) \cdot (1 - \gam ) \cdot \nrows{2}{\x} + (1 - \gam ) \cdot \nrows{2}{\x}$
\end{lemma}

\begin{proof}
    First, by the definition of $\rows{0}{\x}$, $\x$ does not intersect any interval in $\rows{0}{\x}$.
    Thus, no interval in $\rows{0}{\x}$ can be in $\pivotset$.
    By the definition of $\alp$, only $(1 - \alp(\y) ) \cdot \nrows{1}{\x}$ intervals in $\rows{1}{\x}$ are assigned a color strictly larger than $\ffc(\y)$.
    Thus the only intervals in an $\rows{1}{\x}$ that are eligible for $\pivotset$ are those $(1 - \alp(\y) ) \cdot \nrows{1}{\x}$ intervals.
    Since $\ffc(\y) > \ffc(\zbar)$, no interval in an $\rows{3}{\z}$ row is assigned a color larger than $\ffc(\y)$.
    Thus, none of the $\gam \cdot 2\nrows{2}{\x}$ intervals in $\rows{3}{\z} \cap \nb(\x)$ can contribute to $\pivotset$.
    By the definition of $\delt$, only $(1 - \delt(\y) ) \cdot (1 - \gam) \cdot \nrows{2}{\x}$ intervals both in $\nb(\y)$ and $\rows{2}{\x}$ are assigned a color strictly larger than $\ffc(\y)$.
    Furthermore, all $(1 - \gam) \cdot \nrows{2}{\x}$ intervals which are in $\rows{2}{\x}$ but not in $\nb(\y)$ and not in $\rows{3}{\z}$ could potentially be assigned a color larger than $\ffc(\y)$ and therefore contribute to $\pivotset$.
    Finally, as all intervals in $\rows{3}{\x}$ are in $\lmr(\x)$, all intervals in $\rows{3}{\x}$ have a color below $\ffc(\y)$ and cannot contribute to $\pivotset$.
    Thus, $|\pivotset| \leq (1 - \alp(\y)) \cdot \nrows{1}{\x} + (1 - \delt(\z)) \cdot (1 - \gam ) \cdot \nrows{2}{\x} + (1 - \gam ) \cdot \nrows{2}{\x}$.
\end{proof}

As for the set $\pivotset$ when we pick interval $\z$ as pivot, Lemma~\ref{lem:pivotset_z} still suffices.

\medskip


\runtitle{Bound the size of $\ffc(\x)$.}

% The number of R_2(x) rows is bounded
As the bound we are going to prove in the proof of Theorem~\ref{thm:any} is based on the size of set $\Z(\x)$, which in turn is strictly smaller than the size of set $\rows{2}{\x}$, an upper bound on $\nrows{2}{\x}$ improves the bound on the color of $\x$ we would otherwise obtain.
This upper bound on the number of rows in $\rows{2}{\x}$ is obtained by a lower bound on the number of rows in $\rows{3}{\x}$.

\begin{lemma}
    \label{lem:r3_lb}
    For any interval $\x \in \orderedInput$, if $\nrows{3}{\x} < \frac{1}{3}\omega$, then $\ffc(\x) \leq \frac{7}{3} \omega - 2$.
\end{lemma}

\begin{proof}
    We show this using the Neighborhood bound.
    Observe that $\nrows{0}{\x} \geq 1$, since interval $\x$ does not intersect any intervals on its own row.
    Furthermore, note that the Neighborhood bound is maximized when we maximize the number of intervals that intersect interval $\x$.
    That is, we maximize the bound when we maximize $\nrows{3}{\x}$.
    Then, $\nrows{3}{\x} = \frac{1}{3}\omega - 1$.
    For the remaining rows, we maximize the bound by maximizing $\nrows{2}{\x}$, and hence $\nrows{2}{\x} = \frac{2}{3}\omega$.
    Then, $\nrows{1}{\x} = 0$, and we get the following upper bound on the color of interval $\x$,
    \begin{align*}
        \ffc(\x) &\leq \nrows{1}{\x} + 2\nrows{2}{\x} + 3\nrows{3}{\x} + 1 \\
            &\leq 0 + 2 \cdot \frac{2}{3} \omega + 3 \cdot (\frac{1}{3}\omega - 1) + 1\\
            &= \frac{7}{3} \omega - 2
    \end{align*}
\end{proof}

Now we have shown that $\nrows{3}{\x} \geq \frac{1}{3}\omega$, it simply follows that $\nrows{2}{\x} < \frac{2}{3}\omega$.

\begin{corollary}
    \label{cor:r2_bound}
    For any interval $\x \in \orderedInput$, if $\nrows{2}{\x} \geq \frac{2}{3} \omega$, then $\ffc(\x) \leq \frac{7}{3} \omega - 2$. 
\end{corollary}

% Intro to main thm
We have now shown everything we need to show in order to compute a bound on the color of interval $\x$.

\paragraph*{Proof of Theorem~\ref{thm:any}.}
\emph{For any interval $\x \in \orderedInput$, the color of $\x$ is at most $\ffc(\x) \leq \result$.}

\begin{proof}
    By Lemma~\ref{lem:open_interval_small_intersection},~\ref{lem:integral_open_and_closed_assumtion_not_hold}, \ref{lem:c_u>=c_z}, \ref{lem:gamma<delta} and \ref{lem:cy>=cu}, $\ffc(\x) \leq 2\omega$ if $\x$ is open, $\nrows{1}{\x} \leq \nrows{3}{\x} - 2$, $\ffc(\y) < \ffc(\z)$, $\bet  < \gam $ or $\ffc(\zbar) < \ffc(\y)$.
    Thus in the following we focus on closed intervals with $\nrows{1}{\x} \leq \nrows{3}{\x} - 2$, $\ffc(\zbar) < \ffc(\y) < \ffc(\z)$ and $\bet  < \gam $.

    Now we can compute the first bound on the color of interval $\x$, using the Pivot bound with interval $\y$ as pivot $\pivot$.
    According to Lemma~\ref{lem:any_uvw}, $\ffc(\y) \leq  2\nrows{0}{\x} + 2\nrows{1}{\x} + \alp(\y) \cdot \nrows{1}{\x} + \gam  \cdot \nrows{2}{\x} + \nrows{2}{\x} + \delt(\z) \cdot (1 - \gam ) \cdot \nrows{2}{\x} + \nrows{3}{\x}$ and, by taking $\y$ as pivot $\pivot$,  Lemma~\ref{lem:pivotset_y_exact}, $|\pivotset| \leq (1 - \alp(\y)) \cdot \nrows{1}{\x} + (1 - \delt(\z)) \cdot (1 - \gam ) \cdot \nrows{2}{\x} + (1 - \gam ) \cdot \nrows{2}{\x}$.
    Thus,

    \begin{align*}
        \ffc(\x) &\leq \ffc(\pivot) + |\pivotset| + 1\\
        &\leq \ffc(\y) + (1 - \alp(\y)) \cdot \nrows{1}{\x} + (1 - \delt(\z)) \cdot (1 - \gam ) \cdot \nrows{2}{\x} \\ &\hspace{1cm} + (1 - \gam ) \cdot \nrows{2}{\x} + 1 \\
        &\leq 2\nrows{0}{\x} + 3\nrows{1}{\x} + 2\nrows{2}{\x} + (1 - \gam ) \cdot \nrows{2}{\x} + \nrows{3}{\x} + 1 \\
        &\leq 2\nrows{0}{\x} + 2\nrows{1}{\x} + (1 - \gam ) \cdot \nrows{2}{\x} + 2\nrows{2}{\x} + 2\nrows{3}{\x} - 1\\
        &= 2\omega + (1 - \gam ) \cdot \nrows{2}{\x} - 1
    \end{align*}

    Next we move to the other bound.
    By taking $\z$ as pivot $\pivot$, according to Lemma~\ref{lem:any_z}, $\ffc(\z) \leq \omega + \alp(\z) \cdot \nrows{1}{\x} - \bet  \cdot \nrows{2}{\x} + \gam  \cdot \nrows{2}{\x} + \nrows{2}{\x} + \nrows{3}{\x}$, and Lemma~\ref{lem:pivotset_z}, $|\pivotset| = (1 - \alp(\z))\cdot\nrows{1}{\x}$.
    Then, it follows that,

    \begin{align*}
        \ffc(\x) &\leq \ffc(\pivot) + |\pivotset| + 1\\
        &\leq \ffc(\z) + (1 - \alp(\z)) \cdot \nrows{1}{\x} + 1 \\
        &\leq \omega + \nrows{1}{\x} - \bet  \cdot \nrows{2}{\x} + \gam  \cdot \nrows{2}{\x} + \nrows{2}{\x} + \nrows{3}{\x} + 1\\ 
        &\leq 2\omega + \gam  \cdot \nrows{2}{\x} - 1 % beta cannot be 0
    \end{align*}

    We have proven that simultaneously, the color of interval $\x$ is at most $\ffc(\x) \leq 2\omega + \gam  \cdot \nrows{2}{\x} - 1$ and at most $\ffc(\x) \leq 2\omega + (1 - \gam ) \cdot \nrows{2}{\x} - 1$.
    Then, it follows that for any value of $\gam $ the color of interval $\x$ is at most $\ffc(\x) \leq 2\omega + \frac{1}{2}\cdot \nrows{2}{\x} - 1$
    Then, it follows from Corollary~\ref{cor:r2_bound} that the number of $\rows{2}{\x}$ rows must be less than $\nrows{2}{\x}  < \frac{2}{3}\omega$, as otherwise the color of interval $\x$ is bounded by $\ffc(\x) \leq \frac{7}{3} \omega - 2$.
    Thus, the color of interval $\x$ is at most $\ffc(\x) < 2\omega + \frac{1}{3}\cdot \omega - 1 = \frac{7}{3}\omega - 1$.
    Which, by the integrality of $\ffc(\x)$, is at most $\ffc(\x) \leq \result$  
\end{proof}
}

\section{Conclusion}
\label{section:conclusion}
\section{Conclusion}
In this work, we propose a simple yet effective approach, called SMILE, for graph few-shot learning with fewer tasks. Specifically, we introduce a novel dual-level mixup strategy, including within-task and across-task mixup, for enriching the diversity of nodes within each task and the diversity of tasks. Also, we incorporate the degree-based prior information to learn expressive node embeddings. Theoretically, we prove that SMILE effectively enhances the model's generalization performance. Empirically, we conduct extensive experiments on multiple benchmarks and the results suggest that SMILE significantly outperforms other baselines, including both in-domain and cross-domain few-shot settings.

\newpage

%%
%% Bibliography
%%

%% Please use bibtex, 

\bibliography{main}

\newpage



% \appendix

% \renewcommand{\fullversion}[1]{{#1}}
% \renewcommand{\shortversion}[1]{}
% \renewcommand\thesection{\arabic{section}}
% %\setcounter{algorithm}{0}    % Set it if needed.
% \setcounter{theorem}{0}    % Set it if needed.
% \setcounter{figure}{0}    % Set it if needed.
% \clearpage
% \pagenumbering{arabic}

% \title{{\Huge Appendix: Full version}\\On the \ff Algorithm for Online Unit-Interval Coloring}
% \titlerunning{Appendix}


% \maketitle

% \begin{abstract}
%     \begin{abstract}  
Test time scaling is currently one of the most active research areas that shows promise after training time scaling has reached its limits.
Deep-thinking (DT) models are a class of recurrent models that can perform easy-to-hard generalization by assigning more compute to harder test samples.
However, due to their inability to determine the complexity of a test sample, DT models have to use a large amount of computation for both easy and hard test samples.
Excessive test time computation is wasteful and can cause the ``overthinking'' problem where more test time computation leads to worse results.
In this paper, we introduce a test time training method for determining the optimal amount of computation needed for each sample during test time.
We also propose Conv-LiGRU, a novel recurrent architecture for efficient and robust visual reasoning. 
Extensive experiments demonstrate that Conv-LiGRU is more stable than DT, effectively mitigates the ``overthinking'' phenomenon, and achieves superior accuracy.
\end{abstract}  
% \end{abstract}

% \section{Introduction}
% \section{Introduction}
\label{sec:introduction}
The business processes of organizations are experiencing ever-increasing complexity due to the large amount of data, high number of users, and high-tech devices involved \cite{martin2021pmopportunitieschallenges, beerepoot2023biggestbpmproblems}. This complexity may cause business processes to deviate from normal control flow due to unforeseen and disruptive anomalies \cite{adams2023proceddsriftdetection}. These control-flow anomalies manifest as unknown, skipped, and wrongly-ordered activities in the traces of event logs monitored from the execution of business processes \cite{ko2023adsystematicreview}. For the sake of clarity, let us consider an illustrative example of such anomalies. Figure \ref{FP_ANOMALIES} shows a so-called event log footprint, which captures the control flow relations of four activities of a hypothetical event log. In particular, this footprint captures the control-flow relations between activities \texttt{a}, \texttt{b}, \texttt{c} and \texttt{d}. These are the causal ($\rightarrow$) relation, concurrent ($\parallel$) relation, and other ($\#$) relations such as exclusivity or non-local dependency \cite{aalst2022pmhandbook}. In addition, on the right are six traces, of which five exhibit skipped, wrongly-ordered and unknown control-flow anomalies. For example, $\langle$\texttt{a b d}$\rangle$ has a skipped activity, which is \texttt{c}. Because of this skipped activity, the control-flow relation \texttt{b}$\,\#\,$\texttt{d} is violated, since \texttt{d} directly follows \texttt{b} in the anomalous trace.
\begin{figure}[!t]
\centering
\includegraphics[width=0.9\columnwidth]{images/FP_ANOMALIES.png}
\caption{An example event log footprint with six traces, of which five exhibit control-flow anomalies.}
\label{FP_ANOMALIES}
\end{figure}

\subsection{Control-flow anomaly detection}
Control-flow anomaly detection techniques aim to characterize the normal control flow from event logs and verify whether these deviations occur in new event logs \cite{ko2023adsystematicreview}. To develop control-flow anomaly detection techniques, \revision{process mining} has seen widespread adoption owing to process discovery and \revision{conformance checking}. On the one hand, process discovery is a set of algorithms that encode control-flow relations as a set of model elements and constraints according to a given modeling formalism \cite{aalst2022pmhandbook}; hereafter, we refer to the Petri net, a widespread modeling formalism. On the other hand, \revision{conformance checking} is an explainable set of algorithms that allows linking any deviations with the reference Petri net and providing the fitness measure, namely a measure of how much the Petri net fits the new event log \cite{aalst2022pmhandbook}. Many control-flow anomaly detection techniques based on \revision{conformance checking} (hereafter, \revision{conformance checking}-based techniques) use the fitness measure to determine whether an event log is anomalous \cite{bezerra2009pmad, bezerra2013adlogspais, myers2018icsadpm, pecchia2020applicationfailuresanalysispm}. 

The scientific literature also includes many \revision{conformance checking}-independent techniques for control-flow anomaly detection that combine specific types of trace encodings with machine/deep learning \cite{ko2023adsystematicreview, tavares2023pmtraceencoding}. Whereas these techniques are very effective, their explainability is challenging due to both the type of trace encoding employed and the machine/deep learning model used \cite{rawal2022trustworthyaiadvances,li2023explainablead}. Hence, in the following, we focus on the shortcomings of \revision{conformance checking}-based techniques to investigate whether it is possible to support the development of competitive control-flow anomaly detection techniques while maintaining the explainable nature of \revision{conformance checking}.
\begin{figure}[!t]
\centering
\includegraphics[width=\columnwidth]{images/HIGH_LEVEL_VIEW.png}
\caption{A high-level view of the proposed framework for combining \revision{process mining}-based feature extraction with dimensionality reduction for control-flow anomaly detection.}
\label{HIGH_LEVEL_VIEW}
\end{figure}

\subsection{Shortcomings of \revision{conformance checking}-based techniques}
Unfortunately, the detection effectiveness of \revision{conformance checking}-based techniques is affected by noisy data and low-quality Petri nets, which may be due to human errors in the modeling process or representational bias of process discovery algorithms \cite{bezerra2013adlogspais, pecchia2020applicationfailuresanalysispm, aalst2016pm}. Specifically, on the one hand, noisy data may introduce infrequent and deceptive control-flow relations that may result in inconsistent fitness measures, whereas, on the other hand, checking event logs against a low-quality Petri net could lead to an unreliable distribution of fitness measures. Nonetheless, such Petri nets can still be used as references to obtain insightful information for \revision{process mining}-based feature extraction, supporting the development of competitive and explainable \revision{conformance checking}-based techniques for control-flow anomaly detection despite the problems above. For example, a few works outline that token-based \revision{conformance checking} can be used for \revision{process mining}-based feature extraction to build tabular data and develop effective \revision{conformance checking}-based techniques for control-flow anomaly detection \cite{singh2022lapmsh, debenedictis2023dtadiiot}. However, to the best of our knowledge, the scientific literature lacks a structured proposal for \revision{process mining}-based feature extraction using the state-of-the-art \revision{conformance checking} variant, namely alignment-based \revision{conformance checking}.

\subsection{Contributions}
We propose a novel \revision{process mining}-based feature extraction approach with alignment-based \revision{conformance checking}. This variant aligns the deviating control flow with a reference Petri net; the resulting alignment can be inspected to extract additional statistics such as the number of times a given activity caused mismatches \cite{aalst2022pmhandbook}. We integrate this approach into a flexible and explainable framework for developing techniques for control-flow anomaly detection. The framework combines \revision{process mining}-based feature extraction and dimensionality reduction to handle high-dimensional feature sets, achieve detection effectiveness, and support explainability. Notably, in addition to our proposed \revision{process mining}-based feature extraction approach, the framework allows employing other approaches, enabling a fair comparison of multiple \revision{conformance checking}-based and \revision{conformance checking}-independent techniques for control-flow anomaly detection. Figure \ref{HIGH_LEVEL_VIEW} shows a high-level view of the framework. Business processes are monitored, and event logs obtained from the database of information systems. Subsequently, \revision{process mining}-based feature extraction is applied to these event logs and tabular data input to dimensionality reduction to identify control-flow anomalies. We apply several \revision{conformance checking}-based and \revision{conformance checking}-independent framework techniques to publicly available datasets, simulated data of a case study from railways, and real-world data of a case study from healthcare. We show that the framework techniques implementing our approach outperform the baseline \revision{conformance checking}-based techniques while maintaining the explainable nature of \revision{conformance checking}.

In summary, the contributions of this paper are as follows.
\begin{itemize}
    \item{
        A novel \revision{process mining}-based feature extraction approach to support the development of competitive and explainable \revision{conformance checking}-based techniques for control-flow anomaly detection.
    }
    \item{
        A flexible and explainable framework for developing techniques for control-flow anomaly detection using \revision{process mining}-based feature extraction and dimensionality reduction.
    }
    \item{
        Application to synthetic and real-world datasets of several \revision{conformance checking}-based and \revision{conformance checking}-independent framework techniques, evaluating their detection effectiveness and explainability.
    }
\end{itemize}

The rest of the paper is organized as follows.
\begin{itemize}
    \item Section \ref{sec:related_work} reviews the existing techniques for control-flow anomaly detection, categorizing them into \revision{conformance checking}-based and \revision{conformance checking}-independent techniques.
    \item Section \ref{sec:abccfe} provides the preliminaries of \revision{process mining} to establish the notation used throughout the paper, and delves into the details of the proposed \revision{process mining}-based feature extraction approach with alignment-based \revision{conformance checking}.
    \item Section \ref{sec:framework} describes the framework for developing \revision{conformance checking}-based and \revision{conformance checking}-independent techniques for control-flow anomaly detection that combine \revision{process mining}-based feature extraction and dimensionality reduction.
    \item Section \ref{sec:evaluation} presents the experiments conducted with multiple framework and baseline techniques using data from publicly available datasets and case studies.
    \item Section \ref{sec:conclusions} draws the conclusions and presents future work.
\end{itemize}

% \section{Preliminaries, key lemmas, and the framework}
% % !TEX root =  ../main.tex
\section{Background on causality and abstraction}\label{sec:preliminaries}

This section provides the notation and key concepts related to causal modeling and abstraction theory.

\spara{Notation.} The set of integers from $1$ to $n$ is $[n]$.
The vectors of zeros and ones of size $n$ are $\zeros_n$ and $\ones_n$.
The identity matrix of size $n \times n$ is $\identity_n$. The Frobenius norm is $\frob{\mathbf{A}}$.
The set of positive definite matrices over $\reall^{n\times n}$ is $\pd^n$. The Hadamard product is $\odot$.
Function composition is $\circ$.
The domain of a function is $\dom{\cdot}$ and its kernel $\ker$.
Let $\mathcal{M}(\mathcal{X}^n)$ be the set of Borel measures over $\mathcal{X}^n \subseteq \reall^n$. Given a measure $\mu^n \in \mathcal{M}(\mathcal{X}^n)$ and a measurable map $\varphi^{\V}$, $\mathcal{X}^n \ni \mathbf{x} \overset{\varphi^{\V}}{\longmapsto} \V^\top \mathbf{x} \in \mathcal{X}^m$, we denote by $\varphi^{\V}_{\#}(\mu^n) \coloneqq \mu^n(\varphi^{\V^{-1}}(\mathbf{x}))$ the pushforward measure $\mu^m \in \mathcal{M}(\mathcal{X}^m)$. 


We now present the standard definition of SCM.

\begin{definition}[SCM, \citealp{pearl2009causality}]\label{def:SCM}
A (Markovian) structural causal model (SCM) $\scm^n$ is a tuple $\langle \myendogenous, \myexogenous, \myfunctional, \zeta^\myexogenous \rangle$, where \emph{(i)} $\myendogenous = \{X_1, \ldots, X_n\}$ is a set of $n$ endogenous random variables; \emph{(ii)} $\myexogenous =\{Z_1,\ldots,Z_n\}$ is a set of $n$ exogenous variables; \emph{(iii)} $\myfunctional$ is a set of $n$ functional assignments such that $X_i=f_i(\parents_i, Z_i)$, $\forall \; i \in [n]$, with $ \parents_i \subseteq \myendogenous \setminus \{ X_i\}$; \emph{(iv)} $\zeta^\myexogenous$ is a product probability measure over independent exogenous variables $\zeta^\myexogenous=\prod_{i \in [n]} \zeta^i$, where $\zeta^i=P(Z_i)$. 
\end{definition}
A Markovian SCM induces a directed acyclic graph (DAG) $\mathcal{G}_{\scm^n}$ where the nodes represent the variables $\myendogenous$ and the edges are determined by the structural functions $\myfunctional$; $ \parents_i$ constitutes then the parent set for $X_i$. Furthermore, we can recursively rewrite the set of structural function $\myfunctional$ as a set of mixing functions $\mymixing$ dependent only on the exogenous variables (cf. \cref{app:CA}). A key feature for studying causality is the possibility of defining interventions on the model:
\begin{definition}[Hard intervention, \citealp{pearl2009causality}]\label{def:intervention}
Given SCM $\scm^n = \langle \myendogenous, \myexogenous, \myfunctional, \zeta^\myexogenous \rangle$, a (hard) intervention $\iota = \operatorname{do}(\myendogenous^{\iota} = \mathbf{x}^{\iota})$, $\myendogenous^{\iota}\subseteq \myendogenous$,
is an operator that generates a new post-intervention SCM $\scm^n_\iota = \langle \myendogenous, \myexogenous, \myfunctional_\iota, \zeta^\myexogenous \rangle$ by replacing each function $f_i$ for $X_i\in\myendogenous^{\iota}$ with the constant $x_i^\iota\in \mathbf{x}^\iota$. 
Graphically, an intervention mutilates $\mathcal{G}_{\mathsf{M}^n}$ by removing all the incoming edges of the variables in $\myendogenous^{\iota}$.
\end{definition}

Given multiple SCMs describing the same system at different levels of granularity, CA provides the definition of an $\alpha$-abstraction map to relate these SCMs:
\begin{definition}[$\abst$-abstraction, \citealp{rischel2020category}]\label{def:abstraction}
Given low-level $\mathsf{M}^\ell$ and high-level $\mathsf{M}^h$ SCMs, an $\abst$-abstraction is a triple $\abst = \langle \Rset, \amap, \alphamap{} \rangle$, where \emph{(i)} $\Rset \subseteq \datalow$ is a subset of relevant variables in $\mathsf{M}^\ell$; \emph{(ii)} $\amap: \Rset \rightarrow \datahigh$ is a surjective function between the relevant variables of $\mathsf{M}^\ell$ and the endogenous variables of $\mathsf{M}^h$; \emph{(iii)} $\alphamap{}: \dom{\Rset} \rightarrow \dom{\datahigh}$ is a modular function $\alphamap{} = \bigotimes_{i\in[n]} \alphamap{X^h_i}$ made up by surjective functions $\alphamap{X^h_i}: \dom{\amap^{-1}(X^h_i)} \rightarrow \dom{X^h_i}$ from the outcome of low-level variables $\amap^{-1}(X^h_i) \in \datalow$ onto outcomes of the high-level variables $X^h_i \in \datahigh$.
\end{definition}
Notice that an $\abst$-abstraction simultaneously maps variables via the function $\amap$ and values through the function $\alphamap{}$. The definition itself does not place any constraint on these functions, although a common requirement in the literature is for the abstraction to satisfy \emph{interventional consistency} \cite{rubenstein2017causal,rischel2020category,beckers2019abstracting}. An important class of such well-behaved abstractions is \emph{constructive linear abstraction}, for which the following properties hold. By constructivity, \emph{(i)} $\abst$ is interventionally consistent; \emph{(ii)} all low-level variables are relevant $\Rset=\datalow$; \emph{(iii)} in addition to the map $\alphamap{}$ between endogenous variables, there exists a map ${\alphamap{}}_U$ between exogenous variables satisfying interventional consistency \cite{beckers2019abstracting,schooltink2024aligning}. By linearity, $\alphamap{} = \V^\top \in \reall^{h \times \ell}$ \cite{massidda2024learningcausalabstractionslinear}. \cref{app:CA} provides formal definitions for interventional consistency, linear and constructive abstraction.
% \section*{Framework}
% \section{Generalization to Other Circuits}
We extend our analyses to the \textbf{Greater-Than} circuit \cite{hanna2024does}. We find a similar pattern. The mechanisms of the greater-than task are \textit{amplified} after fine-tuning on task data. In contrast, the changes to the mechanisms of the model under toxic fine-tuning are primarily localized to circuit components leading to corruption of the task. Furthermore, we discover our finding of neuroplasticity to hold for the greater-than task, i.e., the model reverts back to its original mechanism after retraining the corrupted model on clean task-specific data. We detail our experiments on this task in \autoref{app:gt}.

% \section{Warm-up: Regular case where endpoints are integral}
% % -Basic idea-
% Key observation that makes integral case easy
As a starter, we consider a special case where all intervals have integral endpoints. 
More formally, for any interval $\x \in \orderedInput$, $I = (i, i+1)$ for some integer $i$ if $I$ is open, and $I = [i, i+1]$ for some integer $i$ if $I$ is closed.
In this section, we show that any $\x \in \orderedInput$ has color $\ffc(\x)\leq 2\omega$, where $\omega$ is the number of colors used by the optimal solution.
Note that in the following, we focus on the closed intervals $\x \in \orderedInput$ with $\nrows{3}{\x} \geq \nrows{1}{\x}+2$ (by Lemma~\ref{lem:open_interval_small_intersection} and Lemma~\ref{lem:integral_open_and_closed_assumtion_not_hold}).

For such a set of integral-endpoints intervals, it is an important property that $\nlmr(\x) = \emptyset$ for any $\x \in \orderedInput$.
That is, the intervals in $\nb(\x)$ are either in $\twins(\x)$ (twins of $\x$) or in $\lmr(\x)$ (twins of the intervals in $\rows{3}{\x}$).
An example is shown in Figure~\ref{fig:column_structure_in_rows}.

\paragraph*{Pick a pivot $\pivot$}

For any closed intervals $\x \in \orderedInput$ with $\nrows{3}{\x} \geq \nrows{1}{\x}+2$, we first identify the interval in $\lmr(\x)$ with the highest color by \ff. 

\begin{definition}\emph{\textbf{(Dominating interval $\y$ in $\lmr(\x)$)}}
    \label{def:y}
    Let interval $\y \in \lmr(\x)$ be the highest colored interval. That is, there is no $I^\prime \in \lmr(\x)$ such that $\ffc(I^\prime) > \ffc(\y)$.
\end{definition}

Note that $\y$ must exist as $|\lmr(\x)| \geq \nrows{3}{\x} \geq \nrows{1}{\x} + 2 \geq 2$.

In this analysis, we want to bound $\ffc(\x)$ by using $\y$ as a pivot $\pivot$ and applying the Pivot bound (Lemma~\ref{lem:prop_ff_equiv}).


\paragraph*{Bound the size of $\pivotset$}
Given that we use $\y$ as the pivot $\pivot$, we bound the size of $\pivotset$.
Recall that $\pivotset$ is the set of intervals in $\nb(\x)$ which are assigned higher color than $\ffc(\pivot)$.
% How interval x can have a higher color
Since all intervals in $\orderedInput$ have integral endpoints, any interval in $\nb(\x)$ is either in $\lmr(\x)$ or in $\twins(\x)$.
Therefore, by the definition of $\y$, only intervals in $\twins(\x)$ are possible to contribute to $\pivotset$.
% Not too many intervals identical to x
Moreover, since we focus on $\x$ that is closed, all intervals in $\twins(\x)$ must be in the rows in $\rows{1}{\x}$. 
We use the following definition to represent the fraction of rows in $\rows{1}{\x}$ that cannot contribute to $\pivotset$.



\begin{definition}
    \label{def:alpha}
    Given any $\hat{I} \in \rows{1}{\x}$, $\alp(\hat{I}) \in [0,1]$ denotes the fraction of rows $\mathcal{R} \in \rows{1}{\x}$ where the interval $I^\prime$ in $\mathcal{R} \cap \nb(\x)$ satisfies \emph{1)} $I^\prime \notin \lmr(\x)$ and \emph{2)} $\ffc(I^\prime) \leq \ffc(\hat{I})$.
    %such that for all $I^\prime \in \rows{1}{\x}$,\footnote{Recall that we say that an interval $I^\prime \in \rows{i}{\x}$ if $I^\prime$ is overlapping with $\x$ and colored by one of the colors in $\nrows{i}{\x}$ by \opt.} $I^\prime \not \in \lmr(\x)$ and $\ffc(I^\prime) \leq \ffc(\hat{I})$.
\end{definition}

Note that we make Definition~\ref{def:alpha} more general for later usage. 
For the integral-endpoints case, it is sufficient to set $\hat{I}$ as $\y$, and $\alp(\y) \cdot \nrows{1}{\x}$ is the number of intervals in $\twins(\x)$ that are not in $\pivotset$.

\begin{lemma}
    \label{lem:integral_S}
    If all intervals in $\orderedInput$ have integral endpoints, by selecting $\y$ as a pivot~$\pivot$, $|\pivotset| = (1-\alp(\y)) \cdot \nrows{1}{\x}$.
\end{lemma}

\begin{proof}
    First, by the definition of $\rows{0}{\x}$, $\x$ does not intersect any interval in $\rows{0}{\x}$. Thus, no interval in $\rows{0}{\x}$ can be in $\pivotset$.
    Next, as all intervals have integral endpoints, all intervals in $\rows{2}{\x}$, $\rows{3}{\x}$, or $\rows{1}{\x} \setminus \twins(\x)$ are in the sets $\lmr(\x)$.
    Thus, all these intervals are assigned a color below $\ffc(\y)$ and cannot contribute to $\pivotset$.
    The remaining intervals are those in $\twins(\x)$.
    By definition, $\alp(\y) \cdot \nrows{1}{\x}$ of these intervals have a color at most $\ffc(\y)$.
    Hence, the $(1 - \alp(\y)) \cdot \nrows{1}{\x}$ intervals are the only intervals intersecting $\x$ that have a color greater than $\ffc(\y)$.
    That is, $|\pivotset| = (1 - \alp(\y)) \cdot \nrows{1}{\x}$.
\end{proof}


\paragraph*{Bound the color $\ffc(\pivot)$}

We bound the color of $\y$ using the Neighborhood bound (Lemma~\ref{lem:prop_ff}).
In general, apart from intervals in $\twins(\x)$, the number of intervals any interval in $\lmr(\x)$ can intersect is shown in the following observation (also see Figure~\ref{fig:intersections_with_v}):

\begin{observation}
    \label{obs:integral}
    Any interval in the set $\lmr(\x)$ intersects
    \begin{enumerate}[(a)]
        \item at most $2$ intervals per row in $\rows{0}{\x}$,
        \item at most $2$ intervals which are not twins with $\x$ per row in $\rows{1}{\x}$,
        \item at most $2$ intervals per row in $\rows{2}{\x}$, and
        \item exactly $1$ interval per row in $\rows{3}{\x}$.
    \end{enumerate}
\end{observation}

% Figures
\begin{figure}[t]
\begin{minipage}{0.4\textwidth}
\centering
\includegraphics[scale=0.4]{pictures/Brick_Rock_integral_rows.pdf}
\caption{
All possible positions of intervals intersecting $\x$ per type of row.
The intervals in $\lmr(\x)$ are drawn in orange, green and blue, the intervals in $\twins(\x)$ in purple.
}
\label{fig:column_structure_in_rows}
\end{minipage}%
\hspace{0.2\textwidth}%
\begin{minipage}{0.4\textwidth}
\centering
\includegraphics[scale=0.4]{pictures/Brick_Rock_integral_intersections.pdf}
\caption{
All possible intersections with interval $\y$ per type of row.
The intervals from $\lmr(\x)$ that intersect with $\y$ are drawn in orange, the intervals in $\twins(\x)$ in purple and the remaining intervals intersecting $\y$ are drawn in yellow.}
\label{fig:intersections_with_v}
\end{minipage}
\end{figure}

% -Start of analysis-
According to the definition of $\y$, Observation~\ref{obs:integral} applies to $\y$. 


\begin{lemma}
    \label{lem:integral_uvw}
    The color of interval $\y$, $\ffc(\y) \leq 2 \omega + \alp(\y) \cdot \nrows{1}{\x} - \nrows{3}{\x}$.
\end{lemma}


\begin{proof}
    By the Neighborhood bound, $\ffc(\y) \leq 1+|N(\y)|$.
    By Observation~\ref{obs:integral}, there are at most $2\nrows{0}{\x} + 2\nrows{1}{\x} + 2\nrows{2}{\x} + \nrows{3}{\x}$ intervals in $\lmr(\x)$ that are in $\nb(\y)$. 
    Together with the $\alp(\y) \cdot \nrows{1}{\x}$ intervals that ``occupied'' the smaller colors from the choice of $\y$ at the moment when $\y$ arrives, $|\nb_{\y}(\y)| \leq 2\nrows{0}{\x} + 2\nrows{1}{\x} + \alp(\y) \cdot \nrows{1}{\x} + 2\nrows{2}{\x} + \nrows{3}{\x}$.
    Observe that by this bound, we count the row that contains interval $\y$ itself. 
    Since $\y$ cannot intersect any interval on this row, we should subtract at least $1$ from this bound.
    Thus,
    \begin{align*}
        \ffc(\y) &\leq 1 + 2\nrows{0}{\x} + 2\nrows{1}{\x} + \alp(\y)\cdot \nrows{1}{\x} + 2\nrows{2}{\x} + \nrows{3}{\x} - 1\\
        &= 2 \omega + \alp(\y)\cdot \nrows{1}{\x} - \nrows{3}{\x}
    \end{align*}
\end{proof}

Note that due to the abstraction of $\alp(\y)$, this proof does not rely on the property of integral endpoints.
Therefore, Observation~\ref{obs:integral} also holds for the general case where the intervals in $\orderedInput$ have arbitrary endpoints.

\begin{corollary}
    \label{cor:integral_LMR}
    Given instance $\orderedInput$ with arbitrary endpoints, the color of interval $\y$ is at most $2\omega + \alp(\y) \cdot \nrows{1}{\x} - \nrows{3}{\x}$.
\end{corollary}

\paragraph*{Proof of Theorem~\ref{thm:integral}}
Now, we are ready to prove Theorem~\ref{thm:integral} that for any $\orderedInput$ with open or closed unit-length intervals that have integral endpoints, $\ffc(\x) \leq 2\omega$ for all $\x \in \orderedInput$, where $\omega$ is the optimal color needed for properly coloring $\orderedInput$.

\begin{proof}
    By Lemma~\ref{lem:integral_open_and_closed_assumtion_not_hold}, we assume that $\nrows{1}{\x} \leq \nrows{3}{\x}-2$. 
    Moreover, by Lemma~\ref{lem:open_interval_small_intersection}, we focus on closed $\x$.
    According to Corollary~\ref{cor:y_exists}, we assume that there exists a dominating interval $\y$ in $\lmr(\x)$ that has the highest color by \ff.
    By Lemma~\ref{lem:integral_uvw}, $\ffc(\y) \leq 2 \omega + \alp(\y) \cdot \nrows{1}{\x} - \nrows{3}{\x}$.

    We now use the Pivot bound to bound the color of interval $\x$ from above, using $\y$ as the role of $\pivot$.
    By Lemma~\ref{lem:integral_S}, $|\pivotset| = (1 - \alp(\y)) \cdot \nrows{1}{\x}$.
    \shortversion{Altogether, we have $\ffc(\x) \leq \ffc(\pivot) + |\pivotset| + 1 \leq 2 \omega - 1$}
    \fullversion{
    It follows that
    \begin{align*}
        \ffc(\x) &\leq \ffc(\pivot) + |\pivotset| + 1 \\
        &= \ffc(\y) + (1 - \alp(\y)) \cdot \nrows{1}{\x} + 1\\
        &\leq 2\omega + \alp(\y)\cdot \nrows{1}{\x} - \nrows{3}{\x} + (1 - \alp(\y)) \cdot \nrows{1}{\x} + 1\\
        &\leq 2\omega + \nrows{1}{\x} - \nrows{3}{\x} + 1 \\
        &\leq 2 \omega - 1
    \end{align*}
    }
\end{proof}

% \section{General case: Intervals with arbitrary endpoints}
% 
In this section, we consider the instance $\orderedInput$, where intervals have arbitrary endpoints. 
More specifically, by Lemma~\ref{lem:integral_open_and_closed_assumtion_not_hold} and Lemma~\ref{lem:open_interval_small_intersection}, we focus on $\x \in \orderedInput$ that are closed and $\nrows{3}{\x} \geq \nrows{1}{\x} + 2$.
Recall from Section~\ref{section:integral_endpoints} that we partition $\nb(\x)$ into $\twins(\x)$, $\lmr(\x)$, and $\nlmr(\x)$, that is, the set of twins, the set of intervals that are twins of intervals in $\rows{3}{\x}$, and the rest of intervals with endpoints ``misaligned'' with the endpoints of $\x$.

\shortversion{Primarily, we pick a pivot interval $\pivot$ and bound its color using the Neighborhood bound. 
The choice of pivot should provide a plausible way to bound the size of the corresponding~$\pivotset$.
However, applying the Pivot bound to bound the color $\ffc(\x)$ in this general case is more challenging than the integral-endpoints case due to the misaligned structure of intervals.
Directly applying the same technique as in the intergral-endpoints case results in a ratio of~$3\omega$ (see the full version).}
\fullversion{
Applying the Pivot bound to bound the color $\ffc(\x)$ in this general case is more challenging than the integral-endpoints case. 
First, unlike in the integral-endpoints case where all intervals that intersect $\x$ must be either in $\twins(\x)$ or in $\lmr(\x)$, the candidates in $\pivotset$ can also be in $\nlmr(\x)$. 
These intervals in $\nlmr(\x)$ can cross-interact with other intervals, making it difficult to bound the size of $\pivotset$.
Second, selecting a good pivot interval $\pivot$ in the general case is more challenging. }
\fullversion{We first show that using the same strategy as we used for the integral-endpoints case only guarantees a bound of $\ffc(\x) \leq 3\omega$.

% All intervals in R_1(x) and R_2(x) can be in set S
\paragraph*{A naive attempt}

As in the integral-endpoints case, let $\pivot$ be $\y$, the interval with the highest color in~$\lmr(\x)$. 
In the general case, any interval in the sets $\rows{1}{\x}$ and $\rows{2}{\x}$ can be in $\nlmr(\x)$ and can therefore be a candidate of $\pivotset$.
If we now apply the same strategy as in the integral-endpoints setting, it is no longer the case that the set $\pivotset$ is a subset of $\rows{1}{\x}$.
Then, in each row $\mathcal{R} \in \rows{2}{\x}$, it can be the case that two intervals (of which one intersects $\x$ and the other does not) are in the neighborhood of the pivot $\pivot$.
Furthermore, the remaining interval on row $\mathcal{R}$ that also intersects with $\x$ can be considered for the set $\pivotset$, resulting in us counting $3$ intervals per row in $\rows{2}{\x}$.
Thus, the bound tends to $3\omega$ when $\nrows{1}{\x}$ and $\nrows{3}{\x}$ are both small.
To work around this, we need to closely look at the color of the intervals in $\rows{2}{\x}$. 
}

% Define interval z
\fullversion{
\subsection{Roadmap}
Primarily, we pick a pivot interval $\pivot$ and bound its color using the Neighborhood bound. 
The choice of pivot should provide a plausible way to bound the size of the corresponding $\pivotset$. }

In the general case, we have another possible candidate of $\pivot$.
Symmetric to the dominating aligned interval $\y$, we define interval $\z$ as the dominating interval in rows in $\rows{2}{\x}$ that is misaligned with $\x$ and assigned the highest color by $\ff$.
Formally,

\begin{definition}
    \shortversion{\label{def:z_short}}
    \fullversion{\label{def:z}}
    \emph{\textbf{(Dominating interval $\z$ in $\rows{2}{\x}\cap\nlmr(\x)$)}}
    Let interval $\z$ be the interval in $\rows{2}{\x}\cap\nlmr(\x)$ with the highest color by $\ff$. That is, for all $I^\prime \in \rows{2}{\x}$, $\ffc(I^\prime) \leq \ffc(\z)$.
\end{definition}

A special property of the selection of $\z$ is that since $\z \in \nlmr(\x)$, any row in $\rows{0}{\x}, \rows{1}{\x}$ or $\rows{3}{\x}$, cannot be in $\rows{3}{\z}$\fullversion{ (Lemma~\ref{lem:R3z_in_R2x_short})}.
Therefore, all rows in $\rows{3}{\z}$ must also be in $\rows{2}{\x}$.

\shortversion{
\begin{lemma}
    \label{lem:R3z_in_R2x_short}
    $\rows{3}{\z} \subseteq \rows{2}{\x}$.
    \end{lemma}
}

% Case distinction
In the general case, we pick one of $\y$ and $\z$ as the pivot and use the Pivot bound to bound the color $\ffc(\x)$.
More specifically, we consider the following two cases.

% CASE 1: c(\y) >= c(\z)

\shortversion{
\paragraph*{Case (1): $\ffc(\z) \leq \ffc(\y)$}
}
\fullversion{
\subparagraph*{(1)}
$\ffc(\z) \leq \ffc(\y)$.}

In this case, the misaligned intervals in $\rows{2}{\x}$ are not in $\pivotset$ since they all have colors of at most $\ffc(\z) \leq \ffc(\y)$.
Using $\y$ as the pivot and following a strategy\shortversion{: }
\fullversion{similar to the case of integral-endpoints, we can show that $\ffc(\x) \leq 2\omega$ (Lemma~\ref{lem:c_u>=c_z}).}

\shortversion{

\smallskip

\begin{lemma}
    \label{lem:c_u>=c_z_short}
    For any interval $\x \in \orderedInput$, if $\ffc(\y) \geq \ffc(\z)$, then $\ffc(\x) \leq 2\omega$.
\end{lemma}
}


% CASE 2: c(\y) < c(\y)
\shortversion{\paragraph*{Case (2): $\ffc(\z) > \ffc(\y)$}}
\fullversion{\subparagraph*{(2)}
% z in S
$\ffc(\z) > \ffc(\y)$.}

% What is interesting about z
In this case, applying the Pivot bound with $\pivot = \y$ leads to a bound of $\ffc(\x) \leq 3 \omega$ as shown in the naive attempt.
To be precise, it is no longer true that $\pivotset \subseteq \rows{1}{\x}$, and any interval in $\rows{2}{\x}$ with a color greater than $\ffc(\y)$ is also a candidate of $\pivotset$.

% Why we use Z
An alternative strategy is to pick the interval $\z$ as the pivot in this case.
However, in the general case, the intervals in $\rows{1}{\z}$ are not particularly structured,
More precisely, we do not know which intervals also intersect with $\x$, and the neighborhood of $\z$ can be huge. 
When~$\z$ has a large number of intersections, using the Neighborhood bound to bound the color $\ffc(\z)$ can lead to a high bound and further lead to a high bound of $\ffc(\x)$ by the Pivot bound.
To deal with this situation, we first identify the rows containing intervals in $\rows{1}{\z}$ that we know their locations, namely, the intervals that are twins with $\z$.

\smallskip

\begin{definition}
    \shortversion{\label{def:Z_short}}
    \fullversion{\label{def:Z}}
    Let the set of rows in $\rows{2}{\x}$ with intervals identical to interval $\z$ be 
    \[ \Z (\x ) = \{ \text{row } \mathcal{R} \in \rows{2}{\x} \mid \mathcal{R} \text{ contains at least one interval that is in } \twins(\z)  \}.\]
\end{definition}

\shortversion{
    In combination with the set $\Z(\x)$ we are interested in the set $\rows{3}{\z}$.       
    Similar to the definition of $\alp$, we define fractions of specific $\rows{2}{\x}$ rows, that it, $\gam$ and $\delt$:
    
    \begin{definition}
    \label{def:gamma_delta_short}
    Given $\z \in \rows{2}{\x}$, 
        \begin{enumerate}[(a)]
            \item $\bet \in [0,1]$ denotes the fraction of rows $\mathcal{R} \in \rows{2}{\x}$ where the interval $I^\prime$ in $\mathcal{R} \cap \nb(\x)$ is in $\Z(\x)$, and
            \item $\gam \in [0,1]$ denotes the fraction of rows $\mathcal{R} \in \rows{2}{\x}$ where the interval $I^\prime$ in $\mathcal{R} \cap \nb(\x)$ is in $\rows{3}{\z}$.
        \end{enumerate}
    \end{definition}
}

% CASE 2.a: |\Z(x)| >= |Z bar(x)|

Note that $\Z(\x) \subseteq \rows{1}{\z}$, and thus $\nrows{3}{\z} \leq |\Z(\x)|$ implies that $\nrows{3}{\z} \leq \nrows{1}{\z}$. On the contrary, $\z$ potentially has a large neighborhood if $\nrows{3}{\z} > |\Z(\x)|$.
According to the size of $|\Z(\x)|$, we consider the following cases \textbf{2.a} and \textbf{2.b}.

\shortversion{\paragraph*{Case (2.a): $\nrows{3}{\z} \leq |\Z(\x)|$}}
\fullversion{\subparagraph*{(2.a)}
$\nrows{3}{\z} \leq |\Z(\x)|$.}

In this case, we pick $\z$ as the pivot $\pivot$.
By the selection of $\z$ and $\ffc(\z) > \ffc(\y)$, the only intervals that can contribute to $\pivotset$ are those in the set $\rows{1}{\x}$.
Furthermore, by $|\Z(\x)| \geq \nrows{3}{\z}$, the neighborhood of $\z$ is small, which enables us to use the Neighborhood bound to prove an upper bound the color of interval $\z$\shortversion{:} \fullversion{(Lemma~\ref{lem:any_z}).}
\shortversion{

\smallskip

\begin{lemma}
    \label{lem:any_z_short}
    The color of interval $\z$ is at most $c(\z) \leq \omega + \alp(\z) \cdot \nrows{1}{\x} - \bet  \cdot \nrows{2}{\x} + \gam  \cdot \nrows{2}{\x}  + \nrows{2}{\x} + \nrows{3}{\x}$. 
\end{lemma}

By selecting $\z$ as the pivot $\pivot$, the following bound of $|\pivotset|$ can be shown by the Neighborhood bound together with bounding the intersection of intervals in $\nlmr(\x)$.
\begin{lemma}
    \label{lem:pivotset_z_short}
    When $\ffc(\z) > \ffc(\y)$, by selecting $\z$ as pivot $\pivot$, $ |\pivotset| = (1 - \alp(\z)) \cdot \nrows{1}{\x}$.
\end{lemma}
}
Then, by the Pivot bound\shortversion{:} \fullversion{($\ffc(\x)$ is bounded by $2\omega$ Lemma~\ref{lem:gamma<delta}).}
\shortversion{
\begin{lemma}
    \shortversion{\label{lem:gamma<delta_short}}
    If $|\Z(\x)| \geq \nrows{3}{\z}$, then $\ffc(\x) \leq 2\omega$.
\end{lemma}
}


% CASE 2.b: |\Z(x)| < |Z bar(x)|
\shortversion{
\paragraph*{Case (2.b): $\nrows{3}{\z} > |\Z(\x)|$}
}
\fullversion{\subparagraph*{(2.b)}
$\nrows{3}{\z} > |\Z(\x)|$.}


This is the most technical part of our analysis.
In this case, $\z$ may intersect more than~$2$ intervals per row on average and have a large neighborhood, and using the Neighborhood bound on $\ffc(\z)$ can lead to a large bound.
We apply our framework recursively on $\z$ by first showing that the intervals in $\rows{3}{\z}$ rows do not intersect many intervals when $\nrows{3}{\z}$ is large\fullversion{ (Lemma~\ref{lem:any_z_bar})}.
Specifically, we let $\zbar$ be the highest colored interval in $\rows{3}{\z}$ that intersects with interval $x$.

\smallskip

\begin{definition}
    \shortversion{\label{def:z_bar_short}}
    \fullversion{\label{def:z_bar}}
    Let interval $\zbar \in \rows{3}{\z}$ be the dominating interval that has the highest color such that there exists no $I^\prime \in \rows{3}{\z}$ that intersects $\x$ such that $\ffc(I^\prime) > \ffc(\zbar)$.\footnote{Recall that we say an interval $I^\prime$ is in $\rows{i}{\x}$ if it is in $\nb(\x)$ and is colored by one of the colors in $\rows{i}{\x}$ by the optimal solution.}
\end{definition}

It is critical to use $\y$, $\z$, or $\zbar$ as the pivot according to whether $\ffc(\zbar)$ is larger than $\ffc(\y)$.

% CASE 2.b.i: c(z bar) >= c(\y)
\shortversion{
\paragraph*{Case (2.b.ii): $\ffc(\zbar) \geq \ffc(\y)$}
}
\fullversion{
\subparagraph*{(2.b.i)}
$\ffc(\zbar) \geq \ffc(\y)$.}

In this case, we apply the Pivot bound by selecting $\zbar$ as the pivot $\pivot$.
Recall that~$\y$ is the dominating interval in $\rows{3}{\x}$.
\shortversion{
Observe that by the definition of interval $\z$, $\ffc(\zbar) < \ffc(\z)$.
Hence, analogous to parameter $\alp$, we define $\delt$.

\smallskip

\begin{definition}
    \label{def:delta_short}
    Given any interval $\hat{I}$, $\delt(\hat{I}) \in [0, 1]$ denotes the fraction of rows $\mathcal{R} \in \rows{2}{\x} \setminus \rows{3}{\z}$ where interval $I^\prime$ in $\mathcal{R} \cap \nb(\x) \cap \nb(\z)$ satisfies $\ffc(I^\prime) \leq \ffc(\hat{I})$.
\end{definition}
}
The condition $\ffc(\zbar) \geq \ffc(\y)$ and the selection of $\zbar$ ensure that no interval in $\rows{3}{\x}$ can be in $\pivotset$.
Furthermore, by the definition of $\zbar$, no interval in $\rows{3}{\z}$ can contribute to $\pivotset$.
Therefore, $\pivotset$ is a subset of intervals in $\rows{1}{\x}$ and $\rows{2}{\x}\setminus \rows{3}{\z}$.

\shortversion{
\begin{lemma}
    \label{lem:pivotset_zbar_short}
    By selecting $\zbar$ as pivot $\pivot$, $|\pivotset| = (1-\alp(\zbar)) \cdot \nrows{1}{\x} + (1 - \delt(\zbar) \cdot (1 - \gam) \cdot 2\nrows{2}{\x}$.
\end{lemma}
}

The definition of $\rows{3}{\z}$ and the selection of $\zbar$ guarantee that a row in the set $\rows{3}{\zbar}$ must contain an interval identical to the interval $\z$ \fullversion{(see Figure~\ref{fig:y_intersections})}.
Thus, $\rows{3}{\zbar} \subseteq \Z(\x)$, and $\nrows{3}{\zbar} \leq |\Z(\x)|$.
Moreover, since $\twins(\zbar) \subseteq \rows{1}{\zbar}$, and every $\rows{3}{\z}$ contains a twin of $\zbar$, $\nrows{1}{\zbar} \geq \nrows{3}{\z}$. 
Together with the condition $\nrows{3}{\z} > |\Z(\x)|$, it guarantees that 
$\nrows{1}{\zbar} \geq \nrows{3}{\z} > |\Z(\x)| \geq \nrows{3}{\zbar}$.
Therefore, $\zbar$ intersects at most $2$ intervals per row on average, and $\ffc(\zbar)$ can be bounded by the Neighborhood bound effectively.
\shortversion{
\begin{lemma}
    \label{lem:any_z_bar_short}
    The color of interval $\zbar$, $\ffc(\zbar) \leq \nrows{0}{\x} + (1 + \alp(\zbar)) \cdot \nrows{1}{\x} + \delt(\zbar) \cdot (1 - \gam ) \cdot 2\nrows{2}{\x} + \bet  \cdot \nrows{2}{\x} + \gam  \cdot \nrows{2}{\x} + 2\nrows{3}{\x}$.
\end{lemma}
}
\fullversion{With the bound $\pivotset \subseteq \rows{1}{\x} \cup (\rows{2}{\x} \setminus \rows{3}{\z})$, the upper bound $\ffc(\x) \leq 2\omega$ (Lemma~\ref{lem:cy>=cu}).}
\shortversion{
Then, by the Pivot bound:
\begin{lemma}
    \label{lem:cy>=cu_short}
    If $\ffc(\zbar) \geq \ffc(\y)$, then $\ffc(\x) \leq 2\omega$   
\end{lemma}
}

% CASE 2.b.ii: c(z bar) < c(\y)
\shortversion{
\paragraph*{Case (2.b.ii): $\ffc(\zbar) < \ffc(\y)$}
}
\fullversion{
\subparagraph*{(2.b.ii)}
$\ffc(\zbar) < \ffc(\y)$.}

% Why we don't use z bar

This case is the tough kernel of the analysis.
In this case, considering using $\y$, $\z$, or $\zbar$ as the pivot only is not sufficient to have an upper bound of $\ffc(\x)$ that is smaller than $3\omega$.
% Why not \y
Recall that this case is a subcase of case $\ffc(\y) < \ffc(\z)$, picking $\y$ as the pivot does not stop any interval in $\rows{2}{\x}$ from contributing to $\pivotset$. 
% Why not \z
On the other hand, as $\nrows{3}{\z} > |\Z(\x)|$, $\nrows{3}{\z}$ may be much larger than $\nrows{1}{\z}$.
Applying the Neighborhood bound yields a huge upper bound of $\ffc(\pivot)$.
% Why not \zbar
Finally, condition $\ffc(\zbar) < \ffc(\y)$ implies that when $\zbar$ is selected as the pivot, $\pivotset$ could contain any interval in $\rows{3}{\x}$ and become too large.


% Why we can use z and y
To deal with this tough case, we express the upper bounds obtained by using $\y$ as the pivot and using $\z$ as the pivot in terms of $|\Z(\x)|$ and $\nrows{2}{\z}$\fullversion{ (Lemmas~\ref{lem:any_z} and~\ref{lem:any_uvw})}. 
\shortversion{
For the color of interval $\z$, Lemma~\ref{lem:any_z_short} still suffices.
However, for interval $\y$ we take a closer look at the neighborhood of interval $\y$ given the current assumptions.
In order to be more precise, let us first make additional observations regarding the intersections of interval $\y$ with intervals in $\rows{2}{\x}$ rows (also see Figure~\ref{fig:y_intersections_short}).

\begin{figure}[t]
\centering
\begin{minipage}{0.5\textwidth}
\centering
\includegraphics[width=\textwidth]{pictures/Brick_Rock_y.pdf}
\end{minipage}%
\caption{
An exemplary non-exhaustive list of possible rows per type of $\rows{2}{\x}$.
Yellow intervals intersect only interval $\y$, purple intervals intersect only $\x$ and orange intervals intersect both $\y$ and $\x$.}
\label{fig:y_intersections_short}
\end{figure}

\medskip

\begin{observation}
    \label{obs:any_y_short}
    Interval $\y$ intersects
    \begin{enumerate}[(a)]
        \item at most $1$ interval that is not in $\nb(\x)$ per row contained in $\rows{2}{\x}$, and
        \item at most $1$ interval that is in $\nb(\x)$ per row contained in $\rows{3}{\z}$,
        \item at most $1$ intervals that is in $\nb(\x)$ per row contained in $\Z(\x)$, and
        \item at most $1$ intervals that is in $\nb(\x)$ per row contained in $\rows{2}{\x} \setminus (\Z(\x) \cup \rows{3}{\z} )$.
    \end{enumerate}
\end{observation}

\begin{lemma}
    \label{lem:any_uvw_short}
    The color of interval $\y$,
    $\ffc(\y) \leq 2\nrows{0}{\x} + 2\nrows{1}{\x} + \alp(\y) \cdot \nrows{1}{\x} + \gam  \cdot \nrows{2}{\x} + \nrows{2}{\x} + \delt(\y) \cdot (1 - \gam ) \cdot \nrows{2}{\x} + \nrows{3}{\x}$. 
\end{lemma}
\begin{proof}
    By the Neighborhood bound, $\ffc(\y) \leq 1 + |\mathcal{N}(\y)|$.
    By Observation~\ref{obs:integral}, there are at most $ 2\nrows{0}{\x} + 2\nrows{1}{\x} + \nrows{3}{\x}$ intervals in $\rows{0}{\x}, \rows{1}{\x}$ and $\rows{3}{\x}$ that are in $\nb(\y)$.
    Together with the at most $\alp(\y) \cdot \nrows{1}{\x}$ intervals that were assigned a color below $\ffc(\y)$ which are either in $\twins(\x)$ or in $\nlmr(\x)$.
    
    By Observation~\ref{obs:any_y_short}, there are at most $\nrows{2}{\x} + \gam \cdot \nrows{2}{\x} + \bet \cdot \nrows{2}{\x} + (1 - \gam - \bet) \cdot \nrows{2}{\x}$ intervals that are in $\rows{2}{\x}$ and in $\nb(\y)$.
    By our assumption that $\ffc(\zbar) < \ffc(\y)$, the $\gam  \cdot \nrows{2}{\x}$ intervals in the set $\rows{3}{\z} \cap \nb(\x)$, must have a color smaller than $\ffc(\y)$.
    For the remaining $(1 - \gam) \cdot \nrows{2}{\x}$ intervals that are in $\nb(\x)$, by the definition of $\delt$,  $(1 - \delt(\y)) \cdot (1 - \gam) \cdot \nrows{2}{\x}$ are assigned a color larger than $\ffc(\z)$ and hence we do not consider them for this bound.
    
    Observe that by this bound we count the row that contains interval $\y$ itself.
    Since interval $\y$ cannot intersect any interval on this row, we should subtract at least $1$ from this bound. 
    Thus,
    \begin{align*}
            \ffc(\y) &\leq |\nb(\y)| + 1\\
            &\leq 2\nrows{0}{\x} + 2\nrows{1}{\x} + \alp(\y) \cdot \nrows{1}{\x} + \gam  \cdot \nrows{2}{\x} \\&\hspace{1cm} + \nrows{2}{\x} + \delt(\y) \cdot (1 - \gam ) \cdot \nrows{2}{\x} + \nrows{3}{\x}
        \end{align*}    
\end{proof}

When we pick interval $\z$ as pivot, the size of $|\pivotset|$ proven in Lemma~\ref{lem:pivotset_z_short} remains true.
For the case where we pick interval $\y$, we show the following bound.

\begin{lemma}
    \label{lem:pivotset_y_exact_short}
    When $\ffc(\y) \geq \ffc(\zbar)$, by selecting $\y$ as pivot $\pivot$, $|\pivotset| \leq (1 - \alp(\y)) \cdot \nrows{1}{\x} + (1 - \delt(\z)) \cdot (1 - \gam ) \cdot \nrows{2}{\x} + (1 - \gam ) \cdot \nrows{2}{\x}$
\end{lemma}
\begin{proof}
    First, by the definition of $\rows{0}{\x}$, $\x$ does not intersect any interval in $\rows{0}{\x}$.
    Thus, no interval in $\rows{0}{\x}$ can be in $\pivotset$.
    By the definition of $\alp$, only $(1 - \alp(\y) ) \cdot \nrows{1}{\x}$ intervals in $\rows{1}{\x}$ are assigned a color strictly larger than $\ffc(\y)$.
    Thus the only intervals in an $\rows{1}{\x}$ that are eligible for $\pivotset$ are those $(1 - \alp(\y) ) \cdot \nrows{1}{\x}$ intervals.
    Since $\ffc(\y) > \ffc(\zbar)$, no interval in an $\rows{3}{\z}$ row is assigned a color larger than $\ffc(\y)$.
    Thus, none of the $\gam \cdot 2\nrows{2}{\x}$ intervals in $\rows{3}{\z} \cap \nb(\x)$ can contribute to $\pivotset$.
    By the definition of $\delt$, only $(1 - \delt(\y) ) \cdot (1 - \gam) \cdot \nrows{2}{\x}$ intervals both in $\nb(\y)$ and $\rows{2}{\x}$ are assigned a color strictly larger than $\ffc(\y)$.
    Furthermore, all $(1 - \gam) \cdot \nrows{2}{\x}$ intervals which are in $\rows{2}{\x}$ but not in $\nb(\y)$ and not in $\rows{3}{\z}$ could potentially be assigned a color larger than $\ffc(\y)$ and therefore contribute to $\pivotset$.
    Finally, as all intervals in $\rows{3}{\x}$ are in $\lmr(\x)$, all intervals in $\rows{3}{\x}$ have a color below $\ffc(\y)$ and cannot contribute to $\pivotset$.
    Thus, $|\pivotset| \leq (1 - \alp(\y)) \cdot \nrows{1}{\x} + (1 - \delt(\z)) \cdot (1 - \gam ) \cdot \nrows{2}{\x} + (1 - \gam ) \cdot \nrows{2}{\x}$.
\end{proof}`}
These formulations show that the upper bounds from applying the Pivot bound using interval $\y$ as $\pivot$ and using interval $\z$ as $\pivot$ form a trade-off.
More specifically, the upper bound using $\y$ is maximized when $|\Z(\x)| = \emptyset$, while upper bound using $\z$ is maximized when $|\Z(\x)| = \rows{2}{\x}$.
Then, we use that in any circumstance, the upper bound is at most $2\omega + \frac{|\Z(\x)|}{2}$, which is strictly less than $2\omega + \frac{\nrows{2}{\x}}{2}\cdot\omega$ 
(Definition~\ref{def:Z}). 
\fullversion{Since $\nrows{2}{\x}$ can be upper bounded by $\frac{2}{3}\omega - 1$ (Corollary~\ref{cor:r2_bound}), the color assigned to interval $\x$ is bounded by $\result$ (Theorem~\ref{thm:any}).}
\shortversion{We next prove a bound on the number of $\rows{2}{\x}$ rows.
\begin{lemma}
    \label{lem:r2_bound_short}
    For any interval $\x \in \orderedInput$, if $\nrows{2}{\x} \geq \frac{2}{3} \omega$, then $\ffc(\x) \leq \frac{7}{3} \omega - 2$. 
\end{lemma}

\begin{proof}
    We show this using the Neighborhood bound.
    Observe that $\nrows{0}{\x} \geq 1$, since interval $\x$ does not intersect any intervals on its own row.
    Furthermore, note that the Neighborhood bound is maximized when we maximize the number of intervals that intersect interval $\x$.
    That is, we maximize the bound when we maximize $\nrows{3}{\x}$.
    Then, it follows by $\nrows{0}{\x} \geq 1$ and $\nrows{2}{\x} \geq \frac{2}{3}\omega$, that $\nrows{3}{\x} = \frac{1}{3}\omega - 1$.
    For the remaining rows, we maximize the bound by maximizing $\nrows{2}{\x}$, and hence $\nrows{2}{\x} = \frac{2}{3}\omega$.
    Then, $\nrows{1}{\x} = 0$, and we get the following upper bound on the color of interval $\x$,
    \begin{align*}
        \ffc(\x) &\leq \nrows{1}{\x} + 2\nrows{2}{\x} + 3\nrows{3}{\x} + 1 \\
            &\leq 0 + 2 \cdot \frac{2}{3} \omega + 3 \cdot (\frac{1}{3}\omega - 1) + 1\\
            &= \frac{7}{3} \omega - 2
    \end{align*}
\end{proof}

Then, we can bound the color assigned to interval $\x$ by $\result$.
\paragraph*{Proof of Theorem~\ref{thm:any}.}
\emph{For any interval $\x \in \orderedInput$, the color of $\x$ is at most $\ffc(\x) \leq \result$.}

\begin{proof}
    By Lemma~\ref{lem:open_interval_small_intersection},~\ref{lem:integral_open_and_closed_assumtion_not_hold}, \ref{lem:c_u>=c_z_short}, \ref{lem:gamma<delta_short} and \ref{lem:cy>=cu_short}, $\ffc(\x) \leq 2\omega$ if $\x$ is open, $\nrows{1}{\x} \leq \nrows{3}{\x} - 2$, $\ffc(\y) < \ffc(\z)$, $\bet  < \gam $ or $\ffc(\zbar) < \ffc(\y)$.
    Thus in the following we focus on closed intervals with $\nrows{1}{\x} \leq \nrows{3}{\x} - 2$, $\ffc(\zbar) < \ffc(\y) < \ffc(\z)$ and $\bet  < \gam $.

    Now we can compute the first bound on the color of interval $\x$, using the Pivot bound with interval $\y$ as pivot $\pivot$.
    According to Lemma~\ref{lem:any_uvw_short}, $\ffc(\y) \leq  2\nrows{0}{\x} + 2\nrows{1}{\x} + \alp(\y) \cdot \nrows{1}{\x} + \gam  \cdot \nrows{2}{\x} + \nrows{2}{\x} + \delt(\z) \cdot (1 - \gam ) \cdot \nrows{2}{\x} + \nrows{3}{\x}$ and, by taking $\y$ as pivot $\pivot$,  Lemma~\ref{lem:pivotset_y_exact_short}, $|\pivotset| \leq (1 - \alp(\y)) \cdot \nrows{1}{\x} + (1 - \delt(\z)) \cdot (1 - \gam ) \cdot \nrows{2}{\x} + (1 - \gam ) \cdot \nrows{2}{\x}$.
    Thus,

    \begin{align*}
        \ffc(\x) &\leq \ffc(\pivot) + |\pivotset| + 1\\
        &\leq \ffc(\y) + (1 - \alp(\y)) \cdot \nrows{1}{\x} + (1 - \delt(\z)) \cdot (1 - \gam ) \cdot \nrows{2}{\x} \\ &\hspace{1cm} + (1 - \gam ) \cdot \nrows{2}{\x} + 1 \\
        &\leq 2\nrows{0}{\x} + 3\nrows{1}{\x} + 2\nrows{2}{\x} + (1 - \gam ) \cdot \nrows{2}{\x} + \nrows{3}{\x} + 1 \\
        &\leq 2\nrows{0}{\x} + 2\nrows{1}{\x} + (1 - \gam ) \cdot \nrows{2}{\x} + 2\nrows{2}{\x} + 2\nrows{3}{\x} - 1\\
        &= 2\omega + (1 - \gam ) \cdot \nrows{2}{\x} - 1
    \end{align*}

    Next we move to the other bound.
    By taking $\z$ as pivot $\pivot$, according to Lemma~\ref{lem:any_z_short}, $\ffc(\z) \leq \omega + \alp(\z) \cdot \nrows{1}{\x} - \bet  \cdot \nrows{2}{\x} + \gam  \cdot \nrows{2}{\x} + \nrows{2}{\x} + \nrows{3}{\x}$, and Lemma~\ref{lem:pivotset_z_short}, $|\pivotset| = (1 - \alp(\z))\cdot\nrows{1}{\x}$.
    Then, it follows that,

    \begin{align*}
        \ffc(\x) &\leq \ffc(\pivot) + |\pivotset| + 1\\
        &\leq \ffc(\z) + (1 - \alp(\z)) \cdot \nrows{1}{\x} + 1 \\
        &\leq \omega + \nrows{1}{\x} - \bet  \cdot \nrows{2}{\x} + \gam  \cdot \nrows{2}{\x} + \nrows{2}{\x} + \nrows{3}{\x} + 1\\ 
        &\leq 2\omega + \gam  \cdot \nrows{2}{\x} - 1 % beta cannot be 0
    \end{align*}

    We have proven that simultaneously, the color of interval $\x$ is at most $\ffc(\x) \leq 2\omega + \gam  \cdot \nrows{2}{\x} - 1$ and at most $\ffc(\x) \leq 2\omega + (1 - \gam ) \cdot \nrows{2}{\x} - 1$.
    Then, it follows that for any value of $\gam $ the color of interval $\x$ is at most $\ffc(\x) \leq 2\omega + \frac{1}{2}\cdot \nrows{2}{\x} - 1$
    Then, it follows from Lemma~\ref{lem:r2_bound_short} that the number of $\rows{2}{\x}$ rows must be less than $\nrows{2}{\x}  < \frac{2}{3}\omega$, as otherwise the color of interval $\x$ is bounded by $\ffc(\x) \leq \frac{7}{3} \omega - 2$.
    Thus, the color of interval $\x$ is at most $\ffc(\x) < 2\omega + \frac{1}{3}\cdot \omega - 1 = \frac{7}{3}\omega - 1$.
    Which, by the integrality of $\ffc(\x)$, is at most $\ffc(\x) \leq \result$  
\end{proof}
}

%%%%%%%%%%%%%%%%%%%%%%%%%%%%%%%%%%%%%%%%%%%%%%%%%%%%%%%%%%%%%%%%%%%%%%%%%%%%%%%%%%%%%%
%% END OF SHORTVERSION
%%%%%%%%%%%%%%%%%%%%%%%%%%%%%%%%%%%%%%%%%%%%%%%%%%%%%%%%%%%%%%%%%%%%%%%%%%%%%%%%%%%%%%
\fullversion{
% Why lemma 11 still holds for any endpoints
\subsection{Analysis}

\paragraph*{Case (1): $\ffc(\y) \geq \ffc(\z)$}

We start by proving that in case (1), that is, when $\ffc(\y) \geq \ffc(\z)$, the color of interval $\x$ can be bounded by $2\omega$.

\medskip

\runtitle{Pick a pivot $\pivot$}.
    Given the condition where interval $\y$ has a color larger or equal to interval $\z$, picking $\y$ as a pivot provides a smaller bound of $|\pivotset|$.
    Therefore, we pick $\y$ as the pivot $\pivot$.

\medskip

\runtitle{Bound the color of $\ffc(\pivot)$.}
    Next, we bound on the color of the pivot $\ffc(\pivot)$.
    By Corollary~\ref{cor:integral_LMR}, $\ffc(\pivot) = \ffc(\y) \leq 2\omega + \alp (\y) \cdot \nrows{1}{\x} - \nrows{3}{\x}$, where $\alp(\y)$ is the fraction of rows in $\rows{1}{\x}$ that contains intervals in $\twins(\x) \cup \nlmr(\x)$ and having color not higher than $\y$.

\medskip

\runtitle{Bound the size of $\pivotset$}.
Given that we use $\y$ as pivot $\pivot$, we bound the size of $\pivotset$.
Recall that $\pivotset$ is the set of intervals in $\nb(\x)$ which are assigned colors strictly higher than $\ffc(\pivot)$.
Since $\ffc(\y) \geq \ffc(\z)$, only intervals in $\twins(\x)$ are possible to contribute to~$\pivotset$.


\begin{lemma}
    \label{lem:pivotset_y}
    When $\ffc(\y) \geq \ffc(\z)$, by selecting $\y$ as pivot $\pivot$, $|\pivotset| = (1 - \alp(\y)) \cdot \nrows{1}{\x}$.
\end{lemma}

\begin{proof}
    First, by the definition of $\rows{0}{\x}$, $\x$ does not intersect any interval in $\rows{0}{\x}$.
    Thus, no interval in $\rows{0}{\x}$ can be in $\pivotset$.
    Next, the intervals in $\rows{2}{\x}$ are either in $\lmr(\x)$ or in $\nlmr(\x)$.
    Moreover, since $\ffc(\y) \geq \ffc(\z)$, all intervals in $\rows{2}{\x}$ have a color below $\ffc(\y)$ and cannot contribute to $\pivotset$.
    Similarly, as all intervals in $\rows{3}{\x}$ are in $\lmr(\x)$, all intervals in $\rows{3}{\x}$ have a color below $\ffc(\y)$ and cannot contribute to $\pivotset$.
    The remaining intervals are the intervals in $\rows{1}{\x}$.
    By definition, $\alp(\y) \cdot \nrows{1}{\x}$ of these intervals have a color at most $\ffc(\y)$.
    Hence, the remaining $(1 - \alp(\y) ) \cdot \nrows{1}{\x}$ are the only intervals intersecting $\x$ that have a color greater than $\ffc(\y)$. That is, $|\pivotset| = (1 - \alp(\y)) \cdot \nrows{1}{\x}$.
\end{proof}

\medskip

\runtitle{Bound the color of $\ffc(\x)$.}
Now, we wrap up the analysis of the case where $\ffc(\y) \geq \ffc(\z)$ using the bounds of $\ffc(\pivot)$ and $|\pivotset|$.

\begin{lemma}
    \fullversion{\label{lem:c_u>=c_z}}
    For any interval $\x \in \orderedInput$, if $\ffc(\y) \geq \ffc(\z)$, then $\ffc(\x) \leq 2\omega$.
\end{lemma}

\begin{proof}
    By Lemma~\ref{lem:integral_open_and_closed_assumtion_not_hold} and Lemma~\ref{lem:open_interval_small_intersection}, $\ffc(\x) \leq 2\omega$ if $\x$ is open or $\nrows{3}{\x} < \nrows{1}{\x} + 2$.
    Thus, in the following, we focus on closed intervals $\x$ with $\nrows{1}{\x} \leq \nrows{3}{\x} - 2$.
    
    By taking $\y$ as pivot $\pivot$, according to Corollary~\ref{cor:integral_LMR}, $\ffc(\y) \leq 2\omega + \alp(\y) \cdot \nrows{1}{\x} - \nrows{3}{\x}$.
    On the other hand, by  Lemma~\ref{lem:pivotset_y}, $|\pivotset| = (1 - \alp(\y)) \cdot \nrows{1}{\x}$.
    Thus,
    \begin{align*}
        \ffc(\x) &\leq \ffc(\pivot) + |\pivotset| + 1\\
        &\leq \ffc(\y) + (1 - \alp(\y) ) \cdot \nrows{1}{\x} + 1\\
        &\leq 2\omega + \nrows{1}{\x} - \nrows{3}{\x} + 1\\
        &\leq 2\omega - 1
    \end{align*}
\end{proof}

\hide{
In the case where interval $\z$ is in $\lmr(\x)$, then, from the definition of $\y$, it must be the case that $\ffc(\y) \geq \ffc(\z)$.
Therefore, the next result follows directly from Lemma~\ref{lem:c_u>=c_z}.

\begin{corollary}
    \label{cor:z_in_pos_u_v}
    If $\z \in \lmr(\x)$, then $\ffc(\x) \leq 2\omega$.
\end{corollary}
}

\medskip

\paragraph*{Case (2): $\ffc(\y) < \ffc(\z)$}

% Intro to next case
From now on, we focus on the case where $\ffc(\y) < \ffc(\z)$.

\medskip

\runtitle{Pick a pivot $\pivot$.}
    The condition where $\ffc(\y) < \ffc(\z)$ guarantees that all intervals in the set $\lmr(\x)$ have colors at most $\ffc(\z)$.
    Therefore, for the sake of the size of the $\pivotset$, it is more efficient to pick the interval $\z$ as a pivot than picking the interval~$\y$.
    Therefore, we pick interval $\z$ as pivot $\pivot$.

%when $\ffc(\z) > \ffc(\y)$. 

Recall from Definition~\ref{def:z} that $\z$ is the interval with the highest color in the rows in~$\rows{2}{\x}$. 
The selection of $\z$ attains a good structural property on $\rows{3}{\z}$ with regards to $\rows{2}{\x}$:

\begin{lemma}
    \label{lem:R3z_in_R2x}
    $\rows{3}{\z} \subseteq \rows{2}{\x}$.
\end{lemma}

\begin{proof}
    By case distinction on $\rows{i}{\x}$, we prove by contradiction that there is no intersection between $\rows{3}{\z} \cap \rows{i}{\x}$.
    
    \textbf{(1)} Suppose on the contrary that $\rows{3}{\z} \cap \rows{0}{\x} \neq \emptyset$.
    If there exists a row with $3$ intervals that intersect $\z$, but none of them intersects $\x$, then the intervals $\x$ and $\z$ must not intersect.
    Thus, $\z \notin \nlmr(\x)$, which is a contradiction by the definition of $\z$.

    \medskip
    
    \textbf{(2)} Suppose on the contrary that $\rows{3}{\z} \cap \rows{1}{\x} \neq \emptyset$.
    If there exists a row with $3$ intervals that intersect $\z$, but only one of those intervals intersects $\x$. 
    Then, either the interval~$\z$ is in the set $\lmr(\x)$ or $\x$ and $\z$ do not intersect.
    In either case, $\z \not \in \nlmr(\x)$, which is a contradiction by the definition of $\z$.
    
    \medskip
    
    \textbf{(3)} Suppose on the contrary that $\rows{3}{\z} \cap \rows{3}{\x} \neq \emptyset$. 
    If there exists a row with $3$ intervals that intersect both $\x$ and $\z$, then interval $\z$ must be in the set $\twins(\x)$.
    Then, $\z \not \in \nlmr(\x)$, which is a contradiction by the definition of $\z$.
    
    \medskip
    
    By \textbf{(1)} to \textbf{(3)}, $\rows{3}{\z} \cap \left( \rows{0}{\x} \cup \rows{1}{\x} \cup \rows{3}{\x} \right) = \emptyset$.
    Therefore, it follows that $\rows{3}{\z} \subseteq \rows{2}{\x}$.
\end{proof}

% Redefine Z Zbar and zbar
Recall from Definition~\ref{def:Z} that the set $\Z(\x) \subset \rows{2}{\x}$ contains all intervals in an~$\rows{2}{\x}$ row that are twins of interval $\z$. 
\hide{And the set $\rows{3}{\z}$ contains all rows in which~$3$ intervals intersect interval $\z$.
This set $\rows{3}{\z}$ must also be a subset of $\rows{2}{\x}$.}
Furthermore, recall from Definition~\ref{def:z_bar} that interval $\zbar$ is the highest colored interval in $\rows{3}{\z}$ that intersects with $\x$.
Observe that the intersection of the sets $\Z(\x)$ and $\rows{3}{\z}$ must be empty, as a row cannot have both $3$ intervals intersecting $\z$ and an interval identical to $\z$, and therefore $|\Z(\x)| + \nrows{3}{\z} \leq \nrows{2}{\x}$.

% Figures
\begin{figure}[t]
\begin{minipage}{0.35\textwidth}
\centering
\includegraphics[width=\textwidth]{pictures/Brick_Rock_r2.pdf}
\end{minipage}%
\hspace{0.1\textwidth}%
\begin{minipage}{0.55\textwidth}
\centering
\includegraphics[width=\textwidth]{pictures/Brick_Rock_z.pdf}
\end{minipage}
\caption{
An exemplary non-exhaustive list of possible rows per type of row.
Yellow intervals intersect only interval $\z$, purple intervals intersect only $\x$ and orange intervals intersect both $\z$ and $\x$.
}
\label{fig:r2_and_z_intersections}
\end{figure}

\medskip

\runtitle{Bound the color of $\ffc(\pivot)$.}
% Observations
We first provide observations on the maximum number of intersections intervals in $\nlmr(\x)$ and specifically $\z$ have in each type of row (also see Figure~\ref{fig:r2_and_z_intersections}).

\begin{observation}
    \label{obs:any_r2}
    Any interval in the set $\nlmr(\x)$ intersects
    \begin{enumerate}[(a)]
        \item at most $1$ interval, other than $\x$, per row in $\rows{0}{\x}$,
        \item at most $1$ interval, which does not intersect interval $\x$, per row in $\rows{1}{\x}$, and
        \item exactly $2$ intervals per row in $\rows{3}{\x}$.
    \end{enumerate}
\end{observation}

\begin{observation}
    \label{obs:any_z}
    Interval $\z$ intersects
    \begin{enumerate}[(a)]
        \item exactly $3$ intervals per row in $\rows{3}{\z}$,
        \item exactly $1$ interval per row in $\Z(\x)$, and
        \item at most $2$ intervals per row in $\rows{2}{\x} \setminus \left( \Z(\x) \cup \rows{3}{\z} \right)$.
    \end{enumerate}
\end{observation}

% Definition of gamma and delta
Furthermore, we define two variables with respect to the sets $\Z(\x)$ and $\rows{3}{\z}$.
Similarly to $\alp(i)$ being a fraction of the $\rows{1}{\x}$ rows, we define $\bet $ and $\gam $ as fractions of the set~$\rows{2}{\x}$ with respect to interval $\z$.

\begin{definition}
    \label{def:gamma_delta}
    Given $\z \in \rows{2}{\x}$, 
    \begin{enumerate}[(a)]
        \item $\bet \in [0,1]$ denotes the fraction of rows $\mathcal{R} \in \rows{2}{\x}$ where the interval $I^\prime$ in $\mathcal{R} \cap \nb(\x)$ is in $\zbar$, and
        \item $\gam \in [0,1]$ denotes the fraction of rows $\mathcal{R} \in \rows{2}{\x}$ where the interval $I^\prime$ in $\mathcal{R} \cap \nb(\x)$ is in $\rows{3}{\z}$.
    \end{enumerate}
\end{definition}

% A sentence introducing the lemma
Now we are equipped to compute an upper bound on the color of interval $\z$.

\begin{lemma}
    \fullversion{\label{lem:any_z}}
    The color of interval $\z$ is at most $c(\z) \leq \omega + \alp(\z) \cdot \nrows{1}{\x} - \bet  \cdot \nrows{2}{\x} + \gam  \cdot \nrows{2}{\x}  + \nrows{2}{\x} + \nrows{3}{\x}$. 
\end{lemma}

\begin{proof}
    Using the Neighborhood bound and Observations~\ref{obs:any_r2} and~\ref{obs:any_z}, we can describe the maximum color of interval $\z$ as $1$ plus the number of intersections per type of row.
    Observe that by doing this, we count the row that contains interval $\z$ itself, and since $\z$ cannot intersect any interval in this row, we may subtract at least $1$.
    Then it follows that
    \begin{align*}
            \ffc(\z) &\leq \nrows{0}{\x} + (1 + \alp(\z)) \cdot \nrows{1}{\x} + \bet \cdot \nrows{2}{\x} + \gam \cdot 3\nrows{2}{\x} \\&\hspace{2cm}+ (1 - \bet - \gam) \cdot 2\nrows{2}{\x} + 2\nrows{3}{\x} + 1 - 1\\
            &= \omega + \alp(\z) \cdot \nrows{1}{\x} - \bet  \cdot \nrows{2}{\x} + \gam \cdot \nrows{2}{\x} + \nrows{2}{\x} + \nrows{3}{\x}
        \end{align*}
\end{proof}

% Case distinction

Next, we make a case distinction based on the relative sizes of the sets $\Z(\x)$ and $\rows{3}{\z}$.
%%%%%%%%%%%%%%%%%%%%%%%%%%%%
% CASE 2.a
%%%%%%%%%%%%%%%%%%%%%%%%%%%%
\paragraph*{Case (2.a): $|\Z(\x)| \geq \nrows{3}{\z}$}
We start with the case where the number of intervals in the set $\Z(\x)$ is at least as large as the number of intervals in the set $\rows{3}{\z}$.

\medskip 

\runtitle{Bound the size of $\pivotset$.}

\begin{lemma}
    \fullversion{\label{lem:pivotset_z}}
    When $\ffc(\z) > \ffc(\y)$, by selecting $\z$ as pivot $\pivot$, $ |\pivotset| = (1 - \alp(\z)) \cdot \nrows{1}{\x}$.
\end{lemma}

\begin{proof}
    First, by the definition of $\rows{0}{\x}$, $\x$ does not intersect any interval in $\rows{0}{\x}$.
    Thus, no interval in $\rows{0}{\x}$ can be in $\pivotset$.
    Next, all intervals in $\rows{2}{\x}$ are either in $\lmr(\x)$ or $\nlmr(\x)$.
    Since $\ffc(\z) > \ffc(\y)$, it follows that all intervals in $\rows{2}{\x}$ have a color below $\ffc(\z)$ and cannot contribute to $\pivotset$.
    Similarly, as all intervals in $\rows{3}{\x}$ are in $\lmr(\x)$, all intervals in $\rows{3}{\x}$ have a color below $\ffc(\z)$ and cannot contribute to $\pivotset$.
    The remaining intervals are those in $\rows{1}{\x}$.
    By definition, $\alp(\x) \cdot \nrows{1}{\x}$ of these intervals have a color at most $\ffc(\z)$.
    Hence, the remaining $(1 - \alp(\z) ) \cdot \nrows{1}{\x}$ are the only intervals intersecting $\x$ that have a color greater than $\ffc(\z)$. That is, $|\pivotset| = (1 - \alp(\z)) \cdot \nrows{1}{\x}$.
\end{proof}

\medskip

\runtitle{Bound the color of $\ffc(\x)$.}

\fullversion{\begin{lemma}
    \fullversion{\label{lem:gamma<delta}}
    If $|\Z(\x)| \geq \nrows{3}{\z}$, then $\ffc(\x) \leq 2\omega$.
\end{lemma}}

\begin{proof}
    Assume that $\ffc(\z) > \ffc(\y)$ and interval $\x$ is a closed interval, as otherwise by Lemma~\ref{lem:c_u>=c_z} and Lemma~\ref{lem:open_interval_small_intersection} it follows that $\ffc(\x) \leq 2\omega$.
    %From Lemma~\ref{lem:any_z} we have a bound for the color of interval $\z$, i.e., $\ffc(\z) \leq \omega + \alp(\z) \cdot \nrows{1}{\x} - \bet  \cdot \nrows{2}{\x} + \gam  \cdot \nrows{2}{\x} + \nrows{2}{\x} + \nrows{3}{\x}$.
    From the definition of $\bet $ and $\gam $ it follows that since $|\Z(\x)| \geq \nrows{3}{\z}$, also $\bet  \geq \gam $.
    
    Now we can compute a bound on the color of interval $\x$, using the Pivot bound where we take interval $\z$ as $\pivot$.
    By Lemma~\ref{lem:any_z}, the color of interval $\z$ is bounded by $\ffc(\z) \leq \omega + \alp(\z) \cdot \nrows{1}{\x} - \bet  \cdot \nrows{2}{\x} + \gam  \cdot \nrows{2}{\x} + \nrows{2}{\x} + \nrows{3}{\x}$.
    And, by Lemma~\ref{lem:pivotset_z}, the size of $\pivotset$ equals $(1 - \alp(\z)) \cdot \nrows{1}{\x}$.
    Then it follows that,

    \begin{align*}
        \ffc(\x) &\leq \ffc(\pivot) + |\pivotset| + 1\\
        &= \ffc(\z) + (1 - \alp(\z)) \cdot \nrows{1}{\x} + 1 \\
        &\leq \omega + \nrows{1}{\x} - \bet  \cdot \nrows{2}{\x} + \gam  \cdot \nrows{2}{\x} + \nrows{2}{\x} + \nrows{3}{\x} + 1 \\
        &\leq \omega + \nrows{1}{\x} + \nrows{2}{\x} + \nrows{3}{\x} + 1 \\
        &\leq 2\omega
    \end{align*}
\end{proof}
%%%%%%%%%%%%%%%%%%%%%%%%%%%%
% CASE 2.b
%%%%%%%%%%%%%%%%%%%%%%%%%%%%
% Continue with the second case of the case distinction. And introduce new one
\paragraph*{Case (2.b): $|\Z(\x)| < \nrows{3}{\z}$}
We continue with the case where the number of intervals in the set $\Z(\x)$ is strictly smaller than the number of intervals in the set $\rows{3}{\z}$.
This relation between $\Z(\x)$ and $\rows{3}{\z}$ indicates that the average number of intersections interval $\z$ has per row can be greater than $2$.
Then, it might be interesting to explore other options as a pivot.
For this we make another case distinction on the relation between the colors of interval $\y$ and interval $\zbar$.

%%%%%%%%%%%%%%%%%%%%%%%%%%%%
% CASE 2.b.i
%%%%%%%%%%%%%%%%%%%%%%%%%%%%
\paragraph*{Case (2.b.i): $\ffc(\zbar) \geq \ffc(\y)$}
First, let us explore the case where the color of interval $\zbar$ is at least as large as the color of interval $\y$.

\medskip

\runtitle{Pick a pivot $\pivot$}
The condition that $\ffc(\zbar) \geq \ffc(\y)$ guarantees that all intervals in the set $\lmr(\x)$ have colors at most $\ffc(\zbar)$.
Therefore, for the sake of the size of the size of the set $\pivotset$, it is more efficient to pick interval $\zbar$ as a pivot compared to picking interval $\y$.
Therefore, we pick interval $\zbar$ as pivot $\pivot$.

\medskip

\runtitle{Bound the color $\ffc(\pivot)$}

% zbar is not in LMR
As we now want to compute an upper bound on the color of interval $\zbar$, it is of importance to know where interval $\zbar$ might be located.

\begin{lemma}
    \label{lem:zbar_not_in_LMR}
    For any interval $I^\prime \in \rows{3}{\z}$, $I^\prime \not \in \lmr(\x)$
\end{lemma}

\begin{proof}
    Assume aiming towards a contradiction that $I^\prime \in \lmr(\x)$.
    \medskip

    \runtitle{(1) $I^\prime$ is a closed interval.} %$i \in L(\x)$. 
    %$L(\x)$ contains only closed intervals.
    Assume without loss of generality that interval $I^\prime$ intersects interval $\x$ on the left-hand side of interval $\x$.
    Since $I^\prime \in \rows{3}{\z}$, and $I^\prime$ is a closed interval, it follows that there are precisely two positions where interval $\z$ could lie.
    The first possible position is to the left of interval $I^\prime$.
    In this case, interval $\z$ and interval $\x$ cannot intersect.
    Then it follows that $\z \not \in \nlmr(\x)$, which is a contradiction by the definition of $\z$.
    The other possible position is to the right of interval $I^\prime$.
    In this case, interval $\z$ must be identical to interval $\x$, i.e., $\z \in \twins(\x)$.
    Then it follows that $\z \not \in \nlmr(\x)$, which is a contradiction by the definition of $\z$.
    
    \medskip
    
    \runtitle{(2) $I^\prime$ is an open interval.} %$i \in M(\x)$.
    %$M(\x)$ contains only open intervals.
    Since $I^\prime \in \lmr(\x)$ and $I^\prime$ is an open interval, it follows that $I^\prime \in \twins(\x)$.
    Similarly, since $I^\prime \in \rows{3}{\z}$, and $I^\prime$ is an open interval, it follows that $I^\prime \in \twins(\z)$.
    An interval can only be in both $\twins(\x)$ and $\twins(\z)$ if interval $\x$ and interval $\z$ are identical, i.e., $\z \in \twins(\x)$.
    Then it follows that $\z \not \in \nlmr(\x)$, which is a contradiction by the definition of $\z$.
\end{proof}

% Figures
\begin{figure}%[ht]
\centering
\begin{minipage}{0.49\textwidth}
\centering
\includegraphics[width=\textwidth]{pictures/Brick_Rock_zbar.pdf}
\end{minipage}%
\hspace{0.01\textwidth}%
\begin{minipage}{0.49\textwidth}
\centering
\includegraphics[width=\textwidth]{pictures/Brick_Rock_y.pdf}
\end{minipage}%
\caption{
An exemplary non-exhaustive list of possible rows per type of $\rows{2}{\x}$.
Yellow intervals intersect only interval $\zbar$ (resp. $\y$), purple intervals intersect only $\x$ and orange intervals intersect both $\zbar$ (resp. $\y$) and $\x$.}
\label{fig:y_intersections}
\end{figure}

% Recall observations from previous proof
Note that Observation~\ref{obs:any_r2} remains true for interval $\zbar$.
% Introduce new observations
Now, let us make an additional observation regarding the intersections interval $\zbar$ has with intervals in $\rows{2}{\x}$ rows (also see Figure~\ref{fig:y_intersections}):
%Exemplary rows are depicted in Figure~\ref{fig:r3z_intersections} to give some intuition.

\begin{observation}
    \label{obs:any_z_bar}
    Interval $\zbar$ intersects
    \begin{enumerate}[(a)]
        \item exactly $1$ interval per row contained in $\rows{3}{\z}$,
        \item exactly $3$ intervals per row contained in $\Z(\x)$, and
        \item at most $2$ intervals per row contained in $\rows{2}{\x} \setminus (\Z(\x) \cup \rows{3}{\z} )$.
    \end{enumerate}
\end{observation}


% Not all intervals are below or above zbar
Furthermore, it is important to specify how many of the intervals in $\rows{2}{\x} \setminus \rows{3}{\z}$ are assigned a color larger than $\ffc(\zbar)$ or smaller than $\ffc(\zbar)$.
This was previously not necessary, when we considered interval $\z$, which is, by definition, the interval in $\rows{2}{\x}$ assigned the largest color.
Analogously, we do not need to consider the intervals in $\rows{3}{\z}$ as per definition interval $\zbar$ is the largest colored interval in this set.

\begin{definition}
    \label{def:delta}
    Given any interval $\hat{I}$, $\delt(\hat{I}) \in [0, 1]$ denotes the fraction of rows $\mathcal{R} \in \rows{2}{\x} \setminus \rows{3}{\z}$ where interval $I^\prime$ in $\mathcal{R} \cap \nb(\x) \cap \nb(\z)$ satisfies $\ffc(I^\prime) \leq \ffc(\hat{I})$.
\end{definition}

% Introduce the next lemma
Now we are equipped to compute a bound on the color of interval $\zbar$.

\begin{lemma}
    \label{lem:any_z_bar}
    The color of interval $\zbar$, $\ffc(\zbar) \leq \nrows{0}{\x} + (1 + \alp(\zbar)) \cdot \nrows{1}{\x} + \delt(\zbar) \cdot (1 - \gam ) \cdot 2\nrows{2}{\x} + \bet  \cdot \nrows{2}{\x} + \gam  \cdot \nrows{2}{\x} + 2\nrows{3}{\x}$.
\end{lemma}

\begin{proof}
    By the Neighborhood bound $\ffc(\zbar) \leq 1 + |\mathcal{N}(\zbar)|$.
    By Observation~\ref{obs:any_r2}, there are at most $\nrows{0}{\x} + 2\nrows{1}{\x} + 2\nrows{3}{\x}$ intervals in $\rows{0}{\x}, \rows{1}{\x}$ and $\rows{3}{\x}$ that are in $\mathcal{N}(\zbar)$.
    However, by Definition~\ref{def:alpha}, $(1 - \alp(\zbar))$ of those intervals have a color larger than $\ffc(\zbar)$ and hence we do not consider them for this bound.
    By Observation~\ref{obs:any_z_bar}, there are at most $\nrows{3}{\z} + 3\cdot|\Z(\x)| + 2\cdot|\rows{2}{\x}\setminus(\Z(\x)\cup\rows{3}{\z})| = \gam \cdot \nrows{2}{\x} + \bet \cdot 3\nrows{2}{\x} + (1 - \bet - \gam) \cdot 2\nrows{2}{\x}$ intervals in $\rows{2}{\x}$ that are in $\mathcal{N}(\zbar)$.
    However, by Definition~\ref{def:delta}, $(1 - \delt(\zbar)) \cdot (\bet \cdot 2\nrows{2}{\x} + (1 - \bet - \gam) \cdot 2\nrows{2}{\x})$ of those intervals have a color larger than $\ffc(\zbar)$ and hence we do not consider them for this bound.
    Observe that by this bound, we count the row that contains interval $\zbar$ itself.
    Since $\zbar$ cannot intersect any interval on this row, we should subtract at least 1 from this bound.
    Thus,

    \begin{align*}
            \ffc(\zbar) &\leq \nrows{0}{\x} + 2\nrows{1}{\x} - (1 - \alp(\zbar)) \cdot \nrows{1}{\x} + 2\nrows{3}{\x}  
            \\ &\hspace{1cm} + \bet  \cdot 3\nrows{2}{\x} + \gam  \cdot \nrows{2}{\x} + (1 - \bet  - \gam ) \cdot 2\nrows{2}{\x}
            \\ &\hspace{1cm} - (1 - \delt(\zbar)) \cdot \left( \bet  \cdot 2\nrows{2}{\x} + (1 - \bet  - \gam ) \cdot 2\nrows{2}{\x} \right) -1 + 1\\
            %
            &= \nrows{0}{\x} + (1 + \alp(\zbar)) \cdot \nrows{1}{\x} + \delt(\zbar) \cdot (1 - \gam ) \cdot 2\nrows{2}{\x} + \bet  \cdot \nrows{2}{\x} \\&\hspace{1cm} + \gam  \cdot \nrows{2}{\x} + 2\nrows{3}{\x}
        \end{align*}
\end{proof}

\medskip

\runtitle{Bound the size of $\pivotset$.}

\begin{lemma}
    \label{lem:pivotset_zbar}
    By selecting $\zbar$ as pivot $\pivot$, $|\pivotset| = (1-\alp(\zbar)) \cdot \nrows{1}{\x} + (1 - \delt(\zbar) \cdot (1 - \gam) \cdot 2\nrows{2}{\x}$.
\end{lemma}

\begin{proof}
    First, by the definition of $\rows{0}{\x}$, $\x$ does not intersect any interval in $\rows{0}{\x}$.
    Thus, no interval in $\rows{0}{\x}$ can be in $\pivotset$.
    Next, all intervals in $\rows{3}{\x}$ are in the set $\lmr(\x)$.
    Thus, all these intervals are assigned a color below $\ffc(\zbar)$ and cannot contribute to $\pivotset$.
    Of the intervals in $\rows{1}{\x}$, by definition of $\alp$, only $(1 - \alp(\zbar))\cdot \nrows{1}{\x}$ intervals have a color larger than $\ffc(\zbar)$, and can contribute to set $\pivotset$.
    Of the $\rows{2}{\x}$ rows, out of the intervals intersecting $\x$ that are in the rows that are also in the set $\rows{3}{\z}$, none have a color larger than $\ffc(\zbar)$ by the definition of $\zbar$.
    Then, of the remaining $(1 - \gam) \cdot \nrows{2}{\x}$ $\rows{2}{\x}$ rows, by the definition of $\delt$, only $(1 - \delt(\zbar)) \cdot (1 - \bet - \gam) \cdot 2\nrows{2}{\x}$ intervals have a color larger than $\ffc(\zbar)$ and can contribute to the set $\pivotset$.
    Then it follows that $|\pivotset| = (1 - \alp(\zbar))\cdot \nrows{1}{\x} + (1 - \delt(\zbar)) \cdot (1 - \gam) \cdot 2\nrows{2}{\x} $
\end{proof}

\medskip

\runtitle{Bound the color $\ffc(\x)$.}

% Something to connect the lemma's?
Now we can use this result to prove an upper bound on the color of interval $\x$, when the color of interval $\zbar$ is at least as large as the color of interval $\y$.

\begin{lemma}
    \label{lem:cy>=cu}
    If $\ffc(\zbar) \geq \ffc(\y)$, then $\ffc(\x) \leq 2\omega$   
\end{lemma}

\begin{proof}
    By Lemma~\ref{lem:gamma<delta}, we assume that $|\Z(\x)| < \nrows{3}{\z}$, and thus $\bet  < \gam$.
    By Lemma~\ref{lem:any_z_bar}, $\ffc(\zbar) \leq \nrows{0}{\x} + (1 + \alp(\zbar)) \cdot \nrows{1}{\x} + \delt(\zbar) \cdot (1 - \gam) \cdot 2\nrows{2}{\x} + \bet  \cdot \nrows{2}{\x} + \gam  \cdot \nrows{2}{\x} + 2\nrows{3}{\x}$.
    By Lemma~\ref{lem:pivotset_zbar}, $|\pivotset| = (1-\alp(\zbar)) \cdot \nrows{1}{\x} + (1 - \delt(\zbar) \cdot (1 - \gam) \cdot 2\nrows{2}{\x}$.
    It follows that,
    
    \begin{align*}
        \ffc(\x) &\leq \ffc(\pivot) + |\pivotset| + 1\\
        &\leq \ffc(\zbar) + (1 - \alp(\zbar)) \cdot \nrows{1}{\x} + ( 1 -  \delt(\zbar) ) \cdot (1 - \gam ) \cdot 2\nrows{2}{\x} + 1\\
        &\leq \nrows{0}{\x} + 2\nrows{1}{\x}  + \bet  \cdot \nrows{2}{\x} + \gam  \cdot \nrows{2}{\x} + (1 - \gam ) \cdot 2\nrows{2}{\x} + 2\nrows{3}{\x} + 1\\
        &< \nrows{0}{\x} + 2\nrows{1}{\x}  + \bet  \cdot \nrows{2}{\x} + (1 - \bet ) \cdot \nrows{2}{\x} + \nrows{2}{\x} + 2\nrows{3}{\x} + 1\\
        &\leq 2\omega
    \end{align*}
\end{proof}

%%%%%%%%%%%%%%%%%%%%%%%%%%%%
% CASE 2.b.ii
%%%%%%%%%%%%%%%%%%%%%%%%%%%%
\paragraph*{Case (2.b.ii): $\ffc(\y) > \ffc(\zbar)$ }

In order to prove Theorem~\ref{thm:any} we still need to consider the tough kernel of the analysis.
That is, the case where $\ffc(\zbar) < \ffc(\y) < \ffc(\z)$ and $\bet  < \gam $.
For this case we abandon our usual mechanism slightly, where instead of focusing on a single pivot $\pivot$ and a single set $\pivotset$, we explore the option where two distinct pairs of a pivot and a set cannot both admit a large solution.

\medskip

\runtitle{Pick the pivots $\pivot$.}
Although we have reasoned earlier that picking interval $\y$ and interval $\z$ as the pivot is not sufficient to have an upper bound on $\ffc(\x)$ that is smaller than $3\omega$, we can deal with this tough case by expressing the bound obtained by using $\y$ or $\z$ as the pivot in terms of $|\Z(\x)|$.

\medskip


\runtitle{Bound the color of $\pivot$}
Let us first take a closer look at the color of interval $\y$.
Although Lemma~\ref{lem:integral_uvw} is technically still correct, knowing that $\ffc(\zbar) < \ffc(\y)$, we can be slightly more precise about which intersections are assigned a color greater than $\ffc(\y)$.
In order to be more precise, let us first make additional observations regarding the intersections of interval $\y$ with interval in $\rows{2}{\x}$ rows (also see Figure~\ref{fig:y_intersections}).

\begin{observation}
    \label{obs:any_y}
    Interval $\y$ intersects
    \begin{enumerate}[(a)]
        \item at most $1$ interval that is not in $\nb(\x)$ per row contained in $\rows{2}{\x}$, and
        \item at most $1$ interval that is in $\nb(\x)$ per row contained in $\rows{3}{\z}$,
        \item at most $1$ intervals that is in $\nb(\x)$ per row contained in $\Z(\x)$, and
        \item at most $1$ intervals that is in $\nb(\x)$ per row contained in $\rows{2}{\x} \setminus (\Z(\x) \cup \rows{3}{\z} )$.
    \end{enumerate}
\end{observation}

\begin{lemma}
    \label{lem:any_uvw}
    The color of interval $\y$,
    $\ffc(\y) \leq 2\nrows{0}{\x} + 2\nrows{1}{\x} + \alp(\y) \cdot \nrows{1}{\x} + \gam  \cdot \nrows{2}{\x} + \nrows{2}{\x} + \delt(\y) \cdot (1 - \gam ) \cdot \nrows{2}{\x} + \nrows{3}{\x}$. 
\end{lemma}

\begin{proof}
    By the Neighborhood bound, $\ffc(\y) \leq 1 + |\mathcal{N}(\y)|$.
    By Observation~\ref{obs:integral}, there are at most $ 2\nrows{0}{\x} + 2\nrows{1}{\x} + \nrows{3}{\x}$ intervals in $\rows{0}{\x}, \rows{1}{\x}$ and $\rows{3}{\x}$ that are in $\nb(\y)$.
    Together with the at most $\alp(\y) \cdot \nrows{1}{\x}$ intervals that were assigned a color below $\ffc(\y)$ which are either in $\twins(\x)$ or do not intersect $\y$ and are in $\nlmr(\x)$.
    
    By Observation~\ref{obs:any_y}, there are at most $\nrows{2}{\x} + \gam \cdot \nrows{2}{\x} + \bet \cdot \nrows{2}{\x} + (1 - \gam - \bet) \cdot \nrows{2}{\x}$ intervals that are in $\rows{2}{\x}$ and in $\nb(\y)$.
    By our assumption that $\ffc(\zbar) < \ffc(\y)$, the $\gam  \cdot \nrows{2}{\x}$ intervals in the set $\rows{3}{\z} \cap \nb(\x)$, must have a color smaller than $\ffc(\y)$.
    For the remaining $(1 - \gam) \cdot \nrows{2}{\x}$ intervals that are in $\nb(\x)$, by the definition of $\delt$,  $(1 - \delt(\y)) \cdot (1 - \gam) \cdot \nrows{2}{\x}$ are assigned a color larger than $\ffc(\z)$ and hence we do not consider them for this bound.
    
    Observe that by this bound we count the row that contains interval $\y$ itself.
    Since interval $\y$ cannot intersect any interval on this row, we should subtract at least 1 from this bound. 
    Thus,

    \begin{align*}
            \ffc(\y) &\leq |\nb(\y)| + 1\\
            &\leq 2\nrows{0}{\x} + 2\nrows{1}{\x} + \alp(\y) \cdot \nrows{1}{\x} + \gam  \cdot \nrows{2}{\x} \\&\hspace{1cm} + \nrows{2}{\x} + \delt(\y) \cdot (1 - \gam ) \cdot \nrows{2}{\x} + \nrows{3}{\x}
        \end{align*}    
\end{proof}

As for the bound on the color of interval $\z$, Lemma~\ref{lem:any_z} still suffices.

\medskip


\runtitle{Bound the sizes of $\pivotset$.}
Given that we use $\y$ as pivot $\pivot$, we bound the size of $\pivotset$.

\begin{lemma}
    \label{lem:pivotset_y_exact}
    When $\ffc(\y) \geq \ffc(\zbar)$, by selecting $\y$ as pivot $\pivot$, $|\pivotset| \leq (1 - \alp(\y)) \cdot \nrows{1}{\x} + (1 - \delt(\z)) \cdot (1 - \gam ) \cdot \nrows{2}{\x} + (1 - \gam ) \cdot \nrows{2}{\x}$
\end{lemma}

\begin{proof}
    First, by the definition of $\rows{0}{\x}$, $\x$ does not intersect any interval in $\rows{0}{\x}$.
    Thus, no interval in $\rows{0}{\x}$ can be in $\pivotset$.
    By the definition of $\alp$, only $(1 - \alp(\y) ) \cdot \nrows{1}{\x}$ intervals in $\rows{1}{\x}$ are assigned a color strictly larger than $\ffc(\y)$.
    Thus the only intervals in an $\rows{1}{\x}$ that are eligible for $\pivotset$ are those $(1 - \alp(\y) ) \cdot \nrows{1}{\x}$ intervals.
    Since $\ffc(\y) > \ffc(\zbar)$, no interval in an $\rows{3}{\z}$ row is assigned a color larger than $\ffc(\y)$.
    Thus, none of the $\gam \cdot 2\nrows{2}{\x}$ intervals in $\rows{3}{\z} \cap \nb(\x)$ can contribute to $\pivotset$.
    By the definition of $\delt$, only $(1 - \delt(\y) ) \cdot (1 - \gam) \cdot \nrows{2}{\x}$ intervals both in $\nb(\y)$ and $\rows{2}{\x}$ are assigned a color strictly larger than $\ffc(\y)$.
    Furthermore, all $(1 - \gam) \cdot \nrows{2}{\x}$ intervals which are in $\rows{2}{\x}$ but not in $\nb(\y)$ and not in $\rows{3}{\z}$ could potentially be assigned a color larger than $\ffc(\y)$ and therefore contribute to $\pivotset$.
    Finally, as all intervals in $\rows{3}{\x}$ are in $\lmr(\x)$, all intervals in $\rows{3}{\x}$ have a color below $\ffc(\y)$ and cannot contribute to $\pivotset$.
    Thus, $|\pivotset| \leq (1 - \alp(\y)) \cdot \nrows{1}{\x} + (1 - \delt(\z)) \cdot (1 - \gam ) \cdot \nrows{2}{\x} + (1 - \gam ) \cdot \nrows{2}{\x}$.
\end{proof}

As for the set $\pivotset$ when we pick interval $\z$ as pivot, Lemma~\ref{lem:pivotset_z} still suffices.

\medskip


\runtitle{Bound the size of $\ffc(\x)$.}

% The number of R_2(x) rows is bounded
As the bound we are going to prove in the proof of Theorem~\ref{thm:any} is based on the size of set $\Z(\x)$, which in turn is strictly smaller than the size of set $\rows{2}{\x}$, an upper bound on $\nrows{2}{\x}$ improves the bound on the color of $\x$ we would otherwise obtain.
This upper bound on the number of rows in $\rows{2}{\x}$ is obtained by a lower bound on the number of rows in $\rows{3}{\x}$.

\begin{lemma}
    \label{lem:r3_lb}
    For any interval $\x \in \orderedInput$, if $\nrows{3}{\x} < \frac{1}{3}\omega$, then $\ffc(\x) \leq \frac{7}{3} \omega - 2$.
\end{lemma}

\begin{proof}
    We show this using the Neighborhood bound.
    Observe that $\nrows{0}{\x} \geq 1$, since interval $\x$ does not intersect any intervals on its own row.
    Furthermore, note that the Neighborhood bound is maximized when we maximize the number of intervals that intersect interval $\x$.
    That is, we maximize the bound when we maximize $\nrows{3}{\x}$.
    Then, $\nrows{3}{\x} = \frac{1}{3}\omega - 1$.
    For the remaining rows, we maximize the bound by maximizing $\nrows{2}{\x}$, and hence $\nrows{2}{\x} = \frac{2}{3}\omega$.
    Then, $\nrows{1}{\x} = 0$, and we get the following upper bound on the color of interval $\x$,
    \begin{align*}
        \ffc(\x) &\leq \nrows{1}{\x} + 2\nrows{2}{\x} + 3\nrows{3}{\x} + 1 \\
            &\leq 0 + 2 \cdot \frac{2}{3} \omega + 3 \cdot (\frac{1}{3}\omega - 1) + 1\\
            &= \frac{7}{3} \omega - 2
    \end{align*}
\end{proof}

Now we have shown that $\nrows{3}{\x} \geq \frac{1}{3}\omega$, it simply follows that $\nrows{2}{\x} < \frac{2}{3}\omega$.

\begin{corollary}
    \label{cor:r2_bound}
    For any interval $\x \in \orderedInput$, if $\nrows{2}{\x} \geq \frac{2}{3} \omega$, then $\ffc(\x) \leq \frac{7}{3} \omega - 2$. 
\end{corollary}

% Intro to main thm
We have now shown everything we need to show in order to compute a bound on the color of interval $\x$.

\paragraph*{Proof of Theorem~\ref{thm:any}.}
\emph{For any interval $\x \in \orderedInput$, the color of $\x$ is at most $\ffc(\x) \leq \result$.}

\begin{proof}
    By Lemma~\ref{lem:open_interval_small_intersection},~\ref{lem:integral_open_and_closed_assumtion_not_hold}, \ref{lem:c_u>=c_z}, \ref{lem:gamma<delta} and \ref{lem:cy>=cu}, $\ffc(\x) \leq 2\omega$ if $\x$ is open, $\nrows{1}{\x} \leq \nrows{3}{\x} - 2$, $\ffc(\y) < \ffc(\z)$, $\bet  < \gam $ or $\ffc(\zbar) < \ffc(\y)$.
    Thus in the following we focus on closed intervals with $\nrows{1}{\x} \leq \nrows{3}{\x} - 2$, $\ffc(\zbar) < \ffc(\y) < \ffc(\z)$ and $\bet  < \gam $.

    Now we can compute the first bound on the color of interval $\x$, using the Pivot bound with interval $\y$ as pivot $\pivot$.
    According to Lemma~\ref{lem:any_uvw}, $\ffc(\y) \leq  2\nrows{0}{\x} + 2\nrows{1}{\x} + \alp(\y) \cdot \nrows{1}{\x} + \gam  \cdot \nrows{2}{\x} + \nrows{2}{\x} + \delt(\z) \cdot (1 - \gam ) \cdot \nrows{2}{\x} + \nrows{3}{\x}$ and, by taking $\y$ as pivot $\pivot$,  Lemma~\ref{lem:pivotset_y_exact}, $|\pivotset| \leq (1 - \alp(\y)) \cdot \nrows{1}{\x} + (1 - \delt(\z)) \cdot (1 - \gam ) \cdot \nrows{2}{\x} + (1 - \gam ) \cdot \nrows{2}{\x}$.
    Thus,

    \begin{align*}
        \ffc(\x) &\leq \ffc(\pivot) + |\pivotset| + 1\\
        &\leq \ffc(\y) + (1 - \alp(\y)) \cdot \nrows{1}{\x} + (1 - \delt(\z)) \cdot (1 - \gam ) \cdot \nrows{2}{\x} \\ &\hspace{1cm} + (1 - \gam ) \cdot \nrows{2}{\x} + 1 \\
        &\leq 2\nrows{0}{\x} + 3\nrows{1}{\x} + 2\nrows{2}{\x} + (1 - \gam ) \cdot \nrows{2}{\x} + \nrows{3}{\x} + 1 \\
        &\leq 2\nrows{0}{\x} + 2\nrows{1}{\x} + (1 - \gam ) \cdot \nrows{2}{\x} + 2\nrows{2}{\x} + 2\nrows{3}{\x} - 1\\
        &= 2\omega + (1 - \gam ) \cdot \nrows{2}{\x} - 1
    \end{align*}

    Next we move to the other bound.
    By taking $\z$ as pivot $\pivot$, according to Lemma~\ref{lem:any_z}, $\ffc(\z) \leq \omega + \alp(\z) \cdot \nrows{1}{\x} - \bet  \cdot \nrows{2}{\x} + \gam  \cdot \nrows{2}{\x} + \nrows{2}{\x} + \nrows{3}{\x}$, and Lemma~\ref{lem:pivotset_z}, $|\pivotset| = (1 - \alp(\z))\cdot\nrows{1}{\x}$.
    Then, it follows that,

    \begin{align*}
        \ffc(\x) &\leq \ffc(\pivot) + |\pivotset| + 1\\
        &\leq \ffc(\z) + (1 - \alp(\z)) \cdot \nrows{1}{\x} + 1 \\
        &\leq \omega + \nrows{1}{\x} - \bet  \cdot \nrows{2}{\x} + \gam  \cdot \nrows{2}{\x} + \nrows{2}{\x} + \nrows{3}{\x} + 1\\ 
        &\leq 2\omega + \gam  \cdot \nrows{2}{\x} - 1 % beta cannot be 0
    \end{align*}

    We have proven that simultaneously, the color of interval $\x$ is at most $\ffc(\x) \leq 2\omega + \gam  \cdot \nrows{2}{\x} - 1$ and at most $\ffc(\x) \leq 2\omega + (1 - \gam ) \cdot \nrows{2}{\x} - 1$.
    Then, it follows that for any value of $\gam $ the color of interval $\x$ is at most $\ffc(\x) \leq 2\omega + \frac{1}{2}\cdot \nrows{2}{\x} - 1$
    Then, it follows from Corollary~\ref{cor:r2_bound} that the number of $\rows{2}{\x}$ rows must be less than $\nrows{2}{\x}  < \frac{2}{3}\omega$, as otherwise the color of interval $\x$ is bounded by $\ffc(\x) \leq \frac{7}{3} \omega - 2$.
    Thus, the color of interval $\x$ is at most $\ffc(\x) < 2\omega + \frac{1}{3}\cdot \omega - 1 = \frac{7}{3}\omega - 1$.
    Which, by the integrality of $\ffc(\x)$, is at most $\ffc(\x) \leq \result$  
\end{proof}
}

% \section{Conclusion}
% \section{Conclusion}
In this work, we propose a simple yet effective approach, called SMILE, for graph few-shot learning with fewer tasks. Specifically, we introduce a novel dual-level mixup strategy, including within-task and across-task mixup, for enriching the diversity of nodes within each task and the diversity of tasks. Also, we incorporate the degree-based prior information to learn expressive node embeddings. Theoretically, we prove that SMILE effectively enhances the model's generalization performance. Empirically, we conduct extensive experiments on multiple benchmarks and the results suggest that SMILE significantly outperforms other baselines, including both in-domain and cross-domain few-shot settings.

% \newpage

%%
%% Bibliography
%%

%% Please use bibtex, 


%\section{Omitted proofs}
%\subsection{Lloyd-Max Algorithm}
\label{subsec:Lloyd-Max}
For a given quantization bitwidth $B$ and an operand $\bm{X}$, the Lloyd-Max algorithm finds $2^B$ quantization levels $\{\hat{x}_i\}_{i=1}^{2^B}$ such that quantizing $\bm{X}$ by rounding each scalar in $\bm{X}$ to the nearest quantization level minimizes the quantization MSE. 

The algorithm starts with an initial guess of quantization levels and then iteratively computes quantization thresholds $\{\tau_i\}_{i=1}^{2^B-1}$ and updates quantization levels $\{\hat{x}_i\}_{i=1}^{2^B}$. Specifically, at iteration $n$, thresholds are set to the midpoints of the previous iteration's levels:
\begin{align*}
    \tau_i^{(n)}=\frac{\hat{x}_i^{(n-1)}+\hat{x}_{i+1}^{(n-1)}}2 \text{ for } i=1\ldots 2^B-1
\end{align*}
Subsequently, the quantization levels are re-computed as conditional means of the data regions defined by the new thresholds:
\begin{align*}
    \hat{x}_i^{(n)}=\mathbb{E}\left[ \bm{X} \big| \bm{X}\in [\tau_{i-1}^{(n)},\tau_i^{(n)}] \right] \text{ for } i=1\ldots 2^B
\end{align*}
where to satisfy boundary conditions we have $\tau_0=-\infty$ and $\tau_{2^B}=\infty$. The algorithm iterates the above steps until convergence.

Figure \ref{fig:lm_quant} compares the quantization levels of a $7$-bit floating point (E3M3) quantizer (left) to a $7$-bit Lloyd-Max quantizer (right) when quantizing a layer of weights from the GPT3-126M model at a per-tensor granularity. As shown, the Lloyd-Max quantizer achieves substantially lower quantization MSE. Further, Table \ref{tab:FP7_vs_LM7} shows the superior perplexity achieved by Lloyd-Max quantizers for bitwidths of $7$, $6$ and $5$. The difference between the quantizers is clear at 5 bits, where per-tensor FP quantization incurs a drastic and unacceptable increase in perplexity, while Lloyd-Max quantization incurs a much smaller increase. Nevertheless, we note that even the optimal Lloyd-Max quantizer incurs a notable ($\sim 1.5$) increase in perplexity due to the coarse granularity of quantization. 

\begin{figure}[h]
  \centering
  \includegraphics[width=0.7\linewidth]{sections/figures/LM7_FP7.pdf}
  \caption{\small Quantization levels and the corresponding quantization MSE of Floating Point (left) vs Lloyd-Max (right) Quantizers for a layer of weights in the GPT3-126M model.}
  \label{fig:lm_quant}
\end{figure}

\begin{table}[h]\scriptsize
\begin{center}
\caption{\label{tab:FP7_vs_LM7} \small Comparing perplexity (lower is better) achieved by floating point quantizers and Lloyd-Max quantizers on a GPT3-126M model for the Wikitext-103 dataset.}
\begin{tabular}{c|cc|c}
\hline
 \multirow{2}{*}{\textbf{Bitwidth}} & \multicolumn{2}{|c|}{\textbf{Floating-Point Quantizer}} & \textbf{Lloyd-Max Quantizer} \\
 & Best Format & Wikitext-103 Perplexity & Wikitext-103 Perplexity \\
\hline
7 & E3M3 & 18.32 & 18.27 \\
6 & E3M2 & 19.07 & 18.51 \\
5 & E4M0 & 43.89 & 19.71 \\
\hline
\end{tabular}
\end{center}
\end{table}

\subsection{Proof of Local Optimality of LO-BCQ}
\label{subsec:lobcq_opt_proof}
For a given block $\bm{b}_j$, the quantization MSE during LO-BCQ can be empirically evaluated as $\frac{1}{L_b}\lVert \bm{b}_j- \bm{\hat{b}}_j\rVert^2_2$ where $\bm{\hat{b}}_j$ is computed from equation (\ref{eq:clustered_quantization_definition}) as $C_{f(\bm{b}_j)}(\bm{b}_j)$. Further, for a given block cluster $\mathcal{B}_i$, we compute the quantization MSE as $\frac{1}{|\mathcal{B}_{i}|}\sum_{\bm{b} \in \mathcal{B}_{i}} \frac{1}{L_b}\lVert \bm{b}- C_i^{(n)}(\bm{b})\rVert^2_2$. Therefore, at the end of iteration $n$, we evaluate the overall quantization MSE $J^{(n)}$ for a given operand $\bm{X}$ composed of $N_c$ block clusters as:
\begin{align*}
    \label{eq:mse_iter_n}
    J^{(n)} = \frac{1}{N_c} \sum_{i=1}^{N_c} \frac{1}{|\mathcal{B}_{i}^{(n)}|}\sum_{\bm{v} \in \mathcal{B}_{i}^{(n)}} \frac{1}{L_b}\lVert \bm{b}- B_i^{(n)}(\bm{b})\rVert^2_2
\end{align*}

At the end of iteration $n$, the codebooks are updated from $\mathcal{C}^{(n-1)}$ to $\mathcal{C}^{(n)}$. However, the mapping of a given vector $\bm{b}_j$ to quantizers $\mathcal{C}^{(n)}$ remains as  $f^{(n)}(\bm{b}_j)$. At the next iteration, during the vector clustering step, $f^{(n+1)}(\bm{b}_j)$ finds new mapping of $\bm{b}_j$ to updated codebooks $\mathcal{C}^{(n)}$ such that the quantization MSE over the candidate codebooks is minimized. Therefore, we obtain the following result for $\bm{b}_j$:
\begin{align*}
\frac{1}{L_b}\lVert \bm{b}_j - C_{f^{(n+1)}(\bm{b}_j)}^{(n)}(\bm{b}_j)\rVert^2_2 \le \frac{1}{L_b}\lVert \bm{b}_j - C_{f^{(n)}(\bm{b}_j)}^{(n)}(\bm{b}_j)\rVert^2_2
\end{align*}

That is, quantizing $\bm{b}_j$ at the end of the block clustering step of iteration $n+1$ results in lower quantization MSE compared to quantizing at the end of iteration $n$. Since this is true for all $\bm{b} \in \bm{X}$, we assert the following:
\begin{equation}
\begin{split}
\label{eq:mse_ineq_1}
    \tilde{J}^{(n+1)} &= \frac{1}{N_c} \sum_{i=1}^{N_c} \frac{1}{|\mathcal{B}_{i}^{(n+1)}|}\sum_{\bm{b} \in \mathcal{B}_{i}^{(n+1)}} \frac{1}{L_b}\lVert \bm{b} - C_i^{(n)}(b)\rVert^2_2 \le J^{(n)}
\end{split}
\end{equation}
where $\tilde{J}^{(n+1)}$ is the the quantization MSE after the vector clustering step at iteration $n+1$.

Next, during the codebook update step (\ref{eq:quantizers_update}) at iteration $n+1$, the per-cluster codebooks $\mathcal{C}^{(n)}$ are updated to $\mathcal{C}^{(n+1)}$ by invoking the Lloyd-Max algorithm \citep{Lloyd}. We know that for any given value distribution, the Lloyd-Max algorithm minimizes the quantization MSE. Therefore, for a given vector cluster $\mathcal{B}_i$ we obtain the following result:

\begin{equation}
    \frac{1}{|\mathcal{B}_{i}^{(n+1)}|}\sum_{\bm{b} \in \mathcal{B}_{i}^{(n+1)}} \frac{1}{L_b}\lVert \bm{b}- C_i^{(n+1)}(\bm{b})\rVert^2_2 \le \frac{1}{|\mathcal{B}_{i}^{(n+1)}|}\sum_{\bm{b} \in \mathcal{B}_{i}^{(n+1)}} \frac{1}{L_b}\lVert \bm{b}- C_i^{(n)}(\bm{b})\rVert^2_2
\end{equation}

The above equation states that quantizing the given block cluster $\mathcal{B}_i$ after updating the associated codebook from $C_i^{(n)}$ to $C_i^{(n+1)}$ results in lower quantization MSE. Since this is true for all the block clusters, we derive the following result: 
\begin{equation}
\begin{split}
\label{eq:mse_ineq_2}
     J^{(n+1)} &= \frac{1}{N_c} \sum_{i=1}^{N_c} \frac{1}{|\mathcal{B}_{i}^{(n+1)}|}\sum_{\bm{b} \in \mathcal{B}_{i}^{(n+1)}} \frac{1}{L_b}\lVert \bm{b}- C_i^{(n+1)}(\bm{b})\rVert^2_2  \le \tilde{J}^{(n+1)}   
\end{split}
\end{equation}

Following (\ref{eq:mse_ineq_1}) and (\ref{eq:mse_ineq_2}), we find that the quantization MSE is non-increasing for each iteration, that is, $J^{(1)} \ge J^{(2)} \ge J^{(3)} \ge \ldots \ge J^{(M)}$ where $M$ is the maximum number of iterations. 
%Therefore, we can say that if the algorithm converges, then it must be that it has converged to a local minimum. 
\hfill $\blacksquare$


\begin{figure}
    \begin{center}
    \includegraphics[width=0.5\textwidth]{sections//figures/mse_vs_iter.pdf}
    \end{center}
    \caption{\small NMSE vs iterations during LO-BCQ compared to other block quantization proposals}
    \label{fig:nmse_vs_iter}
\end{figure}

Figure \ref{fig:nmse_vs_iter} shows the empirical convergence of LO-BCQ across several block lengths and number of codebooks. Also, the MSE achieved by LO-BCQ is compared to baselines such as MXFP and VSQ. As shown, LO-BCQ converges to a lower MSE than the baselines. Further, we achieve better convergence for larger number of codebooks ($N_c$) and for a smaller block length ($L_b$), both of which increase the bitwidth of BCQ (see Eq \ref{eq:bitwidth_bcq}).


\subsection{Additional Accuracy Results}
%Table \ref{tab:lobcq_config} lists the various LOBCQ configurations and their corresponding bitwidths.
\begin{table}
\setlength{\tabcolsep}{4.75pt}
\begin{center}
\caption{\label{tab:lobcq_config} Various LO-BCQ configurations and their bitwidths.}
\begin{tabular}{|c||c|c|c|c||c|c||c|} 
\hline
 & \multicolumn{4}{|c||}{$L_b=8$} & \multicolumn{2}{|c||}{$L_b=4$} & $L_b=2$ \\
 \hline
 \backslashbox{$L_A$\kern-1em}{\kern-1em$N_c$} & 2 & 4 & 8 & 16 & 2 & 4 & 2 \\
 \hline
 64 & 4.25 & 4.375 & 4.5 & 4.625 & 4.375 & 4.625 & 4.625\\
 \hline
 32 & 4.375 & 4.5 & 4.625& 4.75 & 4.5 & 4.75 & 4.75 \\
 \hline
 16 & 4.625 & 4.75& 4.875 & 5 & 4.75 & 5 & 5 \\
 \hline
\end{tabular}
\end{center}
\end{table}

%\subsection{Perplexity achieved by various LO-BCQ configurations on Wikitext-103 dataset}

\begin{table} \centering
\begin{tabular}{|c||c|c|c|c||c|c||c|} 
\hline
 $L_b \rightarrow$& \multicolumn{4}{c||}{8} & \multicolumn{2}{c||}{4} & 2\\
 \hline
 \backslashbox{$L_A$\kern-1em}{\kern-1em$N_c$} & 2 & 4 & 8 & 16 & 2 & 4 & 2  \\
 %$N_c \rightarrow$ & 2 & 4 & 8 & 16 & 2 & 4 & 2 \\
 \hline
 \hline
 \multicolumn{8}{c}{GPT3-1.3B (FP32 PPL = 9.98)} \\ 
 \hline
 \hline
 64 & 10.40 & 10.23 & 10.17 & 10.15 &  10.28 & 10.18 & 10.19 \\
 \hline
 32 & 10.25 & 10.20 & 10.15 & 10.12 &  10.23 & 10.17 & 10.17 \\
 \hline
 16 & 10.22 & 10.16 & 10.10 & 10.09 &  10.21 & 10.14 & 10.16 \\
 \hline
  \hline
 \multicolumn{8}{c}{GPT3-8B (FP32 PPL = 7.38)} \\ 
 \hline
 \hline
 64 & 7.61 & 7.52 & 7.48 &  7.47 &  7.55 &  7.49 & 7.50 \\
 \hline
 32 & 7.52 & 7.50 & 7.46 &  7.45 &  7.52 &  7.48 & 7.48  \\
 \hline
 16 & 7.51 & 7.48 & 7.44 &  7.44 &  7.51 &  7.49 & 7.47  \\
 \hline
\end{tabular}
\caption{\label{tab:ppl_gpt3_abalation} Wikitext-103 perplexity across GPT3-1.3B and 8B models.}
\end{table}

\begin{table} \centering
\begin{tabular}{|c||c|c|c|c||} 
\hline
 $L_b \rightarrow$& \multicolumn{4}{c||}{8}\\
 \hline
 \backslashbox{$L_A$\kern-1em}{\kern-1em$N_c$} & 2 & 4 & 8 & 16 \\
 %$N_c \rightarrow$ & 2 & 4 & 8 & 16 & 2 & 4 & 2 \\
 \hline
 \hline
 \multicolumn{5}{|c|}{Llama2-7B (FP32 PPL = 5.06)} \\ 
 \hline
 \hline
 64 & 5.31 & 5.26 & 5.19 & 5.18  \\
 \hline
 32 & 5.23 & 5.25 & 5.18 & 5.15  \\
 \hline
 16 & 5.23 & 5.19 & 5.16 & 5.14  \\
 \hline
 \multicolumn{5}{|c|}{Nemotron4-15B (FP32 PPL = 5.87)} \\ 
 \hline
 \hline
 64  & 6.3 & 6.20 & 6.13 & 6.08  \\
 \hline
 32  & 6.24 & 6.12 & 6.07 & 6.03  \\
 \hline
 16  & 6.12 & 6.14 & 6.04 & 6.02  \\
 \hline
 \multicolumn{5}{|c|}{Nemotron4-340B (FP32 PPL = 3.48)} \\ 
 \hline
 \hline
 64 & 3.67 & 3.62 & 3.60 & 3.59 \\
 \hline
 32 & 3.63 & 3.61 & 3.59 & 3.56 \\
 \hline
 16 & 3.61 & 3.58 & 3.57 & 3.55 \\
 \hline
\end{tabular}
\caption{\label{tab:ppl_llama7B_nemo15B} Wikitext-103 perplexity compared to FP32 baseline in Llama2-7B and Nemotron4-15B, 340B models}
\end{table}

%\subsection{Perplexity achieved by various LO-BCQ configurations on MMLU dataset}


\begin{table} \centering
\begin{tabular}{|c||c|c|c|c||c|c|c|c|} 
\hline
 $L_b \rightarrow$& \multicolumn{4}{c||}{8} & \multicolumn{4}{c||}{8}\\
 \hline
 \backslashbox{$L_A$\kern-1em}{\kern-1em$N_c$} & 2 & 4 & 8 & 16 & 2 & 4 & 8 & 16  \\
 %$N_c \rightarrow$ & 2 & 4 & 8 & 16 & 2 & 4 & 2 \\
 \hline
 \hline
 \multicolumn{5}{|c|}{Llama2-7B (FP32 Accuracy = 45.8\%)} & \multicolumn{4}{|c|}{Llama2-70B (FP32 Accuracy = 69.12\%)} \\ 
 \hline
 \hline
 64 & 43.9 & 43.4 & 43.9 & 44.9 & 68.07 & 68.27 & 68.17 & 68.75 \\
 \hline
 32 & 44.5 & 43.8 & 44.9 & 44.5 & 68.37 & 68.51 & 68.35 & 68.27  \\
 \hline
 16 & 43.9 & 42.7 & 44.9 & 45 & 68.12 & 68.77 & 68.31 & 68.59  \\
 \hline
 \hline
 \multicolumn{5}{|c|}{GPT3-22B (FP32 Accuracy = 38.75\%)} & \multicolumn{4}{|c|}{Nemotron4-15B (FP32 Accuracy = 64.3\%)} \\ 
 \hline
 \hline
 64 & 36.71 & 38.85 & 38.13 & 38.92 & 63.17 & 62.36 & 63.72 & 64.09 \\
 \hline
 32 & 37.95 & 38.69 & 39.45 & 38.34 & 64.05 & 62.30 & 63.8 & 64.33  \\
 \hline
 16 & 38.88 & 38.80 & 38.31 & 38.92 & 63.22 & 63.51 & 63.93 & 64.43  \\
 \hline
\end{tabular}
\caption{\label{tab:mmlu_abalation} Accuracy on MMLU dataset across GPT3-22B, Llama2-7B, 70B and Nemotron4-15B models.}
\end{table}


%\subsection{Perplexity achieved by various LO-BCQ configurations on LM evaluation harness}

\begin{table} \centering
\begin{tabular}{|c||c|c|c|c||c|c|c|c|} 
\hline
 $L_b \rightarrow$& \multicolumn{4}{c||}{8} & \multicolumn{4}{c||}{8}\\
 \hline
 \backslashbox{$L_A$\kern-1em}{\kern-1em$N_c$} & 2 & 4 & 8 & 16 & 2 & 4 & 8 & 16  \\
 %$N_c \rightarrow$ & 2 & 4 & 8 & 16 & 2 & 4 & 2 \\
 \hline
 \hline
 \multicolumn{5}{|c|}{Race (FP32 Accuracy = 37.51\%)} & \multicolumn{4}{|c|}{Boolq (FP32 Accuracy = 64.62\%)} \\ 
 \hline
 \hline
 64 & 36.94 & 37.13 & 36.27 & 37.13 & 63.73 & 62.26 & 63.49 & 63.36 \\
 \hline
 32 & 37.03 & 36.36 & 36.08 & 37.03 & 62.54 & 63.51 & 63.49 & 63.55  \\
 \hline
 16 & 37.03 & 37.03 & 36.46 & 37.03 & 61.1 & 63.79 & 63.58 & 63.33  \\
 \hline
 \hline
 \multicolumn{5}{|c|}{Winogrande (FP32 Accuracy = 58.01\%)} & \multicolumn{4}{|c|}{Piqa (FP32 Accuracy = 74.21\%)} \\ 
 \hline
 \hline
 64 & 58.17 & 57.22 & 57.85 & 58.33 & 73.01 & 73.07 & 73.07 & 72.80 \\
 \hline
 32 & 59.12 & 58.09 & 57.85 & 58.41 & 73.01 & 73.94 & 72.74 & 73.18  \\
 \hline
 16 & 57.93 & 58.88 & 57.93 & 58.56 & 73.94 & 72.80 & 73.01 & 73.94  \\
 \hline
\end{tabular}
\caption{\label{tab:mmlu_abalation} Accuracy on LM evaluation harness tasks on GPT3-1.3B model.}
\end{table}

\begin{table} \centering
\begin{tabular}{|c||c|c|c|c||c|c|c|c|} 
\hline
 $L_b \rightarrow$& \multicolumn{4}{c||}{8} & \multicolumn{4}{c||}{8}\\
 \hline
 \backslashbox{$L_A$\kern-1em}{\kern-1em$N_c$} & 2 & 4 & 8 & 16 & 2 & 4 & 8 & 16  \\
 %$N_c \rightarrow$ & 2 & 4 & 8 & 16 & 2 & 4 & 2 \\
 \hline
 \hline
 \multicolumn{5}{|c|}{Race (FP32 Accuracy = 41.34\%)} & \multicolumn{4}{|c|}{Boolq (FP32 Accuracy = 68.32\%)} \\ 
 \hline
 \hline
 64 & 40.48 & 40.10 & 39.43 & 39.90 & 69.20 & 68.41 & 69.45 & 68.56 \\
 \hline
 32 & 39.52 & 39.52 & 40.77 & 39.62 & 68.32 & 67.43 & 68.17 & 69.30  \\
 \hline
 16 & 39.81 & 39.71 & 39.90 & 40.38 & 68.10 & 66.33 & 69.51 & 69.42  \\
 \hline
 \hline
 \multicolumn{5}{|c|}{Winogrande (FP32 Accuracy = 67.88\%)} & \multicolumn{4}{|c|}{Piqa (FP32 Accuracy = 78.78\%)} \\ 
 \hline
 \hline
 64 & 66.85 & 66.61 & 67.72 & 67.88 & 77.31 & 77.42 & 77.75 & 77.64 \\
 \hline
 32 & 67.25 & 67.72 & 67.72 & 67.00 & 77.31 & 77.04 & 77.80 & 77.37  \\
 \hline
 16 & 68.11 & 68.90 & 67.88 & 67.48 & 77.37 & 78.13 & 78.13 & 77.69  \\
 \hline
\end{tabular}
\caption{\label{tab:mmlu_abalation} Accuracy on LM evaluation harness tasks on GPT3-8B model.}
\end{table}

\begin{table} \centering
\begin{tabular}{|c||c|c|c|c||c|c|c|c|} 
\hline
 $L_b \rightarrow$& \multicolumn{4}{c||}{8} & \multicolumn{4}{c||}{8}\\
 \hline
 \backslashbox{$L_A$\kern-1em}{\kern-1em$N_c$} & 2 & 4 & 8 & 16 & 2 & 4 & 8 & 16  \\
 %$N_c \rightarrow$ & 2 & 4 & 8 & 16 & 2 & 4 & 2 \\
 \hline
 \hline
 \multicolumn{5}{|c|}{Race (FP32 Accuracy = 40.67\%)} & \multicolumn{4}{|c|}{Boolq (FP32 Accuracy = 76.54\%)} \\ 
 \hline
 \hline
 64 & 40.48 & 40.10 & 39.43 & 39.90 & 75.41 & 75.11 & 77.09 & 75.66 \\
 \hline
 32 & 39.52 & 39.52 & 40.77 & 39.62 & 76.02 & 76.02 & 75.96 & 75.35  \\
 \hline
 16 & 39.81 & 39.71 & 39.90 & 40.38 & 75.05 & 73.82 & 75.72 & 76.09  \\
 \hline
 \hline
 \multicolumn{5}{|c|}{Winogrande (FP32 Accuracy = 70.64\%)} & \multicolumn{4}{|c|}{Piqa (FP32 Accuracy = 79.16\%)} \\ 
 \hline
 \hline
 64 & 69.14 & 70.17 & 70.17 & 70.56 & 78.24 & 79.00 & 78.62 & 78.73 \\
 \hline
 32 & 70.96 & 69.69 & 71.27 & 69.30 & 78.56 & 79.49 & 79.16 & 78.89  \\
 \hline
 16 & 71.03 & 69.53 & 69.69 & 70.40 & 78.13 & 79.16 & 79.00 & 79.00  \\
 \hline
\end{tabular}
\caption{\label{tab:mmlu_abalation} Accuracy on LM evaluation harness tasks on GPT3-22B model.}
\end{table}

\begin{table} \centering
\begin{tabular}{|c||c|c|c|c||c|c|c|c|} 
\hline
 $L_b \rightarrow$& \multicolumn{4}{c||}{8} & \multicolumn{4}{c||}{8}\\
 \hline
 \backslashbox{$L_A$\kern-1em}{\kern-1em$N_c$} & 2 & 4 & 8 & 16 & 2 & 4 & 8 & 16  \\
 %$N_c \rightarrow$ & 2 & 4 & 8 & 16 & 2 & 4 & 2 \\
 \hline
 \hline
 \multicolumn{5}{|c|}{Race (FP32 Accuracy = 44.4\%)} & \multicolumn{4}{|c|}{Boolq (FP32 Accuracy = 79.29\%)} \\ 
 \hline
 \hline
 64 & 42.49 & 42.51 & 42.58 & 43.45 & 77.58 & 77.37 & 77.43 & 78.1 \\
 \hline
 32 & 43.35 & 42.49 & 43.64 & 43.73 & 77.86 & 75.32 & 77.28 & 77.86  \\
 \hline
 16 & 44.21 & 44.21 & 43.64 & 42.97 & 78.65 & 77 & 76.94 & 77.98  \\
 \hline
 \hline
 \multicolumn{5}{|c|}{Winogrande (FP32 Accuracy = 69.38\%)} & \multicolumn{4}{|c|}{Piqa (FP32 Accuracy = 78.07\%)} \\ 
 \hline
 \hline
 64 & 68.9 & 68.43 & 69.77 & 68.19 & 77.09 & 76.82 & 77.09 & 77.86 \\
 \hline
 32 & 69.38 & 68.51 & 68.82 & 68.90 & 78.07 & 76.71 & 78.07 & 77.86  \\
 \hline
 16 & 69.53 & 67.09 & 69.38 & 68.90 & 77.37 & 77.8 & 77.91 & 77.69  \\
 \hline
\end{tabular}
\caption{\label{tab:mmlu_abalation} Accuracy on LM evaluation harness tasks on Llama2-7B model.}
\end{table}

\begin{table} \centering
\begin{tabular}{|c||c|c|c|c||c|c|c|c|} 
\hline
 $L_b \rightarrow$& \multicolumn{4}{c||}{8} & \multicolumn{4}{c||}{8}\\
 \hline
 \backslashbox{$L_A$\kern-1em}{\kern-1em$N_c$} & 2 & 4 & 8 & 16 & 2 & 4 & 8 & 16  \\
 %$N_c \rightarrow$ & 2 & 4 & 8 & 16 & 2 & 4 & 2 \\
 \hline
 \hline
 \multicolumn{5}{|c|}{Race (FP32 Accuracy = 48.8\%)} & \multicolumn{4}{|c|}{Boolq (FP32 Accuracy = 85.23\%)} \\ 
 \hline
 \hline
 64 & 49.00 & 49.00 & 49.28 & 48.71 & 82.82 & 84.28 & 84.03 & 84.25 \\
 \hline
 32 & 49.57 & 48.52 & 48.33 & 49.28 & 83.85 & 84.46 & 84.31 & 84.93  \\
 \hline
 16 & 49.85 & 49.09 & 49.28 & 48.99 & 85.11 & 84.46 & 84.61 & 83.94  \\
 \hline
 \hline
 \multicolumn{5}{|c|}{Winogrande (FP32 Accuracy = 79.95\%)} & \multicolumn{4}{|c|}{Piqa (FP32 Accuracy = 81.56\%)} \\ 
 \hline
 \hline
 64 & 78.77 & 78.45 & 78.37 & 79.16 & 81.45 & 80.69 & 81.45 & 81.5 \\
 \hline
 32 & 78.45 & 79.01 & 78.69 & 80.66 & 81.56 & 80.58 & 81.18 & 81.34  \\
 \hline
 16 & 79.95 & 79.56 & 79.79 & 79.72 & 81.28 & 81.66 & 81.28 & 80.96  \\
 \hline
\end{tabular}
\caption{\label{tab:mmlu_abalation} Accuracy on LM evaluation harness tasks on Llama2-70B model.}
\end{table}

%\section{MSE Studies}
%\textcolor{red}{TODO}


\subsection{Number Formats and Quantization Method}
\label{subsec:numFormats_quantMethod}
\subsubsection{Integer Format}
An $n$-bit signed integer (INT) is typically represented with a 2s-complement format \citep{yao2022zeroquant,xiao2023smoothquant,dai2021vsq}, where the most significant bit denotes the sign.

\subsubsection{Floating Point Format}
An $n$-bit signed floating point (FP) number $x$ comprises of a 1-bit sign ($x_{\mathrm{sign}}$), $B_m$-bit mantissa ($x_{\mathrm{mant}}$) and $B_e$-bit exponent ($x_{\mathrm{exp}}$) such that $B_m+B_e=n-1$. The associated constant exponent bias ($E_{\mathrm{bias}}$) is computed as $(2^{{B_e}-1}-1)$. We denote this format as $E_{B_e}M_{B_m}$.  

\subsubsection{Quantization Scheme}
\label{subsec:quant_method}
A quantization scheme dictates how a given unquantized tensor is converted to its quantized representation. We consider FP formats for the purpose of illustration. Given an unquantized tensor $\bm{X}$ and an FP format $E_{B_e}M_{B_m}$, we first, we compute the quantization scale factor $s_X$ that maps the maximum absolute value of $\bm{X}$ to the maximum quantization level of the $E_{B_e}M_{B_m}$ format as follows:
\begin{align}
\label{eq:sf}
    s_X = \frac{\mathrm{max}(|\bm{X}|)}{\mathrm{max}(E_{B_e}M_{B_m})}
\end{align}
In the above equation, $|\cdot|$ denotes the absolute value function.

Next, we scale $\bm{X}$ by $s_X$ and quantize it to $\hat{\bm{X}}$ by rounding it to the nearest quantization level of $E_{B_e}M_{B_m}$ as:

\begin{align}
\label{eq:tensor_quant}
    \hat{\bm{X}} = \text{round-to-nearest}\left(\frac{\bm{X}}{s_X}, E_{B_e}M_{B_m}\right)
\end{align}

We perform dynamic max-scaled quantization \citep{wu2020integer}, where the scale factor $s$ for activations is dynamically computed during runtime.

\subsection{Vector Scaled Quantization}
\begin{wrapfigure}{r}{0.35\linewidth}
  \centering
  \includegraphics[width=\linewidth]{sections/figures/vsquant.jpg}
  \caption{\small Vectorwise decomposition for per-vector scaled quantization (VSQ \citep{dai2021vsq}).}
  \label{fig:vsquant}
\end{wrapfigure}
During VSQ \citep{dai2021vsq}, the operand tensors are decomposed into 1D vectors in a hardware friendly manner as shown in Figure \ref{fig:vsquant}. Since the decomposed tensors are used as operands in matrix multiplications during inference, it is beneficial to perform this decomposition along the reduction dimension of the multiplication. The vectorwise quantization is performed similar to tensorwise quantization described in Equations \ref{eq:sf} and \ref{eq:tensor_quant}, where a scale factor $s_v$ is required for each vector $\bm{v}$ that maps the maximum absolute value of that vector to the maximum quantization level. While smaller vector lengths can lead to larger accuracy gains, the associated memory and computational overheads due to the per-vector scale factors increases. To alleviate these overheads, VSQ \citep{dai2021vsq} proposed a second level quantization of the per-vector scale factors to unsigned integers, while MX \citep{rouhani2023shared} quantizes them to integer powers of 2 (denoted as $2^{INT}$).

\subsubsection{MX Format}
The MX format proposed in \citep{rouhani2023microscaling} introduces the concept of sub-block shifting. For every two scalar elements of $b$-bits each, there is a shared exponent bit. The value of this exponent bit is determined through an empirical analysis that targets minimizing quantization MSE. We note that the FP format $E_{1}M_{b}$ is strictly better than MX from an accuracy perspective since it allocates a dedicated exponent bit to each scalar as opposed to sharing it across two scalars. Therefore, we conservatively bound the accuracy of a $b+2$-bit signed MX format with that of a $E_{1}M_{b}$ format in our comparisons. For instance, we use E1M2 format as a proxy for MX4.

\begin{figure}
    \centering
    \includegraphics[width=1\linewidth]{sections//figures/BlockFormats.pdf}
    \caption{\small Comparing LO-BCQ to MX format.}
    \label{fig:block_formats}
\end{figure}

Figure \ref{fig:block_formats} compares our $4$-bit LO-BCQ block format to MX \citep{rouhani2023microscaling}. As shown, both LO-BCQ and MX decompose a given operand tensor into block arrays and each block array into blocks. Similar to MX, we find that per-block quantization ($L_b < L_A$) leads to better accuracy due to increased flexibility. While MX achieves this through per-block $1$-bit micro-scales, we associate a dedicated codebook to each block through a per-block codebook selector. Further, MX quantizes the per-block array scale-factor to E8M0 format without per-tensor scaling. In contrast during LO-BCQ, we find that per-tensor scaling combined with quantization of per-block array scale-factor to E4M3 format results in superior inference accuracy across models. 


\end{document}
