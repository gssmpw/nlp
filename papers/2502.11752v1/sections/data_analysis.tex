In this section, we statistically assess whether the object handover condition differs significantly from the two other conditions in gaze and EEG modalities, focusing on the period before movement initiation. Hand motion trajectories are excluded from this analysis since, by design, no hand movement occurs before the \emph{Go!} signal.

\subsection*{Gaze Data Analysis}
%\subsubsection*{Gaze}
As indicated by previous research, human-givers look at the handover location and the target location for object placement in object manipulation. Thus, we analyze the gaze behavior of the participants in our study by segmenting the regions viewed by participants into four zones: \emph{robot}, \emph{position B}, \emph{position C}, and \emph{other}. The ``robot" zone includes the robot's arm and torso, while the ``other" primarily includes the instruction screen. Table~\ref{tab:gaze_a}(a) and ~\ref{tab:gaze_b}(b) show a trend that differentiates the three human actions in terms of these zones for a particular participant (S8).
\begin{table}[h!]
    \centering
    \caption{Frequency (in \%) of gaze locations falling in one of four different regions denoted as Robot, Position B, Position C, and Other for different experiment conditions. \textbf{(a, c)} present values for times $t=-5$ to $t=0$ (presentation of \emph{Go!} signal and movement onset). \textbf{(b, d)} present values for times $t=-5$ to $t=3$ sec including both before and after movement onset. \textbf{(a, b)} correspond to a representative participant (S8) and \textbf{(c, d)} show median values across all participants. The maximum values (excluding Other) per condition are made bold.}
    \label{tab:participant8}
    
        \begin{minipage}{0.4\textwidth}
            \centering
            \resizebox{\textwidth}{!}{
            \begin{tabular}{cccc}
                \toprule
                & \textit{Solo} & \textit{Handover} & Joint \\
                \midrule
                Robot & 11.85 & 15.28 & 2.10\\
                \midrule
                Pos. B & 17.37 & 4.25 & 2.39\\
                \midrule
                Pos. C & 2.29 & 6.40 & 18.74\\
                \midrule
                Other & 68.50 & 74.07 & 76.77\\
                \bottomrule
            \end{tabular}}
            \label{tab:gaze_a}
            \textbf{(a)}
        \end{minipage}
        %&
        \begin{minipage}{0.4\textwidth}
            \centering
            \resizebox{\textwidth}{!}{
            \begin{tabular}{cccc}
                \toprule
                & \textit{Solo} & \textit{Handover} & \textit{Joint} \\
                \midrule
                Robot & 26.27 & \textbf{50.26} & 8.31 \\
                \midrule
                Pos. B & \textbf{37.95} & 17.51 & 15.70\\
                \midrule
                Pos. C & 2.39 & 3.64 & \textbf{50.3} \\
                \midrule
                Other & 45.04 & 48.31 & 22.90\\
                \bottomrule
            \end{tabular}}
            \label{tab:gaze_b}
            \textbf{(b)}
        \end{minipage}
        \\
        \vspace{10pt}
        \begin{minipage}{0.4\textwidth}
            \centering
            \resizebox{\textwidth}{!}{
            \begin{tabular}{cccc}
                \toprule
                & \textit{Solo} & \textit{Handover} & \textit{Joint} \\
                \midrule
                Robot & 9.68 & 15.48 & 3.69 \\
                \midrule
                Pos. B & 13.41 & 4.32 & 4.85\\
                \midrule
                Pos. C & 1.06 & 1.36 & 11.74 \\
                \midrule
                Other & 69.97 & 71.33 & 75.32\\
                \bottomrule
            \end{tabular}}
            \label{tab:gaze_all_before}
            \textbf{(c)}
        \end{minipage}
        %&
        \begin{minipage}{0.4\textwidth}
            \centering
            \resizebox{\textwidth}{!}{
            \begin{tabular}{cccc}
                \toprule
                & \textit{Solo} & \textit{Handover} & \textit{Joint} \\
                \midrule
                Robot & 17.45 & \textbf{33.61} & 16.65 \\
                \midrule
                Pos. B & \textbf{24.22} & 15.35 & 13.31\\
                \midrule
                Pos. C & 2.86 & 3.64 & \textbf{19.1} \\
                \midrule
                Other & 53.55 & 40.16 & 48.41\\
                \bottomrule
            \end{tabular}}
            \label{tab:gaze_all}
            \textbf{(d)}
        \end{minipage}
        \vspace{-7pt}
\end{table} 

For the handover action, the gaze percentage is higher at the robot. For non-handover actions, the gaze percentage is higher at the positions where participants would place the object (B for solo actions and C for joint actions). Before movement onset, as seen in Table~\ref{tab:gaze_a}(a), although the percentages suggest different behaviors for the three actions, these differences are not statistically significant. However, this trend becomes more pronounced after movement onset, as shown in Table~\ref{tab:gaze_b}(b) via the values in bold. Table \ref{tab:gaze_all_before}(c) and \ref{tab:gaze_all}(d) show the median percentages for all participants which also follow the aforementioned trend differentiating the three human actions. The existence of this trend before movement onset suggests that there is indeed some information in the gaze data that can be used to predict the type of upcoming human action before the motion begins. Since this trend becomes more pronounced after movement onset, it hints that the amount of information available in the gaze data increases as the action unfolds. This suggests that while initial predictions can be made using gaze data prior to the human motion onset, more accurate predictions can be made by utilizing gaze data as the human motion begins and progresses.
The percentages for individual participants are provided in the supplementary materials (Supplementary Tables 2 and 3).


\subsection*{EEG Data Analysis}
The motor cortex of the brain is in charge of movement-related functions such as movement planning and execution, which corresponds to the centrally located EEG channels in the 10-20 system \cite{yahya19}. Movement-related cortical potentials (MRCPs) and sensory-motor rhythms (SMRs) are key EEG characteristics during voluntary movement preparation and planning observed over the motor cortex \cite{sumeyra22}. 

MRCPs are event-related potentials (ERPs) that begin with a slow negative shift about two seconds before movement onset, turning positive with the movement \cite{olsen21}. The grand average ERPs for 13 subjects, comparing handover vs. non-handover conditions, are shown in Fig.~\ref{fig:ERP} for central channels C3, C4, Cz, CP1, CP2, FC1, and FC2. A negative trend starts around -1.8 seconds for both conditions, peaking negatively at -1.7 seconds, followed by a second negative peak at -0.7 seconds (Negative Slope) and finally a positive trend begins just before movement execution, peaking at 0.1 seconds (motor potential) which are in-line with what is observed in the literature \cite{olsen21}. MRCP peak and latency parameters are reported to differ with various movement types, task speed, whether the task is self-paced or cued, the uncertainty level, and the existence of force \cite{brunia2003cnv, nascimento2006movement, rektor2003intracerebral}.

 \begin{figure*}[t]
    \centering
        \subfigure[]
   { \includegraphics[width=0.8\linewidth]{figures/ERP_grandAvg_hand_nohand_Avg7Channels.png} \label{fig:ERP} }
    \subfigure[]{\includegraphics[width=0.8\linewidth]{figures/GrandAvgAlphaBetaGammaCz.png}\label{fig:psd}}
        \caption{Grand average over 13 subjects for the two conditions, Blue: Handover, Orange: Non-Handover \textbf{(a)} ERP gathered from the motor cortex (Channels: C3, C4, Cz, CP1, CP2, FC1, FC2) \textbf{(b)} ERDS of mu, beta and gamma Channel Cz, with the variance highlighted by the shaded region.}
   \label{fig:erp_psd}
 \end{figure*}

SMRs are described as oscillations in the mu (8-12 Hz), beta (13-30 Hz), and gamma (31-200 Hz) frequency bands, seen either as a decrease in power called event-related desynchronization (ERD) or an increase in power called event-related synchronization (ERS). The mu, beta, and gamma band powers over the motor cortex decrease approximately two seconds before the onset with the movement preparation corresponding to ERD. The grand average of the relevant frequency band powers for channel Cz is shown in Fig.~\ref{fig:psd} calculated from $n=13$ subjects. It must be noted that since the EEG signals are band-pass filtered between 1-40 Hz during pre-processing, the gamma band contains frequencies between 30 and 40 Hz. ERD is observed between -2 and 0 seconds for both conditions in the mu, beta, and gamma bands as expected. A paired t-test showed that the band powers are significantly different between the handover and non-handover trials for most of the subjects (Supplementary Table 4).
The mu band is significantly different between handover and non-handover for all subjects except for S2 and S9, with a p-value $<0.05$. The beta band is significantly different for all 13 subjects in this time interval, meanwhile, the gamma band is significantly different for all subjects but one, S11, between the two conditions.
