%% 
%% Copyright 2007-2025 Elsevier Ltd
%% 
%% This file is part of the 'Elsarticle Bundle'.
%% ---------------------------------------------
%% 
%% It may be distributed under the conditions of the LaTeX Project Public
%% License, either version 1.3 of this license or (at your option) any
%% later version.  The latest version of this license is in
%%    http://www.latex-project.org/lppl.txt
%% and version 1.3 or later is part of all distributions of LaTeX
%% version 1999/12/01 or later.
%% 
%% The list of all files belonging to the 'Elsarticle Bundle' is
%% given in the file `manifest.txt'.
%% 
%% Template article for Elsevier's document class `elsarticle'
%% with harvard style bibliographic references

\documentclass[preprint,12pt]{elsarticle}

%% Use the option review to obtain double line spacing
%% \documentclass[preprint,review,12pt]{elsarticle}

%% Use the options 1p,twocolumn; 3p; 3p,twocolumn; 5p; or 5p,twocolumn
%% for a journal layout:
%% \documentclass[final,1p,times]{elsarticle}
%% \documentclass[final,1p,times,twocolumn]{elsarticle}
%% \documentclass[final,3p,times]{elsarticle}
%% \documentclass[final,3p,times,twocolumn]{elsarticle}
%% \documentclass[final,5p,times]{elsarticle}
%% \documentclass[final,5p,times,twocolumn]{elsarticle}

%% For including figures, graphicx.sty has been loaded in
%% elsarticle.cls. If you prefer to use the old commands
%% please give \usepackage{epsfig}

%% The amssymb package provides various useful mathematical symbols
\usepackage{amssymb}
%% The amsmath package provides various useful equation environments.
%\usepackage{amsmath}
%% The amsthm package provides extended theorem environments
%% \usepackage{amsthm}

\usepackage{amsmath,amsfonts}
\usepackage{algorithmic}
\usepackage{algorithm}
\usepackage{array}
% \usepackage[caption=false,font=normalsize,labelfont=sf,textfont=sf]{subfig}
\usepackage{textcomp}
\usepackage{stfloats}
\usepackage{url}
\usepackage{verbatim}
\usepackage{graphicx}
%\usepackage{cite}
% updated with editorial comments 8/9/2021
% extra packages
\usepackage{caption}
\usepackage{hyperref}
\usepackage{enumitem}
\usepackage{subfigure}
\usepackage{ragged2e}
\usepackage{booktabs}
\usepackage{multirow}
\usepackage{soul}
\usepackage{xcolor}

\sethlcolor{yellow} % Set your desired color here


\journal{Robotics and Autonomous Systems}
%% The lineno packages adds line numbers. Start line numbering with
%% \begin{linenumbers}, end it with \end{linenumbers}. Or switch it on
%% for the whole article with \linenumbers.
%% \usepackage{lineno}

\begin{document}

\begin{frontmatter}

%% Title, authors and addresses

%% use the tnoteref command within \title for footnotes;
%% use the tnotetext command for theassociated footnote;
%% use the fnref command within \author or \affiliation for footnotes;
%% use the fntext command for theassociated footnote;
%% use the corref command within \author for corresponding author footnotes;
%% use the cortext command for theassociated footnote;
%% use the ead command for the email address,
%% and the form \ead[url] for the home page:
%% \title{Title\tnoteref{label1}}
%% \tnotetext[label1]{}
%% \author{Name\corref{cor1}\fnref{label2}}
%% \ead{email address}
%% \ead[url]{home page}
%% \fntext[label2]{}
%% \cortext[cor1]{}
%% \affiliation{organization={},
%%             addressline={},
%%             city={},
%%             postcode={},
%%             state={},
%%             country={}}
%% \fntext[label3]{}

\title{Early Detection of Human Handover Intentions in Human-Robot Collaboration: \\ Comparing EEG, Gaze, and Hand Motion} %% Article title

%=\author
%{Parag Khanna,$^{1,\ast,+}$ Nona Rajabi,$^{1,+}$ Sumeyra U. Demir Kanik,$^{2}$ Danica Kragic,$^{1}$ \\Mårten Björkman,$^{1}$ Christian Smith$^{1}$\\
%\\
%\normalsize{$^{1}$KTH Royal Institute of Technology, Stockholm, Sweden}\\
%\normalsize{$^{2}$Ericsson Research, Stockholm, Sweden}\\
%\\
%\normalsize{$^{+}$These authors contributed equally to this work}\\
%\normalsize{$^\ast$To whom correspondence should be addressed; E-mail: paragk@kth.se.}
%}

%% use optional labels to link authors explicitly to addresses:
\author[label1]{Parag Khanna \fnref{label3,label4}}
\author[label1]{Nona Rajabi \fnref{label3}}
\author[label2]{Sumeyra U. Demir Kanik}
\author[label1]{Danica Kragic}
\author[label1]{Mårten Björkman}
\author[label1]{Christian Smith}
\fntext[label3]{These authors contributed equally to this work}
\fntext[label4]{To whom correspondence should be addressed; E-mail: paragk@kth.se.}

\affiliation[label1]{organization={KTH Royal Institute of Technology, Stockholm},
             city={Stockholm},
             country={Sweden}}

\affiliation[label2]{organization={Ericsson Research},
             city={Stockholm},
             country={Sweden}}
             

%\author{Parag Khanna} %% Author name

%% Author affiliation
%\affiliation{organization={RPL,KTH Royal Institute of Technology, Stockholm, Sweden},%Department and Organization
%            addressline={}, 
%           city={Stockholm},
%            postcode={}, 
%            state={},
%            country={Sweden}}
%%Graphical abstract
\begin{comment}
    \begin{graphicalabstract}
\vspace{50pt}
\includegraphics[width=0.98\textwidth]{figures/graphical_abstract.pdf}
\end{graphicalabstract}
%%Research highlights
\begin{highlights}
\item Demonstrates the potential of non-movement-based physiological signals for improving HRC applications.
\item Comprehensive comparison of electroencephalography (EEG) brain signals, gaze, and hand motion for predicting handover intent.
\item Gaze data demonstrates the fastest and most accurate detection of handover intentions.
\item A multimodal approach significantly enhances the performance of weaker modalities in detection.
\item Introduces a novel dataset with synchronized EEG, gaze, and hand motion for handovers in HRC context.
\end{highlights}
\end{comment}


%% Abstract
\begin{abstract}
%% Text of abstract
Human-robot collaboration (HRC) relies on accurate and timely recognition of human intentions to ensure seamless interactions. 
Among common HRC tasks, human-to-robot object handovers have been studied extensively for planning the robot's actions during object reception, assuming the human intention for object handover. However, distinguishing handover intentions from other actions has received limited attention. Most research on handovers has focused on visually detecting motion trajectories, which often results in delays or false detections when trajectories overlap. 
This paper investigates whether human intentions for object handovers are reflected in non-movement-based physiological signals. We conduct a multimodal analysis comparing three data modalities: electroencephalogram (EEG), gaze, and hand-motion signals. Our study aims to distinguish between handover-intended human motions and non-handover motions in an HRC setting, evaluating each modality's performance in predicting and classifying these actions before and after human movement initiation.
We develop and evaluate human intention detectors based on these modalities, comparing their accuracy and timing in identifying handover intentions. To the best of our knowledge, this is the first study to systematically develop and test intention detectors across multiple modalities within the same experimental context of human-robot handovers. 
Our analysis reveals that handover intention can be detected from all three modalities. Nevertheless, gaze signals are the earliest as well as the most accurate to classify the motion as intended for handover or non-handover.
\end{abstract}




%% Keywords
\begin{keyword}
Human-Robot Collaboration (HRC)
\sep Human-Robot Handovers
\sep EEG
\sep Gaze
\sep Motion Analysis
%% keywords here, in the form: keyword \sep keyword
%% PACS codes here, in the form: \PACS code \sep code
%% MSC codes here, in the form: \MSC code \sep code
%% or \MSC[2008] code \sep code (2000 is the default)
\end{keyword}
\end{frontmatter}

%% Add \usepackage{lineno} before \begin{document} and uncomment 
%% following line to enable line numbers
%% \linenumbers

%% main text
%%

%% Use \section commands to start a section

\section{Introduction}
\section{Introduction}
\label{sec:introduction}
The business processes of organizations are experiencing ever-increasing complexity due to the large amount of data, high number of users, and high-tech devices involved \cite{martin2021pmopportunitieschallenges, beerepoot2023biggestbpmproblems}. This complexity may cause business processes to deviate from normal control flow due to unforeseen and disruptive anomalies \cite{adams2023proceddsriftdetection}. These control-flow anomalies manifest as unknown, skipped, and wrongly-ordered activities in the traces of event logs monitored from the execution of business processes \cite{ko2023adsystematicreview}. For the sake of clarity, let us consider an illustrative example of such anomalies. Figure \ref{FP_ANOMALIES} shows a so-called event log footprint, which captures the control flow relations of four activities of a hypothetical event log. In particular, this footprint captures the control-flow relations between activities \texttt{a}, \texttt{b}, \texttt{c} and \texttt{d}. These are the causal ($\rightarrow$) relation, concurrent ($\parallel$) relation, and other ($\#$) relations such as exclusivity or non-local dependency \cite{aalst2022pmhandbook}. In addition, on the right are six traces, of which five exhibit skipped, wrongly-ordered and unknown control-flow anomalies. For example, $\langle$\texttt{a b d}$\rangle$ has a skipped activity, which is \texttt{c}. Because of this skipped activity, the control-flow relation \texttt{b}$\,\#\,$\texttt{d} is violated, since \texttt{d} directly follows \texttt{b} in the anomalous trace.
\begin{figure}[!t]
\centering
\includegraphics[width=0.9\columnwidth]{images/FP_ANOMALIES.png}
\caption{An example event log footprint with six traces, of which five exhibit control-flow anomalies.}
\label{FP_ANOMALIES}
\end{figure}

\subsection{Control-flow anomaly detection}
Control-flow anomaly detection techniques aim to characterize the normal control flow from event logs and verify whether these deviations occur in new event logs \cite{ko2023adsystematicreview}. To develop control-flow anomaly detection techniques, \revision{process mining} has seen widespread adoption owing to process discovery and \revision{conformance checking}. On the one hand, process discovery is a set of algorithms that encode control-flow relations as a set of model elements and constraints according to a given modeling formalism \cite{aalst2022pmhandbook}; hereafter, we refer to the Petri net, a widespread modeling formalism. On the other hand, \revision{conformance checking} is an explainable set of algorithms that allows linking any deviations with the reference Petri net and providing the fitness measure, namely a measure of how much the Petri net fits the new event log \cite{aalst2022pmhandbook}. Many control-flow anomaly detection techniques based on \revision{conformance checking} (hereafter, \revision{conformance checking}-based techniques) use the fitness measure to determine whether an event log is anomalous \cite{bezerra2009pmad, bezerra2013adlogspais, myers2018icsadpm, pecchia2020applicationfailuresanalysispm}. 

The scientific literature also includes many \revision{conformance checking}-independent techniques for control-flow anomaly detection that combine specific types of trace encodings with machine/deep learning \cite{ko2023adsystematicreview, tavares2023pmtraceencoding}. Whereas these techniques are very effective, their explainability is challenging due to both the type of trace encoding employed and the machine/deep learning model used \cite{rawal2022trustworthyaiadvances,li2023explainablead}. Hence, in the following, we focus on the shortcomings of \revision{conformance checking}-based techniques to investigate whether it is possible to support the development of competitive control-flow anomaly detection techniques while maintaining the explainable nature of \revision{conformance checking}.
\begin{figure}[!t]
\centering
\includegraphics[width=\columnwidth]{images/HIGH_LEVEL_VIEW.png}
\caption{A high-level view of the proposed framework for combining \revision{process mining}-based feature extraction with dimensionality reduction for control-flow anomaly detection.}
\label{HIGH_LEVEL_VIEW}
\end{figure}

\subsection{Shortcomings of \revision{conformance checking}-based techniques}
Unfortunately, the detection effectiveness of \revision{conformance checking}-based techniques is affected by noisy data and low-quality Petri nets, which may be due to human errors in the modeling process or representational bias of process discovery algorithms \cite{bezerra2013adlogspais, pecchia2020applicationfailuresanalysispm, aalst2016pm}. Specifically, on the one hand, noisy data may introduce infrequent and deceptive control-flow relations that may result in inconsistent fitness measures, whereas, on the other hand, checking event logs against a low-quality Petri net could lead to an unreliable distribution of fitness measures. Nonetheless, such Petri nets can still be used as references to obtain insightful information for \revision{process mining}-based feature extraction, supporting the development of competitive and explainable \revision{conformance checking}-based techniques for control-flow anomaly detection despite the problems above. For example, a few works outline that token-based \revision{conformance checking} can be used for \revision{process mining}-based feature extraction to build tabular data and develop effective \revision{conformance checking}-based techniques for control-flow anomaly detection \cite{singh2022lapmsh, debenedictis2023dtadiiot}. However, to the best of our knowledge, the scientific literature lacks a structured proposal for \revision{process mining}-based feature extraction using the state-of-the-art \revision{conformance checking} variant, namely alignment-based \revision{conformance checking}.

\subsection{Contributions}
We propose a novel \revision{process mining}-based feature extraction approach with alignment-based \revision{conformance checking}. This variant aligns the deviating control flow with a reference Petri net; the resulting alignment can be inspected to extract additional statistics such as the number of times a given activity caused mismatches \cite{aalst2022pmhandbook}. We integrate this approach into a flexible and explainable framework for developing techniques for control-flow anomaly detection. The framework combines \revision{process mining}-based feature extraction and dimensionality reduction to handle high-dimensional feature sets, achieve detection effectiveness, and support explainability. Notably, in addition to our proposed \revision{process mining}-based feature extraction approach, the framework allows employing other approaches, enabling a fair comparison of multiple \revision{conformance checking}-based and \revision{conformance checking}-independent techniques for control-flow anomaly detection. Figure \ref{HIGH_LEVEL_VIEW} shows a high-level view of the framework. Business processes are monitored, and event logs obtained from the database of information systems. Subsequently, \revision{process mining}-based feature extraction is applied to these event logs and tabular data input to dimensionality reduction to identify control-flow anomalies. We apply several \revision{conformance checking}-based and \revision{conformance checking}-independent framework techniques to publicly available datasets, simulated data of a case study from railways, and real-world data of a case study from healthcare. We show that the framework techniques implementing our approach outperform the baseline \revision{conformance checking}-based techniques while maintaining the explainable nature of \revision{conformance checking}.

In summary, the contributions of this paper are as follows.
\begin{itemize}
    \item{
        A novel \revision{process mining}-based feature extraction approach to support the development of competitive and explainable \revision{conformance checking}-based techniques for control-flow anomaly detection.
    }
    \item{
        A flexible and explainable framework for developing techniques for control-flow anomaly detection using \revision{process mining}-based feature extraction and dimensionality reduction.
    }
    \item{
        Application to synthetic and real-world datasets of several \revision{conformance checking}-based and \revision{conformance checking}-independent framework techniques, evaluating their detection effectiveness and explainability.
    }
\end{itemize}

The rest of the paper is organized as follows.
\begin{itemize}
    \item Section \ref{sec:related_work} reviews the existing techniques for control-flow anomaly detection, categorizing them into \revision{conformance checking}-based and \revision{conformance checking}-independent techniques.
    \item Section \ref{sec:abccfe} provides the preliminaries of \revision{process mining} to establish the notation used throughout the paper, and delves into the details of the proposed \revision{process mining}-based feature extraction approach with alignment-based \revision{conformance checking}.
    \item Section \ref{sec:framework} describes the framework for developing \revision{conformance checking}-based and \revision{conformance checking}-independent techniques for control-flow anomaly detection that combine \revision{process mining}-based feature extraction and dimensionality reduction.
    \item Section \ref{sec:evaluation} presents the experiments conducted with multiple framework and baseline techniques using data from publicly available datasets and case studies.
    \item Section \ref{sec:conclusions} draws the conclusions and presents future work.
\end{itemize}

\section{Experimental Design}
\begin{figure}[t]
    \centering
    \includegraphics[width=\linewidth]{figures/experimental_setup.pdf}
    \caption{\justifying Experimental Setup: The participant sits across a table from the Baxter robot, wearing the EEG cap and Tobii eye tracker glasses. An Azure Kinect RGBD camera is installed on the robot's torso.}
    \label{fig:experimental setup}
\end{figure}

We designed a within-subject experiment involving human participants interacting with a Baxter robot~\cite{baxter-Doe:2024:Online}. Each participant sat across from the robot, wearing an EEG cap and Tobii Pro Glasses 2 eye-tracker glasses. An RGBD camera, Microsoft Azure,  mounted on the robot's torso, tracked the participant's hand motion trajectory. The experimental setup is depicted in Fig.~\ref{fig:experimental setup}. From the participant's point of view, there are three marked positions on the table: A, B, and C in a straight line. In each trial, a green sponge is moved from its initial position A, near the participant, to the final position B, near the robot. The movement is executed through one of three types of human actions (also visualized in Supplementary Fig. 1):
\begin{enumerate}[label=(\alph*)]
    \item \textbf{Solo Action:} This action is performed solely by the participant. \textit{The participant picks the object from A} and places it at B.
    \item \textbf{Handover - Human Giver:} This action involves a human-to-robot handover. \textit{The participant picks the object from A} and hands it to the robot over C. The robot then places the object at B. 
    \item \textbf{Joint Action:} This action is a non-handover collaborative action between the participant and the robot. \textit{The participant picks the object from A} and places it at C. Then, the robot picks the object from C and places it at B. 
\end{enumerate}
\begin{figure}[h]
    \centering
    \includegraphics[width=0.9\linewidth]{figures/gui_instructions.pdf}
    \caption{Experimental conditions and timing diagram, Adapted from Fig. 2 in \cite{nonaEEG}. Visual instructions for the three task conditions: \textbf{(a)} solo, \textbf{(b)} handover, and \textbf{(c)} joint actions. \textbf{(d)} Timing diagram of the experiment with $t=0$ at the \emph{Go!} signal.}
    \label{fig:gui-ins}
\end{figure}

\noindent The initial actions, shown in italics, are identical across all three tasks. Further, in both the \emph{Handover} and \emph{Joint} actions, the robot arm was controlled by the experimenter, though participants were unaware of this. 

In our design, joint action specifically refers to a collaborative action with the robot that is not a direct handover.
Each trial corresponds to one of three actions conveyed to the participant via instructions on a screen placed on the table (Fig.~\ref{fig:experimental setup}). There were 90 trials in total, with an equal number, i.e. 30, of each task condition performed in a random order. Fig.~\ref{fig:gui-ins} illustrates the timeline of instructions for each trial. The first image is a cue specifying the action type, displayed for 5 seconds with a countdown from 5 to 1 second. The next image shows a green cross and the word "Go!", signaling the participant to begin the action. For solo actions, this image is displayed for 8 seconds, while for handover and joint actions, it is displayed for 16 seconds to allow sufficient time for task completion based on pilot trials. The final image is a red cross with the word "Rest!", indicating rest time. During this period, the participant returns the object from B to A and presses the space bar on a keyboard to mark the end of the trial and the start of the next one.

\section{Dataset Preparation}
\section{Dataset}
\label{sec:dataset}

\subsection{Data Collection}

To analyze political discussions on Discord, we followed the methodology in \cite{singh2024Cross-Platform}, collecting messages from politically-oriented public servers in compliance with Discord's platform policies.

Using Discord's Discovery feature, we employed a web scraper to extract server invitation links, names, and descriptions, focusing on public servers accessible without participation. Invitation links were used to access data via the Discord API. To ensure relevance, we filtered servers using keywords related to the 2024 U.S. elections (e.g., Trump, Kamala, MAGA), as outlined in \cite{balasubramanian2024publicdatasettrackingsocial}. This resulted in 302 server links, further narrowed to 81 English-speaking, politics-focused servers based on their names and descriptions.

Public messages were retrieved from these servers using the Discord API, collecting metadata such as \textit{content}, \textit{user ID}, \textit{username}, \textit{timestamp}, \textit{bot flag}, \textit{mentions}, and \textit{interactions}. Through this process, we gathered \textbf{33,373,229 messages} from \textbf{82,109 users} across \textbf{81 servers}, including \textbf{1,912,750 messages} from \textbf{633 bots}. Data collection occurred between November 13th and 15th, covering messages sent from January 1st to November 12th, just after the 2024 U.S. election.

\subsection{Characterizing the Political Spectrum}
\label{sec:timeline}

A key aspect of our research is distinguishing between Republican- and Democratic-aligned Discord servers. To categorize their political alignment, we relied on server names and self-descriptions, which often include rules, community guidelines, and references to key ideologies or figures. Each server's name and description were manually reviewed based on predefined, objective criteria, focusing on explicit political themes or mentions of prominent figures. This process allowed us to classify servers into three categories, ensuring a systematic and unbiased alignment determination.

\begin{itemize}
    \item \textbf{Republican-aligned}: Servers referencing Republican and right-wing and ideologies, movements, or figures (e.g., MAGA, Conservative, Traditional, Trump).  
    \item \textbf{Democratic-aligned}: Servers mentioning Democratic and left-wing ideologies, movements, or figures (e.g., Progressive, Liberal, Socialist, Biden, Kamala).  
    \item \textbf{Unaligned}: Servers with no defined spectrum and ideologies or opened to general political debate from all orientations.
\end{itemize}

To ensure the reliability and consistency of our classification, three independent reviewers assessed the classification following the specified set of criteria. The inter-rater agreement of their classifications was evaluated using Fleiss' Kappa \cite{fleiss1971measuring}, with a resulting Kappa value of \( 0.8191 \), indicating an almost perfect agreement among the reviewers. Disagreements were resolved by adopting the majority classification, as there were no instances where a server received different classifications from all three reviewers. This process guaranteed the consistency and accuracy of the final categorization.

Through this process, we identified \textbf{7 Republican-aligned servers}, \textbf{9 Democratic-aligned servers}, and \textbf{65 unaligned servers}.

Table \ref{tab:statistics} shows the statistics of the collected data. Notably, while Democratic- and Republican-aligned servers had a comparable number of user messages, users in the latter servers were significantly more active, posting more than double the number of messages per user compared to their Democratic counterparts. 
This suggests that, in our sample, Democratic-aligned servers attract more users, but these users were less engaged in text-based discussions. Additionally, around 10\% of the messages across all server categories were posted by bots. 

\subsection{Temporal Data} 

Throughout this paper, we refer to the election candidates using the names adopted by their respective campaigns: \textit{Kamala}, \textit{Biden}, and \textit{Trump}. To examine how the content of text messages evolves based on the political alignment of servers, we divided the 2024 election year into three periods: \textbf{Biden vs Trump} (January 1 to July 21), \textbf{Kamala vs Trump} (July 21 to September 20), and the \textbf{Voting Period} (after September 20). These periods reflect key phases of the election: the early campaign dominated by Biden and Trump, the shift in dynamics with Kamala Harris replacing Joe Biden as the Democratic candidate, and the final voting stage focused on electoral outcomes and their implications. This segmentation enables an analysis of how discourse responds to pivotal electoral moments.

Figure \ref{fig:line-plot} illustrates the distribution of messages over time, highlighting trends in total messages volume and mentions of each candidate. Prior to Biden's withdrawal on July 21, mentions of Biden and Trump were relatively balanced. However, following Kamala's entry into the race, mentions of Trump surged significantly, a trend further amplified by an assassination attempt on him, solidifying his dominance in the discourse. The only instance where Trump’s mentions were exceeded occurred during the first debate, as concerns about Biden’s age and cognitive abilities temporarily shifted the focus. In the final stages of the election, mentions of all three candidates rose, with Trump’s mentions peaking as he emerged as the victor.
\section{Data Analysis}
In this section, we statistically assess whether the object handover condition differs significantly from the two other conditions in gaze and EEG modalities, focusing on the period before movement initiation. Hand motion trajectories are excluded from this analysis since, by design, no hand movement occurs before the \emph{Go!} signal.

\subsection*{Gaze Data Analysis}
%\subsubsection*{Gaze}
As indicated by previous research, human-givers look at the handover location and the target location for object placement in object manipulation. Thus, we analyze the gaze behavior of the participants in our study by segmenting the regions viewed by participants into four zones: \emph{robot}, \emph{position B}, \emph{position C}, and \emph{other}. The ``robot" zone includes the robot's arm and torso, while the ``other" primarily includes the instruction screen. Table~\ref{tab:gaze_a}(a) and ~\ref{tab:gaze_b}(b) show a trend that differentiates the three human actions in terms of these zones for a particular participant (S8).
\begin{table}[h!]
    \centering
    \caption{Frequency (in \%) of gaze locations falling in one of four different regions denoted as Robot, Position B, Position C, and Other for different experiment conditions. \textbf{(a, c)} present values for times $t=-5$ to $t=0$ (presentation of \emph{Go!} signal and movement onset). \textbf{(b, d)} present values for times $t=-5$ to $t=3$ sec including both before and after movement onset. \textbf{(a, b)} correspond to a representative participant (S8) and \textbf{(c, d)} show median values across all participants. The maximum values (excluding Other) per condition are made bold.}
    \label{tab:participant8}
    
        \begin{minipage}{0.4\textwidth}
            \centering
            \resizebox{\textwidth}{!}{
            \begin{tabular}{cccc}
                \toprule
                & \textit{Solo} & \textit{Handover} & Joint \\
                \midrule
                Robot & 11.85 & 15.28 & 2.10\\
                \midrule
                Pos. B & 17.37 & 4.25 & 2.39\\
                \midrule
                Pos. C & 2.29 & 6.40 & 18.74\\
                \midrule
                Other & 68.50 & 74.07 & 76.77\\
                \bottomrule
            \end{tabular}}
            \label{tab:gaze_a}
            \textbf{(a)}
        \end{minipage}
        %&
        \begin{minipage}{0.4\textwidth}
            \centering
            \resizebox{\textwidth}{!}{
            \begin{tabular}{cccc}
                \toprule
                & \textit{Solo} & \textit{Handover} & \textit{Joint} \\
                \midrule
                Robot & 26.27 & \textbf{50.26} & 8.31 \\
                \midrule
                Pos. B & \textbf{37.95} & 17.51 & 15.70\\
                \midrule
                Pos. C & 2.39 & 3.64 & \textbf{50.3} \\
                \midrule
                Other & 45.04 & 48.31 & 22.90\\
                \bottomrule
            \end{tabular}}
            \label{tab:gaze_b}
            \textbf{(b)}
        \end{minipage}
        \\
        \vspace{10pt}
        \begin{minipage}{0.4\textwidth}
            \centering
            \resizebox{\textwidth}{!}{
            \begin{tabular}{cccc}
                \toprule
                & \textit{Solo} & \textit{Handover} & \textit{Joint} \\
                \midrule
                Robot & 9.68 & 15.48 & 3.69 \\
                \midrule
                Pos. B & 13.41 & 4.32 & 4.85\\
                \midrule
                Pos. C & 1.06 & 1.36 & 11.74 \\
                \midrule
                Other & 69.97 & 71.33 & 75.32\\
                \bottomrule
            \end{tabular}}
            \label{tab:gaze_all_before}
            \textbf{(c)}
        \end{minipage}
        %&
        \begin{minipage}{0.4\textwidth}
            \centering
            \resizebox{\textwidth}{!}{
            \begin{tabular}{cccc}
                \toprule
                & \textit{Solo} & \textit{Handover} & \textit{Joint} \\
                \midrule
                Robot & 17.45 & \textbf{33.61} & 16.65 \\
                \midrule
                Pos. B & \textbf{24.22} & 15.35 & 13.31\\
                \midrule
                Pos. C & 2.86 & 3.64 & \textbf{19.1} \\
                \midrule
                Other & 53.55 & 40.16 & 48.41\\
                \bottomrule
            \end{tabular}}
            \label{tab:gaze_all}
            \textbf{(d)}
        \end{minipage}
        \vspace{-7pt}
\end{table} 

For the handover action, the gaze percentage is higher at the robot. For non-handover actions, the gaze percentage is higher at the positions where participants would place the object (B for solo actions and C for joint actions). Before movement onset, as seen in Table~\ref{tab:gaze_a}(a), although the percentages suggest different behaviors for the three actions, these differences are not statistically significant. However, this trend becomes more pronounced after movement onset, as shown in Table~\ref{tab:gaze_b}(b) via the values in bold. Table \ref{tab:gaze_all_before}(c) and \ref{tab:gaze_all}(d) show the median percentages for all participants which also follow the aforementioned trend differentiating the three human actions. The existence of this trend before movement onset suggests that there is indeed some information in the gaze data that can be used to predict the type of upcoming human action before the motion begins. Since this trend becomes more pronounced after movement onset, it hints that the amount of information available in the gaze data increases as the action unfolds. This suggests that while initial predictions can be made using gaze data prior to the human motion onset, more accurate predictions can be made by utilizing gaze data as the human motion begins and progresses.
The percentages for individual participants are provided in the supplementary materials (Supplementary Tables 2 and 3).


\subsection*{EEG Data Analysis}
The motor cortex of the brain is in charge of movement-related functions such as movement planning and execution, which corresponds to the centrally located EEG channels in the 10-20 system \cite{yahya19}. Movement-related cortical potentials (MRCPs) and sensory-motor rhythms (SMRs) are key EEG characteristics during voluntary movement preparation and planning observed over the motor cortex \cite{sumeyra22}. 

MRCPs are event-related potentials (ERPs) that begin with a slow negative shift about two seconds before movement onset, turning positive with the movement \cite{olsen21}. The grand average ERPs for 13 subjects, comparing handover vs. non-handover conditions, are shown in Fig.~\ref{fig:ERP} for central channels C3, C4, Cz, CP1, CP2, FC1, and FC2. A negative trend starts around -1.8 seconds for both conditions, peaking negatively at -1.7 seconds, followed by a second negative peak at -0.7 seconds (Negative Slope) and finally a positive trend begins just before movement execution, peaking at 0.1 seconds (motor potential) which are in-line with what is observed in the literature \cite{olsen21}. MRCP peak and latency parameters are reported to differ with various movement types, task speed, whether the task is self-paced or cued, the uncertainty level, and the existence of force \cite{brunia2003cnv, nascimento2006movement, rektor2003intracerebral}.

 \begin{figure*}[t]
    \centering
        \subfigure[]
   { \includegraphics[width=0.8\linewidth]{figures/ERP_grandAvg_hand_nohand_Avg7Channels.png} \label{fig:ERP} }
    \subfigure[]{\includegraphics[width=0.8\linewidth]{figures/GrandAvgAlphaBetaGammaCz.png}\label{fig:psd}}
        \caption{Grand average over 13 subjects for the two conditions, Blue: Handover, Orange: Non-Handover \textbf{(a)} ERP gathered from the motor cortex (Channels: C3, C4, Cz, CP1, CP2, FC1, FC2) \textbf{(b)} ERDS of mu, beta and gamma Channel Cz, with the variance highlighted by the shaded region.}
   \label{fig:erp_psd}
 \end{figure*}

SMRs are described as oscillations in the mu (8-12 Hz), beta (13-30 Hz), and gamma (31-200 Hz) frequency bands, seen either as a decrease in power called event-related desynchronization (ERD) or an increase in power called event-related synchronization (ERS). The mu, beta, and gamma band powers over the motor cortex decrease approximately two seconds before the onset with the movement preparation corresponding to ERD. The grand average of the relevant frequency band powers for channel Cz is shown in Fig.~\ref{fig:psd} calculated from $n=13$ subjects. It must be noted that since the EEG signals are band-pass filtered between 1-40 Hz during pre-processing, the gamma band contains frequencies between 30 and 40 Hz. ERD is observed between -2 and 0 seconds for both conditions in the mu, beta, and gamma bands as expected. A paired t-test showed that the band powers are significantly different between the handover and non-handover trials for most of the subjects (Supplementary Table 4).
The mu band is significantly different between handover and non-handover for all subjects except for S2 and S9, with a p-value $<0.05$. The beta band is significantly different for all 13 subjects in this time interval, meanwhile, the gamma band is significantly different for all subjects but one, S11, between the two conditions.


\section{Handover Intention Detection}
%\section{Results}
In this section, we compare the performance of gaze and EEG for handover detection against motion trajectories recorded by an RGBD camera, evaluating both detection timing and accuracy. We also assess the performance of single modalities versus multiple modalities. 
%All model performances are reported using the area under the receiver operating characteristic curve (AUC-ROC), as detailed in the \emph{Materials and Methods} section.
% \begin{figure}[h!]
%     \centering
%     \includegraphics[width=\linewidth]{figures/experimental_setup.pdf}
%     \caption{\justifying blah blah blah}
%     \label{fig:experimental setup}
% \end{figure}



\subsection*{Model Selection}

The main goal of this study is to discriminate handover intention from other types of movements. Therefore, we combined data from the \emph{Solo} and \emph{Joint} conditions as class 0 and used data from \emph{Handover} trials as class 1. 

We compared the performance of two example classifiers: (1) linear classification using linear discriminant analysis (LDA), and (2) non-linear classification using a long-short-term memory (LSTM) neural network. LDA works by modeling the data distribution as a multivariate Gaussian distribution. It then uses the Bayes theorem to classify new data points. 

We used the Scikit-Learn package~\cite{scikit-learn} in Python to implement the linear classification pipeline. LDA was applied using Scikit-Learn’s default parameters. To evaluate LDA’s performance, we used repeated stratified 10-fold cross-validation with three repeats. This involved splitting the data into 10 folds, using 9 folds for training and 1 fold for testing, and repeating the procedure so each fold served as the test set. This entire process was repeated three times with different randomizations.

For the neural network classifier, we used the PyTorch package~\cite{paszke2019pytorch} in Python to implement data loading, training, and the LSTM model. Different LSTM parameters were used for EEG data compared to gaze and hand motion trajectory data due to EEG's higher dimensionality. Specifically, for EEG, we used a 1-layer LSTM with 128 hidden units. The LSTM's output for the last time step was passed through a ReLU activation function and then mapped to class probabilities using a fully connected layer followed by a Sigmoid function. We used a batch size of 16 and trained the model for 100 epochs, selecting the model with the minimum validation loss for evaluation on the test split. Given the low number of data samples, we aimed to keep the hidden layers of LSTMs to a minimal number and correspondingly selected batch size and other parameters. For the gaze and the motion data, we used a 2-layer LSTM with 10 hidden units while using a batch size of 5 and training for 200 epochs with early stopping based on validation loss after 100 epochs. Therefore, we selected the model with the minimum validation loss for further evaluation on the test split.
%The early stopping was based on validation loss with a patience level of 10 epochs, thus selecting the best model with 
%\todo[inline]{@Parag, please add the way you used LSTM} 

To select the best neural network model based on validation loss, we included both a validation set for model selection and a test set for final evaluation. We employed a nested cross-validation approach to achieve this. Specifically, we used repeated stratified 10-fold cross-validation with three repeats to choose test splits. For each of the nine training folds, we performed an additional 10-fold cross-validation to train the model on various train-validation splits. Finally, a weighted ensemble of these 10 models was used to evaluate the LSTM's performance on the test split.

\subsection*{Performance Evaluation}

We measured the area under the receiver operating characteristic curve (AUC-ROC) to evaluate the classification performance for the classifier models. The ROC curve is obtained by using different threshold values on the model's predictions and computing the true positive rate (TPR) and false positive rate (FPR) given by equations~\ref{eq:tpr} and~\ref{eq:fpr} respectively. 

\begin{equation}
    \label{eq:tpr}
    TPR = \frac{TP}{TP + FN} ,
\end{equation}

\begin{equation}
    \label{eq:fpr}
    FPR = \frac{FP}{FP + TN} ,
\end{equation}

TP, TN, FP, and FN stand for true positive, true negative, false positive, and false negative samples. Plotting TPR against FPR creates the ROC curve. AUC-ROC is by definition the area under the ROC curve. An AUC-ROC greater than 0.5 indicates that the classifier is performing better than a random classifier. An AUC-ROC of 1.0 shows a perfect classification discriminating all positive samples from the negative ones. 
% an auc of 1 depicts perfect classifier that dsicrimantes positive samples form negative ones, 
% mention of med papers, with the tasks.

\subsection*{Feature Extraction}

\subsubsection*{EEG Data}

We applied time-frequency transformation to the EEG data using Morlet wavelets with three cycles to generate input for both classifiers. This transformation covered a frequency range of 5–40 Hz within the selected time window, encompassing the alpha (8–12 Hz), beta (13–30 Hz), and part of the gamma (\(>\text{30 Hz}\)) bands, which are known to encode motor task information in EEG signals~\cite{amo2020induced, yu2022effects}. Frequencies above 40 Hz were excluded due to susceptibility to high-frequency line noise and muscle artifacts. The time-frequency representations were then averaged over channels and reshaped into 1-D vectors suitable for the linear classifier. Before applying LDA, we standardized the data to zero mean and unit standard deviation and used principal component analysis to extract components that capture 99\% of the data variance. Applying PCA before LDA further reduces the dimensionality of the data and helps to regularize the classification problem. We also standardized the EEG data for LSTM training, although no further dimensionality reduction (e.g. by PCA) was applied.  
%. was applied as neural networks can inherently handle high-dimensional inputs.
 
\subsubsection*{Gaze Data}
The features for the gaze data were the X and Y coordinates of the recorded gaze data for the participants, referenced to a fixed point on the robot's torso (Supplementary Fig. 3). These features are denoted by: $GazeX$ and $GazeY$. 

\subsubsection*{Hand Motion Trajectory Data}

The features for the motion data were the X, Y, and Z coordinates of the tracked right hand, denoted by: $HandX$, $HandY$, $HandZ$. These coordinates are referenced to the Azure Kinect camera, which was mounted on the torso of the robot (Fig.~\ref{fig:experimental setup}).



\begin{table*}[t]
\centering
\fontsize{11pt}{11pt}\selectfont
\begin{tabular}{lllllllllllll}
\toprule
\multicolumn{1}{c}{\textbf{task}} & \multicolumn{2}{c}{\textbf{Mir}} & \multicolumn{2}{c}{\textbf{Lai}} & \multicolumn{2}{c}{\textbf{Ziegen.}} & \multicolumn{2}{c}{\textbf{Cao}} & \multicolumn{2}{c}{\textbf{Alva-Man.}} & \multicolumn{1}{c}{\textbf{avg.}} & \textbf{\begin{tabular}[c]{@{}l@{}}avg.\\ rank\end{tabular}} \\
\multicolumn{1}{c}{\textbf{metrics}} & \multicolumn{1}{c}{\textbf{cor.}} & \multicolumn{1}{c}{\textbf{p-v.}} & \multicolumn{1}{c}{\textbf{cor.}} & \multicolumn{1}{c}{\textbf{p-v.}} & \multicolumn{1}{c}{\textbf{cor.}} & \multicolumn{1}{c}{\textbf{p-v.}} & \multicolumn{1}{c}{\textbf{cor.}} & \multicolumn{1}{c}{\textbf{p-v.}} & \multicolumn{1}{c}{\textbf{cor.}} & \multicolumn{1}{c}{\textbf{p-v.}} &  &  \\ \midrule
\textbf{S-Bleu} & 0.50 & 0.0 & 0.47 & 0.0 & 0.59 & 0.0 & 0.58 & 0.0 & 0.68 & 0.0 & 0.57 & 5.8 \\
\textbf{R-Bleu} & -- & -- & 0.27 & 0.0 & 0.30 & 0.0 & -- & -- & -- & -- & - &  \\
\textbf{S-Meteor} & 0.49 & 0.0 & 0.48 & 0.0 & 0.61 & 0.0 & 0.57 & 0.0 & 0.64 & 0.0 & 0.56 & 6.1 \\
\textbf{R-Meteor} & -- & -- & 0.34 & 0.0 & 0.26 & 0.0 & -- & -- & -- & -- & - &  \\
\textbf{S-Bertscore} & \textbf{0.53} & 0.0 & {\ul 0.80} & 0.0 & \textbf{0.70} & 0.0 & {\ul 0.66} & 0.0 & {\ul0.78} & 0.0 & \textbf{0.69} & \textbf{1.7} \\
\textbf{R-Bertscore} & -- & -- & 0.51 & 0.0 & 0.38 & 0.0 & -- & -- & -- & -- & - &  \\
\textbf{S-Bleurt} & {\ul 0.52} & 0.0 & {\ul 0.80} & 0.0 & 0.60 & 0.0 & \textbf{0.70} & 0.0 & \textbf{0.80} & 0.0 & {\ul 0.68} & {\ul 2.3} \\
\textbf{R-Bleurt} & -- & -- & 0.59 & 0.0 & -0.05 & 0.13 & -- & -- & -- & -- & - &  \\
\textbf{S-Cosine} & 0.51 & 0.0 & 0.69 & 0.0 & {\ul 0.62} & 0.0 & 0.61 & 0.0 & 0.65 & 0.0 & 0.62 & 4.4 \\
\textbf{R-Cosine} & -- & -- & 0.40 & 0.0 & 0.29 & 0.0 & -- & -- & -- & -- & - & \\ \midrule
\textbf{QuestEval} & 0.23 & 0.0 & 0.25 & 0.0 & 0.49 & 0.0 & 0.47 & 0.0 & 0.62 & 0.0 & 0.41 & 9.0 \\
\textbf{LLaMa3} & 0.36 & 0.0 & \textbf{0.84} & 0.0 & {\ul{0.62}} & 0.0 & 0.61 & 0.0 &  0.76 & 0.0 & 0.64 & 3.6 \\
\textbf{our (3b)} & 0.49 & 0.0 & 0.73 & 0.0 & 0.54 & 0.0 & 0.53 & 0.0 & 0.7 & 0.0 & 0.60 & 5.8 \\
\textbf{our (8b)} & 0.48 & 0.0 & 0.73 & 0.0 & 0.52 & 0.0 & 0.53 & 0.0 & 0.7 & 0.0 & 0.59 & 6.3 \\  \bottomrule
\end{tabular}
\caption{Pearson correlation on human evaluation on system output. `R-': reference-based. `S-': source-based.}
\label{tab:sys}
\end{table*}



\begin{table}%[]
\centering
\fontsize{11pt}{11pt}\selectfont
\begin{tabular}{llllll}
\toprule
\multicolumn{1}{c}{\textbf{task}} & \multicolumn{1}{c}{\textbf{Lai}} & \multicolumn{1}{c}{\textbf{Zei.}} & \multicolumn{1}{c}{\textbf{Scia.}} & \textbf{} & \textbf{} \\ 
\multicolumn{1}{c}{\textbf{metrics}} & \multicolumn{1}{c}{\textbf{cor.}} & \multicolumn{1}{c}{\textbf{cor.}} & \multicolumn{1}{c}{\textbf{cor.}} & \textbf{avg.} & \textbf{\begin{tabular}[c]{@{}l@{}}avg.\\ rank\end{tabular}} \\ \midrule
\textbf{S-Bleu} & 0.40 & 0.40 & 0.19* & 0.33 & 7.67 \\
\textbf{S-Meteor} & 0.41 & 0.42 & 0.16* & 0.33 & 7.33 \\
\textbf{S-BertS.} & {\ul0.58} & 0.47 & 0.31 & 0.45 & 3.67 \\
\textbf{S-Bleurt} & 0.45 & {\ul 0.54} & {\ul 0.37} & 0.45 & {\ul 3.33} \\
\textbf{S-Cosine} & 0.56 & 0.52 & 0.3 & {\ul 0.46} & {\ul 3.33} \\ \midrule
\textbf{QuestE.} & 0.27 & 0.35 & 0.06* & 0.23 & 9.00 \\
\textbf{LlaMA3} & \textbf{0.6} & \textbf{0.67} & \textbf{0.51} & \textbf{0.59} & \textbf{1.0} \\
\textbf{Our (3b)} & 0.51 & 0.49 & 0.23* & 0.39 & 4.83 \\
\textbf{Our (8b)} & 0.52 & 0.49 & 0.22* & 0.43 & 4.83 \\ \bottomrule
\end{tabular}
\caption{Pearson correlation on human ratings on reference output. *not significant; we cannot reject the null hypothesis of zero correlation}
\label{tab:ref}
\end{table}


\begin{table*}%[]
\centering
\fontsize{11pt}{11pt}\selectfont
\begin{tabular}{lllllllll}
\toprule
\textbf{task} & \multicolumn{1}{c}{\textbf{ALL}} & \multicolumn{1}{c}{\textbf{sentiment}} & \multicolumn{1}{c}{\textbf{detoxify}} & \multicolumn{1}{c}{\textbf{catchy}} & \multicolumn{1}{c}{\textbf{polite}} & \multicolumn{1}{c}{\textbf{persuasive}} & \multicolumn{1}{c}{\textbf{formal}} & \textbf{\begin{tabular}[c]{@{}l@{}}avg. \\ rank\end{tabular}} \\
\textbf{metrics} & \multicolumn{1}{c}{\textbf{cor.}} & \multicolumn{1}{c}{\textbf{cor.}} & \multicolumn{1}{c}{\textbf{cor.}} & \multicolumn{1}{c}{\textbf{cor.}} & \multicolumn{1}{c}{\textbf{cor.}} & \multicolumn{1}{c}{\textbf{cor.}} & \multicolumn{1}{c}{\textbf{cor.}} &  \\ \midrule
\textbf{S-Bleu} & -0.17 & -0.82 & -0.45 & -0.12* & -0.1* & -0.05 & -0.21 & 8.42 \\
\textbf{R-Bleu} & - & -0.5 & -0.45 &  &  &  &  &  \\
\textbf{S-Meteor} & -0.07* & -0.55 & -0.4 & -0.01* & 0.1* & -0.16 & -0.04* & 7.67 \\
\textbf{R-Meteor} & - & -0.17* & -0.39 & - & - & - & - & - \\
\textbf{S-BertScore} & 0.11 & -0.38 & -0.07* & -0.17* & 0.28 & 0.12 & 0.25 & 6.0 \\
\textbf{R-BertScore} & - & -0.02* & -0.21* & - & - & - & - & - \\
\textbf{S-Bleurt} & 0.29 & 0.05* & 0.45 & 0.06* & 0.29 & 0.23 & 0.46 & 4.2 \\
\textbf{R-Bleurt} & - &  0.21 & 0.38 & - & - & - & - & - \\
\textbf{S-Cosine} & 0.01* & -0.5 & -0.13* & -0.19* & 0.05* & -0.05* & 0.15* & 7.42 \\
\textbf{R-Cosine} & - & -0.11* & -0.16* & - & - & - & - & - \\ \midrule
\textbf{QuestEval} & 0.21 & {\ul{0.29}} & 0.23 & 0.37 & 0.19* & 0.35 & 0.14* & 4.67 \\
\textbf{LlaMA3} & \textbf{0.82} & \textbf{0.80} & \textbf{0.72} & \textbf{0.84} & \textbf{0.84} & \textbf{0.90} & \textbf{0.88} & \textbf{1.00} \\
\textbf{Our (3b)} & 0.47 & -0.11* & 0.37 & 0.61 & 0.53 & 0.54 & 0.66 & 3.5 \\
\textbf{Our (8b)} & {\ul{0.57}} & 0.09* & {\ul 0.49} & {\ul 0.72} & {\ul 0.64} & {\ul 0.62} & {\ul 0.67} & {\ul 2.17} \\ \bottomrule
\end{tabular}
\caption{Pearson correlation on human ratings on our constructed test set. 'R-': reference-based. 'S-': source-based. *not significant; we cannot reject the null hypothesis of zero correlation}
\label{tab:con}
\end{table*}

\section{Results}
We benchmark the different metrics on the different datasets using correlation to human judgement. For content preservation, we show results split on data with system output, reference output and our constructed test set: we show that the data source for evaluation leads to different conclusions on the metrics. In addition, we examine whether the metrics can rank style transfer systems similar to humans. On style strength, we likewise show correlations between human judgment and zero-shot evaluation approaches. When applicable, we summarize results by reporting the average correlation. And the average ranking of the metric per dataset (by ranking which metric obtains the highest correlation to human judgement per dataset). 

\subsection{Content preservation}
\paragraph{How do data sources affect the conclusion on best metric?}
The conclusions about the metrics' performance change radically depending on whether we use system output data, reference output, or our constructed test set. Ideally, a good metric correlates highly with humans on any data source. Ideally, for meta-evaluation, a metric should correlate consistently across all data sources, but the following shows that the correlations indicate different things, and the conclusion on the best metric should be drawn carefully.

Looking at the metrics correlations with humans on the data source with system output (Table~\ref{tab:sys}), we see a relatively high correlation for many of the metrics on many tasks. The overall best metrics are S-BertScore and S-BLEURT (avg+avg rank). We see no notable difference in our method of using the 3B or 8B model as the backbone.

Examining the average correlations based on data with reference output (Table~\ref{tab:ref}), now the zero-shoot prompting with LlaMA3 70B is the best-performing approach ($0.59$ avg). Tied for second place are source-based cosine embedding ($0.46$ avg), BLEURT ($0.45$ avg) and BertScore ($0.45$ avg). Our method follows on a 5. place: here, the 8b version (($0.43$ avg)) shows a bit stronger results than 3b ($0.39$ avg). The fact that the conclusions change, whether looking at reference or system output, confirms the observations made by \citet{scialom-etal-2021-questeval} on simplicity transfer.   

Now consider the results on our test set (Table~\ref{tab:con}): Several metrics show low or no correlation; we even see a significantly negative correlation for some metrics on ALL (BLEU) and for specific subparts of our test set for BLEU, Meteor, BertScore, Cosine. On the other end, LlaMA3 70B is again performing best, showing strong results ($0.82$ in ALL). The runner-up is now our 8B method, with a gap to the 3B version ($0.57$ vs $0.47$ in ALL). Note our method still shows zero correlation for the sentiment task. After, ranks BLEURT ($0.29$), QuestEval ($0.21$), BertScore ($0.11$), Cosine ($0.01$).  

On our test set, we find that some metrics that correlate relatively well on the other datasets, now exhibit low correlation. Hence, with our test set, we can now support the logical reasoning with data evidence: Evaluation of content preservation for style transfer needs to take the style shift into account. This conclusion could not be drawn using the existing data sources: We hypothesise that for the data with system-based output, successful output happens to be very similar to the source sentence and vice versa, and reference-based output might not contain server mistakes as they are gold references. Thus, none of the existing data sources tests the limits of the metrics.  


\paragraph{How do reference-based metrics compare to source-based ones?} Reference-based metrics show a lower correlation than the source-based counterpart for all metrics on both datasets with ratings on references (Table~\ref{tab:sys}). As discussed previously, reference-based metrics for style transfer have the drawback that many different good solutions on a rewrite might exist and not only one similar to a reference.


\paragraph{How well can the metrics rank the performance of style transfer methods?}
We compare the metrics' ability to judge the best style transfer methods w.r.t. the human annotations: Several of the data sources contain samples from different style transfer systems. In order to use metrics to assess the quality of the style transfer system, metrics should correctly find the best-performing system. Hence, we evaluate whether the metrics for content preservation provide the same system ranking as human evaluators. We take the mean of the score for every output on each system and the mean of the human annotations; we compare the systems using the Kendall's Tau correlation. 

We find only the evaluation using the dataset Mir, Lai, and Ziegen to result in significant correlations, probably because of sparsity in a number of system tests (App.~\ref{app:dataset}). Our method (8b) is the only metric providing a perfect ranking of the style transfer system on the Lai data, and Llama3 70B the only one on the Ziegen data. Results in App.~\ref{app:results}. 


\subsection{Style strength results}
%Evaluating style strengths is a challenging task. 
Llama3 70B shows better overall results than our method. However, our method scores higher than Llama3 70B on 2 out of 6 datasets, but it also exhibits zero correlation on one task (Table~\ref{tab:styleresults}).%More work i s needed on evaluating style strengths. 
 
\begin{table}%[]
\fontsize{11pt}{11pt}\selectfont
\begin{tabular}{lccc}
\toprule
\multicolumn{1}{c}{\textbf{}} & \textbf{LlaMA3} & \textbf{Our (3b)} & \textbf{Our (8b)} \\ \midrule
\textbf{Mir} & 0.46 & 0.54 & \textbf{0.57} \\
\textbf{Lai} & \textbf{0.57} & 0.18 & 0.19 \\
\textbf{Ziegen.} & 0.25 & 0.27 & \textbf{0.32} \\
\textbf{Alva-M.} & \textbf{0.59} & 0.03* & 0.02* \\
\textbf{Scialom} & \textbf{0.62} & 0.45 & 0.44 \\
\textbf{\begin{tabular}[c]{@{}l@{}}Our Test\end{tabular}} & \textbf{0.63} & 0.46 & 0.48 \\ \bottomrule
\end{tabular}
\caption{Style strength: Pearson correlation to human ratings. *not significant; we cannot reject the null hypothesis of zero corelation}
\label{tab:styleresults}
\end{table}

\subsection{Ablation}
We conduct several runs of the methods using LLMs with variations in instructions/prompts (App.~\ref{app:method}). We observe that the lower the correlation on a task, the higher the variation between the different runs. For our method, we only observe low variance between the runs.
None of the variations leads to different conclusions of the meta-evaluation. Results in App.~\ref{app:results}.

\section{Discussion}
\section{Discussion of Assumptions}\label{sec:discussion}
In this paper, we have made several assumptions for the sake of clarity and simplicity. In this section, we discuss the rationale behind these assumptions, the extent to which these assumptions hold in practice, and the consequences for our protocol when these assumptions hold.

\subsection{Assumptions on the Demand}

There are two simplifying assumptions we make about the demand. First, we assume the demand at any time is relatively small compared to the channel capacities. Second, we take the demand to be constant over time. We elaborate upon both these points below.

\paragraph{Small demands} The assumption that demands are small relative to channel capacities is made precise in \eqref{eq:large_capacity_assumption}. This assumption simplifies two major aspects of our protocol. First, it largely removes congestion from consideration. In \eqref{eq:primal_problem}, there is no constraint ensuring that total flow in both directions stays below capacity--this is always met. Consequently, there is no Lagrange multiplier for congestion and no congestion pricing; only imbalance penalties apply. In contrast, protocols in \cite{sivaraman2020high, varma2021throughput, wang2024fence} include congestion fees due to explicit congestion constraints. Second, the bound \eqref{eq:large_capacity_assumption} ensures that as long as channels remain balanced, the network can always meet demand, no matter how the demand is routed. Since channels can rebalance when necessary, they never drop transactions. This allows prices and flows to adjust as per the equations in \eqref{eq:algorithm}, which makes it easier to prove the protocol's convergence guarantees. This also preserves the key property that a channel's price remains proportional to net money flow through it.

In practice, payment channel networks are used most often for micro-payments, for which on-chain transactions are prohibitively expensive; large transactions typically take place directly on the blockchain. For example, according to \cite{river2023lightning}, the average channel capacity is roughly $0.1$ BTC ($5,000$ BTC distributed over $50,000$ channels), while the average transaction amount is less than $0.0004$ BTC ($44.7k$ satoshis). Thus, the small demand assumption is not too unrealistic. Additionally, the occasional large transaction can be treated as a sequence of smaller transactions by breaking it into packets and executing each packet serially (as done by \cite{sivaraman2020high}).
Lastly, a good path discovery process that favors large capacity channels over small capacity ones can help ensure that the bound in \eqref{eq:large_capacity_assumption} holds.

\paragraph{Constant demands} 
In this work, we assume that any transacting pair of nodes have a steady transaction demand between them (see Section \ref{sec:transaction_requests}). Making this assumption is necessary to obtain the kind of guarantees that we have presented in this paper. Unless the demand is steady, it is unreasonable to expect that the flows converge to a steady value. Weaker assumptions on the demand lead to weaker guarantees. For example, with the more general setting of stochastic, but i.i.d. demand between any two nodes, \cite{varma2021throughput} shows that the channel queue lengths are bounded in expectation. If the demand can be arbitrary, then it is very hard to get any meaningful performance guarantees; \cite{wang2024fence} shows that even for a single bidirectional channel, the competitive ratio is infinite. Indeed, because a PCN is a decentralized system and decisions must be made based on local information alone, it is difficult for the network to find the optimal detailed balance flow at every time step with a time-varying demand.  With a steady demand, the network can discover the optimal flows in a reasonably short time, as our work shows.

We view the constant demand assumption as an approximation for a more general demand process that could be piece-wise constant, stochastic, or both (see simulations in Figure \ref{fig:five_nodes_variable_demand}).
We believe it should be possible to merge ideas from our work and \cite{varma2021throughput} to provide guarantees in a setting with random demands with arbitrary means. We leave this for future work. In addition, our work suggests that a reasonable method of handling stochastic demands is to queue the transaction requests \textit{at the source node} itself. This queuing action should be viewed in conjunction with flow-control. Indeed, a temporarily high unidirectional demand would raise prices for the sender, incentivizing the sender to stop sending the transactions. If the sender queues the transactions, they can send them later when prices drop. This form of queuing does not require any overhaul of the basic PCN infrastructure and is therefore simpler to implement than per-channel queues as suggested by \cite{sivaraman2020high} and \cite{varma2021throughput}.

\subsection{The Incentive of Channels}
The actions of the channels as prescribed by the DEBT control protocol can be summarized as follows. Channels adjust their prices in proportion to the net flow through them. They rebalance themselves whenever necessary and execute any transaction request that has been made of them. We discuss both these aspects below.

\paragraph{On Prices}
In this work, the exclusive role of channel prices is to ensure that the flows through each channel remains balanced. In practice, it would be important to include other components in a channel's price/fee as well: a congestion price  and an incentive price. The congestion price, as suggested by \cite{varma2021throughput}, would depend on the total flow of transactions through the channel, and would incentivize nodes to balance the load over different paths. The incentive price, which is commonly used in practice \cite{river2023lightning}, is necessary to provide channels with an incentive to serve as an intermediary for different channels. In practice, we expect both these components to be smaller than the imbalance price. Consequently, we expect the behavior of our protocol to be similar to our theoretical results even with these additional prices.

A key aspect of our protocol is that channel fees are allowed to be negative. Although the original Lightning network whitepaper \cite{poon2016bitcoin} suggests that negative channel prices may be a good solution to promote rebalancing, the idea of negative prices in not very popular in the literature. To our knowledge, the only prior work with this feature is \cite{varma2021throughput}. Indeed, in papers such as \cite{van2021merchant} and \cite{wang2024fence}, the price function is explicitly modified such that the channel price is never negative. The results of our paper show the benefits of negative prices. For one, in steady state, equal flows in both directions ensure that a channel doesn't loose any money (the other price components mentioned above ensure that the channel will only gain money). More importantly, negative prices are important to ensure that the protocol selectively stifles acyclic flows while allowing circulations to flow. Indeed, in the example of Section \ref{sec:flow_control_example}, the flows between nodes $A$ and $C$ are left on only because the large positive price over one channel is canceled by the corresponding negative price over the other channel, leading to a net zero price.

Lastly, observe that in the DEBT control protocol, the price charged by a channel does not depend on its capacity. This is a natural consequence of the price being the Lagrange multiplier for the net-zero flow constraint, which also does not depend on the channel capacity. In contrast, in many other works, the imbalance price is normalized by the channel capacity \cite{ren2018optimal, lin2020funds, wang2024fence}; this is shown to work well in practice. The rationale for such a price structure is explained well in \cite{wang2024fence}, where this fee is derived with the aim of always maintaining some balance (liquidity) at each end of every channel. This is a reasonable aim if a channel is to never rebalance itself; the experiments of the aforementioned papers are conducted in such a regime. In this work, however, we allow the channels to rebalance themselves a few times in order to settle on a detailed balance flow. This is because our focus is on the long-term steady state performance of the protocol. This difference in perspective also shows up in how the price depends on the channel imbalance. \cite{lin2020funds} and \cite{wang2024fence} advocate for strictly convex prices whereas this work and \cite{varma2021throughput} propose linear prices.

\paragraph{On Rebalancing} 
Recall that the DEBT control protocol ensures that the flows in the network converge to a detailed balance flow, which can be sustained perpetually without any rebalancing. However, during the transient phase (before convergence), channels may have to perform on-chain rebalancing a few times. Since rebalancing is an expensive operation, it is worthwhile discussing methods by which channels can reduce the extent of rebalancing. One option for the channels to reduce the extent of rebalancing is to increase their capacity; however, this comes at the cost of locking in more capital. Each channel can decide for itself the optimum amount of capital to lock in. Another option, which we discuss in Section \ref{sec:five_node}, is for channels to increase the rate $\gamma$ at which they adjust prices. 

Ultimately, whether or not it is beneficial for a channel to rebalance depends on the time-horizon under consideration. Our protocol is based on the assumption that the demand remains steady for a long period of time. If this is indeed the case, it would be worthwhile for a channel to rebalance itself as it can make up this cost through the incentive fees gained from the flow of transactions through it in steady state. If a channel chooses not to rebalance itself, however, there is a risk of being trapped in a deadlock, which is suboptimal for not only the nodes but also the channel.

\section{Conclusion}
This work presents DEBT control: a protocol for payment channel networks that uses source routing and flow control based on channel prices. The protocol is derived by posing a network utility maximization problem and analyzing its dual minimization. It is shown that under steady demands, the protocol guides the network to an optimal, sustainable point. Simulations show its robustness to demand variations. The work demonstrates that simple protocols with strong theoretical guarantees are possible for PCNs and we hope it inspires further theoretical research in this direction.

% use section* for acknowledgment
\section*{Acknowledgment}
This work was supported by Digital Futures at KTH,
the Swedish Foundation for Strategic Research (BOS), Knut
and Alice Wallenberg Foundation, the Swedish Research
Council and ERC BIRD 884807.

\bibliographystyle{elsarticle-num-names} 
\bibliography{jrnl}

\newpage
\appendix

\renewcommand{\figurename}{Supplementary Figure}
\renewcommand{\tablename}{Supplementary Table}
\setcounter{figure}{0}
\setcounter{table}{0}

    



\section{Details of datasets}
This section provides additional details about the dataset used to evaluate the downstream tasks. \Cref{tab:disease_definition} lists the ICD-10 codes and medications used to identify the diagnoses for each disease. \Cref{tab:characteristic} presents the distribution of patient characteristics for each disease. \Cref{fig:nyu_langone_prevalence,fig:nyu_longisland_prevalence} illustrates the prevalence of each disease in the downstream tasks for the NYU Langone and NYU Long Island datasets, highlighting the imbalances present in these tasks.

\begin{table}[!htpb]
    \centering
    \caption{The definition of diseases in EHR by diagnosis codes and medications.}
    \begin{tabular}{lr}
    \toprule
         Disease &  Definition in EHR \\
    \midrule
       IPH  &  I61.0, I61.1, I61.2, I61.3, I61.4, I61.8, I61.9 \\
       IVH  &  I61.5, P52.1, P52.2, P52.3  \\
       ICH  &  IPH + IVH + I61.6, I62.9, P10.9, P52.4, P52.9 \\
       SDH  &  S06.5, I62.0 \\
       EDH  &  S06.4, I62.1 \\
       SAH  &  I60.*, S06.6, P52.5, P10.3  \\
       Tumor  &  C71.*, C79.3, D33.0, D33.1, D33.2, D33.3, D33.7, D33.9  \\
       Hydrocephalus  &  G91.* \\
       Edema  &  G93.1, G93.5, G93.6, G93.82, S06.1 \\
       \multirow{2}{*}{ADRD}  &  G23.1, G30.*, G31.01, G31.09, G31.83, G31.85, G31.9, F01.*, F02.*, F03.*, G31.84, G31.1, \\ 
       & \textbf{Medication:} DONEPEZIL, RIVASTIGMINE, GALANTAMINE, MEMANTINE, TACRINE \\ 
    \bottomrule
    \end{tabular}
    \label{tab:disease_definition}
\end{table}

\begin{table}[!htbp]
\centering
\caption{Demographic characteristics of patients associated with scans from the NYU Langone dataset, matched with electronic health records (EHR) and utilized in downstream tasks.}
\label{tab:characteristic}

 The characteristic table on NYU Langone dataset matched with EHR.
\begin{tabular}{ll|rr|r}
\toprule
                       \textbf{Cohort} &  &           \textbf{Male (\%)} &          \textbf{Female (\%)} &     \textbf{Age (std)} \\
\midrule
 --- & All (n=270,205) & 128,113 (47.41\%) & 142,092 (52.59\%) & 63.64 (19.68) \\
\midrule
       Tumor & Neg (n=260,704) & 123,338 (47.31\%) & 137,366 (52.69\%) & 63.85 (19.72) \\
             & Pos (n=9,501) &   4,775 (50.26\%) &   4,726 (49.74\%) & 57.80 (17.67) \\
\midrule
HCP & Neg (n=253,000) & 118,881 (46.99\%) & 134,119 (53.01\%) & 63.67 (19.72) \\
              & Pos (n=17,205) &   9,232 (53.66\%) &   7,973 (46.34\%) & 63.18 (19.11) \\
\midrule
Edema & Neg (n=242,576) & 112,987 (46.58\%) & 129,589 (53.42\%) & 63.96 (19.84) \\
      & Pos (n=27,629) &  15,126 (54.75\%) &  12,503 (45.25\%) & 60.81 (17.97) \\
\midrule
ADRD  & Neg (n=232,667) & 111,159 (47.78\%) & 121,508 (52.22\%) & 61.31 (19.55) \\
      & Pos (n=37,538) &  16,954 (45.16\%) &  20,584 (54.84\%) & 78.09 (13.30) \\
\midrule
          IPH & Neg (n=251,308) & 117,692 (46.83\%) & 133,616 (53.17\%) & 63.58 (19.82) \\
              & Pos (n=18,897) &  10,421 (55.15\%) &   8,476 (44.85\%) & 64.39 (17.69) \\
\midrule
          IVH & Neg (n=258,232) & 121,686 (47.12\%) & 136,546 (52.88\%) & 63.65 (19.79) \\
              & Pos (n=11,973) &   6,427 (53.68\%) &   5,546 (46.32\%) & 63.45 (17.19) \\
\midrule
          SDH & Neg (n=248,468) & 114,869 (46.23\%) & 133,599 (53.77\%) & 63.44 (19.78) \\
              & Pos (n=21,737) &  13,244 (60.93\%) &   8,493 (39.07\%) & 65.95 (18.33) \\
\midrule
          EDH & Neg (n=265,431) & 125,113 (47.14\%) & 140,318 (52.86\%) & 63.77 (19.64) \\
              & Pos (n=4,774) &   3,000 (62.84\%) &   1,774 (37.16\%) & 56.53 (20.75) \\
\midrule
          SAH & Neg (n=251,594) & 118,424 (47.07\%) & 133,170 (52.93\%) & 63.79 (19.76) \\
              & Pos (n=18,611) &   9,689 (52.06\%) &   8,922 (47.94\%) & 61.59 (18.49) \\
\midrule
          ICH & Neg (n=229,851) & 105,498 (45.90\%) & 124,353 (54.10\%) & 63.41 (19.93) \\
              & Pos (n=40,354) &  22,615 (56.04\%) &  17,739 (43.96\%) & 64.93 (18.14) \\
\bottomrule
\end{tabular}
\end{table}


\begin{table}[!h]
    \centering
    \caption*{\textbf{Supplementary \Cref{tab:characteristic} Continued.} Demographic characteristics of patients associated with scans from the NYU Long Island dataset, matched with electronic health records (EHR) and utilized in downstream tasks.}
\begin{tabular}{ll|rr|r}
\toprule
                       \textbf{Cohort} &  &           \textbf{Male (\%)} &          \textbf{Female (\%)} &     \textbf{Age (std)} \\
\midrule
--- & All (n=22,158) & 9,580 (43.23\%) & 12,578 (56.77\%) & 68.33 (18.14) \\
\midrule
Tumor & Neg (n=21,578) & 9,275 (42.98\%) & 12,303 (57.02\%) & 68.59 (18.08) \\
      & Pos (n=580) &   305 (52.59\%) &    275 (47.41\%) & 58.78 (17.79) \\
\midrule
HCP   & Neg (n=20,653) & 8,718 (42.21\%) & 11,935 (57.79\%) & 69.05 (17.90) \\
      & Pos (n=1,505) &   862 (57.28\%) &    643 (42.72\%) & 58.52 (18.48) \\
\midrule
Edema & Neg (n=19,402) & 8,068 (41.58\%) & 11,334 (58.42\%) & 68.89 (18.27) \\
      & Pos (n=2,756) & 1,512 (54.86\%) &  1,244 (45.14\%) & 64.36 (16.66) \\
\midrule
ADRD  & Neg (n=19,537) & 8,391 (42.95\%) & 11,146 (57.05\%) & 66.78 (18.28) \\
      & Pos (n=2,621) & 1,189 (45.36\%) &  1,432 (54.64\%) & 79.90 (11.77) \\
\midrule
IPH   & Neg (n=19,357) & 7,974 (41.19\%) & 11,383 (58.81\%) & 68.97 (18.27) \\
      & Pos (n=2,801) & 1,606 (57.34\%) &  1,195 (42.66\%) & 63.89 (16.48) \\
\midrule
IVH   & Neg (n=19,636) & 8,164 (41.58\%) & 11,472 (58.42\%) & 68.96 (18.22) \\
      & Pos (n=2,522) & 1,416 (56.15\%) &  1,106 (43.85\%) & 63.43 (16.66) \\
\midrule
SDH   & Neg (n=20,885) & 8,870 (42.47\%) & 12,015 (57.53\%) & 68.33 (18.21) \\
      & Pos (n=1,273) &   710 (55.77\%) &    563 (44.23\%) & 68.37 (16.83) \\
\midrule
EDH   & Neg (n=21,912) & 9,443 (43.10\%) & 12,469 (56.90\%) & 68.33 (18.16) \\
      & Pos (n=246) &   137 (55.69\%) &    109 (44.31\%) & 68.19 (15.59) \\
\midrule
SAH   & Neg (n=20,652) & 8,824 (42.73\%) & 11,828 (57.27\%) & 68.68 (18.12) \\
      & Pos (n=1,506) &   756 (50.20\%) &    750 (49.80\%) & 63.58 (17.65) \\
\midrule
ICH   & Neg (n=18,388) & 7,456 (40.55\%) & 10,932 (59.45\%) & 68.92 (18.35) \\
      & Pos (n=3,770) & 2,124 (56.34\%) &  1,646 (43.66\%) & 65.48 (16.77) \\
\bottomrule
\end{tabular}
\end{table}

\begin{figure}[!ht]
    \centering
    \includegraphics[width=0.8\textwidth]{images/NYU_Langone_prevalence.pdf}
    \caption{Disease prevalence of NYU Langone }
    \label{fig:nyu_langone_prevalence}
\end{figure}

\begin{figure}[!h]
    \centering
    \includegraphics[width=0.8\textwidth]{images/NYU_Longisland_prevalence.pdf}
    \caption{Disease prevalence of NYU Longisland dataset}
    \label{fig:nyu_longisland_prevalence}
\end{figure}



\section{Data augmentation details}
\label{sec:dataaug_details}
We applied Random Flipping across all three dimensions, Random Shift Intensity with offset $0.1$ for both pre-training and fine-tuning. For DINO training. random Gaussian Smoothing with sigma=$(0.5-1.0)$ is applied across all dimensions, Random Gamma Adjust is applied with gamma=$(0.2-1.0)$.


\section{Additional experiment results}
This section provides additional experimental results with more details.
Supplementary \Cref{fig:channels-ablation,fig:patches-ablation} compares the performance of the foundation model using different numbers of channels and patch sizes, demonstrating that the architecture design of our foundation model is optimal. 

Supplementary \Cref{fig:radar-comparison-merlin} compares our foundation model with a foundation CT model from previous studies, Merlin\cite{blankemeier2024merlinvisionlanguagefoundation}, which was trained on abdomen CT scans with corresponding radiology report pairs. Our model demonstrates superior performance on head CT scans.

Supplementary \Cref{fig:probing-comparison-gemini} compares our foundation model with Google CT Foundation model~\cite{yang2024advancingmultimodalmedicalcapabilities}, which was trained on large scale and diverse CT scans from different anatomy with corresponding radiology report pairs. Our model consistently shows improved performance across the board even though our model was pre-trained with less samples.

Supplementary \Cref{fig:probing_comparison} compares the performance on downstream tasks with various supervised tuning methods applied to foundation models pretrained with the MAE and DINO frameworks. Per-pathology comparisons are shown in Supplementary \Cref{fig:probing-comparison-perpath,fig:probing-comparison-perpath-dino}. Meanwhile, supplementary \Cref{fig:boxplot_scaling} complements \Cref{fig:scaling_law}, illustrating the per-pathology performances of foundation models pretrained with different scales of training data.

Supplementary \Cref{fig:batch_effect,fig:thickness-ablation} studies the impact of batch effect caused by different CT scan protocols of slice thickness and machine manufacturer. Detailed per-pathology performances are shown in Supplementary \Cref{fig:slice_thickness_per_pathology,fig:manufacturer_per_pathology}.

\begin{figure}[!htpb]
    \centering
    \makebox[\textwidth][l]{%
        \hspace{0.3\textwidth}\textbf{NYU Langone}
    } \\[0.2cm]
    \includegraphics[trim={0 0 0 0},clip,height=0.3\textwidth, width=0.3\textwidth]{figures/abla_chans/AUC_chans_NYU.pdf}
    \includegraphics[trim={0 0 0 0},clip,height=0.3\textwidth, width=0.55\textwidth]{figures/abla_chans/AP_chans_NYU.pdf}\\
    \makebox[\textwidth][l]{
        \hspace{0.34\textwidth}\textbf{RSNA}
    } \\[0.2cm]
    \includegraphics[trim={0 0 0 0},clip,height=0.3\textwidth, width=0.3\textwidth]{figures/abla_chans/AUC_chans_RSNA.pdf}
    \includegraphics[height=0.3\textwidth, width=0.55\textwidth]{figures/abla_chans/AP_chans_RSNA.pdf} 
    \caption{\textbf{Comparison of Different Channels Performance.} This plot compares the performance of models trained using different numbers of channels (channels from multiple HU intervals vs. a single HU interval). Across two datasets, models using three channels from different HU intervals consistently outperform those using a single channel with a fixed HU interval. All models were pre-trained on $100\%$ of the pretraining data with MAE.}
    \label{fig:channels-ablation}
\end{figure}


\begin{figure}[!htpb]
    \centering
    \makebox[\textwidth][l]{%
        \hspace{0.3\textwidth}\textbf{NYU Langone}
    } \\[0.2cm]
    \includegraphics[trim={0 0 0 0},clip,height=0.3\textwidth, width=0.3\textwidth]{figures/abla_patches/AUC_patches_NYU.pdf}
    \includegraphics[trim={0 0 0 0},clip,height=0.3\textwidth, width=0.55\textwidth]{figures/abla_patches/AP_patches_NYU.pdf}\\
    \makebox[\textwidth][l]{
        \hspace{0.34\textwidth}\textbf{RSNA}
    } \\[0.2cm]
    \includegraphics[trim={0 0 0 0},clip,height=0.3\textwidth, width=0.3\textwidth]{figures/abla_patches/AUC_patches_RSNA.pdf}
    \includegraphics[height=0.3\textwidth, width=0.55\textwidth]{figures/abla_patches/AP_patches_RSNA.pdf} 
    \caption{\textbf{Comparison of Different Patches Performance.} This plot compares the performance of models trained with different patch sizes (12 vs. 16). The results demonstrate that smaller patch sizes consistently achieve better performance. All models were pre-trained on $100\%$ of the pretraining data with MAE.}
    \label{fig:patches-ablation}
\end{figure}


\begin{figure*}
    \centering
    \makebox[\textwidth][l]{%
        \hspace{0.06\textwidth}
        \textbf{NYU Langone} \hspace{0.06\textwidth} \textbf{NYU Long Island} \hspace{0.11\textwidth} \textbf{RSNA} \hspace{0.18\textwidth} \textbf{CQ500}
    } \\[0.2cm]
    \includegraphics[trim={0 0 0 0},clip,height=0.21\textwidth, width=0.21\textwidth]{figures/abla_radarplot_merlin/AUC_NYU.pdf}
    \includegraphics[trim={0 0 0 0},clip,height=0.21\textwidth, width=0.21\textwidth]{figures/abla_radarplot_merlin/AUC_Longisland.pdf}
    \includegraphics[trim={0 0 0 0},clip,height=0.21\textwidth, width=0.21\textwidth]{figures/abla_radarplot_merlin/AUC_RSNA.pdf}
    \includegraphics[trim={0 0 0 0},clip,height=0.21\textwidth, width=0.35\textwidth]{figures/abla_radarplot_merlin/AUC_CQ500.pdf}\\[0.2cm]
    \includegraphics[height=0.21\textwidth, width=0.21\textwidth]{figures/abla_radarplot_merlin/AP_NYU.pdf} 
    \includegraphics[height=0.21\textwidth, width=0.21\textwidth]{figures/abla_radarplot_merlin/AP_Longisland.pdf} 
    \includegraphics[height=0.21\textwidth, width=0.21\textwidth]{figures/abla_radarplot_merlin/AP_RSNA.pdf}
    \includegraphics[height=0.21\textwidth, width=0.35\textwidth]{figures/abla_radarplot_merlin/AP_CQ500.pdf}
    \caption{\textbf{Comparison to previous 3D Foundation Model.} This plot compares the performance of our model with Merlin~\cite{blankemeier2024merlinvisionlanguagefoundation} and models trained from scratch across four datasets for our model and ResNet50-3D. Our DINO trained model is used in this comparison. Our model demonstrates consistently superior performance across majority of diseases, with the exception of epidural hemorrhage (EDH) in the CQ500 dataset.}
    \label{fig:radar-comparison-merlin}
\end{figure*}



\begin{figure*}
    \centering
    \makebox[\textwidth][l]{%
        \hspace{0.10\textwidth}
        \textbf{NYU Langone} \hspace{0.08\textwidth} \textbf{NYU Long Island} \hspace{0.1\textwidth} \textbf{RSNA} \hspace{0.15\textwidth} \textbf{CQ500}
    } \\[0.2cm]
    \includegraphics[trim={0 0 0 0},clip, width=0.22\textwidth]{figures/abla_probing/AUC_NYU.pdf}
    \includegraphics[trim={0 0 0 0},clip, width=0.22\textwidth]{figures/abla_probing/AUC_Longisland.pdf}
    \includegraphics[trim={0 0 0 0},clip, width=0.22\textwidth]{figures/abla_probing/AUC_RSNA.pdf}
    \includegraphics[trim={0 0 0 0},clip, width=0.28\textwidth]{figures/abla_probing/AUC_CQ500.pdf}
    \\[0.2cm]
    \includegraphics[width=0.22\textwidth]{figures/abla_probing/AP_NYU.pdf} 
    \includegraphics[width=0.22\textwidth]{figures/abla_probing/AP_Longisland.pdf} 
    \includegraphics[width=0.22\textwidth]{figures/abla_probing/AP_RSNA.pdf}
    \includegraphics[width=0.28\textwidth]{figures/abla_probing/AP_CQ500.pdf}
    \caption{\textbf{Comparison of Different Downstream Training Methods.} This plot illustrates the downstream performance of models evaluated using fine-tuning and various probing methods across four datasets. In most cases, the DINO pre-trained model outperforms the MAE pre-trained model. All models were pre-trained on $100\%$ of the available pretraining data.}
    \label{fig:probing_comparison}
\end{figure*}


\begin{figure}
\centering
\makebox[\textwidth][l]{%
    \hspace{0.39\textwidth}\textbf{RSNA}
} \\[0.2cm]
\includegraphics[trim={0 0 0mm 0},clip,height=0.27\textwidth]{figures/abla_gemini/AUC_RSNA_Gemini.pdf}
\includegraphics[trim={0 0 5mm 0},clip,height=0.27\textwidth]{figures/abla_gemini/AP_RSNA_Gemini.pdf}

\makebox[\textwidth][l]{%
    \hspace{0.38\textwidth}\textbf{CQ500}
} \\[0.2cm]
\includegraphics[trim={0 0 10mm 0},clip,height=0.345\textwidth]{figures/abla_gemini/AUC_CQ500_Gemini.pdf}
\includegraphics[trim={0 0 5mm 0},clip,height=0.345\textwidth]{figures/abla_gemini/AP_CQ500_Gemini.pdf}

\caption{\textbf{Performance comparison of linear probing for Our Model vs. Google CT Foundation model} This plot compares our model performance vs. Google CT Foundation model\cite{yang2024advancing} and Merlin \cite{blankemeier2024merlinvisionlanguagefoundation} across all diseases on RSNA and CQ500. Since Google CT Foundation moudel requires uploading data to Google Cloud (not allowed on our private data) for requesting model embeddings with model weights inaccessible, only public dataset comparison is provided in this study. Similar to other evaluations, we observed that our model outperforms Google CT Foundation model across the board with the only exception on Midline Shift for Google CT Foundation model and EDH for Merlin.}
\label{fig:probing-comparison-gemini}
\end{figure}



\begin{figure}
    \centering
    \makebox[\textwidth][l]{%
        \hspace{0.35\textwidth}\textbf{NYU Langone}
    } \\[0.2cm]
    \includegraphics[trim={0 0 120mm 0},clip,height=0.255\textwidth]{figures/abla_probing_perpath/DINO_AUC_NYU_Langone.pdf}
    \includegraphics[trim={0 0 0 0},clip,height=0.255\textwidth]{figures/abla_probing_perpath/DINO_AP_NYU_Langone.pdf} \\
    \makebox[\textwidth][l]{
        \hspace{0.35\textwidth}\textbf{NYU Long Island}
    } \\[0.2cm]
    \includegraphics[trim={0 0 120mm 0},clip,height=0.255\textwidth]{figures/abla_probing_perpath/DINO_AUC_NYU_Long_Island.pdf}
    \includegraphics[trim={0 0 0 0},clip,height=0.255\textwidth]{figures/abla_probing_perpath/DINO_AP_NYU_Long_Island.pdf} 
    \makebox[\textwidth][l]{
        \hspace{0.4\textwidth}\textbf{RSNA}
    } \\[0.2cm]
    \includegraphics[trim={0 0 120mm 0},clip,height=0.24\textwidth]{figures/abla_probing_perpath/DINO_AUC_RSNA.pdf}
    \hspace{5mm}
    \includegraphics[trim={0 0 0 0},clip,height=0.24\textwidth]{figures/abla_probing_perpath/DINO_AP_RSNA.pdf} 
    \makebox[\textwidth][l]{
        \hspace{0.4\textwidth}\textbf{CQ500}
    } \\[0.2cm]
    \includegraphics[trim={0 0 120mm 0},clip,height=0.30\textwidth]{figures/abla_probing_perpath/DINO_AUC_CQ500.pdf} \hspace{5mm}
    \includegraphics[trim={0 0 0 0},clip,height=0.30\textwidth]{figures/abla_probing_perpath/DINO_AP_CQ500.pdf} 
    \caption{\textbf{Performance comparison of supervised finetuning methods per pathology on the foundation model trained with DINO.} This plot breaks down the average performance across all diseases shown in Supplementary \Cref{fig:probing_comparison}. The results show that fine-tuning the entire network achieves the best performance in most scenarios. However, linear probing closely approaches finetuning performance for many diseases especially on small or imbalanced dataset, underscoring the capability of our pre-trained models to generate representations that adapt effectively to diverse disease detection tasks.}
    \label{fig:probing-comparison-perpath-dino}
\end{figure}

\begin{figure}
    \centering
    \makebox[\textwidth][l]{%
        \hspace{0.35\textwidth}\textbf{NYU Langone}
    } \\[0.2cm]
    \includegraphics[trim={0 0 0 0},clip,height=0.24\textwidth, width=0.3\textwidth]{figures/abla_probing_perpath/AUC_NYU.pdf}
    \includegraphics[trim={0 0 0 0},clip,height=0.24\textwidth, width=0.45\textwidth]{figures/abla_probing_perpath/AP_NYU.pdf}\\
    \makebox[\textwidth][l]{
        \hspace{0.35\textwidth}\textbf{NYU Long Island}
    } \\[0.2cm]
    \includegraphics[trim={0 0 0 0},clip,height=0.24\textwidth, width=0.3\textwidth]{figures/abla_probing_perpath/AUC_Longisland.pdf}
    \includegraphics[trim={0 0 0 0},clip,height=0.24\textwidth, width=0.45\textwidth]{figures/abla_probing_perpath/AP_Longisland.pdf} 
    \makebox[\textwidth][l]{
        \hspace{0.4\textwidth}\textbf{RSNA}
    } \\[0.2cm]
    \includegraphics[trim={0 0 0 0},clip,height=0.24\textwidth, width=0.3\textwidth]{figures/abla_probing_perpath/AUC_RSNA.pdf}
    \includegraphics[height=0.24\textwidth, width=0.45\textwidth]{figures/abla_probing_perpath/AP_RSNA.pdf} 
    \makebox[\textwidth][l]{
        \hspace{0.4\textwidth}\textbf{CQ500}
    } \\[0.2cm]
    \includegraphics[trim={0 0 120mm 0},clip,height=0.24\textwidth]{figures/abla_probing_perpath/AUC_CQ500.pdf}
    \includegraphics[trim={0 0 0 0},clip,height=0.24\textwidth]{figures/abla_probing_perpath/AP_CQ500.pdf} 
    \caption{\textbf{Performance comparison of supervised finetuning methods per pathology on the foundation model trained with MAE.} The results reveal that attentive probing is significantly more effective than linear probing, consistent with findings from~\cite{Chen2024}. Furthermore, for many diseases, the performance of probing models approaches that of fine-tuning, demonstrating that our pre-trained models produce adaptable representations capable of detecting diverse diseases.}
    \label{fig:probing-comparison-perpath}
\end{figure}









\begin{figure}
    \centering
    \textbf{NYU Langone} \\
    \includegraphics[trim={0 0 0 0},clip,height=0.24\textwidth, width=0.38\textwidth]{figures/abla_perpath_perf/AUC_NYU.pdf}
    \includegraphics[height=0.24\textwidth, width=0.45\textwidth]{figures/abla_perpath_perf/AP_NYU.pdf} \\
    \textbf{NYU Long Island} \\
    \includegraphics[trim={0 0 0 0},clip,height=0.24\textwidth, width=0.38\textwidth]{figures/abla_perpath_perf/AUC_Longisland.pdf}
    \includegraphics[height=0.24\textwidth, width=0.45\textwidth]{figures/abla_perpath_perf/AP_Longisland.pdf} \\
    \textbf{RSNA} \\
    \includegraphics[trim={0 0 0 0},clip,height=0.24\textwidth, width=0.38\textwidth]{figures/abla_perpath_perf/AUC_RSNA.pdf}
    \includegraphics[height=0.24\textwidth, width=0.45\textwidth]{figures/abla_perpath_perf/AP_RSNA.pdf}\\
    \textbf{CQ500} \\
    \includegraphics[trim={0 0 0 0},clip,height=0.24\textwidth, width=0.38\textwidth]{figures/abla_perpath_perf/AUC_CQ500.pdf}
    \includegraphics[height=0.24\textwidth, width=0.45\textwidth]{figures/abla_perpath_perf/AP_CQ500.pdf}
    \caption{\textbf{Performance for Different Percentage of Pre-training Samples (Per-Pathology).} This plot illustrates label efficiency for individual pathologies using Tukey plots, alongside the average performance across all diseases shown in \Cref{fig:scaling_law}. The results indicate that the majority of pathologies show improved downstream performance as the amount of pretraining data increases.}
    \label{fig:boxplot_scaling}
\end{figure}


\newpage

\section{Time complexity increase with reduced patch size}
\label{apd:self_attention_rate}
Assume we have 3D image input of shape $H\times W\times D$, patch size $P$ and reducing factor $s$. By time complexity of self-attention $O(n^2 d)$ for sequence length $n$ and embedding dimension $d$, the new time complexity after reducing patch size can be derived as
\begin{align*}
    O(n^2d)&=O((\frac{H\times W\times D}{(\frac{P}{s})^3})^2d) \\
           &=O((\frac{H\times W\times D}{P^3})^2 s^6d)  \\
           &=O(s^6)O(n_{ori}^2d)
\end{align*}
where $n_{ori}=\frac{H\times W\times D}{P^3}$ is the original sequence length before reducing patch size.



















\newpage
\begin{figure}[ht]
    \centering
    \includegraphics[width=\textwidth]{images/tsne_embedding_visualization_per_pathology.png}
    \caption{The 2D projection with t-SNE of CT volume representation extracted from the foundation model. Interestingly, certain subgroups still exhibited slightly better AUCs. For instance, scans with slice thicknesses between 1–4 mm (represented by light green points in the upper panel of \Cref{fig:batch_effect}) align with a specialized protocol for CT angiography (CTA), which uses contrast dye to improve diagnosis on particular diseases.}
    \label{fig:batch_effect}
\end{figure}


\begin{figure*}[ht]
    \centering
    \begin{subfigure}[b]{0.33\textwidth}
        \centering
        \includegraphics[width=\textwidth]{images/AUROC_vs_Slice_thickness_binned.png}
        \caption{AUROC Performance}
    \end{subfigure}
    \hfill
    \begin{subfigure}[b]{0.33\textwidth}
        \centering
        \includegraphics[width=\textwidth]{images/AUPRC_vs_Slice_thickness_binned.png}
        \caption{AUPRC Performance}
    \end{subfigure}
    \hfill
    \begin{subfigure}[b]{0.33\textwidth}
        \centering
        \includegraphics[width=\textwidth]{images/Histogram_of_slice_thickness_distribution_across_scans.png}
        \caption{Histogram of slice thickness distribution}
    \end{subfigure}
    \caption{The downstream task performances on various ranges of slice thickness.}
    \label{fig:thickness-ablation}
\end{figure*}


\begin{figure*}[ht]
    \centering
    \begin{subfigure}[b]{\textwidth}
        \centering
        \includegraphics[width=\textwidth]{images/AUROC_vs_slice_thickness_for_each_disease_category.png}
        \caption{AUROC Performance}
    \end{subfigure}
    \hfill
    \begin{subfigure}[b]{\textwidth}
        \centering
        \includegraphics[width=\textwidth]{images/AUPRC_vs_slice_thickness_for_eachdisease_category.png}
        \caption{AUPRC Performance}
    \end{subfigure}
    \hfill
    \begin{subfigure}[b]{\textwidth}
        \centering
        \includegraphics[width=\textwidth]{images/Ratio_of_positive_labels_vs_slice_thickness_for_each_disease_category.png}
        \caption{Ratio of Positive Labels}
    \end{subfigure}
    \caption{Performance for Each Slice Thickness Bin (Per Pathology).}
    \label{fig:slice_thickness_per_pathology}
\end{figure*}


\begin{figure*}[ht]
    \centering
    \begin{subfigure}[b]{0.3\textwidth}
        \centering
        \includegraphics[width=\textwidth]{images/AUROC_by_Disease_and_Manufacturer.png}
        \caption{AUROC Performance}
    \end{subfigure}
    \hfill
    \begin{subfigure}[b]{0.3\textwidth}
        \centering
        \includegraphics[width=\textwidth]{images/AUPRC_by_Disease_and_Manufacturer.png}
        \caption{AUPRC Performance}
    \end{subfigure}
    \hfill
    \begin{subfigure}[b]{0.39\textwidth}
        \centering
        \includegraphics[width=\textwidth]{images/Positive_Label_Ratio_by_Disease_and_Manufacturer.png}
        \caption{Distribution of Scans from Each Manufacturer}
    \end{subfigure}
    \caption{Performance for Each Manufacturer (Per Pathology).}
    \label{fig:manufacturer_per_pathology}
\end{figure*}






%% else use the following coding to input the bibitems directly in the
%% TeX file.

%% Refer following link for more details about bibliography and citations.
%% https://en.wikibooks.org/wiki/LaTeX/Bibliography_Management

\begin{comment}

\section{Example Section}
\label{sec1}
%% Labels are used to cross-reference an item using \ref command.

Section text. See Subsection \ref{subsec1}.

%% Use \subsection commands to start a subsection.
\subsection{Example Subsection}
\label{subsec1}

Subsection text.

%% Use \subsubsection, \paragraph, \subparagraph commands to 
%% start 3rd, 4th and 5th level sections.
%% Refer following link for more details.
%% https://en.wikibooks.org/wiki/LaTeX/Document_Structure#Sectioning_commands

\subsubsection{Mathematics}
%% Inline mathematics is tagged between $ symbols.
This is an example for the symbol $\alpha$ tagged as inline mathematics.

%% Displayed equations can be tagged using various environments. 
%% Single line equations can be tagged using the equation environment.
\begin{equation}
f(x) = (x+a)(x+b)
\end{equation}

%% Unnumbered equations are tagged using starred versions of the environment.
%% amsmath package needs to be loaded for the starred version of equation environment.
\begin{equation*}
f(x) = (x+a)(x+b)
\end{equation*}

%% align or eqnarray environments can be used for multi line equations.
%% & is used to mark alignment points in equations.
%% \\ is used to end a row in a multiline equation.
\begin{align}
 f(x) &= (x+a)(x+b) \\
      &= x^2 + (a+b)x + ab
\end{align}

\begin{eqnarray}
 f(x) &=& (x+a)(x+b) \nonumber\\ %% If equation numbering is not needed for a row use \nonumber.
      &=& x^2 + (a+b)x + ab
\end{eqnarray}

%% Unnumbered versions of align and eqnarray
\begin{align*}
 f(x) &= (x+a)(x+b) \\
      &= x^2 + (a+b)x + ab
\end{align*}

\begin{eqnarray*}
 f(x)&=& (x+a)(x+b) \\
     &=& x^2 + (a+b)x + ab
\end{eqnarray*}

%% Refer following link for more details.
%% https://en.wikibooks.org/wiki/LaTeX/Mathematics
%% https://en.wikibooks.org/wiki/LaTeX/Advanced_Mathematics

%% Use a table environment to create tables.
%% Refer following link for more details.
%% https://en.wikibooks.org/wiki/LaTeX/Tables
\begin{table}[t]%% placement specifier
%% Use tabular environment to tag the tabular data.
%% https://en.wikibooks.org/wiki/LaTeX/Tables#The_tabular_environment
\centering%% For centre alignment of tabular.
\begin{tabular}{l c r}%% Table column specifiers
%% Tabular cells are separated by &
  1 & 2 & 3 \\ %% A tabular row ends with \\
  4 & 5 & 6 \\
  7 & 8 & 9 \\
\end{tabular}
%% Use \caption command for table caption and label.
\caption{Table Caption}\label{fig1}
\end{table}


%% Use figure environment to create figures
%% Refer following link for more details.
%% https://en.wikibooks.org/wiki/LaTeX/Floats,_Figures_and_Captions
\begin{figure}[t]%% placement specifier
%% Use \includegraphics command to insert graphic files. Place graphics files in 
%% working directory.
\centering%% For centre alignment of image.
\includegraphics{example-image-a}
%% Use \caption command for figure caption and label.
\caption{Figure Caption}\label{fig1}
%% https://en.wikibooks.org/wiki/LaTeX/Importing_Graphics#Importing_external_graphics
\end{figure}


%% The Appendices part is started with the command \appendix;
%% appendix sections are then done as normal sections
\appendix
\section{Example Appendix Section}
\label{app1}

Appendix text.

%% For citations use: 
%%       \citet{<label>} ==> Lamport [21]
%%       \citep{<label>} ==> [21]
%%
Example citation, See \citet{lamport94}.

%% If you have bib database file and want bibtex to generate the
%% bibitems, please use
%%

\begin{thebibliography}{00}

%% For authoryear reference style
%% \bibitem[Author(year)]{label}
%% Text of bibliographic item

\bibitem[Lamport(1994)]{lamport94}
  Leslie Lamport,
  \textit{\LaTeX: a document preparation system},
  Addison Wesley, Massachusetts,
  2nd edition,
  1994.

\end{thebibliography}    
\end{comment}

\end{document}

\endinput
%%
%% End of file `elsarticle-template-num-names.tex'.
