\section{Preliminaries} \label{sec:pre}
This section introduces the fundamental knowledge of graph signal processing (GSP) and global placement.

\subsection{Graph Signal Processing}~\label{sec:gsp}
Given a weighted undirected graph $G = (V, E)$ with a set of nodes $V$ and edges $E$, it can be represented as an adjacency matrix $A =\{w_{i,j}\}\in \mathbb{R}^{N \times N}$ with $w_{i,j}>0$ if node $v_i$ and $v_j$ are connected by edges, and $w_{i,j}=0$ otherwise. The graph Laplacian matrix is defined as $L = D-A$, where $D =diag(d_1, d_2, \ldots, d_n )\in \mathbb{R}^{N \times N}$ represents the degree matrix of $A$ with $d_i=\sum_{v_j \in V}w_{i,j}$. The normalized graph Laplacian matrix is defined as $\tilde{L}=D^{-\frac{1}{2}}LD^{-\frac{1}{2}}$.

The graph signal $g$ is defined as a mapping $g : V \to \mathbb{R}$ and it can be expressed as an n-dimensional vector where each element $g_i$ can encapsulate diverse types of information associated with node $v_i$~\cite{graph_signal}. 
GSP is a field dedicated to the analysis and manipulation of these signals on graphs, intending to extract meaningful insights or accomplish various tasks.

\textbf{Smoothness.}
The graph gradient of signal $g$ at node $v_i$ is defined as $\nabla_ig:=[\{\frac{\partial f}{\partial e}|_i\}_{e \in E} ]=[\sqrt{w_{i,j}}(g_i-g_j)].$
% The edge derivative of a signal $g$ with respect to edge $e=(i, j)$ at node $v_i$ is defined as $\frac{\partial f}{\partial e}|_i=\sqrt{A_{i,j}}(g_i-g_j)$ and 
The smoothness of the graph signal can be quantified using the graph Laplacian quadratic form, as calculated by the following equation:
\begin{equation}\label{eq:smoothness}
S(g)=\frac{1}{2}\sum_{v_i\in V}||\nabla_ig||_2^2=\sum_{(v_i,v_j)\in E}w_{i,j}(g_j-g_i)^2
\end{equation}
% The normalized form can be calculated using the Rayleigh quotient as follows:
% \begin{equation}\label{eq:smoothness_ray}
% R(g)=\frac{S(g)}{||g||_2}=\frac{\sum_{(v_i,v_j)\in E}w_{i,j}(g_j-g_i)^2}{\sum_{v_i \in V}g_i^2}=\frac{g^TLg}{g^Tg}
% \end{equation}

\textbf{Graph Fourier Transform.}
Since the graph Laplacian matrix $L$ is real and symmetric, it can be decomposed into $L=U\Lambda U^T$, where $\Lambda=diag(\lambda_1, \ldots, \lambda_n) $ denotes eigenvalues with $\lambda_1 \leq \lambda_2, \ldots, \leq \lambda_n$ and $U=[u_1, \ldots, u_n]$ is the corresponding eigenvectors. Using eigenvector matrix $U$ as the Graph Fourier Transform (GFT) basis, we call $\hat{g}=U^Tg$ as GFT that transforms the graph signal $g$ from the spatial domain to the spectral domain. And its inverse transform is defined as $g = U\hat{g}$. 

% Next, we interpret the relation between smoothness and frequency of graph signal. 
% Based on GFT, the measurement of smoothness in Eq.~\ref{eq:smoothness_ray} can be transformed as follows:
% \begin{equation}\label{eq:smoothness_ray}
% R(g)=\frac{g^TLg}{g^Tg}=\frac{g^TU\Lambda U^Tg}{g^TUU^Tg}=\frac{\hat{g}^T\Lambda\hat{g}}{\hat{g}^T\hat{g}}=\frac{\sum_{v_i \in V} \lambda_i\hat{g_i}^2}{\sum_{v_i \in V}\hat{g_i}^2}
% \end{equation}

% From Eq.~\ref{eq:smoothness_ray}, it's evident that $R(g_i)=\lambda_i$. This illustrates that graph signals associated with lower eigenvalues (i.e. lower frequencies) exhibits a greater degree of smoothness.


\textbf{Graph Filter.}
In the spectral domain, the graph filter can be used to selectively filter some unwanted frequencies present in the graph signal, which is formally defined as:
\begin{equation}\label{eq:filter}
\mathcal{H}=Udiag(h(\lambda_1),h(\lambda_2), \ldots, h(\lambda_n))U^T,
\end{equation}
where $h(\cdot)$ is a filter function applied to frequencies.
 % (i.e., eigenvalues).

% Usually, non-smooth high frequency components in signals contains more noises. To extract useful information, low-pass filters are needed to save low frequencies in signals and promote the smoothness of graph signal for denoising. The graph low-pass filter are defined as follows.

% \textbf{Graph Convolution.}
% The graph convolution of a graph signal $g$ is defined as follows:
% \begin{equation}\label{eq:graph_convolution}
% \mathcal{H}=Udiag(h(\lambda_1),h(\lambda_2), \ldots, h(\lambda_n))U^Tg.
% \end{equation}
% The process of graph convolution can be interpreted from the graph signal processing perspective. In essence, given a graph signal $g$, it is initially transformed from spatial domain to spectral domain via the Graph Fourier Transform.  
% Subsequently, unwanted frequencies within the signal are removed using the filter $h(\cdot)$, and finally, the filtered signal is transformed back into the spatial domain through the Inverse Graph Fourier Transform.


\subsection{Global Placement}

A placement instance can be formulated as a graph $G=(V, E)$ with a set of objects $V$ (e.g. IOs, macros, and standard cells) and edges $E$.
The main objective of placement $f(g)$ (see Eq.~\ref{eq:global_placement_eq}) is to find a solution $g \in \mathbb{R}^{N \times 2} $ with minimized total wirelength $S(g)$ subject to density constraints (i.e., the density $\rho_b(g)$ does not exceed a predetermined density $\rho_t$), where $N$ is the number of objects.

\begin{equation}\label{eq:global_placement_eq}
\begin{aligned}
&\min f(g) = S(g) \quad\quad s.t. \quad \rho_b(g) \leq \rho_t. 
\end{aligned}
\end{equation}

The total wirelength $S(g)$ can be estimated as the weighted sum of squared distances between connected objects:

\begin{equation}\label{eq:wire_eq}
\begin{aligned}
S(g) = \sum_{(v_i,v_j)\in E} w_{i,j}((x_i- x_j)^2 + (y_i- y_j)^2).
\end{aligned}
\end{equation}
Considering one net with $M$ pins, we set the weight $w = \frac{2}{M}$.  


The placement area can be evenly partitioned into a series of grids (bins)~\cite{2015eplace}. The density of each grid $b$ is computed using the formula in Eq.~\ref{eq:density_eq}:
\begin{equation}\label{eq:density_eq}
\rho_b(g) = \sum_{v \in V} l_x(v, b)l_y(v, b)
\end{equation}
where $l_x(v, b)$ and $l_y(v, b)$ quantify the horizontal and vertical overlaps between the grid $b$ and the object $v$.

% For a given netlist, $w_{i,j}$ are known. According to Eq.~\ref{eq:global_placement_eq} and Eq.~\ref{eq:wire_eq}, the essence of placement is to establish an optimized mapping from circuit connectivity to physical locations while meeting all constraints (e.g. density constraints, location constraints, etc.). 
% In light of this, an efficient initial placer that generates initial locations with such optimized mapping (i.e., minimized total wirelength, reduced overlap, and adhering to location constraints) becomes instrumental in reducing cost of subsequent placement optimization and leads to high-quality placement results with highly reduced iteration time.



% \textbf{Motivation of Using GSP for Placement.}
% As demonstrated in Eq~\ref{eq:smoothness} and Eq~\ref{eq:wire_eq}, the optimization objective in placement, which aims to minimize quadratic wirelength, aligns with the goal of enhancing graph signal smoothness from the perspective of graph signal processing. Hence, it is reasonable to transform the placement problem to promote smoothness of graph signal on given circuit graphs. 
% However, it is essential to note that globally smooth signals may overlook valuable local information. This local information can be globally high-frequency, yet locally smooth. Neglecting these signals can result in high overlap in local areas and violations of density constraints.
% To this end, we need to carefully design a graph filter that considers both globally smooth signals and locally smooth but globally rough signals, thereby enabling us to explore comprehensive structural information throughout the entire graph, facilitating the generation of optimized placement solutions with reduced overlap.