\section{Discussion} \label{sec:discussion}

This section provide the theoretical foundations about why GiFt can drive the gradient descent-based method (e.g. DREAMPlace) to efficiently solve the placement problem.

The placement problem is a non-convex nonlinear optimization problem. Gradient-descent methods such as DREAMPlace and RePlAce require a large number of optimization iterations and are prone to falling into local optima. A critical factor influencing the solution quality of these gradient-descent methods is their dependency on the initial solution~\cite{eplace_ms}. 
Our GSP-based method provides an optimized starting point for subsequent placers, that is, it allows gradient descent-based placers to begin the optimization process from a more advantageous position. As a result, the placement problem can be tackled more effectively, leading to high-quality solutions with a significantly reduced number of iterations.

