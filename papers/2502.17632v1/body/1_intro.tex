\section{Introduction}

Placement is a critical and complex task in VLSI physical designs. It determines the locations of connected cells to minimize total wirelength while adhering to constraints (e.g. density constraints).
Despite continuous efforts to accelerate and improve the placement process, it remains challenging due to the growing complexity of modern designs.
The state-of-the-art analytical placers~\cite{2015eplace, replace} model the placement problem as an electrostatic system and iteratively optimize it, which involves long iteration time.
Recently proposed DREAMPlace~\cite{dreamplace3} has enhanced this process via GPU acceleration. However, long iterations remain a challenge to analytical placers. 

To expedite the placement process, deep learning-based methods have drawn significant attention.
Recently proposed graph convolution network (GCN) based methods represent circuit netlist as graph and produce initial chip placement through efficient model inference. For instance, GraphPlanner~\cite{graphplanner} utilizes a variational GCN-based model to generate initial placement solutions, streamlining the placement procedure for improved efficiency. Researches~\cite{CY2021, google21} apply GCNs and their variants to encode the connectivity information for generating node embeddings. These embeddings are subsequently employed in reinforcement learning calculations to yield superior placement results.


However, existing empirical GCN-based approaches face two key limitations. 
Firstly, these approaches require time-consuming parameter learning with high computation and memory costs.  
Specifically, GCN-based deep models equipped with a large number of learnable parameters demand a significant amount of training data and considerable training time. As the scale of modern designs continues to increase, the computational complexity and memory demands of GCN-based deep models increase accordingly.  
Secondly, it is challenging to guarantee the generalization of GCN-based models. The statistical characteristics of graph structures across various designs can vary significantly, and the existence of predetermined fixed cells (e.g. fixed IOs) exacerbates the issue. Consequently, even well-trained GCNs may struggle to generalize to other unseen designs, limiting their practical applicability. 
Therefore, an interesting question arises: \emph{How to minimize the required optimization iterations of analytical solvers and yet avoid the high training costs of GCN-based methods? }



% This paper propose a parameter-free and highly efficient placement approach, which we have coined as Multi-frequency Graph Filter based Placer (GF-Placer). GF-Placer is proficient in efficiently exploring comprehensive structural information throughout the entire graph, enabling the generation of optimized placement solutions. It achieves this without the necessity for time-consuming model training, while also substantially diminishing the iterations required by analytical placers.
% Specifically, our approach is rooted in the principles of graph signal processing, where we emphasize the crucial role of smoothness in enhancing the placement problem. 
% Building upon this foundation, we design a multi-frequency graph filter with low computation complexity to promote both the local and global smoothness of signals on circuit graphs, thus enabling the generation of optimized placement solutions. 
% In addition, it is proved that both the classic eigenvector-based placer and recently emerged GCN-based placers are considered special cases of the proposed graph filter. However, we recognize that these approaches introduce redundant computations, which are unnecessary for efficient chip placement. By integrating the proposed graph filter with a state-of-the-art analytical placer running on GPUs, we develop GF-placer, an ultra-fast placement tool. Experimental results on both academic benchmarks and real-world industry designs show that GF-Placer achieves competitive or superior performance compared to the state-of-the-art placers while providing significantly efficiency improvement. 
% Notably, it surpasses even the performance of analytical placers optimized for GPU usage, delivering a remarkable runtime improvement of over 50\%.


This paper presents GiFt, an efficient parameter-free approach for accelerating chip placement. GiFt can be seamlessly integrated with modern analytical placers to construct an ultra-fast placement flow GiFt-Placer, which significantly minimizes the numerous optimization iterations of placers without the need for time-consuming model training.
% GiFt has the capability to explore and capture comprehensive structural information of circuit graphs. 
% It can be seamlessly integrated with modern analytical placers to significantly minimize the numerous optimization iterations of placers without the need for time-consuming model training. 
In essence, GiFt is theoretically rooted in graph signal processing (GSP) and we emphasize the crucial role of \emph{smoothness}, a key concept in GSP, in enhancing chip placement. Specifically, GiFt functions as a multi-frequency \emph{Graph Filter} with low computation complexity to promote both local and global signal smoothness on circuit graphs. This feature facilitates the generation of optimized initial placement solutions, which drives the subsequent placers to generate high-quality placement results with highly reduced iteration time.
Moreover, we prove that both the classic eigenvector-based placers~\cite{eigen_placer, eigen_placer2} and recently emerged GCN-based placers~\cite{graphplanner, CY2021} are special cases of GiFt-Placer. Our work further points out that these approaches introduce redundant computations, which are unnecessary for high-quality chip placement. Experimental results on academic benchmarks show that GiFt-Placer significantly improves placement efficiency, while achieving competitive or superior performance compared to state-of-the-art placers. In particular, compared to DREAMPlace, the recently proposed GPU-accelerated analytical placer, GiFt-DREAMPlace improves runtime over 45\%.

The key contributions are listed as follows.


\begin{enumerate}


\item We propose a new multi-frequency graph filter-based placement acceleration approach called GiFt, which can comprehensively capture the graph structure and efficiently generate optimized cell locations. 

\item GiFt offers high efficiency through sparse matrix multiplication and does not require a time-consuming model training process due to its parameter-free nature.

\item GiFt can be seamlessly integrated with modern analytical placers to construct GiFt-Placer placement flow, which effectively reduces the number of iterations required for analytical placers and significantly improves placement efficiency, even surpassing the performance of analytical placers running on GPU versions.

\item By examining classic eigenvector-based placers and recent GCN-based placers from the perspective of graph signal processing, we prove that they can be considered as special cases of GiFt-Placer, but they introduce unnecessary computation costs to chip placement.

% Experimental results on both academic benchmarks and real-world industry designs show GF-Placer achieves competitive or superior performance compared to both GCN-based placer and the analytical placer running on GPU version while remarkably improving the placement efficiency.  

\end{enumerate}

The rest of this paper is organized as follows.  Section~\ref{sec:pre} describes preliminaries for the rest of this paper. Section~\ref{sec:method} introduces the details of the proposed approach. Section~\ref{sec:experiment} presents the experimental results. Section~\ref{sec:discussion} discusses the theoretical foundations of this work. We conclude the paper in Section~\ref{sec:conclusion}.

