% This must be in the first 5 lines to tell arXiv to use pdfLaTeX, which is strongly recommended.
\pdfoutput=1
% In particular, the hyperref package requires pdfLaTeX in order to break URLs across lines.

\documentclass[11pt]{article}

% Change "review" to "final" to generate the final (sometimes called camera-ready) version.
% Change to "preprint" to generate a non-anonymous version with page numbers.
\usepackage[preprint]{acl}

% Standard package includes
\usepackage{times}
\usepackage{latexsym}

% For proper rendering and hyphenation of words containing Latin characters (including in bib files)
\usepackage[T1]{fontenc}
% For Vietnamese characters
% \usepackage[T5]{fontenc}
% See https://www.latex-project.org/help/documentation/encguide.pdf for other character sets

% This assumes your files are encoded as UTF8
\usepackage[utf8]{inputenc}

% This is not strictly necessary, and may be commented out,
% but it will improve the layout of the manuscript,
% and will typically save some space.
\usepackage{microtype}

% This is also not strictly necessary, and may be commented out.
% However, it will improve the aesthetics of text in
% the typewriter font.
\usepackage{inconsolata}

%Including images in your LaTeX document requires adding
%additional package(s)
\usepackage{graphicx}

% If the title and author information does not fit in the area allocated, uncomment the following
%
%\setlength\titlebox{<dim>}
%
% and set <dim> to something 5cm or larger.
%%%%% NEW MATH DEFINITIONS %%%%%

\usepackage{amsmath,amsfonts,bm}
\usepackage{derivative}
% Mark sections of captions for referring to divisions of figures
\newcommand{\figleft}{{\em (Left)}}
\newcommand{\figcenter}{{\em (Center)}}
\newcommand{\figright}{{\em (Right)}}
\newcommand{\figtop}{{\em (Top)}}
\newcommand{\figbottom}{{\em (Bottom)}}
\newcommand{\captiona}{{\em (a)}}
\newcommand{\captionb}{{\em (b)}}
\newcommand{\captionc}{{\em (c)}}
\newcommand{\captiond}{{\em (d)}}

% Highlight a newly defined term
\newcommand{\newterm}[1]{{\bf #1}}

% Derivative d 
\newcommand{\deriv}{{\mathrm{d}}}

% Figure reference, lower-case.
\def\figref#1{figure~\ref{#1}}
% Figure reference, capital. For start of sentence
\def\Figref#1{Figure~\ref{#1}}
\def\twofigref#1#2{figures \ref{#1} and \ref{#2}}
\def\quadfigref#1#2#3#4{figures \ref{#1}, \ref{#2}, \ref{#3} and \ref{#4}}
% Section reference, lower-case.
\def\secref#1{section~\ref{#1}}
% Section reference, capital.
\def\Secref#1{Section~\ref{#1}}
% Reference to two sections.
\def\twosecrefs#1#2{sections \ref{#1} and \ref{#2}}
% Reference to three sections.
\def\secrefs#1#2#3{sections \ref{#1}, \ref{#2} and \ref{#3}}
% Reference to an equation, lower-case.
\def\eqref#1{equation~\ref{#1}}
% Reference to an equation, upper case
\def\Eqref#1{Equation~\ref{#1}}
% A raw reference to an equation---avoid using if possible
\def\plaineqref#1{\ref{#1}}
% Reference to a chapter, lower-case.
\def\chapref#1{chapter~\ref{#1}}
% Reference to an equation, upper case.
\def\Chapref#1{Chapter~\ref{#1}}
% Reference to a range of chapters
\def\rangechapref#1#2{chapters\ref{#1}--\ref{#2}}
% Reference to an algorithm, lower-case.
\def\algref#1{algorithm~\ref{#1}}
% Reference to an algorithm, upper case.
\def\Algref#1{Algorithm~\ref{#1}}
\def\twoalgref#1#2{algorithms \ref{#1} and \ref{#2}}
\def\Twoalgref#1#2{Algorithms \ref{#1} and \ref{#2}}
% Reference to a part, lower case
\def\partref#1{part~\ref{#1}}
% Reference to a part, upper case
\def\Partref#1{Part~\ref{#1}}
\def\twopartref#1#2{parts \ref{#1} and \ref{#2}}

\def\ceil#1{\lceil #1 \rceil}
\def\floor#1{\lfloor #1 \rfloor}
\def\1{\bm{1}}
\newcommand{\train}{\mathcal{D}}
\newcommand{\valid}{\mathcal{D_{\mathrm{valid}}}}
\newcommand{\test}{\mathcal{D_{\mathrm{test}}}}

\def\eps{{\epsilon}}


% Random variables
\def\reta{{\textnormal{$\eta$}}}
\def\ra{{\textnormal{a}}}
\def\rb{{\textnormal{b}}}
\def\rc{{\textnormal{c}}}
\def\rd{{\textnormal{d}}}
\def\re{{\textnormal{e}}}
\def\rf{{\textnormal{f}}}
\def\rg{{\textnormal{g}}}
\def\rh{{\textnormal{h}}}
\def\ri{{\textnormal{i}}}
\def\rj{{\textnormal{j}}}
\def\rk{{\textnormal{k}}}
\def\rl{{\textnormal{l}}}
% rm is already a command, just don't name any random variables m
\def\rn{{\textnormal{n}}}
\def\ro{{\textnormal{o}}}
\def\rp{{\textnormal{p}}}
\def\rq{{\textnormal{q}}}
\def\rr{{\textnormal{r}}}
\def\rs{{\textnormal{s}}}
\def\rt{{\textnormal{t}}}
\def\ru{{\textnormal{u}}}
\def\rv{{\textnormal{v}}}
\def\rw{{\textnormal{w}}}
\def\rx{{\textnormal{x}}}
\def\ry{{\textnormal{y}}}
\def\rz{{\textnormal{z}}}

% Random vectors
\def\rvepsilon{{\mathbf{\epsilon}}}
\def\rvphi{{\mathbf{\phi}}}
\def\rvtheta{{\mathbf{\theta}}}
\def\rva{{\mathbf{a}}}
\def\rvb{{\mathbf{b}}}
\def\rvc{{\mathbf{c}}}
\def\rvd{{\mathbf{d}}}
\def\rve{{\mathbf{e}}}
\def\rvf{{\mathbf{f}}}
\def\rvg{{\mathbf{g}}}
\def\rvh{{\mathbf{h}}}
\def\rvu{{\mathbf{i}}}
\def\rvj{{\mathbf{j}}}
\def\rvk{{\mathbf{k}}}
\def\rvl{{\mathbf{l}}}
\def\rvm{{\mathbf{m}}}
\def\rvn{{\mathbf{n}}}
\def\rvo{{\mathbf{o}}}
\def\rvp{{\mathbf{p}}}
\def\rvq{{\mathbf{q}}}
\def\rvr{{\mathbf{r}}}
\def\rvs{{\mathbf{s}}}
\def\rvt{{\mathbf{t}}}
\def\rvu{{\mathbf{u}}}
\def\rvv{{\mathbf{v}}}
\def\rvw{{\mathbf{w}}}
\def\rvx{{\mathbf{x}}}
\def\rvy{{\mathbf{y}}}
\def\rvz{{\mathbf{z}}}

% Elements of random vectors
\def\erva{{\textnormal{a}}}
\def\ervb{{\textnormal{b}}}
\def\ervc{{\textnormal{c}}}
\def\ervd{{\textnormal{d}}}
\def\erve{{\textnormal{e}}}
\def\ervf{{\textnormal{f}}}
\def\ervg{{\textnormal{g}}}
\def\ervh{{\textnormal{h}}}
\def\ervi{{\textnormal{i}}}
\def\ervj{{\textnormal{j}}}
\def\ervk{{\textnormal{k}}}
\def\ervl{{\textnormal{l}}}
\def\ervm{{\textnormal{m}}}
\def\ervn{{\textnormal{n}}}
\def\ervo{{\textnormal{o}}}
\def\ervp{{\textnormal{p}}}
\def\ervq{{\textnormal{q}}}
\def\ervr{{\textnormal{r}}}
\def\ervs{{\textnormal{s}}}
\def\ervt{{\textnormal{t}}}
\def\ervu{{\textnormal{u}}}
\def\ervv{{\textnormal{v}}}
\def\ervw{{\textnormal{w}}}
\def\ervx{{\textnormal{x}}}
\def\ervy{{\textnormal{y}}}
\def\ervz{{\textnormal{z}}}

% Random matrices
\def\rmA{{\mathbf{A}}}
\def\rmB{{\mathbf{B}}}
\def\rmC{{\mathbf{C}}}
\def\rmD{{\mathbf{D}}}
\def\rmE{{\mathbf{E}}}
\def\rmF{{\mathbf{F}}}
\def\rmG{{\mathbf{G}}}
\def\rmH{{\mathbf{H}}}
\def\rmI{{\mathbf{I}}}
\def\rmJ{{\mathbf{J}}}
\def\rmK{{\mathbf{K}}}
\def\rmL{{\mathbf{L}}}
\def\rmM{{\mathbf{M}}}
\def\rmN{{\mathbf{N}}}
\def\rmO{{\mathbf{O}}}
\def\rmP{{\mathbf{P}}}
\def\rmQ{{\mathbf{Q}}}
\def\rmR{{\mathbf{R}}}
\def\rmS{{\mathbf{S}}}
\def\rmT{{\mathbf{T}}}
\def\rmU{{\mathbf{U}}}
\def\rmV{{\mathbf{V}}}
\def\rmW{{\mathbf{W}}}
\def\rmX{{\mathbf{X}}}
\def\rmY{{\mathbf{Y}}}
\def\rmZ{{\mathbf{Z}}}

% Elements of random matrices
\def\ermA{{\textnormal{A}}}
\def\ermB{{\textnormal{B}}}
\def\ermC{{\textnormal{C}}}
\def\ermD{{\textnormal{D}}}
\def\ermE{{\textnormal{E}}}
\def\ermF{{\textnormal{F}}}
\def\ermG{{\textnormal{G}}}
\def\ermH{{\textnormal{H}}}
\def\ermI{{\textnormal{I}}}
\def\ermJ{{\textnormal{J}}}
\def\ermK{{\textnormal{K}}}
\def\ermL{{\textnormal{L}}}
\def\ermM{{\textnormal{M}}}
\def\ermN{{\textnormal{N}}}
\def\ermO{{\textnormal{O}}}
\def\ermP{{\textnormal{P}}}
\def\ermQ{{\textnormal{Q}}}
\def\ermR{{\textnormal{R}}}
\def\ermS{{\textnormal{S}}}
\def\ermT{{\textnormal{T}}}
\def\ermU{{\textnormal{U}}}
\def\ermV{{\textnormal{V}}}
\def\ermW{{\textnormal{W}}}
\def\ermX{{\textnormal{X}}}
\def\ermY{{\textnormal{Y}}}
\def\ermZ{{\textnormal{Z}}}

% Vectors
\def\vzero{{\bm{0}}}
\def\vone{{\bm{1}}}
\def\vmu{{\bm{\mu}}}
\def\vtheta{{\bm{\theta}}}
\def\vphi{{\bm{\phi}}}
\def\va{{\bm{a}}}
\def\vb{{\bm{b}}}
\def\vc{{\bm{c}}}
\def\vd{{\bm{d}}}
\def\ve{{\bm{e}}}
\def\vf{{\bm{f}}}
\def\vg{{\bm{g}}}
\def\vh{{\bm{h}}}
\def\vi{{\bm{i}}}
\def\vj{{\bm{j}}}
\def\vk{{\bm{k}}}
\def\vl{{\bm{l}}}
\def\vm{{\bm{m}}}
\def\vn{{\bm{n}}}
\def\vo{{\bm{o}}}
\def\vp{{\bm{p}}}
\def\vq{{\bm{q}}}
\def\vr{{\bm{r}}}
\def\vs{{\bm{s}}}
\def\vt{{\bm{t}}}
\def\vu{{\bm{u}}}
\def\vv{{\bm{v}}}
\def\vw{{\bm{w}}}
\def\vx{{\bm{x}}}
\def\vy{{\bm{y}}}
\def\vz{{\bm{z}}}

% Elements of vectors
\def\evalpha{{\alpha}}
\def\evbeta{{\beta}}
\def\evepsilon{{\epsilon}}
\def\evlambda{{\lambda}}
\def\evomega{{\omega}}
\def\evmu{{\mu}}
\def\evpsi{{\psi}}
\def\evsigma{{\sigma}}
\def\evtheta{{\theta}}
\def\eva{{a}}
\def\evb{{b}}
\def\evc{{c}}
\def\evd{{d}}
\def\eve{{e}}
\def\evf{{f}}
\def\evg{{g}}
\def\evh{{h}}
\def\evi{{i}}
\def\evj{{j}}
\def\evk{{k}}
\def\evl{{l}}
\def\evm{{m}}
\def\evn{{n}}
\def\evo{{o}}
\def\evp{{p}}
\def\evq{{q}}
\def\evr{{r}}
\def\evs{{s}}
\def\evt{{t}}
\def\evu{{u}}
\def\evv{{v}}
\def\evw{{w}}
\def\evx{{x}}
\def\evy{{y}}
\def\evz{{z}}

% Matrix
\def\mA{{\bm{A}}}
\def\mB{{\bm{B}}}
\def\mC{{\bm{C}}}
\def\mD{{\bm{D}}}
\def\mE{{\bm{E}}}
\def\mF{{\bm{F}}}
\def\mG{{\bm{G}}}
\def\mH{{\bm{H}}}
\def\mI{{\bm{I}}}
\def\mJ{{\bm{J}}}
\def\mK{{\bm{K}}}
\def\mL{{\bm{L}}}
\def\mM{{\bm{M}}}
\def\mN{{\bm{N}}}
\def\mO{{\bm{O}}}
\def\mP{{\bm{P}}}
\def\mQ{{\bm{Q}}}
\def\mR{{\bm{R}}}
\def\mS{{\bm{S}}}
\def\mT{{\bm{T}}}
\def\mU{{\bm{U}}}
\def\mV{{\bm{V}}}
\def\mW{{\bm{W}}}
\def\mX{{\bm{X}}}
\def\mY{{\bm{Y}}}
\def\mZ{{\bm{Z}}}
\def\mBeta{{\bm{\beta}}}
\def\mPhi{{\bm{\Phi}}}
\def\mLambda{{\bm{\Lambda}}}
\def\mSigma{{\bm{\Sigma}}}

% Tensor
\DeclareMathAlphabet{\mathsfit}{\encodingdefault}{\sfdefault}{m}{sl}
\SetMathAlphabet{\mathsfit}{bold}{\encodingdefault}{\sfdefault}{bx}{n}
\newcommand{\tens}[1]{\bm{\mathsfit{#1}}}
\def\tA{{\tens{A}}}
\def\tB{{\tens{B}}}
\def\tC{{\tens{C}}}
\def\tD{{\tens{D}}}
\def\tE{{\tens{E}}}
\def\tF{{\tens{F}}}
\def\tG{{\tens{G}}}
\def\tH{{\tens{H}}}
\def\tI{{\tens{I}}}
\def\tJ{{\tens{J}}}
\def\tK{{\tens{K}}}
\def\tL{{\tens{L}}}
\def\tM{{\tens{M}}}
\def\tN{{\tens{N}}}
\def\tO{{\tens{O}}}
\def\tP{{\tens{P}}}
\def\tQ{{\tens{Q}}}
\def\tR{{\tens{R}}}
\def\tS{{\tens{S}}}
\def\tT{{\tens{T}}}
\def\tU{{\tens{U}}}
\def\tV{{\tens{V}}}
\def\tW{{\tens{W}}}
\def\tX{{\tens{X}}}
\def\tY{{\tens{Y}}}
\def\tZ{{\tens{Z}}}


% Graph
\def\gA{{\mathcal{A}}}
\def\gB{{\mathcal{B}}}
\def\gC{{\mathcal{C}}}
\def\gD{{\mathcal{D}}}
\def\gE{{\mathcal{E}}}
\def\gF{{\mathcal{F}}}
\def\gG{{\mathcal{G}}}
\def\gH{{\mathcal{H}}}
\def\gI{{\mathcal{I}}}
\def\gJ{{\mathcal{J}}}
\def\gK{{\mathcal{K}}}
\def\gL{{\mathcal{L}}}
\def\gM{{\mathcal{M}}}
\def\gN{{\mathcal{N}}}
\def\gO{{\mathcal{O}}}
\def\gP{{\mathcal{P}}}
\def\gQ{{\mathcal{Q}}}
\def\gR{{\mathcal{R}}}
\def\gS{{\mathcal{S}}}
\def\gT{{\mathcal{T}}}
\def\gU{{\mathcal{U}}}
\def\gV{{\mathcal{V}}}
\def\gW{{\mathcal{W}}}
\def\gX{{\mathcal{X}}}
\def\gY{{\mathcal{Y}}}
\def\gZ{{\mathcal{Z}}}

% Sets
\def\sA{{\mathbb{A}}}
\def\sB{{\mathbb{B}}}
\def\sC{{\mathbb{C}}}
\def\sD{{\mathbb{D}}}
% Don't use a set called E, because this would be the same as our symbol
% for expectation.
\def\sF{{\mathbb{F}}}
\def\sG{{\mathbb{G}}}
\def\sH{{\mathbb{H}}}
\def\sI{{\mathbb{I}}}
\def\sJ{{\mathbb{J}}}
\def\sK{{\mathbb{K}}}
\def\sL{{\mathbb{L}}}
\def\sM{{\mathbb{M}}}
\def\sN{{\mathbb{N}}}
\def\sO{{\mathbb{O}}}
\def\sP{{\mathbb{P}}}
\def\sQ{{\mathbb{Q}}}
\def\sR{{\mathbb{R}}}
\def\sS{{\mathbb{S}}}
\def\sT{{\mathbb{T}}}
\def\sU{{\mathbb{U}}}
\def\sV{{\mathbb{V}}}
\def\sW{{\mathbb{W}}}
\def\sX{{\mathbb{X}}}
\def\sY{{\mathbb{Y}}}
\def\sZ{{\mathbb{Z}}}

% Entries of a matrix
\def\emLambda{{\Lambda}}
\def\emA{{A}}
\def\emB{{B}}
\def\emC{{C}}
\def\emD{{D}}
\def\emE{{E}}
\def\emF{{F}}
\def\emG{{G}}
\def\emH{{H}}
\def\emI{{I}}
\def\emJ{{J}}
\def\emK{{K}}
\def\emL{{L}}
\def\emM{{M}}
\def\emN{{N}}
\def\emO{{O}}
\def\emP{{P}}
\def\emQ{{Q}}
\def\emR{{R}}
\def\emS{{S}}
\def\emT{{T}}
\def\emU{{U}}
\def\emV{{V}}
\def\emW{{W}}
\def\emX{{X}}
\def\emY{{Y}}
\def\emZ{{Z}}
\def\emSigma{{\Sigma}}

% entries of a tensor
% Same font as tensor, without \bm wrapper
\newcommand{\etens}[1]{\mathsfit{#1}}
\def\etLambda{{\etens{\Lambda}}}
\def\etA{{\etens{A}}}
\def\etB{{\etens{B}}}
\def\etC{{\etens{C}}}
\def\etD{{\etens{D}}}
\def\etE{{\etens{E}}}
\def\etF{{\etens{F}}}
\def\etG{{\etens{G}}}
\def\etH{{\etens{H}}}
\def\etI{{\etens{I}}}
\def\etJ{{\etens{J}}}
\def\etK{{\etens{K}}}
\def\etL{{\etens{L}}}
\def\etM{{\etens{M}}}
\def\etN{{\etens{N}}}
\def\etO{{\etens{O}}}
\def\etP{{\etens{P}}}
\def\etQ{{\etens{Q}}}
\def\etR{{\etens{R}}}
\def\etS{{\etens{S}}}
\def\etT{{\etens{T}}}
\def\etU{{\etens{U}}}
\def\etV{{\etens{V}}}
\def\etW{{\etens{W}}}
\def\etX{{\etens{X}}}
\def\etY{{\etens{Y}}}
\def\etZ{{\etens{Z}}}

% The true underlying data generating distribution
\newcommand{\pdata}{p_{\rm{data}}}
\newcommand{\ptarget}{p_{\rm{target}}}
\newcommand{\pprior}{p_{\rm{prior}}}
\newcommand{\pbase}{p_{\rm{base}}}
\newcommand{\pref}{p_{\rm{ref}}}

% The empirical distribution defined by the training set
\newcommand{\ptrain}{\hat{p}_{\rm{data}}}
\newcommand{\Ptrain}{\hat{P}_{\rm{data}}}
% The model distribution
\newcommand{\pmodel}{p_{\rm{model}}}
\newcommand{\Pmodel}{P_{\rm{model}}}
\newcommand{\ptildemodel}{\tilde{p}_{\rm{model}}}
% Stochastic autoencoder distributions
\newcommand{\pencode}{p_{\rm{encoder}}}
\newcommand{\pdecode}{p_{\rm{decoder}}}
\newcommand{\precons}{p_{\rm{reconstruct}}}

\newcommand{\laplace}{\mathrm{Laplace}} % Laplace distribution

\newcommand{\E}{\mathbb{E}}
\newcommand{\Ls}{\mathcal{L}}
\newcommand{\R}{\mathbb{R}}
\newcommand{\emp}{\tilde{p}}
\newcommand{\lr}{\alpha}
\newcommand{\reg}{\lambda}
\newcommand{\rect}{\mathrm{rectifier}}
\newcommand{\softmax}{\mathrm{softmax}}
\newcommand{\sigmoid}{\sigma}
\newcommand{\softplus}{\zeta}
\newcommand{\KL}{D_{\mathrm{KL}}}
\newcommand{\Var}{\mathrm{Var}}
\newcommand{\standarderror}{\mathrm{SE}}
\newcommand{\Cov}{\mathrm{Cov}}
% Wolfram Mathworld says $L^2$ is for function spaces and $\ell^2$ is for vectors
% But then they seem to use $L^2$ for vectors throughout the site, and so does
% wikipedia.
\newcommand{\normlzero}{L^0}
\newcommand{\normlone}{L^1}
\newcommand{\normltwo}{L^2}
\newcommand{\normlp}{L^p}
\newcommand{\normmax}{L^\infty}

\newcommand{\parents}{Pa} % See usage in notation.tex. Chosen to match Daphne's book.

\DeclareMathOperator*{\argmax}{arg\,max}
\DeclareMathOperator*{\argmin}{arg\,min}

\DeclareMathOperator{\sign}{sign}
\DeclareMathOperator{\Tr}{Tr}
\let\ab\allowbreak

\usepackage{xcolor}
\usepackage{xspace}
\usepackage{soul}
\usepackage{tcolorbox}
\usepackage{listings}
\usepackage{booktabs}
\usepackage{multirow}
\usepackage{multicol}
\usepackage{makecell}
\usepackage{enumitem}
\usepackage{balance}
\usepackage{graphicx}
\usepackage{subfigure}
\usepackage{amsmath}
\usepackage{amsthm}
\usepackage{amssymb}
\usepackage{amsfonts}
\usepackage{mathtools}
\usepackage{tikz}
\usepackage{array}
\usepackage{tabularx}  % 自动调整列宽
\usepackage{ragged2e}  % 更好的对齐控制
\usetikzlibrary{matrix,positioning,fit,shapes.geometric}
\usetikzlibrary{arrows.meta}
\usepackage{cleveref}
\definecolor{mycitecolor}{HTML}{3A76CF}
\hypersetup{
    colorlinks=true,
    citecolor=mycitecolor
} 

\newcommand{\citeneeded}[1][]{\colorbox{yellow}{\textbf{[CITE\ifx\empty#1\empty\else: #1\fi]}}}
 

\definecolor{bg1}{HTML}{D9EBFC}
\definecolor{bg2}{HTML}{CEDDF3}
% 子结论:浅灰色高亮
\newcommand{\subconc}[1]{%
    \setulcolor{black!30}
    \setul{3.2pt}{0.2pt}
    \ul{#1}%
}

% 主结论:浅黄色高亮 + 微粗体
\newcommand{\mainconc}[1]{%
    %\sethlcolor{lightblue}%
    %\hl{\textmd{#1}}%
    \textbf{#1}
}

\definecolor{darkred}{HTML}{C23B22}
\definecolor{green}{HTML}{1cc650}
\definecolor{darkergreen}{HTML}{006400}
\newenvironment{promptbox}[4][] % [Title (optional)]{Prompt}{Baseline}{Intervention}
{
  % Cannot use empty lines inside the arguments
  \begin{tcolorbox}[left=1.5mm, right=1.5mm, top=1.5mm, bottom=1.5mm]
    \raggedright
    \small
    \ifx\relax#1\relax\else
      \begin{center}
        {\normalsize \textbf{\color{black} #1}}
      \end{center}
    \fi
    \textcolor{black}{\textbf{Prompt:} {\texttt{#2}}} \\[2pt]
    \textcolor{darkergreen}{\textbf{Generation (w/o TempPatch):} {\texttt{#3}}} \\[2pt]
    \textcolor{darkred}{\textbf{Generation (w/ TempPatch):} {\texttt{#4}}}
  \end{tcolorbox}
}{}



\title{\textit{Why Safeguarded Ships Run Aground?} Aligned Large Language Models' Safety Mechanisms Tend to Be Anchored in The Template Region}

% Author information can be set in various styles:
% For several authors from the same institution:
% \author{Author 1 \and ... \and Author n \\
%         Address line \\ ... \\ Address line}
% if the names do not fit well on one line use
%         Author 1 \\ {\bf Author 2} \\ ... \\ {\bf Author n} \\
% For authors from different institutions:
% \author{Author 1 \\ Address line \\  ... \\ Address line
%         \And  ... \And
%         Author n \\ Address line \\ ... \\ Address line}
% To start a separate ``row'' of authors use \AND, as in
% \author{Author 1 \\ Address line \\  ... \\ Address line
%         \AND
%         Author 2 \\ Address line \\ ... \\ Address line \And
%         Author 3 \\ Address line \\ ... \\ Address line}

\author{
Chak Tou Leong$^1$, Qingyu Yin$^{2}$, Jian Wang$^{1 \dagger}$, Wenjie Li$^{1 \dagger}$ \\
$^1$ Department of Computing, The Hong Kong Polytechnic University \\
$^2$ Zhejiang University \\
\texttt{chak-tou.leong@connect.polyu.hk} ~~
\texttt{qingyu.yin@zju.edu.cn} \\
\texttt{jian51.wang@polyu.edu.hk} ~~
\texttt{cswjli@comp.polyu.edu.hk}
}



%\author{
%  \textbf{First Author\textsuperscript{1}},
%  \textbf{Second Author\textsuperscript{1,2}},
%  \textbf{Third T. Author\textsuperscript{1}},
%  \textbf{Fourth Author\textsuperscript{1}},
%\\
%  \textbf{Fifth Author\textsuperscript{1,2}},
%  \textbf{Sixth Author\textsuperscript{1}},
%  \textbf{Seventh Author\textsuperscript{1}},
%  \textbf{Eighth Author \textsuperscript{1,2,3,4}},
%\\
%  \textbf{Ninth Author\textsuperscript{1}},
%  \textbf{Tenth Author\textsuperscript{1}},
%  \textbf{Eleventh E. Author\textsuperscript{1,2,3,4,5}},
%  \textbf{Twelfth Author\textsuperscript{1}},
%\\
%  \textbf{Thirteenth Author\textsuperscript{3}},
%  \textbf{Fourteenth F. Author\textsuperscript{2,4}},
%  \textbf{Fifteenth Author\textsuperscript{1}},
%  \textbf{Sixteenth Author\textsuperscript{1}},
%\\
%  \textbf{Seventeenth S. Author\textsuperscript{4,5}},
%  \textbf{Eighteenth Author\textsuperscript{3,4}},
%  \textbf{Nineteenth N. Author\textsuperscript{2,5}},
%  \textbf{Twentieth Author\textsuperscript{1}}
%\\
%\\
%  \textsuperscript{1}Affiliation 1,
%  \textsuperscript{2}Affiliation 2,
%  \textsuperscript{3}Affiliation 3,
%  \textsuperscript{4}Affiliation 4,
%  \textsuperscript{5}Affiliation 5
%\\
%  \small{
%    \textbf{Correspondence:} \href{mailto:email@domain}{email@domain}
%  }
%}

\begin{document}

\maketitle

% self-define the footnote symbol
\renewcommand{\thefootnote}{$\dagger$}
\footnotetext[1]{Corresponding authors.}
% reset footnote counter
\setcounter{footnote}{0}
\renewcommand{\thefootnote}{\arabic{footnote}}
%\footnotetext{$^\dagger$ Corresponding authors.}

\begin{abstract}

The safety alignment of large language models (LLMs) remains vulnerable, as their initial behavior can be easily jailbroken by even relatively simple attacks. Since infilling a fixed template between the input instruction and initial model output is a common practice for existing LLMs, we hypothesize that this template is a key factor behind their vulnerabilities: LLMs' safety-related decision-making overly relies on the aggregated information from the template region, which largely influences these models' safety behavior. We refer to this issue as \textit{template-anchored safety alignment}.
In this paper, we conduct extensive experiments and verify that template-anchored safety alignment is widespread across various aligned LLMs. Our mechanistic analyses demonstrate how it leads to models' susceptibility when encountering inference-time jailbreak attacks. Furthermore, we show that detaching safety mechanisms from the template region is promising in mitigating vulnerabilities to jailbreak attacks. We encourage future research to develop more robust safety alignment techniques that reduce reliance on the template region.

\end{abstract}

\section{Introduction}
Backdoor attacks pose a concealed yet profound security risk to machine learning (ML) models, for which the adversaries can inject a stealth backdoor into the model during training, enabling them to illicitly control the model's output upon encountering predefined inputs. These attacks can even occur without the knowledge of developers or end-users, thereby undermining the trust in ML systems. As ML becomes more deeply embedded in critical sectors like finance, healthcare, and autonomous driving \citep{he2016deep, liu2020computing, tournier2019mrtrix3, adjabi2020past}, the potential damage from backdoor attacks grows, underscoring the emergency for developing robust defense mechanisms against backdoor attacks.

To address the threat of backdoor attacks, researchers have developed a variety of strategies \cite{liu2018fine,wu2021adversarial,wang2019neural,zeng2022adversarial,zhu2023neural,Zhu_2023_ICCV, wei2024shared,wei2024d3}, aimed at purifying backdoors within victim models. These methods are designed to integrate with current deployment workflows seamlessly and have demonstrated significant success in mitigating the effects of backdoor triggers \cite{wubackdoorbench, wu2023defenses, wu2024backdoorbench,dunnett2024countering}.  However, most state-of-the-art (SOTA) backdoor purification methods operate under the assumption that a small clean dataset, often referred to as \textbf{auxiliary dataset}, is available for purification. Such an assumption poses practical challenges, especially in scenarios where data is scarce. To tackle this challenge, efforts have been made to reduce the size of the required auxiliary dataset~\cite{chai2022oneshot,li2023reconstructive, Zhu_2023_ICCV} and even explore dataset-free purification techniques~\cite{zheng2022data,hong2023revisiting,lin2024fusing}. Although these approaches offer some improvements, recent evaluations \cite{dunnett2024countering, wu2024backdoorbench} continue to highlight the importance of sufficient auxiliary data for achieving robust defenses against backdoor attacks.

While significant progress has been made in reducing the size of auxiliary datasets, an equally critical yet underexplored question remains: \emph{how does the nature of the auxiliary dataset affect purification effectiveness?} In  real-world  applications, auxiliary datasets can vary widely, encompassing in-distribution data, synthetic data, or external data from different sources. Understanding how each type of auxiliary dataset influences the purification effectiveness is vital for selecting or constructing the most suitable auxiliary dataset and the corresponding technique. For instance, when multiple datasets are available, understanding how different datasets contribute to purification can guide defenders in selecting or crafting the most appropriate dataset. Conversely, when only limited auxiliary data is accessible, knowing which purification technique works best under those constraints is critical. Therefore, there is an urgent need for a thorough investigation into the impact of auxiliary datasets on purification effectiveness to guide defenders in  enhancing the security of ML systems. 

In this paper, we systematically investigate the critical role of auxiliary datasets in backdoor purification, aiming to bridge the gap between idealized and practical purification scenarios.  Specifically, we first construct a diverse set of auxiliary datasets to emulate real-world conditions, as summarized in Table~\ref{overall}. These datasets include in-distribution data, synthetic data, and external data from other sources. Through an evaluation of SOTA backdoor purification methods across these datasets, we uncover several critical insights: \textbf{1)} In-distribution datasets, particularly those carefully filtered from the original training data of the victim model, effectively preserve the model’s utility for its intended tasks but may fall short in eliminating backdoors. \textbf{2)} Incorporating OOD datasets can help the model forget backdoors but also bring the risk of forgetting critical learned knowledge, significantly degrading its overall performance. Building on these findings, we propose Guided Input Calibration (GIC), a novel technique that enhances backdoor purification by adaptively transforming auxiliary data to better align with the victim model’s learned representations. By leveraging the victim model itself to guide this transformation, GIC optimizes the purification process, striking a balance between preserving model utility and mitigating backdoor threats. Extensive experiments demonstrate that GIC significantly improves the effectiveness of backdoor purification across diverse auxiliary datasets, providing a practical and robust defense solution.

Our main contributions are threefold:
\textbf{1) Impact analysis of auxiliary datasets:} We take the \textbf{first step}  in systematically investigating how different types of auxiliary datasets influence backdoor purification effectiveness. Our findings provide novel insights and serve as a foundation for future research on optimizing dataset selection and construction for enhanced backdoor defense.
%
\textbf{2) Compilation and evaluation of diverse auxiliary datasets:}  We have compiled and rigorously evaluated a diverse set of auxiliary datasets using SOTA purification methods, making our datasets and code publicly available to facilitate and support future research on practical backdoor defense strategies.
%
\textbf{3) Introduction of GIC:} We introduce GIC, the \textbf{first} dedicated solution designed to align auxiliary datasets with the model’s learned representations, significantly enhancing backdoor mitigation across various dataset types. Our approach sets a new benchmark for practical and effective backdoor defense.



\section{Preliminary} \label{sec:preliminary}
\paragraph{Random variable and distribution.} Let $\mathcal{X} = \mathcal{X}_v \times \mathcal{X}_t$ denote the input space, where $\mathcal{X}_v$ and $\mathcal{X}_t$ correspond to the visual and textual feature spaces, respectively. Similarly, let $\mathcal{Y}$ denote the response space. We define the random variables $\mathbf{X} = (X_v, X_t) \in \mathcal{X}$ and $Y \in \mathcal{Y}$, where $\mathbf{X}$ is the sequence of tokens that combine visual and text input queries, and $Y$ represents the associated response tokens. The joint population is denoted by $P_{\mathbf{X}Y}$, with marginals $P_{\mathbf{X}}$, $P_{Y}$, and the conditional distribution $P_{Y|\mathbf{X}}$. For subsequent sections, $P_{\mathbf{X}Y}$ refers to the instruction tuning data distribution which we consider as in-distribution (ID). 

\paragraph{MLLM and visual instruction tuning.} MLLM usually consists of three components: (1) a visual encoder, (2) a vision-to-language projector, and (3) an LLM that processes a multimodal input sequence to generate a valid textual output $y$ in response to an input query $\mathbf{x}$. An MLLM can be regarded as modeling a conditional distribution $P_{\theta}(y|\mathbf{x})$, where $\theta$ is the model parameters. To attain the multimodal conversation capability, MLLMs commonly undergo a phase so-called \textit{visual instruction tuning} \cite{liu2023visual, dai2023instructblip} with an autoregressive objective as follows:
{
\begin{align} \label{eq::1}
    % & \min_{\theta\in\Theta} \mathbb{E}_{\mathbf{x},y\sim P_{\mathbf{X}Y}} [-\log P_{\theta}(y|\mathbf{x})] \nonumber \\
     \min_{\theta\in\Theta} \mathbb{E}_{\mathbf{x},y\sim P_{\mathbf{X}Y}} [\sum_{l=0}^{L}-\log P_{\theta}(y_{l}|\mathbf{x},y_{<l})],
\end{align}}
where $L$ is a sequence length and $y=(y_{0},...,y_{L})$. After being trained by Eq. \eqref{eq::1}, MLLM produces a response given a query of any possible tasks represented by text.

\paragraph{Evaluation of open-ended generations.} 

(M)LLM-as-a-judge method \cite{zheng2023judging, kim2023prometheus} is commonly adopted to evaluate open-ended generation. In this paradigm, a judge model produces preference scores or rankings for the responses given a query, model responses, and a scoring rubric. Among the evaluation metrics, the \emph{win rate} (Eq. \eqref{eq:win_rate}) is one of the most widely used and representative.


\begin{definition}[\textbf{Win Rate}] Given a parametric reward function $r:\mathcal{X}\times \mathcal{Y}\rightarrow \mathbb{R}$, the 
win rate (WR) of model $P_{\theta}$ w.r.t. $P_{\mathbf{X}Y}$ are defined as follows: 
\begin{equation} \label{eq:win_rate}
\begin{split}
    &\text{WR}(P_{\mathbf{X}Y};\theta):=\mathbb{E}_{\begin{subarray}{l} \mathbf{x},y \sim P_{\mathbf{X}Y} \\ \hat{y} \sim P_{\theta}(\cdot|\mathbf{x}) \end{subarray}}[\mathbb{I}(r(\mathbf{x},\hat{y}) > r(\mathbf{x},y))],
\end{split}
\end{equation}
where $\mathbb{I}(\cdot)$ is  the indicator function.
\end{definition}
Here, the reward function $r(\cdot,\cdot)$, can be any possible (multimodal) LLMs such as GPT-4o \cite{hurst2024gpt}.


\section{The Template-Anchored Safety
Alignment in Aligned LLMs}
\label{sec:rq1}

% In this section, we provide evidence that (1) aligned LLMs shift their attention from the instruction region to the template region (\Cref{subsec:attn_shift}), and (2) their refusal capability causally relies on the information processed in the template region (\Cref{subsec:temp_patching}).
%In this section, we demonstrate that the template-anchored safety alignment issue is widespread across various aligned LLMs.

\begin{figure*}[t]
    \centering
    \begin{tikzpicture}[scale=0.85] % Added scale factor to fit double columns
        % Define colors
        \definecolor{orangetext}{RGB}{255,99,71}
        \definecolor{bluetext}{RGB}{30,144,255}
        
       % Left figure
        \node at (6,3) {\includegraphics[width=0.7\textwidth]{Figures/2_1_attn-sums.pdf}};
        
        % Right figure
        \node at (15.5,3) {\includegraphics[width=0.28\textwidth]{Figures/2_2_Meta-Llama-3-8B-Instruct_L17H21.pdf}};
        
         % Vertical dashed line between figures
        \draw[dashed, line width=0.8pt] (12.7,-0.5) -- (12.7,6.8);
        
        % Split the left text into two lines
        \node[anchor=west, text width=10cm, font=\tiny] at (0,6.7) {More heads become \textbf{\textcolor{orangetext}{less}} focused on this region.};
        \node[anchor=east, text width=10cm, align=right, font=\tiny] at (12.6,6.7) {More heads become \textbf{\textcolor{bluetext}{more}} focused on this region.};
        
        % Title for the right figure
        \node[anchor=west, font=\scriptsize] at (13.5,6.8) {L17H21 of Llama-3-8B-Instruct};
        
        % Arrows - moved up to match new text position
        \draw[{Triangle[fill=orangetext, length=1.2mm, width=1.2mm]}-, line width=1pt, orangetext] (-0.1,6.5) -- (5.1,6.5);
        \draw[-{Triangle[fill=orangetext, length=1.2mm, width=1.2mm]}, line width=1pt, bluetext] (7.3,6.5) -- (12.5,6.5);
    \end{tikzpicture}
    \caption{\textbf{Left:} Attention distributions across different LLMs demonstrate that their attentions shift systematically from the \textit{instruction} to the \textit{template} region when processing harmful inputs. \textbf{Right:} Attention heatmaps (17th-layer, 21st-head) from Llama-3-8B-Instruct consistently illustrate this distinct pattern.}
    \label{fig:attn_shift}
\end{figure*}

\subsection{Preliminaries}

\paragraph{Datasets.}
We construct two datasets, $\gD_{\text{anlz}}$ and $\gD_{\text{eval}}$, designed to analyze the behavioral differences of LLMs when handling harmless versus harmful inputs and to evaluate their refusal capabilities, respectively. 
Each dataset consists of paired \textit{harmful} and \textit{harmless} instructions. For $\gD_{\text{anlz}}$, harmful instructions are sourced from JailbreakBench \cite{chao2024jailbreakbench}, while for $\gD_{\text{anlz}}$, they are drawn from HarmBench's standard behavior test set \cite{mazeika2024harmbench}. The harmless counterparts in both datasets are sampled from Alpaca-Cleaned,
\footnote{https://huggingface.co/datasets/yahma/alpaca-cleaned}
a filtered version of Alpaca \cite{alpaca} that excludes refusal-triggering content.
To ensure a precise comparative analysis, each harmless instruction matches its harmful counterpart in token length. Since tokenization methods vary across models, we maintained separate versions of $\gD_{\text{anlz}}$ and $\gD_{\text{eval}}$ for each model.


\paragraph{Models.}
To validate the generality of our findings, we study a diverse set of safety fine-tuned models: Gemma-2 (2b-it, 9b-it) \cite{team2024gemma}, Llama-2-7b-Chat \cite{touvron2023llama}, Llama-3 (3.2-3b-Instruct, 8B-Instruct) \cite{dubey2024llama}, and Mistral-7B-Instruct \cite{jiang2023mistral}.



\subsection{Attention Shifts to The Template Region}
\label{subsec:attn_shift}
In modern LLMs based on attention mechanisms, the distribution of attention weights across different heads reflects which regions of information collectively influence the model's next token predictions \cite{bibal2022attention}. A notable observation is that when the model refuses harmful requests, its response often exhibits distinct patterns from the outset, for instance, initiating with the token `\texttt{Sorry}' as the first output \cite{zou2023universal, qi2024safety}. 
This suggests that if the model's safety function primarily depends on the template region, then when processing harmful inputs, the attention weights at the final input position should focus more on the template region, while exhibiting comparatively less focus on the instruction region.

\begin{figure*}[htb] 
    \centering
    \includegraphics[width=1.\textwidth]{figures/fig_sentence.pdf} 
     \centering
     \begin{minipage}{0.49\textwidth}
        \centering
        
        \begin{tabular}{p{0.3\textwidth} p{0.7\textwidth}}
            \textbf{\small  B} \\ \\
            \textbf{True:} & \typing{las teorias reducen los numeros} \\
            \textbf{Best subject:} & \typing{\textcolor[rgb]{0.204, 0.596, 0.859}{las teorias reducen los numeros}} \\
            \textbf{Median subject:} & \typing{\textcolor[rgb]{0.204, 0.596, 0.859}{las teorias }\textcolor[rgb]{0.882, 0.071, 0.188}{exig}\textcolor[rgb]{0.204, 0.596, 0.859}{en los }\textcolor[rgb]{0.882, 0.071, 0.188}{homb}\textcolor[rgb]{0.204, 0.596, 0.859}{ros}} \\
            \textbf{Worst subject:} & \typing{\textcolor[rgb]{0.204, 0.596, 0.859}{las} \textcolor[rgb]{0.882, 0.071, 0.188}{ranc}\textcolor[rgb]{0.204, 0.596, 0.859}{ias re}\textcolor[rgb]{0.882, 0.071, 0.188}{vis}\textcolor[rgb]{0.204, 0.596, 0.859}{en los numer}\textcolor[rgb]{0.882, 0.071, 0.188}{ad}} \\
            \addlinespace

        \end{tabular}
    \end{minipage}
    \hfill
    \begin{minipage}{0.49\textwidth}
        \centering
        \begin{tabular}{p{0.3\textwidth} p{0.7\textwidth}}
            \textbf{\small  } \\ \\
            \textbf{True:} & \typing{la estadistica sigue la distribucion} \\
            \textbf{Best subject:} & \typing{\textcolor[rgb]{0.204, 0.596, 0.859}{la estadistica sigue la distribucion}} \\
            \textbf{Median subject:} & \typing{\textcolor[rgb]{0.882, 0.071, 0.188}{stamistosa} \textcolor[rgb]{0.204, 0.596, 0.859}{sigue la distribucion}} \\
            \textbf{Worst subject:} & \typing{\textcolor[rgb]{0.204, 0.596, 0.859}{la estadistica} \textcolor[rgb]{0.882, 0.071, 0.188}{f}\textcolor[rgb]{0.204, 0.596, 0.859}{igu}\textcolor[rgb]{0.882, 0.071, 0.188}{ra de petrilla lo}} \\

            \addlinespace
    % \textbf{True:} & \typing{el centro describe las parabolas }\\
    % \textbf{Type:} & \typing{\underline{w}l centro describe las parabolas }\\
    % \textbf{Decode:} & \typing{\textcolor[rgb]{0.882, 0.071, 0.188}{\underline{e}}\textcolor[rgb]{0.204, 0.596, 0.859}{l centro} \textcolor[rgb]{0.882, 0.071, 0.188}{trece de} \textcolor[rgb]{0.204, 0.596, 0.859}{las} \textcolor[rgb]{0.882, 0.071, 0.188}{carabin}\textcolor[rgb]{0.204, 0.596, 0.859}{as} }\\
    % \addlinespace
    % \textbf{True:} & \typing{los usos ofrecen las ventajas energeticas }\\
    % \textbf{Type:} & \typing{los usos ofrecen las ventajas energeticas }\\
    % \textbf{Decode:} & \typing{\textcolor[rgb]{0.204, 0.596, 0.859}{los} \textcolor[rgb]{0.882, 0.071, 0.188}{para me}\textcolor[rgb]{0.204, 0.596, 0.859}{recen las venta}\textcolor[rgb]{0.882, 0.071, 0.188}{n}\textcolor[rgb]{0.204, 0.596, 0.859}{as e}\textcolor[rgb]{0.882, 0.071, 0.188}{structur}\textcolor[rgb]{0.204, 0.596, 0.859}{as} }\\
    % \addlinespace
    % \textbf{True:} & \typing{la explicacion aclara la pregunta de la evaluacion }\\
    % \textbf{Type:} & \typing{la explicacion aclara la pregunta de la evaluacion }\\
    % \textbf{Decode:} & \typing{\textcolor[rgb]{0.204, 0.596, 0.859}{la explicacion}\textcolor[rgb]{0.882, 0.071, 0.188}{es para} \textcolor[rgb]{0.204, 0.596, 0.859}{la pre}\textcolor[rgb]{0.882, 0.071, 0.188}{sentan a} \textcolor[rgb]{0.204, 0.596, 0.859}{la }\textcolor[rgb]{0.882, 0.071, 0.188}{respirar}\textcolor[rgb]{0.204, 0.596, 0.859}{on} }\\
    \end{tabular}
    \end{minipage}

    \caption{
        \textbf{Sentence-level performance for Best, Median and Worst MEG subjects.} \\
        \textbf{A.} Character-error-rate for three representative subjects. Each dot represents a unique sentence, with error bars indicating the standard error of the mean across repetitions. White dots corresponds to the sentences displayed below. 
        \textbf{B.} Decoding predictions for two sentences.  %Brain2Qwerty achieves CERs of 0.06, 0.19, and 0.25 for the best, median and worst subjects in the first sentence (left) and 0, 0.23 and 0.4 in the second sentence (right). 
        Several splitting seeds were used to obtain the predictions across sentences.
    }
    \label{fig:performance_sentence} % Label for referencing the figure
\end{figure*}
\paragraph{Method.}
To investigate whether the attention weights exhibit increased focus on the template region when processing harmful inputs, we analyze attention weight distributions across all heads for both the instruction and template regions. More importantly, we examine how these distributions differ between harmless and harmful inputs.

Formally, for $h$-th attention head in layer $\ell$, we compute the average attention weight accumulation over regions of interest. Let $\mathbf{A}^{\ell,h,j}_{T,i}$ denote the attention weight at the final position $T$ of the input that attends to the position $i$ in $j$-example, we define the regional attention accumulation for harmless (\(+\)) and harmful (\(-\)) inputs as:
\vspace{-0.5em}
\begin{equation}
\alpha^{\pm}_R(\ell,h) = \frac{1}{|\gD_{\text{anlz}}|} \sum_{j=1}^{|\gD_{\text{anlz}}|} \sum_{i \in \gI_R} \mathbf{A}^{\ell,h,j,\pm}_{T,i},
\end{equation}
where $R \in \{\text{inst}, \text{temp}\}$ indicates the region, with $\gI_{\text{inst}} = \{1,\dots,S\}$ and $\gI_{\text{temp}} = \{S+1,\dots,T\}$ being the position indices for the instruction and template region, respectively.

When processing harmful inputs compared to harmless ones, the attention shift is computed as:
\begin{equation}
\delta_R(\ell,h) = \alpha^{-}_R(\ell,h) - \alpha^{+}_R(\ell,h),
\end{equation}
where a positive \( \delta_R(\ell,h) \) indicates that region $R$ receives more attention from the given head when processing harmful inputs relative to harmless ones, whereas a negative value suggests the opposite.


\paragraph{Results.}

\Cref{fig:attn_shift} shows the distribution histograms of \( \delta_R \) from all heads across the compared LLMs. We observe that the template distributions exhibit longer and more pronounced tails on the positive side compared to the negative side, while the instruction distributions show the opposite trend. This consistent phenomenon observed across various safety-tuned LLMs suggests that \subconc{these models tend to focus more on the template region when processing harmful inputs, providing strong evidence for the existence of TASA}.

To illustrate this phenomenon more concretely, we showcase the behavior of a specific attention head (17th-layer, 21st-head) from Llama-3-8B-Instruct on the right side of \Cref{fig:attn_shift}. This example demonstrates how an individual head behaves differently when processing harmless versus harmful inputs. We observe that the attention weights at the final input position (i.e., `\texttt{\textbackslash n\textbackslash n}') show a clear focus shift from a concrete noun `\texttt{tea}' in the instruction to a role-indicating token `\texttt{assistant}' in the template region when the input is harmful.


\subsection{Causal Role of The Template Region}
\label{subsec:temp_patching}
While safety-tuned LLMs shift their attention toward the template region when processing harmful inputs, \textit{does this shift indicate a reliance on template information for safety-related decisions?} To confirm this, we verify whether intermediate states from the template region exert a greater influence on models' safety capabilities than those from the instruction region.


\paragraph{Evaluation Metric.}

Quantifying the influence of intermediate states typically involves causal effects, such as IE (see \Cref{para:patching}).
However, evaluating an LLM's safety capability by analyzing complete responses for each of its numerous internal states would be highly inefficient.
To address this, we adopt a lightweight surrogate metric following prior work \cite{lee2024mechanistic, arditi2024refusal}. This approach uses a linear probe on the last hidden states to estimate a model's likelihood of complying with harmful inputs.
The predicted logits for harmful inputs serve as an efficient proxy to measure the causal effects of intermediate states on safety capability, where higher logits for harmful inputs indicate weaker safety capability. Following difference-in-mean method \cite{arditi2024refusal, marks2024geometry}, we obtain the probe \(\vd^{+}\in \sR^d\) as follows:
\vspace{-0.6em}
\begin{equation}
    \vd^+ = \frac{1}{|\gD_{\text{anlz}}|}\sum_{j=1}^{|\gD_{\text{anlz}}|} \vx^{L,j,+}_{T} - \frac{1}{|\gD_{\text{anlz}}|}\sum_{j=1}^{|\gD_{\text{anlz}}|} \vx^{L,j,-}_{T},
\label{eq:diff_prob}
\end{equation}
where \( \vx^{L,j,\pm}_{T} \) is the residual stream from example \(j\) of either harmless (\(+\)) or harmful (\(-\)).  We then compute \(m(x) = \vx_{T}^{L} \vd^+\) and refer to it as the \textit{compliance metric}. 

A 5-fold cross-validation of the probe achieves an average accuracy of $98.7 \pm 0.7\%$ across models, demonstrating its effectiveness in distinguishing between safe and unsafe model behaviors.

\paragraph{Method.}

Consider a scenario where we input the last token in the template and aim to obtain whether the model intends to comply the input, as measured by the compliance probe. 
In this forward pass, the residual stream of the last token aggregates context information by fusing the previous value states \( \vv^{\ell,h}_{<T}\coloneqq\vx_{<T}^\ell\mW^{\ell,h}_V\) in every attention head.
To compute the causal effects of intermediate states from different regions, we calculate the IE when patching the value states of harmful input with those of harmless input for one region, while leaving the states unchanged for the other region. Specifically, we compute the IE as:
\vspace{-0.3em}
\begin{align}
     &\mathrm{IE}^{\ell,h}_{R^\prime}\left(m;\gD_{\text{anlz}}\right) = \nonumber\\ &\resizebox{\linewidth}{!}{$\underset{\left(x^{+},x^{-}\right)\sim\gD_{\text{anlz}}}{\E}\left[m\left(x^{-}|\mathrm{do}\left(\vv_{\gI_{R^\prime}}^{\ell,h}=\vv_{\gI_{R^\prime}}^{\ell,h,+}\right)\right) - m(x^{+})\right],$}
\end{align}
where \(R^\prime \in \{\text{inst}, \text{temp}^\prime, \text{all}\}\) indicates a specific region, with $\gI_{\text{inst}} = \{1,\dots,K\}$, $\gI_{\text{temp}^\prime} = \{K+1,\dots,T-1\}$ and $\gI_{\text{all}} = \{1,\dots,T-1\}$. Notably, we exclude the last position $T$ from patching to avoid direct impact on the compliance probe.


Given that different heads have varying influences on safety capability, we first patch two regions together to quantify the importance of each head by \(\mathrm{IE}^{\ell,h}_{\text{all}}\left(m;\gD_{\text{anlz}}\right)\). Then we cumulatively patch the value states of heads for each region, starting from the most important head to the least, to obtain \(\mathrm{IE}^\gH_{R^\prime}\left(m;\gD_{\text{anlz}}\right)\). Here, \(\gH=\{(\ell_1,h_1),\dots\}\) represents the head indexes sorted by their importance scores. A higher \( \mathrm{IE}^\gH_{R^\prime} \) indicates the information from region \( R^\prime \) has a greater causal effect on the model's compliance decision, and vice versa. For a fair cross-model comparison, we use the \textit{normalized indirect effect} (NIE) by dividing the IE of each pair by \( (m(x^-)-m(x^+)) \). 

\paragraph{Results.}

\Cref{fig:region_patching} shows the trend of NIE in different regions as the number of patched heads increases. We have these key observations: (1) When patching the template region, a substantial increase in NIE is achieved by patching only a small number of heads that are critical to safety capabilities. In contrast, patching the instruction region does not bring significant improvement. This indicates that \subconc{the core computation of safety functions primarily occurs in heads processing information from the template region}. (2) For most models, even as the number of patched heads increases steadily, the NIE of the instruction region remains a remarkable gap compared to that of the template region. This indicates that \subconc{safety-tuned LLMs tend to rely on information from the template region rather than the instruction region when making initial compliance decisions.} Even when reversed instruction information is forcibly injected, it has limited influence on the prediction results. 

Overall, these results confirm that the safety alignment of LLMs is indeed anchored: \mainconc{current safety alignment mechanisms primarily rely on information aggregated from the template region to make initial safety-related decisions}.


\section{How Does TASA Cause Inference-time Vulnerabilities of LLMs}
\label{sec:rq2}


While TASA has been broadly observed across various safety-tuned LLMs, its role in causing vulnerabilities, particularly in the context of jailbreak attacks, remains unclear. To investigate this, we address two key questions: First, to what extent does TASA influence the model's initial output and affect its overall safety? Second, how is TASA connected to jailbreak attacks during generation?

\begin{figure}[t!]
  \centering
  \includegraphics[width=\linewidth]{Figures/4_asr_eval.pdf}
  \caption{Performance of different attack methods. Surprisingly, simply intervening information from the template region (i.e., \textsc{TempPatch}) can significantly increase attack success rates.}
  \label{fig:asr_eval}
\end{figure}


\begin{figure*}[t]
  \centering
  \includegraphics[width=\textwidth]{Figures/6_Meta-Llama-3-8B-Instruct_probe.pdf}
  %\caption{Probing harmfulness features in the residual streams across layers and template positions (from the 5th to the 1st closest to the ending position) of Llama-3-8B-Instruct, measured by the proportion of representations classified as harmful. The background intensity reflects the criticality of each layer's states for safety-related decisions, as aligned with \Cref{fig:temp_patching}.}
  \caption{Probed harmful rates in the residual streams across layers and template positions (from the 5th to the 1st closest to the ending position) of Llama-3-8B-Instruct. The background intensity reflects the importance of each layer's states for safety-related decisions, as aligned with \Cref{fig:temp_patching}.}
  \label{fig:prob_in_temp}
\end{figure*}



\subsection{TASA's Impact on Response Generation}
\label{subsec:tasa_resp}

To investigate the impact of TASA on the model's safety capability, we intervene in the information from template positions during response generation for harmful requests, and evaluate whether the model can still produce refusal responses.



\paragraph{Method.} 
During the forward process of each token in the response, we replace the value states of a specific proportion of attention heads at template positions with the corresponding value states from processing the harmful input (See \Cref{appendix:temp_patch}).
We refer to this operation as \textsc{TempPatch} and evaluate its performance on the Harmbench test set. For comparison, we also evaluate three representative jailbreak attack methods: (1) \textbf{AIM} \cite{wei2023jailbroken}, a carefully crafted attack prompt; (2) \textbf{PAIR} \cite{chao2023jailbreaking}, which iteratively optimizes attack instructions using an attacker LLM; and (3) \textbf{AmpleGCG} \cite{liao2024amplegcg}, an efficient approach for generating adversarial suffixes \cite{zou2023universal} (See \Cref{appendix:jb_details}). To assess compliance, we employ a compliance detector \cite{xie2024sorry} to identify whether the model complies with the provided inputs. The effectiveness of each method is measured by the \textit{attack success rate} (\textbf{ASR}), defined as the proportion of inputs for which the model complies.



\paragraph{Results.} 
As shown in \Cref{fig:asr_eval}, \textsc{TempPatch} significantly increases the ASRs of LLMs, achieving results that are comparable to or even surpass those of other specialized jailbreak attack methods. These findings further validate the deep connection between TASA and the safety mechanisms of LLMs. Moreover, while other attack methods demonstrate limited effectiveness against certain models, particularly the Llama-3 8B and 3B variants, \textsc{TempPatch} achieves notably higher ASR in comparison. This contrast suggests that \subconc{what might seem like stronger safety alignment could actually depend more on shortcut-based safety mechanisms, which may potentially introduce unseen vulnerabilities when faced with scenarios outside the training distribution}.


\subsection{Probing Attack Effects on Template}

\label{subsec:prob_attack}

To understand how jailbreak attacks affect information processing in the template region, we probe how harmfulness features are represented in the intermediate states under different attack scenarios.


\paragraph{Method.} 
We feed both harmful and harmless inputs from \( \gD_{\text{anlz}} \) into Llama-3-8B-Instruct and collect residual streams at the template region across all layers. At each intermediate location, we construct a probe \( \vd^{-} \coloneqq -\vd^{+} \), using the method described in \Cref{eq:diff_prob}, but applied in the reverse direction. This probe is used to determine whether a state is harmful, defined as the predicted logit exceeding a decision threshold. The threshold is set at the midpoint between the average logits of harmful and harmless inputs. To quantify the harmfulness features at a specific intermediate location, we calculate the \emph{harmful rate}, defined as the proportion of intermediate states classified as harmful.




\paragraph{Results.}
\Cref{fig:prob_in_temp} illustrates the harmful rate of residual streams across different layers and template positions. Our analysis highlights two key findings:
(1) Successful attacks consistently reduce the harmful rate in residual streams across all template positions, indicating a uniform disruption in the processing of harmfulness features throughout the template region.
(2) Notable patterns emerge at the last positions close to the ending (e.g., from `\texttt{assistant}' to \texttt{`\texttt{\textbackslash n\textbackslash n}'}): For failed attacks, the harmful rate starts low but rises sharply in the middle layers, eventually plateauing at levels comparable to those of typical harmful inputs. In contrast, successful attacks exhibit only a modest increase across layers.
These observations suggest that intermediate template regions are critical for aggregating harmful information: \subconc{Successful attacks deeply suppress this aggregation process, whereas failed attacks are ultimately ``exposed''}.

Recalling the insights about TASA (\Cref{sec:rq1}), \mainconc{the loss of harmfulness information in the template region caused by attacks disrupts initial safety evaluations, leading to incorrect decisions and ultimately resulting in unsafe behaviors}.

\section{Detaching Safety Mechanism from The Template Region}
\label{sec:rq3}

Since an anchored safety mechanism likely causes vulnerabilities, it is worth exploring whether a detached safety mechanism during generation could, conversely, improve the model's overall safety robustness. This would involve detaching its safety functions from two aspects: (\romannumeral 1) the process of identifying harmful content and (\romannumeral 2) the way this processed information is utilized during generation.


\paragraph{Transferability of Probes.} 
Regarding the first aspect, we inspect whether the harmfulness processing functions in the template region can transfer effectively to response generation. 
To investigate this, we collect harmful responses from successful jailbreaking attempts and harmless responses using instructions in \(\gD_{\text{anlz}} \). We then evaluate whether the harmfulness probes derived from the template region in \Cref{subsec:prob_attack} can still distinguish if a response is harmful.
Specifically, we collect the residual streams from all layers at the first 50 positions of each response and measure the probes' accuracy in classifying harmfulness.

\arrayrulecolor{black}
\begin{table*}[!t]
    \centering
        \renewcommand{\arraystretch}{1.3}
        \caption{Non-MLLM detectors for AI-generated media, spanning from unimodal to multimodal content. \textbf{Au} means Authenticity detection, \textbf{Ex} means Explainability, \textbf{Lo} means Localization.}
        \resizebox{\linewidth}{!}{
        \begin{tabular}{c|c|ccc|c|l}
\hline 
&  & \multicolumn{3}{c|}{\textbf{Task}}                                      &                                                                             \\ 
\cline{3-5} 
\multirow{-2}{*}{\textbf{Method}} & \multirow{-2}{*}{\textbf{Venue}} & \textbf{Au} & \textbf{Ex} &\textbf{Lo} & \multirow{-2}{*}{\textbf{Category}}  & \makecell[c]{\multirow{-2}{*}{\textbf{Highlight}}}                                    \\ \hline 
\rowcolor{lightorange}
\multicolumn{7}{c}{\textbf{Text}}\\ 
DeTeCtive~\cite{guo2024detective}                                             & \lightgraytext{{[}ArXiv'24{]}}                                            
& \CheckmarkBold      %Au           
& -      %Ex                
& -       %Lo                    
& Stylistic-based  
& Learn distinct writing styles\\
Shah et al.~\cite{shah2023detecting}                                             & \lightgraytext{{[}IJACSA'23{]}}                                            
& \CheckmarkBold      %Au           
& -      %Ex                
& -       %Lo                  
& Stylistic-based
& Discuss various factors that need to be considered while detecting AI-generated text                              \\
Kumarage et al.~\cite{kumarage2023stylometric}                            & \lightgraytext{{[}Arxiv'23{]}}                                   
& \CheckmarkBold      %Au           
& -      %Ex                
& -       %Lo                  
& Stylistic-based            
& Use stylometric signals                                   \\
Hamed et al.~\cite{hamed2023improving}                            & \lightgraytext{{[}Preprint'23{]}}                                         
& \CheckmarkBold      %Au           
& -      %Ex                
& -       %Lo                  
& Linguistics-based   
& Extract the TF-IDF bigrams to train supervised Machine Learning algorithm           \\
Gallé et al.~\cite{galle2021unsupervised}                            & \lightgraytext{{[}Arxiv'21{]}}                                         
& \CheckmarkBold      %Au           
& -      %Ex                
& -       %Lo                  
& Linguistics-based             
& Leveraging repeated higher-order n-grams as detection signal           \\
Yoo et al.~\cite{yoo2023robust}                            & \lightgraytext{{[}Arxiv'23{]}}                                          
& \CheckmarkBold      %Au           
& -      %Ex                
& -       %Lo                  
& Watermarking               
& Use invariant features of natural language to embed robust watermarks to corruptions        \\
DeepTextMark~\cite{munyer2024deeptextmark}                            & \lightgraytext{{[}IEEE'24{]}}                                         
& \CheckmarkBold      %Au           
& -      %Ex                
& -       %Lo                  
& Watermarking         
& Use Word2Vec, Sentence Encoding, and transformer-based classifier for watermark insertion and detection          \\
Yang et al.~\cite{yang2023watermarking}                            & \lightgraytext{{[}Arxiv'23{]}}                                      
& \CheckmarkBold      %Au           
& -      %Ex                
& -       %Lo                  
& Watermarking                
& Inject watermarks by replacing synonyms with different hash values.      \\
AWT~\cite{abdelnabi2021adversarial}                            & \lightgraytext{{[}IEEE'21{]}}                                         
& \CheckmarkBold      %Au           
& -      %Ex                
& -       %Lo                  
& Watermarking                     
&  Learn word substitutions along with their locations to hide watermarks          \\
REMARK-LLM~\cite{zhang2024remark}                            & \lightgraytext{{[}USENIX'24{]}}                              
& \CheckmarkBold      %Au           
& -      %Ex                
& -       %Lo                  
& Watermarking       
&  Insert watermarks into LLM-generated texts without compromising the semantic integrity          \\
\iffalse
GPTZero~\cite{gptzero}                            & \lightgraytext{{[}-{]}}                                           & Text           & Ex          & -    
& \CheckmarkBold      %txt           
& -      %img                
& -       %vid             
& -       %aud                    
& (找不到论文)            \\
\fi
Mitrovic et al.~\cite{mitrovic2023chatgpt}                            & \lightgraytext{{[}Arxiv'23{]}}                                         
& -      %Au           
& \CheckmarkBold      %Ex                
& -       %Lo                  
& -                
&  Apply Shapley Additive Explanations to uncover the detection model's reasoning         \\
Ji et al.~\cite{ji2024detecting}                            & \lightgraytext{{[}Arxiv'24{]}}                                          
& -      %Au           
& \CheckmarkBold      %Ex                
& -       %Lo                  
& -     
&  Introduce novel ternary text classification scheme to enhance explainability          \\
Zhang et al.~\cite{zhang2024machine}                            & \lightgraytext{{[}Arxiv'24{]}}                                       
& -      %Au           
& -      %Ex                
& \CheckmarkBold       %Lo                  
& -                      
&  Provide additional context by including multiple sentences at once but predict each one individually        \\
MFD~\cite{tao2024unveiling}                            & \lightgraytext{{[}Arxiv'24{]}}                                       
& -      %Au           
& -      %Ex                
& \CheckmarkBold       %Lo                  
& -               
&  Integrate low-level structural, high-level semantic, and deep-level linguistic features          \\
\rowcolor{lightorange}
\multicolumn{7}{c}{\textbf{Image}}\\ 
FHAD~\cite{wang2024generated}                            & \lightgraytext{{[}Arxiv'24{]}}                                           
& \CheckmarkBold       %Au           
& -      %Ex                
& -      %Lo                  
& High-Level                  
&  Use correlation of body parts to detect absent abnormalities     \\
Farid~\cite{farid2022lighting}                            & \lightgraytext{{[}Arxiv'22{]}}                                 
& \CheckmarkBold       %Au           
& -      %Ex                
& -      %Lo                  
& High-Level              
& Explore if physics-based forensic analyses will prove fruitful in detecting synthetic media           \\
Sarkar et al.~\cite{sarkar2024shadows}                            & \lightgraytext{{[}CVPR'24{]}}                 
& \CheckmarkBold       %Au           
& -      %Ex                
& -      %Lo                  
& High-Level              
& Use geometric properties         \\
AIDE~\cite{yan2024sanity}                            & \lightgraytext{{[}Arxiv'24{]}}                                        
& \CheckmarkBold       %Au           
& -      %Ex                
& -      %Lo                  
& High-Level                   
& Use multiple experts to simultaneously extract visual artifacts and noise patterns           \\
\iffalse
PatchCraft~\cite{zhong2024patchcraft}                            & \lightgraytext{{[}Arxiv'24{]}}                           
& Image           & Au          & Low-Level       
& -       %txt           
& \CheckmarkBold       %img                
& -       %vid             
& -       %aud       
& 论文只有引用           \\
\fi
LGrad~\cite{tan2023learning}                            & \lightgraytext{{[}CVPR'23{]}}                                       
& \CheckmarkBold       %Au           
& -      %Ex                
& -      %Lo                  
& Low-Level                
& Use gradients as the representation of artifacts in GAN-generated images           \\
AUSOME~\cite{poredi2023ausome}                            & \lightgraytext{{[}SPIE'23{]}}                                      
& \CheckmarkBold       %Au           
& -      %Ex                
& -      %Lo                  
& Low-Level  
&   Use spectral analysis and machine learning        \\
Wolter et al.~\cite{wolter2022wavelet}                            & \lightgraytext{{[}ML'22{]}}                    
& \CheckmarkBold       %Au           
& -      %Ex                
& -      %Lo                  
& Low-Level 
&  Use wavelet-packet-based analysis and boundary wavelets       \\
Synthbuster~\cite{bammey2023synthbuster}                            & \lightgraytext{{[}IEEE'23{]}}                            
& \CheckmarkBold       %Au           
& -      %Ex                
& -      %Lo                  
& Low-Level 
&  Use spectral analysis to highlight the artifacts in the Fourier transform of a residual image        \\
Frank et al.~\cite{frank2020leveraging}                            & \lightgraytext{{[}ICML'20{]}}                             
& \CheckmarkBold       %Au           
& -      %Ex                
& -      %Lo                  
& Low-Level   
&  Employ frequency representations for detecting         \\
Corvi et al.~\cite{corvi2023intriguing}                            & \lightgraytext{{[}CVPR'23{]}}                   
& \CheckmarkBold       %Au           
& -      %Ex                
& -      %Lo                  
& Low-Level   
&    Consider second-order statistics both in the spatial domain and in the frequency domains         \\
SeDID~\cite{ma2023exposing}                            & \lightgraytext{{[}Arxiv'23{]}}                            
& \CheckmarkBold       %Au           
& -      %Ex                
& -      %Lo                  
& Low-Level      
&    Exploit diffusion models' deterministic reverse and deterministic to denoise computation errors         \\
E3~\cite{azizpour2024e3}                            & \lightgraytext{{[}CVPR'24{]}}                            
& \CheckmarkBold       %Au           
& -      %Ex                
& -      %Lo                  
& Low-Level    
&   Create a set of expert embedders to accurately capture traces from each new target generator       \\
DIRE~\cite{wang2023dire}                            & \lightgraytext{{[}ICCV'23{]}}                           
& \CheckmarkBold       %Au           
& -      %Ex                
& -      %Lo                  
& Reconstruction Error    
&   Measure error between the input image and its reconstruction counterpart by pre-trained diffusion model         \\
AEROBLADE~\cite{ricker2024aeroblade}                            & \lightgraytext{{[}CVPR'24{]}}            
& \CheckmarkBold       %Au           
& -      %Ex                
& -      %Lo                  
& Reconstruction Error         
&   Compute images' AE reconstruction error         \\
FIRE~\cite{chu2024fire}                            & \lightgraytext{{[}Arxiv'24{]}}                       
& \CheckmarkBold       %Au           
& -      %Ex                
& -      %Lo                  
& Reconstruction Error        
&   Investigate the influence of frequency decomposition on reconstruction error         \\
DRCT~\cite{chendrct}                            & \lightgraytext{{[}ICML'24{]}}                           
& \CheckmarkBold       %Au           
& -      %Ex                
& -      %Lo                  
& Reconstruction Error    
&   Generate hard samples and adopt contrastive training to guide the learning of diffusion artifacts         \\
SemGIR~\cite{yu2024semgir}                            & \lightgraytext{{[}MM'24{]}}                      
& \CheckmarkBold       %Au           
& -      %Ex                
& -      %Lo                  
& Reconstruction Error    
&   Compel detector to focus on the inherent characteristic of the model expressed within them         \\
EditGuard~\cite{zhang2024editguard}                            & \lightgraytext{{[}CVPR'24{]}}                           
& \CheckmarkBold       %Au           
& -      %Ex                
& -      %Lo                  
& Watermarking         
& Train united Image-Bit Steganography Network to embed dual invisible watermarks into original images      \\
DiffusionShield~\cite{cui2023diffusionshield}                            & \lightgraytext{{[}Arxiv'23{]}}             
& \CheckmarkBold       %Au           
& -      %Ex                
& -      %Lo                  
& Watermarking  
&   Protect images from infringement by encoding the ownership message into an imperceptible watermark        \\
ZoDiac~\cite{zhang2024robust}                            & \lightgraytext{{[}Arxiv'24{]}}           
& \CheckmarkBold       %Au           
& -      %Ex                
& -      %Lo                  
& Watermarking    
&  Inject watermarks into trainable latent space for protection  \\
LaWa~\cite{rezaei2024lawa}                            & \lightgraytext{{[}Arxiv'24{]}}                  
& \CheckmarkBold       %Au           
& -      %Ex                
& -      %Lo                  
& Watermarking  
&  Change latent feature of pre-trained LDMs to integrate watermarking into the generation process  \\
WMAdapter~\cite{ci2024wmadapter}                            & \lightgraytext{{[}Arxiv'24{]}}               
& \CheckmarkBold       %Au           
& -      %Ex                
& -      %Lo                  
& Watermarking    
&  Use pretrained watermark decoder and minimal training pipeline to design a lightweight structure \\
Cifake~\cite{bird2024cifake}                            & \lightgraytext{{[}IEEE'24{]}}                           
& -       %Au           
& \CheckmarkBold      %Ex                
& -      %Lo                  
& -
&   Benchmarks of mirroring ten classes of the already available CIFAR-10 dataset with latent diffusion     \\
ASAP~\cite{huang2024asap}                            & \lightgraytext{{[}Arxiv'24{]}}              
& -       %Au           
& \CheckmarkBold      %Ex                
& -      %Lo                  
& -  
&  Extract distinct patterns and allow users to interactively explore them using various views.          \\
DA-HFNet~\cite{liu2024hfnet}                            & \lightgraytext{{[}Arxiv'24{]}}                    
& -       %Au           
& -      %Ex                
& \CheckmarkBold      %Lo                  
& -
&   Use dual-attention mechanism for deeper feature fusion and multi-scale feature interaction       \\
DiffForensics~\cite{yu2024diffforensics}                            & \lightgraytext{{[}CVPR'24{]}}                
& -       %Au           
& -      %Ex                
& \CheckmarkBold      %Lo                  
& -      
&   Propose a two-stage learning framework for IFDL tasks combining macro-features and micro-features        \\
MoNFAP~\cite{miao2024mixture}                            & \lightgraytext{{[}Arxiv'24{]}}            
& -       %Au           
& -      %Ex                
& \CheckmarkBold      %Lo                  
& -
&    Integrate detection and localization processing into a single predictor for face manipulation localization        \\
HiFi-Net++~\cite{guo2024language}                            & \lightgraytext{{[}IJCV'24{]}}                    
& -       %Au           
& -      %Ex                
& \CheckmarkBold      %Lo                  
& -    
&   Use additional language-guided forgery localization enhancer       \\
SAFIRE~\cite{kwon2024safire}                            & \lightgraytext{{[}Arxiv'24{]}}                             
& -       %Au           
& -      %Ex                
& \CheckmarkBold      %Lo                  
& -    
&   Capitalize on SAM’s point prompting capability to distinguish each source when an image has been forged        \\
\rowcolor{lightorange}
\multicolumn{7}{c}{\textbf{Video}}\\ 
Bohacek et al.~\cite{bohacek2024human}                            & \lightgraytext{{[}Arxiv'24{]}}                     
& \CheckmarkBold       %Au           
& -      %Ex                
& -      %Lo                  
& Frame-Level  
&   Leverage multi-modal semantic embedding to make it robust to the types of laundering      \\
AIGVDet~\cite{bai2024ai}                            & \lightgraytext{{[}Arxiv'24{]}}                      
& \CheckmarkBold       %Au           
& -      %Ex                
& -      %Lo                  
& Frame-Level  
&   Capture the forensic traces with a two-branch spatio-temporal convolutional neural network      \\
DIVID~\cite{liu2024turns}                            & \lightgraytext{{[}Arxiv'24{]}}                         
& \CheckmarkBold       %Au           
& -      %Ex                
& -      %Lo                  
& Video-Level    
&   Use CNN and LSTM to capture different levels of abstraction features and temporal dependencies     \\
He et al.~\cite{he2024exposing}                            & \lightgraytext{{[}Arxiv'24{]}}                        
& \CheckmarkBold       %Au           
& -      %Ex                
& -      %Lo                  
& Video-Level      
&   Design channel attention-based feature fusion by combining local and global temporal clues adaptively  \\
Yan et al.~\cite{yan2024generalizing}                            & \lightgraytext{{[}Arxiv'24{]}}               
& \CheckmarkBold       %Au           
& -      %Ex                
& -      %Lo                  
& Video-Level   
&    Blend original image and its warped version frame-by-frame to implement Facial Feature Drift   \\
DuB3D~\cite{ji2024distinguish}                            & \lightgraytext{{[}Arxiv'24{]}}                    
& \CheckmarkBold       %Au           
& -      %Ex                
& -      %Lo                  
& Video-Level  
&    Use a dual-branch architecture that adaptively leverages and fuses raw spatio-temporal data and optical flows      \\
Demamba~\cite{chen2024demamba}                            & \lightgraytext{{[}Arxiv'24{]}}                             
& \CheckmarkBold       %Au           
& -      %Ex                
& -      %Lo                  
& Video-Level   
&    Leverage a structured state space model to capture spatial-temporal inconsistencies across different regions      \\
Vahdati et al.~\cite{vahdati2024beyond}                            & \lightgraytext{{[}CVPR'24{]}}            
& \CheckmarkBold       %Au           
& -      %Ex                
& -      %Lo                  
& Video-Level    
&    Use synthetic video traces to perform reliable synthetic video detection or generator source attribution     \\
DVMark~\cite{luo2023dvmark}                            & \lightgraytext{{[}IEEE'23{]}}               
& \CheckmarkBold       %Au           
& -      %Ex                
& -      %Lo                  
& Watermarking
&   Use multi-scale design to make watermarks distributed across multiple spatial-temporal scales  \\
REVMark~\cite{zhang2023novel}                            & \lightgraytext{{[}MM'23{]}}              
& \CheckmarkBold       %Au           
& -      %Ex                
& -      %Lo                  
& Watermarking 
&  Use encoder/decoder structure with pre-processing block to extract temporal-associated features on aligned frames  \\
\rowcolor{lightorange}
\multicolumn{7}{c}{\textbf{Audio}}\\ 
Salvi et al.~\cite{salvi2024listening}                            & \lightgraytext{{[}Arxiv'24{]}}          
& \CheckmarkBold       %Au           
& -      %Ex                
& -      %Lo                  
& Fingerprint       
&    Indicate that analyzing the background noise alone leads to better classification results across diverse scenarios   \\
DeAR~\cite{liu2023dear}                            & \lightgraytext{{[}AAAI'23{]}}                                 
& \CheckmarkBold       %Au           
& -      %Ex                
& -      %Lo                  
& Watermarking   
&    Resist AR distortion at different distances in the real world   \\
AudioSeal~\cite{roman2024proactive}                            & \lightgraytext{{[}ICML'24{]}}                   
& \CheckmarkBold       %Au           
& -      %Ex                
& -      %Lo                  
& Watermarking       
&    Jointly train generator and detector for localized speech watermarking \\
Wu et al.~\cite{wu2023adversarial}                            & \lightgraytext{{[}ICME'23{]}}                          
& \CheckmarkBold       %Au           
& -      %Ex                
& -      %Lo                  
& Watermarking      
&    Embed a watermark into a feature domain mapped by a deep neural network   \\
SLIM~\cite{zhu2024slim}                            & \lightgraytext{{[}Arxiv'24{]}}                         
& -       %Au           
& \CheckmarkBold      %Ex                
& -      %Lo                  
& -   
&    Use style-linguistics mismatch in fake speech to separate style and linguistics contents from real speech   \\
SFAT-Net-3~\cite{cuccovillo2024audio}                            & \lightgraytext{{[}CVPR'24{]}}        
& -       %Au           
& \CheckmarkBold      %Ex                
& -      %Lo                  
& -   
&    Encode magnitude and phase of input speech to predict the trajectory of first phonetic formants\\
Pascu et al.~\cite{pascu2024easy}                            & \lightgraytext{{[}Arxiv'24{]}}                   
& -       %Au           
& \CheckmarkBold      %Ex                
& -      %Lo                  
& -         
&    Demonstrate that attacks can be identified with surprising accuracy using small subset of simplistic features  \\
HarmoNet~\cite{liu2024harmonet}                            & \lightgraytext{{[}ISCA'24{]}}                                   
& -       %Au           
& -      %Ex                
& \CheckmarkBold      %Lo                  
& -      
&  Use latent representations extraction capability of SSL along with harmonic F0 characteristic of speech\\
CFPRF~\cite{wu2024coarse}                            & \lightgraytext{{[}MM'24{]}}            
& -       %Au           
& -      %Ex                
& \CheckmarkBold      %Lo                  
& -      
&  Mine temporal inconsistency cues\\ %by perceiving subtle differences between different frames and capturing contextual information of multiple transition boundaries\\
\rowcolor{lightorange}
\multicolumn{7}{c}{\textbf{Multimodal}}\\ 
HAMMER~\cite{Shao2023CVPR}                            & \lightgraytext{{[}CVPR'23{]}}                     
& \CheckmarkBold        %Au           
& -      %Ex                
& -     %Lo                  
& Text-Image       
&    Capture interaction of image-texts based on embeddings alignment and multi-modal embedding aggregation\\
Li et al.~\cite{li2024zero}                            & \lightgraytext{{[}Arxiv'24{]}}                           
& \CheckmarkBold        %Au           
& -      %Ex                
& -     %Lo                  
& Visual-Audio    
&    Employ pre-trained ASR and VSR models to edit distance between audio and video sequences\\
Yoon et al.~\cite{yoon2024triple}                            & \lightgraytext{{[}IF'24{]}}            
& \CheckmarkBold        %Au           
& -      %Ex                
& -     %Lo                  
& Visual-Audio        
&    Propose a baseline approach based on zero-shot identity and one-shot deepfake detection with limited data\\
DiMoDif~\cite{koutlis2024dimodif}                            & \lightgraytext{{[}Arxiv'24{]}}                 
& -        %Au           
& -      %Ex                
& \CheckmarkBold     %Lo                  
& -
&    Exploit inter-modality differences in machine perception of speech \\
MMMS-BA~\cite{katamneni2024contextual}                            & \lightgraytext{{[}IJCB'24{]}}     
& -        %Au           
& -      %Ex                
& \CheckmarkBold     %Lo                  
& -   
&   Leverage attention from neighboring sequences and multi-modal representations \\ \hline
\end{tabular}
        }
    \label{table:non-mllm-detector}
\end{table*}

% \\
% ImgTrojan~\cite{tao2024imgtrojan}                                                 & \lightgraytext{{[}arXiv'24{]}}                                           & White           & Generator      & Tuning-based              & \cellcolor{LightRed}I + T → T           & Inject poisoned image-text pairs into the training of VLMs as triggers.                             

\looseness=-1 As shown in \Cref{fig:probe_in_resp} (see others in \Cref{appendix:probes}), our analysis of Llama-3-8B-Instruct reveals that harmfulness probes from the middle layers achieve relatively high accuracy and remain consistent across response positions. This result suggests that harmfulness probes from specific layers in the template region can be effectively transferred to identify harmful content in generated responses. 





\paragraph{Detaching Safety Mechanism.} 

To address the harmfulness-to-generation aspect, we need to examine how harmfulness features evolve during the generation process. The right-most plot in \Cref{fig:prob_in_temp} highlights distinct patterns between successful and failed attacks when generating the first response token. In failed attacks, the harmfulness feature quickly peaks and sustains that level throughout the generation process, whereas in successful attacks, it decreases and remains at a low level.
This observation suggests that additional harmfulness features should be injected during generation to counteract their decline in effective attacks. 

Based on this finding, we propose a simple straightforward method to detach the safety mechanism: use the probe to monitor whether the model is generating harmful content during response generation and, if detected, inject harmfulness features to trigger refusal behavior.
Formally, for a harmful probe \(\vd^{\ell,-}_{\tau}\) obtained from position \(\tau\) and layer \(\ell\), the representation at position \(i\) during generation is steered as follows:
\begin{equation}
\label{eq:detach}
    \vx_i^\ell \leftarrow \begin{cases} 
    \vx_i^\ell + \alpha\vd^{\ell,-}_{\tau} & \text{if } \vx_i^\ell\vd^{\ell,-}_{\tau} > \lambda \\
    \vx_i^\ell & \text{otherwise}
\end{cases},
\end{equation}
where \(\alpha\) is a factor controlling the strength of injection and \(\lambda\) is a decision threshold (See \Cref{appendix:detaching} for further details).

We evaluate this approach against AIM, AmpleGCG, and PAIR attacks.
We compare ASRs for response generations with and without detaching the safety mechanism, as shown in \Cref{tab:steering}. The results demonstrate that detaching the safety mechanism from the template and applying it directly to response generation effectively reduces ASRs, strengthening the model's safety robustness.


\begin{table*}
\centering
\renewcommand{\dashlinedash}{0.5pt} 
\renewcommand{\dashlinegap}{2pt}    
\begin{tabular}{lllll}
\hline\hline
 & \shortstack[l]{\space \\ \space \\ \textbf{Methods} \\ \space } & 
   \shortstack[l]{\space \\ \space \\ \textbf{Venue} \\ \space } & 
   \shortstack[l]{\space \\ \space \\ \textbf{Scheme} \\ \space } & 
   \shortstack[l]{\space \\ \space \\ \textbf{Cross-stage fusion} \\ \space} \\
\bottomrule   \\
Single-stage
& Zhang et al.~\cite{zhang2018single} & CVPR 2018 & $I \rightarrow [T, R]$ & - \\ 
& ERRNet~\cite{wei2019single} & CVPR 2019 & $I \rightarrow T$ & - \\ 
& RobustSIRR~\cite{song2023robust} & CVPR 2023 & $I_{multiscale} \rightarrow T$ & - \\
& YTMT~\cite{hu2021trash} & NeurIPS 2021 & $I \rightarrow [T, R]$ & - \\   \bottomrule   \\
Two-stage
& CoRRN~\cite{wan2019corrn} & TPAMI 2019 & \shortstack[l]{\space \\$I \rightarrow E_{T}$ \\ $[I, E_{T}] \rightarrow T$} & Convolutional Fusion \\ \cline{2-5}
& DMGN~\cite{feng2021deep} & TIP 2021 & \shortstack[l]{\space \\ $I \rightarrow [T_{1}, R]$ \\ $[I, T_{1}, R] \rightarrow T$} & Convolutional Fusion \\ \cline{2-5}
& RAGNet~\cite{li2023two} & Appl. Intell. 2023 & \shortstack[l]{\space \\ $I \rightarrow R$ \\ $ [I, R] \rightarrow T$} & Convolutional Fusion \\ \cline{2-5}
& CEILNet~\cite{fan2017generic} & ICCV 2017 & \shortstack[l]{\space \\ $[I, E_{I}] \rightarrow E_{T}$ \\ $[I, E_{T}] \rightarrow T$ } & Concat \\ \cline{2-5}
& DSRNet~\cite{hu2023single} & ICCV 2023 & \shortstack[l]{\space \\ $I \rightarrow (T_{1}, R_{1})$ \\ $(R_{1}, T_{1}) \rightarrow (R, T, residue)$} & N/A \\ \cline{2-5}
& SP-net BT-net~\cite{kim2020single} & CVPR 2020 & \shortstack[l]{\space \\ $I \rightarrow [T_{1}, R_{1}]$ \\ $R_{1} \rightarrow R$} & N/A \\ \cline{2-5}
& Wan et al.~\cite{wan2020reflection} & CVPR 2020  & \shortstack[l]{\space \\ $[I, E_{I}] \rightarrow R_{1}$ \\ $R_{1} \rightarrow R$} & N/A \\ \cline{2-5}
& Zheng et al.~\cite{zheng2021single} & CVPR 2021 & \shortstack[l]{\space \\ $I \rightarrow e$ \\ $[I, e] \rightarrow T$} & Concat \\ \cline{2-5}
& Zhu et al.~\cite{zhu2024revisiting} & CVPR 2024 & \shortstack[l]{\space \\ $I \rightarrow E_{R}$ \\ $ [I, E_{R}] \rightarrow T$} & Concat \\ \cline{2-5}
& Language-Guided~\cite{zhong2024language} & CVPR 2024 & \shortstack[l]{\space \\ $[I, Texts] \rightarrow R\ or\ T$ \\ $[I, R\ or\ T] \rightarrow T\ or\ R$} & Feature-Level Concat
\\   \bottomrule   \\
Multi-stage
& BDN~\cite{yang2018seeing} & ECCV 2018 & \shortstack[l]{\space \\ $I \rightarrow T_{1}$ \\ $[I, T_{1}] \rightarrow R$ \\ $[I, R] \rightarrow T$ } & Concat \\ \cline{2-5}
& IBCLN~\cite{li2020single} & CVPR 2020 & \shortstack[l]{\space \\ $[I, R_{0}, T_{0}] \rightarrow [R_{1}, T_{1}]$ \\ $[I, R_{1}, T_{1}] \rightarrow [R_{2}, T_{2}]$ \\ \ldots} & \shortstack[l]{Concat \\ Recurrent} \\ \cline{2-5}
& Chang et al.~\cite{chang2021single} & WACV 2021 & \shortstack[l]{\space \\ $I \rightarrow E_{T}$ \\ $[I, E_{T}] \rightarrow T_{1} \rightarrow R_{1} \rightarrow T_{2}$ \\ $[I, E_{T}, T_{2}] \rightarrow R \rightarrow T$} & \shortstack[l]{Concat \\ Recurrent} \\ \cline{2-5}
& LANet~\cite{dong2021location} & ICCV 2021 & \shortstack[l]{\space \\ $[I, T_{0}] \rightarrow R_{1} \rightarrow T_{1}$ \\ $[I, T_{1}] \rightarrow R_{2} \rightarrow T_{2}$ \\ \ldots} & \shortstack[l]{Concat \\ Recurrent} \\ \cline{2-5}
& V-DESIRR~\cite{prasad2021v} & ICCV 2021 & \shortstack[l]{\space \\ $I_{1} \rightarrow T_{1}$ \\ $[I_{1}, T_{1}, I_{2}] \rightarrow T_{2}$ \\ \ldots \\ $[I_{n-1}, T_{n-1}, I_{n}] \rightarrow T$ } & \shortstack[l]{Convolutional Fusion \\ Recurrent} \\
\hline\hline
\end{tabular}
\caption{\textbf{I}, \textbf{R}, \textbf{T}, and \textbf{E} represent the \textbf{I}nput, \textbf{R}eflection, \textbf{T}ransmission, and \textbf{E}dge map, respectively. The subscripts of \textbf{T} and \textbf{R} represent intermediate process outputs. The Absorption Effect $e$ is introduced in~\cite{zheng2021single} to describe light attenuation as it passes through the glass. The output $residue$ term, proposed in~\cite{hu2023single}, is used to correct errors in the additive reconstruction of the reflection and transmission layers. Language descriptions in~\cite{zhong2024language} provide contextual information about the image layers, assisting in addressing the ill-posed nature of the reflection separation problem.}
\label{tab:1}
\end{table*}

\subsection{Related Works}
\label{sec:related_works}
% In this section we will focus on differentiating from other related works (Table~\ref{tab:related-works}) and group work related to asynchronous planning. The field of LLM agent benchmarks for task planning performance consists of a large literature. \robotouille{} aims to extend the existing work with a LLM assessment environment that uses multi-agent asynchronous planning on long-horizon tasks.

In this section we will focus on our desiderata for LLM assistants and how \robotouille{} is different from other related works (Table~\ref{tab:related-works}).

% \begin{table*}[!h]
\centering
\setlength\tabcolsep{2.5pt}
\scalebox{0.80}{
\begin{tabular}{ccccccccccc}
    \toprule 
    Benchmark & \makecell{High-Level\\Actions} & Multi-agent & \makecell{Procedural\\Level Generation} & \makecell{Time\\Delays} & \makecell{Number of Tasks} & \makecell{Longest Plan \\Horizon} \\
    \midrule
    ALFWorld \citep{shridhar2021alfworldaligningtextembodied} & \cmark & \xmark & \xmark & \xmark & 3827 & 50 \\
    CuisineWorld \citep{gong2023mindagent} & \cmark & \cmark & \cmark & \xmark & 33 & 11 \\
    MiniWoB++ \citep{liu2018reinforcementlearningwebinterfaces} & \cmark & \xmark & \xmark & \xmark & 40 & 13 \\ 
    Overcooked-AI \citep{carroll2020utilitylearninghumanshumanai} & \xmark & \cmark & \xmark & \cmark & 1 & 100 \\
    PlanBench \citep{valmeekam2023planbenchextensiblebenchmarkevaluating} & \cmark & \xmark & \cmark & \xmark & 885 & 48\\
    $\tau$-bench \citep{yao2024taubenchbenchmarktoolagentuserinteraction} & \cmark & \xmark & \cmark & \xmark & 165 & 30 \\
    WebArena \citep{zhou2024webarenarealisticwebenvironment} & \cmark & \xmark & \cmark & \xmark & 812 & 30 \\
    WebShop \citep{yao2023webshopscalablerealworldweb} & \cmark & \xmark & \xmark & \xmark & 12087 & 90 \\
    AgentBench \citep{liu2023agentbenchevaluatingllmsagents} & \cmark & \cmark & \xmark & \xmark & 8 & 35 \\
    ARA \citep{kinniment2024evaluatinglanguagemodelagentsrealistic} & \cmark & \xmark & \xmark & \xmark & 12 & 4 \\
    AsyncHow \citep{lin2024graphenhancedlargelanguagemodels} & \cmark & \xmark & \xmark & \cmark & 1600 & 9 \\
    MAgIC \citep{xu2023magicinvestigationlargelanguage} & \cmark & \cmark & \xmark & \xmark & 5 & 20 \\
    T-Eval \citep{chen2024tevalevaluatingtoolutilization} & \cmark & \cmark & \xmark & \xmark & 23305 & 19 \\
    MLAgentBench \citep{huang2024mlagentbenchevaluatinglanguageagents} & \cmark & \xmark & \xmark & \xmark & 13 & 50 \\
    GAIA \citep{mialon2023gaiabenchmarkgeneralai} & \cmark & \xmark & \xmark & \xmark & 466 & 45 \\
    VirtualHome \citep{puig2018virtualhomesimulatinghouseholdactivities} & \cmark & \cmark & \cmark & \xmark & 2821 & 96 \\\midrule
    \robotouille (Ours) & \cmark & \cmark & \cmark & \cmark & 30 & 82 \\
    \bottomrule
\end{tabular}
}
\caption{Comparison between \robotouille and other benchmarks. See Appendix~\ref{sec:related_works} for more details.}
\label{tab:related-works}
\end{table*}

\textbf{Asynchronous Planning}
% While there exist many benchmarks that test the task planning abilities of LLMs agents (\cite{shridhar2021alfworldaligningtextembodied}; \cite{gong2023mindagent}; \cite{liu2018reinforcementlearningwebinterfaces}; \cite{valmeekam2023planbenchextensiblebenchmarkevaluating}; \cite{yao2024taubenchbenchmarktoolagentuserinteraction}; \cite{zhou2024webarenarealisticwebenvironment};\cite{yao2022webshop}), very few test the ability to plan asynchronously. The field of asynchronous planning benchmarks begins with benchmarks testing the temporal logic abilities of LLMs like TRAM (\cite{wang2024trambenchmarkingtemporalreasoning}). Of those benchmarks which test asynchronous planning, many use graph-based algorithms like GoT or GNNs (\cite{wu2024graphlearningimprovetask}; \cite{Besta_2024}). In fact, AsyncHow (\cite{lin2024graphenhancedlargelanguagemodels}), which extends both asynchronous and graph-based planning, proposes Plan Like a Graph (PLaG).
% When planning to bake potatoes or cut onions, it is important to realize that it is more efficient to cut onions while the potatoes are baking. Thus, a temporal component in tasks can be critical in task planning. While Overcooked AI (\cite{carroll2020utilitylearninghumanshumanai}) includes time delays for cooking in their environment, it has only 1 objective-based task so it is not able to test situations where multiple temporal tasks occur asynchronously. AsyncHow (\cite{lin2024graphenhancedlargelanguagemodels}) creates a benchmark for asynchronous planning with LLMs but uses GPT-4 to calculate estimates for each task. In contrast, \robotouille{} tests asynchronous task planning in a kitchen environment using temporal tasks like cutting and frying. In \robotouille{}, multiple tasks can occur sequentially or simultaneously which mimics real world interactions and planning.
Many benchmarks evaluate the task planning abilities of LLM agents \citep{shridhar2021alfworldaligningtextembodied, gong2023mindagent,liu2018reinforcementlearningwebinterfaces,valmeekam2023planbenchextensiblebenchmarkevaluating,yao2024taubenchbenchmarktoolagentuserinteraction,zhou2024webarenarealisticwebenvironment,yao2023webshopscalablerealworldweb} but few test the ability to plan asynchronously. Existing work relevant to asynchronous planning evaluate LLM capabilities on temporal logic \citep{wang2024trambenchmarkingtemporalreasoning} or use graph-based techniques \citep{wu2024graphlearningimprovetask}; \citep{Besta_2024}) but do not focus on it. \citep{lin2024graphenhancedlargelanguagemodels} proposes the Plan Like a Graph technique and a benchmark AsyncHow that focuses on asynchronous planning but makes a strong assumption that infinite agents exist. \citep{carroll2020utilitylearninghumanshumanai} proposes a benchmark, Overcooked-AI, that involves cooking onion soup which has time delays but has limited tasks and focuses on lower-level planning without LLM agents. \robotouille{} has a dataset focused on asynchronous planning that involves actions including cooking, frying, filling a pot with water, and boiling water.

% \textbf{Diverse Long-Horizon Task Planning}
% LLMs have a great store of semantic knowledge about the world but it is difficult for them to make decisions as agents because they have not experienced the real world. Works like SayCan (\cite{ahn2022icanisay}) have tried to solve this problem by grounding the LLM in real-world, long horizon tasks. Inner Monologue (\cite{huang2022innermonologueembodiedreasoning}) highlights the application of LLMs to embodied environments like robots where closed-loop language feedback allows better planning for robotic situations. Because of their wealth of knowledge about high-level tasks, reasoning capabilities, and ability to turn natural language into plans, LLMs can make proficient planners (\cite{singh2022progpromptgeneratingsituatedrobot}; \cite{liu2023llmpempoweringlargelanguage}; \cite{Lin_2023}; \cite{Wang_2024}; \cite{huang2024understandingplanningllmagents}). In many papers, LLMs shine as zero-shot generalizers, planners, and reasoners (\cite{huang2022languagemodelszeroshotplanners}; \cite{zeng2022socraticmodelscomposingzeroshot}). There have also been improvements in few-shot prompting with LLMs for the purpose of planning (\cite{yang2023couplinglargelanguagemodels}; \cite{liang2023codepolicieslanguagemodel}; \cite{song2023llmplannerfewshotgroundedplanning}). Now that LLMs can be used as agents in planners, how can we evaluate the effectiveness of the agents?

% Existing LLM agent benchmarks evaluate agents on only short-horizon tasks (\cite{shridhar2020alfworld}; \cite{puig2018virtualhome}; \cite{gong2023mindagent}). Firstly, many benchmarks contain very few goal-based tasks at all. The benchmarks AgentBench (\cite{liu2023agentbenchevaluatingllmsagents}), ARA (\cite{kinniment2024evaluatinglanguagemodelagentsrealistic}), and MLAgentBench (\cite{huang2024mlagentbenchevaluatinglanguageagents}) have 8, 12, 11 tasks respectively. Highlighting long-horizon tasks, the majority of the agent benchmarks in (Table~\ref{tab:related-works}) have less than 50 steps for their longest horizon plans. However, \robotouille{} creates a benchmark with 30 diverse long-horizon tasks with 80 steps as its longest plan, focused on cooking and better evaluates agents’ ability to solve plans with long sequences of steps.

% In addition, \robotouille{} leverages procedural generation. While other agent benchmarks like MiniWoB++ (\cite{liu2018reinforcementlearningwebinterfaces}), PlanBench (\cite{valmeekam2023planbenchextensiblebenchmarkevaluating}), $\tau$-bench (\cite{yao2024taubenchbenchmarktoolagentuserinteraction}), WebArena (\cite{zhou2024webarenarealisticwebenvironment}), and MAgIC (\cite{xu2023magicinvestigationlargelanguage}) support procedural generation, they contain only short-horizon tasks. \robotouille{}’s combination of long-horizon tasks and procedural generation gives it the ability to provide thorough testing of LLM agents on the randomization of tasks and environments.

\textbf{Diverse Long-Horizon Task Planning}
There is vast amount of work that use LLMs to plan \citep{ahn2022icanisay,huang2022innermonologueembodiedreasoning,zeng2022socraticmodelscomposingzeroshot,liang2023codepolicieslanguagemodel,singh2022progpromptgeneratingsituatedrobot,song2023llmplannerfewshotgroundedplanning,yang2023couplinglargelanguagemodels,song2023llmplannerfewshotgroundedplanning} but they tend to evaluate on short-horizon tasks with limited diversity in tasks. We present the number of tasks, longest plan horizon, and procedural generation capability of various benchmarks in Table~\ref{tab:related-works} to capture these axes. Notable LLM agent benchmarks that capture these axes include PlanBench \citep{valmeekam2023planbenchextensiblebenchmarkevaluating}, WebShop \citep{yao2023webshopscalablerealworldweb}, and VirtualHome \citep{puig2018virtualhomesimulatinghouseholdactivities}. \robotouille{} provides a focused set of diverse long-horizon tasks that can be procedurally generated.

% \textbf{Multi-agent Planning}
% Recently, after the development of using a singular LLM as an autonomous agent, multi-agent planners have made advancements in complex decision making and task planning (\cite{guo2024largelanguagemodelbased}; \cite{xi2023risepotentiallargelanguage}). Because of a better division of labor, having multiple agents can break down complex tasks and efficiently finish tasks. Agents can specialize in their roles, allowing for more collaboration. AutoGen (\cite{wu2023autogenenablingnextgenllm}) and ProAgent (\cite{zhang2024proagentbuildingproactivecooperative}) are frameworks used to build LLM-based multi-agent applications in conversation-based and cooperative task settings respectively.

% Many benchmarks have used LLMs as agents for collaboration in a shared environment (\cite{gong2023mindagent}; \cite{carroll2020utilitylearninghumanshumanai}; \cite{yao2024taubenchbenchmarktoolagentuserinteraction}; \cite{liu2023agentbenchevaluatingllmsagents}; \cite{xu2023magicinvestigationlargelanguage}; \cite{ma2024agentboardanalyticalevaluationboard}). However, many of these benchmarks focus on two-agent interactions with uneven task distribution like Overcooked AI (\cite{carroll2020utilitylearninghumanshumanai}). In response, CUISINEWORLD (\cite{gong2023mindagent}) tries to address this problem by assuming equal responsibilities across agents for tasks with competing resources. \robotouille{} instead extends this approach with turn-based environments where multiple LLM agents can control multiple agents to reach an objective within an environment. T-Eval (\cite{chen2024tevalevaluatingtoolutilization}) uses a three pronged approach to multi-agent frameworks with a planner, executor, and reviewers in which each agent can finish its job without switching the roles as dictated by the current plan. But, T-Eval uses multiple agents to annotate solutions which is different from the multi-agent planning of \robotouille{}. While MAgIC (\cite{xu2023magicinvestigationlargelanguage}) also investigates LLM powered multi-agents, it focuses on classic decision-making games like the Prisoner’s Dilemma and Chameleon. \robotouille{}’s cooking-based temporal tasks allow for better evaluation of a complex multi-agent environment.

\textbf{Multi-agent Planning} LLM agent benchmarks like \citep{liu2023agentbenchevaluatingllmsagents,xu2023magicinvestigationlargelanguage,ma2024agentboardanalyticalevaluationboard, gong2023mindagent} evaluate multi-agent interactions but do not involve time delays. OvercookedAI \citep{carroll2020utilitylearninghumanshumanai}, while not an LLM agent benchmark, incorporates time delays which brings the complexity of asynchronous planning to multi-agent settings. \robotouille{} provides a multi-agent dataset for 2-4 agents, a choice between turn-based or realtime planning, and incorporates asynchronous tasks for added complexity.

\section{Conclusion}

This work investigates template-anchored safety alignment (TASA), a widespread yet understudied phenomenon in aligned LLMs. We reveal how it relates to vulnerabilities during inference and suggest preliminary approaches to address this problem. Our work emphasizes the need to develop more robust safety alignment techniques that reduce the risk of learning potential shortcuts.




\section*{Limitations}


\textit{Limited Generalization.} While we have conducted systematic analysis on multiple mainstream models to demonstrate the widespread existence of the TASA issue, we acknowledge that this does not mean that all safety-aligned LLMs necessarily have significant TASA vulnerabilities. Our primary contribution lies in empirically demonstrating the existence of such vulnerabilities in real-world systems, rather than asserting their universality. Some aligned LLMs may actively or passively mitigate this issue through the following mechanisms: 1) Training data accidentally included defense patterns for relevant adversarial samples \cite{lyu2024keeping, zhang2024backtracking, qi2024safety}; 2) Feature suppression methods used in the safety alignment process happened to affect the activation conditions of the TASA trigger mechanism \cite{zou2024improving, rosati2024representation}; 3) The model scale has not reached the critical threshold for vulnerability to emerge.

\noindent\textit{Limited Solution.} As a direct response to the TASA issue analysis, in \Cref{sec:rq3} we attempt to detach the safety mechanism from the template region using activation steering \cite{leong2023self, zou2023representation, arditi2024refusal}. Since we haven't updated the model itself, we acknowledge that this method doesn't eliminate the learned safety shortcuts. We view this approach as a proof-of-concept for detachable safety mechanisms rather than a comprehensive solution. Building on our findings, robust mitigation may require systematic integration of adversarial defense patterns during training \cite{lyu2024keeping, zhang2024backtracking, qi2024safety}, or proactive suppression of shortcut-prone features during alignment \cite{zou2024improving, rosati2024representation}, which we leave for future work.

\section*{Ethic Statements}

This work reveals a new vulnerability in aligned LLMs, namely that LLMs' alignment may learn shortcut-based safety mechanisms, causing them to rely on information from template regions to make safety decisions. Although exposing new vulnerabilities could potentially be exploited by malicious actors, given that direct interference with information processing at template region can only be performed on white-box models, we believe the benefits of new insights into current safety alignment deficiencies far outweigh the risks. We hope these new findings will promote the development of more robust safety alignment methods.


% Bibliography entries for the entire Anthology, followed by custom entries
%\bibliography{anthology,custom}
% Custom bibliography entries only
\bibliography{custom,anthology}
\newpage
\appendix



\newpage
\appendix
\section{Applicability of SparseTransX for dense graphs} 
\label{A:density}
Even for fully dense graphs, our KGE computations remain highly sparse. This is because our SpMM leverages the incidence matrix for triplets, rather than the graph's adjacency matrix. In the paper, the sparse matrix $A \in \{-1,0,1\}^{M \times (N+R)}$ represents the triplets, where $N$ is the number of entities, $R$ is the number of relations, and $M$ is the number of triplets. This representation remains extremely sparse, as each row contains exactly three non-zero values (or two in the case of the "ht" representation). Hence, the sparsity of this formulation is independent of the graph's structure, ensuring computational efficiency even for dense graphs.

\section{Computational Complexity}
\label{A:complexity}
 For a sparse matrix $A$ with $m \times k$ having $nnz(A)=$ number of non zeros and dense matrix $X$ with $k \times n$ dimension, the computational complexity of the SpMM is $O(nnz(A) \cdot n)$ since there are a total of $nnz(A)$ number of dot products each involving $n$ components. Since our sparse matrix contains exactly three non-zeros in each row, $nnz(A) = 3m$. Therefore, the complexity of SpMM is $O(3m \cdot n)$ or $O(m \cdot n)$, meaning the complexity increases when triplet counts or embedding dimension is increased. Memory access pattern will change when the number of entities is increased and it will affect the runtime, but the algorithmic complexity will not be affected by the number of entities/relations.

\section{Applicability to Non-translational Models}
\label{A:non_trans}
Our paper focused on translational models using sparse operations, but the concept extends broadly to various other knowledge graph embedding (KGE) methods. Neural network-based models, which are inherently matrix-multiplication-based, can be seamlessly integrated into this framework. Additionally, models such as DistMult, ComplEx, and RotatE can be implemented with simple modifications to the SpMM operations. Implementing these KGE models requires modifying the addition and multiplication operators in SpMM, effectively changing the semiring that governs the multiplication.   

In the paper, the sparse matrix $A \in \{-1,0,1\}^{M \times (N+R)}$ represents the triplets, and the dense matrix $E \in \mathbb{R}^{(N+R) \times d}$ represents the embedding matrix, where $N$ is the number of entities, $R$ is the number of relations, and $M$ is the number of triplets. TransE’s score function, defined as $h + r - t$, is computed by multiplying $A$ and $E$ using an SpMM followed by the L2 norm. This operation can be generalized using a semiring-based SpMM model: $Z_{ij} = \bigoplus_{k=1}^{n} (A_{ik} \otimes E_{kj})$

Here, $\oplus$ represents the semiring addition operator, and $\otimes$ represents the semiring multiplication operator. For TransE, these operators correspond to standard arithmetic addition and multiplication, respectively.

\subsection*{DistMult} 
DistMult’s score function has the expression $h \odot r \odot t$. To adapt SpMM for this model, two key adjustments are required: The sparse matrix $A$ stores $+1$ at the positions corresponding to $h_{\text{idx}}$, $t_{\text{idx}}$, and $r_{\text{idx}}$. Both the semiring addition and multiplication operators are set to arithmetic multiplication. These changes enable the use of SpMM for the DistMult score function.

\subsection*{ComplEx} 
ComplEx’s score function has $h \odot r \odot \bar{t}$, where embeddings are stored as complex numbers (e.g., using PyTorch). In this case, the semiring operations are similar to DistMult, but with complex number multiplication replacing real number multiplication.

\subsection*{RotatE} 
RotatE’s score function has $h \odot r - t$. For this model, the semiring requires both arithmetic multiplication and subtraction for $\oplus$. With minor modifications to our SpMM implementation, the semiring addition operator can be adapted to compute $h \odot r - t$.

\subsection*{Support from other libraries}
Many existing libraries, such as GraphBLAS (Kimmerer, Raye, et al., 2024), Ginkgo (Anzt, Hartwig, et al., 2022), and Gunrock (Wang, Yangzihao, et al., 2017), already support custom semirings in SpMM. We can leverage C++ templates to extend support for KGE models with minimal effort.


\begin{figure*}[t]
\centering     %%% not \center
\includegraphics[width=\textwidth]{figures/all-eval.pdf}
\caption{Loss curve for sparse and non-sparse approach. Sparse approach eventually reaches the same loss value with similar Hits@10 test accuracy.}
\label{fig:loss_curve}
\end{figure*}

\section{Model Performance Evaluation and Convergence}
\label{A:eval}
SpTransX follows a slightly different loss curve (see Figure \ref{fig:loss_curve}) and eventually converges with the same loss as other non-sparse implementations such as TorchKGE. We test SpTransX with the WN18 dataset having embedding size 512 (128 for TransR and TransH due to memory limitation) and run 200-1000 epochs. We compute average Hits@10 of 9 runs with different initial seeds and a learning rate scheduler. The results are shown below. We find that Hits@10 is generally comparable to or better than the Hits@10 achieved by TorchKGE.

\begin{table}[h]
\centering
\caption{Average of 9 Hits@10 Accuracy for WN18 dataset}
\begin{tabular}{|c|c|c|}
\hline
\textbf{Model} & \textbf{TorchKGE} & \textbf{SpTransX} \\ \hline
TransE         & 0.79 ± 0.001700   & 0.79 ± 0.002667   \\ \hline
TransR         & 0.29 ± 0.005735   & 0.33 ± 0.006154   \\ \hline
TransH         & 0.76 ± 0.012285   & 0.79 ± 0.001832   \\ \hline
TorusE         & 0.73 ± 0.003258   & 0.73 ± 0.002780   \\ \hline
\end{tabular}
\label{table:perf_eval}
\end{table}

% We also plot the loss curve for different models in Figure \ref{fig:loss_curve}. We observe that the sparse approach follows a similar loss curve and eventually converges to the same final loss.

\section{Distributed SpTransX and Its Applicability to Large KGs}
\label{A:dist}
SpTransX framework includes several features to support distributed KGE training across multi-CPU, multi-GPU, and multi-node setups. Additionally, it incorporates modules for model and dataset streaming to handle massive datasets efficiently. 

Distributed SpTransX relies on PyTorch Distributed Data Parallel (DDP) and Fully Sharded Data Parallel (FSDP) support to distribute sparse computations across multiple GPUs. 

\begin{table}[h]
\centering
\caption{Average Time of 15 Epochs (seconds). Training time of TransE model with Freebase dataset (250M triplets, 77M entities. 74K relations, batch size 393K)  on 32 NVIDIA A100 GPUs. FSDP enables model training with larger embedding when DDP fails.}
\begin{tabular}{|p{2cm}|p{2.5cm}|p{2.5cm}|}
\hline
\textbf{Embedding Size} & \textbf{DDP (Distributed Data Parallel)} & \textbf{FSDP (Fully Sharded Data Parallel)} \\ \hline
16                      & 65.07 ± 1.641                            & 63.35 ± 1.258                               \\ \hline
20                      & Out of Memory                            & 96.44 ± 1.490                               \\ \hline
\end{tabular}
\end{table}

We run an experiment with a large-scale KG to showcase the performance of distributed SpTransX. Freebase (250M triplets, 77M entities. 74K relations, batch size 393K) dataset is trained using the TransE model on 32 NVIDIA A100 GPUs of NERSC using various distributed settings. SpTransX’s Streaming dataset module allows fetching only the necessary batch from the dataset and enables memory-efficient training. FSDP enables model training with larger embedding when DDP fails.

\section{Scaling and Communication Bottlenecks for Large KG Training}
\label{A:scaling}
Communication can be a significant bottleneck in distributed KGE training when using SpMM. However, by leveraging Distributed Data-Parallel (DDP) in PyTorch, we successfully scale distributed SpTransX to 64 NVIDIA A100 GPUs with reasonable efficiency. The training time for the COVID-19 dataset with 60,820 entities, 62 relations, and 1,032,939 triplets is in Table \ref{table:scaling}. 
% \vspace{-.3cm}
\begin{table}[h]
\centering
\caption{Scaling TransE model on COVID-19 dataset}
\begin{tabular}{|c|c|}
\hline
\textbf{Number of GPUs} & \textbf{500 epoch time (seconds)} \\ \hline
4                       & 706.38                            \\ \hline
8                       & 586.03                            \\ \hline
16                      & 340.00                               \\ \hline
32                      & 246.02                            \\ \hline
64                      & 179.95                            \\ \hline
\end{tabular}
\label{table:scaling}
\end{table}
% \vspace{-.2cm}
It indicates that communication is not a bottleneck up to 64 GPUs. If communication becomes a performance bottleneck at larger scales, we plan to explore alternative communication-reducing algorithms, including 2D and 3D matrix distribution techniques, which are known to minimize communication overhead at extreme scales. Additionally, we will incorporate model parallelism alongside data parallelism for large-scale knowledge graphs.

\section{Backpropagation of SpMM}
\label{A:backprop}
 Our main computational kernel is the sparse-dense matrix multiplication (SpMM). The computation of backpropagation of an SpMM w.r.t. the dense matrix is also another SpMM. To see how, let's consider the sparse-dense matrix multiplication $AX = C$ which is part of the training process. As long as the computational graph reduces to a single scaler loss $\mathfrak{L}$, it can be shown that $\frac{\partial C}{\partial X} = A^T$. Here, $X$ is the learnable parameter (embeddings), and $A$ is the sparse matrix. Since $A^T$ is also a sparse matrix and $\frac{\partial \mathfrak{L}}{\partial C}$ is a dense matrix, the computation $\frac{\partial \mathfrak{L}}{\partial X} = \frac{\partial C}{\partial X} \times \frac{\partial \mathfrak{L}}{\partial C} = A^T \times \frac{\partial \mathfrak{L}}{\partial C} $ is an SpMM. This means that both forward and backward propagation of our approach benefit from the efficiency of a high-performance SpMM.

\subsection*{Proof that $\frac{\partial C}{\partial X} = A^T$}
 To see why $\frac{\partial C}{\partial X} = A^T$ is used in the gradient calculation, we can consider the following small matrix multiplication without loss of generality.
\begin{align*}
A &= \begin{bmatrix}
a_1 & a_2 \\
a_3 & a_4
\end{bmatrix} \\ 
 X &= \begin{bmatrix}
x_1 & x_2 \\
x_3 & x_4
\end{bmatrix} \\
 C &=  \begin{bmatrix}
c_1 & c_2 \\
c_3 & c_4
\end{bmatrix}
\end{align*}
Where $C=AX$, thus-
\begin{align*}
c_1&=f(x_1, x_3) \\
c_2&=f(x_2, x_4) \\
c_3&=f(x_1, x_3) \\
c_4&=f(x_2, x_4) \\
\end{align*}
Therefore-
\begin{align*}
\frac{\partial \mathfrak{L}}{\partial x_1} &= \frac{\partial \mathfrak{L}}{\partial c_1} \times \frac{\partial c_1}{\partial x_1} + \frac{\partial \mathfrak{L}}{\partial c_2} \times \frac{\partial c_2}{\partial x_1} + \frac{\partial \mathfrak{L}}{\partial c_3} \times \frac{\partial c_3}{\partial x_1} + \frac{\partial \mathfrak{L}}{\partial c_4} \times \frac{\partial c_4}{\partial x_1}\\
&= \frac{\partial \mathfrak{L}}{\partial c_1} \times \frac{\partial \mathfrak{c_1}}{\partial x_1} + 0 + \frac{\partial \mathfrak{L}}{\partial c_3} \times \frac{\partial \mathfrak{c_3}}{\partial x_1} + 0\\
&= a_1 \times \frac{\partial \mathfrak{L}}{\partial c_1} + a_3 \times \frac{\partial \mathfrak{L}}{\partial c_3}\\
\end{align*}

Similarly-
\begin{align*}
\frac{\partial \mathfrak{L}}{\partial x_2}
&= a_1 \times \frac{\partial \mathfrak{L}}{\partial c_2} + a_3 \times \frac{\partial \mathfrak{L}}{\partial c_4}\\
\frac{\partial \mathfrak{L}}{\partial x_3}
&= a_2 \times \frac{\partial \mathfrak{L}}{\partial c_1} + a_4 \times \frac{\partial \mathfrak{L}}{\partial c_3}\\
\frac{\partial \mathfrak{L}}{\partial x_4}
&= a_2 \times \frac{\partial \mathfrak{L}}{\partial c_2} + a_4 \times \frac{\partial \mathfrak{L}}{\partial c_4}\\
\end{align*}
This can be expressed as a matrix equation in the following manner-
\begin{align*}
\frac{\partial \mathfrak{L}}{\partial X} &= \frac{\partial C}{\partial X} \times \frac{\partial \mathfrak{L}}{\partial C}\\
\implies \begin{bmatrix}
\frac{\partial \mathfrak{L}}{\partial x_1} & \frac{\partial \mathfrak{L}}{\partial x_2} \\
\frac{\partial \mathfrak{L}}{\partial x_3} & \frac{\partial \mathfrak{L}}{\partial x_4}
\end{bmatrix} &= \frac{\partial C}{\partial X} \times \begin{bmatrix}
\frac{\partial \mathfrak{L}}{\partial c_1} & \frac{\partial \mathfrak{L}}{\partial c_2} \\
\frac{\partial \mathfrak{L}}{\partial c_3} & \frac{\partial \mathfrak{L}}{\partial c_4}
\end{bmatrix}
\end{align*}
By comparing the individual partial derivatives computed earlier, we can say-

\begin{align*}
\begin{bmatrix}
\frac{\partial \mathfrak{L}}{\partial x_1} & \frac{\partial \mathfrak{L}}{\partial x_2} \\
\frac{\partial \mathfrak{L}}{\partial x_3} & \frac{\partial \mathfrak{L}}{\partial x_4}
\end{bmatrix} &= \begin{bmatrix}
a_1 & a_3 \\
a_2 & a_4
\end{bmatrix} \times \begin{bmatrix}
\frac{\partial \mathfrak{L}}{\partial c_1} & \frac{\partial \mathfrak{L}}{\partial c_2} \\
\frac{\partial \mathfrak{L}}{\partial c_3} & \frac{\partial \mathfrak{L}}{\partial c_4}
\end{bmatrix}\\
\implies \begin{bmatrix}
\frac{\partial \mathfrak{L}}{\partial x_1} & \frac{\partial \mathfrak{L}}{\partial x_2} \\
\frac{\partial \mathfrak{L}}{\partial x_3} & \frac{\partial \mathfrak{L}}{\partial x_4}
\end{bmatrix} &= A^T \times \begin{bmatrix}
\frac{\partial \mathfrak{L}}{\partial c_1} & \frac{\partial \mathfrak{L}}{\partial c_2} \\
\frac{\partial \mathfrak{L}}{\partial c_3} & \frac{\partial \mathfrak{L}}{\partial c_4}
\end{bmatrix}\\
\implies \frac{\partial \mathfrak{L}}{\partial X} &= A^T \times \frac{\partial \mathfrak{L}}{\partial C}\\
\therefore \frac{\partial C}{\partial X} &= A^T \qed
\end{align*}


\end{document}
