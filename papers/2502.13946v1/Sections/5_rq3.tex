\section{Detaching Safety Mechanism from The Template Region}
\label{sec:rq3}

Since an anchored safety mechanism likely causes vulnerabilities, it is worth exploring whether a detached safety mechanism during generation could, conversely, improve the model's overall safety robustness. This would involve detaching its safety functions from two aspects: (\romannumeral 1) the process of identifying harmful content and (\romannumeral 2) the way this processed information is utilized during generation.


\paragraph{Transferability of Probes.} 
Regarding the first aspect, we inspect whether the harmfulness processing functions in the template region can transfer effectively to response generation. 
To investigate this, we collect harmful responses from successful jailbreaking attempts and harmless responses using instructions in \(\gD_{\text{anlz}} \). We then evaluate whether the harmfulness probes derived from the template region in \Cref{subsec:prob_attack} can still distinguish if a response is harmful.
Specifically, we collect the residual streams from all layers at the first 50 positions of each response and measure the probes' accuracy in classifying harmfulness.

\arrayrulecolor{black}
\begin{table*}[!t]
    \centering
        \renewcommand{\arraystretch}{1.3}
        \caption{Non-MLLM detectors for AI-generated media, spanning from unimodal to multimodal content. \textbf{Au} means Authenticity detection, \textbf{Ex} means Explainability, \textbf{Lo} means Localization.}
        \resizebox{\linewidth}{!}{
        \begin{tabular}{c|c|ccc|c|l}
\hline 
&  & \multicolumn{3}{c|}{\textbf{Task}}                                      &                                                                             \\ 
\cline{3-5} 
\multirow{-2}{*}{\textbf{Method}} & \multirow{-2}{*}{\textbf{Venue}} & \textbf{Au} & \textbf{Ex} &\textbf{Lo} & \multirow{-2}{*}{\textbf{Category}}  & \makecell[c]{\multirow{-2}{*}{\textbf{Highlight}}}                                    \\ \hline 
\rowcolor{lightorange}
\multicolumn{7}{c}{\textbf{Text}}\\ 
DeTeCtive~\cite{guo2024detective}                                             & \lightgraytext{{[}ArXiv'24{]}}                                            
& \CheckmarkBold      %Au           
& -      %Ex                
& -       %Lo                    
& Stylistic-based  
& Learn distinct writing styles\\
Shah et al.~\cite{shah2023detecting}                                             & \lightgraytext{{[}IJACSA'23{]}}                                            
& \CheckmarkBold      %Au           
& -      %Ex                
& -       %Lo                  
& Stylistic-based
& Discuss various factors that need to be considered while detecting AI-generated text                              \\
Kumarage et al.~\cite{kumarage2023stylometric}                            & \lightgraytext{{[}Arxiv'23{]}}                                   
& \CheckmarkBold      %Au           
& -      %Ex                
& -       %Lo                  
& Stylistic-based            
& Use stylometric signals                                   \\
Hamed et al.~\cite{hamed2023improving}                            & \lightgraytext{{[}Preprint'23{]}}                                         
& \CheckmarkBold      %Au           
& -      %Ex                
& -       %Lo                  
& Linguistics-based   
& Extract the TF-IDF bigrams to train supervised Machine Learning algorithm           \\
Gallé et al.~\cite{galle2021unsupervised}                            & \lightgraytext{{[}Arxiv'21{]}}                                         
& \CheckmarkBold      %Au           
& -      %Ex                
& -       %Lo                  
& Linguistics-based             
& Leveraging repeated higher-order n-grams as detection signal           \\
Yoo et al.~\cite{yoo2023robust}                            & \lightgraytext{{[}Arxiv'23{]}}                                          
& \CheckmarkBold      %Au           
& -      %Ex                
& -       %Lo                  
& Watermarking               
& Use invariant features of natural language to embed robust watermarks to corruptions        \\
DeepTextMark~\cite{munyer2024deeptextmark}                            & \lightgraytext{{[}IEEE'24{]}}                                         
& \CheckmarkBold      %Au           
& -      %Ex                
& -       %Lo                  
& Watermarking         
& Use Word2Vec, Sentence Encoding, and transformer-based classifier for watermark insertion and detection          \\
Yang et al.~\cite{yang2023watermarking}                            & \lightgraytext{{[}Arxiv'23{]}}                                      
& \CheckmarkBold      %Au           
& -      %Ex                
& -       %Lo                  
& Watermarking                
& Inject watermarks by replacing synonyms with different hash values.      \\
AWT~\cite{abdelnabi2021adversarial}                            & \lightgraytext{{[}IEEE'21{]}}                                         
& \CheckmarkBold      %Au           
& -      %Ex                
& -       %Lo                  
& Watermarking                     
&  Learn word substitutions along with their locations to hide watermarks          \\
REMARK-LLM~\cite{zhang2024remark}                            & \lightgraytext{{[}USENIX'24{]}}                              
& \CheckmarkBold      %Au           
& -      %Ex                
& -       %Lo                  
& Watermarking       
&  Insert watermarks into LLM-generated texts without compromising the semantic integrity          \\
\iffalse
GPTZero~\cite{gptzero}                            & \lightgraytext{{[}-{]}}                                           & Text           & Ex          & -    
& \CheckmarkBold      %txt           
& -      %img                
& -       %vid             
& -       %aud                    
& (找不到论文)            \\
\fi
Mitrovic et al.~\cite{mitrovic2023chatgpt}                            & \lightgraytext{{[}Arxiv'23{]}}                                         
& -      %Au           
& \CheckmarkBold      %Ex                
& -       %Lo                  
& -                
&  Apply Shapley Additive Explanations to uncover the detection model's reasoning         \\
Ji et al.~\cite{ji2024detecting}                            & \lightgraytext{{[}Arxiv'24{]}}                                          
& -      %Au           
& \CheckmarkBold      %Ex                
& -       %Lo                  
& -     
&  Introduce novel ternary text classification scheme to enhance explainability          \\
Zhang et al.~\cite{zhang2024machine}                            & \lightgraytext{{[}Arxiv'24{]}}                                       
& -      %Au           
& -      %Ex                
& \CheckmarkBold       %Lo                  
& -                      
&  Provide additional context by including multiple sentences at once but predict each one individually        \\
MFD~\cite{tao2024unveiling}                            & \lightgraytext{{[}Arxiv'24{]}}                                       
& -      %Au           
& -      %Ex                
& \CheckmarkBold       %Lo                  
& -               
&  Integrate low-level structural, high-level semantic, and deep-level linguistic features          \\
\rowcolor{lightorange}
\multicolumn{7}{c}{\textbf{Image}}\\ 
FHAD~\cite{wang2024generated}                            & \lightgraytext{{[}Arxiv'24{]}}                                           
& \CheckmarkBold       %Au           
& -      %Ex                
& -      %Lo                  
& High-Level                  
&  Use correlation of body parts to detect absent abnormalities     \\
Farid~\cite{farid2022lighting}                            & \lightgraytext{{[}Arxiv'22{]}}                                 
& \CheckmarkBold       %Au           
& -      %Ex                
& -      %Lo                  
& High-Level              
& Explore if physics-based forensic analyses will prove fruitful in detecting synthetic media           \\
Sarkar et al.~\cite{sarkar2024shadows}                            & \lightgraytext{{[}CVPR'24{]}}                 
& \CheckmarkBold       %Au           
& -      %Ex                
& -      %Lo                  
& High-Level              
& Use geometric properties         \\
AIDE~\cite{yan2024sanity}                            & \lightgraytext{{[}Arxiv'24{]}}                                        
& \CheckmarkBold       %Au           
& -      %Ex                
& -      %Lo                  
& High-Level                   
& Use multiple experts to simultaneously extract visual artifacts and noise patterns           \\
\iffalse
PatchCraft~\cite{zhong2024patchcraft}                            & \lightgraytext{{[}Arxiv'24{]}}                           
& Image           & Au          & Low-Level       
& -       %txt           
& \CheckmarkBold       %img                
& -       %vid             
& -       %aud       
& 论文只有引用           \\
\fi
LGrad~\cite{tan2023learning}                            & \lightgraytext{{[}CVPR'23{]}}                                       
& \CheckmarkBold       %Au           
& -      %Ex                
& -      %Lo                  
& Low-Level                
& Use gradients as the representation of artifacts in GAN-generated images           \\
AUSOME~\cite{poredi2023ausome}                            & \lightgraytext{{[}SPIE'23{]}}                                      
& \CheckmarkBold       %Au           
& -      %Ex                
& -      %Lo                  
& Low-Level  
&   Use spectral analysis and machine learning        \\
Wolter et al.~\cite{wolter2022wavelet}                            & \lightgraytext{{[}ML'22{]}}                    
& \CheckmarkBold       %Au           
& -      %Ex                
& -      %Lo                  
& Low-Level 
&  Use wavelet-packet-based analysis and boundary wavelets       \\
Synthbuster~\cite{bammey2023synthbuster}                            & \lightgraytext{{[}IEEE'23{]}}                            
& \CheckmarkBold       %Au           
& -      %Ex                
& -      %Lo                  
& Low-Level 
&  Use spectral analysis to highlight the artifacts in the Fourier transform of a residual image        \\
Frank et al.~\cite{frank2020leveraging}                            & \lightgraytext{{[}ICML'20{]}}                             
& \CheckmarkBold       %Au           
& -      %Ex                
& -      %Lo                  
& Low-Level   
&  Employ frequency representations for detecting         \\
Corvi et al.~\cite{corvi2023intriguing}                            & \lightgraytext{{[}CVPR'23{]}}                   
& \CheckmarkBold       %Au           
& -      %Ex                
& -      %Lo                  
& Low-Level   
&    Consider second-order statistics both in the spatial domain and in the frequency domains         \\
SeDID~\cite{ma2023exposing}                            & \lightgraytext{{[}Arxiv'23{]}}                            
& \CheckmarkBold       %Au           
& -      %Ex                
& -      %Lo                  
& Low-Level      
&    Exploit diffusion models' deterministic reverse and deterministic to denoise computation errors         \\
E3~\cite{azizpour2024e3}                            & \lightgraytext{{[}CVPR'24{]}}                            
& \CheckmarkBold       %Au           
& -      %Ex                
& -      %Lo                  
& Low-Level    
&   Create a set of expert embedders to accurately capture traces from each new target generator       \\
DIRE~\cite{wang2023dire}                            & \lightgraytext{{[}ICCV'23{]}}                           
& \CheckmarkBold       %Au           
& -      %Ex                
& -      %Lo                  
& Reconstruction Error    
&   Measure error between the input image and its reconstruction counterpart by pre-trained diffusion model         \\
AEROBLADE~\cite{ricker2024aeroblade}                            & \lightgraytext{{[}CVPR'24{]}}            
& \CheckmarkBold       %Au           
& -      %Ex                
& -      %Lo                  
& Reconstruction Error         
&   Compute images' AE reconstruction error         \\
FIRE~\cite{chu2024fire}                            & \lightgraytext{{[}Arxiv'24{]}}                       
& \CheckmarkBold       %Au           
& -      %Ex                
& -      %Lo                  
& Reconstruction Error        
&   Investigate the influence of frequency decomposition on reconstruction error         \\
DRCT~\cite{chendrct}                            & \lightgraytext{{[}ICML'24{]}}                           
& \CheckmarkBold       %Au           
& -      %Ex                
& -      %Lo                  
& Reconstruction Error    
&   Generate hard samples and adopt contrastive training to guide the learning of diffusion artifacts         \\
SemGIR~\cite{yu2024semgir}                            & \lightgraytext{{[}MM'24{]}}                      
& \CheckmarkBold       %Au           
& -      %Ex                
& -      %Lo                  
& Reconstruction Error    
&   Compel detector to focus on the inherent characteristic of the model expressed within them         \\
EditGuard~\cite{zhang2024editguard}                            & \lightgraytext{{[}CVPR'24{]}}                           
& \CheckmarkBold       %Au           
& -      %Ex                
& -      %Lo                  
& Watermarking         
& Train united Image-Bit Steganography Network to embed dual invisible watermarks into original images      \\
DiffusionShield~\cite{cui2023diffusionshield}                            & \lightgraytext{{[}Arxiv'23{]}}             
& \CheckmarkBold       %Au           
& -      %Ex                
& -      %Lo                  
& Watermarking  
&   Protect images from infringement by encoding the ownership message into an imperceptible watermark        \\
ZoDiac~\cite{zhang2024robust}                            & \lightgraytext{{[}Arxiv'24{]}}           
& \CheckmarkBold       %Au           
& -      %Ex                
& -      %Lo                  
& Watermarking    
&  Inject watermarks into trainable latent space for protection  \\
LaWa~\cite{rezaei2024lawa}                            & \lightgraytext{{[}Arxiv'24{]}}                  
& \CheckmarkBold       %Au           
& -      %Ex                
& -      %Lo                  
& Watermarking  
&  Change latent feature of pre-trained LDMs to integrate watermarking into the generation process  \\
WMAdapter~\cite{ci2024wmadapter}                            & \lightgraytext{{[}Arxiv'24{]}}               
& \CheckmarkBold       %Au           
& -      %Ex                
& -      %Lo                  
& Watermarking    
&  Use pretrained watermark decoder and minimal training pipeline to design a lightweight structure \\
Cifake~\cite{bird2024cifake}                            & \lightgraytext{{[}IEEE'24{]}}                           
& -       %Au           
& \CheckmarkBold      %Ex                
& -      %Lo                  
& -
&   Benchmarks of mirroring ten classes of the already available CIFAR-10 dataset with latent diffusion     \\
ASAP~\cite{huang2024asap}                            & \lightgraytext{{[}Arxiv'24{]}}              
& -       %Au           
& \CheckmarkBold      %Ex                
& -      %Lo                  
& -  
&  Extract distinct patterns and allow users to interactively explore them using various views.          \\
DA-HFNet~\cite{liu2024hfnet}                            & \lightgraytext{{[}Arxiv'24{]}}                    
& -       %Au           
& -      %Ex                
& \CheckmarkBold      %Lo                  
& -
&   Use dual-attention mechanism for deeper feature fusion and multi-scale feature interaction       \\
DiffForensics~\cite{yu2024diffforensics}                            & \lightgraytext{{[}CVPR'24{]}}                
& -       %Au           
& -      %Ex                
& \CheckmarkBold      %Lo                  
& -      
&   Propose a two-stage learning framework for IFDL tasks combining macro-features and micro-features        \\
MoNFAP~\cite{miao2024mixture}                            & \lightgraytext{{[}Arxiv'24{]}}            
& -       %Au           
& -      %Ex                
& \CheckmarkBold      %Lo                  
& -
&    Integrate detection and localization processing into a single predictor for face manipulation localization        \\
HiFi-Net++~\cite{guo2024language}                            & \lightgraytext{{[}IJCV'24{]}}                    
& -       %Au           
& -      %Ex                
& \CheckmarkBold      %Lo                  
& -    
&   Use additional language-guided forgery localization enhancer       \\
SAFIRE~\cite{kwon2024safire}                            & \lightgraytext{{[}Arxiv'24{]}}                             
& -       %Au           
& -      %Ex                
& \CheckmarkBold      %Lo                  
& -    
&   Capitalize on SAM’s point prompting capability to distinguish each source when an image has been forged        \\
\rowcolor{lightorange}
\multicolumn{7}{c}{\textbf{Video}}\\ 
Bohacek et al.~\cite{bohacek2024human}                            & \lightgraytext{{[}Arxiv'24{]}}                     
& \CheckmarkBold       %Au           
& -      %Ex                
& -      %Lo                  
& Frame-Level  
&   Leverage multi-modal semantic embedding to make it robust to the types of laundering      \\
AIGVDet~\cite{bai2024ai}                            & \lightgraytext{{[}Arxiv'24{]}}                      
& \CheckmarkBold       %Au           
& -      %Ex                
& -      %Lo                  
& Frame-Level  
&   Capture the forensic traces with a two-branch spatio-temporal convolutional neural network      \\
DIVID~\cite{liu2024turns}                            & \lightgraytext{{[}Arxiv'24{]}}                         
& \CheckmarkBold       %Au           
& -      %Ex                
& -      %Lo                  
& Video-Level    
&   Use CNN and LSTM to capture different levels of abstraction features and temporal dependencies     \\
He et al.~\cite{he2024exposing}                            & \lightgraytext{{[}Arxiv'24{]}}                        
& \CheckmarkBold       %Au           
& -      %Ex                
& -      %Lo                  
& Video-Level      
&   Design channel attention-based feature fusion by combining local and global temporal clues adaptively  \\
Yan et al.~\cite{yan2024generalizing}                            & \lightgraytext{{[}Arxiv'24{]}}               
& \CheckmarkBold       %Au           
& -      %Ex                
& -      %Lo                  
& Video-Level   
&    Blend original image and its warped version frame-by-frame to implement Facial Feature Drift   \\
DuB3D~\cite{ji2024distinguish}                            & \lightgraytext{{[}Arxiv'24{]}}                    
& \CheckmarkBold       %Au           
& -      %Ex                
& -      %Lo                  
& Video-Level  
&    Use a dual-branch architecture that adaptively leverages and fuses raw spatio-temporal data and optical flows      \\
Demamba~\cite{chen2024demamba}                            & \lightgraytext{{[}Arxiv'24{]}}                             
& \CheckmarkBold       %Au           
& -      %Ex                
& -      %Lo                  
& Video-Level   
&    Leverage a structured state space model to capture spatial-temporal inconsistencies across different regions      \\
Vahdati et al.~\cite{vahdati2024beyond}                            & \lightgraytext{{[}CVPR'24{]}}            
& \CheckmarkBold       %Au           
& -      %Ex                
& -      %Lo                  
& Video-Level    
&    Use synthetic video traces to perform reliable synthetic video detection or generator source attribution     \\
DVMark~\cite{luo2023dvmark}                            & \lightgraytext{{[}IEEE'23{]}}               
& \CheckmarkBold       %Au           
& -      %Ex                
& -      %Lo                  
& Watermarking
&   Use multi-scale design to make watermarks distributed across multiple spatial-temporal scales  \\
REVMark~\cite{zhang2023novel}                            & \lightgraytext{{[}MM'23{]}}              
& \CheckmarkBold       %Au           
& -      %Ex                
& -      %Lo                  
& Watermarking 
&  Use encoder/decoder structure with pre-processing block to extract temporal-associated features on aligned frames  \\
\rowcolor{lightorange}
\multicolumn{7}{c}{\textbf{Audio}}\\ 
Salvi et al.~\cite{salvi2024listening}                            & \lightgraytext{{[}Arxiv'24{]}}          
& \CheckmarkBold       %Au           
& -      %Ex                
& -      %Lo                  
& Fingerprint       
&    Indicate that analyzing the background noise alone leads to better classification results across diverse scenarios   \\
DeAR~\cite{liu2023dear}                            & \lightgraytext{{[}AAAI'23{]}}                                 
& \CheckmarkBold       %Au           
& -      %Ex                
& -      %Lo                  
& Watermarking   
&    Resist AR distortion at different distances in the real world   \\
AudioSeal~\cite{roman2024proactive}                            & \lightgraytext{{[}ICML'24{]}}                   
& \CheckmarkBold       %Au           
& -      %Ex                
& -      %Lo                  
& Watermarking       
&    Jointly train generator and detector for localized speech watermarking \\
Wu et al.~\cite{wu2023adversarial}                            & \lightgraytext{{[}ICME'23{]}}                          
& \CheckmarkBold       %Au           
& -      %Ex                
& -      %Lo                  
& Watermarking      
&    Embed a watermark into a feature domain mapped by a deep neural network   \\
SLIM~\cite{zhu2024slim}                            & \lightgraytext{{[}Arxiv'24{]}}                         
& -       %Au           
& \CheckmarkBold      %Ex                
& -      %Lo                  
& -   
&    Use style-linguistics mismatch in fake speech to separate style and linguistics contents from real speech   \\
SFAT-Net-3~\cite{cuccovillo2024audio}                            & \lightgraytext{{[}CVPR'24{]}}        
& -       %Au           
& \CheckmarkBold      %Ex                
& -      %Lo                  
& -   
&    Encode magnitude and phase of input speech to predict the trajectory of first phonetic formants\\
Pascu et al.~\cite{pascu2024easy}                            & \lightgraytext{{[}Arxiv'24{]}}                   
& -       %Au           
& \CheckmarkBold      %Ex                
& -      %Lo                  
& -         
&    Demonstrate that attacks can be identified with surprising accuracy using small subset of simplistic features  \\
HarmoNet~\cite{liu2024harmonet}                            & \lightgraytext{{[}ISCA'24{]}}                                   
& -       %Au           
& -      %Ex                
& \CheckmarkBold      %Lo                  
& -      
&  Use latent representations extraction capability of SSL along with harmonic F0 characteristic of speech\\
CFPRF~\cite{wu2024coarse}                            & \lightgraytext{{[}MM'24{]}}            
& -       %Au           
& -      %Ex                
& \CheckmarkBold      %Lo                  
& -      
&  Mine temporal inconsistency cues\\ %by perceiving subtle differences between different frames and capturing contextual information of multiple transition boundaries\\
\rowcolor{lightorange}
\multicolumn{7}{c}{\textbf{Multimodal}}\\ 
HAMMER~\cite{Shao2023CVPR}                            & \lightgraytext{{[}CVPR'23{]}}                     
& \CheckmarkBold        %Au           
& -      %Ex                
& -     %Lo                  
& Text-Image       
&    Capture interaction of image-texts based on embeddings alignment and multi-modal embedding aggregation\\
Li et al.~\cite{li2024zero}                            & \lightgraytext{{[}Arxiv'24{]}}                           
& \CheckmarkBold        %Au           
& -      %Ex                
& -     %Lo                  
& Visual-Audio    
&    Employ pre-trained ASR and VSR models to edit distance between audio and video sequences\\
Yoon et al.~\cite{yoon2024triple}                            & \lightgraytext{{[}IF'24{]}}            
& \CheckmarkBold        %Au           
& -      %Ex                
& -     %Lo                  
& Visual-Audio        
&    Propose a baseline approach based on zero-shot identity and one-shot deepfake detection with limited data\\
DiMoDif~\cite{koutlis2024dimodif}                            & \lightgraytext{{[}Arxiv'24{]}}                 
& -        %Au           
& -      %Ex                
& \CheckmarkBold     %Lo                  
& -
&    Exploit inter-modality differences in machine perception of speech \\
MMMS-BA~\cite{katamneni2024contextual}                            & \lightgraytext{{[}IJCB'24{]}}     
& -        %Au           
& -      %Ex                
& \CheckmarkBold     %Lo                  
& -   
&   Leverage attention from neighboring sequences and multi-modal representations \\ \hline
\end{tabular}
        }
    \label{table:non-mllm-detector}
\end{table*}

% \\
% ImgTrojan~\cite{tao2024imgtrojan}                                                 & \lightgraytext{{[}arXiv'24{]}}                                           & White           & Generator      & Tuning-based              & \cellcolor{LightRed}I + T → T           & Inject poisoned image-text pairs into the training of VLMs as triggers.                             

\looseness=-1 As shown in \Cref{fig:probe_in_resp} (see others in \Cref{appendix:probes}), our analysis of Llama-3-8B-Instruct reveals that harmfulness probes from the middle layers achieve relatively high accuracy and remain consistent across response positions. This result suggests that harmfulness probes from specific layers in the template region can be effectively transferred to identify harmful content in generated responses. 





\paragraph{Detaching Safety Mechanism.} 

To address the harmfulness-to-generation aspect, we need to examine how harmfulness features evolve during the generation process. The right-most plot in \Cref{fig:prob_in_temp} highlights distinct patterns between successful and failed attacks when generating the first response token. In failed attacks, the harmfulness feature quickly peaks and sustains that level throughout the generation process, whereas in successful attacks, it decreases and remains at a low level.
This observation suggests that additional harmfulness features should be injected during generation to counteract their decline in effective attacks. 

Based on this finding, we propose a simple straightforward method to detach the safety mechanism: use the probe to monitor whether the model is generating harmful content during response generation and, if detected, inject harmfulness features to trigger refusal behavior.
Formally, for a harmful probe \(\vd^{\ell,-}_{\tau}\) obtained from position \(\tau\) and layer \(\ell\), the representation at position \(i\) during generation is steered as follows:
\begin{equation}
\label{eq:detach}
    \vx_i^\ell \leftarrow \begin{cases} 
    \vx_i^\ell + \alpha\vd^{\ell,-}_{\tau} & \text{if } \vx_i^\ell\vd^{\ell,-}_{\tau} > \lambda \\
    \vx_i^\ell & \text{otherwise}
\end{cases},
\end{equation}
where \(\alpha\) is a factor controlling the strength of injection and \(\lambda\) is a decision threshold (See \Cref{appendix:detaching} for further details).

We evaluate this approach against AIM, AmpleGCG, and PAIR attacks.
We compare ASRs for response generations with and without detaching the safety mechanism, as shown in \Cref{tab:steering}. The results demonstrate that detaching the safety mechanism from the template and applying it directly to response generation effectively reduces ASRs, strengthening the model's safety robustness.


\begin{table*}
\centering
\renewcommand{\dashlinedash}{0.5pt} 
\renewcommand{\dashlinegap}{2pt}    
\begin{tabular}{lllll}
\hline\hline
 & \shortstack[l]{\space \\ \space \\ \textbf{Methods} \\ \space } & 
   \shortstack[l]{\space \\ \space \\ \textbf{Venue} \\ \space } & 
   \shortstack[l]{\space \\ \space \\ \textbf{Scheme} \\ \space } & 
   \shortstack[l]{\space \\ \space \\ \textbf{Cross-stage fusion} \\ \space} \\
\bottomrule   \\
Single-stage
& Zhang et al.~\cite{zhang2018single} & CVPR 2018 & $I \rightarrow [T, R]$ & - \\ 
& ERRNet~\cite{wei2019single} & CVPR 2019 & $I \rightarrow T$ & - \\ 
& RobustSIRR~\cite{song2023robust} & CVPR 2023 & $I_{multiscale} \rightarrow T$ & - \\
& YTMT~\cite{hu2021trash} & NeurIPS 2021 & $I \rightarrow [T, R]$ & - \\   \bottomrule   \\
Two-stage
& CoRRN~\cite{wan2019corrn} & TPAMI 2019 & \shortstack[l]{\space \\$I \rightarrow E_{T}$ \\ $[I, E_{T}] \rightarrow T$} & Convolutional Fusion \\ \cline{2-5}
& DMGN~\cite{feng2021deep} & TIP 2021 & \shortstack[l]{\space \\ $I \rightarrow [T_{1}, R]$ \\ $[I, T_{1}, R] \rightarrow T$} & Convolutional Fusion \\ \cline{2-5}
& RAGNet~\cite{li2023two} & Appl. Intell. 2023 & \shortstack[l]{\space \\ $I \rightarrow R$ \\ $ [I, R] \rightarrow T$} & Convolutional Fusion \\ \cline{2-5}
& CEILNet~\cite{fan2017generic} & ICCV 2017 & \shortstack[l]{\space \\ $[I, E_{I}] \rightarrow E_{T}$ \\ $[I, E_{T}] \rightarrow T$ } & Concat \\ \cline{2-5}
& DSRNet~\cite{hu2023single} & ICCV 2023 & \shortstack[l]{\space \\ $I \rightarrow (T_{1}, R_{1})$ \\ $(R_{1}, T_{1}) \rightarrow (R, T, residue)$} & N/A \\ \cline{2-5}
& SP-net BT-net~\cite{kim2020single} & CVPR 2020 & \shortstack[l]{\space \\ $I \rightarrow [T_{1}, R_{1}]$ \\ $R_{1} \rightarrow R$} & N/A \\ \cline{2-5}
& Wan et al.~\cite{wan2020reflection} & CVPR 2020  & \shortstack[l]{\space \\ $[I, E_{I}] \rightarrow R_{1}$ \\ $R_{1} \rightarrow R$} & N/A \\ \cline{2-5}
& Zheng et al.~\cite{zheng2021single} & CVPR 2021 & \shortstack[l]{\space \\ $I \rightarrow e$ \\ $[I, e] \rightarrow T$} & Concat \\ \cline{2-5}
& Zhu et al.~\cite{zhu2024revisiting} & CVPR 2024 & \shortstack[l]{\space \\ $I \rightarrow E_{R}$ \\ $ [I, E_{R}] \rightarrow T$} & Concat \\ \cline{2-5}
& Language-Guided~\cite{zhong2024language} & CVPR 2024 & \shortstack[l]{\space \\ $[I, Texts] \rightarrow R\ or\ T$ \\ $[I, R\ or\ T] \rightarrow T\ or\ R$} & Feature-Level Concat
\\   \bottomrule   \\
Multi-stage
& BDN~\cite{yang2018seeing} & ECCV 2018 & \shortstack[l]{\space \\ $I \rightarrow T_{1}$ \\ $[I, T_{1}] \rightarrow R$ \\ $[I, R] \rightarrow T$ } & Concat \\ \cline{2-5}
& IBCLN~\cite{li2020single} & CVPR 2020 & \shortstack[l]{\space \\ $[I, R_{0}, T_{0}] \rightarrow [R_{1}, T_{1}]$ \\ $[I, R_{1}, T_{1}] \rightarrow [R_{2}, T_{2}]$ \\ \ldots} & \shortstack[l]{Concat \\ Recurrent} \\ \cline{2-5}
& Chang et al.~\cite{chang2021single} & WACV 2021 & \shortstack[l]{\space \\ $I \rightarrow E_{T}$ \\ $[I, E_{T}] \rightarrow T_{1} \rightarrow R_{1} \rightarrow T_{2}$ \\ $[I, E_{T}, T_{2}] \rightarrow R \rightarrow T$} & \shortstack[l]{Concat \\ Recurrent} \\ \cline{2-5}
& LANet~\cite{dong2021location} & ICCV 2021 & \shortstack[l]{\space \\ $[I, T_{0}] \rightarrow R_{1} \rightarrow T_{1}$ \\ $[I, T_{1}] \rightarrow R_{2} \rightarrow T_{2}$ \\ \ldots} & \shortstack[l]{Concat \\ Recurrent} \\ \cline{2-5}
& V-DESIRR~\cite{prasad2021v} & ICCV 2021 & \shortstack[l]{\space \\ $I_{1} \rightarrow T_{1}$ \\ $[I_{1}, T_{1}, I_{2}] \rightarrow T_{2}$ \\ \ldots \\ $[I_{n-1}, T_{n-1}, I_{n}] \rightarrow T$ } & \shortstack[l]{Convolutional Fusion \\ Recurrent} \\
\hline\hline
\end{tabular}
\caption{\textbf{I}, \textbf{R}, \textbf{T}, and \textbf{E} represent the \textbf{I}nput, \textbf{R}eflection, \textbf{T}ransmission, and \textbf{E}dge map, respectively. The subscripts of \textbf{T} and \textbf{R} represent intermediate process outputs. The Absorption Effect $e$ is introduced in~\cite{zheng2021single} to describe light attenuation as it passes through the glass. The output $residue$ term, proposed in~\cite{hu2023single}, is used to correct errors in the additive reconstruction of the reflection and transmission layers. Language descriptions in~\cite{zhong2024language} provide contextual information about the image layers, assisting in addressing the ill-posed nature of the reflection separation problem.}
\label{tab:1}
\end{table*}
