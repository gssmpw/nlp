% \documentclass[journal,draftclsnofoot,onecolumn,12pt]{IEEEtran}
\documentclass[final]{IEEEtran}
\usepackage{amsthm,amssymb,graphicx,multirow,amsmath,color,amsfonts,physics}%,ulem}
\usepackage[update,prepend]{epstopdf}
\usepackage[noadjust]{cite}
% \usepackage[latin1]{inputenc}
\usepackage{tikz}
\usepackage{bbm} % for \nbb1 
\usepackage{pdfpages}
\usepackage{balance}
%\usepackage{flushend}
%\usepackage{tabulary}
\usepackage{multirow}
\usepackage{comment}
\usepackage{subfigure}
% \usepackage{subcaption}

%\usepackage[justification=centering]{caption}
% Colors
\def\chr#1{{\color{red} #1}}
\def\chb#1{{\color{blue} #1}}
\newtheorem{corollary}{Corollary}
\allowdisplaybreaks % Allows breaking of eqnarray over multiple pages (avoids unnecessary blanks in the document before eqnarray)

\begin{document}
\newcommand{\ours}{$\text{Q}$LASS}
\pagenumbering{gobble}
\graphicspath{{./Figures/}}
\title{
Performance Analysis of Infrastructure Sharing Techniques in Cellular Networks:\\ A Percolation Theory Approach}
\author{
 Hao Lin,  Mustafa A. Kishk and Mohamed-Slim Alouini
\thanks{Hao Lin is with the Electrical and Computer Engineering Program, Computer, Electrical, and Mathematical Sciences and Engineering Division (CEMSE), King Abdullah University of Science and Technology (KAUST),
Thuwal 23955-6900, Saudi Arabia (e-mail: hao.lin.std@gmail.com).\\
\indent Mustafa A. Kishk is with the Department of Electronic Engineering,
Maynooth University, Maynooth, W23 F2H6 Ireland (e-mail:
mustafa.kishk@mu.ie).\\
\indent Mohamed-Slim Alouini is with the CEMSE Division, King Abdullah
University of Science and Technology (KAUST), Thuwal 23955-6900,
Saudi Arabia (e-mail: slim.alouini@kaust.edu.sa).}
}

\maketitle
\vspace{-2cm}
%\thispagestyle{empty}
%\pagestyle{empty}
\begin{abstract}
In the context of 5G, infrastructure sharing has been identified as a potential solution to reduce the investment costs of cellular networks. In particular, it can help low-income regions build 5G networks more affordably and further bridge the digital divide. There are two main kinds of infrastructure sharing: passive sharing (\ie site sharing) and active sharing (\ie access sharing), which require mobile network operators (MNOs) to share their non-electronic elements or electronic elements, respectively. Because co-construction and sharing can achieve broader coverage with lower investment, through percolation theory, we investigate how different sharing strategies can deliver large-scale continuous services. First, we examine the percolation characteristics in signal-to-interference-plus-noise ratio (SINR) coverage graphs and the necessary conditions for percolation. Second, we propose an `average coverage radius' to approximate the SINR graph with a low base station (BS) density based on the Gilbert disk model. Finally, we estimate the critical conditions of BS densities of MNOs for different sharing strategies and compare the percolation probabilities under different infrastructure sharing strategies.
\end{abstract}
% \vspace{-0.3cm}
\begin{IEEEkeywords}
% \vspace{-0.3cm}
Infrastructure sharing, stochastic geometry, graph theory, percolation theory, Gilbert disk model.
\end{IEEEkeywords}

\section{Introduction} \label{sec:Intro}
\indent The advent of the Fifth Generation (5G) mobile communication technology enables digital transformation and accelerates future digital economic growth. Due to its significant commercial potential, accelerating 5G network deployment has become a key priority for global mobile network operators (MNOs). At the same time, reducing network construction and operation costs, particularly in regions without advanced 5G infrastructure, has emerged as a critical challenge \cite{GSMA5GWhitepaper,ISoverview}. In \cite{8951153}, the potential of infrastructure sharing, data sharing and spectrum sharing was investigated, showing that leveraging existing public infrastructure can reduce the anticipated cost by approximately 40\% to 60\%.\\% \textcolor{red}{(5G, cost and infra sharing)}\\
\indent Infrastructure sharing involves sharing existing infrastructure and jointly deploying new facilities among MNOs. It has become an important research topic in recent years and a commercial reality in the 5G context \cite{cano2020evolution}. Based on which network elements that MNOs share, there are two main types of infrastructure sharing: \textit{passive sharing} and \textit{active sharing}. Passive sharing, which is also referred to as site sharing, implies the sharing of non-electrical elements at a site, such as shelter, cabinet, mast, power supply, management system. Since no additional spectrum resources are required, passive sharing allows MNOs to quickly expand their range of services by installing transceivers on other MNOs' base stations (BSs). Differently, active sharing involves the sharing of electronic components such as antennas, radio access networks, backhaul networks, and parts of core network. Therefore, active sharing is also called access sharing, where providers and operators offer access to others' resources to serve their own customers better \cite{IS}. In specific applications, infrastructure sharing is often accompanied by spectrum sharing, where different MNOs share their spectrum resources to obtain a wider bandwidth and higher data rate \cite{7876864}. Unlike traditional MNOs, mobile virtual network operators (MVNOs) do not have their own equipment and infrastructure. Instead, they provide services to their customers through other MNOs' infrastructure. Therefore, they have different sharing behaviors from MNOs \cite{7105671}.\\
\indent Since 2019, China Telecom and China Unicom have built the world's first, largest, and fastest 5G Standalone shared network, realizing one physical network correlated with two logical networks and multiple customized private networks. The two parties have built more than 1 million 5G shared BSs. At the same time, China Mobile has built more than 1.27 million 5G BSs, of which about 850,000 were jointly built and shared with China Broadcast Network. On May 17, 2023, telecom operators in China announced that they will jointly launch what they claim to be the world’s first commercial 5G inter-network roaming service trial \cite{5Groaming}. Infrastructure sharing is not only the main trend of future 5G development but also an important method to bridge the digital divide \cite{chaoub20216g}. Since 2015, the high cost of basic internet access in emerging markets, especially the poorest countries, has limited the access of the poor to the digital economy. Infrastructure sharing can reduce the substantial sunk costs of installation and provide extensive opportunities in other fields. For example, in the 6G era, infrastructure sharing could also provide opportunities for data sharing to help meet the considerable computational demands of multi-modal learning \cite{du2024distributed}.\\
\indent For cellular networks, it is important to better quantify the improvement in quality of service (QoS) arising from infrastructure sharing \cite{10077468}, which requires attention to the proportion and continuity of service coverage. Among various potential mathematical tools for analyzing coverage performance in cellular networks, percolation theory has its unique value. Percolation theory focus on whether giant connected components exist in a network \cite{Percolation,haenggi2012stochastic}. In cellular networks, percolation refers to the existence of large-scale continuous service areas. For example, users in mobile vehicles often require uninterrupted network access. In addition to the Doppler effect caused by the physical speed and the shadow effect caused by obstructions, users do not want to lose connection due to insufficient BS density. For the Internet of Vehicles, continuous services involve road information updates, the backup of driving data, navigation, and security. The monitoring network or the Internet of Things on the road should also be fully covered. For administrative boundaries or natural boundaries that need to be protected, it is also crucial to provide large-scale, continuous, safe, and reliable services. In this paper, we focus on the no sharing, passive sharing, and active sharing strategies between traditional MNOs with their own spectrum resource, transceivers and BSs, without considering MVNOs. We aim to analyze the ability of cellular networks with infrastructure sharing to form large-scale continuous coverage areas by studying discrete percolation and continuous percolation.
\subsection{Related Work}
 Infrastructure sharing, as a potential solution to improve network performance in multi-operator regions, becomes vital to reduce the high costs of BS deployment in the 5G era. However, few studies discussed the impact of infrastructure sharing on the service continuity of user equipment (UEs) from the perspective of percolation theory. For cellular networks, we consider SINR as the main metric to judge the continuity of BSs' coverage areas. Therefore, we divide the related work into: i) infrastructure sharing and ii) percolation theory and SINR analysis. \\
\indent \textit{Infrastructure Sharing}: As an important concept in 5G cellular networks, infrastructure sharing has been analyzed in various works in literature, which can save the cost, share the risk, boost quality of service and avoid environmental emissions \cite{dlamini2021remote}. There exist various models of infrastructure sharing. Authors in \cite{7105671} identified the characteristics of existing and future multi-operator network architectures, including mobile virtual network operators (MVNOs), trusted third parties, unique infrastructure providers, and standalone cases. A novel BS switching-off scheme was proposed to achieve significant energy and cost savings. Different sharing strategies have different advantages and also challenges, therefore they can be compared or combined in different application scenarios. 
In \cite{7343930}, authors assessed the fundamental trade-offs between spectrum and radio access infrastructure sharing. 
Authors in \cite{7562085} accurately modeled the channel propagation and antenna characterization and showed that a full spectrum and infrastructure sharing configuration can help increase user rate and bring economical advantages. 
In \cite{7500354}, authors introduced a mathematical framework to analyze multi-operator cellular networks that share spectrum licenses and infrastructure elements. Authors in \cite{8315130} proposed a mathematical framework to model a multi-operator mmWave cellular network with co-located BSs. They derived the SINR distribution and the coverage probability. 
In \cite{8594671}, authors proposed a multi-operator cooperation framework for sharing base stations among $N$ number of co-located radio access networks to improve energy efficiency. 
The proposed algorithm had great capacity in saving energy as well. They modeled locations of BSs in each network using independent Hardcore Poisson point processes (HCPPs). In infrastructure sharing, different MNOs' operation or sharing methods are also various.
In \cite{8329530}, authors modeled a single buyer MNO and multiple seller MNO infrastructure sharing system. Considering a given QoS in terms of the SINR coverage probability, they analyzed the trade-off between the transmit power of a BS and the intensity of BSs of the buyer MNO. Especially, MVNOs do not have their own infrastructure, so that their main expenses only come from renting transceivers and infrastructure \cite{7105671}. In \cite{zheng2017economic}, authors studied the behavior of MVNOs and Internet service providers, and derived the conditions for cross-carrier MVNOs to make profits and reduce costs for their users.
In a multi-operator cellular network, how to choose the best sharing solution requires reference to factors such as profit growth and performance improvement. Authors in \cite{cano2017optimal} considered regions with different areas and user amount, and discussed the optimal sharing strategy and number of active base stations under different service unit prices. Focusing on remote and rural areas, the authors in \cite{dlamini2021remote} reduced energy consumption while maintaining a quality of service comparable to that in urban areas. To better model random networks without losing accuracy and tractability, stochastic geometry has been widely used to evaluate the network performance with different infrastructure sharing strategies \cite{hmamouche2021new}.
They employed the homogeneous Poisson point process (PPP) and Gauss-Poisson process (GPP) to analyze the coverage probability and average user data rate. They extended their work in \cite{7876864}, differentiated the spectrum sharing experiencing flat or frequency-selective power fading, and considered the impact of network density imbalance between sharing MNOs. In summary, infrastructure sharing has been analyzed from many indicators, but there is still a lack of research on the existence of large-scale connected service areas.

\indent \textit{Percolation Theory and SINR analysis}: Percolation theory has been widely used to analyze the existence of large-scale multi-hop links or connected coverage areas in wireless networks \cite{haenggi2012stochastic, elsawy2023tutorial}. Authors in \cite{zhang2018robustness} studied the robustness of two spatially embedded networks that are interdependent. Focusing on the Gilbert disk model (GDM) and random Gilbert disk model (RGDM), authors in \cite{anjum2019percolation} derived the bounds of critical density regime in these two cases that are used to analyze the percolation in large-scale wireless balloon networks. They also analyzed the critical density of unmanned aerial vehicle (UAV) networks to ensure large-scale network coverage in \cite{9049663}. For sensing and monitoring applications, the path exposure problem was characterized in \cite{8794718,liu2012optimal}, where the authors find the critical density of sensors or cameras to detect moving objects over arbitrary paths through a given region. In \cite{9214384}, a novel and low-cost countermeasure against malware epidemics in large-scale wireless networks (LSWN) was proposed, which was denoted as spatial firewalls. In \cite{9214384} and \cite{9240972}, the critical density of such spatial firewalls and the percentage of secured devices were characterized using the tools of percolation probability. Authors in \cite{yemini2019simultaneous} proved the possibility of the coexistence of random primary and random secondary cognitive networks, both of which can include an unbounded connected component. Authors in \cite{wu2023connectivity} investigated the BS networks assisted by reconfigurable intelligent surfaces (RISs), and derived the lower bound of the critical density of RISs. Based on dynamic bond percolation, authors in \cite{han2024dynamic} proposed an evolution model to characterize the reliable topology evolution affected by the nodes and links states. In cellular networks, the effect of SINR on network performance and connectivity can not be ignored. Authors in \cite{jahnel2022sinr} derived a vital relationship between the interference cancellation factor, SINR threshold, and the status of percolation. In \cite{tobias2020signal}, authors showed that an SINR graph has an infinite connected component when the device density is large enough and the interferences are reduced sufficiently. They also estimated the relationship between the critical interference cancellation factor and device density. Authors in \cite{elsawy2023tutorial} summarized several classical percolation models. They investigated the critical relationship between device density, interference reduction factor and SINR threshold in the SINR graphs. However, there is still a lack of research that can provide critical conditions or approximate critical conditions for percolation in SINR graphs with infrastructure sharing. Therefore, it is necessary to further utilize SINR analysis to evaluate the network connectivity through percolation theory.

\subsection{Contributions}

Different from the existing literature on infrastructure sharing, this paper uses the percolation theory to study the possibility of continuous effective services under different infrastructure sharing strategies, where there are two considered MNOs in the entire network. The main contributions of this paper are as follows:\\
\indent \textit{A new perspective for performance comparison between different infrastructure sharing strategies.} In this paper, we compare the probabilities of forming large-scale continuous service areas under `no sharing', `active sharing', and `passive sharing' strategies. We show that percolation probability has its unique advantage in capturing the coverage and handover performance together. We compare the influence of MNOs' BS densities on the percolation probability under different sharing strategies, and show that `active sharing' can obtain not only the highest coverage performance but also the highest percolation probability. \\
\indent \textit{Evaluation method of SINR coverage.} In cellular networks, the impact of interference can not be ignore, and the signal-to-noise ratio (SNR) method overestimates the coverage of BSs. At the same time, the random distribution of BSs makes the coverage area of each BS irregular. Therefore, we propose an `average coverage radius' to analyze the BSs' SINR coverage. We verify that such an approximate evaluation method is reasonable through the comparison with simulated coverage probability. Computing the average coverage radius as a function of the BS densities, their transmission powers, and the SINR threshold, leads to tractable and insightful results. Especially from the perspective of percolation, we investigate the necessary conditions for large-scale continuous SINR coverage. 

\indent \textit{Percolation model and critical condition analysis.} In this paper, we confirm an important concept: continuous percolation of coverage areas can be analyzed using discrete percolation in hexagons whose side length is much less than coverage radius. We prove that when the coverage probability transits from less than $1/2$ to greater than $1/2$, the percolation probability also experiences the phase transition from zero to non-zero. Based on the `average coverage radius', we use the Gilbert disk model or the superposition of multiple Gilbert disk models to analyze the SINR coverage of BSs. We also study the critical condition of BS densities for the phase transition of percolation probability under different infrastructure sharing strategies. 

\section{System Model} \label{sec:SysMod}


Infrastructure sharing is a key concept in 5G cellular networks, and a common practice is to encourage two MNOs to share their core techs, shelters, or power cables. To understand different sharing strategies better, we focus on two MNOs (MNO $a$ and MNO $b$) and assume that the locations of BSs in each of these two MNOs follow independent Poisson point processes (PPPs) $\Phi_a$ and $\Phi_b$, respectively. The density of $\Phi_a$ is  $\lambda_a$ and the density of $\Phi_b$ is $\lambda_b$. We consider typical user equipment (UE) that subscribes to MNO $a$'s services. The basic concept of infrastructure sharing is shown in Fig.\ref{fig:SharingCases} and the differences between them are introduced as follows: 
\begin{figure}[ht]
    \centering
    \includegraphics[width=1\linewidth]{Figures/SharingCases2.pdf}
    \caption{Comparison of `no sharing', `active sharing (access sharing)' and `passive sharing (site sharing)'. Active sharing allows UEs of each MNO to switch the spectrum and access another MNO's BSs. Passive sharing allows each MNO to deploy its transceivers on another MNO's BSs.}
    \label{fig:SharingCases}
\end{figure}

    \indent \textbf{No Sharing:} In the `no sharing' case, the UEs subscribing to services of MNO $a$ (red UEs) can only access the network through BSs that belong to MNO $a$ (the nearest red BSs). The SINR and coverage range of a BS of MNO $a$ is only affected by the signals from MNO $a$'s other BSs (other red BSs).\\
    \indent \textbf{Active Sharing:} In the `active sharing' case, MNOs share access to their networks. Therefore, UEs subscribing to MNO $a$ or MNO $b$'s service can access the Internet through all BSs. As shown in Fig.\ref{fig:SharingCases}, UEs equipped with all required antennas are marked in purple. When they access the Internet through BSs of MNO $a$ (red BSs), they are only interfered by other red BSs. When they access the Internet through BSs of MNO $b$ (blue BSs), they are only interfered by other blue BSs.\\
    \indent \textbf{Passive Sharing:} In the `passive sharing' case, MNOs share the locations of their BSs and each MNO can deploy its transceivers on other MNOs' BSs. As shown in Fig.\ref{fig:SharingCases}, BSs with transceivers from different MNOs are marked in purple, \ie purple BSs with red transceivers and blue transceivers. A typical UE subscribing to the MNO $a$'s service (red UE) can receive the signals from the red transceivers installed in the nearest purple BS. However, its service is also interfered by the red transceivers on other purple BSs. \\
\indent We consider a typical UE that subscribes to the service of MNO $a$. For the case of no sharing, it directly chooses the nearest MNO $a$'s BS to connect. For the case of active sharing, it can choose the nearest BS of MNO $a$ to connect. It can also switch the working frequency and access the nearest BS of MNO $b$. The selection depends on which MNO can provide a higher signal-to-interference-plus-noise ratio (SINR). In passive sharing, since all BSs are equipped with both MNO $a$ and MNO $b$'S transceivers, this UE can choose the nearest BS to access and communicate with the MNO $a$'s transceiver on this BS. Once the SINR at the typical UE is less than the threshold $\beta$, the typical UE can not obtain effective service from any BS. In this paper, we assume that the power from each BS received by a UE at a unit distance is $P_{a}=P_{b}=P_t$, and use $N_0$ to represent the power of noise. In general, when the signal source is located at $x_i$ and the UE is located at $z$, the SINR is expressed as:
\begin{equation}
    \beta_i(z)=\frac{P_t l(x_i-z)}{N_0+\gamma\sum\limits_{x_j\in \Phi\setminus x_i}P_t l(x_j-z)},
\label{betaiz}
\end{equation}
where $\Phi$ is the set of BSs that use the same spectrum as $x_i$ and $0\leq \gamma \leq 1$ is the interference cancellation factor. The distance attenuation $l(x)$ is written
as:
\begin{equation}
    l(x)=\left\{\begin{matrix}
 1, &\|x\|\leq 1\, {\rm m},\\
\|x\|^{-\alpha}, &\|x\|>1\,{\rm m},
\end{matrix}\right.
\end{equation}
where $\|\cdot\|$ means the Euclidean norm and $\alpha$ is the path loss exponent. \\%\textcolor{red}{(SINR metric)}\\
\indent For future communication networks, providing continuous and high-quality service for UEs is important, especially for communications in the future vision of Internet of Vehicles. This requires that the serving areas of BSs can be connected. Based on graph theory, the continuity of BSs' coverage areas can be analyzed using a random graph $G(V,E)=\{V,E\}$, where $V$ is the set of locations of BSs that provide MNO $a$'s service, and $E$ is the edge set that shows whether the coverage areas of these BSs are connected. The edge set $E$ can be expressed as:
\begin{equation}
    E=\{\overline{x_i x_j}: \exists z\in \mathbb{R}^2,\, \beta_i(z)\geq\beta\;{\rm and}\;\beta_j(z)\geq\beta\},
\label{edgeset}
\end{equation}
where $x_i,\,x_j\in V$, $\beta_i(z)$ and $\beta_j(z)$ represent the SINR from BSs located at $x_i$ and $x_j$ to $z$, respectively.\\%, regardless which MNO builds the BSs at $x_i$ and $x_j$. \\%\textcolor{red}{(graph definition)}\\
\indent In this paper, we focus on the coverage of downlink services. When there exist large-scale connected coverage areas of BSs, UEs moving inside such giant service areas can achieve continuous high-quality services. Based on percolation theory, we use $K\subseteq G(V,E)$ to denote a connected component inside the whole graph and let $K(0)$ denote the connected component where the origin $O$ is covered by BSs in $K$. It is worth noting that analyzing whether $K(0)$ has infinite set cardinality is equivalent to studying the probability of large-scale continuous coverage areas. Since the BS densities of two MNOs are the main factor, we define the percolation probability using $\lambda_a$ and $\lambda_b$:
\begin{equation}\label{perprodef}
    \theta(\lambda_a,\lambda_b)\overset{\triangle}{=}{\P}\{|K(0)|=\infty\},
\end{equation}
where $|K(0)|$ denotes the set cardinality of the connected component $K(0)$. In this paper, we focus on the phase transition of percolation probability from zero to non-zero, \ie the transition from $\theta(\lambda_a,\lambda_b)=0$ to $\theta(\lambda_a,\lambda_b)>0$, which is a critical indicator of the ability to form large-scale continuous service in cellular networks. For different sharing strategies, the percolation probability of the whole system $\theta(\lambda_a,\lambda_b)$ can be expressed in different forms.\\
\indent In graph theory, the Gilbert disk model (GDM) is an important tool to analyze the percolation probability in wireless networks. For a classic GDM $D(\lambda,r)$, where $\lambda$ is the BS density and $r$ is the coverage radius, the critical condition for phase transition of percolation probability is written as: \\
\begin{equation}\label{GDMcricon}
    \lambda>\frac{\lambda_c(1)}{4r^2},
\end{equation}
where $\lambda_c(1)$ is the critical density of a Gilbert disk model with the radius of $1/2$. The value of such a parameter is an open research problem where existing literature has only provided approximations and upper/lower bounds. Focusing on the continuous percolation, we estimate the value of $\lambda_c(1)$ later in this paper, particularly in Theorem \ref{theo:lambdac1}. It is worth noting that, the Gilbert disk model has been widely used in signal-to-noise ratio (SNR) graphs, where the coverage radius is not related to the BS density. In this paper, even though the edge set $E$ is defined based on SINR, we find an approximate method to theoretically analyze the critical condition of BS densities.  \\
\indent In the following section, we first investigate the properties of the SINR coverage graph and the necessary conditions for the main factors in cellular networks. We propose the `average coverage radius' to approximate the SINR coverage of each BS in low density case. Next, we give the approximate expression of critical conditions for different infrastructure sharing strategies in Sec.\ref{sec:sharing}. For ease of reading, we summarize most of the symbols in Table \ref{tab:TableOfNotations}.


\begin{table*}[htbp]\caption{Table of Notations}
\centering
\begin{center}
\resizebox{\textwidth}{!}{
\renewcommand{\arraystretch}{1}%1.4
    \begin{tabular}{ {c} | {l} }
    \hline
        \hline
    \textbf{Notation} & \textbf{Description} \\ \hline
    $\Phi_{a}$; $\lambda_a$ & The set of locations of MNO $a$'s BSs; the BS density of MNO $a$\\ \hline
    $\Phi_{b}$; $\lambda_b$ & The set of locations of MNO $b$'s BSs; the BS density of MNO $b$\\ \hline
    $G(V,E)=\{V,E\}$ & The random graph with vertice set $V$ and edge set $E$\\ \hline
    $\beta$; $\gamma$; $\alpha$ & The threshold of SINR metric; the interference cancellation factor $(0\leq \gamma \leq 1)$; the path loss exponent $(\alpha>2)$\\ \hline
    $P_t$; $N_0$ & The received power at a unit distance to a BS; the power of noise\\ \hline
    $K$; $K(0)$; $|K(0)|$ & A connected component; the connected component covering the origin; the set cardinality of $K(0)$\\ \hline
    $l(x)$; $\|x\|$ & The distance attenuation function of a distance vector $x$; the Euclidean norm of vector $x$\\ \hline
    $\Psi_i$; $\Omega_i$ & The coverage area of the BS at $x_i$; the Voronoi cell of the BS located at $x_i$\\ \hline
    $\textbf{b}(x,r)$; $D(\lambda,r)$ & The circular area centered at $x$ with radius $r$; the GDM with density $\lambda$ and coverage radius $r$\\ \hline
    $\theta(\lambda_a,\lambda_b)$ & The general expression of percolation probability of $G(V,E)$ in the whole paper\\ \hline
    $\theta(p_{cov})$ & The special expression of percolation probability which depends on the homogeneous coverage probability $p_{cov}$\\ \hline
    $\theta(\lambda,r)$ & The special expression of percolation probability of a single-layer GDM $D(\lambda,r)$\\ \hline
    $r_{a}$; $r_{b}$ & The average coverage radii of MNO $a$ and MNO $b$ in `no sharing' and `active sharing' strategies\\ \hline
    $r_{ab}$ & The average coverage radius of MNO $a$ in `passive sharing' strategy\\ \hline
    $\lambda_c(1)$ & The critical density for phase transition of percolation probability in a Gilbert disk model with half of unit coverage radius.\\ \hline
     \hline
    \end{tabular}
    }
\end{center}
\label{tab:TableOfNotations}
%\vspace{-8mm}
\end{table*}


\section{Percolation in SINR coverage graph} \label{sec:percolation}
Percolation in the SINR graph, especially regarding communications between BSs or devices, has been investigated in the related literature. However, in this paper, the SINR coverage graph is defined based on the SINR at users, which has not been investigated sufficiently before. Therefore, in this section, we introduce the properties of percolation in a `single layer SINR coverage graph', where all BSs interfere with each other:
\begin{itemize}
    \item We first define the serving areas of BSs, containing the `strongest coverage areas' (SCAs) and `coverage areas' (CAs) of BSs. We show that both of them can be used to analyze the existence of large-scale continuous serving areas, \ie percolation of BSs' coverage.
    \item Next, we introduce sub-critical cases where the SCAs of BSs can not be connected so that there is no percolation in the SINR graph. 
    \item Finally, we propose the `average coverage radius' to approximate the serving areas of BSs. We also introduce the relationship between coverage probability and phase transition of percolation probability from zero to non-zero.
\end{itemize}

\subsection{Serving areas of BSs}
\indent In this paper, we focus on the continuity of BSs' coverage areas, which depends on the value of SINR at UEs. For a typical BS at $x_i$, its coverage area (CA) can be defined as:
\begin{equation}
\begin{array}{r@{}l}
    \Psi_i
    &\overset{\triangle}{=}\bigg\{z\in\mathbb{R}^2:\beta_i(z)\geq\beta\bigg\},
\label{CA}
\end{array}
\end{equation}
where the expression of $\beta_i(z)$ is shown in (\ref{betaiz}). Especially, when the interference cancellation factor $\gamma=0$, the CA $\Psi_i$ is a circular area $\textbf{b}\big(x_i,(\frac{\beta}{P_t/N_0})^{-\frac{1}{\alpha}}\big)$ centered at $x_i$ with radius $(\frac{\beta}{P_t/N_0})^{-\frac{1}{\alpha}}$. We assume that the BSs' coverage areas always contain the circular areas around them with unit radius, therefore, ${P_t}/{N_0}>\beta$ is always satisfied. When $0<\gamma\leq 1$, the shape of CA $\Psi_i$ becomes irregular and not tractable.\\
\indent In cellular networks, it's common for users to choose the nearest BSs to them as the serving ones, which can provide the highest average signal power and best average QoS. Therefore, each BS's actual service range is contained by its Voronoi cell. Notice that the SINR of the closest BS is larger than others, we can define the Voronoi cell of the BS located at $x_i$ as:
\begin{equation}
\begin{array}{r@{}l}
    \Omega_i&\overset{\triangle}{=}\bigg\{z\in\mathbb{R}^2:\|x_i-z\|\leq \|x_j-z\|,\forall j\neq i\bigg\}\\
    &=\bigg\{z\in\mathbb{R}^2:\beta_i(z)\geq\beta_j(z),\forall j\neq i\bigg\}.
\end{array}
\end{equation}

Considering both the SINR and closest BS selection strategy, we define the `strongest coverage area' (SCA) of BS located at $x_i$ as the intersection between its CA $\Psi_i$ and its Voronoi cell $\Omega_i$, that is $\Psi_i\cap\Omega_i$, where
\begin{equation}
\begin{array}{r@{}l}
\displaystyle\Psi_i&\bigcap\Omega_i\\&\overset{\triangle}{=}\bigg\{z\in\mathbb{R}^2:\beta_i(z)\geq\beta,\;\beta_i(z)\geq\beta_j(z),\forall j\neq i \bigg\}.
\end{array}
\end{equation}
For the users inside $\Psi_i \cap \Omega_i$, they can obtain the best Internet service through the BS located at $x_i$.\\
\indent In this paper, we aim to discuss the percolation of BSs' coverage areas (CAs). To analyze percolation in SINR coverage graphs, we first introduce Theorem \ref{theo:SCAandCA} to show the relationship between the percolation of SCAs and CAs of BSs.
\begin{theorem}
    For a large-scale cellular network, the union of the strongest coverage areas (SCAs) of BSs is the same as the union of coverage areas (CAs) of BSs, that is:
\begin{equation}\label{CASCA}
\bigcup\limits_{i}\Psi_i\cap\Omega_i=\bigcup\limits_{i}\Psi_i.
\end{equation}
Therefore, the face percolation of CAs $\{\Psi_i\}$ is equivalent to the percolation of SCAs $\{\Psi_i\cap\Omega_i\}$.
\label{theo:SCAandCA}
\end{theorem}
\begin{IEEEproof}
    See Appendix~\ref{app:SCAandCA}.
\end{IEEEproof}

Therefore, for an SINR coverage graph, both of SCAs and CAs can be used to define the edges in $G(V,E)$: 
\begin{itemize}
    \item If $\Psi_i\cap\Psi_j$ is not empty, we can say that these two CA $\Psi_i$ and $\Psi_j$ are connected, \ie $\overline{x_i x_j}\in E$. 
    \item If any user on the Voronoi boundary between $\Omega_i$ and $\Omega_j$ can be covered by $\Psi_i$ and $\Psi_j$, we can say that these two strongest coverage areas $\Psi_i\cap\Omega_i$ and $\Psi_j\cap\Omega_j$ are connected, \ie $\overline{x_i x_j}\in E$. 
\end{itemize}
Especially, when there is only one BS in a large area, the required large-scale continuous coverage path can not be formed. If two adjacent BS's CAs are possible to be connected considering the effect of interference, a dense cellular network can be well-designed and almost all areas can be successfully covered. Therefore, when the BS density is at a certain value, random BS deployment can achieve large-scale continuous coverage with a non-zero probability, that is the probability of face percolation. The face percolation on the SINR coverage graph is equivalent to percolation of the random geometric graph (RGG) $G(V,E)$ where $V$ contains the location of BSs and $E$ represent the connections between BSs' CAs or SCAs. Next, using the definition of the edges through SCAs and CAs, we introduce the sub-critical cases where there is no percolation on the SINR graph.

\subsection{Sub-critical regions}
To discuss the percolation between serving areas of BSs, we start with the percolation of SCAs. A key problem is whether the points on the Voronoi boundary of neighbour BSs can be covered. If all points on Voronoi bounds can not be covered, there is no percolation on the SINR coverage graph. Therefore,  we introduce a necessary condition for the percolation of SCAs on the SINR coverage graph in Theorem \ref{theo:betagamma}.

\begin{theorem}
\label{theo:betagamma}
    For a cellular network with interference cancellation factor $\gamma$ and SINR threshold $\beta$, a necessary condition for percolation is:
\begin{equation}
    \beta\gamma<1,
\label{betagamma1}
\end{equation}
where $\beta>0$ and $0\leq \gamma\leq 1$.
\end{theorem}
\begin{IEEEproof}
    See Appendix~\ref{app:betagamma}.
\end{IEEEproof}
\begin{remark}
    It is worth noting that, when $\gamma=1$, to achieve a non-zero percolation probability, $\beta$ has to be strictly less than 1. When $0<\gamma<1$, the upper bound of $\beta$ can be higher than 1. When $\gamma=0$, this necessary condition is always satisfied, where the SINR graph becomes an SNR graph. This means both signal processing and interference controlling are significant in generating large-scale continuous services.
\end{remark}

The necessary condition (\ref{betagamma1}) can be also explained using CAs, that is, the $k$th coverage problem that we introduce in Theorem \ref{theo:devicedegree}.
\begin{theorem}
\label{theo:devicedegree}
    Consider a cellular network with interference cancellation factor $\gamma$ and SINR threshold $\beta$. Let $N$ denote `the number of potential serving BSs except for the closest one', it satisfies:
\begin{equation}
    N<\frac{1}{\beta\gamma}.
\end{equation}
\end{theorem}
\begin{IEEEproof}
    See Appendix~\ref{app:devicedegree}.
\end{IEEEproof}
\begin{remark}
    If $\beta\gamma\geq 1$, $0<\frac{1}{\beta\gamma}\leq 1$. Therefore, $N=0$. This represents that the typical user can not access the Internet service through the BSs which is not the closest one.
\end{remark}
In conclusion, we have:
\begin{itemize}
    \item When $\beta\gamma\geq 1$, $\forall \lambda>0$, any two BSs' coverage areas can not be connected. Thus, the percolation probability is 0.
    \item When $\gamma=0$, $\beta\gamma=0$ and the SINR graph reduces to an SNR graph, which can be modelled using a classic Gilbert disk model (GDM) $D(\lambda,(\frac{\beta}{P_t/N_0})^{-\frac{1}{\alpha}})$. For a classical GDM, there exists a phase transition of percolation probability from zero to non-zero with the increase in BS density $\lambda$. The critical condition for phase transition of percolation probability in a GDM is shown in (\ref{GDMcricon}).
    \item When $0<\beta\gamma<1$, whether any two BSs' SCA can be connected depends on the concrete deployment, which is not tractable. However, for the whole cellular network, the percolation probability depends on $\beta$, $\gamma$ and $\lambda$.
\end{itemize}

In this paper, we mainly discuss the case where $\beta>0$, $0<\gamma\leq1$ and $\beta\gamma<1$. Because the SINR coverage graph $G_{\rm SINR}$ is the subset of the SNR coverage graph $G_{\rm SNR}$, we introduce a necessary condition of BS density for non-zero percolation probability:
\begin{theorem}
    For a cellular network where BSs follow a PPP with density $\lambda$, a necessary condition of BS density for non-zero percolation probability is:
\begin{equation}
    \lambda >\frac{\lambda_c(1)}{4r_{\rm SNR}^2},
\label{necconforlambda}
\end{equation}
where $r_{\rm SNR}=(\frac{\beta}{P_t/N_0})^{-\frac{1}{\alpha}}$ is the SNR coverage radius.
\end{theorem}
\begin{IEEEproof}
    Let $G_{\rm SINR}$ denote the SINR coverage graph and $G_{\rm SNR}$ denote the corresponding SNR coverage graph, $G_{\rm SINR}\subseteq G_{\rm SNR}$ always holds. Therefore, no percolation on $G_{\rm SNR}$ leads to no percolation on $G_{\rm SINR}$. As shown in (\ref{GDMcricon}), we can obtain the sufficient and necessary condition for non-zero percolation probability of $G_{\rm SNR}$, which is also the necessary condition for non-zero percolation probability of $G_{\rm SINR}$.
\end{IEEEproof}
Therefore, when $\gamma\beta<1$, if the BS density $\lambda$ does not satisfy the necessary condition (\ref{necconforlambda}), percolation probability is zero. However, the value of $\lambda_c(1)$ is an open question. Using discrete percolation in a hexagonal lattice, we discuss what value $\lambda_c(1)$ converges to in Theorem \ref{theo:lambdac1}.
\begin{theorem}\label{theo:lambdac1}
    For a Gilbert disk model $D(\lambda,r)$ with vertex density $\lambda$ and coverage radius $r$, the product of $\lambda$ and $(2r)^2$ is defined as $\lambda(1)$. In a large-scale network, the critical value of $\lambda(1)$ for phase transition of percolation probability is 
\begin{equation}\label{lambdac1}
    \lambda_c(1)=\frac{4\ln 2}{\pi}.
\end{equation}
\end{theorem}
\begin{IEEEproof}
    See Appendix~\ref{app:lambdac1}.
\end{IEEEproof}

In this paper, we focus on the phase transition of percolation probability from zero to non-zero in different sharing strategies, thus the BS density $\lambda$ is the main factor we consider. Therefore, we focus on low-density cases and aim to find a proper method to approximate the coverage area of each BS. 
\subsection{Average coverage radius for low-density network}
To obtain the critical condition of BS density for phase transition of percolation probability, we first focus on a single-layer network with low BS density. Since the shapes of BSs' SINR coverage areas are irregular, we define a parameter named `average coverage radius' to help approximate the coverage areas of BSs, where we consider the global interference which is related to the distance to the closest BS. The expression of average coverage radius is introduced in Theorem \ref{theo:acr} as below:
\begin{theorem}
\label{theo:acr}
    Consider a cellular network where all BSs interfere with each other. The locations of BSs follow a PPP with BS density $\lambda$ which is low enough. In this case, the `average coverage radius' of any BS in this network is the unique solution of $\mathcal{F}(r_m,\lambda)=1$, where

\begin{equation}
    \mathcal{F}(r_m,\lambda)=\frac{\beta}{P_t/N_0}r_{m}^{\alpha}+\frac{2\pi\gamma\beta\lambda}{\alpha-2}r^2_{m}.
\label{betaratio}
\end{equation}
The `average coverage radius' can be rewritten as a decreasing function of $\lambda$, i.e.
\begin{equation}
    r=r_m(\lambda).
\label{rewrite}
\end{equation}

\end{theorem}
\begin{IEEEproof}
    See Appendix~\ref{app:acr}.
\end{IEEEproof}

\indent Especially, if the path loss exponent $\alpha=4$, the `average coverage radius' can be written as a closed form expression of BS density $\lambda$, which is shown in Corollary \ref{cor:alpha4}.
\begin{corollary}\label{cor:alpha4}
    When $\alpha=4$, the average coverage radius $r_{a}$ can be written as a closed form expression:
\begin{equation}
    r_{m}(\lambda)=\sqrt{\sqrt{\frac{\beta_0}{\beta}+\bigg(\frac{\pi\gamma\lambda\beta_0}{2}\bigg)^2}-\frac{\pi\gamma\lambda\beta_0}{2}},
\end{equation}
where $\beta_0=P_t/N_0$.
\end{corollary}
\begin{IEEEproof}
    Substitute $\alpha=4$ into Theorem \ref{theo:acr}.
\end{IEEEproof}

\indent When $\gamma$ is small enough (approaches 0), the interference can be reduced efficiently, thus the SINR graph can be approximated well using the Gilbert disk model. In (\ref{betaratio}), $r_m$ depends on the product of $\gamma\lambda$. Therefore, especially for a low BS density network, $r_m$ can approximate the coverage areas well. Also, because the locations of BSs are homogeneous, it is feasible to adopt this method to analyze the critical condition for phase transition of percolation probability. \\
\indent Adopting the average coverage radius, we can approximately model a cellular network with a low BS density as a Gilbert disk model with BS density $\lambda$ and coverage radius $r_m$, \ie $D(\lambda,r_m)$. Next, we introduce some properties of percolation probability in the classical Gilbert disk model, where the coverage probability is a significant indicator.
\subsection{Continuous Percolation in Gilbert disk model}
\indent In this paper, we aim to adopt the Gilbert disk model to discuss the phase transition of percolation probability in `no sharing', `active sharing', and `passive sharing' strategies, respectively. Therefore, we need to provide the properties of continuous percolation in the Gilbert disk model. Firstly, based on a classical Gilbert disk model, the critical condition of coverage probability for phase transition of percolation probability in Theorem \ref{theo:percov}.
\begin{theorem}\label{theo:percov}
    Consider a large-scale wireless communication system where the distribution of BSs is homogeneous and all users have the same probability of being covered. If and only if the coverage probability $p_{cov}$ is larger than $1/2$, the percolation probability $\theta$ is non-zero. Such a phase transition of percolation probability can be expressed as: 
\begin{equation}\left.
    \begin{array}{@{}r@{}l}
    \theta(p_{cov})=0, &\;{\rm if}\;p_{cov}\leq1/2,\\
    \theta(p_{cov})>0, &\;{\rm if}\;p_{cov}>1/2.
    \end{array}\right.
    \label{percov}
\end{equation}
\end{theorem}
\begin{IEEEproof}
    See Appendix~\ref{app:pcov}.
\end{IEEEproof}
\indent Using this critical condition for phase transition of the probability of continuous percolation, we show the critical condition of BS densities in the next section.

\section{Percolation on Different Sharing Strategy}
\label{sec:sharing}
In Sec.\ref{sec:percolation}, we have provided the properties of a single-layer SINR coverage graph. In this section, adopting the `average coverage radius' and properties of continuous percolation under GDMs, we discuss the phase transition of percolation probability in different sharing strategies in detail. Considering the UEs subscribing to the service of MNO $a$, we simplify the coverage model under these three sharing strategies as:
\begin{itemize}
    \item \textbf{No sharing}: A single-layer GDM $D(\lambda_a,r_{a})$ with BS density $\lambda_a$ and the coverage radius $r_{a}$.
    \item \textbf{Active sharing}: A union of two independent GDMs $D(\lambda_a,r_{a})$ and $D(\lambda_b,r_{b})$. For MNO $a$, the BS density is $\lambda_a$ and the coverage radius is $r_{a}$. For MNO $b$, the BS density is $\lambda_b$ and the coverage radius is $r_{b}$.
    \item \textbf{Passive sharing}: A single-layer GDM $D(\lambda_a+\lambda_b,r_{ab})$ with BS density $\lambda_a+\lambda_b$ and the coverage radius $r_{ab}$.
\end{itemize}

\begin{figure*}[htbp]
    \centering
    \includegraphics[width=0.75\linewidth]{Figures/Geometric.pdf}
    \caption{Mathematical models and geometric interpretations considered when studying different sharing strategies based on the Gilbert disk model.}
    \label{fig:Geometric}
\end{figure*}
\subsection{No Sharing}
\indent In the `no sharing' strategy, we consider a typical UE of MNO $a$ that can be served by MNO $a$'s BSs. The density of MNO $a$'s BSs is $\lambda_a$ and the vertice set $V=\Phi_a$. The RGG $G(V,E)$ can be approximately modelled as a GDM $D(\lambda_a,r_{a})$ and the critical condition for phase transition of percolation probability of $G(V,E)$ can be approximately expressed as:
\begin{equation}\label{ama2}
    \lambda_a r_{a}^2>\frac{\lambda_c(1)}{4}.
\end{equation}
\indent As introduced in Theorem \ref{theo:acr}, the average coverage radius is:% $r_a$.
\begin{equation}
    r_a=r_m(\lambda_a).
\end{equation} Since $r_{a}$ depends on $\lambda_a$, we further propose the critical condition of BS density in Lemma \ref{lem:criconlam}.
\begin{lemma}\label{lem:criconlam}
    In `no sharing' strategy, the critical condition for phase transition of percolation probability can be written as the critical condition for MNO $a$'s BS density $\lambda_a$:
\begin{equation}\label{criconlama}
    \lambda_a>\frac{\lambda_c(1)}{4}\bigg(\frac{\beta_0}{\beta}\bigg(1-\frac{\pi \gamma \beta\lambda_c(1)}{2(\alpha-2)}\bigg)\bigg)^{-\frac{2}{\alpha}},
\end{equation}
where $\beta_0=P_t/N_0.$
\end{lemma}
\begin{IEEEproof}
    See Appendix~\ref{app:criconlam}.
\end{IEEEproof}
\indent In the condition (\ref{criconlama}), the critical value of $\lambda_a$ for phase transition of percolation probability in `no sharing' is related to the parameters  $\beta$, $P_t$, $N_0$, $\gamma$ and $\alpha$.

\subsection{Active Sharing}
In the `active sharing' case, MNOs share their access rights without deploying more BSs and transceivers. The vertice set $V=\Phi_a \cup \Phi_b$. The MNO $a$'s network and MNO $b$'s network do not interfere with each other. When the UE accesses MNO $a$'s network, the considered BS density is $\lambda_a$ and average coverage radius is $r_a=r_m(\lambda_a)$. Similarly, when the UE accesses MNO $b$'s network, the average coverage radius is: 
\begin{equation}
    r_{b}=r_m(\lambda_b),
\end{equation}
where $\lambda_b$ is the density of BSs of MNO $b$.

\indent Because BSs of MNO $a$ and MNO $b$ have different coverage radii $r_{a}$ and $r_{b}$ and different densities $\lambda_a$ and $\lambda_b$ respectively, we need to find the method to analyze the critical condition for percolation probability. We introduce the restriction region of the critical BS density condition of the superposition of two GDMs in Lemma \ref{lem:2GDpre}.
\begin{lemma}\label{lem:2GDpre}
    For the superposition of Gilbert disk models $D(\lambda_a,r_{a})$ and $D(\lambda_b,r_{b})$ with different densities and radii. The restriction region of the critical BS density condition is
    \begin{equation}\label{limit}
        \frac{\lambda_c(1)}{4\max\{r_{a}^2,r_{b}^2\}}<\lambda_a+\lambda_b<\frac{\lambda_c(1)}{4\min\{r_{a}^2,r_{b}^2\}},
    \end{equation}
where $r_{a}=r_{m}(\lambda_a)$ and $r_{b}=r_{m}(\lambda_b)$.
\end{lemma}
\begin{IEEEproof}
    See Appendix~\ref{app:2GDpre}.
\end{IEEEproof}

Further, using the conclusion in Theorem \ref{theo:percov}, we introduce the critical BS density condition for phase transition of percolation probability in Lemma \ref{lem:2GD}.
\begin{lemma}\label{lem:2GD}
    When there are two kinds of BSs: i) BSs of MNO $a$ with density $\lambda_a$ and coverage radius $r_{a}$ and ii) BSs of MNO $b$ with density $\lambda_b$ and coverage radius $r_{b}$ at the same time, the condition for phase transition of percolation probability of all BSs' coverage areas is written as:
\begin{equation}
    \lambda_a r_{a}^2+\lambda_b r_{b}^2>\frac{\lambda_c(1)}{4},
\end{equation}
where $r_{a}=r_{m}(\lambda_a)$ and $r_{b}=r_{m}(\lambda_b)$.
\end{lemma}
\begin{IEEEproof}
    See Appendix~{\ref{app:2GD}}.
\end{IEEEproof}
\indent We notice that the critical condition in Lemma \ref{lem:2GD} satisfies the restriction in (\ref{limit}). 

\subsection{Passive Sharing}
\indent In the `passive sharing' case, MNO $a$ deploys its transceivers on all BSs, including its own BSs and MNO $b$'s BSs, therefore, $V=\Phi_a\cup\Phi_b$. Since interference is caused by the signals from all BSs, the average coverage radius is:
\begin{equation}
    r_{ab} = r_m(\lambda_a+\lambda_b).
\end{equation}

In this case, the critical condition for phase transition of percolation probability can be approximately expressed as:
\begin{equation}\label{pscricon}
    (\lambda_a+\lambda_b)r_{ab}^2>\frac{\lambda_c(1)}{4}.
\end{equation}
\indent Since we consider that `passive sharing' only increases the density of transceivers, the expression of the average coverage radius is also similar to that of `no sharing'. Therefore, the critical condition (\ref{pscricon}) can be also written as the critical condition for MNOs' BS densities, \ie
\begin{equation}
    \lambda_a+\lambda_b>\frac{\lambda_c(1)}{4}\bigg(\frac{\beta_0}{\beta}\bigg(1-\frac{\pi \gamma \beta\lambda_c(1)}{2(\alpha-2)}\bigg)\bigg)^{-\frac{2}{\alpha}}.
\end{equation}

\section{Simulation results and discussion}\label{sec:simulation}

\indent In this paper, a novel concept is to use the `average coverage radius' and Gilbert disk model to approximate the coverage areas of BSs. First, we prove that such a method is valid to analyze the phase transition of percolation probability. We focus on an SINR graph where there is only one MNO. Following \cite{kouzayha2022coexisting,isabona2023accurate,kumari2019short}, we set $P_t=13 {\rm\,dB}$, $N_0=-104\,{\rm dB}$ and $\alpha=4$ to simulate a cellular network in a $4000\,{\rm m}\times 4000\,{\rm m}$ residential area with micro-cell BSs. We choose $\beta=-3\,{\rm dB}$, $\gamma=1$ to satisfy the necessary conditions for percolation (\ref{betagamma1}) and conduct 100,000 Monte Carlo experiments.  
Fig.\ref{fig:Covpro} shows the SINR coverage proportion of MNO $a$ through simulation and the theoretic curve of SINR coverage probability based on the `average coverage radius'. It shows that the average coverage radius does not overestimate the actual SINR coverage probability.

\begin{figure}[ht]
    \centering
    \includegraphics[width=0.9\linewidth]{eps/CoverageProp_beta05_gamma1.pdf}
    \caption{The relationship between the proportion of areas effectively covered by BSs providing the same MNO's service.}
    \label{fig:Covpro}
\end{figure}

\begin{figure}[htbp]
\centering
\subfigure[Percolation probability in `no sharing' strategy.]{
\begin{minipage}[t]{1\linewidth}
\centering 
\includegraphics[width=0.85\textwidth]{eps/PerP_ns.pdf}
\label{fig:PerP_ns}
\end{minipage}}

\subfigure[Percolation probability in `active sharing' strategy.]{
\begin{minipage}[t]{1\linewidth}
\centering 
\includegraphics[width=0.85\textwidth]{eps/PerP_as.pdf}
\label{fig:PerP_as}
\end{minipage}}

\subfigure[Percolation probability in `passive sharing' strategy.]{
\begin{minipage}[t]{1\linewidth}
\centering 
\includegraphics[width=0.85\textwidth]{eps/PerP_ss.pdf}
\label{fig:PerP_ss}
\end{minipage}}

\caption{Percolation probability in different sharing strategies.}
\label{fig:PerP}
\end{figure}

\begin{figure}[htbp]
\centering
\subfigure[SINR coverage proportion in `no sharing' strategy.]{
\begin{minipage}[t]{1\linewidth}
\centering 
\includegraphics[width=0.85\textwidth]{eps/Prop_ns3.pdf}
\label{fig:Prop_ns}
\end{minipage}}

\subfigure[SINR coverage proportion in `active sharing' strategy.]{
\begin{minipage}[t]{1\linewidth}
\centering 
\includegraphics[width=0.85\textwidth]{eps/Prop_as3.pdf}
\label{fig:Prop_as}
\end{minipage}}

\subfigure[SINR coverage proportion in `passive sharing' strategy.]{
\begin{minipage}[t]{1\linewidth}
\centering 
\includegraphics[width=0.85\textwidth]{eps/Prop_ss3.pdf}
\label{fig:Prop_ss}
\end{minipage}}

\caption{SINR coverage proportion in different sharing strategies.}
\label{fig:Prop}
\end{figure}

\indent Next, we can observe the relationship between percolation probability and SINR coverage proportion in Fig.\ref{fig:PerP} and Fig.\ref{fig:Prop}. Especially, focusing on the `no sharing' strategy, the phase transition of percolation probability happens when the BS density is within the range $(2\times 10^{-7},3\times 10^{-7})\;{\rm BSs/m^2}$, which means that approximate critical BS density (which is $2.73\times 10^{-7}\;{\rm BSs/m^2}$) obtained using our method in (12) is valid. When the density is less than the critical value, the percolation probability is less than $0.1$. Theoretically, broadening the simulation range and increasing simulation times can make the transition from a low level ($<0.1$) to a higher level be sharper. In each of the three sharing strategies, such a phase transition is accompanied by the increase in the SINR coverage proportion from $p_{cov}<1/2$ to $p_{cov}>1/2$, which is shown in Fig.\ref{fig:PerP} and Fig.\ref{fig:Prop}.

\indent From the perspective of the ability to form large-scale continuous service areas, we aim to obtain a high percolation probability. Suppose that a network with a percolation probability larger than 0.9 can be judged to have high global connectivity. Considering a network where MNO $b$'s BS density $\lambda_b=5\times10^{-7}\;{\rm BSs/m^2}$, no sharing strategy requires MNO $a$ to deploy its BSs at a density larger than $1\times10^{-6}\;{\rm BSs/m^2}$. Under active sharing strategy, the required BS density is $2.5\times10^{-7}\;{\rm BSs/m^2}$. Differently, under the passive sharing strategy, the required BS density is $5\times10^{-7}\;{\rm BSs/m^2}$, which is larger than that of the active sharing strategy but smaller than that of no sharing strategy. \\
\indent With the same BS densities, these three infrastructure sharing strategies show different percolation probability. For example, we consider the case where $\lambda_{a}=\lambda_{b}=5\times10^{-7}\;{\rm BSs/m^2}$. Without cooperation, `no sharing' strategy leads to only about 0.5 percolation probability. Under `passive sharing' strategy, the percolation probability can reach 0.85, while the `active sharing' leads to almost 1 percolation probability.\\
\indent Based on the simulation results, we can give suggestions to different MNOs. For the new MNOs with few BS deployment, our critical condition of phase transition can help them make the initial plan to realize a large-scale continuous service area at the lowest cost, which depends on the infrastructure sharing strategy that it adopts. For MNOs whose BS deployment is not enough (e.g. $\lambda<5\times10^{-7}\;{\rm BSs/m^2}$), active sharing and passive sharing can both help them further generate large-scale continuous service areas to a great extent. Compared to active sharing which performs better, passive sharing does not require the antenna and transceivers of other MNOs which might be expensive. For an MNO whose BS density is high enough (e.g. $\lambda>5\times10^{-7}\;{\rm BSs/m^2}$), cooperation with a new MNO with a low BS density is a good choice, where it can not only achieve a higher percolation probability at a low cost, but also earn profit from sharing their infrastructure.\\
\indent Consequently, this manuscript adopts percolation theory to provide connectivity analysis for cooperation between different MNOs. MNOs can analyze the connectivity improvement brought by infrastructure sharing based on their own and their partners' capabilities. In addition, MNOs need to consider the benefits that infrastructure sharing can bring. Regardless of the sharing strategy adopted, MNOs still need to consider the market changes after sharing. For two MNOs who adopt active sharing, since most UEs are already equipped with antenna for different MNOs, they need to further consider the potential risks and costs that may be brought by sharing access rights and protocols. For two MNOs who adopt passive sharing, they need to consider the cost of renting BSs or the profit of renting out BSs, as well as the installation cost and maintenance cost of more transceivers. In addition, this manuscript focuses on the percolation of SINR graphs. In the future, factors affecting service performance such as network congestion need to be considered, and dynamic percolation needs to be further discussed.

\section{Conclusion}
In this paper, we built mathematical models to analyze the connectivity of coverage areas in different infrastructure sharing strategies. We emphasized that percolation probability has its unique advantage in evaluating the ability to generate large-scale continuous coverage areas. Firstly, we analyzed the necessary conditions for percolation in SINR coverage graphs. Based on GDMs, we discussed the relationship between percolation probability and coverage probability. After that, we proposed an approximate tool to study the coverage of SINR graphs, \ie the `average coverage radius'. Using `average coverage radius', we studied the critical condition for phase transition of percolation probability under different infrastructure sharing strategies. To prove our concept, we conducted Monte Carlo experiments for `no sharing', `active sharing', and `passive sharing', and compared the percolation probability and SINR coverage proportion in different sharing strategies, among which we showed that active sharing can provide the best coverage performance.  In addition, we provided different suggestions for the MNOs with different sizes of BS deployment.



\appendices
\section{Proof of Theorem \ref{theo:SCAandCA}}\label{app:SCAandCA}
To prove that the face percolation of CAs of BSs is the same as the face percolation of SCAs of BSs, we need to prove that:
\begin{equation}
\bigcup\limits_{i}\Psi_i\cap\Omega_i=\bigcup\limits_{i}\Psi_i.
\end{equation}
Using the SINR $\beta(z)$, the CAs can be defined as:
\begin{equation}
\begin{array}{r@{}l}
    \Psi_i
    &\overset{\triangle}{=}\bigg\{z\in\mathbb{R}^2:\beta_i(z)\geq\beta\bigg\},
\end{array}
\end{equation}
and the Voronoi cells can be defined as:
\begin{equation}
\begin{array}{r@{}l}
    \Omega_i
    &\overset{\triangle}{=}\bigg\{z\in\mathbb{R}^2:\beta_i(z)\geq\beta_j(z),\forall j\neq i\bigg\}.
\end{array}
\end{equation}
 We assume that $\bigcup_i \Omega_i=\Omega$, where  $\Omega$ is the universal set. Since $\Psi_i\cap \Omega_i \subseteq \Psi_i$ for any $i$, we can obtain:
\begin{equation}
    \bigcup_i \Psi_i\cap \Omega_i \subseteq \bigcup_i \Psi_i.
\end{equation}
Under our assumption, if a typical user at $z$ is covered by BS $x_j$ but  the closest BS is located at $x_i$, $z\in \Omega_i\cap\Psi_j$, we have $\beta_j(z)\geq \beta$ and $\beta_i(z)\geq \beta_j(z)$. Thus,  we have $\beta_i(z)\geq\beta$. Therefore, this UE is also covered by BS located at $x_i$, \ie $z\in\Psi_i$. Therefore, $\forall i\neq j$, $\Psi_j\cap \Omega_i\subseteq\Psi_i\cap\Omega_i$, and we have
\begin{equation}
\begin{array}{r@{}l}
    \displaystyle(\bigcup_j \Psi_j)\cap \Omega_i&\displaystyle=(\Psi_i\cap\Omega_i)\cup(\bigcup_{j\neq i}\Psi_j\cup\Omega_i)\\
    &=\Psi_i\cap\Omega_i,
\end{array}
\end{equation}
which means that the SCA $\Psi_i\cap\Omega_i$ contains all coverage areas inside the Voronoi cell $\Omega_i$. Next, we can also obtain:
\begin{equation}
\begin{array}{r@{}l}
    \displaystyle\bigcup_{i}\Psi_i\cap\Omega_i&\displaystyle=\bigcup_{i}\bigg((\bigcup_j \Psi_j)\cap \Omega_i\bigg)\\
    &\displaystyle=(\bigcup_j \Psi_j)\cap(\bigcup_{i}\Omega_i)\\
    &\displaystyle=(\bigcup_j \Psi_j)\cap\Omega\displaystyle=\bigcup_j \Psi_j,
\end{array}
\end{equation}
that is, the union of SCAs is the same as the union CAs.\\ 
\indent For CAs $\Psi_i$ and $\Psi_j$, if there exists $z\in\Psi_i\cap\Psi_j$, they are connected. But for SCAs $\Psi_i\cap\Omega_i$ and $\Psi_j\cap\Omega_j$, all the points on their intersection are located on their Voronoi boundary. Therefore, if there exist some points on the common boundary where the SINR $\beta_i(z)$ and $\beta_j(z)$ are larger or equal to the threshold $\beta$, $\Psi_i\cap\Omega_i$ and $\Psi_j\cap\Omega_j$ are connected.

\section{Proof of Theorem~\ref{theo:betagamma}}\label{app:betagamma}

We focus on a typical user $z$ on the common Voronoi boundary of two neighbour BSs located at $x_i$ and $x_j$. We assume that the distance between the user and these two BSs are both $d$, \ie $\|x_i-z\|=\|x_j-z\|=d$. The BSs at $x_i$ and $x_j$ provide the strongest received power for the typical user. Therefore, the SINR at $z$ satisfies:
\begin{equation}
\begin{array}{r@{}l}
    \beta_i&(z)=\beta_j(z)\\
    &\displaystyle=\frac{P_t d^{-\alpha}}{N_0+\gamma P_t d^{-\alpha} + \gamma\sum\limits_{x_k\in\Phi\backslash \{x_i,x_j\}}P_t l(x_k-z)}\\
    &\displaystyle<\frac{P_t d^{-\alpha}}{\gamma P_t d^{-\alpha}}=\frac{1}{\gamma}.
\end{array}
\end{equation}

\indent If $\frac{1}{\gamma}\leq \beta$, \ie $\beta\gamma\geq 1$, any points on the Voronoi boundary can not achieve enough SINR. Thus, no user can realize handover between different BSs' services, and there is no percolation. Therefore, a necessary condition for the non-zero percolation probability of a single-layer GDM is $\beta\gamma<1$. 
\section{Proof of Theorem \ref{theo:devicedegree}}\label{app:devicedegree}
\indent In cellular networks, we name the closest BS as the serving BS, and name other BSs that can also provide enough SINR larger than $\beta$ as `potential serving BSs'. Denote $N$ as the number of potential serving BSs. Let $x_{(k)}$ denote the $k$th closest potential serving BS to the typical user at $z$ and $x_{(0)}$ denote the location of the serving BS. Therefore, the SINR at $z$ from BS located at $x_{(N)}$ satisfies:
\begin{equation}
\begin{array}{r@{}l}
    \beta_{(N)}(z)&=\displaystyle
    \frac{P_t \|x_{(N)}-z\|^{-\alpha}}{N_0+\gamma\sum\limits_{j=0}^{N-1} P_t \|x_{(j)}-z\|^{-\alpha}}\\
    &< \displaystyle
    \frac{P_t \|x_{(N)}-z\|^{-\alpha}}{\gamma\sum\limits_{j=0}^{N-1} P_t \|x_{(j)}-z\|^{-\alpha}}\\
    &< \displaystyle
    \frac{P_t \|x_{(N)}-z\|^{-\alpha}}{\gamma N P_t \|x_{(N)}-z\|^{-\alpha}}=\frac{1}{\gamma N}.
\end{array}
\end{equation}

Because the $x_{(N)}$ is the farthest potential serving BS, its SINR at $z$ should be larger or equal to beta, \ie $\beta_{(N)}(z)\geq\beta$. Therefore, $\frac{1}{\gamma N}>\beta$, that is, $N<\frac{1}{\gamma \beta}$.
When $\beta\gamma\geq 1$, $N=0$ and there is no potential serving BSs and all users can be only served by the closest BS. In this case, percolation can not be realized because handovers between different BSs' coverage areas are impossible.


\section{Proof of Theorem~\ref{theo:lambdac1}}\label{app:lambdac1}

 \indent To analyze the continuity of coverage areas in cellular networks, we focus on the Gilbert disk model $D(\lambda,r)$ where $\lambda$ is the density of BSs and $r$ is defined as the coverage radius. The sufficient and necessary condition for phase transition of percolation probability is:
 \begin{equation}
     \lambda>\frac{\lambda_c(1)}{4r^2}.
     \label{conditionGD}
 \end{equation}
 \begin{figure}
    \centering
    \includegraphics[width=0.5\linewidth]{Figures/InnerEnvelope.pdf}
    \caption{The `inner envelope' in Theorem \ref{theo:lambdac1}.}% and Theorem \ref{theo:increase}.}
    \label{fig:InnerEnvelope}
\end{figure}

 \indent But the value of 
$\lambda_c(1)$ is an open question. Let us focus on the first `special case'. If there is at least one BS inside the `inner envelope' shown in Fig.\ref{fig:InnerEnvelope} ($r\gg 2a$), the entire hexagon $\ncalH$ is completely covered. We have
\begin{equation}
    \P\{\ncalH {\rm \;is\; completely\; covered}\}=1-e^{-\lambda S_{\rm in}},
\end{equation}
where
\begin{equation}
    S_{\rm in}=6(\theta r^2-2S_{\bigtriangleup ABC}).
\end{equation}
is the area inside the inner envelope, $\theta$ is the radian of $\angle BAC$, and $S_{\bigtriangleup ABC}$ is the area of the triangle $\bigtriangleup ABC$. We define that $b=\|\overline{BC}\|$. Using the law of cosines, we have
\begin{equation}
\cos \frac{5\pi}{6}=\frac{a^2+b^2-r^2}{2ab}=-\frac{\sqrt{3}}{2}, 
\end{equation}
and we have $b=\sqrt{r^2-\frac{a^2}{4}}-\frac{\sqrt{3}}{2}a$. Using the law of sines, we have
\begin{equation}
    \frac{\sin \theta}{b}=\frac{\sin \frac{5\pi}{6}}{r},
\end{equation}
and $\theta=\arcsin{\frac{b}{2r}}$. The area of the triangle $\bigtriangleup ABC$ is 
\begin{equation}
    S_{\bigtriangleup ABC}=\frac{1}{2}ab\sin \frac{5\pi}{6}=\frac{1}{4}ab.
\end{equation}
\indent Therefore, the area of the inner envelope is 
\begin{equation}
    S_{\rm in}=6\theta r^2-3ab,
\end{equation}
which increases as $a$ decreases ($0<a<\frac{r}{2}$). When $a$ approaches 0, $S_{\rm in}$ approaches its maximum value $\pi r^2$.\\
\indent To achieve percolation of covered hexagons, the sufficient condition is:
\begin{equation}
    \P\{\ncalH {\rm \;is\; completely\; covered}\}=1-e^{-\lambda S_{\rm in}}>\frac{1}{2},
\end{equation}
which is equivalent to 
\begin{equation} \label{Sin}
    \lambda S_{\rm in}> \ln 2.
\end{equation}
\indent When the side length $a$ of hexagons approaches 0, all hexagons are much smaller than the coverage areas so that they can be considered discrete points. At the same time, (\ref{Sin}) becomes $\lambda \pi r^2 >\ln 2$. Notice that $\lambda \pi r^2 >\ln 2$ is the necessary but not sufficient condition for $\lambda S_{\rm in}> \ln 2$. Only when $a$ approaches 0, they are equivalent.

\begin{figure}
    \centering
    \includegraphics[width=0.8\linewidth]{Figures/OuterWall.pdf}
    \caption{The `outer envelope' and `wall' in Theorem \ref{theo:lambdac1}.}% and Theorem \ref{theo:increase}. }
    \label{fig:OuterWall}
\end{figure}

\begin{figure}
    \centering
    \includegraphics[width=0.85\linewidth]{Figures/ClosedCircuit.pdf}
    \caption{A closed circuit formed by many uncovered hexagons. There are many adjacent uncovered hexagons and `walls' to prevent coverage areas inside or outside the closed circuit from being connected.}
    \label{fig:ClosedCircuit}
\end{figure}
\indent Next, let us focus on another `special case'. If there is no BS inside the `outer envelope' shown in Fig.\ref{fig:OuterWall}, the entire hexagon $\ncalH$ is completely not covered. We have
\begin{equation}
    \P\{\ncalH {\rm \;is\; completely\; not\; covered}\}=e^{-\lambda S_{\rm out}}>\frac{1}{2},
\end{equation}
where 
\begin{equation}
    S_{\rm out}=\frac{3\sqrt{3}}{2}a^2+\pi r^2+6ar.
\end{equation}
\indent As $a$ decreases, the area of the outer envelope $S_{\rm out}$ decreases. When $a$ approaches 0, $S_{\rm out}$ approaches its minimum value $\pi r^2$. In order to achieve percolation of uncovered hexagons, the sufficient condition is:
\begin{equation}
    \P\{\ncalH {\rm \;is\; completely\;not \; covered}\}=e^{-\lambda S_{\rm out}}>\frac{1}{2},
\end{equation}
which is equivalent to 
\begin{equation} \label{Sout}
    \lambda S_{\rm out}<\ln 2.
\end{equation}
\indent Similarly, decreasing the value of side length $a$, all hexagons are much smaller than coverage areas of BSs and (\ref{Sout}) becomes $\lambda \pi r^2 <\ln 2$. Notice that $\lambda \pi r^2 <\ln 2$ is also only the necessary but not sufficient condition for $\lambda S_{\rm out}< \ln 2$. They are equivalent when $a$ approaches 0 (much less than $r$).\\
\indent Only the percolation of uncovered hexagons is not enough to prove that the percolation probability of coverage areas is $0$. As shown in Fig.\ref{fig:OuterWall}, two adjacent uncovered hexagons form a `wall' to avoid the continuity of coverage areas. The minimum distance between BSs on different sides of the `wall' is $a+\frac{4}{\sqrt{3}}r$ which is always larger than $2r$. That means the wall formed by two adjacent uncovered hexagons prevents the connection of coverage areas of BSs on both sides of this wall. We can assume that there is a BS in the origin. When $\P\{\ncalH {\rm \;is\; completely\; not\; covered}\}=e^{-\lambda S_{\rm out}}>\frac{1}{2}$, a circuit of uncovered hexagons (as shown in Fig.\ref{fig:ClosedCircuit}) is formed around the origin, which consists of many walls that can prevent the continuity of coverage areas.\\
\indent For the Gilbert disk model, the transformation of scaling does not affect the percolation state of a random graph and the percolation probability of $D(\lambda,r)$. Mathematically, $\forall A>0$, $D(\lambda,r)$ and $D(A^2\lambda,r/A)$ have the same percolation probability. This indicates that the critical value of $\lambda \pi r^2$, the product of density $\lambda$ and coverage area $\pi r^2$, is a vital factor. Such a concept is also mentioned in \cite{haenggi2012stochastic}, which means the continuous percolation can be analyzed using discrete percolation. When the side length of considered hexagons is much less than the coverage radius, the critical value of $\lambda\pi r^2$ is $\ln 2$. Because we have defined $\lambda(1)=\lambda (2r)^2$, its critical value is $\lambda_c(1)=4\ln 2/\pi$. 

\section{Proof of Theorem~\ref{theo:acr}}\label{app:acr}
We consider a single-layer cellular network where all BSs interfere with each other. Assume that $\|x_{i}-z\|$ is the distance from a typical UE at $z$ to a considered BS located at $x_{i}$. We define $r_{1}=\|x_{1}-z\|=\min\{\|x_{i}-z\|\}$, where $x_{1}$ is the location of the nearest BS to typical UE and we assume that $r_{1}>1\,{\rm m}$. Based on (\ref{betaiz}), the SINR from the considered BS to the typical UE is 
\begin{equation}
    \beta_{i}(z)=\frac{P_t \|x_{i}-z\|^{-\alpha}}{N_0+\gamma\sum\limits_{x_{j}\in\Phi\setminus x_{i}}P_t\|x_{j}-z\|^{-\alpha}}.
\end{equation}
\indent Therefore, the typical UE at $y$ can be covered by cellular networks when $\max\{\beta_{i}(z)\}\geq\beta$, where $\beta$ is the SINR threshold. This event can be also expressed as:
\begin{equation}
\frac{P_t r_{1}^{-\alpha}}{N_0+\gamma P_t\sum\limits_{x_{j}\in\Phi\setminus x_{1}}\|x_{j}-z\|^{-\alpha}}\geq\beta.
\end{equation}

Consider the users at the boundary of CAs, the distances to BSs satisfy:
\begin{equation}\label{nosharing1}
    r_{1}^{-\alpha}=\frac{\beta}{P_t/N_0}+\gamma\beta \sum\limits_{x_{j}\in\Phi\setminus x_{1}}\|x_{j}-z\|^{-\alpha}.
\end{equation}
Especially for the users on the boundary of SCAs, the distances to all interfering BSs are larger than $r_1$. \\
\indent In order to apply the Gilbert disk model to analyze the critical condition for percolation, we propose a parameter `average coverage radius' to approximate the BS's coverage range. The serving BS is at a distance $r_m$ to the typical user and the locations of interfering BSs follow a PPP with density $\lambda$ outside the circular area centered at the user with radius $r$. 
Using Campbell's theorem, we use an integral to calculate the expectation of interference conditioned on $r_m$:
\begin{equation}
\begin{array}{@{}r@{}l}
  \E\bigg[\sum\limits_{x_{j}\in\Phi\setminus x_{1}}\|x_{j}-z\|^{-\alpha}\bigg]&=\displaystyle\int_{r_{m}}^{\infty}r^{-\alpha}2\lambda\pi rdr\\
  &\displaystyle=\frac{2\pi \lambda}{\alpha-2}r_{m}^{2-\alpha}.
  \end{array}  
\end{equation}
\indent Therefore, $r_m$ is the solution of the following equation:
\begin{equation}
    r^{-\alpha}=\frac{\beta}{P_t/N_0}+\frac{2\pi \gamma\beta\lambda}{\alpha-2}r^{2-\alpha}.
\label{appro}
\end{equation}
Multiply $r^{\alpha}$ on both sides of (\ref{appro}), we have:
\begin{equation}\label{appro2}
    \frac{\beta}{P_t/N_0}r^{\alpha}+\frac{2\pi \gamma\beta\lambda}{\alpha-2}r^{2}=1.
\end{equation}
\indent Define that
\begin{equation}
\ncalF(r,\lambda)=\frac{\beta}{P_t/N_0}r^{\alpha}+\frac{2\pi \gamma\beta\lambda}{\alpha-2}r^{2},
\label{Frlam}
\end{equation}
which is an increasing function of $r$ when $\alpha>2$. When $r=0$,  $\ncalF(r,\lambda)=0$. When $r>(\frac{P_t/N_0}{\beta})^{\frac{1}{\alpha}}$ or $r>(\frac{\alpha-2}{2\pi\gamma\beta\lambda})^{\frac{1}{2}}$, $\ncalF(r,\lambda)>1$. Therefore, the solution of (\ref{appro2}), $r_m$, is unique which satisfies $0<r_{m}<\min\{(\frac{P_t/N_0}{\beta})^{\frac{1}{\alpha}},(\frac{\alpha-2}{2\pi\gamma\beta\lambda})^{\frac{1}{2}}\}$. For a higher $\lambda$, the value of $r_m$ becomes smaller. Therefore, the average coverage radius $r_{m}$ can be written as a non-increasing function of $\lambda$, and the `border condition' in Fig.\ref{fig:onlysolution} is the curve of the implicit function $r_{m}(\lambda)$ hiding in (\ref{appro2}). \\
\indent To approximate the actual network deployment as much as possible, we assume that all BSs can always cover the UEs inside their circular coverage areas with a unit coverage radius. Therefore, the average coverage radius $r_m$ should be larger than 1\;{\rm m}. Based on this, we have 
\begin{equation}
\ncalF(1,\lambda)=\displaystyle\frac{\beta}{P_t/N_0}+\frac{2\pi \gamma\beta\lambda}{\alpha-2}<1.
\end{equation}

\indent From (\ref{Frlam}), because $r_m^{\alpha}>r_m^{2}$, we have
\begin{equation}
\displaystyle\mathcal{F}(1,\lambda)r_{m}^{2}<\ncalF(r_{m},\lambda)
<\displaystyle\mathcal{F}(1,\lambda)r_{m}^{\alpha}.
\label{inequality2}
\end{equation}

Since $\mathcal{F}(r_m,\lambda)=1$,  $r_{m}$ satisfies 
\begin{equation}
\begin{array}{@{}r@{}l}
\displaystyle\mathcal{F}(1,\lambda)^{-\frac{1}{\alpha}}<r_{m}<\mathcal{F}(1,\lambda)^{-\frac{1}{2}}.
\end{array}
\label{inequality3}
\end{equation}
\indent The inequality (\ref{inequality3}) is also shown in Fig.\ref{fig:onlysolution}. The curve of $r_{m}(\lambda)$, \ie `border condition', is between the `lower bound' $r_L(\lambda)=\mathcal{F}(1,\lambda)^{-\frac{1}{\alpha}}$ and the `upper bound' $r_U(\lambda)=\mathcal{F}(1,\lambda)^{-\frac{1}{2}}$.\\
\begin{figure}
    \centering
    \includegraphics[width=0.9\linewidth]{eps/OnlySolution.pdf}
    \caption{The implicit function $r_{m}(\lambda)$ introduced in (\ref{appro}) and (\ref{appro2}) is shown as the curve `Border Condition'. It is between the lower bound and upper bound that are shown in (\ref{inequality3}). For each density $\lambda$, the solution of approximate coverage radius is unique. Equation (\ref{percona}) is also shown as the `Percolation Condition' curve. There is only one intersection of the `Percolation Condition' and `Border Condition' curves.}
    \label{fig:onlysolution}
\end{figure}


\section{Proof of Theorem \ref{theo:percov}}\label{app:pcov}
\indent To analyze continuous percolation using discrete percolation in hexagons, we focus on the `inner envelope' and `outer envelope' in Fig.\ref{fig:InnerEnvelope} and Fig.\ref{fig:OuterWall}. The areas of them satisfy:
\begin{equation}
    S_{\rm in}<\pi r^2<S_{\rm out}.
\end{equation}
\indent When the side length $a$ of considered hexagons is much less than the coverage radius, the shapes of `inner envelope' and `outer envelope' are almost the same as a circular area with radius $r$ and the areas of them both converge to $\pi r^2$, \ie
\begin{equation}
    \lim\limits_{a\ll r, a\rightarrow 0^+}S_{\rm in}=\lim\limits_{a\ll r, a\rightarrow 0^+}S_{\rm out}=\pi r^2.
\end{equation}

When $a$ approaches 0, the discrete percolation of hexagons is used to approximate the continuous percolation of points. The probability of a point being covered (\ie coverage probability) is 
\begin{equation}
    p_{cov}=\lim\limits_{a\rightarrow 0^+}1-e^{-\lambda S_{\rm in}}=1-e^{-\lambda\pi r^2},
\end{equation} 
and the probability of a point being not covered is
\begin{equation}
    \lim\limits_{a\rightarrow 0^+}e^{-\lambda S_{\rm out}}=e^{-\lambda\pi r^2}=1-p_{cov}.
\end{equation} 
\indent Similar to Theorem \ref{theo:lambdac1}, if $p_{cov}$ is larger than $1/2$, the percolation probability is non-zero. If $1-p_{cov}$ is larger than $1/2$, the percolation probability is zero, while $p_{cov}$ is less than $1/2$ at the same time. Therefore, for continuous percolation, the phase of percolation probability (zero or non-zero) depends on whether $p_{cov}$ is greater than $1/2$ or not.

\section{Proof of Lemma~\ref{lem:criconlam}}\label{app:criconlam}
\indent In Theorem \ref{theo:acr}, we obtain the relationship between BS density $\lambda_a$ and its corresponding average coverage radius $r_{a}$. Based on (\ref{GDMcricon}), the critical condition for phase transition of percolation probability is 
\begin{equation}\label{GDMcricona}
\lambda_a r_{a}^2=\frac{\lambda_c(1)}{4}.
\end{equation}
\indent Let $\beta_0=P_t/N_0$, $r_a=r_{m}(\lambda_a)$ and substitute (\ref{GDMcricona}) into (\ref{appro2}), we have:
\begin{equation}\label{percona}
    \frac{\beta}{\beta_0}r_{a}^{\alpha}+\frac{\pi\gamma\beta\lambda_c(1)}{2(\alpha-2)}=1.
\end{equation}

Therefore, the average coverage radius can be derived as:
\begin{equation}\label{crirad}
    r_{a}=\bigg(\frac{\beta_0}{\beta}\bigg(1-\frac{\pi \gamma \beta\lambda_c(1)}{2(\alpha-2)}\bigg)\bigg)^{\frac{1}{\alpha}}.
\end{equation}
\indent Substitute (\ref{crirad}) into (\ref{GDMcricona}), the critical value of BS density is 
\begin{equation}\label{lambdaac}
    \lambda_a=\frac{\lambda_c(1)}{4}\bigg(\frac{\beta_0}{\beta}\bigg(1-\frac{\pi \gamma \beta\lambda_c(1)}{2(\alpha-2)}\bigg)\bigg)^{-\frac{2}{\alpha}}.
\end{equation}
\indent The `percolation condition' in Fig.\ref{fig:onlysolution} shows the critical relationship in (\ref{GDMcricona}). There is only one intersection between the `border condition' and `percolation condition', which shows the unique solution of critical BS density in (\ref{lambdaac}).  

% \section{Proof of Lemma~\ref{lem:activesharing}}\label{app:activesharing}
% \indent In addition, the existence of such an only solution and its range are also shown in Fig.\ref{fig:onlysolution}
% \begin{figure}
%     \centering
%     \includegraphics[width=1\linewidth]{eps/OnlySolution.pdf}
%     \caption{The existence of the only solution of (\ref{appro2}) and its range.}
%     \label{fig:onlysolution}
% \end{figure}

\section{Proof of Lemma~\ref{lem:2GDpre}}\label{app:2GDpre}
\indent In the graph $G(V,E)$, the sets of BSs' locations of MNO $a$ and MNO $b$ are modeled using independent PPPs $\Phi_a$ with density $\lambda_a$ and $\Phi_b$ with density $\lambda_b$. The vertice set $V$ is the superposition of $\Phi_a$ and $\Phi_b$, \ie $V=\Phi_a \cup \Phi_b$, which can be modeled as a PPP with density $\lambda_a+\lambda_b$. Approximated by the superposition of two Gilbert disk models $D(\lambda_a,r_{a})$ and $D(\lambda_b,r_{b})$, the edge set $E$ can be approximated using $E_a\cup E_b \cup E_{ab}$. The edge set $E_a$ is
\begin{equation}
    E_a=\{\overline{x_{i}x_{j}}:\|x_{i}-x_{j}\|\leq 2r_{a},\,\forall\,x_{i},x_{j}\in\Phi_a\},
\end{equation}
which represents the connectivity between coverage areas of MNO $a$'s BSs. Similarly, the edge set $E_b$ represents the connectivity between coverage areas of MNO $b$'s BSs:
\begin{equation}
    E_b=\{\overline{x_{i}x_{j}}:\|x_{i}-x_{j}\|\leq 2r_{b},\,\forall\,x_{i},x_{j}\in\Phi_b\}.
\end{equation}
\indent Differently, the set $E_{ab}$ represents the connectivity between MNO $a$'s coverage areas and MNO $b$'s coverage areas, which is expressed as:
\begin{equation}
\begin{array}{@{}r@{}l}
    E_{ab}=\{\overline{x_{i}x_{j}}:&\|x_{i}-x_{j}\|\leq r_{a}+r_{b},\\ &\;\;\;\;\;\;\;\;\;\;\;\;\;\;\;\;\;\;\;\;\forall\,x_{i}\in\Phi_a,x_{j}\in\Phi_b\}.  
\end{array}
\end{equation}
\indent Next, we prove the sufficient condition for zero and non-zero percolation probability. We first consider an extreme graph $G(V,E_L)=\{V,E_L\}$, where the edge set $E_L$ is
\begin{equation}
\begin{array}{@{}r@{}l}
    E_L=\{\overline{x_{i}x_{j}}:\|x_{i}-x_{j}\|\leq & 2\max\{r_{a},r_{b}\},\\
    &\;\;\;\;\;\;\;\;\;\;\;\forall\,x_{i},x_{j}\in V\}.
\end{array}
\end{equation}
\indent At the same time, consider another extreme graph $G(V,E_U)=\{V,E_U\}$, where the edge set $E_U$ is
\begin{equation}
\begin{array}{@{}r@{}l}
    E_U=\{\overline{x_{i}x_{j}}:\|x_{i}-x_{j}\|\leq &2\min\{r_{a},r_{b}\},\\
    &\;\;\;\;\;\;\;\;\;\;\;\forall\,x_{i},x_{j}\in V\}.
\end{array}
\end{equation}
\indent We can obtain the relationship between the edge sets:
\begin{equation}
    E_U\subseteq E,\, E\subseteq E_L.
\end{equation}
\indent Therefore, the relationship between $G(V,E)$, $G(V,E_L)$ and $G(V,E_U)$ is shown as follows:
\begin{equation}
    G(V,E_U)\subseteq G(V,E),\,G(V,E)\subseteq G(V,E_L).
    \label{relationship}
\end{equation}
\indent As a common understanding, the critical value of BS density $\lambda$ for a GDM $D(\lambda,r)$ is $\lambda_c(1)/4r^2$. Therefore, the sufficient condition for $\theta(\lambda_a+\lambda_b,\max\{r_{a},r_{b}\})=0$ is 
\begin{equation}
    \lambda_a+\lambda_b<\frac{\lambda_c(1)}{4\max\{r_{a},r_{b}\}^2}.
\label{lower}
\end{equation}
\indent Similarly, we can also obtain the sufficient condition for $\theta(\lambda_a+\lambda_b,\min\{r_{a},r_{b}\})>0$: 
\begin{equation}
    \lambda_a+\lambda_b>\frac{\lambda_c(1)}{4\min\{r_{a},r_{b}\}^2}.
\label{upper}
\end{equation}
\indent According to the relationship between $G(V,E)$, $G(V,E_L)$ and $G(V,E_U)$ shown in (\ref{relationship}), (\ref{lower}) is the sufficient condition for $\theta(\lambda_a,\lambda_b)=0$ and (\ref{upper}) is the sufficient condition for $\theta(\lambda_a,\lambda_b)>0$. These two conditions form the restriction region of the critical condition for phase transition of percolation probability. 
\section{Proof of Lemma~\ref{lem:2GD}} \label{app:2GD}

\indent For active sharing case, we want to analyze the critical condition for percolation of coverage areas when we have two independent Gilbert disk models $D(\lambda_a,r_{a})$ and $D(\lambda_b,r_{b})$. In Theorem \ref{theo:percov}, we prove that the percolation probability is related to the coverage probability of any point. In this case, the probability of a point being covered is $p_{cov}=1-e^{-\lambda_a \pi r_{a}^2 -\lambda_b \pi r_{b}^2}$ and the probability of a point being not covered is $1-p_{cov}=e^{-\lambda_a \pi r_{a}^2 -\lambda_b \pi r_{b}^2}$. Based on (\ref{percov}) in Theorem \ref{theo:percov}, the critical condition for the phase transition of percolation probability can be simplified as: 
\begin{equation}
    \lambda_a \pi r_a^2 + \lambda_b \pi r_b^2 = \ln 2.
\end{equation}


\ifCLASSOPTIONcaptionsoff
  \newpage
\fi

\bibliographystyle{IEEEtran}
\documentclass{MITstyle}

%\usepackage[table]{xcolor}
\usepackage{chngcntr}
\usepackage{hyperref}
\usepackage{microtype}

\title{A Lightweight and Extensible Cell Segmentation and Classification Model for Whole Slide Images}

\author{Nikita Shvetsov~$^{1, }$\footnote{Correspondence e-mail: nikita.shvetsov@uit.no}, Thomas K. Kilvaer~$^{2, 3}$, Masoud Tafavvoghi~$^{4}$, Anders Sildnes~$^{1}$, \\ Kajsa Møllersen~$^{4}$, Lill-Tove Rasmussen Busund~$^{5, 6}$, Lars Ailo Bongo~$^{1}$ \\
%
\vspace{1em} % Space between authors and afilliations
%
\normalfont{\small $^{1}$Department of Computer Science, UiT The Arctic University of Norway}\\
\normalfont{\small $^{2}$Department of Oncology, University Hospital of North Norway}\\
\normalfont{\small $^{3}$Department of Clinical Medicine, UiT The Arctic University of Norway}\\
\normalfont{\small $^{4}$Department of Community Medicine, UiT The Arctic University of Norway}\\
\normalfont{\small $^{5}$Department of Medical Biology, UiT The Arctic University of Norway} \\
\normalfont{\small $^{6}$Department of Clinical Pathology, University Hospital of North Norway} %\vspace{2em}
}

\begin{document}
\maketitle

\section*{Abstract}

% \begin{abstract}
% Developing clinically useful cell-level analysis tools in digital pathology remains challenging due to limitations in dataset granularity, inconsistent annotations, computational demands of advanced models, and difficulties in integrating new technologies into clinical workflows. To address these challenges, we propose a multi-faceted solution that enhances data quality, model performance, and usability to create a lightweight and extensible cell segmentation and classification model.

% First, we update data labels by employing a cross-relabeling process that refines the labels of two existing datasets, PanNuke and MoNuSAC, to create a new unified dataset with enhanced granularity, encompassing seven distinct cell types. Second, we leverage the H-Optimus foundation model as a fixed encoder to improve feature representation for simultaneous cell segmentation and classification tasks. Third, to address the computational demands of foundation models, we employ knowledge distillation to reduce model size and complexity while maintaining comparable performance. Finally, to facilitate integration into clinical workflows, we integrate the distilled model into the QuPath software, a widely used open-source platform in digital pathology.

% Our results demonstrate improvements in cell segmentation and classification performance using the H‑Optimus-based model compared to a CNN-based model. Specifically, the average $R^2$ improved from 0.575 to 0.871, and the average $PQ$ score improved from 0.450 to 0.492, indicating better alignment with actual cell counts and enhanced segmentation and classification quality. Furthermore, the distilled student model maintains performance comparable to the larger foundation model while reducing the parameter count by a factor of 48.
% Overall, by reducing computational complexity and integrating it into existing workflows, the proposed approach may significantly impact diagnostic processes, reduce the workload of pathologists, and contribute to improved patient outcomes. Though our approach shows potential enhancements in efficiency and usability of cell segmentation and classification models in digital pathology, extensive validation is needed to deploy these models in clinical practice.
% \end{abstract}

%%% shortened abstract
\begin{abstract}
Developing clinically useful cell-level analysis tools in digital pathology remains challenging due to limitations in dataset granularity, inconsistent annotations, high computational demands, and difficulties integrating new technologies into workflows. To address these issues, we propose a solution that enhances data quality, model performance, and usability by creating a lightweight, extensible cell segmentation and classification model. 

First, we update data labels through cross-relabeling to refine annotations of PanNuke and MoNuSAC, producing a unified dataset with seven distinct cell types. Second, we leverage the H-Optimus foundation model as a fixed encoder to improve feature representation for simultaneous segmentation and classification tasks. Third, to address foundation models' computational demands, we distill knowledge to reduce model size and complexity while maintaining comparable performance. Finally, we integrate the distilled model into QuPath, a widely used open-source digital pathology platform. 

Results demonstrate improved segmentation and classification performance using the H-Optimus-based model compared to a CNN-based model. Specifically, average $R^2$ improved from 0.575 to 0.871, and average $PQ$ score improved from 0.450 to 0.492, indicating better alignment with actual cell counts and enhanced segmentation quality. The distilled model maintains comparable performance while reducing parameter count by a factor of 48. By reducing computational complexity and integrating into workflows, this approach may significantly impact diagnostics, reduce pathologist workload, and improve outcomes. Although the method shows promise, extensive validation is necessary prior to clinical deployment.
\end{abstract}
\clearpage

\section{Introduction}
In digital pathology, accurate segmentation and classification of cells are crucial for many diagnostic, prognostic, and predictive analyses \cite{Jaber_Beziaeva_etal._2019,Lin_Pan_etal._2022,Park_Ock_etal._2022,Shen_Choi_etal._2024}. Nowadays, developments in computational pathology offer multiple solutions \cite{H._Qu_P._Wu_etal._2020,Javed_Mahmood_etal._2020} to utilize cell-level datasets to train machine learning models that solve these problems. The quality and specificity of training datasets are critical for robust and accurate models. Adhering to the principle of "garbage in, garbage out", it is essential to ensure that these datasets are extensively and accurately labeled with distinct classes that reflect the diverse biological characteristics of different cell types. Unfortunately, the number of open-source datasets comprising such high-quality annotations is limited. Existing cell segmentation datasets \cite{Gamper_Koohbanani_etal._2019,Graham_Vu_etal._2019,Verma_Kumar_etal._2021} may offer extensive annotations for certain cell types while providing more general labels for others. For example, in PanNuke, which is one of the largest open-source datasets comprising labeled cells, various types of morphologically and functionally different inflammatory cells like macrophages and lymphocytes are clustered in a broad "inflammatory" class. Consequently, these classes are frequently omitted from analyses or aggregated into broader meta-classes \cite{Gamper_Koohbanani_etal._2020} and likely interfere with other cell classes included in the dataset. This and similar inconsistencies in annotation granularity limit the ability of machine learning models to learn the comprehensive and nuanced features necessary for accurate cell segmentation and classification. To address these challenges, methods for refining and standardizing dataset annotations are essential to enhance the quality of training data.

A complementary approach to mitigate the absence of high-quality training data is the use of foundation models. Foundation models as encoders are defined as large-scale, versatile networks pre-trained on vast, diverse datasets using self-supervised learning, contrasting with convolutional neural network (CNN) pre-trained encoders that rely on supervised learning with labeled data. In practice, foundation models leverage enormous amounts of weakly or unlabeled data from millions of whole slide images (WSIs) and employ self-attention mechanisms to capture long-range dependencies and global context \cite{Chen_Ding_etal._2024,Saillard_Jenatton_etal._2024,Vorontsov_Bozkurt_etal._2024,Xu_Usuyama_etal._2024}. As a consequence, foundation models are able to produce transferable feature representations across different cell types and tissue environments. The feature representations can be leveraged by decoder networks to produce segmentation masks and pixel-level classifications. Because foundation models have comprehensive feature representations, they can be effectively fine-tuned using much smaller amounts of cell-level data compared to the large datasets needed to train models from scratch. Furthermore, foundation models incorporate adversarial training elements or contrastive learning \cite{Chen_Ding_etal._2024,Xu_Usuyama_etal._2024}, enhancing their resilience and adaptability by exposing them to challenging and varied scenarios during training. This may result in more generalizable models, often making them well-suited for diverse and complex tasks in digital pathology.

Despite the inherent advantages of foundation models, their deployment for practical use faces its own obstacles. In particular, they require substantial computational power, financial investments and rigorous testing to ensure reliability and efficacy for a given task \cite{Akkus_Dangott_etal._2022,Dragomir_Cocuz_etal._2022,Go_2022,Jafri_Farooqui_etal._2024}. Moreover, while foundation models enhance feature representation and performance, they depend on the quality of available annotations for decoder fine-tuning and, like any other model, cannot resolve existing inconsistencies or ambiguities in data labels. Therefore, there remains a critical need for solutions that address both data quality and practical deployment considerations.
Further, integrating new technologies into existing clinical workflows often encounters resistance, as it necessitates adjustments to established diagnostic processes. So, there is a need to develop solutions that could be integrated into current practices, minimizing the burden on medical professionals to adopt new tools \cite{King_Williams_etal._2023}.

Existing solutions \cite{Goldsborough_Philps_etal._2024,Hörst_Rempe_etal._2024}, while addressing some aspects of these challenges, fall short in providing a comprehensive approach. To address the data quality and clinical deployment issues, we propose a multi-faceted solution that encompasses data refinement, model optimization, and integration with existing pathology tools (\hyperref[fig:fig1]{Figure 1}). The outcome is a lightweight cell segmentation and classification model that can be integrated into digital pathology workflows for practical clinical use.

\begin{figure}[h!]
    \centering
    \includegraphics[width=\textwidth, height=0.82\textheight, keepaspectratio]{images/Figure_1.pdf}
    \caption{Overview of the proposed solution, including 1) Data refinement using cross-relabeling, 2) Teacher model development and fine tuning, 3) Student model optimization with knowledge distillation and 4) Student model and QuPath integration}
    \label{fig:fig1}
\end{figure}
\clearpage

Our approach begins with preparing the data for the fine-tuning and training of the machine learning models. We create a refined dataset, acquired via cross-relabeling two cell-level datasets, enhancing annotation specificity and consistency of the labeled data. Subsequently, we create a cell segmentation and classification model based on the foundation model. We leverage the foundation model as a fixed encoder and fine-tune a decoder using the refined dataset to improve generalization across diverse tissue- and cell types.
To ensure that the model remains lightweight and deployable in a possibly resource-constrained environment, we employ knowledge distillation to approximate the functionality of the foundation model. Finally, to facilitate the practical application of our model in digital pathology workflows, we integrate it with the QuPath \cite{Bankhead_Loughrey_etal._2017} application. Each methodological component contributes to the overarching goal of enhancing model performance, generalizability, and usability in clinical settings.

The primary contributions of this paper are:
\begin{enumerate}
    \item \textit{Data labels refinement through cross-relabeling:}
    
    We propose a new method for refining labels of cell-level datasets through cross-relabeling. This method employs classification models to re-label broad and ambiguous instances, resulting in a more diverse dataset. Our evaluation demonstrates that these classification models achieve high accuracy on test subsets, indicating the reliability of the method for label refinement.

    \item \textit{Enhanced model performance via foundation models:}
    
    We employ a foundation model as a feature extractor for the cell segmentation and classification task. In comparison with training a CNN model from scratch, the foundation model backbone only needs fine-tuning, which significantly reduces training time, computational resources and data requirements. We show that using a foundation model encoder leads to better performance in cell segmentation and classification networks than using a CNN-based encoder. This improvement may enable the model to generalize more effectively across various tissue types and imaging methods.
    
    \item \textit{Model optimization through knowledge distillation:}
    
    We show that a smaller student model trained using knowledge distillation on the refined dataset obtained via our cross-relabeling approach from a foundation model achieves comparable performance in cell segmentation and quantification tasks. As a result, this model is more suitable for deployment in environments without high-performance computing resources.
    
    \item \textit{Integration with QuPath:}
    
    We integrate the distilled cell segmentation and classification model into QuPath, a widely used open-source digital pathology platform, to accelerate clinical adaptation by enabling pathologists to more easily incorporate advanced computational tools into their existing workflows.
\end{enumerate}

Through these methodological steps, we aim to bridge the gap between advanced machine learning techniques and practical clinical applications, making accurate and efficient digital pathology accessible in a broader range of healthcare settings.

\section{Refining Existing Datasets Using Cross-Relabeling}
To address the limitations of sparse and ambiguous labeling of cell-level datasets, we propose a generalizable cross-relabeling strategy that can be applied to any dataset containing broadly categorized or imprecisely labeled cell types. This approach involves training and subsequently leveraging classification models to refine broad categories into more specific or biologically relevant classes.
When applied to cell-level data, the methodology includes extracting individual cell images from the dataset patches, preprocessing these images to standardize the size and accommodate partial cells, and then training deep learning classifiers capable of distinguishing between the finer cell subtypes within the coarser categories. 
To illustrate our approach, we focus on the PanNuke \cite{Gamper_Koohbanani_etal._2020, Gamper_Koohbanani_etal._2019} and MoNuSAC \cite{Verma_Kumar_etal._2021} datasets that we have used to train models for cell quantification in our previous works \cite{Shvetsov_Grønnesby_etal._2022,Shvetsov_Sildnes_etal._2024}. We find that for better cell differentiation we have to introduce more granular labels. PanNuke includes a broad classification of "inflammatory" cells, encompassing lymphocytes, macrophages, and neutrophils. Each cell type differs significantly in structure, function, and clinical relevance. Conversely, MoNuSAC uses the label "epithelial" for a class that comprises both benign epithelial cells and malignant neoplastic cells. This practice makes it challenging to differentiate between benign and malignant epithelial cells in the dataset, which is a critical distinction when identifying tumor areas within tissue samples. To address these issues, we implement a cross-relabeling strategy as shown in \hyperref[fig:fig2]{Figure 2}. The key components are two classification models: one is trained on singular cell images from PanNuke data to classify the epithelial meta-class into epithelial and neoplastic classes. The other is trained on MoNuSAC to refine the inflammatory class into lymphocytes, neutrophils, and macrophages.

\begin{figure}[h!]
    \centering
    \includegraphics[width=\textwidth]{images/Figure_2.pdf}
    \caption{Refined dataset generation via cross relabeling}
    \label{fig:fig2}
\end{figure}

The refining approach consists of three consecutive steps. The first is the preprocessing step, in which we extract individual cells from both datasets (\hyperref[fig:fig3]{Figure 3}). The specifics of PanNuke and MoNuSAC patch preparation before cell preprocessing are provided in \hyperref[chap:S1]{Appendix S1}.

\begin{figure}[h!]
    \centering
    \includegraphics[width=\textwidth]{images/Figure_3.pdf}
    \caption{Cell instances preprocessing including (1) cell map extraction, (2) bounding box delineation, (3) adjusting cell boxes and (4) cropping and resizing of cell images}
    \label{fig:fig3}
\end{figure}

During preprocessing, we extract cell type maps from the ground truth label mask and calculate bounding boxes around each cell instance. To accommodate partial cells at patch borders, a common issue in cropped patch images, we employ mirror padding and extend the field of view of the cell label by 15 pixels to capture adjacent cells. We then crop and resize the identified regions to $64 \times 64$ pixels using bicubic interpolation.

The preprocessed PanNuke dataset comprises 68,031 neoplastic and 23,207 epithelial cell images, while MoNuSAC comprises  33,104 lymphocytes, 1,252 neutrophils, and 1,695 macrophages, which we subsequently use in training cell classification models and classifying the cell image data \hyperref[fig:S2]{Appendix Figure S2 (1)}. 

The next step is to train two distinct ResNet50-based classifiers tailored to address the specific labeling challenges inherent in each dataset. We use ResNet50 for classification models due to its proven effectiveness for image classification tasks in histopathology \cite{pan2022reviewmachinelearningapproaches}, and its compatibility with small images. For the PanNuke dataset, we design the classifier, trained on MoNuSAC data, to disaggregate the heterogeneous "inflammatory" cell category into distinct subtypes: lymphocytes, macrophages, and neutrophils. Similarly, for the MoNuSAC dataset, the classifier is trained on PanNuke data and distinguishes between benign and malignant epithelial cells within the overarching "epithelial" label. By applying these targeted classifiers to their respective datasets, we assign more specific labels to individual cell instances, thus enabling us to create a unified dataset.
To ensure a balanced representation of classes, we train both models on datasets that had been equalized to match the size of the least represented class. Thus, we obtain datasets comprising 23,207 samples per class for PanNuke and 1,252 samples per class for MoNuSAC data. Next, we partition both of them into training (70\%), validation (20\%), and testing (10\%) subsets. To mitigate the risk of overfitting, we use a single dropout layer with a rate of p=0.5 in both models and data augmentation using randomized color perturbations, rotation, and horizontal and vertical flipping. We employ AdamW optimizer and the cross-entropy loss function for the training criterion.

To evaluate the two trained models, we measure the classification accuracy on the respective test subsets. The accuracies on the test subset for both classifiers are presented in \hyperref[tab:1]{Table 1}. The PanNuke model achieves an average accuracy of 93.57\%, with higher accuracy for neoplastic cells (96.06\%) compared to epithelial cells (86.26\%). The confusion matrix in Figure A3.1 shows that the model predominantly distinguishes accurately between epithelial and neoplastic tissues, with a substantial number of correct classifications and relatively few misclassifications. The MoNuSAC model demonstrates an average accuracy of 98.92\%, excelling in classifying lymphocytes (99.67\%) and macrophages (94.12\%), with lower performance for neutrophils (85.71\%). The confusion matrix in Figure A3.2 shows that the model identifies lymphocytes and performs reasonably well with macrophages and neutrophils.

\begin{table}[h!]
\renewcommand{\arraystretch}{1.5}
  \centering
  \caption{Cell classification results for PanNuke and MoNuSAC trained models (CI 95\%).}
  \label{tab:1}
  \begin{tabular}{|l|c|c|}
   \hline
   %\rowcolor{gray!30}
    Accuracy               & PanNuke model              & MoNuSAC model              \\
    \hline
    Average      & 0.936 (0.931--0.941)         & 0.989 (0.986--0.993)        \\
    \hline
    Neoplastic   & 0.961 (0.956--0.965)         & -                          \\
    \hline
    Epithelial   & 0.863 (0.849--0.877)         & -                          \\
    \hline
    Lymphocytes  & -                          & 0.997 (0.995--0.999)        \\
    \hline
    Neutrophils  & -                          & 0.857 (0.796--0.918)        \\
    \hline
    Macrophages  & -                          & 0.941 (0.906--0.976)        \\
    \hline
  \end{tabular}
\end{table}

Finally, during the last step, we use the model trained on PanNuke data for epithelial cells in MoNuSAC and the model trained on MoNuSAC for the inflammatory cells class in PanNuke. Specifically, we use classifier models to relabel epithelial cells in MoNuSAC and inflammatory cells in PanNuke data. Then we combine cells with refined labels and the rest of the cells in both datasets to create a refined dataset (\hyperref[fig:S2]{Appendix Figure S2 (2)}). The process of relabeling cells and visualizing them on a patch is shown in \hyperref[fig:fig4]{Figure 4}. The cell counts in the refined dataset are provided in \hyperref[tab:S4]{Appendix Table S4}.

\begin{figure}[h!]
    \centering
    \includegraphics[width=\textwidth, height=0.42\textheight, keepaspectratio]{images/Figure_4.pdf}
    \caption{Cell relabeling procedure for epithelial and inflammatory cell classes}
    \label{fig:fig4}
\end{figure}

%\hfill

Relabeling and combining datasets have been explored in a prior study \cite{Parulekar_Kanwat_etal._2023}, where consecutive fine-tuning on multiple datasets was employed to account for hierarchical class label structures. While the method presented in \cite{Parulekar_Kanwat_etal._2023} is intuitive, it often lacks consistency and requires multiple fine-tuning runs, which can be cumbersome and time-consuming. 
In contrast, cross-relabeling simplifies this process by using specialized classification models tailored to each dataset's specific labeling challenges. This approach provides better transparency and produces a unified dataset encompassing seven distinct cell types across multiple tissue samples, enhancing data diversity for further model training or fine-tuning.

Despite these improvements, cross-relabeling does not entirely resolve issues related to poor labeling quality or the amount of labeled data. Specifically, our results show lower accuracies persist for underrepresented classes, such as macrophages, which may stem from a limited sample availability and intrinsic challenges in distinguishing these cells based solely on H\&E staining. Furthermore, while our method enhances label specificity, it relies on the initial quality of the broad labels; thus, any fundamental inaccuracies in the original annotations can propagate through the relabeling process. Addressing the overall problem of limited data labels may require integrating additional data sources or utilizing complementary immunohistochemical staining methods.
Although the reported performance metrics are obtained from evaluations on the native test sets of each dataset, it is important to note that the primary application of these classifiers is to perform cross-relabeling, where a model trained on one dataset (e.g., PanNuke) is applied to another (e.g., MoNuSAC) and vice versa. We acknowledge that a more systematic evaluation of cross-dataset generalization is needed and could be performed in future work.

Overall, the refined dataset produced by our approach can enhance the supervised training or fine-tuning of cell segmentation and classification models, especially those that utilize pre-trained foundation models to improve feature extraction robustness. In addition, these models can detect nuanced classes that enable researchers to conduct more detailed analyses of biological processes in computational pathology.

\section{Foundation models for robust cell segmentation and classification}

Accurate cell segmentation and classification in digital pathology are hindered by limited labeled data and the fact that conventional CNNs are unable to capture global contextual information due to their local receptive field constraints \cite{Gheflati_Rivaz_2022,Yang_Marcus_etal.}. Traditional approaches in cell quantification have predominantly relied on CNN encoders, such as ResNet50, given their proven effectiveness in semantic segmentation tasks \cite{Deshmane_2023,Graham_Vu_etal._2019,Mukasheva_Koishiyeva_etal._2024,Stringer_Wang_etal._2021}. However, approaches that include fine-tuning of pretrained CNNs, data augmentation, and stain normalization to partially increase data variability and address staining differences often fail to achieve the necessary generalization and robustness across diverse tissue types and staining conditions \cite{G._Wang_W._Li_etal._2018,Gao_Bagci_etal._2018,Karim_El_Khoury_Martin_Fockedey_etal._2021}.

To overcome these challenges, we leverage an encoder-decoder network that uses a foundation model as the encoder and a CNN upsampling decoder (\hyperref[fig:fig5]{Figure 5}) for simultaneous cell segmentation and classification in 2D patches extracted from WSIs. Foundation models with transformer-based architectures are viable alternatives to CNN-based encoders \cite{Shamshad_Khan_etal._2023,Sourget_2023}. They enable the creation of more advanced architectures that can decode or transform learned features more effectively \cite{Chen_Duan_etal._2023,Cheng_Misra_etal._2022,Xie_Wang_etal._2021}.

\begin{figure}[h!]
    \centering
    \includegraphics[width=\textwidth]{images/Figure_5.pdf}
    \caption{UNETR-like model with foundational model as backbone}
    \label{fig:fig5}
\end{figure}

By utilizing a transformer-based encoder, we incorporate global contextual information into the feature extraction process, which is a key advantage of such architectures \cite{Chen_Lu_etal._2021}. This foundation model integration facilitates accurate pixel-wise segmentation and classification without the need for extensive encoder training, thereby potentially improving generalization across varied cellular structures and tissue types.
In our implementation, we employ a modified UNETR \cite{Hatamizadeh_Tang_etal._2021} architecture that combines a vision transformer (ViT) \cite{Dosovitskiy_Beyer_etal._2021} encoder with a CNN-based decoder. The encoder utilizes the pretrained H-Optimus foundation model, which contains 1.1 billion parameters and is trained on over 500,000 H\&E stained WSIs \cite{Saillard_Jenatton_etal._2024}. We extract outputs from four evenly spaced transformer blocks $Z_i$, where $i \in [1, 14, 26, 38]$, to serve as residual connections for the CNN decoder. We select these blocks based on our observation that features from non-adjacent levels of the encoder lead to better overall performance on the test subset.

The CNN decoder upsamples the feature representations, acquired from the transformer blocks, to generate an intermediate vector that is handled by two task-specific layers that generate cell segmentation and classification masks. The first task-specific layer is the ‘Cellpose head’,  which is used to delineate cell instances. The layer generates horizontal and vertical gradient maps to form vector fields that are refined through gradient tracking in a post-processing step using the Cellpose algorithm \cite{Stringer_Wang_etal._2021}, known for its efficacy in cell segmentation tasks and generalizability across multiple domains \cite{Pachitariu_Stringer_2022,Stringer_Pachitariu_2024}. The second task-specific layer is the "Cell type head", which assigns labels to individual pixels. In the post-processing step, we determine the output classification label of each segmented cell instance by majority voting over the labeled pixels that comprise the cell in the segmentation map.

To evaluate model performance and measure the impact of adding a foundation model as backbone, we compare it to a ResNet50-based model. ResNet50 is a widely used solution for encoders in segmentation architectures in the medical domain \cite{Deshmane_2023,Graham_Vu_etal._2019,Mukasheva_Koishiyeva_etal._2024,Stringer_Wang_etal._2021}. For the H-Optimus-based model, we utilize frozen weights for the encoder and only fine-tune the decoder to take advantage of the extensive pre-training of the foundation model. For the ResNet50-based model we start with ImageNet \cite{Deng_Dong_etal.} weights and train both encoder and decoder parts. Hyperparameters for the training step are set to be identical, where possible, for comparable evaluation. 
For this evaluation, we deliberately use the PanNuke dataset to provide a standardized and controlled comparison between the H‑Optimus and ResNet50-based models (\hyperref[fig:S2]{Appendix Figure S2 (3)}). Specifically, we use two of the default PanNuke dataset splits (66\%) for training and validation, and reserve the third split (33\%) for testing.

To address the challenge of cell class imbalance in the PanNuke dataset, which is a common characteristic in most cell-level H\&E patch datasets, both models’ training processes employ a weighted loss function comprising cross-entropy and focal loss \cite{Lin_Goyal_etal._2018}. The focal loss component is adjusted with coefficients derived from each cell class' instance frequency, emphasizing learning from underrepresented classes and enhancing the model's sensitivity to rare but significant cellular patterns. The cross-entropy loss is augmented with spectral decoupling regularization \cite{Pezeshki_Kaba_etal._2021,Pohjonen_Stürenberg_etal._2022} and spatially varying label smoothing \cite{Islam_Glocker_2021}, which potentially stabilizes training and improves generalization in case of complex tissue morphologies. For optimization, we employ the \textit{AdamW} \cite{Loshchilov_Hutter_2019} to counter unbalanced class scenarios, with cosine annealing learning rate scheduler.

We utilize the scikit-learn library \cite{Van_der_Walt_Schönberger_etal._2014} and HoVer-Net \cite{Graham_Vu_etal._2019} implementations of $R^2$ (the coefficient of determination) and $PQ$ (panoptic quality) to evaluate our experiments. Complete mathematical formulations and detailed explanations of these metrics are provided in \hyperref[chap:S5]{Appendix S5}. To compute confidence intervals, we use nonparametric bootstrapping, where after calculating the metric on the full sample, we generated 1000 bootstrap replicates by resampling with replacement and then determined the 95\% confidence intervals as the 2.5th and 97.5th percentiles of the resulting empirical distribution.

%\hfill

The model comparisons are summarized in \hyperref[tab:2]{Table 2}. The H‑Optimus-based model achieves higher $R^2$ across all cell classes compared to the ResNet50-based model, which means that its predictions are more closely aligned with the PanNuke cell counts, indicating a stronger correlation with the observed data. Notably, the improvement of $R^2_{dead}$ may be an indicator of better global contextual representations provided by the foundation model backbone. In terms of segmentation and classification quality combined, measured by the PQ score, the H‑Optimus-based model demonstrates notable improvements across most cell classes. Overall, the average $R^2$ improved from 0.575 to 0.871, while the average $PQ$ score improved from 0.450 to 0.492, demonstrating better performance of the H-Optimus-based model.

\begin{table}[h!]
\renewcommand{\arraystretch}{1.5}
  \centering
  \caption{Cell quantification metrics for baseline and proposed models (CI 95\%).}
  \label{tab:2}
  \begin{tabular}{|l|c|c|}
    \hline
    %\rowcolor{gray!30}
    Metric             & Resnet50-based            & H-optimus-based              \\
    \hline
    $R^2_{neoplastic}$    & 0.681 (0.576--0.769)       & \textbf{0.941 (0.917--0.960)} \\
    \hline
    $R^2_{inflammatory}$  & 0.863 (0.778--0.903)       & \textbf{0.949 (0.918--0.966)} \\
    \hline
    $R^2_{connective}$    & 0.600 (0.488--0.698)       & 0.609 (0.436--0.772)          \\
    \hline
    $R^2_{dead}$          & 0.097 (-11.389--0.669)     & 0.925 (0.404--0.982)          \\
    \hline
    $R^2_{epithelial}$    & 0.635 (0.490--0.747)       & \textbf{0.930 (0.886--0.964)} \\
    \hline
    $PQ_{neoplastic}$       & 0.517 (0.499--0.535)       & \textbf{0.589 (0.575--0.604)} \\
    \hline
    $PQ_{inflammatory}$     & 0.455 (0.429--0.482)       & \textbf{0.528 (0.507--0.549)} \\
    \hline
    $PQ_{connective}$       & 0.416 (0.400--0.431)       & \textbf{0.451 (0.436--0.465)} \\
    \hline
    $PQ_{dead}$             & 0.374 (0.342--0.408)       & 0.292 (0.209--0.365)          \\
    \hline
    $PQ_{epithelial}$       & 0.488 (0.460--0.519)       & \textbf{0.599 (0.579--0.618)} \\
    \hline
  \end{tabular}
\end{table}

Our results  show that integrating the H‑Optimus foundation model within the UNETR architecture enhances the model's ability to segment and classify cells across diverse tissues from PanNuke data. The pretrained transformer encoder provides robust feature representations, resulting in higher average $R^2$ and $PQ$ scores compared to the CNN-based model. This leads to more reliable cell quantification and more accurate downstream analysis. Additionally, the streamlined fine-tuning process reduces computational overhead and training time, making the model more adaptable for new data.

Despite these advancements, the foundation model-based approach does not fully resolve all challenges related to cell segmentation and classification. We observe lower metric scores for underrepresented classes in the training data. Furthermore, foundation models typically encompass billions of parameters, resulting in substantial computational and memory requirements. It therefore poses challenges for deployment in resource-constrained environments, limiting their practical applicability in certain clinical settings.

\section{Model optimization via Knowledge Distillation}

To address the limitations posed by the extensive size of foundation models, we implement knowledge distillation — a model compression technique that leverages the teacher-student paradigm \cite{Hinton_Vinyals_etal._2015}. By training a smaller, more efficient student model to replicate the output of a larger, pre-trained teacher model, we retain performance while significantly reducing the model's complexity and resource requirements (\hyperref[fig:fig6]{Figure 6}).

\begin{figure}[h!]
    \centering
    \includegraphics[width=\textwidth, height=0.45\textheight, keepaspectratio]{images/Figure_6.pdf}
    \caption{Knowledge distillation framework for training a student model using a pre-trained teacher}
    \label{fig:fig6}
\end{figure}

We employ knowledge distillation to compress the H‑Optimus-based teacher model into a more efficient student model. The teacher model is the modified UNETR architecture with the H‑Optimus foundation model described in the previous chapter. The student model is based on a UNet architecture augmented with residual connections and incorporates a smaller ViT encoder with 9 million parameters \cite{Steiner_Kolesnikov_etal._2022,Wightman_2019}. 

First, we fine-tune the teacher model using the refined dataset from the cross-relabeling procedure (Section 2). Initially we train the decoder of the teacher model while keeping the encoder weights frozen. We split the refined dataset into train (70\%), validation (20\%) and test (10\%) subsets (\hyperref[fig:S2]{Appendix Figure S2 (4)}). During fine-tuning, we use the train and validation subsets, while leaving the test subset for model evaluation. We set the training procedure and model hyperparameters to be identical to those that were used to demonstrate the utility of foundation models for the simultaneous cell segmentation and classification task.

Next, we perform knowledge distillation from teacher to student using the refined dataset used to fine-tune the teacher model. The student model is trained to replicate the teacher model's outputs. We utilize a specialized loss function that aligns the student's predicted probability distribution with the teacher's, incorporating the teacher's class probability distribution derived from the output. Following the methodology of Hinton et al. \cite{Hinton_Vinyals_etal._2015}, we experiment with various hyperparameter settings for the temperature ($T$) and the balancing coefficients ($\alpha$ and $\beta$) in the loss function. We vary $T$ from 1 to 20 and adjust $\alpha$ and $\beta$ to balance the distillation and student losses. Through iterative tuning and evaluation, we identify that setting $T=14$, $\alpha=0.3$, and $\beta=0.7$ yields a configuration that converges and closely approximates the teacher model's performance during training.

Finally, we assess the performance of both models using the $R^2$ and $PQ$ (defined in \hyperref[chap:S5]{Appendix S5}) on the test set of the refined dataset (\hyperref[tab:3]{Table 3}). We observe that the 95\% confidence intervals overlap for most cell types, so we cannot claim statistically significant performance differences between the teacher and student models. One exception appears in the neoplastic class. The teacher model produces an $R^2$ of 0.919, while the student model shows an $R^2$ of 0.852. In addition, the student model achieves higher $PQ$ values for the neoplastic and connective classes, though the confidence intervals show overlap.

\begin{table}[h!]
\renewcommand{\arraystretch}{1.5}
  \centering
  \caption{Cell quantification metrics for teacher and distilled student models (CI 95\%).}
  \label{tab:3}
  \begin{tabular}{|l|c|c|}
    \hline
    %\rowcolor{gray!30}
    Metric & Teacher & Student \\
    \hline
    $R^2_{neoplastic}$    & \textbf{0.919} (0.898--0.939) & 0.852 (0.800--0.891) \\
    \hline
    $R^2_{lymphocyte}$    & 0.969 (0.956--0.977)         & 0.969 (0.956--0.978) \\
    \hline
    $R^2_{connective}$    & 0.694 (0.548--0.809)         & 0.618 (0.469--0.741) \\
    \hline
    $R^2_{dead}$          & 0.755 (0.400--0.908)         & 0.424 (0.100--0.731) \\
    \hline
    $R^2_{epithelial}$    & 0.922 (0.870--0.958)         & 0.843 (0.738--0.917) \\
    \hline
    $R^2_{macrophage}$    & 0.384 (-0.369--0.724)        & 0.704 (0.352--0.859) \\
    \hline
    $R^2_{neutrofil}$     & 0.854 (0.578--0.929)         & 0.833 (0.502--0.925) \\
    \hline
    $PQ_{neoplastic}$       & 0.581 (0.569--0.593)         & 0.601 (0.588--0.613) \\
    \hline
    $PQ_{lymphocyte}$       & 0.536 (0.520--0.553)         & 0.563 (0.544--0.579) \\
    \hline
    $PQ_{connective}$       & 0.436 (0.421--0.451)         & 0.457 (0.441--0.474) \\
    \hline
    $PQ_{dead}$             & 0.272 (0.235--0.315)         & 0.279 (0.201--0.369) \\
    \hline
    $PQ_{epithelial}$       & 0.522 (0.500--0.545)         & 0.530 (0.506--0.555) \\
    \hline
    $PQ_{macrophage}$       & 0.524 (0.459--0.588)         & 0.474 (0.405--0.543) \\
    \hline
    $PQ_{neutrofil}$        & 0.541 (0.490--0.592)         & 0.565 (0.522--0.607) \\
    \hline
  \end{tabular}
\end{table}


We further decompose the $PQ$ metric into its $SQ$ and $DQ$ components (\hyperref[tab:S6]{Appendix Table S6}). Both models produce nearly identical $SQ$ values, which indicates that they predict instance boundaries with similar precision. Although the student model shows some improvement in $DQ$ scores for certain classes, the confidence intervals overlap and do not confirm a statistically significant difference.

We observe that the student and teacher models yield comparable detection performance despite the student model using a much smaller and simpler architecture. A model with fewer parameters reduces the risk of overfitting when training data are scarce relative to the model’s complexity \cite{Farias_Ludermir_etal._2022}. The knowledge distillation process also encourages the student model to focus on the most generalizable detection features learned from the teacher. These factors enable the student model to achieve similar detection performance across different cell types.

Additionally, considering the model sizes reported in \hyperref[tab:4]{Table 4}, the distilled model achieves a significant reduction compared to the teacher model, with a 48-fold decrease in parameter count and a 5.5-fold reduction in on-disk size. In inference mode, the teacher model requires 16 GB of VRAM for a batch size of 32, while the distilled model only needs 3 GB of VRAM for the same batch size. These reductions make the distilled model significantly more practical for fine-tuning and deployment in resource-constrained environments.

\begin{table}[h!]
\renewcommand{\arraystretch}{1.5}
  \centering
  \caption{Parameter counts and size of teacher and distilled model}
  \label{tab:4}
  \adjustbox{max width=\textwidth}{%
  \begin{tabular}{|l|c|c|c|}
    \hline
    %\rowcolor{gray!30}
    Metric & H-optimus-based (Teacher) & mobileViT-based (Student) & Magnitude of difference \\
    \hline
    Parameters count       & 1,158,917,906   & \textbf{24,093,393}   & \textbf{48x}  \\
    \hline
    Estimated Total Size (MB) & 87,912       & \textbf{15,935}    & \textbf{5.5x} \\
    \hline
  \end{tabular}%
}
\end{table}

%\hfill

With recent advancements in complex network architectures and the use of pretrained encoders to achieve state-of-the-art performance \cite{Baumann_Dislich_etal._2024,Hörst_Rempe_etal._2024} in cell segmentation and classification tasks, model size, computational complexity, and processing times have increased. This limits the scalability and accessibility of these models. As we demonstrate, this may be mitigated using knowledge distillation. Studies in the field of natural language processing have demonstrated the efficacy of knowledge distillation in retaining the capabilities of the teacher model while achieving significant reductions in size and complexity \cite{Huangpu_Gao_2024,Sun_Yu_etal.}. 

We demonstrate the feasibility of knowledge distillation in digital pathology, specifically for cell segmentation and classification tasks. Moreover, we achieve this performance while also significantly reducing the parameter count. In addressing the challenge of knowledge transfer, we found that distillation from a transformer-based model to a smaller transformer is more straightforward than attempting to map transformer features to CNN blocks. In our experiments, using a CNN-based network as a student results in worse cell quantification performance due to the structural constraints of CNN feature space dimensions. 

Although our primary approach relies on a transformer-based student model that performs well, it can be further optimized to incorporate advantages from CNN architectures. For example, employing alternative techniques such as using ViT adapters \cite{Chen_Duan_etal._2023} or $1 \times 1$ convolutions to adjust feature map sizes may be beneficial for harnessing CNN advantages like enhanced local feature extraction. Moreover, if additional performance improvements are desired, the process can be further enhanced by applying supplementary knowledge distillation techniques, such as self-distillation \cite{Zhang_Song_etal._2019} or online distillation \cite{Houyon_Cioppa_etal._2023}.

Despite these promising results, further validation on independent datasets is necessary to fully understand the model's limitations. Underrepresented classes may pose challenges when addressing complex cases. Pathologists need to validate these models to adopt them in clinical settings. While the distilled models are smaller and more deployable, a technological gap persists because pathologists traditionally rely on established methods for inspecting WSIs and diagnosing diseases. Addressing the complexities involved in deploying models for inference and supporting pathologists in adopting new tools is essential for integrating these models into clinical workflows.

\section{Model integration with QuPath}
Digital pathology tools with graphical user interfaces are essential for visualizing and analyzing WSIs. To make our student model useful in clinical pathology workflows, it needs to be integrated into a tool that enables inspecting regions, creating annotations, and providing quantitative analyses of biomarkers. Therefore, we integrate the trained student model from the previous chapter into the QuPath open‑source platform \cite{Bankhead_Loughrey_etal._2017}. QuPath provides the required annotation, visualization, and analysis tools to interpret complex histological data, including workflows for cell segmentation, classification, and quantification (\hyperref[fig:fig7]{Figure 7}). 

\begin{figure}[h!]
    \centering
    \includegraphics[width=\textwidth]{images/Figure_7.pdf}
    \caption{Visualization of model-generated cell quantification annotations (left) and the corresponding unannotated slide (right) in QuPath}
    \label{fig:fig7}
\end{figure}

To identify the regions in a WSI critical for prognosticating tumor development, such as specific tumor areas or border regions without overlapping healthy tissue, the pathologist uses QuPath to outline these regions. Then, the pathologist initiates a cell segmentation and classification script through the QuPath interface for the selected regions. The resulting annotations and quantified cell information are then directly overlaid onto the WSI in the QuPath interface. Additional design and implementation details are in \hyperref[chap:S7]{Appendix S7}. 

Two common approaches for integrating deep learning models into QuPath are Java‑based native QuPath extensions \cite{Goldsborough_Philps_etal._2024} and the execution of RESTful API requests to a model server coupled with handling the response via an extension, as demonstrated in the application of cell segmentation models applied to immunofluorescence images \cite{Sugawara_2023}. While the community is actively working on these integration strategies, there is currently no universal solution that fully addresses all integration and performance requirements.

Extensions may offer better integration with QuPath, allowing slightly improved performance and more widespread usage of the built-in QuPath models, but they lack the flexibility to customize models and modify their behavior. For example, the newest version of QuPath includes models such as StarDist \cite{Weigert_Schmidt} and InstanSeg \cite{Goldsborough_Philps_etal._2024} that can perform cell segmentation. Both models pose limitations when applied to simultaneous cell segmentation and classification. StarDist performs well only on convex, round shapes by design, whereas some neoplastic, inflammatory, and connective cells exhibit complex and non-convex shapes. InstanSeg provides only semantic segmentation without assigning classes to the segmented cells.

%\hfill

In contrast, our approach offers an alternative integration strategy. It utilizes the paquo library to directly interact with QuPath’s internal application programming interface from within Python. This enables data exchange and processing without the need for intermediate conversion steps and provides greater control over model customization, retraining, and the incorporation of custom processing steps.

The integration of our custom model with QuPath underscores its potential to significantly enhance the diagnostic process by reducing the time burden on pathologists and enabling them to focus on more complex interpretative tasks using familiar software. Leveraging a tool that is already well-established among pathologists increases the likelihood of its adoption into daily clinical workflows. The quantitative data generated through the automated workflow is critical for both clinical decision-making and research, facilitating more accurate biomarker analysis, enabling robust statistical evaluations, and supporting hypothesis generation and testing. Additionally, by streamlining cell segmentation and classification, the tool enhances the scalability and reproducibility of pathological assessments, ultimately contributing to improved diagnostic accuracy and patient outcomes.

\section{Conclusion and future work}

In this study, we address critical challenges in digital pathology and tackle the usability and deployment issues of the developed models in standard computing environments without the need for high-performance computing systems. Our multi-faceted approach encompasses data refinement through cross-relabeling, leveraging foundation models for robust cell segmentation and classification, optimizing model performance via knowledge distillation, and integrating the optimized model into the QuPath software for practical application. This approach is used to construct a capable, versatile, and adjustable model for cell segmentation and classification, with enhanced performance and usability.

\begin{sloppypar}
While our approach shows potential in the field of computational pathology, certain limitations persist. 
For example, our implementation currently exhibits lower performance in detecting macrophages. 
This serves as an instance of the broader challenge of accurately identifying complex cell types. In order to address this issue, extending our approach to incorporate additional data sources, exploring alternative modeling approaches, and integrating other imaging modalities such as immunohistochemical staining may help improve detection accuracy. Moreover, although the distilled model reduces computational demands, integrating advanced deep learning models into clinical practice requires addressing technological gaps and potential resistance to adopting new tools within established diagnostic processes.
\end{sloppypar}

Future work could focus on several key areas to refine the proposed approach and facilitate its adoption in clinical environments. Enhancing the cell-relabeling process with additional datasets \cite{Graham_Jahanifar_etal._2021} could improve the representation of underrepresented cell types and enhance overall model performance. Also, incorporating additional data sources, such as multi-modal imaging or complementary staining methods, may address limitations related to cell type differentiation and class imbalance. Exploring other foundation models \cite{Vorontsov_Bozkurt_etal._2024,Zimmermann_Vorontsov_etal._2024} or introducing additional modalities \cite{Ding_Wagner_etal._2024,Vaidya_Zhang_etal._2025} may provide alternative architectures better suited to specific tasks or offer improved efficiency. Implementing more complex knowledge distillation techniques \cite{Houyon_Cioppa_etal._2023,Zhang_Song_etal._2019} could further optimize the model's performance and adaptability. Additionally, deeper integration with QuPath or other digital pathology software could provide pathologists more control over cell quantification analysis directly within the QuPath interface, thereby increasing accessibility and usability. Such enhancements would not only refine model performance but also ensure greater adaptability and scalability within various clinical environments. Finally, extensive validation of the model by pathologists and benchmarking against independent datasets are essential steps toward establishing the model's reliability and fostering confidence in its clinical utility.

\section*{Acknowledgments} 
This work was funded in part by the Research Council of Norway grant no. 309439 SFI Visual Intelligence, and the North Norwegian Health Authority grant no. HNF1521-20.

\bibliographystyle{IEEEtran}
\begin{sloppypar}
\begin{thebibliography}{99}

\bibitem{chaplot2020neural} Chaplot, Devendra Singh, et al. "Neural topological slam for visual navigation." Proceedings of the IEEE/CVF conference on computer vision and pattern recognition. 2020.

\bibitem{maksymets2021thda} Maksymets, Oleksandr, et al. "Thda: Treasure hunt data augmentation for semantic navigation." Proceedings of the IEEE/CVF International Conference on Computer Vision. 2021.

\bibitem{mezghan2022memory} Mezghan, Lina, et al. "Memory-augmented reinforcement learning for image-goal navigation." 2022 IEEE/RSJ International Conference on Intelligent Robots and Systems (IROS). IEEE, 2022.

\bibitem{al2022zero} Al-Halah, Ziad, Santhosh Kumar Ramakrishnan, and Kristen Grauman. "Zero experience required: Plug \& play modular transfer learning for semantic visual navigation." Proceedings of the IEEE/CVF Conference on Computer Vision and Pattern Recognition. 2022.

\bibitem{ye2021auxiliary} Ye, Joel, et al. "Auxiliary tasks and exploration enable objectgoal navigation." Proceedings of the IEEE/CVF international conference on computer vision. 2021.

\bibitem{chaplot2020object} Chaplot, Devendra Singh, et al. "Object goal navigation using goal-oriented semantic exploration." Advances in Neural Information Processing Systems 33 (2020)

\bibitem{ramakrishnan2022poni} Ramakrishnan, Santhosh Kumar, et al. "Poni: Potential functions for objectgoal navigation with interaction-free learning." Proceedings of the IEEE/CVF Conference on Computer Vision and Pattern Recognition. 2022.

\bibitem{ramrakhya2022habitat} Ramrakhya, Ram, et al. "Habitat-web: Learning embodied object-search strategies from human demonstrations at scale." Proceedings of the IEEE/CVF Conference on Computer Vision and Pattern Recognition. 2022.

\bibitem{mousavian2019visual} Mousavian, Arsalan, et al. "Visual representations for semantic target driven navigation." 2019 International Conference on Robotics and Automation (ICRA). IEEE, 2019.

\bibitem{dhariwal2021diffusion} Dhariwal, Prafulla, and Alexander Nichol. "Diffusion models beat gans on image synthesis." Advances in neural information processing systems 34 (2021)

\bibitem{ho2022classifier} Ho, Jonathan, and Tim Salimans. "Classifier-free diffusion guidance." arXiv preprint arXiv:2207.12598 (2022).

\bibitem{nichol2021glide} Nichol, Alex, et al. "Glide: Towards photorealistic image generation and editing with text-guided diffusion models." arXiv preprint arXiv:2112.10741 (2021)

\bibitem{brooks2023instructpix2pix} Brooks, Tim, Aleksander Holynski, and Alexei A. Efros. "Instructpix2pix: Learning to follow image editing instructions." Proceedings of the IEEE/CVF Conference on Computer Vision and Pattern Recognition. 2023.

\bibitem{fu2023guiding} Fu, Tsu-Jui, et al. "Guiding instruction-based image editing via multimodal large language models." arXiv preprint arXiv:2309.17102 (2023).

\bibitem{geng2024instructdiffusion} Geng, Zigang, et al. "Instructdiffusion: A generalist modeling interface for vision tasks." Proceedings of the IEEE/CVF Conference on Computer Vision and Pattern Recognition. 2024.

\bibitem{zhou2024minedreamer} Zhou, Enshen, et al. "Minedreamer: Learning to follow instructions via chain-of-imagination for simulated-world control." arXiv preprint arXiv:2403.12037 (2024).

\bibitem{zhou2023esc} Zhou, Kaiwen, et al. "Esc: Exploration with soft commonsense constraints for zero-shot object navigation." International Conference on Machine Learning. PMLR, 2023.

\bibitem{yu2023l3mvn} Yu, Bangguo, Hamidreza Kasaei, and Ming Cao. "L3mvn: Leveraging large language models for visual target navigation." 2023 IEEE/RSJ International Conference on Intelligent Robots and Systems (IROS). IEEE, 2023.

\bibitem{gadre2023cows} Gadre, Samir Yitzhak, et al. "Cows on pasture: Baselines and benchmarks for language-driven zero-shot object navigation." Proceedings of the IEEE/CVF Conference on Computer Vision and Pattern Recognition. 2023.

\bibitem{shah2023navigation} Shah, Dhruv, et al. "Navigation with large language models: Semantic guesswork as a heuristic for planning." Conference on Robot Learning. PMLR, 2023.

\bibitem{cai2024bridging} Cai, Wenzhe, et al. "Bridging zero-shot object navigation and foundation models through pixel-guided navigation skill." 2024 IEEE International Conference on Robotics and Automation (ICRA). IEEE, 2024.

\bibitem{yu2023co} Yu, Bangguo, Hamidreza Kasaei, and Ming Cao. "Co-NavGPT: Multi-robot cooperative visual semantic navigation using large language models." arXiv preprint arXiv:2310.07937 (2023).

\bibitem{wu2024voronav} Wu, Pengying, et al. "Voronav: Voronoi-based zero-shot object navigation with large language model." arXiv preprint arXiv:2401.02695 (2024).

\bibitem{qin2023mp5} Qin, Yiran, et al. "Mp5: A multi-modal open-ended embodied system in minecraft via active perception." arXiv preprint arXiv:2312.07472 (2023).

\bibitem{du2024learning} Du, Yilun, et al. "Learning universal policies via text-guided video generation." Advances in Neural Information Processing Systems 36 (2024).

\bibitem{ajay2024compositional} Ajay, Anurag, et al. "Compositional foundation models for hierarchical planning." Advances in Neural Information Processing Systems 36 (2024).

\bibitem{liang2024skilldiffuser} Liang, Zhixuan, et al. "Skilldiffuser: Interpretable hierarchical planning via skill abstractions in diffusion-based task execution." Proceedings of the IEEE/CVF Conference on Computer Vision and Pattern Recognition. 2024.

\bibitem{heusel2017gans} Heusel, Martin, et al. "Gans trained by a two time-scale update rule converge to a local nash equilibrium." Advances in neural information processing systems 30 (2017).

\bibitem{zhang2018unreasonable} Zhang, Richard, et al. "The unreasonable effectiveness of deep features as a perceptual metric." Proceedings of the IEEE conference on computer vision and pattern recognition. 2018.

\bibitem{brown2020language} Brown, Tom B. "Language models are few-shot learners." arXiv preprint arXiv:2005.14165 (2020).

\bibitem{podell2023sdxl} Podell, Dustin, et al. "Sdxl: Improving latent diffusion models for high-resolution image synthesis." arXiv preprint arXiv:2307.01952 (2023).

\bibitem{brohan2022rt} Brohan, Anthony, et al. "Rt-1: Robotics transformer for real-world control at scale." arXiv preprint arXiv:2212.06817 (2022).

\bibitem{brohan2023rt} Brohan, Anthony, et al. "Rt-2: Vision-language-action models transfer web knowledge to robotic control." arXiv preprint arXiv:2307.15818 (2023).

\bibitem{li2024manipllm} Li, Xiaoqi, et al. "Manipllm: Embodied multimodal large language model for object-centric robotic manipulation." Proceedings of the IEEE/CVF Conference on Computer Vision and Pattern Recognition. 2024.

\bibitem{shah2023vint} Shah, Dhruv, et al. "ViNT: A foundation model for visual navigation." arXiv preprint arXiv:2306.14846 (2023).

\bibitem{liu2024visual} Liu, Haotian, et al. "Visual instruction tuning." Advances in neural information processing systems 36 (2024).

\bibitem{hu2021lora} Hu, Edward J., et al. "Lora: Low-rank adaptation of large language models." arXiv preprint arXiv:2106.09685 (2021).

\bibitem{qin2023supfusion} Qin, Yiran, et al. "SupFusion: Supervised LiDAR-camera fusion for 3D object detection." Proceedings of the IEEE/CVF International Conference on Computer Vision. 2023.

\bibitem{qin2024worldsimbench} Qin, Yiran, et al. "Worldsimbench: Towards video generation models as world simulators." arXiv preprint arXiv:2410.18072 (2024).

\bibitem{yu2025gamefactory} Yu, Jiwen, et al. "GameFactory: Creating New Games with Generative Interactive Videos." arXiv preprint arXiv:2501.08325 (2025).

\bibitem{zhou2024code} Zhou, Enshen, et al. "Code-as-Monitor: Constraint-aware Visual Programming for Reactive and Proactive Robotic Failure Detection." arXiv preprint arXiv:2412.04455 (2024).

\bibitem{zhang2024ad} Zhang, Zaibin, et al. "AD-H: Autonomous Driving with Hierarchical Agents." arXiv preprint arXiv:2406.03474 (2024).

\bibitem{wang2024toward} Wang, Chaoqun, et al. "Toward Accurate Camera-based 3D Object Detection via Cascade Depth Estimation and Calibration." arXiv preprint arXiv:2402.04883 (2024).

\bibitem{huang2024story3d} Huang, Yuzhou, et al. "Story3d-agent: Exploring 3d storytelling visualization with large language models." arXiv preprint arXiv:2408.11801 (2024).

\bibitem{savinov2018semi} Savinov, Nikolay, Alexey Dosovitskiy, and Vladlen Koltun. "Semi-parametric topological memory for navigation." arXiv preprint arXiv:1803.00653 (2018).

\bibitem{majumdar2022zson} Majumdar, Arjun, et al. "Zson: Zero-shot object-goal navigation using multimodal goal embeddings." Advances in Neural Information Processing Systems 35 (2022): 32340-32352.

\bibitem{yadav2023offline} Yadav, Karmesh, et al. "Offline visual representation learning for embodied navigation." Workshop on Reincarnating Reinforcement Learning at ICLR 2023. 2023.

\bibitem{yadav2023ovrl} Yadav, Karmesh, et al. "Ovrl-v2: A simple state-of-art baseline for imagenav and objectnav." arXiv preprint arXiv:2303.07798 (2023).

\bibitem{sun2024fgprompt} Sun, Xinyu, et al. "FGPrompt: fine-grained goal prompting for image-goal navigation." Advances in Neural Information Processing Systems 36 (2024).

\bibitem{zhu2017target} Zhu, Yuke, et al. "Target-driven visual navigation in indoor scenes using deep reinforcement learning." 2017 IEEE international conference on robotics and automation (ICRA). IEEE, 2017.

\bibitem{koh2024generating} Koh, Jing Yu, Daniel Fried, and Russ R. Salakhutdinov. "Generating images with multimodal language models." Advances in Neural Information Processing Systems 36 (2024).

\bibitem{krantz2022instance} Krantz, Jacob, et al. "Instance-specific image goal navigation: Training embodied agents to find object instances." arXiv preprint arXiv:2211.15876 (2022).

\bibitem{schulman2017proximal} Schulman, John, et al. "Proximal policy optimization algorithms." arXiv preprint arXiv:1707.06347 (2017).

\bibitem{anderson2018evaluation} Anderson, Peter, et al. "On evaluation of embodied navigation agents." arXiv preprint arXiv:1807.06757 (2018).

\bibitem{lin2024navcot} Lin, Bingqian, et al. "NavCoT: Boosting LLM-Based Vision-and-Language Navigation via Learning Disentangled Reasoning." arXiv preprint arXiv:2403.07376 (2024).

\bibitem{NavGPT} Zhou, Gengze, Yicong Hong, and Qi Wu. "Navgpt: Explicit reasoning in vision-and-language navigation with large language models." Proceedings of the AAAI Conference on Artificial Intelligence.

\bibitem{hahn2021no} Hahn, Meera, et al. "No rl, no simulation: Learning to navigate without navigating." Advances in Neural Information Processing Systems 34 (2021): 26661-26673.

\bibitem{li2025t2isafety} Li, Lijun, et al. "T2ISafety: Benchmark for Assessing Fairness, Toxicity, and Privacy in Image Generation." arXiv preprint arXiv:2501.12612 (2025).

\bibitem{an2024agfsync} An, Jingkun, et al. "AGFSync: Leveraging AI-Generated Feedback for Preference Optimization in Text-to-Image Generation." arXiv preprint arXiv:2403.13352 (2024).


\end{thebibliography}
\end{sloppypar}

\clearpage
\beginsupplement
\section*{Appendix}
\renewcommand{\thesubsection}{S\arabic{subsection}}

\subsection{\label{chap:S1}PanNuke and MoNuSAC preprocessing}
The PanNuke dataset comprises a set of 7,901 RGB patches, each with dimensions of $256 \times 256$ pixels, which we set as the standard patch size for our analysis. In contrast, the MoNuSAC dataset encompasses 294 images of heterogeneous dimensions. To standardize the MoNuSAC images with our experiments, we implement a standardization protocol. Specifically, for images exceeding the dimensions of $256 \times 256$ pixels, we segment them into equal-sized patches and apply mirror padding to the remaining portions to avoid information loss at the peripherals. Patches with dimensions less than $128 \times 128$ pixels are excluded from the dataset due to the insufficient resolution to capture relevant cellular details. For patches where either dimension falls between 128 and 256 pixels, we employ upsampling to achieve the standard patch size. As a result, we obtain a total of 2,823 RGB patches derived from the MoNuSAC dataset for subsequent analysis. For additional details on the MoNuSAC data preparation process, refer to the source code \cite{Shvetsov_2025a}.
\clearpage

\subsection{\label{chap:S2}Data usage for the methodology}

\counterwithin{figure}{subsection}
\renewcommand{\thefigure}{S\arabic{subsection}}

\begin{figure}[h!]
    \centering
    \includegraphics[width=\textwidth, height=0.85\textheight, keepaspectratio]{images/A2.pdf}
    \caption{Overview of the methodology for cross-labeling, dataset refinement, and model comparison. (1) Cross-relabeling - training and testing cell classification models, (2) Cross-relabeling - using cell classification models to create refined dataset, (3) Fine-tuning and training models for comparison, (4) Student knowledge distillation with refined dataset}
    \label{fig:S2}
\end{figure}
\clearpage

\subsection{\label{chap:S3}Confusion matrices for classification models}
\counterwithin{figure}{subsection}
\renewcommand{\thefigure}{S\arabic{subsection}.\arabic{figure}}

\begin{figure}[h!]
    \centering
    \includegraphics[width=\textwidth, height=0.4\textheight, keepaspectratio]{images/A3_1.pdf}
    \caption{Confusion matrix for PanNuke trained model}
    \label{fig:S3.1}
\end{figure}

\begin{figure}[h!]
    \centering
    \includegraphics[width=\textwidth, height=0.4\textheight, keepaspectratio]{images/A3_2.pdf}
    \caption{Confusion matrix for MoNuSAC trained model}
    \label{fig:S3.2}
\end{figure}

\clearpage

\subsection{\label{chap:S4}Datasets cell counts}

\counterwithin{table}{subsection}
\renewcommand{\thetable}{S\arabic{subsection}}

\begin{table}[h!]
\renewcommand{\arraystretch}{2.0}
\centering
\caption{\label{tab:S4}Cell counts for PanNuke, MoNuSAC and refined datasets. Numbers in parentheses indicate preprocessed cell counts for cell classifier models training and testing.}
%\adjustbox{max width=\textwidth}{%
\begin{tabular}{|l|c|c|c|}
\hline
%\rowcolor{gray!30}
Cell type & PanNuke & MoNuSAC & Refined \\
\hline
Neoplastic & 77,403 (68,031) & - & 105,451 \\
\hline
Epithelial & 26,572 (23,207) & - & 29,926 \\
\hline
Epithelial (benign and malignant) & - & 31,402 & - \\
\hline
Inflammatory & 32,276 & - & - \\
\hline
Lymphocytes & - & 37,045 (33,104) & 65,275 \\
\hline
Neutrophils & - & 1,355 (1,252) & 3,833 \\
\hline
Macrophage & - & 1,842 (1,695) & 3,410 \\
\hline
Dead & 2,908 & - & 2,908 \\
\hline
Connective & 50,585 & - & 50,585 \\
\hline
\end{tabular}
%
%}
\end{table}



\clearpage

\subsection{\label{chap:S5}Definition of validation metrics}
\counterwithin{equation}{subsection}
\renewcommand{\theequation}{\arabic{equation}}

\subsubsection{\label{chap:S5.1}R\textsuperscript{2}}
The coefficient of determination, denoted as $R^2$, is a statistical measure that represents the proportion of variance in the dependent variable that is predictable from the independent variables. In the context of cell quantification in pathology, $R^2$ is used to assess how well the predicted quantities of different cell types in a patch align with the actual quantities observed in the ground truth data, with higher values representing more accurate quantification. $R^2$ is defined as
\begin{equation*}
R^2 = 1 - \frac{\sum_{i=1}^n (y_i - \hat{y}_i)^2}{\sum_{i=1}^n (y_i - \bar{y})^2},
\end{equation*}
where $y_i$ represents the actual number of cells of a specific type in the $i$-th image, $\hat{y}_i$ represents the predicted number of cells of that type in the $i$-th image, $\bar{y}$ is the mean of the actual numbers across all images, and $n$ is the total number of images in the dataset.

The $R^2$ metric has a range of $(-\infty, 1]$. An $R^2$ of 1 indicates perfect prediction, where all predicted values exactly match the actual values. An $R^2$ of 0 suggests that the model explains none of the variability of the response data around its mean. If $R^2$ is negative, it indicates that the model performs worse than a model that simply predicts the mean of the actual values for all observations.

\subsubsection{\label{chap:S5.2}PQ}
Panoptic Quality ($PQ$) is a comprehensive metric used to evaluate the performance of segmentation models in tasks that require both instance segmentation and classification. $PQ$ provides a single score that encapsulates both the detection accuracy (i.e., how many objects were correctly identified) and the segmentation quality (i.e., how accurately the objects' boundaries were delineated). This metric is particularly useful in multiclass scenarios where each pixel is classified into distinct categories, such as different cell types in pathology images.

$PQ$ is calculated as the product of two terms: Detection Quality ($DQ$) and Segmentation Quality ($SQ$). It can be expressed as
\begin{equation*}
PQ = DQ \cdot SQ,
\end{equation*}
where
\begin{equation*}
DQ = \frac{TP}{TP + 0.5\, FP + 0.5\, FN},
\end{equation*}
\begin{equation*}
SQ = \frac{\sum_{(p, g) \in \mathcal{M}} IoU(p, g)}{TP}.
\end{equation*}
In these formulas, $TP$ denotes the number of correctly matched instances between ground truth and prediction, $FP$ denotes the predicted instances that have no corresponding ground truth, $FN$ denotes the ground truth instances that were not detected, $IoU(p, g)$ is the Intersection over Union for a pair of matched instances $p$ (prediction) and $g$ (ground truth), and $\mathcal{M}$ is the set of matched pairs.

The $PQ$ metric is calculated for each class and is averaged across classes to provide a global performance measure.

The $PQ$ score has a range of $[0, 1.0]$, where a higher score indicates better performance in both detecting and segmenting the instances correctly. A $PQ$ of 1 signifies perfect identification and segmentation of all instances, whereas a $PQ$ of 0 indicates that no instances were correctly identified and segmented.

\clearpage

\subsection{\label{chap:S6}Segmentation and Detection quality metrics for teacher and student models}

\begin{table}[h!]
\renewcommand{\arraystretch}{2.0}
\centering
\caption{Segmentation and detection quality for student and teacher models (CI 95\%)}
\label{tab:S6}
%\adjustbox{max width=\textwidth}{%
\begin{tabular}{|l|c|c|}
\hline
%\rowcolor{gray!30}
Metric & Teacher & Student \\
\hline
$SQ_{neoplastic}$ & 0.819 (0.815--0.823) & 0.824 (0.819--0.828) \\
\hline
$SQ_{lymphocyte}$ & 0.795 (0.788--0.802) & 0.790 (0.783--0.796) \\
\hline
$SQ_{connective}$ & 0.770 (0.762--0.776) & 0.780 (0.772--0.786) \\
\hline
$SQ_{dead}$ & 0.659 (0.623--0.688) & 0.657 (0.624--0.695) \\
\hline
$SQ_{epithelial}$ & 0.780 (0.770--0.790) & 0.788 (0.779--0.797) \\
\hline
$SQ_{macrophage}$ & 0.788 (0.760--0.810) & 0.757 (0.730--0.783) \\
\hline
$SQ_{neutrofil}$ & 0.782 (0.761--0.801) & 0.775 (0.759--0.792) \\
\hline
$DQ_{neoplastic}$ & 0.706 (0.692--0.719) & 0.727 (0.712--0.741) \\
\hline
$DQ_{lymphocyte}$ & 0.675 (0.656--0.698) & 0.713 (0.691--0.734) \\
\hline
$DQ_{connective}$ & 0.566 (0.546--0.584) & 0.583 (0.565--0.602) \\
\hline
$DQ_{dead}$ & 0.410 (0.361--0.465) & 0.435 (0.306--0.561) \\
\hline
$DQ_{epithelial}$ & 0.668 (0.639--0.694) & 0.673 (0.644--0.702) \\
\hline
$DQ_{macrophage}$ & 0.657 (0.583--0.727) & 0.615 (0.531--0.703) \\
\hline
$DQ_{neutrofil}$ & 0.691 (0.625--0.753) & 0.729 (0.679--0.778) \\
\hline
\end{tabular}
%
%}
\end{table}

\clearpage

\subsection{\label{chap:S7}QuPath integration method}
We adopt an integration strategy leveraging the paquo \cite{Bayer_AG} library, a Python package that enables direct interaction with QuPath’s internal API, thereby facilitating seamless data exchange without intermediate conversion steps. The data processing pipeline (\hyperref[fig:S7]{Appendix Figure S7}) begins with the acquisition of WSIs and their associated annotations from QuPath, which are represented as Shapely \cite{Gillies_Wel_etal._2024} polygons. Utilizing paquo, we directly read, create, and modify these annotations and detections within a QuPath project in the Python environment. Images are then cropped using these polygons and processed by cell segmentation and classification models employing standard vision processing toolkits such as OpenCV, pyvips, and PyTorch. Additionally, QuPath employs Groovy scripts to initiate a Python process that starts the entire pipeline from QuPath graphical interface: fetching polygons, extracting images from them, and running deep learning model inference on the cropped images. 
The results are returned to QuPath, leveraging paquo's Python bindings to manipulate QuPath data while minimizing the computational overhead typically associated with cross-environment communication.

\counterwithin{figure}{subsection}
\renewcommand{\thefigure}{S\arabic{subsection}}

\begin{figure}[h!]
    \centering
    \includegraphics[width=\textwidth]{images/A7.pdf}
    \caption{QuPath integration workflow using Python environment}
    \label{fig:S7}
\end{figure}

Compared to traditional workflows that involve exporting annotations as GeoJSON, classifying them in Python, and reimporting them into QuPath, our approach offers several advantages. We eliminate the need to switch between programming languages, providing a cohesive and streamlined development process entirely within QuPath software and removing the necessity to use other tools. Meanwhile, we avoid storing annotations as intermediate JSON files unless required for external use or archiving. By conducting the entire inference and post-processing workflow within the Python environment, we leverage the power and flexibility of Python libraries for image processing and machine learning. This approach also enables adjustments to any set of labels and models, thereby improving its applicability.

%\hfill

The distilled model and QuPath integration code are packaged into a Docker container, enabling streamlined execution with the Docker engine. Detailed integration code and deployment instructions can be found in the GitHub repository \cite{Shvetsov_2025b}.

Despite these benefits, we acknowledge that the paquo library is a proof‑of‑concept project in its early development stage and has not been tested across all versions of QuPath.

\clearpage

\subsection{\label{chap:S8}Data and code availability statement}
All datasets, models, and code used in this study are publicly available and can be obtained from the repositories listed below. 
The PanNuke \cite{Gamper_Koohbanani_etal._2019} and MoNuSAC \cite{Verma_Kumar_etal._2021} datasets are publicly accessible, and download information along with detailed descriptions can be found in their respective articles. Preprocessing scripts for PanNuke and MoNuSAC data, as well as individual cell extraction scripts, are available on GitHub \cite{Shvetsov_2025a}. The H-Optimus foundation model used in our experiments can be downloaded from the HuggingFace repository \cite{hoptimus2024}, and model information is available on GitHub \cite{Saillard_Jenatton_etal._2024}. In addition, the integration code for QuPath and the distilled model packaged in a Docker container are provided in the repository \cite{Shvetsov_2025b}, and paquo Python library is available from the authors GitHub repository \cite{Bayer_AG}.
\clearpage

\end{document}

% \bibliography{ref}

% \begin{IEEEbiography}
% % [{\includegraphics[width=1in,height=1.25in,clip,keepaspectratio]{./AuthorPhotos/Linhao.jpg}}]{Lin Hao}
% A
% \end{IEEEbiography}
% \begin{IEEEbiography}
% % [{\includegraphics[width=1in,height=1.25in,clip,keepaspectratio]{./AuthorPhotos/MustafaKishk.jpg}}]{Mustafa A. Kishk} 
% (Member, IEEE) received the
% B.Sc. and M.Sc. degrees from Cairo University
% in 2013 and 2015, respectively, and the Ph.D. degree
% from Virginia Tech in 2018. He is currently a
% Post-Doctoral Research Fellow with the CTL at
% KAUST. His current research interests include stochastic geometry, energy harvesting wireless networks, UAV-enabled communication systems, and
% satellite communications.
% \end{IEEEbiography}



\end{document}

