\section{Related Work}
Infrastructure sharing, as a potential solution to improve network performance in multi-operator regions, becomes vital to reduce the high costs of BS deployment in the 5G era. However, few studies discussed the impact of infrastructure sharing on the service continuity of user equipment (UEs) from the perspective of percolation theory. For cellular networks, we consider SINR as the main metric to judge the continuity of BSs' coverage areas. Therefore, we divide the related work into: i) infrastructure sharing and ii) percolation theory and SINR analysis. \\
\indent \textit{Infrastructure Sharing}: As an important concept in 5G cellular networks, infrastructure sharing has been analyzed in various works in literature, which can save the cost, share the risk, boost quality of service and avoid environmental emissions \cite{dlamini2021remote}. There exist various models of infrastructure sharing. Authors in \cite{7105671} identified the characteristics of existing and future multi-operator network architectures, including mobile virtual network operators (MVNOs), trusted third parties, unique infrastructure providers, and standalone cases. A novel BS switching-off scheme was proposed to achieve significant energy and cost savings. Different sharing strategies have different advantages and also challenges, therefore they can be compared or combined in different application scenarios. 
In \cite{7343930}, authors assessed the fundamental trade-offs between spectrum and radio access infrastructure sharing. 
Authors in \cite{7562085} accurately modeled the channel propagation and antenna characterization and showed that a full spectrum and infrastructure sharing configuration can help increase user rate and bring economical advantages. 
In \cite{7500354}, authors introduced a mathematical framework to analyze multi-operator cellular networks that share spectrum licenses and infrastructure elements. Authors in \cite{8315130} proposed a mathematical framework to model a multi-operator mmWave cellular network with co-located BSs. They derived the SINR distribution and the coverage probability. 
In \cite{8594671}, authors proposed a multi-operator cooperation framework for sharing base stations among $N$ number of co-located radio access networks to improve energy efficiency. 
The proposed algorithm had great capacity in saving energy as well. They modeled locations of BSs in each network using independent Hardcore Poisson point processes (HCPPs). In infrastructure sharing, different MNOs' operation or sharing methods are also various.
In \cite{8329530}, authors modeled a single buyer MNO and multiple seller MNO infrastructure sharing system. Considering a given QoS in terms of the SINR coverage probability, they analyzed the trade-off between the transmit power of a BS and the intensity of BSs of the buyer MNO. Especially, MVNOs do not have their own infrastructure, so that their main expenses only come from renting transceivers and infrastructure \cite{7105671}. In \cite{zheng2017economic}, authors studied the behavior of MVNOs and Internet service providers, and derived the conditions for cross-carrier MVNOs to make profits and reduce costs for their users.
In a multi-operator cellular network, how to choose the best sharing solution requires reference to factors such as profit growth and performance improvement. Authors in \cite{cano2017optimal} considered regions with different areas and user amount, and discussed the optimal sharing strategy and number of active base stations under different service unit prices. Focusing on remote and rural areas, the authors in \cite{dlamini2021remote} reduced energy consumption while maintaining a quality of service comparable to that in urban areas. To better model random networks without losing accuracy and tractability, stochastic geometry has been widely used to evaluate the network performance with different infrastructure sharing strategies \cite{hmamouche2021new}.
They employed the homogeneous Poisson point process (PPP) and Gauss-Poisson process (GPP) to analyze the coverage probability and average user data rate. They extended their work in \cite{7876864}, differentiated the spectrum sharing experiencing flat or frequency-selective power fading, and considered the impact of network density imbalance between sharing MNOs. In summary, infrastructure sharing has been analyzed from many indicators, but there is still a lack of research on the existence of large-scale connected service areas.

\indent \textit{Percolation Theory and SINR analysis}: Percolation theory has been widely used to analyze the existence of large-scale multi-hop links or connected coverage areas in wireless networks \cite{haenggi2012stochastic, elsawy2023tutorial}. Authors in \cite{zhang2018robustness} studied the robustness of two spatially embedded networks that are interdependent. Focusing on the Gilbert disk model (GDM) and random Gilbert disk model (RGDM), authors in \cite{anjum2019percolation} derived the bounds of critical density regime in these two cases that are used to analyze the percolation in large-scale wireless balloon networks. They also analyzed the critical density of unmanned aerial vehicle (UAV) networks to ensure large-scale network coverage in \cite{9049663}. For sensing and monitoring applications, the path exposure problem was characterized in \cite{8794718,liu2012optimal}, where the authors find the critical density of sensors or cameras to detect moving objects over arbitrary paths through a given region. In \cite{9214384}, a novel and low-cost countermeasure against malware epidemics in large-scale wireless networks (LSWN) was proposed, which was denoted as spatial firewalls. In \cite{9214384} and \cite{9240972}, the critical density of such spatial firewalls and the percentage of secured devices were characterized using the tools of percolation probability. Authors in \cite{yemini2019simultaneous} proved the possibility of the coexistence of random primary and random secondary cognitive networks, both of which can include an unbounded connected component. Authors in \cite{wu2023connectivity} investigated the BS networks assisted by reconfigurable intelligent surfaces (RISs), and derived the lower bound of the critical density of RISs. Based on dynamic bond percolation, authors in \cite{han2024dynamic} proposed an evolution model to characterize the reliable topology evolution affected by the nodes and links states. In cellular networks, the effect of SINR on network performance and connectivity can not be ignored. Authors in \cite{jahnel2022sinr} derived a vital relationship between the interference cancellation factor, SINR threshold, and the status of percolation. In \cite{tobias2020signal}, authors showed that an SINR graph has an infinite connected component when the device density is large enough and the interferences are reduced sufficiently. They also estimated the relationship between the critical interference cancellation factor and device density. Authors in \cite{elsawy2023tutorial} summarized several classical percolation models. They investigated the critical relationship between device density, interference reduction factor and SINR threshold in the SINR graphs. However, there is still a lack of research that can provide critical conditions or approximate critical conditions for percolation in SINR graphs with infrastructure sharing. Therefore, it is necessary to further utilize SINR analysis to evaluate the network connectivity through percolation theory.