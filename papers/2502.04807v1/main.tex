%%%%%%%%%%%%%%%%%%%%%%%%%%%%%%%%
%%%%%%%%%%% ARXIV %%%%%%%%%%%%%%
%%%%%%%%%%%%%%%%%%%%%%%%%%%%%%%%
\documentclass{article} % For LaTeX2e
\usepackage[table]{xcolor}
\usepackage{times}
\usepackage{eso-pic} % used by \AddToShipoutPicture
\usepackage[margin=1in]{geometry}
\usepackage{authblk}
\renewcommand\Authfont{\large}
\renewcommand\Affilfont{\small\color{black}}
\usepackage[square,numbers]{natbib}
\bibliographystyle{abbrvnat}
%%%%%%%%%%%%%%%%%%%%%%%%%%%%%%
%%%%%%%%%%%%%%%%%%%%%%%%%%%%%%


\usepackage[utf8]{inputenc} % allow utf-8 input
\usepackage[T1]{fontenc}    % use 8-bit T1 fonts
\usepackage{hyperref}       % hyperlinks
\usepackage{url}            % simple URL typesetting
\usepackage{booktabs}       % professional-quality tables
\usepackage{amsfonts}       % blackboard math symbols
\usepackage{nicefrac}       % compact symbols for 1/2, etc.
\usepackage{microtype}      % microtypography
\usepackage{xcolor}         % colors
\usepackage{placeins}         % colors


%%%%%%%%%%%%%%%%%%%%%%%%%% more packages
\usepackage{amsthm}
\usepackage{amsmath}
\usepackage{graphicx}
\usepackage[capitalize,noabbrev]{cleveref}
\usepackage{algorithm}
\usepackage{algorithmic}
\usepackage{subcaption}
\usepackage[export]{adjustbox}% http://ctan.org/pkg/adjustbox

%%%%%%%%%%%%%%%%%%%%%%%%%%%%%%%%
% THEOREMS
%%%%%%%%%%%%%%%%%%%%%%%%%%%%%%%%
\theoremstyle{plain}
\newtheorem{theorem}{Theorem}[section]
\newtheorem{proposition}[theorem]{Proposition}
\newtheorem{lemma}[theorem]{Lemma}
\newtheorem{corollary}[theorem]{Corollary}
\theoremstyle{definition}
\newtheorem{definition}[theorem]{Definition}
\newtheorem{assumption}[theorem]{Assumption}
\theoremstyle{remark}
\newtheorem{remark}[theorem]{Remark}
%%%%%%%%%%%%%%%%%%%%%%%%%%%%%%%%%%%%%%%%%%%%%%%%%%%%%%%%%%%%%%%%%%%%%%% 
\usepackage[textsize=tiny]{todonotes}
\usepackage[export]{adjustbox}
% ################################################################
\usepackage{amsmath, amssymb} % Ensure these packages are included
\newcommand{\I}[1]{\mathbb{I} \left\{ #1 \right\}}
\renewcommand{\P}[1]{\mathbb{P} \left[ #1 \right]}
\newcommand{\EV}[1]{\mathbb{E} \left[ #1 \right]}

\usepackage{subcaption}
\DeclareMathOperator{\E}{\mathbb{E}}
\DeclareMathOperator{\p}{\mathbb{P}}
\DeclareMathOperator{\D}{\mathcal{D}}
\DeclareMathOperator{\H0}{\mathcal{H}_0}

\definecolor{Green}{RGB}{119,221,119}
\usepackage{pdflscape}
\usepackage{placeins}
\newcommand\numberthis{\addtocounter{equation}{1}\tag{\theequation}}

\usepackage{enumitem}
\allowdisplaybreaks
% ################################################################
\usepackage{xr}
\title{Robust Conformal Outlier Detection under\\Contaminated Reference Data}
\author[1]{Meshi Bashari}
\author[2,3]{Matteo Sesia}
\author[1,4]{Yaniv Romano}
\affil[1]{Department of Electrical and Computer Engineering, Technion IIT, Haifa, Israel}
\affil[2]{Department of Data Sciences and Operations, University of Southern California, Los Angeles, California, USA}
\affil[3]{Department of Computer Science, University of Southern California, Los Angeles, California, USA}
\affil[4]{Department of Computer Science, Technion IIT, Haifa, Israel}



\begin{document}

\date{}
\maketitle


\begin{abstract}
Conformal prediction is a flexible framework for calibrating machine learning predictions, providing distribution-free statistical guarantees. In outlier detection, this calibration relies on a reference set of labeled inlier data to control the type-I error rate. However, obtaining a perfectly labeled inlier reference set is often unrealistic, and a more practical scenario involves access to a contaminated reference set containing a small fraction of outliers. This paper analyzes the impact of such contamination on the validity of conformal methods. We prove that under realistic, non-adversarial settings, calibration on contaminated data yields conservative type-I error control, shedding light on the inherent robustness of conformal methods. This conservativeness, however, typically results in a loss of power. To alleviate this limitation, we propose a novel, active data-cleaning framework that leverages a limited labeling budget and an outlier detection model to selectively annotate data points in the contaminated reference set that are suspected as outliers. By removing only the annotated outliers in this ``suspicious'' subset, we can effectively enhance power while mitigating the risk of inflating the type-I error rate, as supported by our theoretical analysis. Experiments on real datasets validate the conservative behavior of conformal methods under contamination and show that the proposed data-cleaning strategy improves power without sacrificing validity.

\end{abstract}

\noindent \textbf{Keywords:}  Conformal Prediction, Hypothesis Testing, Out-of-Distribution Detection, Contaminated Data
\section{Introduction}
\label{sec:intro}
\subsection{Background and Motivation}
\label{sec:background}

This paper studies the problem of outlier detection: given a reference dataset (e.g., a collection of legitimate financial transactions) and an unlabeled test point (a new transaction), our goal is to determine whether the test point is an outlier (a fraudulent transaction) by assessing its deviation from the reference data distribution. Naturally, we aim to maximize the detection of outliers by harnessing the capabilities of complex machine learning (ML) models. However, these models typically lack type-I error rate control, potentially resulting in unreliable detections. In our running example, the type-I error is the probability of falsely flagging a legitimate transaction as fraudulent. As such, uncontrolled error rates can lead to costly unnecessary investigations of legitimate transactions and negatively impact customer experience. 

The broad need for reliable ML systems has sparked a surge of interest in conformal prediction---a versatile framework that can provide statistical guarantees for any ``black-box'' predictive model \cite{vovk2005algorithmic}. This framework formulates the outlier detection task as a statistical test, where the null hypothesis is that the new data point is not an outlier \cite{laxhammar2015inductive,conformal-p-values}. To derive a decision rule guaranteeing type-I error control, conformal inference relies on a reference (calibration) set of inlier data points. These points are assumed to be sampled i.i.d.~from an unknown distribution, independent of the data used to train the outlier detection model.

In practice, however, it is often difficult to obtain a perfectly clean reference dataset that contains no outliers \cite{park2021wrong,zhao2019robust,chalapathy2019deep,jiang2022softpatch}. 
In our example, a more realistic scenario would assume instead access to a slightly \emph{contaminated} reference set, mostly legitimate transactions with a few unnoticed outliers \cite{zhao2019robust}.
But this setting poses new challenges for conformal prediction methods, potentially invalidating the error control guarantees or, as we shall see, often reducing the power to detect true outliers at test time.

\subsection{Outline and Contributions}

While type-I error control in conformal inference theoretically requires perfectly clean reference data, in practice contaminated data often makes these methods overly conservative, reducing the power to detect true outliers rather than inflating the type-I error rate. This empirical observation motivates the first question explored in this paper:

\textbf{Q1:} \emph{When does conformal outlier detection with contaminated reference data yield valid type-I error control?}

In Section \ref{sec:conservativeness}, we present the first contribution of this paper: a novel theoretical analysis that identifies common conditions under which this conservative behavior arises. Unfortunately, this conservativeness often comes at the cost of reduced detection power, particularly when targeting low type-I error rates. To address this issue, we investigate data-driven cleaning strategies aimed at mitigating the contamination in the reference dataset.

A straightforward approach to cleaning the contaminated set is to remove all data points flagged as likely outliers by the detection model. However, this method is unsatisfactory, as it risks inadvertently removing inliers along with outliers, resulting in an "overly clean" reference set. This, in turn, distorts the inlier distribution and inflates the type-I error rate above the desired nominal level.

This challenge motivates our second and main contribution. In Section \ref{sec:label-trim}, we introduce an approach to clean the contaminated reference set by leveraging a limited labeling budget (e.g., 50 annotations). The outlier detection model is first used to identify suspected outliers within the contaminated reference set. The limited budget is then strategically allocated to annotate these points, thereby avoiding the unintended removal of inliers. While this is a practical and intuitive approach, it naturally prompts a critical question:

\textbf{Q2:} \emph{How does the selective annotation and partial removal of outliers from a contaminated reference set affect the validity of conformal inferences?}

We analyze the validity of this active labeling approach for trimming outliers in the contaminated set. Our theoretical results identify the conditions required to achieve approximate type-I error control, even when the data are selectively annotated and not all outliers are removed. This analysis also highlights key factors that can inflate the error rate, offering practitioners guiding principles to enhance the power of conformal methods in the presence of contaminated data.

Finally, in Section \ref{sec:experiments}, we empirically validate our theory and proposed data-cleaning approach through comprehensive experiments on real-world datasets. The experiments confirm that conformal inference with contaminated data tends to be conservative. Furthermore, they demonstrate that our method significantly boosts power, particularly when the target type-I error rate is low and the number of outliers in the contaminated set is small.
Software for reproducing the experiments is available at \href{https://github.com/Meshiba/robust-conformal-od}{https://github.com/Meshiba/robust-conformal-od}.

\subsection{Related Work}

Recently, there has been growing interest in studying the statistical properties of conformal inference methods under more realistic scenarios, moving beyond the idealized assumption of perfectly clean and exchangeable observations to account for various types of {\em distribution shift} \citep{tibshirani2019conformal,einbinder2022conformal,sesia2023adaptive,barber2023conformal,gibbs2021adaptive,zaffran2022adaptive,feldman2022achieving,gibbs2024conformal,podkopaev2021distribution,si2023pac,prinsterconformal}. This paper draws inspiration from several prior works in this area.

\citet{tibshirani2019conformal} introduced a weighted conformal prediction approach to address covariate shift between calibration and test data, later extended by \citet{podkopaev2021distribution} to accommodate label shift. Both settings, however, involve a different form of distribution shift from the one we study here. \citet{barber2023conformal} extend this line of work by analyzing the effects of general distribution shifts on the validity of conformal methods, focusing however on worst-case scenarios; see also \citet{farinhas2024nonexchangeable}.

In contrast, our work moves away from this worst-case perspective. We aim to explain why conformal outlier detection with contaminated reference data often results in a conservative type-I error rate, rather than investigating type-I error inflation, which, while theoretically possible in adversarial settings, appears less common in practice. Furthermore, we focus on developing methods to address this over-conservativeness, boosting detection power.

A more closely related line of work investigates the robustness of conformal prediction to label noise \cite{einbinder2022conformal, sesia2023adaptive, clarkson2024splitconformalpredictiondata,penso2025estimating} or other forms of data contamination \cite{pmlr-v202-zaffran23a, zaffran2024predictive, feldman2024robust}. Specifically, \citet{einbinder2022conformal} and \citet{sesia2023adaptive} show that, under certain assumptions, conformal prediction for classification with noisy labels often results in conservative type-I error rates. Furthermore, \citet{sesia2023adaptive} propose a method to address this conservativeness by leveraging an explicit ``label noise model'' that captures the relationship between the true and contaminated labels in the calibration dataset.

In contrast, this paper avoids relying on an explicit model for the contaminated data, as such models can be difficult to estimate in practice within our context. Instead, we utilize a pre-trained black-box outlier detection model and a limited annotation budget to selectively and reliably trim outliers from the contaminated set. Furthermore, it is important to emphasize that the method proposed by \citet{sesia2023adaptive} is primarily designed for classification tasks with relatively balanced data, whereas outlier detection naturally involves extreme class imbalance. This distinction underscores the need for solutions specifically tailored to outlier detection.

\section{Setup and Preliminary Results}

\subsection{Inference with Clean Calibration Data} \label{sec:cp}

Conformal inference for outlier detection requires a reference (or \emph{calibration}) set, $\D_{\mathrm{cal}} = [n] := \{1,\ldots,n\}$, containing $n$ data points. The reference set is typically assumed to be \emph{clean}, consisting solely of \emph{inliers}, which are i.i.d.~samples from an unknown distribution $\p_0$ (exchangeability may sometimes suffice, but this work assumes i.i.d.~inliers). Under this assumption, $\D_{\mathrm{cal}}$ may be referred to as $\D_{\mathrm{inlier}}$.

The goal is to determine whether a new observation, $X_{n+1}$, is an inlier—independently sampled from $\p_0$---or an \emph{outlier}, sampled from a different distribution $\p_1 \neq \p_0$. This can be formulated as a hypothesis testing problem, where the null hypothesis $\mathcal{H}_0$ claims that $X_{n+1}$ is an inlier:
\begin{align} \label{eq:setup-clean}
\begin{split}
  & X_i \overset{\text{i.i.d.}}{\sim} \p_0, \; \forall i \in \D_{\mathrm{inlier}}, \quad
  \D_{\mathrm{inlier}} = \D_{\mathrm{cal}} = [n], \\
  & \mathcal{H}_0 : X_{n+1} \overset{\text{ind.}}{\sim} \p_0.
\end{split}
\end{align}

The split-conformal method, a simple and computationally efficient approach, uses a pre-trained outlier detection model—potentially any machine learning model—to compute \emph{nonconformity scores} that quantify how different a data point is from the reference distribution. The model, represented by a score function $s$, is trained on a separate dataset $\D_{\mathrm{train}}$, which is similar to but independent of $\D_{\mathrm{cal}}$. Typically, a larger dataset of inliers, assumed to be i.i.d.~samples from $\p_0$, is randomly split into $\D_{\mathrm{train}}$ and $\D_{\mathrm{cal}}$.

The model tries to learn a score function $s$ such that larger values of $s(X_{n+1})$ indicate stronger evidence that the test point may be an outlier. Conformal inference rigorously quantifies this evidence, providing a principled decision rule for rejecting $\mathcal{H}_0$ when the evidence is strong enough, while controlling the type-I error rate—the probability of incorrectly rejecting $\mathcal{H}_0$ when $X_{n+1}$ is actually an inlier.

This statistical evidence is quantified by computing a \emph{conformal p-value}, defined as: 
\begin{align} \label{eq:conformal-p-value}
  \hat{p}_{n+1} = \frac{1 + \sum_{i=1}^{n} \mathbb{I}[s(X_i) \geq s(X_{n+1})]}{1 + n}. 
\end{align} 
Thus, larger values of $s(X_{n+1})$ correspond to smaller values of $\hat{p}_{n+1}$, and the test point $X_{n+1}$ can be confidently classified as an outlier (rejecting $\mathcal{H}_0$) when $\hat{p}_{n+1}$ is smaller than a given significance level $\alpha \in (0,1)$.

\begin{proposition}[from \citet{vovk2005algorithmic}] \label{prop:standard-conformal}
Under~\eqref{eq:setup-clean}, if the null hypothesis $\mathcal{H}_0$ is true, then for any $\alpha \in (0,1)$:
$\p \left( \hat{p}_{n+1} \leq \alpha \right) \leq \alpha$.
Further, if $s(X)$ has a continuous distribution under $\p_0$, then
$\p \left( \hat{p}_{n+1} \leq \alpha \right) \geq \alpha - 1/(n+1)$.
\end{proposition}

Proposition~\ref{prop:standard-conformal} intuitively states that the conformal p-value defined in~\eqref{eq:conformal-p-value} provides a well-calibrated rule for flagging a new data point as a likely outlier. Rejecting $\mathcal{H}_0$ when $\hat{p}_{n+1} \leq \alpha$ ensures type-I error control at level $\alpha$ while avoiding excessive conservatism. Specifically, the type-I error rate closely matches $\alpha$ when the sample size $n$ is large and the nonconformity scores have a continuous distribution with no ties—a mild condition that can be achieved in practice by adding small random noise to the scores.

What remains unclear, and serves as the starting point of this paper, is how conformal p-values behave when the calibration dataset is contaminated, containing not only inliers but also a fraction of misplaced outliers.


\subsection{Inference with Contaminated Calibration Data} \label{sec:notations}

In this paper, we consider a more general setting where the calibration dataset, indexed by $\D_{\mathrm{cal}} = [n]$, may contain both inliers ($\D_{\mathrm{inlier}}$), sampled i.i.d.~from a distribution $\p_0$, and outliers ($\D_{\mathrm{outlier}}$), sampled i.i.d.~from a different distribution $\p_1 \neq \p_0$. Thus, $\D_{\mathrm{cal}} = \D_{\mathrm{inlier}} \cup \D_{\mathrm{outlier}}$.
The numbers of inliers and outliers, respectively $n_0 = |\D_{\mathrm{inlier}}|$ and $n_1 = |\D_{\mathrm{outlier}}|$, are treated as fixed, with $n = n_0 + n_1$. 
The goal remains to test the null hypothesis $\mathcal{H}_0$ that a new data point $X_{n+1}$ is an inlier, independently sampled from $\p_0$. Formally, this setup can be written as:
\begin{align} \label{eq:setup-contaminated}
\begin{split}
  & X_i \overset{\text{i.i.d.}}{\sim} \p_0, \; \forall i \in \D_{\mathrm{inlier}}, \quad X_i \overset{\text{i.i.d.}}{\sim} \p_1, \; \forall i \in \D_{\mathrm{outlier}}, \\
  & \D_{\mathrm{inlier}} \cup \D_{\mathrm{outlier}} = \D_{\mathrm{cal}} = [n], \\
  & \mathcal{H}_0 : X_{n+1} \overset{\text{ind.}}{\sim} \p_0.
\end{split}
\end{align}

In the following, we first analyze the behavior of standard conformal p-values, computed as in~\eqref{eq:conformal-p-value}, when applied to contaminated data scenarios described by~\eqref{eq:setup-contaminated}. Subsequently, we will propose a novel method for computing more refined conformal p-values by approximately cleaning the calibration set to remove undesired outliers.

\subsection{Explaining the Conservativeness}\label{sec:conservativeness}

Empirical results suggest that contamination by outliers in the calibration data often makes standard conformal p-values overly conservative, resulting in a type-I error rate significantly lower than the desired nominal level $\alpha$.

We begin by examining~\Cref{fig:scores-shuttle}, which provides some insight into this behavior based on the analysis of the ``shuttle'' dataset \cite{shuttle}, as detailed in Section~\ref{sec:experiments}. In this example, the nonconformity scores of outlier data points in the contaminated calibration set are typically larger than those of the inliers.
This pattern reflects the goal of a well-designed outlier detection model: to differentiate outliers from inliers by assigning higher scores to the former. Consequently, conformal p-values computed using~\eqref{eq:conformal-p-value} will be inflated relative to the ideal scenario in which all calibration points are inliers, reducing our power to detect true outliers at test time.

\begin{figure}[!htb]
\centering
    \includegraphics[width=0.5\textwidth]{figures/exp/real_data/shuttle/hist_shuttle_n_2500_p_0.05_p_train_0.05_a_0.02.pdf}
    \caption{Histogram of nonconformity scores for inliers and outliers in a contaminated calibration subset of the ``shuttle" data, with a contamination rate of 5\%. The vertical lines indicate the $(1-\alpha)$ empirical quantile of all calibration scores (black), as well as separately for inliers (blue) and outliers (red), with $\alpha = 0.02$.}
    \label{fig:scores-shuttle}
\end{figure}

This phenomenon is further corroborated by extensive numerical experiments presented in Section~\ref{sec:experiments} and \Cref{app-sec:real-data-exp,app-sec:images-data-exp}, which consistently demonstrate this conservative behavior across nine different datasets.

While the conservativeness of conformal prediction methods in the presence of contaminated calibration data has already been observed and studied theoretically in different contexts \citep{einbinder2022conformal,sesia2023adaptive,clarkson2024splitconformalpredictiondata}, prior works did not focus on outlier detection.
Therefore, it is helpful to introduce a new theoretical result that precisely quantifies the inflation of standard conformal p-values that we often observe in practice. 
This will serve as a foundation for the novel method of computing adaptive conformal p-values presented in the next section.

Let $\hat{F}_{1}$ denote the empirical cumulative distribution function (CDF) of the scores in $\D_{\mathrm{outlier}}$ and $\hat{Q}_{1-\alpha}^{\mathrm{cal}}$ represent the $\lceil (1-\alpha)(1+n)\rceil$-th smallest score in the calibration set.

\begin{lemma}
\label{lem:conservativeness}
Under the setup defined in~\eqref{eq:setup-contaminated}, if $\mathcal{H}_0$ is true, then for any  $\alpha\in(0,1)$,
    \begin{align*}
    \p &\left( \hat{p}_{n+1}\leq \alpha \right) \leq \alpha -\frac{n_1}{n_0 + 1} \left( 1-\alpha -   \E\left[ \hat{F}_{1} \left( \hat{Q}^{\mathrm{cal}}_{1-\alpha} \right) \right]\right).
    \end{align*}
\end{lemma}

This result is related to Theorem 1 in \citet{sesia2023adaptive}, which studies the behavior of conformal prediction sets for multi-class classification \citep{lei2013distribution,romano2020classification} calibrated with contaminated data. The key distinction is in our treatment of the calibration set: we assume that $n_0$ and $n_1$ are fixed, whereas \citet{sesia2023adaptive} consider a mixture model where the observed proportions of data points from different classes in the calibration set are random.
While treating $n_0$ and $n_1$ as fixed is convenient for this paper, we also include an additional result (\Cref{cor:conservativeness-p-values}) in Appendix~\ref{app-sec:proofs}, which reaches qualitatively similar conclusions by adopting an approach more closely aligned with Theorem 1 from \citet{sesia2023adaptive}.

A direct corollary of \Cref{lem:conservativeness} is that standard conformal p-values are conservative when outlier scores are typically larger than inlier scores. Formally, this condition is:
\begin{assumption}\label{asm:model-scores} 
$\E [ \hat{F}_{1} ( \hat{Q}_{1-\alpha}^{\mathrm{cal}} ) ] < 1-\alpha$. 
\end{assumption}
If \Cref{asm:model-scores} fails—for instance, when outlier scores are smaller than inlier scores—data contamination may invalidate standard conformal p-values, inflating the type-I error rate.
However, it is more common in practice that \Cref{asm:model-scores} holds, in which case contamination tends to reduce calibration power, and more so if $n_1$ is large.
In particular, \Cref{asm:model-scores} holds if:
(i) the outlier detection model is relatively accurate, and
(ii) the outlier distribution $\p_1$ is not adversarial.
Our experiments will show this power loss can be substantial, motivating the need for new methods that can approximately ``clean up'' the calibration data.


\section{Methods}\label{sec:trim}

\subsection{Key Idea: Boosting Power by Cleaning the Data}

Ideally, we would like to remove all $n_1$ outliers from $\D_{\mathrm{cal}}$, restoring the ideal behavior of conformal p-values calibrated on a clean dataset, as described in \Cref{prop:standard-conformal}, and likely boosting power. However, manually labeling the entire contaminated calibration set $\D_{\mathrm{cal}}$ is often impractical, especially when $n = |\D_{\mathrm{cal}}|$ is large. At the same time, utilizing only a small calibration set is not always desirable.

A large calibration set is often needed because the smallest conformal p-value obtainable through~\eqref{eq:conformal-p-value} scales as $1/n$. Thus, a large $n$ is critical for achieving high confidence in identifying outliers, especially in ``needle-in-a-haystack'' scenarios \citep{conformal-p-values}, where a few outliers must be detected in a large test set dominated by inliers. In such cases, the ability to obtain very small p-values is essential to achieve non-trivial power while controlling the false discovery rate \citep{BH}.

\subsection{A Simple but Unsatisfactory Approach: Naive-Trim} \label{sec:naive-trim}

The above challenge underscores the need for a method to mitigate the impact of outliers in the calibration dataset without requiring exhaustive annotation. An intuitive approach is to forgo annotations and simply remove all ``suspicious'' data points with large nonconformity scores. For instance, one could remove the top $m$ scores from $\D_{\mathrm{cal}}$, where $m$ is a fixed guess of the true number of outliers $n_1$ in the calibration set. We refer to this approach as \texttt{Naive-Trim}.

While \texttt{Naive-Trim} can reduce conservativeness by removing some outliers, it is not a satisfactory solution as it risks ``over-compensating''. By potentially removing also true inliers with large nonconformity scores, it can significantly skew the inlier score distribution to the left. This side effect is problematic, as it tends to invalidate conformal p-values and inflate the type-I error rate, over-correcting the conservativeness of standard conformal p-values.

This issue is particularly pronounced when $m > n_1$ or in noisy settings where the outlier detection model cannot perfectly distinguish between inliers and outliers. For example, as shown in Section~\ref{sec:experiments}, applying \texttt{Naive-Trim} to the dataset illustrated in \Cref{fig:scores-shuttle} results in uncontrolled inflation of the type-I error rate.

To address this challenge, we will now present a more sophisticated method, which we refer to as  \texttt{Label-Trim}. These approach utilize a limited labeling budget to remove outliers from $\D_{\mathrm{cal}}$ in a more reliable manner, mitigating the risk of over-correcting the conformal p-value.

\subsection{The \texttt{Label-Trim} Method}\label{sec:label-trim}

Consider having a limited budget to label $m < n$ calibration samples, where $m$ is much smaller than $n$. We aim to utilize this budget to remove as many outliers as possible from the calibration set without altering the inlier score distribution.

A practical approach is to annotate the $m$ largest scores in $\D_{\mathrm{cal}}$, as these are most likely outliers based on the model. Denote these annotated samples as $\D_{\mathrm{labeled}} \subseteq \D_{\mathrm{cal}}$, and let $\D_{\mathrm{labeled}}^{\mathrm{outlier}}$ denote the subset of annotated data points that are true outliers. Removing these outliers from $\D_{\mathrm{cal}}$ yields a smaller, cleaner calibration set, which we call $\D_{\mathrm{cal}}^{\mathrm{LT}} \subseteq \D_{\mathrm{cal}}$.

The \texttt{Label-Trim} method then calculates a refined conformal p-value, now denoted as $\hat{p}^{\mathrm{LT}}_{n+1}$, following the same procedure as in~\eqref{eq:conformal-p-value} with $\D_{\mathrm{cal}}$ replaced by the (partially) cleaned calibration set $\D_{\mathrm{cal}}^{\mathrm{LT}}$:
\begin{align} \label{eq:LT-p-value}
  \hat{p}^{\mathrm{LT}}_{n+1} = \frac{1 + \sum_{i \in \D_{\mathrm{cal}}^{\mathrm{LT}}} \mathbb{I}[s(X_i) \geq s(X_{n+1})]}{1 + |\D_{\mathrm{cal}}^{\mathrm{LT}}|}. 
\end{align} 
Algorithms~\ref{algo:label-trim-construction} and~\ref{algo:label-trim-testing} summarize this procedure, which intuitively offers advantages over both the standard method for computing $\hat{p}_{n+1}$ in~\eqref{eq:conformal-p-value}, by potentially increasing power, and the \texttt{Naive-Trim} approach, by mitigating the risk of over-correcting $\hat{p}_{n+1}$.

\begin{algorithm}[!htb]
\caption{Label-trim calibration  (construction phase)}
\label{algo:label-trim-construction}
\begin{algorithmic}[1]
\STATE \textbf{Input:} labeling budget $m$; contaminate calibration-set $\D_{\mathrm{cal}} = \left\{ X_i \right\}_{i=1}^{n}$; score function $s(\cdot)$, obtained by a pre-trained outlier detection model;

\STATE Compute the calibration scores $S_i = s(X_i)$, $\forall i\in \D_{\mathrm{cal}}$.
\STATE Sort the calibration scores, such that $S_{\pi(1)} \leq \dots \leq S_{\pi(n)}$ where $\pi : [n] \rightarrow [n]$ is the corresponding permutation of the indices.
\STATE Annotate the $m$ largest scores  $\D_{\mathrm{labeled}}$:= $\{ (S_{\pi(i)}, Y_{\pi(i)}) : i > n-m\}$, with $Y_{\pi(i)}=0$ if $X_{\pi(i)}$ is an inlier and $Y_{\pi(i)}=1$ otherwise. 
\STATE Construct the trimmed calibration set $\D_{\mathrm{cal}}^{\mathrm{LT}} = \left\{\pi (i) : i \leq n - m\right\}\cup \left\{j:  j\in \D_{\mathrm{labeled}}\text{ and }Y_j=0\right\}$.
% \STATE Construct the trimmed calibration set $\D_{\mathrm{cal}}^{\mathrm{LT}} = \left\{S_{\pi(i)}\right\}_{i=1}^{n-m} \cup \left\{S_{j}:  j\in \D_{\mathrm{labeled}}\text{ and }Y_j=0\right\}$

\STATE \textbf{Output:} trimmed calibration set $\D_{\mathrm{cal}}^{\mathrm{LT}}$.
\end{algorithmic}
\end{algorithm}

\begin{algorithm}[!htb]
\caption{Label-trim calibration (testing phase)}
\label{algo:label-trim-testing}
\begin{algorithmic}[1]
\STATE \textbf{Input:} test point $X_{n+1}$; score function $s$; trimmed calibration set $\D_{\mathrm{cal}}^{\mathrm{LT}}$; type-I error level $\alpha$;
% \STATE Compute the test score $S_{n+1}=s(X_{n+1})$
\STATE Compute the conformal p-value $\hat{p}^{\mathrm{LT}}_{n+1}$ according to \eqref{eq:LT-p-value}.
\STATE \textbf{Output:} reject the null hypothesis $\mathcal{H}_0$ if $\hat{p}_{n+1}^{\mathrm{LT}} \leq \alpha$, classifying $X_{n+1}$ as an outlier.
\end{algorithmic}
\end{algorithm}

The following theorem provides justification for \texttt{Label-Trim}, demonstrating that $\hat{p}^{\mathrm{LT}}_{n+1}$ is an approximately valid p-value under relatively mild conditions. While our method is intuitive, this result is nontrivial for two reasons.
First, \texttt{Label-Trim} cannot guarantee the removal of all outliers from the calibration set, as it may be that $m < n_1$ or some outliers are not among the $m$ largest scores. Second, it involves annotating the $m$ largest scores, revealing the true labels of some calibration points but not others, which could disrupt the exchangeability typically assumed among inlier data points in conformal inference.
Therefore, this justification requires novel proof techniques and does not follow directly from existing results.

Following a notation similar to that of Lemma~\ref{lem:conservativeness}, let $\hat{F}^{\mathrm{LT}}_{1}$ denote the empirical CDF of the scores in $\D^{\mathrm{LT}}_{\mathrm{outlier}}$. Define also $\hat{Q}^{\mathrm{LT}}_{1-\alpha}$ as the $\hat{i}_{\mathrm{LT}}$-th smallest element in $\{S_i\}_{i\in \D_{\mathrm{cal}}^{\mathrm{LT}}}\cup\{\infty\}$, with $\hat{i}_{\mathrm{LT}} :=\lceil (1-\alpha)(n^{\mathrm{LT}}+1)\rceil$ and $n^{\mathrm{LT}}:=\left|\D_{\mathrm{cal}}^{\mathrm{LT}}\right|$.

\begin{theorem}
    \label{thm:labeled-trim}
    Consider the setup in~\eqref{eq:setup-contaminated}, with $\mathcal{H}_0$ being true.
    For any fixed $\alpha \in (0,1)$, assume that $m \leq \alpha (n+1)$.
    Then,
    \begin{align*}
     \p \left( \hat{p}_{n+1}^{\mathrm{LT}} \leq \alpha \right) \leq \alpha + \frac{1}{n_0+1}
         - \E\left[ \frac{\hat{n}^{\mathrm{LT}}_{1}}{n_0+1} \left( (1-\alpha) - \hat{F}_{1}^{\mathrm{LT}} \left( \hat{Q}_{1-\alpha}^{\mathrm{LT}} \right) \right)\right].
    \end{align*}
\end{theorem}

The upper bound on the type-I error rate provided by Theorem~\ref{thm:labeled-trim} resembles that of Lemma~\ref{lem:conservativeness} and can be interpreted as follows: \texttt{Label-Trim} produces approximately valid conformal p-values if: (i) the labeling budget is small relative to the calibration set size, i.e., $m \leq \alpha(n+1)$; and (ii) the calibration set contains a large number of inliers, $n_0$.

However, the upper bound in Theorem~\ref{thm:labeled-trim} also suggests that \texttt{Label-Trim} may remain overly conservative, similar to standard conformal p-values, if (i) not all outliers are removed from the calibration set ($\hat{n}^{\mathrm{LT}}_{1} > 0$ with high probability), and (ii) the remaining outlier scores are generally larger than the remaining inlier scores, consistent with $\E [ \hat{F}^{\mathrm{LT}}_{1} ( \hat{Q}^{\mathrm{LT}}_{1-\alpha} ) ] < 1-\alpha$, akin to Assumption~\ref{asm:model-scores}. 

This potential conservative behavior arises naturally from the use of a limited labeling budget, especially when the model guiding the construction of the set $\D_{\mathrm{labeled}}$ fails to effectively detect true outliers. Nevertheless, as we will see in the next section, \texttt{Label-Trim} often enhances power.

\section{Experiments} \label{sec:experiments}

We turn to evaluate the performance of conformal outlier detection methods under contaminated data. The experiments presented in this section are conducted on nine benchmark datasets: three tabular datasets, listed in Section~\ref{sec:real-data-exp}, and six visual datasets, listed in \Cref{sec:img-exp}.

\paragraph{Methods} We compare the following methods:
\begin{itemize}[noitemsep, topsep=0pt]
    \item \texttt{Standard}: The basic conformal method that uses the contaminated reference set $\mathcal{D}_\text{cal}=\mathcal{D}_\text{inlier}\cup\mathcal{D}_\text{outlier}$.
    \item \texttt{Oracle}: An infeasible benchmark method where the reference set contains only inliers, i.e., $\mathcal{D}_\text{cal}=\mathcal{D}_\text{inlier}$.
    \item \texttt{Naive-Trim}: The baseline method from Section~\ref{sec:naive-trim}, which removes the top $r\%$ non-conformity scores from $\mathcal{D}_\text{cal}$, where $r = n_1 / (n_0 + n_1)$.
    \item \texttt{Label-Trim}: Our proposed reliable data-cleaning method from Section~\ref{sec:label-trim}, applied with a labeling budget of $m=50$ annotations to label the $m$ data points with the largest non-conformity scores from $\mathcal{D}_\text{cal}$.
    \item \texttt{Small-Clean}: A baseline method that uses the labeling budget to construct a small, clean reference set by (i) randomly selecting $m$ data points from $\mathcal{D}_\text{cal}$ and (ii) extracting the true inliers from this subset.
\end{itemize}

\paragraph{Setup and performance metrics} In all experiments, we randomly split a given dataset into disjoint training $\mathcal{D}_\text{train}$, calibration $\mathcal{D}_\text{cal}$, and test sets of inliers $\mathcal{D}_\text{test}^\text{inlier}$ and outliers $\mathcal{D}_\text{test}^\text{outlier}$. To simulate a realistic setting, we construct the training and contaminated calibration sets with the same contamination rate of $r\%$. The inlier $\mathcal{D}_\text{test}^\text{inlier}$ and outlier $\mathcal{D}_\text{test}^\text{outlier}$ test tests are used to compute the type-I error and power of the outlier detection model, respectively. To ensure fair comparisons, all conformal methods use the same outlier detection model, trained on $\mathcal{D}_\text{train}$. Performance metrics are evaluated across 100 random splits of the data. The size of each dataset, along with the details of how  $\mathcal{D}_\text{train}$, $\mathcal{D}_\text{cal}$, and $\mathcal{D}_\text{test}$ are constructed are provided in~\Cref{app-sec:data}. 

\subsection{Tabular Data} \label{sec:real-data-exp}

We now compare the performance of the different methods on three benchmark tabular datasets for outlier detection, previously used in the conformal literature \citep{conformal-p-values}. 
Since conclusions are similar across datasets, we focus here on results for the {\em shuttle} dataset \citep{shuttle}. Results for the {\em credit card} \citep{creditcard} and {\em KDDCup99} \citep{KDDCup99} datasets are presented in \Cref{app-sec:real-data-exp}. For all conformal methods, we use Isolation Forest~\citep{liu2008isolation} as the base outlier detection model, implemented using \texttt{scikit-learn} with default hyperparameters~\citep{sklearn_api}.

\begin{figure*}[!h]
    % \centering 
    \includegraphics[height=3.3cm, valign=t]{figures/exp/real_data/shuttle/outlier_prop/IF_e_100_s_auto_train_5000_exp_outliers_calib_shuttle_1_fdr_model_0.5_initial_50_cal_2500_p_0.05_test_1000_p_0.05_q_0.02/Type-1-Error_point_no_legend.pdf}
    \includegraphics[height=3.3cm, valign=t]{figures/exp/real_data/shuttle/outlier_prop/IF_e_100_s_auto_train_5000_exp_outliers_calib_shuttle_1_fdr_model_0.5_initial_50_cal_2500_p_0.05_test_1000_p_0.05_q_0.02/Power_point_no_legend.pdf}
    \includegraphics[height=3.3cm, valign=t]{figures/exp/real_data/shuttle/outlier_prop/IF_e_100_s_auto_train_5000_exp_outliers_calib_shuttle_1_fdr_model_0.5_initial_50_cal_2500_p_0.05_test_1000_p_0.05_q_0.02/Trimmed_point_no_legend.pdf}
    \includegraphics[width=3.3cm, valign=t]{figures/exp/legend.pdf}
    \caption{Comparison of conformal outlier detection methods on a tabular dataset (``shuttle'') as a function of the contamination rate  $r$ . The target type-I error rate is  $\alpha = 0.02$. Left: Empirical type-I error. Middle: Average detection rate (power), where higher values indicate better performance. Right: Number of outliers trimmed by the \texttt{Label-Trim} method. Results are averaged across 100 random splits of the data. 
}
    \label{fig:shuttle-outlier-prop}
\end{figure*}

\Cref{fig:shuttle-outlier-prop} presents the performance metrics of each method as a function of the contamination rate $r$. 
Following the left panel in that figure, we can see that the \texttt{Standard} conformal method results in conservative type-I error control, with a decrease in the error rate as the outlier proportion increases---a behavior that is aligned with~\Cref{lem:conservativeness}. Notably, the type-I error of the \texttt{Oracle} method is tightly centered around $\alpha$, as guaranteed by~\Cref{prop:standard-conformal}. The \texttt{Naive-Trim} method does not control the type-I error rate, emphasizing the need for reliable data-cleaning procedures. In striking contrast, our \texttt{Label-Trim} method, achieves a valid type-I error rate. At lower outlier proportions, the empirical type-I error is close to  $\alpha$, but the method becomes more conservative as the outlier proportion increases. This observation aligns with the upper bound on the error rate derived in~\Cref{thm:labeled-trim}. Notably, as the contamination rate in the training data increases, the outlier detection model’s ability to distinguish between inliers and outliers weakens. This, in turn, adversely affects the effectiveness of forming a subset of data points for annotation, as demonstrated in the right panel of \Cref{fig:shuttle-outlier-prop}. The \texttt{Small-Clean} method also controls the type-I error but is more conservative than \texttt{Label-Trim} due to its much smaller reference set, which becomes even smaller as the contamination rate increases. Observe how the power of the \texttt{Small-Clean} method is lower than that of the \texttt{Standard} approach, despite the latter using a contaminated reference set. By contrast, our proposed \texttt{Label-Trim} method significantly improves the power of the \texttt{Standard} method and even achieves near-oracle performance when the outlier proportion is low.

Next, we study the effect of the labeling budget on the performance of our \texttt{Label-Trim} method. As shown in \Cref{fig:shuttle-labeled-exp}, increasing the labeling budget brings the \texttt{Label-Trim} method closer to the \texttt{Oracle} in terms of both type-I error and power. Notably, even with a modest budget of $40$–$50$ annotations, the power of \texttt{Label-Trim} is nearly indistinguishable from that of the \texttt{Oracle}. This is attributed to the method’s effective trimming of outliers, as shown in the right panel. Notably, for labeling budgets $m > 50$, the condition in~\Cref{thm:labeled-trim} no longer holds, and yet the \texttt{Label-Trim} method still achieves valid type-I error control at level  $\alpha$ in practice. This highlights the robustness of the proposed method to the choice of $m$ beyond the restrictions specified in \Cref{thm:labeled-trim}, where we attribute this robustness to the non-adversarial nature of the outlier distribution and the underlying detection model. 

\Cref{fig:shuttle-labeled-exp} also illustrates that the \texttt{Small-Clean} method lags behind \texttt{Label-Trim} both in terms of power and conservativeness. For small labeling budgets of $m < 45$, the coarse granularity of conformal p-values \eqref{eq:conformal-p-value} renders the method powerless; the smallest achievable p-value in this case is $1/(m+1) > 0.02 = \alpha$. Even for slightly larger labeling budgets, the conservative nature of the conformal p-value---specifically, the `plus 1’ term in \eqref{eq:conformal-p-value}---continues to have a significant impact. This effect is rigorously quantified by the lower bound on type-I error provided in \Cref{prop:standard-conformal}. For instance, with $m = 80$ and $\alpha = 0.02$, the lower bound is approximately $\alpha - 1/(m+1) \approx 0.0076$, which aligns closely with the empirical error rate shown in the left panel of \Cref{fig:shuttle-labeled-exp}. Overall, these results highlight the benefits of selectively cleaning a relatively large contaminated set compared to relying on a small clean reference set, offering both improved stability and higher power.

\begin{figure*}[htb]
    % \centering 
    \includegraphics[height=3.3cm, valign=t]{figures/exp/real_data/shuttle/labeled_size/IF_e_100_s_auto_train_5000_exp_clean_calib_size_shuttle_1_fdr_model_0.5_initial_50_cal_2500_p_0.03_test_1000_p_0.05_q_0.02/Type-1-Error_point_no_legend.pdf}
    \includegraphics[height=3.3cm, valign=t]{figures/exp/real_data/shuttle/labeled_size/IF_e_100_s_auto_train_5000_exp_clean_calib_size_shuttle_1_fdr_model_0.5_initial_50_cal_2500_p_0.03_test_1000_p_0.05_q_0.02/Power_point_no_legend.pdf}
    \includegraphics[height=3.3cm, valign=t]{figures/exp/real_data/shuttle/labeled_size/IF_e_100_s_auto_train_5000_exp_clean_calib_size_shuttle_1_fdr_model_0.5_initial_50_cal_2500_p_0.03_test_1000_p_0.05_q_0.02/Trimmed_point_no_legend.pdf}
    \includegraphics[width=3.3cm, valign=t]{figures/exp/legend_wo_n.pdf}
    \caption{Comparison of conformal outlier detection methods on a tabular dataset (``shuttle'') as a function of the labeling budget $m$. The contamination rate is fixed to $r=0.03$. Other details are as in \Cref{fig:shuttle-outlier-prop}.}
    \label{fig:shuttle-labeled-exp}
\end{figure*}


Next, we examine how the target error level $\alpha$ affects the performance of different methods. \Cref{fig:shuttle-levels} shows that our \texttt{Label-Trim} method performs particularly well at low type-I error rates, especially when $\alpha$ is smaller than the contamination rate ($r=3\%$). This behavior can be explained as follows. For a relatively accurate model, the outliers primarily distort the tail of the empirical distribution of nonconformity scores---see \Cref{fig:scores-shuttle}. Consequently, the influence of these outliers on the rejection rule $\hat{p}_{n+1} \leq \alpha$ from \eqref{eq:conformal-p-value}, or $\hat{p}^{\text{LT}}_{n+1} \leq \alpha$ from \eqref{eq:LT-p-value}, diminishes as $\alpha$ increases.

\begin{figure*}[htb]
    \centering 
    \includegraphics[height=3.3cm, valign=t]{figures/exp/real_data/shuttle/levels/IF_e_100_s_auto_train_5000_exp_levels_shuttle_1_fdr_model_0.5_initial_50_cal_2500_p_0.03_test_1000_p_0.05_q_0.02/Type-1-Error_point_no_legend.pdf}
    \includegraphics[height=3.3cm, valign=t]{figures/exp/real_data/shuttle/levels/IF_e_100_s_auto_train_5000_exp_levels_shuttle_1_fdr_model_0.5_initial_50_cal_2500_p_0.03_test_1000_p_0.05_q_0.02/Power_point_no_legend.pdf}
    \includegraphics[width=3.3cm, valign=t]{figures/exp/legend_wo_trm.pdf}
    \caption{Comparison of conformal outlier detection methods on a tabular dataset (``shuttle'') as a function of the target type-I error rate $\alpha$. The contamination rate $r$ is fixed to 3\%. Other details are as in \Cref{fig:shuttle-outlier-prop}.
}
    \label{fig:shuttle-levels}
\end{figure*}

\subsection{Visual Data}\label{sec:img-exp}

In what follows, we compare all methods using benchmark visual datasets for outlier detection. Similar to \citet{zhang2023openood}, we construct six datasets, where the inlier samples are always images from CIFAR10~\citep{cifar-10, cifar-data} and the outlier samples vary across datasets. Specifically, the outliers are drawn from (1) MNIST~\citep{deng2012mnist}, (2) SVHN~\citep{svhn}, (3) Texture~\citep{texture}, (4) Places365~\citep{texture}, (5) TinyImageNet~\citep{tinyimages}, and (6) CIFAR100~\citep{cifar-data}. For all datasets, we use the outlier detection model proposed by \citet{react}, which operates on feature representations extracted by a pre-trained ResNet18 model. More details are in~\Cref{app-sec:data}.

\Cref{tab:avg-images} summarizes the results for all six datasets. Overall, we can see a trend similar to the one of the tabular data: the \texttt{Standard} and \texttt{Small-Clean} methods are valid but conservative, the \texttt{Naive-Trim} fails to control the type-I error, and our \texttt{Label-Trim} achieves a significant boost in power while practically controlling the type-I error. Notably, our \texttt{Label-Trim} method attains near-oracle performance for low contamination rates. Detailed results for each dataset are provided in \Cref{app-sec:images-data-exp}.

\begin{table*}[!htb]
\caption{Comparison of conformal outlier detection methods on six visual datasets for varying contamination rate $r$ and target type-I error level $\alpha$. The empirical type-I error values are averaged across all datasets. The empirical power is presented relative to the \texttt{Standard} method (higher is better), and averaged across all datasets. Results are averaged across 100 random splits of the data, with standard errors presented in parentheses.
}
\label{tab:avg-images}
\centering
\resizebox{\textwidth}{!}{
\begin{tabular}{l|ll|ll|ll}
\hline
& \multicolumn{6}{c}{Contamination rate} \\
\hline
 & \multicolumn{2}{c|}{1\%} & \multicolumn{2}{c|}{3\%} & \multicolumn{2}{c}{5\%} \\ \hline
 Method      & Power & Type-I Error & Power & Type-I Error & Power & Type-I Error \\ \hline
Standard & \bfseries \cellcolor{Green!30} 1.0 ($\pm$ 0.0317) & \cellcolor{white} 0.008 ($\pm$ 0.0003)  & \bfseries \cellcolor{Green!30} 1.0 ($\pm$ 0.0354) & \cellcolor{white} 0.005 ($\pm$ 0.0003)  & \bfseries \cellcolor{Green!30} 1.0 ($\pm$ 0.0408) & \cellcolor{white} 0.004 ($\pm$ 0.0002) \\

Oracle (infeasible) & \bfseries \cellcolor{Green!100} 1.166 ($\pm$ 0.0336) & \cellcolor{white} 0.01 ($\pm$ 0.0003)  & \bfseries \cellcolor{Green!100} 1.549 ($\pm$ 0.0425) & \cellcolor{white} 0.01 ($\pm$ 0.0003)  & \bfseries \cellcolor{Green!100} 1.961 ($\pm$ 0.0531) & \cellcolor{white} 0.009 ($\pm$ 0.0004) \\

Naive-Trim (invalid) & \cellcolor{red!20} 1.659 ($\pm$ 0.0342) & \cellcolor{red!20} 0.017 ($\pm$ 0.0004)  & \cellcolor{red!20} 2.79 ($\pm$ 0.045) & \cellcolor{red!20} 0.027 ($\pm$ 0.0006)  & \cellcolor{red!20} 4.16 ($\pm$ 0.0596) & \cellcolor{red!20} 0.036 ($\pm$ 0.0007) \\

Small-Clean & \cellcolor{white} 0.0 ($\pm$ 0.0) & \cellcolor{white} 0.0 ($\pm$ 0.0)  & \cellcolor{white} 0.0 ($\pm$ 0.0) & \cellcolor{white} 0.0 ($\pm$ 0.0)  & \cellcolor{white} 0.0 ($\pm$ 0.0) & \cellcolor{white} 0.0 ($\pm$ 0.0) \\

Label-Trim & \bfseries \cellcolor{Green!100} 1.166 ($\pm$ 0.0336) & \cellcolor{white} 0.01 ($\pm$ 0.0003)  & \bfseries \cellcolor{Green!60} 1.517 ($\pm$ 0.042) & \cellcolor{white} 0.01 ($\pm$ 0.0003)  & \bfseries \cellcolor{Green!60} 1.786 ($\pm$ 0.0498) & \cellcolor{white} 0.008 ($\pm$ 0.0003) \\
\end{tabular}
}
\subcaption{Target type-I error rate $\alpha=0.01$}

\resizebox{\textwidth}{!}{
\begin{tabular}{l|ll|ll|ll}
\hline
& \multicolumn{6}{c}{Contamination rate} \\
\hline
 & \multicolumn{2}{c|}{1\%} & \multicolumn{2}{c|}{3\%} & \multicolumn{2}{c}{5\%} \\ \hline
 Method      & Power & Type-I Error & Power & Type-I Error & Power & Type-I Error \\ \hline
Standard & \bfseries \cellcolor{Green!30} 1.0 ($\pm$ 0.0174) & \cellcolor{white} 0.027 ($\pm$ 0.0006)  & \bfseries \cellcolor{Green!30} 1.0 ($\pm$ 0.0189) & \cellcolor{white} 0.019 ($\pm$ 0.0005)  & \cellcolor{white} 1.0 ($\pm$ 0.0212) & \cellcolor{white} 0.015 ($\pm$ 0.0005) \\

Oracle (infeasible) & \bfseries \cellcolor{Green!100} 1.062 ($\pm$ 0.0176) & \cellcolor{white} 0.03 ($\pm$ 0.0006)  & \bfseries \cellcolor{Green!100} 1.235 ($\pm$ 0.0192) & \cellcolor{white} 0.029 ($\pm$ 0.0006)  & \bfseries \cellcolor{Green!100} 1.448 ($\pm$ 0.023) & \cellcolor{white} 0.03 ($\pm$ 0.0007) \\

Naive-Trim (invalid) & \cellcolor{red!20} 1.146 ($\pm$ 0.0175) & \cellcolor{red!20} 0.035 ($\pm$ 0.0006)  & \cellcolor{red!20} 1.487 ($\pm$ 0.0186) & \cellcolor{red!20} 0.043 ($\pm$ 0.0007)  & \cellcolor{red!20} 1.882 ($\pm$ 0.0224) & \cellcolor{red!20} 0.052 ($\pm$ 0.0008) \\

Small-Clean & \cellcolor{white} 0.714 ($\pm$ 0.0448) & \cellcolor{white} 0.02 ($\pm$ 0.0021)  & \cellcolor{white} 0.869 ($\pm$ 0.0501) & \cellcolor{white} 0.02 ($\pm$ 0.002)  & \bfseries \cellcolor{Green!30} 1.033 ($\pm$ 0.0613) & \cellcolor{white} 0.021 ($\pm$ 0.0023) \\

Label-Trim & \bfseries \cellcolor{Green!60} 1.041 ($\pm$ 0.0177) & \cellcolor{white} 0.029 ($\pm$ 0.0006)  & \bfseries \cellcolor{Green!60} 1.139 ($\pm$ 0.019) & \cellcolor{white} 0.025 ($\pm$ 0.0006)  & \bfseries \cellcolor{Green!60} 1.215 ($\pm$ 0.0226) & \cellcolor{white} 0.021 ($\pm$ 0.0006) \\
\end{tabular}
}
\subcaption{Target type-I error rate $\alpha=0.03$}
\end{table*}
% \FloatBarrier
\section{Discussion}

In this work, we studied the robustness of conformal prediction under contaminated reference data. Motivated by empirical evidence, we characterized the conditions under which conformal outlier detection methods become too conservative. To improve power, we proposed the \texttt{Label-Trim} method, which leverages an outlier detection model and a limited labeling budget to remove outliers from the contaminated reference set. We also provided a theoretical justification for this approach, employing novel proof techniques. Numerical experiments with real data confirmed that standard conformal outlier detection methods are conservative under contaminated data and demonstrated that our \texttt{Label-Trim} method can significantly enhance power.

However, the experiments also reveal a limitation of our \texttt{Label-Trim} method: while it improves power compared to standard conformal inference, it often remains too conservative, particularly when the labeling budget is very limited, leaving room for further improvement. A promising direction for future research is to enhance \texttt{Label-Trim} with {\em active learning} strategies \cite{makili2012active,fannjiang2022conformal,prinsterconformal}, enabling the removal of more outliers without increasing the labeling budget.

A second limitation of the \texttt{Label-Trim} approach is its reliance on actively collecting new annotations. In scenarios where a flexible labeling budget is unavailable but access to a small, clean reference set is feasible, this dependency becomes restrictive. As our experiments demonstrate, the limited sample size imposes a fundamental constraint on the power of conformal outlier detection methods. This raises an intriguing question for future research: given a small clean reference set and a larger, contaminated reference set, how can we effectively and safely clean the contaminated data to enhance detection power at test time?

One potential solution could involve using the small clean reference set to calibrate a base outlier detection model. This calibrated model could then be employed to clean the larger contaminated set by removing detected outliers, while carefully accounting for inliers mistakenly classified as outliers. Exploring such a semi-supervised data-cleaning approach represents a promising direction for future work, though we anticipate that establishing the theoretical validity of such a method may not be straightforward.

\section*{Acknowledgments}
M.~S.~was partly supported by NSF grant DMS 2210637 and by a Capital One CREDIF Research Award.
Y.~R. and M.~B. were funded by the European Union (ERC, SafetyBounds, 101163414). Views and opinions expressed are however those of the authors only and do not necessarily reflect those of the European Union or the European Research Council Executive Agency. Neither the European Union nor the granting authority can be held responsible for them.


% \bibliography{bib}
\documentclass{MITstyle}

%\usepackage[table]{xcolor}
\usepackage{chngcntr}
\usepackage{hyperref}
\usepackage{microtype}

\title{A Lightweight and Extensible Cell Segmentation and Classification Model for Whole Slide Images}

\author{Nikita Shvetsov~$^{1, }$\footnote{Correspondence e-mail: nikita.shvetsov@uit.no}, Thomas K. Kilvaer~$^{2, 3}$, Masoud Tafavvoghi~$^{4}$, Anders Sildnes~$^{1}$, \\ Kajsa Møllersen~$^{4}$, Lill-Tove Rasmussen Busund~$^{5, 6}$, Lars Ailo Bongo~$^{1}$ \\
%
\vspace{1em} % Space between authors and afilliations
%
\normalfont{\small $^{1}$Department of Computer Science, UiT The Arctic University of Norway}\\
\normalfont{\small $^{2}$Department of Oncology, University Hospital of North Norway}\\
\normalfont{\small $^{3}$Department of Clinical Medicine, UiT The Arctic University of Norway}\\
\normalfont{\small $^{4}$Department of Community Medicine, UiT The Arctic University of Norway}\\
\normalfont{\small $^{5}$Department of Medical Biology, UiT The Arctic University of Norway} \\
\normalfont{\small $^{6}$Department of Clinical Pathology, University Hospital of North Norway} %\vspace{2em}
}

\begin{document}
\maketitle

\section*{Abstract}

% \begin{abstract}
% Developing clinically useful cell-level analysis tools in digital pathology remains challenging due to limitations in dataset granularity, inconsistent annotations, computational demands of advanced models, and difficulties in integrating new technologies into clinical workflows. To address these challenges, we propose a multi-faceted solution that enhances data quality, model performance, and usability to create a lightweight and extensible cell segmentation and classification model.

% First, we update data labels by employing a cross-relabeling process that refines the labels of two existing datasets, PanNuke and MoNuSAC, to create a new unified dataset with enhanced granularity, encompassing seven distinct cell types. Second, we leverage the H-Optimus foundation model as a fixed encoder to improve feature representation for simultaneous cell segmentation and classification tasks. Third, to address the computational demands of foundation models, we employ knowledge distillation to reduce model size and complexity while maintaining comparable performance. Finally, to facilitate integration into clinical workflows, we integrate the distilled model into the QuPath software, a widely used open-source platform in digital pathology.

% Our results demonstrate improvements in cell segmentation and classification performance using the H‑Optimus-based model compared to a CNN-based model. Specifically, the average $R^2$ improved from 0.575 to 0.871, and the average $PQ$ score improved from 0.450 to 0.492, indicating better alignment with actual cell counts and enhanced segmentation and classification quality. Furthermore, the distilled student model maintains performance comparable to the larger foundation model while reducing the parameter count by a factor of 48.
% Overall, by reducing computational complexity and integrating it into existing workflows, the proposed approach may significantly impact diagnostic processes, reduce the workload of pathologists, and contribute to improved patient outcomes. Though our approach shows potential enhancements in efficiency and usability of cell segmentation and classification models in digital pathology, extensive validation is needed to deploy these models in clinical practice.
% \end{abstract}

%%% shortened abstract
\begin{abstract}
Developing clinically useful cell-level analysis tools in digital pathology remains challenging due to limitations in dataset granularity, inconsistent annotations, high computational demands, and difficulties integrating new technologies into workflows. To address these issues, we propose a solution that enhances data quality, model performance, and usability by creating a lightweight, extensible cell segmentation and classification model. 

First, we update data labels through cross-relabeling to refine annotations of PanNuke and MoNuSAC, producing a unified dataset with seven distinct cell types. Second, we leverage the H-Optimus foundation model as a fixed encoder to improve feature representation for simultaneous segmentation and classification tasks. Third, to address foundation models' computational demands, we distill knowledge to reduce model size and complexity while maintaining comparable performance. Finally, we integrate the distilled model into QuPath, a widely used open-source digital pathology platform. 

Results demonstrate improved segmentation and classification performance using the H-Optimus-based model compared to a CNN-based model. Specifically, average $R^2$ improved from 0.575 to 0.871, and average $PQ$ score improved from 0.450 to 0.492, indicating better alignment with actual cell counts and enhanced segmentation quality. The distilled model maintains comparable performance while reducing parameter count by a factor of 48. By reducing computational complexity and integrating into workflows, this approach may significantly impact diagnostics, reduce pathologist workload, and improve outcomes. Although the method shows promise, extensive validation is necessary prior to clinical deployment.
\end{abstract}
\clearpage

\section{Introduction}
In digital pathology, accurate segmentation and classification of cells are crucial for many diagnostic, prognostic, and predictive analyses \cite{Jaber_Beziaeva_etal._2019,Lin_Pan_etal._2022,Park_Ock_etal._2022,Shen_Choi_etal._2024}. Nowadays, developments in computational pathology offer multiple solutions \cite{H._Qu_P._Wu_etal._2020,Javed_Mahmood_etal._2020} to utilize cell-level datasets to train machine learning models that solve these problems. The quality and specificity of training datasets are critical for robust and accurate models. Adhering to the principle of "garbage in, garbage out", it is essential to ensure that these datasets are extensively and accurately labeled with distinct classes that reflect the diverse biological characteristics of different cell types. Unfortunately, the number of open-source datasets comprising such high-quality annotations is limited. Existing cell segmentation datasets \cite{Gamper_Koohbanani_etal._2019,Graham_Vu_etal._2019,Verma_Kumar_etal._2021} may offer extensive annotations for certain cell types while providing more general labels for others. For example, in PanNuke, which is one of the largest open-source datasets comprising labeled cells, various types of morphologically and functionally different inflammatory cells like macrophages and lymphocytes are clustered in a broad "inflammatory" class. Consequently, these classes are frequently omitted from analyses or aggregated into broader meta-classes \cite{Gamper_Koohbanani_etal._2020} and likely interfere with other cell classes included in the dataset. This and similar inconsistencies in annotation granularity limit the ability of machine learning models to learn the comprehensive and nuanced features necessary for accurate cell segmentation and classification. To address these challenges, methods for refining and standardizing dataset annotations are essential to enhance the quality of training data.

A complementary approach to mitigate the absence of high-quality training data is the use of foundation models. Foundation models as encoders are defined as large-scale, versatile networks pre-trained on vast, diverse datasets using self-supervised learning, contrasting with convolutional neural network (CNN) pre-trained encoders that rely on supervised learning with labeled data. In practice, foundation models leverage enormous amounts of weakly or unlabeled data from millions of whole slide images (WSIs) and employ self-attention mechanisms to capture long-range dependencies and global context \cite{Chen_Ding_etal._2024,Saillard_Jenatton_etal._2024,Vorontsov_Bozkurt_etal._2024,Xu_Usuyama_etal._2024}. As a consequence, foundation models are able to produce transferable feature representations across different cell types and tissue environments. The feature representations can be leveraged by decoder networks to produce segmentation masks and pixel-level classifications. Because foundation models have comprehensive feature representations, they can be effectively fine-tuned using much smaller amounts of cell-level data compared to the large datasets needed to train models from scratch. Furthermore, foundation models incorporate adversarial training elements or contrastive learning \cite{Chen_Ding_etal._2024,Xu_Usuyama_etal._2024}, enhancing their resilience and adaptability by exposing them to challenging and varied scenarios during training. This may result in more generalizable models, often making them well-suited for diverse and complex tasks in digital pathology.

Despite the inherent advantages of foundation models, their deployment for practical use faces its own obstacles. In particular, they require substantial computational power, financial investments and rigorous testing to ensure reliability and efficacy for a given task \cite{Akkus_Dangott_etal._2022,Dragomir_Cocuz_etal._2022,Go_2022,Jafri_Farooqui_etal._2024}. Moreover, while foundation models enhance feature representation and performance, they depend on the quality of available annotations for decoder fine-tuning and, like any other model, cannot resolve existing inconsistencies or ambiguities in data labels. Therefore, there remains a critical need for solutions that address both data quality and practical deployment considerations.
Further, integrating new technologies into existing clinical workflows often encounters resistance, as it necessitates adjustments to established diagnostic processes. So, there is a need to develop solutions that could be integrated into current practices, minimizing the burden on medical professionals to adopt new tools \cite{King_Williams_etal._2023}.

Existing solutions \cite{Goldsborough_Philps_etal._2024,Hörst_Rempe_etal._2024}, while addressing some aspects of these challenges, fall short in providing a comprehensive approach. To address the data quality and clinical deployment issues, we propose a multi-faceted solution that encompasses data refinement, model optimization, and integration with existing pathology tools (\hyperref[fig:fig1]{Figure 1}). The outcome is a lightweight cell segmentation and classification model that can be integrated into digital pathology workflows for practical clinical use.

\begin{figure}[h!]
    \centering
    \includegraphics[width=\textwidth, height=0.82\textheight, keepaspectratio]{images/Figure_1.pdf}
    \caption{Overview of the proposed solution, including 1) Data refinement using cross-relabeling, 2) Teacher model development and fine tuning, 3) Student model optimization with knowledge distillation and 4) Student model and QuPath integration}
    \label{fig:fig1}
\end{figure}
\clearpage

Our approach begins with preparing the data for the fine-tuning and training of the machine learning models. We create a refined dataset, acquired via cross-relabeling two cell-level datasets, enhancing annotation specificity and consistency of the labeled data. Subsequently, we create a cell segmentation and classification model based on the foundation model. We leverage the foundation model as a fixed encoder and fine-tune a decoder using the refined dataset to improve generalization across diverse tissue- and cell types.
To ensure that the model remains lightweight and deployable in a possibly resource-constrained environment, we employ knowledge distillation to approximate the functionality of the foundation model. Finally, to facilitate the practical application of our model in digital pathology workflows, we integrate it with the QuPath \cite{Bankhead_Loughrey_etal._2017} application. Each methodological component contributes to the overarching goal of enhancing model performance, generalizability, and usability in clinical settings.

The primary contributions of this paper are:
\begin{enumerate}
    \item \textit{Data labels refinement through cross-relabeling:}
    
    We propose a new method for refining labels of cell-level datasets through cross-relabeling. This method employs classification models to re-label broad and ambiguous instances, resulting in a more diverse dataset. Our evaluation demonstrates that these classification models achieve high accuracy on test subsets, indicating the reliability of the method for label refinement.

    \item \textit{Enhanced model performance via foundation models:}
    
    We employ a foundation model as a feature extractor for the cell segmentation and classification task. In comparison with training a CNN model from scratch, the foundation model backbone only needs fine-tuning, which significantly reduces training time, computational resources and data requirements. We show that using a foundation model encoder leads to better performance in cell segmentation and classification networks than using a CNN-based encoder. This improvement may enable the model to generalize more effectively across various tissue types and imaging methods.
    
    \item \textit{Model optimization through knowledge distillation:}
    
    We show that a smaller student model trained using knowledge distillation on the refined dataset obtained via our cross-relabeling approach from a foundation model achieves comparable performance in cell segmentation and quantification tasks. As a result, this model is more suitable for deployment in environments without high-performance computing resources.
    
    \item \textit{Integration with QuPath:}
    
    We integrate the distilled cell segmentation and classification model into QuPath, a widely used open-source digital pathology platform, to accelerate clinical adaptation by enabling pathologists to more easily incorporate advanced computational tools into their existing workflows.
\end{enumerate}

Through these methodological steps, we aim to bridge the gap between advanced machine learning techniques and practical clinical applications, making accurate and efficient digital pathology accessible in a broader range of healthcare settings.

\section{Refining Existing Datasets Using Cross-Relabeling}
To address the limitations of sparse and ambiguous labeling of cell-level datasets, we propose a generalizable cross-relabeling strategy that can be applied to any dataset containing broadly categorized or imprecisely labeled cell types. This approach involves training and subsequently leveraging classification models to refine broad categories into more specific or biologically relevant classes.
When applied to cell-level data, the methodology includes extracting individual cell images from the dataset patches, preprocessing these images to standardize the size and accommodate partial cells, and then training deep learning classifiers capable of distinguishing between the finer cell subtypes within the coarser categories. 
To illustrate our approach, we focus on the PanNuke \cite{Gamper_Koohbanani_etal._2020, Gamper_Koohbanani_etal._2019} and MoNuSAC \cite{Verma_Kumar_etal._2021} datasets that we have used to train models for cell quantification in our previous works \cite{Shvetsov_Grønnesby_etal._2022,Shvetsov_Sildnes_etal._2024}. We find that for better cell differentiation we have to introduce more granular labels. PanNuke includes a broad classification of "inflammatory" cells, encompassing lymphocytes, macrophages, and neutrophils. Each cell type differs significantly in structure, function, and clinical relevance. Conversely, MoNuSAC uses the label "epithelial" for a class that comprises both benign epithelial cells and malignant neoplastic cells. This practice makes it challenging to differentiate between benign and malignant epithelial cells in the dataset, which is a critical distinction when identifying tumor areas within tissue samples. To address these issues, we implement a cross-relabeling strategy as shown in \hyperref[fig:fig2]{Figure 2}. The key components are two classification models: one is trained on singular cell images from PanNuke data to classify the epithelial meta-class into epithelial and neoplastic classes. The other is trained on MoNuSAC to refine the inflammatory class into lymphocytes, neutrophils, and macrophages.

\begin{figure}[h!]
    \centering
    \includegraphics[width=\textwidth]{images/Figure_2.pdf}
    \caption{Refined dataset generation via cross relabeling}
    \label{fig:fig2}
\end{figure}

The refining approach consists of three consecutive steps. The first is the preprocessing step, in which we extract individual cells from both datasets (\hyperref[fig:fig3]{Figure 3}). The specifics of PanNuke and MoNuSAC patch preparation before cell preprocessing are provided in \hyperref[chap:S1]{Appendix S1}.

\begin{figure}[h!]
    \centering
    \includegraphics[width=\textwidth]{images/Figure_3.pdf}
    \caption{Cell instances preprocessing including (1) cell map extraction, (2) bounding box delineation, (3) adjusting cell boxes and (4) cropping and resizing of cell images}
    \label{fig:fig3}
\end{figure}

During preprocessing, we extract cell type maps from the ground truth label mask and calculate bounding boxes around each cell instance. To accommodate partial cells at patch borders, a common issue in cropped patch images, we employ mirror padding and extend the field of view of the cell label by 15 pixels to capture adjacent cells. We then crop and resize the identified regions to $64 \times 64$ pixels using bicubic interpolation.

The preprocessed PanNuke dataset comprises 68,031 neoplastic and 23,207 epithelial cell images, while MoNuSAC comprises  33,104 lymphocytes, 1,252 neutrophils, and 1,695 macrophages, which we subsequently use in training cell classification models and classifying the cell image data \hyperref[fig:S2]{Appendix Figure S2 (1)}. 

The next step is to train two distinct ResNet50-based classifiers tailored to address the specific labeling challenges inherent in each dataset. We use ResNet50 for classification models due to its proven effectiveness for image classification tasks in histopathology \cite{pan2022reviewmachinelearningapproaches}, and its compatibility with small images. For the PanNuke dataset, we design the classifier, trained on MoNuSAC data, to disaggregate the heterogeneous "inflammatory" cell category into distinct subtypes: lymphocytes, macrophages, and neutrophils. Similarly, for the MoNuSAC dataset, the classifier is trained on PanNuke data and distinguishes between benign and malignant epithelial cells within the overarching "epithelial" label. By applying these targeted classifiers to their respective datasets, we assign more specific labels to individual cell instances, thus enabling us to create a unified dataset.
To ensure a balanced representation of classes, we train both models on datasets that had been equalized to match the size of the least represented class. Thus, we obtain datasets comprising 23,207 samples per class for PanNuke and 1,252 samples per class for MoNuSAC data. Next, we partition both of them into training (70\%), validation (20\%), and testing (10\%) subsets. To mitigate the risk of overfitting, we use a single dropout layer with a rate of p=0.5 in both models and data augmentation using randomized color perturbations, rotation, and horizontal and vertical flipping. We employ AdamW optimizer and the cross-entropy loss function for the training criterion.

To evaluate the two trained models, we measure the classification accuracy on the respective test subsets. The accuracies on the test subset for both classifiers are presented in \hyperref[tab:1]{Table 1}. The PanNuke model achieves an average accuracy of 93.57\%, with higher accuracy for neoplastic cells (96.06\%) compared to epithelial cells (86.26\%). The confusion matrix in Figure A3.1 shows that the model predominantly distinguishes accurately between epithelial and neoplastic tissues, with a substantial number of correct classifications and relatively few misclassifications. The MoNuSAC model demonstrates an average accuracy of 98.92\%, excelling in classifying lymphocytes (99.67\%) and macrophages (94.12\%), with lower performance for neutrophils (85.71\%). The confusion matrix in Figure A3.2 shows that the model identifies lymphocytes and performs reasonably well with macrophages and neutrophils.

\begin{table}[h!]
\renewcommand{\arraystretch}{1.5}
  \centering
  \caption{Cell classification results for PanNuke and MoNuSAC trained models (CI 95\%).}
  \label{tab:1}
  \begin{tabular}{|l|c|c|}
   \hline
   %\rowcolor{gray!30}
    Accuracy               & PanNuke model              & MoNuSAC model              \\
    \hline
    Average      & 0.936 (0.931--0.941)         & 0.989 (0.986--0.993)        \\
    \hline
    Neoplastic   & 0.961 (0.956--0.965)         & -                          \\
    \hline
    Epithelial   & 0.863 (0.849--0.877)         & -                          \\
    \hline
    Lymphocytes  & -                          & 0.997 (0.995--0.999)        \\
    \hline
    Neutrophils  & -                          & 0.857 (0.796--0.918)        \\
    \hline
    Macrophages  & -                          & 0.941 (0.906--0.976)        \\
    \hline
  \end{tabular}
\end{table}

Finally, during the last step, we use the model trained on PanNuke data for epithelial cells in MoNuSAC and the model trained on MoNuSAC for the inflammatory cells class in PanNuke. Specifically, we use classifier models to relabel epithelial cells in MoNuSAC and inflammatory cells in PanNuke data. Then we combine cells with refined labels and the rest of the cells in both datasets to create a refined dataset (\hyperref[fig:S2]{Appendix Figure S2 (2)}). The process of relabeling cells and visualizing them on a patch is shown in \hyperref[fig:fig4]{Figure 4}. The cell counts in the refined dataset are provided in \hyperref[tab:S4]{Appendix Table S4}.

\begin{figure}[h!]
    \centering
    \includegraphics[width=\textwidth, height=0.42\textheight, keepaspectratio]{images/Figure_4.pdf}
    \caption{Cell relabeling procedure for epithelial and inflammatory cell classes}
    \label{fig:fig4}
\end{figure}

%\hfill

Relabeling and combining datasets have been explored in a prior study \cite{Parulekar_Kanwat_etal._2023}, where consecutive fine-tuning on multiple datasets was employed to account for hierarchical class label structures. While the method presented in \cite{Parulekar_Kanwat_etal._2023} is intuitive, it often lacks consistency and requires multiple fine-tuning runs, which can be cumbersome and time-consuming. 
In contrast, cross-relabeling simplifies this process by using specialized classification models tailored to each dataset's specific labeling challenges. This approach provides better transparency and produces a unified dataset encompassing seven distinct cell types across multiple tissue samples, enhancing data diversity for further model training or fine-tuning.

Despite these improvements, cross-relabeling does not entirely resolve issues related to poor labeling quality or the amount of labeled data. Specifically, our results show lower accuracies persist for underrepresented classes, such as macrophages, which may stem from a limited sample availability and intrinsic challenges in distinguishing these cells based solely on H\&E staining. Furthermore, while our method enhances label specificity, it relies on the initial quality of the broad labels; thus, any fundamental inaccuracies in the original annotations can propagate through the relabeling process. Addressing the overall problem of limited data labels may require integrating additional data sources or utilizing complementary immunohistochemical staining methods.
Although the reported performance metrics are obtained from evaluations on the native test sets of each dataset, it is important to note that the primary application of these classifiers is to perform cross-relabeling, where a model trained on one dataset (e.g., PanNuke) is applied to another (e.g., MoNuSAC) and vice versa. We acknowledge that a more systematic evaluation of cross-dataset generalization is needed and could be performed in future work.

Overall, the refined dataset produced by our approach can enhance the supervised training or fine-tuning of cell segmentation and classification models, especially those that utilize pre-trained foundation models to improve feature extraction robustness. In addition, these models can detect nuanced classes that enable researchers to conduct more detailed analyses of biological processes in computational pathology.

\section{Foundation models for robust cell segmentation and classification}

Accurate cell segmentation and classification in digital pathology are hindered by limited labeled data and the fact that conventional CNNs are unable to capture global contextual information due to their local receptive field constraints \cite{Gheflati_Rivaz_2022,Yang_Marcus_etal.}. Traditional approaches in cell quantification have predominantly relied on CNN encoders, such as ResNet50, given their proven effectiveness in semantic segmentation tasks \cite{Deshmane_2023,Graham_Vu_etal._2019,Mukasheva_Koishiyeva_etal._2024,Stringer_Wang_etal._2021}. However, approaches that include fine-tuning of pretrained CNNs, data augmentation, and stain normalization to partially increase data variability and address staining differences often fail to achieve the necessary generalization and robustness across diverse tissue types and staining conditions \cite{G._Wang_W._Li_etal._2018,Gao_Bagci_etal._2018,Karim_El_Khoury_Martin_Fockedey_etal._2021}.

To overcome these challenges, we leverage an encoder-decoder network that uses a foundation model as the encoder and a CNN upsampling decoder (\hyperref[fig:fig5]{Figure 5}) for simultaneous cell segmentation and classification in 2D patches extracted from WSIs. Foundation models with transformer-based architectures are viable alternatives to CNN-based encoders \cite{Shamshad_Khan_etal._2023,Sourget_2023}. They enable the creation of more advanced architectures that can decode or transform learned features more effectively \cite{Chen_Duan_etal._2023,Cheng_Misra_etal._2022,Xie_Wang_etal._2021}.

\begin{figure}[h!]
    \centering
    \includegraphics[width=\textwidth]{images/Figure_5.pdf}
    \caption{UNETR-like model with foundational model as backbone}
    \label{fig:fig5}
\end{figure}

By utilizing a transformer-based encoder, we incorporate global contextual information into the feature extraction process, which is a key advantage of such architectures \cite{Chen_Lu_etal._2021}. This foundation model integration facilitates accurate pixel-wise segmentation and classification without the need for extensive encoder training, thereby potentially improving generalization across varied cellular structures and tissue types.
In our implementation, we employ a modified UNETR \cite{Hatamizadeh_Tang_etal._2021} architecture that combines a vision transformer (ViT) \cite{Dosovitskiy_Beyer_etal._2021} encoder with a CNN-based decoder. The encoder utilizes the pretrained H-Optimus foundation model, which contains 1.1 billion parameters and is trained on over 500,000 H\&E stained WSIs \cite{Saillard_Jenatton_etal._2024}. We extract outputs from four evenly spaced transformer blocks $Z_i$, where $i \in [1, 14, 26, 38]$, to serve as residual connections for the CNN decoder. We select these blocks based on our observation that features from non-adjacent levels of the encoder lead to better overall performance on the test subset.

The CNN decoder upsamples the feature representations, acquired from the transformer blocks, to generate an intermediate vector that is handled by two task-specific layers that generate cell segmentation and classification masks. The first task-specific layer is the ‘Cellpose head’,  which is used to delineate cell instances. The layer generates horizontal and vertical gradient maps to form vector fields that are refined through gradient tracking in a post-processing step using the Cellpose algorithm \cite{Stringer_Wang_etal._2021}, known for its efficacy in cell segmentation tasks and generalizability across multiple domains \cite{Pachitariu_Stringer_2022,Stringer_Pachitariu_2024}. The second task-specific layer is the "Cell type head", which assigns labels to individual pixels. In the post-processing step, we determine the output classification label of each segmented cell instance by majority voting over the labeled pixels that comprise the cell in the segmentation map.

To evaluate model performance and measure the impact of adding a foundation model as backbone, we compare it to a ResNet50-based model. ResNet50 is a widely used solution for encoders in segmentation architectures in the medical domain \cite{Deshmane_2023,Graham_Vu_etal._2019,Mukasheva_Koishiyeva_etal._2024,Stringer_Wang_etal._2021}. For the H-Optimus-based model, we utilize frozen weights for the encoder and only fine-tune the decoder to take advantage of the extensive pre-training of the foundation model. For the ResNet50-based model we start with ImageNet \cite{Deng_Dong_etal.} weights and train both encoder and decoder parts. Hyperparameters for the training step are set to be identical, where possible, for comparable evaluation. 
For this evaluation, we deliberately use the PanNuke dataset to provide a standardized and controlled comparison between the H‑Optimus and ResNet50-based models (\hyperref[fig:S2]{Appendix Figure S2 (3)}). Specifically, we use two of the default PanNuke dataset splits (66\%) for training and validation, and reserve the third split (33\%) for testing.

To address the challenge of cell class imbalance in the PanNuke dataset, which is a common characteristic in most cell-level H\&E patch datasets, both models’ training processes employ a weighted loss function comprising cross-entropy and focal loss \cite{Lin_Goyal_etal._2018}. The focal loss component is adjusted with coefficients derived from each cell class' instance frequency, emphasizing learning from underrepresented classes and enhancing the model's sensitivity to rare but significant cellular patterns. The cross-entropy loss is augmented with spectral decoupling regularization \cite{Pezeshki_Kaba_etal._2021,Pohjonen_Stürenberg_etal._2022} and spatially varying label smoothing \cite{Islam_Glocker_2021}, which potentially stabilizes training and improves generalization in case of complex tissue morphologies. For optimization, we employ the \textit{AdamW} \cite{Loshchilov_Hutter_2019} to counter unbalanced class scenarios, with cosine annealing learning rate scheduler.

We utilize the scikit-learn library \cite{Van_der_Walt_Schönberger_etal._2014} and HoVer-Net \cite{Graham_Vu_etal._2019} implementations of $R^2$ (the coefficient of determination) and $PQ$ (panoptic quality) to evaluate our experiments. Complete mathematical formulations and detailed explanations of these metrics are provided in \hyperref[chap:S5]{Appendix S5}. To compute confidence intervals, we use nonparametric bootstrapping, where after calculating the metric on the full sample, we generated 1000 bootstrap replicates by resampling with replacement and then determined the 95\% confidence intervals as the 2.5th and 97.5th percentiles of the resulting empirical distribution.

%\hfill

The model comparisons are summarized in \hyperref[tab:2]{Table 2}. The H‑Optimus-based model achieves higher $R^2$ across all cell classes compared to the ResNet50-based model, which means that its predictions are more closely aligned with the PanNuke cell counts, indicating a stronger correlation with the observed data. Notably, the improvement of $R^2_{dead}$ may be an indicator of better global contextual representations provided by the foundation model backbone. In terms of segmentation and classification quality combined, measured by the PQ score, the H‑Optimus-based model demonstrates notable improvements across most cell classes. Overall, the average $R^2$ improved from 0.575 to 0.871, while the average $PQ$ score improved from 0.450 to 0.492, demonstrating better performance of the H-Optimus-based model.

\begin{table}[h!]
\renewcommand{\arraystretch}{1.5}
  \centering
  \caption{Cell quantification metrics for baseline and proposed models (CI 95\%).}
  \label{tab:2}
  \begin{tabular}{|l|c|c|}
    \hline
    %\rowcolor{gray!30}
    Metric             & Resnet50-based            & H-optimus-based              \\
    \hline
    $R^2_{neoplastic}$    & 0.681 (0.576--0.769)       & \textbf{0.941 (0.917--0.960)} \\
    \hline
    $R^2_{inflammatory}$  & 0.863 (0.778--0.903)       & \textbf{0.949 (0.918--0.966)} \\
    \hline
    $R^2_{connective}$    & 0.600 (0.488--0.698)       & 0.609 (0.436--0.772)          \\
    \hline
    $R^2_{dead}$          & 0.097 (-11.389--0.669)     & 0.925 (0.404--0.982)          \\
    \hline
    $R^2_{epithelial}$    & 0.635 (0.490--0.747)       & \textbf{0.930 (0.886--0.964)} \\
    \hline
    $PQ_{neoplastic}$       & 0.517 (0.499--0.535)       & \textbf{0.589 (0.575--0.604)} \\
    \hline
    $PQ_{inflammatory}$     & 0.455 (0.429--0.482)       & \textbf{0.528 (0.507--0.549)} \\
    \hline
    $PQ_{connective}$       & 0.416 (0.400--0.431)       & \textbf{0.451 (0.436--0.465)} \\
    \hline
    $PQ_{dead}$             & 0.374 (0.342--0.408)       & 0.292 (0.209--0.365)          \\
    \hline
    $PQ_{epithelial}$       & 0.488 (0.460--0.519)       & \textbf{0.599 (0.579--0.618)} \\
    \hline
  \end{tabular}
\end{table}

Our results  show that integrating the H‑Optimus foundation model within the UNETR architecture enhances the model's ability to segment and classify cells across diverse tissues from PanNuke data. The pretrained transformer encoder provides robust feature representations, resulting in higher average $R^2$ and $PQ$ scores compared to the CNN-based model. This leads to more reliable cell quantification and more accurate downstream analysis. Additionally, the streamlined fine-tuning process reduces computational overhead and training time, making the model more adaptable for new data.

Despite these advancements, the foundation model-based approach does not fully resolve all challenges related to cell segmentation and classification. We observe lower metric scores for underrepresented classes in the training data. Furthermore, foundation models typically encompass billions of parameters, resulting in substantial computational and memory requirements. It therefore poses challenges for deployment in resource-constrained environments, limiting their practical applicability in certain clinical settings.

\section{Model optimization via Knowledge Distillation}

To address the limitations posed by the extensive size of foundation models, we implement knowledge distillation — a model compression technique that leverages the teacher-student paradigm \cite{Hinton_Vinyals_etal._2015}. By training a smaller, more efficient student model to replicate the output of a larger, pre-trained teacher model, we retain performance while significantly reducing the model's complexity and resource requirements (\hyperref[fig:fig6]{Figure 6}).

\begin{figure}[h!]
    \centering
    \includegraphics[width=\textwidth, height=0.45\textheight, keepaspectratio]{images/Figure_6.pdf}
    \caption{Knowledge distillation framework for training a student model using a pre-trained teacher}
    \label{fig:fig6}
\end{figure}

We employ knowledge distillation to compress the H‑Optimus-based teacher model into a more efficient student model. The teacher model is the modified UNETR architecture with the H‑Optimus foundation model described in the previous chapter. The student model is based on a UNet architecture augmented with residual connections and incorporates a smaller ViT encoder with 9 million parameters \cite{Steiner_Kolesnikov_etal._2022,Wightman_2019}. 

First, we fine-tune the teacher model using the refined dataset from the cross-relabeling procedure (Section 2). Initially we train the decoder of the teacher model while keeping the encoder weights frozen. We split the refined dataset into train (70\%), validation (20\%) and test (10\%) subsets (\hyperref[fig:S2]{Appendix Figure S2 (4)}). During fine-tuning, we use the train and validation subsets, while leaving the test subset for model evaluation. We set the training procedure and model hyperparameters to be identical to those that were used to demonstrate the utility of foundation models for the simultaneous cell segmentation and classification task.

Next, we perform knowledge distillation from teacher to student using the refined dataset used to fine-tune the teacher model. The student model is trained to replicate the teacher model's outputs. We utilize a specialized loss function that aligns the student's predicted probability distribution with the teacher's, incorporating the teacher's class probability distribution derived from the output. Following the methodology of Hinton et al. \cite{Hinton_Vinyals_etal._2015}, we experiment with various hyperparameter settings for the temperature ($T$) and the balancing coefficients ($\alpha$ and $\beta$) in the loss function. We vary $T$ from 1 to 20 and adjust $\alpha$ and $\beta$ to balance the distillation and student losses. Through iterative tuning and evaluation, we identify that setting $T=14$, $\alpha=0.3$, and $\beta=0.7$ yields a configuration that converges and closely approximates the teacher model's performance during training.

Finally, we assess the performance of both models using the $R^2$ and $PQ$ (defined in \hyperref[chap:S5]{Appendix S5}) on the test set of the refined dataset (\hyperref[tab:3]{Table 3}). We observe that the 95\% confidence intervals overlap for most cell types, so we cannot claim statistically significant performance differences between the teacher and student models. One exception appears in the neoplastic class. The teacher model produces an $R^2$ of 0.919, while the student model shows an $R^2$ of 0.852. In addition, the student model achieves higher $PQ$ values for the neoplastic and connective classes, though the confidence intervals show overlap.

\begin{table}[h!]
\renewcommand{\arraystretch}{1.5}
  \centering
  \caption{Cell quantification metrics for teacher and distilled student models (CI 95\%).}
  \label{tab:3}
  \begin{tabular}{|l|c|c|}
    \hline
    %\rowcolor{gray!30}
    Metric & Teacher & Student \\
    \hline
    $R^2_{neoplastic}$    & \textbf{0.919} (0.898--0.939) & 0.852 (0.800--0.891) \\
    \hline
    $R^2_{lymphocyte}$    & 0.969 (0.956--0.977)         & 0.969 (0.956--0.978) \\
    \hline
    $R^2_{connective}$    & 0.694 (0.548--0.809)         & 0.618 (0.469--0.741) \\
    \hline
    $R^2_{dead}$          & 0.755 (0.400--0.908)         & 0.424 (0.100--0.731) \\
    \hline
    $R^2_{epithelial}$    & 0.922 (0.870--0.958)         & 0.843 (0.738--0.917) \\
    \hline
    $R^2_{macrophage}$    & 0.384 (-0.369--0.724)        & 0.704 (0.352--0.859) \\
    \hline
    $R^2_{neutrofil}$     & 0.854 (0.578--0.929)         & 0.833 (0.502--0.925) \\
    \hline
    $PQ_{neoplastic}$       & 0.581 (0.569--0.593)         & 0.601 (0.588--0.613) \\
    \hline
    $PQ_{lymphocyte}$       & 0.536 (0.520--0.553)         & 0.563 (0.544--0.579) \\
    \hline
    $PQ_{connective}$       & 0.436 (0.421--0.451)         & 0.457 (0.441--0.474) \\
    \hline
    $PQ_{dead}$             & 0.272 (0.235--0.315)         & 0.279 (0.201--0.369) \\
    \hline
    $PQ_{epithelial}$       & 0.522 (0.500--0.545)         & 0.530 (0.506--0.555) \\
    \hline
    $PQ_{macrophage}$       & 0.524 (0.459--0.588)         & 0.474 (0.405--0.543) \\
    \hline
    $PQ_{neutrofil}$        & 0.541 (0.490--0.592)         & 0.565 (0.522--0.607) \\
    \hline
  \end{tabular}
\end{table}


We further decompose the $PQ$ metric into its $SQ$ and $DQ$ components (\hyperref[tab:S6]{Appendix Table S6}). Both models produce nearly identical $SQ$ values, which indicates that they predict instance boundaries with similar precision. Although the student model shows some improvement in $DQ$ scores for certain classes, the confidence intervals overlap and do not confirm a statistically significant difference.

We observe that the student and teacher models yield comparable detection performance despite the student model using a much smaller and simpler architecture. A model with fewer parameters reduces the risk of overfitting when training data are scarce relative to the model’s complexity \cite{Farias_Ludermir_etal._2022}. The knowledge distillation process also encourages the student model to focus on the most generalizable detection features learned from the teacher. These factors enable the student model to achieve similar detection performance across different cell types.

Additionally, considering the model sizes reported in \hyperref[tab:4]{Table 4}, the distilled model achieves a significant reduction compared to the teacher model, with a 48-fold decrease in parameter count and a 5.5-fold reduction in on-disk size. In inference mode, the teacher model requires 16 GB of VRAM for a batch size of 32, while the distilled model only needs 3 GB of VRAM for the same batch size. These reductions make the distilled model significantly more practical for fine-tuning and deployment in resource-constrained environments.

\begin{table}[h!]
\renewcommand{\arraystretch}{1.5}
  \centering
  \caption{Parameter counts and size of teacher and distilled model}
  \label{tab:4}
  \adjustbox{max width=\textwidth}{%
  \begin{tabular}{|l|c|c|c|}
    \hline
    %\rowcolor{gray!30}
    Metric & H-optimus-based (Teacher) & mobileViT-based (Student) & Magnitude of difference \\
    \hline
    Parameters count       & 1,158,917,906   & \textbf{24,093,393}   & \textbf{48x}  \\
    \hline
    Estimated Total Size (MB) & 87,912       & \textbf{15,935}    & \textbf{5.5x} \\
    \hline
  \end{tabular}%
}
\end{table}

%\hfill

With recent advancements in complex network architectures and the use of pretrained encoders to achieve state-of-the-art performance \cite{Baumann_Dislich_etal._2024,Hörst_Rempe_etal._2024} in cell segmentation and classification tasks, model size, computational complexity, and processing times have increased. This limits the scalability and accessibility of these models. As we demonstrate, this may be mitigated using knowledge distillation. Studies in the field of natural language processing have demonstrated the efficacy of knowledge distillation in retaining the capabilities of the teacher model while achieving significant reductions in size and complexity \cite{Huangpu_Gao_2024,Sun_Yu_etal.}. 

We demonstrate the feasibility of knowledge distillation in digital pathology, specifically for cell segmentation and classification tasks. Moreover, we achieve this performance while also significantly reducing the parameter count. In addressing the challenge of knowledge transfer, we found that distillation from a transformer-based model to a smaller transformer is more straightforward than attempting to map transformer features to CNN blocks. In our experiments, using a CNN-based network as a student results in worse cell quantification performance due to the structural constraints of CNN feature space dimensions. 

Although our primary approach relies on a transformer-based student model that performs well, it can be further optimized to incorporate advantages from CNN architectures. For example, employing alternative techniques such as using ViT adapters \cite{Chen_Duan_etal._2023} or $1 \times 1$ convolutions to adjust feature map sizes may be beneficial for harnessing CNN advantages like enhanced local feature extraction. Moreover, if additional performance improvements are desired, the process can be further enhanced by applying supplementary knowledge distillation techniques, such as self-distillation \cite{Zhang_Song_etal._2019} or online distillation \cite{Houyon_Cioppa_etal._2023}.

Despite these promising results, further validation on independent datasets is necessary to fully understand the model's limitations. Underrepresented classes may pose challenges when addressing complex cases. Pathologists need to validate these models to adopt them in clinical settings. While the distilled models are smaller and more deployable, a technological gap persists because pathologists traditionally rely on established methods for inspecting WSIs and diagnosing diseases. Addressing the complexities involved in deploying models for inference and supporting pathologists in adopting new tools is essential for integrating these models into clinical workflows.

\section{Model integration with QuPath}
Digital pathology tools with graphical user interfaces are essential for visualizing and analyzing WSIs. To make our student model useful in clinical pathology workflows, it needs to be integrated into a tool that enables inspecting regions, creating annotations, and providing quantitative analyses of biomarkers. Therefore, we integrate the trained student model from the previous chapter into the QuPath open‑source platform \cite{Bankhead_Loughrey_etal._2017}. QuPath provides the required annotation, visualization, and analysis tools to interpret complex histological data, including workflows for cell segmentation, classification, and quantification (\hyperref[fig:fig7]{Figure 7}). 

\begin{figure}[h!]
    \centering
    \includegraphics[width=\textwidth]{images/Figure_7.pdf}
    \caption{Visualization of model-generated cell quantification annotations (left) and the corresponding unannotated slide (right) in QuPath}
    \label{fig:fig7}
\end{figure}

To identify the regions in a WSI critical for prognosticating tumor development, such as specific tumor areas or border regions without overlapping healthy tissue, the pathologist uses QuPath to outline these regions. Then, the pathologist initiates a cell segmentation and classification script through the QuPath interface for the selected regions. The resulting annotations and quantified cell information are then directly overlaid onto the WSI in the QuPath interface. Additional design and implementation details are in \hyperref[chap:S7]{Appendix S7}. 

Two common approaches for integrating deep learning models into QuPath are Java‑based native QuPath extensions \cite{Goldsborough_Philps_etal._2024} and the execution of RESTful API requests to a model server coupled with handling the response via an extension, as demonstrated in the application of cell segmentation models applied to immunofluorescence images \cite{Sugawara_2023}. While the community is actively working on these integration strategies, there is currently no universal solution that fully addresses all integration and performance requirements.

Extensions may offer better integration with QuPath, allowing slightly improved performance and more widespread usage of the built-in QuPath models, but they lack the flexibility to customize models and modify their behavior. For example, the newest version of QuPath includes models such as StarDist \cite{Weigert_Schmidt} and InstanSeg \cite{Goldsborough_Philps_etal._2024} that can perform cell segmentation. Both models pose limitations when applied to simultaneous cell segmentation and classification. StarDist performs well only on convex, round shapes by design, whereas some neoplastic, inflammatory, and connective cells exhibit complex and non-convex shapes. InstanSeg provides only semantic segmentation without assigning classes to the segmented cells.

%\hfill

In contrast, our approach offers an alternative integration strategy. It utilizes the paquo library to directly interact with QuPath’s internal application programming interface from within Python. This enables data exchange and processing without the need for intermediate conversion steps and provides greater control over model customization, retraining, and the incorporation of custom processing steps.

The integration of our custom model with QuPath underscores its potential to significantly enhance the diagnostic process by reducing the time burden on pathologists and enabling them to focus on more complex interpretative tasks using familiar software. Leveraging a tool that is already well-established among pathologists increases the likelihood of its adoption into daily clinical workflows. The quantitative data generated through the automated workflow is critical for both clinical decision-making and research, facilitating more accurate biomarker analysis, enabling robust statistical evaluations, and supporting hypothesis generation and testing. Additionally, by streamlining cell segmentation and classification, the tool enhances the scalability and reproducibility of pathological assessments, ultimately contributing to improved diagnostic accuracy and patient outcomes.

\section{Conclusion and future work}

In this study, we address critical challenges in digital pathology and tackle the usability and deployment issues of the developed models in standard computing environments without the need for high-performance computing systems. Our multi-faceted approach encompasses data refinement through cross-relabeling, leveraging foundation models for robust cell segmentation and classification, optimizing model performance via knowledge distillation, and integrating the optimized model into the QuPath software for practical application. This approach is used to construct a capable, versatile, and adjustable model for cell segmentation and classification, with enhanced performance and usability.

\begin{sloppypar}
While our approach shows potential in the field of computational pathology, certain limitations persist. 
For example, our implementation currently exhibits lower performance in detecting macrophages. 
This serves as an instance of the broader challenge of accurately identifying complex cell types. In order to address this issue, extending our approach to incorporate additional data sources, exploring alternative modeling approaches, and integrating other imaging modalities such as immunohistochemical staining may help improve detection accuracy. Moreover, although the distilled model reduces computational demands, integrating advanced deep learning models into clinical practice requires addressing technological gaps and potential resistance to adopting new tools within established diagnostic processes.
\end{sloppypar}

Future work could focus on several key areas to refine the proposed approach and facilitate its adoption in clinical environments. Enhancing the cell-relabeling process with additional datasets \cite{Graham_Jahanifar_etal._2021} could improve the representation of underrepresented cell types and enhance overall model performance. Also, incorporating additional data sources, such as multi-modal imaging or complementary staining methods, may address limitations related to cell type differentiation and class imbalance. Exploring other foundation models \cite{Vorontsov_Bozkurt_etal._2024,Zimmermann_Vorontsov_etal._2024} or introducing additional modalities \cite{Ding_Wagner_etal._2024,Vaidya_Zhang_etal._2025} may provide alternative architectures better suited to specific tasks or offer improved efficiency. Implementing more complex knowledge distillation techniques \cite{Houyon_Cioppa_etal._2023,Zhang_Song_etal._2019} could further optimize the model's performance and adaptability. Additionally, deeper integration with QuPath or other digital pathology software could provide pathologists more control over cell quantification analysis directly within the QuPath interface, thereby increasing accessibility and usability. Such enhancements would not only refine model performance but also ensure greater adaptability and scalability within various clinical environments. Finally, extensive validation of the model by pathologists and benchmarking against independent datasets are essential steps toward establishing the model's reliability and fostering confidence in its clinical utility.

\section*{Acknowledgments} 
This work was funded in part by the Research Council of Norway grant no. 309439 SFI Visual Intelligence, and the North Norwegian Health Authority grant no. HNF1521-20.

\bibliographystyle{IEEEtran}
\begin{sloppypar}
\begin{thebibliography}{99}

\bibitem{chaplot2020neural} Chaplot, Devendra Singh, et al. "Neural topological slam for visual navigation." Proceedings of the IEEE/CVF conference on computer vision and pattern recognition. 2020.

\bibitem{maksymets2021thda} Maksymets, Oleksandr, et al. "Thda: Treasure hunt data augmentation for semantic navigation." Proceedings of the IEEE/CVF International Conference on Computer Vision. 2021.

\bibitem{mezghan2022memory} Mezghan, Lina, et al. "Memory-augmented reinforcement learning for image-goal navigation." 2022 IEEE/RSJ International Conference on Intelligent Robots and Systems (IROS). IEEE, 2022.

\bibitem{al2022zero} Al-Halah, Ziad, Santhosh Kumar Ramakrishnan, and Kristen Grauman. "Zero experience required: Plug \& play modular transfer learning for semantic visual navigation." Proceedings of the IEEE/CVF Conference on Computer Vision and Pattern Recognition. 2022.

\bibitem{ye2021auxiliary} Ye, Joel, et al. "Auxiliary tasks and exploration enable objectgoal navigation." Proceedings of the IEEE/CVF international conference on computer vision. 2021.

\bibitem{chaplot2020object} Chaplot, Devendra Singh, et al. "Object goal navigation using goal-oriented semantic exploration." Advances in Neural Information Processing Systems 33 (2020)

\bibitem{ramakrishnan2022poni} Ramakrishnan, Santhosh Kumar, et al. "Poni: Potential functions for objectgoal navigation with interaction-free learning." Proceedings of the IEEE/CVF Conference on Computer Vision and Pattern Recognition. 2022.

\bibitem{ramrakhya2022habitat} Ramrakhya, Ram, et al. "Habitat-web: Learning embodied object-search strategies from human demonstrations at scale." Proceedings of the IEEE/CVF Conference on Computer Vision and Pattern Recognition. 2022.

\bibitem{mousavian2019visual} Mousavian, Arsalan, et al. "Visual representations for semantic target driven navigation." 2019 International Conference on Robotics and Automation (ICRA). IEEE, 2019.

\bibitem{dhariwal2021diffusion} Dhariwal, Prafulla, and Alexander Nichol. "Diffusion models beat gans on image synthesis." Advances in neural information processing systems 34 (2021)

\bibitem{ho2022classifier} Ho, Jonathan, and Tim Salimans. "Classifier-free diffusion guidance." arXiv preprint arXiv:2207.12598 (2022).

\bibitem{nichol2021glide} Nichol, Alex, et al. "Glide: Towards photorealistic image generation and editing with text-guided diffusion models." arXiv preprint arXiv:2112.10741 (2021)

\bibitem{brooks2023instructpix2pix} Brooks, Tim, Aleksander Holynski, and Alexei A. Efros. "Instructpix2pix: Learning to follow image editing instructions." Proceedings of the IEEE/CVF Conference on Computer Vision and Pattern Recognition. 2023.

\bibitem{fu2023guiding} Fu, Tsu-Jui, et al. "Guiding instruction-based image editing via multimodal large language models." arXiv preprint arXiv:2309.17102 (2023).

\bibitem{geng2024instructdiffusion} Geng, Zigang, et al. "Instructdiffusion: A generalist modeling interface for vision tasks." Proceedings of the IEEE/CVF Conference on Computer Vision and Pattern Recognition. 2024.

\bibitem{zhou2024minedreamer} Zhou, Enshen, et al. "Minedreamer: Learning to follow instructions via chain-of-imagination for simulated-world control." arXiv preprint arXiv:2403.12037 (2024).

\bibitem{zhou2023esc} Zhou, Kaiwen, et al. "Esc: Exploration with soft commonsense constraints for zero-shot object navigation." International Conference on Machine Learning. PMLR, 2023.

\bibitem{yu2023l3mvn} Yu, Bangguo, Hamidreza Kasaei, and Ming Cao. "L3mvn: Leveraging large language models for visual target navigation." 2023 IEEE/RSJ International Conference on Intelligent Robots and Systems (IROS). IEEE, 2023.

\bibitem{gadre2023cows} Gadre, Samir Yitzhak, et al. "Cows on pasture: Baselines and benchmarks for language-driven zero-shot object navigation." Proceedings of the IEEE/CVF Conference on Computer Vision and Pattern Recognition. 2023.

\bibitem{shah2023navigation} Shah, Dhruv, et al. "Navigation with large language models: Semantic guesswork as a heuristic for planning." Conference on Robot Learning. PMLR, 2023.

\bibitem{cai2024bridging} Cai, Wenzhe, et al. "Bridging zero-shot object navigation and foundation models through pixel-guided navigation skill." 2024 IEEE International Conference on Robotics and Automation (ICRA). IEEE, 2024.

\bibitem{yu2023co} Yu, Bangguo, Hamidreza Kasaei, and Ming Cao. "Co-NavGPT: Multi-robot cooperative visual semantic navigation using large language models." arXiv preprint arXiv:2310.07937 (2023).

\bibitem{wu2024voronav} Wu, Pengying, et al. "Voronav: Voronoi-based zero-shot object navigation with large language model." arXiv preprint arXiv:2401.02695 (2024).

\bibitem{qin2023mp5} Qin, Yiran, et al. "Mp5: A multi-modal open-ended embodied system in minecraft via active perception." arXiv preprint arXiv:2312.07472 (2023).

\bibitem{du2024learning} Du, Yilun, et al. "Learning universal policies via text-guided video generation." Advances in Neural Information Processing Systems 36 (2024).

\bibitem{ajay2024compositional} Ajay, Anurag, et al. "Compositional foundation models for hierarchical planning." Advances in Neural Information Processing Systems 36 (2024).

\bibitem{liang2024skilldiffuser} Liang, Zhixuan, et al. "Skilldiffuser: Interpretable hierarchical planning via skill abstractions in diffusion-based task execution." Proceedings of the IEEE/CVF Conference on Computer Vision and Pattern Recognition. 2024.

\bibitem{heusel2017gans} Heusel, Martin, et al. "Gans trained by a two time-scale update rule converge to a local nash equilibrium." Advances in neural information processing systems 30 (2017).

\bibitem{zhang2018unreasonable} Zhang, Richard, et al. "The unreasonable effectiveness of deep features as a perceptual metric." Proceedings of the IEEE conference on computer vision and pattern recognition. 2018.

\bibitem{brown2020language} Brown, Tom B. "Language models are few-shot learners." arXiv preprint arXiv:2005.14165 (2020).

\bibitem{podell2023sdxl} Podell, Dustin, et al. "Sdxl: Improving latent diffusion models for high-resolution image synthesis." arXiv preprint arXiv:2307.01952 (2023).

\bibitem{brohan2022rt} Brohan, Anthony, et al. "Rt-1: Robotics transformer for real-world control at scale." arXiv preprint arXiv:2212.06817 (2022).

\bibitem{brohan2023rt} Brohan, Anthony, et al. "Rt-2: Vision-language-action models transfer web knowledge to robotic control." arXiv preprint arXiv:2307.15818 (2023).

\bibitem{li2024manipllm} Li, Xiaoqi, et al. "Manipllm: Embodied multimodal large language model for object-centric robotic manipulation." Proceedings of the IEEE/CVF Conference on Computer Vision and Pattern Recognition. 2024.

\bibitem{shah2023vint} Shah, Dhruv, et al. "ViNT: A foundation model for visual navigation." arXiv preprint arXiv:2306.14846 (2023).

\bibitem{liu2024visual} Liu, Haotian, et al. "Visual instruction tuning." Advances in neural information processing systems 36 (2024).

\bibitem{hu2021lora} Hu, Edward J., et al. "Lora: Low-rank adaptation of large language models." arXiv preprint arXiv:2106.09685 (2021).

\bibitem{qin2023supfusion} Qin, Yiran, et al. "SupFusion: Supervised LiDAR-camera fusion for 3D object detection." Proceedings of the IEEE/CVF International Conference on Computer Vision. 2023.

\bibitem{qin2024worldsimbench} Qin, Yiran, et al. "Worldsimbench: Towards video generation models as world simulators." arXiv preprint arXiv:2410.18072 (2024).

\bibitem{yu2025gamefactory} Yu, Jiwen, et al. "GameFactory: Creating New Games with Generative Interactive Videos." arXiv preprint arXiv:2501.08325 (2025).

\bibitem{zhou2024code} Zhou, Enshen, et al. "Code-as-Monitor: Constraint-aware Visual Programming for Reactive and Proactive Robotic Failure Detection." arXiv preprint arXiv:2412.04455 (2024).

\bibitem{zhang2024ad} Zhang, Zaibin, et al. "AD-H: Autonomous Driving with Hierarchical Agents." arXiv preprint arXiv:2406.03474 (2024).

\bibitem{wang2024toward} Wang, Chaoqun, et al. "Toward Accurate Camera-based 3D Object Detection via Cascade Depth Estimation and Calibration." arXiv preprint arXiv:2402.04883 (2024).

\bibitem{huang2024story3d} Huang, Yuzhou, et al. "Story3d-agent: Exploring 3d storytelling visualization with large language models." arXiv preprint arXiv:2408.11801 (2024).

\bibitem{savinov2018semi} Savinov, Nikolay, Alexey Dosovitskiy, and Vladlen Koltun. "Semi-parametric topological memory for navigation." arXiv preprint arXiv:1803.00653 (2018).

\bibitem{majumdar2022zson} Majumdar, Arjun, et al. "Zson: Zero-shot object-goal navigation using multimodal goal embeddings." Advances in Neural Information Processing Systems 35 (2022): 32340-32352.

\bibitem{yadav2023offline} Yadav, Karmesh, et al. "Offline visual representation learning for embodied navigation." Workshop on Reincarnating Reinforcement Learning at ICLR 2023. 2023.

\bibitem{yadav2023ovrl} Yadav, Karmesh, et al. "Ovrl-v2: A simple state-of-art baseline for imagenav and objectnav." arXiv preprint arXiv:2303.07798 (2023).

\bibitem{sun2024fgprompt} Sun, Xinyu, et al. "FGPrompt: fine-grained goal prompting for image-goal navigation." Advances in Neural Information Processing Systems 36 (2024).

\bibitem{zhu2017target} Zhu, Yuke, et al. "Target-driven visual navigation in indoor scenes using deep reinforcement learning." 2017 IEEE international conference on robotics and automation (ICRA). IEEE, 2017.

\bibitem{koh2024generating} Koh, Jing Yu, Daniel Fried, and Russ R. Salakhutdinov. "Generating images with multimodal language models." Advances in Neural Information Processing Systems 36 (2024).

\bibitem{krantz2022instance} Krantz, Jacob, et al. "Instance-specific image goal navigation: Training embodied agents to find object instances." arXiv preprint arXiv:2211.15876 (2022).

\bibitem{schulman2017proximal} Schulman, John, et al. "Proximal policy optimization algorithms." arXiv preprint arXiv:1707.06347 (2017).

\bibitem{anderson2018evaluation} Anderson, Peter, et al. "On evaluation of embodied navigation agents." arXiv preprint arXiv:1807.06757 (2018).

\bibitem{lin2024navcot} Lin, Bingqian, et al. "NavCoT: Boosting LLM-Based Vision-and-Language Navigation via Learning Disentangled Reasoning." arXiv preprint arXiv:2403.07376 (2024).

\bibitem{NavGPT} Zhou, Gengze, Yicong Hong, and Qi Wu. "Navgpt: Explicit reasoning in vision-and-language navigation with large language models." Proceedings of the AAAI Conference on Artificial Intelligence.

\bibitem{hahn2021no} Hahn, Meera, et al. "No rl, no simulation: Learning to navigate without navigating." Advances in Neural Information Processing Systems 34 (2021): 26661-26673.

\bibitem{li2025t2isafety} Li, Lijun, et al. "T2ISafety: Benchmark for Assessing Fairness, Toxicity, and Privacy in Image Generation." arXiv preprint arXiv:2501.12612 (2025).

\bibitem{an2024agfsync} An, Jingkun, et al. "AGFSync: Leveraging AI-Generated Feedback for Preference Optimization in Text-to-Image Generation." arXiv preprint arXiv:2403.13352 (2024).


\end{thebibliography}
\end{sloppypar}

\clearpage
\beginsupplement
\section*{Appendix}
\renewcommand{\thesubsection}{S\arabic{subsection}}

\subsection{\label{chap:S1}PanNuke and MoNuSAC preprocessing}
The PanNuke dataset comprises a set of 7,901 RGB patches, each with dimensions of $256 \times 256$ pixels, which we set as the standard patch size for our analysis. In contrast, the MoNuSAC dataset encompasses 294 images of heterogeneous dimensions. To standardize the MoNuSAC images with our experiments, we implement a standardization protocol. Specifically, for images exceeding the dimensions of $256 \times 256$ pixels, we segment them into equal-sized patches and apply mirror padding to the remaining portions to avoid information loss at the peripherals. Patches with dimensions less than $128 \times 128$ pixels are excluded from the dataset due to the insufficient resolution to capture relevant cellular details. For patches where either dimension falls between 128 and 256 pixels, we employ upsampling to achieve the standard patch size. As a result, we obtain a total of 2,823 RGB patches derived from the MoNuSAC dataset for subsequent analysis. For additional details on the MoNuSAC data preparation process, refer to the source code \cite{Shvetsov_2025a}.
\clearpage

\subsection{\label{chap:S2}Data usage for the methodology}

\counterwithin{figure}{subsection}
\renewcommand{\thefigure}{S\arabic{subsection}}

\begin{figure}[h!]
    \centering
    \includegraphics[width=\textwidth, height=0.85\textheight, keepaspectratio]{images/A2.pdf}
    \caption{Overview of the methodology for cross-labeling, dataset refinement, and model comparison. (1) Cross-relabeling - training and testing cell classification models, (2) Cross-relabeling - using cell classification models to create refined dataset, (3) Fine-tuning and training models for comparison, (4) Student knowledge distillation with refined dataset}
    \label{fig:S2}
\end{figure}
\clearpage

\subsection{\label{chap:S3}Confusion matrices for classification models}
\counterwithin{figure}{subsection}
\renewcommand{\thefigure}{S\arabic{subsection}.\arabic{figure}}

\begin{figure}[h!]
    \centering
    \includegraphics[width=\textwidth, height=0.4\textheight, keepaspectratio]{images/A3_1.pdf}
    \caption{Confusion matrix for PanNuke trained model}
    \label{fig:S3.1}
\end{figure}

\begin{figure}[h!]
    \centering
    \includegraphics[width=\textwidth, height=0.4\textheight, keepaspectratio]{images/A3_2.pdf}
    \caption{Confusion matrix for MoNuSAC trained model}
    \label{fig:S3.2}
\end{figure}

\clearpage

\subsection{\label{chap:S4}Datasets cell counts}

\counterwithin{table}{subsection}
\renewcommand{\thetable}{S\arabic{subsection}}

\begin{table}[h!]
\renewcommand{\arraystretch}{2.0}
\centering
\caption{\label{tab:S4}Cell counts for PanNuke, MoNuSAC and refined datasets. Numbers in parentheses indicate preprocessed cell counts for cell classifier models training and testing.}
%\adjustbox{max width=\textwidth}{%
\begin{tabular}{|l|c|c|c|}
\hline
%\rowcolor{gray!30}
Cell type & PanNuke & MoNuSAC & Refined \\
\hline
Neoplastic & 77,403 (68,031) & - & 105,451 \\
\hline
Epithelial & 26,572 (23,207) & - & 29,926 \\
\hline
Epithelial (benign and malignant) & - & 31,402 & - \\
\hline
Inflammatory & 32,276 & - & - \\
\hline
Lymphocytes & - & 37,045 (33,104) & 65,275 \\
\hline
Neutrophils & - & 1,355 (1,252) & 3,833 \\
\hline
Macrophage & - & 1,842 (1,695) & 3,410 \\
\hline
Dead & 2,908 & - & 2,908 \\
\hline
Connective & 50,585 & - & 50,585 \\
\hline
\end{tabular}
%
%}
\end{table}



\clearpage

\subsection{\label{chap:S5}Definition of validation metrics}
\counterwithin{equation}{subsection}
\renewcommand{\theequation}{\arabic{equation}}

\subsubsection{\label{chap:S5.1}R\textsuperscript{2}}
The coefficient of determination, denoted as $R^2$, is a statistical measure that represents the proportion of variance in the dependent variable that is predictable from the independent variables. In the context of cell quantification in pathology, $R^2$ is used to assess how well the predicted quantities of different cell types in a patch align with the actual quantities observed in the ground truth data, with higher values representing more accurate quantification. $R^2$ is defined as
\begin{equation*}
R^2 = 1 - \frac{\sum_{i=1}^n (y_i - \hat{y}_i)^2}{\sum_{i=1}^n (y_i - \bar{y})^2},
\end{equation*}
where $y_i$ represents the actual number of cells of a specific type in the $i$-th image, $\hat{y}_i$ represents the predicted number of cells of that type in the $i$-th image, $\bar{y}$ is the mean of the actual numbers across all images, and $n$ is the total number of images in the dataset.

The $R^2$ metric has a range of $(-\infty, 1]$. An $R^2$ of 1 indicates perfect prediction, where all predicted values exactly match the actual values. An $R^2$ of 0 suggests that the model explains none of the variability of the response data around its mean. If $R^2$ is negative, it indicates that the model performs worse than a model that simply predicts the mean of the actual values for all observations.

\subsubsection{\label{chap:S5.2}PQ}
Panoptic Quality ($PQ$) is a comprehensive metric used to evaluate the performance of segmentation models in tasks that require both instance segmentation and classification. $PQ$ provides a single score that encapsulates both the detection accuracy (i.e., how many objects were correctly identified) and the segmentation quality (i.e., how accurately the objects' boundaries were delineated). This metric is particularly useful in multiclass scenarios where each pixel is classified into distinct categories, such as different cell types in pathology images.

$PQ$ is calculated as the product of two terms: Detection Quality ($DQ$) and Segmentation Quality ($SQ$). It can be expressed as
\begin{equation*}
PQ = DQ \cdot SQ,
\end{equation*}
where
\begin{equation*}
DQ = \frac{TP}{TP + 0.5\, FP + 0.5\, FN},
\end{equation*}
\begin{equation*}
SQ = \frac{\sum_{(p, g) \in \mathcal{M}} IoU(p, g)}{TP}.
\end{equation*}
In these formulas, $TP$ denotes the number of correctly matched instances between ground truth and prediction, $FP$ denotes the predicted instances that have no corresponding ground truth, $FN$ denotes the ground truth instances that were not detected, $IoU(p, g)$ is the Intersection over Union for a pair of matched instances $p$ (prediction) and $g$ (ground truth), and $\mathcal{M}$ is the set of matched pairs.

The $PQ$ metric is calculated for each class and is averaged across classes to provide a global performance measure.

The $PQ$ score has a range of $[0, 1.0]$, where a higher score indicates better performance in both detecting and segmenting the instances correctly. A $PQ$ of 1 signifies perfect identification and segmentation of all instances, whereas a $PQ$ of 0 indicates that no instances were correctly identified and segmented.

\clearpage

\subsection{\label{chap:S6}Segmentation and Detection quality metrics for teacher and student models}

\begin{table}[h!]
\renewcommand{\arraystretch}{2.0}
\centering
\caption{Segmentation and detection quality for student and teacher models (CI 95\%)}
\label{tab:S6}
%\adjustbox{max width=\textwidth}{%
\begin{tabular}{|l|c|c|}
\hline
%\rowcolor{gray!30}
Metric & Teacher & Student \\
\hline
$SQ_{neoplastic}$ & 0.819 (0.815--0.823) & 0.824 (0.819--0.828) \\
\hline
$SQ_{lymphocyte}$ & 0.795 (0.788--0.802) & 0.790 (0.783--0.796) \\
\hline
$SQ_{connective}$ & 0.770 (0.762--0.776) & 0.780 (0.772--0.786) \\
\hline
$SQ_{dead}$ & 0.659 (0.623--0.688) & 0.657 (0.624--0.695) \\
\hline
$SQ_{epithelial}$ & 0.780 (0.770--0.790) & 0.788 (0.779--0.797) \\
\hline
$SQ_{macrophage}$ & 0.788 (0.760--0.810) & 0.757 (0.730--0.783) \\
\hline
$SQ_{neutrofil}$ & 0.782 (0.761--0.801) & 0.775 (0.759--0.792) \\
\hline
$DQ_{neoplastic}$ & 0.706 (0.692--0.719) & 0.727 (0.712--0.741) \\
\hline
$DQ_{lymphocyte}$ & 0.675 (0.656--0.698) & 0.713 (0.691--0.734) \\
\hline
$DQ_{connective}$ & 0.566 (0.546--0.584) & 0.583 (0.565--0.602) \\
\hline
$DQ_{dead}$ & 0.410 (0.361--0.465) & 0.435 (0.306--0.561) \\
\hline
$DQ_{epithelial}$ & 0.668 (0.639--0.694) & 0.673 (0.644--0.702) \\
\hline
$DQ_{macrophage}$ & 0.657 (0.583--0.727) & 0.615 (0.531--0.703) \\
\hline
$DQ_{neutrofil}$ & 0.691 (0.625--0.753) & 0.729 (0.679--0.778) \\
\hline
\end{tabular}
%
%}
\end{table}

\clearpage

\subsection{\label{chap:S7}QuPath integration method}
We adopt an integration strategy leveraging the paquo \cite{Bayer_AG} library, a Python package that enables direct interaction with QuPath’s internal API, thereby facilitating seamless data exchange without intermediate conversion steps. The data processing pipeline (\hyperref[fig:S7]{Appendix Figure S7}) begins with the acquisition of WSIs and their associated annotations from QuPath, which are represented as Shapely \cite{Gillies_Wel_etal._2024} polygons. Utilizing paquo, we directly read, create, and modify these annotations and detections within a QuPath project in the Python environment. Images are then cropped using these polygons and processed by cell segmentation and classification models employing standard vision processing toolkits such as OpenCV, pyvips, and PyTorch. Additionally, QuPath employs Groovy scripts to initiate a Python process that starts the entire pipeline from QuPath graphical interface: fetching polygons, extracting images from them, and running deep learning model inference on the cropped images. 
The results are returned to QuPath, leveraging paquo's Python bindings to manipulate QuPath data while minimizing the computational overhead typically associated with cross-environment communication.

\counterwithin{figure}{subsection}
\renewcommand{\thefigure}{S\arabic{subsection}}

\begin{figure}[h!]
    \centering
    \includegraphics[width=\textwidth]{images/A7.pdf}
    \caption{QuPath integration workflow using Python environment}
    \label{fig:S7}
\end{figure}

Compared to traditional workflows that involve exporting annotations as GeoJSON, classifying them in Python, and reimporting them into QuPath, our approach offers several advantages. We eliminate the need to switch between programming languages, providing a cohesive and streamlined development process entirely within QuPath software and removing the necessity to use other tools. Meanwhile, we avoid storing annotations as intermediate JSON files unless required for external use or archiving. By conducting the entire inference and post-processing workflow within the Python environment, we leverage the power and flexibility of Python libraries for image processing and machine learning. This approach also enables adjustments to any set of labels and models, thereby improving its applicability.

%\hfill

The distilled model and QuPath integration code are packaged into a Docker container, enabling streamlined execution with the Docker engine. Detailed integration code and deployment instructions can be found in the GitHub repository \cite{Shvetsov_2025b}.

Despite these benefits, we acknowledge that the paquo library is a proof‑of‑concept project in its early development stage and has not been tested across all versions of QuPath.

\clearpage

\subsection{\label{chap:S8}Data and code availability statement}
All datasets, models, and code used in this study are publicly available and can be obtained from the repositories listed below. 
The PanNuke \cite{Gamper_Koohbanani_etal._2019} and MoNuSAC \cite{Verma_Kumar_etal._2021} datasets are publicly accessible, and download information along with detailed descriptions can be found in their respective articles. Preprocessing scripts for PanNuke and MoNuSAC data, as well as individual cell extraction scripts, are available on GitHub \cite{Shvetsov_2025a}. The H-Optimus foundation model used in our experiments can be downloaded from the HuggingFace repository \cite{hoptimus2024}, and model information is available on GitHub \cite{Saillard_Jenatton_etal._2024}. In addition, the integration code for QuPath and the distilled model packaged in a Docker container are provided in the repository \cite{Shvetsov_2025b}, and paquo Python library is available from the authors GitHub repository \cite{Bayer_AG}.
\clearpage

\end{document}



%%%%%%%%%%%%%%%%%%%%%%%%%%%%%%%%%%%%%%%%%%%%%%%%%%%%%%%%%%%%%%%%%%%%%%%%%%%%%%%
%%%%%%%%%%%%%%%%%%%%%%%%%%%%%%%%%%%%%%%%%%%%%%%%%%%%%%%%%%%%%%%%%%%%%%%%%%%%%%%
% APPENDIX
%%%%%%%%%%%%%%%%%%%%%%%%%%%%%%%%%%%%%%%%%%%%%%%%%%%%%%%%%%%%%%%%%%%%%%%%%%%%%%%
%%%%%%%%%%%%%%%%%%%%%%%%%%%%%%%%%%%%%%%%%%%%%%%%%%%%%%%%%%%%%%%%%%%%%%%%%%%%%%%
\newpage
\appendix


\section{Mathematical Proofs}
\label{app-sec:proofs}

\subsection{Auxiliary Technical Results}

In this section, we begin by introducing two useful propositions, \Cref{app-prop:p-value-to-quantile} and \Cref{app-prop:quantiles}, which will be used later in the proofs presented here. 


\begin{proposition}
\label{app-prop:p-value-to-quantile}
    Let $\D$ be a dataset containing $n$ scores, and define the threshold $\hat{Q}_{1-\alpha}$ as
    \begin{align*}
\hat{Q}_{1-\alpha} &:= \hat{i} \text{-th smallest element in } \D\cup \{\infty\},
\end{align*}
where 
\begin{align*}
    \hat{i} :=\lceil (1-\alpha)(n+1)\rceil.
\end{align*}
For any test point $X_{n+1}$, the following holds:
    \begin{align*}
        s(X_{n+1}) > \hat{Q}_{1-\alpha} \quad\text{ if and only if }\quad \hat{p}_{n+1} \leq \alpha,
    \end{align*}
    where $\hat{p}_{n+1}$ is the conformal p-value \eqref{eq:conformal-p-value}.
\end{proposition}

\begin{proof}[Proof of \Cref{app-prop:p-value-to-quantile}]
The proof follows the definition of conformal p-value from \eqref{eq:conformal-p-value}, and its relation to the empirical quantile function:
\begin{alignat*}{2}
    \hat{p}_{n+1} = \frac{1 + \sum_{i=1}^{n} \mathbb{I}[s(X_i) \geq s(X_{n+1})]}{n+1}
    &\leq \alpha &&\overset{(i)}{\Longleftrightarrow} \\
    \hat{p}_{n+1} = \frac{1 + \sum_{i=1}^{n} \mathbb{I}[s(X_i) \geq s(X_{n+1})]}{n+1}
    &\leq \frac{\lfloor \alpha(n+1) \rfloor }{n+1} &&\Longleftrightarrow \\
    1 + \sum_{i=1}^{n} \mathbb{I}[s(X_i) \geq s(X_{n+1})] &\leq \lfloor \alpha (n+1) \rfloor &&\Longleftrightarrow \\
    1 + n - \sum_{i=1}^{n} \mathbb{I}[s(X_i) < s(X_{n+1})] &\leq \lfloor \alpha (n+1) \rfloor &&\Longleftrightarrow \\
     \sum_{i=1}^{n} \mathbb{I}[s(X_i) < s(X_{n+1})] &\geq n + 1 - \lfloor \alpha (n+1) \rfloor &&\overset{(ii)}{\Longleftrightarrow} \\
    \sum_{i=1}^{n} \mathbb{I}[s(X_i) < s(X_{n+1})] &\geq  \lceil (1-\alpha) (n+1) \rceil && \numberthis \label{app-eq:prop-inequality}
\end{alignat*}
The labeled steps above can be explained as follows.
\begin{itemize}
    \item (i) The values of $\hat{p}_{n+1}$ are discrete, taking values from $\{\frac{1}{n+1}, \frac{2}{n+1}, \dots, 1\}$. Therefore, $\hat{p}_{n+1} = \frac{k}{n+1}$ for some $k\in[n+1]$. We explicitly prove that $\hat{p}_{n+1}\leq \alpha$ iff $\hat{p}_{n+1}\leq \frac{\lfloor \alpha(n+1)\rfloor}{n+1}$ as follows:
    \begin{itemize}
        \item[$\Leftarrow$] Assume $\hat{p}_{n+1} \leq \frac{\lfloor \alpha(n+1)\rfloor}{n+1}$. Therefore, $\hat{p}_{n+1} \leq \frac{\lfloor \alpha(n+1)\rfloor}{n+1} \leq \frac{\alpha(n+1)}{n+1} = \alpha$.
        \item[$\Rightarrow$] Assume $\hat{p}_{n+1} \leq \alpha$, then $\frac{k}{n+1}\leq \alpha$. This implies that 
        $k \leq \alpha (n+1)$. Since $k$ is an integer, it follows that $k \leq \lfloor \alpha (n+1) \rfloor$. Therefore, $\hat{p}_{n+1} = \frac{k}{n+1} \leq \frac{\lfloor \alpha (n+1) \rfloor}{n+1}$.
    \end{itemize}
    \item (ii) This step follows directly from the equality $n + 1 = \lceil (1-\alpha)(n+1)\rceil + \lfloor \alpha(n+1)\rfloor$. We explicitly prove this equality as follows:
    \begin{itemize}
        \item The term $\lceil (1-\alpha) (n+1)\rceil$ represents the smallest integer greater than or equal to $(1-\alpha)(n+1)$. Hence, we can write:
        \begin{align*}
            \lceil (1-\alpha)(n+1)\rceil = (1-\alpha)(n+1) + \delta_1, 
        \end{align*}
        where $0\leq\delta_1<1$.
        \item Similarly, the term $\lfloor \alpha (n+1)\rfloor$ represents the largest integer less than or equal to $\alpha(n+1)$. Thus:
        \begin{align*}
            \lfloor \alpha(n+1)\rfloor = \alpha(n+1) - \delta_2, 
        \end{align*}
        where $0\leq\delta_2<1$.
        \item Adding these two terms gives:
    \begin{align*}
        \lfloor \alpha(n+1)\rfloor + \lceil (1-\alpha)(n+1)\rceil = \alpha(n+1) - \delta_2 + (1-\alpha)(n+1) + \delta_1 = n + 1 + (\delta_1 - \delta_2).
    \end{align*}
    Since $\lfloor \alpha(n+1)\rfloor + \lceil (1-\alpha)(n+1)\rceil$ must be an integer and $\delta_1,\delta_2\in [0,1)$, it follows that $\delta_1 - \delta_2 = 0$.
    \end{itemize}
    
    Therefore, $\lfloor \alpha(n+1)\rfloor + \lceil (1-\alpha)(n+1)\rceil = n + 1$.
\end{itemize}

To complete the proof, we now show that \eqref{app-eq:prop-inequality} holds if and only if $s(X_{n+1}) > \hat{Q}_{1-\alpha}$. 
\begin{itemize}
    \item[$\Leftarrow$] Assume $s(X_{n+1}) > \hat{Q}_{1-\alpha}$. 
    By definition, $\sum_{i=1}^{n} \mathbb{I}[s(X_i) \leq \hat{Q}_{1-\alpha}] = \lceil (1-\alpha)(n+1)\rceil$.
    Then, \eqref{app-eq:prop-inequality} holds since
    \begin{align*}
        \sum_{i=1}^{n} \mathbb{I}[s(X_i) < s(X_{n+1})] \geq \sum_{i=1}^{n} \mathbb{I}[s(X_i) \leq \hat{Q}_{1-\alpha}] =  \lceil (1-\alpha)(n+1)\rceil.
    \end{align*}
    \item[$\Rightarrow$] We prove this direction by contradiction, assuming that \eqref{app-eq:prop-inequality} holds. Now, suppose that $s(X_{n+1})\leq \hat{Q}_{1-\alpha}$ also holds, implying that
    \begin{align*}
        \sum_{i=1}^{n} \mathbb{I}[s(X_i) < s(X_{n+1})] \leq \sum_{i=1}^{n} \mathbb{I}[s(X_i) < \hat{Q}_{1-\alpha}] \overset{(i)}{<} \lceil (1-\alpha) (n+1) \rceil,
    \end{align*}
    which contradicts the assumption \eqref{app-eq:prop-inequality}. Therefore, we conclude that $s(X_{n+1}) > \hat{Q}_{1-\alpha}$. The last step above can be explained as follows.
    \begin{itemize}
        \item (i) Recall that by definition, $\hat{Q}_{1-\alpha}$ is a specific value in $\{s(X_i)\}_{i=1}^{n}$ and $\sum_{i=1}^{n} \mathbb{I}[s(X_i) \leq \hat{Q}_{1-\alpha}] = \lceil (1-\alpha)(n+1)\rceil$.
This implies that
\begin{align*}
%\label{eq:quantile_and_p}
    \sum_{i=1}^{n} \mathbb{I}[s(X_i) < \hat{Q}_{1-\alpha}] 
 = \lceil (1-\alpha)(n+1)\rceil - \sum_{i=1}^{n} \mathbb{I}[s(X_i) = \hat{Q}_{1-\alpha}] < \lceil (1-\alpha)(n+1)\rceil.
\end{align*}
    \end{itemize}
\end{itemize}
In sum, $\hat{p}_{n+1}\leq \alpha$ holds if and only if \eqref{app-eq:prop-inequality} holds, and the latter holds if and only if $s(X_{n+1}) > \hat{Q}_{1-\alpha}$. This completes the proof.
\end{proof}

\begin{proposition}\label{app-prop:quantiles}
    Let $\D$ be a dataset containing $n$ scores, and let $S_{n+1}$ be a test score. Define the following thresholds:
        \begin{align*}
\hat{Q}_{1-\alpha} &:= \hat{i} \text{-th smallest element in } \D\cup \{\infty\},
\end{align*}
where 
\begin{align*}
    \hat{i} :=\lceil (1-\alpha)(n+1)\rceil.
\end{align*}
Similarly, let
    \begin{align*}
\hat{Q}_{1-\alpha}^{n+1} &:= \hat{i} \text{-th smallest element in } \D\cup \{S_{n+1}\}.
\end{align*}
It follows that
\begin{align*}
    \hat{Q}_{1-\alpha} \geq \hat{Q}_{1-\alpha}^{n+1} \text{ almost surely}.
\end{align*}
\end{proposition}

\begin{proof}[Proof of \Cref{app-prop:quantiles}]
    Since the largest possible score is $\infty$, the set $\D \cup \{\infty \}$ almost surely contains scores that are greater or equal to those in $\D \cup \{S_{n+1}\}$. Consequently, $\hat{Q}_{1-\alpha} := \hat{i}$-th smallest element in $\D \cup \{ \infty \}$ is almost surely greater than or equal to $\hat{Q}_{1-\alpha}^{n+1} := \hat{i}$-th smallest element in $\D \cup \{S_{n+1}\}$.
\end{proof}

\subsection{Explaining the Conservativeness of Standard Conformal p-Values}

\subsubsection{Proof of~\Cref{lem:conservativeness}}

\begin{proof}[Proof of~\Cref{lem:conservativeness}]
\label{prf:lem}
To simplify the notation define the random score $S_i := s(X_i)$ for all $i\in \D_{\mathrm{cal}} \cup \{n+1\}$. Throughout the proof, we refer to the calibration set as the set of nonconformity scores corresponding to the calibration points.
Without loss of generality, assume that the inliers in $\D_{\mathrm{cal}}$ are located at the first $n_0$ indices. Let $\D_{\mathrm{inlier}}=[n_0]$ denote the set of indices corresponding to the inlier scores. Consequently, define $\D_{\mathrm{outlier}} = \{n_0+1, n_0+2, \ldots, n\}$ as the set of indices corresponding to the outlier scores in $\D_{\mathrm{cal}}$.
We assume the scores have no ties (which can always be achieved by adding a negligible random noise to the scores output by any model).

Given a fixed realization of the score vector $(s_1,\ldots,s_n,s_{n+1}) \in \mathbb{R}^{n+1}$, define the following two events: 
    \begin{itemize}
        \item $E_{\mathrm{in}}$: the unordered set of inlier scores, including the test score, is $\{S_1,\dots,S_{n_0},S_{n+1}\} = \{s_1,\dots,s_{n_0},s_{n+1}\}$;
        \item $E_{\mathrm{out}}$: the unordered set of outlier scores is $\{S_{n_0+1}, \dots, S_n\} = \{s_{n_0+1}, \dots, s_n\}$.
    \end{itemize}

Under the setup defined in~\eqref{eq:setup-contaminated}, when $\mathcal{H}_0$ is true, the test score $S_{n+1}$ and the inlier scores in the calibration set are i.i.d.~from $\p_0$. 
    Therefore, by exchangeability, the following holds for each inlier index $i\in \D_{\mathrm{inlier}}\cup\{n+1\}$:
    \begin{align}\label{eq:exch-inlier}
        \p \left( S_{n+1} = s_i \mid E_{\mathrm{in}}, E_{\mathrm{out}}\right) =         \frac{1}{n_0 +1}.
    \end{align}
    Since the calibration scores are almost-surely distinct, the probability of a null test point obtaining any outlier score is zero. Therefore, for each outlier index $j\in \D_{\mathrm{outlier}}$:
    \begin{align}\label{eq:exch-outlier}
        \p \left( S_{n+1} = s_j \mid E_{\mathrm{in}}, E_{\mathrm{out}}\right) = 0.
    \end{align}

To obtain an upper bound on the type-I error rate, $\p \left( \hat{p}_{n+1} \leq \alpha \right)$, we use the equivalence established in~\Cref{app-prop:p-value-to-quantile}. According to this result, the following holds:
\begin{align*}
    \p \left( \hat{p}_{n+1} \leq \alpha \right) = \p \left( S_{n+1} > \hat{Q}_{1-\alpha}^{\mathrm{cal}}\right),
\end{align*}
where $\hat{Q}^{\mathrm{cal}}_{1-\alpha}$ is the $\hat{i}_{\mathrm{cal}}$-th smallest element in $\{S_i\}_{i=1}^{n}\cup\{\infty\}$ and $\hat{i}_{\mathrm{cal}} :=\lceil (1-\alpha)(n+1)\rceil$.

Moreover, define $\hat{Q}_{1-\alpha}^{n+1}$ as the $\hat{i}_{\mathrm{cal}}$-th smallest score in $\{S_i\}_{i=1}^{n+1}$.
By~\Cref{app-prop:quantiles}, $\hat{Q}^{\mathrm{cal}}_{1-\alpha} \geq \hat{Q}_{1-\alpha}^{n+1}$ almost surely.\\


Now, we obtain an upper bound for $\p \left( S_{n+1} > \hat{Q}^{\mathrm{cal}}_{1-\alpha} \mid E_{\mathrm{in}}, E_{\mathrm{out}}\right)$, where the probability is taken over random permutations of the scores conditional on $E_{\mathrm{in}}, E_{\mathrm{out}}$.
    \begin{align*} 
        \p \left( S_{n+1} > \hat{Q}^{\mathrm{cal}}_{1-\alpha} \mid E_{\mathrm{in}}, E_{\mathrm{out}}\right) &\leq \p \left( S_{n+1} > \hat{Q}^{\mathrm{n+1}}_{1-\alpha} \mid E_{\mathrm{in}}, E_{\mathrm{out}}\right)\\
        &= \E \left[ \mathbb{I} \left[ S_{n+1} > \hat{Q}^{\mathrm{n+1}}_{1-\alpha}\right]\mid E_{\mathrm{in}}, E_{\mathrm{out}} \right] \\
     &= \sum\limits_{i\in\D_{\mathrm{inlier}}\cup\{n+1\}} \E \left[ \mathbb{I} \left[S_{n+1}=s_i \right] \mathbb{I}\left[ s_i > \hat{Q}^{\mathrm{n+1}}_{1-\alpha}\right] \mid E_{\mathrm{in}}, E_{\mathrm{out}}\right]\\
          &\overset{(i)}{=} \sum\limits_{i\in\D_{\mathrm{inlier}}\cup\{n+1\}} \mathbb{I} \left[ s_i > \hat{Q}^{\mathrm{n+1}}_{1-\alpha}\right] \p \left(  S_{n+1}=s_i   \mid E_{\mathrm{in}}, E_{\mathrm{out}}\right)\\
        &= \frac{1}{n_0+1} \sum\limits_{i\in  \D_{\mathrm{inlier}} \cup \{ n+1\}} \mathbb{I} \left[ s_i > \hat{Q}^{\mathrm{n+1}}_{1-\alpha}\right] \\
        &\overset{(ii)}{\leq}  \frac{1}{n_0+1} \left(\alpha(n+1) - \sum\limits_{i\in  \D_{\mathrm{outlier}}} \mathbb{I} \left[ s_i > \hat{Q}^{\mathrm{n+1}}_{1-\alpha}\right]\right)\\
        &= \alpha + \frac{1}{n_0+1} \left( \alpha n_1 - \sum\limits_{i\in  \D_{\mathrm{outlier}}} \mathbb{I} \left[ s_i > \hat{Q}^{\mathrm{n+1}}_{1-\alpha}\right]\right)\\
        &= \alpha - \frac{n_1}{n_0+1} \left( 1-\alpha - \frac{1}{n_1} \sum\limits_{i\in  \D_{\mathrm{outlier}}} \mathbb{I} \left[ s_i \leq \hat{Q}^{\mathrm{n+1}}_{1-\alpha}\right]\right) \\
        &\overset{(iii)}{\leq} \alpha - \frac{n_1}{n_0+1} \left( 1-\alpha - \frac{1}{n_1} \sum\limits_{i\in  \D_{\mathrm{outlier}}} \mathbb{I} \left[ s_i \leq \hat{Q}^{\mathrm{cal}}_{1-\alpha}\right]\right) \\
        &= \alpha - \frac{n_1}{n_0 + 1} \left( 1-\alpha -\hat{F}_{1} \left( \hat{Q}^{\mathrm{cal}}_{1-\alpha} \right) \right),
    \end{align*}
where $\hat{F}_{1}$ is the empirical CDF of the outlier scores.
The labeled steps above can be explained as follows.
\begin{itemize}
\item (i) $\mathbb{I} \left[ s_i > \hat{Q}^{\mathrm{n+1}}_{1-\alpha} \right]$ is measurable with respect to the $\sigma$-algebra generated by $E_{\mathrm{in}}, E_{\mathrm{out}}$. This follows because $\hat{Q}_{1-\alpha}^{\mathrm{n+1}}$ is the $\hat{i}_{\mathrm{cal}}$-th smallest element of $\{s_1,\dots,s_{n+1}\}$, which is fully determined by these variables. Thus, we can pull it out of the expectation.\\
\item (ii) $\hat{Q}^{\mathrm{n+1}}_{1-\alpha}$ is the $\hat{i}_{\mathrm{cal}}$-th smallest score in $\{s_1,\dots,s_n,s_{n+1}\}$ 
  and $[n+1] = \D_{\mathrm{outlier}} \cup \D_{\mathrm{inlier}} \cup \{n+1\}$. By definition,\\ 
  \begin{align*}
    &\sum\limits_{i\in  \D_{\mathrm{inlier}} \cup \{ n+1\}} \mathbb{I}\left[ s_i > \hat{Q}^{\mathrm{n+1}}_{1-\alpha} \right] + \sum\limits_{i\in  \D_{\mathrm{outlier}}} \mathbb{I} \left[ s_i > \hat{Q}^{\mathrm{n+1}}_{1-\alpha} \right] = 
      \sum\limits_{i=1}^{n+1} \mathbb{I} \left[ s_i > \hat{Q}^{\mathrm{n+1}}_{1-\alpha} \right] = \lfloor\alpha(n+1) \rfloor
      \leq \alpha(n+1)\\
    \intertext{and therefore,}\\
    &\sum\limits_{i\in  \D_{\mathrm{inlier}} \cup \{ n+1\}} \mathbb{I}\left[ s_i > \hat{Q}^{\mathrm{n+1}}_{1-\alpha} \right]  \leq \alpha(n+1) - \sum\limits_{i\in  \D_{\mathrm{outlier}}} \mathbb{I}\left[ s_i > \hat{Q}^{\mathrm{n+1}}_{1-\alpha} \right]
  \end{align*}
\item (iii) Since $\hat{Q}^{\mathrm{cal}}_{1-\alpha} \geq \hat{Q}_{1-\alpha}^{n+1}$ almost surely, increasing the threshold (i.e., using $\hat{Q}^{\mathrm{cal}}_{1-\alpha}$) results in an equal or larger value of the sum.
\end{itemize}


Now, we can derive an upper bound for $\p \left(\hat{p}_{n+1} \leq \alpha\right)$ as follows:
\begin{align*}
\p \left(\hat{p}_{n+1} \leq \alpha\right) &= 
\p \left( S_{n+1} > \hat{Q}^{\mathrm{cal}}_{1-\alpha} \right)\\
&= \E\left[ \p \left( S_{n+1} > \hat{Q}^{\mathrm{cal}}_{1-\alpha} \mid E_{\mathrm{in}}, E_{\mathrm{out}}\right) \right]\\
    &\leq \E\left[\alpha - \frac{n_1}{n_0 + 1} \left( 1-\alpha - \hat{F}_{1} \left( \hat{Q}^{\mathrm{cal}}_{1-\alpha} \right)\ \right)\right]\\
    &= \alpha -\frac{n_1}{n_0 + 1} \left( 1-\alpha -   \E\left[ \hat{F}_{1} \left( \hat{Q}^{\mathrm{cal}}_{1-\alpha} \right) \right]\right),
\end{align*}
with this expectation being taken over different realizations of the inlier and outlier nonconformity scores.
\end{proof}

\subsubsection{An Alternative View Based on Mixture Distributions}

Next, we provide an additional theoretical result concerning the conservativeness of conformal outlier detection methods, to supplement the result presented in Section~\ref{sec:conservativeness} from a point of view closer to that of \citet{sesia2023adaptive}. 

Specifically, we consider a contaminated calibration set, $\D_{\mathrm{cal}}$, which may include both inliers (samples i.i.d. from $\p_0$) and outliers (samples i.i.d. from $\p_1 \neq \p_0$). The goal remains to test the null hypothesis $\mathcal{H}_0$ that a new data point $X_{n+1}$ is an inlier, independently sampled from $\p_0$.

This setup differs from \eqref{eq:setup-contaminated} in that the calibration set is drawn from a mixed distribution, where the proportion of outliers in the population is denoted by $\delta\in [0,1)$. Hence, the numbers of inliers and outliers in the calibration set are random, rather than fixed. Formally, this setup is expressed as:

\begin{align} \label{eq:setup-contaminated-random} 
\begin{split}
& X_i \overset{\text{i.i.d.}}{\sim} \p_{\mathrm{mixed}} = (1-\delta)\cdot \p_0 + \delta \cdot P_1, \quad \forall i \in \D_{\mathrm{cal}},\\ 
& \mathcal{H}_0 : X_{n+1} \overset{\text{ind.}}{\sim} \p_0.  
\end{split}\end{align}

Let $F_0$ and $F_1$ denote the CDFs of $\p_0$ and $\p_1$, respectively, and $\hat{Q}_{1-\alpha}^{\mathrm{cal}}$ represent the $\lceil (1-\alpha)(n+1)\rceil$-th smallest score in the calibration set.

\begin{corollary}[Conservativeness]
\label{cor:conservativeness-p-values}
    Under the setup defined in~\eqref{eq:setup-contaminated-random}, if $\mathcal{H}_0$ is true, then, for any $\alpha\in(0,1)$,
    \begin{align*}
    \p &\left( \hat{p}_{n+1}\leq \alpha \right) \leq \alpha - \delta \E \left[ F_0(\hat{Q}_{1-\alpha}^{\mathrm{cal}}) - F_1(\hat{Q}_{1-\alpha}^{\mathrm{cal}} )\right].
    \end{align*}
\end{corollary}

\Cref{cor:conservativeness-p-values} reformulates Theorem 1 in \citet{sesia2023adaptive} under the setup in \eqref{eq:setup-contaminated-random}. This result quantifies the behavior of conformal outlier detection methods in the presence of contaminated data and establishes guarantees on the type-I error rate. 
This result complements our analysis of the conservativeness of these methods.

\begin{proof}[Proof of~\Cref{cor:conservativeness-p-values}]
The proof adapts the argument of Theorem 1 in \citet{sesia2023adaptive} to the outlier detection setting considered here. Specifically, we follow the structure of the original proof, making adjustments to account for the presence of inliers and outliers in the calibration set.

By~\Cref{app-prop:p-value-to-quantile}, we have
\begin{align*}
    \hat{p}_{n+1} \leq \alpha \Longleftrightarrow S_{n+1} > \hat{Q}_{1-\alpha}^{\mathrm{cal}}.
\end{align*}

Under the null, we upper bound $\p \left( \hat{p}_{n+1}\leq \alpha\right)$ as follows:
    \begin{align*}
        \p_0 \left( \hat{p}_{n+1} \leq \alpha \right) &= \p_0 \left( \hat{p}_{n+1} \leq \alpha\right) + \p_{\mathrm{mixed}} \left( \hat{p}_{n+1} \leq \alpha\right) - \p_{\mathrm{mixed}} \left( \hat{p}_{n+1} \leq \alpha\right) \\
        &= \p_{\mathrm{mixed}} \left( \hat{p}_{n+1} \leq \alpha\right) - \left[ \p_{\mathrm{mixed}} \left( \hat{p}_{n+1} \leq \alpha\right) - \p_0 \left( \hat{p}_{n+1} \leq \alpha\right)\right]\\
        &\leq \alpha - \left[ \p_{\mathrm{mixed}} \left( \hat{p}_{n+1} \leq \alpha\right) - \p_0 \left( \hat{p}_{n+1} \leq \alpha\right)\right]\\
        &= \alpha - \left[ \left( (1-\delta)\p_0 \left( \hat{p}_{n+1} \leq \alpha\right) + \delta\p_1 \left( \hat{p}_{n+1} \leq \alpha\right)\right) - \p_0 \left( \hat{p}_{n+1} \leq \alpha\right) \right]\\
        &= \alpha - \delta\left[ \p_1 \left( \hat{p}_{n+1} \leq \alpha\right) - \p_0 \left( \hat{p}_{n+1} \leq \alpha\right)\right]\\
        &= \alpha - \delta\left[ \p_1 \left( S_{n+1} > \hat{Q}_{1-\alpha}^{\mathrm{cal}}\right) - \p_0 \left( S_{n+1} > \hat{Q}_{1-\alpha}^{\mathrm{cal}}\right) \right]\\
        &= \alpha - \delta\left[ \p_0 \left( S_{n+1} \leq \hat{Q}_{1-\alpha}^{\mathrm{cal}}\right) - \p_1 \left( S_{n+1} \leq \hat{Q}_{1-\alpha}^{\mathrm{cal}}\right) \right]\\
        &= \alpha - \delta\E \left[ \p_0 \left( S_{n+1} \leq \hat{Q}_{1-\alpha}^{\mathrm{cal}} \mid \D_{\mathrm{cal}}\right) - \p_1 \left( S_{n+1} \leq \hat{Q}_{1-\alpha}^{\mathrm{cal}} \mid \D_{\mathrm{cal}}\right) \right]\\
        &= \alpha - \delta\E \left[ F_0 \left( \hat{Q}_{1-\alpha}^{\mathrm{cal}} \right) - F_1 \left( \hat{Q}_{1-\alpha}^{\mathrm{cal}} \right) \right].
    \end{align*}
\end{proof}


\subsection{Validity of the \texttt{Label-Trim} Method}

\subsubsection{Proof of~\Cref{thm:labeled-trim} | Main Steps}

\begin{proof}[Proof of~\Cref{thm:labeled-trim}]
\label{prf:labeled-trim}
As in the proof of~\Cref{lem:conservativeness}, define the random score $S_i := s(X_i)$ for all $i\in \D_{\mathrm{cal}} \cup \{n+1\}$. 
By~\Cref{app-prop:p-value-to-quantile}, for any fixed $\alpha \in (0,1)$, the probability of a type-I error, $\p \left( \hat{p}_{n+1}^{\mathrm{LT}} \leq \alpha \right)$, can be expressed as
\begin{align*}
    \p \left( \hat{p}_{n+1}^{\mathrm{LT}} \leq \alpha \right) = \p \left( S_{n+1} > \hat{Q}_{1-\alpha}^{\mathrm{LT}}\right).
\end{align*}

Consider the augmented set $\{S_i\}_{i\in \D_{\mathrm{cal}}^{\mathrm{LT}}}\cup \{ S_{n+1}\}$, which includes the test score $S_{n+1}$. Define $\hat{Q}_{1-\alpha}^{\mathrm{LT,n+1}}$ as follows:
\begin{align*}
\hat{Q}_{1-\alpha}^{\mathrm{LT,n+1}} &:= \hat{i}_{\mathrm{LT}} \text{-th smallest element in } \{S_i\}_{i\in \D_{\mathrm{cal}}^{\mathrm{LT}}}\cup \{S_{n+1}\}.
\end{align*}
By~\Cref{app-prop:quantiles}, it holds that $\hat{Q}_{1-\alpha}^{\mathrm{LT}} \geq \hat{Q}_{1-\alpha}^{\mathrm{LT, n+1}}$ almost surely. 

Now, consider an imaginary ``mirror" version of this method that applies the label-trim algorithm with two key differences:
\begin{itemize}
\item it uses a larger labeling budget, $\tilde{m}=m+1$;
\item it treats $\{S_i\}_{i=1}^{n+1}$ as the calibration set instead of $\{S_i\}_{i=1}^{n}$---that is, it includes the test point in the annotation process, preserving the exchangeability with the calibration inliers.
\end{itemize}
Let $\tilde{\D}_{\mathrm{cal}}^{\mathrm{LT}}\cup \{ n+1\}$ denote the indices of the trimmed augmented calibration set produced by the mirror procedure, let $\tilde{\D}_{\mathrm{labeled}}$ denote the indices of the corresponding labeled data points, and define $\tilde{\D}_{\mathrm{labeled}}^{\mathrm{inlier}}, \tilde{\D}_{\mathrm{labeled}}^{\mathrm{outlier}}$ as the corresponding subsets of inliers and outliers, respectively.
Under this mirror procedure, the empirical quantile $\hat{Q}_{1-\alpha}^{\mathrm{LT,n+1}}$ corresponds to
\begin{align*}
\tilde{Q}_{1-\alpha}^{\mathrm{LT,n+1}} &:= \tilde{i}_{\mathrm{LT}} \text{-th smallest element in } \{S_i\}_{i \in \tilde{\D}_{\mathrm{cal}}^{\mathrm{LT}}} \cup \{S_{n+1}\},
\end{align*}
where $\tilde{i}_{\mathrm{LT}} := \lceil (1- \alpha ) (\tilde{n}^{\mathrm{LT}}+1) \rceil$ and $\tilde{n}_{\mathrm{LT}} := |\tilde{\D}_{\mathrm{cal}}^{\mathrm{LT}}|$.

By construction of this mirror procedure, $\tilde{i}_{\mathrm{LT}} \leq \hat{i}_{\mathrm{LT}}$ almost surely, because $\tilde{n}^{\mathrm{LT}} \leq n^{\mathrm{LT}}$ almost surely and thus
\begin{align*}
    \tilde{i}_{\mathrm{LT}} 
    & = \left \lceil ( 1-\alpha ) (\tilde{n}^{\mathrm{LT}}+1) \right \rceil
     \leq \lceil (1- \alpha )  (n^{\mathrm{LT}}+1) \rceil
     = \hat{i}_{\mathrm{LT}}.
\end{align*}

Using the fact that $\tilde{i}_{\mathrm{LT}}  \leq \hat{i}_{\mathrm{LT}}$ almost surely, we prove in Appendix~\ref{app:proof-label-trim-a1} that, almost surely,
\begin{align}\label{app-eq:tilde-hat-relation}
\tilde{Q}_{1-\alpha}^{\mathrm{LT,n+1}} \leq \hat{Q}_{1-\alpha}^{\mathrm{LT,n+1}}.
\end{align}
Since we already knew that $\hat{Q}_{1-\alpha}^{\mathrm{LT}} \geq \hat{Q}_{1-\alpha}^{\mathrm{LT,n+1}}$, this implies:
\begin{align} \label{app-eq:tilde-hat-relation-b}
    \hat{Q}_{1-\alpha}^{\mathrm{LT}} \geq \hat{Q}_{1-\alpha}^{\mathrm{LT,n+1}} \geq \tilde{Q}_{1-\alpha}^{\mathrm{LT,n+1}}.
\end{align}
Therefore, the type-I error rate of the Label-Trim approach can be bounded from above by the type-I error rate of the mirror procedure, which can be studied with an approach similar to that of the proof of~\Cref{lem:conservativeness}.


Let $\tilde{\D}_{\mathrm{outlier}}^{\mathrm{LT}}$ denote the outlier indices remaining in $\tilde{\D}_{\mathrm{cal}}^{\mathrm{LT}}$, with $\tilde{n}_{1}^{\mathrm{LT}}=|\tilde{\D}_{\mathrm{outlier}}^{\mathrm{LT}}|$.
 As in the proof of~\Cref{lem:conservativeness}, define $E_{\mathrm{in}}$ and $E_{\mathrm{out}}$ as two unordered realizations of the inlier and outlier scores in $\{S_i\}_{i=1}^{n+1}$, respectively. Appendix~\ref{app:proof-label-trim-a0} proves that
    \begin{align} \label{app-eq:type-I-error-cond}
    \p \left( S_{n+1} > \hat{Q}_{1-\alpha}^{\mathrm{LT}} \mid E_{\mathrm{in}}, E_{\mathrm{out}}, \tilde{\D}_{\mathrm{labeled}}, \tilde{\D}_{\mathrm{labeled}}^{\mathrm{inlier}}\right)
        \leq \alpha + \frac{1}{n_0+1} - \frac{\hat{n}_1^{\mathrm{LT}}}{n_0+1} \left( (1-\alpha) - \hat{F}_{1}^{\mathrm{LT}}(\hat{Q}_{1-\alpha}^{\mathrm{LT}}) \right),
    \end{align}
from which it follows immediately, by marginalizing over $E_{\mathrm{in}}, E_{\mathrm{out}}, \tilde{\D}_{\mathrm{labeled}},$ and $\tilde{\D}_{\mathrm{labeled}}^{\mathrm{inlier}}$, that
 \begin{align*}
    \p \left( S_{n+1} > \hat{Q}_{1-\alpha}^{\mathrm{LT}} \right)
   & \leq \E\left[ \alpha + \frac{1}{n_0+1} - \frac{\hat{n}^{\mathrm{LT}}_{1}}{n_0+1} \left( (1-\alpha) - \hat{F}_{1}^{\mathrm{LT}} \left( \hat{Q}_{1-\alpha}^{\mathrm{LT}} \right) \right)\right].
    \end{align*}
\end{proof}


\subsubsection{Proof of~\Cref{thm:labeled-trim} | Proof of Equation~\eqref{app-eq:tilde-hat-relation}} \label{app:proof-label-trim-a1}

\begin{proof}
We prove~\eqref{app-eq:tilde-hat-relation} by analyzing two distinct cases, depending on whether $S_{n+1}$ is among the $m + 1$ largest scores or not.

Recall that $m \leq \alpha (n+1)$, by assumption, and $n^{\mathrm{LT}} \leq n$.
It is easy to see that this implies that, almost surely,
    \begin{align} \label{eq:app-C1}
        \hat{i}_{\mathrm{LT}} = \lceil (1-\alpha)(n^{\mathrm{LT}}+1)\rceil \leq n+1 - (m+1).
    \end{align}

The mirror procedure labels and potentially removes the largest $m+1$ scores out of the $n+1$ total scores. Thus, $\hat{Q}_{1-\alpha}^{\mathrm{LT,n+1}}$ is smaller than all the outliers removed during trimming, i.e., $\hat{Q}^{\mathrm{LT,n+1}}_{1-\alpha} < S_j$ for all $S_j \in \tilde{\D}_{\mathrm{labeled}}^{\mathrm{outlier}}$.


\begin{itemize}
\item Suppose $S_{n+1}$ is among the $m + 1$ largest scores in $\{S_i\}_{i=1}^{n+1}$. This case is illustrated below:\\
\begin{minipage}{.4\textwidth}
In this scenario, the mirror trimming approach additionally labels the test score $S_{n+1}$, which, under the null hypothesis, is an inlier and is thus not removed. As a result, both trimming procedures yield the same set; i.e.,
$\tilde{\D}_{\mathrm{cal}}^{\mathrm{LT}} = \D_{\mathrm{cal}}^{\mathrm{LT}}$. Consequently, $\hat{Q}_{1-\alpha}^{\mathrm{LT,n+1}} = \tilde{Q}_{1-\alpha}^{\mathrm{LT,n+1}}$.
\end{minipage}
\begin{minipage}{.05\textwidth}
\hspace{1pt}
\end{minipage}
\begin{minipage}{.5\textwidth}
\centering
\includegraphics[width=\linewidth]{figures/theorem3.1_proof/idx-thm3.1-case-1.pdf}
\end{minipage}

\item Suppose $S_{n+1}$ is not among the $m + 1$ largest scores in $\{S_i\}_{i=1}^{n+1}$. Within this case, there are two sub-cases to consider.
\begin{itemize}
    \item  The $(m+1)$-th largest score in an inlier:\\
    \begin{minipage}[t]{.4\textwidth}
In this case, the mirror trimming approach additionally labels an inlier, which is not removed. As a result, both trimming procedures yield the same set; i.e.,
$\tilde{\D}_{\mathrm{cal}}^{\mathrm{LT}} = \D_{\mathrm{cal}}^{\mathrm{LT}}$. Consequently, $\hat{Q}_{1-\alpha}^{\mathrm{LT,n+1}} = \tilde{Q}_{1-\alpha}^{\mathrm{LT,n+1}}$.
\end{minipage}
\begin{minipage}[t]{0.05\textwidth}
    \hspace{1pt}
\end{minipage}
\begin{minipage}[t]{.5\textwidth}
\vspace{-40pt}
\centering
\includegraphics[width=\linewidth, valign=t]{figures/theorem3.1_proof/idx-thm3.1-case-2-1.pdf}
\end{minipage}
    \item The $(m+1)$-th largest score in an outlier:\\
    \begin{minipage}[t]{.4\textwidth}
This is the interesting case where $\tilde{n}^{\mathrm{LT}} = n^{\mathrm{LT}}-1$ and the set $\tilde{\D}_{\mathrm{cal}}^{\mathrm{LT}}$ contains one fewer outlier score than $\D_{\mathrm{cal}}^{\mathrm{LT}}$. 
It follows from~\eqref{eq:app-C1} that the threshold $\hat{Q}_{1-\alpha}^{\mathrm{LT,n+1}}$ is smaller than the $m+1$ largest scores. Then, in this region,  $\D_{\mathrm{cal}}^{\mathrm{LT}}$ and $\tilde{\D}_{\mathrm{cal}}^{\mathrm{LT}}$ contain the same scores. Therefore, the $\hat{i}_{\mathrm{LT}}$-th smallest score corresponds to the same score in both $\D_{\mathrm{cal}}^{\mathrm{LT}}\cup\{n+1\}$ and $\tilde{\D}_{\mathrm{cal}}^{\mathrm{LT}} \cup \{n+1\}$.
Since $\tilde{i}_{\mathrm{LT}} \leq \hat{i}_{\mathrm{LT}}$, it follows that $\tilde{Q}_{1-\alpha}^{\mathrm{LT,n+1}} \leq \hat{Q}_{1-\alpha}^{\mathrm{LT,n+1}}$.
\end{minipage}
\begin{minipage}[t]{0.05\textwidth}
    \hspace{1pt}
\end{minipage}
\begin{minipage}[t]{.5\textwidth}
\centering
\includegraphics[width=\linewidth, valign=t]{figures/theorem3.1_proof/idx-thm3.1-case-2-2.pdf}
\end{minipage}
\end{itemize}
\end{itemize}

\end{proof}


\subsubsection{Proof of~\Cref{thm:labeled-trim} | Proof of Equation~\eqref{app-eq:type-I-error-cond}} \label{app:proof-label-trim-a0}

\begin{proof}
    \begin{align*}
    & \p \left( S_{n+1} > \hat{Q}_{1-\alpha}^{\mathrm{LT}} \mid E_{\mathrm{in}}, E_{\mathrm{out}}, \tilde{\D}_{\mathrm{labeled}}, \tilde{\D}_{\mathrm{labeled}}^{\mathrm{inlier}}\right) \\
        & \qquad \leq \p \left( S_{n+1} > \tilde{Q}_{1-\alpha}^{\mathrm{LT,n+1}} \mid E_{\mathrm{in}}, E_{\mathrm{out}}, \tilde{\D}_{\mathrm{labeled}}, \tilde{\D}_{\mathrm{labeled}}^{\mathrm{inlier}}\right) \\
        & \qquad = \sum\limits_{i\in \D_{\mathrm{inlier}} \cup \{ n+1\} } \E \left[ \mathbb{I} \left[ s_i > \tilde{Q}_{1-\alpha}^{\mathrm{LT,n+1}} \right] \cdot \mathbb{I} \left[ S_{n+1} = s_i \right]\mid E_{\mathrm{in}}, E_{\mathrm{out}}, \tilde{\D}_{\mathrm{labeled}}, \tilde{\D}_{\mathrm{labeled}}^{\mathrm{inlier}}\right] \\
        & \qquad \overset{(i)}{=} \sum\limits_{i\in \D_{\mathrm{inlier}} \cup \{ n+1\} }         \mathbb{I} \left[ s_i > \tilde{Q}_{1-\alpha}^{\mathrm{LT,n+1}} \right] \cdot \p \left( S_{n+1} = s_i \mid E_{\mathrm{in}}, E_{\mathrm{out}}, \tilde{\D}_{\mathrm{labeled}}, \tilde{\D}_{\mathrm{labeled}}^{\mathrm{inlier}}\right) \\
        & \qquad \overset{(ii)}{=} \sum\limits_{i\in \D_{\mathrm{inlier}} \cup \{ n+1\} }  \mathbb{I} \left[ s_i > \tilde{Q}_{1-\alpha}^{\mathrm{LT,n+1}}\right] \cdot \p \left( S_{n+1} = s_i \mid E_{\mathrm{in}}, E_{\mathrm{out}} \right) 
        \\
        & \qquad = \frac{1}{n_0+1}  \sum\limits_{i\in  \D_{\mathrm{inlier}} \cup \{ n+1\}} \mathbb{I} \left[ s_i > \tilde{Q}_{1-\alpha}^{\mathrm{LT,n+1}}\right] \\
        & \qquad \overset{(iii)}{\leq}  \frac{1}{n_0+1} \left( \alpha(\tilde{n}^{\mathrm{LT}}+1) - \sum\limits_{i\in  \tilde{\D}_{\mathrm{outlier}}^{\mathrm{LT}}} \mathbb{I} \left[ s_i > \tilde{Q}_{1-\alpha}^{\mathrm{LT,n+1}}\right] \right)\\
        & \qquad \overset{(iv)}{\leq}  \frac{1}{n_0+1} \left( \alpha(n^{\mathrm{LT}}+1) - \sum\limits_{i\in \D_{\mathrm{outlier}}^{\mathrm{LT}}} \mathbb{I} \left[ s_i > \tilde{Q}_{1-\alpha}^{\mathrm{LT,n+1}}\right] + 1\right)\\
        & \qquad \overset{(v)}{\leq}  \frac{1}{n_0+1} \left( \alpha(n^{\mathrm{LT}}+1) - \sum\limits_{i\in \D_{\mathrm{outlier}}^{\mathrm{LT}}} \mathbb{I} \left[ s_i > \hat{Q}_{1-\alpha}^{\mathrm{LT}}\right] + 1\right)\\
        & \qquad = \frac{1}{n_0+1} \left( \alpha(n^{\mathrm{LT}}+1) - \hat{n}_1^{\mathrm{LT}} + \sum\limits_{i\in \D_{\mathrm{outlier}}^{\mathrm{LT}}} \mathbb{I} \left[ s_i \leq \hat{Q}_{1-\alpha}^{\mathrm{LT}}\right] + 1\right)\\
        & \qquad = \alpha + \frac{1}{n_0+1} - \frac{\hat{n}_1^{\mathrm{LT}}}{n_0+1} \left( (1-\alpha) - \hat{F}_{1}^{\mathrm{LT}}(\hat{Q}_{1-\alpha}^{\mathrm{LT}}) \right).
    \end{align*}


The labeled steps above can be explained as follows.
\begin{itemize}
\item     (i) $\mathbb{I}\left[ s_i > \tilde{Q}_{1-\alpha}^{\mathrm{LT,n+1}}\right]$ is measurable with respect to the $\sigma$-algebra generated by $E_{\mathrm{in}}, E_{\mathrm{out}}, \tilde{\D}_{\mathrm{labeled}},$ and $\tilde{\D}_{\mathrm{labeled}}^{\mathrm{inlier}}$ since $\tilde{Q}_{1-\alpha}^{\mathrm{LT,n+1}}$ is the $\tilde{i}_{\mathrm{LT}}$-th smallest element of $\{s_1,\dots,s_{n+1}\} \setminus \left(\tilde{\D}_{\mathrm{labeled}} \setminus \tilde{\D}_{\mathrm{labeled}}^{\mathrm{inlier}}\right)$.
\item    (ii) The mirror procedure is applied on $\{S_i\}_{i=1}^{n+1}$, preserving the exchangeability of the test score $S_{n+1}$ with the calibration inliers. Hence, the resulting labeled sets $\tilde{\D}_{\mathrm{labeled}}$ and $\tilde{\D}_{\mathrm{labeled}}^{\mathrm{inlier}}$ contain no additional information about $S_{n+1}$ beyond $E_{\mathrm{in}}, E_{\mathrm{out}}$.
\item    (iii) By definition, $\tilde{Q}_{1-\alpha}^{\mathrm{LT,n+1}}$ is the $\tilde{i}_{\mathrm{LT}}$-th smallest element of $\{S_i\}_{i\in \tilde{\D}_{\mathrm{cal}}^{\mathrm{LT}}} \cup \{S_{n+1}\}$, where $\tilde{i}_{\mathrm{LT}}=\lceil(1-\alpha)(\tilde{n}^{\mathrm{LT}}+1)\rceil$. Consequently, $\lfloor\alpha(\tilde{n}^{\mathrm{LT}}+1)\rfloor$ scores in $\{S_i\}_{i\in \tilde{\D}_{\mathrm{cal}}^{\mathrm{LT}}} \cup \{S_{n+1}\}$ are larger than $\tilde{Q}_{1-\alpha}^{\mathrm{LT,n+1}}$.
\item (iv) The set $\tilde{\D}_{\mathrm{outlier}}^{\mathrm{LT}}$ is either equal to $\D_{\mathrm{outlier}}^{\mathrm{LT}}$ or contains one fewer outlier, and $\tilde{n}^{\mathrm{LT}} \in \{n^{\mathrm{LT}}, n^{\mathrm{LT}}-1\}$ almost surely.
\item  (v) Recall from~\eqref{app-eq:tilde-hat-relation-b} that $\hat{Q}_{1-\alpha}^{\mathrm{LT}} \geq \tilde{Q}_{1-\alpha}^{\mathrm{LT,n+1}}$ almost surely.
\end{itemize}

\end{proof}



\section{Supplementary Experiments and Implementation Details}
\subsection{Datasets}\label{app-sec:data}
\Cref{app-tab:tabular-info} summarizes details of the three tabular benchmark datasets. For all tabular datasets, we perform random subsampling to construct contaminated train, calibration, and test sets. Specifically, for the shuttle and KDDCup99 datasets, the train set contains 5,000 samples, and the calibration set contains 2,500 samples, both with a contamination rate of $r=3\%$, unless stated otherwise. The inlier and outlier test sets consist of 950 and 50 samples, respectively. For the credit-card dataset, the train set contains 2,000 samples, while the calibration and test sets follow the same setup as the Shuttle and KDDCup99 datasets.
\begin{table}[ht]
\centering
\caption{Summary of tabular datasets}
\label{app-tab:tabular-info}
\begin{tabular}{lccc}
\toprule
Dataset              & Shuttle \citep{shuttle} & Credit-card \citep{creditcard} & KDDCup99 \citep{KDDCup99} \\ 
\midrule
Total Samples        & 58,000           & 284,807              & 494,020           \\ 
Number of Outliers   & 12,414           & 492                  & 396,743           \\ 
Number of Features   & 9                & 29                   & 41                \\ 
\bottomrule
\end{tabular}
\end{table}

For visual datasets, we use the OpenOOD benchmark \citep{zhang2023openood, yang2022openood}. Specifically, for each dataset and contamination rate, we perform a one-time training of the ReAct outlier detection model~\citep{react}, which operates on feature representations extracted from a pre-trained ResNet-18~\citep{zhang2023openood,he2016deep}. The model applies a percentile-based threshold (set to 90\%) to truncate activations, where the threshold is computed on the contaminated train set. These truncated activations then pass through the fully connected layer of the pre-trained ResNet-18.
The outlier score is computed using an energy-based log-sum-exp function applied to these truncated activations. 
After training, we save the outlier scores for the remaining outlier samples and the CIFAR-10 test set. We randomly subsample this pool of scores to construct the calibration and test sets, ensuring that all sets are disjoint.

The sizes of the train and calibration sets are 2,000 and 3,000, respectively, with the same contamination rate.
The inlier and outlier test sets consist of 950 and 50 samples, respectively.
\FloatBarrier

\subsection{Tabular Datasets}
\label{app-sec:real-data-exp}
In this section, we provide additional real-data experiments conducted on the {\em credit card} \citep{creditcard} and {\em KDDCup99} datasets,
complementing the analysis provided for the shuttle dataset in the main manuscript. Each figure in this section corresponds to and extends the figures presented in the main text.

\paragraph{Results as a function of the contamination rate} In~\Cref{fig:shuttle-outlier-prop} of the main manuscript, we analyze the performance of conformal outlier methods as a function of the contamination rate $r$. Here, we repeat the same experiment on the credit-card and KDDCup99 datasets (Figures \ref{app-fig:creditcard-outlier-prop} and \ref{app-fig:KDDCup99-outlier-prop}). The performance trends are similar to the one presented in the main manuscript: both the \texttt{Standard} and \texttt{Small-Clean} methods achieve valid type-I error rate but exhibit conservative behavior. In contrast, the \texttt{Naive-Trim} method fails to control the type-I error rate. The \texttt{Label-Trim} method attains improved power while practically controlling the type-I error at level $\alpha$.

\begin{figure*}[htb]
    % \centering 
    \includegraphics[height=3.3cm, valign=t]{figures/exp/real_data/creditcard/outlier_prop/IF_e_100_s_auto_train_2000_exp_outliers_calib_creditcard_1_fdr_model_0.5_initial_50_cal_2500_p_0.05_test_1000_p_0.05_q_0.02/Type-1-Error_point_no_legend.pdf}
    \includegraphics[height=3.3cm, valign=t]{figures/exp/real_data/creditcard/outlier_prop/IF_e_100_s_auto_train_2000_exp_outliers_calib_creditcard_1_fdr_model_0.5_initial_50_cal_2500_p_0.05_test_1000_p_0.05_q_0.02/Power_point_no_legend.pdf}
    \includegraphics[height=3.3cm, valign=t]{figures/exp/real_data/creditcard/outlier_prop/IF_e_100_s_auto_train_2000_exp_outliers_calib_creditcard_1_fdr_model_0.5_initial_50_cal_2500_p_0.05_test_1000_p_0.05_q_0.02/Trimmed_point_no_legend.pdf}
    \includegraphics[width=3.3cm, valign=t]{figures/exp/legend.pdf}
    \caption{Comparison of conformal outlier detection methods on real dataset ``credit-card'' as a function of the contamination rate  $r$. Other details are as in \Cref{fig:shuttle-outlier-prop}. 
}
    \label{app-fig:creditcard-outlier-prop}
\end{figure*}

\begin{figure*}[htb]
    % \centering 
    \includegraphics[height=3.3cm, valign=t]{figures/exp/real_data/KDDCup99/outlier_prop/IF_e_100_s_auto_train_5000_exp_outliers_calib_KDDCup99_1_fdr_model_0.5_initial_50_cal_2500_p_0.05_test_1000_p_0.05_q_0.02/Type-1-Error_point_no_legend.pdf}
    \includegraphics[height=3.3cm, valign=t]{figures/exp/real_data/KDDCup99/outlier_prop/IF_e_100_s_auto_train_5000_exp_outliers_calib_KDDCup99_1_fdr_model_0.5_initial_50_cal_2500_p_0.05_test_1000_p_0.05_q_0.02/Power_point_no_legend.pdf}
    \includegraphics[height=3.3cm, valign=t]{figures/exp/real_data/KDDCup99/outlier_prop/IF_e_100_s_auto_train_5000_exp_outliers_calib_KDDCup99_1_fdr_model_0.5_initial_50_cal_2500_p_0.05_test_1000_p_0.05_q_0.02/Trimmed_point_no_legend.pdf}
    \includegraphics[width=3.3cm, valign=t]{figures/exp/legend.pdf}
    \caption{Comparison of conformal outlier detection methods on real dataset ``KDDCup99'' as a function of the contamination rate  $r$. Other details are as in \Cref{fig:shuttle-outlier-prop}. 
}
    \label{app-fig:KDDCup99-outlier-prop}
\end{figure*}

\paragraph{Results as a function of the labeling budget} In~\Cref{fig:shuttle-labeled-exp} of the main manuscript, we evaluate the performance of the \texttt{Label-Trim}, \texttt{Small-Clean} and, \texttt{Oracle} methods as a function of the labeling budget $m$. \Cref{app-fig:creditcard-labeled-exp,app-fig:KDDCup99-labeled-exp} extend this analysis to the credit-card and KDDCup99 datasets, respectively. Consistent with the trends observed in~\Cref{fig:shuttle-labeled-exp}, increasing the labeling budget improves the performance of the \texttt{Label-Trim} method in terms of both type-I error and power. Notably, although the condition in \Cref{thm:labeled-trim} is no longer satisfied for $m > 50$, the \texttt{Label-Trim} method still maintains valid type-I error control at the desired level $\alpha$. 

For the KDDCup99 dataset, however, we observe that the \texttt{Small-Clean} method shows higher power than the \texttt{Oracle} across several labeling budgets ($m \geq 55$). This is due to the high variability in the dataset and the small sample size used by this method, resulting in significant variance in type-I error. To illustrate this, \Cref{app-fig:KDDCup99-labeled-exp-55-105} presents a box plot showing the variability and instability of the \texttt{Small-Clean} method in this regime. While this leads to a higher average power, this variability is undesirable in practice.

\begin{figure*}[!htb]
    % \centering 
    \includegraphics[height=3.3cm, valign=t]{figures/exp/real_data/creditcard/labeled_size/IF_e_100_s_auto_train_2000_exp_clean_calib_size_creditcard_1_fdr_model_0.5_initial_50_cal_2500_p_0.03_test_1000_p_0.05_q_0.02/Type-1-Error_point_no_legend.pdf}
    \includegraphics[height=3.3cm, valign=t]{figures/exp/real_data/creditcard/labeled_size/IF_e_100_s_auto_train_2000_exp_clean_calib_size_creditcard_1_fdr_model_0.5_initial_50_cal_2500_p_0.03_test_1000_p_0.05_q_0.02/Power_point_no_legend.pdf}
    \includegraphics[height=3.3cm, valign=t]{figures/exp/real_data/creditcard/labeled_size/IF_e_100_s_auto_train_2000_exp_clean_calib_size_creditcard_1_fdr_model_0.5_initial_50_cal_2500_p_0.03_test_1000_p_0.05_q_0.02/Trimmed_point_no_legend.pdf}
    \includegraphics[width=3.3cm, valign=t]{figures/exp/legend_wo_n.pdf}
    \caption{Performance on real dataset ``credit-card'' as a function of the labeling budget $m$. The contamination rate is $r=0.03$. Other details are as in \Cref{fig:shuttle-outlier-prop}.
}
    \label{app-fig:creditcard-labeled-exp}
\end{figure*}

\begin{figure*}[!htb]
    % \centering 
    \includegraphics[height=3.3cm, valign=t]{figures/exp/real_data/KDDCup99/labeled_size/IF_e_100_s_auto_train_5000_exp_clean_calib_size_KDDCup99_1_fdr_model_0.5_initial_50_cal_2500_p_0.03_test_1000_p_0.05_q_0.02_n_seeds_400/Type-1-Error_point_no_legend.pdf}
    \includegraphics[height=3.3cm, valign=t]{figures/exp/real_data/KDDCup99/labeled_size/IF_e_100_s_auto_train_5000_exp_clean_calib_size_KDDCup99_1_fdr_model_0.5_initial_50_cal_2500_p_0.03_test_1000_p_0.05_q_0.02_n_seeds_400/Power_point_no_legend.pdf}
    \includegraphics[height=3.3cm, valign=t]{figures/exp/real_data/KDDCup99/labeled_size/IF_e_100_s_auto_train_5000_exp_clean_calib_size_KDDCup99_1_fdr_model_0.5_initial_50_cal_2500_p_0.03_test_1000_p_0.05_q_0.02_n_seeds_400/Trimmed_point_no_legend.pdf}
    \includegraphics[width=3.3cm, valign=t]{figures/exp/legend_wo_n.pdf}
    \caption{Performance on real dataset ``KDDCup99'' as a function of the labeling budget $m$. The contamination rate is $r=0.03$. Results are averages across 400 random splits of the data. Other details are as in \Cref{fig:shuttle-outlier-prop}.
}
    \label{app-fig:KDDCup99-labeled-exp}
\end{figure*}

\begin{figure*}[!htb]
    \centering 
    \includegraphics[width=0.33\textwidth, valign=t]{figures/exp/real_data/KDDCup99/labeled_size/IF_e_100_s_auto_train_5000_exp_clean_calib_size_KDDCup99_1_fdr_model_0.5_initial_50_cal_2500_p_0.03_test_1000_p_0.05_q_0.02_n_seeds_400/55-105/Type-1-Error_no_legend.pdf}
    \includegraphics[width=0.33\textwidth, valign=t]{figures/exp/real_data/KDDCup99/labeled_size/IF_e_100_s_auto_train_5000_exp_clean_calib_size_KDDCup99_1_fdr_model_0.5_initial_50_cal_2500_p_0.03_test_1000_p_0.05_q_0.02_n_seeds_400/55-105/Power_no_legend.pdf}
    \includegraphics[width=3.3cm, valign=t]{figures/exp/legend_wo_n_box.pdf}
    \caption{Performance on real dataset ``KDDCup99'' as a function of the labeling budget $m$. The contamination rate is $r=0.03$. Results are averages across 400 random splits of the data. Other details are as in \Cref{fig:shuttle-outlier-prop}.
}
    \label{app-fig:KDDCup99-labeled-exp-55-105}
\end{figure*}

\paragraph{Results as a function of the target type-I error level} In~\Cref{fig:shuttle-levels} we examine the performance of conformal outlier detection methods as a function of the target type-I error level $\alpha$. Here, we replicate the experiments on the credit-card and KDDCup99 datasets (\Cref{app-fig:creditcard-levels,app-fig:KDDCup99-levels}). In line with the trends observed in~\Cref{fig:shuttle-levels}, the \texttt{Label-Trim} method performs well at low type-I error rates. Notably, for $\alpha=0.01$, the \texttt{Label-Trim} method outperforms the baselines, while practically controlling the type-I error at level $\alpha$, even though the condition of~\Cref{thm:labeled-trim} is not satisfied in this case. This highlights the robustness of our approach.

\begin{figure*}[!htb]
    \centering 
    \includegraphics[height=3.3cm, valign=t]{figures/exp/real_data/creditcard/levels/IF_e_100_s_auto_train_2000_exp_levels_creditcard_1_fdr_model_0.5_initial_50_cal_2500_p_0.03_test_1000_p_0.05_q_0.02/Type-1-Error_point_no_legend.pdf}
    \includegraphics[height=3.3cm, valign=t]{figures/exp/real_data/creditcard/levels/IF_e_100_s_auto_train_2000_exp_levels_creditcard_1_fdr_model_0.5_initial_50_cal_2500_p_0.03_test_1000_p_0.05_q_0.02/Power_point_no_legend.pdf}
    \includegraphics[width=3.3cm, valign=t]{figures/exp/legend_wo_trm.pdf}
    \caption{Comparison of conformal outlier detection methods on real dataset ``credit-card'' as a function of the target type-I error rate $\alpha$. The contamination rate $r$ is fixed to 3\%; other details are as in \Cref{fig:shuttle-outlier-prop}.
}
    \label{app-fig:creditcard-levels}
\end{figure*}

\begin{figure*}[!htb]
    \centering 
    \includegraphics[height=3.3cm, valign=t]{figures/exp/real_data/KDDCup99/levels/IF_e_100_s_auto_train_5000_exp_levels_KDDCup99_1_fdr_model_0.5_initial_50_cal_2500_p_0.03_test_1000_p_0.05_q_0.02/Type-1-Error_point_no_legend.pdf}
    \includegraphics[height=3.3cm, valign=t]{figures/exp/real_data/KDDCup99/levels/IF_e_100_s_auto_train_5000_exp_levels_KDDCup99_1_fdr_model_0.5_initial_50_cal_2500_p_0.03_test_1000_p_0.05_q_0.02/Power_point_no_legend.pdf}
    \includegraphics[width=3.3cm, valign=t]{figures/exp/legend_wo_trm.pdf}
    \caption{Comparison of conformal outlier detection methods on real dataset ``KDDCup99'' as a function of the target type-I error rate $\alpha$. The contamination rate $r$ is fixed to 3\%; other details are as in \Cref{fig:shuttle-outlier-prop}.
}
    \label{app-fig:KDDCup99-levels}
\end{figure*}

% \clearpage
\FloatBarrier

\subsection{Visual Datasets}\label{app-sec:images-data-exp}
This section provides detailed results for each dataset, covering various contamination rates and target type-I error levels.

\Cref{app-tab:all} summarizes the results across all six datasets for a target type-I error rate of $\alpha = 0.02$, showing trends consistent with those observed in \Cref{tab:avg-images} of the main manuscript.

\Cref{app-tab:texture,app-tab:svhn,app-tab:places365,app-tab:mnist,app-tab:cifar100,app-tab:tin} present the type-I error rates and the power of each method, with the power reported relative to the \texttt{Standard} method (normalized to 1). The actual power of the \texttt{Standard} method is included in the last row of each table.

\begin{table}[!htb]
\caption{Comparison of conformal outlier detection methods on six visual datasets for varying contamination rate $r$ and target type-I error level $\alpha = 0.02$. The empirical type-I error values are averaged across all datasets. The empirical power is presented relative to the \texttt{Standard} method (higher is better), and averaged across all datasets. Results are averaged across 100 random splits of the data, with standard errors presented in parentheses.}
\centering
\label{app-tab:all}

\resizebox{\textwidth}{!}{
\begin{tabular}{l|ll|ll|ll}
\hline
& \multicolumn{6}{c}{Contamination rate} \\
\hline
 & \multicolumn{2}{c|}{1\%} & \multicolumn{2}{c|}{3\%} & \multicolumn{2}{c}{5\%} \\ \hline
 Method      & Power & Type-I Error & Power & Type-I Error & Power & Type-I Error \\ \hline
Standard & \bfseries \cellcolor{Green!30} 1.0 ($\pm$ 0.0204) & \cellcolor{white} 0.017 ($\pm$ 0.0004)  & \bfseries \cellcolor{Green!30} 1.0 ($\pm$ 0.0238) & \cellcolor{white} 0.012 ($\pm$ 0.0004)  & \bfseries \cellcolor{Green!30} 1.0 ($\pm$ 0.0272) & \cellcolor{white} 0.009 ($\pm$ 0.0003) \\

Oracle (infeasible) & \bfseries \cellcolor{Green!100} 1.096 ($\pm$ 0.0214) & 0.021 ($\pm$ 0.0005)  & \bfseries \cellcolor{Green!100} 1.33 ($\pm$ 0.025) & \cellcolor{white} 0.02 ($\pm$ 0.0005)  & \bfseries \cellcolor{Green!100} 1.674 ($\pm$ 0.0311) & \cellcolor{white} 0.02 ($\pm$ 0.0006) \\

Naive-Trim (invalid) & \cellcolor{red!20} 1.249 ($\pm$ 0.0219) & \cellcolor{red!20} 0.026 ($\pm$ 0.0005)  & \cellcolor{red!20} 1.777 ($\pm$ 0.0254) & \cellcolor{red!20} 0.035 ($\pm$ 0.0006)  & \cellcolor{red!20} 2.452 ($\pm$ 0.0324) & \cellcolor{red!20} 0.044 ($\pm$ 0.0007) \\

Small-Clean & \cellcolor{white} 0.819 ($\pm$ 0.0592) & \cellcolor{white} 0.018 ($\pm$ 0.002)  & \cellcolor{white} 0.613 ($\pm$ 0.0753) & \cellcolor{white} 0.011 ($\pm$ 0.0018)  & \cellcolor{white} 0.406 ($\pm$ 0.0769) & \cellcolor{white} 0.006 ($\pm$ 0.0013) \\

Label-Trim & \bfseries \cellcolor{Green!60} 1.079 ($\pm$ 0.0212) & \cellcolor{white} 0.02 ($\pm$ 0.0005)  & \bfseries \cellcolor{Green!60} 1.23 ($\pm$ 0.0247) & \cellcolor{white} 0.017 ($\pm$ 0.0004)  & \bfseries \cellcolor{Green!60} 1.381 ($\pm$ 0.0298) & \cellcolor{white} 0.014 ($\pm$ 0.0005) \\
\end{tabular}
}
\subcaption{Target type-I error rate $\alpha=0.02$}
\end{table}


\begin{table}[!htb]
\caption{Comparison of conformal outlier detection methods on Texture dataset (outliers) and CIFAR-10 dataset (inliers) for varying contamination rate $r$ and target type-I error level $\alpha$. The empirical power is presented relative to the \texttt{Standard} method (higher is better). Results are averaged across 100 random splits of the data, with standard errors presented in parentheses.}
\label{app-tab:texture}
\centering
\resizebox{\textwidth}{!}{
\begin{tabular}{l|ll|ll|ll}
\hline
& \multicolumn{6}{c}{Contamination rate} \\
\hline
 & \multicolumn{2}{c|}{1\%} & \multicolumn{2}{c|}{3\%} & \multicolumn{2}{c}{5\%} \\ \hline
 Method      & Power & Type-I Error & Power & Type-I Error & Power & Type-I Error \\ \hline
Standard & \bfseries \cellcolor{Green!30} 1.0 ($\pm$ 0.0295) & \cellcolor{white} 0.008 ($\pm$ 0.0003)  & \bfseries \cellcolor{Green!30} 1.0 ($\pm$ 0.0369) & \cellcolor{white} 0.006 ($\pm$ 0.0003)  & \bfseries \cellcolor{Green!30} 1.0 ($\pm$ 0.0405) & \cellcolor{white} 0.004 ($\pm$ 0.0002) \\

Oracle (infeasible) & \bfseries \cellcolor{Green!100} 1.173 ($\pm$ 0.0295) & \cellcolor{white} 0.01 ($\pm$ 0.0003)  & \bfseries \cellcolor{Green!100} 1.455 ($\pm$ 0.0432) & \cellcolor{white} 0.01 ($\pm$ 0.0003)  & \bfseries \cellcolor{Green!100} 1.824 ($\pm$ 0.051) & \cellcolor{white} 0.009 ($\pm$ 0.0004) \\

Naive-Trim (invalid) & \cellcolor{red!20} 1.722 ($\pm$ 0.035) & \cellcolor{red!20} 0.017 ($\pm$ 0.0004)  & \cellcolor{red!20} 2.668 ($\pm$ 0.0428) & \cellcolor{red!20} 0.028 ($\pm$ 0.0006)  & \cellcolor{red!20} 4.008 ($\pm$ 0.0599) & \cellcolor{red!20} 0.037 ($\pm$ 0.0007) \\

Small-Clean & \cellcolor{white} 0.0 ($\pm$ 0.0) & \cellcolor{white} 0.0 ($\pm$ 0.0)  & \cellcolor{white} 0.0 ($\pm$ 0.0) & \cellcolor{white} 0.0 ($\pm$ 0.0)  & \cellcolor{white} 0.0 ($\pm$ 0.0) & \cellcolor{white} 0.0 ($\pm$ 0.0) \\

Label-Trim & \bfseries \cellcolor{Green!100} 1.173 ($\pm$ 0.0295) & \cellcolor{white} 0.01 ($\pm$ 0.0003)  & \bfseries \cellcolor{Green!60} 1.448 ($\pm$ 0.0428) & \cellcolor{white} 0.01 ($\pm$ 0.0003)  & \bfseries \cellcolor{Green!60} 1.73 ($\pm$ 0.0477) & \cellcolor{white} 0.009 ($\pm$ 0.0003) \\
\hline
Standard Power & \cellcolor{white} 0.194 ($\pm$ 0.0057) &  & \cellcolor{white} 0.159 ($\pm$ 0.0059) &  & \cellcolor{white} 0.123 ($\pm$ 0.005) & \\
\end{tabular}
}
\subcaption{Target type-I error rate $\alpha=0.01$}

\resizebox{\textwidth}{!}{
\begin{tabular}{l|ll|ll|ll}
\hline
& \multicolumn{6}{c}{Contamination rate} \\
\hline
 & \multicolumn{2}{c|}{1\%} & \multicolumn{2}{c|}{3\%} & \multicolumn{2}{c}{5\%} \\ \hline
 Method      & Power & Type-I Error & Power & Type-I Error & Power & Type-I Error \\ \hline
Standard & \bfseries \cellcolor{Green!30} 1.0 ($\pm$ 0.0202) & \cellcolor{white} 0.017 ($\pm$ 0.0005)  & \bfseries \cellcolor{Green!30} 1.0 ($\pm$ 0.0239) & \cellcolor{white} 0.012 ($\pm$ 0.0004)  & \bfseries \cellcolor{Green!30} 1.0 ($\pm$ 0.0264) & \cellcolor{white} 0.009 ($\pm$ 0.0003) \\

Oracle (infeasible) & \bfseries \cellcolor{Green!100} 1.109 ($\pm$ 0.0199) & 0.021 ($\pm$ 0.0005)  & \bfseries \cellcolor{Green!100} 1.315 ($\pm$ 0.0259) & \cellcolor{white} 0.02 ($\pm$ 0.0005)  & \bfseries \cellcolor{Green!100} 1.623 ($\pm$ 0.0346) & \cellcolor{white} 0.02 ($\pm$ 0.0006) \\

Naive-Trim (invalid) & \cellcolor{red!20} 1.24 ($\pm$ 0.0193) & \cellcolor{red!20} 0.026 ($\pm$ 0.0006)  & \cellcolor{red!20} 1.781 ($\pm$ 0.026) & \cellcolor{red!20} 0.035 ($\pm$ 0.0007)  & \cellcolor{red!20} 2.431 ($\pm$ 0.0334) & \cellcolor{red!20} 0.045 ($\pm$ 0.0007) \\

Small-Clean & \cellcolor{white} 0.819 ($\pm$ 0.0599) & \cellcolor{white} 0.018 ($\pm$ 0.002)  & \cellcolor{white} 0.702 ($\pm$ 0.0824) & \cellcolor{white} 0.014 ($\pm$ 0.0025)  & \cellcolor{white} 0.478 ($\pm$ 0.0842) & \cellcolor{white} 0.007 ($\pm$ 0.0016) \\

Label-Trim & \bfseries \cellcolor{Green!60} 1.089 ($\pm$ 0.0199) & \cellcolor{white} 0.02 ($\pm$ 0.0005)  & \bfseries \cellcolor{Green!60} 1.219 ($\pm$ 0.0253) & \cellcolor{white} 0.017 ($\pm$ 0.0004)  & \bfseries \cellcolor{Green!60} 1.353 ($\pm$ 0.0303) & \cellcolor{white} 0.014 ($\pm$ 0.0005) \\
\hline
Standard Power & \cellcolor{white} 0.337 ($\pm$ 0.0068) &  & \cellcolor{white} 0.272 ($\pm$ 0.0065) &   & \cellcolor{white} 0.224 ($\pm$ 0.0059) &  \\
\end{tabular}
}
\subcaption{Target type-I error rate $\alpha=0.02$}

\resizebox{\textwidth}{!}{
\begin{tabular}{l|ll|ll|ll}
\hline
& \multicolumn{6}{c}{Contamination rate} \\
\hline
 & \multicolumn{2}{c|}{1\%} & \multicolumn{2}{c|}{3\%} & \multicolumn{2}{c}{5\%} \\ \hline
 Method      & Power & Type-I Error & Power & Type-I Error & Power & Type-I Error \\ \hline
Standard & \bfseries \cellcolor{Green!30} 1.0 ($\pm$ 0.0155) & \cellcolor{white} 0.027 ($\pm$ 0.0006)  & \bfseries \cellcolor{Green!30} 1.0 ($\pm$ 0.0197) & \cellcolor{white} 0.02 ($\pm$ 0.0005)  & \cellcolor{white} 1.0 ($\pm$ 0.0215) & \cellcolor{white} 0.015 ($\pm$ 0.0005) \\

Oracle (infeasible) & \bfseries \cellcolor{Green!100} 1.068 ($\pm$ 0.0152) & \cellcolor{white} 0.03 ($\pm$ 0.0006)  & \bfseries \cellcolor{Green!100} 1.224 ($\pm$ 0.0191) & \cellcolor{white} 0.03 ($\pm$ 0.0006)  & \bfseries \cellcolor{Green!100} 1.427 ($\pm$ 0.0249) & \cellcolor{white} 0.03 ($\pm$ 0.0007) \\

Naive-Trim (invalid) & \cellcolor{red!20} 1.15 ($\pm$ 0.0152) & \cellcolor{red!20} 0.036 ($\pm$ 0.0006)  & \cellcolor{red!20} 1.514 ($\pm$ 0.0178) & \cellcolor{red!20} 0.043 ($\pm$ 0.0007)  & \cellcolor{red!20} 1.919 ($\pm$ 0.0229) & \cellcolor{red!20} 0.052 ($\pm$ 0.0008) \\

Small-Clean & \cellcolor{white} 0.743 ($\pm$ 0.0441) & \cellcolor{white} 0.02 ($\pm$ 0.002)  & \cellcolor{white} 0.883 ($\pm$ 0.0549) & \cellcolor{white} 0.021 ($\pm$ 0.0025)  & \bfseries \cellcolor{Green!30} 1.062 ($\pm$ 0.0665) & \cellcolor{white} 0.024 ($\pm$ 0.0027) \\

Label-Trim & \bfseries \cellcolor{Green!60} 1.046 ($\pm$ 0.0153) & \cellcolor{white} 0.029 ($\pm$ 0.0006)  & \bfseries \cellcolor{Green!60} 1.13 ($\pm$ 0.019) & \cellcolor{white} 0.025 ($\pm$ 0.0005)  & \bfseries \cellcolor{Green!60} 1.215 ($\pm$ 0.0245) & \cellcolor{white} 0.021 ($\pm$ 0.0005) \\
\hline
Standard Power & \cellcolor{white} 0.421 ($\pm$ 0.0065) &   & \cellcolor{white} 0.357 ($\pm$ 0.007) &   & \cellcolor{white} 0.309 ($\pm$ 0.0067) &  \\
\end{tabular}
}
\subcaption{Target type-I error rate $\alpha=0.03$}

\end{table}


\begin{table}[!htb]
\caption{Comparison of conformal outlier detection methods on SVHN dataset (outliers) and CIFAR-10 dataset (inliers) for varying contamination rate $r$ and target type-I error level $\alpha$. The empirical power is presented relative to the \texttt{Standard} method (higher is better). Results are averaged across 100 random splits of the data, with standard errors presented in parentheses.}
\label{app-tab:svhn}
\centering
\resizebox{\textwidth}{!}{
\begin{tabular}{l|ll|ll|ll}
\hline
& \multicolumn{6}{c}{Contamination rate} \\
\hline
 & \multicolumn{2}{c|}{1\%} & \multicolumn{2}{c|}{3\%} & \multicolumn{2}{c}{5\%} \\ \hline
 Method      & Power & Type-I Error & Power & Type-I Error & Power & Type-I Error \\ \hline
Standard & \bfseries \cellcolor{Green!30} 1.0 ($\pm$ 0.0255) & \cellcolor{white} 0.008 ($\pm$ 0.0003)  & \bfseries \cellcolor{Green!30} 1.0 ($\pm$ 0.0316) & \cellcolor{white} 0.004 ($\pm$ 0.0002)  & \bfseries \cellcolor{Green!30} 1.0 ($\pm$ 0.0388) & \cellcolor{white} 0.002 ($\pm$ 0.0002) \\

Oracle (infeasible) & \bfseries \cellcolor{Green!100} 1.192 ($\pm$ 0.0262) & \cellcolor{white} 0.01 ($\pm$ 0.0003)  & \bfseries \cellcolor{Green!100} 1.654 ($\pm$ 0.0368) & \cellcolor{white} 0.01 ($\pm$ 0.0003)  & \bfseries \cellcolor{Green!100} 2.208 ($\pm$ 0.052) & \cellcolor{white} 0.009 ($\pm$ 0.0004) \\

Naive-Trim (invalid) & \cellcolor{red!20} 1.573 ($\pm$ 0.0271) & \cellcolor{red!20} 0.016 ($\pm$ 0.0004)  & \cellcolor{red!20} 2.689 ($\pm$ 0.0353) & \cellcolor{red!20} 0.025 ($\pm$ 0.0006)  & \cellcolor{red!20} 4.073 ($\pm$ 0.0533) & \cellcolor{red!20} 0.033 ($\pm$ 0.0007) \\

Small-Clean & \cellcolor{white} 0.0 ($\pm$ 0.0) & \cellcolor{white} 0.0 ($\pm$ 0.0)  & \cellcolor{white} 0.0 ($\pm$ 0.0) & \cellcolor{white} 0.0 ($\pm$ 0.0)  & \cellcolor{white} 0.0 ($\pm$ 0.0) & \cellcolor{white} 0.0 ($\pm$ 0.0) \\

Label-Trim & \bfseries \cellcolor{Green!100} 1.192 ($\pm$ 0.0262) & \cellcolor{white} 0.01 ($\pm$ 0.0003)  & \bfseries \cellcolor{Green!60} 1.601 ($\pm$ 0.0367) & \cellcolor{white} 0.009 ($\pm$ 0.0003)  & \bfseries \cellcolor{Green!60} 1.895 ($\pm$ 0.0448) & \cellcolor{white} 0.007 ($\pm$ 0.0003) \\
\hline
Standard Power & \cellcolor{white} 0.271 ($\pm$ 0.0069) &  & \cellcolor{white} 0.191 ($\pm$ 0.006) &  & \cellcolor{white} 0.137 ($\pm$ 0.0053) & \\
\end{tabular}
}
\subcaption{Target type-I error rate $\alpha=0.01$}

\resizebox{\textwidth}{!}{
\begin{tabular}{l|ll|ll|ll}
\hline
& \multicolumn{6}{c}{Contamination rate} \\
\hline
 & \multicolumn{2}{c|}{1\%} & \multicolumn{2}{c|}{3\%} & \multicolumn{2}{c}{5\%} \\ \hline
 Method      & Power & Type-I Error & Power & Type-I Error & Power & Type-I Error \\ \hline
Standard & \bfseries \cellcolor{Green!30} 1.0 ($\pm$ 0.0168) & \cellcolor{white} 0.016 ($\pm$ 0.0004)  & \bfseries \cellcolor{Green!30} 1.0 ($\pm$ 0.0218) & \cellcolor{white} 0.01 ($\pm$ 0.0003)  & \bfseries \cellcolor{Green!30} 1.0 ($\pm$ 0.0245) & \cellcolor{white} 0.007 ($\pm$ 0.0003) \\

Oracle (infeasible) & \bfseries \cellcolor{Green!100} 1.096 ($\pm$ 0.0173) & 0.021 ($\pm$ 0.0005)  & \bfseries \cellcolor{Green!100} 1.404 ($\pm$ 0.0211) & \cellcolor{white} 0.02 ($\pm$ 0.0005)  & \bfseries \cellcolor{Green!100} 1.828 ($\pm$ 0.0304) & \cellcolor{white} 0.02 ($\pm$ 0.0006) \\

Naive-Trim (invalid) & \cellcolor{red!20} 1.2 ($\pm$ 0.0175) & \cellcolor{red!20} 0.026 ($\pm$ 0.0005)  & \cellcolor{red!20} 1.721 ($\pm$ 0.0216) & \cellcolor{red!20} 0.032 ($\pm$ 0.0006)  & \cellcolor{red!20} 2.42 ($\pm$ 0.0281) & \cellcolor{red!20} 0.041 ($\pm$ 0.0007) \\

Small-Clean & \cellcolor{white} 0.833 ($\pm$ 0.053) & \cellcolor{white} 0.02 ($\pm$ 0.0026)  & \cellcolor{white} 0.598 ($\pm$ 0.0754) & \cellcolor{white} 0.01 ($\pm$ 0.0016)  & \cellcolor{white} 0.509 ($\pm$ 0.0929) & \cellcolor{white} 0.008 ($\pm$ 0.0019) \\

Label-Trim & \bfseries \cellcolor{Green!60} 1.08 ($\pm$ 0.0176) & \cellcolor{white} 0.02 ($\pm$ 0.0005)  & \bfseries \cellcolor{Green!60} 1.277 ($\pm$ 0.0228) & \cellcolor{white} 0.016 ($\pm$ 0.0004)  & \bfseries \cellcolor{Green!60} 1.469 ($\pm$ 0.0288) & \cellcolor{white} 0.013 ($\pm$ 0.0004) \\
\hline
Standard Power & \cellcolor{white} 0.429 ($\pm$ 0.0072) &  & \cellcolor{white} 0.332 ($\pm$ 0.0072) &  & \cellcolor{white} 0.25 ($\pm$ 0.0061) &  \\
\end{tabular}
}
\subcaption{Target type-I error rate $\alpha=0.02$}

\resizebox{\textwidth}{!}{
\begin{tabular}{l|ll|ll|ll}
\hline
& \multicolumn{6}{c}{Contamination rate} \\
\hline
 & \multicolumn{2}{c|}{1\%} & \multicolumn{2}{c|}{3\%} & \multicolumn{2}{c}{5\%} \\ \hline
 Method      & Power & Type-I Error & Power & Type-I Error & Power & Type-I Error \\ \hline
Standard & \bfseries \cellcolor{Green!30} 1.0 ($\pm$ 0.0145) & \cellcolor{white} 0.026 ($\pm$ 0.0005)  & \bfseries \cellcolor{Green!30} 1.0 ($\pm$ 0.0169) & \cellcolor{white} 0.017 ($\pm$ 0.0004)  & \cellcolor{white} 1.0 ($\pm$ 0.0204) & \cellcolor{white} 0.012 ($\pm$ 0.0004) \\

Oracle (infeasible) & \bfseries \cellcolor{Green!100} 1.063 ($\pm$ 0.014) & \cellcolor{white} 0.03 ($\pm$ 0.0006)  & \bfseries \cellcolor{Green!100} 1.246 ($\pm$ 0.0158) & \cellcolor{white} 0.029 ($\pm$ 0.0006)  & \bfseries \cellcolor{Green!100} 1.513 ($\pm$ 0.0212) & \cellcolor{white} 0.03 ($\pm$ 0.0007) \\

Naive-Trim (invalid) & \cellcolor{red!20} 1.115 ($\pm$ 0.0137) & \cellcolor{red!20} 0.035 ($\pm$ 0.0006)  & \cellcolor{red!20} 1.395 ($\pm$ 0.0148) & \cellcolor{red!20} 0.041 ($\pm$ 0.0007)  & \cellcolor{red!20} 1.825 ($\pm$ 0.0199) & \cellcolor{red!20} 0.049 ($\pm$ 0.0008) \\

Small-Clean & \cellcolor{white} 0.73 ($\pm$ 0.0419) & \cellcolor{white} 0.021 ($\pm$ 0.0027)  & \cellcolor{white} 0.948 ($\pm$ 0.0447) & \cellcolor{white} 0.022 ($\pm$ 0.0019)  & \bfseries \cellcolor{Green!30} 1.092 ($\pm$ 0.0625) & \cellcolor{white} 0.022 ($\pm$ 0.0023) \\

Label-Trim & \bfseries \cellcolor{Green!60} 1.042 ($\pm$ 0.0143) & \cellcolor{white} 0.029 ($\pm$ 0.0006)  & \bfseries \cellcolor{Green!60} 1.144 ($\pm$ 0.0151) & \cellcolor{white} 0.024 ($\pm$ 0.0005)  & \bfseries \cellcolor{Green!60} 1.266 ($\pm$ 0.0209) & \cellcolor{white} 0.019 ($\pm$ 0.0006) \\
\hline
Standard Power & \cellcolor{white} 0.517 ($\pm$ 0.0075) &  & \cellcolor{white} 0.44 ($\pm$ 0.0074) &  & \cellcolor{white} 0.355 ($\pm$ 0.0072) &  \\
\end{tabular}
}
\subcaption{Target type-I error rate $\alpha=0.03$}

\end{table}

\begin{table}[!htb]
\caption{Comparison of conformal outlier detection methods on Places365 dataset (outliers) and CIFAR-10 dataset (inliers) for varying contamination rate $r$ and target type-I error level $\alpha$. The empirical power is presented relative to the \texttt{Standard} method (higher is better). Results are averaged across 100 random splits of the data, with standard errors presented in parentheses.}
\label{app-tab:places365}
\centering
\resizebox{\textwidth}{!}{
\begin{tabular}{l|ll|ll|ll}
\hline
& \multicolumn{6}{c}{Contamination rate} \\
\hline
 & \multicolumn{2}{c|}{1\%} & \multicolumn{2}{c|}{3\%} & \multicolumn{2}{c}{5\%} \\ \hline
 Method      & Power & Type-I Error & Power & Type-I Error & Power & Type-I Error \\ \hline
Standard & \bfseries \cellcolor{Green!30} 1.0 ($\pm$ 0.033) & \cellcolor{white} 0.009 ($\pm$ 0.0003)  & \bfseries \cellcolor{Green!30} 1.0 ($\pm$ 0.0364) & \cellcolor{white} 0.006 ($\pm$ 0.0003)  & \bfseries \cellcolor{Green!30} 1.0 ($\pm$ 0.0431) & \cellcolor{white} 0.004 ($\pm$ 0.0002) \\

Oracle (infeasible) & \bfseries \cellcolor{Green!100} 1.139 ($\pm$ 0.0339) & \cellcolor{white} 0.01 ($\pm$ 0.0003)  & \bfseries \cellcolor{Green!100} 1.47 ($\pm$ 0.0423) & \cellcolor{white} 0.01 ($\pm$ 0.0003)  & \bfseries \cellcolor{Green!100} 1.839 ($\pm$ 0.0558) & \cellcolor{white} 0.009 ($\pm$ 0.0004) \\

Naive-Trim (invalid) & \cellcolor{red!20} 1.624 ($\pm$ 0.0339) & \cellcolor{red!20} 0.017 ($\pm$ 0.0004)  & \cellcolor{red!20} 2.73 ($\pm$ 0.0487) & \cellcolor{red!20} 0.028 ($\pm$ 0.0006)  & \cellcolor{red!20} 4.12 ($\pm$ 0.069) & \cellcolor{red!20} 0.039 ($\pm$ 0.0007) \\

Small-Clean & \cellcolor{white} 0.0 ($\pm$ 0.0) & \cellcolor{white} 0.0 ($\pm$ 0.0)  & \cellcolor{white} 0.0 ($\pm$ 0.0) & \cellcolor{white} 0.0 ($\pm$ 0.0)  & \cellcolor{white} 0.0 ($\pm$ 0.0) & \cellcolor{white} 0.0 ($\pm$ 0.0) \\

Label-Trim & \bfseries \cellcolor{Green!100} 1.139 ($\pm$ 0.0339) & \cellcolor{white} 0.01 ($\pm$ 0.0003)  & \bfseries \cellcolor{Green!100} 1.466 ($\pm$ 0.0421) & \cellcolor{white} 0.01 ($\pm$ 0.0003)  & \bfseries \cellcolor{Green!60} 1.707 ($\pm$ 0.0531) & \cellcolor{white} 0.009 ($\pm$ 0.0003) \\
\hline
Standard Power & \cellcolor{white} 0.193 ($\pm$ 0.0064) &  & \cellcolor{white} 0.148 ($\pm$ 0.0054) &  & \cellcolor{white} 0.115 ($\pm$ 0.005) &  \\
\end{tabular}
}
\subcaption{Target type-I error rate $\alpha=0.01$}

\resizebox{\textwidth}{!}{
\begin{tabular}{l|ll|ll|ll}
\hline
& \multicolumn{6}{c}{Contamination rate} \\
\hline
 & \multicolumn{2}{c|}{1\%} & \multicolumn{2}{c|}{3\%} & \multicolumn{2}{c}{5\%} \\ \hline
 Method      & Power & Type-I Error & Power & Type-I Error & Power & Type-I Error \\ \hline
Standard & \bfseries \cellcolor{Green!30} 1.0 ($\pm$ 0.0209) & \cellcolor{white} 0.017 ($\pm$ 0.0005)  & \bfseries \cellcolor{Green!30} 1.0 ($\pm$ 0.0241) & \cellcolor{white} 0.013 ($\pm$ 0.0004)  & \bfseries \cellcolor{Green!30} 1.0 ($\pm$ 0.0298) & \cellcolor{white} 0.009 ($\pm$ 0.0003) \\

Oracle (infeasible) & \bfseries \cellcolor{Green!100} 1.091 ($\pm$ 0.0224) & 0.021 ($\pm$ 0.0005)  & \bfseries \cellcolor{Green!100} 1.271 ($\pm$ 0.0258) & \cellcolor{white} 0.02 ($\pm$ 0.0005)  & \bfseries \cellcolor{Green!100} 1.599 ($\pm$ 0.0308) & \cellcolor{white} 0.02 ($\pm$ 0.0006) \\

Naive-Trim (invalid) & \cellcolor{red!20} 1.246 ($\pm$ 0.0225) & \cellcolor{red!20} 0.027 ($\pm$ 0.0006)  & \cellcolor{red!20} 1.753 ($\pm$ 0.0277) & \cellcolor{red!20} 0.036 ($\pm$ 0.0007)  & \cellcolor{red!20} 2.462 ($\pm$ 0.0376) & \cellcolor{red!20} 0.046 ($\pm$ 0.0007) \\

Small-Clean & \cellcolor{white} 0.799 ($\pm$ 0.0632) & \cellcolor{white} 0.018 ($\pm$ 0.0021)  & \cellcolor{white} 0.549 ($\pm$ 0.0716) & \cellcolor{white} 0.01 ($\pm$ 0.0016)  & \cellcolor{white} 0.255 ($\pm$ 0.0595) & \cellcolor{white} 0.003 ($\pm$ 0.0009) \\

Label-Trim & \bfseries \cellcolor{Green!60} 1.071 ($\pm$ 0.0219) & \cellcolor{white} 0.02 ($\pm$ 0.0005)  & \bfseries \cellcolor{Green!60} 1.187 ($\pm$ 0.0247) & \cellcolor{white} 0.017 ($\pm$ 0.0004)  & \bfseries \cellcolor{Green!60} 1.361 ($\pm$ 0.0306) & \cellcolor{white} 0.015 ($\pm$ 0.0005) \\
\hline
Standard Power & \cellcolor{white} 0.316 ($\pm$ 0.0066) &  & \cellcolor{white} 0.261 ($\pm$ 0.0063) &  & \cellcolor{white} 0.213 ($\pm$ 0.0064) &  \\
\end{tabular}
}
\subcaption{Target type-I error rate $\alpha=0.02$}

\resizebox{\textwidth}{!}{
\begin{tabular}{l|ll|ll|ll}
\hline
& \multicolumn{6}{c}{Contamination rate} \\
\hline
 & \multicolumn{2}{c|}{1\%} & \multicolumn{2}{c|}{3\%} & \multicolumn{2}{c}{5\%} \\ \hline
 Method      & Power & Type-I Error & Power & Type-I Error & Power & Type-I Error \\ \hline
Standard & \bfseries \cellcolor{Green!30} 1.0 ($\pm$ 0.018) & \cellcolor{white} 0.027 ($\pm$ 0.0006)  & \bfseries \cellcolor{Green!30} 1.0 ($\pm$ 0.0204) & \cellcolor{white} 0.02 ($\pm$ 0.0005)  & \cellcolor{white} 1.0 ($\pm$ 0.021) & \cellcolor{white} 0.015 ($\pm$ 0.0005) \\

Oracle (infeasible) & \bfseries \cellcolor{Green!100} 1.055 ($\pm$ 0.0189) & \cellcolor{white} 0.03 ($\pm$ 0.0006)  & \bfseries \cellcolor{Green!100} 1.231 ($\pm$ 0.0213) & \cellcolor{white} 0.029 ($\pm$ 0.0006)  & \bfseries \cellcolor{Green!100} 1.403 ($\pm$ 0.025) & \cellcolor{white} 0.03 ($\pm$ 0.0007) \\

Naive-Trim (invalid) & \cellcolor{red!20} 1.145 ($\pm$ 0.0199) & \cellcolor{red!20} 0.036 ($\pm$ 0.0006)  & \cellcolor{red!20} 1.496 ($\pm$ 0.0217) & \cellcolor{red!20} 0.044 ($\pm$ 0.0007)  & \cellcolor{red!20} 1.883 ($\pm$ 0.0261) & \cellcolor{red!20} 0.054 ($\pm$ 0.0008) \\

Small-Clean & \cellcolor{white} 0.746 ($\pm$ 0.0469) & \cellcolor{white} 0.021 ($\pm$ 0.002)  & \cellcolor{white} 0.833 ($\pm$ 0.0504) & \cellcolor{white} 0.018 ($\pm$ 0.0018)  & \bfseries \cellcolor{Green!30} 1.029 ($\pm$ 0.0593) & \cellcolor{white} 0.022 ($\pm$ 0.0022) \\

Label-Trim & \bfseries \cellcolor{Green!60} 1.038 ($\pm$ 0.0188) & \cellcolor{white} 0.029 ($\pm$ 0.0006)  & \bfseries \cellcolor{Green!60} 1.149 ($\pm$ 0.0211) & \cellcolor{white} 0.025 ($\pm$ 0.0006)  & \bfseries \cellcolor{Green!60} 1.186 ($\pm$ 0.0228) & \cellcolor{white} 0.022 ($\pm$ 0.0006) \\
\hline
Standard Power & \cellcolor{white} 0.395 ($\pm$ 0.0071) &  & \cellcolor{white} 0.335 ($\pm$ 0.0068) &  & \cellcolor{white} 0.302 ($\pm$ 0.0063) &  \\
\end{tabular}
}
\subcaption{Target type-I error rate $\alpha=0.03$}

\end{table}
\begin{table}[!htb]
\caption{Comparison of conformal outlier detection methods on MNIST dataset (outliers) and CIFAR-10 dataset (inliers) for varying contamination rate $r$ and target type-I error level $\alpha$. The empirical power is presented relative to the \texttt{Standard} method (higher is better). Results are averaged across 100 random splits of the data, with standard errors presented in parentheses.}
\label{app-tab:mnist}
\centering
\resizebox{\textwidth}{!}{
\begin{tabular}{l|ll|ll|ll}
\hline
& \multicolumn{6}{c}{Contamination rate} \\
\hline
 & \multicolumn{2}{c|}{1\%} & \multicolumn{2}{c|}{3\%} & \multicolumn{2}{c}{5\%} \\ \hline
 Method      & Power & Type-I Error & Power & Type-I Error & Power & Type-I Error \\ \hline
Standard & \bfseries \cellcolor{Green!30} 1.0 ($\pm$ 0.0248) & \cellcolor{white} 0.008 ($\pm$ 0.0003)  & \bfseries \cellcolor{Green!30} 1.0 ($\pm$ 0.0283) & \cellcolor{white} 0.004 ($\pm$ 0.0002)  & \bfseries \cellcolor{Green!30} 1.0 ($\pm$ 0.0363) & \cellcolor{white} 0.003 ($\pm$ 0.0002) \\

Oracle (infeasible) & \bfseries \cellcolor{Green!100} 1.241 ($\pm$ 0.0291) & \cellcolor{white} 0.01 ($\pm$ 0.0003)  & \bfseries \cellcolor{Green!100} 1.758 ($\pm$ 0.0352) & \cellcolor{white} 0.01 ($\pm$ 0.0003)  & \bfseries \cellcolor{Green!100} 2.554 ($\pm$ 0.0601) & \cellcolor{white} 0.009 ($\pm$ 0.0004) \\

Naive-Trim (invalid) & \cellcolor{red!20} 1.629 ($\pm$ 0.0292) & \cellcolor{red!20} 0.016 ($\pm$ 0.0004)  & \cellcolor{red!20} 2.722 ($\pm$ 0.0363) & \cellcolor{red!20} 0.023 ($\pm$ 0.0005)  & \cellcolor{red!20} 4.718 ($\pm$ 0.0552) & \cellcolor{red!20} 0.03 ($\pm$ 0.0006) \\

Small-Clean & \cellcolor{white} 0.0 ($\pm$ 0.0) & \cellcolor{white} 0.0 ($\pm$ 0.0)  & \cellcolor{white} 0.0 ($\pm$ 0.0) & \cellcolor{white} 0.0 ($\pm$ 0.0)  & \cellcolor{white} 0.0 ($\pm$ 0.0) & \cellcolor{white} 0.0 ($\pm$ 0.0) \\

Label-Trim & \bfseries \cellcolor{Green!100} 1.241 ($\pm$ 0.0291) & \cellcolor{white} 0.01 ($\pm$ 0.0003)  & \bfseries \cellcolor{Green!60} 1.636 ($\pm$ 0.0331) & \cellcolor{white} 0.009 ($\pm$ 0.0003)  & \bfseries \cellcolor{Green!60} 2.113 ($\pm$ 0.0555) & \cellcolor{white} 0.007 ($\pm$ 0.0003) \\
\hline
Standard Power & \cellcolor{white} 0.282 ($\pm$ 0.007) &  & \cellcolor{white} 0.208 ($\pm$ 0.0059) &  & \cellcolor{white} 0.131 ($\pm$ 0.0048) & \\
\end{tabular}
}
\subcaption{Target type-I error rate $\alpha=0.01$}

\resizebox{\textwidth}{!}{
\begin{tabular}{l|ll|ll|ll}
\hline
& \multicolumn{6}{c}{Contamination rate} \\
\hline
 & \multicolumn{2}{c|}{1\%} & \multicolumn{2}{c|}{3\%} & \multicolumn{2}{c}{5\%} \\ \hline
 Method      & Power & Type-I Error & Power & Type-I Error & Power & Type-I Error \\ \hline
Standard & \bfseries \cellcolor{Green!30} 1.0 ($\pm$ 0.0173) & \cellcolor{white} 0.016 ($\pm$ 0.0004)  & \bfseries \cellcolor{Green!30} 1.0 ($\pm$ 0.0205) & \cellcolor{white} 0.01 ($\pm$ 0.0003)  & \bfseries \cellcolor{Green!30} 1.0 ($\pm$ 0.027) & \cellcolor{white} 0.007 ($\pm$ 0.0003) \\

Oracle (infeasible) & \bfseries \cellcolor{Green!100} 1.117 ($\pm$ 0.0179) & 0.021 ($\pm$ 0.0005)  & \bfseries \cellcolor{Green!100} 1.452 ($\pm$ 0.0209) & \cellcolor{white} 0.02 ($\pm$ 0.0005)  & \bfseries \cellcolor{Green!100} 1.971 ($\pm$ 0.0297) & \cellcolor{white} 0.02 ($\pm$ 0.0006) \\

Naive-Trim (invalid) & \cellcolor{red!20} 1.231 ($\pm$ 0.0179) & \cellcolor{red!20} 0.025 ($\pm$ 0.0005)  & \cellcolor{red!20} 1.759 ($\pm$ 0.0207) & \cellcolor{red!20} 0.031 ($\pm$ 0.0006)  & \cellcolor{red!20} 2.531 ($\pm$ 0.0273) & \cellcolor{red!20} 0.038 ($\pm$ 0.0007) \\

Small-Clean & \cellcolor{white} 0.844 ($\pm$ 0.0514) & \cellcolor{white} 0.019 ($\pm$ 0.0019)  & \cellcolor{white} 0.617 ($\pm$ 0.0738) & \cellcolor{white} 0.01 ($\pm$ 0.0017)  & \cellcolor{white} 0.438 ($\pm$ 0.0817) & \cellcolor{white} 0.005 ($\pm$ 0.0011) \\

Label-Trim & \bfseries \cellcolor{Green!60} 1.096 ($\pm$ 0.0181) & \cellcolor{white} 0.02 ($\pm$ 0.0005)  & \bfseries \cellcolor{Green!60} 1.308 ($\pm$ 0.0206) & \cellcolor{white} 0.015 ($\pm$ 0.0004)  & \bfseries \cellcolor{Green!60} 1.493 ($\pm$ 0.0295) & \cellcolor{white} 0.012 ($\pm$ 0.0004) \\
\hline
Standard Power & \cellcolor{white} 0.465 ($\pm$ 0.008) &  & \cellcolor{white} 0.36 ($\pm$ 0.0074) & & \cellcolor{white} 0.264 ($\pm$ 0.0071) &  \\
\end{tabular}
}
\subcaption{Target type-I error rate $\alpha=0.02$}

\resizebox{\textwidth}{!}{
\begin{tabular}{l|ll|ll|ll}
\hline
& \multicolumn{6}{c}{Contamination rate} \\
\hline
 & \multicolumn{2}{c|}{1\%} & \multicolumn{2}{c|}{3\%} & \multicolumn{2}{c}{5\%} \\ \hline
 Method      & Power & Type-I Error & Power & Type-I Error & Power & Type-I Error \\ \hline
Standard & \bfseries \cellcolor{Green!30} 1.0 ($\pm$ 0.0147) & \cellcolor{white} 0.025 ($\pm$ 0.0005)  & \bfseries \cellcolor{Green!30} 1.0 ($\pm$ 0.0151) & \cellcolor{white} 0.016 ($\pm$ 0.0004)  & \cellcolor{white} 1.0 ($\pm$ 0.0201) & \cellcolor{white} 0.011 ($\pm$ 0.0004) \\

Oracle (infeasible) & \bfseries \cellcolor{Green!100} 1.072 ($\pm$ 0.0149) & \cellcolor{white} 0.03 ($\pm$ 0.0006)  & \bfseries \cellcolor{Green!100} 1.295 ($\pm$ 0.0155) & \cellcolor{white} 0.029 ($\pm$ 0.0006)  & \bfseries \cellcolor{Green!100} 1.617 ($\pm$ 0.0186) & \cellcolor{white} 0.03 ($\pm$ 0.0007) \\

Naive-Trim (invalid) & \cellcolor{red!20} 1.116 ($\pm$ 0.0144) & \cellcolor{red!20} 0.034 ($\pm$ 0.0006)  & \cellcolor{red!20} 1.424 ($\pm$ 0.0142) & \cellcolor{red!20} 0.039 ($\pm$ 0.0007)  & \cellcolor{red!20} 1.851 ($\pm$ 0.0178) & \cellcolor{red!20} 0.046 ($\pm$ 0.0007) \\

Small-Clean & \cellcolor{white} 0.699 ($\pm$ 0.0394) & \cellcolor{white} 0.019 ($\pm$ 0.0019)  & \cellcolor{white} 0.884 ($\pm$ 0.0464) & \cellcolor{white} 0.019 ($\pm$ 0.0018)  & \bfseries \cellcolor{Green!30} 1.164 ($\pm$ 0.0576) & \cellcolor{white} 0.02 ($\pm$ 0.002) \\

Label-Trim & \bfseries \cellcolor{Green!60} 1.049 ($\pm$ 0.0148) & \cellcolor{white} 0.029 ($\pm$ 0.0006)  & \bfseries \cellcolor{Green!60} 1.158 ($\pm$ 0.016) & \cellcolor{white} 0.023 ($\pm$ 0.0005)  & \bfseries \cellcolor{Green!60} 1.273 ($\pm$ 0.0218) & \cellcolor{white} 0.017 ($\pm$ 0.0005) \\
\hline
Standard Power & \cellcolor{white} 0.576 ($\pm$ 0.0085) &  & \cellcolor{white} 0.483 ($\pm$ 0.0073) &  & \cellcolor{white} 0.382 ($\pm$ 0.0077) &  \\
\end{tabular}
}
\subcaption{Target type-I error rate $\alpha=0.03$}

\end{table}

\begin{table}[!htb]
\caption{Comparison of conformal outlier detection methods on CIFAR-100 dataset (outliers) and CIFAR-10 dataset (inliers) for varying contamination rate $r$ and target type-I error level $\alpha$. The empirical power is presented relative to the \texttt{Standard} method (higher is better). Results are averaged across 100 random splits of the data, with standard errors presented in parentheses.}
\label{app-tab:cifar100}
\centering
\resizebox{\textwidth}{!}{
\begin{tabular}{l|ll|ll|ll}
\hline
& \multicolumn{6}{c}{Contamination rate} \\
\hline
 & \multicolumn{2}{c|}{1\%} & \multicolumn{2}{c|}{3\%} & \multicolumn{2}{c}{5\%} \\ \hline
 Method      & Power & Type-I Error & Power & Type-I Error & Power & Type-I Error \\ \hline
Standard & \bfseries \cellcolor{Green!30} 1.0 ($\pm$ 0.0393) & \cellcolor{white} 0.009 ($\pm$ 0.0003)  & \bfseries \cellcolor{Green!30} 1.0 ($\pm$ 0.0391) & \cellcolor{white} 0.007 ($\pm$ 0.0003)  & \bfseries \cellcolor{Green!30} 1.0 ($\pm$ 0.042) & \cellcolor{white} 0.005 ($\pm$ 0.0003) \\

Oracle (infeasible) & \bfseries \cellcolor{Green!100} 1.116 ($\pm$ 0.0417) & \cellcolor{white} 0.01 ($\pm$ 0.0003)  & \bfseries \cellcolor{Green!100} 1.463 ($\pm$ 0.05) & \cellcolor{white} 0.01 ($\pm$ 0.0003)  & \bfseries \cellcolor{Green!100} 1.588 ($\pm$ 0.0477) & \cellcolor{white} 0.009 ($\pm$ 0.0004) \\

Naive-Trim (invalid) & \cellcolor{red!20} 1.733 ($\pm$ 0.0397) & \cellcolor{red!20} 0.018 ($\pm$ 0.0005)  & \cellcolor{red!20} 3.06 ($\pm$ 0.0585) & \cellcolor{red!20} 0.03 ($\pm$ 0.0006)  & \cellcolor{red!20} 4.054 ($\pm$ 0.0631) & \cellcolor{red!20} 0.041 ($\pm$ 0.0007) \\

Small-Clean & \cellcolor{white} 0.0 ($\pm$ 0.0) & \cellcolor{white} 0.0 ($\pm$ 0.0)  & \cellcolor{white} 0.0 ($\pm$ 0.0) & \cellcolor{white} 0.0 ($\pm$ 0.0)  & \cellcolor{white} 0.0 ($\pm$ 0.0) & \cellcolor{white} 0.0 ($\pm$ 0.0) \\

Label-Trim & \bfseries \cellcolor{Green!100} 1.116 ($\pm$ 0.0417) & \cellcolor{white} 0.01 ($\pm$ 0.0003)  & \bfseries \cellcolor{Green!100} 1.463 ($\pm$ 0.05) & \cellcolor{white} 0.01 ($\pm$ 0.0003)  & \bfseries \cellcolor{Green!100} 1.587 ($\pm$ 0.0468) & \cellcolor{white} 0.009 ($\pm$ 0.0004) \\
\hline
Standard Power & \cellcolor{white} 0.145 ($\pm$ 0.0057) &  & \cellcolor{white} 0.118 ($\pm$ 0.0046) &  & \cellcolor{white} 0.104 ($\pm$ 0.0044) & \\
\end{tabular}
}
\subcaption{Target type-I error rate $\alpha=0.01$}

\resizebox{\textwidth}{!}{
\begin{tabular}{l|ll|ll|ll}
\hline
& \multicolumn{6}{c}{Contamination rate} \\
\hline
 & \multicolumn{2}{c|}{1\%} & \multicolumn{2}{c|}{3\%} & \multicolumn{2}{c}{5\%} \\ \hline
 Method      & Power & Type-I Error & Power & Type-I Error & Power & Type-I Error \\ \hline
Standard & \bfseries \cellcolor{Green!30} 1.0 ($\pm$ 0.023) & \cellcolor{white} 0.018 ($\pm$ 0.0005)  & \bfseries \cellcolor{Green!30} 1.0 ($\pm$ 0.0265) & \cellcolor{white} 0.014 ($\pm$ 0.0004)  & \bfseries \cellcolor{Green!30} 1.0 ($\pm$ 0.0264) & \cellcolor{white} 0.011 ($\pm$ 0.0004) \\

Oracle (infeasible) & \bfseries \cellcolor{Green!100} 1.082 ($\pm$ 0.0256) & 0.021 ($\pm$ 0.0005)  & \bfseries \cellcolor{Green!100} 1.265 ($\pm$ 0.0299) & \cellcolor{white} 0.02 ($\pm$ 0.0005)  & \bfseries \cellcolor{Green!100} 1.476 ($\pm$ 0.0317) & \cellcolor{white} 0.02 ($\pm$ 0.0006) \\

Naive-Trim (invalid) & \cellcolor{red!20} 1.294 ($\pm$ 0.0266) & \cellcolor{red!20} 0.027 ($\pm$ 0.0006)  & \cellcolor{red!20} 1.831 ($\pm$ 0.0288) & \cellcolor{red!20} 0.038 ($\pm$ 0.0006)  & \cellcolor{red!20} 2.464 ($\pm$ 0.0372) & \cellcolor{red!20} 0.049 ($\pm$ 0.0008) \\

Small-Clean & \cellcolor{white} 0.844 ($\pm$ 0.0709) & \cellcolor{white} 0.019 ($\pm$ 0.0022)  & \cellcolor{white} 0.686 ($\pm$ 0.0797) & \cellcolor{white} 0.012 ($\pm$ 0.0017)  & \cellcolor{white} 0.392 ($\pm$ 0.078) & \cellcolor{white} 0.006 ($\pm$ 0.0014) \\

Label-Trim & \bfseries \cellcolor{Green!60} 1.064 ($\pm$ 0.0247) & \cellcolor{white} 0.02 ($\pm$ 0.0005)  & \bfseries \cellcolor{Green!60} 1.193 ($\pm$ 0.0293) & \cellcolor{white} 0.018 ($\pm$ 0.0004)  & \bfseries \cellcolor{Green!60} 1.288 ($\pm$ 0.0296) & \cellcolor{white} 0.016 ($\pm$ 0.0005) \\
\hline
Standard Power & \cellcolor{white} 0.255 ($\pm$ 0.0059) &   & \cellcolor{white} 0.225 ($\pm$ 0.006) &  & \cellcolor{white} 0.19 ($\pm$ 0.005) & \\
\end{tabular}
}
\subcaption{Target type-I error rate $\alpha=0.02$}

\resizebox{\textwidth}{!}{
\begin{tabular}{l|ll|ll|ll}
\hline
& \multicolumn{6}{c}{Contamination rate} \\
\hline
 & \multicolumn{2}{c|}{1\%} & \multicolumn{2}{c|}{3\%} & \multicolumn{2}{c}{5\%} \\ \hline
 Method      & Power & Type-I Error & Power & Type-I Error & Power & Type-I Error \\ \hline
Standard & \bfseries \cellcolor{Green!30} 1.0 ($\pm$ 0.0204) & \cellcolor{white} 0.027 ($\pm$ 0.0006)  & \bfseries \cellcolor{Green!30} 1.0 ($\pm$ 0.0214) & \cellcolor{white} 0.022 ($\pm$ 0.0005)  & \bfseries \cellcolor{Green!30} 1.0 ($\pm$ 0.0228) & \cellcolor{white} 0.018 ($\pm$ 0.0005) \\

Oracle (infeasible) & \bfseries \cellcolor{Green!100} 1.054 ($\pm$ 0.0212) & 0.031 ($\pm$ 0.0006)  & \bfseries \cellcolor{Green!100} 1.184 ($\pm$ 0.0228) & \cellcolor{white} 0.03 ($\pm$ 0.0006)  & \bfseries \cellcolor{Green!100} 1.33 ($\pm$ 0.0246) & \cellcolor{white} 0.03 ($\pm$ 0.0007) \\

Naive-Trim (invalid) & \cellcolor{red!20} 1.184 ($\pm$ 0.0213) & \cellcolor{red!20} 0.036 ($\pm$ 0.0006)  & \cellcolor{red!20} 1.523 ($\pm$ 0.0222) & \cellcolor{red!20} 0.046 ($\pm$ 0.0007)  & \cellcolor{red!20} 1.94 ($\pm$ 0.026) & \cellcolor{red!20} 0.056 ($\pm$ 0.0009) \\

Small-Clean & \cellcolor{white} 0.709 ($\pm$ 0.0527) & \cellcolor{white} 0.021 ($\pm$ 0.0023)  & \cellcolor{white} 0.842 ($\pm$ 0.0545) & \cellcolor{white} 0.021 ($\pm$ 0.0019)  & \cellcolor{white} 0.947 ($\pm$ 0.0628) & \cellcolor{white} 0.02 ($\pm$ 0.002) \\

Label-Trim & \bfseries \cellcolor{Green!60} 1.029 ($\pm$ 0.0212) & \cellcolor{white} 0.029 ($\pm$ 0.0006)  & \bfseries \cellcolor{Green!60} 1.114 ($\pm$ 0.0222) & \cellcolor{white} 0.026 ($\pm$ 0.0006)  & \bfseries \cellcolor{Green!60} 1.16 ($\pm$ 0.0229) & \cellcolor{white} 0.023 ($\pm$ 0.0006) \\
\hline
Standard Power & \cellcolor{white} 0.332 ($\pm$ 0.0068) &  & \cellcolor{white} 0.302 ($\pm$ 0.0064) &  & \cellcolor{white} 0.263 ($\pm$ 0.006) &  \\
\end{tabular}
}
\subcaption{Target type-I error rate $\alpha=0.03$}

\end{table}


\begin{table}[!htb]
\caption{Comparison of conformal outlier detection methods on TinyImageNet dataset (outliers) and CIFAR-10 dataset (inliers) for varying contamination rate $r$ and target type-I error level $\alpha$. The empirical power is presented relative to the \texttt{Standard} method (higher is better). Results are averaged across 100 random splits of the data, with standard errors presented in parentheses.}
\label{app-tab:tin}
\centering
\resizebox{\textwidth}{!}{
\begin{tabular}{l|ll|ll|ll}
\hline
& \multicolumn{6}{c}{Contamination rate} \\
\hline
 & \multicolumn{2}{c|}{1\%} & \multicolumn{2}{c|}{3\%} & \multicolumn{2}{c}{5\%} \\ \hline
 Method      & Power & Type-I Error & Power & Type-I Error & Power & Type-I Error \\ \hline
Standard & \bfseries \cellcolor{Green!30} 1.0 ($\pm$ 0.0381) & \cellcolor{white} 0.009 ($\pm$ 0.0003)  & \bfseries \cellcolor{Green!30} 1.0 ($\pm$ 0.0398) & \cellcolor{white} 0.006 ($\pm$ 0.0003)  & \bfseries \cellcolor{Green!30} 1.0 ($\pm$ 0.0439) & \cellcolor{white} 0.004 ($\pm$ 0.0002) \\

Oracle (infeasible) & \bfseries \cellcolor{Green!100} 1.137 ($\pm$ 0.0409) & \cellcolor{white} 0.01 ($\pm$ 0.0003)  & \bfseries \cellcolor{Green!100} 1.492 ($\pm$ 0.0474) & \cellcolor{white} 0.01 ($\pm$ 0.0003)  & \bfseries \cellcolor{Green!100} 1.754 ($\pm$ 0.0518) & \cellcolor{white} 0.009 ($\pm$ 0.0004) \\

Naive-Trim (invalid) & \cellcolor{red!20} 1.675 ($\pm$ 0.0401) & \cellcolor{red!20} 0.018 ($\pm$ 0.0004)  & \cellcolor{red!20} 2.872 ($\pm$ 0.0485) & \cellcolor{red!20} 0.028 ($\pm$ 0.0006)  & \cellcolor{red!20} 3.986 ($\pm$ 0.0568) & \cellcolor{red!20} 0.039 ($\pm$ 0.0007) \\

Small-Clean & \cellcolor{white} 0.0 ($\pm$ 0.0) & \cellcolor{white} 0.0 ($\pm$ 0.0)  & \cellcolor{white} 0.0 ($\pm$ 0.0) & \cellcolor{white} 0.0 ($\pm$ 0.0)  & \cellcolor{white} 0.0 ($\pm$ 0.0) & \cellcolor{white} 0.0 ($\pm$ 0.0) \\

Label-Trim & \bfseries \cellcolor{Green!100} 1.137 ($\pm$ 0.0409) & \cellcolor{white} 0.01 ($\pm$ 0.0003)  & \bfseries \cellcolor{Green!100} 1.488 ($\pm$ 0.0472) & \cellcolor{white} 0.01 ($\pm$ 0.0003)  & \bfseries \cellcolor{Green!60} 1.681 ($\pm$ 0.0511) & \cellcolor{white} 0.009 ($\pm$ 0.0003) \\
\hline
Standard Power & \cellcolor{white} 0.168 ($\pm$ 0.0064) &   & \cellcolor{white} 0.133 ($\pm$ 0.0053) &  & \cellcolor{white} 0.115 ($\pm$ 0.005) &  \\
\end{tabular}
}
\subcaption{Target type-I error rate $\alpha=0.01$}

\resizebox{\textwidth}{!}{
\begin{tabular}{l|ll|ll|ll}
\hline
& \multicolumn{6}{c}{Contamination rate} \\
\hline
 & \multicolumn{2}{c|}{1\%} & \multicolumn{2}{c|}{3\%} & \multicolumn{2}{c}{5\%} \\ \hline
 Method      & Power & Type-I Error & Power & Type-I Error & Power & Type-I Error \\ \hline
Standard & \bfseries \cellcolor{Green!30} 1.0 ($\pm$ 0.0239) & \cellcolor{white} 0.018 ($\pm$ 0.0004)  & \bfseries \cellcolor{Green!30} 1.0 ($\pm$ 0.0262) & \cellcolor{white} 0.013 ($\pm$ 0.0004)  & \bfseries \cellcolor{Green!30} 1.0 ($\pm$ 0.0293) & \cellcolor{white} 0.01 ($\pm$ 0.0004) \\

Oracle (infeasible) & \bfseries \cellcolor{Green!100} 1.084 ($\pm$ 0.0253) & 0.021 ($\pm$ 0.0005)  & \bfseries \cellcolor{Green!100} 1.275 ($\pm$ 0.0263) & \cellcolor{white} 0.02 ($\pm$ 0.0005)  & \bfseries \cellcolor{Green!100} 1.544 ($\pm$ 0.0293) & \cellcolor{white} 0.02 ($\pm$ 0.0006) \\

Naive-Trim (invalid) & \cellcolor{red!20} 1.284 ($\pm$ 0.0274) & \cellcolor{red!20} 0.027 ($\pm$ 0.0006)  & \cellcolor{red!20} 1.819 ($\pm$ 0.0277) & \cellcolor{red!20} 0.036 ($\pm$ 0.0006)  & \cellcolor{red!20} 2.406 ($\pm$ 0.0308) & \cellcolor{red!20} 0.046 ($\pm$ 0.0008) \\

Small-Clean & \cellcolor{white} 0.777 ($\pm$ 0.0566) & \cellcolor{white} 0.015 ($\pm$ 0.0015)  & \cellcolor{white} 0.525 ($\pm$ 0.0687) & \cellcolor{white} 0.009 ($\pm$ 0.0015)  & \cellcolor{white} 0.366 ($\pm$ 0.0653) & \cellcolor{white} 0.005 ($\pm$ 0.0011) \\

Label-Trim & \bfseries \cellcolor{Green!60} 1.074 ($\pm$ 0.0249) & \cellcolor{white} 0.02 ($\pm$ 0.0005)  & \bfseries \cellcolor{Green!60} 1.193 ($\pm$ 0.0254) & \cellcolor{white} 0.017 ($\pm$ 0.0004)  & \bfseries \cellcolor{Green!60} 1.322 ($\pm$ 0.03) & \cellcolor{white} 0.015 ($\pm$ 0.0005) \\
\hline
Standard Power & \cellcolor{white} 0.285 ($\pm$ 0.0068) &  & \cellcolor{white} 0.243 ($\pm$ 0.0064) & & \cellcolor{white} 0.209 ($\pm$ 0.0061) &  \\
\end{tabular}
}
\subcaption{Target type-I error rate $\alpha=0.02$}

\resizebox{\textwidth}{!}{
\begin{tabular}{l|ll|ll|ll}
\hline
& \multicolumn{6}{c}{Contamination rate} \\
\hline
 & \multicolumn{2}{c|}{1\%} & \multicolumn{2}{c|}{3\%} & \multicolumn{2}{c}{5\%} \\ \hline
 Method      & Power & Type-I Error & Power & Type-I Error & Power & Type-I Error \\ \hline
Standard & \bfseries \cellcolor{Green!30} 1.0 ($\pm$ 0.0213) & \cellcolor{white} 0.027 ($\pm$ 0.0006)  & \bfseries \cellcolor{Green!30} 1.0 ($\pm$ 0.0202) & \cellcolor{white} 0.02 ($\pm$ 0.0005)  & \bfseries \cellcolor{Green!30} 1.0 ($\pm$ 0.0216) & \cellcolor{white} 0.016 ($\pm$ 0.0005) \\

Oracle (infeasible) & \bfseries \cellcolor{Green!100} 1.062 ($\pm$ 0.0216) & \cellcolor{white} 0.03 ($\pm$ 0.0006)  & \bfseries \cellcolor{Green!100} 1.23 ($\pm$ 0.0205) & \cellcolor{white} 0.03 ($\pm$ 0.0006)  & \bfseries \cellcolor{Green!100} 1.397 ($\pm$ 0.0236) & \cellcolor{white} 0.03 ($\pm$ 0.0007) \\

Naive-Trim (invalid) & \cellcolor{red!20} 1.168 ($\pm$ 0.0203) & \cellcolor{red!20} 0.036 ($\pm$ 0.0006)  & \cellcolor{red!20} 1.569 ($\pm$ 0.0207) & \cellcolor{red!20} 0.045 ($\pm$ 0.0007)  & \cellcolor{red!20} 1.877 ($\pm$ 0.0217) & \cellcolor{red!20} 0.054 ($\pm$ 0.0009) \\

Small-Clean & \cellcolor{white} 0.656 ($\pm$ 0.0438) & \cellcolor{white} 0.017 ($\pm$ 0.0017)  & \cellcolor{white} 0.825 ($\pm$ 0.0496) & \cellcolor{white} 0.019 ($\pm$ 0.0019)  & \cellcolor{white} 0.904 ($\pm$ 0.0589) & \cellcolor{white} 0.02 ($\pm$ 0.0024) \\

Label-Trim & \bfseries \cellcolor{Green!60} 1.04 ($\pm$ 0.0219) & \cellcolor{white} 0.029 ($\pm$ 0.0006)  & \bfseries \cellcolor{Green!60} 1.139 ($\pm$ 0.0208) & \cellcolor{white} 0.026 ($\pm$ 0.0006)  & \bfseries \cellcolor{Green!60} 1.19 ($\pm$ 0.0225) & \cellcolor{white} 0.022 ($\pm$ 0.0006) \\
\hline
Standard Power & \cellcolor{white} 0.368 ($\pm$ 0.0078) &  & \cellcolor{white} 0.318 ($\pm$ 0.0064) &  & \cellcolor{white} 0.29 ($\pm$ 0.0063) &  \\
\end{tabular}
}
\subcaption{Target type-I error rate $\alpha=0.03$}

\end{table}


\FloatBarrier

\end{document}
