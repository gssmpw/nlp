%%%%%%%%%%%%%%%%%%%%%%%%%%%%%%%%
%%%%%%%%%%% ARXIV %%%%%%%%%%%%%%
%%%%%%%%%%%%%%%%%%%%%%%%%%%%%%%%
\documentclass{article} % For LaTeX2e
\usepackage[table]{xcolor}
\usepackage{times}
\usepackage{eso-pic} % used by \AddToShipoutPicture
\usepackage[margin=1in]{geometry}
\usepackage{authblk}
\renewcommand\Authfont{\large}
\renewcommand\Affilfont{\small\color{black}}
\usepackage[square,numbers]{natbib}
\bibliographystyle{abbrvnat}
%%%%%%%%%%%%%%%%%%%%%%%%%%%%%%
%%%%%%%%%%%%%%%%%%%%%%%%%%%%%%


\usepackage[utf8]{inputenc} % allow utf-8 input
\usepackage[T1]{fontenc}    % use 8-bit T1 fonts
\usepackage{hyperref}       % hyperlinks
\usepackage{url}            % simple URL typesetting
\usepackage{booktabs}       % professional-quality tables
\usepackage{amsfonts}       % blackboard math symbols
\usepackage{nicefrac}       % compact symbols for 1/2, etc.
\usepackage{microtype}      % microtypography
\usepackage{xcolor}         % colors
\usepackage{placeins}         % colors


%%%%%%%%%%%%%%%%%%%%%%%%%% more packages
\usepackage{amsthm}
\usepackage{amsmath}
\usepackage{graphicx}
\usepackage[capitalize,noabbrev]{cleveref}
\usepackage{algorithm}
\usepackage{algorithmic}
\usepackage{subcaption}
\usepackage[export]{adjustbox}% http://ctan.org/pkg/adjustbox

%%%%%%%%%%%%%%%%%%%%%%%%%%%%%%%%
% THEOREMS
%%%%%%%%%%%%%%%%%%%%%%%%%%%%%%%%
\theoremstyle{plain}
\newtheorem{theorem}{Theorem}[section]
\newtheorem{proposition}[theorem]{Proposition}
\newtheorem{lemma}[theorem]{Lemma}
\newtheorem{corollary}[theorem]{Corollary}
\theoremstyle{definition}
\newtheorem{definition}[theorem]{Definition}
\newtheorem{assumption}[theorem]{Assumption}
\theoremstyle{remark}
\newtheorem{remark}[theorem]{Remark}
%%%%%%%%%%%%%%%%%%%%%%%%%%%%%%%%%%%%%%%%%%%%%%%%%%%%%%%%%%%%%%%%%%%%%%% 
\usepackage[textsize=tiny]{todonotes}
\usepackage[export]{adjustbox}
% ################################################################
\usepackage{amsmath, amssymb} % Ensure these packages are included
\newcommand{\I}[1]{\mathbb{I} \left\{ #1 \right\}}
\renewcommand{\P}[1]{\mathbb{P} \left[ #1 \right]}
\newcommand{\EV}[1]{\mathbb{E} \left[ #1 \right]}

\usepackage{subcaption}
\DeclareMathOperator{\E}{\mathbb{E}}
\DeclareMathOperator{\p}{\mathbb{P}}
\DeclareMathOperator{\D}{\mathcal{D}}
\DeclareMathOperator{\H0}{\mathcal{H}_0}

\definecolor{Green}{RGB}{119,221,119}
\usepackage{pdflscape}
\usepackage{placeins}
\newcommand\numberthis{\addtocounter{equation}{1}\tag{\theequation}}

\usepackage{enumitem}
\allowdisplaybreaks
% ################################################################
\usepackage{xr}
\title{Robust Conformal Outlier Detection under\\Contaminated Reference Data}
\author[1]{Meshi Bashari}
\author[2,3]{Matteo Sesia}
\author[1,4]{Yaniv Romano}
\affil[1]{Department of Electrical and Computer Engineering, Technion IIT, Haifa, Israel}
\affil[2]{Department of Data Sciences and Operations, University of Southern California, Los Angeles, California, USA}
\affil[3]{Department of Computer Science, University of Southern California, Los Angeles, California, USA}
\affil[4]{Department of Computer Science, Technion IIT, Haifa, Israel}



\begin{document}

\date{}
\maketitle


\begin{abstract}
Conformal prediction is a flexible framework for calibrating machine learning predictions, providing distribution-free statistical guarantees. In outlier detection, this calibration relies on a reference set of labeled inlier data to control the type-I error rate. However, obtaining a perfectly labeled inlier reference set is often unrealistic, and a more practical scenario involves access to a contaminated reference set containing a small fraction of outliers. This paper analyzes the impact of such contamination on the validity of conformal methods. We prove that under realistic, non-adversarial settings, calibration on contaminated data yields conservative type-I error control, shedding light on the inherent robustness of conformal methods. This conservativeness, however, typically results in a loss of power. To alleviate this limitation, we propose a novel, active data-cleaning framework that leverages a limited labeling budget and an outlier detection model to selectively annotate data points in the contaminated reference set that are suspected as outliers. By removing only the annotated outliers in this ``suspicious'' subset, we can effectively enhance power while mitigating the risk of inflating the type-I error rate, as supported by our theoretical analysis. Experiments on real datasets validate the conservative behavior of conformal methods under contamination and show that the proposed data-cleaning strategy improves power without sacrificing validity.

\end{abstract}

\noindent \textbf{Keywords:}  Conformal Prediction, Hypothesis Testing, Out-of-Distribution Detection, Contaminated Data
\section{Introduction}
\label{sec:intro}
\subsection{Background and Motivation}
\label{sec:background}

This paper studies the problem of outlier detection: given a reference dataset (e.g., a collection of legitimate financial transactions) and an unlabeled test point (a new transaction), our goal is to determine whether the test point is an outlier (a fraudulent transaction) by assessing its deviation from the reference data distribution. Naturally, we aim to maximize the detection of outliers by harnessing the capabilities of complex machine learning (ML) models. However, these models typically lack type-I error rate control, potentially resulting in unreliable detections. In our running example, the type-I error is the probability of falsely flagging a legitimate transaction as fraudulent. As such, uncontrolled error rates can lead to costly unnecessary investigations of legitimate transactions and negatively impact customer experience. 

The broad need for reliable ML systems has sparked a surge of interest in conformal prediction---a versatile framework that can provide statistical guarantees for any ``black-box'' predictive model \cite{vovk2005algorithmic}. This framework formulates the outlier detection task as a statistical test, where the null hypothesis is that the new data point is not an outlier \cite{laxhammar2015inductive,conformal-p-values}. To derive a decision rule guaranteeing type-I error control, conformal inference relies on a reference (calibration) set of inlier data points. These points are assumed to be sampled i.i.d.~from an unknown distribution, independent of the data used to train the outlier detection model.

In practice, however, it is often difficult to obtain a perfectly clean reference dataset that contains no outliers \cite{park2021wrong,zhao2019robust,chalapathy2019deep,jiang2022softpatch}. 
In our example, a more realistic scenario would assume instead access to a slightly \emph{contaminated} reference set, mostly legitimate transactions with a few unnoticed outliers \cite{zhao2019robust}.
But this setting poses new challenges for conformal prediction methods, potentially invalidating the error control guarantees or, as we shall see, often reducing the power to detect true outliers at test time.

\subsection{Outline and Contributions}

While type-I error control in conformal inference theoretically requires perfectly clean reference data, in practice contaminated data often makes these methods overly conservative, reducing the power to detect true outliers rather than inflating the type-I error rate. This empirical observation motivates the first question explored in this paper:

\textbf{Q1:} \emph{When does conformal outlier detection with contaminated reference data yield valid type-I error control?}

In Section \ref{sec:conservativeness}, we present the first contribution of this paper: a novel theoretical analysis that identifies common conditions under which this conservative behavior arises. Unfortunately, this conservativeness often comes at the cost of reduced detection power, particularly when targeting low type-I error rates. To address this issue, we investigate data-driven cleaning strategies aimed at mitigating the contamination in the reference dataset.

A straightforward approach to cleaning the contaminated set is to remove all data points flagged as likely outliers by the detection model. However, this method is unsatisfactory, as it risks inadvertently removing inliers along with outliers, resulting in an "overly clean" reference set. This, in turn, distorts the inlier distribution and inflates the type-I error rate above the desired nominal level.

This challenge motivates our second and main contribution. In Section \ref{sec:label-trim}, we introduce an approach to clean the contaminated reference set by leveraging a limited labeling budget (e.g., 50 annotations). The outlier detection model is first used to identify suspected outliers within the contaminated reference set. The limited budget is then strategically allocated to annotate these points, thereby avoiding the unintended removal of inliers. While this is a practical and intuitive approach, it naturally prompts a critical question:

\textbf{Q2:} \emph{How does the selective annotation and partial removal of outliers from a contaminated reference set affect the validity of conformal inferences?}

We analyze the validity of this active labeling approach for trimming outliers in the contaminated set. Our theoretical results identify the conditions required to achieve approximate type-I error control, even when the data are selectively annotated and not all outliers are removed. This analysis also highlights key factors that can inflate the error rate, offering practitioners guiding principles to enhance the power of conformal methods in the presence of contaminated data.

Finally, in Section \ref{sec:experiments}, we empirically validate our theory and proposed data-cleaning approach through comprehensive experiments on real-world datasets. The experiments confirm that conformal inference with contaminated data tends to be conservative. Furthermore, they demonstrate that our method significantly boosts power, particularly when the target type-I error rate is low and the number of outliers in the contaminated set is small.
Software for reproducing the experiments is available at \href{https://github.com/Meshiba/robust-conformal-od}{https://github.com/Meshiba/robust-conformal-od}.

\subsection{Related Work}

Recently, there has been growing interest in studying the statistical properties of conformal inference methods under more realistic scenarios, moving beyond the idealized assumption of perfectly clean and exchangeable observations to account for various types of {\em distribution shift} \citep{tibshirani2019conformal,einbinder2022conformal,sesia2023adaptive,barber2023conformal,gibbs2021adaptive,zaffran2022adaptive,feldman2022achieving,gibbs2024conformal,podkopaev2021distribution,si2023pac,prinsterconformal}. This paper draws inspiration from several prior works in this area.

\citet{tibshirani2019conformal} introduced a weighted conformal prediction approach to address covariate shift between calibration and test data, later extended by \citet{podkopaev2021distribution} to accommodate label shift. Both settings, however, involve a different form of distribution shift from the one we study here. \citet{barber2023conformal} extend this line of work by analyzing the effects of general distribution shifts on the validity of conformal methods, focusing however on worst-case scenarios; see also \citet{farinhas2024nonexchangeable}.

In contrast, our work moves away from this worst-case perspective. We aim to explain why conformal outlier detection with contaminated reference data often results in a conservative type-I error rate, rather than investigating type-I error inflation, which, while theoretically possible in adversarial settings, appears less common in practice. Furthermore, we focus on developing methods to address this over-conservativeness, boosting detection power.

A more closely related line of work investigates the robustness of conformal prediction to label noise \cite{einbinder2022conformal, sesia2023adaptive, clarkson2024splitconformalpredictiondata,penso2025estimating} or other forms of data contamination \cite{pmlr-v202-zaffran23a, zaffran2024predictive, feldman2024robust}. Specifically, \citet{einbinder2022conformal} and \citet{sesia2023adaptive} show that, under certain assumptions, conformal prediction for classification with noisy labels often results in conservative type-I error rates. Furthermore, \citet{sesia2023adaptive} propose a method to address this conservativeness by leveraging an explicit ``label noise model'' that captures the relationship between the true and contaminated labels in the calibration dataset.

In contrast, this paper avoids relying on an explicit model for the contaminated data, as such models can be difficult to estimate in practice within our context. Instead, we utilize a pre-trained black-box outlier detection model and a limited annotation budget to selectively and reliably trim outliers from the contaminated set. Furthermore, it is important to emphasize that the method proposed by \citet{sesia2023adaptive} is primarily designed for classification tasks with relatively balanced data, whereas outlier detection naturally involves extreme class imbalance. This distinction underscores the need for solutions specifically tailored to outlier detection.

\section{Setup and Preliminary Results}

\subsection{Inference with Clean Calibration Data} \label{sec:cp}

Conformal inference for outlier detection requires a reference (or \emph{calibration}) set, $\D_{\mathrm{cal}} = [n] := \{1,\ldots,n\}$, containing $n$ data points. The reference set is typically assumed to be \emph{clean}, consisting solely of \emph{inliers}, which are i.i.d.~samples from an unknown distribution $\p_0$ (exchangeability may sometimes suffice, but this work assumes i.i.d.~inliers). Under this assumption, $\D_{\mathrm{cal}}$ may be referred to as $\D_{\mathrm{inlier}}$.

The goal is to determine whether a new observation, $X_{n+1}$, is an inlier—independently sampled from $\p_0$---or an \emph{outlier}, sampled from a different distribution $\p_1 \neq \p_0$. This can be formulated as a hypothesis testing problem, where the null hypothesis $\mathcal{H}_0$ claims that $X_{n+1}$ is an inlier:
\begin{align} \label{eq:setup-clean}
\begin{split}
  & X_i \overset{\text{i.i.d.}}{\sim} \p_0, \; \forall i \in \D_{\mathrm{inlier}}, \quad
  \D_{\mathrm{inlier}} = \D_{\mathrm{cal}} = [n], \\
  & \mathcal{H}_0 : X_{n+1} \overset{\text{ind.}}{\sim} \p_0.
\end{split}
\end{align}

The split-conformal method, a simple and computationally efficient approach, uses a pre-trained outlier detection model—potentially any machine learning model—to compute \emph{nonconformity scores} that quantify how different a data point is from the reference distribution. The model, represented by a score function $s$, is trained on a separate dataset $\D_{\mathrm{train}}$, which is similar to but independent of $\D_{\mathrm{cal}}$. Typically, a larger dataset of inliers, assumed to be i.i.d.~samples from $\p_0$, is randomly split into $\D_{\mathrm{train}}$ and $\D_{\mathrm{cal}}$.

The model tries to learn a score function $s$ such that larger values of $s(X_{n+1})$ indicate stronger evidence that the test point may be an outlier. Conformal inference rigorously quantifies this evidence, providing a principled decision rule for rejecting $\mathcal{H}_0$ when the evidence is strong enough, while controlling the type-I error rate—the probability of incorrectly rejecting $\mathcal{H}_0$ when $X_{n+1}$ is actually an inlier.

This statistical evidence is quantified by computing a \emph{conformal p-value}, defined as: 
\begin{align} \label{eq:conformal-p-value}
  \hat{p}_{n+1} = \frac{1 + \sum_{i=1}^{n} \mathbb{I}[s(X_i) \geq s(X_{n+1})]}{1 + n}. 
\end{align} 
Thus, larger values of $s(X_{n+1})$ correspond to smaller values of $\hat{p}_{n+1}$, and the test point $X_{n+1}$ can be confidently classified as an outlier (rejecting $\mathcal{H}_0$) when $\hat{p}_{n+1}$ is smaller than a given significance level $\alpha \in (0,1)$.

\begin{proposition}[from \citet{vovk2005algorithmic}] \label{prop:standard-conformal}
Under~\eqref{eq:setup-clean}, if the null hypothesis $\mathcal{H}_0$ is true, then for any $\alpha \in (0,1)$:
$\p \left( \hat{p}_{n+1} \leq \alpha \right) \leq \alpha$.
Further, if $s(X)$ has a continuous distribution under $\p_0$, then
$\p \left( \hat{p}_{n+1} \leq \alpha \right) \geq \alpha - 1/(n+1)$.
\end{proposition}

Proposition~\ref{prop:standard-conformal} intuitively states that the conformal p-value defined in~\eqref{eq:conformal-p-value} provides a well-calibrated rule for flagging a new data point as a likely outlier. Rejecting $\mathcal{H}_0$ when $\hat{p}_{n+1} \leq \alpha$ ensures type-I error control at level $\alpha$ while avoiding excessive conservatism. Specifically, the type-I error rate closely matches $\alpha$ when the sample size $n$ is large and the nonconformity scores have a continuous distribution with no ties—a mild condition that can be achieved in practice by adding small random noise to the scores.

What remains unclear, and serves as the starting point of this paper, is how conformal p-values behave when the calibration dataset is contaminated, containing not only inliers but also a fraction of misplaced outliers.


\subsection{Inference with Contaminated Calibration Data} \label{sec:notations}

In this paper, we consider a more general setting where the calibration dataset, indexed by $\D_{\mathrm{cal}} = [n]$, may contain both inliers ($\D_{\mathrm{inlier}}$), sampled i.i.d.~from a distribution $\p_0$, and outliers ($\D_{\mathrm{outlier}}$), sampled i.i.d.~from a different distribution $\p_1 \neq \p_0$. Thus, $\D_{\mathrm{cal}} = \D_{\mathrm{inlier}} \cup \D_{\mathrm{outlier}}$.
The numbers of inliers and outliers, respectively $n_0 = |\D_{\mathrm{inlier}}|$ and $n_1 = |\D_{\mathrm{outlier}}|$, are treated as fixed, with $n = n_0 + n_1$. 
The goal remains to test the null hypothesis $\mathcal{H}_0$ that a new data point $X_{n+1}$ is an inlier, independently sampled from $\p_0$. Formally, this setup can be written as:
\begin{align} \label{eq:setup-contaminated}
\begin{split}
  & X_i \overset{\text{i.i.d.}}{\sim} \p_0, \; \forall i \in \D_{\mathrm{inlier}}, \quad X_i \overset{\text{i.i.d.}}{\sim} \p_1, \; \forall i \in \D_{\mathrm{outlier}}, \\
  & \D_{\mathrm{inlier}} \cup \D_{\mathrm{outlier}} = \D_{\mathrm{cal}} = [n], \\
  & \mathcal{H}_0 : X_{n+1} \overset{\text{ind.}}{\sim} \p_0.
\end{split}
\end{align}

In the following, we first analyze the behavior of standard conformal p-values, computed as in~\eqref{eq:conformal-p-value}, when applied to contaminated data scenarios described by~\eqref{eq:setup-contaminated}. Subsequently, we will propose a novel method for computing more refined conformal p-values by approximately cleaning the calibration set to remove undesired outliers.

\subsection{Explaining the Conservativeness}\label{sec:conservativeness}

Empirical results suggest that contamination by outliers in the calibration data often makes standard conformal p-values overly conservative, resulting in a type-I error rate significantly lower than the desired nominal level $\alpha$.

We begin by examining~\Cref{fig:scores-shuttle}, which provides some insight into this behavior based on the analysis of the ``shuttle'' dataset \cite{shuttle}, as detailed in Section~\ref{sec:experiments}. In this example, the nonconformity scores of outlier data points in the contaminated calibration set are typically larger than those of the inliers.
This pattern reflects the goal of a well-designed outlier detection model: to differentiate outliers from inliers by assigning higher scores to the former. Consequently, conformal p-values computed using~\eqref{eq:conformal-p-value} will be inflated relative to the ideal scenario in which all calibration points are inliers, reducing our power to detect true outliers at test time.

\begin{figure}[!htb]
\centering
    \includegraphics[width=0.5\textwidth]{figures/exp/real_data/shuttle/hist_shuttle_n_2500_p_0.05_p_train_0.05_a_0.02.pdf}
    \caption{Histogram of nonconformity scores for inliers and outliers in a contaminated calibration subset of the ``shuttle" data, with a contamination rate of 5\%. The vertical lines indicate the $(1-\alpha)$ empirical quantile of all calibration scores (black), as well as separately for inliers (blue) and outliers (red), with $\alpha = 0.02$.}
    \label{fig:scores-shuttle}
\end{figure}

This phenomenon is further corroborated by extensive numerical experiments presented in Section~\ref{sec:experiments} and \Cref{app-sec:real-data-exp,app-sec:images-data-exp}, which consistently demonstrate this conservative behavior across nine different datasets.

While the conservativeness of conformal prediction methods in the presence of contaminated calibration data has already been observed and studied theoretically in different contexts \citep{einbinder2022conformal,sesia2023adaptive,clarkson2024splitconformalpredictiondata}, prior works did not focus on outlier detection.
Therefore, it is helpful to introduce a new theoretical result that precisely quantifies the inflation of standard conformal p-values that we often observe in practice. 
This will serve as a foundation for the novel method of computing adaptive conformal p-values presented in the next section.

Let $\hat{F}_{1}$ denote the empirical cumulative distribution function (CDF) of the scores in $\D_{\mathrm{outlier}}$ and $\hat{Q}_{1-\alpha}^{\mathrm{cal}}$ represent the $\lceil (1-\alpha)(1+n)\rceil$-th smallest score in the calibration set.

\begin{lemma}
\label{lem:conservativeness}
Under the setup defined in~\eqref{eq:setup-contaminated}, if $\mathcal{H}_0$ is true, then for any  $\alpha\in(0,1)$,
    \begin{align*}
    \p &\left( \hat{p}_{n+1}\leq \alpha \right) \leq \alpha -\frac{n_1}{n_0 + 1} \left( 1-\alpha -   \E\left[ \hat{F}_{1} \left( \hat{Q}^{\mathrm{cal}}_{1-\alpha} \right) \right]\right).
    \end{align*}
\end{lemma}

This result is related to Theorem 1 in \citet{sesia2023adaptive}, which studies the behavior of conformal prediction sets for multi-class classification \citep{lei2013distribution,romano2020classification} calibrated with contaminated data. The key distinction is in our treatment of the calibration set: we assume that $n_0$ and $n_1$ are fixed, whereas \citet{sesia2023adaptive} consider a mixture model where the observed proportions of data points from different classes in the calibration set are random.
While treating $n_0$ and $n_1$ as fixed is convenient for this paper, we also include an additional result (\Cref{cor:conservativeness-p-values}) in Appendix~\ref{app-sec:proofs}, which reaches qualitatively similar conclusions by adopting an approach more closely aligned with Theorem 1 from \citet{sesia2023adaptive}.

A direct corollary of \Cref{lem:conservativeness} is that standard conformal p-values are conservative when outlier scores are typically larger than inlier scores. Formally, this condition is:
\begin{assumption}\label{asm:model-scores} 
$\E [ \hat{F}_{1} ( \hat{Q}_{1-\alpha}^{\mathrm{cal}} ) ] < 1-\alpha$. 
\end{assumption}
If \Cref{asm:model-scores} fails—for instance, when outlier scores are smaller than inlier scores—data contamination may invalidate standard conformal p-values, inflating the type-I error rate.
However, it is more common in practice that \Cref{asm:model-scores} holds, in which case contamination tends to reduce calibration power, and more so if $n_1$ is large.
In particular, \Cref{asm:model-scores} holds if:
(i) the outlier detection model is relatively accurate, and
(ii) the outlier distribution $\p_1$ is not adversarial.
Our experiments will show this power loss can be substantial, motivating the need for new methods that can approximately ``clean up'' the calibration data.


\section{Methods}\label{sec:trim}

\subsection{Key Idea: Boosting Power by Cleaning the Data}

Ideally, we would like to remove all $n_1$ outliers from $\D_{\mathrm{cal}}$, restoring the ideal behavior of conformal p-values calibrated on a clean dataset, as described in \Cref{prop:standard-conformal}, and likely boosting power. However, manually labeling the entire contaminated calibration set $\D_{\mathrm{cal}}$ is often impractical, especially when $n = |\D_{\mathrm{cal}}|$ is large. At the same time, utilizing only a small calibration set is not always desirable.

A large calibration set is often needed because the smallest conformal p-value obtainable through~\eqref{eq:conformal-p-value} scales as $1/n$. Thus, a large $n$ is critical for achieving high confidence in identifying outliers, especially in ``needle-in-a-haystack'' scenarios \citep{conformal-p-values}, where a few outliers must be detected in a large test set dominated by inliers. In such cases, the ability to obtain very small p-values is essential to achieve non-trivial power while controlling the false discovery rate \citep{BH}.

\subsection{A Simple but Unsatisfactory Approach: Naive-Trim} \label{sec:naive-trim}

The above challenge underscores the need for a method to mitigate the impact of outliers in the calibration dataset without requiring exhaustive annotation. An intuitive approach is to forgo annotations and simply remove all ``suspicious'' data points with large nonconformity scores. For instance, one could remove the top $m$ scores from $\D_{\mathrm{cal}}$, where $m$ is a fixed guess of the true number of outliers $n_1$ in the calibration set. We refer to this approach as \texttt{Naive-Trim}.

While \texttt{Naive-Trim} can reduce conservativeness by removing some outliers, it is not a satisfactory solution as it risks ``over-compensating''. By potentially removing also true inliers with large nonconformity scores, it can significantly skew the inlier score distribution to the left. This side effect is problematic, as it tends to invalidate conformal p-values and inflate the type-I error rate, over-correcting the conservativeness of standard conformal p-values.

This issue is particularly pronounced when $m > n_1$ or in noisy settings where the outlier detection model cannot perfectly distinguish between inliers and outliers. For example, as shown in Section~\ref{sec:experiments}, applying \texttt{Naive-Trim} to the dataset illustrated in \Cref{fig:scores-shuttle} results in uncontrolled inflation of the type-I error rate.

To address this challenge, we will now present a more sophisticated method, which we refer to as  \texttt{Label-Trim}. These approach utilize a limited labeling budget to remove outliers from $\D_{\mathrm{cal}}$ in a more reliable manner, mitigating the risk of over-correcting the conformal p-value.

\subsection{The \texttt{Label-Trim} Method}\label{sec:label-trim}

Consider having a limited budget to label $m < n$ calibration samples, where $m$ is much smaller than $n$. We aim to utilize this budget to remove as many outliers as possible from the calibration set without altering the inlier score distribution.

A practical approach is to annotate the $m$ largest scores in $\D_{\mathrm{cal}}$, as these are most likely outliers based on the model. Denote these annotated samples as $\D_{\mathrm{labeled}} \subseteq \D_{\mathrm{cal}}$, and let $\D_{\mathrm{labeled}}^{\mathrm{outlier}}$ denote the subset of annotated data points that are true outliers. Removing these outliers from $\D_{\mathrm{cal}}$ yields a smaller, cleaner calibration set, which we call $\D_{\mathrm{cal}}^{\mathrm{LT}} \subseteq \D_{\mathrm{cal}}$.

The \texttt{Label-Trim} method then calculates a refined conformal p-value, now denoted as $\hat{p}^{\mathrm{LT}}_{n+1}$, following the same procedure as in~\eqref{eq:conformal-p-value} with $\D_{\mathrm{cal}}$ replaced by the (partially) cleaned calibration set $\D_{\mathrm{cal}}^{\mathrm{LT}}$:
\begin{align} \label{eq:LT-p-value}
  \hat{p}^{\mathrm{LT}}_{n+1} = \frac{1 + \sum_{i \in \D_{\mathrm{cal}}^{\mathrm{LT}}} \mathbb{I}[s(X_i) \geq s(X_{n+1})]}{1 + |\D_{\mathrm{cal}}^{\mathrm{LT}}|}. 
\end{align} 
Algorithms~\ref{algo:label-trim-construction} and~\ref{algo:label-trim-testing} summarize this procedure, which intuitively offers advantages over both the standard method for computing $\hat{p}_{n+1}$ in~\eqref{eq:conformal-p-value}, by potentially increasing power, and the \texttt{Naive-Trim} approach, by mitigating the risk of over-correcting $\hat{p}_{n+1}$.

\begin{algorithm}[!htb]
\caption{Label-trim calibration  (construction phase)}
\label{algo:label-trim-construction}
\begin{algorithmic}[1]
\STATE \textbf{Input:} labeling budget $m$; contaminate calibration-set $\D_{\mathrm{cal}} = \left\{ X_i \right\}_{i=1}^{n}$; score function $s(\cdot)$, obtained by a pre-trained outlier detection model;

\STATE Compute the calibration scores $S_i = s(X_i)$, $\forall i\in \D_{\mathrm{cal}}$.
\STATE Sort the calibration scores, such that $S_{\pi(1)} \leq \dots \leq S_{\pi(n)}$ where $\pi : [n] \rightarrow [n]$ is the corresponding permutation of the indices.
\STATE Annotate the $m$ largest scores  $\D_{\mathrm{labeled}}$:= $\{ (S_{\pi(i)}, Y_{\pi(i)}) : i > n-m\}$, with $Y_{\pi(i)}=0$ if $X_{\pi(i)}$ is an inlier and $Y_{\pi(i)}=1$ otherwise. 
\STATE Construct the trimmed calibration set $\D_{\mathrm{cal}}^{\mathrm{LT}} = \left\{\pi (i) : i \leq n - m\right\}\cup \left\{j:  j\in \D_{\mathrm{labeled}}\text{ and }Y_j=0\right\}$.
% \STATE Construct the trimmed calibration set $\D_{\mathrm{cal}}^{\mathrm{LT}} = \left\{S_{\pi(i)}\right\}_{i=1}^{n-m} \cup \left\{S_{j}:  j\in \D_{\mathrm{labeled}}\text{ and }Y_j=0\right\}$

\STATE \textbf{Output:} trimmed calibration set $\D_{\mathrm{cal}}^{\mathrm{LT}}$.
\end{algorithmic}
\end{algorithm}

\begin{algorithm}[!htb]
\caption{Label-trim calibration (testing phase)}
\label{algo:label-trim-testing}
\begin{algorithmic}[1]
\STATE \textbf{Input:} test point $X_{n+1}$; score function $s$; trimmed calibration set $\D_{\mathrm{cal}}^{\mathrm{LT}}$; type-I error level $\alpha$;
% \STATE Compute the test score $S_{n+1}=s(X_{n+1})$
\STATE Compute the conformal p-value $\hat{p}^{\mathrm{LT}}_{n+1}$ according to \eqref{eq:LT-p-value}.
\STATE \textbf{Output:} reject the null hypothesis $\mathcal{H}_0$ if $\hat{p}_{n+1}^{\mathrm{LT}} \leq \alpha$, classifying $X_{n+1}$ as an outlier.
\end{algorithmic}
\end{algorithm}

The following theorem provides justification for \texttt{Label-Trim}, demonstrating that $\hat{p}^{\mathrm{LT}}_{n+1}$ is an approximately valid p-value under relatively mild conditions. While our method is intuitive, this result is nontrivial for two reasons.
First, \texttt{Label-Trim} cannot guarantee the removal of all outliers from the calibration set, as it may be that $m < n_1$ or some outliers are not among the $m$ largest scores. Second, it involves annotating the $m$ largest scores, revealing the true labels of some calibration points but not others, which could disrupt the exchangeability typically assumed among inlier data points in conformal inference.
Therefore, this justification requires novel proof techniques and does not follow directly from existing results.

Following a notation similar to that of Lemma~\ref{lem:conservativeness}, let $\hat{F}^{\mathrm{LT}}_{1}$ denote the empirical CDF of the scores in $\D^{\mathrm{LT}}_{\mathrm{outlier}}$. Define also $\hat{Q}^{\mathrm{LT}}_{1-\alpha}$ as the $\hat{i}_{\mathrm{LT}}$-th smallest element in $\{S_i\}_{i\in \D_{\mathrm{cal}}^{\mathrm{LT}}}\cup\{\infty\}$, with $\hat{i}_{\mathrm{LT}} :=\lceil (1-\alpha)(n^{\mathrm{LT}}+1)\rceil$ and $n^{\mathrm{LT}}:=\left|\D_{\mathrm{cal}}^{\mathrm{LT}}\right|$.

\begin{theorem}
    \label{thm:labeled-trim}
    Consider the setup in~\eqref{eq:setup-contaminated}, with $\mathcal{H}_0$ being true.
    For any fixed $\alpha \in (0,1)$, assume that $m \leq \alpha (n+1)$.
    Then,
    \begin{align*}
     \p \left( \hat{p}_{n+1}^{\mathrm{LT}} \leq \alpha \right) \leq \alpha + \frac{1}{n_0+1}
         - \E\left[ \frac{\hat{n}^{\mathrm{LT}}_{1}}{n_0+1} \left( (1-\alpha) - \hat{F}_{1}^{\mathrm{LT}} \left( \hat{Q}_{1-\alpha}^{\mathrm{LT}} \right) \right)\right].
    \end{align*}
\end{theorem}

The upper bound on the type-I error rate provided by Theorem~\ref{thm:labeled-trim} resembles that of Lemma~\ref{lem:conservativeness} and can be interpreted as follows: \texttt{Label-Trim} produces approximately valid conformal p-values if: (i) the labeling budget is small relative to the calibration set size, i.e., $m \leq \alpha(n+1)$; and (ii) the calibration set contains a large number of inliers, $n_0$.

However, the upper bound in Theorem~\ref{thm:labeled-trim} also suggests that \texttt{Label-Trim} may remain overly conservative, similar to standard conformal p-values, if (i) not all outliers are removed from the calibration set ($\hat{n}^{\mathrm{LT}}_{1} > 0$ with high probability), and (ii) the remaining outlier scores are generally larger than the remaining inlier scores, consistent with $\E [ \hat{F}^{\mathrm{LT}}_{1} ( \hat{Q}^{\mathrm{LT}}_{1-\alpha} ) ] < 1-\alpha$, akin to Assumption~\ref{asm:model-scores}. 

This potential conservative behavior arises naturally from the use of a limited labeling budget, especially when the model guiding the construction of the set $\D_{\mathrm{labeled}}$ fails to effectively detect true outliers. Nevertheless, as we will see in the next section, \texttt{Label-Trim} often enhances power.

\section{Experiments} \label{sec:experiments}

We turn to evaluate the performance of conformal outlier detection methods under contaminated data. The experiments presented in this section are conducted on nine benchmark datasets: three tabular datasets, listed in Section~\ref{sec:real-data-exp}, and six visual datasets, listed in \Cref{sec:img-exp}.

\paragraph{Methods} We compare the following methods:
\begin{itemize}[noitemsep, topsep=0pt]
    \item \texttt{Standard}: The basic conformal method that uses the contaminated reference set $\mathcal{D}_\text{cal}=\mathcal{D}_\text{inlier}\cup\mathcal{D}_\text{outlier}$.
    \item \texttt{Oracle}: An infeasible benchmark method where the reference set contains only inliers, i.e., $\mathcal{D}_\text{cal}=\mathcal{D}_\text{inlier}$.
    \item \texttt{Naive-Trim}: The baseline method from Section~\ref{sec:naive-trim}, which removes the top $r\%$ non-conformity scores from $\mathcal{D}_\text{cal}$, where $r = n_1 / (n_0 + n_1)$.
    \item \texttt{Label-Trim}: Our proposed reliable data-cleaning method from Section~\ref{sec:label-trim}, applied with a labeling budget of $m=50$ annotations to label the $m$ data points with the largest non-conformity scores from $\mathcal{D}_\text{cal}$.
    \item \texttt{Small-Clean}: A baseline method that uses the labeling budget to construct a small, clean reference set by (i) randomly selecting $m$ data points from $\mathcal{D}_\text{cal}$ and (ii) extracting the true inliers from this subset.
\end{itemize}

\paragraph{Setup and performance metrics} In all experiments, we randomly split a given dataset into disjoint training $\mathcal{D}_\text{train}$, calibration $\mathcal{D}_\text{cal}$, and test sets of inliers $\mathcal{D}_\text{test}^\text{inlier}$ and outliers $\mathcal{D}_\text{test}^\text{outlier}$. To simulate a realistic setting, we construct the training and contaminated calibration sets with the same contamination rate of $r\%$. The inlier $\mathcal{D}_\text{test}^\text{inlier}$ and outlier $\mathcal{D}_\text{test}^\text{outlier}$ test tests are used to compute the type-I error and power of the outlier detection model, respectively. To ensure fair comparisons, all conformal methods use the same outlier detection model, trained on $\mathcal{D}_\text{train}$. Performance metrics are evaluated across 100 random splits of the data. The size of each dataset, along with the details of how  $\mathcal{D}_\text{train}$, $\mathcal{D}_\text{cal}$, and $\mathcal{D}_\text{test}$ are constructed are provided in~\Cref{app-sec:data}. 

\subsection{Tabular Data} \label{sec:real-data-exp}

We now compare the performance of the different methods on three benchmark tabular datasets for outlier detection, previously used in the conformal literature \citep{conformal-p-values}. 
Since conclusions are similar across datasets, we focus here on results for the {\em shuttle} dataset \citep{shuttle}. Results for the {\em credit card} \citep{creditcard} and {\em KDDCup99} \citep{KDDCup99} datasets are presented in \Cref{app-sec:real-data-exp}. For all conformal methods, we use Isolation Forest~\citep{liu2008isolation} as the base outlier detection model, implemented using \texttt{scikit-learn} with default hyperparameters~\citep{sklearn_api}.

\begin{figure*}[!h]
    % \centering 
    \includegraphics[height=3.3cm, valign=t]{figures/exp/real_data/shuttle/outlier_prop/IF_e_100_s_auto_train_5000_exp_outliers_calib_shuttle_1_fdr_model_0.5_initial_50_cal_2500_p_0.05_test_1000_p_0.05_q_0.02/Type-1-Error_point_no_legend.pdf}
    \includegraphics[height=3.3cm, valign=t]{figures/exp/real_data/shuttle/outlier_prop/IF_e_100_s_auto_train_5000_exp_outliers_calib_shuttle_1_fdr_model_0.5_initial_50_cal_2500_p_0.05_test_1000_p_0.05_q_0.02/Power_point_no_legend.pdf}
    \includegraphics[height=3.3cm, valign=t]{figures/exp/real_data/shuttle/outlier_prop/IF_e_100_s_auto_train_5000_exp_outliers_calib_shuttle_1_fdr_model_0.5_initial_50_cal_2500_p_0.05_test_1000_p_0.05_q_0.02/Trimmed_point_no_legend.pdf}
    \includegraphics[width=3.3cm, valign=t]{figures/exp/legend.pdf}
    \caption{Comparison of conformal outlier detection methods on a tabular dataset (``shuttle'') as a function of the contamination rate  $r$ . The target type-I error rate is  $\alpha = 0.02$. Left: Empirical type-I error. Middle: Average detection rate (power), where higher values indicate better performance. Right: Number of outliers trimmed by the \texttt{Label-Trim} method. Results are averaged across 100 random splits of the data. 
}
    \label{fig:shuttle-outlier-prop}
\end{figure*}

\Cref{fig:shuttle-outlier-prop} presents the performance metrics of each method as a function of the contamination rate $r$. 
Following the left panel in that figure, we can see that the \texttt{Standard} conformal method results in conservative type-I error control, with a decrease in the error rate as the outlier proportion increases---a behavior that is aligned with~\Cref{lem:conservativeness}. Notably, the type-I error of the \texttt{Oracle} method is tightly centered around $\alpha$, as guaranteed by~\Cref{prop:standard-conformal}. The \texttt{Naive-Trim} method does not control the type-I error rate, emphasizing the need for reliable data-cleaning procedures. In striking contrast, our \texttt{Label-Trim} method, achieves a valid type-I error rate. At lower outlier proportions, the empirical type-I error is close to  $\alpha$, but the method becomes more conservative as the outlier proportion increases. This observation aligns with the upper bound on the error rate derived in~\Cref{thm:labeled-trim}. Notably, as the contamination rate in the training data increases, the outlier detection model’s ability to distinguish between inliers and outliers weakens. This, in turn, adversely affects the effectiveness of forming a subset of data points for annotation, as demonstrated in the right panel of \Cref{fig:shuttle-outlier-prop}. The \texttt{Small-Clean} method also controls the type-I error but is more conservative than \texttt{Label-Trim} due to its much smaller reference set, which becomes even smaller as the contamination rate increases. Observe how the power of the \texttt{Small-Clean} method is lower than that of the \texttt{Standard} approach, despite the latter using a contaminated reference set. By contrast, our proposed \texttt{Label-Trim} method significantly improves the power of the \texttt{Standard} method and even achieves near-oracle performance when the outlier proportion is low.

Next, we study the effect of the labeling budget on the performance of our \texttt{Label-Trim} method. As shown in \Cref{fig:shuttle-labeled-exp}, increasing the labeling budget brings the \texttt{Label-Trim} method closer to the \texttt{Oracle} in terms of both type-I error and power. Notably, even with a modest budget of $40$–$50$ annotations, the power of \texttt{Label-Trim} is nearly indistinguishable from that of the \texttt{Oracle}. This is attributed to the method’s effective trimming of outliers, as shown in the right panel. Notably, for labeling budgets $m > 50$, the condition in~\Cref{thm:labeled-trim} no longer holds, and yet the \texttt{Label-Trim} method still achieves valid type-I error control at level  $\alpha$ in practice. This highlights the robustness of the proposed method to the choice of $m$ beyond the restrictions specified in \Cref{thm:labeled-trim}, where we attribute this robustness to the non-adversarial nature of the outlier distribution and the underlying detection model. 

\Cref{fig:shuttle-labeled-exp} also illustrates that the \texttt{Small-Clean} method lags behind \texttt{Label-Trim} both in terms of power and conservativeness. For small labeling budgets of $m < 45$, the coarse granularity of conformal p-values \eqref{eq:conformal-p-value} renders the method powerless; the smallest achievable p-value in this case is $1/(m+1) > 0.02 = \alpha$. Even for slightly larger labeling budgets, the conservative nature of the conformal p-value---specifically, the `plus 1’ term in \eqref{eq:conformal-p-value}---continues to have a significant impact. This effect is rigorously quantified by the lower bound on type-I error provided in \Cref{prop:standard-conformal}. For instance, with $m = 80$ and $\alpha = 0.02$, the lower bound is approximately $\alpha - 1/(m+1) \approx 0.0076$, which aligns closely with the empirical error rate shown in the left panel of \Cref{fig:shuttle-labeled-exp}. Overall, these results highlight the benefits of selectively cleaning a relatively large contaminated set compared to relying on a small clean reference set, offering both improved stability and higher power.

\begin{figure*}[htb]
    % \centering 
    \includegraphics[height=3.3cm, valign=t]{figures/exp/real_data/shuttle/labeled_size/IF_e_100_s_auto_train_5000_exp_clean_calib_size_shuttle_1_fdr_model_0.5_initial_50_cal_2500_p_0.03_test_1000_p_0.05_q_0.02/Type-1-Error_point_no_legend.pdf}
    \includegraphics[height=3.3cm, valign=t]{figures/exp/real_data/shuttle/labeled_size/IF_e_100_s_auto_train_5000_exp_clean_calib_size_shuttle_1_fdr_model_0.5_initial_50_cal_2500_p_0.03_test_1000_p_0.05_q_0.02/Power_point_no_legend.pdf}
    \includegraphics[height=3.3cm, valign=t]{figures/exp/real_data/shuttle/labeled_size/IF_e_100_s_auto_train_5000_exp_clean_calib_size_shuttle_1_fdr_model_0.5_initial_50_cal_2500_p_0.03_test_1000_p_0.05_q_0.02/Trimmed_point_no_legend.pdf}
    \includegraphics[width=3.3cm, valign=t]{figures/exp/legend_wo_n.pdf}
    \caption{Comparison of conformal outlier detection methods on a tabular dataset (``shuttle'') as a function of the labeling budget $m$. The contamination rate is fixed to $r=0.03$. Other details are as in \Cref{fig:shuttle-outlier-prop}.}
    \label{fig:shuttle-labeled-exp}
\end{figure*}


Next, we examine how the target error level $\alpha$ affects the performance of different methods. \Cref{fig:shuttle-levels} shows that our \texttt{Label-Trim} method performs particularly well at low type-I error rates, especially when $\alpha$ is smaller than the contamination rate ($r=3\%$). This behavior can be explained as follows. For a relatively accurate model, the outliers primarily distort the tail of the empirical distribution of nonconformity scores---see \Cref{fig:scores-shuttle}. Consequently, the influence of these outliers on the rejection rule $\hat{p}_{n+1} \leq \alpha$ from \eqref{eq:conformal-p-value}, or $\hat{p}^{\text{LT}}_{n+1} \leq \alpha$ from \eqref{eq:LT-p-value}, diminishes as $\alpha$ increases.

\begin{figure*}[htb]
    \centering 
    \includegraphics[height=3.3cm, valign=t]{figures/exp/real_data/shuttle/levels/IF_e_100_s_auto_train_5000_exp_levels_shuttle_1_fdr_model_0.5_initial_50_cal_2500_p_0.03_test_1000_p_0.05_q_0.02/Type-1-Error_point_no_legend.pdf}
    \includegraphics[height=3.3cm, valign=t]{figures/exp/real_data/shuttle/levels/IF_e_100_s_auto_train_5000_exp_levels_shuttle_1_fdr_model_0.5_initial_50_cal_2500_p_0.03_test_1000_p_0.05_q_0.02/Power_point_no_legend.pdf}
    \includegraphics[width=3.3cm, valign=t]{figures/exp/legend_wo_trm.pdf}
    \caption{Comparison of conformal outlier detection methods on a tabular dataset (``shuttle'') as a function of the target type-I error rate $\alpha$. The contamination rate $r$ is fixed to 3\%. Other details are as in \Cref{fig:shuttle-outlier-prop}.
}
    \label{fig:shuttle-levels}
\end{figure*}

\subsection{Visual Data}\label{sec:img-exp}

In what follows, we compare all methods using benchmark visual datasets for outlier detection. Similar to \citet{zhang2023openood}, we construct six datasets, where the inlier samples are always images from CIFAR10~\citep{cifar-10, cifar-data} and the outlier samples vary across datasets. Specifically, the outliers are drawn from (1) MNIST~\citep{deng2012mnist}, (2) SVHN~\citep{svhn}, (3) Texture~\citep{texture}, (4) Places365~\citep{texture}, (5) TinyImageNet~\citep{tinyimages}, and (6) CIFAR100~\citep{cifar-data}. For all datasets, we use the outlier detection model proposed by \citet{react}, which operates on feature representations extracted by a pre-trained ResNet18 model. More details are in~\Cref{app-sec:data}.

\Cref{tab:avg-images} summarizes the results for all six datasets. Overall, we can see a trend similar to the one of the tabular data: the \texttt{Standard} and \texttt{Small-Clean} methods are valid but conservative, the \texttt{Naive-Trim} fails to control the type-I error, and our \texttt{Label-Trim} achieves a significant boost in power while practically controlling the type-I error. Notably, our \texttt{Label-Trim} method attains near-oracle performance for low contamination rates. Detailed results for each dataset are provided in \Cref{app-sec:images-data-exp}.

\begin{table*}[!htb]
\caption{Comparison of conformal outlier detection methods on six visual datasets for varying contamination rate $r$ and target type-I error level $\alpha$. The empirical type-I error values are averaged across all datasets. The empirical power is presented relative to the \texttt{Standard} method (higher is better), and averaged across all datasets. Results are averaged across 100 random splits of the data, with standard errors presented in parentheses.
}
\label{tab:avg-images}
\centering
\resizebox{\textwidth}{!}{
\begin{tabular}{l|ll|ll|ll}
\hline
& \multicolumn{6}{c}{Contamination rate} \\
\hline
 & \multicolumn{2}{c|}{1\%} & \multicolumn{2}{c|}{3\%} & \multicolumn{2}{c}{5\%} \\ \hline
 Method      & Power & Type-I Error & Power & Type-I Error & Power & Type-I Error \\ \hline
Standard & \bfseries \cellcolor{Green!30} 1.0 ($\pm$ 0.0317) & \cellcolor{white} 0.008 ($\pm$ 0.0003)  & \bfseries \cellcolor{Green!30} 1.0 ($\pm$ 0.0354) & \cellcolor{white} 0.005 ($\pm$ 0.0003)  & \bfseries \cellcolor{Green!30} 1.0 ($\pm$ 0.0408) & \cellcolor{white} 0.004 ($\pm$ 0.0002) \\

Oracle (infeasible) & \bfseries \cellcolor{Green!100} 1.166 ($\pm$ 0.0336) & \cellcolor{white} 0.01 ($\pm$ 0.0003)  & \bfseries \cellcolor{Green!100} 1.549 ($\pm$ 0.0425) & \cellcolor{white} 0.01 ($\pm$ 0.0003)  & \bfseries \cellcolor{Green!100} 1.961 ($\pm$ 0.0531) & \cellcolor{white} 0.009 ($\pm$ 0.0004) \\

Naive-Trim (invalid) & \cellcolor{red!20} 1.659 ($\pm$ 0.0342) & \cellcolor{red!20} 0.017 ($\pm$ 0.0004)  & \cellcolor{red!20} 2.79 ($\pm$ 0.045) & \cellcolor{red!20} 0.027 ($\pm$ 0.0006)  & \cellcolor{red!20} 4.16 ($\pm$ 0.0596) & \cellcolor{red!20} 0.036 ($\pm$ 0.0007) \\

Small-Clean & \cellcolor{white} 0.0 ($\pm$ 0.0) & \cellcolor{white} 0.0 ($\pm$ 0.0)  & \cellcolor{white} 0.0 ($\pm$ 0.0) & \cellcolor{white} 0.0 ($\pm$ 0.0)  & \cellcolor{white} 0.0 ($\pm$ 0.0) & \cellcolor{white} 0.0 ($\pm$ 0.0) \\

Label-Trim & \bfseries \cellcolor{Green!100} 1.166 ($\pm$ 0.0336) & \cellcolor{white} 0.01 ($\pm$ 0.0003)  & \bfseries \cellcolor{Green!60} 1.517 ($\pm$ 0.042) & \cellcolor{white} 0.01 ($\pm$ 0.0003)  & \bfseries \cellcolor{Green!60} 1.786 ($\pm$ 0.0498) & \cellcolor{white} 0.008 ($\pm$ 0.0003) \\
\end{tabular}
}
\subcaption{Target type-I error rate $\alpha=0.01$}

\resizebox{\textwidth}{!}{
\begin{tabular}{l|ll|ll|ll}
\hline
& \multicolumn{6}{c}{Contamination rate} \\
\hline
 & \multicolumn{2}{c|}{1\%} & \multicolumn{2}{c|}{3\%} & \multicolumn{2}{c}{5\%} \\ \hline
 Method      & Power & Type-I Error & Power & Type-I Error & Power & Type-I Error \\ \hline
Standard & \bfseries \cellcolor{Green!30} 1.0 ($\pm$ 0.0174) & \cellcolor{white} 0.027 ($\pm$ 0.0006)  & \bfseries \cellcolor{Green!30} 1.0 ($\pm$ 0.0189) & \cellcolor{white} 0.019 ($\pm$ 0.0005)  & \cellcolor{white} 1.0 ($\pm$ 0.0212) & \cellcolor{white} 0.015 ($\pm$ 0.0005) \\

Oracle (infeasible) & \bfseries \cellcolor{Green!100} 1.062 ($\pm$ 0.0176) & \cellcolor{white} 0.03 ($\pm$ 0.0006)  & \bfseries \cellcolor{Green!100} 1.235 ($\pm$ 0.0192) & \cellcolor{white} 0.029 ($\pm$ 0.0006)  & \bfseries \cellcolor{Green!100} 1.448 ($\pm$ 0.023) & \cellcolor{white} 0.03 ($\pm$ 0.0007) \\

Naive-Trim (invalid) & \cellcolor{red!20} 1.146 ($\pm$ 0.0175) & \cellcolor{red!20} 0.035 ($\pm$ 0.0006)  & \cellcolor{red!20} 1.487 ($\pm$ 0.0186) & \cellcolor{red!20} 0.043 ($\pm$ 0.0007)  & \cellcolor{red!20} 1.882 ($\pm$ 0.0224) & \cellcolor{red!20} 0.052 ($\pm$ 0.0008) \\

Small-Clean & \cellcolor{white} 0.714 ($\pm$ 0.0448) & \cellcolor{white} 0.02 ($\pm$ 0.0021)  & \cellcolor{white} 0.869 ($\pm$ 0.0501) & \cellcolor{white} 0.02 ($\pm$ 0.002)  & \bfseries \cellcolor{Green!30} 1.033 ($\pm$ 0.0613) & \cellcolor{white} 0.021 ($\pm$ 0.0023) \\

Label-Trim & \bfseries \cellcolor{Green!60} 1.041 ($\pm$ 0.0177) & \cellcolor{white} 0.029 ($\pm$ 0.0006)  & \bfseries \cellcolor{Green!60} 1.139 ($\pm$ 0.019) & \cellcolor{white} 0.025 ($\pm$ 0.0006)  & \bfseries \cellcolor{Green!60} 1.215 ($\pm$ 0.0226) & \cellcolor{white} 0.021 ($\pm$ 0.0006) \\
\end{tabular}
}
\subcaption{Target type-I error rate $\alpha=0.03$}
\end{table*}
% \FloatBarrier
\section{Discussion}

In this work, we studied the robustness of conformal prediction under contaminated reference data. Motivated by empirical evidence, we characterized the conditions under which conformal outlier detection methods become too conservative. To improve power, we proposed the \texttt{Label-Trim} method, which leverages an outlier detection model and a limited labeling budget to remove outliers from the contaminated reference set. We also provided a theoretical justification for this approach, employing novel proof techniques. Numerical experiments with real data confirmed that standard conformal outlier detection methods are conservative under contaminated data and demonstrated that our \texttt{Label-Trim} method can significantly enhance power.

However, the experiments also reveal a limitation of our \texttt{Label-Trim} method: while it improves power compared to standard conformal inference, it often remains too conservative, particularly when the labeling budget is very limited, leaving room for further improvement. A promising direction for future research is to enhance \texttt{Label-Trim} with {\em active learning} strategies \cite{makili2012active,fannjiang2022conformal,prinsterconformal}, enabling the removal of more outliers without increasing the labeling budget.

A second limitation of the \texttt{Label-Trim} approach is its reliance on actively collecting new annotations. In scenarios where a flexible labeling budget is unavailable but access to a small, clean reference set is feasible, this dependency becomes restrictive. As our experiments demonstrate, the limited sample size imposes a fundamental constraint on the power of conformal outlier detection methods. This raises an intriguing question for future research: given a small clean reference set and a larger, contaminated reference set, how can we effectively and safely clean the contaminated data to enhance detection power at test time?

One potential solution could involve using the small clean reference set to calibrate a base outlier detection model. This calibrated model could then be employed to clean the larger contaminated set by removing detected outliers, while carefully accounting for inliers mistakenly classified as outliers. Exploring such a semi-supervised data-cleaning approach represents a promising direction for future work, though we anticipate that establishing the theoretical validity of such a method may not be straightforward.

\section*{Acknowledgments}
M.~S.~was partly supported by NSF grant DMS 2210637 and by a Capital One CREDIF Research Award.
Y.~R. and M.~B. were funded by the European Union (ERC, SafetyBounds, 101163414). Views and opinions expressed are however those of the authors only and do not necessarily reflect those of the European Union or the European Research Council Executive Agency. Neither the European Union nor the granting authority can be held responsible for them.


% \bibliography{bib}
% This must be in the first 5 lines to tell arXiv to use pdfLaTeX, which is strongly recommended.
\pdfoutput=1
% In particular, the hyperref package requires pdfLaTeX in order to break URLs across lines.

\documentclass[11pt]{article}

% Change "review" to "final" to generate the final (sometimes called camera-ready) version.
% Change to "preprint" to generate a non-anonymous version with page numbers.
\usepackage{acl}

% Standard package includes
\usepackage{times}
\usepackage{latexsym}

% Draw tables
\usepackage{booktabs}
\usepackage{multirow}
\usepackage{xcolor}
\usepackage{colortbl}
\usepackage{array} 
\usepackage{amsmath}

\newcolumntype{C}{>{\centering\arraybackslash}p{0.07\textwidth}}
% For proper rendering and hyphenation of words containing Latin characters (including in bib files)
\usepackage[T1]{fontenc}
% For Vietnamese characters
% \usepackage[T5]{fontenc}
% See https://www.latex-project.org/help/documentation/encguide.pdf for other character sets
% This assumes your files are encoded as UTF8
\usepackage[utf8]{inputenc}

% This is not strictly necessary, and may be commented out,
% but it will improve the layout of the manuscript,
% and will typically save some space.
\usepackage{microtype}
\DeclareMathOperator*{\argmax}{arg\,max}
% This is also not strictly necessary, and may be commented out.
% However, it will improve the aesthetics of text in
% the typewriter font.
\usepackage{inconsolata}

%Including images in your LaTeX document requires adding
%additional package(s)
\usepackage{graphicx}
% If the title and author information does not fit in the area allocated, uncomment the following
%
%\setlength\titlebox{<dim>}
%
% and set <dim> to something 5cm or larger.

\title{Wi-Chat: Large Language Model Powered Wi-Fi Sensing}

% Author information can be set in various styles:
% For several authors from the same institution:
% \author{Author 1 \and ... \and Author n \\
%         Address line \\ ... \\ Address line}
% if the names do not fit well on one line use
%         Author 1 \\ {\bf Author 2} \\ ... \\ {\bf Author n} \\
% For authors from different institutions:
% \author{Author 1 \\ Address line \\  ... \\ Address line
%         \And  ... \And
%         Author n \\ Address line \\ ... \\ Address line}
% To start a separate ``row'' of authors use \AND, as in
% \author{Author 1 \\ Address line \\  ... \\ Address line
%         \AND
%         Author 2 \\ Address line \\ ... \\ Address line \And
%         Author 3 \\ Address line \\ ... \\ Address line}

% \author{First Author \\
%   Affiliation / Address line 1 \\
%   Affiliation / Address line 2 \\
%   Affiliation / Address line 3 \\
%   \texttt{email@domain} \\\And
%   Second Author \\
%   Affiliation / Address line 1 \\
%   Affiliation / Address line 2 \\
%   Affiliation / Address line 3 \\
%   \texttt{email@domain} \\}
% \author{Haohan Yuan \qquad Haopeng Zhang\thanks{corresponding author} \\ 
%   ALOHA Lab, University of Hawaii at Manoa \\
%   % Affiliation / Address line 2 \\
%   % Affiliation / Address line 3 \\
%   \texttt{\{haohany,haopengz\}@hawaii.edu}}
  
\author{
{Haopeng Zhang$\dag$\thanks{These authors contributed equally to this work.}, Yili Ren$\ddagger$\footnotemark[1], Haohan Yuan$\dag$, Jingzhe Zhang$\ddagger$, Yitong Shen$\ddagger$} \\
ALOHA Lab, University of Hawaii at Manoa$\dag$, University of South Florida$\ddagger$ \\
\{haopengz, haohany\}@hawaii.edu\\
\{yiliren, jingzhe, shen202\}@usf.edu\\}



  
%\author{
%  \textbf{First Author\textsuperscript{1}},
%  \textbf{Second Author\textsuperscript{1,2}},
%  \textbf{Third T. Author\textsuperscript{1}},
%  \textbf{Fourth Author\textsuperscript{1}},
%\\
%  \textbf{Fifth Author\textsuperscript{1,2}},
%  \textbf{Sixth Author\textsuperscript{1}},
%  \textbf{Seventh Author\textsuperscript{1}},
%  \textbf{Eighth Author \textsuperscript{1,2,3,4}},
%\\
%  \textbf{Ninth Author\textsuperscript{1}},
%  \textbf{Tenth Author\textsuperscript{1}},
%  \textbf{Eleventh E. Author\textsuperscript{1,2,3,4,5}},
%  \textbf{Twelfth Author\textsuperscript{1}},
%\\
%  \textbf{Thirteenth Author\textsuperscript{3}},
%  \textbf{Fourteenth F. Author\textsuperscript{2,4}},
%  \textbf{Fifteenth Author\textsuperscript{1}},
%  \textbf{Sixteenth Author\textsuperscript{1}},
%\\
%  \textbf{Seventeenth S. Author\textsuperscript{4,5}},
%  \textbf{Eighteenth Author\textsuperscript{3,4}},
%  \textbf{Nineteenth N. Author\textsuperscript{2,5}},
%  \textbf{Twentieth Author\textsuperscript{1}}
%\\
%\\
%  \textsuperscript{1}Affiliation 1,
%  \textsuperscript{2}Affiliation 2,
%  \textsuperscript{3}Affiliation 3,
%  \textsuperscript{4}Affiliation 4,
%  \textsuperscript{5}Affiliation 5
%\\
%  \small{
%    \textbf{Correspondence:} \href{mailto:email@domain}{email@domain}
%  }
%}

\begin{document}
\maketitle
\begin{abstract}
Recent advancements in Large Language Models (LLMs) have demonstrated remarkable capabilities across diverse tasks. However, their potential to integrate physical model knowledge for real-world signal interpretation remains largely unexplored. In this work, we introduce Wi-Chat, the first LLM-powered Wi-Fi-based human activity recognition system. We demonstrate that LLMs can process raw Wi-Fi signals and infer human activities by incorporating Wi-Fi sensing principles into prompts. Our approach leverages physical model insights to guide LLMs in interpreting Channel State Information (CSI) data without traditional signal processing techniques. Through experiments on real-world Wi-Fi datasets, we show that LLMs exhibit strong reasoning capabilities, achieving zero-shot activity recognition. These findings highlight a new paradigm for Wi-Fi sensing, expanding LLM applications beyond conventional language tasks and enhancing the accessibility of wireless sensing for real-world deployments.
\end{abstract}

\section{Introduction}

In today’s rapidly evolving digital landscape, the transformative power of web technologies has redefined not only how services are delivered but also how complex tasks are approached. Web-based systems have become increasingly prevalent in risk control across various domains. This widespread adoption is due their accessibility, scalability, and ability to remotely connect various types of users. For example, these systems are used for process safety management in industry~\cite{kannan2016web}, safety risk early warning in urban construction~\cite{ding2013development}, and safe monitoring of infrastructural systems~\cite{repetto2018web}. Within these web-based risk management systems, the source search problem presents a huge challenge. Source search refers to the task of identifying the origin of a risky event, such as a gas leak and the emission point of toxic substances. This source search capability is crucial for effective risk management and decision-making.

Traditional approaches to implementing source search capabilities into the web systems often rely on solely algorithmic solutions~\cite{ristic2016study}. These methods, while relatively straightforward to implement, often struggle to achieve acceptable performances due to algorithmic local optima and complex unknown environments~\cite{zhao2020searching}. More recently, web crowdsourcing has emerged as a promising alternative for tackling the source search problem by incorporating human efforts in these web systems on-the-fly~\cite{zhao2024user}. This approach outsources the task of addressing issues encountered during the source search process to human workers, leveraging their capabilities to enhance system performance.

These solutions often employ a human-AI collaborative way~\cite{zhao2023leveraging} where algorithms handle exploration-exploitation and report the encountered problems while human workers resolve complex decision-making bottlenecks to help the algorithms getting rid of local deadlocks~\cite{zhao2022crowd}. Although effective, this paradigm suffers from two inherent limitations: increased operational costs from continuous human intervention, and slow response times of human workers due to sequential decision-making. These challenges motivate our investigation into developing autonomous systems that preserve human-like reasoning capabilities while reducing dependency on massive crowdsourced labor.

Furthermore, recent advancements in large language models (LLMs)~\cite{chang2024survey} and multi-modal LLMs (MLLMs)~\cite{huang2023chatgpt} have unveiled promising avenues for addressing these challenges. One clear opportunity involves the seamless integration of visual understanding and linguistic reasoning for robust decision-making in search tasks. However, whether large models-assisted source search is really effective and efficient for improving the current source search algorithms~\cite{ji2022source} remains unknown. \textit{To address the research gap, we are particularly interested in answering the following two research questions in this work:}

\textbf{\textit{RQ1: }}How can source search capabilities be integrated into web-based systems to support decision-making in time-sensitive risk management scenarios? 
% \sq{I mention ``time-sensitive'' here because I feel like we shall say something about the response time -- LLM has to be faster than humans}

\textbf{\textit{RQ2: }}How can MLLMs and LLMs enhance the effectiveness and efficiency of existing source search algorithms? 

% \textit{\textbf{RQ2:}} To what extent does the performance of large models-assisted search align with or approach the effectiveness of human-AI collaborative search? 

To answer the research questions, we propose a novel framework called Auto-\
S$^2$earch (\textbf{Auto}nomous \textbf{S}ource \textbf{Search}) and implement a prototype system that leverages advanced web technologies to simulate real-world conditions for zero-shot source search. Unlike traditional methods that rely on pre-defined heuristics or extensive human intervention, AutoS$^2$earch employs a carefully designed prompt that encapsulates human rationales, thereby guiding the MLLM to generate coherent and accurate scene descriptions from visual inputs about four directional choices. Based on these language-based descriptions, the LLM is enabled to determine the optimal directional choice through chain-of-thought (CoT) reasoning. Comprehensive empirical validation demonstrates that AutoS$^2$-\ 
earch achieves a success rate of 95–98\%, closely approaching the performance of human-AI collaborative search across 20 benchmark scenarios~\cite{zhao2023leveraging}. 

Our work indicates that the role of humans in future web crowdsourcing tasks may evolve from executors to validators or supervisors. Furthermore, incorporating explanations of LLM decisions into web-based system interfaces has the potential to help humans enhance task performance in risk control.






\section{Related Work}
\label{sec:relatedworks}

% \begin{table*}[t]
% \centering 
% \renewcommand\arraystretch{0.98}
% \fontsize{8}{10}\selectfont \setlength{\tabcolsep}{0.4em}
% \begin{tabular}{@{}lc|cc|cc|cc@{}}
% \toprule
% \textbf{Methods}           & \begin{tabular}[c]{@{}c@{}}\textbf{Training}\\ \textbf{Paradigm}\end{tabular} & \begin{tabular}[c]{@{}c@{}}\textbf{$\#$ PT Data}\\ \textbf{(Tokens)}\end{tabular} & \begin{tabular}[c]{@{}c@{}}\textbf{$\#$ IFT Data}\\ \textbf{(Samples)}\end{tabular} & \textbf{Code}  & \begin{tabular}[c]{@{}c@{}}\textbf{Natural}\\ \textbf{Language}\end{tabular} & \begin{tabular}[c]{@{}c@{}}\textbf{Action}\\ \textbf{Trajectories}\end{tabular} & \begin{tabular}[c]{@{}c@{}}\textbf{API}\\ \textbf{Documentation}\end{tabular}\\ \midrule 
% NexusRaven~\citep{srinivasan2023nexusraven} & IFT & - & - & \textcolor{green}{\CheckmarkBold} & \textcolor{green}{\CheckmarkBold} &\textcolor{red}{\XSolidBrush}&\textcolor{red}{\XSolidBrush}\\
% AgentInstruct~\citep{zeng2023agenttuning} & IFT & - & 2k & \textcolor{green}{\CheckmarkBold} & \textcolor{green}{\CheckmarkBold} &\textcolor{red}{\XSolidBrush}&\textcolor{red}{\XSolidBrush} \\
% AgentEvol~\citep{xi2024agentgym} & IFT & - & 14.5k & \textcolor{green}{\CheckmarkBold} & \textcolor{green}{\CheckmarkBold} &\textcolor{green}{\CheckmarkBold}&\textcolor{red}{\XSolidBrush} \\
% Gorilla~\citep{patil2023gorilla}& IFT & - & 16k & \textcolor{green}{\CheckmarkBold} & \textcolor{green}{\CheckmarkBold} &\textcolor{red}{\XSolidBrush}&\textcolor{green}{\CheckmarkBold}\\
% OpenFunctions-v2~\citep{patil2023gorilla} & IFT & - & 65k & \textcolor{green}{\CheckmarkBold} & \textcolor{green}{\CheckmarkBold} &\textcolor{red}{\XSolidBrush}&\textcolor{green}{\CheckmarkBold}\\
% LAM~\citep{zhang2024agentohana} & IFT & - & 42.6k & \textcolor{green}{\CheckmarkBold} & \textcolor{green}{\CheckmarkBold} &\textcolor{green}{\CheckmarkBold}&\textcolor{red}{\XSolidBrush} \\
% xLAM~\citep{liu2024apigen} & IFT & - & 60k & \textcolor{green}{\CheckmarkBold} & \textcolor{green}{\CheckmarkBold} &\textcolor{green}{\CheckmarkBold}&\textcolor{red}{\XSolidBrush} \\\midrule
% LEMUR~\citep{xu2024lemur} & PT & 90B & 300k & \textcolor{green}{\CheckmarkBold} & \textcolor{green}{\CheckmarkBold} &\textcolor{green}{\CheckmarkBold}&\textcolor{red}{\XSolidBrush}\\
% \rowcolor{teal!12} \method & PT & 103B & 95k & \textcolor{green}{\CheckmarkBold} & \textcolor{green}{\CheckmarkBold} & \textcolor{green}{\CheckmarkBold} & \textcolor{green}{\CheckmarkBold} \\
% \bottomrule
% \end{tabular}
% \caption{Summary of existing tuning- and pretraining-based LLM agents with their training sample sizes. "PT" and "IFT" denote "Pre-Training" and "Instruction Fine-Tuning", respectively. }
% \label{tab:related}
% \end{table*}

\begin{table*}[ht]
\begin{threeparttable}
\centering 
\renewcommand\arraystretch{0.98}
\fontsize{7}{9}\selectfont \setlength{\tabcolsep}{0.2em}
\begin{tabular}{@{}l|c|c|ccc|cc|cc|cccc@{}}
\toprule
\textbf{Methods} & \textbf{Datasets}           & \begin{tabular}[c]{@{}c@{}}\textbf{Training}\\ \textbf{Paradigm}\end{tabular} & \begin{tabular}[c]{@{}c@{}}\textbf{\# PT Data}\\ \textbf{(Tokens)}\end{tabular} & \begin{tabular}[c]{@{}c@{}}\textbf{\# IFT Data}\\ \textbf{(Samples)}\end{tabular} & \textbf{\# APIs} & \textbf{Code}  & \begin{tabular}[c]{@{}c@{}}\textbf{Nat.}\\ \textbf{Lang.}\end{tabular} & \begin{tabular}[c]{@{}c@{}}\textbf{Action}\\ \textbf{Traj.}\end{tabular} & \begin{tabular}[c]{@{}c@{}}\textbf{API}\\ \textbf{Doc.}\end{tabular} & \begin{tabular}[c]{@{}c@{}}\textbf{Func.}\\ \textbf{Call}\end{tabular} & \begin{tabular}[c]{@{}c@{}}\textbf{Multi.}\\ \textbf{Step}\end{tabular}  & \begin{tabular}[c]{@{}c@{}}\textbf{Plan}\\ \textbf{Refine}\end{tabular}  & \begin{tabular}[c]{@{}c@{}}\textbf{Multi.}\\ \textbf{Turn}\end{tabular}\\ \midrule 
\multicolumn{13}{l}{\emph{Instruction Finetuning-based LLM Agents for Intrinsic Reasoning}}  \\ \midrule
FireAct~\cite{chen2023fireact} & FireAct & IFT & - & 2.1K & 10 & \textcolor{red}{\XSolidBrush} &\textcolor{green}{\CheckmarkBold} &\textcolor{green}{\CheckmarkBold}  & \textcolor{red}{\XSolidBrush} &\textcolor{green}{\CheckmarkBold} & \textcolor{red}{\XSolidBrush} &\textcolor{green}{\CheckmarkBold} & \textcolor{red}{\XSolidBrush} \\
ToolAlpaca~\cite{tang2023toolalpaca} & ToolAlpaca & IFT & - & 4.0K & 400 & \textcolor{red}{\XSolidBrush} &\textcolor{green}{\CheckmarkBold} &\textcolor{green}{\CheckmarkBold} & \textcolor{red}{\XSolidBrush} &\textcolor{green}{\CheckmarkBold} & \textcolor{red}{\XSolidBrush}  &\textcolor{green}{\CheckmarkBold} & \textcolor{red}{\XSolidBrush}  \\
ToolLLaMA~\cite{qin2023toolllm} & ToolBench & IFT & - & 12.7K & 16,464 & \textcolor{red}{\XSolidBrush} &\textcolor{green}{\CheckmarkBold} &\textcolor{green}{\CheckmarkBold} &\textcolor{red}{\XSolidBrush} &\textcolor{green}{\CheckmarkBold}&\textcolor{green}{\CheckmarkBold}&\textcolor{green}{\CheckmarkBold} &\textcolor{green}{\CheckmarkBold}\\
AgentEvol~\citep{xi2024agentgym} & AgentTraj-L & IFT & - & 14.5K & 24 &\textcolor{red}{\XSolidBrush} & \textcolor{green}{\CheckmarkBold} &\textcolor{green}{\CheckmarkBold}&\textcolor{red}{\XSolidBrush} &\textcolor{green}{\CheckmarkBold}&\textcolor{red}{\XSolidBrush} &\textcolor{red}{\XSolidBrush} &\textcolor{green}{\CheckmarkBold}\\
Lumos~\cite{yin2024agent} & Lumos & IFT  & - & 20.0K & 16 &\textcolor{red}{\XSolidBrush} & \textcolor{green}{\CheckmarkBold} & \textcolor{green}{\CheckmarkBold} &\textcolor{red}{\XSolidBrush} & \textcolor{green}{\CheckmarkBold} & \textcolor{green}{\CheckmarkBold} &\textcolor{red}{\XSolidBrush} & \textcolor{green}{\CheckmarkBold}\\
Agent-FLAN~\cite{chen2024agent} & Agent-FLAN & IFT & - & 24.7K & 20 &\textcolor{red}{\XSolidBrush} & \textcolor{green}{\CheckmarkBold} & \textcolor{green}{\CheckmarkBold} &\textcolor{red}{\XSolidBrush} & \textcolor{green}{\CheckmarkBold}& \textcolor{green}{\CheckmarkBold}&\textcolor{red}{\XSolidBrush} & \textcolor{green}{\CheckmarkBold}\\
AgentTuning~\citep{zeng2023agenttuning} & AgentInstruct & IFT & - & 35.0K & - &\textcolor{red}{\XSolidBrush} & \textcolor{green}{\CheckmarkBold} & \textcolor{green}{\CheckmarkBold} &\textcolor{red}{\XSolidBrush} & \textcolor{green}{\CheckmarkBold} &\textcolor{red}{\XSolidBrush} &\textcolor{red}{\XSolidBrush} & \textcolor{green}{\CheckmarkBold}\\\midrule
\multicolumn{13}{l}{\emph{Instruction Finetuning-based LLM Agents for Function Calling}} \\\midrule
NexusRaven~\citep{srinivasan2023nexusraven} & NexusRaven & IFT & - & - & 116 & \textcolor{green}{\CheckmarkBold} & \textcolor{green}{\CheckmarkBold}  & \textcolor{green}{\CheckmarkBold} &\textcolor{red}{\XSolidBrush} & \textcolor{green}{\CheckmarkBold} &\textcolor{red}{\XSolidBrush} &\textcolor{red}{\XSolidBrush}&\textcolor{red}{\XSolidBrush}\\
Gorilla~\citep{patil2023gorilla} & Gorilla & IFT & - & 16.0K & 1,645 & \textcolor{green}{\CheckmarkBold} &\textcolor{red}{\XSolidBrush} &\textcolor{red}{\XSolidBrush}&\textcolor{green}{\CheckmarkBold} &\textcolor{green}{\CheckmarkBold} &\textcolor{red}{\XSolidBrush} &\textcolor{red}{\XSolidBrush} &\textcolor{red}{\XSolidBrush}\\
OpenFunctions-v2~\citep{patil2023gorilla} & OpenFunctions-v2 & IFT & - & 65.0K & - & \textcolor{green}{\CheckmarkBold} & \textcolor{green}{\CheckmarkBold} &\textcolor{red}{\XSolidBrush} &\textcolor{green}{\CheckmarkBold} &\textcolor{green}{\CheckmarkBold} &\textcolor{red}{\XSolidBrush} &\textcolor{red}{\XSolidBrush} &\textcolor{red}{\XSolidBrush}\\
API Pack~\cite{guo2024api} & API Pack & IFT & - & 1.1M & 11,213 &\textcolor{green}{\CheckmarkBold} &\textcolor{red}{\XSolidBrush} &\textcolor{green}{\CheckmarkBold} &\textcolor{red}{\XSolidBrush} &\textcolor{green}{\CheckmarkBold} &\textcolor{red}{\XSolidBrush}&\textcolor{red}{\XSolidBrush}&\textcolor{red}{\XSolidBrush}\\ 
LAM~\citep{zhang2024agentohana} & AgentOhana & IFT & - & 42.6K & - & \textcolor{green}{\CheckmarkBold} & \textcolor{green}{\CheckmarkBold} &\textcolor{green}{\CheckmarkBold}&\textcolor{red}{\XSolidBrush} &\textcolor{green}{\CheckmarkBold}&\textcolor{red}{\XSolidBrush}&\textcolor{green}{\CheckmarkBold}&\textcolor{green}{\CheckmarkBold}\\
xLAM~\citep{liu2024apigen} & APIGen & IFT & - & 60.0K & 3,673 & \textcolor{green}{\CheckmarkBold} & \textcolor{green}{\CheckmarkBold} &\textcolor{green}{\CheckmarkBold}&\textcolor{red}{\XSolidBrush} &\textcolor{green}{\CheckmarkBold}&\textcolor{red}{\XSolidBrush}&\textcolor{green}{\CheckmarkBold}&\textcolor{green}{\CheckmarkBold}\\\midrule
\multicolumn{13}{l}{\emph{Pretraining-based LLM Agents}}  \\\midrule
% LEMUR~\citep{xu2024lemur} & PT & 90B & 300.0K & - & \textcolor{green}{\CheckmarkBold} & \textcolor{green}{\CheckmarkBold} &\textcolor{green}{\CheckmarkBold}&\textcolor{red}{\XSolidBrush} & \textcolor{red}{\XSolidBrush} &\textcolor{green}{\CheckmarkBold} &\textcolor{red}{\XSolidBrush}&\textcolor{red}{\XSolidBrush}\\
\rowcolor{teal!12} \method & \dataset & PT & 103B & 95.0K  & 76,537  & \textcolor{green}{\CheckmarkBold} & \textcolor{green}{\CheckmarkBold} & \textcolor{green}{\CheckmarkBold} & \textcolor{green}{\CheckmarkBold} & \textcolor{green}{\CheckmarkBold} & \textcolor{green}{\CheckmarkBold} & \textcolor{green}{\CheckmarkBold} & \textcolor{green}{\CheckmarkBold}\\
\bottomrule
\end{tabular}
% \begin{tablenotes}
%     \item $^*$ In addition, the StarCoder-API can offer 4.77M more APIs.
% \end{tablenotes}
\caption{Summary of existing instruction finetuning-based LLM agents for intrinsic reasoning and function calling, along with their training resources and sample sizes. "PT" and "IFT" denote "Pre-Training" and "Instruction Fine-Tuning", respectively.}
\vspace{-2ex}
\label{tab:related}
\end{threeparttable}
\end{table*}

\noindent \textbf{Prompting-based LLM Agents.} Due to the lack of agent-specific pre-training corpus, existing LLM agents rely on either prompt engineering~\cite{hsieh2023tool,lu2024chameleon,yao2022react,wang2023voyager} or instruction fine-tuning~\cite{chen2023fireact,zeng2023agenttuning} to understand human instructions, decompose high-level tasks, generate grounded plans, and execute multi-step actions. 
However, prompting-based methods mainly depend on the capabilities of backbone LLMs (usually commercial LLMs), failing to introduce new knowledge and struggling to generalize to unseen tasks~\cite{sun2024adaplanner,zhuang2023toolchain}. 

\noindent \textbf{Instruction Finetuning-based LLM Agents.} Considering the extensive diversity of APIs and the complexity of multi-tool instructions, tool learning inherently presents greater challenges than natural language tasks, such as text generation~\cite{qin2023toolllm}.
Post-training techniques focus more on instruction following and aligning output with specific formats~\cite{patil2023gorilla,hao2024toolkengpt,qin2023toolllm,schick2024toolformer}, rather than fundamentally improving model knowledge or capabilities. 
Moreover, heavy fine-tuning can hinder generalization or even degrade performance in non-agent use cases, potentially suppressing the original base model capabilities~\cite{ghosh2024a}.

\noindent \textbf{Pretraining-based LLM Agents.} While pre-training serves as an essential alternative, prior works~\cite{nijkamp2023codegen,roziere2023code,xu2024lemur,patil2023gorilla} have primarily focused on improving task-specific capabilities (\eg, code generation) instead of general-domain LLM agents, due to single-source, uni-type, small-scale, and poor-quality pre-training data. 
Existing tool documentation data for agent training either lacks diverse real-world APIs~\cite{patil2023gorilla, tang2023toolalpaca} or is constrained to single-tool or single-round tool execution. 
Furthermore, trajectory data mostly imitate expert behavior or follow function-calling rules with inferior planning and reasoning, failing to fully elicit LLMs' capabilities and handle complex instructions~\cite{qin2023toolllm}. 
Given a wide range of candidate API functions, each comprising various function names and parameters available at every planning step, identifying globally optimal solutions and generalizing across tasks remains highly challenging.



\section{Preliminaries}
\label{Preliminaries}
\begin{figure*}[t]
    \centering
    \includegraphics[width=0.95\linewidth]{fig/HealthGPT_Framework.png}
    \caption{The \ourmethod{} architecture integrates hierarchical visual perception and H-LoRA, employing a task-specific hard router to select visual features and H-LoRA plugins, ultimately generating outputs with an autoregressive manner.}
    \label{fig:architecture}
\end{figure*}
\noindent\textbf{Large Vision-Language Models.} 
The input to a LVLM typically consists of an image $x^{\text{img}}$ and a discrete text sequence $x^{\text{txt}}$. The visual encoder $\mathcal{E}^{\text{img}}$ converts the input image $x^{\text{img}}$ into a sequence of visual tokens $\mathcal{V} = [v_i]_{i=1}^{N_v}$, while the text sequence $x^{\text{txt}}$ is mapped into a sequence of text tokens $\mathcal{T} = [t_i]_{i=1}^{N_t}$ using an embedding function $\mathcal{E}^{\text{txt}}$. The LLM $\mathcal{M_\text{LLM}}(\cdot|\theta)$ models the joint probability of the token sequence $\mathcal{U} = \{\mathcal{V},\mathcal{T}\}$, which is expressed as:
\begin{equation}
    P_\theta(R | \mathcal{U}) = \prod_{i=1}^{N_r} P_\theta(r_i | \{\mathcal{U}, r_{<i}\}),
\end{equation}
where $R = [r_i]_{i=1}^{N_r}$ is the text response sequence. The LVLM iteratively generates the next token $r_i$ based on $r_{<i}$. The optimization objective is to minimize the cross-entropy loss of the response $\mathcal{R}$.
% \begin{equation}
%     \mathcal{L}_{\text{VLM}} = \mathbb{E}_{R|\mathcal{U}}\left[-\log P_\theta(R | \mathcal{U})\right]
% \end{equation}
It is worth noting that most LVLMs adopt a design paradigm based on ViT, alignment adapters, and pre-trained LLMs\cite{liu2023llava,liu2024improved}, enabling quick adaptation to downstream tasks.


\noindent\textbf{VQGAN.}
VQGAN~\cite{esser2021taming} employs latent space compression and indexing mechanisms to effectively learn a complete discrete representation of images. VQGAN first maps the input image $x^{\text{img}}$ to a latent representation $z = \mathcal{E}(x)$ through a encoder $\mathcal{E}$. Then, the latent representation is quantized using a codebook $\mathcal{Z} = \{z_k\}_{k=1}^K$, generating a discrete index sequence $\mathcal{I} = [i_m]_{m=1}^N$, where $i_m \in \mathcal{Z}$ represents the quantized code index:
\begin{equation}
    \mathcal{I} = \text{Quantize}(z|\mathcal{Z}) = \arg\min_{z_k \in \mathcal{Z}} \| z - z_k \|_2.
\end{equation}
In our approach, the discrete index sequence $\mathcal{I}$ serves as a supervisory signal for the generation task, enabling the model to predict the index sequence $\hat{\mathcal{I}}$ from input conditions such as text or other modality signals.  
Finally, the predicted index sequence $\hat{\mathcal{I}}$ is upsampled by the VQGAN decoder $G$, generating the high-quality image $\hat{x}^\text{img} = G(\hat{\mathcal{I}})$.



\noindent\textbf{Low Rank Adaptation.} 
LoRA\cite{hu2021lora} effectively captures the characteristics of downstream tasks by introducing low-rank adapters. The core idea is to decompose the bypass weight matrix $\Delta W\in\mathbb{R}^{d^{\text{in}} \times d^{\text{out}}}$ into two low-rank matrices $ \{A \in \mathbb{R}^{d^{\text{in}} \times r}, B \in \mathbb{R}^{r \times d^{\text{out}}} \}$, where $ r \ll \min\{d^{\text{in}}, d^{\text{out}}\} $, significantly reducing learnable parameters. The output with the LoRA adapter for the input $x$ is then given by:
\begin{equation}
    h = x W_0 + \alpha x \Delta W/r = x W_0 + \alpha xAB/r,
\end{equation}
where matrix $ A $ is initialized with a Gaussian distribution, while the matrix $ B $ is initialized as a zero matrix. The scaling factor $ \alpha/r $ controls the impact of $ \Delta W $ on the model.

\section{HealthGPT}
\label{Method}


\subsection{Unified Autoregressive Generation.}  
% As shown in Figure~\ref{fig:architecture}, 
\ourmethod{} (Figure~\ref{fig:architecture}) utilizes a discrete token representation that covers both text and visual outputs, unifying visual comprehension and generation as an autoregressive task. 
For comprehension, $\mathcal{M}_\text{llm}$ receives the input joint sequence $\mathcal{U}$ and outputs a series of text token $\mathcal{R} = [r_1, r_2, \dots, r_{N_r}]$, where $r_i \in \mathcal{V}_{\text{txt}}$, and $\mathcal{V}_{\text{txt}}$ represents the LLM's vocabulary:
\begin{equation}
    P_\theta(\mathcal{R} \mid \mathcal{U}) = \prod_{i=1}^{N_r} P_\theta(r_i \mid \mathcal{U}, r_{<i}).
\end{equation}
For generation, $\mathcal{M}_\text{llm}$ first receives a special start token $\langle \text{START\_IMG} \rangle$, then generates a series of tokens corresponding to the VQGAN indices $\mathcal{I} = [i_1, i_2, \dots, i_{N_i}]$, where $i_j \in \mathcal{V}_{\text{vq}}$, and $\mathcal{V}_{\text{vq}}$ represents the index range of VQGAN. Upon completion of generation, the LLM outputs an end token $\langle \text{END\_IMG} \rangle$:
\begin{equation}
    P_\theta(\mathcal{I} \mid \mathcal{U}) = \prod_{j=1}^{N_i} P_\theta(i_j \mid \mathcal{U}, i_{<j}).
\end{equation}
Finally, the generated index sequence $\mathcal{I}$ is fed into the decoder $G$, which reconstructs the target image $\hat{x}^{\text{img}} = G(\mathcal{I})$.

\subsection{Hierarchical Visual Perception}  
Given the differences in visual perception between comprehension and generation tasks—where the former focuses on abstract semantics and the latter emphasizes complete semantics—we employ ViT to compress the image into discrete visual tokens at multiple hierarchical levels.
Specifically, the image is converted into a series of features $\{f_1, f_2, \dots, f_L\}$ as it passes through $L$ ViT blocks.

To address the needs of various tasks, the hidden states are divided into two types: (i) \textit{Concrete-grained features} $\mathcal{F}^{\text{Con}} = \{f_1, f_2, \dots, f_k\}, k < L$, derived from the shallower layers of ViT, containing sufficient global features, suitable for generation tasks; 
(ii) \textit{Abstract-grained features} $\mathcal{F}^{\text{Abs}} = \{f_{k+1}, f_{k+2}, \dots, f_L\}$, derived from the deeper layers of ViT, which contain abstract semantic information closer to the text space, suitable for comprehension tasks.

The task type $T$ (comprehension or generation) determines which set of features is selected as the input for the downstream large language model:
\begin{equation}
    \mathcal{F}^{\text{img}}_T =
    \begin{cases}
        \mathcal{F}^{\text{Con}}, & \text{if } T = \text{generation task} \\
        \mathcal{F}^{\text{Abs}}, & \text{if } T = \text{comprehension task}
    \end{cases}
\end{equation}
We integrate the image features $\mathcal{F}^{\text{img}}_T$ and text features $\mathcal{T}$ into a joint sequence through simple concatenation, which is then fed into the LLM $\mathcal{M}_{\text{llm}}$ for autoregressive generation.
% :
% \begin{equation}
%     \mathcal{R} = \mathcal{M}_{\text{llm}}(\mathcal{U}|\theta), \quad \mathcal{U} = [\mathcal{F}^{\text{img}}_T; \mathcal{T}]
% \end{equation}
\subsection{Heterogeneous Knowledge Adaptation}
We devise H-LoRA, which stores heterogeneous knowledge from comprehension and generation tasks in separate modules and dynamically routes to extract task-relevant knowledge from these modules. 
At the task level, for each task type $ T $, we dynamically assign a dedicated H-LoRA submodule $ \theta^T $, which is expressed as:
\begin{equation}
    \mathcal{R} = \mathcal{M}_\text{LLM}(\mathcal{U}|\theta, \theta^T), \quad \theta^T = \{A^T, B^T, \mathcal{R}^T_\text{outer}\}.
\end{equation}
At the feature level for a single task, H-LoRA integrates the idea of Mixture of Experts (MoE)~\cite{masoudnia2014mixture} and designs an efficient matrix merging and routing weight allocation mechanism, thus avoiding the significant computational delay introduced by matrix splitting in existing MoELoRA~\cite{luo2024moelora}. Specifically, we first merge the low-rank matrices (rank = r) of $ k $ LoRA experts into a unified matrix:
\begin{equation}
    \mathbf{A}^{\text{merged}}, \mathbf{B}^{\text{merged}} = \text{Concat}(\{A_i\}_1^k), \text{Concat}(\{B_i\}_1^k),
\end{equation}
where $ \mathbf{A}^{\text{merged}} \in \mathbb{R}^{d^\text{in} \times rk} $ and $ \mathbf{B}^{\text{merged}} \in \mathbb{R}^{rk \times d^\text{out}} $. The $k$-dimension routing layer generates expert weights $ \mathcal{W} \in \mathbb{R}^{\text{token\_num} \times k} $ based on the input hidden state $ x $, and these are expanded to $ \mathbb{R}^{\text{token\_num} \times rk} $ as follows:
\begin{equation}
    \mathcal{W}^\text{expanded} = \alpha k \mathcal{W} / r \otimes \mathbf{1}_r,
\end{equation}
where $ \otimes $ denotes the replication operation.
The overall output of H-LoRA is computed as:
\begin{equation}
    \mathcal{O}^\text{H-LoRA} = (x \mathbf{A}^{\text{merged}} \odot \mathcal{W}^\text{expanded}) \mathbf{B}^{\text{merged}},
\end{equation}
where $ \odot $ represents element-wise multiplication. Finally, the output of H-LoRA is added to the frozen pre-trained weights to produce the final output:
\begin{equation}
    \mathcal{O} = x W_0 + \mathcal{O}^\text{H-LoRA}.
\end{equation}
% In summary, H-LoRA is a task-based dynamic PEFT method that achieves high efficiency in single-task fine-tuning.

\subsection{Training Pipeline}

\begin{figure}[t]
    \centering
    \hspace{-4mm}
    \includegraphics[width=0.94\linewidth]{fig/data.pdf}
    \caption{Data statistics of \texttt{VL-Health}. }
    \label{fig:data}
\end{figure}
\noindent \textbf{1st Stage: Multi-modal Alignment.} 
In the first stage, we design separate visual adapters and H-LoRA submodules for medical unified tasks. For the medical comprehension task, we train abstract-grained visual adapters using high-quality image-text pairs to align visual embeddings with textual embeddings, thereby enabling the model to accurately describe medical visual content. During this process, the pre-trained LLM and its corresponding H-LoRA submodules remain frozen. In contrast, the medical generation task requires training concrete-grained adapters and H-LoRA submodules while keeping the LLM frozen. Meanwhile, we extend the textual vocabulary to include multimodal tokens, enabling the support of additional VQGAN vector quantization indices. The model trains on image-VQ pairs, endowing the pre-trained LLM with the capability for image reconstruction. This design ensures pixel-level consistency of pre- and post-LVLM. The processes establish the initial alignment between the LLM’s outputs and the visual inputs.

\noindent \textbf{2nd Stage: Heterogeneous H-LoRA Plugin Adaptation.}  
The submodules of H-LoRA share the word embedding layer and output head but may encounter issues such as bias and scale inconsistencies during training across different tasks. To ensure that the multiple H-LoRA plugins seamlessly interface with the LLMs and form a unified base, we fine-tune the word embedding layer and output head using a small amount of mixed data to maintain consistency in the model weights. Specifically, during this stage, all H-LoRA submodules for different tasks are kept frozen, with only the word embedding layer and output head being optimized. Through this stage, the model accumulates foundational knowledge for unified tasks by adapting H-LoRA plugins.

\begin{table*}[!t]
\centering
\caption{Comparison of \ourmethod{} with other LVLMs and unified multi-modal models on medical visual comprehension tasks. \textbf{Bold} and \underline{underlined} text indicates the best performance and second-best performance, respectively.}
\resizebox{\textwidth}{!}{
\begin{tabular}{c|lcc|cccccccc|c}
\toprule
\rowcolor[HTML]{E9F3FE} &  &  &  & \multicolumn{2}{c}{\textbf{VQA-RAD \textuparrow}} & \multicolumn{2}{c}{\textbf{SLAKE \textuparrow}} & \multicolumn{2}{c}{\textbf{PathVQA \textuparrow}} &  &  &  \\ 
\cline{5-10}
\rowcolor[HTML]{E9F3FE}\multirow{-2}{*}{\textbf{Type}} & \multirow{-2}{*}{\textbf{Model}} & \multirow{-2}{*}{\textbf{\# Params}} & \multirow{-2}{*}{\makecell{\textbf{Medical} \\ \textbf{LVLM}}} & \textbf{close} & \textbf{all} & \textbf{close} & \textbf{all} & \textbf{close} & \textbf{all} & \multirow{-2}{*}{\makecell{\textbf{MMMU} \\ \textbf{-Med}}\textuparrow} & \multirow{-2}{*}{\textbf{OMVQA}\textuparrow} & \multirow{-2}{*}{\textbf{Avg. \textuparrow}} \\ 
\midrule \midrule
\multirow{9}{*}{\textbf{Comp. Only}} 
& Med-Flamingo & 8.3B & \Large \ding{51} & 58.6 & 43.0 & 47.0 & 25.5 & 61.9 & 31.3 & 28.7 & 34.9 & 41.4 \\
& LLaVA-Med & 7B & \Large \ding{51} & 60.2 & 48.1 & 58.4 & 44.8 & 62.3 & 35.7 & 30.0 & 41.3 & 47.6 \\
& HuatuoGPT-Vision & 7B & \Large \ding{51} & 66.9 & 53.0 & 59.8 & 49.1 & 52.9 & 32.0 & 42.0 & 50.0 & 50.7 \\
& BLIP-2 & 6.7B & \Large \ding{55} & 43.4 & 36.8 & 41.6 & 35.3 & 48.5 & 28.8 & 27.3 & 26.9 & 36.1 \\
& LLaVA-v1.5 & 7B & \Large \ding{55} & 51.8 & 42.8 & 37.1 & 37.7 & 53.5 & 31.4 & 32.7 & 44.7 & 41.5 \\
& InstructBLIP & 7B & \Large \ding{55} & 61.0 & 44.8 & 66.8 & 43.3 & 56.0 & 32.3 & 25.3 & 29.0 & 44.8 \\
& Yi-VL & 6B & \Large \ding{55} & 52.6 & 42.1 & 52.4 & 38.4 & 54.9 & 30.9 & 38.0 & 50.2 & 44.9 \\
& InternVL2 & 8B & \Large \ding{55} & 64.9 & 49.0 & 66.6 & 50.1 & 60.0 & 31.9 & \underline{43.3} & 54.5 & 52.5\\
& Llama-3.2 & 11B & \Large \ding{55} & 68.9 & 45.5 & 72.4 & 52.1 & 62.8 & 33.6 & 39.3 & 63.2 & 54.7 \\
\midrule
\multirow{5}{*}{\textbf{Comp. \& Gen.}} 
& Show-o & 1.3B & \Large \ding{55} & 50.6 & 33.9 & 31.5 & 17.9 & 52.9 & 28.2 & 22.7 & 45.7 & 42.6 \\
& Unified-IO 2 & 7B & \Large \ding{55} & 46.2 & 32.6 & 35.9 & 21.9 & 52.5 & 27.0 & 25.3 & 33.0 & 33.8 \\
& Janus & 1.3B & \Large \ding{55} & 70.9 & 52.8 & 34.7 & 26.9 & 51.9 & 27.9 & 30.0 & 26.8 & 33.5 \\
& \cellcolor[HTML]{DAE0FB}HealthGPT-M3 & \cellcolor[HTML]{DAE0FB}3.8B & \cellcolor[HTML]{DAE0FB}\Large \ding{51} & \cellcolor[HTML]{DAE0FB}\underline{73.7} & \cellcolor[HTML]{DAE0FB}\underline{55.9} & \cellcolor[HTML]{DAE0FB}\underline{74.6} & \cellcolor[HTML]{DAE0FB}\underline{56.4} & \cellcolor[HTML]{DAE0FB}\underline{78.7} & \cellcolor[HTML]{DAE0FB}\underline{39.7} & \cellcolor[HTML]{DAE0FB}\underline{43.3} & \cellcolor[HTML]{DAE0FB}\underline{68.5} & \cellcolor[HTML]{DAE0FB}\underline{61.3} \\
& \cellcolor[HTML]{DAE0FB}HealthGPT-L14 & \cellcolor[HTML]{DAE0FB}14B & \cellcolor[HTML]{DAE0FB}\Large \ding{51} & \cellcolor[HTML]{DAE0FB}\textbf{77.7} & \cellcolor[HTML]{DAE0FB}\textbf{58.3} & \cellcolor[HTML]{DAE0FB}\textbf{76.4} & \cellcolor[HTML]{DAE0FB}\textbf{64.5} & \cellcolor[HTML]{DAE0FB}\textbf{85.9} & \cellcolor[HTML]{DAE0FB}\textbf{44.4} & \cellcolor[HTML]{DAE0FB}\textbf{49.2} & \cellcolor[HTML]{DAE0FB}\textbf{74.4} & \cellcolor[HTML]{DAE0FB}\textbf{66.4} \\
\bottomrule
\end{tabular}
}
\label{tab:results}
\end{table*}
\begin{table*}[ht]
    \centering
    \caption{The experimental results for the four modality conversion tasks.}
    \resizebox{\textwidth}{!}{
    \begin{tabular}{l|ccc|ccc|ccc|ccc}
        \toprule
        \rowcolor[HTML]{E9F3FE} & \multicolumn{3}{c}{\textbf{CT to MRI (Brain)}} & \multicolumn{3}{c}{\textbf{CT to MRI (Pelvis)}} & \multicolumn{3}{c}{\textbf{MRI to CT (Brain)}} & \multicolumn{3}{c}{\textbf{MRI to CT (Pelvis)}} \\
        \cline{2-13}
        \rowcolor[HTML]{E9F3FE}\multirow{-2}{*}{\textbf{Model}}& \textbf{SSIM $\uparrow$} & \textbf{PSNR $\uparrow$} & \textbf{MSE $\downarrow$} & \textbf{SSIM $\uparrow$} & \textbf{PSNR $\uparrow$} & \textbf{MSE $\downarrow$} & \textbf{SSIM $\uparrow$} & \textbf{PSNR $\uparrow$} & \textbf{MSE $\downarrow$} & \textbf{SSIM $\uparrow$} & \textbf{PSNR $\uparrow$} & \textbf{MSE $\downarrow$} \\
        \midrule \midrule
        pix2pix & 71.09 & 32.65 & 36.85 & 59.17 & 31.02 & 51.91 & 78.79 & 33.85 & 28.33 & 72.31 & 32.98 & 36.19 \\
        CycleGAN & 54.76 & 32.23 & 40.56 & 54.54 & 30.77 & 55.00 & 63.75 & 31.02 & 52.78 & 50.54 & 29.89 & 67.78 \\
        BBDM & {71.69} & {32.91} & {34.44} & 57.37 & 31.37 & 48.06 & \textbf{86.40} & 34.12 & 26.61 & {79.26} & 33.15 & 33.60 \\
        Vmanba & 69.54 & 32.67 & 36.42 & {63.01} & {31.47} & {46.99} & 79.63 & 34.12 & 26.49 & 77.45 & 33.53 & 31.85 \\
        DiffMa & 71.47 & 32.74 & 35.77 & 62.56 & 31.43 & 47.38 & 79.00 & {34.13} & {26.45} & 78.53 & {33.68} & {30.51} \\
        \rowcolor[HTML]{DAE0FB}HealthGPT-M3 & \underline{79.38} & \underline{33.03} & \underline{33.48} & \underline{71.81} & \underline{31.83} & \underline{43.45} & {85.06} & \textbf{34.40} & \textbf{25.49} & \underline{84.23} & \textbf{34.29} & \textbf{27.99} \\
        \rowcolor[HTML]{DAE0FB}HealthGPT-L14 & \textbf{79.73} & \textbf{33.10} & \textbf{32.96} & \textbf{71.92} & \textbf{31.87} & \textbf{43.09} & \underline{85.31} & \underline{34.29} & \underline{26.20} & \textbf{84.96} & \underline{34.14} & \underline{28.13} \\
        \bottomrule
    \end{tabular}
    }
    \label{tab:conversion}
\end{table*}

\noindent \textbf{3rd Stage: Visual Instruction Fine-Tuning.}  
In the third stage, we introduce additional task-specific data to further optimize the model and enhance its adaptability to downstream tasks such as medical visual comprehension (e.g., medical QA, medical dialogues, and report generation) or generation tasks (e.g., super-resolution, denoising, and modality conversion). Notably, by this stage, the word embedding layer and output head have been fine-tuned, only the H-LoRA modules and adapter modules need to be trained. This strategy significantly improves the model's adaptability and flexibility across different tasks.


\section{Experiment}
\label{s:experiment}

\subsection{Data Description}
We evaluate our method on FI~\cite{you2016building}, Twitter\_LDL~\cite{yang2017learning} and Artphoto~\cite{machajdik2010affective}.
FI is a public dataset built from Flickr and Instagram, with 23,308 images and eight emotion categories, namely \textit{amusement}, \textit{anger}, \textit{awe},  \textit{contentment}, \textit{disgust}, \textit{excitement},  \textit{fear}, and \textit{sadness}. 
% Since images in FI are all copyrighted by law, some images are corrupted now, so we remove these samples and retain 21,828 images.
% T4SA contains images from Twitter, which are classified into three categories: \textit{positive}, \textit{neutral}, and \textit{negative}. In this paper, we adopt the base version of B-T4SA, which contains 470,586 images and provides text descriptions of the corresponding tweets.
Twitter\_LDL contains 10,045 images from Twitter, with the same eight categories as the FI dataset.
% 。
For these two datasets, they are randomly split into 80\%
training and 20\% testing set.
Artphoto contains 806 artistic photos from the DeviantArt website, which we use to further evaluate the zero-shot capability of our model.
% on the small-scale dataset.
% We construct and publicly release the first image sentiment analysis dataset containing metadata.
% 。

% Based on these datasets, we are the first to construct and publicly release metadata-enhanced image sentiment analysis datasets. These datasets include scenes, tags, descriptions, and corresponding confidence scores, and are available at this link for future research purposes.


% 
\begin{table}[t]
\centering
% \begin{center}
\caption{Overall performance of different models on FI and Twitter\_LDL datasets.}
\label{tab:cap1}
% \resizebox{\linewidth}{!}
{
\begin{tabular}{l|c|c|c|c}
\hline
\multirow{2}{*}{\textbf{Model}} & \multicolumn{2}{c|}{\textbf{FI}}  & \multicolumn{2}{c}{\textbf{Twitter\_LDL}} \\ \cline{2-5} 
  & \textbf{Accuracy} & \textbf{F1} & \textbf{Accuracy} & \textbf{F1}  \\ \hline
% (\rownumber)~AlexNet~\cite{krizhevsky2017imagenet}  & 58.13\% & 56.35\%  & 56.24\%& 55.02\%  \\ 
% (\rownumber)~VGG16~\cite{simonyan2014very}  & 63.75\%& 63.08\%  & 59.34\%& 59.02\%  \\ 
(\rownumber)~ResNet101~\cite{he2016deep} & 66.16\%& 65.56\%  & 62.02\% & 61.34\%  \\ 
(\rownumber)~CDA~\cite{han2023boosting} & 66.71\%& 65.37\%  & 64.14\% & 62.85\%  \\ 
(\rownumber)~CECCN~\cite{ruan2024color} & 67.96\%& 66.74\%  & 64.59\%& 64.72\% \\ 
(\rownumber)~EmoVIT~\cite{xie2024emovit} & 68.09\%& 67.45\%  & 63.12\% & 61.97\%  \\ 
(\rownumber)~ComLDL~\cite{zhang2022compound} & 68.83\%& 67.28\%  & 65.29\% & 63.12\%  \\ 
(\rownumber)~WSDEN~\cite{li2023weakly} & 69.78\%& 69.61\%  & 67.04\% & 65.49\% \\ 
(\rownumber)~ECWA~\cite{deng2021emotion} & 70.87\%& 69.08\%  & 67.81\% & 66.87\%  \\ 
(\rownumber)~EECon~\cite{yang2023exploiting} & 71.13\%& 68.34\%  & 64.27\%& 63.16\%  \\ 
(\rownumber)~MAM~\cite{zhang2024affective} & 71.44\%  & 70.83\% & 67.18\%  & 65.01\%\\ 
(\rownumber)~TGCA-PVT~\cite{chen2024tgca}   & 73.05\%  & 71.46\% & 69.87\%  & 68.32\% \\ 
(\rownumber)~OEAN~\cite{zhang2024object}   & 73.40\%  & 72.63\% & 70.52\%  & 69.47\% \\ \hline
(\rownumber)~\shortname  & \textbf{79.48\%} & \textbf{79.22\%} & \textbf{74.12\%} & \textbf{73.09\%} \\ \hline
\end{tabular}
}
\vspace{-6mm}
% \end{center}
\end{table}
% 

\subsection{Experiment Setting}
% \subsubsection{Model Setting.}
% 
\textbf{Model Setting:}
For feature representation, we set $k=10$ to select object tags, and adopt clip-vit-base-patch32 as the pre-trained model for unified feature representation.
Moreover, we empirically set $(d_e, d_h, d_k, d_s) = (512, 128, 16, 64)$, and set the classification class $L$ to 8.

% 

\textbf{Training Setting:}
To initialize the model, we set all weights such as $\boldsymbol{W}$ following the truncated normal distribution, and use AdamW optimizer with the learning rate of $1 \times 10^{-4}$.
% warmup scheduler of cosine, warmup steps of 2000.
Furthermore, we set the batch size to 32 and the epoch of the training process to 200.
During the implementation, we utilize \textit{PyTorch} to build our entire model.
% , and our project codes are publicly available at https://github.com/zzmyrep/MESN.
% Our project codes as well as data are all publicly available on GitHub\footnote{https://github.com/zzmyrep/KBCEN}.
% Code is available at \href{https://github.com/zzmyrep/KBCEN}{https://github.com/zzmyrep/KBCEN}.

\textbf{Evaluation Metrics:}
Following~\cite{zhang2024affective, chen2024tgca, zhang2024object}, we adopt \textit{accuracy} and \textit{F1} as our evaluation metrics to measure the performance of different methods for image sentiment analysis. 



\subsection{Experiment Result}
% We compare our model against the following baselines: AlexNet~\cite{krizhevsky2017imagenet}, VGG16~\cite{simonyan2014very}, ResNet101~\cite{he2016deep}, CECCN~\cite{ruan2024color}, EmoVIT~\cite{xie2024emovit}, WSCNet~\cite{yang2018weakly}, ECWA~\cite{deng2021emotion}, EECon~\cite{yang2023exploiting}, MAM~\cite{zhang2024affective} and TGCA-PVT~\cite{chen2024tgca}, and the overall results are summarized in Table~\ref{tab:cap1}.
We compare our model against several baselines, and the overall results are summarized in Table~\ref{tab:cap1}.
We observe that our model achieves the best performance in both accuracy and F1 metrics, significantly outperforming the previous models. 
This superior performance is mainly attributed to our effective utilization of metadata to enhance image sentiment analysis, as well as the exceptional capability of the unified sentiment transformer framework we developed. These results strongly demonstrate that our proposed method can bring encouraging performance for image sentiment analysis.

\setcounter{magicrownumbers}{0} 
\begin{table}[t]
\begin{center}
\caption{Ablation study of~\shortname~on FI dataset.} 
% \vspace{1mm}
\label{tab:cap2}
\resizebox{.9\linewidth}{!}
{
\begin{tabular}{lcc}
  \hline
  \textbf{Model} & \textbf{Accuracy} & \textbf{F1} \\
  \hline
  (\rownumber)~Ours (w/o vision) & 65.72\% & 64.54\% \\
  (\rownumber)~Ours (w/o text description) & 74.05\% & 72.58\% \\
  (\rownumber)~Ours (w/o object tag) & 77.45\% & 76.84\% \\
  (\rownumber)~Ours (w/o scene tag) & 78.47\% & 78.21\% \\
  \hline
  (\rownumber)~Ours (w/o unified embedding) & 76.41\% & 76.23\% \\
  (\rownumber)~Ours (w/o adaptive learning) & 76.83\% & 76.56\% \\
  (\rownumber)~Ours (w/o cross-modal fusion) & 76.85\% & 76.49\% \\
  \hline
  (\rownumber)~Ours  & \textbf{79.48\%} & \textbf{79.22\%} \\
  \hline
\end{tabular}
}
\end{center}
\vspace{-5mm}
\end{table}


\begin{figure}[t]
\centering
% \vspace{-2mm}
\includegraphics[width=0.42\textwidth]{fig/2dvisual-linux4-paper2.pdf}
\caption{Visualization of feature distribution on eight categories before (left) and after (right) model processing.}
% 
\label{fig:visualization}
\vspace{-5mm}
\end{figure}

\subsection{Ablation Performance}
In this subsection, we conduct an ablation study to examine which component is really important for performance improvement. The results are reported in Table~\ref{tab:cap2}.

For information utilization, we observe a significant decline in model performance when visual features are removed. Additionally, the performance of \shortname~decreases when different metadata are removed separately, which means that text description, object tag, and scene tag are all critical for image sentiment analysis.
Recalling the model architecture, we separately remove transformer layers of the unified representation module, the adaptive learning module, and the cross-modal fusion module, replacing them with MLPs of the same parameter scale.
In this way, we can observe varying degrees of decline in model performance, indicating that these modules are indispensable for our model to achieve better performance.

\subsection{Visualization}
% 


% % 开始使用minipage进行左右排列
% \begin{minipage}[t]{0.45\textwidth}  % 子图1宽度为45%
%     \centering
%     \includegraphics[width=\textwidth]{2dvisual.pdf}  % 插入图片
%     \captionof{figure}{Visualization of feature distribution.}  % 使用captionof添加图片标题
%     \label{fig:visualization}
% \end{minipage}


% \begin{figure}[t]
% \centering
% \vspace{-2mm}
% \includegraphics[width=0.45\textwidth]{fig/2dvisual.pdf}
% \caption{Visualization of feature distribution.}
% \label{fig:visualization}
% % \vspace{-4mm}
% \end{figure}

% \begin{figure}[t]
% \centering
% \vspace{-2mm}
% \includegraphics[width=0.45\textwidth]{fig/2dvisual-linux3-paper.pdf}
% \caption{Visualization of feature distribution.}
% \label{fig:visualization}
% % \vspace{-4mm}
% \end{figure}



\begin{figure}[tbp]   
\vspace{-4mm}
  \centering            
  \subfloat[Depth of adaptive learning layers]   
  {
    \label{fig:subfig1}\includegraphics[width=0.22\textwidth]{fig/fig_sensitivity-a5}
  }
  \subfloat[Depth of fusion layers]
  {
    % \label{fig:subfig2}\includegraphics[width=0.22\textwidth]{fig/fig_sensitivity-b2}
    \label{fig:subfig2}\includegraphics[width=0.22\textwidth]{fig/fig_sensitivity-b2-num.pdf}
  }
  \caption{Sensitivity study of \shortname~on different depth. }   
  \label{fig:fig_sensitivity}  
\vspace{-2mm}
\end{figure}

% \begin{figure}[htbp]
% \centerline{\includegraphics{2dvisual.pdf}}
% \caption{Visualization of feature distribution.}
% \label{fig:visualization}
% \end{figure}

% In Fig.~\ref{fig:visualization}, we use t-SNE~\cite{van2008visualizing} to reduce the dimension of data features for visualization, Figure in left represents the metadata features before model processing, the features are obtained by embedding through the CLIP model, and figure in right shows the features of the data after model processing, it can be observed that after the model processing, the data with different label categories fall in different regions in the space, therefore, we can conclude that the Therefore, we can conclude that the model can effectively utilize the information contained in the metadata and use it to guide the model for classification.

In Fig.~\ref{fig:visualization}, we use t-SNE~\cite{van2008visualizing} to reduce the dimension of data features for visualization.
The left figure shows metadata features before being processed by our model (\textit{i.e.}, embedded by CLIP), while the right shows the distribution of features after being processed by our model.
We can observe that after the model processing, data with the same label are closer to each other, while others are farther away.
Therefore, it shows that the model can effectively utilize the information contained in the metadata and use it to guide the classification process.

\subsection{Sensitivity Analysis}
% 
In this subsection, we conduct a sensitivity analysis to figure out the effect of different depth settings of adaptive learning layers and fusion layers. 
% In this subsection, we conduct a sensitivity analysis to figure out the effect of different depth settings on the model. 
% Fig.~\ref{fig:fig_sensitivity} presents the effect of different depth settings of adaptive learning layers and fusion layers. 
Taking Fig.~\ref{fig:fig_sensitivity} (a) as an example, the model performance improves with increasing depth, reaching the best performance at a depth of 4.
% Taking Fig.~\ref{fig:fig_sensitivity} (a) as an example, the performance of \shortname~improves with the increase of depth at first, reaching the best performance at a depth of 4.
When the depth continues to increase, the accuracy decreases to varying degrees.
Similar results can be observed in Fig.~\ref{fig:fig_sensitivity} (b).
Therefore, we set their depths to 4 and 6 respectively to achieve the best results.

% Through our experiments, we can observe that the effect of modifying these hyperparameters on the results of the experiments is very weak, and the surface model is not sensitive to the hyperparameters.


\subsection{Zero-shot Capability}
% 

% (1)~GCH~\cite{2010Analyzing} & 21.78\% & (5)~RA-DLNet~\cite{2020A} & 34.01\% \\ \hline
% (2)~WSCNet~\cite{2019WSCNet}  & 30.25\% & (6)~CECCN~\cite{ruan2024color} & 43.83\% \\ \hline
% (3)~PCNN~\cite{2015Robust} & 31.68\%  & (7)~EmoVIT~\cite{xie2024emovit} & 44.90\% \\ \hline
% (4)~AR~\cite{2018Visual} & 32.67\% & (8)~Ours (Zero-shot) & 47.83\% \\ \hline


\begin{table}[t]
\centering
\caption{Zero-shot capability of \shortname.}
\label{tab:cap3}
\resizebox{1\linewidth}{!}
{
\begin{tabular}{lc|lc}
\hline
\textbf{Model} & \textbf{Accuracy} & \textbf{Model} & \textbf{Accuracy} \\ \hline
(1)~WSCNet~\cite{2019WSCNet}  & 30.25\% & (5)~MAM~\cite{zhang2024affective} & 39.56\%  \\ \hline
(2)~AR~\cite{2018Visual} & 32.67\% & (6)~CECCN~\cite{ruan2024color} & 43.83\% \\ \hline
(3)~RA-DLNet~\cite{2020A} & 34.01\%  & (7)~EmoVIT~\cite{xie2024emovit} & 44.90\% \\ \hline
(4)~CDA~\cite{han2023boosting} & 38.64\% & (8)~Ours (Zero-shot) & 47.83\% \\ \hline
\end{tabular}
}
\vspace{-5mm}
\end{table}

% We use the model trained on the FI dataset to test on the artphoto dataset to verify the model's generalization ability as well as robustness to other distributed datasets.
% We can observe that the MESN model shows strong competitiveness in terms of accuracy when compared to other trained models, which suggests that the model has a good generalization ability in the OOD task.

To validate the model's generalization ability and robustness to other distributed datasets, we directly test the model trained on the FI dataset, without training on Artphoto. 
% As observed in Table 3, compared to other models trained on Artphoto, we achieve highly competitive zero-shot performance, indicating that the model has good generalization ability in out-of-distribution tasks.
From Table~\ref{tab:cap3}, we can observe that compared with other models trained on Artphoto, we achieve competitive zero-shot performance, which shows that the model has good generalization ability in out-of-distribution tasks.


%%%%%%%%%%%%
%  E2E     %
%%%%%%%%%%%%


\section{Conclusion}
In this paper, we introduced Wi-Chat, the first LLM-powered Wi-Fi-based human activity recognition system that integrates the reasoning capabilities of large language models with the sensing potential of wireless signals. Our experimental results on a self-collected Wi-Fi CSI dataset demonstrate the promising potential of LLMs in enabling zero-shot Wi-Fi sensing. These findings suggest a new paradigm for human activity recognition that does not rely on extensive labeled data. We hope future research will build upon this direction, further exploring the applications of LLMs in signal processing domains such as IoT, mobile sensing, and radar-based systems.

\section*{Limitations}
While our work represents the first attempt to leverage LLMs for processing Wi-Fi signals, it is a preliminary study focused on a relatively simple task: Wi-Fi-based human activity recognition. This choice allows us to explore the feasibility of LLMs in wireless sensing but also comes with certain limitations.

Our approach primarily evaluates zero-shot performance, which, while promising, may still lag behind traditional supervised learning methods in highly complex or fine-grained recognition tasks. Besides, our study is limited to a controlled environment with a self-collected dataset, and the generalizability of LLMs to diverse real-world scenarios with varying Wi-Fi conditions, environmental interference, and device heterogeneity remains an open question.

Additionally, we have yet to explore the full potential of LLMs in more advanced Wi-Fi sensing applications, such as fine-grained gesture recognition, occupancy detection, and passive health monitoring. Future work should investigate the scalability of LLM-based approaches, their robustness to domain shifts, and their integration with multimodal sensing techniques in broader IoT applications.


% Bibliography entries for the entire Anthology, followed by custom entries
%\bibliography{anthology,custom}
% Custom bibliography entries only
\bibliography{main}
\newpage
\appendix

\section{Experiment prompts}
\label{sec:prompt}
The prompts used in the LLM experiments are shown in the following Table~\ref{tab:prompts}.

\definecolor{titlecolor}{rgb}{0.9, 0.5, 0.1}
\definecolor{anscolor}{rgb}{0.2, 0.5, 0.8}
\definecolor{labelcolor}{HTML}{48a07e}
\begin{table*}[h]
	\centering
	
 % \vspace{-0.2cm}
	
	\begin{center}
		\begin{tikzpicture}[
				chatbox_inner/.style={rectangle, rounded corners, opacity=0, text opacity=1, font=\sffamily\scriptsize, text width=5in, text height=9pt, inner xsep=6pt, inner ysep=6pt},
				chatbox_prompt_inner/.style={chatbox_inner, align=flush left, xshift=0pt, text height=11pt},
				chatbox_user_inner/.style={chatbox_inner, align=flush left, xshift=0pt},
				chatbox_gpt_inner/.style={chatbox_inner, align=flush left, xshift=0pt},
				chatbox/.style={chatbox_inner, draw=black!25, fill=gray!7, opacity=1, text opacity=0},
				chatbox_prompt/.style={chatbox, align=flush left, fill=gray!1.5, draw=black!30, text height=10pt},
				chatbox_user/.style={chatbox, align=flush left},
				chatbox_gpt/.style={chatbox, align=flush left},
				chatbox2/.style={chatbox_gpt, fill=green!25},
				chatbox3/.style={chatbox_gpt, fill=red!20, draw=black!20},
				chatbox4/.style={chatbox_gpt, fill=yellow!30},
				labelbox/.style={rectangle, rounded corners, draw=black!50, font=\sffamily\scriptsize\bfseries, fill=gray!5, inner sep=3pt},
			]
											
			\node[chatbox_user] (q1) {
				\textbf{System prompt}
				\newline
				\newline
				You are a helpful and precise assistant for segmenting and labeling sentences. We would like to request your help on curating a dataset for entity-level hallucination detection.
				\newline \newline
                We will give you a machine generated biography and a list of checked facts about the biography. Each fact consists of a sentence and a label (True/False). Please do the following process. First, breaking down the biography into words. Second, by referring to the provided list of facts, merging some broken down words in the previous step to form meaningful entities. For example, ``strategic thinking'' should be one entity instead of two. Third, according to the labels in the list of facts, labeling each entity as True or False. Specifically, for facts that share a similar sentence structure (\eg, \textit{``He was born on Mach 9, 1941.''} (\texttt{True}) and \textit{``He was born in Ramos Mejia.''} (\texttt{False})), please first assign labels to entities that differ across atomic facts. For example, first labeling ``Mach 9, 1941'' (\texttt{True}) and ``Ramos Mejia'' (\texttt{False}) in the above case. For those entities that are the same across atomic facts (\eg, ``was born'') or are neutral (\eg, ``he,'' ``in,'' and ``on''), please label them as \texttt{True}. For the cases that there is no atomic fact that shares the same sentence structure, please identify the most informative entities in the sentence and label them with the same label as the atomic fact while treating the rest of the entities as \texttt{True}. In the end, output the entities and labels in the following format:
                \begin{itemize}[nosep]
                    \item Entity 1 (Label 1)
                    \item Entity 2 (Label 2)
                    \item ...
                    \item Entity N (Label N)
                \end{itemize}
                % \newline \newline
                Here are two examples:
                \newline\newline
                \textbf{[Example 1]}
                \newline
                [The start of the biography]
                \newline
                \textcolor{titlecolor}{Marianne McAndrew is an American actress and singer, born on November 21, 1942, in Cleveland, Ohio. She began her acting career in the late 1960s, appearing in various television shows and films.}
                \newline
                [The end of the biography]
                \newline \newline
                [The start of the list of checked facts]
                \newline
                \textcolor{anscolor}{[Marianne McAndrew is an American. (False); Marianne McAndrew is an actress. (True); Marianne McAndrew is a singer. (False); Marianne McAndrew was born on November 21, 1942. (False); Marianne McAndrew was born in Cleveland, Ohio. (False); She began her acting career in the late 1960s. (True); She has appeared in various television shows. (True); She has appeared in various films. (True)]}
                \newline
                [The end of the list of checked facts]
                \newline \newline
                [The start of the ideal output]
                \newline
                \textcolor{labelcolor}{[Marianne McAndrew (True); is (True); an (True); American (False); actress (True); and (True); singer (False); , (True); born (True); on (True); November 21, 1942 (False); , (True); in (True); Cleveland, Ohio (False); . (True); She (True); began (True); her (True); acting career (True); in (True); the late 1960s (True); , (True); appearing (True); in (True); various (True); television shows (True); and (True); films (True); . (True)]}
                \newline
                [The end of the ideal output]
				\newline \newline
                \textbf{[Example 2]}
                \newline
                [The start of the biography]
                \newline
                \textcolor{titlecolor}{Doug Sheehan is an American actor who was born on April 27, 1949, in Santa Monica, California. He is best known for his roles in soap operas, including his portrayal of Joe Kelly on ``General Hospital'' and Ben Gibson on ``Knots Landing.''}
                \newline
                [The end of the biography]
                \newline \newline
                [The start of the list of checked facts]
                \newline
                \textcolor{anscolor}{[Doug Sheehan is an American. (True); Doug Sheehan is an actor. (True); Doug Sheehan was born on April 27, 1949. (True); Doug Sheehan was born in Santa Monica, California. (False); He is best known for his roles in soap operas. (True); He portrayed Joe Kelly. (True); Joe Kelly was in General Hospital. (True); General Hospital is a soap opera. (True); He portrayed Ben Gibson. (True); Ben Gibson was in Knots Landing. (True); Knots Landing is a soap opera. (True)]}
                \newline
                [The end of the list of checked facts]
                \newline \newline
                [The start of the ideal output]
                \newline
                \textcolor{labelcolor}{[Doug Sheehan (True); is (True); an (True); American (True); actor (True); who (True); was born (True); on (True); April 27, 1949 (True); in (True); Santa Monica, California (False); . (True); He (True); is (True); best known (True); for (True); his roles in soap operas (True); , (True); including (True); in (True); his portrayal (True); of (True); Joe Kelly (True); on (True); ``General Hospital'' (True); and (True); Ben Gibson (True); on (True); ``Knots Landing.'' (True)]}
                \newline
                [The end of the ideal output]
				\newline \newline
				\textbf{User prompt}
				\newline
				\newline
				[The start of the biography]
				\newline
				\textcolor{magenta}{\texttt{\{BIOGRAPHY\}}}
				\newline
				[The ebd of the biography]
				\newline \newline
				[The start of the list of checked facts]
				\newline
				\textcolor{magenta}{\texttt{\{LIST OF CHECKED FACTS\}}}
				\newline
				[The end of the list of checked facts]
			};
			\node[chatbox_user_inner] (q1_text) at (q1) {
				\textbf{System prompt}
				\newline
				\newline
				You are a helpful and precise assistant for segmenting and labeling sentences. We would like to request your help on curating a dataset for entity-level hallucination detection.
				\newline \newline
                We will give you a machine generated biography and a list of checked facts about the biography. Each fact consists of a sentence and a label (True/False). Please do the following process. First, breaking down the biography into words. Second, by referring to the provided list of facts, merging some broken down words in the previous step to form meaningful entities. For example, ``strategic thinking'' should be one entity instead of two. Third, according to the labels in the list of facts, labeling each entity as True or False. Specifically, for facts that share a similar sentence structure (\eg, \textit{``He was born on Mach 9, 1941.''} (\texttt{True}) and \textit{``He was born in Ramos Mejia.''} (\texttt{False})), please first assign labels to entities that differ across atomic facts. For example, first labeling ``Mach 9, 1941'' (\texttt{True}) and ``Ramos Mejia'' (\texttt{False}) in the above case. For those entities that are the same across atomic facts (\eg, ``was born'') or are neutral (\eg, ``he,'' ``in,'' and ``on''), please label them as \texttt{True}. For the cases that there is no atomic fact that shares the same sentence structure, please identify the most informative entities in the sentence and label them with the same label as the atomic fact while treating the rest of the entities as \texttt{True}. In the end, output the entities and labels in the following format:
                \begin{itemize}[nosep]
                    \item Entity 1 (Label 1)
                    \item Entity 2 (Label 2)
                    \item ...
                    \item Entity N (Label N)
                \end{itemize}
                % \newline \newline
                Here are two examples:
                \newline\newline
                \textbf{[Example 1]}
                \newline
                [The start of the biography]
                \newline
                \textcolor{titlecolor}{Marianne McAndrew is an American actress and singer, born on November 21, 1942, in Cleveland, Ohio. She began her acting career in the late 1960s, appearing in various television shows and films.}
                \newline
                [The end of the biography]
                \newline \newline
                [The start of the list of checked facts]
                \newline
                \textcolor{anscolor}{[Marianne McAndrew is an American. (False); Marianne McAndrew is an actress. (True); Marianne McAndrew is a singer. (False); Marianne McAndrew was born on November 21, 1942. (False); Marianne McAndrew was born in Cleveland, Ohio. (False); She began her acting career in the late 1960s. (True); She has appeared in various television shows. (True); She has appeared in various films. (True)]}
                \newline
                [The end of the list of checked facts]
                \newline \newline
                [The start of the ideal output]
                \newline
                \textcolor{labelcolor}{[Marianne McAndrew (True); is (True); an (True); American (False); actress (True); and (True); singer (False); , (True); born (True); on (True); November 21, 1942 (False); , (True); in (True); Cleveland, Ohio (False); . (True); She (True); began (True); her (True); acting career (True); in (True); the late 1960s (True); , (True); appearing (True); in (True); various (True); television shows (True); and (True); films (True); . (True)]}
                \newline
                [The end of the ideal output]
				\newline \newline
                \textbf{[Example 2]}
                \newline
                [The start of the biography]
                \newline
                \textcolor{titlecolor}{Doug Sheehan is an American actor who was born on April 27, 1949, in Santa Monica, California. He is best known for his roles in soap operas, including his portrayal of Joe Kelly on ``General Hospital'' and Ben Gibson on ``Knots Landing.''}
                \newline
                [The end of the biography]
                \newline \newline
                [The start of the list of checked facts]
                \newline
                \textcolor{anscolor}{[Doug Sheehan is an American. (True); Doug Sheehan is an actor. (True); Doug Sheehan was born on April 27, 1949. (True); Doug Sheehan was born in Santa Monica, California. (False); He is best known for his roles in soap operas. (True); He portrayed Joe Kelly. (True); Joe Kelly was in General Hospital. (True); General Hospital is a soap opera. (True); He portrayed Ben Gibson. (True); Ben Gibson was in Knots Landing. (True); Knots Landing is a soap opera. (True)]}
                \newline
                [The end of the list of checked facts]
                \newline \newline
                [The start of the ideal output]
                \newline
                \textcolor{labelcolor}{[Doug Sheehan (True); is (True); an (True); American (True); actor (True); who (True); was born (True); on (True); April 27, 1949 (True); in (True); Santa Monica, California (False); . (True); He (True); is (True); best known (True); for (True); his roles in soap operas (True); , (True); including (True); in (True); his portrayal (True); of (True); Joe Kelly (True); on (True); ``General Hospital'' (True); and (True); Ben Gibson (True); on (True); ``Knots Landing.'' (True)]}
                \newline
                [The end of the ideal output]
				\newline \newline
				\textbf{User prompt}
				\newline
				\newline
				[The start of the biography]
				\newline
				\textcolor{magenta}{\texttt{\{BIOGRAPHY\}}}
				\newline
				[The ebd of the biography]
				\newline \newline
				[The start of the list of checked facts]
				\newline
				\textcolor{magenta}{\texttt{\{LIST OF CHECKED FACTS\}}}
				\newline
				[The end of the list of checked facts]
			};
		\end{tikzpicture}
        \caption{GPT-4o prompt for labeling hallucinated entities.}\label{tb:gpt-4-prompt}
	\end{center}
\vspace{-0cm}
\end{table*}
% \section{Full Experiment Results}
% \begin{table*}[th]
    \centering
    \small
    \caption{Classification Results}
    \begin{tabular}{lcccc}
        \toprule
        \textbf{Method} & \textbf{Accuracy} & \textbf{Precision} & \textbf{Recall} & \textbf{F1-score} \\
        \midrule
        \multicolumn{5}{c}{\textbf{Zero Shot}} \\
                Zero-shot E-eyes & 0.26 & 0.26 & 0.27 & 0.26 \\
        Zero-shot CARM & 0.24 & 0.24 & 0.24 & 0.24 \\
                Zero-shot SVM & 0.27 & 0.28 & 0.28 & 0.27 \\
        Zero-shot CNN & 0.23 & 0.24 & 0.23 & 0.23 \\
        Zero-shot RNN & 0.26 & 0.26 & 0.26 & 0.26 \\
DeepSeek-0shot & 0.54 & 0.61 & 0.54 & 0.52 \\
DeepSeek-0shot-COT & 0.33 & 0.24 & 0.33 & 0.23 \\
DeepSeek-0shot-Knowledge & 0.45 & 0.46 & 0.45 & 0.44 \\
Gemma2-0shot & 0.35 & 0.22 & 0.38 & 0.27 \\
Gemma2-0shot-COT & 0.36 & 0.22 & 0.36 & 0.27 \\
Gemma2-0shot-Knowledge & 0.32 & 0.18 & 0.34 & 0.20 \\
GPT-4o-mini-0shot & 0.48 & 0.53 & 0.48 & 0.41 \\
GPT-4o-mini-0shot-COT & 0.33 & 0.50 & 0.33 & 0.38 \\
GPT-4o-mini-0shot-Knowledge & 0.49 & 0.31 & 0.49 & 0.36 \\
GPT-4o-0shot & 0.62 & 0.62 & 0.47 & 0.42 \\
GPT-4o-0shot-COT & 0.29 & 0.45 & 0.29 & 0.21 \\
GPT-4o-0shot-Knowledge & 0.44 & 0.52 & 0.44 & 0.39 \\
LLaMA-0shot & 0.32 & 0.25 & 0.32 & 0.24 \\
LLaMA-0shot-COT & 0.12 & 0.25 & 0.12 & 0.09 \\
LLaMA-0shot-Knowledge & 0.32 & 0.25 & 0.32 & 0.28 \\
Mistral-0shot & 0.19 & 0.23 & 0.19 & 0.10 \\
Mistral-0shot-Knowledge & 0.21 & 0.40 & 0.21 & 0.11 \\
        \midrule
        \multicolumn{5}{c}{\textbf{4 Shot}} \\
GPT-4o-mini-4shot & 0.58 & 0.59 & 0.58 & 0.53 \\
GPT-4o-mini-4shot-COT & 0.57 & 0.53 & 0.57 & 0.50 \\
GPT-4o-mini-4shot-Knowledge & 0.56 & 0.51 & 0.56 & 0.47 \\
GPT-4o-4shot & 0.77 & 0.84 & 0.77 & 0.73 \\
GPT-4o-4shot-COT & 0.63 & 0.76 & 0.63 & 0.53 \\
GPT-4o-4shot-Knowledge & 0.72 & 0.82 & 0.71 & 0.66 \\
LLaMA-4shot & 0.29 & 0.24 & 0.29 & 0.21 \\
LLaMA-4shot-COT & 0.20 & 0.30 & 0.20 & 0.13 \\
LLaMA-4shot-Knowledge & 0.15 & 0.23 & 0.13 & 0.13 \\
Mistral-4shot & 0.02 & 0.02 & 0.02 & 0.02 \\
Mistral-4shot-Knowledge & 0.21 & 0.27 & 0.21 & 0.20 \\
        \midrule
        
        \multicolumn{5}{c}{\textbf{Suprevised}} \\
        SVM & 0.94 & 0.92 & 0.91 & 0.91 \\
        CNN & 0.98 & 0.98 & 0.97 & 0.97 \\
        RNN & 0.99 & 0.99 & 0.99 & 0.99 \\
        % \midrule
        % \multicolumn{5}{c}{\textbf{Conventional Wi-Fi-based Human Activity Recognition Systems}} \\
        E-eyes & 1.00 & 1.00 & 1.00 & 1.00 \\
        CARM & 0.98 & 0.98 & 0.98 & 0.98 \\
\midrule
 \multicolumn{5}{c}{\textbf{Vision Models}} \\
           Zero-shot SVM & 0.26 & 0.25 & 0.25 & 0.25 \\
        Zero-shot CNN & 0.26 & 0.25 & 0.26 & 0.26 \\
        Zero-shot RNN & 0.28 & 0.28 & 0.29 & 0.28 \\
        SVM & 0.99 & 0.99 & 0.99 & 0.99 \\
        CNN & 0.98 & 0.99 & 0.98 & 0.98 \\
        RNN & 0.98 & 0.99 & 0.98 & 0.98 \\
GPT-4o-mini-Vision & 0.84 & 0.85 & 0.84 & 0.84 \\
GPT-4o-mini-Vision-COT & 0.90 & 0.91 & 0.90 & 0.90 \\
GPT-4o-Vision & 0.74 & 0.82 & 0.74 & 0.73 \\
GPT-4o-Vision-COT & 0.70 & 0.83 & 0.70 & 0.68 \\
LLaMA-Vision & 0.20 & 0.23 & 0.20 & 0.09 \\
LLaMA-Vision-Knowledge & 0.22 & 0.05 & 0.22 & 0.08 \\

        \bottomrule
    \end{tabular}
    \label{full}
\end{table*}




\end{document}



%%%%%%%%%%%%%%%%%%%%%%%%%%%%%%%%%%%%%%%%%%%%%%%%%%%%%%%%%%%%%%%%%%%%%%%%%%%%%%%
%%%%%%%%%%%%%%%%%%%%%%%%%%%%%%%%%%%%%%%%%%%%%%%%%%%%%%%%%%%%%%%%%%%%%%%%%%%%%%%
% APPENDIX
%%%%%%%%%%%%%%%%%%%%%%%%%%%%%%%%%%%%%%%%%%%%%%%%%%%%%%%%%%%%%%%%%%%%%%%%%%%%%%%
%%%%%%%%%%%%%%%%%%%%%%%%%%%%%%%%%%%%%%%%%%%%%%%%%%%%%%%%%%%%%%%%%%%%%%%%%%%%%%%
\newpage
\appendix


\section{Mathematical Proofs}
\label{app-sec:proofs}

\subsection{Auxiliary Technical Results}

In this section, we begin by introducing two useful propositions, \Cref{app-prop:p-value-to-quantile} and \Cref{app-prop:quantiles}, which will be used later in the proofs presented here. 


\begin{proposition}
\label{app-prop:p-value-to-quantile}
    Let $\D$ be a dataset containing $n$ scores, and define the threshold $\hat{Q}_{1-\alpha}$ as
    \begin{align*}
\hat{Q}_{1-\alpha} &:= \hat{i} \text{-th smallest element in } \D\cup \{\infty\},
\end{align*}
where 
\begin{align*}
    \hat{i} :=\lceil (1-\alpha)(n+1)\rceil.
\end{align*}
For any test point $X_{n+1}$, the following holds:
    \begin{align*}
        s(X_{n+1}) > \hat{Q}_{1-\alpha} \quad\text{ if and only if }\quad \hat{p}_{n+1} \leq \alpha,
    \end{align*}
    where $\hat{p}_{n+1}$ is the conformal p-value \eqref{eq:conformal-p-value}.
\end{proposition}

\begin{proof}[Proof of \Cref{app-prop:p-value-to-quantile}]
The proof follows the definition of conformal p-value from \eqref{eq:conformal-p-value}, and its relation to the empirical quantile function:
\begin{alignat*}{2}
    \hat{p}_{n+1} = \frac{1 + \sum_{i=1}^{n} \mathbb{I}[s(X_i) \geq s(X_{n+1})]}{n+1}
    &\leq \alpha &&\overset{(i)}{\Longleftrightarrow} \\
    \hat{p}_{n+1} = \frac{1 + \sum_{i=1}^{n} \mathbb{I}[s(X_i) \geq s(X_{n+1})]}{n+1}
    &\leq \frac{\lfloor \alpha(n+1) \rfloor }{n+1} &&\Longleftrightarrow \\
    1 + \sum_{i=1}^{n} \mathbb{I}[s(X_i) \geq s(X_{n+1})] &\leq \lfloor \alpha (n+1) \rfloor &&\Longleftrightarrow \\
    1 + n - \sum_{i=1}^{n} \mathbb{I}[s(X_i) < s(X_{n+1})] &\leq \lfloor \alpha (n+1) \rfloor &&\Longleftrightarrow \\
     \sum_{i=1}^{n} \mathbb{I}[s(X_i) < s(X_{n+1})] &\geq n + 1 - \lfloor \alpha (n+1) \rfloor &&\overset{(ii)}{\Longleftrightarrow} \\
    \sum_{i=1}^{n} \mathbb{I}[s(X_i) < s(X_{n+1})] &\geq  \lceil (1-\alpha) (n+1) \rceil && \numberthis \label{app-eq:prop-inequality}
\end{alignat*}
The labeled steps above can be explained as follows.
\begin{itemize}
    \item (i) The values of $\hat{p}_{n+1}$ are discrete, taking values from $\{\frac{1}{n+1}, \frac{2}{n+1}, \dots, 1\}$. Therefore, $\hat{p}_{n+1} = \frac{k}{n+1}$ for some $k\in[n+1]$. We explicitly prove that $\hat{p}_{n+1}\leq \alpha$ iff $\hat{p}_{n+1}\leq \frac{\lfloor \alpha(n+1)\rfloor}{n+1}$ as follows:
    \begin{itemize}
        \item[$\Leftarrow$] Assume $\hat{p}_{n+1} \leq \frac{\lfloor \alpha(n+1)\rfloor}{n+1}$. Therefore, $\hat{p}_{n+1} \leq \frac{\lfloor \alpha(n+1)\rfloor}{n+1} \leq \frac{\alpha(n+1)}{n+1} = \alpha$.
        \item[$\Rightarrow$] Assume $\hat{p}_{n+1} \leq \alpha$, then $\frac{k}{n+1}\leq \alpha$. This implies that 
        $k \leq \alpha (n+1)$. Since $k$ is an integer, it follows that $k \leq \lfloor \alpha (n+1) \rfloor$. Therefore, $\hat{p}_{n+1} = \frac{k}{n+1} \leq \frac{\lfloor \alpha (n+1) \rfloor}{n+1}$.
    \end{itemize}
    \item (ii) This step follows directly from the equality $n + 1 = \lceil (1-\alpha)(n+1)\rceil + \lfloor \alpha(n+1)\rfloor$. We explicitly prove this equality as follows:
    \begin{itemize}
        \item The term $\lceil (1-\alpha) (n+1)\rceil$ represents the smallest integer greater than or equal to $(1-\alpha)(n+1)$. Hence, we can write:
        \begin{align*}
            \lceil (1-\alpha)(n+1)\rceil = (1-\alpha)(n+1) + \delta_1, 
        \end{align*}
        where $0\leq\delta_1<1$.
        \item Similarly, the term $\lfloor \alpha (n+1)\rfloor$ represents the largest integer less than or equal to $\alpha(n+1)$. Thus:
        \begin{align*}
            \lfloor \alpha(n+1)\rfloor = \alpha(n+1) - \delta_2, 
        \end{align*}
        where $0\leq\delta_2<1$.
        \item Adding these two terms gives:
    \begin{align*}
        \lfloor \alpha(n+1)\rfloor + \lceil (1-\alpha)(n+1)\rceil = \alpha(n+1) - \delta_2 + (1-\alpha)(n+1) + \delta_1 = n + 1 + (\delta_1 - \delta_2).
    \end{align*}
    Since $\lfloor \alpha(n+1)\rfloor + \lceil (1-\alpha)(n+1)\rceil$ must be an integer and $\delta_1,\delta_2\in [0,1)$, it follows that $\delta_1 - \delta_2 = 0$.
    \end{itemize}
    
    Therefore, $\lfloor \alpha(n+1)\rfloor + \lceil (1-\alpha)(n+1)\rceil = n + 1$.
\end{itemize}

To complete the proof, we now show that \eqref{app-eq:prop-inequality} holds if and only if $s(X_{n+1}) > \hat{Q}_{1-\alpha}$. 
\begin{itemize}
    \item[$\Leftarrow$] Assume $s(X_{n+1}) > \hat{Q}_{1-\alpha}$. 
    By definition, $\sum_{i=1}^{n} \mathbb{I}[s(X_i) \leq \hat{Q}_{1-\alpha}] = \lceil (1-\alpha)(n+1)\rceil$.
    Then, \eqref{app-eq:prop-inequality} holds since
    \begin{align*}
        \sum_{i=1}^{n} \mathbb{I}[s(X_i) < s(X_{n+1})] \geq \sum_{i=1}^{n} \mathbb{I}[s(X_i) \leq \hat{Q}_{1-\alpha}] =  \lceil (1-\alpha)(n+1)\rceil.
    \end{align*}
    \item[$\Rightarrow$] We prove this direction by contradiction, assuming that \eqref{app-eq:prop-inequality} holds. Now, suppose that $s(X_{n+1})\leq \hat{Q}_{1-\alpha}$ also holds, implying that
    \begin{align*}
        \sum_{i=1}^{n} \mathbb{I}[s(X_i) < s(X_{n+1})] \leq \sum_{i=1}^{n} \mathbb{I}[s(X_i) < \hat{Q}_{1-\alpha}] \overset{(i)}{<} \lceil (1-\alpha) (n+1) \rceil,
    \end{align*}
    which contradicts the assumption \eqref{app-eq:prop-inequality}. Therefore, we conclude that $s(X_{n+1}) > \hat{Q}_{1-\alpha}$. The last step above can be explained as follows.
    \begin{itemize}
        \item (i) Recall that by definition, $\hat{Q}_{1-\alpha}$ is a specific value in $\{s(X_i)\}_{i=1}^{n}$ and $\sum_{i=1}^{n} \mathbb{I}[s(X_i) \leq \hat{Q}_{1-\alpha}] = \lceil (1-\alpha)(n+1)\rceil$.
This implies that
\begin{align*}
%\label{eq:quantile_and_p}
    \sum_{i=1}^{n} \mathbb{I}[s(X_i) < \hat{Q}_{1-\alpha}] 
 = \lceil (1-\alpha)(n+1)\rceil - \sum_{i=1}^{n} \mathbb{I}[s(X_i) = \hat{Q}_{1-\alpha}] < \lceil (1-\alpha)(n+1)\rceil.
\end{align*}
    \end{itemize}
\end{itemize}
In sum, $\hat{p}_{n+1}\leq \alpha$ holds if and only if \eqref{app-eq:prop-inequality} holds, and the latter holds if and only if $s(X_{n+1}) > \hat{Q}_{1-\alpha}$. This completes the proof.
\end{proof}

\begin{proposition}\label{app-prop:quantiles}
    Let $\D$ be a dataset containing $n$ scores, and let $S_{n+1}$ be a test score. Define the following thresholds:
        \begin{align*}
\hat{Q}_{1-\alpha} &:= \hat{i} \text{-th smallest element in } \D\cup \{\infty\},
\end{align*}
where 
\begin{align*}
    \hat{i} :=\lceil (1-\alpha)(n+1)\rceil.
\end{align*}
Similarly, let
    \begin{align*}
\hat{Q}_{1-\alpha}^{n+1} &:= \hat{i} \text{-th smallest element in } \D\cup \{S_{n+1}\}.
\end{align*}
It follows that
\begin{align*}
    \hat{Q}_{1-\alpha} \geq \hat{Q}_{1-\alpha}^{n+1} \text{ almost surely}.
\end{align*}
\end{proposition}

\begin{proof}[Proof of \Cref{app-prop:quantiles}]
    Since the largest possible score is $\infty$, the set $\D \cup \{\infty \}$ almost surely contains scores that are greater or equal to those in $\D \cup \{S_{n+1}\}$. Consequently, $\hat{Q}_{1-\alpha} := \hat{i}$-th smallest element in $\D \cup \{ \infty \}$ is almost surely greater than or equal to $\hat{Q}_{1-\alpha}^{n+1} := \hat{i}$-th smallest element in $\D \cup \{S_{n+1}\}$.
\end{proof}

\subsection{Explaining the Conservativeness of Standard Conformal p-Values}

\subsubsection{Proof of~\Cref{lem:conservativeness}}

\begin{proof}[Proof of~\Cref{lem:conservativeness}]
\label{prf:lem}
To simplify the notation define the random score $S_i := s(X_i)$ for all $i\in \D_{\mathrm{cal}} \cup \{n+1\}$. Throughout the proof, we refer to the calibration set as the set of nonconformity scores corresponding to the calibration points.
Without loss of generality, assume that the inliers in $\D_{\mathrm{cal}}$ are located at the first $n_0$ indices. Let $\D_{\mathrm{inlier}}=[n_0]$ denote the set of indices corresponding to the inlier scores. Consequently, define $\D_{\mathrm{outlier}} = \{n_0+1, n_0+2, \ldots, n\}$ as the set of indices corresponding to the outlier scores in $\D_{\mathrm{cal}}$.
We assume the scores have no ties (which can always be achieved by adding a negligible random noise to the scores output by any model).

Given a fixed realization of the score vector $(s_1,\ldots,s_n,s_{n+1}) \in \mathbb{R}^{n+1}$, define the following two events: 
    \begin{itemize}
        \item $E_{\mathrm{in}}$: the unordered set of inlier scores, including the test score, is $\{S_1,\dots,S_{n_0},S_{n+1}\} = \{s_1,\dots,s_{n_0},s_{n+1}\}$;
        \item $E_{\mathrm{out}}$: the unordered set of outlier scores is $\{S_{n_0+1}, \dots, S_n\} = \{s_{n_0+1}, \dots, s_n\}$.
    \end{itemize}

Under the setup defined in~\eqref{eq:setup-contaminated}, when $\mathcal{H}_0$ is true, the test score $S_{n+1}$ and the inlier scores in the calibration set are i.i.d.~from $\p_0$. 
    Therefore, by exchangeability, the following holds for each inlier index $i\in \D_{\mathrm{inlier}}\cup\{n+1\}$:
    \begin{align}\label{eq:exch-inlier}
        \p \left( S_{n+1} = s_i \mid E_{\mathrm{in}}, E_{\mathrm{out}}\right) =         \frac{1}{n_0 +1}.
    \end{align}
    Since the calibration scores are almost-surely distinct, the probability of a null test point obtaining any outlier score is zero. Therefore, for each outlier index $j\in \D_{\mathrm{outlier}}$:
    \begin{align}\label{eq:exch-outlier}
        \p \left( S_{n+1} = s_j \mid E_{\mathrm{in}}, E_{\mathrm{out}}\right) = 0.
    \end{align}

To obtain an upper bound on the type-I error rate, $\p \left( \hat{p}_{n+1} \leq \alpha \right)$, we use the equivalence established in~\Cref{app-prop:p-value-to-quantile}. According to this result, the following holds:
\begin{align*}
    \p \left( \hat{p}_{n+1} \leq \alpha \right) = \p \left( S_{n+1} > \hat{Q}_{1-\alpha}^{\mathrm{cal}}\right),
\end{align*}
where $\hat{Q}^{\mathrm{cal}}_{1-\alpha}$ is the $\hat{i}_{\mathrm{cal}}$-th smallest element in $\{S_i\}_{i=1}^{n}\cup\{\infty\}$ and $\hat{i}_{\mathrm{cal}} :=\lceil (1-\alpha)(n+1)\rceil$.

Moreover, define $\hat{Q}_{1-\alpha}^{n+1}$ as the $\hat{i}_{\mathrm{cal}}$-th smallest score in $\{S_i\}_{i=1}^{n+1}$.
By~\Cref{app-prop:quantiles}, $\hat{Q}^{\mathrm{cal}}_{1-\alpha} \geq \hat{Q}_{1-\alpha}^{n+1}$ almost surely.\\


Now, we obtain an upper bound for $\p \left( S_{n+1} > \hat{Q}^{\mathrm{cal}}_{1-\alpha} \mid E_{\mathrm{in}}, E_{\mathrm{out}}\right)$, where the probability is taken over random permutations of the scores conditional on $E_{\mathrm{in}}, E_{\mathrm{out}}$.
    \begin{align*} 
        \p \left( S_{n+1} > \hat{Q}^{\mathrm{cal}}_{1-\alpha} \mid E_{\mathrm{in}}, E_{\mathrm{out}}\right) &\leq \p \left( S_{n+1} > \hat{Q}^{\mathrm{n+1}}_{1-\alpha} \mid E_{\mathrm{in}}, E_{\mathrm{out}}\right)\\
        &= \E \left[ \mathbb{I} \left[ S_{n+1} > \hat{Q}^{\mathrm{n+1}}_{1-\alpha}\right]\mid E_{\mathrm{in}}, E_{\mathrm{out}} \right] \\
     &= \sum\limits_{i\in\D_{\mathrm{inlier}}\cup\{n+1\}} \E \left[ \mathbb{I} \left[S_{n+1}=s_i \right] \mathbb{I}\left[ s_i > \hat{Q}^{\mathrm{n+1}}_{1-\alpha}\right] \mid E_{\mathrm{in}}, E_{\mathrm{out}}\right]\\
          &\overset{(i)}{=} \sum\limits_{i\in\D_{\mathrm{inlier}}\cup\{n+1\}} \mathbb{I} \left[ s_i > \hat{Q}^{\mathrm{n+1}}_{1-\alpha}\right] \p \left(  S_{n+1}=s_i   \mid E_{\mathrm{in}}, E_{\mathrm{out}}\right)\\
        &= \frac{1}{n_0+1} \sum\limits_{i\in  \D_{\mathrm{inlier}} \cup \{ n+1\}} \mathbb{I} \left[ s_i > \hat{Q}^{\mathrm{n+1}}_{1-\alpha}\right] \\
        &\overset{(ii)}{\leq}  \frac{1}{n_0+1} \left(\alpha(n+1) - \sum\limits_{i\in  \D_{\mathrm{outlier}}} \mathbb{I} \left[ s_i > \hat{Q}^{\mathrm{n+1}}_{1-\alpha}\right]\right)\\
        &= \alpha + \frac{1}{n_0+1} \left( \alpha n_1 - \sum\limits_{i\in  \D_{\mathrm{outlier}}} \mathbb{I} \left[ s_i > \hat{Q}^{\mathrm{n+1}}_{1-\alpha}\right]\right)\\
        &= \alpha - \frac{n_1}{n_0+1} \left( 1-\alpha - \frac{1}{n_1} \sum\limits_{i\in  \D_{\mathrm{outlier}}} \mathbb{I} \left[ s_i \leq \hat{Q}^{\mathrm{n+1}}_{1-\alpha}\right]\right) \\
        &\overset{(iii)}{\leq} \alpha - \frac{n_1}{n_0+1} \left( 1-\alpha - \frac{1}{n_1} \sum\limits_{i\in  \D_{\mathrm{outlier}}} \mathbb{I} \left[ s_i \leq \hat{Q}^{\mathrm{cal}}_{1-\alpha}\right]\right) \\
        &= \alpha - \frac{n_1}{n_0 + 1} \left( 1-\alpha -\hat{F}_{1} \left( \hat{Q}^{\mathrm{cal}}_{1-\alpha} \right) \right),
    \end{align*}
where $\hat{F}_{1}$ is the empirical CDF of the outlier scores.
The labeled steps above can be explained as follows.
\begin{itemize}
\item (i) $\mathbb{I} \left[ s_i > \hat{Q}^{\mathrm{n+1}}_{1-\alpha} \right]$ is measurable with respect to the $\sigma$-algebra generated by $E_{\mathrm{in}}, E_{\mathrm{out}}$. This follows because $\hat{Q}_{1-\alpha}^{\mathrm{n+1}}$ is the $\hat{i}_{\mathrm{cal}}$-th smallest element of $\{s_1,\dots,s_{n+1}\}$, which is fully determined by these variables. Thus, we can pull it out of the expectation.\\
\item (ii) $\hat{Q}^{\mathrm{n+1}}_{1-\alpha}$ is the $\hat{i}_{\mathrm{cal}}$-th smallest score in $\{s_1,\dots,s_n,s_{n+1}\}$ 
  and $[n+1] = \D_{\mathrm{outlier}} \cup \D_{\mathrm{inlier}} \cup \{n+1\}$. By definition,\\ 
  \begin{align*}
    &\sum\limits_{i\in  \D_{\mathrm{inlier}} \cup \{ n+1\}} \mathbb{I}\left[ s_i > \hat{Q}^{\mathrm{n+1}}_{1-\alpha} \right] + \sum\limits_{i\in  \D_{\mathrm{outlier}}} \mathbb{I} \left[ s_i > \hat{Q}^{\mathrm{n+1}}_{1-\alpha} \right] = 
      \sum\limits_{i=1}^{n+1} \mathbb{I} \left[ s_i > \hat{Q}^{\mathrm{n+1}}_{1-\alpha} \right] = \lfloor\alpha(n+1) \rfloor
      \leq \alpha(n+1)\\
    \intertext{and therefore,}\\
    &\sum\limits_{i\in  \D_{\mathrm{inlier}} \cup \{ n+1\}} \mathbb{I}\left[ s_i > \hat{Q}^{\mathrm{n+1}}_{1-\alpha} \right]  \leq \alpha(n+1) - \sum\limits_{i\in  \D_{\mathrm{outlier}}} \mathbb{I}\left[ s_i > \hat{Q}^{\mathrm{n+1}}_{1-\alpha} \right]
  \end{align*}
\item (iii) Since $\hat{Q}^{\mathrm{cal}}_{1-\alpha} \geq \hat{Q}_{1-\alpha}^{n+1}$ almost surely, increasing the threshold (i.e., using $\hat{Q}^{\mathrm{cal}}_{1-\alpha}$) results in an equal or larger value of the sum.
\end{itemize}


Now, we can derive an upper bound for $\p \left(\hat{p}_{n+1} \leq \alpha\right)$ as follows:
\begin{align*}
\p \left(\hat{p}_{n+1} \leq \alpha\right) &= 
\p \left( S_{n+1} > \hat{Q}^{\mathrm{cal}}_{1-\alpha} \right)\\
&= \E\left[ \p \left( S_{n+1} > \hat{Q}^{\mathrm{cal}}_{1-\alpha} \mid E_{\mathrm{in}}, E_{\mathrm{out}}\right) \right]\\
    &\leq \E\left[\alpha - \frac{n_1}{n_0 + 1} \left( 1-\alpha - \hat{F}_{1} \left( \hat{Q}^{\mathrm{cal}}_{1-\alpha} \right)\ \right)\right]\\
    &= \alpha -\frac{n_1}{n_0 + 1} \left( 1-\alpha -   \E\left[ \hat{F}_{1} \left( \hat{Q}^{\mathrm{cal}}_{1-\alpha} \right) \right]\right),
\end{align*}
with this expectation being taken over different realizations of the inlier and outlier nonconformity scores.
\end{proof}

\subsubsection{An Alternative View Based on Mixture Distributions}

Next, we provide an additional theoretical result concerning the conservativeness of conformal outlier detection methods, to supplement the result presented in Section~\ref{sec:conservativeness} from a point of view closer to that of \citet{sesia2023adaptive}. 

Specifically, we consider a contaminated calibration set, $\D_{\mathrm{cal}}$, which may include both inliers (samples i.i.d. from $\p_0$) and outliers (samples i.i.d. from $\p_1 \neq \p_0$). The goal remains to test the null hypothesis $\mathcal{H}_0$ that a new data point $X_{n+1}$ is an inlier, independently sampled from $\p_0$.

This setup differs from \eqref{eq:setup-contaminated} in that the calibration set is drawn from a mixed distribution, where the proportion of outliers in the population is denoted by $\delta\in [0,1)$. Hence, the numbers of inliers and outliers in the calibration set are random, rather than fixed. Formally, this setup is expressed as:

\begin{align} \label{eq:setup-contaminated-random} 
\begin{split}
& X_i \overset{\text{i.i.d.}}{\sim} \p_{\mathrm{mixed}} = (1-\delta)\cdot \p_0 + \delta \cdot P_1, \quad \forall i \in \D_{\mathrm{cal}},\\ 
& \mathcal{H}_0 : X_{n+1} \overset{\text{ind.}}{\sim} \p_0.  
\end{split}\end{align}

Let $F_0$ and $F_1$ denote the CDFs of $\p_0$ and $\p_1$, respectively, and $\hat{Q}_{1-\alpha}^{\mathrm{cal}}$ represent the $\lceil (1-\alpha)(n+1)\rceil$-th smallest score in the calibration set.

\begin{corollary}[Conservativeness]
\label{cor:conservativeness-p-values}
    Under the setup defined in~\eqref{eq:setup-contaminated-random}, if $\mathcal{H}_0$ is true, then, for any $\alpha\in(0,1)$,
    \begin{align*}
    \p &\left( \hat{p}_{n+1}\leq \alpha \right) \leq \alpha - \delta \E \left[ F_0(\hat{Q}_{1-\alpha}^{\mathrm{cal}}) - F_1(\hat{Q}_{1-\alpha}^{\mathrm{cal}} )\right].
    \end{align*}
\end{corollary}

\Cref{cor:conservativeness-p-values} reformulates Theorem 1 in \citet{sesia2023adaptive} under the setup in \eqref{eq:setup-contaminated-random}. This result quantifies the behavior of conformal outlier detection methods in the presence of contaminated data and establishes guarantees on the type-I error rate. 
This result complements our analysis of the conservativeness of these methods.

\begin{proof}[Proof of~\Cref{cor:conservativeness-p-values}]
The proof adapts the argument of Theorem 1 in \citet{sesia2023adaptive} to the outlier detection setting considered here. Specifically, we follow the structure of the original proof, making adjustments to account for the presence of inliers and outliers in the calibration set.

By~\Cref{app-prop:p-value-to-quantile}, we have
\begin{align*}
    \hat{p}_{n+1} \leq \alpha \Longleftrightarrow S_{n+1} > \hat{Q}_{1-\alpha}^{\mathrm{cal}}.
\end{align*}

Under the null, we upper bound $\p \left( \hat{p}_{n+1}\leq \alpha\right)$ as follows:
    \begin{align*}
        \p_0 \left( \hat{p}_{n+1} \leq \alpha \right) &= \p_0 \left( \hat{p}_{n+1} \leq \alpha\right) + \p_{\mathrm{mixed}} \left( \hat{p}_{n+1} \leq \alpha\right) - \p_{\mathrm{mixed}} \left( \hat{p}_{n+1} \leq \alpha\right) \\
        &= \p_{\mathrm{mixed}} \left( \hat{p}_{n+1} \leq \alpha\right) - \left[ \p_{\mathrm{mixed}} \left( \hat{p}_{n+1} \leq \alpha\right) - \p_0 \left( \hat{p}_{n+1} \leq \alpha\right)\right]\\
        &\leq \alpha - \left[ \p_{\mathrm{mixed}} \left( \hat{p}_{n+1} \leq \alpha\right) - \p_0 \left( \hat{p}_{n+1} \leq \alpha\right)\right]\\
        &= \alpha - \left[ \left( (1-\delta)\p_0 \left( \hat{p}_{n+1} \leq \alpha\right) + \delta\p_1 \left( \hat{p}_{n+1} \leq \alpha\right)\right) - \p_0 \left( \hat{p}_{n+1} \leq \alpha\right) \right]\\
        &= \alpha - \delta\left[ \p_1 \left( \hat{p}_{n+1} \leq \alpha\right) - \p_0 \left( \hat{p}_{n+1} \leq \alpha\right)\right]\\
        &= \alpha - \delta\left[ \p_1 \left( S_{n+1} > \hat{Q}_{1-\alpha}^{\mathrm{cal}}\right) - \p_0 \left( S_{n+1} > \hat{Q}_{1-\alpha}^{\mathrm{cal}}\right) \right]\\
        &= \alpha - \delta\left[ \p_0 \left( S_{n+1} \leq \hat{Q}_{1-\alpha}^{\mathrm{cal}}\right) - \p_1 \left( S_{n+1} \leq \hat{Q}_{1-\alpha}^{\mathrm{cal}}\right) \right]\\
        &= \alpha - \delta\E \left[ \p_0 \left( S_{n+1} \leq \hat{Q}_{1-\alpha}^{\mathrm{cal}} \mid \D_{\mathrm{cal}}\right) - \p_1 \left( S_{n+1} \leq \hat{Q}_{1-\alpha}^{\mathrm{cal}} \mid \D_{\mathrm{cal}}\right) \right]\\
        &= \alpha - \delta\E \left[ F_0 \left( \hat{Q}_{1-\alpha}^{\mathrm{cal}} \right) - F_1 \left( \hat{Q}_{1-\alpha}^{\mathrm{cal}} \right) \right].
    \end{align*}
\end{proof}


\subsection{Validity of the \texttt{Label-Trim} Method}

\subsubsection{Proof of~\Cref{thm:labeled-trim} | Main Steps}

\begin{proof}[Proof of~\Cref{thm:labeled-trim}]
\label{prf:labeled-trim}
As in the proof of~\Cref{lem:conservativeness}, define the random score $S_i := s(X_i)$ for all $i\in \D_{\mathrm{cal}} \cup \{n+1\}$. 
By~\Cref{app-prop:p-value-to-quantile}, for any fixed $\alpha \in (0,1)$, the probability of a type-I error, $\p \left( \hat{p}_{n+1}^{\mathrm{LT}} \leq \alpha \right)$, can be expressed as
\begin{align*}
    \p \left( \hat{p}_{n+1}^{\mathrm{LT}} \leq \alpha \right) = \p \left( S_{n+1} > \hat{Q}_{1-\alpha}^{\mathrm{LT}}\right).
\end{align*}

Consider the augmented set $\{S_i\}_{i\in \D_{\mathrm{cal}}^{\mathrm{LT}}}\cup \{ S_{n+1}\}$, which includes the test score $S_{n+1}$. Define $\hat{Q}_{1-\alpha}^{\mathrm{LT,n+1}}$ as follows:
\begin{align*}
\hat{Q}_{1-\alpha}^{\mathrm{LT,n+1}} &:= \hat{i}_{\mathrm{LT}} \text{-th smallest element in } \{S_i\}_{i\in \D_{\mathrm{cal}}^{\mathrm{LT}}}\cup \{S_{n+1}\}.
\end{align*}
By~\Cref{app-prop:quantiles}, it holds that $\hat{Q}_{1-\alpha}^{\mathrm{LT}} \geq \hat{Q}_{1-\alpha}^{\mathrm{LT, n+1}}$ almost surely. 

Now, consider an imaginary ``mirror" version of this method that applies the label-trim algorithm with two key differences:
\begin{itemize}
\item it uses a larger labeling budget, $\tilde{m}=m+1$;
\item it treats $\{S_i\}_{i=1}^{n+1}$ as the calibration set instead of $\{S_i\}_{i=1}^{n}$---that is, it includes the test point in the annotation process, preserving the exchangeability with the calibration inliers.
\end{itemize}
Let $\tilde{\D}_{\mathrm{cal}}^{\mathrm{LT}}\cup \{ n+1\}$ denote the indices of the trimmed augmented calibration set produced by the mirror procedure, let $\tilde{\D}_{\mathrm{labeled}}$ denote the indices of the corresponding labeled data points, and define $\tilde{\D}_{\mathrm{labeled}}^{\mathrm{inlier}}, \tilde{\D}_{\mathrm{labeled}}^{\mathrm{outlier}}$ as the corresponding subsets of inliers and outliers, respectively.
Under this mirror procedure, the empirical quantile $\hat{Q}_{1-\alpha}^{\mathrm{LT,n+1}}$ corresponds to
\begin{align*}
\tilde{Q}_{1-\alpha}^{\mathrm{LT,n+1}} &:= \tilde{i}_{\mathrm{LT}} \text{-th smallest element in } \{S_i\}_{i \in \tilde{\D}_{\mathrm{cal}}^{\mathrm{LT}}} \cup \{S_{n+1}\},
\end{align*}
where $\tilde{i}_{\mathrm{LT}} := \lceil (1- \alpha ) (\tilde{n}^{\mathrm{LT}}+1) \rceil$ and $\tilde{n}_{\mathrm{LT}} := |\tilde{\D}_{\mathrm{cal}}^{\mathrm{LT}}|$.

By construction of this mirror procedure, $\tilde{i}_{\mathrm{LT}} \leq \hat{i}_{\mathrm{LT}}$ almost surely, because $\tilde{n}^{\mathrm{LT}} \leq n^{\mathrm{LT}}$ almost surely and thus
\begin{align*}
    \tilde{i}_{\mathrm{LT}} 
    & = \left \lceil ( 1-\alpha ) (\tilde{n}^{\mathrm{LT}}+1) \right \rceil
     \leq \lceil (1- \alpha )  (n^{\mathrm{LT}}+1) \rceil
     = \hat{i}_{\mathrm{LT}}.
\end{align*}

Using the fact that $\tilde{i}_{\mathrm{LT}}  \leq \hat{i}_{\mathrm{LT}}$ almost surely, we prove in Appendix~\ref{app:proof-label-trim-a1} that, almost surely,
\begin{align}\label{app-eq:tilde-hat-relation}
\tilde{Q}_{1-\alpha}^{\mathrm{LT,n+1}} \leq \hat{Q}_{1-\alpha}^{\mathrm{LT,n+1}}.
\end{align}
Since we already knew that $\hat{Q}_{1-\alpha}^{\mathrm{LT}} \geq \hat{Q}_{1-\alpha}^{\mathrm{LT,n+1}}$, this implies:
\begin{align} \label{app-eq:tilde-hat-relation-b}
    \hat{Q}_{1-\alpha}^{\mathrm{LT}} \geq \hat{Q}_{1-\alpha}^{\mathrm{LT,n+1}} \geq \tilde{Q}_{1-\alpha}^{\mathrm{LT,n+1}}.
\end{align}
Therefore, the type-I error rate of the Label-Trim approach can be bounded from above by the type-I error rate of the mirror procedure, which can be studied with an approach similar to that of the proof of~\Cref{lem:conservativeness}.


Let $\tilde{\D}_{\mathrm{outlier}}^{\mathrm{LT}}$ denote the outlier indices remaining in $\tilde{\D}_{\mathrm{cal}}^{\mathrm{LT}}$, with $\tilde{n}_{1}^{\mathrm{LT}}=|\tilde{\D}_{\mathrm{outlier}}^{\mathrm{LT}}|$.
 As in the proof of~\Cref{lem:conservativeness}, define $E_{\mathrm{in}}$ and $E_{\mathrm{out}}$ as two unordered realizations of the inlier and outlier scores in $\{S_i\}_{i=1}^{n+1}$, respectively. Appendix~\ref{app:proof-label-trim-a0} proves that
    \begin{align} \label{app-eq:type-I-error-cond}
    \p \left( S_{n+1} > \hat{Q}_{1-\alpha}^{\mathrm{LT}} \mid E_{\mathrm{in}}, E_{\mathrm{out}}, \tilde{\D}_{\mathrm{labeled}}, \tilde{\D}_{\mathrm{labeled}}^{\mathrm{inlier}}\right)
        \leq \alpha + \frac{1}{n_0+1} - \frac{\hat{n}_1^{\mathrm{LT}}}{n_0+1} \left( (1-\alpha) - \hat{F}_{1}^{\mathrm{LT}}(\hat{Q}_{1-\alpha}^{\mathrm{LT}}) \right),
    \end{align}
from which it follows immediately, by marginalizing over $E_{\mathrm{in}}, E_{\mathrm{out}}, \tilde{\D}_{\mathrm{labeled}},$ and $\tilde{\D}_{\mathrm{labeled}}^{\mathrm{inlier}}$, that
 \begin{align*}
    \p \left( S_{n+1} > \hat{Q}_{1-\alpha}^{\mathrm{LT}} \right)
   & \leq \E\left[ \alpha + \frac{1}{n_0+1} - \frac{\hat{n}^{\mathrm{LT}}_{1}}{n_0+1} \left( (1-\alpha) - \hat{F}_{1}^{\mathrm{LT}} \left( \hat{Q}_{1-\alpha}^{\mathrm{LT}} \right) \right)\right].
    \end{align*}
\end{proof}


\subsubsection{Proof of~\Cref{thm:labeled-trim} | Proof of Equation~\eqref{app-eq:tilde-hat-relation}} \label{app:proof-label-trim-a1}

\begin{proof}
We prove~\eqref{app-eq:tilde-hat-relation} by analyzing two distinct cases, depending on whether $S_{n+1}$ is among the $m + 1$ largest scores or not.

Recall that $m \leq \alpha (n+1)$, by assumption, and $n^{\mathrm{LT}} \leq n$.
It is easy to see that this implies that, almost surely,
    \begin{align} \label{eq:app-C1}
        \hat{i}_{\mathrm{LT}} = \lceil (1-\alpha)(n^{\mathrm{LT}}+1)\rceil \leq n+1 - (m+1).
    \end{align}

The mirror procedure labels and potentially removes the largest $m+1$ scores out of the $n+1$ total scores. Thus, $\hat{Q}_{1-\alpha}^{\mathrm{LT,n+1}}$ is smaller than all the outliers removed during trimming, i.e., $\hat{Q}^{\mathrm{LT,n+1}}_{1-\alpha} < S_j$ for all $S_j \in \tilde{\D}_{\mathrm{labeled}}^{\mathrm{outlier}}$.


\begin{itemize}
\item Suppose $S_{n+1}$ is among the $m + 1$ largest scores in $\{S_i\}_{i=1}^{n+1}$. This case is illustrated below:\\
\begin{minipage}{.4\textwidth}
In this scenario, the mirror trimming approach additionally labels the test score $S_{n+1}$, which, under the null hypothesis, is an inlier and is thus not removed. As a result, both trimming procedures yield the same set; i.e.,
$\tilde{\D}_{\mathrm{cal}}^{\mathrm{LT}} = \D_{\mathrm{cal}}^{\mathrm{LT}}$. Consequently, $\hat{Q}_{1-\alpha}^{\mathrm{LT,n+1}} = \tilde{Q}_{1-\alpha}^{\mathrm{LT,n+1}}$.
\end{minipage}
\begin{minipage}{.05\textwidth}
\hspace{1pt}
\end{minipage}
\begin{minipage}{.5\textwidth}
\centering
\includegraphics[width=\linewidth]{figures/theorem3.1_proof/idx-thm3.1-case-1.pdf}
\end{minipage}

\item Suppose $S_{n+1}$ is not among the $m + 1$ largest scores in $\{S_i\}_{i=1}^{n+1}$. Within this case, there are two sub-cases to consider.
\begin{itemize}
    \item  The $(m+1)$-th largest score in an inlier:\\
    \begin{minipage}[t]{.4\textwidth}
In this case, the mirror trimming approach additionally labels an inlier, which is not removed. As a result, both trimming procedures yield the same set; i.e.,
$\tilde{\D}_{\mathrm{cal}}^{\mathrm{LT}} = \D_{\mathrm{cal}}^{\mathrm{LT}}$. Consequently, $\hat{Q}_{1-\alpha}^{\mathrm{LT,n+1}} = \tilde{Q}_{1-\alpha}^{\mathrm{LT,n+1}}$.
\end{minipage}
\begin{minipage}[t]{0.05\textwidth}
    \hspace{1pt}
\end{minipage}
\begin{minipage}[t]{.5\textwidth}
\vspace{-40pt}
\centering
\includegraphics[width=\linewidth, valign=t]{figures/theorem3.1_proof/idx-thm3.1-case-2-1.pdf}
\end{minipage}
    \item The $(m+1)$-th largest score in an outlier:\\
    \begin{minipage}[t]{.4\textwidth}
This is the interesting case where $\tilde{n}^{\mathrm{LT}} = n^{\mathrm{LT}}-1$ and the set $\tilde{\D}_{\mathrm{cal}}^{\mathrm{LT}}$ contains one fewer outlier score than $\D_{\mathrm{cal}}^{\mathrm{LT}}$. 
It follows from~\eqref{eq:app-C1} that the threshold $\hat{Q}_{1-\alpha}^{\mathrm{LT,n+1}}$ is smaller than the $m+1$ largest scores. Then, in this region,  $\D_{\mathrm{cal}}^{\mathrm{LT}}$ and $\tilde{\D}_{\mathrm{cal}}^{\mathrm{LT}}$ contain the same scores. Therefore, the $\hat{i}_{\mathrm{LT}}$-th smallest score corresponds to the same score in both $\D_{\mathrm{cal}}^{\mathrm{LT}}\cup\{n+1\}$ and $\tilde{\D}_{\mathrm{cal}}^{\mathrm{LT}} \cup \{n+1\}$.
Since $\tilde{i}_{\mathrm{LT}} \leq \hat{i}_{\mathrm{LT}}$, it follows that $\tilde{Q}_{1-\alpha}^{\mathrm{LT,n+1}} \leq \hat{Q}_{1-\alpha}^{\mathrm{LT,n+1}}$.
\end{minipage}
\begin{minipage}[t]{0.05\textwidth}
    \hspace{1pt}
\end{minipage}
\begin{minipage}[t]{.5\textwidth}
\centering
\includegraphics[width=\linewidth, valign=t]{figures/theorem3.1_proof/idx-thm3.1-case-2-2.pdf}
\end{minipage}
\end{itemize}
\end{itemize}

\end{proof}


\subsubsection{Proof of~\Cref{thm:labeled-trim} | Proof of Equation~\eqref{app-eq:type-I-error-cond}} \label{app:proof-label-trim-a0}

\begin{proof}
    \begin{align*}
    & \p \left( S_{n+1} > \hat{Q}_{1-\alpha}^{\mathrm{LT}} \mid E_{\mathrm{in}}, E_{\mathrm{out}}, \tilde{\D}_{\mathrm{labeled}}, \tilde{\D}_{\mathrm{labeled}}^{\mathrm{inlier}}\right) \\
        & \qquad \leq \p \left( S_{n+1} > \tilde{Q}_{1-\alpha}^{\mathrm{LT,n+1}} \mid E_{\mathrm{in}}, E_{\mathrm{out}}, \tilde{\D}_{\mathrm{labeled}}, \tilde{\D}_{\mathrm{labeled}}^{\mathrm{inlier}}\right) \\
        & \qquad = \sum\limits_{i\in \D_{\mathrm{inlier}} \cup \{ n+1\} } \E \left[ \mathbb{I} \left[ s_i > \tilde{Q}_{1-\alpha}^{\mathrm{LT,n+1}} \right] \cdot \mathbb{I} \left[ S_{n+1} = s_i \right]\mid E_{\mathrm{in}}, E_{\mathrm{out}}, \tilde{\D}_{\mathrm{labeled}}, \tilde{\D}_{\mathrm{labeled}}^{\mathrm{inlier}}\right] \\
        & \qquad \overset{(i)}{=} \sum\limits_{i\in \D_{\mathrm{inlier}} \cup \{ n+1\} }         \mathbb{I} \left[ s_i > \tilde{Q}_{1-\alpha}^{\mathrm{LT,n+1}} \right] \cdot \p \left( S_{n+1} = s_i \mid E_{\mathrm{in}}, E_{\mathrm{out}}, \tilde{\D}_{\mathrm{labeled}}, \tilde{\D}_{\mathrm{labeled}}^{\mathrm{inlier}}\right) \\
        & \qquad \overset{(ii)}{=} \sum\limits_{i\in \D_{\mathrm{inlier}} \cup \{ n+1\} }  \mathbb{I} \left[ s_i > \tilde{Q}_{1-\alpha}^{\mathrm{LT,n+1}}\right] \cdot \p \left( S_{n+1} = s_i \mid E_{\mathrm{in}}, E_{\mathrm{out}} \right) 
        \\
        & \qquad = \frac{1}{n_0+1}  \sum\limits_{i\in  \D_{\mathrm{inlier}} \cup \{ n+1\}} \mathbb{I} \left[ s_i > \tilde{Q}_{1-\alpha}^{\mathrm{LT,n+1}}\right] \\
        & \qquad \overset{(iii)}{\leq}  \frac{1}{n_0+1} \left( \alpha(\tilde{n}^{\mathrm{LT}}+1) - \sum\limits_{i\in  \tilde{\D}_{\mathrm{outlier}}^{\mathrm{LT}}} \mathbb{I} \left[ s_i > \tilde{Q}_{1-\alpha}^{\mathrm{LT,n+1}}\right] \right)\\
        & \qquad \overset{(iv)}{\leq}  \frac{1}{n_0+1} \left( \alpha(n^{\mathrm{LT}}+1) - \sum\limits_{i\in \D_{\mathrm{outlier}}^{\mathrm{LT}}} \mathbb{I} \left[ s_i > \tilde{Q}_{1-\alpha}^{\mathrm{LT,n+1}}\right] + 1\right)\\
        & \qquad \overset{(v)}{\leq}  \frac{1}{n_0+1} \left( \alpha(n^{\mathrm{LT}}+1) - \sum\limits_{i\in \D_{\mathrm{outlier}}^{\mathrm{LT}}} \mathbb{I} \left[ s_i > \hat{Q}_{1-\alpha}^{\mathrm{LT}}\right] + 1\right)\\
        & \qquad = \frac{1}{n_0+1} \left( \alpha(n^{\mathrm{LT}}+1) - \hat{n}_1^{\mathrm{LT}} + \sum\limits_{i\in \D_{\mathrm{outlier}}^{\mathrm{LT}}} \mathbb{I} \left[ s_i \leq \hat{Q}_{1-\alpha}^{\mathrm{LT}}\right] + 1\right)\\
        & \qquad = \alpha + \frac{1}{n_0+1} - \frac{\hat{n}_1^{\mathrm{LT}}}{n_0+1} \left( (1-\alpha) - \hat{F}_{1}^{\mathrm{LT}}(\hat{Q}_{1-\alpha}^{\mathrm{LT}}) \right).
    \end{align*}


The labeled steps above can be explained as follows.
\begin{itemize}
\item     (i) $\mathbb{I}\left[ s_i > \tilde{Q}_{1-\alpha}^{\mathrm{LT,n+1}}\right]$ is measurable with respect to the $\sigma$-algebra generated by $E_{\mathrm{in}}, E_{\mathrm{out}}, \tilde{\D}_{\mathrm{labeled}},$ and $\tilde{\D}_{\mathrm{labeled}}^{\mathrm{inlier}}$ since $\tilde{Q}_{1-\alpha}^{\mathrm{LT,n+1}}$ is the $\tilde{i}_{\mathrm{LT}}$-th smallest element of $\{s_1,\dots,s_{n+1}\} \setminus \left(\tilde{\D}_{\mathrm{labeled}} \setminus \tilde{\D}_{\mathrm{labeled}}^{\mathrm{inlier}}\right)$.
\item    (ii) The mirror procedure is applied on $\{S_i\}_{i=1}^{n+1}$, preserving the exchangeability of the test score $S_{n+1}$ with the calibration inliers. Hence, the resulting labeled sets $\tilde{\D}_{\mathrm{labeled}}$ and $\tilde{\D}_{\mathrm{labeled}}^{\mathrm{inlier}}$ contain no additional information about $S_{n+1}$ beyond $E_{\mathrm{in}}, E_{\mathrm{out}}$.
\item    (iii) By definition, $\tilde{Q}_{1-\alpha}^{\mathrm{LT,n+1}}$ is the $\tilde{i}_{\mathrm{LT}}$-th smallest element of $\{S_i\}_{i\in \tilde{\D}_{\mathrm{cal}}^{\mathrm{LT}}} \cup \{S_{n+1}\}$, where $\tilde{i}_{\mathrm{LT}}=\lceil(1-\alpha)(\tilde{n}^{\mathrm{LT}}+1)\rceil$. Consequently, $\lfloor\alpha(\tilde{n}^{\mathrm{LT}}+1)\rfloor$ scores in $\{S_i\}_{i\in \tilde{\D}_{\mathrm{cal}}^{\mathrm{LT}}} \cup \{S_{n+1}\}$ are larger than $\tilde{Q}_{1-\alpha}^{\mathrm{LT,n+1}}$.
\item (iv) The set $\tilde{\D}_{\mathrm{outlier}}^{\mathrm{LT}}$ is either equal to $\D_{\mathrm{outlier}}^{\mathrm{LT}}$ or contains one fewer outlier, and $\tilde{n}^{\mathrm{LT}} \in \{n^{\mathrm{LT}}, n^{\mathrm{LT}}-1\}$ almost surely.
\item  (v) Recall from~\eqref{app-eq:tilde-hat-relation-b} that $\hat{Q}_{1-\alpha}^{\mathrm{LT}} \geq \tilde{Q}_{1-\alpha}^{\mathrm{LT,n+1}}$ almost surely.
\end{itemize}

\end{proof}



\section{Supplementary Experiments and Implementation Details}
\subsection{Datasets}\label{app-sec:data}
\Cref{app-tab:tabular-info} summarizes details of the three tabular benchmark datasets. For all tabular datasets, we perform random subsampling to construct contaminated train, calibration, and test sets. Specifically, for the shuttle and KDDCup99 datasets, the train set contains 5,000 samples, and the calibration set contains 2,500 samples, both with a contamination rate of $r=3\%$, unless stated otherwise. The inlier and outlier test sets consist of 950 and 50 samples, respectively. For the credit-card dataset, the train set contains 2,000 samples, while the calibration and test sets follow the same setup as the Shuttle and KDDCup99 datasets.
\begin{table}[ht]
\centering
\caption{Summary of tabular datasets}
\label{app-tab:tabular-info}
\begin{tabular}{lccc}
\toprule
Dataset              & Shuttle \citep{shuttle} & Credit-card \citep{creditcard} & KDDCup99 \citep{KDDCup99} \\ 
\midrule
Total Samples        & 58,000           & 284,807              & 494,020           \\ 
Number of Outliers   & 12,414           & 492                  & 396,743           \\ 
Number of Features   & 9                & 29                   & 41                \\ 
\bottomrule
\end{tabular}
\end{table}

For visual datasets, we use the OpenOOD benchmark \citep{zhang2023openood, yang2022openood}. Specifically, for each dataset and contamination rate, we perform a one-time training of the ReAct outlier detection model~\citep{react}, which operates on feature representations extracted from a pre-trained ResNet-18~\citep{zhang2023openood,he2016deep}. The model applies a percentile-based threshold (set to 90\%) to truncate activations, where the threshold is computed on the contaminated train set. These truncated activations then pass through the fully connected layer of the pre-trained ResNet-18.
The outlier score is computed using an energy-based log-sum-exp function applied to these truncated activations. 
After training, we save the outlier scores for the remaining outlier samples and the CIFAR-10 test set. We randomly subsample this pool of scores to construct the calibration and test sets, ensuring that all sets are disjoint.

The sizes of the train and calibration sets are 2,000 and 3,000, respectively, with the same contamination rate.
The inlier and outlier test sets consist of 950 and 50 samples, respectively.
\FloatBarrier

\subsection{Tabular Datasets}
\label{app-sec:real-data-exp}
In this section, we provide additional real-data experiments conducted on the {\em credit card} \citep{creditcard} and {\em KDDCup99} datasets,
complementing the analysis provided for the shuttle dataset in the main manuscript. Each figure in this section corresponds to and extends the figures presented in the main text.

\paragraph{Results as a function of the contamination rate} In~\Cref{fig:shuttle-outlier-prop} of the main manuscript, we analyze the performance of conformal outlier methods as a function of the contamination rate $r$. Here, we repeat the same experiment on the credit-card and KDDCup99 datasets (Figures \ref{app-fig:creditcard-outlier-prop} and \ref{app-fig:KDDCup99-outlier-prop}). The performance trends are similar to the one presented in the main manuscript: both the \texttt{Standard} and \texttt{Small-Clean} methods achieve valid type-I error rate but exhibit conservative behavior. In contrast, the \texttt{Naive-Trim} method fails to control the type-I error rate. The \texttt{Label-Trim} method attains improved power while practically controlling the type-I error at level $\alpha$.

\begin{figure*}[htb]
    % \centering 
    \includegraphics[height=3.3cm, valign=t]{figures/exp/real_data/creditcard/outlier_prop/IF_e_100_s_auto_train_2000_exp_outliers_calib_creditcard_1_fdr_model_0.5_initial_50_cal_2500_p_0.05_test_1000_p_0.05_q_0.02/Type-1-Error_point_no_legend.pdf}
    \includegraphics[height=3.3cm, valign=t]{figures/exp/real_data/creditcard/outlier_prop/IF_e_100_s_auto_train_2000_exp_outliers_calib_creditcard_1_fdr_model_0.5_initial_50_cal_2500_p_0.05_test_1000_p_0.05_q_0.02/Power_point_no_legend.pdf}
    \includegraphics[height=3.3cm, valign=t]{figures/exp/real_data/creditcard/outlier_prop/IF_e_100_s_auto_train_2000_exp_outliers_calib_creditcard_1_fdr_model_0.5_initial_50_cal_2500_p_0.05_test_1000_p_0.05_q_0.02/Trimmed_point_no_legend.pdf}
    \includegraphics[width=3.3cm, valign=t]{figures/exp/legend.pdf}
    \caption{Comparison of conformal outlier detection methods on real dataset ``credit-card'' as a function of the contamination rate  $r$. Other details are as in \Cref{fig:shuttle-outlier-prop}. 
}
    \label{app-fig:creditcard-outlier-prop}
\end{figure*}

\begin{figure*}[htb]
    % \centering 
    \includegraphics[height=3.3cm, valign=t]{figures/exp/real_data/KDDCup99/outlier_prop/IF_e_100_s_auto_train_5000_exp_outliers_calib_KDDCup99_1_fdr_model_0.5_initial_50_cal_2500_p_0.05_test_1000_p_0.05_q_0.02/Type-1-Error_point_no_legend.pdf}
    \includegraphics[height=3.3cm, valign=t]{figures/exp/real_data/KDDCup99/outlier_prop/IF_e_100_s_auto_train_5000_exp_outliers_calib_KDDCup99_1_fdr_model_0.5_initial_50_cal_2500_p_0.05_test_1000_p_0.05_q_0.02/Power_point_no_legend.pdf}
    \includegraphics[height=3.3cm, valign=t]{figures/exp/real_data/KDDCup99/outlier_prop/IF_e_100_s_auto_train_5000_exp_outliers_calib_KDDCup99_1_fdr_model_0.5_initial_50_cal_2500_p_0.05_test_1000_p_0.05_q_0.02/Trimmed_point_no_legend.pdf}
    \includegraphics[width=3.3cm, valign=t]{figures/exp/legend.pdf}
    \caption{Comparison of conformal outlier detection methods on real dataset ``KDDCup99'' as a function of the contamination rate  $r$. Other details are as in \Cref{fig:shuttle-outlier-prop}. 
}
    \label{app-fig:KDDCup99-outlier-prop}
\end{figure*}

\paragraph{Results as a function of the labeling budget} In~\Cref{fig:shuttle-labeled-exp} of the main manuscript, we evaluate the performance of the \texttt{Label-Trim}, \texttt{Small-Clean} and, \texttt{Oracle} methods as a function of the labeling budget $m$. \Cref{app-fig:creditcard-labeled-exp,app-fig:KDDCup99-labeled-exp} extend this analysis to the credit-card and KDDCup99 datasets, respectively. Consistent with the trends observed in~\Cref{fig:shuttle-labeled-exp}, increasing the labeling budget improves the performance of the \texttt{Label-Trim} method in terms of both type-I error and power. Notably, although the condition in \Cref{thm:labeled-trim} is no longer satisfied for $m > 50$, the \texttt{Label-Trim} method still maintains valid type-I error control at the desired level $\alpha$. 

For the KDDCup99 dataset, however, we observe that the \texttt{Small-Clean} method shows higher power than the \texttt{Oracle} across several labeling budgets ($m \geq 55$). This is due to the high variability in the dataset and the small sample size used by this method, resulting in significant variance in type-I error. To illustrate this, \Cref{app-fig:KDDCup99-labeled-exp-55-105} presents a box plot showing the variability and instability of the \texttt{Small-Clean} method in this regime. While this leads to a higher average power, this variability is undesirable in practice.

\begin{figure*}[!htb]
    % \centering 
    \includegraphics[height=3.3cm, valign=t]{figures/exp/real_data/creditcard/labeled_size/IF_e_100_s_auto_train_2000_exp_clean_calib_size_creditcard_1_fdr_model_0.5_initial_50_cal_2500_p_0.03_test_1000_p_0.05_q_0.02/Type-1-Error_point_no_legend.pdf}
    \includegraphics[height=3.3cm, valign=t]{figures/exp/real_data/creditcard/labeled_size/IF_e_100_s_auto_train_2000_exp_clean_calib_size_creditcard_1_fdr_model_0.5_initial_50_cal_2500_p_0.03_test_1000_p_0.05_q_0.02/Power_point_no_legend.pdf}
    \includegraphics[height=3.3cm, valign=t]{figures/exp/real_data/creditcard/labeled_size/IF_e_100_s_auto_train_2000_exp_clean_calib_size_creditcard_1_fdr_model_0.5_initial_50_cal_2500_p_0.03_test_1000_p_0.05_q_0.02/Trimmed_point_no_legend.pdf}
    \includegraphics[width=3.3cm, valign=t]{figures/exp/legend_wo_n.pdf}
    \caption{Performance on real dataset ``credit-card'' as a function of the labeling budget $m$. The contamination rate is $r=0.03$. Other details are as in \Cref{fig:shuttle-outlier-prop}.
}
    \label{app-fig:creditcard-labeled-exp}
\end{figure*}

\begin{figure*}[!htb]
    % \centering 
    \includegraphics[height=3.3cm, valign=t]{figures/exp/real_data/KDDCup99/labeled_size/IF_e_100_s_auto_train_5000_exp_clean_calib_size_KDDCup99_1_fdr_model_0.5_initial_50_cal_2500_p_0.03_test_1000_p_0.05_q_0.02_n_seeds_400/Type-1-Error_point_no_legend.pdf}
    \includegraphics[height=3.3cm, valign=t]{figures/exp/real_data/KDDCup99/labeled_size/IF_e_100_s_auto_train_5000_exp_clean_calib_size_KDDCup99_1_fdr_model_0.5_initial_50_cal_2500_p_0.03_test_1000_p_0.05_q_0.02_n_seeds_400/Power_point_no_legend.pdf}
    \includegraphics[height=3.3cm, valign=t]{figures/exp/real_data/KDDCup99/labeled_size/IF_e_100_s_auto_train_5000_exp_clean_calib_size_KDDCup99_1_fdr_model_0.5_initial_50_cal_2500_p_0.03_test_1000_p_0.05_q_0.02_n_seeds_400/Trimmed_point_no_legend.pdf}
    \includegraphics[width=3.3cm, valign=t]{figures/exp/legend_wo_n.pdf}
    \caption{Performance on real dataset ``KDDCup99'' as a function of the labeling budget $m$. The contamination rate is $r=0.03$. Results are averages across 400 random splits of the data. Other details are as in \Cref{fig:shuttle-outlier-prop}.
}
    \label{app-fig:KDDCup99-labeled-exp}
\end{figure*}

\begin{figure*}[!htb]
    \centering 
    \includegraphics[width=0.33\textwidth, valign=t]{figures/exp/real_data/KDDCup99/labeled_size/IF_e_100_s_auto_train_5000_exp_clean_calib_size_KDDCup99_1_fdr_model_0.5_initial_50_cal_2500_p_0.03_test_1000_p_0.05_q_0.02_n_seeds_400/55-105/Type-1-Error_no_legend.pdf}
    \includegraphics[width=0.33\textwidth, valign=t]{figures/exp/real_data/KDDCup99/labeled_size/IF_e_100_s_auto_train_5000_exp_clean_calib_size_KDDCup99_1_fdr_model_0.5_initial_50_cal_2500_p_0.03_test_1000_p_0.05_q_0.02_n_seeds_400/55-105/Power_no_legend.pdf}
    \includegraphics[width=3.3cm, valign=t]{figures/exp/legend_wo_n_box.pdf}
    \caption{Performance on real dataset ``KDDCup99'' as a function of the labeling budget $m$. The contamination rate is $r=0.03$. Results are averages across 400 random splits of the data. Other details are as in \Cref{fig:shuttle-outlier-prop}.
}
    \label{app-fig:KDDCup99-labeled-exp-55-105}
\end{figure*}

\paragraph{Results as a function of the target type-I error level} In~\Cref{fig:shuttle-levels} we examine the performance of conformal outlier detection methods as a function of the target type-I error level $\alpha$. Here, we replicate the experiments on the credit-card and KDDCup99 datasets (\Cref{app-fig:creditcard-levels,app-fig:KDDCup99-levels}). In line with the trends observed in~\Cref{fig:shuttle-levels}, the \texttt{Label-Trim} method performs well at low type-I error rates. Notably, for $\alpha=0.01$, the \texttt{Label-Trim} method outperforms the baselines, while practically controlling the type-I error at level $\alpha$, even though the condition of~\Cref{thm:labeled-trim} is not satisfied in this case. This highlights the robustness of our approach.

\begin{figure*}[!htb]
    \centering 
    \includegraphics[height=3.3cm, valign=t]{figures/exp/real_data/creditcard/levels/IF_e_100_s_auto_train_2000_exp_levels_creditcard_1_fdr_model_0.5_initial_50_cal_2500_p_0.03_test_1000_p_0.05_q_0.02/Type-1-Error_point_no_legend.pdf}
    \includegraphics[height=3.3cm, valign=t]{figures/exp/real_data/creditcard/levels/IF_e_100_s_auto_train_2000_exp_levels_creditcard_1_fdr_model_0.5_initial_50_cal_2500_p_0.03_test_1000_p_0.05_q_0.02/Power_point_no_legend.pdf}
    \includegraphics[width=3.3cm, valign=t]{figures/exp/legend_wo_trm.pdf}
    \caption{Comparison of conformal outlier detection methods on real dataset ``credit-card'' as a function of the target type-I error rate $\alpha$. The contamination rate $r$ is fixed to 3\%; other details are as in \Cref{fig:shuttle-outlier-prop}.
}
    \label{app-fig:creditcard-levels}
\end{figure*}

\begin{figure*}[!htb]
    \centering 
    \includegraphics[height=3.3cm, valign=t]{figures/exp/real_data/KDDCup99/levels/IF_e_100_s_auto_train_5000_exp_levels_KDDCup99_1_fdr_model_0.5_initial_50_cal_2500_p_0.03_test_1000_p_0.05_q_0.02/Type-1-Error_point_no_legend.pdf}
    \includegraphics[height=3.3cm, valign=t]{figures/exp/real_data/KDDCup99/levels/IF_e_100_s_auto_train_5000_exp_levels_KDDCup99_1_fdr_model_0.5_initial_50_cal_2500_p_0.03_test_1000_p_0.05_q_0.02/Power_point_no_legend.pdf}
    \includegraphics[width=3.3cm, valign=t]{figures/exp/legend_wo_trm.pdf}
    \caption{Comparison of conformal outlier detection methods on real dataset ``KDDCup99'' as a function of the target type-I error rate $\alpha$. The contamination rate $r$ is fixed to 3\%; other details are as in \Cref{fig:shuttle-outlier-prop}.
}
    \label{app-fig:KDDCup99-levels}
\end{figure*}

% \clearpage
\FloatBarrier

\subsection{Visual Datasets}\label{app-sec:images-data-exp}
This section provides detailed results for each dataset, covering various contamination rates and target type-I error levels.

\Cref{app-tab:all} summarizes the results across all six datasets for a target type-I error rate of $\alpha = 0.02$, showing trends consistent with those observed in \Cref{tab:avg-images} of the main manuscript.

\Cref{app-tab:texture,app-tab:svhn,app-tab:places365,app-tab:mnist,app-tab:cifar100,app-tab:tin} present the type-I error rates and the power of each method, with the power reported relative to the \texttt{Standard} method (normalized to 1). The actual power of the \texttt{Standard} method is included in the last row of each table.

\begin{table}[!htb]
\caption{Comparison of conformal outlier detection methods on six visual datasets for varying contamination rate $r$ and target type-I error level $\alpha = 0.02$. The empirical type-I error values are averaged across all datasets. The empirical power is presented relative to the \texttt{Standard} method (higher is better), and averaged across all datasets. Results are averaged across 100 random splits of the data, with standard errors presented in parentheses.}
\centering
\label{app-tab:all}

\resizebox{\textwidth}{!}{
\begin{tabular}{l|ll|ll|ll}
\hline
& \multicolumn{6}{c}{Contamination rate} \\
\hline
 & \multicolumn{2}{c|}{1\%} & \multicolumn{2}{c|}{3\%} & \multicolumn{2}{c}{5\%} \\ \hline
 Method      & Power & Type-I Error & Power & Type-I Error & Power & Type-I Error \\ \hline
Standard & \bfseries \cellcolor{Green!30} 1.0 ($\pm$ 0.0204) & \cellcolor{white} 0.017 ($\pm$ 0.0004)  & \bfseries \cellcolor{Green!30} 1.0 ($\pm$ 0.0238) & \cellcolor{white} 0.012 ($\pm$ 0.0004)  & \bfseries \cellcolor{Green!30} 1.0 ($\pm$ 0.0272) & \cellcolor{white} 0.009 ($\pm$ 0.0003) \\

Oracle (infeasible) & \bfseries \cellcolor{Green!100} 1.096 ($\pm$ 0.0214) & 0.021 ($\pm$ 0.0005)  & \bfseries \cellcolor{Green!100} 1.33 ($\pm$ 0.025) & \cellcolor{white} 0.02 ($\pm$ 0.0005)  & \bfseries \cellcolor{Green!100} 1.674 ($\pm$ 0.0311) & \cellcolor{white} 0.02 ($\pm$ 0.0006) \\

Naive-Trim (invalid) & \cellcolor{red!20} 1.249 ($\pm$ 0.0219) & \cellcolor{red!20} 0.026 ($\pm$ 0.0005)  & \cellcolor{red!20} 1.777 ($\pm$ 0.0254) & \cellcolor{red!20} 0.035 ($\pm$ 0.0006)  & \cellcolor{red!20} 2.452 ($\pm$ 0.0324) & \cellcolor{red!20} 0.044 ($\pm$ 0.0007) \\

Small-Clean & \cellcolor{white} 0.819 ($\pm$ 0.0592) & \cellcolor{white} 0.018 ($\pm$ 0.002)  & \cellcolor{white} 0.613 ($\pm$ 0.0753) & \cellcolor{white} 0.011 ($\pm$ 0.0018)  & \cellcolor{white} 0.406 ($\pm$ 0.0769) & \cellcolor{white} 0.006 ($\pm$ 0.0013) \\

Label-Trim & \bfseries \cellcolor{Green!60} 1.079 ($\pm$ 0.0212) & \cellcolor{white} 0.02 ($\pm$ 0.0005)  & \bfseries \cellcolor{Green!60} 1.23 ($\pm$ 0.0247) & \cellcolor{white} 0.017 ($\pm$ 0.0004)  & \bfseries \cellcolor{Green!60} 1.381 ($\pm$ 0.0298) & \cellcolor{white} 0.014 ($\pm$ 0.0005) \\
\end{tabular}
}
\subcaption{Target type-I error rate $\alpha=0.02$}
\end{table}


\begin{table}[!htb]
\caption{Comparison of conformal outlier detection methods on Texture dataset (outliers) and CIFAR-10 dataset (inliers) for varying contamination rate $r$ and target type-I error level $\alpha$. The empirical power is presented relative to the \texttt{Standard} method (higher is better). Results are averaged across 100 random splits of the data, with standard errors presented in parentheses.}
\label{app-tab:texture}
\centering
\resizebox{\textwidth}{!}{
\begin{tabular}{l|ll|ll|ll}
\hline
& \multicolumn{6}{c}{Contamination rate} \\
\hline
 & \multicolumn{2}{c|}{1\%} & \multicolumn{2}{c|}{3\%} & \multicolumn{2}{c}{5\%} \\ \hline
 Method      & Power & Type-I Error & Power & Type-I Error & Power & Type-I Error \\ \hline
Standard & \bfseries \cellcolor{Green!30} 1.0 ($\pm$ 0.0295) & \cellcolor{white} 0.008 ($\pm$ 0.0003)  & \bfseries \cellcolor{Green!30} 1.0 ($\pm$ 0.0369) & \cellcolor{white} 0.006 ($\pm$ 0.0003)  & \bfseries \cellcolor{Green!30} 1.0 ($\pm$ 0.0405) & \cellcolor{white} 0.004 ($\pm$ 0.0002) \\

Oracle (infeasible) & \bfseries \cellcolor{Green!100} 1.173 ($\pm$ 0.0295) & \cellcolor{white} 0.01 ($\pm$ 0.0003)  & \bfseries \cellcolor{Green!100} 1.455 ($\pm$ 0.0432) & \cellcolor{white} 0.01 ($\pm$ 0.0003)  & \bfseries \cellcolor{Green!100} 1.824 ($\pm$ 0.051) & \cellcolor{white} 0.009 ($\pm$ 0.0004) \\

Naive-Trim (invalid) & \cellcolor{red!20} 1.722 ($\pm$ 0.035) & \cellcolor{red!20} 0.017 ($\pm$ 0.0004)  & \cellcolor{red!20} 2.668 ($\pm$ 0.0428) & \cellcolor{red!20} 0.028 ($\pm$ 0.0006)  & \cellcolor{red!20} 4.008 ($\pm$ 0.0599) & \cellcolor{red!20} 0.037 ($\pm$ 0.0007) \\

Small-Clean & \cellcolor{white} 0.0 ($\pm$ 0.0) & \cellcolor{white} 0.0 ($\pm$ 0.0)  & \cellcolor{white} 0.0 ($\pm$ 0.0) & \cellcolor{white} 0.0 ($\pm$ 0.0)  & \cellcolor{white} 0.0 ($\pm$ 0.0) & \cellcolor{white} 0.0 ($\pm$ 0.0) \\

Label-Trim & \bfseries \cellcolor{Green!100} 1.173 ($\pm$ 0.0295) & \cellcolor{white} 0.01 ($\pm$ 0.0003)  & \bfseries \cellcolor{Green!60} 1.448 ($\pm$ 0.0428) & \cellcolor{white} 0.01 ($\pm$ 0.0003)  & \bfseries \cellcolor{Green!60} 1.73 ($\pm$ 0.0477) & \cellcolor{white} 0.009 ($\pm$ 0.0003) \\
\hline
Standard Power & \cellcolor{white} 0.194 ($\pm$ 0.0057) &  & \cellcolor{white} 0.159 ($\pm$ 0.0059) &  & \cellcolor{white} 0.123 ($\pm$ 0.005) & \\
\end{tabular}
}
\subcaption{Target type-I error rate $\alpha=0.01$}

\resizebox{\textwidth}{!}{
\begin{tabular}{l|ll|ll|ll}
\hline
& \multicolumn{6}{c}{Contamination rate} \\
\hline
 & \multicolumn{2}{c|}{1\%} & \multicolumn{2}{c|}{3\%} & \multicolumn{2}{c}{5\%} \\ \hline
 Method      & Power & Type-I Error & Power & Type-I Error & Power & Type-I Error \\ \hline
Standard & \bfseries \cellcolor{Green!30} 1.0 ($\pm$ 0.0202) & \cellcolor{white} 0.017 ($\pm$ 0.0005)  & \bfseries \cellcolor{Green!30} 1.0 ($\pm$ 0.0239) & \cellcolor{white} 0.012 ($\pm$ 0.0004)  & \bfseries \cellcolor{Green!30} 1.0 ($\pm$ 0.0264) & \cellcolor{white} 0.009 ($\pm$ 0.0003) \\

Oracle (infeasible) & \bfseries \cellcolor{Green!100} 1.109 ($\pm$ 0.0199) & 0.021 ($\pm$ 0.0005)  & \bfseries \cellcolor{Green!100} 1.315 ($\pm$ 0.0259) & \cellcolor{white} 0.02 ($\pm$ 0.0005)  & \bfseries \cellcolor{Green!100} 1.623 ($\pm$ 0.0346) & \cellcolor{white} 0.02 ($\pm$ 0.0006) \\

Naive-Trim (invalid) & \cellcolor{red!20} 1.24 ($\pm$ 0.0193) & \cellcolor{red!20} 0.026 ($\pm$ 0.0006)  & \cellcolor{red!20} 1.781 ($\pm$ 0.026) & \cellcolor{red!20} 0.035 ($\pm$ 0.0007)  & \cellcolor{red!20} 2.431 ($\pm$ 0.0334) & \cellcolor{red!20} 0.045 ($\pm$ 0.0007) \\

Small-Clean & \cellcolor{white} 0.819 ($\pm$ 0.0599) & \cellcolor{white} 0.018 ($\pm$ 0.002)  & \cellcolor{white} 0.702 ($\pm$ 0.0824) & \cellcolor{white} 0.014 ($\pm$ 0.0025)  & \cellcolor{white} 0.478 ($\pm$ 0.0842) & \cellcolor{white} 0.007 ($\pm$ 0.0016) \\

Label-Trim & \bfseries \cellcolor{Green!60} 1.089 ($\pm$ 0.0199) & \cellcolor{white} 0.02 ($\pm$ 0.0005)  & \bfseries \cellcolor{Green!60} 1.219 ($\pm$ 0.0253) & \cellcolor{white} 0.017 ($\pm$ 0.0004)  & \bfseries \cellcolor{Green!60} 1.353 ($\pm$ 0.0303) & \cellcolor{white} 0.014 ($\pm$ 0.0005) \\
\hline
Standard Power & \cellcolor{white} 0.337 ($\pm$ 0.0068) &  & \cellcolor{white} 0.272 ($\pm$ 0.0065) &   & \cellcolor{white} 0.224 ($\pm$ 0.0059) &  \\
\end{tabular}
}
\subcaption{Target type-I error rate $\alpha=0.02$}

\resizebox{\textwidth}{!}{
\begin{tabular}{l|ll|ll|ll}
\hline
& \multicolumn{6}{c}{Contamination rate} \\
\hline
 & \multicolumn{2}{c|}{1\%} & \multicolumn{2}{c|}{3\%} & \multicolumn{2}{c}{5\%} \\ \hline
 Method      & Power & Type-I Error & Power & Type-I Error & Power & Type-I Error \\ \hline
Standard & \bfseries \cellcolor{Green!30} 1.0 ($\pm$ 0.0155) & \cellcolor{white} 0.027 ($\pm$ 0.0006)  & \bfseries \cellcolor{Green!30} 1.0 ($\pm$ 0.0197) & \cellcolor{white} 0.02 ($\pm$ 0.0005)  & \cellcolor{white} 1.0 ($\pm$ 0.0215) & \cellcolor{white} 0.015 ($\pm$ 0.0005) \\

Oracle (infeasible) & \bfseries \cellcolor{Green!100} 1.068 ($\pm$ 0.0152) & \cellcolor{white} 0.03 ($\pm$ 0.0006)  & \bfseries \cellcolor{Green!100} 1.224 ($\pm$ 0.0191) & \cellcolor{white} 0.03 ($\pm$ 0.0006)  & \bfseries \cellcolor{Green!100} 1.427 ($\pm$ 0.0249) & \cellcolor{white} 0.03 ($\pm$ 0.0007) \\

Naive-Trim (invalid) & \cellcolor{red!20} 1.15 ($\pm$ 0.0152) & \cellcolor{red!20} 0.036 ($\pm$ 0.0006)  & \cellcolor{red!20} 1.514 ($\pm$ 0.0178) & \cellcolor{red!20} 0.043 ($\pm$ 0.0007)  & \cellcolor{red!20} 1.919 ($\pm$ 0.0229) & \cellcolor{red!20} 0.052 ($\pm$ 0.0008) \\

Small-Clean & \cellcolor{white} 0.743 ($\pm$ 0.0441) & \cellcolor{white} 0.02 ($\pm$ 0.002)  & \cellcolor{white} 0.883 ($\pm$ 0.0549) & \cellcolor{white} 0.021 ($\pm$ 0.0025)  & \bfseries \cellcolor{Green!30} 1.062 ($\pm$ 0.0665) & \cellcolor{white} 0.024 ($\pm$ 0.0027) \\

Label-Trim & \bfseries \cellcolor{Green!60} 1.046 ($\pm$ 0.0153) & \cellcolor{white} 0.029 ($\pm$ 0.0006)  & \bfseries \cellcolor{Green!60} 1.13 ($\pm$ 0.019) & \cellcolor{white} 0.025 ($\pm$ 0.0005)  & \bfseries \cellcolor{Green!60} 1.215 ($\pm$ 0.0245) & \cellcolor{white} 0.021 ($\pm$ 0.0005) \\
\hline
Standard Power & \cellcolor{white} 0.421 ($\pm$ 0.0065) &   & \cellcolor{white} 0.357 ($\pm$ 0.007) &   & \cellcolor{white} 0.309 ($\pm$ 0.0067) &  \\
\end{tabular}
}
\subcaption{Target type-I error rate $\alpha=0.03$}

\end{table}


\begin{table}[!htb]
\caption{Comparison of conformal outlier detection methods on SVHN dataset (outliers) and CIFAR-10 dataset (inliers) for varying contamination rate $r$ and target type-I error level $\alpha$. The empirical power is presented relative to the \texttt{Standard} method (higher is better). Results are averaged across 100 random splits of the data, with standard errors presented in parentheses.}
\label{app-tab:svhn}
\centering
\resizebox{\textwidth}{!}{
\begin{tabular}{l|ll|ll|ll}
\hline
& \multicolumn{6}{c}{Contamination rate} \\
\hline
 & \multicolumn{2}{c|}{1\%} & \multicolumn{2}{c|}{3\%} & \multicolumn{2}{c}{5\%} \\ \hline
 Method      & Power & Type-I Error & Power & Type-I Error & Power & Type-I Error \\ \hline
Standard & \bfseries \cellcolor{Green!30} 1.0 ($\pm$ 0.0255) & \cellcolor{white} 0.008 ($\pm$ 0.0003)  & \bfseries \cellcolor{Green!30} 1.0 ($\pm$ 0.0316) & \cellcolor{white} 0.004 ($\pm$ 0.0002)  & \bfseries \cellcolor{Green!30} 1.0 ($\pm$ 0.0388) & \cellcolor{white} 0.002 ($\pm$ 0.0002) \\

Oracle (infeasible) & \bfseries \cellcolor{Green!100} 1.192 ($\pm$ 0.0262) & \cellcolor{white} 0.01 ($\pm$ 0.0003)  & \bfseries \cellcolor{Green!100} 1.654 ($\pm$ 0.0368) & \cellcolor{white} 0.01 ($\pm$ 0.0003)  & \bfseries \cellcolor{Green!100} 2.208 ($\pm$ 0.052) & \cellcolor{white} 0.009 ($\pm$ 0.0004) \\

Naive-Trim (invalid) & \cellcolor{red!20} 1.573 ($\pm$ 0.0271) & \cellcolor{red!20} 0.016 ($\pm$ 0.0004)  & \cellcolor{red!20} 2.689 ($\pm$ 0.0353) & \cellcolor{red!20} 0.025 ($\pm$ 0.0006)  & \cellcolor{red!20} 4.073 ($\pm$ 0.0533) & \cellcolor{red!20} 0.033 ($\pm$ 0.0007) \\

Small-Clean & \cellcolor{white} 0.0 ($\pm$ 0.0) & \cellcolor{white} 0.0 ($\pm$ 0.0)  & \cellcolor{white} 0.0 ($\pm$ 0.0) & \cellcolor{white} 0.0 ($\pm$ 0.0)  & \cellcolor{white} 0.0 ($\pm$ 0.0) & \cellcolor{white} 0.0 ($\pm$ 0.0) \\

Label-Trim & \bfseries \cellcolor{Green!100} 1.192 ($\pm$ 0.0262) & \cellcolor{white} 0.01 ($\pm$ 0.0003)  & \bfseries \cellcolor{Green!60} 1.601 ($\pm$ 0.0367) & \cellcolor{white} 0.009 ($\pm$ 0.0003)  & \bfseries \cellcolor{Green!60} 1.895 ($\pm$ 0.0448) & \cellcolor{white} 0.007 ($\pm$ 0.0003) \\
\hline
Standard Power & \cellcolor{white} 0.271 ($\pm$ 0.0069) &  & \cellcolor{white} 0.191 ($\pm$ 0.006) &  & \cellcolor{white} 0.137 ($\pm$ 0.0053) & \\
\end{tabular}
}
\subcaption{Target type-I error rate $\alpha=0.01$}

\resizebox{\textwidth}{!}{
\begin{tabular}{l|ll|ll|ll}
\hline
& \multicolumn{6}{c}{Contamination rate} \\
\hline
 & \multicolumn{2}{c|}{1\%} & \multicolumn{2}{c|}{3\%} & \multicolumn{2}{c}{5\%} \\ \hline
 Method      & Power & Type-I Error & Power & Type-I Error & Power & Type-I Error \\ \hline
Standard & \bfseries \cellcolor{Green!30} 1.0 ($\pm$ 0.0168) & \cellcolor{white} 0.016 ($\pm$ 0.0004)  & \bfseries \cellcolor{Green!30} 1.0 ($\pm$ 0.0218) & \cellcolor{white} 0.01 ($\pm$ 0.0003)  & \bfseries \cellcolor{Green!30} 1.0 ($\pm$ 0.0245) & \cellcolor{white} 0.007 ($\pm$ 0.0003) \\

Oracle (infeasible) & \bfseries \cellcolor{Green!100} 1.096 ($\pm$ 0.0173) & 0.021 ($\pm$ 0.0005)  & \bfseries \cellcolor{Green!100} 1.404 ($\pm$ 0.0211) & \cellcolor{white} 0.02 ($\pm$ 0.0005)  & \bfseries \cellcolor{Green!100} 1.828 ($\pm$ 0.0304) & \cellcolor{white} 0.02 ($\pm$ 0.0006) \\

Naive-Trim (invalid) & \cellcolor{red!20} 1.2 ($\pm$ 0.0175) & \cellcolor{red!20} 0.026 ($\pm$ 0.0005)  & \cellcolor{red!20} 1.721 ($\pm$ 0.0216) & \cellcolor{red!20} 0.032 ($\pm$ 0.0006)  & \cellcolor{red!20} 2.42 ($\pm$ 0.0281) & \cellcolor{red!20} 0.041 ($\pm$ 0.0007) \\

Small-Clean & \cellcolor{white} 0.833 ($\pm$ 0.053) & \cellcolor{white} 0.02 ($\pm$ 0.0026)  & \cellcolor{white} 0.598 ($\pm$ 0.0754) & \cellcolor{white} 0.01 ($\pm$ 0.0016)  & \cellcolor{white} 0.509 ($\pm$ 0.0929) & \cellcolor{white} 0.008 ($\pm$ 0.0019) \\

Label-Trim & \bfseries \cellcolor{Green!60} 1.08 ($\pm$ 0.0176) & \cellcolor{white} 0.02 ($\pm$ 0.0005)  & \bfseries \cellcolor{Green!60} 1.277 ($\pm$ 0.0228) & \cellcolor{white} 0.016 ($\pm$ 0.0004)  & \bfseries \cellcolor{Green!60} 1.469 ($\pm$ 0.0288) & \cellcolor{white} 0.013 ($\pm$ 0.0004) \\
\hline
Standard Power & \cellcolor{white} 0.429 ($\pm$ 0.0072) &  & \cellcolor{white} 0.332 ($\pm$ 0.0072) &  & \cellcolor{white} 0.25 ($\pm$ 0.0061) &  \\
\end{tabular}
}
\subcaption{Target type-I error rate $\alpha=0.02$}

\resizebox{\textwidth}{!}{
\begin{tabular}{l|ll|ll|ll}
\hline
& \multicolumn{6}{c}{Contamination rate} \\
\hline
 & \multicolumn{2}{c|}{1\%} & \multicolumn{2}{c|}{3\%} & \multicolumn{2}{c}{5\%} \\ \hline
 Method      & Power & Type-I Error & Power & Type-I Error & Power & Type-I Error \\ \hline
Standard & \bfseries \cellcolor{Green!30} 1.0 ($\pm$ 0.0145) & \cellcolor{white} 0.026 ($\pm$ 0.0005)  & \bfseries \cellcolor{Green!30} 1.0 ($\pm$ 0.0169) & \cellcolor{white} 0.017 ($\pm$ 0.0004)  & \cellcolor{white} 1.0 ($\pm$ 0.0204) & \cellcolor{white} 0.012 ($\pm$ 0.0004) \\

Oracle (infeasible) & \bfseries \cellcolor{Green!100} 1.063 ($\pm$ 0.014) & \cellcolor{white} 0.03 ($\pm$ 0.0006)  & \bfseries \cellcolor{Green!100} 1.246 ($\pm$ 0.0158) & \cellcolor{white} 0.029 ($\pm$ 0.0006)  & \bfseries \cellcolor{Green!100} 1.513 ($\pm$ 0.0212) & \cellcolor{white} 0.03 ($\pm$ 0.0007) \\

Naive-Trim (invalid) & \cellcolor{red!20} 1.115 ($\pm$ 0.0137) & \cellcolor{red!20} 0.035 ($\pm$ 0.0006)  & \cellcolor{red!20} 1.395 ($\pm$ 0.0148) & \cellcolor{red!20} 0.041 ($\pm$ 0.0007)  & \cellcolor{red!20} 1.825 ($\pm$ 0.0199) & \cellcolor{red!20} 0.049 ($\pm$ 0.0008) \\

Small-Clean & \cellcolor{white} 0.73 ($\pm$ 0.0419) & \cellcolor{white} 0.021 ($\pm$ 0.0027)  & \cellcolor{white} 0.948 ($\pm$ 0.0447) & \cellcolor{white} 0.022 ($\pm$ 0.0019)  & \bfseries \cellcolor{Green!30} 1.092 ($\pm$ 0.0625) & \cellcolor{white} 0.022 ($\pm$ 0.0023) \\

Label-Trim & \bfseries \cellcolor{Green!60} 1.042 ($\pm$ 0.0143) & \cellcolor{white} 0.029 ($\pm$ 0.0006)  & \bfseries \cellcolor{Green!60} 1.144 ($\pm$ 0.0151) & \cellcolor{white} 0.024 ($\pm$ 0.0005)  & \bfseries \cellcolor{Green!60} 1.266 ($\pm$ 0.0209) & \cellcolor{white} 0.019 ($\pm$ 0.0006) \\
\hline
Standard Power & \cellcolor{white} 0.517 ($\pm$ 0.0075) &  & \cellcolor{white} 0.44 ($\pm$ 0.0074) &  & \cellcolor{white} 0.355 ($\pm$ 0.0072) &  \\
\end{tabular}
}
\subcaption{Target type-I error rate $\alpha=0.03$}

\end{table}

\begin{table}[!htb]
\caption{Comparison of conformal outlier detection methods on Places365 dataset (outliers) and CIFAR-10 dataset (inliers) for varying contamination rate $r$ and target type-I error level $\alpha$. The empirical power is presented relative to the \texttt{Standard} method (higher is better). Results are averaged across 100 random splits of the data, with standard errors presented in parentheses.}
\label{app-tab:places365}
\centering
\resizebox{\textwidth}{!}{
\begin{tabular}{l|ll|ll|ll}
\hline
& \multicolumn{6}{c}{Contamination rate} \\
\hline
 & \multicolumn{2}{c|}{1\%} & \multicolumn{2}{c|}{3\%} & \multicolumn{2}{c}{5\%} \\ \hline
 Method      & Power & Type-I Error & Power & Type-I Error & Power & Type-I Error \\ \hline
Standard & \bfseries \cellcolor{Green!30} 1.0 ($\pm$ 0.033) & \cellcolor{white} 0.009 ($\pm$ 0.0003)  & \bfseries \cellcolor{Green!30} 1.0 ($\pm$ 0.0364) & \cellcolor{white} 0.006 ($\pm$ 0.0003)  & \bfseries \cellcolor{Green!30} 1.0 ($\pm$ 0.0431) & \cellcolor{white} 0.004 ($\pm$ 0.0002) \\

Oracle (infeasible) & \bfseries \cellcolor{Green!100} 1.139 ($\pm$ 0.0339) & \cellcolor{white} 0.01 ($\pm$ 0.0003)  & \bfseries \cellcolor{Green!100} 1.47 ($\pm$ 0.0423) & \cellcolor{white} 0.01 ($\pm$ 0.0003)  & \bfseries \cellcolor{Green!100} 1.839 ($\pm$ 0.0558) & \cellcolor{white} 0.009 ($\pm$ 0.0004) \\

Naive-Trim (invalid) & \cellcolor{red!20} 1.624 ($\pm$ 0.0339) & \cellcolor{red!20} 0.017 ($\pm$ 0.0004)  & \cellcolor{red!20} 2.73 ($\pm$ 0.0487) & \cellcolor{red!20} 0.028 ($\pm$ 0.0006)  & \cellcolor{red!20} 4.12 ($\pm$ 0.069) & \cellcolor{red!20} 0.039 ($\pm$ 0.0007) \\

Small-Clean & \cellcolor{white} 0.0 ($\pm$ 0.0) & \cellcolor{white} 0.0 ($\pm$ 0.0)  & \cellcolor{white} 0.0 ($\pm$ 0.0) & \cellcolor{white} 0.0 ($\pm$ 0.0)  & \cellcolor{white} 0.0 ($\pm$ 0.0) & \cellcolor{white} 0.0 ($\pm$ 0.0) \\

Label-Trim & \bfseries \cellcolor{Green!100} 1.139 ($\pm$ 0.0339) & \cellcolor{white} 0.01 ($\pm$ 0.0003)  & \bfseries \cellcolor{Green!100} 1.466 ($\pm$ 0.0421) & \cellcolor{white} 0.01 ($\pm$ 0.0003)  & \bfseries \cellcolor{Green!60} 1.707 ($\pm$ 0.0531) & \cellcolor{white} 0.009 ($\pm$ 0.0003) \\
\hline
Standard Power & \cellcolor{white} 0.193 ($\pm$ 0.0064) &  & \cellcolor{white} 0.148 ($\pm$ 0.0054) &  & \cellcolor{white} 0.115 ($\pm$ 0.005) &  \\
\end{tabular}
}
\subcaption{Target type-I error rate $\alpha=0.01$}

\resizebox{\textwidth}{!}{
\begin{tabular}{l|ll|ll|ll}
\hline
& \multicolumn{6}{c}{Contamination rate} \\
\hline
 & \multicolumn{2}{c|}{1\%} & \multicolumn{2}{c|}{3\%} & \multicolumn{2}{c}{5\%} \\ \hline
 Method      & Power & Type-I Error & Power & Type-I Error & Power & Type-I Error \\ \hline
Standard & \bfseries \cellcolor{Green!30} 1.0 ($\pm$ 0.0209) & \cellcolor{white} 0.017 ($\pm$ 0.0005)  & \bfseries \cellcolor{Green!30} 1.0 ($\pm$ 0.0241) & \cellcolor{white} 0.013 ($\pm$ 0.0004)  & \bfseries \cellcolor{Green!30} 1.0 ($\pm$ 0.0298) & \cellcolor{white} 0.009 ($\pm$ 0.0003) \\

Oracle (infeasible) & \bfseries \cellcolor{Green!100} 1.091 ($\pm$ 0.0224) & 0.021 ($\pm$ 0.0005)  & \bfseries \cellcolor{Green!100} 1.271 ($\pm$ 0.0258) & \cellcolor{white} 0.02 ($\pm$ 0.0005)  & \bfseries \cellcolor{Green!100} 1.599 ($\pm$ 0.0308) & \cellcolor{white} 0.02 ($\pm$ 0.0006) \\

Naive-Trim (invalid) & \cellcolor{red!20} 1.246 ($\pm$ 0.0225) & \cellcolor{red!20} 0.027 ($\pm$ 0.0006)  & \cellcolor{red!20} 1.753 ($\pm$ 0.0277) & \cellcolor{red!20} 0.036 ($\pm$ 0.0007)  & \cellcolor{red!20} 2.462 ($\pm$ 0.0376) & \cellcolor{red!20} 0.046 ($\pm$ 0.0007) \\

Small-Clean & \cellcolor{white} 0.799 ($\pm$ 0.0632) & \cellcolor{white} 0.018 ($\pm$ 0.0021)  & \cellcolor{white} 0.549 ($\pm$ 0.0716) & \cellcolor{white} 0.01 ($\pm$ 0.0016)  & \cellcolor{white} 0.255 ($\pm$ 0.0595) & \cellcolor{white} 0.003 ($\pm$ 0.0009) \\

Label-Trim & \bfseries \cellcolor{Green!60} 1.071 ($\pm$ 0.0219) & \cellcolor{white} 0.02 ($\pm$ 0.0005)  & \bfseries \cellcolor{Green!60} 1.187 ($\pm$ 0.0247) & \cellcolor{white} 0.017 ($\pm$ 0.0004)  & \bfseries \cellcolor{Green!60} 1.361 ($\pm$ 0.0306) & \cellcolor{white} 0.015 ($\pm$ 0.0005) \\
\hline
Standard Power & \cellcolor{white} 0.316 ($\pm$ 0.0066) &  & \cellcolor{white} 0.261 ($\pm$ 0.0063) &  & \cellcolor{white} 0.213 ($\pm$ 0.0064) &  \\
\end{tabular}
}
\subcaption{Target type-I error rate $\alpha=0.02$}

\resizebox{\textwidth}{!}{
\begin{tabular}{l|ll|ll|ll}
\hline
& \multicolumn{6}{c}{Contamination rate} \\
\hline
 & \multicolumn{2}{c|}{1\%} & \multicolumn{2}{c|}{3\%} & \multicolumn{2}{c}{5\%} \\ \hline
 Method      & Power & Type-I Error & Power & Type-I Error & Power & Type-I Error \\ \hline
Standard & \bfseries \cellcolor{Green!30} 1.0 ($\pm$ 0.018) & \cellcolor{white} 0.027 ($\pm$ 0.0006)  & \bfseries \cellcolor{Green!30} 1.0 ($\pm$ 0.0204) & \cellcolor{white} 0.02 ($\pm$ 0.0005)  & \cellcolor{white} 1.0 ($\pm$ 0.021) & \cellcolor{white} 0.015 ($\pm$ 0.0005) \\

Oracle (infeasible) & \bfseries \cellcolor{Green!100} 1.055 ($\pm$ 0.0189) & \cellcolor{white} 0.03 ($\pm$ 0.0006)  & \bfseries \cellcolor{Green!100} 1.231 ($\pm$ 0.0213) & \cellcolor{white} 0.029 ($\pm$ 0.0006)  & \bfseries \cellcolor{Green!100} 1.403 ($\pm$ 0.025) & \cellcolor{white} 0.03 ($\pm$ 0.0007) \\

Naive-Trim (invalid) & \cellcolor{red!20} 1.145 ($\pm$ 0.0199) & \cellcolor{red!20} 0.036 ($\pm$ 0.0006)  & \cellcolor{red!20} 1.496 ($\pm$ 0.0217) & \cellcolor{red!20} 0.044 ($\pm$ 0.0007)  & \cellcolor{red!20} 1.883 ($\pm$ 0.0261) & \cellcolor{red!20} 0.054 ($\pm$ 0.0008) \\

Small-Clean & \cellcolor{white} 0.746 ($\pm$ 0.0469) & \cellcolor{white} 0.021 ($\pm$ 0.002)  & \cellcolor{white} 0.833 ($\pm$ 0.0504) & \cellcolor{white} 0.018 ($\pm$ 0.0018)  & \bfseries \cellcolor{Green!30} 1.029 ($\pm$ 0.0593) & \cellcolor{white} 0.022 ($\pm$ 0.0022) \\

Label-Trim & \bfseries \cellcolor{Green!60} 1.038 ($\pm$ 0.0188) & \cellcolor{white} 0.029 ($\pm$ 0.0006)  & \bfseries \cellcolor{Green!60} 1.149 ($\pm$ 0.0211) & \cellcolor{white} 0.025 ($\pm$ 0.0006)  & \bfseries \cellcolor{Green!60} 1.186 ($\pm$ 0.0228) & \cellcolor{white} 0.022 ($\pm$ 0.0006) \\
\hline
Standard Power & \cellcolor{white} 0.395 ($\pm$ 0.0071) &  & \cellcolor{white} 0.335 ($\pm$ 0.0068) &  & \cellcolor{white} 0.302 ($\pm$ 0.0063) &  \\
\end{tabular}
}
\subcaption{Target type-I error rate $\alpha=0.03$}

\end{table}
\begin{table}[!htb]
\caption{Comparison of conformal outlier detection methods on MNIST dataset (outliers) and CIFAR-10 dataset (inliers) for varying contamination rate $r$ and target type-I error level $\alpha$. The empirical power is presented relative to the \texttt{Standard} method (higher is better). Results are averaged across 100 random splits of the data, with standard errors presented in parentheses.}
\label{app-tab:mnist}
\centering
\resizebox{\textwidth}{!}{
\begin{tabular}{l|ll|ll|ll}
\hline
& \multicolumn{6}{c}{Contamination rate} \\
\hline
 & \multicolumn{2}{c|}{1\%} & \multicolumn{2}{c|}{3\%} & \multicolumn{2}{c}{5\%} \\ \hline
 Method      & Power & Type-I Error & Power & Type-I Error & Power & Type-I Error \\ \hline
Standard & \bfseries \cellcolor{Green!30} 1.0 ($\pm$ 0.0248) & \cellcolor{white} 0.008 ($\pm$ 0.0003)  & \bfseries \cellcolor{Green!30} 1.0 ($\pm$ 0.0283) & \cellcolor{white} 0.004 ($\pm$ 0.0002)  & \bfseries \cellcolor{Green!30} 1.0 ($\pm$ 0.0363) & \cellcolor{white} 0.003 ($\pm$ 0.0002) \\

Oracle (infeasible) & \bfseries \cellcolor{Green!100} 1.241 ($\pm$ 0.0291) & \cellcolor{white} 0.01 ($\pm$ 0.0003)  & \bfseries \cellcolor{Green!100} 1.758 ($\pm$ 0.0352) & \cellcolor{white} 0.01 ($\pm$ 0.0003)  & \bfseries \cellcolor{Green!100} 2.554 ($\pm$ 0.0601) & \cellcolor{white} 0.009 ($\pm$ 0.0004) \\

Naive-Trim (invalid) & \cellcolor{red!20} 1.629 ($\pm$ 0.0292) & \cellcolor{red!20} 0.016 ($\pm$ 0.0004)  & \cellcolor{red!20} 2.722 ($\pm$ 0.0363) & \cellcolor{red!20} 0.023 ($\pm$ 0.0005)  & \cellcolor{red!20} 4.718 ($\pm$ 0.0552) & \cellcolor{red!20} 0.03 ($\pm$ 0.0006) \\

Small-Clean & \cellcolor{white} 0.0 ($\pm$ 0.0) & \cellcolor{white} 0.0 ($\pm$ 0.0)  & \cellcolor{white} 0.0 ($\pm$ 0.0) & \cellcolor{white} 0.0 ($\pm$ 0.0)  & \cellcolor{white} 0.0 ($\pm$ 0.0) & \cellcolor{white} 0.0 ($\pm$ 0.0) \\

Label-Trim & \bfseries \cellcolor{Green!100} 1.241 ($\pm$ 0.0291) & \cellcolor{white} 0.01 ($\pm$ 0.0003)  & \bfseries \cellcolor{Green!60} 1.636 ($\pm$ 0.0331) & \cellcolor{white} 0.009 ($\pm$ 0.0003)  & \bfseries \cellcolor{Green!60} 2.113 ($\pm$ 0.0555) & \cellcolor{white} 0.007 ($\pm$ 0.0003) \\
\hline
Standard Power & \cellcolor{white} 0.282 ($\pm$ 0.007) &  & \cellcolor{white} 0.208 ($\pm$ 0.0059) &  & \cellcolor{white} 0.131 ($\pm$ 0.0048) & \\
\end{tabular}
}
\subcaption{Target type-I error rate $\alpha=0.01$}

\resizebox{\textwidth}{!}{
\begin{tabular}{l|ll|ll|ll}
\hline
& \multicolumn{6}{c}{Contamination rate} \\
\hline
 & \multicolumn{2}{c|}{1\%} & \multicolumn{2}{c|}{3\%} & \multicolumn{2}{c}{5\%} \\ \hline
 Method      & Power & Type-I Error & Power & Type-I Error & Power & Type-I Error \\ \hline
Standard & \bfseries \cellcolor{Green!30} 1.0 ($\pm$ 0.0173) & \cellcolor{white} 0.016 ($\pm$ 0.0004)  & \bfseries \cellcolor{Green!30} 1.0 ($\pm$ 0.0205) & \cellcolor{white} 0.01 ($\pm$ 0.0003)  & \bfseries \cellcolor{Green!30} 1.0 ($\pm$ 0.027) & \cellcolor{white} 0.007 ($\pm$ 0.0003) \\

Oracle (infeasible) & \bfseries \cellcolor{Green!100} 1.117 ($\pm$ 0.0179) & 0.021 ($\pm$ 0.0005)  & \bfseries \cellcolor{Green!100} 1.452 ($\pm$ 0.0209) & \cellcolor{white} 0.02 ($\pm$ 0.0005)  & \bfseries \cellcolor{Green!100} 1.971 ($\pm$ 0.0297) & \cellcolor{white} 0.02 ($\pm$ 0.0006) \\

Naive-Trim (invalid) & \cellcolor{red!20} 1.231 ($\pm$ 0.0179) & \cellcolor{red!20} 0.025 ($\pm$ 0.0005)  & \cellcolor{red!20} 1.759 ($\pm$ 0.0207) & \cellcolor{red!20} 0.031 ($\pm$ 0.0006)  & \cellcolor{red!20} 2.531 ($\pm$ 0.0273) & \cellcolor{red!20} 0.038 ($\pm$ 0.0007) \\

Small-Clean & \cellcolor{white} 0.844 ($\pm$ 0.0514) & \cellcolor{white} 0.019 ($\pm$ 0.0019)  & \cellcolor{white} 0.617 ($\pm$ 0.0738) & \cellcolor{white} 0.01 ($\pm$ 0.0017)  & \cellcolor{white} 0.438 ($\pm$ 0.0817) & \cellcolor{white} 0.005 ($\pm$ 0.0011) \\

Label-Trim & \bfseries \cellcolor{Green!60} 1.096 ($\pm$ 0.0181) & \cellcolor{white} 0.02 ($\pm$ 0.0005)  & \bfseries \cellcolor{Green!60} 1.308 ($\pm$ 0.0206) & \cellcolor{white} 0.015 ($\pm$ 0.0004)  & \bfseries \cellcolor{Green!60} 1.493 ($\pm$ 0.0295) & \cellcolor{white} 0.012 ($\pm$ 0.0004) \\
\hline
Standard Power & \cellcolor{white} 0.465 ($\pm$ 0.008) &  & \cellcolor{white} 0.36 ($\pm$ 0.0074) & & \cellcolor{white} 0.264 ($\pm$ 0.0071) &  \\
\end{tabular}
}
\subcaption{Target type-I error rate $\alpha=0.02$}

\resizebox{\textwidth}{!}{
\begin{tabular}{l|ll|ll|ll}
\hline
& \multicolumn{6}{c}{Contamination rate} \\
\hline
 & \multicolumn{2}{c|}{1\%} & \multicolumn{2}{c|}{3\%} & \multicolumn{2}{c}{5\%} \\ \hline
 Method      & Power & Type-I Error & Power & Type-I Error & Power & Type-I Error \\ \hline
Standard & \bfseries \cellcolor{Green!30} 1.0 ($\pm$ 0.0147) & \cellcolor{white} 0.025 ($\pm$ 0.0005)  & \bfseries \cellcolor{Green!30} 1.0 ($\pm$ 0.0151) & \cellcolor{white} 0.016 ($\pm$ 0.0004)  & \cellcolor{white} 1.0 ($\pm$ 0.0201) & \cellcolor{white} 0.011 ($\pm$ 0.0004) \\

Oracle (infeasible) & \bfseries \cellcolor{Green!100} 1.072 ($\pm$ 0.0149) & \cellcolor{white} 0.03 ($\pm$ 0.0006)  & \bfseries \cellcolor{Green!100} 1.295 ($\pm$ 0.0155) & \cellcolor{white} 0.029 ($\pm$ 0.0006)  & \bfseries \cellcolor{Green!100} 1.617 ($\pm$ 0.0186) & \cellcolor{white} 0.03 ($\pm$ 0.0007) \\

Naive-Trim (invalid) & \cellcolor{red!20} 1.116 ($\pm$ 0.0144) & \cellcolor{red!20} 0.034 ($\pm$ 0.0006)  & \cellcolor{red!20} 1.424 ($\pm$ 0.0142) & \cellcolor{red!20} 0.039 ($\pm$ 0.0007)  & \cellcolor{red!20} 1.851 ($\pm$ 0.0178) & \cellcolor{red!20} 0.046 ($\pm$ 0.0007) \\

Small-Clean & \cellcolor{white} 0.699 ($\pm$ 0.0394) & \cellcolor{white} 0.019 ($\pm$ 0.0019)  & \cellcolor{white} 0.884 ($\pm$ 0.0464) & \cellcolor{white} 0.019 ($\pm$ 0.0018)  & \bfseries \cellcolor{Green!30} 1.164 ($\pm$ 0.0576) & \cellcolor{white} 0.02 ($\pm$ 0.002) \\

Label-Trim & \bfseries \cellcolor{Green!60} 1.049 ($\pm$ 0.0148) & \cellcolor{white} 0.029 ($\pm$ 0.0006)  & \bfseries \cellcolor{Green!60} 1.158 ($\pm$ 0.016) & \cellcolor{white} 0.023 ($\pm$ 0.0005)  & \bfseries \cellcolor{Green!60} 1.273 ($\pm$ 0.0218) & \cellcolor{white} 0.017 ($\pm$ 0.0005) \\
\hline
Standard Power & \cellcolor{white} 0.576 ($\pm$ 0.0085) &  & \cellcolor{white} 0.483 ($\pm$ 0.0073) &  & \cellcolor{white} 0.382 ($\pm$ 0.0077) &  \\
\end{tabular}
}
\subcaption{Target type-I error rate $\alpha=0.03$}

\end{table}

\begin{table}[!htb]
\caption{Comparison of conformal outlier detection methods on CIFAR-100 dataset (outliers) and CIFAR-10 dataset (inliers) for varying contamination rate $r$ and target type-I error level $\alpha$. The empirical power is presented relative to the \texttt{Standard} method (higher is better). Results are averaged across 100 random splits of the data, with standard errors presented in parentheses.}
\label{app-tab:cifar100}
\centering
\resizebox{\textwidth}{!}{
\begin{tabular}{l|ll|ll|ll}
\hline
& \multicolumn{6}{c}{Contamination rate} \\
\hline
 & \multicolumn{2}{c|}{1\%} & \multicolumn{2}{c|}{3\%} & \multicolumn{2}{c}{5\%} \\ \hline
 Method      & Power & Type-I Error & Power & Type-I Error & Power & Type-I Error \\ \hline
Standard & \bfseries \cellcolor{Green!30} 1.0 ($\pm$ 0.0393) & \cellcolor{white} 0.009 ($\pm$ 0.0003)  & \bfseries \cellcolor{Green!30} 1.0 ($\pm$ 0.0391) & \cellcolor{white} 0.007 ($\pm$ 0.0003)  & \bfseries \cellcolor{Green!30} 1.0 ($\pm$ 0.042) & \cellcolor{white} 0.005 ($\pm$ 0.0003) \\

Oracle (infeasible) & \bfseries \cellcolor{Green!100} 1.116 ($\pm$ 0.0417) & \cellcolor{white} 0.01 ($\pm$ 0.0003)  & \bfseries \cellcolor{Green!100} 1.463 ($\pm$ 0.05) & \cellcolor{white} 0.01 ($\pm$ 0.0003)  & \bfseries \cellcolor{Green!100} 1.588 ($\pm$ 0.0477) & \cellcolor{white} 0.009 ($\pm$ 0.0004) \\

Naive-Trim (invalid) & \cellcolor{red!20} 1.733 ($\pm$ 0.0397) & \cellcolor{red!20} 0.018 ($\pm$ 0.0005)  & \cellcolor{red!20} 3.06 ($\pm$ 0.0585) & \cellcolor{red!20} 0.03 ($\pm$ 0.0006)  & \cellcolor{red!20} 4.054 ($\pm$ 0.0631) & \cellcolor{red!20} 0.041 ($\pm$ 0.0007) \\

Small-Clean & \cellcolor{white} 0.0 ($\pm$ 0.0) & \cellcolor{white} 0.0 ($\pm$ 0.0)  & \cellcolor{white} 0.0 ($\pm$ 0.0) & \cellcolor{white} 0.0 ($\pm$ 0.0)  & \cellcolor{white} 0.0 ($\pm$ 0.0) & \cellcolor{white} 0.0 ($\pm$ 0.0) \\

Label-Trim & \bfseries \cellcolor{Green!100} 1.116 ($\pm$ 0.0417) & \cellcolor{white} 0.01 ($\pm$ 0.0003)  & \bfseries \cellcolor{Green!100} 1.463 ($\pm$ 0.05) & \cellcolor{white} 0.01 ($\pm$ 0.0003)  & \bfseries \cellcolor{Green!100} 1.587 ($\pm$ 0.0468) & \cellcolor{white} 0.009 ($\pm$ 0.0004) \\
\hline
Standard Power & \cellcolor{white} 0.145 ($\pm$ 0.0057) &  & \cellcolor{white} 0.118 ($\pm$ 0.0046) &  & \cellcolor{white} 0.104 ($\pm$ 0.0044) & \\
\end{tabular}
}
\subcaption{Target type-I error rate $\alpha=0.01$}

\resizebox{\textwidth}{!}{
\begin{tabular}{l|ll|ll|ll}
\hline
& \multicolumn{6}{c}{Contamination rate} \\
\hline
 & \multicolumn{2}{c|}{1\%} & \multicolumn{2}{c|}{3\%} & \multicolumn{2}{c}{5\%} \\ \hline
 Method      & Power & Type-I Error & Power & Type-I Error & Power & Type-I Error \\ \hline
Standard & \bfseries \cellcolor{Green!30} 1.0 ($\pm$ 0.023) & \cellcolor{white} 0.018 ($\pm$ 0.0005)  & \bfseries \cellcolor{Green!30} 1.0 ($\pm$ 0.0265) & \cellcolor{white} 0.014 ($\pm$ 0.0004)  & \bfseries \cellcolor{Green!30} 1.0 ($\pm$ 0.0264) & \cellcolor{white} 0.011 ($\pm$ 0.0004) \\

Oracle (infeasible) & \bfseries \cellcolor{Green!100} 1.082 ($\pm$ 0.0256) & 0.021 ($\pm$ 0.0005)  & \bfseries \cellcolor{Green!100} 1.265 ($\pm$ 0.0299) & \cellcolor{white} 0.02 ($\pm$ 0.0005)  & \bfseries \cellcolor{Green!100} 1.476 ($\pm$ 0.0317) & \cellcolor{white} 0.02 ($\pm$ 0.0006) \\

Naive-Trim (invalid) & \cellcolor{red!20} 1.294 ($\pm$ 0.0266) & \cellcolor{red!20} 0.027 ($\pm$ 0.0006)  & \cellcolor{red!20} 1.831 ($\pm$ 0.0288) & \cellcolor{red!20} 0.038 ($\pm$ 0.0006)  & \cellcolor{red!20} 2.464 ($\pm$ 0.0372) & \cellcolor{red!20} 0.049 ($\pm$ 0.0008) \\

Small-Clean & \cellcolor{white} 0.844 ($\pm$ 0.0709) & \cellcolor{white} 0.019 ($\pm$ 0.0022)  & \cellcolor{white} 0.686 ($\pm$ 0.0797) & \cellcolor{white} 0.012 ($\pm$ 0.0017)  & \cellcolor{white} 0.392 ($\pm$ 0.078) & \cellcolor{white} 0.006 ($\pm$ 0.0014) \\

Label-Trim & \bfseries \cellcolor{Green!60} 1.064 ($\pm$ 0.0247) & \cellcolor{white} 0.02 ($\pm$ 0.0005)  & \bfseries \cellcolor{Green!60} 1.193 ($\pm$ 0.0293) & \cellcolor{white} 0.018 ($\pm$ 0.0004)  & \bfseries \cellcolor{Green!60} 1.288 ($\pm$ 0.0296) & \cellcolor{white} 0.016 ($\pm$ 0.0005) \\
\hline
Standard Power & \cellcolor{white} 0.255 ($\pm$ 0.0059) &   & \cellcolor{white} 0.225 ($\pm$ 0.006) &  & \cellcolor{white} 0.19 ($\pm$ 0.005) & \\
\end{tabular}
}
\subcaption{Target type-I error rate $\alpha=0.02$}

\resizebox{\textwidth}{!}{
\begin{tabular}{l|ll|ll|ll}
\hline
& \multicolumn{6}{c}{Contamination rate} \\
\hline
 & \multicolumn{2}{c|}{1\%} & \multicolumn{2}{c|}{3\%} & \multicolumn{2}{c}{5\%} \\ \hline
 Method      & Power & Type-I Error & Power & Type-I Error & Power & Type-I Error \\ \hline
Standard & \bfseries \cellcolor{Green!30} 1.0 ($\pm$ 0.0204) & \cellcolor{white} 0.027 ($\pm$ 0.0006)  & \bfseries \cellcolor{Green!30} 1.0 ($\pm$ 0.0214) & \cellcolor{white} 0.022 ($\pm$ 0.0005)  & \bfseries \cellcolor{Green!30} 1.0 ($\pm$ 0.0228) & \cellcolor{white} 0.018 ($\pm$ 0.0005) \\

Oracle (infeasible) & \bfseries \cellcolor{Green!100} 1.054 ($\pm$ 0.0212) & 0.031 ($\pm$ 0.0006)  & \bfseries \cellcolor{Green!100} 1.184 ($\pm$ 0.0228) & \cellcolor{white} 0.03 ($\pm$ 0.0006)  & \bfseries \cellcolor{Green!100} 1.33 ($\pm$ 0.0246) & \cellcolor{white} 0.03 ($\pm$ 0.0007) \\

Naive-Trim (invalid) & \cellcolor{red!20} 1.184 ($\pm$ 0.0213) & \cellcolor{red!20} 0.036 ($\pm$ 0.0006)  & \cellcolor{red!20} 1.523 ($\pm$ 0.0222) & \cellcolor{red!20} 0.046 ($\pm$ 0.0007)  & \cellcolor{red!20} 1.94 ($\pm$ 0.026) & \cellcolor{red!20} 0.056 ($\pm$ 0.0009) \\

Small-Clean & \cellcolor{white} 0.709 ($\pm$ 0.0527) & \cellcolor{white} 0.021 ($\pm$ 0.0023)  & \cellcolor{white} 0.842 ($\pm$ 0.0545) & \cellcolor{white} 0.021 ($\pm$ 0.0019)  & \cellcolor{white} 0.947 ($\pm$ 0.0628) & \cellcolor{white} 0.02 ($\pm$ 0.002) \\

Label-Trim & \bfseries \cellcolor{Green!60} 1.029 ($\pm$ 0.0212) & \cellcolor{white} 0.029 ($\pm$ 0.0006)  & \bfseries \cellcolor{Green!60} 1.114 ($\pm$ 0.0222) & \cellcolor{white} 0.026 ($\pm$ 0.0006)  & \bfseries \cellcolor{Green!60} 1.16 ($\pm$ 0.0229) & \cellcolor{white} 0.023 ($\pm$ 0.0006) \\
\hline
Standard Power & \cellcolor{white} 0.332 ($\pm$ 0.0068) &  & \cellcolor{white} 0.302 ($\pm$ 0.0064) &  & \cellcolor{white} 0.263 ($\pm$ 0.006) &  \\
\end{tabular}
}
\subcaption{Target type-I error rate $\alpha=0.03$}

\end{table}


\begin{table}[!htb]
\caption{Comparison of conformal outlier detection methods on TinyImageNet dataset (outliers) and CIFAR-10 dataset (inliers) for varying contamination rate $r$ and target type-I error level $\alpha$. The empirical power is presented relative to the \texttt{Standard} method (higher is better). Results are averaged across 100 random splits of the data, with standard errors presented in parentheses.}
\label{app-tab:tin}
\centering
\resizebox{\textwidth}{!}{
\begin{tabular}{l|ll|ll|ll}
\hline
& \multicolumn{6}{c}{Contamination rate} \\
\hline
 & \multicolumn{2}{c|}{1\%} & \multicolumn{2}{c|}{3\%} & \multicolumn{2}{c}{5\%} \\ \hline
 Method      & Power & Type-I Error & Power & Type-I Error & Power & Type-I Error \\ \hline
Standard & \bfseries \cellcolor{Green!30} 1.0 ($\pm$ 0.0381) & \cellcolor{white} 0.009 ($\pm$ 0.0003)  & \bfseries \cellcolor{Green!30} 1.0 ($\pm$ 0.0398) & \cellcolor{white} 0.006 ($\pm$ 0.0003)  & \bfseries \cellcolor{Green!30} 1.0 ($\pm$ 0.0439) & \cellcolor{white} 0.004 ($\pm$ 0.0002) \\

Oracle (infeasible) & \bfseries \cellcolor{Green!100} 1.137 ($\pm$ 0.0409) & \cellcolor{white} 0.01 ($\pm$ 0.0003)  & \bfseries \cellcolor{Green!100} 1.492 ($\pm$ 0.0474) & \cellcolor{white} 0.01 ($\pm$ 0.0003)  & \bfseries \cellcolor{Green!100} 1.754 ($\pm$ 0.0518) & \cellcolor{white} 0.009 ($\pm$ 0.0004) \\

Naive-Trim (invalid) & \cellcolor{red!20} 1.675 ($\pm$ 0.0401) & \cellcolor{red!20} 0.018 ($\pm$ 0.0004)  & \cellcolor{red!20} 2.872 ($\pm$ 0.0485) & \cellcolor{red!20} 0.028 ($\pm$ 0.0006)  & \cellcolor{red!20} 3.986 ($\pm$ 0.0568) & \cellcolor{red!20} 0.039 ($\pm$ 0.0007) \\

Small-Clean & \cellcolor{white} 0.0 ($\pm$ 0.0) & \cellcolor{white} 0.0 ($\pm$ 0.0)  & \cellcolor{white} 0.0 ($\pm$ 0.0) & \cellcolor{white} 0.0 ($\pm$ 0.0)  & \cellcolor{white} 0.0 ($\pm$ 0.0) & \cellcolor{white} 0.0 ($\pm$ 0.0) \\

Label-Trim & \bfseries \cellcolor{Green!100} 1.137 ($\pm$ 0.0409) & \cellcolor{white} 0.01 ($\pm$ 0.0003)  & \bfseries \cellcolor{Green!100} 1.488 ($\pm$ 0.0472) & \cellcolor{white} 0.01 ($\pm$ 0.0003)  & \bfseries \cellcolor{Green!60} 1.681 ($\pm$ 0.0511) & \cellcolor{white} 0.009 ($\pm$ 0.0003) \\
\hline
Standard Power & \cellcolor{white} 0.168 ($\pm$ 0.0064) &   & \cellcolor{white} 0.133 ($\pm$ 0.0053) &  & \cellcolor{white} 0.115 ($\pm$ 0.005) &  \\
\end{tabular}
}
\subcaption{Target type-I error rate $\alpha=0.01$}

\resizebox{\textwidth}{!}{
\begin{tabular}{l|ll|ll|ll}
\hline
& \multicolumn{6}{c}{Contamination rate} \\
\hline
 & \multicolumn{2}{c|}{1\%} & \multicolumn{2}{c|}{3\%} & \multicolumn{2}{c}{5\%} \\ \hline
 Method      & Power & Type-I Error & Power & Type-I Error & Power & Type-I Error \\ \hline
Standard & \bfseries \cellcolor{Green!30} 1.0 ($\pm$ 0.0239) & \cellcolor{white} 0.018 ($\pm$ 0.0004)  & \bfseries \cellcolor{Green!30} 1.0 ($\pm$ 0.0262) & \cellcolor{white} 0.013 ($\pm$ 0.0004)  & \bfseries \cellcolor{Green!30} 1.0 ($\pm$ 0.0293) & \cellcolor{white} 0.01 ($\pm$ 0.0004) \\

Oracle (infeasible) & \bfseries \cellcolor{Green!100} 1.084 ($\pm$ 0.0253) & 0.021 ($\pm$ 0.0005)  & \bfseries \cellcolor{Green!100} 1.275 ($\pm$ 0.0263) & \cellcolor{white} 0.02 ($\pm$ 0.0005)  & \bfseries \cellcolor{Green!100} 1.544 ($\pm$ 0.0293) & \cellcolor{white} 0.02 ($\pm$ 0.0006) \\

Naive-Trim (invalid) & \cellcolor{red!20} 1.284 ($\pm$ 0.0274) & \cellcolor{red!20} 0.027 ($\pm$ 0.0006)  & \cellcolor{red!20} 1.819 ($\pm$ 0.0277) & \cellcolor{red!20} 0.036 ($\pm$ 0.0006)  & \cellcolor{red!20} 2.406 ($\pm$ 0.0308) & \cellcolor{red!20} 0.046 ($\pm$ 0.0008) \\

Small-Clean & \cellcolor{white} 0.777 ($\pm$ 0.0566) & \cellcolor{white} 0.015 ($\pm$ 0.0015)  & \cellcolor{white} 0.525 ($\pm$ 0.0687) & \cellcolor{white} 0.009 ($\pm$ 0.0015)  & \cellcolor{white} 0.366 ($\pm$ 0.0653) & \cellcolor{white} 0.005 ($\pm$ 0.0011) \\

Label-Trim & \bfseries \cellcolor{Green!60} 1.074 ($\pm$ 0.0249) & \cellcolor{white} 0.02 ($\pm$ 0.0005)  & \bfseries \cellcolor{Green!60} 1.193 ($\pm$ 0.0254) & \cellcolor{white} 0.017 ($\pm$ 0.0004)  & \bfseries \cellcolor{Green!60} 1.322 ($\pm$ 0.03) & \cellcolor{white} 0.015 ($\pm$ 0.0005) \\
\hline
Standard Power & \cellcolor{white} 0.285 ($\pm$ 0.0068) &  & \cellcolor{white} 0.243 ($\pm$ 0.0064) & & \cellcolor{white} 0.209 ($\pm$ 0.0061) &  \\
\end{tabular}
}
\subcaption{Target type-I error rate $\alpha=0.02$}

\resizebox{\textwidth}{!}{
\begin{tabular}{l|ll|ll|ll}
\hline
& \multicolumn{6}{c}{Contamination rate} \\
\hline
 & \multicolumn{2}{c|}{1\%} & \multicolumn{2}{c|}{3\%} & \multicolumn{2}{c}{5\%} \\ \hline
 Method      & Power & Type-I Error & Power & Type-I Error & Power & Type-I Error \\ \hline
Standard & \bfseries \cellcolor{Green!30} 1.0 ($\pm$ 0.0213) & \cellcolor{white} 0.027 ($\pm$ 0.0006)  & \bfseries \cellcolor{Green!30} 1.0 ($\pm$ 0.0202) & \cellcolor{white} 0.02 ($\pm$ 0.0005)  & \bfseries \cellcolor{Green!30} 1.0 ($\pm$ 0.0216) & \cellcolor{white} 0.016 ($\pm$ 0.0005) \\

Oracle (infeasible) & \bfseries \cellcolor{Green!100} 1.062 ($\pm$ 0.0216) & \cellcolor{white} 0.03 ($\pm$ 0.0006)  & \bfseries \cellcolor{Green!100} 1.23 ($\pm$ 0.0205) & \cellcolor{white} 0.03 ($\pm$ 0.0006)  & \bfseries \cellcolor{Green!100} 1.397 ($\pm$ 0.0236) & \cellcolor{white} 0.03 ($\pm$ 0.0007) \\

Naive-Trim (invalid) & \cellcolor{red!20} 1.168 ($\pm$ 0.0203) & \cellcolor{red!20} 0.036 ($\pm$ 0.0006)  & \cellcolor{red!20} 1.569 ($\pm$ 0.0207) & \cellcolor{red!20} 0.045 ($\pm$ 0.0007)  & \cellcolor{red!20} 1.877 ($\pm$ 0.0217) & \cellcolor{red!20} 0.054 ($\pm$ 0.0009) \\

Small-Clean & \cellcolor{white} 0.656 ($\pm$ 0.0438) & \cellcolor{white} 0.017 ($\pm$ 0.0017)  & \cellcolor{white} 0.825 ($\pm$ 0.0496) & \cellcolor{white} 0.019 ($\pm$ 0.0019)  & \cellcolor{white} 0.904 ($\pm$ 0.0589) & \cellcolor{white} 0.02 ($\pm$ 0.0024) \\

Label-Trim & \bfseries \cellcolor{Green!60} 1.04 ($\pm$ 0.0219) & \cellcolor{white} 0.029 ($\pm$ 0.0006)  & \bfseries \cellcolor{Green!60} 1.139 ($\pm$ 0.0208) & \cellcolor{white} 0.026 ($\pm$ 0.0006)  & \bfseries \cellcolor{Green!60} 1.19 ($\pm$ 0.0225) & \cellcolor{white} 0.022 ($\pm$ 0.0006) \\
\hline
Standard Power & \cellcolor{white} 0.368 ($\pm$ 0.0078) &  & \cellcolor{white} 0.318 ($\pm$ 0.0064) &  & \cellcolor{white} 0.29 ($\pm$ 0.0063) &  \\
\end{tabular}
}
\subcaption{Target type-I error rate $\alpha=0.03$}

\end{table}


\FloatBarrier

\end{document}
