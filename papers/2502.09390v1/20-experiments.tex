
\begin{table*}[ht]
    \centering
    % \small
    \caption{The main results of our experimentation. Each row group corresponds to the results for the given dataset, with each row showcasing the metric results for each model. The columns include all the main approaches, with \textbf{bold} highlighting the best result across all approaches.}
    \small
    \begin{tabular}{llccccc}
      \toprule
      Dataset & Model & Baseline & RAG & CoT & RaR & \rephrase \\
      \midrule
      \multirow[l]{3}{*}{TriviaQA}
          & Llama-3.2 3B  & 59.5 & 82.0 & 87.5  & 86.0 &  \textbf{88.5}    \\
          & Llama-3.1 8B  & 76.5 & 89.5 & 90.5  & 89.5 &  \textbf{92.5}    \\
          & GPT-4o    & 88.7 & 92.7 & 92.7  & 94.7 &  \textbf{96.7}    \\
      \midrule
      \multirow[l]{3}{*}{HotpotQA}
          & Llama-3.2 3B  &  17.5  & 26.0  & 26.5   & 25.0  &  \textbf{31.5}   \\
          & Llama-3.1 8B  &  23.0  & 26.5  & 31.0   & 28.5  &  \textbf{33.5}   \\
          & GPT-4o    &  44.0  & 45.3  & 46.7   & \textbf{47.3}  &  46.7   \\
      \midrule
      \multirow[l]{3}{*}{ASQA}
          & Llama-3.2 3B  &  14.2 & 21.5  & 21.9  & 23.5  &  \textbf{26.6}   \\ 
          & Llama-3.1 8B  &  14.6 & 23.1  & 24.8  & 25.5  &  \textbf{28.8}   \\ 
          & GPT-4o    &  26.8 & 30.4  & \textbf{31.9}  & 30.1 & 31.7 \\ 
      \bottomrule
    \end{tabular}
    \label{tab:main}
\end{table*}



\section{Experiments}
\label{sec:experiments}

In this section, we detail the experimental setup and the evaluations conducted to assess the effectiveness of the \rephrase{} technique across various datasets and models. Our approach is compared to several existing methods to ascertain its relative performance.


\subsection{Datasets}

We evaluate our models on \mbox{\textbf{TriviaQA}} \cite{joshi-etal-2017-triviaqa}, \textbf{HotpotQA} \cite{yang-etal-2018-hotpotqa}, and \textbf{ASQA} \cite{stelmakh-etal-2022-asqa} which are knowledge intensive question-answering datasets which benefit from external context. Context retrieval was done over a Wikipedia corpus \cite{zhangRetrieveAnythingAugment2023}. We randomly sampled 200 examples from each dataset.  Results are reported using the following metrics: for TriviaQA and HotpotQA sub-string exact match (subEM) is reported \cite{asaiSelfRAGLearningRetrieve2023,yenHELMETHowEvaluate2024}. For ASQA, recall-EM is reported \cite{gaoEnablingLargeLanguage2023a}. For more details, see \Cref{sec:datasets}.

\subsection{Models}
\label{sec:models}

Our experiments utilize two open-source Llama models \cite{grattafioriLlama3Herd2024}: Llama-3.2 3B and Llama-3.1 8B. Both models are instruction-tuned to optimize their performance on complex tasks. In addition, we employed the OpenAI GPT-4o system\footnote{Version \textit{2024-05-13}.} \cite{openaiGPT4oSystemCard2024} to provide a benchmark for comparison. We use greedy decoding with local models.

\subsection{Configurations}
\label{sec:configs}

Our experimental setup is composed of the following configuration settings:

\begin{itemize}[noitemsep,topsep=0.5em,parsep=0.4em,leftmargin=0.8em]
\item \textbf{Baseline}: Standard application without any augmentative techniques.
\item \textbf{CoT}: Methodology as outlined by \citet{weiChainofThoughtPromptingElicits2023} that leverages intermediate reasoning steps leading to a final answer; instruction described in \Cref{tab:cot_prompt}.
\item \textbf{RaR:} A rephrasing strategy that prompts for a rephrasing of the original request before answering it, as proposed by \citet{dengRephraseRespondLet2024}; instruction described in \Cref{tab:rar_prompt}.
\item \textbf{\rephrase}: This configuration employs our prompt and is run with a default $N\!=\!3$ question-answer pairs.
\end{itemize}

We augment the requests with a pair of query-answer examples (few-shot) to facilitate understanding and improve prediction formatting and accuracy. All prompts and few-shot examples are presented in \Cref{sec:prompts} for reproducibility.

\pagebreak

Notably, in configurations containing reasoning instructions, we employ a regular expression\footnote{Regex pattern: \texttt{.*answer(.*)}. It has a 99.2\% capture rate.} to extract the final answer. This extraction is crucial as it assists in mitigating incorrect answers when correct phrases appear throughout reasoning chains \textbf{but not in the final answer}. For an example of this phenomena, see \Cref{tab:bad-example}.
