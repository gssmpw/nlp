\section{Related Works}
\label{sec:related-works}

Chain-of-Thought (CoT) prompting, introduced by \citet{wu-etal-2023-chain}, and explored further by \citet{weiChainofThoughtPromptingElicits2023}, has been instrumental in enhancing language models, by encouraging them to articulate their reasoning processes explicitly. This approach has been shown to substantially improve model performance across a wide range of tasks, including question answering. 

\citet{dengRephraseRespondLet2024} propose a novel rephrasing prompt, which involves requesting the model to rephrase the initial question before providing an answer. This method has demonstrated performance improvements on various datasets, highlighting its efficacy in refining model responses. Our work expands upon this approach, by utilizing multiple query-answer pairs, that enable the model to better examine the topic at hand, and provide a better answer.

\citet{wangSelfConsistencyImprovesChain2023} and \citet{chenUniversalSelfConsistencyLarge2023} leverage self-consistency techniques by generating multiple response samples (by using sample decoding) and incorporating an aggregation step to increase accuracy, thereby enhancing the reliability of model conclusions. While our approach does generate multiple variations of the possible answer, they are dedicated for answering specific automatically generated inquiries regarding the topic at hand.

\citet{snellScalingLLMTestTime2024} demonstrate that extra test-time compute boosts LLM performance on difficult prompts, with smaller models sometimes surpassing larger ones. They propose a \textit{compute-optimal} method that adaptively explores multiple next steps, maximizing inference efficiency. Building on this idea, our approach focuses on question answering, where diverse perspectives substantially improve response quality. As previously mentioned, while our approach benefits from generating multiple responses for a given query, we focus on specific query-answer pair generation.   
