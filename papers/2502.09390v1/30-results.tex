
\section{Results}
\label{sec:results}

Table~\ref{tab:main} presents the main results of our method compared against several baselines on three benchmark QA datasets: TriviaQA, HotpotQA, and ASQA. Across the smaller Llama 3.2 3B and Llama 3.1 8B models, our approach consistently outperforms or matches the strongest baselines in each dataset. For example, with Llama 3.2 3B on \mbox{TriviaQA}, \rephrase{} improves performance by 6.5\% and 2.5\% over RAG and RaR, respectively, achieving an overall score of 88.5\%. On HotpotQA, Llama 3.2 3B also sees a notable boost, from 26.5\% (CoT) to 31.5\% with our method. These gains become even more pronounced with Llama 3.1 8B, where improvements of up to 3\% (TriviaQA) and 7\% (HotpotQA) are observed compared to alternative methods. We also observe notable gains on ASQA. For Llama-3.2 3B, \rephrase{} lifts performance from 21.5\% (RAG) and 23.5\% (RaR) to 26.6\%, nearly doubling the baseline of 14.2\%. 

When using GPT-4o, \rephrase{} remains highly competitive. On TriviaQA, our method reaches 96.7\%, outperforming other settings by at least 2.0\%. On HotpotQA, RaR and \rephrase{} are close, with RaR exhibiting a slight edge (47.3\% versus 46.7\%). For ASQA, CoT and \rephrase{} yield nearly identical performance (31.9\% versus 31.7\%), indicating that GPT-4o is already adept at leveraging additional reasoning steps or retrieved facts in these tasks. Nevertheless, \rephrase{} demonstrates robust performance across all three datasets and is especially beneficial for smaller-scale models, where sequential questioning can substantially bolster the final answer quality.


\begin{table}[t]
    \centering
    \small
    \begin{tabular}{llccc}
      \toprule
      Dataset & $N$ &  \rephrase & +Summarize  & +Vote \\
      \midrule
      \multirow[l]{3}{*}{TriviaQA}
              & 3  & 92.5 & 87.5 & 81.0  \\
              & 5  & \textbf{94.0} & 85.5 & 78.0  \\
              & 10 & \textbf{94.0} & 88.0 & 89.0  \\
      \midrule
      \multirow[l]{3}{*}{HotpotQA}
              & 3  & \textbf{33.5} & 30.0 & 23.5    \\
              &  5 & 31.5 & 31.5 & 22.5    \\
              &  10 & \textbf{33.5} & 29.0 & 23.5    \\
      \midrule
      \multirow[l]{3}{*}{ASQA}
              & 3 & \textbf{28.8} & 20.9 & 23.9   \\ 
              & 5 & 27.9 & 22.1 & 23.5  \\ 
              & 10 & 27.8 & 23.1 & 22.7  \\ 
      \bottomrule
    \end{tabular}
    \caption{Comparison of two aggregation methods in addition to \rephrase{}, and the effect of varying the number of sub-questions ($N$). Results showcase the Llama-3.1 8B model with few-shot examples adapted for each approach, as detailed in \Cref{sec:prompts}.}
    \label{tab:variations}
\end{table}


\subsection{Ablation Study}
\label{sec:ablation-study}

To highlight the contribution of each component in \rephrase, we performed an ablation study analyzing (1) the number of generated questions ($N$), (2) the role of few-shot examples, and (3) an optional aggregation step.

\paragraph{Number of Generated Questions:}
We conducted an evaluation using $N\in\{3,5,10\}$. As shown in Table~\ref{tab:variations}, for TriviaQA, increasing $N$ from 3 to 5 or 10 boosts performance from 92.5\% to 94.0\%. On HotpotQA, $N\!=\!5$ (31.5\%) dips slightly below $N\!=\!3$, but returns to 33.5\% at $N\!=\!10$. In ASQA, performance drops from 28.8\% at $N\!=\!3$ to 27.8\% at $N\!=\!10$, suggesting that while additional questions can add useful context, they can also introduce redundancy or noise. For more comparisons, see \Cref{tab:more_results}.

\subsection{Few-Shot Adaptation}

\label{S:fewshot}
A particularly useful consequence of learning a policy committee $\Pi$ that is a $(\epsilon,1-\delta)$-cover is that we can leverage it in meta-learning for few-shot adaptation. The algorithmic idea is straightforward: evaluate each of $K$ policies in $\Pi$ by computing a sample average sum of rewards over $N$ randomly initialized episodes, and choose the best policy $\pi \in \Pi$ in terms of empirical average reward.
%This yields the following sample complexity bound.

In particular, suppose that $\gamma = 1$.
We now show that this translates into a few-shot sample complexity on a previously unseen task $\tau$ that is linear in $K$ (the size of the committee). Details of the proof are in Appendix~\ref{A:adaptation}.
%\begin{definition}
    %We define the best expert as $\pi^* = \mathrm{argmax}_{\pi\in \Pi} V^{\pi}$, the expert which yields the highest expected average cumulative reward $V^{\pi}$ from its steady-state  $\mu_\pi$. The cumulative regret, $r(n)$ after $n$ iterations of a multi-armed bandit algorithm is defined as: $r(n)=nV^{*}-\sum_{k=0}^{n}\frac{1}{T_k}\sum_{t=t_k}^{t_{k+1}-1} r_t$ By Theorem~\ref{thm:linear_reg} $V^{\pi*}$ is at least $2LT\epsilon-$optimal to $V*$ since $V^{\pi*} \ge V^{\pi'} $ where $\pi'\in \Pi$ is inside the cover.\end{definition}

%\begin{theorem}(Regret decomposition identity). If the induced Markov chains for each policy committee member $\pi \in \Pi$ are irreducible and aperiodic, then the expected cumulative regret at time n can be bounded with:$\mathbb{E}[r(n)] \le \sum \mathbb{E}[T(n)][\delta_\pi+\frac{C_\pi}{T_0(1-\beta_pi)}]+\frac{C_*}{}(1-\beta_*)\sum_{k=0}^{n-1}\frac{1}{T_k}$ \end{theorem}


%TODO: formal bounds for few-shot adaptation; idea 1: $K$ policy evaluation steps followed by playing the best policy; what is the bound on regret?  How does it compare to regret bounds in standard RL?  Is there a UCB bound that we can use and treat policies in $\Pi$ as ``experts''? 

%Having solved the committee formation problem, we can now utilize the RLPA algorithm to quickly identify the best policy member in the committee with regret bound measured \emph{against the optimal policy for the underlying MDP}, rather than by the best policy in the committee.

%One application 


\iffalse
We can then leverage \emph{online learning} approaches which treat $\Pi$ as a set of policy \emph{experts}.
To illustrate, we can use the RLPA algorithm due to \citet{azar2013regret}, which combined with our methods above can yield provable regret bounds, as we show next.


\fi

\iffalse
\begin{theorem}\citep{azar2013regret}
\label{T:regret}
   Let $f$ be an increasing function and $S^+$ the span of the best policy in $\Pi$ with average reward $\mu_\Pi^*$.
   Under Assumption~\ref{A:online}, for any number of trials $N \ge f^{-1}(S^+)$ the regret of the RLPA algorithm with $K$ deterministic policies with respect to $\mu_\Pi^*$ is bounded by
$\Delta(s)\le 24(f(N)+1)\sqrt{3NK(\log(N/\alpha))}+\sqrt{N} +6f(N)K(\log_2(S^+)+2\log_2(N))$
with probability at least $1-\alpha$ for any initial state $s \in \mathcal{S}$.
\end{theorem}




\begin{corollary}\label{cor:RLPA}
Suppose $\Pi$ with $|\Pi|=K$ is an $(\epsilon,1-\delta)$-cover for $\Gamma$. Then under Assumption~\ref{A:online}, with probability at least $1-\delta-\alpha$, for any number of trials $N \ge f^{-1}(S^+)$, the regret of a  task $\tau \sim \Gamma$ with respect to the optimal average reward $\mu^* = V_\tau^*/h$ for $\tau$ is bounded by
$\Delta(s)\le 24(f(N)+1)\sqrt{3NK(\log(N/\alpha))}+\sqrt{N} +6f(N)K(\log_2(N^+)+2\log_2(N))+\frac{\epsilon}{h}$, for any $s\in \mathcal{S}$.
\end{corollary}
A notable aspect of this result is that while conventional regret in online learning, such as in Theorem~\ref{T:regret}, is measured with respect to the best policy in the (small) set of experts, we obtain low regret with respect to the \emph{optimal} policy for the task faced at execution time.
This is the key consequence of the committee $\Pi$ constituting a $(\epsilon,1-\delta)$ cover.
\fi

%on the efficacy of policy committees.
%For small $K$,  Azuma’s inequality have guaranteed us high confidence that we can also evaluate each policy repeatedly and select the one with the highest reward.

%We now connect this result with the notion of $(\epsilon,1-\delta)$-cover that we have, noting that $V^\pi_\tau = h\mu^\pi_\tau$ for any $\pi,\tau$.
\begin{theorem}\label{thm:online_repetition}
Suppose $\Pi$ is a $(\epsilon,1-\delta)$-cover for $\Gamma$ and let $\tau \sim \Gamma$. Under some mild conditions,
if we run 
%the number of episodes ran for each policy 
$p \ge\frac{32h(H+1)^2\log(4/\alpha)}{(\beta-2H)^2}$ episodes for each policy $\pi \in \Pi$, where $H$ is a constant, the policy $\pi$ that maximizes the empirical return yields $V_\tau^\pi \ge V_\tau^* -\epsilon-\beta$ with probability at least $1-\delta-\alpha$, where $V_\tau^*$ is the optimal reward for $\tau$.
\end{theorem}

%\vspace{-15pt}

%\vspace{-5pt}
%While this result assumes that each policy $\pi \in \Pi$ has been fixed for the duration of adaptation (that is, we are only doing policy evaluation for each), in practice a simple improvement is to actually fine-tune each policy as we get more experience.
%This is the variation that we use in the experiments below.

%Notably, this suggests that a simple few-shot learning algorithm in which we perform policy evaluation for each of $K$ policies, and then use the one with the best empirical performance, will lead to effective few-shot learning.



% \pagebreak

\paragraph{Impact of Few-Shot Examples:}
We inspected how incorporating few-shot examples substantially boosts accuracy, as seen in Figure~\ref{fig:fewshot}. We observe that both CoT and \rephrase{} benefit strongly from these examples, indicating that better exposure to task-relevant scenarios helps the model generate answers with correct and properly formed final answers. Interestingly, zero-shot experiments exhibit lower regex capture rate (85.0\%, see \Cref{sec:configs}) which could play a role in the diminished performance. For full results, see \Cref{tab:more_results}.

\paragraph{Aggregation Methods:}
Finally, we explored two aggregation strategies, before producing the final answer: \textit{Summarize} and \textit{Vote}. The \textit{Summarize} method involves the model summarizing the information learned from the generated questions and answers, whereas the \textit{Vote} method relies on majority voting to determine the final answer. According to \Cref{tab:variations}, \textit{Summarize} generally outperforms \textit{Vote} on TriviaQA and HotpotQA. However, using no aggregation step outperforms both in nearly all instances, suggesting that further post-processing can sometimes hurt the quality of the answer.
