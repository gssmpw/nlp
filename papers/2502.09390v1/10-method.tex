\section{\rephrase}
\label{sec:phrasing}

In this section, we introduce the \rephrase{} technique in more detail. Building upon the foundation laid by \citet{dengRephraseRespondLet2024}, our method alters the system instructions to prompt the model to generate a set of $N$ question-and-answer pairs. Figure~\ref{fig:illustration} illustrates a simple example in which the model receives a query, generates two sub-questions and their corresponding answers, and then arrives at a correct final solution. The system prompt used by our method is presented in Table~\ref{tab:square_prompt}.

The rationale behind \rephrase{} is to guide the model into an iterative cycle of inquiry and response, encouraging it to explore various facets of a topic before forming a conclusion. In contrast to standard chain-of-thought prompts, which often present a single stream of reasoning, \rephrase{} nudges the model toward self-interrogation pathways. This design also makes \rephrase{} relatively straightforward to integrate with other prompting techniques. In practice, $N$ can be tuned to balance the thoroughness of exploration with computational cost and response length; our experiments in Section~\ref{sec:experiments} show that even a small set of sub-questions can significantly improve the final answers' correctness.

