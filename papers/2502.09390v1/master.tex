% This must be in the first 5 lines to tell arXiv to use pdfLaTeX, which is strongly recommended.
\pdfoutput=1
% In particular, the hyperref package requires pdfLaTeX in order to break URLs across lines.

\documentclass[11pt]{article}

% Change "review" to "final" to generate the final (sometimes called camera-ready) version.
% Change to "preprint" to generate a non-anonymous version with page numbers.
\usepackage[preprint]{acl}

% Standard package includes
\usepackage{times}
\usepackage{latexsym}

% For proper rendering and hyphenation of words containing Latin characters (including in bib files)
\usepackage[T1]{fontenc}
% For Vietnamese characters
% \usepackage[T5]{fontenc}
% See https://www.latex-project.org/help/documentation/encguide.pdf for other character sets

% This assumes your files are encoded as UTF8
\usepackage[utf8]{inputenc}

% This is not strictly necessary, and may be commented out,
% but it will improve the layout of the manuscript,
% and will typically save some space.
\usepackage{microtype}

% This is also not strictly necessary, and may be commented out.
% However, it will improve the aesthetics of text in
% the typewriter font.
\usepackage{inconsolata}

%Including images in your LaTeX document requires adding
%additional package(s)
\usepackage{graphicx}

% Table related
\usepackage{tabularx}
\usepackage{multirow}
\usepackage{booktabs}
\usepackage{amsfonts}
\usepackage{pifont}
\usepackage{wrapfig}

\usepackage{float}
\restylefloat{table}

\usepackage[skip=4pt]{caption}  % Reduce space after captions
\raggedbottom

% Lists related
\usepackage{enumitem}
\newlist{todolist}{itemize}{2}
\setlist[todolist]{label=$\square$}
\usepackage{pifont}
\newcommand{\cmark}{\ding{51}}%
\newcommand{\done}{\rlap{$\square$}{\raisebox{2pt}{\large\hspace{1pt}\cmark}}\hspace{-1.1pt}}

\usepackage{cleveref}
% If the title and author information does not fit in the area allocated, uncomment the following
%
%\setlength\titlebox{<dim>}
%
% and set <dim> to something 5cm or larger.


\newcommand{\moshe}[1]{\textcolor{cyan}{[\textbf{Moshe}: #1]}}
\newcommand{\gad}[1]{\textcolor{blue}{[\textbf{Gad}: #1]}}
\newcommand{\daniel}[1]{\textcolor{magenta}{[\textbf{Daniel}: #1]}}


\title{SQuARE: Sequential Question Answering Reasoning Engine for Enhanced Chain-of-Thought in Large Language Models}
\newcommand{\rephrase}{SQuARE}

\author{Daniel Fleischer\thanks{Corresponding author: \href{mailto:daniel.fleischer@intel.com}{daniel.fleischer@intel.com}} 
\qquad Moshe Berchansky \qquad Gad Markovits \qquad Moshe Wasserblat
\vspace{0.3em} \\
Intel Labs
}


\begin{document}
\maketitle


During the early stages of interface design, designers need to produce multiple sketches to explore a design space.  Design tools often fail to support this critical stage, because they insist on specifying more details than necessary. Although recent advances in generative AI have raised hopes of solving this issue, in practice they fail because expressing loose ideas in a prompt is impractical. In this paper, we propose a diffusion-based approach to the low-effort generation of interface sketches. It breaks new ground by allowing flexible control of the generation process via three types of inputs: A) prompts, B) wireframes, and C) visual flows. The designer can provide any combination of these as input at any level of detail, and will get a diverse gallery of low-fidelity solutions in response. The unique benefit is that large design spaces can be explored rapidly with very little effort in input-specification. We present qualitative results for various combinations of input specifications. Additionally, we demonstrate that our model aligns more accurately with these specifications than other models. 

% OLD ABSTRACT
%When sketching Graphical User Interfaces (GUIs), designers need to explore several aspects of visual design simultaneously, such as how to guide the user’s attention to the right aspects of the design while making the intended functionality visible. Although current Large Language Models (LLMs) can generate GUIs, they do not offer the finer level of control necessary for this kind of exploration. To address this, we propose a diffusion-based model with multi-modal conditional generation. In practice, our model optionally takes semantic segmentation, prompt guidance, and flow direction to generate multiple GUIs that are aligned with the input design specifications. It produces multiple examples. We demonstrate that our approach outperforms baseline methods in producing desirable GUIs and meets the desired visual flow.

% Designing visually engaging Graphical User Interfaces (GUIs) is a challenge in HCI research. Effective GUI design must balance visual properties, like color and positioning, with user behaviors to ensure GUIs easy to comprehend and guide attention to critical elements. Modern GUIs, with their complex combinations of text, images, and interactive components, make it difficult to maintain a coherent visual flow during design.
% Although current Large Language Models (LLMs) can generate GUIs, they often lack the fine control necessary for ensuring a coherent visual flow. To address this, we propose a diffusion-based model that effectively handles multi-modal conditional generation. Our model takes semantic segmentation, optional prompt guidance, and ordered viewing elements to generate high-fidelity GUIs that are aligned with the input design specifications.
% We demonstrate that our approach outperforms baseline methods in producing desirable GUIs and meets the desired visual flow. Moreover, a user study involving XX designers indicates that our model enhances the efficiency of the GUI design ideation process and provides designers with greater control compared to existing methods.    



% %%%%%%%%%%%%%%%%%%%%%%%%%%%%%%%%%%%%%%%%%%%%%%%%%%%%%%
% % Writing Clinic Comments:
% %%%%%%%%%%%%%%%%%%%%%%%%%%%%%%%%%%%%%%%%%%%%%%%%%%%%%%
% % Define: Effective UI design
% % Motivate GANs and write in full form.
% % LLMs vs ControlNet vs GANs
% % Say something about the Figma plugin?
% % Write the work is novel or what has been done before
% % What is desirable UI and how to evalutate that?
% % Visual Flow - main theme (center around it)
% % Re-Title: use word Flow!
% % Use ControlNet++ & SPADE for abstract.
% % Write about input/output. 
% % Why better than previous work?
% %%%%%%%%%%%%%%%%%%%%%%%%%%%%%%%%%%%%%%%%%%%%%%%%%%%%%

% % v2:
% % \noindent \textcolor{red}{\textbf{NEW Abstract!} (Post Writing Clinic 1 - 25-Jun)}

% % \noindent \textcolor{red}{----------------------------------------------------------------------}

% % \noindent Designing user interfaces (UIs) is a time-consuming process, particularly for novice designers. 
% % Creating UI designs that are effective in market funneling or any other designer defined goal requires a good understanding of the visual flow to guide users' attention to UI elements in the desired order. 
% % While current Large Language Models (LLMs) can generate UIs from just prompts, they often lack finer pixel-precise control and fail to consider visual flow. 
% % In this work, we present a UI synthesis method that incorporates visual flow alongside prompts and semantic layouts. 
% % Our efficient approach uses a carefully designed Generative Adversarial Network (GAN) optimized for scenarios with limited data, making it more suitable than diffusion-based and large vision-language models.
% % We demonstrate that our method produces more "desirable" UIs according to the well-known contrast, repetition, alignment, and proximity principles of design. 
% % We further validate our method through comprehensive automatic non-reference, human-preference aligned network scoring and subjective human evaluations.
% % Finally, an evaluation with xx non-expert designers using our contributed Figma plugin shows that <method-name> improves the time-efficiency as well as the overall quality of the UI design development cycle.

% % \noindent \textcolor{red}{----------------------------------------------------------------------}


% \noindent \textcolor{blue}{\textbf{NEW Abstract!} (Pre Writing Clinic 9-July)}

% \noindent \textcolor{blue}{----------------------------------------------------------------------}

% \noindent Exploring different graphical user interface (GUI) design ideas is time-consuming, particularly for novice designers. 
% Given the segmentation masks, design requirement as prompt, and/or preferred visual flow, we aim to facilitate creative exploration for GUI design and generate different UI designs for inspiration.
% While current Vision Language Models (VLMs) can generate GUIs from just prompts, they often lack control over visual concepts and flow that are difficult to convey through language during the generation process. 
% In this work, we present FlowGenUI, a semantic map-guided GUI synthesis method that optionally incorporates visual flow information based on the user's choice alongside language prompts. 
% We demonstrate that our model not only creates more realistic GUIs but also creates "predictable" (how users pay attention to and order of looking at GUI elements) GUIs.
% Our approach uses Stable Diffusion (SD), a large paired image-text pretrained diffusion model with a rich latent space that we steer toward realistic GUIs using a trainable copy of SD's encoder for every condition (segmentation masks, prompts, and visual flow). 
% We further provide a semantic typography feature to create custom text-fonts and styles while also alleviating SD's inherent limitations in drawing coherent, meaningful and correct aspect-ratio text. 
% Finally, a subjective evaluation study of XX non-expert and expert designers demonstrates the efficiency and fidelity of our method.


% This process encourages creativity and prevents designers from falling into habitual patterns.


% ------------------------------------------------------------------
% Joongi Why is it important to create realistic GUI?
% I do not see how the Visual Flow given on the left hand side is reflected in the results on the right hand side. 
% I’d avoid making unsubstantiated claims about designers (falling into habitual patterns).
% The UIs you generate do not “align with users’ attention patterns” but rather try to control it (that’s what visual flow means)
% ------------------------------------------------------------------
% Comments - Writing Clinic - 9th July:
% Improve title. More names: FlowGen
% Figure 1: Use an inference time hand-drawn mask
% Figure 1: Show both workflows. Add a designer --> Input.
% Figure 1: Make them more diverse
% ------------------------------------------------------------------
% Designing graphical user interfaces (GUIs) requires human creativity and time. Designers often fall into habitual patterns, which can limit the exploration of new ideas. 
% To address this, we introduce FlowGenUI, a method that facilitates creative exploration and generates diverse GUI designs for inspiration. By using segmentation masks, design requirements as prompts, and/or selected visual flows, our approach enhances control over the visual concepts and flows during the generation process, which current Vision Language Models (VLMs) often lack.
% FlowGenUI uses Stable Diffusion (SD), a largely pretrained text-to-image diffusion model, and guides it to create realistic GUIs. 
% We achieve this by using a trainable copy of SD's encoder for each condition (segmentation masks, prompts, and visual flow). 
% This method enables the creation of more realistic and predictable GUIs that align with users' attention patterns and their preferred order of viewing elements.
% We also offer a semantic typography feature that creates custom text fonts and styles while addressing SD's limitations in generating coherent, meaningful, and correctly aspect-ratio text.
% Our approach's efficiency and fidelity are evaluated through a subjective user study involving XX designers. 
% The results demonstrate the effectiveness of FlowGenUI in generating high-quality GUI designs that meet user requirements and visual expectations.

% ---------------------------------------


%A critical and general issue remains while using such deep generative priors: creating coherent, meaningful and correct aspect-ratio text. 
%We tackle this issue within our framework and additionally provide a semantic typography feature to create custom text-fonts and styles. 


% %Creating UI designs that are effective in market funneling or any other designer-defined goal requires a good understanding of the visual flow to guide users' attention to UI elements in the desired order. 
% %While current largely pre-trained Vision Language Models (VLMs) can generate GUIs from just prompts, they often lack finer or pixel-precise control which can be crucial for many easy-to-understand visual concepts but difficult to convey through language. 
% % However, obtaining such pixe-level labels is an extremely expensive so we
% % For example - overlaying text on images with certain aspect ratios and two equally separated buttons 
% Additionally, all prior GUI generation work fails to consider visual flow information during the generation process. 
% We demonstrate that visual flow-informed generation not only creates more realistic and human-friendly GUIs but also creates "predictable" (how users pay attention to and order of looking at GUI elements) UIs that could be beneficial for designers for tasks like creating effective market funnels.
% In this work, we present a semantic map-guided GUI synthesis method that optionally incorporates visual flow information based on the user's choice alongside language prompts. 
% Our approach uses Stable Diffusion, a large (billions) paired image-text pretrained diffusion model with a rich latent space that we steer toward realistic GUIs using an ensemble of ControlNets. 
% % TODO: Mention it in 1 sentence:
% A critical and general issue remains while using such deep generative priors: creating coherent, meaningful and correct aspect-ratio text. 
% We tackle this issue within our framework and additionally provide a semantic typography feature to create custom text-fonts and styles. 
% To evaluate our method, we demonstrate that our method produces more "desirable" UIs according to the well-known contrast, repetition, alignment, and proximity principles of design. 
% % We further validate our method through comprehensive automatic non-reference and human-preference aligned scores. (TODO: Maybe Unskip if we get UIClip from Jason!)
% % TODO: Re-word this and only keep ideation cycles and time-efficiency.
% Finally, a subjective evaluation study of XX non-expert and expert designers demonstrates the efficiency and fidelity of our method.
% % improves the time-efficiency by quick iterations of the UI design ideation process.
% %Finally, an evaluation with xx non-expert designers using our contributed <method-name> improves the time-efficiency by quick iterations of the UI design ideation cycle.

%\noindent \textcolor{blue}{----------------------------------------------------------------------}


%In an evaluation with xx designers, we found that GenerativeLayout: 1) enhances designers' exploration by expanding the coverage of the design space, 2) reduces the time required for exploration, and 3) maintains a perceived level of control similar to that of manual exploration.



% Present-day graphical user interfaces (GUIs) exhibit diverse arrangements of text, graphics, and interactive elements such as buttons and menus, but representations of GUIs have not kept up. They do not encapsulate both semantic and visuo-spatial relationships among elements. %\color{red} 
% To seize machine learning's potential for GUIs more efficiently, \papername~ exploits graph neural networks to capture individual elements' properties and their semantic—visuo-spatial constraints in a layout. The learned representation demonstrated its effectiveness in multiple tasks, especially generating designs in a challenging GUI autocompletion task, which involved predicting the positions of remaining unplaced elements in a partially completed GUI. The new model's suggestions showed alignment and visual appeal superior to the baseline method and received higher subjective ratings for preference. 
% Furthermore, we demonstrate the practical benefits and efficiency advantages designers perceive when utilizing our model as an autocompletion plug-in.


% Overall pipeline: Maybe drop semantic typography / visual flow?

% humans are sensitive to the way information is presented.

% introduce framing as the way we address framing. say something about political views and how information is represented.

% in this paper we explore if models show similar sensitivity.

% why is it important/interesting.



% thought - it would be interesting to test it on real world data, but it would be hard to test humans because they come already biased about real world stuff, so we tested artificial.


% LLMs have recently been shown to mimic cognitive biases, typically associated with human behavior~\citep{ malberg2024comprehensive, itzhak-etal-2024-instructed}. This resemblance has significant implications for how we perceive these models and what we can expect from them in real-world interactions and decisionmaking~\citep{eigner2024determinants, echterhoff-etal-2024-cognitive}.

The \textit{framing effect} is a well-known cognitive phenomenon, where different presentations of the same underlying facts affect human perception towards them~\citep{tversky1981framing}.
For example, presenting an economic policy as only creating 50,000 new jobs, versus also reporting that it would cost 2B USD, can dramatically shift public opinion~\cite{sniderman2004structure}. 
%%%%%%%% 图1:  %%%%%%%%%%%%%%%%
\begin{figure}[t]
    \centering
    \includegraphics[width=\columnwidth]{Figs/01.pdf}
    \caption{Performance comparison (Top-1 Acc (\%)) under various open-vocabulary evaluation settings where the video learners except for CLIP are tuned on Kinetics-400~\cite{k400} with frozen text encoders. The satisfying in-context generalizability on UCF101~\cite{UCF101} (a) can be severely affected by static bias when evaluating on out-of-context SCUBA-UCF101~\cite{li2023mitigating} (b) by replacing the video background with other images.}
    \label{fig:teaser}
\end{figure}


Previous research has shown that LLMs exhibit various cognitive biases, including the framing effect~\cite{lore2024strategic,shaikh2024cbeval,malberg2024comprehensive,echterhoff-etal-2024-cognitive}. However, these either rely on synthetic datasets or evaluate LLMs on different data from what humans were tested on. In addition, comparisons between models and humans typically treat human performance as a baseline rather than comparing patterns in human behavior. 
% \gabis{looks good! what do we mean by ``most studies'' or ``rarely'' can we remove those? or we want to say that we don't know of previous work doing both at the same time?}\gili{yeah the main point is that some work has done each separated, but not all of it together. how about now?}

In this work, we evaluate LLMs on real-world data. Rather than measuring model performance in terms of accuracy, we analyze how closely their responses align with human annotations. Furthermore, while previous studies have examined the effect of framing on decision making, we extend this analysis to sentiment analysis, as sentiment perception plays a key explanatory role in decision-making \cite{lerner2015emotion}. 
%Based on this, we argue that examining sentiment shifts in response to reframing can provide deeper insights into the framing effect. \gabis{I don't understand this last claim. Maybe remove and just say we extend to sentiment analysis?}

% Understanding how LLMs respond to framing is crucial, as they are increasingly integrated into real-world applications~\citep{gan2024application, hurlin2024fairness}.
% In some applications, e.g., in virtual companions, framing can be harnessed to produce human-like behavior leading to better engagement.
% In contrast, in other applications, such as financial or legal advice, mitigating the effect of framing can lead to less biased decisions.
% In both cases, a better understanding of the framing effect on LLMs can help develop strategies to mitigate its negative impacts,
% while utilizing its positive aspects. \gabis{$\leftarrow$ reading this again, maybe this isn't the right place for this paragraph. Consider putting in the conclusion? I think that after we said that people have worked on it, we don't necessarily need this here and will shorten the long intro}


% If framing can influence their outputs, this could have significant societal effects,
% from spreading biases in automated decision-making~\citep{ghasemaghaei2024understanding} to reducing public trust in AI-generated content~\citep{afroogh2024trust}. 
% However, framing is not inherently negative -- understanding how it affects LLM outputs can offer valuable insights into both human and machine cognition.
% By systematically investigating the framing effect,


%It is therefore crucial to systematically investigate the framing effect, to better understand and mitigate its impact. \gabis{This paragraph is important - I think that right now it's saying that we don't want models to be influenced by framing (since we want to mitigate its impact, right?) When we talked I think we had a more nuanced position?}




To better understand the framing effect in LLMs in comparison to human behavior,
we introduce the \name{} dataset (Section~\ref{sec:data}), comprising 1,000 statements, constructed through a three-step process, as shown in Figure~\ref{fig:fig1}.
First, we collect a set of real-world statements that express a clear negative or positive sentiment (e.g., ``I won the highest prize'').
%as exemplified in Figure~\ref{fig:fig1} -- ``I won the highest prize'' positive base statement. (2) next,
Second, we \emph{reframe} the text by adding a prefix or suffix with an opposite sentiment (e.g., ``I won the highest prize, \emph{although I lost all my friends on the way}'').
Finally, we collect human annotations by asking different participants
if they consider the reframed statement to be overall positive or negative.
% \gabist{This allows us to quantify the extent of \textit{sentiment shifts}, which is defined as labeling the sentiment aligning with the opposite framing, rather then the base sentiment -- e.g., voting ``negative'' for the statement ``I won the highest prize, although I lost all my friends on the way'', as it aligns with the opposite framing sentiment.}
We choose to annotate Amazon reviews, where sentiment is more robust, compared to e.g., the news domain which introduces confounding variables such as prior political leaning~\cite{druckman2004political}.


%While the implications of framing on sensitive and controversial topics like politics or economics are highly relevant to real-world applications, testing these subjects in a controlled setting is challenging. Such topics can introduce confounding variables, as annotators might rely on their personal beliefs or emotions rather than focusing solely on the framing, particularly when the content is emotionally charged~\cite{druckman2004political}. To balance real-world relevance with experimental reliability, we chose to focus on statements derived from Amazon reviews. These are naturally occurring, sentiment-rich texts that are less likely to trigger strong preexisting biases or emotional reactions. For instance, a review like ``The book was engaging'' can be framed negatively without invoking specific cultural or political associations. 

 In Section~\ref{sec:results}, we evaluate eight state-of-the-art LLMs
 % including \gpt{}~\cite{openai2024gpt4osystemcard}, \llama{}~\cite{dubey2024llama}, \mistral{}~\cite{jiang2023mistral}, \mixtral{}~\cite{mistral2023mixtral}, and \gemma{}~\cite{team2024gemma}, 
on the \name{} dataset and compare them against human annotations. We find  that LLMs are influenced by framing, somewhat similar to human behavior. All models show a \emph{strong} correlation ($r>0.57$) with human behavior.
%All models show a correlation with human responses of more than $0.55$ in Pearson's $r$ \gabis{@Gili check how people report this?}.
Moreover, we find that both humans and LLMs are more influenced by positive reframing rather than negative reframing. We also find that larger models tend to be more correlated with human behavior. Interestingly, \gpt{} shows the lowest correlation with human behavior. This raises questions about how architectural or training differences might influence susceptibility to framing. 
%\gabis{this last finding about \gpt{} stands in opposition to the start of the statement, right? Even though it's probably one of the largest models, it doesn't correlate with humans? If so, better to state this explicitly}

This work contributes to understanding the parallels between LLM and human cognition, offering insights into how cognitive mechanisms such as the framing effect emerge in LLMs.\footnote{\name{} data available at \url{https://huggingface.co/datasets/gililior/WildFrame}\\Code: ~\url{https://github.com/SLAB-NLP/WildFrame-Eval}}

%\gabist{It also raises fundamental philosophical and practical questions -- should LLMs aim to emulate human-like behavior, even when such behavior is susceptible to harmful cognitive biases? or should they strive to deviate from human tendencies to avoid reproducing these pitfalls?}\gabis{$\leftarrow$ also following Itay's comment, maybe this is better in the dicsussion, since we don't address these questions in the paper.} %\gabis{This last statement brings the nuance back, so I think it contradicts the previous parapgraph where we talked about ``mitigating'' the effect of framing. Also, I think it would be nice to discuss this a bit more in depth, maybe in the discussion section.}






\section{\rephrase}
\label{sec:phrasing}

In this section, we introduce the \rephrase{} technique in more detail. Building upon the foundation laid by \citet{dengRephraseRespondLet2024}, our method alters the system instructions to prompt the model to generate a set of $N$ question-and-answer pairs. Figure~\ref{fig:illustration} illustrates a simple example in which the model receives a query, generates two sub-questions and their corresponding answers, and then arrives at a correct final solution. The system prompt used by our method is presented in Table~\ref{tab:square_prompt}.

The rationale behind \rephrase{} is to guide the model into an iterative cycle of inquiry and response, encouraging it to explore various facets of a topic before forming a conclusion. In contrast to standard chain-of-thought prompts, which often present a single stream of reasoning, \rephrase{} nudges the model toward self-interrogation pathways. This design also makes \rephrase{} relatively straightforward to integrate with other prompting techniques. In practice, $N$ can be tuned to balance the thoroughness of exploration with computational cost and response length; our experiments in Section~\ref{sec:experiments} show that even a small set of sub-questions can significantly improve the final answers' correctness.



\begin{table*}[ht]
    \centering
    % \small
    \caption{The main results of our experimentation. Each row group corresponds to the results for the given dataset, with each row showcasing the metric results for each model. The columns include all the main approaches, with \textbf{bold} highlighting the best result across all approaches.}
    \small
    \begin{tabular}{llccccc}
      \toprule
      Dataset & Model & Baseline & RAG & CoT & RaR & \rephrase \\
      \midrule
      \multirow[l]{3}{*}{TriviaQA}
          & Llama-3.2 3B  & 59.5 & 82.0 & 87.5  & 86.0 &  \textbf{88.5}    \\
          & Llama-3.1 8B  & 76.5 & 89.5 & 90.5  & 89.5 &  \textbf{92.5}    \\
          & GPT-4o    & 88.7 & 92.7 & 92.7  & 94.7 &  \textbf{96.7}    \\
      \midrule
      \multirow[l]{3}{*}{HotpotQA}
          & Llama-3.2 3B  &  17.5  & 26.0  & 26.5   & 25.0  &  \textbf{31.5}   \\
          & Llama-3.1 8B  &  23.0  & 26.5  & 31.0   & 28.5  &  \textbf{33.5}   \\
          & GPT-4o    &  44.0  & 45.3  & 46.7   & \textbf{47.3}  &  46.7   \\
      \midrule
      \multirow[l]{3}{*}{ASQA}
          & Llama-3.2 3B  &  14.2 & 21.5  & 21.9  & 23.5  &  \textbf{26.6}   \\ 
          & Llama-3.1 8B  &  14.6 & 23.1  & 24.8  & 25.5  &  \textbf{28.8}   \\ 
          & GPT-4o    &  26.8 & 30.4  & \textbf{31.9}  & 30.1 & 31.7 \\ 
      \bottomrule
    \end{tabular}
    \label{tab:main}
\end{table*}



\section{Experiments}
\label{sec:experiments}

In this section, we detail the experimental setup and the evaluations conducted to assess the effectiveness of the \rephrase{} technique across various datasets and models. Our approach is compared to several existing methods to ascertain its relative performance.


\subsection{Datasets}

We evaluate our models on \mbox{\textbf{TriviaQA}} \cite{joshi-etal-2017-triviaqa}, \textbf{HotpotQA} \cite{yang-etal-2018-hotpotqa}, and \textbf{ASQA} \cite{stelmakh-etal-2022-asqa} which are knowledge intensive question-answering datasets which benefit from external context. Context retrieval was done over a Wikipedia corpus \cite{zhangRetrieveAnythingAugment2023}. We randomly sampled 200 examples from each dataset.  Results are reported using the following metrics: for TriviaQA and HotpotQA sub-string exact match (subEM) is reported \cite{asaiSelfRAGLearningRetrieve2023,yenHELMETHowEvaluate2024}. For ASQA, recall-EM is reported \cite{gaoEnablingLargeLanguage2023a}. For more details, see \Cref{sec:datasets}.

\subsection{Models}
\label{sec:models}

Our experiments utilize two open-source Llama models \cite{grattafioriLlama3Herd2024}: Llama-3.2 3B and Llama-3.1 8B. Both models are instruction-tuned to optimize their performance on complex tasks. In addition, we employed the OpenAI GPT-4o system\footnote{Version \textit{2024-05-13}.} \cite{openaiGPT4oSystemCard2024} to provide a benchmark for comparison. We use greedy decoding with local models.

\subsection{Configurations}
\label{sec:configs}

Our experimental setup is composed of the following configuration settings:

\begin{itemize}[noitemsep,topsep=0.5em,parsep=0.4em,leftmargin=0.8em]
\item \textbf{Baseline}: Standard application without any augmentative techniques.
\item \textbf{CoT}: Methodology as outlined by \citet{weiChainofThoughtPromptingElicits2023} that leverages intermediate reasoning steps leading to a final answer; instruction described in \Cref{tab:cot_prompt}.
\item \textbf{RaR:} A rephrasing strategy that prompts for a rephrasing of the original request before answering it, as proposed by \citet{dengRephraseRespondLet2024}; instruction described in \Cref{tab:rar_prompt}.
\item \textbf{\rephrase}: This configuration employs our prompt and is run with a default $N\!=\!3$ question-answer pairs.
\end{itemize}

We augment the requests with a pair of query-answer examples (few-shot) to facilitate understanding and improve prediction formatting and accuracy. All prompts and few-shot examples are presented in \Cref{sec:prompts} for reproducibility.

\pagebreak

Notably, in configurations containing reasoning instructions, we employ a regular expression\footnote{Regex pattern: \texttt{.*answer(.*)}. It has a 99.2\% capture rate.} to extract the final answer. This extraction is crucial as it assists in mitigating incorrect answers when correct phrases appear throughout reasoning chains \textbf{but not in the final answer}. For an example of this phenomena, see \Cref{tab:bad-example}.


\section{Results}
\label{sec:results}

Table~\ref{tab:main} presents the main results of our method compared against several baselines on three benchmark QA datasets: TriviaQA, HotpotQA, and ASQA. Across the smaller Llama 3.2 3B and Llama 3.1 8B models, our approach consistently outperforms or matches the strongest baselines in each dataset. For example, with Llama 3.2 3B on \mbox{TriviaQA}, \rephrase{} improves performance by 6.5\% and 2.5\% over RAG and RaR, respectively, achieving an overall score of 88.5\%. On HotpotQA, Llama 3.2 3B also sees a notable boost, from 26.5\% (CoT) to 31.5\% with our method. These gains become even more pronounced with Llama 3.1 8B, where improvements of up to 3\% (TriviaQA) and 7\% (HotpotQA) are observed compared to alternative methods. We also observe notable gains on ASQA. For Llama-3.2 3B, \rephrase{} lifts performance from 21.5\% (RAG) and 23.5\% (RaR) to 26.6\%, nearly doubling the baseline of 14.2\%. 

When using GPT-4o, \rephrase{} remains highly competitive. On TriviaQA, our method reaches 96.7\%, outperforming other settings by at least 2.0\%. On HotpotQA, RaR and \rephrase{} are close, with RaR exhibiting a slight edge (47.3\% versus 46.7\%). For ASQA, CoT and \rephrase{} yield nearly identical performance (31.9\% versus 31.7\%), indicating that GPT-4o is already adept at leveraging additional reasoning steps or retrieved facts in these tasks. Nevertheless, \rephrase{} demonstrates robust performance across all three datasets and is especially beneficial for smaller-scale models, where sequential questioning can substantially bolster the final answer quality.


\begin{table}[t]
    \centering
    \small
    \begin{tabular}{llccc}
      \toprule
      Dataset & $N$ &  \rephrase & +Summarize  & +Vote \\
      \midrule
      \multirow[l]{3}{*}{TriviaQA}
              & 3  & 92.5 & 87.5 & 81.0  \\
              & 5  & \textbf{94.0} & 85.5 & 78.0  \\
              & 10 & \textbf{94.0} & 88.0 & 89.0  \\
      \midrule
      \multirow[l]{3}{*}{HotpotQA}
              & 3  & \textbf{33.5} & 30.0 & 23.5    \\
              &  5 & 31.5 & 31.5 & 22.5    \\
              &  10 & \textbf{33.5} & 29.0 & 23.5    \\
      \midrule
      \multirow[l]{3}{*}{ASQA}
              & 3 & \textbf{28.8} & 20.9 & 23.9   \\ 
              & 5 & 27.9 & 22.1 & 23.5  \\ 
              & 10 & 27.8 & 23.1 & 22.7  \\ 
      \bottomrule
    \end{tabular}
    \caption{Comparison of two aggregation methods in addition to \rephrase{}, and the effect of varying the number of sub-questions ($N$). Results showcase the Llama-3.1 8B model with few-shot examples adapted for each approach, as detailed in \Cref{sec:prompts}.}
    \label{tab:variations}
\end{table}


\subsection{Ablation Study}
\label{sec:ablation-study}

To highlight the contribution of each component in \rephrase, we performed an ablation study analyzing (1) the number of generated questions ($N$), (2) the role of few-shot examples, and (3) an optional aggregation step.

\paragraph{Number of Generated Questions:}
We conducted an evaluation using $N\in\{3,5,10\}$. As shown in Table~\ref{tab:variations}, for TriviaQA, increasing $N$ from 3 to 5 or 10 boosts performance from 92.5\% to 94.0\%. On HotpotQA, $N\!=\!5$ (31.5\%) dips slightly below $N\!=\!3$, but returns to 33.5\% at $N\!=\!10$. In ASQA, performance drops from 28.8\% at $N\!=\!3$ to 27.8\% at $N\!=\!10$, suggesting that while additional questions can add useful context, they can also introduce redundancy or noise. For more comparisons, see \Cref{tab:more_results}.

\subsection{Few-Shot Adaptation}

\label{S:fewshot}
A particularly useful consequence of learning a policy committee $\Pi$ that is a $(\epsilon,1-\delta)$-cover is that we can leverage it in meta-learning for few-shot adaptation. The algorithmic idea is straightforward: evaluate each of $K$ policies in $\Pi$ by computing a sample average sum of rewards over $N$ randomly initialized episodes, and choose the best policy $\pi \in \Pi$ in terms of empirical average reward.
%This yields the following sample complexity bound.

In particular, suppose that $\gamma = 1$.
We now show that this translates into a few-shot sample complexity on a previously unseen task $\tau$ that is linear in $K$ (the size of the committee). Details of the proof are in Appendix~\ref{A:adaptation}.
%\begin{definition}
    %We define the best expert as $\pi^* = \mathrm{argmax}_{\pi\in \Pi} V^{\pi}$, the expert which yields the highest expected average cumulative reward $V^{\pi}$ from its steady-state  $\mu_\pi$. The cumulative regret, $r(n)$ after $n$ iterations of a multi-armed bandit algorithm is defined as: $r(n)=nV^{*}-\sum_{k=0}^{n}\frac{1}{T_k}\sum_{t=t_k}^{t_{k+1}-1} r_t$ By Theorem~\ref{thm:linear_reg} $V^{\pi*}$ is at least $2LT\epsilon-$optimal to $V*$ since $V^{\pi*} \ge V^{\pi'} $ where $\pi'\in \Pi$ is inside the cover.\end{definition}

%\begin{theorem}(Regret decomposition identity). If the induced Markov chains for each policy committee member $\pi \in \Pi$ are irreducible and aperiodic, then the expected cumulative regret at time n can be bounded with:$\mathbb{E}[r(n)] \le \sum \mathbb{E}[T(n)][\delta_\pi+\frac{C_\pi}{T_0(1-\beta_pi)}]+\frac{C_*}{}(1-\beta_*)\sum_{k=0}^{n-1}\frac{1}{T_k}$ \end{theorem}


%TODO: formal bounds for few-shot adaptation; idea 1: $K$ policy evaluation steps followed by playing the best policy; what is the bound on regret?  How does it compare to regret bounds in standard RL?  Is there a UCB bound that we can use and treat policies in $\Pi$ as ``experts''? 

%Having solved the committee formation problem, we can now utilize the RLPA algorithm to quickly identify the best policy member in the committee with regret bound measured \emph{against the optimal policy for the underlying MDP}, rather than by the best policy in the committee.

%One application 


\iffalse
We can then leverage \emph{online learning} approaches which treat $\Pi$ as a set of policy \emph{experts}.
To illustrate, we can use the RLPA algorithm due to \citet{azar2013regret}, which combined with our methods above can yield provable regret bounds, as we show next.


\fi

\iffalse
\begin{theorem}\citep{azar2013regret}
\label{T:regret}
   Let $f$ be an increasing function and $S^+$ the span of the best policy in $\Pi$ with average reward $\mu_\Pi^*$.
   Under Assumption~\ref{A:online}, for any number of trials $N \ge f^{-1}(S^+)$ the regret of the RLPA algorithm with $K$ deterministic policies with respect to $\mu_\Pi^*$ is bounded by
$\Delta(s)\le 24(f(N)+1)\sqrt{3NK(\log(N/\alpha))}+\sqrt{N} +6f(N)K(\log_2(S^+)+2\log_2(N))$
with probability at least $1-\alpha$ for any initial state $s \in \mathcal{S}$.
\end{theorem}




\begin{corollary}\label{cor:RLPA}
Suppose $\Pi$ with $|\Pi|=K$ is an $(\epsilon,1-\delta)$-cover for $\Gamma$. Then under Assumption~\ref{A:online}, with probability at least $1-\delta-\alpha$, for any number of trials $N \ge f^{-1}(S^+)$, the regret of a  task $\tau \sim \Gamma$ with respect to the optimal average reward $\mu^* = V_\tau^*/h$ for $\tau$ is bounded by
$\Delta(s)\le 24(f(N)+1)\sqrt{3NK(\log(N/\alpha))}+\sqrt{N} +6f(N)K(\log_2(N^+)+2\log_2(N))+\frac{\epsilon}{h}$, for any $s\in \mathcal{S}$.
\end{corollary}
A notable aspect of this result is that while conventional regret in online learning, such as in Theorem~\ref{T:regret}, is measured with respect to the best policy in the (small) set of experts, we obtain low regret with respect to the \emph{optimal} policy for the task faced at execution time.
This is the key consequence of the committee $\Pi$ constituting a $(\epsilon,1-\delta)$ cover.
\fi

%on the efficacy of policy committees.
%For small $K$,  Azuma’s inequality have guaranteed us high confidence that we can also evaluate each policy repeatedly and select the one with the highest reward.

%We now connect this result with the notion of $(\epsilon,1-\delta)$-cover that we have, noting that $V^\pi_\tau = h\mu^\pi_\tau$ for any $\pi,\tau$.
\begin{theorem}\label{thm:online_repetition}
Suppose $\Pi$ is a $(\epsilon,1-\delta)$-cover for $\Gamma$ and let $\tau \sim \Gamma$. Under some mild conditions,
if we run 
%the number of episodes ran for each policy 
$p \ge\frac{32h(H+1)^2\log(4/\alpha)}{(\beta-2H)^2}$ episodes for each policy $\pi \in \Pi$, where $H$ is a constant, the policy $\pi$ that maximizes the empirical return yields $V_\tau^\pi \ge V_\tau^* -\epsilon-\beta$ with probability at least $1-\delta-\alpha$, where $V_\tau^*$ is the optimal reward for $\tau$.
\end{theorem}

%\vspace{-15pt}

%\vspace{-5pt}
%While this result assumes that each policy $\pi \in \Pi$ has been fixed for the duration of adaptation (that is, we are only doing policy evaluation for each), in practice a simple improvement is to actually fine-tune each policy as we get more experience.
%This is the variation that we use in the experiments below.

%Notably, this suggests that a simple few-shot learning algorithm in which we perform policy evaluation for each of $K$ policies, and then use the one with the best empirical performance, will lead to effective few-shot learning.



% \pagebreak

\paragraph{Impact of Few-Shot Examples:}
We inspected how incorporating few-shot examples substantially boosts accuracy, as seen in Figure~\ref{fig:fewshot}. We observe that both CoT and \rephrase{} benefit strongly from these examples, indicating that better exposure to task-relevant scenarios helps the model generate answers with correct and properly formed final answers. Interestingly, zero-shot experiments exhibit lower regex capture rate (85.0\%, see \Cref{sec:configs}) which could play a role in the diminished performance. For full results, see \Cref{tab:more_results}.

\paragraph{Aggregation Methods:}
Finally, we explored two aggregation strategies, before producing the final answer: \textit{Summarize} and \textit{Vote}. The \textit{Summarize} method involves the model summarizing the information learned from the generated questions and answers, whereas the \textit{Vote} method relies on majority voting to determine the final answer. According to \Cref{tab:variations}, \textit{Summarize} generally outperforms \textit{Vote} on TriviaQA and HotpotQA. However, using no aggregation step outperforms both in nearly all instances, suggesting that further post-processing can sometimes hurt the quality of the answer.

\section{Background and related work}
% 重点看Artistic data visualization: Beyond visual analytics 和Visualization criticism-the missing link between information visualization and art 的被引


This section reviews the background on artistic data visualization and previous research related to this topic.

\subsection{Artistic Data Visualization in Art History Context}
\label{ssec:contemporary}

Art history has been marked by transformative periods characterized by different aesthetic pursuits, materials, and methods. Since the time of Plato, imitation (or \textit{mimesis}, which views art as a mirror to the world around us) has been an important pursuit~\cite{pooke2021art}. Successful artworks, such as Greek sculptures and the convincing illusions of depth and space in Renaissance paintings, exemplify this tradition.
The advent of modern society and new technology, especially photography, posed a significant challenge to the notion of art as imitation~\cite{perry2004themes}. By the 1850s, modern art began to emerge with the core goal of transcending traditional forms and conventions. Movements like Post Impressionism, Expressionism, and Cubism revolutionized art through innovative uses of form (\eg color, texture, composition), moving art towards abstraction and experimentation. 
After World War II, the Cold War and the surge of various social problems heightened skepticism about the progress narrative of modernity and the superiority of the capitalist system, leading to the rise of postmodernism and the birth of contemporary art~\cite{hopkins2000after,harrison1992art}. One prominent feature of contemporary art is the absence of fixed standards or genres historically defined by the church or the academy. Postmodern design neither defines a cohesive set of aesthetic values nor restricts the range of media used~\cite{pooke2021art}, sparking novel practices such as installations, performances, lens-based media, videos, and land-based art~\cite{hopkins2000after}.
Meanwhile, artists have increasingly drawn energy from various philosophical and critical theories such as gender studies, psychoanalysis, Marxism, and post-structuralism~\cite{pooke2021art}. As a result, as described by Foster~\cite{foster1999recodings}, artists have increasingly become ``manipulators of signs and symbols... and the viewer an active reader of messages rather than a passive contemplator of the aesthetic''. Hopkins~\cite{hopkins2000after} described this shift as the ``death of the object'' and ``the move to conceptualism''. 
% Joseph Kosuth, one of the most important representatives of conceptual artists, also once said that “all art (after Duchamp) is conceptual (in nature) because art only exists conceptually”
% As argued by Danto~\cite{danto2015after}, traditional notions of aesthetics can no longer apply to contemporary art. ``“All there is at the end,” Danto wrote, “is theory, art having finally become vaporized in a dazzle of pure thought about itself, and remaining, as it were, solely as the object of its own theoretical consciousness.''
% The Anti-aesthetic (1983) has described these as ‘anti-aesthetic’ strategies – typified, as we have seen, by a conceptually driven approach to the art object and to the process of its production.

Emerging within the contemporary art historical context, data art has been significantly propelled by the advent of big data. An early milestone was Kynaston McShine's 1970 exhibition \textit{Information} at the Museum of Modern Art (MoMA). 
% In the exhibition catalogue, McShine wrote~\cite{information_moma}: ``Increasingly artists use mail, telegrams, telex machines, etc., for transmission of works themselves—photographs, films, documents—or of information about their activity.'' 
% The millennium era has witnessed substantial growth in this area.
In 2008, Google’s Data Arts Team was founded to explore the advancement of what creativity and technology can do together~\cite{google}.
% data artist Aaron Koblin
In 2012, Viégas and Wattenberg's \textit{Wind Map}, an artwork that visualizes real-time air movement, became the first web-based artwork to be included in MoMA's permanent collection~\cite{wind}.
Since 2013, the academic conference IEEE VIS has included an Arts Program (IEEE VISAP), showcasing artistic data visualizations through accepted papers and curated exhibitions. 
As noted by Barabási~\cite{dataism} (a Fellow of the American Physical Society and the head of a data art lab), data has become a vital medium for artists to deal with the complexities of our society: ``Humanity is facing a complexity explosion. We are confronted with too much data for any of us to make sense of...The traditional tools and mediums of art, be they canvas or chisel, are woefully inadequate for this task...today’s and tomorrow’s artists can embrace new tools and mediums that scale to the challenge, ensuring that their practice can continue to reflect our changing epistemology.''
% a physicist and head of a data art lab, has noted, 

% Artists are now exploring new mediums and methods, incorporating datasets, computer technology, and algorithms into their work.



\subsection{Research on Artistic Data Visualization}
\label{ssec:artisticvis}

Artistic data visualization is also referred to as artistic visualization, data art, or information art~\cite{holmquist2003informative,rodgers2011exploring,few,viegas2007artistic}. Despite the varying terminologies, there is a basic consensus that artistic data visualization must be art practice grounded in real data~\cite{viegas2007artistic}.
Since the early 2000s, a series of papers introduced innovative artistic systems for visualizing everyday data, such as museum visit routes and bus schedule information~\cite{skog2003between,holmquist2003informative,viegas2004artifacts}.
In 2007, Viégas and Wattenberg~\cite{viegas2007artistic} explicitly proposed the concept of \textit{artistic data visualization} and viewed it as a promising domain beyond visual analytics.
% and defined it as ``visualization of data done by artists with the intent of making art''. 
Kosara~\cite{kosara2007visualization} drew a spectrum of visualization design, positioning artistic visualization and pragmatic visualization at opposite ends of this spectrum to demonstrate that the goals of these two types of design often diverge. 
% advocating that analytical visualizations prioritize practicality, while artistic data visualizations emphasize sublime quality, that is, the capacity to inspire awe and grandeur and elicit profound emotional or intellectual responses. 
% In 2011, Rodgers and Bartram~\cite{rodgers2011exploring} utilized artistic data visualization to enhance residential energy use feedback. 
However, overall, research on this subject has been sparse. Among those relevant papers, most have focused on specific applications of artistic data visualization. 
%~\cite{rodgers2011exploring,schroeder2015visualization,perovich2020chemicals}
For instance, Rodgers and Bartram~\cite{rodgers2011exploring} utilized ambient artistic data visualization to enhance residential energy use feedback. Samsel~\etal~\cite{samsel2018art} transferred artistic styles from paintings into scientific visualization.
Artistic practice has also been observed in fields such as data physicalization~\cite{hornecker2023design,perovich2020chemicals,offenhuber2019data} and sonification~\cite{enge2024open}. For example, Hornecker~\etal~\cite{hornecker2023design} found that many artists are practicing transforming data into tangible artifacts or installations. Enge~\etal~\cite{enge2024open} discussed a set of representative artistic cases that combine sonification and visualization.
% dragicevic2020data
% Offenhuber~\cite{offenhuber2019data} created a set of artworks in urban settings that transform air quality data into situated displays, allowing people to encounter visualizations in their daily lives.

% But in contrast, empirical studies that describe the characteristics of artistic visualization and how they are created are extremely scarce. This scarcity forms a stark contrast to the increasingly rich and diverse practices by artists in the field.
% As for the difference between artistic data visualization and traditional visualizations for analytics, Vi{\'e}gas and Wattenberg~\cite{viegas2007artistic} thought that the most salient feature of artistic data visualizations is their forceful expression of viewpoints.
% In Ramirez~\cite{ramirez2008information}'s opinion, functional information visualizations are concerned with usability and performance while aesthetic information visualizations are concerned with visually attractive forms of representation design.
% Donath~\etal~\cite{donath2010data} presented a series of tools developed to integrate artistic expressions in generating unique and customized visualizations based on users' personal data, such as health monitoring data, album records, and e-mail records. 

On the other hand, some studies, while not focusing on artistic data visualization, have explored a key art-related concept: aesthetics. 
% ~\cite{moere2012evaluating,cawthon2007effect,lau2007towards} explored the aesthetics of visualization design in their research. They
For example, Moere~\etal~\cite{moere2012evaluating} compared analytical, magazine, and artistic visualization styles, noting that analytical styles enhance the discovery of analytical insights, while the other two induce more meaning-based insights. Cawthon~\etal~\cite{cawthon2007effect} asked participants to rank eleven visualization types on an aesthetic scale from ``ugly'' to ``beautiful'', finding that some visualizations (\eg sunburst) were perceived as more beautiful than others (\eg beam trees).
% Moere~\etal~\cite{moere2012evaluating} displayed data in three different styles (analytical style, magazine style, artistic style) and found that these styles led to different perceptions of usability and types of insights.
% More importantly, the authors found that the sunburst chart ranks the highest in aesthetics and is one of the top-performing visualizations in both efficiency and effectiveness, thus exemplifying the notion that "beautiful is indeed usable".
Factors such as embellishment~\cite{bateman2010useful}, colorfulness~\cite{harrison2015infographic}, and interaction~\cite{stoll2024investigating} have also been found to influence aesthetics. 
% borkin2013makes,tanahashi2012design
However, most of these studies have simplified aesthetics to hedonic features (\eg beauty), without delving into the nuanced connotations of aesthetics.
% most of these studies have simplified aesthetics to concepts like 'beauty,' 'preference,' or 'pleasing,' without exploring the deeper essence of aesthetics as the core of art.

The value of artistic data visualization to the visualization community is still in controversy. For instance, Few~\cite{few} argued for a clearer distinction between data art and data visualization; he highlighted the negative consequences when data art ``masquerades as data visualization'', such as making visualizations difficult to perceive. Willers~\cite{willers2014show} thought the artistic approach is ``unlikely be appreciated if the intention was for rapid decision making.''
% In an interview, American artist and technologist Harris commented that ``data can be made pretty by design, but this is a superficial prettiness, like a boring woman wearing too much makeup.''~\cite{harris2015beauty} 
To address these gaps, more empirical research needs to be conducted to explore how artistic data visualizations are designed, their underlying pursuits, and how they might inspire our community.




% Examining this field not only helps us understand the latest application of data visualization in various domains but also extends our understanding of the aesthetic and humanistic aspects of data visualization.
% there should be more theoretical investigation into artistic data visualization. 

% These three concepts emphasize placing or installing visualizations at physical places that people will encounter in their daily lives. 

% ~\cite{wang2019emotional}


% gap between art and science~\cite{judelman2004aesthetics}
% constructive visualization~\cite{huron2014constructive}
% data feminism~\cite{d2020data}
% critical infovis~\cite{dork2013critical}
% citizen data and participation~\cite{valkanova2015public}

% \x{Lee~\etal~\cite{lee2013sketchstory}, give users artistic freedom to create their own visualizations.}


% Aesthetics refers to the study of beauty, taste, and sensory perception and is deeply intertwined with art.
% a particular taste for or approach to what is pleasing to the senses and especially sight

% why shouldn't all charts be scatter plot~\cite{bertini2020shouldn}
% aesthetic model~\cite{lau2007towards}
% Aesthetics for Communicative Visualization : a Brief Review
% Stacked graphs--geometry \& aesthetics~\cite{byron2008stacked}
% storyline optimization~\cite{tanahashi2012design}
% graphic designers rate the attractiveness of non-standard and pictorial visualizations higher than standard and abstract ones, whereas the opposite is true for laypeople.~\cite{quispel2014would}
% evaluate aesthetics - wordcloud
% An Evaluation of Semantically Grouped Word Cloud Designs, tag cloud, wordle

% On the other hand, empirical studies conducted with designers have shown that functionality is not the only design goal of visualization. For example, Bigelow~\etal~\cite{bigelow2014reflections} found that designers would frequently fine-tune the non-data elements in their designs, and data encoding was even "a later consideration with respect to other visual elements of the infographic".
% Moere~\cite{moere2011role} - design
% Quispel~\etal~\cite{quispel2018aesthetics} found that for information designers, clarity and aesthetics are both important for making a design attractive.
\section{Conclusions and Summary}
\label{sec:conclusions-summary}

This study introduced a multi-question chain-of-thought prompt strategy that significantly enhances the reasoning capabilities of large language models. By generating and answering a series of sub-questions before addressing the primary query, our method improves response accuracy over traditional baselines and established techniques such as canonical chain-of-thought and RaR \cite{dengRephraseRespondLet2024}. Experiments with Llama 3 models and GPT-4o on several Q\&A datasets show that our approach outperforms existing methods, highlighting its effectiveness.

These results show how carefully designed prompts can improve multi-step reasoning in large language models. They also point to the value of exploring adaptive prompt techniques across different NLP tasks. As these models evolve, multi-question prompting may further sharpen automated reasoning and foster more dependable AI interactions.


\section*{Limitations and Future Plans}

While our multi-question chain of thought prompt strategy has demonstrated notable improvements in reasoning capabilities and response accuracy of large language models, several limitations should be acknowledged. Firstly, the method requires fine-tuning of the number of intermediate questions (3, 5, 10 or other), and this may not be optimal or applicable across varying query complexities or domains. Choosing the appropriate number of questions is important, as an incorrect configuration might lead to redundancy or insufficient exploration of the query context.

Secondly, our approach was evaluated only on specific Q\&A datasets, which may not encompass the full spectrum of topics and question types. Therefore, the generalizability of this technique to other domains, such as dialogue systems or more complex multi-turn interactions, remains to be tested. Additionally, while our experiments utilized the Llama 3 models and GPT-4o, the effectiveness of this strategy across other architectures or smaller-scale models could differ.

Another limitation is the potential increase in computational resources required to generate and answer multiple intermediate questions, which could impact the efficiency and scalability of deploying these models in real-time applications. 

% Lastly, since we utilized instruction-following models, they inherently understand how to integrate and handle external knowledge; however, this is not universally applicable to all language models, which may sometimes result in logical inconsistencies or off-topic intermediate questions.

Future research should focus on addressing these limitations by exploring adaptive mechanisms for intermediate question generation, extending validation across more diverse datasets and models, and optimizing computational requirements to ensure broader applicability and effectiveness.

\section*{Ethics Statement}

Throughout our research, we carefully considered the ethical aspects of developing advanced language models. Our technique aims to enhance reasoning and accuracy, but we recognize the need to address potential ethical issues.

One concern is that improved reasoning could result in producing more persuasive but misleading or harmful content. To counteract this, it is essential to implement safeguards ensuring responses are accurate, unbiased, and factual. Future efforts should continue to monitor outputs for bias and misinformation, incorporating methods to mitigate these risks.

Additionally, the increased computational demand for generating intermediate questions raises environmental concerns about energy consumption. We advocate for continued research into optimizing the efficiency of these processes to minimize ecological impact.

We prioritized privacy and security by using only publicly available data in our experiments, free of private information. Adhering to transparency and reproducibility principles, we documented our methodologies, to facilitate replication of our findings by others.


\bibliography{anthology,library}

\appendix
\newpage
\clearpage
% \section{Framework Details}
% Our framework is described in Algorithm~\ref{algorithm}, and compared with former baselines in Table~\ref{table:comparison}. Distinct with several methods generating Python code for visualization directly, we use VQL as an intermediate representation to bridge natural language queries and visualization code. Additionally, our framework can be easily optimized by adding some useful tools such as Retrieval Augmented Generation. Moreover, our method supports handling multi-table data and the visualization can be customized according to humans' preferences. Our framework utilizes the agent-based collaborative workflow, which consists of data preprocessing, generation, and error correction, organized with the modular design.

% \begin{algorithm}
% \small
% \caption{\system Framework}
% \label{algorithm}
% \begin{algorithmic}[1]
% \Function{\nlvis}{$Q$, $S$}
%     \State Initialize $Mem \gets \{Q,S\}$
%     \State $(S', A) \gets \textsc{Processor}(Mem)$
%     \State $Mem.update(S', A)$
%     \State $V \gets \textsc{Composer}(Mem)$
%     \State $Mem.update(V)$
%     \State $Chart, isValid \gets \textsc{Validator}(Mem)$
%     \While{not $isValid$}
%         \State $V \gets \textsc{Refine}(Mem)$
%         \State $Mem.update(V)$
%         \State $Chart, isValid \gets \textsc{Validator}(Mem)$
%     \EndWhile
%     \State \Return $Chart$
% \EndFunction
% \end{algorithmic}

% \end{algorithm}




% \begin{table*}[!t]
%     \centering
    
%     \vspace{-1em}
%     \scalebox{0.68}{
%     \begin{tabular}{lccccccc}
%         \toprule[1.5pt]
%         \multirow{3}{*}{\textbf{Framework}} & \multicolumn{2}{c}{\textbf{System Features}} & \multicolumn{2}{c}{\textbf{Visualization Capabilities}} & \multicolumn{3}{c}{\textbf{Agentic Workflow}} \\
%         \cmidrule(lr){2-3} \cmidrule(lr){4-5} \cmidrule(lr){6-8}
%         & \textbf{VQL as} & \textbf{Extensible} & \textbf{Multi-Table} & \textbf{Customizable} & \textbf{Data} & \textbf{Modular} & \textbf{Error-} \\
%         & \textbf{Thoughts} & \textbf{Optimization} & \textbf{Support} & \textbf{Styling} & \textbf{Preprocess} & \textbf{Design} & \textbf{Correction} \\
%         \midrule
%         Chat2VIS~\cite{chat2vis} & \textcolor{red}{\ding{56}} & \textcolor{red}{\ding{56}} & \textcolor{red}{\ding{56}} & \textcolor{red}{\ding{56}} & \textcolor{green!60!black}{\ding{52}} & \textcolor{red}{\ding{56}} & \textcolor{red}{\ding{56}} \\
%         Mirror~\cite{mirror} & \textcolor{red}{\ding{56}} & \textcolor{red}{\ding{56}} & \textcolor{red}{\ding{56}} & \textcolor{red}{\ding{56}} & \textcolor{red}{\ding{56}} & \textcolor{green!60!black}{\ding{52}} & \textcolor{red}{\ding{56}} \\
        
%         LIDA~\cite{lida} & \textcolor{red}{\ding{56}} & \textcolor{green!60!black}{\ding{52}} & \textcolor{red}{\ding{56}} & \textcolor{green!60!black}{\ding{52}} & \textcolor{green!60!black}{\ding{52}} & \textcolor{green!60!black}{\ding{52}} & \textcolor{red}{\ding{56}} \\
%         CoML4VIS~\cite{coml} & \textcolor{red}{\ding{56}} & \textcolor{red}{\ding{56}} & \textcolor{green!60!black}{\ding{52}} & \textcolor{red}{\ding{56}} & \textcolor{green!60!black}{\ding{52}} & \textcolor{red}{\ding{56}} & \textcolor{red}{\ding{56}} \\
        
%         Prompt4VIS~\cite{prompt4vis} & \textcolor{green!60!black}{\ding{52}} & \textcolor{red}{\ding{56}} & \textcolor{green!60!black}{\ding{52}} & \textcolor{red}{\ding{56}} & \textcolor{green!60!black}{\ding{52}} & \textcolor{green!60!black}{\ding{52}} & \textcolor{red}{\ding{56}} \\
        
%         CoT-Vis~\cite{cotvis} & \textcolor{green!60!black}{\ding{52}} & \textcolor{red}{\ding{56}} & \textcolor{red}{\ding{56}} & \textcolor{red}{\ding{56}} & \textcolor{green!60!black}{\ding{52}} & \textcolor{red}{\ding{56}} & \textcolor{red}{\ding{56}} \\

%         \midrule
%         \SystemName (Ours) & \textcolor{green!60!black}{\ding{52}} & \textcolor{green!60!black}{\ding{52}} & \textcolor{green!60!black}{\ding{52}} & \textcolor{green!60!black}{\ding{52}} & \textcolor{green!60!black}{\ding{52}} & \textcolor{green!60!black}{\ding{52}} & \textcolor{green!60!black}{\ding{52}} \\
%         \bottomrule[1.5pt]
%     \end{tabular}}
% \caption{Comparison of various \nlvis frameworks. }  \label{table:comparison}
% \vspace{-1em}
% \end{table*}

\section{Detailed Experiment Setups}
\label{detailed_experiment_setups}
\paragraph{Baselines.}
\label{detailed_baselines}
% We implemented our experiment compared with three recent baselines. Note that, we also tried to use Code Interpreter as a baseline, but due to the rate limit of API constraint, the evaluation failed to generate visualizations via direct .csv files.
This study compares our approach with three state-of-the-art baselines. We also attempted to include Code Interpreter as a baseline; however, API rate limitations prevent the direct generation of visualizations from CSV files.

\begin{itemize}[leftmargin=*, itemsep=0pt] 
    \item \textbf{Chat2Vis} \cite{chat2vis}: It generates data visualizations by leveraging prompt engineering to translate natural language descriptions into visualizations. It uses a language-based table description, which includes column types and sample values, to inform the visualization generation process.\item \textbf{LIDA} \cite{lida}: It structures visualization generation as a four-step process, where each step builds on the previous one to incrementally translate natural language inputs into visualizations. It uses a JSON format to describe column statistics and samples, making it adaptable across various visualization tasks.
    \item \textbf{CoML4Vis} \cite{coml}: 
    % Building on a data science code generation framework, CoML4Vis 
    It utilizes a few-shot prompt that integrates multiple tables into a single visualization task. It summarizes data table information, including column names and samples, and then applies a few-shot prompt to guide visualization generation.
\end{itemize}

\paragraph{Metrics.}
\label{detailed_metrics}
Our evaluation framework involves five main metrics:
\begin{itemize}[leftmargin=*, itemsep=0pt] 
    \item \textbf{Invalid Rate} represents the percentage of visualizations that fail to render due to issues like incorrect API usage or other code errors.
    \item \textbf{Illegal Rate} indicates the percentage of visualizations that do not meet query requirements, which can include incorrect data transformations, mismatched chart types, or improper visualizations.
    \item \textbf{Readability Score} is the average score (range 1-5) assigned by a vision language model, like GPT-4V, for valid and legal visualizations, assessing their visual clarity and ease of interpretation.
    \item \textbf{Pass Rate} measures the proportion of visualizations in the evaluation set that are both valid (able to render) and legal (meet the query requirements).
    \item \textbf{Quality Score} is set to 0 for invalid or illegal visualizations; otherwise, it is equal to the readability score, providing an overall assessment of visualization quality factoring in both functionality and clarity.
\end{itemize}
To thoroughly evaluate each main metric, we further break them down into the following detailed assessment criteria:
\begin{itemize}[leftmargin=4mm, itemsep=0.05mm] 
    \item \textbf{Code Execution Check} verifies that the Python code generated by the model can be successfully executed.
    \item \textbf{Surface-form Check} ensures that the generated code includes necessary elements to produce a visualization like function calls to display the chart.
    \item \textbf{Chart Type Check} verifies whether the extracted chart type from the visualization matches the ground truth.
    \item \textbf{Data Check} assesses if the data used in the visualization matches the ground truth, taking into consideration potential channel swaps based on specified channels.
    \item \textbf{Order Check} evaluates whether the sorting of visual elements follows the specified query requirements.
    \item \textbf{Layout Check} examines issues like text overflow or element overlap within visualizations.
    \item \textbf{Scale \& Ticks Check} ensures that scales and ticks are appropriately chosen, avoiding unconventional representations.
    \item \textbf{Overall Readability Rating} integrates various readability checks to provide a comprehensive score considering layout, scale, text clarity, and arrangement.
\end{itemize}

% For all evaluation results, these metrics are averaged across the dataset to provide an overarching view of model performance. These metrics collectively ensure that visualizations are not only correct in terms of execution but also effective in communicating the intended data narratives.
The evaluation metrics are averaged across the dataset to provide a comprehensive overview of the model's performance. Together, these metrics ensure that the visualizations are both accurate in execution and effective in conveying the intended data narratives.



\begin{table}[!t]
\centering
\setlength{\belowcaptionskip}{0em} 
% \vspace{-1em}
\begin{tabular}{lcc}
\toprule[1.5pt]
\textbf{Model} & \textbf{P-corr} & \textbf{P-value} \\
\midrule
GPT-4o-mini & \textbf{0.6503} & 0.000 \\
GPT-4o & 0.5648 & 0.000 \\
\bottomrule[1.5pt]
\end{tabular}
\caption{ The Pearson correlations of GPT-4o-mini and GPT-4o with human judgments on readability scores. }
\label{tab:pearson_corr}
\vspace{-1em}
\end{table}

\begin{table*}[!ht]
\centering

\vspace{-1em}
\begin{tabular}{l|ccc|ccc}
\toprule
\multirow{2}{*}{Method} & \multicolumn{3}{c|}{Single Table} & \multicolumn{3}{c}{Multiple Tables} \\
\cmidrule(l){2-4} \cmidrule(l){5-7}
 & prompt & response & total & prompt & response & total \\
\midrule
LIDA & 1386.23 & 237.90 & 1624.13 & \multicolumn{3}{c}{N/A} \\
Chat2Vis & 414.35 & 451.30 & 865.65 & \multicolumn{3}{c}{N/A} \\
CoML4Vis & 2614.76 & 279.86 & 2894.62 & 3069.62 & 307.67 & 3377.29 \\
\system & 5122.99 & 777.63 & 5900.62 & 5613.96 & 1014.10 & 6628.06 \\
\bottomrule
\end{tabular}
\caption{Token usage comparison for different methods. N/A indicates that LIDA and Chat2Vis cannot handle multiple table scenarios.}
\label{tab:token_usage}
\end{table*}

\begin{table}[ht]
\centering
\scalebox{1}{
\begin{tabular}{l|ccc}
\toprule
Agent & \#Input & \#Output & \#Total \\
\midrule
Processor & 1486.07 & 569.58 & 1755.65\\
Composer & 3268.32 & 221.74 & 3490.07 \\
Validator & 1051.82 & 127.85 & 1179.67  \\
\bottomrule
\end{tabular}}
\caption{Token usage of three agents in \system.} \label{tab:token_agent} 
\vspace{-1em}
\end{table}

\paragraph{Implement Details.}
Our system is implemented in Python 3.9, utilizing GPT-4o \citep{openai_gpt4o_2024}, GPT-4o-mini~\cite{openai2024gpt4omini}, and GPT-3.5-turbo~\cite{chatgpt3.5} as the backbone model for all approaches, with the temperature set to 0 for consistent outputs. GPT-4o-mini serves as the vision language model for readability evaluation. We interact with these models through the Azure OpenAI API. The specific prompt templates for each agent, crucial for guiding their respective roles in the visualization generation process, are detailed in Appendix~\ref{prompt_details}. Token usages of \system and baselines are demonstrated in Table~\ref{tab:token_usage}, and usage for each agent in our \system is shown in Table~\ref{tab:token_agent}. Additionally, our evaluations are conducted in VisEval Benchmark (with MIT license).

\paragraph{Human Annotation.}
\label{human}
The annotation is conducted by 5 authors of this paper independently. As acknowledged, the diversity of annotators plays a crucial role in reducing bias and enhancing the reliability of the benchmark. These annotators have knowledge in the data visualization domain, with different genders, ages, and educational backgrounds. The educational backgrounds of annotators are above undergraduate. To ensure the annotators can proficiently mark the data, we provide them with detailed tutorials, teaching them how to judge the quality of data visualization. We also provide them with detailed criteria and task requirements in each annotation process shown in Figure~\ref{fig:annotation}. Two experiments requiring human annotation are detailed as follows:

\begin{figure}[!ht]
    \centering
    \includegraphics[width=\linewidth]{figure/score_distribution.pdf}
    \caption{Comparison of score density distribution between GPT-4o, GPT-4o-mini and human average score.}
    \label{fig:score_distribution}
\end{figure}

\begin{table*}[!ht]
\centering
\begin{tabular}{l|ccc}
\toprule
& Invalid Rate & Illegal Rate & Pass Rate \\
\midrule
\system & 4.66\% & 23.97\% & 71.35\% \\
w. CoT for Validator & 5.82\% & 23.39\% & 70.78\% \\
w. original schema for Validator & 4.80\% & 24.22\% & 70.97\% \\
\bottomrule
\end{tabular}
\caption{Additional exploration for Validator (using GPT-3.5-turbo).} 
\vspace{-1em} 
\label{tab:ablation_validator}
\end{table*}

\begin{itemize}[leftmargin=*, itemsep=0pt]
    \item \textbf{Pearson Correlation of Visual Language Model.} We conduct human annotation frameworks to compare the ability of the visual language model for MLLM-as-a-Judge~\cite{chen2024mllm}, providing the readability score. Our annotation framework is shown in Figure~\ref{fig:annotation}. The final Pearson scores are demonstrated in Table~\ref{tab:pearson_corr}, with its density distribution in Figure~\ref{fig:score_distribution}. The detailed instructions can be found in Figure~\ref{fig:scoring_instructions}.
    \item \textbf{Qualitative comparison to calculate ELO Scores.} We conduct human-judgments evaluations to compare which visualization generated by different models meets the query requirement more precisely. The leaderboard is shown in Table~\ref{tab:elo_rankings}, and Figure~\ref{fig:elo} shows the judgment framework. Each model starts with a base ELO score of 1500. After each pairwise comparison, the scores are updated based on the outcome and the current scores of the models involved. The hyperparameters are set as follows: the $K$-factor is set to 32, which determines the maximum change in rating after a single comparison. We conduct two sets of evaluations: one for single-table queries and another for multiple-table queries, with 1000 bootstrap iterations for each set to ensure statistical robustness. For each model's ELO rating, we report the 95\% confidence intervals computed through bootstrap resampling, providing a measure of rating stability. The evaluation process involves presenting human judges with a query and two visualizations, asking them to select the one that better meets the query requirements. This process is repeated across all model pairs and queries in our test set. The detailed guidance provides to the human evaluators can be found in Figure~\ref{fig:evaluation_instructions}, which outlines the criteria for judging visualization quality and relevance to the given query.


\end{itemize}

\begin{figure}[!ht]
	\centering
    \setlength{\belowcaptionskip}{-1em}
	\includegraphics[width=0.98\linewidth,scale=1.0]
    {./figure/library.pdf}
    \vspace{-1em}
	\caption{Performance of different models using \texttt{Matplotlib} and \texttt{Seaborn} libraries, using GPT-3.5-turbo.
    % \yao{larger fontsize?}
    }
\label{fig: library}
\end{figure}

\begin{figure*}[!h]
    \centering
    \includegraphics[width=0.98\linewidth]{figure/annotation.pdf}
    \caption{Screenshot of human annotation process in readability score.}
    \label{fig:annotation}
\end{figure*}

\begin{figure*}[ht]
\centering
\vspace{1em}
\begin{tcolorbox}[enhanced,attach boxed title to top center={yshift=-3mm,yshifttext=-1mm},boxrule=0.9pt, 
  colback=gray!00,colframe=black!50,colbacktitle=gray,
  title=Readability Scoring Instruction,
  boxed title style={size=small,colframe=gray} ]
\small
\textbf{Scoring Instructions:} Please evaluate the charts based on the following criteria, with a score range from 1 to 5, where 1 indicates very poor quality and 5 indicates excellent quality. You should focus on the following aspects:

\vspace{0.5em}
\textbf{1. Chart Colors:}
\begin{itemize}
    \item Are the colors clear and natural, effectively conveying the information?
    \item Color blindness accessibility: Are the color combinations easy to distinguish, especially for users with color blindness?
\end{itemize}

\vspace{0.5em}
\textbf{2. Title and Axis Labels:}
\begin{itemize}
    \item Ensure the chart has a clear title.
    \item Do the X-axis and Y-axis labels exist, and are they complete?
    \item Check if the labels are difficult to read, e.g., are they written vertically instead of horizontally?
    \item The title should not be a direct question; instead, it should describe the data or trends being presented.
\end{itemize}

\vspace{0.5em}
\textbf{3. Legend Completeness:}
\begin{itemize}
    \item Is the legend complete, and does it clearly indicate the color labels for different data series?
    \item Ensure each color has a corresponding legend, making it easy for users to understand what the data represents.
\end{itemize}

\vspace{0.5em}
\textbf{Scoring Scale:}
\begin{itemize}
    \item \textbf{1 Point:} Very poor, unable to understand or severely lacking information.
    \item \textbf{2 Points:} Poor quality, multiple issues present, difficult to extract information.
    \item \textbf{3 Points:} Fair, conveys some information but still has room for improvement.
    \item \textbf{4 Points:} Good, generally clear charts with minor areas for improvement.
    \item \textbf{5 Points:} Excellent, outstanding chart design with clear and effective information presentation.
\end{itemize}

Please consider the above factors when assessing the charts and provide the appropriate score. Thank you for your cooperation and effort!
\end{tcolorbox}
\vspace{-7pt}
\caption{Instructions for human annorators in annotating readability scoring.}
\label{fig:scoring_instructions}
\vspace{1em}
\end{figure*}

\begin{figure*}[!ht]
    \centering
    \includegraphics[width=0.98\linewidth]{figure/elo.pdf}
    \caption{Screenshot of ELO score evaluation framework for Human-as-a-Judge.}
    \label{fig:elo}
\end{figure*}

\begin{figure*}[ht]
\centering
\vspace{1em}
\begin{tcolorbox}[enhanced,attach boxed title to top center={yshift=-3mm,yshifttext=-1mm},boxrule=0.9pt, 
  colback=gray!00,colframe=black!50,colbacktitle=gray,
  title=Visualization Comparison Guidance,
  boxed title style={size=small,colframe=gray} ]
\small
Welcome to the visualization comparison evaluation. Your task is to judge which model-generated visualization better meets the requirements of the natural language query.

\vspace{0.5em}
\textbf{Evaluation criteria:}
\begin{enumerate}
    \item \textbf{Appropriateness of chart type:} Check if the selected chart type is suitable for expressing the data and relationships required by the query.
    \item \textbf{Data completeness:} Ensure the chart includes all necessary data required by the query.
    \item \textbf{Readability:} Assess the clarity of the chart, accuracy of labels, and overall layout.
    \item \textbf{Aesthetics:} Consider if the chart's color scheme, proportions, and overall design are visually pleasing.
    \item \textbf{Information conveyance:} Judge if the chart effectively conveys the main information or insights required by the query.
\end{enumerate}

\vspace{0.5em}
\textbf{Evaluation process:}
\begin{enumerate}
    \item Carefully read the natural language query.
    \item Observe the visualization results generated by two models.
    \item Based on the above criteria, choose the better visualization or select a tie if they are equally good.
    \item If neither visualization satisfies the query requirements well, please choose the relatively better one.
\end{enumerate}

Remember, your evaluation will help us improve and compare different visualization models. Thank you for your participation!
\end{tcolorbox}
\vspace{-7pt}
\caption{Instructions for human annorators in visualization comparison.}
\label{fig:evaluation_instructions}
\vspace{1em}
\end{figure*}


\section{Additional Experiment Results}
\label{additional_experiment_result}

We also conducted a comparison experiment of different methods using matplotlib or seaborn library. Figure~\ref{fig: library} demonstrates the results, indicating that our method outperforms obviously other baselines not only with matplotlib but also seaborn.

In addition, we test techniques in the Validator Agent, such as Chain-of-Thought. As is shown in Table~\ref{tab:ablation_validator}, integrating Chain-of-Thought reasoning, may affect its performance badly, likely due to the simple refining task with complex reasoning. Moreover, using the original schema to check for false schema filtering seems to be useless in this case.

\section{Evaluation Results with Detailed Metrics}
We demonstrated the main results in Table~\ref{tab:performance_comparison}, and here we reported more detailed results of other metrics in Table~\ref{tab:detailed_results}, which underscored the error rates for each stage, including \textit{Invalid}, \textit{Illegal}, and \textit{Low Readability}. 

\begin{table*}[!ht]
\centering
\footnotesize
\scalebox{0.98}{
\begin{tabular}{ll|cc|cccc|cc}
\toprule[1.5pt]
\multirow{2}{*}{Method} & \multirow{2}{*}{Dataset} & \multicolumn{2}{c|}{Invalid} & \multicolumn{4}{c|}{Illegal} & \multicolumn{2}{c}{Low Readability} \\
&  & Execution & Surface. & Decon. & Chart Type & Data & Order & Layout & Scale\&Ticks \\
\midrule
\multicolumn{10}{c}{ \textbf{\textit{GPT-4o}}}\\
\midrule
\multirow{3}{*}{CoML4Vis} & All & 1.15 & 0.00 & 0.26 & 1.75 & 14.28 & 10.36 & 32.02 & 32.55 \\
& Single & 0.67 & 0.00 & 0.43 & 1.93 & 13.54 & 10.16 & 31.08 & 32.76 \\
& Multiple & 1.87 & 0.00 & 0.00 & 1.48 & 15.39 & 10.66 & 33.43 & 32.23 \\
\cmidrule{2-10}
\multirow{3}{*}{LIDA} & All & 6.61 & 0.00 & 1.60 & 3.24 & 40.53 & 4.07 & 32.68 & 15.77 \\
& Single & 1.13 & 0.00 & 2.11 & 0.89 & 12.26 & 6.79 & 53.93 & 26.22 \\
& Multiple & 14.80 & 0.00 & 0.79 & 8.51 & 80.53 & 0.00 & 1.24 & 0.21 \\
\cmidrule{2-10}
\multirow{3}{*}{Chat2Vis} & All & 16.05 & 0.00 & 0.62 & 3.99 & 30.14 & 5.96 & 2.37 & 20.88 \\
& Single & 0.86 & 0.00 & 0.75 & 2.30 & 10.78 & 9.73 & 3.97 & 34.63 \\
& Multiple & 38.74 & 0.00 & 0.43 & 6.51 & 59.08 & 0.32 & 0.00 & 0.34 \\
\cmidrule{2-10}
\multirow{3}{*}{nvAgent} & All & 0.97 & 0.00 & 0.08 & 1.28 & 11.07 & 4.05 & 5.07 & 40.03 \\
& Single & 0.72 & 0.00 & 0.14 & 1.27 & 9.88 & 3.60 & 3.92 & 39.36 \\
& Multiple & 1.34 & 0.00 & 0.00 & 1.30 & 12.84 & 4.73 & 6.79 & 41.03 \\
\midrule
\multicolumn{10}{c}{ \textbf{\textit{GPT-4o-mini}}}\\
\midrule
\multirow{3}{*}{CoML4Vis} & All & 4.23 & 0.00 & 0.20 & 2.31 & 16.64 & 11.83 & 35.23 & 29.35 \\
& Single & 0.36 & 0.00 & 0.26 & 2.32 & 13.80 & 11.67 & 35.92 & 32.22 \\
& Multiple & 10.01 & 0.00 & 0.10 & 2.31 & 20.87 & 12.07 & 34.19 & 25.05 \\
\cmidrule{2-10}
\multirow{3}{*}{LIDA} & All & 12.50 & 0.00 & 0.40 & 4.92 & 40.02 & 5.80 & 27.87 & 17.05 \\
& Single & 9.09 & 0.00 & 0.44 & 2.53 & 12.91 & 9.68 & 45.69 & 28.32 \\
& Multiple & 17.61 & 0.00 & 0.33 & 8.51 & 80.53 & 0.00 & 1.24 & 0.21 \\
\cmidrule{2-10}
\multirow{3}{*}{Chat2Vis} & All & 15.45 & 0.17 & 0.17 & 4.21 & 31.90 & 8.20 & 2.14 & 18.97 \\
& Single & 2.14 & 0.29 & 0.41 & 2.53 & 11.99 & 9.68 & 45.69 & 28.32 \\
& Multiple & 35.78 & 0.00 & 0.00 & 6.70 & 61.66 & 0.00 & 0.92 & 0.32 \\
\cmidrule{2-10}
\multirow{3}{*}{nvAgent} & All & 5.14 & 0.00 & 0.00 & 2.40 & 16.33 & 10.61 & 41.06 & 27.00 \\
& Single & 1.97 & 0.00 & 0.14 & 2.97 & 15.21 & 7.49 & 39.30 & 32.39 \\
& Multiple & 8.15 & 0.00 & 0.00 & 2.31 & 20.87 & 12.07 & 34.19 & 25.05 \\
\midrule
\multicolumn{10}{c}{ \textbf{\textit{GPT-3.5-turbo}}}\\
\midrule
\multirow{3}{*}{CoML4Vis} & All & 9.28 & 0.00 & 0.62 & 1.91 & 15.83 & 12.86 & 25.09 & 27.73 \\ 
& Single & 6.17 & 0.00 & 0.89 & 2.50 & 14.71 & 13.20 & 26.10 & 29.93 \\ 
& Multiple & 13.92 & 0.00 & 0.21 & 1.04 & 17.51 & 12.36 & 23.57 & 24.43 \\ 
\cmidrule{2-10} 
\multirow{3}{*}{LIDA} & All & 53.43 & 0.00 & 1.27 & 3.56 & 22.33 & 0.53 & 14.90 & 6.62 \\ 
& Single & 47.32 & 0.00 & 1.91 & 2.81 & 13.03 & 0.89 & 24.43 & 11.05 \\ 
& Multiple & 62.57 & 0.00 & 0.32 & 4.68 & 36.23 & 0.00 & 0.65 & 0.00 \\ 
\cmidrule{2-10} 
\multirow{3}{*}{Chat2Vis} & All & 18.68 & 0.00 & 0.28 & 3.66 & 32.47 & 7.20 & 25.45 & 20.15 \\ 
& Single & 3.90 & 0.00 & 0.47 & 2.78 & 15.62 & 12.01 & 41.74 & 33.38 \\ 
& Multiple & 40.77 & 0.00 & 0.00 & 4.97 & 57.66 & 0.00 & 1.12 & 0.37 \\ 
\cmidrule{2-10} 
\multirow{3}{*}{nvAgent} & All & 4.66 & 0.00 & 0.08 & 3.06 & 18.24 & 5.64 & 5.25 & 35.34 \\ 
& Single & 2.98 & 0.00 & 0.14 & 2.84 & 15.08 & 5.69 & 3.62 & 37.57 \\ 
& Multiple & 7.18 & 0.00 & 0.00 & 3.38 & 22.95 & 5.56 & 7.69 & 32.02 \\
\bottomrule[1.5pt]
\end{tabular}
}
\caption{Detailed error rates (\%) for different methods.} 
\label{tab:detailed_results}
\end{table*}

\section{Case Study}
\label{example}
% To demonstrate our approach's effectiveness, we present several illustrative examples. Figure~\ref{fig:nl_vql} shows how our system translates natural language into a structured VQL representation. Figure~\ref{python code} and Figure~\ref{fig:example_chart} demonstrate the complete pipeline from query to visualization.
Figure~\ref{fig:nl_vql} shows an example of a natural language query with its corresponding VQL representation. The output Python code for visualization and the final bar chart are demonstrated in Figure~\ref{python code} and Figure~\ref{fig:example_chart}, respectively.
Furthermore, we provide a case study of \system performance on four hardness-level NL2Vis problems in VisEval in Figure \ref{hardness case}.

The easy case demonstrates accurate grouping in scatter plot relationships. The medium case shows correct handling of multi-table joins for continent-wise statistics. The hard case exhibits temporal data visualization with proper filtering. The extra hard case showcases complex operations including weekday binning and stacked visualization. These cases highlight our system's consistent performance across varying task complexities, particularly excelling in multiple table scenarios and complex aggregations.

\begin{figure*}[htbp]
\centering
\begin{tcolorbox}[enhanced,attach boxed title to top center={yshift=-3mm,yshifttext=-1mm},boxrule=0.9pt, 
  colback=gray!00,colframe=black!50,colbacktitle=gray,
  title=An Example of Natural Language Query and  Corresponding VQL,,
  boxed title style={size=small,colframe=gray} ]

\textbf{Natural Language Query:}\\
How many documents are stored? Bin the store date by weekday in a bar chart.\\
\tcbline
\textbf{Corresponding VQL:}\\
Visualize BAR \\
SELECT Date\_Stored, COUNT(Document\_ID)\\
FROM All\_Documents \\
GROUP BY Date\_Stored \\
BIN Date\_Stored BY WEEKDAY\\
\end{tcolorbox}
\caption{The natural language query case and its corresponding output VQL representation.}
\label{fig:nl_vql}
\end{figure*}

\lstset{
    basicstyle=\ttfamily\small,
    breaklines=true,
    numbers=left,
    numberstyle=\tiny,
    frame=single,
    showstringspaces=false,
    tabsize=4,
    keywordstyle=\color{blue},
    commentstyle=\color{green!60!black},
    stringstyle=\color{purple},
    breakatwhitespace=false,
    keepspaces=true
}
\begin{figure*}[htbp]
\begin{lstlisting}[
    language=Python,
    frame=single,
    numbers=left,
    numberstyle=\tiny,
    basicstyle=\tiny\ttfamily,    
    breaklines=true,
    breakatwhitespace=false,
    showstringspaces=false,
    framesep=1mm,                 % 减小框架边距
    framexleftmargin=1mm,         % 减小框架左边距
    columns=flexible,             % 灵活列宽
    keepspaces=true
]
import matplotlib.pyplot as plt 
import pandas as pd 
import os 
import duckdb

# Set data folder path
data_folder = 'E:/visEval_dataset/databases/cre_Doc_Tracking_DB'  

# Connect to database 
con = duckdb.connect(database=':memory:')  

# Read all CSV files and create view
csv_files = [f for f in os.listdir(data_folder) if f.endswith('.csv')] 
for file in csv_files:
    table_name = os.path.splitext(file)[0]
    con.execute(f"CREATE VIEW {table_name} AS SELECT * FROM read_csv_auto('{os.path.join(data_folder, file)}')")  
    
# Execute SQL query 
sql = f''' 
    SELECT Date_Stored, COUNT(Document_ID) AS count_Document_ID 
    FROM All_Documents 
    GROUP BY Date_Stored 
''' 
df = con.execute(sql).fetchdf() 
con.close()  

# Rename columns 
df.columns = ['Date_Stored','count_Document_ID'] 

# Apply binning operation
flag = True 
df['Date_Stored'] = pd.to_datetime(df['Date_Stored']) 
df['Date_Stored'] = df['Date_Stored'].dt.day_name()  

# Group by and calculate count 
if flag:
    df = df.groupby('Date_Stored').sum().reset_index() 

# Ensure all seven days of the week are included 
weekday_order = ['Monday', 'Tuesday', 'Wednesday', 'Thursday', 
                 'Friday', 'Saturday', 'Sunday'] 
df = df.set_index('Date_Stored').reindex(weekday_order, fill_value=0).reset_index()
df['Date_Stored'] = pd.Categorical(df['Date_Stored'], 
                                  categories=weekday_order, ordered=True) 
df = df.sort_values('Date_Stored')

# Create visualization 
fig, ax = plt.subplots(1, 1, figsize=(10, 4)) 
ax.spines['top'].set_visible(False) 
ax.spines['right'].set_visible(False) 
ax.bar(df['Date_Stored'], df['count_Document_ID']) 
ax.set_xlabel('Date_Stored') 
ax.set_ylabel('count_Document_ID') 
ax.set_title(f'BAR Chart of count_Document_ID by Date_Stored') 
plt.xticks(rotation=45) 
plt.tight_layout()  
plt.show()
\end{lstlisting}
\caption{An example of python code generating module within \system.}
\label{python code}
\end{figure*}


\begin{figure*}[!ht]
    \centering
    \includegraphics[width=0.98\linewidth,scale=1.0]{figure/bar_chart.pdf}
    \caption{An example of generated bar chart using \system.}
    \label{fig:example_chart}
\end{figure*}

\begin{figure*}[htbp]
\centering
\begin{tcolorbox}[enhanced,attach boxed title to top center={yshift=-3mm,yshifttext=-1mm},boxrule=0.9pt, 
  colback=gray!00,colframe=black!50,colbacktitle=gray,
  title=Examples of \textsc{nvAgent} performance on different hardness levels,
  boxed title style={size=small,colframe=gray} ]
  
\textbf{Hardness Level:} Easy \\
\begin{minipage}{0.45\linewidth}
    \textbf{Dataset}: \textit{Single}\\
    \textbf{Input Tables}: basketball\_match\\
    \textbf{Input Query}: Show the relation between acc percent and all\_games\_percent for each ACC\_Home using a grouped scatter chart.\\
\end{minipage}\hfill
\begin{minipage}{0.45\linewidth}
    \centering
    \textbf{Response}:
    \includegraphics[width=\linewidth]{figure/easy_3085.pdf} 
\end{minipage}
\tcbline

\textbf{Hardness Level:} Medium \\
\begin{minipage}{0.45\linewidth}
    \textbf{Dataset}: \textit{Multiple}\\
    \textbf{Input Tables}: car\_makers, car\_names, cars\_data, continents, countries, model\_list\\
    \textbf{Input Query}: Display a pie chart for what is the name of each continent and how many car makers are there in each one?\\
\end{minipage}\hfill
\begin{minipage}{0.55\linewidth}
    \centering
    \textbf{Response}:
    \includegraphics[width=\linewidth]{figure/medium_433.pdf} 
\end{minipage}
\tcbline

\textbf{Hardness Level:} Hard \\[1em]
\begin{minipage}{0.45\linewidth}
    \textbf{Dataset}: \textit{Multiple}\\
    \textbf{Input Tables}: advisor, classroom, course, department, instructor, prereq, section, student, takes, teaches, time\_slot\\
    \textbf{Input Query}: Find the number of courses offered by Psychology department in each year with a line chart.\\
\end{minipage}\hfill
\begin{minipage}{0.45\linewidth}
    \centering
    \textbf{Response}:
    \includegraphics[width=\linewidth]{figure/hard_611.pdf} 
\end{minipage}
\tcbline

\textbf{Hardness Level:} Extra Hard \\[1em]
\begin{minipage}{0.45\linewidth}
    \textbf{Dataset}: \textit{Multiple}\\
    \textbf{Input Tables}: Accounts, Documents, Documents\_with\_Expenses, Projects, Ref- \_Budget\_Codes, Ref\_Document\_Types, Statements\\
    \textbf{Input Query}: How many documents are created in each day? Bin the document date by weekday and group by document type description with a stacked bar chart, I want to sort Y in desc order.\\
\end{minipage}\hfill
\begin{minipage}{0.45\linewidth}
    \centering
    \textbf{Response}:
    \includegraphics[width=\linewidth]{figure/extra_851.pdf} 
\end{minipage}

\end{tcolorbox}
    \caption{Examples of \textsc{nvAgent}'s performance on different hardness levels in VisEval (easy, medium, hard, and extra hard.}
    \label{hardness case}
\end{figure*}


\clearpage
\onecolumn
\section{Prompts Details}
\label{prompt_details}
We provide detailed prompt design of our \system as follows.



\begin{promptbox}[Prompt template for Processor Agent]
You are an experienced and professional database administrator. Given a database schema and a user query, your task is to analyze the query, filter the relevant schema, generate an optimized representation, and classify the query difficulty. \\
\\
Now you can think step by step, following these instructions below. \\
\textbf{[Instructions]} \\
1. Schema Filtering: \\
\text{\ \ \ \ }- Identify the tables and columns that are relevant to the user query.\\
\text{\ \ \ \ }- Only exclude columns that are completely irrelevant.\\
\text{\ \ \ \ }- The output should be \{\{tables: [columns]\}\}.\\
\text{\ \ \ \ }- Keep the columns needed to be primary keys and foreign keys in the filtered schema.\\
\text{\ \ \ \ }- Keep the columns that seem to be similar with other columns of another table.\\
\\
2. New Schema Generation:\\
\text{\ \ \ \ }- Generate a new schema of the filtered schema, based on the given database schema and your filtered schema.\\
\\
3. Augmented Explanation:\\
\text{\ \ \ \ }- Provide a concise summary of the filtered schema to give additional knowledge.\\
\text{\ \ \ \ }- Include the number of tables, total columns, and any notable relationships or patterns.\\
\\
4. Classification:\\
For the database new schema, classify it as SINGLE or MULTIPLE based on the tables number.\\
\text{\ \ \ \ }- if tables number >= 2: predict MULTIPLE\\
\text{\ \ \ \ }- elif only one table: predict SINGLE\\
\\
==============================\\
Here is a typical example:\\
\textbf{[Database Schema]}\\
\textbf{[DB\_ID]} dorm\_1\\
\textbf{[Schema]}\\
\# Table: Student\\
\text{[}\\
  \text{\ \ \ \ }(stuid, And This is a id type column),\\
  \text{\ \ \ \ }(lname, Value examples: [`Smith', `Pang', `Lee', `Adams', `Nelson', `Wilson'].),\\
  \text{\ \ \ \ }(fname, Value examples: [`Eric', `Lisa', `David', `Sarah', `Paul', `Michael'].),\\
  \text{\ \ \ \ }(age, Value examples: [18, 20, 17, 19, 21, 22].),\\
  \text{\ \ \ \ }(sex, Value examples: [`M', `F'].),\\
  \text{\ \ \ \ }(major, Value examples: [600, 520, 550, 50, 540, 100].),\\
  \text{\ \ \ \ }(advisor, And this is a number type column),\\
  \text{\ \ \ \ }(city code, Value examples: [`PIT', `BAL', `NYC', `WAS', `HKG', `PHL'].)\\
\text{]}\\
% \end{promptbox}
% \end{figure*}
% \begin{figure*}[!h]
% \begin{promptbox}[Prompt template for Processor Agent]
\# Table: Dorm\\
\text{[}\\
  \text{\ \ \ \ }(dormid, And This is a id type column),\\
  \text{\ \ \ \ }(dorm name, Value examples: [`Anonymous Donor Hall', `Bud Jones Hall', `Dorm-plex 2000', `Fawlty Towers', `Grad Student Asylum', `Smith Hall'].),\\
  \text{\ \ \ \ }(student capacity, Value examples: [40, 85, 116, 128, 256, 355].),
  (gender, Value examples: [`X', `F', `M'].)\\
\text{]}\\
\# Table: Dorm\_amenity\\
\text{[}\\
  \text{\ \ \ \ }(amenid, And This is a id type column),\\
  \text{\ \ \ \ }(amenity name, Value examples: [`4 Walls', `Air Conditioning', `Allows Pets', `Carpeted Rooms', `Ethernet Ports', `Heat'].)\\
\text{]}\\
\# Table: Has\_amenity\\
\text{[}\\
  \text{\ \ \ \ }(dormid, And This is a id type column),\\
  \text{\ \ \ \ }(amenid, And This is a id type column)\\
\text{]}\\
\# Table: Lives\_in\\
\text{[}\\
  \text{\ \ \ \ }(stuid, And This is a id type column),\\
  \text{\ \ \ \ }(dormid, And This is a id type column),\\
  \text{\ \ \ \ }(room number, And this is a number type column)\\
\text{]}\\
\\
\textbf{[Query]}\\
Find the first name of students who are living in the Smith Hall, and count them by a pie chart\\
\\
Now we can think step by step\\
\textbf{[Filtered Schema]}\\
\{\\
  "Student": ["stuid", "fname"],\\
  "Dorm": ["dormid", "dorm name"],\\
  "Lives\_in": ["stuid", "dormid"]\\
\}\\
\\
\textbf{[New Schema]}\\
\# Table: Student\\
\text{[}\\
  (stuid, And This is a id type column),\\
  (fname, Value examples: [`Eric', `Lisa', `David', `Sarah', `Paul', `Michael'].),\\
\text{]}\\
% \end{promptbox}
% \end{figure*}
% \begin{figure*}[!h]
% \begin{promptbox}[Prompt template for Processor Agent]
\# Table: Dorm
\text{[}\\
  (dormid, And This is a id type column),\\
  (dorm name, Value examples: [`Anonymous Donor Hall', `Bud Jones Hall', `Dorm-plex 2000', `Fawlty Towers', `Grad Student Asylum', `Smith Hall'].),\\
\text{]}\\
\# Table: Lives\_in\\
\text{[}\\
  (stuid, And This is a id type column),\\
  (dormid, And This is a id type column),\\
\text{]}\\
\textbf{[Augmented Explanation]}\\
The filtered schema consists of 3 tables (Student, Dorm, and Lives\_in) with a total of 6 relevant columns. There is a many-to-one relationship between Student and Dorm through the Lives\_in junction table. The query involves joining these three tables to find students living in a specific dorm (Smith Hall).\\
\\
Key points:\\
1. The Lives\_in table acts as a bridge between Student and Dorm, allowing for the association of students with their dorms.\\
2. The `dorm name' column in the Dorm table is crucial for filtering the specific dorm (Smith Hall).\\
3. The `fname' column from the Student table is required for the final output.\\
\\
\textbf{[Classification]}\\
MULTIPLE\\
\\
==============================\\
Here is a new question:\\
\\
\textbf{[DB\_ID]} \{db\_id\}\\
\textbf{[Database Schema]}\\
\{db\_schema\}\\
\\
\textbf{[Query]}\\
\{query\}\\
\\
Now give your answer following this format strictly without other explanation:\\
\\
\textbf{[Filtered Schema]}\\
\\
\textbf{[New Schema]}\\
\\
\textbf{[Augmented Explanation]}\\
\\
\textbf{[Classification]}\\
\\
\end{promptbox}
% \end{figure*}

% \subsection{Composer Agent Prompt}
% \label{composer_prompt}
% \begin{figure*}[!h]
\begin{promptbox}[Prompt template for multiple classification]
Given a [Database schema] with [Augmented Explanation] and a [Question], generate a valid VQL (Visualization Query Language) sentence. VQL is similar to SQL but includes visualization components. \\
\\
Now you can think step by step, following these instructions below. \\
\textbf{[Background]} \\
VQL Structure:\\
Visualize [TYPE] SELECT [COLUMNS] FROM [TABLES] [JOIN] [WHERE] [GROUP BY] [ORDER BY] [BIN BY]\\
\\
You can consider a VQL sentence as "VIS TYPE + SQL + BINNING"\\
You must consider which part in the sketch is necessary, which is unnecessary, and construct a specific sketch for the natural language query.\\
\\
Key Components:\\
1. Visualization Type: bar, pie, line, scatter, stacked bar, grouped line, grouped scatter\\
2. SQL Components: SELECT, FROM, JOIN, WHERE, GROUP BY, ORDER BY\\
3. Binning: BIN [COLUMN] BY [INTERVAL], [INTERVAL]: [YEAR, MONTH, DAY, WEEKDAY]\\
\\
When generating VQL, we should always consider special rules and constraints:\\
\textbf{[Special Rules]} \\
a. For simple visualizations:\\
    \text{\ \ \ \ }- SELECT exactly TWO columns, X-axis and Y-axis(usually aggregate function)\\
b. For complex visualizations (STACKED BAR, GROUPED LINE, GROUPED SCATTER):\\
    \text{\ \ \ \ }- SELECT exactly THREE columns in this order!!!:\\
        \text{\ \ \ \ }\text{\ \ \ \ }1. X-axis\\
        \text{\ \ \ \ }\text{\ \ \ \ }2. Y-axis (aggregate function)\\
        \text{\ \ \ \ }\text{\ \ \ \ }3. Grouping column\\
c. When "COLORED BY" is mentioned in the question:\\
    \text{\ \ \ \ }- Use complex visualization type(STACKED BAR for bar charts, GROUPED LINE for line charts, GROUPED SCATTER for scatter charts)\\
    \text{\ \ \ \ }- Make the "COLORED BY" column the third SELECT column\\
    \text{\ \ \ \ }- Do NOT include "COLORED BY" in the final VQL\\     
d. Aggregate Functions:\\
    \text{\ \ \ \ }- Use COUNT for counting occurrences\\
    \text{\ \ \ \ }- Use SUM only for numeric columns\\
    \text{\ \ \ \ }- When in doubt, prefer COUNT over SUM\\
e. Time based questions:\\
    \text{\ \ \ \ }- Always use BIN BY clause at the end of VQL sentence\\
    \text{\ \ \ \ }- When you meet the questions including "year", "month", "day", "weekday"\\
    \text{\ \ \ \ }- Avoid using window function, just use BIN BY to deal with time base queries\\
% \end{promptbox}
% \end{figure*}
% \begin{figure*}[!h]
% \begin{promptbox}[Prompt template for multiple classification]
\textbf{[Constraints]} \\
- In SELECT <column>, make sure there are at least two selected!!!\\
- In FROM <table> or JOIN <table>, do not include unnecessary table\\
- Use only table names and column names from the given database schema\\
- Enclose string literals in single quotes\\
- If [Value examples] of <column> has `None' or None, use JOIN <table> or WHERE <column> is NOT NULL is better\\
- Ensure GROUP BY precedes ORDER BY for distinct values\\
- NEVER use window functions in SQL\\
\\
Now we could think step by step:\\
1. First choose visualize type and binning, then construct a specific sketch for the natural language query\\
2. Second generate SQL components following the sketch.\\
3. Third add Visualize type and BINNING into the SQL components to generate final VQL\\
\\
==============================\\
Here is a typical example:\\
\textbf{[Database Schema]}\\
\# Table: Orders, (orders)\\
\text{[}\\
  \text{\ \ \ \ }(order\_id, order id, And this is a id type column),\\
  \text{\ \ \ \ }(customer\_id, customer id, And this is a id type column),\\
  \text{\ \ \ \ }(order\_date, order date, Value examples: [`2023-01-15', `2023-02-20', `2023-03-10'].),\\
  \text{\ \ \ \ }(total\_amount, total amount, Value examples: [100.00, 200.00, 300.00, 400.00, 500.00].)\\
\text{]}\\
\# Table: Customers, (customers)\\
\text{[}\\
  \text{\ \ \ \ }(customer\_id, customer id, And this is a id type column),\\
  \text{\ \ \ \ }(customer\_name, customer name, Value examples: [`John', `Emma', `Michael', `Sophia', `William'].),\\
  \text{\ \ \ \ }(customer\_type, customer type, Value examples: [`Regular', `VIP', `New'].)\\
\text{]}\\
\textbf{[Augmented Explanation]}\\
The filtered schema consists of 2 tables (Orders and Customers) with a total of 7 relevant columns. There is a one-to-many relationship between Customers and Orders through the customer\_id foreign key.\\
\\
Key points:\\
1. The Orders table contains information about individual orders, including the order date and total amount.\\
2. The Customers table contains customer information, including their name and type (Regular, VIP, or New).\\
3. The customer\_id column links the two tables, allowing us to associate orders with specific customers.\\
% \end{promptbox}
% \end{figure*}
% \begin{figure*}[!h]
% \begin{promptbox}[Prompt template for multiple classification]
4. The order\_date column in the Orders table will be used for monthly grouping and binning.\\
5. The total\_amount column in the Orders table needs to be summed for each group.\\
6. The customer\_type column in the Customers table will be used for further grouping and as the third dimension in the stacked bar chart.\\
\\

The query involves joining these two tables to analyze order amounts by customer type and month, which requires aggregation and time-based binning.\\
\\
\textbf{[Question]}\\
Show the total order amount for each customer type by month in a stacked bar chart.\\
\\
Decompose the task into sub tasks, considering [Background] [Special Rules] [Constraints], and generate the VQL after thinking step by step:\\
\\
\textbf{Sub task 1:} First choose visualize type and binning, then construct a specific sketch for the natural language query\\
Visualize type: STACKED BAR, BINNING: True\\
VQL Sketch:\\
Visualize STACKED BAR SELECT \_ , \_ , \_ FROM \_ JOIN \_ ON \_ GROUP BY \_ BIN \_ BY MONTH\\
\\
\textbf{Sub task 2:} Second generate SQL components following the sketch.\\
Let's think step by step:\\
1. We need to select 3 columns for STACKED BAR chart, order\_date as X-axis, SUM(total\_amout) as Y-axis, customer\_type as group column.\\
2. We need to join the Orders and Customers tables.\\
3. We need to group by customer type.\\
4. We do not need to use any window function for MONTH.\\
\\
\text{sql}\\
```sql\\
SELECT O.order\_date, SUM(O.total\_amount), C.customer\_type\\
FROM Orders AS O\\
JOIN Customers AS C ON O.customer\_id = C.customer\_id\\
GROUP BY C.customer\_type\\
```\\
\\
\textbf{Sub task 3:} Third add Visualize type and BINNING into the SQL components to generate final VQL\\
\textbf{Final VQL:}\\
Visualize STACKED BAR SELECT O.order\_date, SUM(O.total\_amount), C.customer\_type FROM Orders O JOIN Customers C ON O.customer\_id = C.customer\_id GROUP BY C.customer\_type BIN O.order\_date BY MONTH\\
\\
% \end{promptbox}
% \end{figure*}
% \begin{figure*}[!h]
% \begin{promptbox}[Prompt template for multiple classification]
==============================\\
Here is a new question:\\
\\
\textbf{[Database Schema]}\\
\{desc\_str\}\\
\\
\textbf{[Augmented Explanation]}\\
\{augmented\_explanation\}\\
\\
\textbf{[Query]}\\
\{query\}\\
\\
Now, please generate a VQL sentence for the database schema and question after thinking step by step.\\

\end{promptbox}
% \end{figure*}


% \begin{figure*}[!h]
\begin{promptbox}[Prompt template for single classification]
Given a [Database schema] with [Augmented Explanation] and a [Question], generate a valid VQL (Visualization Query Language) sentence. VQL is similar to SQL but includes visualization components. \\
\\
Now you can think step by step, following these instructions below. \\
\textbf{[Background]} \\
VQL Structure:\\
Visualize [TYPE] SELECT [COLUMNS] FROM [TABLES] [JOIN] [WHERE] [GROUP BY] [ORDER BY] [BIN BY]\\
\\
You can consider a VQL sentence as "VIS TYPE + SQL + BINNING"\\
You must consider which part in the sketch is necessary, which is unnecessary, and construct a specific sketch for the natural language query.\\
\\
Key Components:\\
1. Visualization Type: bar, pie, line, scatter, stacked bar, grouped line, grouped scatter\\
2. SQL Components: SELECT, FROM, JOIN, WHERE, GROUP BY, ORDER BY\\
3. Binning: BIN [COLUMN] BY [INTERVAL], [INTERVAL]: [YEAR, MONTH, DAY, WEEKDAY]\\
\\
When generating VQL, we should always consider special rules and constraints:\\
\textbf{[Special Rules]} \\
a. For simple visualizations:\\
    \text{\ \ \ \ }- SELECT exactly TWO columns, X-axis and Y-axis(usually aggregate function)\\
b. For complex visualizations (STACKED BAR, GROUPED LINE, GROUPED SCATTER):\\
    \text{\ \ \ \ }- SELECT exactly THREE columns in this order!!!:\\
        \text{\ \ \ \ }\text{\ \ \ \ }1. X-axis\\
        \text{\ \ \ \ }\text{\ \ \ \ }2. Y-axis (aggregate function)\\
        \text{\ \ \ \ }\text{\ \ \ \ }3. Grouping column\\
c. When "COLORED BY" is mentioned in the question:\\
    \text{\ \ \ \ }- Use complex visualization type(STACKED BAR for bar charts, GROUPED LINE for line charts, GROUPED SCATTER for scatter charts)\\
    \text{\ \ \ \ }- Make the "COLORED BY" column the third SELECT column\\
    \text{\ \ \ \ }- Do NOT include "COLORED BY" in the final VQL\\     
d. Aggregate Functions:\\
    \text{\ \ \ \ }- Use COUNT for counting occurrences\\
    \text{\ \ \ \ }- Use SUM only for numeric columns\\
    \text{\ \ \ \ }- When in doubt, prefer COUNT over SUM\\
e. Time based questions:\\
    \text{\ \ \ \ }- Always use BIN BY clause at the end of VQL sentence\\
    \text{\ \ \ \ }- When you meet the questions including "year", "month", "day", "weekday"\\
    \text{\ \ \ \ }- Avoid using window function, just use BIN BY to deal with time base queries\\
% \end{promptbox}
% \end{figure*}
% \begin{figure*}[!h]
% \begin{promptbox}[Prompt template for single classification]
\textbf{[Constraints]} \\
- In SELECT <column>, make sure there are at least two selected!!!\\
- In FROM <table> or JOIN <table>, do not include unnecessary table\\
- Use only table names and column names from the given database schema\\
- Enclose string literals in single quotes\\
- If [Value examples] of <column> has `None' or None, use JOIN <table> or WHERE <column> is NOT NULL is better\\
- Ensure GROUP BY precedes ORDER BY for distinct values\\
- NEVER use window functions in SQL\\
\\
Now we could think step by step:\\
1. First choose visualize type and binning, then construct a specific sketch for the natural language query\\
2. Second generate SQL components following the sketch.\\
3. Third add Visualize type and BINNING into the SQL components to generate final VQL\\
\\
==============================\\
Here is a typical example:\\
\textbf{[Database Schema]}\\
\# Table: course, (course)\\
\text{[}\\
  \text{\ \ \ \ }(course\_id, course id, Value examples: [101, 696, 656, 659]. And this is an id type column),\\
  \text{\ \ \ \ }(title, title, Value examples: [`Geology', `Differential Geometry', `Compiler Design', `International Trade', `Composition and Literature', `Environmental Law'].),\\
  \text{\ \ \ \ }(dept\_name, dept name, Value examples: [`Cybernetics', `Finance', `Psychology', `Accounting', `Mech. Eng.', `Physics'].),\\
  \text{\ \ \ \ }(credits, credits, Value examples: [3, 4].)\\
\text{]}\\
\# Table: section, (section)\\
\text{[}\\
  \text{\ \ \ \ }(course\_id, course id, Value examples: [362, 105, 960, 468]. And this is an id type column),\\
  \text{\ \ \ \ }(sec\_id, sec id, Value examples: [1, 2, 3]. And this is an id type column),\\
  \text{\ \ \ \ }(semester, semester, Value examples: [`Fall', `Spring'].),\\
  \text{\ \ \ \ }(year, year, Value examples: [2002, 2006, 2003, 2007, 2010, 2008].),\\
  \text{\ \ \ \ }(building, building, Value examples: [`Saucon', `Taylor', `Lamberton', `Power', `Fairchild', `Main'].),\\
  \text{\ \ \ \ }(room\_number, room number, Value examples: [180, 183, 134, 143].),\\
  \text{\ \ \ \ }(time\_slot\_id, time slot id, Value examples: [`D', `J', `M', `C', `E', `F']. And this is an id type column)\\
\text{]}\\
\textbf{[Augmented Explanation]}\\
The filtered schema consists of 2 tables (course and section) with a total of 11 relevant columns. There is a one-to-many relationship between course and section through the course\_id foreign key.\\
\\
% \end{promptbox}
% \end{figure*}
% \begin{figure*}[!h]
% \begin{promptbox}[Prompt template for single classification]
Key points:\\
1. The course table contains information about individual courses, including the course title, department, and credits.\\
2. The section table contains information about specific sections of courses, including the semester, year, building, room number, and time slot.\\
3. The course\_id column links the two tables, allowing us to associate sections with specific courses.\\
4. The dept\_name column in the course table will be used to filter for Psychology department courses.\\
5. The year column in the section table will be used for yearly grouping and binning.\\
6. We need to count the number of courses offered each year, which requires aggregation and time-based binning.\\
\\
The query involves joining these two tables to analyze the number of courses offered by the Psychology department each year, which requires aggregation and time-based binning.\\
\\
\textbf{[Question]}\\
Find the number of courses offered by Psychology department in each year with a line chart.\\
\\
Decompose the task into sub tasks, considering [Background] [Special Rules] [Constraints], and generate the VQL after thinking step by step:\\
\\
\textbf{Sub task 1:} First choose visualize type and binning, then construct a specific sketch for the natural language query\\
Visualize type: LINE, BINNING: True\\
VQL Sketch:\\
Visualize LINE SELECT \_ , \_ FROM \_ JOIN \_ ON \_ WHERE \_ BIN \_ BY YEAR\\
\\
\textbf{Sub task 2:} Second generate SQL components following the sketch.\\
Let's think step by step:\\
1. We need to select 2 columns for LINE chart, year as X-axis, COUNT(year) as Y-axis.\\
2. We need to join the course and section tables to get the number of courses offered by the Psychology department in each year.\\
3. We need to filter the courses by the Psychology department.\\
4. We do not need to use any window function for YEAR.\\
\\
\text{sql}\\
```sql\\
SELECT S.year, COUNT(S.year)\\
FROM course AS C\\
JOIN section AS S ON C.course\_id = S.course\_id\\
WHERE C.dept\_name = `Psychology'\\
```\\
\\
% \end{promptbox}
% \end{figure*}
% \begin{figure*}[!h]
% \begin{promptbox}[Prompt template for single classification]
\textbf{Sub task 3:} Third add Visualize type and BINNING into the SQL components to generate final VQL\\
\textbf{Final VQL:}\\
Visualize LINE SELECT S.year, COUNT(S.year) FROM course C JOIN section S ON C.course\_id = S.course\_id WHERE C.dept\_name = `Psychology' BIN S.year BY YEAR\\
\\
==============================\\
Here is a new question:\\
\\
\textbf{[Database Schema]}\\
\{desc\_str\}\\
\\
\textbf{[Augmented Explanation]}\\
\{augmented\_explanation\}\\
\\
\textbf{[Query]}\\
\{query\}\\
\\
Now, please generate a VQL sentence for the database schema and question after thinking step by step.\\

\end{promptbox}
% \end{figure*}

% \subsection{Validator Agent Prompt}
% \label{validator_prompt}
% \begin{figure*}
\begin{promptbox}[Prompt template for Validator Agent]
As an AI assistant specializing in data visualization and VQL (Visualization Query Language), your task is to refine a VQL query that has resulted in an error. Please approach this task systematically, thinking step by step.\\
\textbf{[Background]}\\
VQL Structure:\\
Visualize [TYPE] SELECT [COLUMNS] FROM [TABLES] [JOIN] [WHERE] [GROUP BY] [ORDER BY] [BIN BY]\\
\\
You can consider a VQL sentence as "VIS TYPE + SQL + BINNING"\\
\\
Key Components:\\
1. Visualization Type: bar, pie, line, scatter, stacked bar, grouped line, grouped scatter\\
2. SQL Components: SELECT, FROM, JOIN, WHERE, GROUP BY, ORDER BY\\
3. Binning: BIN [COLUMN] BY [INTERVAL], [INTERVAL]: [YEAR, MONTH, DAY, WEEKDAY]\\
\\
When refining VQL, we should always consider special rules and constraints:\\
\textbf{[Special Rules]} \\
a. For simple visualizations:\\
    \text{\ \ \ \ }- SELECT exactly TWO columns, X-axis and Y-axis(usually aggregate function)\\
b. For complex visualizations (STACKED BAR, GROUPED LINE, GROUPED SCATTER):\\
    \text{\ \ \ \ }- SELECT exactly THREE columns in this order!!!:\\
        \text{\ \ \ \ }\text{\ \ \ \ }1. X-axis\\
        \text{\ \ \ \ }\text{\ \ \ \ }2. Y-axis (aggregate function)\\
        \text{\ \ \ \ }\text{\ \ \ \ }3. Grouping column\\
c. When "COLORED BY" is mentioned in the question:\\
    \text{\ \ \ \ }- Use complex visualization type(STACKED BAR for bar charts, GROUPED LINE for line charts, GROUPED SCATTER for scatter charts)\\
    \text{\ \ \ \ }- Make the "COLORED BY" column the third SELECT column\\
    \text{\ \ \ \ }- Do NOT include "COLORED BY" in the final VQL\\     
d. Aggregate Functions:\\
    \text{\ \ \ \ }- Use COUNT for counting occurrences\\
    \text{\ \ \ \ }- Use SUM only for numeric columns\\
    \text{\ \ \ \ }- When in doubt, prefer COUNT over SUM
% \end{promptbox}
% \end{figure*}

% \begin{figure*}
% \begin{promptbox}[Prompt template for Validator Agent]
e. Time based questions:\\
    \text{\ \ \ \ }- Always use BIN BY clause at the end of VQL sentence\\
    \text{\ \ \ \ }- When you meet the questions including "year", "month", "day", "weekday"\\
    \text{\ \ \ \ }- Avoid using time function, just use BIN BY to deal with time base queries\\
\\
\textbf{[Constraints]} \\
- In FROM <table> or JOIN <table>, do not include unnecessary table\\
- Use only table names and column names from the given database schema\\
- Enclose string literals in single quotes\\
- If [Value examples] of <column> has `None' or None, use JOIN <table> or WHERE <column> is NOT NULL is better\\
- ENSURE GROUP BY clause cannot contain aggregates\\
- NEVER use date functions in SQL\\
\\
\textbf{[Query]} \\
\{query\}\\
\\
\textbf{[Database info]} \\
\{db\_info\}\\
\\
\textbf{[Current VQL]} \\
\{vql\}\\
\\
\textbf{[Error]} \\
\{error\}\\
\\
Now, please analyze and refine the VQL, please provide:\\
\\
\textbf{[Explanation]}\\
\text{[}Provide a detailed explanation of your analysis process, the issues identified, and the changes made. Reference specific steps where relevant.\text{]}\\
\\
\textbf{[Corrected VQL]}\\
\text{[}Present your corrected VQL here. Ensure it's on a single line without any line breaks.\text{]}\\
\\
Remember:\\
- The SQL components must be parseable by DuckDB.\\
- Do not change rows when you generate the VQL.\\
- Always verify your answer carefully before submitting.\\
\end{promptbox}
% \end{figure*}

\end{document}
