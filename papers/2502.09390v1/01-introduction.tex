
\begin{table*}[ht]
    \caption{Main prompt for the \rephrase{} technique. 
    % \moshe{Are we certain we wish to put this prompt here, instead of the appendix?}
    }
    \centering
    \small
    \begin{tabular}{>{\raggedright\arraybackslash\tt}p{0.98\textwidth}<{}}
      \toprule
      You are a helpful question answerer who can provide an answer given a question and relevant context.\\
      Generate \{N\} questions based on the given question and context, and shortly answer them.\\
      Finally, provide an answer to the original question using what you learned from answering the questions you created.
      The answer should be a short span, just a few words. \\
      \bottomrule
    \end{tabular}
    \label{tab:square_prompt}
\end{table*}


\section{Introduction}
\label{sec:introduction}

Large Language Models (LLMs) have rapidly transformed Natural Language Processing (NLP), excelling in tasks like text generation, machine translation, and dialogue systems \cite{brownLanguageModelsAre2020,kojimaLargeLanguageModels2022}. These models owe their flexibility to the Transformer architecture \cite{vaswaniAttentionAllYou2017}, and benefit from large-scale pretraining followed by fine-tuning or instruction tuning to align with human objectives \cite{ouyangTrainingLanguageModels2022,weiFinetunedLanguageModels2022}. A key technique for enhancing these models is chain-of-thought (CoT) prompting, which has gained notable attention for its ability to improve reasoning by encouraging models to work through problems step by step \citep{weiChainofThoughtPromptingElicits2023}. This approach has shown efficacy in complex tasks like multi-step arithmetic and commonsense question answering, by making intermediate processes transparent and facilitating more accurate outcomes \cite{snellScalingLLMTestTime2024}. While some CoT variants explore iterative reasoning, there is still limited exploration of self-interrogation paradigms that prompt models to pose and resolve their own intermediate queries.

\begin{figure*}[t]
    \centering
    \includegraphics[width=\linewidth]{figs/illustration_v2.pdf}
    \vspace{-17mm}
    \caption{Illustration of the critique-correction process for a coding problem. Top: An initial solution is proposed by the task-performing using a min-heap approach. Bottom: The critic identifies flaws in the implementation (incorrect heap access and inefficient query handling) and suggests specific improvements, leading to a corrected max-heap solution. This example is taken from critiques of {\ours} on LiveCodeBench, which demonstrates how structured feedback from the critic can guide meaningful improvements in code generation.}
    \label{fig:illustration}
\end{figure*}


In this paper, we introduce \rephrase{} (Sequential Question Answering Reasoning Engine), a prompting technique that instructs an LLM to generate and answer multiple sub-questions before addressing the main query. By decomposing queries into iterative steps, \rephrase{} draws on chain-of-thought frameworks and prior prompting methodologies~\citep{dengRephraseRespondLet2024} to produce more comprehensive solutions. In extensive evaluations on multiple question-answering datasets using Llama 3~\cite{grattafioriLlama3Herd2024} (3B and 8B) and \mbox{GPT-4o} \citep{openaiGPT4oSystemCard2024}, \mbox{\rephrase{} outperforms} chain-of-thought prompts and existing rephrase-and-respond strategies. This work highlights how systematically breaking down queries advances LLM reasoning capabilities.

