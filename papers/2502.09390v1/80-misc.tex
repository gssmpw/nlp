\section*{Limitations and Future Plans}

While our multi-question chain of thought prompt strategy has demonstrated notable improvements in reasoning capabilities and response accuracy of large language models, several limitations should be acknowledged. Firstly, the method requires fine-tuning of the number of intermediate questions (3, 5, 10 or other), and this may not be optimal or applicable across varying query complexities or domains. Choosing the appropriate number of questions is important, as an incorrect configuration might lead to redundancy or insufficient exploration of the query context.

Secondly, our approach was evaluated only on specific Q\&A datasets, which may not encompass the full spectrum of topics and question types. Therefore, the generalizability of this technique to other domains, such as dialogue systems or more complex multi-turn interactions, remains to be tested. Additionally, while our experiments utilized the Llama 3 models and GPT-4o, the effectiveness of this strategy across other architectures or smaller-scale models could differ.

Another limitation is the potential increase in computational resources required to generate and answer multiple intermediate questions, which could impact the efficiency and scalability of deploying these models in real-time applications. 

% Lastly, since we utilized instruction-following models, they inherently understand how to integrate and handle external knowledge; however, this is not universally applicable to all language models, which may sometimes result in logical inconsistencies or off-topic intermediate questions.

Future research should focus on addressing these limitations by exploring adaptive mechanisms for intermediate question generation, extending validation across more diverse datasets and models, and optimizing computational requirements to ensure broader applicability and effectiveness.

\section*{Ethics Statement}

Throughout our research, we carefully considered the ethical aspects of developing advanced language models. Our technique aims to enhance reasoning and accuracy, but we recognize the need to address potential ethical issues.

One concern is that improved reasoning could result in producing more persuasive but misleading or harmful content. To counteract this, it is essential to implement safeguards ensuring responses are accurate, unbiased, and factual. Future efforts should continue to monitor outputs for bias and misinformation, incorporating methods to mitigate these risks.

Additionally, the increased computational demand for generating intermediate questions raises environmental concerns about energy consumption. We advocate for continued research into optimizing the efficiency of these processes to minimize ecological impact.

We prioritized privacy and security by using only publicly available data in our experiments, free of private information. Adhering to transparency and reproducibility principles, we documented our methodologies, to facilitate replication of our findings by others.
