%%%%%%%%%%%%%%%%%%%%%%%%%%%%%%%%%%%%%%%%%%%%%%%%%%%%%%%%%%%%%%%%%%%%%%%%%%%%%%%%
%2345678901234567890123456789012345678901234567890123456789012345678901234567890
%        1         2         3         4         5         6         7         8

\documentclass[letterpaper, 10 pt, journal, twoside]{IEEEtran}  % Comment this line out if you need a4paper
% \documentclass[letterpaper, 10 pt, conference]{ieeeconf}

\IEEEoverridecommandlockouts                              % This command is only needed if 
                                                          % you want to use the \thanks command
\pdfminorversion=4
% \overrideIEEEmargins  
% Needed to meet printer requirements.

% See the \addtolength command later in the file to balance the column lengths
% on the last page of the document

% The following packages can be found on http:\\www.ctan.org
\usepackage{graphicx} % for pdf, bitmapped graphics files
%\usepackage{epsfig} % for postscript graphics files
%\usepackage{mathptmx} % assumes new font selection scheme installed
%\usepackage{times} % assumes new font selection scheme installed
\usepackage{amsmath} % assumes amsmath package installed
\usepackage{amssymb}  % assumes amsmath package installed
\usepackage[usenames, dvipsnames]{color}
\usepackage{siunitx}
\usepackage{cases}
\usepackage{lipsum}
\usepackage{color}
\usepackage{cite}
\usepackage[ruled,vlined]{algorithm2e}
\usepackage{diagbox} 
\usepackage{tabu}
\usepackage{array}
\usepackage{multirow}
\usepackage{booktabs}
\usepackage{makecell}
\usepackage{empheq}
\usepackage{threeparttable}
\usepackage{tabularx}
\usepackage{hyperref}

\newcolumntype{P}[1]{>{\centering\arraybackslash}p{#1}}
\newcolumntype{M}[1]{>{\centering\arraybackslash}m{#1}}

\DeclareMathOperator*{\argmin}{arg\,min}
\DeclareMathOperator*{\argmax}{arg\,max}
\newcommand{\todo}[1]{\textcolor{red}{\textbf{\emph{TODO:~{#1}}}}}
\newcommand{\checkme}[1]{\textcolor{blue}{\textbf{\emph{{#1}}}}}
\newcommand{\revision}[1]{\textcolor{black}{#1}} %for displaying blue texts
\newcommand{\final}[1]{\textcolor{black}{#1}} %for displaying blue texts
\newcolumntype{C}[1]{>{\centering\let\newline\\\arraybackslash\hspace{0pt}}m{#1}}
\newcolumntype{L}[1]{>{\raggedright\let\newline\\\arraybackslash\hspace{0pt}}m{#1}}
% \linespread{0.97}

% \title{
% Automatic Normal Positioning of Robotic Ultrasound Probe based only on Confidence Map Optimization and Force Measurement
% }

\title{
Gaze-Guided Robotic Vascular Ultrasound Leveraging Human Intention Estimation
% Stabilized Gaze-Guided Robotic Vascular Ultrasound
% Gaze-Guided Robotic Vascular Ultrasound
% Gaze-Guided Robotic Ultrasound System for Vascular Screening
}



\author{Yuan Bi$^{1, 2}$, Yang Su$^{1}$, Nassir Navab$^{1, 2}$, \textit{Fellow, IEEE}, and Zhongliang Jiang$^{1, 2}$ % <-this % stops a space

\thanks{Manuscript received: September 10, 2024; Revised: December 7, 2024; Accepted: January 15, 2025.}%Use only for final RAL version
\thanks{This paper was recommended for publication by Editor Jessica Burgner-Kahrs upon evaluation of the Associate Editor and Reviewers' comments.}
% \thanks{$^{*}$ Authors are with equal contributions.}
\thanks{This work was partially supported by SINO-German Mobility Project (M-0221). (Corresponding Author: Zhongliang Jiang, zl.jiang@tum.de)}
\thanks{$^{1}$Y. Bi, Y. Su, N. Navab, and Z. Jiang are with the Chair for Computer-Aided Medical Procedures and Augmented Reality, Technical University of Munich, Boltzmannstr. 3, 85748 Garching, Germany}%% 
\thanks{$^{2}$Y. Bi, N. Navab, and Z. Jiang are with the Munich Center for Machine Learning, Arcisstraße 21, 80333 M\"unchen, Germany}
% \thanks{The authors would like to thank Dr. Reza Ghotbi from vascular surgery department of Helios Klinikum M\"unchen West and Mr. Wei Zhang from Klinikum rechts der Isar for their feedback and discussions.}
\thanks{Digital Object Identifier (DOI): see top of this page.}
}


\markboth{IEEE Robotics and Automation Letters. Preprint Version. Accepted JANUARY, 2025}
{Bi \MakeLowercase{\textit{et al.}}: Gaze-Guided Robotic Vascular Ultrasound Leveraging Human Intention Estimation} 




\begin{document}

\maketitle

%%%%%%%%%%%%%%%%%%%%%%%%%%%%%%%%%%%%%%%%%%%%%%%%%%%%%%%%%%%%%%%%%%%%%%%%%%%%%%%%
\begin{abstract}
Medical ultrasound has been widely used to examine vascular structure in modern clinical practice. However, traditional ultrasound examination often faces challenges related to inter- and intra-operator variation. The robotic ultrasound system (RUSS) appears as a potential solution for such challenges because of its superiority in stability and reproducibility.
Given the complex anatomy of human vasculature, multiple vessels often appear in ultrasound images, or a single vessel bifurcates into branches, complicating the examination process.
To tackle this challenge, this work presents a gaze-guided RUSS for vascular applications. A gaze tracker captures the eye movements of the operator. The extracted gaze signal guides the RUSS to follow the correct vessel when it bifurcates. Additionally, a gaze-guided segmentation network is proposed to enhance segmentation robustness by exploiting gaze information.
However, gaze signals are often noisy, requiring interpretation to accurately discern the operator's true intentions. To this end, this study proposes a stabilization module to process raw gaze data. The inferred attention heatmap is utilized as a region proposal to aid segmentation and serve as a trigger signal when the operator needs to adjust the scanning target, such as when a bifurcation appears. To ensure appropriate contact between the probe and surface during scanning, an automatic ultrasound confidence-based orientation correction method is developed.
In experiments, we demonstrated the efficiency of the proposed gaze-guided segmentation pipeline by comparing it with other methods. Besides, the performance of the proposed gaze-guided RUSS was also validated as a whole on a realistic arm phantom with an uneven surface.
% A confidence-based probe orientation correction method is proposed to control the robot establish proper contact with the scanning surface
\end{abstract}

% \markboth{IEEE Robotics and Automation Letters. Preprint Version. Accepted April, 2022}
% {Bi \MakeLowercase{\textit{et al.}}: VesNet-RL: Simulation-based Reinforcement Learning for Real-World US Probe Navigation} 




%%%%%%%%%%%%%%%%%%%%%%%%%%%%%%%%%%%%%%%%%%%%%%%%%%%%%%%%%%%%%%%%%%%%%%%%%%%%%%%%
% \vspace{1em}
% %%%%%%%%%%%%%%%%%%%%%%%%%%%%%%%%%%%%%%%%%%%%%%%%%%%%%%%%%%%%%%%%%%%%%%%%%%%%%%%%
\begin{IEEEkeywords}
Robotic ultrasound, Gaze tracker, Gaze-guided system, Ultrasound segmentation
\end{IEEEkeywords}


%%%%%%%%%%%%%%%%%%%%%%%%%%%%%%%%%%%%%%%%%%%%%%%%%%%%%%%%%%%%%%%%%%%%%%%%%%%%%%%%
\bstctlcite{IEEEexample:BSTcontrol}
% \section{Introduction}
\label{sec:introduction}
The business processes of organizations are experiencing ever-increasing complexity due to the large amount of data, high number of users, and high-tech devices involved \cite{martin2021pmopportunitieschallenges, beerepoot2023biggestbpmproblems}. This complexity may cause business processes to deviate from normal control flow due to unforeseen and disruptive anomalies \cite{adams2023proceddsriftdetection}. These control-flow anomalies manifest as unknown, skipped, and wrongly-ordered activities in the traces of event logs monitored from the execution of business processes \cite{ko2023adsystematicreview}. For the sake of clarity, let us consider an illustrative example of such anomalies. Figure \ref{FP_ANOMALIES} shows a so-called event log footprint, which captures the control flow relations of four activities of a hypothetical event log. In particular, this footprint captures the control-flow relations between activities \texttt{a}, \texttt{b}, \texttt{c} and \texttt{d}. These are the causal ($\rightarrow$) relation, concurrent ($\parallel$) relation, and other ($\#$) relations such as exclusivity or non-local dependency \cite{aalst2022pmhandbook}. In addition, on the right are six traces, of which five exhibit skipped, wrongly-ordered and unknown control-flow anomalies. For example, $\langle$\texttt{a b d}$\rangle$ has a skipped activity, which is \texttt{c}. Because of this skipped activity, the control-flow relation \texttt{b}$\,\#\,$\texttt{d} is violated, since \texttt{d} directly follows \texttt{b} in the anomalous trace.
\begin{figure}[!t]
\centering
\includegraphics[width=0.9\columnwidth]{images/FP_ANOMALIES.png}
\caption{An example event log footprint with six traces, of which five exhibit control-flow anomalies.}
\label{FP_ANOMALIES}
\end{figure}

\subsection{Control-flow anomaly detection}
Control-flow anomaly detection techniques aim to characterize the normal control flow from event logs and verify whether these deviations occur in new event logs \cite{ko2023adsystematicreview}. To develop control-flow anomaly detection techniques, \revision{process mining} has seen widespread adoption owing to process discovery and \revision{conformance checking}. On the one hand, process discovery is a set of algorithms that encode control-flow relations as a set of model elements and constraints according to a given modeling formalism \cite{aalst2022pmhandbook}; hereafter, we refer to the Petri net, a widespread modeling formalism. On the other hand, \revision{conformance checking} is an explainable set of algorithms that allows linking any deviations with the reference Petri net and providing the fitness measure, namely a measure of how much the Petri net fits the new event log \cite{aalst2022pmhandbook}. Many control-flow anomaly detection techniques based on \revision{conformance checking} (hereafter, \revision{conformance checking}-based techniques) use the fitness measure to determine whether an event log is anomalous \cite{bezerra2009pmad, bezerra2013adlogspais, myers2018icsadpm, pecchia2020applicationfailuresanalysispm}. 

The scientific literature also includes many \revision{conformance checking}-independent techniques for control-flow anomaly detection that combine specific types of trace encodings with machine/deep learning \cite{ko2023adsystematicreview, tavares2023pmtraceencoding}. Whereas these techniques are very effective, their explainability is challenging due to both the type of trace encoding employed and the machine/deep learning model used \cite{rawal2022trustworthyaiadvances,li2023explainablead}. Hence, in the following, we focus on the shortcomings of \revision{conformance checking}-based techniques to investigate whether it is possible to support the development of competitive control-flow anomaly detection techniques while maintaining the explainable nature of \revision{conformance checking}.
\begin{figure}[!t]
\centering
\includegraphics[width=\columnwidth]{images/HIGH_LEVEL_VIEW.png}
\caption{A high-level view of the proposed framework for combining \revision{process mining}-based feature extraction with dimensionality reduction for control-flow anomaly detection.}
\label{HIGH_LEVEL_VIEW}
\end{figure}

\subsection{Shortcomings of \revision{conformance checking}-based techniques}
Unfortunately, the detection effectiveness of \revision{conformance checking}-based techniques is affected by noisy data and low-quality Petri nets, which may be due to human errors in the modeling process or representational bias of process discovery algorithms \cite{bezerra2013adlogspais, pecchia2020applicationfailuresanalysispm, aalst2016pm}. Specifically, on the one hand, noisy data may introduce infrequent and deceptive control-flow relations that may result in inconsistent fitness measures, whereas, on the other hand, checking event logs against a low-quality Petri net could lead to an unreliable distribution of fitness measures. Nonetheless, such Petri nets can still be used as references to obtain insightful information for \revision{process mining}-based feature extraction, supporting the development of competitive and explainable \revision{conformance checking}-based techniques for control-flow anomaly detection despite the problems above. For example, a few works outline that token-based \revision{conformance checking} can be used for \revision{process mining}-based feature extraction to build tabular data and develop effective \revision{conformance checking}-based techniques for control-flow anomaly detection \cite{singh2022lapmsh, debenedictis2023dtadiiot}. However, to the best of our knowledge, the scientific literature lacks a structured proposal for \revision{process mining}-based feature extraction using the state-of-the-art \revision{conformance checking} variant, namely alignment-based \revision{conformance checking}.

\subsection{Contributions}
We propose a novel \revision{process mining}-based feature extraction approach with alignment-based \revision{conformance checking}. This variant aligns the deviating control flow with a reference Petri net; the resulting alignment can be inspected to extract additional statistics such as the number of times a given activity caused mismatches \cite{aalst2022pmhandbook}. We integrate this approach into a flexible and explainable framework for developing techniques for control-flow anomaly detection. The framework combines \revision{process mining}-based feature extraction and dimensionality reduction to handle high-dimensional feature sets, achieve detection effectiveness, and support explainability. Notably, in addition to our proposed \revision{process mining}-based feature extraction approach, the framework allows employing other approaches, enabling a fair comparison of multiple \revision{conformance checking}-based and \revision{conformance checking}-independent techniques for control-flow anomaly detection. Figure \ref{HIGH_LEVEL_VIEW} shows a high-level view of the framework. Business processes are monitored, and event logs obtained from the database of information systems. Subsequently, \revision{process mining}-based feature extraction is applied to these event logs and tabular data input to dimensionality reduction to identify control-flow anomalies. We apply several \revision{conformance checking}-based and \revision{conformance checking}-independent framework techniques to publicly available datasets, simulated data of a case study from railways, and real-world data of a case study from healthcare. We show that the framework techniques implementing our approach outperform the baseline \revision{conformance checking}-based techniques while maintaining the explainable nature of \revision{conformance checking}.

In summary, the contributions of this paper are as follows.
\begin{itemize}
    \item{
        A novel \revision{process mining}-based feature extraction approach to support the development of competitive and explainable \revision{conformance checking}-based techniques for control-flow anomaly detection.
    }
    \item{
        A flexible and explainable framework for developing techniques for control-flow anomaly detection using \revision{process mining}-based feature extraction and dimensionality reduction.
    }
    \item{
        Application to synthetic and real-world datasets of several \revision{conformance checking}-based and \revision{conformance checking}-independent framework techniques, evaluating their detection effectiveness and explainability.
    }
\end{itemize}

The rest of the paper is organized as follows.
\begin{itemize}
    \item Section \ref{sec:related_work} reviews the existing techniques for control-flow anomaly detection, categorizing them into \revision{conformance checking}-based and \revision{conformance checking}-independent techniques.
    \item Section \ref{sec:abccfe} provides the preliminaries of \revision{process mining} to establish the notation used throughout the paper, and delves into the details of the proposed \revision{process mining}-based feature extraction approach with alignment-based \revision{conformance checking}.
    \item Section \ref{sec:framework} describes the framework for developing \revision{conformance checking}-based and \revision{conformance checking}-independent techniques for control-flow anomaly detection that combine \revision{process mining}-based feature extraction and dimensionality reduction.
    \item Section \ref{sec:evaluation} presents the experiments conducted with multiple framework and baseline techniques using data from publicly available datasets and case studies.
    \item Section \ref{sec:conclusions} draws the conclusions and presents future work.
\end{itemize}
% \section{Text capabilities }\label{sec:text_ability}
We assess the text-related capabilities of \ourM{} (8x7B) in this section. We first present several examples to provide readers with an intuitive understanding (Section \ref{sec:case_study}), and then benchmark results from AlpacaEval (Section \ref{sec:alpahaeval}).

\subsection{Case study}\label{sec:case_study}
%\TODO{@Yuan-Jyue}, pls input several text cases here.
Here we demonstrate our model's language capabilities in scientific discovery research. Specifically, we assessed the model's understanding of CRISPR gene editing, cancer immunotherapy, and drug discovery. A few selected cases are presented in this section for brevity. Additional cases can be found in Fig. \ref{SI_fig:case_study:CRISPR}, \ref{SI_fig:case_study:cancer_immunotherapy} and \ref{SI_fig:case_study:drug_discovery}.


\subsubsection{CRISPR gene editing}
 CRISPR gene editing has rapidly emerged as a powerful tool in biotechnology. Our model demonstrates a good understanding of CRISPR and is able to effectively analyze the advantages and disadvantages of commonly used CRISPR tools. 

  
\begin{mdframed}[backgroundcolor=white, linecolor=black, linewidth=1pt]
    \noindent
    \textbf{Instruction:} Could you provide me the advantages and disadvantages of Cas9? \\
    \textbf{Response:} \\
    Advantages: 1. Highly efficient and specific gene editing with a low risk of off-target effects. 2. Can target a wide range of sequences.  3. Cam be used to make specific alterations to genes in virtually any organism.  Disadvantages: 1. Potential for off-target effects when using Cas9. 2. The delivery of Cas9 can be inefficient. 3. Lack of control over the repair mechanism. 4. May cause chromosomal abnormalities. \\

    \noindent
    \textbf{Instruction:} Could you provide me the advantages and disadvantages of Cas12a, another widely used CRISPR protein? \\
    \textbf{Response:} \\
    Advantages: \\
    1. High specificty: Cas12a has higher specificity compared to Cas 9, which reduces off-target effects. 
    2. Simpler PAM requirements: Cas 12a requires a simpler PAM sequence than Cas9, making it easier to target specific genomic loci. 
    3. Multiplexing: Cas12a can be used for multiplexing genome editing, meaning multiple genomic loci can be edited in a single reaction. 
    Disadvantages: \\
    4. Lower efficiency: Cas12a is less efficient in terms of indel formation than Cas9. 
    5. Larger size: Cas12a is a larger protein than Cas9, which can make it harder to deliver the Cas12a system into cells. \\
    ... 
\end{mdframed}  
\captionof{figure}{Cases about CRISPR gene editing.} \label{fig:case_study:CRISPR}

\subsubsection{Cancer immunotherapy}
Our model is knowledgeable about existing checkpoint inhibitors used in cancer therapy and provides accurate explanations of their mechanisms of action. Specifically, in our experiment, it generates a valid peptide sequence for inhibiting the overexpressed PD-L1 protein (see Figure \ref{SI_fig:case_study:cancer_immunotherapy}), although the length of the peptide is incorrect. It demonstrates a solid understanding of the inhibition cascade between drugs and proteins.

\begin{mdframed}[backgroundcolor=white, linecolor=black, linewidth=1pt]  
    \noindent
    \textbf{Instruction:} Tell me existing checkpoint inhibitors for PD-L1. \\
    \textbf{Response:} The checkpoint inhibitors for PD-L1 that are currently approved by the FDA include atezolizumab, avelumab, duravalumab, and perbrolizumab.\\
    
    \noindent
    \textbf{Instruction:} PD-L1 is a protein that inhibits PD-1, a recptor found on T cells. This new drug X, is designed to inhibit PD-L1. If we introduce drug X, will T cell function be restored? Explain it step by step. \\
    \textbf{Response:}
    PD-1 is inhibited by interaction with PD-L1. If drug X inhibits PD-L1, PD-1 will no longer be inhibited. Therefore, T cell function will beestored. Therefore, the answer is yes.

\end{mdframed}  
\captionof{figure}{Cases about cancer immunotherapy.} \label{fig:case_study:cancer_immunotherapy}

\subsubsection{Drug discovery}
Drug discovery is an evolving process that incorporates empirical rules, computational design, and experimental validation. The model demonstrates an understanding of the fundamental principles of drug discovery. Specifically, when being asked questions about Lipinski’s Rule of Five and strategies (see Figure \ref{SI_fig:case_study:drug_discovery}) to improve a drug’s ADME properties, the model provides mostly accurate answers.

\begin{mdframed}[backgroundcolor=white, linecolor=black, linewidth=1pt]  
    \textbf{Instruction:} What approaches can drug experts use to enhance a drug's ADME properties? \\
    \textbf{Response:} Structure-based drug design: Computational techniques can be used to modify existing drugs or design new ones with desired ADME properties. 2. Lipophilicity: Lipophilicity can affect the ADME profile of a drug. 3. Hepatic and extracellular clearance: Understanding the hepatic and extracellular clearance of a drug is important for assessing its ADME profile. 4. Microsomal stability: Microsomal stability is a measure of how quickly a drug is broken down by liver enzymes. 5. Oral absorption: Oral absorption is determined by the solubility and permeability of a drug. 6. Permeability: The ability of a drug to cross cell membranes affects its bioavailability. 
    ... %7. Plasma protein binding: Plasma protein binding affects the concentration of free drug available to interact with its target. 8. Efflux transporters: Efflux transporters can affect the absorption, distribution, and excretion of a drug. 

\end{mdframed}  
\captionof{figure}{Case about drug discovery.} \label{fig:case_study:drug
_discovery}

\subsection{Comparison on AlpacaEval evaluation}\label{sec:alpahaeval}
To assess the linguistic capabilities of \ourM{}, we compared \ourM{} (8x7B) with the original Mixtral  (8x7B) using the AlpacaEval \cite{dubois2024length} dataset and method. AlpacaEval is a benchmark for LLM designed to gauge a model's proficiency in executing general user instructions. \ourM{}'s responses are compared against reference responses generated by original Mixtral using GPT-4 as the annotator. We focused on two metrics from AlpacaEval: the general win rate and the LC win rate (length-controlled win rate), which controls the bias introduced by the length of the responses. Our NatureLM model achieved a LC win rate of 31.8\% and a win rate of 25.2\% when benchmarked against Mixtral, showcasing that while the model possesses a foundational level of language comprehension, it exhibits constraints in its linguistic capacity.   A potential contributing factor to this limited performance is the overrepresentation of scientific data within the training dataset, which may skew the model's proficiency towards technical language and away from a more generalized understanding required for diverse user instructions. Fig. \ref{fig:case_study:alpacaeval} presents two illustrative examples from the AlpacaEval comparison. We will improve the text capabilities of \ourM{} in our future work. 
    
\section{Text capabilities }\label{sec:text_ability}
We assess the text-related capabilities of \ourM{} (8x7B) in this section. We first present several examples to provide readers with an intuitive understanding (Section \ref{sec:case_study}), and then benchmark results from AlpacaEval (Section \ref{sec:alpahaeval}).

\subsection{Case study}\label{sec:case_study}
%\TODO{@Yuan-Jyue}, pls input several text cases here.
Here we demonstrate our model's language capabilities in scientific discovery research. Specifically, we assessed the model's understanding of CRISPR gene editing, cancer immunotherapy, and drug discovery. A few selected cases are presented in this section for brevity. Additional cases can be found in Fig. \ref{SI_fig:case_study:CRISPR}, \ref{SI_fig:case_study:cancer_immunotherapy} and \ref{SI_fig:case_study:drug_discovery}.


\subsubsection{CRISPR gene editing}
 CRISPR gene editing has rapidly emerged as a powerful tool in biotechnology. Our model demonstrates a good understanding of CRISPR and is able to effectively analyze the advantages and disadvantages of commonly used CRISPR tools. 

  
\begin{mdframed}[backgroundcolor=white, linecolor=black, linewidth=1pt]
    \noindent
    \textbf{Instruction:} Could you provide me the advantages and disadvantages of Cas9? \\
    \textbf{Response:} \\
    Advantages: 1. Highly efficient and specific gene editing with a low risk of off-target effects. 2. Can target a wide range of sequences.  3. Cam be used to make specific alterations to genes in virtually any organism.  Disadvantages: 1. Potential for off-target effects when using Cas9. 2. The delivery of Cas9 can be inefficient. 3. Lack of control over the repair mechanism. 4. May cause chromosomal abnormalities. \\

    \noindent
    \textbf{Instruction:} Could you provide me the advantages and disadvantages of Cas12a, another widely used CRISPR protein? \\
    \textbf{Response:} \\
    Advantages: \\
    1. High specificty: Cas12a has higher specificity compared to Cas 9, which reduces off-target effects. 
    2. Simpler PAM requirements: Cas 12a requires a simpler PAM sequence than Cas9, making it easier to target specific genomic loci. 
    3. Multiplexing: Cas12a can be used for multiplexing genome editing, meaning multiple genomic loci can be edited in a single reaction. 
    Disadvantages: \\
    4. Lower efficiency: Cas12a is less efficient in terms of indel formation than Cas9. 
    5. Larger size: Cas12a is a larger protein than Cas9, which can make it harder to deliver the Cas12a system into cells. \\
    ... 
\end{mdframed}  
\captionof{figure}{Cases about CRISPR gene editing.} \label{fig:case_study:CRISPR}

\subsubsection{Cancer immunotherapy}
Our model is knowledgeable about existing checkpoint inhibitors used in cancer therapy and provides accurate explanations of their mechanisms of action. Specifically, in our experiment, it generates a valid peptide sequence for inhibiting the overexpressed PD-L1 protein (see Figure \ref{SI_fig:case_study:cancer_immunotherapy}), although the length of the peptide is incorrect. It demonstrates a solid understanding of the inhibition cascade between drugs and proteins.

\begin{mdframed}[backgroundcolor=white, linecolor=black, linewidth=1pt]  
    \noindent
    \textbf{Instruction:} Tell me existing checkpoint inhibitors for PD-L1. \\
    \textbf{Response:} The checkpoint inhibitors for PD-L1 that are currently approved by the FDA include atezolizumab, avelumab, duravalumab, and perbrolizumab.\\
    
    \noindent
    \textbf{Instruction:} PD-L1 is a protein that inhibits PD-1, a recptor found on T cells. This new drug X, is designed to inhibit PD-L1. If we introduce drug X, will T cell function be restored? Explain it step by step. \\
    \textbf{Response:}
    PD-1 is inhibited by interaction with PD-L1. If drug X inhibits PD-L1, PD-1 will no longer be inhibited. Therefore, T cell function will beestored. Therefore, the answer is yes.

\end{mdframed}  
\captionof{figure}{Cases about cancer immunotherapy.} \label{fig:case_study:cancer_immunotherapy}

\subsubsection{Drug discovery}
Drug discovery is an evolving process that incorporates empirical rules, computational design, and experimental validation. The model demonstrates an understanding of the fundamental principles of drug discovery. Specifically, when being asked questions about Lipinski’s Rule of Five and strategies (see Figure \ref{SI_fig:case_study:drug_discovery}) to improve a drug’s ADME properties, the model provides mostly accurate answers.

\begin{mdframed}[backgroundcolor=white, linecolor=black, linewidth=1pt]  
    \textbf{Instruction:} What approaches can drug experts use to enhance a drug's ADME properties? \\
    \textbf{Response:} Structure-based drug design: Computational techniques can be used to modify existing drugs or design new ones with desired ADME properties. 2. Lipophilicity: Lipophilicity can affect the ADME profile of a drug. 3. Hepatic and extracellular clearance: Understanding the hepatic and extracellular clearance of a drug is important for assessing its ADME profile. 4. Microsomal stability: Microsomal stability is a measure of how quickly a drug is broken down by liver enzymes. 5. Oral absorption: Oral absorption is determined by the solubility and permeability of a drug. 6. Permeability: The ability of a drug to cross cell membranes affects its bioavailability. 
    ... %7. Plasma protein binding: Plasma protein binding affects the concentration of free drug available to interact with its target. 8. Efflux transporters: Efflux transporters can affect the absorption, distribution, and excretion of a drug. 

\end{mdframed}  
\captionof{figure}{Case about drug discovery.} \label{fig:case_study:drug
_discovery}

\subsection{Comparison on AlpacaEval evaluation}\label{sec:alpahaeval}
To assess the linguistic capabilities of \ourM{}, we compared \ourM{} (8x7B) with the original Mixtral  (8x7B) using the AlpacaEval \cite{dubois2024length} dataset and method. AlpacaEval is a benchmark for LLM designed to gauge a model's proficiency in executing general user instructions. \ourM{}'s responses are compared against reference responses generated by original Mixtral using GPT-4 as the annotator. We focused on two metrics from AlpacaEval: the general win rate and the LC win rate (length-controlled win rate), which controls the bias introduced by the length of the responses. Our NatureLM model achieved a LC win rate of 31.8\% and a win rate of 25.2\% when benchmarked against Mixtral, showcasing that while the model possesses a foundational level of language comprehension, it exhibits constraints in its linguistic capacity.   A potential contributing factor to this limited performance is the overrepresentation of scientific data within the training dataset, which may skew the model's proficiency towards technical language and away from a more generalized understanding required for diverse user instructions. Fig. \ref{fig:case_study:alpacaeval} presents two illustrative examples from the AlpacaEval comparison. We will improve the text capabilities of \ourM{} in our future work. 

%%%%%%%%%%%%%%%%%%%%%%%%%%%%%%%%%%%%%%%%%%%%%%%%%%%%%%%%%%%%%%%%%%%%%%%%%%%%%%%%

%\section*{ACKNOWLEDGMENT}
%This work partially received funding by the project IOTMA supported by the central innovation program for SMEs (ZIM), the BayMED project 5G-MedServices funded by the Bavarian state, as well as from the European Union’s Horizon 2020 research and innovation program EDEN2020 under Grant Agreement No. 688279. The motion capture data used in this project was obtained from mocap.cs.cmu.edu which was created with funding from NSF EIA-0196217. We gratefully acknowledge the support of NVIDIA Corporation with the donation of the Titan X Pascal GPU used here. 

%%%%%%%%%%%%%%%%%%%%%%%%%%%%%%%%%%%%%%%%%%%%%%%%%%%%%%%%%%%%%%%%%%%%%%%%%%%%%%%%
\bibliographystyle{IEEEtran}
\bibliography{IEEEabrv,references}




\end{document}