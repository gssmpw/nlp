\subsection{NDVI Dataset Preprocessing}
\label{dataset}
%VI Summary of changes: I tried to add a little more detail on how this works, some context for the scale of the dataset and reformatted it to remove the bullets.
To support applications from land use tracking to vegetation health we develop the first longitudinal NDVI dataset using Sentinel-2 data from 2020-2023. We select the state of California [423,970km$^2$] due to its agricultural diversity~\cite{USDA1}. To support small farms, we select a sub-kilometer grid scale of 500x500~m. This however results in almost 1.7 M locations, each of which requires multi-year time series data. To facilitate large datasets, we divide the state into eight regions as seen in Fig.~\ref{fig:Fig2}. 

We divide each region into a 0.25~km$^2$ grid and follow the procedure above to extract cloudless data. Next, we filter the date range to capture the full growing cycles of various crops and show vegetation dynamics. We then compute the NDVI with Eq.\ref{eq:1} using the Sentinal-2 B8 NIR band and B4 red bands. This 70 million point dataset provides a time series of vegetation health for each location and serves as the foundational input for generating signature curves for further analysis.

%Land use understanding
%spatial factors- topography, soil types
%temporal factors- seasonal changes, long term trends
%crop patterns- rotations, growth cycles
%risk assessmentdeforestation and land degradation

%Questions:
%1. What is the 



