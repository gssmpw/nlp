\section{Initial Demonstration and Results}
We perform an initial evaluation to show TerraTrace's potential. We input coordinates from a corn farm in Stockton, CA. Preliminary analysis using features extracted from the NDVI curve show the crop has expected growth patterns, notably an initial increase in vegetative indices during April and May, followed by a peak growing season spanning June, July, and August. The maximum value (a point in the growing season when the vegetation reaches its highest level) value of 0.97 and a median (measure of the overall vegetation health throughout the growing season) of 0.8722, which confirms the hypothesis of sustained healthy vegetation throughout the growing cycle. The loss rate of -0.7729 is consistent with the rapid dry-down phase in corn harvests. To validate our findings, we cross-referenced the results with Crop Data Layer information \cite{CropScape}, which confirmed that the coordinates indeed represent corn cultivation areas. We also load the MODIS dataset for the US, calculate distance between wildfires and the target location to determine the region's fire history. 

%Collectively, these metrics strongly suggest that the coordinates under investigation correspond to a specific agricultural land use type.

%Moved the EUDR use case discussion to the next section

\begin{figure}[t]
    %\hspace*{-2.5cm} 
    \includegraphics[width=\linewidth]{Photos/TerraTrace-Fig4.pdf}
    \vspace{-0.5cm}
    \caption{Screenshots of TerraTrace}
    \label{fig:Fig4}
    \vspace{-0.7cm}
\end{figure}