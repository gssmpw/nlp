\begin{figure}[t]
    %\hspace*{-2.8cm} 
    \includegraphics[width=\linewidth]{Photos/TerraTrace-Fig2.pdf}
    \vspace{-0.5cm}
    \caption{\textbf{NDVI Signature Curves.} (A) Comparison of  NDVI on a wheat farm versus adjacent pine trees in Farmington, WA from Apr-Dec 2020. (B) NDVI curves for a chickpea farm in Grand Rapids, ND, an apple orchard in Wenatchee, WA and a citrus farm in Fresno County, CA. (C) NDVI for deciduous forest in MO, USA versus evergreen forest in WA, USA. (D) NDVI data for coffee farms in Buon Ma Thuot, Vietnam and El Paraiso, Honduras. (E) NDVI dataset created for CA, USA.}
    \label{fig:Fig2}
    \vspace{-0.6cm}
\end{figure}

\section{NDVI Crop Signature Curves}
\label{signatures}
The NDVI metric is a widely utilized remote sensing vegetation indicator~\cite{USGS}. Molecules such as chlorophyll in live green plants have higher absorption for specific wavelengths of light (e.g. 400-500~nm, 600-700~nm) while longer near infrared (NIR) light not used in photosynthesis are reflected by the leaf cell structure to prevent overheating.%Live green plants appear darker in red versus NIR wavelengths~\cite{gates2012biophysical}. 
NDVI is calculated as a ratio of spectral reflectance measurements in the NIR and visible red bands:
\vspace{-0.10cm}
\begin{equation}\label{eq:1}
    \text{NDVI} = \frac{\text{NIR} - \text{Red}}{\text{NIR} + \text{Red}}
    \vspace{-0.25cm}
\end{equation}

Our key insight is that NDVI values vary uniquely over time based on growth cycles of specific vegetation and farming practices. To compute this unique ``signature curve,'' we first pre-process Sentinel-2 imagery from Google Earth Engine (GEE) and Microsoft Planetary Computers (MPC)~\cite{GEE, Sentinel}, removing those with a cloud mask ratio over 10\%. Next, we compute daily NDVI values at a 10-meter resolution for the target region and dates.  We shift the raw NDVI to be positive to exclude bare soil or clouds and normalize them to a 0-1 scale for consistent analysis across datasets.

We begin by analyzing data from April through December 2020. Fig~\ref{fig:Fig2}(A) shows an aerial image of a wheat farm in eastern Washington along with corresponding mean NDVI values over a year in a section of a wheat field and nearby stand of pine trees. The graph shows both the raw NDVI values as well as a degree eight polynomial fit line to illustrate the trend.

% What does this mean?
%The data undergoes non-linear least-squares minimization and curve-fitting procedures to refine these estimations. 

This wheat data shows a defined change in NDVI in key agricultural periods. After the spring wheat planting and germination in April, the NDVI values increase to a peak as the plants grow. Next, after day 200, we see a sharp drop for the pre-harvest dry down and harvest ending in late August. In contrast, the adjacent pine trees maintain a consistent high NDVI throughout the year. This data shows ability to distinguish crops from adjacent forested land.

Next we analyze 20 farms in the US and show examples in Fig\ref{fig:Fig2}(B). These include a chickpea farm in Grand Rapids, ND, USA an apple orchard in Wenatchee, WA, USA and a citrus farm in Fresno County, CA, USA. The data shows changes in NDVI that reflect agricultural practices. We also analyze forests as seen in Fig~\ref{fig:Fig2}(C). The decidious forest vegetation in Missouri, USA increases in spring and declines in fall unlike the consistent data from an evergreen forest in Washington, USA.

These curves also appear to be consistent globally, due to the similarity in growing cycles. Fig~\ref{fig:Fig2}(D) compares a coffee farm in Buon Ma Thuot, Vietnam and El Paraiso, Honduras. The data shows a similar trend in NDVI. We note these regions have less data available, and we adjust our curve estimate to use a degree three polynomial. Additionally there is a fixed offset which could be due to the difference in geography or specific crop variety.





