\begin{abstract}
Understanding land use over time is critical to tracking events related to climate change, like deforestation. However, satellite-based remote sensing tools which are used for monitoring struggle to differentiate vegetation types in farms and orchards from forests. We observe that metrics such as the Normalized Difference Vegetation Index (NDVI), based on plant photosynthesis, have unique temporal signatures that reflect agricultural practices and seasonal cycles. We analyze yearly NDVI changes on 20 farms for 10 unique crops. Initial results show that NDVI curves are coherent with agricultural practices, are unique to each crop, consistent globally, and can differentiate farms from forests. We develop a novel longitudinal NDVI dataset for the state of California from 2020-2023 with 500~m resolution and over 70 million points. We use this to develop the TerraTrace platform, an end-to-end analytic tool that classifies land use using NDVI signatures and allows users to query the system through an LLM chatbot and graphical interface.
%This work presents the idea to compute yearly spatio-temporal vegetation signatures from Spectral Indices measured through satellite imaging and to perform multi-year analysis based on them to estimate land-use. We further demonstrate how such cross-year signatures can be used to infer other information about a given geolocation, such as deforestation, agricultural crop growth, land degradation and wildfire risk. This workshop paper presents an end-to-end system which demonstrates the ability to leverage the Normalized Difference Vegetation Index(NDVI) signature curves to distinguish crops. Lastly, TerraTrace also shows the benefits of using a math-based Index computation with GPT's analytical skills to obtain land use inference, estimate risks and verify them with other modalities and data sources.
\end{abstract}