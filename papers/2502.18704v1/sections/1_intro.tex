\section{Introduction}
% @Vaishnavi- we 
%As wildfires and natural disasters become more frequent due to climate change, maintaining an accurate history of land use over time is essential for food security and agricultural planning. Land use classification can help track changes in specific areas, providing critical data to understand prior land degradation and predict future risks. One such issue is deforestation. 
Every year humanity clears 10 million hectares of forests which releases over 5.6 billion tonnes of greenhouse gases annually ~\cite{UNEP2024}. This significant contribution to climate change has prompted passage of the European Union Deforestation Regulation (EUDR) ~\cite{EUDR}, which aims to ensure that products linked to deforestation are excluded from the EU market from December 2024. There is now a critical need for satellite and remote sensing technologies to help implement such regulations globally. However, these systems frequently misclassify vegetation types with similar visual characteristics. For instance, the Advanced Land Observation Satellite (ALOS) Forest Map is widely used to monitor forest growth and track deforestation \cite{ALOS}, but it falsely classifies orchards and pine plantations, as forests, as seen in Fig.\ref{fig:Fig1}(A). Similarly Fig~\ref{fig:Fig1}(B) shows that prior works do not provide data on historical land use changes at variable spatial scale which is needed to answer questions about deforestation. Additionally, they require substantial computing resources.~\cite{tseng2023lightweight, ravirathinam2024combining}.

%A comprehensive understanding of historical land use patterns is essential for accurate classification. To better understand land use across the years, a range of factors, including spatial considerations (e.g., topography), temporal variables (e.g., seasonal changes), agricultural practices (e.g., crop rotations and growth cycles), and risk assessments related to deforestation and land degradation need to be well understood.

To address these challenges we present TerraTrace: a temporal signature mapping system that combines Spectral Vegetation Indices, Satellite Imagery and Crop-Data Layer (CDL) \cite{montero,CropScape} to measure historical land use. The key insight here is that temporal variables and agricultural practices like crop rotations and growth cycles of plants are visible in satellite-based spectral index data. Specifically, we show that yearly patterns of Normalized Difference Vegetation Index (NDVI), based on plant photosynthesis, can distinguish wild forests from crops. We further find that these NDVI signatures are unique to different crops and follow consistent patterns throughout the world. Leveraging this we make the following contributions: \\
\textbf{1) Exploring NDVI Signatures.} We analyze yearly NDVI change on 20 farms with 10 unique crops. We verify the curves are consistent with agricultural practices, can distinguish between unique crops and forests. We also show they are consistent globally on coffee farms in Vietnam and Honduras. \\
\textbf{2) NDVI Dataset.} We develop a novel NDVI dataset for the state of California from 2020-2023 with 500~m resolution containing over 70 million points. \\
\textbf{3) TerraTrace Platform.} We develop the TerraTrace platform, an end-to-end analysis tool that classifies land use using mathematical analysis on NDVI signatures and allows users to query the system through an LLM chatbot interface and graphical interface.

%\textcolor{red}{TODO: Add point aout NDVI dataset - how many points for california, the fact that it doesn't exist but needs to be created}

%system with a GUI allowing users to input a set of coordinates  can take coordinates and provide detailed insights on land use.

%We develop and demonstrate the capabilities of TerraTrace for Agricultural land use in the state of California. Lastly, we demonstrate the possibility of using signature curves to identify the land use at a global scale. We display this through the example of coffee cultivation across three distinct regions, Guatemala, Vietnam and Honduras, all of which exhibit similar curves which correspond to coffee as shown in Fig.\ref{fig:Fig2}(C). We also discuss the possibility of scaling TerraTrace with multi-modal inputs and tools like SATClip \cite{klemmer2024satclip} to understand the geolocation-based features.
\begin{figure}[t]
    %\hspace*{-1.0cm} 
    \centering
    \includegraphics[width=\linewidth]{Photos/TerraTrace-Fig1.pdf}
    \vspace{-0.6cm} 
    \caption{\textbf{Land use classification challenges} (A) Current forest probability maps based on prior datasets~\cite{ALOS-paper, pittman2019global} cannot distinguish between farms and wild forests. (B) Unlike prior work \cite{presto} \cite{wstatt}, TerraTrace tracks historical land use with low computational overhead and variable scales.}
    \label{fig:Fig1}
    \vspace{-0.3cm} 
\end{figure}