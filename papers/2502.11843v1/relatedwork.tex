\section{Related Work}
Personality traits matter since LLMs mimic humans, but their structured psychological evaluation remains an unexplored gap that needs further research \cite{zhu2025investigating}.
The recent literature has looked at designing \cite{klinkert2024evaluating}, improving\cite{huang2024designing}, investigating\cite{frisch-giulianelli-2024-llm,zhu2025investigating}, customizing~\cite{han2024psydial,dan2024ptailor,zhang-etal-2018-personalizing} and exploring \cite{zhu2025investigating,han2024psydial} personality traits. The scope of our work lies both in generating and extracting personality traits embedded within discourse. 



% \mn{how your work is different from the following \cite{han2024psydial}?--(korean dataset- add, less models and data- focus on only one dimension)}
\citet{han2024psydial} contribute towards the generation of synthetic dialogues through LLMs. A five-step generation process is used where personality is induced through personality character. Special consideration on prompts is made to infer Pre-trained Language Models (PLM) in generating dialogues. This is because dialogue generation is a challenging task, especially with many constraints and maintaining personality traits. Unlike traditional methods of curating datasets by humans, the authors leverage the capability of PLM to generate synthetic data that is easily scalable. The use of these synthetic datasets significantly improved the ability of LLMs to generate content that is more tailored towards personality traits. While the research is broad, its dataset is limited to Korean and focuses on a single personality trait, which may hinder balanced trait prediction.

While designing and customising the personality traits for LLMs is an intriguing field of study, the focus of this work lies in inducing and investigating the personality traits through discourse generation \cite{yeo-etal-2025-pado}. \citet{jiang2023personallm} investigate the ability of LLMs to express personality traits through essay generation. Using both humans and LLMs as evaluators they explore the personality traits in the generated content. Evaluation through linguistic patterns (LIWC analysis) and human annotation is carried out for GPT models. They show a positive correlation between the generated content and personality traits. However, several gaps are identified such as focusing on closed models, limited data generation and conversations focused on single-ended generation(essays) which does not address the personality expression in scenarios consisting shift of context. Furthermore, the authors suggest models other than OpenAI's GPT models do not follow the instructions well, which results in discarding the content generated by these models for further evaluation. We aim to address this problem through systematic and structural prompting techniques which increases the scope of the analysis. 



% \noindent
\citet{sun2024revealing} argue that personality detection should be evidence-based rather than a classification task, enhancing explainability. They introduce the Chain of Personality Evidence (CoPE) dataset for personality recognition in dialogues, addressing state and trait recognition. However, limitations include model specialisation and the availability of a small dataset in Chinese, leaving gaps in the personality trait recognition research.



\noindent\textbf{Prompting methods:} Different methods for assigning personality traits are used in literature, mainly categorising explicit or implicit mention of personality traits or training-based methods. Most studies focus on implementing the OCEAN models to the agents \cite{bhandari2025evaluating, xi2025rise}. One common way of assigning personality traits is through direct allocation of personalities and assigning the personality traits to the agents\cite{}. Another commonly followed methodology is passing content that infers the traits but does not directly mention them \cite{sun2024revealing, han2024psydial}. Personality is also assigned through fine-tuning where distinct fine-tuned models represent distinct personalities. We believe that providing clear instructions about the personas would clear the ambiguity and hence prompt the use of the direct allocation method. 

\noindent\textbf{Evaluation:} LLMs are increasingly used to evaluate personality traits from the text. While their accuracy is still under study, they offer a cost-effective and efficient approach.

\citet{zhu2025investigating} use closed-source models (GPT-4o and GPT-4o-mini) to infer the BFI traits and extract the scores. 

Authors present the findings that the effectiveness of LLMs in predicting personality traits increased as they were prompted with an intermediate step of BFI-10~\cite{bfi10} questionnaires. Two main metrics were used to benchmark the ability of LLMs: correlation and mean difference, where correlation measured the ability to capture structural relationships and mean difference captured absolute prediction accuracy. We also adapt these metrics to evaluate the content produced by LLMs in our agent ecosystem. Different validation datasets relating to personality traits include: Essay Dataset~\cite{yeo-etal-2025-pado}, myPersonality~\cite{mypersonality}, and Twitter Dataset~\cite{twitterdata}.
 

In summary, the main problems identified in the literature are the use of closed-source models, the lack of analysis in content generation consisting of context-shifting behaviour, and the lack of use of standard evaluation metrics. Furthermore, one of the main challenges in incorporating personality traits is understanding whether all five traits are effectively adhered to in the content that is produced. We aim to address some of these problems through this research.