% This must be in the first 5 lines to tell arXiv to use pdfLaTeX, which is strongly recommended.
\pdfoutput=1
% In particular, the hyperref package requires pdfLaTeX in order to break URLs across lines.

% \documentclass[11pt]{article}

% Change "review" to "final" to generate the final (sometimes called camera-ready) version.
% Change to "preprint" to generate a non-anonymous version with page numbers.
\usepackage[review]{acl}

% Standard package includes
\usepackage{times}
\usepackage{latexsym}
\usepackage{subcaption}
\usepackage{listings}
% For proper rendering and hyphenation of words containing Latin characters (including in bib files)
\usepackage[T1]{fontenc}
% For Vietnamese characters
% \usepackage[T5]{fontenc}
% See https://www.latex-project.org/help/documentation/encguide.pdf for other character sets

% This assumes your files are encoded as UTF8
\usepackage[utf8]{inputenc}

% This is not strictly necessary, and may be commented out,
% but it will improve the layout of the manuscript,
% and will typically save some space.
\usepackage{microtype}

% This is also not strictly necessary, and may be commented out.
% However, it will improve the aesthetics of text in
% the typewriter font.
\usepackage{inconsolata}

%Including images in your LaTeX document requires adding
%additional package(s)
\usepackage{graphicx}
\usepackage{pdflscape}  % For landscape tables
\usepackage{array}       % For column formatting
\usepackage{adjustbox}
\usepackage{booktabs}    % Professional-quality tables

% \usepackage{graphicx}
% \usepackage{subcaption}
% \usepackage{multirow}    % For multi-row cells
\usepackage{ragged2e} 
\usepackage{xcolor}
\usepackage{cellspace} % For cell padding
\usepackage{subfigure}

\usepackage{caption}
\usepackage{hyperref}
\newcommand{\mn}[1]{{\color{blue}[[MN: {#1}]]}}
% Better text alignment% If the title and author information does not fit in the area allocated, uncomment the following
%
%\setlength\titlebox{<dim>}
%
% and set <dim> to something 5cm or larger.

%\title{Simulating Personality in AI-Enabled Agents: Evaluating Trait Consistency in Generated Discourse}
\title{Conversations with Personality: Generation and Evaluation of Trait Adherent Discourse with LLM Agents.}

% Author information can be set in various styles:
% For several authors from the same institution:
% \author{Author 1 \and ... \and Author n \\
%         Address line \\ ... \\ Address line}
% if the names do not fit well on one line use
%         Author 1 \\ {\bf Author 2} \\ ... \\ {\bf Author n} \\
% For authors from different institutions:
% \author{Author 1 \\ Address line \\  ... \\ Address line
%         \And  ... \And
%         Author n \\ Address line \\ ... \\ Address line}
% To start a separate ``row'' of authors use \AND, as in
% \author{Author 1 \\ Address line \\  ... \\ Address line
%         \AND
%         Author 2 \\ Address line \\ ... \\ Address line \And
%         Author 3 \\ Address line \\ ... \\ Address line}

\author{First Author \\
  Affiliation / Address line 1 \\
  Affiliation / Address line 2 \\
  Affiliation / Address line 3 \\
  \texttt{email@domain} \\\And
  Second Author \\
  Affiliation / Address line 1 \\
  Affiliation / Address line 2 \\
  Affiliation / Address line 3 \\
  \texttt{email@domain} \\}

%\author{
%  \textbf{First Author\textsuperscript{1}},
%  \textbf{Second Author\textsuperscript{1,2}},
%  \textbf{Third T. Author\textsuperscript{1}},
%  \textbf{Fourth Author\textsuperscript{1}},
%\\
%  \textbf{Fifth Author\textsuperscript{1,2}},
%  \textbf{Sixth Author\textsuperscript{1}},
%  \textbf{Seventh Author\textsuperscript{1}},
%  \textbf{Eighth Author \textsuperscript{1,2,3,4}},
%\\
%  \textbf{Ninth Author\textsuperscript{1}},
%  \textbf{Tenth Author\textsuperscript{1}},
%  \textbf{Eleventh E. Author\textsuperscript{1,2,3,4,5}},
%  \textbf{Twelfth Author\textsuperscript{1}},
%\\
%  \textbf{Thirteenth Author\textsuperscript{3}},
%  \textbf{Fourteenth F. Author\textsuperscript{2,4}},
%  \textbf{Fifteenth Author\textsuperscript{1}},
%  \textbf{Sixteenth Author\textsuperscript{1}},
%\\
%  \textbf{Seventeenth S. Author\textsuperscript{4,5}},
%  \textbf{Eighteenth Author\textsuperscript{3,4}},
%  \textbf{Nineteenth N. Author\textsuperscript{2,5}},
%  \textbf{Twentieth Author\textsuperscript{1}}
%\\
%\\
%  \textsuperscript{1}Affiliation 1,
%  \textsuperscript{2}Affiliation 2,
%  \textsuperscript{3}Affiliation 3,
%  \textsuperscript{4}Affiliation 4,
%  \textsuperscript{5}Affiliation 5
%\\
%  \small{
%    \textbf{Correspondence:} \href{mailto:email@domain}{email@domain}
%  }
%}
\lstset{
    basicstyle=\ttfamily\footnotesize,
    backgroundcolor=\color{gray!10},
    frame=single,
    breaklines=true,   % Enables automatic line wrapping
    breakatwhitespace=true,
    linewidth=\linewidth,  % Ensures content fits within the column
    xleftmargin=0pt,   % Adjusts left margin to align with the column
    keywordstyle=\color{blue},
    commentstyle=\color{green!50!black},
    stringstyle=\color{red!80!black}
}


\begin{document}
% \bibliography{custom}

\appendix

\label{sec:appendix}

\section{Sample of Topics and Trait Combinations Used}
Samples of \emph{topics} used for debate:
\begin{lstlisting}
"Is the concept of a universal language beneficial?",
"Should the government regulate the pharmaceutical industry?",
"Is the use of nuclear energy justified?",
"Should the government provide free public transportation?",
"Is the concept of a cashless society beneficial?",
"Should the government regulate the gaming industry?"
\end{lstlisting}

\emph{Trait} combinations samples to assign personas to Agents: 

\begin{lstlisting}
{"Agreeableness": "High", "Openness": "Low", "Conscientiousness": "High", "Extraversion": "Low", "Neuroticism": "High"},
{"Agreeableness": "Low", "Openness": "High", "Conscientiousness": "Low", "Extraversion": "High", "Neuroticism": "Low"},
{"Agreeableness": "High", "Openness": "High", "Conscientiousness": "Low", "Extraversion": "High", "Neuroticism": "High"},
{"Agreeableness": "Low", "Openness": "Low", "Conscientiousness": "High", "Extraversion": "Low", "Neuroticism": "Low"},
{"Agreeableness": "High", "Openness": "High", "Conscientiousness": "High", "Extraversion": "Low", "Neuroticism": "Low"}
\end{lstlisting}

\section{System and User prompts}

We use, different \emph{System and User} prompts to extract the discourses and ratings from the conversing and judge agents. 

\subsection{Discourse Generation }
The \emph{system prompt} to generate the discourses:

     
\begin{lstlisting}
SYSTEM_PROMPT = ''' f"You are participating in a structured debate on: '{topic}'\n"
"Your responses should reflect these personality traits:\n"
f"- Agreeableness: {traits['Agreeableness']}\n"
f"- Openness: {traits['Openness']}\n"
f"- Conscientiousness: {traits['Conscientiousness']}\n"
f"- Extraversion: {traits['Extraversion']}\n"
f"- Neuroticism: {traits['Neuroticism']}\n\n"
"Rules:\n"
"- Maintain these personality traits (DO NOT EXPLICITLY MENTION IN TEXT) at all
times during your conversation\n"
"- Keep responses under 50 words\n"
"- Maintain your personality consistently\n"
"- Address previous arguments directly but do not repeat what
the other speaker said.\n"
"- End with proper punctuation" ''''
\end{lstlisting} 

The \emph{user prompt} carries the previous argument :
\begin{lstlisting}
USER_PROMPT = """Previous Argument:f"{previous_arguement}" """
\end{lstlisting}


\subsection{Extracting Personalities from the Judge Agents.}
The \emph{system prompt} to extract the personality traits:

\begin{lstlisting}
SYSTEM_PROMPT = """Analyze text segments from two anonymous debaters (Person One and Person Two) for:
1. Big Five Inventory (BFI) traits (High/Low for each dimension)
2. Consistency with typical behavior for those traits (Yes/No)

For each person, return:
{
    "predicted_bfi": {
        "Agreeableness": "High/Low",
        "Openness": "High/Low",
        "Conscientiousness": "High/Low",
        "Extraversion": "High/Low",
        "Neuroticism": "High/Low"
    }
}
"""
\end{lstlisting}
The \emph{user prompt} is: 

\begin{lstlisting}
USER_PROMPT= '''f"Analyze{persona}'s text:\n{text}'''    
\end{lstlisting}
where the \emph{persona} contains Participant 1 and 2  and the \emph{text} contains the discourses for each of the participants respectively. 

\newpage
\section{Metadata of the Discourses.}
% For professional tables


% Table 1: GPT-4o vs GPT-4o-mini
\begin{table}[h]
    \centering
    \begin{tabular}{lc}
        \toprule
        \textbf{Metric} & \textbf{GPT-4o vs GPT-4o-mini} \\
        \midrule
        % Total Utterances & 16,160 \\
        Total Sentences & 70,750 \\
        Total Words & 781,330 \\
        Assertions & 14,653 \\
        Questions & 1,507 \\
        Logical Structures & 690 \\
        Total Dialogues & 2,020 \\
        Avg. Words per Sentence & 11.04 \\
        Avg. Utterance Length & 48.35 \\
        \bottomrule
    \end{tabular}
    \caption{Metadata analysis for GPT-4o vs GPT-4o-mini}
    \label{tab:gpt4o_vs_gpt4o_mini}
\end{table}

% Table 2: LLaMA-3 vs GPT-4
\begin{table}[h]
    \centering
    \begin{tabular}{lc}
        \toprule
        \textbf{Metric} & \textbf{LLaMA-3 vs GPT-4o} \\
        \midrule
        % Total Utterances & 18,180 \\
        Total Sentences & 44,964 \\
        Total Words & 541,603 \\
        Assertions & 15,577 \\
        Questions & 2,603 \\
        Logical Structures & 767 \\
        Total Dialogues & 2,020 \\
        Avg. Words per Sentence & 12.05 \\
        Avg. Utterance Length & 29.79 \\
        \bottomrule
    \end{tabular}
    \caption{Metadata analysis for LLaMA-3 vs GPT-40}
    \label{tab:llama3_vs_gpt4}
\end{table}

% Table 3: DeepSeek vs LLaMA
\begin{table}[h]
    \centering
    \begin{tabular}{lc}
        \toprule
        \textbf{Metric} & \textbf{DeepSeek vs GPT-4o} \\
        \midrule
        % Total Utterances & 18,180 \\
        Total Sentences & 44,387 \\
        Total Words & 1,033,592 \\
        Assertions & 17,800 \\
        Questions & 380 \\
        Logical Structures & 4,697 \\
        Total Dialogues & 2,020 \\
        Avg. Words per Sentence & 23.29 \\
        Avg. Utterance Length & 56.85 \\
        \bottomrule
    \end{tabular}
    \caption{Metadata analysis for DeepSeek vs GPT-4o}
    \label{tab:deepseek_vs_llama}
\end{table}









\end{document}