% This must be in the first 5 lines to tell arXiv to use pdfLaTeX, which is strongly recommended.
\pdfoutput=1
% In particular, the hyperref package requires pdfLaTeX in order to break URLs across lines.

\documentclass[11pt]{article}
\usepackage{titlesec}
% Change "review" to "final" to generate the final (sometimes called camera-ready) version.
% Change to "preprint" to generate a non-anonymous version with page numbers.
\usepackage[preprint]{acl}

% Standard package includes
\usepackage{times}
\usepackage{latexsym}
\usepackage{subcaption}
\usepackage{amsmath}
\usepackage{amssymb}% For proper rendering and hyphenation of words containing Latin characters (including in bib files)
%\usepackage[font=footnotesize,labelformat=simple]{subcaption}
%\renewcommand\thesubfigure{(\alph{subfigure})}\usepackage[T1]{fontenc}

\usepackage{listings}
% For Vietnamese characters
% \usepackage[T5]{fontenc}
% See https://www.latex-project.org/help/documentation/encguide.pdf for other character sets

% This assumes your files are encoded as UTF8
\usepackage[utf8]{inputenc}
\usepackage{subfiles}
% This is not strictly necessary, and may be commented out,
% but it will improve the layout of the manuscript,
% and will typically save some space.
\usepackage{microtype}

% This is also not strictly necessary, and may be commented out.
% However, it will improve the aesthetics of text in
% the typewriter font.
\usepackage{inconsolata}

%Including images in your LaTeX document requires adding
%additional package(s)
\usepackage{graphicx}
\usepackage{pdflscape}  % For landscape tables
\usepackage{array}       % For column formatting
\usepackage{adjustbox}
\usepackage{booktabs}    % Professional-quality tables

% \usepackage{graphicx}
% \usepackage{subcaption}
\usepackage{multirow}    % For multi-row cells
\usepackage{ragged2e} 
\usepackage{xcolor}
\usepackage{cellspace} % For cell padding
%\usepackage{subfig}
\usepackage{enumitem}
%%%%%%%%%%%%%%%%%%%%%%% to fix arxiv undefined seq problem
\usepackage[font=footnotesize,labelformat=simple]{subcaption}
\captionsetup[sub]{labelformat=simple}\renewcommand\thesubfigure{(\alph{subfigure})}\usepackage[T1]{fontenc}

%%%%%%%%%%%%%%%%%%%%%%%%%%%%%%%
%\usepackage{caption}
\usepackage{hyperref}
% \newcommand{\mn}[1]{{\color{blue}[[MN: {#1}]]}}

% \newcommand{\un}[1]{{\color{red}[[UN: {#1}]]}}

% \newlist{RQ}{enumerate}{1}
% \setlist[RQ]{label=\textbf{RQ\,\arabic*},ref={RQ\,\arabic*}}
% Better text alignment% If the title and author information does not fit in the area allocated, uncomment the following
%

\lstset{
    basicstyle=\ttfamily\footnotesize,
    backgroundcolor=\color{gray!10},
    frame=single,
    breaklines=true,   % Enables automatic line wrapping
    breakatwhitespace=true,
    linewidth=\linewidth,  % Ensures content fits within the column
    xleftmargin=0pt,   % Adjusts left margin to align with the column
    keywordstyle=\color{blue},
    commentstyle=\color{green!50!black},
    stringstyle=\color{red!80!black}
}
%\setlength\titlebox{<dim>}
%
% and set <dim> to something 5cm or larger.

%\title{Simulating Personality in AI-Enabled Agents: Evaluating Trait Consistency in Generated Discourse}
%\title{Conversations with Personality: Generation and Evaluation of Trait Adherent Discourse with LLM Agents.}

\title{Can LLM Agents Maintain a Persona in Discourse?}

% Author information can be set in various styles:
% For several authors from the same institution:
% \author{Author 1 \and ... \and Author n \\
%         Address line \\ ... \\ Address line}
% If the names do not fit well on one line use
%         Author 1 \\ {\bf Author 2} \\ ... \\ {\bf Author n} \\
% For authors from different institutions:
% \author{Author 1 \\ Address line \\  ... \\ Address line
%         \And  ... \And
%         Author n \\ Address line \\ ... \\ Address line}
% To start a separate ``row'' of authors use \AND, as in
% \author{Author 1 \\ Address line \\  ... \\ Address line
%         \AND
%         Author 2 \\ Address line \\ ... \\ Address line \And
%         Author 3 \\ Address line \\ ... \\ Address line}


\author{
 \textbf{Pranav Bhandari\textsuperscript{1}},
 \textbf{Nicolas Fay\textsuperscript{2}},
 \textbf{Michael Wise\textsuperscript{1}},
 \textbf{Amitava Datta\textsuperscript{1}},
\\
 \textbf{Stephanie Meek\textsuperscript{3}},
 \textbf{Usman Naseem\textsuperscript{4}},
 \textbf{Mehwish Nasim\textsuperscript{1}},
\\
 \textsuperscript{1}School of Physics Mathematics and Computing, University of Western Australia,
 \\
 \textsuperscript{2}School of Psychological Sciences, University of Western Australia,
 \\
 \textsuperscript{3}School of Business and Law, Edith Cowan University,
 \\
 \textsuperscript{4}School of Computing, Macquarie University
\\
 \small{
   \textbf{Correspondence:} \href{mailto:mehwish.nasim@uwa.edu.au}{mehwish.nasim@uwa.edu.au}
 } %I have put myself as the corresponding author for the preprint version.
}

\begin{document}
\maketitle
\begin{abstract}


%background % research problem %objective%method %key finding %conclusion

%Large Language Models (LLMs) are widely used as conversational agents exploiting their capabilities in various sectors such as education, law, medicine, and more. However, LLMs are often subjected to context-shifting behaviour, resulting in a lack of consistent and interpretable personality-aligned interactions. This work investigates whether and how LLM agents maintain assigned personality traits during dyadic conversations. We employ a novel agent-based evaluation framework: two LLM agent agents, each assigned a distinct OCEAN personality profile (Openness, Conscientiousness, Extraversion, Agreeableness, and Neuroticism as High/Low for each trait) and multiple judge agents to map back the original traits assigned to explore prediction consistency, inter-model agreement, alignment with assigned personality. Our findings highlight the significant challenges in achieving consistent, interpretable, and consistent personality-adherent interactions.



Large Language Models (LLMs) are widely used as conversational agents exploiting their capabilities in various sectors such as education, law, medicine, and more. 
%The personalities these agents reflect are a key factor determining their conversation style, linguistic choice and overall engagement in the discourse. However, from a psychological perspective, the adherence to these psychological traits by the agents lacks comprehensive analysis, especially in the cases of dyadic (pairwise) conversations where the shift of context is prominent and personality adaptations are required according to the opponent.
However, LLMs are often subjected to context-shifting behaviour, resulting in a lack of consistent and interpretable personality-aligned interactions. Adherence to psychological traits lacks comprehensive analysis, especially in the case of dyadic (pairwise) conversations. 
We examine this challenge from two viewpoints, initially using two conversation agents to generate a discourse on a certain topic with an assigned personality from the OCEAN framework (Openness, Conscientiousness, Extraversion, Agreeableness, and Neuroticism) as High/Low for each trait. This is followed by using multiple judge agents to infer the original traits assigned to explore prediction consistency, inter-model agreement, and alignment with the assigned personality. Our findings indicate that while LLMs can be guided toward personality-driven dialogue, their ability to maintain personality traits varies significantly depending on the combination of models and discourse settings. These inconsistencies emphasise the challenges in achieving stable and interpretable personality-aligned interactions in LLMs.
% \un{can you complete the abstract - draft}
% \mn{we need to think of  a catchy title}

%Authors: Pranav Bhandari, Usman Naseem, Michael Wise, Amitava Datta, Stephanie Meek, Nicolas Fay and Mehwish Nasim

\end{abstract}

\section{Introduction}
%sections to write>
%background
\begin{figure}[th]
  \centering
  %\includegraphics[width = 7.9cm, height = 6cm]{acl_firstfig.pdf}
  \includegraphics[scale =0.24]{acl_firstfig}
  \caption{An example of inducing personality in LLM agents, followed by a discourse. A judge agent evaluates whether personality traits were adhered to in the discourse. }
  \label{fig:ToyExample}
  \vspace{-0.35cm}
\end{figure}

Large language models (LLMs) have evolved from task solvers and general-purpose chatbots to sophisticated conversational agents capable of embodying distinct personas. This shift towards personalised agents, driven by LLMs' capacity for perception, planning, generalisation, and learning \cite{xi2025rise}, has enabled context-sensitive discourse and opened up new possibilities across diverse domains.  Persona, defined as conditioning AI models to adopt specific roles and characteristics \cite{li2024steerability}, is a key element in this evolution.  Personalised agents show promise in areas such as emotional support, training, and social skills development \cite{dan2024ptailor}, and are increasingly explored for applications ranging from social science research \cite{zhu2025investigating} to mimicking human behaviour \cite{jiang2023personallm}.  While various personalisation approaches exist, incorporating personas has proven particularly effective in generating contextually appropriate responses and enhancing overall performance \cite{tseng2024twotales, dan2024ptailor}.
%problem statement


% While research on personas offers insights into LLM behavior, a critical challenge lies in understanding how these models express and maintain specific personality traits, especially in dynamic conversational settings.  Although LLMs often generate neutral and balanced content, consistent and nuanced personality expression is crucial for many applications.  
%summarised below:

Understanding how LLMs express and sustain personality traits in dynamic conversations is crucial, despite their tendency to generate neutral, balanced content. Existing work has explored personality in text using tools like the Big Five Inventory (BFI) \cite{john1991bfi} to infer and analyse personality profiles \cite{bhandari2025evaluating}. However, two key gaps remain. First, it is unclear how consistently LLMs portray assigned personality traits during extended interactions, particularly in pairwise (dyadic) conversations where context shifts and adaptation are necessary. Second, robust methods are needed to evaluate the alignment between the expressed traits in the generated text and the intended psychological profile. We present an example in Figure \ref{fig:ToyExample}. 
%\un{Pranav - can we add a toy example (motivation and also explain dyadic - not many people know this- work in p } \mn {Done. }

% research gap 
% Previous studies \cite{jiang2023personallm,kim2025can}  have successfully worked on the first problem to address whether LLMs reflect personality traits. Through the administration of various personality traits questionnaires, the tests validate that LLMs reflect the personality traits assigned to them. Considering these research outputs we focus on the other research gap that needs to be addressed if these traits are maintained across their generated content, especially in the case of debates or dyadic conversations that consist of frequent context shifting. Although the results obtained from administering personality traits to these conversational agents result in positive outcomes, it does not validate that the content generated reflects the traits and to what degree. Hence, we approach the problem of determining whether the personality traits are reflected by the LLMs through the generation and evaluation of content. 

While previous studies \cite{jiang2023personallm,kim2025can} have made progress in demonstrating that LLMs can reflect assigned personality traits (often through personality questionnaires), a critical gap remains in understanding how consistently these traits are maintained in generated content, particularly within dynamic conversational settings.  Although assigning personality traits to conversational agents often yields positive results in controlled settings, this does not guarantee that the generated content effectively expresses those traits, nor does it quantify the degree of expression.  Our work addresses this gap by focusing on the generation and evaluation of trait-adherent discourse, specifically within dyadic conversations involving frequent context shifts.  We investigate whether and how LLMs maintain assigned personalities during these dynamic interactions, beyond simply demonstrating the potential for personality reflection to assessing its actual manifestation in conversation.
%\mn{Plural of content in this case is content as we are not referring to a countable noun}.

%objectives 
% The overarching aim of this work is to understand whether LLMs reflect personality traits that are assigned to them in the content they create. This is done through the direct implication of the personality traits(\emph{five domains from the BFI called OCEAN model: Openness, Conscientiousness, Extraversion, Agreeableness, Neuroticism}) to different state-of-the-art models, generating a discourse on a given topic and evaluating the content for personality reflection. In an agent-based setting, a pair of LLMs interact with each other to generate conversational-level content while an external agent makes judgment about the generated content. 

This work aims to investigate how effectively LLMs express assigned personality traits in generated dialogue.  Specifically, we explore whether and how LLMs maintain Big Five Personality traits, which are represented as the \textbf{OCEAN} framework \cite{husain2025reliability} (\emph{Openness, Conscientiousness, Extraversion, Agreeableness, and Neuroticism}), during dyadic conversations.  We employ a novel agent-based evaluation framework where two LLM agents, each assigned a distinct OCEAN personality profile, engage in a conversation on a given topic.  Subsequently, independent LLM agents (\emph{judges}) assess the generated dialogue to determine the consistency between expressed and assigned traits. This approach allows us to analyse not only whether LLMs reflect personality, but also the peculiarities in trait expression and the challenges of maintaining personality consistency within dynamic conversational contexts.
% \un{Pranav pls give the reference or link or some definition as a footnote abt OCEAN framework---referenced and summarized as bfi trait given}

%hypothesis??? do we have any
% Thus, we posit the following research questions:
This work seeks to address the following research questions:

 % \begin{RQ}[align=parleft, leftmargin=!,itemsep=0pt,labelsep=12pt] 
 %        \item How accurately can "judge" LLMs predict assigned traits from dialogue? \label{RQA}
 %        \item How consistently do LLM agents express assigned personality traits in conversation? \label{RQB}
 %        \item Are all OCEAN traits equally prominent in generated conversations? \label{RQC}
 %    \end{RQ}
%\mn{same Rqs, just changed formattting}


\noindent\textbf{RQ1:} How accurately LLMs as a \emph{judge agent} predict assigned traits from discourse?

\noindent\textbf{RQ2:} How consistently do LLM agents express assigned personality traits in conversations?

\noindent\textbf{RQ3:} Are all OCEAN traits equally prominent in generated conversations?

% \noindent\textbf{RQ4:} Does agent role (creator vs. judge) affect personality-related performance?

% \textbf{RQ1:}Do LLMs effectively demonstrate the assigned personality traits in the content that they generate?

% \textbf{RQ2:}Can LLMs accurately classify the personality traits from the given content? 


% \textbf{RQ3:}Are all five traits from the OCEAN framework are modelled in a balanced manner in the content produced by LLMs? 

% \textbf{RQ4:}Are the agents better judges or better conversation creators?
% \mn{I think in addtion to the above we are also trying to judge whether P1 gets influenced by the personality traits of P2? For example if P1 and P2 are high on agreeableness, does one lower the score of agreeableness or vicevers?}



%contribution 
% \mn{If we have RQs then we may not need the contribution. However, we should categorically address each RQ perhaps in the discussion and present a crisp finding}
% \textbf{Contribution:}(Write in the end.)


\begin{figure*}[th]
  \centering
  \includegraphics[scale=0.39]{methodology_acl} % Scale reduces both width & height proportionally
  \caption{Methodology of the paper. \textbf{System prompt} inducing traits and topic of discourse are passed with the \textbf{User prompt} containing previous utterance. The conversations are then extracted and analysed by \textbf{Judge Agents} to report the findings.}
  \label{fig:methodology}
\end{figure*}



\section{Related Work}


Personality traits matter since LLMs mimic humans, but their structured psychological evaluation remains an unexplored gap that needs further research \cite{zhu2025investigating}.
The recent literature has looked at designing \cite{klinkert2024evaluating}, improving\cite{huang2024designing}, investigating\cite{frisch-giulianelli-2024-llm,zhu2025investigating}, customizing~\cite{han2024psydial,dan2024ptailor,zhang-etal-2018-personalizing} and exploring \cite{zhu2025investigating,han2024psydial} personality traits. The scope of our work lies both in generating and extracting personality traits embedded within discourse. 



% \mn{how your work is different from the following \cite{han2024psydial}?--(korean dataset- add, less models and data- focus on only one dimension)}
\citet{han2024psydial} contribute towards the generation of synthetic dialogues through LLMs. A five-step generation process is used where personality is induced through personality character. Special consideration on prompts is made to infer Pre-trained Language Models (PLM) in generating dialogues. This is because dialogue generation is a challenging task, especially with many constraints and maintaining personality traits. Unlike traditional methods of curating datasets by humans, the authors leverage the capability of PLM to generate synthetic data that is easily scalable. The use of these synthetic datasets significantly improved the ability of LLMs to generate content that is more tailored towards personality traits. While the research is broad, its dataset is limited to Korean and focuses on a single personality trait, which may hinder balanced trait prediction.

While designing and customising the personality traits for LLMs is an intriguing field of study, the focus of this work lies in inducing and investigating the personality traits through discourse generation \cite{yeo-etal-2025-pado}. \citet{jiang2023personallm} investigate the ability of LLMs to express personality traits through essay generation. Using both humans and LLMs as evaluators they explore the personality traits in the generated content. Evaluation through linguistic patterns (LIWC analysis) and human annotation is carried out for GPT models. They show a positive correlation between the generated content and personality traits. However, several gaps are identified such as focusing on closed models, limited data generation and conversations focused on single-ended generation(essays) which does not address the personality expression in scenarios consisting shift of context. Furthermore, the authors suggest models other than OpenAI's GPT models do not follow the instructions well, which results in discarding the content generated by these models for further evaluation. We aim to address this problem through systematic and structural prompting techniques which increases the scope of the analysis. 



% \noindent
\citet{sun2024revealing} argue that personality detection should be evidence-based rather than a classification task, enhancing explainability. They introduce the Chain of Personality Evidence (CoPE) dataset for personality recognition in dialogues, addressing state and trait recognition. However, limitations include model specialisation and the availability of a small dataset in Chinese, leaving gaps in the personality trait recognition research.



\noindent\textbf{Prompting methods:} Different methods for assigning personality traits are used in literature, mainly categorising explicit or implicit mention of personality traits or training-based methods. Most studies focus on implementing the OCEAN models to the agents \cite{bhandari2025evaluating, xi2025rise}. One common way of assigning personality traits is through direct allocation of personalities and assigning the personality traits to the agents\cite{}. Another commonly followed methodology is passing content that infers the traits but does not directly mention them \cite{sun2024revealing, han2024psydial}. Personality is also assigned through fine-tuning where distinct fine-tuned models represent distinct personalities. We believe that providing clear instructions about the personas would clear the ambiguity and hence prompt the use of the direct allocation method. 

\noindent\textbf{Evaluation:} LLMs are increasingly used to evaluate personality traits from the text. While their accuracy is still under study, they offer a cost-effective and efficient approach.

\citet{zhu2025investigating} use closed-source models (GPT-4o and GPT-4o-mini) to infer the BFI traits and extract the scores. 

Authors present the findings that the effectiveness of LLMs in predicting personality traits increased as they were prompted with an intermediate step of BFI-10~\cite{bfi10} questionnaires. Two main metrics were used to benchmark the ability of LLMs: correlation and mean difference, where correlation measured the ability to capture structural relationships and mean difference captured absolute prediction accuracy. We also adapt these metrics to evaluate the content produced by LLMs in our agent ecosystem. Different validation datasets relating to personality traits include: Essay Dataset~\cite{yeo-etal-2025-pado}, myPersonality~\cite{mypersonality}, and Twitter Dataset~\cite{twitterdata}.
 

In summary, the main problems identified in the literature are the use of closed-source models, the lack of analysis in content generation consisting of context-shifting behaviour, and the lack of use of standard evaluation metrics. Furthermore, one of the main challenges in incorporating personality traits is understanding whether all five traits are effectively adhered to in the content that is produced. We aim to address some of these problems through this research. 


\section{Methodology}

We present the methodology of this work in Figure \ref{fig:methodology}. In an agent-based setting the methodology is operationalised in 4 phases: 
\emph{\textbf{P}ersonifying agents}, 
\emph{\textbf{G}enerating discourse}, 
\emph{\textbf{E}xtracting personality within discourse}, and 
\emph{\textbf{E}valuation}.
%using multiple models to generate and explore the nature of the generated content.
A detailed explanation of the modular approach is presented in subsequent sections.
In summary, the \emph{psychological personas} are assigned to two agents and asked to converse on a topic. The discourse is evaluated using independent agents --- \emph{judge agents} through several evaluation metrics. 




We adopted an iterative approach to refine the methodology. Various problems were encountered while producing the discourse between the models, starting with synchronization issues, over-generalisation, repeating the prompts, and explicitly mentioning the personality that the LLMs have assumed. 
%We go through multiple stages to address these problems by refining the prompts and methodology. 
Furthermore, in a dyadic conversation between two agents, the subsequent dialogues are highly dependent on the previous conversation, hence one unjustified/bad response can cause the whole conversation to deviate from its original objective. 
%Hence, special consideration to have full and sensible conversations have been made. 
Hence, special consideration has been given to achieving complete and sensible conversations.
To validate that LLMs are not generating the same dialogues as before, we perform a similarity check across all the dyadic conversations and validate them.

We selected GPT models from OpenAI\cite{openai2024gpt4omini} and LLaMA models from Meta\cite{llama_cite} due to their popularity and reach. As the landscape rapidly evolved, we expanded our scope to include DeepSeek\footnote{\href{https://huggingface.co/deepseek-ai/deepseek-llm-67b-chat}{DeepSeek models}} to ensure broader coverage and comparison across architectures.
 
Since the generation of essays on a particular topic has been explored in literature such as \cite{kim2025can,yeo-etal-2025-pado}, we wanted to explore the generation of discourses, particularly for two reasons \textbf{1)} The complexity of the topic increases and maintaining a progressive discussion given the explicit persona is a difficult task. \textbf{2)} It is also interesting to understand the consistency in the personality during a conversation. 

%as they shift their conversations in a discourse. 


\noindent\textbf{Dataset:} We have carefully selected 100 different topics that require, ethical, moral, social or political considerations
\footnote{\href{https://tpd.edu.au/most-controversial-debate-topics/?srsltid=AfmBOooSOI5B5SmGeneZVV9jMuIyqskncYVGpKYepmcgntSW15Czt6Vb}{Debate Topics}} and 20 different combinations of random traits (more in Appendix).

\subsection{Prompt formation} \label{sec:prompt}

There are two basic requirements to create the discourse between two agents. The first one is the assigned persona of the OCEAN model 
(the Big Five Inventory)~\cite{john1991bfi} that is to be maintained at all times while producing an utterance and second is the consideration of the previous utterance in the dyadic conversation so that the current utterance reflects the understanding of the previous utterance and is not an independent reply. In addition, the context of the utterances must be lexically similar to the topic given. 

The prompt formation is an essential part of our methodology. Since the discourse is analysed by other agents and we draw the results based on the discourse, it must be structured robustly to ensure reliability and objective evaluation.

Prompting for LLMs is carried out through specific prompting methods where agents are assigned roles to convey requirements and expected outcomes. Usually, the \emph{system and user} roles are passed as arguments \cite{yeo-etal-2025-pado} in which the system role is responsible for defining the behaviour and limiting the scope of response and the user role is used for defining the input. Despite strict adherence to these techniques, agents may still be overwhelmed by excessive constraints. 

\noindent\textbf{System Prompt}: The system prompt in our case contains the rules for debates carried out on a specific topic. Structured prompts enhance clarity for agents, improve effectiveness, and help users create inclusive prompts despite multiple constraints. Although the formatting of the prompts varies according to the model specifications, they contain the following information. 
\begin{itemize}[noitemsep,leftmargin=*]
    \item The traits are assigned in two forms of extremities: \emph{High or Low}. 
    \item You are a participant in a discourse in which the topic is \emph{${topic}$} and presented with the following traits \emph{${traits}$}.
    % \item You have been assigned the following personality traits \emph{${traits}$}.
    \item Assigned personality traits must be maintained throughout the conversation but not explicitly mentioned in the utterances.
    \item Each utterance must be under 50 words and the previous utterance needs to be addressed.
    % \item Points from the previous utterance need to be addressed.
    % \item Use natural conversational English and conclude well with punctuation. 
\end{itemize}

%Providing structural prompts to these agents has two benefits: First, the objective is clear and understandable for the agents, and the prompt becomes more effective. Second, it is easier for the users to create an inclusive prompt when multiple constraining factors are present. 

\noindent\textbf{User Prompt:} User prompt in this case contributes to an important role in shaping the conversation because the previous discussions are passed through the user prompt to generate the next utterance. 

%not sure to include this section in the paper but it is a worthy mention in the thesis itself.
%\textbf{Ability of systems to understand and follow the prompts:} Based on our experience of inferencing different models across experiments, 

During the experiments, we noted that GPT models followed instructions effectively in a zero-shot setting with minimal guidance, while models like Llama and DeepSeek required more detailed explanations and constraints. This suggests that GPT models are more adaptable to imperfect prompts compared to other state-of-the-art models.
% A wide range of experiments were conducted to evaluate model inference. While different models vary in their ability to process prompts, we find that \emph{GPT models are the most prompt-friendly}, requiring fewer modifications to generate responses aligned with assigned traits. Compared to LLaMA and Deepseek they exhibit greater responsiveness and consistency in personality-constrained dialogue.

% Prompting techniques also known as prompt engineering are an essential part of interacting with the models because they determine the quality and the nature of the output. 

% This analysis is derived from our experience of inferencing models for different experiments. While creating the discourse, GPT models easily followed all the instructions in a zero-shot setting with a minimum explanation of what was expected out of it. On the other hand, other advanced models like Llama and Deepseek models needed more detailed explanations, examples and extended constraints as compared to GPT. This behaviour leads us to believe that GPT models are well responsive to (not so perfect?) prompts compared to other high-end models like LLama and Deepseek. 

% deepseek being a reasoning model, was hard to inference because of it including the think part in the responses it generated which consumed all the space and did not produce a good result. So we moved from a reasoning model to a chat model that made the output easier.

% \subsection{Response selection}
% Once the discourse is generated, it is important to validate it to complete the discourse generation process. The following steps are taken as a measure for discourse validation. \textbf{1)} A human observation of 10-15 discourses is made randomly for each of the categories for the length, content, coherence and quality of the discourse. Personality trait portrayal is not the topmost priority because this is evaluated through other methods further. \textbf{2)} For each course of discourse, we analyse the similarity scores between all the utterances to make sure that the same arguments are not repeated. 

\subsection{Validation} \label{sec:validation}
Validation involves both human assessment and agent-based evaluation. Discourse quality and coherence are checked via:  \textbf{1)} A human observation of 10-15 discourses is made randomly for each of the categories for the length, content, coherence and quality of the discourse. \textbf{2)} For each course of discourse, we analyse the similarity scores between all the utterances to make sure that the same arguments are not repeated. 
% Various studies use LLMs as agents to extract the personality traits from contents \cite{zhu2025investigating,sun2024revealing}. We use various PLMs to analyse the dialogues and generate results to evaluate the results. Since the agents are initially assigned specific personality traits for discourse generation, we use this ground truth to compute various evaluation metrics.
LLMs are used in the literature for personality trait extraction \cite{zhu2025investigating, sun2024revealing}. We employ PLMs to analyse dialogues to infer personality traits and then use pre-assigned personality traits as ground-truth data for evaluation in Section \ref{sec:evaluation}. 




\setlength{\cellspacetoplimit}{12pt}
\setlength{\cellspacebottomlimit}{12pt}
\newcolumntype{C}[1]{>{\centering\arraybackslash}m{#1}}
\newcolumntype{L}[1]{>{\RaggedRight\arraybackslash}m{#1}}

% Custom colors
\definecolor{headerblue}{RGB}{25,113,194}
\definecolor{rowgray}{RGB}{240,240,240}

% \begin{document}

\begin{table*}[h!]
    \centering
    % \caption{Model Comparison Across Different Evaluation Scenarios}
    % \label{tab:comparison}
    \renewcommand{\arraystretch}{1.1}
    \footnotesize  % Smaller font size for compact appearance
    \begin{tabular}{@{}L{.2cm}C{0.3\textwidth}C{0.3\textwidth}C{0.3\textwidth}@{}}
        \toprule
        % \rowcolor{headerblue!15}
        \textbf{\color{headerblue}Judge} & 
        \textbf{\color{headerblue}GPT-4o vs GPT-4o-mini} & 
        \textbf{\color{headerblue}GPT-4o vs LLaMA-3.3-70B-Instruct} & 
        \textbf{\color{headerblue}GPT-4o vs DeepSeek} \\
        \midrule
        
        % \rowcolor{rowgray!30}
       \rotatebox{90}{GPT-4o} & 
        \includegraphics[width=\linewidth]{heatmap_GPT-4o_GPT_vs_GPT.pdf} & 
        \includegraphics[width=\linewidth]{heatmap_GPT-4o_GPT_vs_Llama.pdf} & 
        \includegraphics[width=\linewidth]{heatmap_GPT-4o_GPT_vs_Deepseek.pdf} \\
        \addlinespace[3pt]
        
        \rotatebox{90}{GPT-4o-mini} & 
        \includegraphics[width=\linewidth]{heatmap_GPT-4o-mini_GPT_vs_GPT.pdf} & 
        \includegraphics[width=\linewidth]{heatmap_GPT-4o-mini_GPT_vs_Llama.pdf} & 
        \includegraphics[width=\linewidth]{heatmap_GPT-4o-mini_GPT_vs_Deepseek.pdf} \\
        \addlinespace[3pt]
        
        \rotatebox{90}{LLaMA} & 
        \includegraphics[width=\linewidth]{heatmap_LLaMA_GPT_vs_GPT.pdf} & 
        \includegraphics[width=\linewidth]{heatmap_LLaMA_GPT_vs_Llama.pdf} & 
        \includegraphics[width=\linewidth]{heatmap_LLaMA_GPT_vs_Deepseek.pdf} \\
        \addlinespace[3pt]
        
        \rotatebox{90}{Qwen} & 
        \includegraphics[width=\linewidth]{heatmap_Qwen_GPT_vs_GPT.pdf} & 
        \includegraphics[width=\linewidth]{heatmap_Qwen_GPT_vs_Llama.pdf} & 
        \includegraphics[width=\linewidth]{heatmap_Qwen_GPT_vs_Deepseek.pdf} \\
        \bottomrule
    \end{tabular}
    \vspace{-0.2cm}
     \caption{Calculation of High Trait Classification Accuracy\textbf{(HTA)} and Low Trait Classification Accuracy\textbf{(LTA)} for \textbf{Participants 1 and 2} across all the conversations for all the \textbf{Judge Agents}.}
     \label{tab:comparison}
\end{table*} 
% Once the objectives mentioned above are satisfied, the collected data is further processed for judgment and evaluation.



% \subsection{Evaluations of the responses from agents}
% Average dialogues? average words per dialogues? average length of utterances? number of sentences ? dataset description+ PACC analysis +BLEU (Bilingual Evaluation Understudy)!+ ?

% \mn{In the previous paper, you deduced values for the personality traits. If you threshold them into High and Low categories, it will be interesting to see that despite assigning personality traits to these models for discourse, these models still go towards their original personality that you inferred in the last paper. This would be a serendipity finding!}

\section{Evaluation} \label{sec:evaluation}
%\emph{we are driven by two different kind of evaluations, one is whether the discourse that we present contains the personality traits and another is whether the models predicting the personality traits can do them well.}
Once the discourses are generated, each of the discourses is evaluated by \emph{Judge agents}. The judge agents return data in a \emph{json} format with their prediction of each speaker's personality traits in the text. To reduce the bias of human vs agent-generated content, we provide the utterances to the Judge agents specifying that they are `human-generated'. The following evaluations are made:  


% \mn{this first one is agreement between judges? If so, you can write "...across Judge Agents". Please clarify}
% We evaluate the results of the discourse from multiple perspectives. The following evaluations are made. Since the use of LLM agents for the evaluation of contents is still questionable, we use several metrics to validate our research:


    \subsection{Personality prediction consistency Across Models:}

    Personality prediction consistency Across Models:
    With access to both the assigned traits (Section \ref{sec:prompt}) and inferred traits (Section \ref{sec:validation}) using different judge agents, we begin by calculating the accuracy of the models' predictions (a.k.a. inferred traits). We calculate the accuracy of prediction in two different ways: the accuracy of predicted \emph{High} for each trait as High Trait Classification Accuracy(HTA) and finally accuracy of predicted \emph{Low} for each trait as Low Trait Classification Accuracy(LTA). Recall, that we assign a high or a low value for each \emph{OCEAN} trait while assigning personalities in Section \ref{sec:prompt}. We create a confusion matrix for this labelling all the True and False predictions of High and Low values to compute the HTA and LTA values. 

    HTA measures how well the models classify traits assigned as High originally. This is computed by creating a confusion matrix for correct and incorrect classifications. HTA is calculated by dividing the total correctly classified High by the total number of High cases. 

    LTA on the other hand measures how well the models classify traits assigned as Low originally. It is calculated by dividing the total correctly classified Low by the total number of Low cases. An important aspect of this study is understanding potential bias in classification into High or Low traits. While overall accuracy may be high, we focus on whether both categories are proportionately represented.
    %%end of 1 st eval the next part will go to appendix if needed?!
    %%%
%     Initially, we calculate the personality prediction consistency for various models. For each speaker in the discourse, we compare the predicted traits against the assigned traits to generate the results. While stable personality expression is expected from LLMs, context shifting can be a contributing factor to drifting in personality. A frequent shift in personality trait expression would mean inconsistency in personality representation in a more complex scenario like dialogue generation. 
%     Three different analyses are made to test the consistency: First explores the exact match of prediction of all five personality traits that are assigned(meaning all five matches) given as Exact Match Accuracy(EMA):
% \begin{equation}
%     \scriptsize{EMA = \frac{\sum_{i=1}^{N} \mathbb{1} ( P_i = A_i )}{N} \times 100}
% \end{equation}
% where:
% \begin{itemize}
%     \item $N$ is the total discourse instances,$P_i$ is the predicted traits for instance $i$,$A_i$ is the assigned traits for instance $i$,$\mathbb{1} ( P_i = A_i )$ is 1 if all traits match, else 0.
% \end{itemize}
%     second explores the consistency in each domain of the five bands of personality traits(partial matches) given as Per Trait Accuracy(PTA):
%     \begin{equation}
%    \scriptsize{ PTA_{t} = \frac{\sum_{i=1}^{N} \mathbb{1} ( P_{i,t} = A_{i,t} )}{N} \times 100}
%     \end{equation}
%     where:
%     \begin{itemize}
%     \item $t$ represents a specific trait, $N$ is the total discourse instances, $P_{i,t}$ and $A_{i,t}$ are the predicted and assigned values for trait $t$, $\mathbb{1} ( P_{i,t} = A_{i,t} )$ is 1 if they match, else 0.
%     \end{itemize}    
%     , and third explores the consistency of scores provided to the speakers given as Speaker Level Consistency(SLC): 
%     \begin{equation}
%     \scriptsize{SLC = \frac{\sum_{i=1}^{N} \sum_{t=1}^{T} \mathbb{1} ( P_{i,t}^{(1)} = P_{i,t}^{(2)} )}{N \times T} \times 100}
%     \end{equation}
%     where:
%     \begin{itemize}
%     \item $N$ is the total discourse instances, $T$ is the total personality traits, $P_{i,t}^{(1)}$ and $P_{i,t}^{(2)}$ are the predicted values for Speaker One and Speaker Two, $\mathbb{1} ( P_{i,t}^{(1)} = P_{i,t}^{(2)} )$ is 1 if both match, else 0.
% \end{itemize}    
    % \item \textbf{Personality Stability in Multi-Turn Conversation:} One of our main concern and the purpose of this experiments is to understand if the personality traits assigned to the LLM agents are maintained through out the conversation. If the personality traits are constant across multiple turns, it denotes that LLMs can infact maintain personality traits when there is a shift of content and when they counter the other arguments. 
    % For this we evaluate the personality trait for each utterance and compare it with the next utterance that is produced to check if the traits are consistent. Inonsistent traits are scored as 1 and consistent traits are scored as 0. The final results for Personality Stability Score(PSS)is given as: 
    % \begin{equation}
    % \scriptsize{PSS = 1 - \frac{\sum_{i=1}^{N} \sum_{t=1}^{T} \sum_{n=2}^{M} I ( P_{i,t}^{(n)} \neq P_{i,t}^{(n-1)} )}{N \times T \times (M - 1)}}
    % \end{equation}
    % where:
    % \begin{itemize}
    % \item $N$ is the total discourse instances,$T$ is the number of personality traits,$M$ is the number of turns in the discourse,$P_{i,t}^{(n)}$ is the predicted trait $t$ at turn $n$ for instance $i$,$I ( P_{i,t}^{(n)} \neq P_{i,t}^{(n-1)} )$ is 1 if the trait changes, else 0.
    % \end{itemize}
    
    \subsection{Inter-rater reliability among the models:} Inter-rater reliability is the measure to understand the agreement between the models. Kappa statistics($\kappa$) is a common method to assess the consistency of ratings among raters (Judge LLMs) \cite{perez2020systematic}. 
     % Among different Kappa statistics, Cohens Kappa\cite{cohen1960coefficient} and Fleiss' Kappa\cite{fleiss1971measuring} are commonly used in cases of categorical values. Since we analyse our results in between 4 judge models Fleiss' Kappa is the best fit as it can assess multiple raters. 
    
    We computed Fleiss' Kappa by first gathering personality trait predictions from five different judge models. Each model analysed debates across multiple topics and rated Big Five personality traits for two participants (P1 \& P2). We structured the data so that all model ratings for the same Topic-Trait pair were aligned, ensuring consistency in comparison. After validation, we reformatted the dataset into a matrix where each row represented a topic-trait combination. The matrix contained counts of how many models classified the trait as \emph{High} or \emph{Low} for both P1 and P2 separately. We calculated the inter-model agreement for each trait using Python’s `statsmodels'\footnote{\href{https://www.statsmodels.org/stable/index.html}{statsmodels}} package, specifically the fleiss\_kappa function to extract the consistency of various judge models across all topics.  

    While the first measure explores the accuracy with which the models correctly identify \emph{High} and \emph{Low}, respective to the ground values, this method explores the agreement between the models for a particular trait at a time, irrespective of the base values. 
    % \textbf{Step 1: Compute Proportion of Ratings}
    % \begin{equation}
    % \scriptsize P_{i,h} &= \frac{n_{i,h}}{N}, \quad P_{i,l} = \frac{n_{i,l}}{N}
    % \end{equation}

    % \textbf{Step 2: Compute Agreement per Topic-Trait}
    % \begin{equation}
    % \scriptsize P_i = P_{i,h}^2 + P_{i,l}^2
    % \end{equation}
    
    % \textbf{Step 3: Compute Mean Agreement}
    % \begin{equation}
    % \scriptsize\bar{P} = \frac{1}{T} \sum_{i=1}^{T} P_i
    % \end{equation}
    
    % \textbf{Step 4: Compute Expected Agreement}
    % \begin{equation}
    % \scriptsize\bar{P}_e = P_{h}^2 + P_{l}^2
    % \end{equation}
    % \begin{align}
    % \scriptsize P_h &= \frac{1}{T} \sum_{i=1}^{T} P_{i,h}, \quad
    % \scriptsize P_l = \frac{1}{T} \sum_{i=1}^{T} P_{i,l}
    % \end{align}
    
    % \textbf{Step 5: Compute Fleiss' Kappa}
    % \begin{equation}
    % \scriptsize \kappa_F = \frac{\bar{P} - \bar{P}_e}{1 - \bar{P}_e}
    % \end{equation}
    
        
    % Inter model agreement is the degree of agreeableness between various models on the personalty trait prediction. This measure addresses if different models from different family agree on the personality traits derived from the discourses. Personality predictions may either converge or diverge depending on the models prediction when multiple models are used. We calculate the Inter Model Agreement for each trait across all the agents as: 
    % \begin{equation}
    % \scriptsize{IMA_t = \frac{\sum_{i=1}^{N} \sum_{j=1}^{M} \sum_{k=j+1}^{M} I ( P_{i,t}^{(j)} = P_{i,t}^{(k)} )}{N \times T \times \binom{M}_{2}} \times 100}
    % \end{equation}
    % where:
    % \begin{itemize}
    % \item $N$ is the total discourse instances,$T$ is the total personality traits, $M$ is the total number of models,$P_{i,t}^{(j)}$ is the predicted value for trait $t$ by model $j$, $I ( P_{i,t}^{(j)} = P_{i,t}^{(k)} )$ is 1 if both models $j$ and $k$ agree, else 0,$\binom{M}_{2}$ represents the number of unique model pairs.
    % \end{itemize}

    \subsection{Discourse alignment with Assigned Personality Traits:} The discourse alignment with assigned personality traits is an important part of this analysis as it depicts if the personality traits are reflected in the contents generated by the agents. We analyse if the discourses linguistically align with the assigned personality traits. Various factors like language, tone and argument structures contribute towards the alignment of personality traits with the content produced \cite{pennebaker1999linguistic}. 
    Linguistic Inquiry and Word Count (LIWC-22)\cite{boyd2022development} analysis is a widely used tool for this category that classifies words into psychological and linguistic categories. \cite{ireland2014natural} explain how natural language and linguistic markers can effectively serve as an indicator of personality traits. For instance, extroverts tend to use more positive words and social process words to reflect their sociable nature. 
    % Similarly, a neurotic behaviour would use more words denoting negative emotions such as anxiety, anger or sadness.
    Linguistic markers are successfully able to understand and predict the personality traits in given text \cite{mairesse2007usinglinguistcmarker}. We use the capabilities of LIWC-22 to extract the linguistic features and systematically map the five personality traits from the data to analyse the results. 

  
    % Lexical diversity is a key term that determines if the words used during the discourse align with human personality traits. LIWC analysis identifies the frequencies of words that correlate with personality traits that provide the alignment metrics. Once the LIWC scores are presented, we calculate the alignment these scores compared the ground truth. This Linguistic Alignment Scores(LAS) is given as:

    % \begin{equation}
    % \scriptsize LAS = \frac{\sum_{i=1}^{N} \sum_{t=1}^{T} I ( L_{i,t} = A_{i,t} )}{N \times T} \times 100
    % \end{equation}
    % where:
    % \begin{itemize}
    % \item $N$ is the total discourse instances,$T$ is the number of personality traits, $L_{i,t}$ is the linguistic marker presence for trait $t$ in instance $i$,$A_{i,t}$ is the assigned personality trait for trait $t$,$I ( L_{i,t} = A_{i,t} )$ is 1 if the linguistic marker matches the expected trait, else 0.
% \end{itemize}


% \end{document}
%     \item \textbf{Bais Analysis}:The personality traits must be reflected by the LLM agents as intended. Furthermore, the judge agents must evaluate not only the correctness of trait classification but also the directional shifts in predictions. Net Trait Shift (NTS) analysis investigates whether these agents systematically push trait classifications higher or lower than their assigned values. Additionally, it would be insightful to determine if an agent tends to shift certain traits more consistently in one direction than others. This kind of analysis is lacking in the literature, and we believe it is an important characteristic to observe.  

%     We initially calculate the average directional shift between the assigned and predicted personality traits. Since our data contain categorical values for the level of personality traits assigned, we map them numerically and compute the average difference between assigned and predicted traits. The following method is used to calculate the overall trait shift tendency :
    
%   \begin{equation}
%     \scriptsize\text{NTS}_t = \frac{\sum_{i=1}^{N} (P_{i,t} - A_{i,t})}{N}
% \end{equation}

% where:
% \begin{itemize}
%     \item \( N \) is the total discourse instances.
%     \item \( P_{i,t} \) is the predicted trait value (\(+1\) for High, \(-1\) for Low).
%     \item \( A_{i,t} \) is the assigned trait value (\(+1\) for High, \(-1\) for Low).
% \end{itemize}
% \end{enumerate}

% A positive \( \text{NTS}_t \) indicates over-prediction of High traits, while a negative \( \text{NTS}_t \) suggests under-prediction.

% Analysis of fluency, coherence and plausibility\cite{sun2024revealing}. Larger dataset,

\section{Results}
% \mn{at some point address the RQs. IF you like you can modify the RQs based on the results, but refer to each one of them in bold, either here on in conclusions.}

The experiments are carried out in two phases: \textbf{1).}
Agents are personified and discourse is generated on a given topic; \textbf{2).} Personality traits are extracted from the discourses and evaluation is performed. This evaluation is critical for determining the controllability of personality traits in language models and validating their alignment with intended psychological characteristics.

% First is the \emph{generation of discourse} using multiple agents constrained with certain personality traits and discussions within a topic. Second, is the \emph{evaluation of personality adherence} by the models while producing utterances for the discourse. This phase is crucial for assessing the controllability of personality expression in language models and validating whether the generated dialogues align with the intended psychological characteristics.
% \mn{we need to ponder on this a bit more. In the second phase the judge agents are "determining" the personality, but it is not really to understand the validity of the discourse. It is mainly to check whether the models followed the assigned personality.. think of something that you can sell better.}

Four models are involved in the creation of discourse in different combinations (GPT-4o vs. GPT-4o-mini, GPT-4o vs. Llama-3.3-70B-Instruct, GPT-4o vs. Deepseek-llm-67B-Chat). All of these models have been set up at higher temperatures ($>$0.8) to allow creativity during discourse generation. Limited by resources(NVIDIA A6000 GPU), the larger models such as  Llama-3.3-70B-Instruct and Deepseek-llm-67B-Chat, were quantized to generate discourse. The max\_tokens were limited to 150 to prevent the model from generating verbose utterances.

For the evaluations of the generated discourse, we used {five} different models: GPT-4o, GPT-4o-mini, Llama-3.3-70B-Instruct, Qwen-2.5-14B-Instruct-1M, and Deepseek-llm-67B-Chat --- \emph{the judge agents}. The idea is to include a variety of models(both small and large) and understand the consistency in the results.  

Utterances from LLaMA-3.3-70B-Instruct and DeepSeek-LLM-67B-Chat required filtration due to prompt repetition and inline tags whereas GPT models adhered to instructions effectively. 
\begin{table}[h]
    \centering
    \small % Reduce font size
    \setlength{\tabcolsep}{4pt} % Reduce column spacing
    \resizebox{0.5\textwidth}{!}{ % Ensure it fits in half-column width
    \begin{tabular}{lcccccc}
        \toprule
        \multirow{2}{*}{\textbf{Trait}} & \multicolumn{2}{c}{\textbf{Discourse 1}} & \multicolumn{2}{c}{\textbf{Discourse 2}} & \multicolumn{2}{c}{\textbf{Discourse 3}} \\
        \cmidrule(lr){2-3} \cmidrule(lr){4-5} \cmidrule(lr){6-7}
        & \textbf{P1} & \textbf{P2} & \textbf{P1} & \textbf{P2} & \textbf{P1} & \textbf{P2} \\
        \midrule
        Agr   & 0.500 & 0.557 & 0.242 & 0.692 & 0.518 & 0.532 \\
        Ope   & 0.699 & 0.420 & 0.534 & 0.631 & 0.250 & 0.430 \\
        Con   & 0.352 & 0.366 & 0.502 & 0.421 & 0.330 & 0.367 \\
        Ext   & 0.123 & 0.097 & 0.235 & 0.105 & 0.287 & 0.260 \\
        Neu   & 0.480 & 0.293 & 0.233 & 0.463 & 0.351 & 0.389 \\
        \bottomrule
    \end{tabular}
    }
        \caption{Fleiss' Kappa Scores for Personality Trait Agreement. \emph{Discourse 1 :} \textbf{GPT-4o vs. GPT-4o-mini}, \emph{Discourse 2}: \textbf{GPT-4o vs. Llama-3.3-70B-Instruct} and \emph{Discourse 3}: \textbf{GPT-4o vs. Deepseek-llm-67b-chat}. P1 and P2: Participants 1 and 2 respectively. }
    \label{tab:fleiss_kappa}
\end{table}
% \begin{enumerate}
    % \item
    
\subsection{Personality Prediction Consistency across models} Figures in Table \ref{tab:comparison} represent the result of personality prediction for each of the Judge models. We now describe various interesting patterns observed with different models as Judges. 
   
   \textbf{Analysis across judge models}:
    We note that for Agreeableness, Openness, and Conscientiousness, GPT-4o, GPT-4o-mini, and LLaMA-3.3-70B-Instruct achieve comparable and high-quality results for both Person 1 and Person 2, exceeding 90\% accuracy. However, for the same categories of traits, Qwen-2.5-14B-1M produces significantly low numbers for Openness and Conscientiousness while the scores for Agreeableness are comparable. From the perspective of the size of the models, larger models (GPT-4o and Llama-3.3-70B-Instruct) have higher accuracy in predicting the \emph{High} classification compared to smaller models (Qwen-2.5-14B-Instruct-1M). However, the accuracy of predicting the \emph{Low} trait was significantly high for Openness and Conscientious with Qwen-2.5 as a Judge as compared to other models for both persons 1 and 2. Overall, for Agreeableness, Openness and Conscientiousness the ability of the (GPT-4o, GPT-4o-mini and Llama-3.3)models to predict their High values is significantly higher than predicting the Low values. 


\begin{figure*}[t]
    \centering
    \subfloat[GPT-4o vs. GPT-4o-mini\label{fig:first}]{        \includegraphics[width=0.3\textwidth]{trait_accuracy_bar_chart_gptvgpt.pdf}}
    \hfill % Adds flexible horizontal space
    \subfloat[GPT-4o vs. LLaMA-3.3\label{fig:second}]{
        \includegraphics[width=0.3\textwidth]{trait_accuracy_bar_chart_gptvllama.pdf} }
    \hfill % Adds flexible horizontal space
    \subfloat[GPT-4o vs. DeepSeek\label{fig:third}]{
        \includegraphics[width=0.3\textwidth]{trait_accuracy_bar_chart_gptvdeepseek.pdf}}
    \caption{LIWC analysis depicting the accuracy of conveying the assigned personality traits to Participants 1 and 2.}
    \label{fig:three_figures}
\end{figure*}


        Judgments for Neuroticism and Extraversion show a distinct pattern, with High values predicted less frequently across all discourses and participants. When observed with scrutiny, detecting High Neuroticism is particularly challenging, likely due to judge models failing to recognise it in text or conversational models avoiding highly neurotic responses. However, some divergent cases occur where GPT-4o detects neuroticism with 62\% precision, significantly higher than in other models. Also, it is worth noticing that detecting High Neuroticism in discourse between the GPT-4o vs. Deepseek is more challenging than the other two combinations. 
        
        We used DeepSeek\footnote{\url{https://huggingface.co/deepseek-ai}} as a judge for pairwise conversation analysis. While LLaMA-3.3 and Qwen-2.5 required refinement, DeepSeek proved unreliable, with over 40\% invalid responses, leading to its exclusion from Table \ref{tab:comparison}.
           

        % In general, GPT-4o as judge effectively classifies the \emph{High} trait prediction across Agreeableness, Openness and Conscientiousness for all three discourses, but performs mostly poor for the \emph{Low} value prediction in similar category. However, the trend reverses in the case of Extraversion and Neuroticism where the \emph{Low} values in these categories are predicted more accurately than the \emph{High} values. There is a significant variation in accuracy across different traits, suggesting that models \textbf{struggle with certain personality traits more than others}. 
       
        % On an individual analysis, the accuracy of predicting Conscientiousness across all the conversations are high and consistent for both person 1 and 2 inferring this trait to be classified most credibly among all the traits. On the other hand, the struggle to effectively calculate higher values of Extraversion and Neuroticism is consistent among all the conversations for GPT-4o but it struggles the most in GPT-4o vs. Deepseek. 
        \textbf{Analysis Across Conversations}:  Compared to GPT-4o vs. GPT-4o-mini and GPT-4o vs Llama-3.3, the accuracy of High trait prediction in Neuroticism and Extraversion was significantly lower for GPT-4o vs. Deepseek conversation for both participants 1 and 2. This suggests that while exploration of low Neuroticism and Extraversion is comparable to the other two conversations, the complexity increases when these domains are High in the GPT-4o vs. Deepseek conversations. While observing individual participants across all the conversations, the results tend to be constant among the judges meaning if GPT rates high Agreeableness to participant 1 in one conversation, other judge models are likely to present similar results. 

%         noindent\textbf{RQ1:} How accurately can "judge" LLMs predict assigned traits from dialogue?
% \noindent\textbf{RQ2:} How consistently do LLM agents express assigned personality traits in conversation?

% \noindent\textbf{RQ3:} Are all OCEAN traits equally prominent in generated conversations?

% \noindent\textbf{RQ4:} Does agent role (creator vs. judge) affect personality-related performance?

    This addresses \textbf{RQ1 \& RQ3}. We observed a conditional capability of these agents as judges to accurately classify  traits from the discourses. This is true within various traits and also for the High and Low classification of the traits. Also, this finding provides an impression of inconsistency and bias towards certain OCEAN traits more than others. 
    % \item 
    \subsection{Inter Model Agreement}

Table \ref{tab:fleiss_kappa} presents the Fleiss' Kappa statistics, measuring inter-model agreement on personality trait judgments for Participants 1 and 2 across all dialogues.

In Discourse 1, Agreeableness showed moderate agreement ($\kappa$ > 0.5) for both participants. Openness agreement was substantial for Participant 1 but moderate for Participant 2. Conscientiousness and Neuroticism exhibited fair to moderate agreement.  Notably, Extraversion showed the lowest agreement, indicating poor reliability in its assessment.

Discourse 2 revealed minimal Agreeableness agreement for Participant 1 but substantially higher agreement for Participant 2, highlighting fluctuations in judging this trait. Openness maintained moderate to substantial agreement. Conscientiousness and Extraversion agreement increased compared to Discourse 1, though Extraversion remained low overall.  Neuroticism agreement showed a reversed trend, with lower agreement for Participant 1 and higher for Participant 2.

In Discourse 3, Agreeableness agreement remained moderate. Openness agreement decreased drastically. Conscientiousness, Extraversion, and Neuroticism agreement was stable between participants but only slight to fair.

These results address \textbf{RQ2}, demonstrating inconsistent inter-model agreement on personality traits.  Agreeableness and Openness agreement fluctuated across dialogues. The consistently low Extraversion agreement indicates significant challenges in its reliable assessment.  This variability underscores the non-uniformity of personality alignment in LLMs, highlighting difficulties in achieving stable and interpretable personality-driven interactions.



    
    % We calculate the Fleiss' Kappa statistics presented in Table \ref{tab:fleiss_kappa} across all the personality traits for both participants 1 and 2. 
    % For Discourse 1, Agreeableness for both P1 and P2 are $>$0.5 which according to the Fleiss' measures\cite{fleiss1971measuring} reports moderate alignment between the models. For Openness, the value of P1 is substantially high while it is moderate for P2. Furthermore, Conscientiousness and Neuroticism present fair to moderate inter-model agreement for both P1 and P2. However, the values for Extraversion are the least for both P1 and P2 meaning the judgement of Extraversion between the models is the least reliable among all. 
    
    % For Discourse 2, the inter-model agreement for Agreeableness for P1 is only minimal whereas for P2 is substantially high compared to other values. This difference shows the fluctuating values of judge models in predicting Agreeableness for P1. Similar to the previous discourse, consistency is maintained in Openness, where it demonstrates moderate to substantial agreement. Furthermore, the values of Conscientiousness and Extraversion are higher than in previous conversations. However, for Extraversion, the values are still significantly lower overall. For Neuroicism, the trend has reversed where P1 has lower inter-model scores as P2 demonstrates higher values compared to previous readings. 
    
    % For Discourse 3, the intermodal agreement for Agreeableness is moderate, continuing the previous trend, whereas values for Openness drastically reduced compared to previous records. For Conscientiousness, Extraversion, and Neuroticism, the readings are stable between P1 and P2, but the inter-modal agreement is only slight to fair.
  
    % Overall, we answer \textbf{RQ2} with a conclusion that LLMs exhibit inconsistent personality trait judgments, with some traits such as Agreeableness and Openness showing fluctuating agreement levels across discourses. The persistent low reliability in Extraversion indicates that models struggle to assess this trait consistently. These variations imply that personality alignment in LLMs is not uniform, highlighting potential challenges in ensuring stable and interpretable personality-driven interactions.

\subsection{Discourse Alignment with assigned personality traits}

Figure \ref{fig:three_figures} presents the accuracy of personality trait depiction for Participants 1 and 2, measured using LIWC-22.  GPT-4o-mini achieved the highest accuracy for Agreeableness across all dialogues.  However, GPT-4o's Agreeableness accuracy decreased substantially (from ~68\% and ~65\% to ~52\%) when conversing with Deepseek than GPT-4o-mini and Llama-3.3, suggesting a potential shift in personality expression depending on the interlocutor, similar to human behaviour \cite{atherton2022stability}.

Openness was the trait least accurately represented in all dialogues, with a maximum accuracy of ~51\%. This suggests that expressing Openness is particularly challenging for these LLMs.  Llama-3.3 exhibited the highest Conscientiousness, while GPT-4o showed the highest Extraversion. However, these differences were not statistically significant, and trait expression varied depending on the conversational partner.  GPT-4o's Neuroticism depiction was most accurate when interacting with Llama-3.3. This variability in traits and conversational settings directly addresses \textbf{RQ3}, confirming that all OCEAN traits are not equally prominent in generated conversations.

When comparing pairwise dialogues, GPT-4o vs. GPT-4o-mini and GPT-4o vs. Llama-3.3 showed similar performance.  However, GPT-4o vs. Deepseek dialogues exhibited significantly different results.  We observed that Deepseek struggled to consistently follow instructions from the prompts (even though the prompts were minimally adapted across models). Deepseek's generated text was also the most inconsistent in length compared to other models, which may have contributed to the observed differences.




% Figure \ref{fig:three_figures} shows the accuracy of personality trait depiction in discourses for Participants 1 and 2 using LIWC-22. GPT-4o-mini achieves the highest accuracy for Agreeableness across all discourses. It is important to note that, while GPT-4o produced fair results ($\approx$
% 68\% \& $\approx$65\%) for Agreeableness when it was participating with GPT-4o-mini and Llama-3.3, the numbers drastically reduced when it was conversing with Deepseek ($\approx$52\%) which suggests the possibility of a shift in personality with the change of opponent as per humans~\cite{atherton2022stability}.

% Openness in all the discourses for both Participants 1 and 2 is reported most poorly among the BFI traits with the highest accuracy ranging to $\approx$ 51\%. This suggests the ability of adopting the Openness traits by the LLMs when directly instructed to do so, is the lowest. Furthermore, the Conscientiousness of the Llama-3.3 model is the highest in the discourses whereas GPT-4o wins in Extraversion. However, the differences are not significant and the values fluctuate according to the debate partners. Finally, GPT-4o follows the assigned Neuroticism most precisely while conversing with Llama-3.3, resulting in the highest accuracy. This inconsistency addresses and concludes our \textbf{RQ3} that all OCEAN traits are not prominent across all the generated conversations. 

% When pairwise discourse situations are analysed, the results for GPT-4o vs GPT-4o-mini and GPT-4o vs Llama-3.3 are comparable, however, the numbers significantly defer for GPT-4o vs Deepseek conversations. Several reasons may contribute to these factors, we present some of our observations. Initially, compared to GPT and LLama models, it is challenging for Deepseek to instructions from the prompts that are given (although the same prompts with minimal changes were given to all models for a fair comparison). In addition, the results generated by Deepseek were the most inconsistent among others, on observation(in length). 
% \end{enumerate}

\section{Conclusion}
% Our study highlights the varying degrees of personality adherence and judgment consistency across different LLMs in personality-constrained discourse. While Agreeableness, Openness, and Conscientiousness were predicted with high accuracy across most models, Neuroticism and Extraversion proved more challenging, with frequent under-prediction of High values. The inter-model agreement, assessed using Fleiss’ Kappa, revealed inconsistencies in personality trait judgments, with lower reliability in Extraversion across all discourses. Models varied in their alignment with assigned traits, with higher accuracy in Agreeableness and Conscientiousness, while Openness remained the most challenging. The evaluation also showed a noticeable decline in personality alignment in GPT-4o vs. DeepSeek conversations, suggesting that conversational dynamics influence personality consistency. 

% Overall, our findings indicate that while LLMs can be guided toward personality-driven dialogue, their ability to maintain personality traits varies significantly depending on the model and discourse setting. These inconsistencies underscore the challenges in achieving stable and interpretable personality-aligned interactions in LLMs.



% \un{check if below is good} \mn{I think this is good}

This paper provides a comprehensive evaluation of trait adherence in LLM agents engaged in dyadic conversations.  Our findings highlight the significant challenges in achieving consistent and interpretable personality-aligned interactions. While LLMs can be guided to exhibit certain personality traits, their ability to maintain these traits across dynamic conversations varies considerably.  Future work should explore more sophisticated methods for instilling and evaluating personality, investigating the impact of dialogue context and developing metrics for assessing the nuances of personality expression in LLMs.  Exploring fine-tuning strategies or reinforcement learning approaches for improving consistency would also be valuable.


\section{Limitations}
One of the key challenges in this study is the absence of a standardized benchmarking system that all evaluations adhere to, making direct comparisons across different approaches more difficult. While strict rules were enforced to structure the discourse, models did not always fully comply, occasionally deviating from expected dialogue patterns. Additionally, there is a risk of bias, as language models may incorporate their own implicit judgments into discussions, potentially influencing personality assessments. Another important consideration is the length of dyadic conversations, there is no widely accepted standard for how long a dialogue should be to ensure a reliable evaluation. This uncertainty raises questions about whether longer or shorter exchanges might yield different insights, adding a layer of complexity to the interpretation of results.

% \un{pls chk... do not give v explicit limitations. They will pick those and add as a weakness of ur study...}

% \textcolor{red}{pls chk... do not give v explicit limitations. They will pick those and add as a weakness of ur study...}


\section{Ethical Considerations}
% \un{add ethical consideration paragraph (talk abt LLMs, bias and stuff related to ur data and work)}

We do not collect any personal information and views for the creation of the discourse dataset or refer to any kind of personal traits from any sources to judge the nature of conversations. All the discourses are created by LLM agents. Topics provided for discussion for the agents are debatable but do not involve or promote the thought of violence, hatred or extremism of any kind to anyone. 

We use open and closed-source models that are available off the self and accessible to the general public. No changes in the model architecture have been made. Some hyperparameters have been adjusted to meet our expectations of the results, but they have been mentioned clearly in the paper. LLMs have the possibility of introducing bias in their results as per numerous studies. The dataset generated by the conversing agents has not been made public, but we do plan to publish it for further studies with careful ethical consideration and approvals. The results do present bias in predicting the BFI from the discourses but are solely limited to LLMs as judges. 

The content of LLM agents is subject to change if they are altered, fine-tuned, and tempered in different ways, which is a potential risk. 

% \textcolor{red}{add ethical consideration paragraph (talk abt LLMs, bias and stuff related to ur data and work)}

\bibliography{custom}
\newpage
\appendix

\label{sec:appendix}

\section{Sample of Topics and Trait Combinations Used}
Samples of \emph{topics} used for debate:
\begin{lstlisting}
"Is the concept of a universal language beneficial?",
"Should the government regulate the pharmaceutical industry?",
"Is the use of nuclear energy justified?",
"Should the government provide free public transportation?",
"Is the concept of a cashless society beneficial?",
"Should the government regulate the gaming industry?"
\end{lstlisting}

\emph{Trait} combinations samples to assign personas to Agents: 

\begin{lstlisting}
{"Agreeableness": "High", "Openness": "Low", "Conscientiousness": "High", "Extraversion": "Low", "Neuroticism": "High"},
{"Agreeableness": "Low", "Openness": "High", "Conscientiousness": "Low", "Extraversion": "High", "Neuroticism": "Low"},
{"Agreeableness": "High", "Openness": "High", "Conscientiousness": "Low", "Extraversion": "High", "Neuroticism": "High"},
{"Agreeableness": "Low", "Openness": "Low", "Conscientiousness": "High", "Extraversion": "Low", "Neuroticism": "Low"},
{"Agreeableness": "High", "Openness": "High", "Conscientiousness": "High", "Extraversion": "Low", "Neuroticism": "Low"}
\end{lstlisting}

\section{System and User prompts}

We use, different \emph{System and User} prompts to extract the discourses and ratings from the conversing and judge agents. 

\subsection{Discourse Generation }
The \emph{system prompt} to generate the discourses:

     
\begin{lstlisting}
SYSTEM_PROMPT = ''' f"You are participating in a structured debate on: '{topic}'\n"
"Your responses should reflect these personality traits:\n"
f"- Agreeableness: {traits['Agreeableness']}\n"
f"- Openness: {traits['Openness']}\n"
f"- Conscientiousness: {traits['Conscientiousness']}\n"
f"- Extraversion: {traits['Extraversion']}\n"
f"- Neuroticism: {traits['Neuroticism']}\n\n"
"Rules:\n"
"- Maintain these personality traits (DO NOT EXPLICITLY MENTION IN TEXT) at all
times during your conversation\n"
"- Keep responses under 50 words\n"
"- Maintain your personality consistently\n"
"- Address previous arguments directly but do not repeat what
the other speaker said.\n"
"- End with proper punctuation" ''''
\end{lstlisting} 

The \emph{user prompt} carries the previous argument :
\begin{lstlisting}
USER_PROMPT = """Previous Argument:f"{previous_arguement}" """
\end{lstlisting}


\subsection{Extracting Personalities from the Judge Agents.}
The \emph{system prompt} to extract the personality traits:

\begin{lstlisting}
SYSTEM_PROMPT = """Analyze text segments from two anonymous debaters (Person One and Person Two) for:
1. Big Five Inventory (BFI) traits (High/Low for each dimension)
2. Consistency with typical behavior for those traits (Yes/No)

For each person, return:
{
    "predicted_bfi": {
        "Agreeableness": "High/Low",
        "Openness": "High/Low",
        "Conscientiousness": "High/Low",
        "Extraversion": "High/Low",
        "Neuroticism": "High/Low"
    }
}
"""
\end{lstlisting}
The \emph{user prompt} is: 

\begin{lstlisting}
USER_PROMPT= '''f"Analyze{persona}'s text:\n{text}'''    
\end{lstlisting}
where the \emph{persona} contains Participant 1 and 2  and the \emph{text} contains the discourses for each of the participants respectively. 

\newpage
\section{Metadata of the Discourses.}
% For professional tables


% Table 1: GPT-4o vs GPT-4o-mini
\begin{table}[h]
    \centering
    \begin{tabular}{lc}
        \toprule
        \textbf{Metric} & \textbf{GPT-4o vs GPT-4o-mini} \\
        \midrule
        % Total Utterances & 16,160 \\
        Total Sentences & 70,750 \\
        Total Words & 781,330 \\
        Assertions & 14,653 \\
        Questions & 1,507 \\
        Logical Structures & 690 \\
        Total Dialogues & 2,020 \\
        Avg. Words per Sentence & 11.04 \\
        Avg. Utterance Length & 48.35 \\
        \bottomrule
    \end{tabular}
    \caption{Metadata analysis for GPT-4o vs GPT-4o-mini}
    \label{tab:gpt4o_vs_gpt4o_mini}
\end{table}

% Table 2: LLaMA-3 vs GPT-4
\begin{table}[h]
    \centering
    \begin{tabular}{lc}
        \toprule
        \textbf{Metric} & \textbf{LLaMA-3 vs GPT-4o} \\
        \midrule
        % Total Utterances & 18,180 \\
        Total Sentences & 44,964 \\
        Total Words & 541,603 \\
        Assertions & 15,577 \\
        Questions & 2,603 \\
        Logical Structures & 767 \\
        Total Dialogues & 2,020 \\
        Avg. Words per Sentence & 12.05 \\
        Avg. Utterance Length & 29.79 \\
        \bottomrule
    \end{tabular}
    \caption{Metadata analysis for LLaMA-3 vs GPT-40}
    \label{tab:llama3_vs_gpt4}
\end{table}

% Table 3: DeepSeek vs LLaMA
\begin{table}[h]
    \centering
    \begin{tabular}{lc}
        \toprule
        \textbf{Metric} & \textbf{DeepSeek vs GPT-4o} \\
        \midrule
        % Total Utterances & 18,180 \\
        Total Sentences & 44,387 \\
        Total Words & 1,033,592 \\
        Assertions & 17,800 \\
        Questions & 380 \\
        Logical Structures & 4,697 \\
        Total Dialogues & 2,020 \\
        Avg. Words per Sentence & 23.29 \\
        Avg. Utterance Length & 56.85 \\
        \bottomrule
    \end{tabular}
    \caption{Metadata analysis for DeepSeek vs GPT-4o}
    \label{tab:deepseek_vs_llama}
\end{table}






% \section{Information about the Generated Discourses.}
% \section{Calculation of Fleiss Kappa}
% \section{System and User Prompt}
% % \begin{table*}[t!]
%     \centering
%     \small
%     \setlength{\tabcolsep}{3pt}
%     \caption{Comparison of Personality Consistency Metrics Across Judges and Discourses}
%     \resizebox{0.9\textwidth}{!}{
%     \begin{tabular}{@{}llcccccccccccccc@{}}
%         \toprule
%         \multirow{3}{*}{\textbf{Discourse}} & \multirow{3}{*}{\textbf{Judge}} & \multicolumn{2}{c}{\textbf{EMA (\%)}} & \multicolumn{5}{c}{\textbf{PTA (\%)}} & \multicolumn{5}{c}{\textbf{SLC (\%)}} \\
%         \cmidrule(l){3-4} \cmidrule(l){5-9} \cmidrule(l){10-14}
%         & & P1 & P2 & Agr & Ope & Con & Ext & Neu & Agr & Ope & Con & Ext & Neu \\
%         \midrule
%         \multirow{5}{*}{GPT-4o vs Mini} & GPT-4o & 13.4 & 4.31 & P1:79.9 & P1:63.16 & P1:66.03 & P1:76.56 & P1:66.03 & P1:75.12 & P1:85.17 & P1:93.3 & P1:79.43 & P1:79.9 \\
%          &  &  &  & P2:79.9 & P2:63.16 & P2:66.03 & P2:76.56 & P2:66.03 & P2:75.12 & P2:85.17 & P2:93.3 & P2:79.43 & P2:79.9\\\cmidrule(l){2-14}
%         & LLaMA3.2 & 13.4 & 4.31 & P1:79.9 & P1:63.16 & P1:66.03 & P1:76.56 & P1:66.03 & P1:75.12 & P1:85.17 & P1:93.3 & P1:79.43 & P1:79.9 \\
%          &  &  &  & P2:79.9 & P2:63.16 & P2:66.03 & P2:76.56 & P2:66.03 & P2:75.12 & P2:85.17 & P2:93.3 & P2:79.43 & P2:79.9\\\cmidrule(l){2-14}
%         & LLaMA3.3 & 13.4 & 4.31 & P1:79.9 & P1:63.16 & P1:66.03 & P1:76.56 & P1:66.03 & P1:75.12 & P1:85.17 & P1:93.3 & P1:79.43 & P1:79.9 \\
%          &  &  &  & P2:79.9 & P2:63.16 & P2:66.03 & P2:76.56 & P2:66.03 & P2:75.12 & P2:85.17 & P2:93.3 & P2:79.43 & P2:79.9\\\cmidrule(l){2-14}
%         & Qwen & 13.4 & 4.31 & P1:79.9 & P1:63.16 & P1:66.03 & P1:76.56 & P1:66.03 & P1:75.12 & P1:85.17 & P1:93.3 & P1:79.43 & P1:79.9 \\
%          &  &  &  & P2:79.9 & P2:63.16 & P2:66.03 & P2:76.56 & P2:66.03 & P2:75.12 & P2:85.17 & P2:93.3 & P2:79.43 & P2:79.9\\ \cmidrule(l){2-14}
%         & DeepSeek& 13.4 & 4.31 & P1:79.9 & P1:63.16 & P1:66.03 & P1:76.56 & P1:66.03 & P1:75.12 & P1:85.17 & P1:93.3 & P1:79.43 & P1:79.9 \\
%          &  &  &  & P2:79.9 & P2:63.16 & P2:66.03 & P2:76.56 & P2:66.03 & P2:75.12 & P2:85.17 & P2:93.3 & P2:79.43 & P2:79.9\\ 
%         \midrule
%         \multirow{5}{*}{GPT-4o vs LLaMA} & GPT-4o& 13.4 & 4.31 & P1:79.9 & P1:63.16 & P1:66.03 & P1:76.56 & P1:66.03 & P1:75.12 & P1:85.17 & P1:93.3 & P1:79.43 & P1:79.9 \\ 
%          &  &  &  & P2:79.9 & P2:63.16 & P2:66.03 & P2:76.56 & P2:66.03 & P2:75.12 & P2:85.17 & P2:93.3 & P2:79.43 & P2:79.9\\ \cmidrule(l){2-14}
%         & LLaMA3.2 & 13.4 & 4.31 & P1:79.9 & P1:63.16 & P1:66.03 & P1:76.56 & P1:66.03 & P1:75.12 & P1:85.17 & P1:93.3 & P1:79.43 & P1:79.9 \\
%          &  &  &  & P2:79.9 & P2:63.16 & P2:66.03 & P2:76.56 & P2:66.03 & P2:75.12 & P2:85.17 & P2:93.3 & P2:79.43 & P2:79.9\\ \cmidrule(l){2-14}
%         & LLaMA3.3& 13.4 & 4.31 & P1:79.9 & P1:63.16 & P1:66.03 & P1:76.56 & P1:66.03 & P1:75.12 & P1:85.17 & P1:93.3 & P1:79.43 & P1:79.9 \\
%          &  &  &  & P2:79.9 & P2:63.16 & P2:66.03 & P2:76.56 & P2:66.03 & P2:75.12 & P2:85.17 & P2:93.3 & P2:79.43 & P2:79.9\\ \cmidrule(l){2-14}
%         & Qwen & 13.4 & 4.31 & P1:79.9 & P1:63.16 & P1:66.03 & P1:76.56 & P1:66.03 & P1:75.12 & P1:85.17 & P1:93.3 & P1:79.43 & P1:79.9 \\
%          &  &  &  & P2:79.9 & P2:63.16 & P2:66.03 & P2:76.56 & P2:66.03 & P2:75.12 & P2:85.17 & P2:93.3 & P2:79.43 & P2:79.9\\ \cmidrule(l){2-14}
%         & DeepSeek & 13.4 & 4.31 & P1:79.9 & P1:63.16 & P1:66.03 & P1:76.56 & P1:66.03 & P1:75.12 & P1:85.17 & P1:93.3 & P1:79.43 & P1:79.9 \\
%          &  &  &  & P2:79.9 & P2:63.16 & P2:66.03 & P2:76.56 & P2:66.03 & P2:75.12 & P2:85.17 & P2:93.3 & P2:79.43 & P2:79.9\\
%         \midrule
%         \multirow{5}{*}{GPT-4o vs DeepSeek} & GPT-4o& 13.4 & 4.31 & P1:79.9 & P1:63.16 & P1:66.03 & P1:76.56 & P1:66.03 & P1:75.12 & P1:85.17 & P1:93.3 & P1:79.43 & P1:79.9 \\
%          &  &  &  & P2:79.9 & P2:63.16 & P2:66.03 & P2:76.56 & P2:66.03 & P2:75.12 & P2:85.17 & P2:93.3 & P2:79.43 & P2:79.9\\ \cmidrule(l){2-14}
%         & LLaMA3.2 & 13.4 & 4.31 & P1:79.9 & P1:63.16 & P1:66.03 & P1:76.56 & P1:66.03 & P1:75.12 & P1:85.17 & P1:93.3 & P1:79.43 & P1:79.9 \\
%          &  &  &  & P2:79.9 & P2:63.16 & P2:66.03 & P2:76.56 & P2:66.03 & P2:75.12 & P2:85.17 & P2:93.3 & P2:79.43 & P2:79.9\\ \cmidrule(l){2-14}
%         & LLaMA3.3& 13.4 & 4.31 & P1:79.9 & P1:63.16 & P1:66.03 & P1:76.56 & P1:66.03 & P1:75.12 & P1:85.17 & P1:93.3 & P1:79.43 & P1:79.9 \\
%          &  &  &  & P2:79.9 & P2:63.16 & P2:66.03 & P2:76.56 & P2:66.03 & P2:75.12 & P2:85.17 & P2:93.3 & P2:79.43 & P2:79.9\\ \cmidrule(l){2-14}
%         & Qwen& 13.4 & 4.31 & P1:79.9 & P1:63.16 & P1:66.03 & P1:76.56 & P1:66.03 & P1:75.12 & P1:85.17 & P1:93.3 & P1:79.43 & P1:79.9 \\
%          &  &  &  & P2:79.9 & P2:63.16 & P2:66.03 & P2:76.56 & P2:66.03 & P2:75.12 & P2:85.17 & P2:93.3 & P2:79.43 & P2:79.9\\ \cmidrule(l){2-14}
%         & DeepSeek& 13.4 & 4.31 & P1:79.9 & P1:63.16 & P1:66.03 & P1:76.56 & P1:66.03 & P1:75.12 & P1:85.17 & P1:93.3 & P1:79.43 & P1:79.9 \\
%          &  &  &  & P2:79.9 & P2:63.16 & P2:66.03 & P2:76.56 & P2:66.03 & P2:75.12 & P2:85.17 & P2:93.3 & P2:79.43 & P2:79.9\\
%         \bottomrule
%     \end{tabular}}
%     \label{tab:personality-metrics}
% \end{table*}
\end{document}
