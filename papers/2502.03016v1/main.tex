\documentclass[smallextended]{svjour3}
\listfiles
\makeatletter % https://tex.stackexchange.com/a/499541
\let\cl@chapter\undefined
\makeatletter
% Language setting
% Replace `english' with e.g. `spanish' to change the document language
\usepackage[english]{babel}

% Set page size and margins
% Replace `letterpaper' with `a4paper' for UK/EU standard size
\usepackage[letterpaper,top=2cm,bottom=2cm,left=3cm,right=3cm,marginparwidth=1.75cm]{geometry}

% Useful packages
\usepackage{natbib}
\usepackage{xparse}
\usepackage{amsmath}
\usepackage{amssymb}
\usepackage{mathtools}
\usepackage{graphicx}
\usepackage{subcaption}
\usepackage[colorlinks=true, allcolors=blue]{hyperref}
%\usepackage{authblk}
\usepackage{pgfplots}
\usepackage{import}
\usepackage{color}
\usepackage{xcolor}
\usepackage{csquotes}
\usepackage{multirow}
\usepackage{todonotes}
\usepackage{varioref}
\usepackage[capitalise,noabbrev]{cleveref}
\usepackage[title]{appendix}
\usepackage{float}
\usepackage{booktabs}

\pgfplotsset{compat=1.16}

\title{An analysis of optimization problems involving ReLU neural networks}
\author{Christoph Plate \and 
            Mirko Hahn  \and
            Alexander Klimek \and
            Caroline Ganzer \and
            Kai Sundmacher \and
            Sebastian Sager}
\institute{Christoph Plate, Mirko Hahn, Kai Sundmacher, Sebastian Sager \at
         Otto von Guericke University Magdeburg, Magdeburg, Germany \\
         \email{\{christoph.plate, mirhahn, kai.sundmacher, sager\}@ovgu.de}
         \and
         Christoph Plate, Alexander Klimek, Caroline Ganzer, Kai Sundmacher, Sebastian Sager \at
         Max Planck Institute for Dynamics of Complex Technical Systems, Magdeburg, Germany \\
         \email{\{plate, klimek, cganzer, sundmacher, sager\}@mpi-magdeburg.mpg.de}
         \and
         Corresponding author: Christoph Plate
}

\date{Received: date / Accepted: date}
\newcommand{\myReLU}[1]{\textrm{ReLU}\left(#1\right)}
\newcommand{\myClippedReLU}[2][M]{\textrm{ReLU}_{#1}\left(#2\right)}

% Prohibit line breaking in inline math
\binoppenalty=10000
\relpenalty=10000

% Mark overfull boxes with a black bar
\overfullrule=1ex

% ORCID
% Alex: 
% Caroline: 0000-0002-7081-4523
% Sebastian: 0000-0002-0283-9075
% Sundmacher: 0000-0003-3251-0593
% Christoph: 0000-0003-0354-8904
% Mirko: 0000-0002-2442-3978

\begin{document}
\maketitle

\begin{abstract}
Solving mixed-integer optimization problems with embedded neural networks with ReLU activation functions is challenging.
Big-M coefficients that arise in relaxing binary decisions related to these functions grow exponentially with the number of layers. 
We survey and propose different approaches to analyze and improve the run time behavior of mixed-integer programming solvers in this context.
Among them are clipped variants and regularization techniques applied during training as well as optimization-based bound tightening and a novel scaling for given ReLU networks. %which yields a functionally equivalent neural network with lower big-M coefficients. 
We numerically compare these approaches for three benchmark problems from the literature.
We use the number of linear regions, the percentage of stable neurons, and overall computational effort as indicators. 
As a major takeaway we observe and quantify a trade-off between the often desired redundancy of neural network models versus the computational costs for solving related optimization problems.
\keywords{optimization \and machine learning \and neural network \and integer programming}
% \PACS{PACS code1 \and PACS code2 \and more}
%\subclass{MSC code1 \and MSC code2 \and more}
\end{abstract}


\section{Introduction}

Tutoring has long been recognized as one of the most effective methods for enhancing human learning outcomes and addressing educational disparities~\citep{hill2005effects}. 
By providing personalized guidance to students, intelligent tutoring systems (ITS) have proven to be nearly as effective as human tutors in fostering deep understanding and skill acquisition, with research showing comparable learning gains~\citep{vanlehn2011relative,rus2013recent}.
More recently, the advancement of large language models (LLMs) has offered unprecedented opportunities to replicate these benefits in tutoring agents~\citep{dan2023educhat,jin2024teach,chen2024empowering}, unlocking the enormous potential to solve knowledge-intensive tasks such as answering complex questions or clarifying concepts.


\begin{figure}[t!]
\centering
\includegraphics[width=1.0\linewidth]{Figs/Fig.intro.pdf}
\caption{An illustration of coding tutoring, where a tutor aims to proactively guide students toward completing a target coding task while adapting to students' varying levels of background knowledge. \vspace{-5pt}}
\label{fig:example}
\end{figure}

\begin{figure}[t!]
\centering
\includegraphics[width=1.0\linewidth]{Figs/Fig.scaling.pdf}
\caption{\textsc{Traver} with the trained verifier shows inference-time scaling for coding tutoring (detailed in \S\ref{sec:scaling_analysis}). \textbf{Left}: Performance vs. sampled candidate utterances per turn. \textbf{Right}: Performance vs. total tokens consumed per tutoring session. \vspace{-15pt}}
\label{fig:scale}
\end{figure}


Previous research has extensively explored tutoring in educational fields, including language learning~\cite{swartz2012intelligent,stasaski-etal-2020-cima}, math reasoning~\cite{demszky-hill-2023-ncte,macina-etal-2023-mathdial}, and scientific concept education~\cite{yuan-etal-2024-boosting,yang2024leveraging}. 
Most aim to enhance students' understanding of target knowledge by employing pedagogical strategies such as recommending exercises~\cite{deng2023towards} or selecting teaching examples~\cite{ross-andreas-2024-toward}. 
However, these approaches fall short in broader situations requiring both understanding and practical application of specific pieces of knowledge to solve real-world, goal-driven problems. 
Such scenarios demand tutors to proactively guide people toward completing targeted tasks (e.g., coding).
Furthermore, the tutoring outcomes are challenging to assess since targeted tasks can often be completed by open-ended solutions.



To bridge this gap, we introduce \textbf{coding tutoring}, a promising yet underexplored task for LLM agents.
As illustrated in Figure~\ref{fig:example}, the tutor is provided with a target coding task and task-specific knowledge (e.g., cross-file dependencies and reference solutions), while the student is given only the coding task. The tutor does not know the student's prior knowledge about the task.
Coding tutoring requires the tutor to proactively guide the student toward completing the target task through dialogue.
This is inherently a goal-oriented process where tutors guide students using task-specific knowledge to achieve predefined objectives. 
Effective tutoring requires personalization, as tutors must adapt their guidance and communication style to students with varying levels of prior knowledge. 


Developing effective tutoring agents is challenging because off-the-shelf LLMs lack grounding to task-specific knowledge and interaction context.
Specifically, tutoring requires \textit{epistemic grounding}~\citep{tsai2016concept}, where domain expertise and assessment can vary significantly, and \textit{communicative grounding}~\citep{chai2018language}, necessary for proactively adapting communications to students' current knowledge.
To address these challenges, we propose the \textbf{Tra}ce-and-\textbf{Ver}ify (\textbf{\model}) agent workflow for building effective LLM-powered coding tutors. 
Leveraging knowledge tracing (KT)~\citep{corbett1994knowledge,scarlatos2024exploring}, \model explicitly estimates a student's knowledge state at each turn, which drives the tutor agents to adapt their language to fill the gaps in task-specific knowledge during utterance generation. 
Drawing inspiration from value-guided search mechanisms~\citep{lightman2023let,wang2024math,zhang2024rest}, \model incorporates a turn-by-turn reward model as a verifier to rank candidate utterances. 
By sampling more candidate tutor utterances during inference (see Figure~\ref{fig:scale}), \model ensures the selection of optimal utterances that prioritize goal-driven guidance and advance the tutoring progression effectively. 
Furthermore, we present \textbf{Di}alogue for \textbf{C}oding \textbf{T}utoring (\textbf{\eval}), an automatic protocol designed to assess the performance of tutoring agents. 
\eval employs code generation tests and simulated students with varying levels of programming expertise for evaluation. While human evaluation remains the gold standard for assessing tutoring agents, its reliance on time-intensive and costly processes often hinders rapid iteration during development. 
By leveraging simulated students, \eval serves as an efficient and scalable proxy, enabling reproducible assessments and accelerated agent improvement prior to final human validation. 



Through extensive experiments, we show that agents developed by \model consistently demonstrate higher success rates in guiding students to complete target coding tasks compared to baseline methods. We present detailed ablation studies, human evaluations, and an inference time scaling analysis, highlighting the transferability and scalability of our tutoring agent workflow.


\section{Methods}\label{sec:methods}

In this section we discuss several methods that have an impact on the overall performance of a MINLP solver such as \texttt{Gurobi}, when applied to optimization problems of type \eqref{prob:embedded}.
%
We start by introducing two measures of complexity in this context in Section~\ref{subsec:measures}, namely the number of regions partitioning the input domain in which the function $h(x)$ has identical linear output behavior, and the number of stable ReLU neurons.

Then we examine methods that are applicable to trained ANNs. In Section~\ref{subsec:boundtightening} bound tightening approaches for the optimization problem are presented. In Section~\ref{sec:scaling} we propose a novel scaling method that improves the $\ell^1$ regularization term of a pre-trained network without changing its encoded function. This method can be used after completed training of the ANN (a posteriori) and before the optimization is started (a priori).

In Sections~\ref{sec:regularization}, \ref{sec:clipped}, and \ref{sec:dropout} we investigate modifications to the training of the ANN, in particular regularization of training weights, clipped ReLU formulations, and the use of dropout during training.


%The goal of this paper is to examine and expand upon some of the acceleration methods enumerated in the previous section. 
%As the big-M formulation \eqref{eq:bigM} is used in the majority of publications investigating optimization problems with embedded ReLU neural networks, we restrict ourselves to this formulation. 
%Regarding the methods proposed for accelerating optimization algorithms, we particularly focus on the effects of bound tightening and training with regularization. We also propose a simple scaling method that improves the $\ell^1$ regularization term of a pre-trained network without changing its encoded function. Apart from these acceleration methods, we also consider dropout as it is a popular method applied during training. Before analyzing these methods in more detail, we introduce two important characteristics of ReLU networks which are an indicator of the complexity of optimization problems with embedded neural networks. 

\subsection{Measures of complexity of ReLU ANNs}\label{subsec:measures}

While the solution of mixed-integer optimization problems is difficult ($\mathcal{NP}$-complete) in general, it is well known that the number of optimization variables and the tightness of relaxations of the integer variables have a major impact on computational runtimes. In the context of embedded ANNs, we shall consider two particular indicators of complexity.
%From the point of view of optimization, several aspects complicate optimizing over neural networks. These include large number of variables and weak relaxations, among others. In this section, we introduce properties of neural networks which are indicators of the complexity, which we use later in our numerical study.

\subsubsection{Number of linear regions of ReLU networks}

ReLU ANN describe piecewise affine-linear functions \citep{Grigsby2022}. Therefore, the network partitions the input domain $\mathcal{X} \subseteq \mathbb{R}^{n_x}$ into regions in which $h(x)$ is affine linear. These regions are typically called \textit{linear regions}. The bounds on the number of linear regions of a neural network with given depth and width was investigated in \citet{Montufar2014} and later improved on in \citet{Raghu2017}. In general, the number of linear regions of a neural network corresponds to the number of feasible activation patterns in \eqref{eq:bigM}, i.e., the binary decisions whether a neuron is on or off for all neurons in the neural network. Thus, it is an important statistic when considering the complexity of optimizing over neural networks, e.g., in branch-and-bound frameworks, where the variables to branch on represent active or inactive neurons. 

\subsubsection{Number of stable ReLU neurons}

The number of variables a branch-and-bound method has to branch on is an important statistic for estimating the complexity of the optimization problem. In ReLU networks, the variables to branch on are the binary variables $z$ in \eqref{eq:bigM} of every neuron in the network. However, if a neuron can be identified as stable, no binary variable has to be added to model the neuron. To identify stable neurons, their pre-activation bounds are used. The neuron $i$ in layer $j$ is called stably active if $L^{(j)}_i > 0$ and stably inactive if $U^{(j)}_i < 0$, for $j \in [J]$ and $i \in [n_j]$ for all inputs in the input domain $\mathcal{X} \subseteq \mathbb{R}^{n_x}$. 

A regularization to induce ReLU stability was proposed in \citet{Xiao2019} to speed up verification of ReLU networks, whereas in \citet{Serra2020} stable neurons are used to compress neural networks. 
To enumerate the linear regions of a ReLU ANN, we exploit the fact that, within a given linear region, the input and output of each neuron is an affine linear functional in the ANN's overall input space. We use forward sensitivity propagation to calculate the gradient of each neuron's regional input functional and simultaneously perform a forward evaluation of the linearized ANN at the input space's coordinate origin to determine each affine input functional's output shift. With both gradient and shift, we can determine a hyperplane in input space along which the neuron's ReLU activation would switch. We then construct a linear equation system that describes the intersection of halfspaces within which all neurons would retain their current activation pattern. We add the bounds of the input domain to this equation system to ensure boundedness of the linear region. We then use a variant of the \texttt{QuickHull} algorithm~\citep{Barber1996} via the \texttt{SciPy} library~\citep{SciPy2020} to reduce this equation system into an irredundant one and to determine the vertices of the linear region. This also reveals information on which neurons define the facets of the linear region, which means that we can jump across these facets to adjacent regions by switching the activity of those neuron's activation functions. Assuming that there is no facet along which two neurons switch simultaneously, this allows us to enumerate all linear regions that intersect the input domain. We can detect the edge case of two neurons switching simultaneously because it would cause us to enter a region with an empty interior. We do not observe this behavior. 

\subsection{bound tightening} \label{subsec:boundtightening}

Calibrating the big-M coefficients in MILP formulations is crucial for performance of optimization algorithms. bound tightening plays an important role in this context. With ReLU ANNs, there are different ways to compute the big-M coefficients.
%
\subsubsection{Interval arithmetic}\label{subsec:ia_bounds}
%
In the presence of input bounds $L^{(0)} \leq x^{(0)} \leq U^{(0)}$ with $L^{(0)}, U^{(0)} \in \mathbb{R}^{n_x}$, big-M coefficients of formulation \eqref{eq:bigM} can be computed via interval arithmetic (IA).
%
\begin{align}
    L^{(k)}_i &= \sum_{j=1}^{n_{k-1}} \min \{ W_{i,j}^{(k)} L_j^{(k-1)}, W_{i,j}^{(k)} U_j^{(k-1)} \} + b^{(k)}_i, \quad k \in [J],  i \in [n_k] \\
    U^{(k)}_i &= \sum_{j=1}^{n_{k-1}} \max \{ W_{i,j}^{(k)} L_j^{(k-1)}, W_{i,j}^{(k)} U_j^{(k-1)} \} + b^{(k)}_i, \quad k \in [J],  i \in [n_k] 
\end{align}
%
This forward propagation yields valid bounds. However, it ignores the fact that the activation of neurons, i.e., whether they are on the left or right arm of the ReLU function, is not independent between neurons. This results in overly relaxed approximations of the actual bounds. As a result, there is typically an exponential increase of big-M coefficients with increasing depth. This behaviour is exemplified in \vref{fig:IA_bounds}.


\subsubsection{LP-based bound tightening}\label{subsec:lr_bounds}

The bounds from \cref{subsec:ia_bounds} can be tightened by taking advantage of dependencies between the neurons as well as potentially existing bounds on the output of the neural network $L^{(J)} \leq x^{(J)} \leq U^{(J)}$. This is achieved by solving two auxiliary optimization problems per neuron, minimizing and maximizing, respectively, the pre-activation value of each neuron. The optimization problem for computing tighter bounds for neuron $k$ in layer $j$, with $j \in [J],\,k \in [n_k]$ in its general form as an MILP reads

\begin{equation}\label{prob:obbt}
    \begin{aligned}
    \min_{x,z}\ & W^{(j)}_k x^{(j-1)} + b^{(j)}_k \\ 
    \textrm{s.t.}\ & \begin{alignedat}[t]{3}
            x^{(j)}_i & \geq 0, \quad && j \in [J],\, i \in [n_j] \\
            x^{(j)}_i & \geq W^{(j)}_i x^{(j-1)} + b^{(j)}_i, \quad && j \in [J],\, i \in [n_j] \\
            x^{(j)}_i & \leq W^{(j)}_i x^{(j-1)} + b^{(j)}_i - L^{(j)}_i (1-z^{(j)}_i), \quad && j \in [J],\, i \in [n_j] \\
            x^{(j)}_i & \leq U^{(j)}_i z^{(j)}_i, \quad && j \in [J],\, i \in [n_j] \\
            x^{(0)}_i  & \leq U^{(0)}_i,        \quad && i \in [n_x]\\
            x^{(0)}_i  & \geq L^{(0)}_i,        \quad && i \in [n_x]\\
            z^{(j)}_i & \in \{0,1\}. \quad && j \in [J],\, i \in [n_j]
        \end{alignedat}
    \end{aligned}
\end{equation}
Solving \eqref{prob:obbt} yields a valid lower bound $L^{(j)}_k$, while the corresponding upper bound $U^{(j)}_k$ is computed by maximizing instead of minimizing in \eqref{prob:obbt}. In order to reduce the computational effort, typically the LP relaxation of formulation \eqref{prob:obbt} is considered. Hence, the auxiliary problems are linear programs (LPs) and can be solved efficiently. Solving the MILP directly is considered in \citet{Badilla2023,Grimstad2019}. However, the reduction in computational effort in subsequent optimization is quickly outweighed by the effort spent on solving the bound tightening MILPs. Therefore, we only consider the LP-based bound tightening procedure in this paper. One degree of freedom when performing bound tightening is the ordering of variables for which bounds are tightened. As the direction of bound propagation is from the input to the output layer, this is also the natural order to perform the tightening. However, within each layer the order may be chosen arbitrarily. Different methods to choose this order are discussed in \citet{Rossig2021}. However, they do not find any advantage of more advanced methods over a simple, fixed ordering of variables. Therefore, in this contribution, we apply bound tightening in a fixed ordering of variables.% The effects of LP-based bound tightening on a ReLU network with ten hidden layers is illustrated in \vref{fig:big_Ms}.

\subsection{A posteriori scaling of ReLU ANNs} \label{sec:scaling}
Weights of neural networks are not uniquely determined by the training process and the training data, i.e., there are different realizations of weights and biases that define the same functional relationship of input and output. This observation can be exploited to design algorithms that transform a trained neural network into a functionally identical network with some desired property. This could be, e.g., a lower norm of the weight matrices. With the input bounds remaining unchanged, this would lead to a reduction of big-M coefficients, which could be beneficial in subsequent optimization problems.

In case of the ReLU activation function, one can exploit its positive homogeneity. For a single neuron $i$ in layer $k$, with $k \in [J], \ i \in [n_k]$ and a scalar $c^{(k)}_i > 0$ it holds, that
\begin{align}
    \myReLU{c^{(k)}_i \left( W^{(k)}_i x^{(k-1)} + b_i \right)} = c^{(k)}_i \cdot \myReLU{ W^{(k)}_i x^{(k-1)} + b_i},
\end{align}
with $W^{(k)}_i \in \mathbb{R}^{1 \times n_{k-1}}$ being the $i$-th row of the weight matrix in layer $k$. 
%
To ensure the functional equivalence of the neural network, the $i$-th column of the weight matrix of layer $k+1$, corresponding to the scaled neuron $i$ in layer $k$, needs to be multiplied with the reciprocal of $c^{(k)}_i$. As all neurons of the neural network may be scaled, all weight matrices except the first and the last are scaled with the ratio of the two scaling factors of their surrounding layers. As the bias is not multiplied with the output from the previous layer, no multiplication with the reciprocal is needed. In the final layer $J$, no more new scaling factors may be introduced as they can no longer be compensated in subsequent layers. Therefore, only the scaling of layer $J-1$ is compensated by multiplying $W^{(J)}$ with the reciprocals of the scaling factors of the penultimate layer. For any set of scaling factors $c^{(k)}_i > 0,\,k \in [J], i \in [n_k]$, scaled weights and biases  $\tilde{W}$ and $\tilde b$, computed as 
\begin{equation}
    \begin{alignedat}{3}
        \tilde{W}_{i,j}^{(1)} &= W_{i,j}^{(1)} \cdot c_i^{(1)}, \quad && i \in [n_1],\, j \in [n_x] \\       
        \tilde{W}_{i,j}^{(k)} &= W_{i,j}^{(k)} \cdot \frac{c_i^{(k)}}{c_j^{(k-1)}}, \quad && k \in \{2, \ldots, J-1\},\, i \in [n_k],\, j \in [n_{k-1}],\\
        \tilde{W}_{i,j}^{(J)} &= W_{i,j}^{(J)} \cdot \frac{1}{c_j^{(J-1)}}, \quad && i \in [n_J],\, j \in [n_{J-1}] \\
        \tilde{b}_i^{(k)} &= b_i^{(k)} \cdot c_i^{(k)}, \quad && k \in [J-1], \, i \in [n_k]
    \end{alignedat}
\end{equation}
define functionally equivalent neural networks.
This basic idea of an equivalent transformation of ReLU networks via scaling one layer and compensating the effect of scaling in the next layer is illustrated in \vref{fig:rescale}. 

\begin{figure}
    \centering
    \includegraphics[width=.6\linewidth]{figures/hahnRescale.png}
    \caption{Equivalent scaling of ReLU ANNs. Scalar factor $c$ is multiplied row-wise to weight matrix and corresponding bias of current layer, resulting in a scaling of the output of the neuron by a factor of $c$. To compensate this, the weight matrix in the subsequent layer needs to be multiplied column-wise with the reciprocal of $c$.}
    \label{fig:rescale}
\end{figure}
%
The scaling factors $c_i^{(k)}$ can be chosen arbitrarily. However, we can specifically choose them such that the resulting network has favorable properties. We propose formulating an optimization problem to obtain scaling factors that minimize the absolute value of the scaled weights $\tilde{W}$ and biases $\tilde{b}$. This lower norm of the weights then yields lower big-M coefficients. As noted in \cref{subsec:ia_bounds}, these are determined solely\footnote{While bounds on the output of the network can be propagated backwards through the network and thus influence the big-M coefficients \citep{Grimstad2019}, we refer only to the big-M coefficients derived via interval arithmetic and forward propagation of input bounds as explained in \cref{subsec:ia_bounds}.} by the input bounds and the magnitude of weights and biases. Hence, this approach can be applied to networks that were not initially trained with regularization in order to generate an equivalent neural network with lower big-M coefficients. Of course, other effects of regularization, e.g., weight sparsity, cannot be obtained by this method. The proposed optimization problem is
%
\begin{equation}
    \begin{aligned}
        \min_{c}\ & \sum_{i=1}^{n_1} \sum_{j=1}^{n_x} \lvert W_{i,j}^{(1)} \rvert \cdot c_i^{(1)} +  \sum_{k=2}^{J-1} \sum_{i=1}^{n_k} \sum_{j=1}^{n_{k-1}}  \lvert W_{i,j}^{(k)} \rvert \cdot \frac{c_i^{(k)}}{c_j^{(k-1)}} \\*
        & + \sum_{k=1}^{J-1} \sum_{i=1}^{n_k} \lvert b_i^{(k)} \rvert \cdot c_i^{(k)} + \sum_{i=1}^{n_J} \sum_{j=1}^{n_{J-1}} \lvert W_{i,j}^{(J)} \rvert \cdot \frac{1}{c_i^{(J)}} \\
        \textrm{s.t.}\ & \begin{alignedat}[t]{3}
            c_i^{(k)} & > 0, \quad && k \in [J],\, i \in [n_k] \\
            c & \in \bigtimes_{k=1}^J \mathbb{R}^{n_k} \quad
        \end{alignedat}
    \end{aligned}
\end{equation}
%
This optimization problem is not trivial to solve directly because it involves fractions and strict inequality constraints. However, because all $c_i^{(k)}$ have to be strictly positive, we can convert it into a convex optimization problem on a closed set by replacing each $c_i^{(k)}$ with its logarithm. Each summand in the objective function then becomes an evaluation of the exponential function, multiplication becomes addition, and division becomes subtraction. With the logarithm of $c^{(k)}_i$  referred to as $\tilde{c}^{(k)}_i$, the transformed optimization problem reads
%
\begin{equation}\label{prob:scaling}
    \begin{aligned}
        \min_{\tilde{c}}\ & \sum_{i=1}^{n_1} \sum_{j=1}^{n_x} \exp \left( \log \left( \lvert W_{i,j}^{(1)} \rvert\right) + \tilde{c}_i^{(1)} \right)   + \sum_{k=2}^{J} \sum_{i=1}^{n_k} \sum_{j=1}^{n_{k-1}} \exp \left( \log \left( \lvert W_{i,j}^{(k)} \rvert \right)+ \tilde{c}_i^{(k)} - \tilde{c}_j^{(k-1)} \right)\\*
        & + \displaystyle \sum_{k=1}^{J} \sum_{i=1}^{n_k} \exp \left( \log \left( \lvert b_i^{(k)} \rvert \right) +  \tilde{c}_i^{(k)} \right) + \sum_{i=1}^{n_J} \sum_{j=1}^{n_{J-1}}  \exp \left( \log\left(\lvert W_{i,j}^{(J)} \rvert \right) - \tilde{c}_i^{(J)} \right) \\
        \textrm{s.t.}\ & \tilde{c} \in \bigtimes_{k=1}^J \mathbb{R}^{n_k}
    \end{aligned}
\end{equation}
%


\subsection{Regularization} \label{sec:regularization}

The objective function for training neural networks typically consists of two terms. The first 
accounts for the mismatch between prediction and data, while the second term aims at preventing overfitting and thus allowing for a better generalization of the model to unseen data. With $W \in \mathbb{R}^d$ denoting the vector of all weights and biases and $N \in  \mathbb{N}$ representing the number of training samples of inputs and outputs $(x_i, y_i), \, i \in [N]$, the objective reads
%
\begin{equation}\label{prob:training}
    \begin{aligned}
        \min_{W}\ & \frac{1}{N} \sum_{i=1}^{N} \left( h(x_i) -y_i\right)^2 + \lambda \Omega(W)
    \end{aligned}
\end{equation}
%
Popular choices for the regularization term $\Omega : \mathbb{R}^d \mapsto \mathbb{R}$ are the penalization of large magnitudes of weights and biases by using some vector norm, e.g., $\Omega(W) = \| W \|_p$, with typically $p=1$ and $p=2$. Typical ways to measure the generalization performance of a model is to compute the mean absolute percentage error (MAPE) defined as 
\begin{align}
    \text{MAPE}\left(\hat{y}, y\right) = \frac{1}{n} \sum_{i=1}^{n} \frac{|\hat{y}_i - y_i|}{\max\{\varepsilon, |y_i|\}}
\end{align}
for predictions $\hat{y}$ on the test dataset.

While it is known that $\ell^1$ regularization leads to sparser regression models \citep{Tibshirani1996}, \citet{Xiao2019,Serra2020} found that applying $\ell^1$ regularization also increased ReLU stability, i.e., the percentage of stable neurons. The authors of \citet{Xiao2019} also propose a dedicated ReLU stability regularization \eqref{eq:stability_regularization}, which penalizes the sign differences in the pre-activation bounds of each neuron, thus encouraging stability.
\begin{align}\label{eq:stability_regularization}
    \Omega_{\text{RS}}(W) = - \sum_{i=1}^{J} \sum_{j=1}^{n_i} \text{sign}(U_j^{(i)}) \cdot \text{sign}(L_j^{(i)})
\end{align}
For practical purposes, a smooth reformulation of \eqref{eq:stability_regularization} is used, and \citet{Xiao2019} show that verification problems of neural networks trained using this regularization can be solved faster than with $\ell^1$ regularization due to a higher number of stable neurons. In this paper, we will however focus on investigating the effect of varying levels of $\ell^1$ regularization on the performance of optimization algorithms, as it is one of the most commonly used types of regularization.

\subsection{Clipped ReLU} \label{sec:clipped}
%
One of the reasons why big-M coefficients in ReLU networks increase quickly with increasing network depth is that the ReLU activation function is unbounded. A variation of the ReLU function is the clipped ReLU function proposed in \citet{Hannun2014}. In the clipped ReLU function, the output of the function is bounded by an upper value $M \in \mathbb{R}$, i.e.,
\begin{align}\label{eq:relu_m}
    \myClippedReLU[M]{x} = \max\bigl\{0,\, \min\{M,\, x\}\bigr\}
\end{align}
%
Using standard disjunctive programming notation, the feasible set of $x_i^{(j)} = \myClippedReLU[M]{ W^{(j)}_i x^{(j-1)} + b_i}$ can be written as
%
\[
\begin{bmatrix}
    x_i^{(j)} = 0 \\
    W^{(j)}_i x^{(j-1)} + b_i \leq 0
\end{bmatrix}
\vee 
\begin{bmatrix}
    x_i^{(j)} = W^{(j)}_i x^{(j-1)} + b_i\\
    0 < W^{(j)}_i x^{(j-1)} + b_i < M 
\end{bmatrix}
\vee
\begin{bmatrix}
    x_i^{(j)} = M \\
    W^{(j)}_i x^{(j-1)} + b_i \geq M 
\end{bmatrix}
\]
%
We formulate a big-M relaxation of this feasible set as
\begin{align}
    \begin{split}
        x_i^{(j)} & \geq 0, \\
        x_i^{(j)} & \leq M z_{1_i}^{(j)}, \\
        x_i^{(j)} & \leq U_i^{(j)} z_{1_i}^{(j)}, \\
        x_i^{(j)} & \leq W^{(j)}_i x^{(j-1)} + b_i - L_i^{(j)} \cdot (1-z_{1_i}^{(j)}),\\
        x_i^{(j)} & \geq M z_{2_i}^{(j)}, \\
        x_i^{(j)} & \geq W^{(j)}_i x^{(j-1)} + b_i - (U_i^{(j)}-M) z_{2_i}^{(j)}, \\
        z_{1_i}^{(j)}, z_{2_i}^{(j)} & \in \{0,1\},\\
        z_{1_i}^{(j)} & \geq z_{2_i}^{(j)},
    \end{split}
    \label{eq:bigM_clipped}
\end{align}
similar to the formulation suggested in a preprint version of \citet{Anderson2020}.
This formulation comes at the cost of an additional binary variable compared to the standard big-M formulation \eqref{eq:bigM}. If both binary variables are zero, the neuron is inactive and $x_i^{(j)}=0$. In the case $z_{1_i}^{(j)}=1, z_{2_i}^{(j)}=0$, the neuron is active and $0 \leq x_i^{(j)} =  W^{(j)}_i x^{(j-1)} + b_i \leq M$. If both binary variables are non-zero, the neuron's output is limited by the threshold $M$. 

\subsection{Dropout} \label{sec:dropout}

Dropout is a technique applied during training proposed in \citet{Srivastava2014} to prevent overfitting the data by randomly turning off a percentage of the neurons in some or all layers. Therefore, redundancies have to be established in the neural network to achieve an adequate accuracy. There is empirical evidence that neural networks trained with dropout have more linear regions \citep{Zhang2020a} than those trained without. Hence, in contrast to the aforementioned methods, it is expected that applying dropout during training leads to more complex neural networks which makes optimizing over them more difficult.
We will thus apply dropout as an antithesis to validate our conjecture that the runtime of MINLP solvers increases for more redundant and decreases for less redundant ANN models.


\section{Numerical results}\label{sec:results}

In the \cref{sec:methods}, we have enumerated some methods to formulate, train, and scale feed-forward neural networks with ReLU activation (or variations thereof), as well as to tighten their relaxation prior to optimization through bound tightening. In this section, we evaluate how these methods affect global optimization performance. In order to do so, we train neural networks as surrogates for several non-convex benchmark functions and compare solver performance with various post-processing steps.

We first present numerical results on relevant characteristics of ReLU ANNs in the context of optimization. These include their expressive power as measured by the number of linear regions they define and the percentage of stable neurons that can be determined from the pre-activation bounds, introduced in the beginning of \cref{sec:methods}.  We count only those linear regions that intersect the relevant input domain of each function.

We show how the methods presented in \cref{sec:methods} impact these quantities and improve the performance of optimization algorithms. For this, we restrict ourselves to minimizing the output of feed-forward ReLU ANNs, i.e., the optimization problem we solve reads
\begin{equation}\label{prob:minANN}
    \min_{x} \ h(x)
\end{equation}
where $h \colon \mathbb{R}^{{n_x}} \mapsto \mathbb{R}$ is the trained neural network. 
The benchmark functions we consider for approximation and subsequent minimization are:

\begin{figure}[t]
    \centering
    \begin{subfigure}{.32\linewidth}
        \includegraphics[width=\linewidth]{figures/surf_peak.png}
        \subcaption{Peaks function \eqref{fun:peaks}}\label{fig:peaks-function}
    \end{subfigure}
    \begin{subfigure}{.32\linewidth}
        \includegraphics[width=\linewidth]{figures/surf_ack.png}
        \subcaption{Ackley's function \eqref{fun:ack}}\label{fig:ack-function}
    \end{subfigure}
    \begin{subfigure}{.32\linewidth}
        \includegraphics[width=\linewidth]{figures/surf_him.png}
        \subcaption{Himmelblau's function \eqref{fun:him}}\label{fig:him-function}
    \end{subfigure}
    \caption{Surface plots of the benchmark functions for surrogate model training and optimization.}
    \label{fig:different_functions}
\end{figure}


\begin{enumerate}
    \item The Peaks function $f_{\text{peaks}} \colon \mathbb{R}^2 \mapsto \mathbb{R}$ is given by
    \begin{align}
        \begin{split}
        f_{\text{peaks}}(x,y) &= - 3 (1-x)^2 \exp\bigl(-x^2 - (y+1)^2\bigr)   - 10 \Bigl( \frac{x}{5} - x^3 - y^5\Bigr) \exp(-x^2 - y^2) \\
            & \quad - \frac{1}{3}  \exp\bigl(-(x+1)^2 - y^2\bigr).
        \end{split}
        \label{fun:peaks}
    \end{align}
    It is commonly used as a benchmark function, e.g., in \citet{Schweidtmann2019a} and has multiple local minima and maxima on the domain $x,y \in [-2,2]$. The global minimum is $(0.228, -1.626)$ with objective value $-6.551$. The function is depicted in \vref{fig:peaks-function}.

    \item Ackley's function $f_{\text{ackley}} \colon \mathbb{R}^2 \mapsto \mathbb{R}$ is defined by
    \begin{align}
        \begin{split}
            f_{\text{ackley}}(x,y) & = -20 \cdot \exp \left(-\frac{1}{5}\sqrt{\frac{1}{2}(x^2+y^2)}\right) \\
                            & \quad  - \exp \left(\frac{1}{2} \bigl(\cos(2\pi x) + \cos(2\pi y)\bigr) \right) + \exp(1) + 20    
        \end{split}
        \label{fun:ack}
    \end{align}
    and is often used as a benchmark function for optimization algorithms. For instance, it is used in \citet{Tsay2021}. It is considered on the domain $x,y \in [-3.5,3.5]$. It is non-convex, has several local minima and one global minimum at $x=y=0$ with objective value $0$. A surface plot is depicted in \vref{fig:ack-function}.

    \item Himmelblau's function $f_{\text{himmelblau}} \colon \mathbb{R}^2 \mapsto \mathbb{R}$ with
    \begin{align}\label{fun:him}
        f_{\text{himmelblau}}(x,y) =  (x^2 + y - 11)^2 + (x + y^2 - 7)^2
    \end{align}
    is considered on the domain $x,y \in [-5,5]$, where it has four equivalent local (and global) minima: $(3.0,2.0)$, $(-2.805, 3.131)$, $(-3.779, -3.283)$ and $(3.584,-1.848)$. All have objective function value $0$. A surface plot is depicted in \vref{fig:him-function}.
\end{enumerate}

In our numerical study, we consider a total of 1080 different neural networks. This number of instances stems from considering the three benchmark functions used for the approximation of \cref{fun:peaks,fun:ack,fun:him} and the specific choices for the hyperparameters of the trained neural networks. These differ both in their width and depth, as well as the activation function and the level of $\ell^1$ regularization applied during training. The specific options for these hyperparameters are given in \Cref{table:hyperparameter}. Using Latin Hypercube sampling, we generated training data of 100,000 samples for the Peaks function \eqref{fun:peaks} and Himmelblau's function \eqref{fun:him}, and 150,000 samples for Ackley's function \eqref{fun:ack}, to account for its higher nonconvexity. For training, we first normalize both input and output data, and reserve 30\% of the data as a test set to evaluate the generalization of the networks. All networks are then trained for 300 epochs using the Adam algorithm \citep{Kingma2017}. To study the effect of scaling and bound tightening on each of the trained networks, we solve problems \eqref{prob:scaling} and \eqref{prob:obbt}, where applicable. As the scaling method is not designed for the clipped ReLU, we can only solve \eqref{prob:scaling} for the 360 instances with standard ReLU activation. We use \texttt{OMLT} \citep{Ceccon2022} to set up the constraints for the ReLU ANNs via \texttt{Pyomo}~\citep{Bynum2021,Hart2011}, and \texttt{Gurobi}~\citep{gurobi}~{v11.0.1} with default options and a time limit of 300 seconds to solve the resulting optimization problems. 
%
\begin{figure}
    \centering
    \begin{subfigure}{0.49\textwidth}
        \includegraphics[width=\textwidth]{figures/big_M_unscaled}
        \caption{Big-M coefficients $U^{(k)}$ determined via IA for standard ReLU ANN.}
        \label{fig:IA_bounds}
    \end{subfigure}
    \hfill
    \begin{subfigure}{0.49\textwidth}
        \includegraphics[width=\textwidth]{figures/big_M_unscaled_OBBT}
        \caption{Big-M coefficients $U^{(k)}$ determined via OBBT for standard ReLU ANN.}
        \label{fig:LR_bounds}
    \end{subfigure}
    
    \begin{subfigure}{0.49\textwidth}
        \includegraphics[width=\textwidth]{figures/big_M_scaled}
        \caption{Big-M coefficients $U^{(k)}$ determined via IA for ReLU ANN after ReLU scaling.}
        \label{fig:scaled_IA_bounds}
    \end{subfigure}
    \hfill
    \begin{subfigure}{0.49\textwidth}
        \includegraphics[width=\textwidth]{figures/big_M_scaled_OBBT.pdf}
        \caption{Big-M coefficients $U^{(k)}$ determined via OBBT for ReLU ANN after ReLU scaling.}
        \label{fig:scaled_LR_bounds}
    \end{subfigure}
            
    \caption{Comparison of pre-activation bounds  $U^{(k)}$ for functionally equivalent neural networks with ten hidden layers. The original bounds derived via interval arithmetic shown in \ref{fig:IA_bounds} are characterized by the typical exponential increase due to forward propagation of the input bounds. Solving auxiliary LPs yields tighter bounds, although the exponential increase is still present, as shown in \ref{fig:LR_bounds}. Comparable bounds can be computed via solving the scaling problem \eqref{prob:scaling}, with the distinction that the bounds on the output of the network are equivalent to those derived from interval arithmetic. For the scaled neural network, solving the bound tightening problem \eqref{prob:obbt} in addition yields even tighter bounds on the big-M coefficients in the hidden layers with ReLU activation, as can be seen in \ref{fig:scaled_LR_bounds}, while the output bounds are equivalent to those in \ref{fig:LR_bounds}.}
    \label{fig:big_Ms}
\end{figure}


\begin{table}[ht]
    \centering
    \caption{Hyperparameter options for training of neural networks. Besides varying the depth and width of the networks, we investigate two variants of the clipped ReLU activation \eqref{eq:relu_m} and five levels of $\ell^1$ regularization. All hidden layers have the same dimension.}
    \label{table:hyperparameter}
    \begin{tabular}{l c } 
        \toprule 
        Hyperparameter & Options \\ \midrule
        Hidden Layers & $1,\ldots,10$  \\
        Layer Width &  $25,50$ \\
        Activation & ReLU, ReLU$_2$, ReLU$_5$ \\ 
        $\lambda$ &  $0.0, 10^{-7}, 10^{-6},10^{-5}, 10^{-4}, 10^{-3}$\\ \bottomrule 
    \end{tabular}
\end{table}

\subsection{Effect of OBBT}

For the 1080 trained neural networks, we solve the LP-relaxation of \eqref{prob:obbt} to compute tighter big-M coefficients for formulation \eqref{eq:bigM}, and use them in the optimization problem \eqref{prob:minANN}. The effect on a network with ten hidden layers is illustrated in \vref{fig:big_Ms}. Compared to the IA bounds, there is a reduction in big-M coefficients of the last layer by roughly two orders of magnitude. As \vref{tab:results} shows, OBBT is effective for all trained networks. We assess the reduction in big-M coefficients across all networks by comparing the averaged distances between upper and lower bound $U^{(j)}_k -L^{(j)}_k$ for bounds based on OBBT and IA. Then, over all networks, we calculate the geometric mean over the ratio of these averages. The resulting geometric mean of 0.54 suggests, that, as a rough estimate, OBBT is reducing the big-M coefficients by half. 
As a side effect of these tighter bounds, the percentage of stable neurons increases by 5.5 percent on average. We assess the resulting improvement in computational times by calculating the ratios of the measured  computational times with tightened bounds and those with the original bounds, restricted to instances that were solved to global optimality in both cases. Over these ratios, we again form the geometric mean. With a geometric mean of 0.57, bound tightening brings a significant computational speedup, though it does not substantially increase the number of instances that are solved within the time limit. \vref{fig:improvement_obbt} illustrates the parities of percentage of stable neurons and computational time.


\subsection{Effect of ReLU scaling}

\Vref{fig:big_Ms} illustrates the effects of solving the scaling problem \eqref{prob:scaling} on the big-M coefficients of a neural network with ten hidden layers. The first observation is that the output bounds remain unchanged compared to the original neural network, which is expected as the functional relationship is equivalent. However, the lower $\ell^1$ norm of the weights leads to a reduction in the big-M coefficients for the hidden layers. They are roughly on the same order of magnitude as those obtained via LP-based bound tightening. When both scaling and bound tightening are applied sequentially, the bounds for the hidden layers are tighter than those achieved by OBBT on its own. Also, with the sequential application of scaling and tightening we do not observe any clear sign of an exponential increase in bounds with increasing depth.

Using the big-M formulation with standard bounds obtained via IA as a baseline, we compare the following options:
\begin{enumerate}
    \item ReLU scaling only: We solve Problem~\eqref{prob:scaling} to obtain equivalent weights and biases with lower $\ell^1$ norm;
    \item ReLU scaling and subsequent LP-based bound tightening: a combination of the two methods.
\end{enumerate}
As shown in \vref{tab:results}, ReLU scaling on its own, as well as combined with OBBT, is able to reduce the big-M coefficients more than applying OBBT on an unscaled network. This is clearly illustrated by the geometric means over the ratios of averaged distances of upper and lower bounds $L$ and $U$ of 0.388 and 0.16 for ReLU scaling and ReLU scaling combined with OBBT compared to unscaled networks, respectively. 
Again, we compute the improvement in computational times as a geometric mean over the ratios of computational times with improved bounds and those with interval arithmetic bounds. We observe that scaling the neural network weights by solving \eqref{prob:scaling} yields only a marginal improvement with a geometric mean of {0.936}. However, combining this scaling with subsequent bound tightening yields a more substantial computational speedup as indicated by a geometric mean of {0.467}. This seems to stem from the tighter big-M coefficients, but also from an increased percentage of stable neurons. Compared to the default bounds, there is an average increase by {7.2} percent. In \vref{fig:improvement_scaling_obbt}, the parities of  computational times for the two comparisons are shown. We note that the parity plot in \vref{subfig:IA_vs_scaler_and_OBBT} suggests that the average speedup may be driven by a few outlier instances in which in the combined method performs exceptionally well.

Overall, with the scaled ReLU networks and their default bounds from interval arithmetic, 307 instances can be solved within the time limit. With tightened bounds, there is a slight reduction to 299 instances.


\begin{figure}
    \centering
    \begin{subfigure}{.47\linewidth}
        \centering
        Percentage of stable neurons
        \includegraphics[width=\linewidth]{figures/parities/obbt/parity_percentage_fixed}
        \subcaption{Parity plot for percentage of stable neurons compared for bounds from IA and LP-based OBBT.}
        \label{subfig:percentage_fixed_IA_vs_OBBT}
    \end{subfigure}
    \begin{subfigure}{.47\linewidth}
        \centering
        Computational time
        \includegraphics[width=\linewidth]{figures/parities/obbt/parity_time}
        \subcaption{Parity plot for computational time compared for bounds from IA and LP-based OBBT.}
        \label{subfig:time_IA_vs_OBBT}
    \end{subfigure}

    \caption{Parity plots comparing percentage of stable neurons and computational times of optimally solved instances of \eqref{prob:minANN} for bounds derived from IA and LP-based OBBT. Solving \eqref{prob:obbt} leads to an increase of 5.5 percentage points in stable neurons on average. This carries over to a reduction in computational time shown in \subref{subfig:time_IA_vs_OBBT}. The ratios of times with OBBT and IA bounds have a geometric mean of 0.57.}
    \label{fig:improvement_obbt}
\end{figure}


\begin{figure}
    \centering
    \begin{subfigure}{.47\linewidth}
        \centering
        Computational time
        \includegraphics[width=\linewidth]{figures/parities/scaled/parity_time}
        \subcaption{Runtime comparison between IA bounds for the baseline network (\enquote{Default}) and IA for the scaled ANN (\enquote{ReLU scaling}).}
        \label{subfig:IA_vs_scaler}
    \end{subfigure}
    \begin{subfigure}{.47\linewidth}
        \centering
        Computational time
        \includegraphics[width=\linewidth]{figures/parities/scaled_obbt/parity_time}
        \subcaption{Runtime comparison between IA bounds for the baseline network (\enquote{Default}) and OBBT for the scaled ANN (\enquote{ReLU scaling + OBBT}).}
        \label{subfig:IA_vs_scaler_and_OBBT}
    \end{subfigure}

    \caption{Parity plots comparing computational times for optimally solved instances of \eqref{prob:minANN} in different versions: \subref{subfig:IA_vs_scaler}: IA bounds for baseline vs. scaled ReLU network with a geometric mean ratio of {0.936}; \subref{subfig:IA_vs_scaler_and_OBBT}: IA bounds for baseline network vs. OBBT bounds for scaled network with a geometric mean ratio of {0.467}.}
    \label{fig:improvement_scaling_obbt}
\end{figure}

\subsection{Effect of regularization}

\begin{table}
    \addtolength{\tabcolsep}{-0.2em}
    \centering
    \caption{Influence of training options, bound tightening and ReLU scaling on all trained neural networks and their optimization problems \eqref{prob:minANN}. In each row, the effect of the listed method is evaluated by comparing it to similar networks that differ only in this particular method, e.g., for $\ell^1$ regularization we compare neural networks that were trained with the specified level of regularization to those that were trained without regularization. The first and second column show the number of solved instances without and with the applied technique and the number of instances in total in this comparison. The third column lists the reduction of big-M coefficients as measured by the geometric mean of the ratio of averaged distances of pre-activation bounds $U- L$ of the adapted network and that of the baseline network. The fourth column shows the arithmetic mean of the increase in percentage points of stable neurons due to the applied method. The fifth column shows the geometric mean of the ratio between the number of linear regions of the adapted network and that of the baseline network. The last column shows the geometric mean of the ratio between the computational time with the adapted network and that observed with the baseline network, but is limited to instances in which the optimization problems for both networks are solved within the time limit. We observe a computational speedup with regularization, bound tightening and ReLU-scaling, while dropout leads to a deterioration in performance.}
    \label{tab:results}
    \begin{tabular}{lc|cccccc}
    \multicolumn{1}{c}{}    &  & \begin{tabular}[c]{@{}c@{}}Solved instances\\ (adapted vs. baseline)\end{tabular} & \begin{tabular}[c]{@{}c@{}}Instances\\  total\end{tabular} & 
    \begin{tabular}[c]{@{}c@{}}Geom. mean \\ $\overline{U-L}$ \end{tabular} &
    \begin{tabular}[c]{@{}c@{}}Improvement \\ stable neurons\end{tabular} &    \begin{tabular}[c]{@{}c@{}}Geom. mean \\ lin. regions\end{tabular} & \begin{tabular}[c]{@{}c@{}}Geom. mean \\ time\end{tabular} \\ \hline
    \multirow{5}{*}{$\lambda$} & 1e-3   & 349 vs. 151 & 360 & 0.009  & 0.379    & 0.283  & 0.028 \\
                             & 1e-4   & 358 vs. 151 & 360  & 0.024  & 0.216  & 0.463 & 0.059  \\
                             & 1e-5   & 352 vs. 151 & 360  & 0.052  & 0.122  & 0.817 & 0.109  \\
                             & 1e-6   & 326 vs. 151  & 360 & 0.133  & 0.146  & 1.089 & 0.208  \\
                             & 1e-7   & 287 vs. 151  & 360 & 0.261  & 0.213  & 0.996 & 0.280  \\ \hline
    \multirow{2}{*}{\begin{tabular}[c]{@{}l@{}}Clipped\\ ReLU\end{tabular}} 
                            & M=2    &609 vs. 608 & 720   & 0.415   & 0.029 & 1.094 & 0.931  \\
                            & M=5    & 606 vs. 608 & 720  & 0.560   & 0.011 & 1.055 & 0.974  \\ \hline
    \multirow{2}{*}{Dropout}    & 10\%   & 146 vs. 217 & 240 & 12.761  & -0.204 & 4.098 & 5.825  \\
                                & 20\%   & 152 vs. 217& 240  & 15.403  & -0.210 & 3.513 & 4.845  \\ \hline \hline
    \multicolumn{2}{l|}{OBBT}   & 912 vs. 911  & 1080  & 0.541 & 0.055  & 1.0   & 0.570  \\ \hline
    \multirow{2}{*}{\begin{tabular}[c]{@{}l@{}}ReLU\\ scaling\end{tabular}} &        & 307 vs. 303  & 360   & 0.388 & 0.0    & 1.0   & 0.936  \\
                & OBBT & 299 vs. 303  & 360  & 0.160  & 0.072  & 1.0   & 0.467                                                     
    \end{tabular}
\end{table}

In \vref{fig:model_statistics}, we depict how the mean absolute error on the test set, the number of linear regions, the percentage of neurons of fixed activation, and the solver runtime correlate with the depth of the neural networks for networks with 50 neurons per layer with different regularization parameters. 
In the first row, we see the performance of the neural networks on the test data as measured by the MAPE. We see, that large regularization parameters lead to a degradation of accuracy on the test dataset, especially for Ackley's function. For small regularization parameters there is a high level of agreement between the predictions and the ground truth on the test data. Further, in some instances, training with moderate levels of $\ell^1$ regularization does in fact lead to be better generalization of the neural network. 
%
The second row of \cref{fig:model_statistics} shows the number of linear regions as an indicator of the complexity, or expressive power of the neural network. With increasing levels of regularization, we obtain neural networks with a lower number of linear regions. This is also illustrated in \vref{fig:linear_regions}. Comparing the number of linear regions among the three different functions, the neural networks which approximate Ackley's function have the most linear regions. This is plausible comparing the surface plots in \vref{fig:different_functions}, because Ackley's function exhibits a large number of local oscillations.
%
In the third row of \cref{fig:model_statistics}, we plot the percentage of stable neurons. These are neurons whose input bounds are either non-negative or non-positive, which means that they are in a fixed state of activation regardless of input. No binary variables have to be added to model the activation function of such neurons. Confirming the findings of \citet{Xiao2019,Serra2020}, higher values of $\lambda$ lead to a higher percentage of stable neurons. 
%
The last row shows the computational times in the optimization problem \eqref{prob:minANN}. Comparing the runtimes among the three functions, Ackley's function appears to be the hardest to minimize. Here, we cannot solve unregularized networks with as little as three hidden layers to global optimality within the specified time limit. Based on the observation that ANNs approximating this function have an increased number of linear regions and that several local minima exist in the input domain, this is expected behavior. Increasing the regularization generally lowers the time to compute global minima for all three functions. While the global minima of unregularized networks cannot be determined for any network with more than four layers, applying moderate levels of regularization makes almost all instances tractable. The only exception here is Ackley's function, which remains unsolved for the lowest regularization parameter $\lambda = 10^{-7}$ as well. While \cref{fig:model_statistics} shows the results for all networks with 50 neurons per hidden layer, we obtain similar results for those with 25 neurons (data shown in the appendix).
In combination with the results in \vref{tab:results}, this illustrates that regularization proved the most effective method by improving big-M coefficients, increasing the number of stable neurons and decreasing the number of linear regions, thus enabling the computational speedup.
%
\begin{figure}[H]
    \centering
    \begin{subfigure}{.32\linewidth}
        \centering
        Peaks
        \includegraphics[width=\linewidth]{figures/statistics/regularization/peaks_mape_50_neurons}
        \subcaption{MAPE on test set of ReLU ANNs approximating \eqref{fun:peaks}.}
    \end{subfigure}
    \begin{subfigure}{.32\linewidth}
        \centering
        Ackley
        \includegraphics[width=\linewidth]{figures/statistics/regularization/ackley_mape_50_neurons}
        \subcaption{MAPE on test set of ReLU ANNs approximating \eqref{fun:ack}.}
    \end{subfigure}
    \begin{subfigure}{.32\linewidth}
        \centering
        Himmelblau
        \includegraphics[width=\linewidth]{figures/statistics/regularization/himmelblau_mape_50_neurons}
        \subcaption{MAPE on test set of ReLU ANNs approximating \eqref{fun:him}.}
    \end{subfigure}
    
    %\begin{subfigure}{.32\linewidth}
    %    \centering
    %    \includegraphics[width=\linewidth]{figures/statistics/regularization/peaks_r2_50_neurons}
    %    \subcaption{$R^2$ on test set of ReLU ANNs approximating \eqref{fun:peaks}.}
    %\end{subfigure}
    %\begin{subfigure}{.32\linewidth}
    %    \centering
    %    \includegraphics[width=\linewidth]{figures/statistics/regularization/ackley_r2_50_neurons}
    %    \subcaption{$R^2$ on test set of ReLU ANNs approximating \eqref{fun:ack}.}
    %\end{subfigure}
    %\begin{subfigure}{.32\linewidth}
    %    \centering
    %    \includegraphics[width=\linewidth]{figures/statistics/regularization/himmelblau_r2_50_neurons}
    %    \subcaption{$R^2$ on test set of ReLU ANNs approximating \eqref{fun:him}.}
    %\end{subfigure}

    \begin{subfigure}{.32\linewidth}
        \centering
        \includegraphics[width=\linewidth]{figures/statistics/regularization/peaks_num_regions_50_neurons}
        \subcaption{Number of linear regions of ReLU ANNs approximating \eqref{fun:peaks}.}
    \end{subfigure}
    \begin{subfigure}{.32\linewidth}
        \centering
        \includegraphics[width=\linewidth]{figures/statistics/regularization/ackley_num_regions_50_neurons}
        \subcaption{Number of linear regions of ReLU ANNs approximating \eqref{fun:ack}.}
    \end{subfigure}
    \begin{subfigure}{.32\linewidth}
        \centering
        \includegraphics[width=\linewidth]{figures/statistics/regularization/himmelblau_num_regions_50_neurons}
        \subcaption{Number of linear regions of ReLU ANNs approximating \eqref{fun:him}.}
    \end{subfigure}

    \begin{subfigure}{.32\linewidth}
        \includegraphics[width=\linewidth]{figures/statistics/regularization/peaks_percentage_fixed_50_neurons}
        \subcaption{Percentage of stable neurons for big-M coefficients based on interval arithmetic bounds.}
    \end{subfigure}
    \begin{subfigure}{.32\linewidth}
        \includegraphics[width=\linewidth]{figures/statistics/regularization/ackley_percentage_fixed_50_neurons}
        \subcaption{Percentage of stable neurons for big-M coefficients based on interval arithmetic bounds.}
    \end{subfigure}
    \begin{subfigure}{.32\linewidth}
        \includegraphics[width=\linewidth]{figures/statistics/regularization/himmelblau_percentage_fixed_50_neurons}
        \subcaption{Percentage of stable neurons for big-M coefficients based on interval arithmetic bounds.}
    \end{subfigure}


    \begin{subfigure}{.32\linewidth}
        \includegraphics[width=\linewidth]{figures/statistics/regularization/peaks_time_50_neurons}
        \subcaption{Solution time of problem \eqref{prob:minANN} for Peaks function.}
        \typeout{PLOT LINE WIDTH: \the\linewidth}%
    \end{subfigure}
    \begin{subfigure}{.32\linewidth}
        \includegraphics[width=\linewidth]{figures/statistics/regularization/ackley_time_50_neurons}
        \subcaption{Solution time of problem \eqref{prob:minANN} for Ackley's function.}
    \end{subfigure}
    \begin{subfigure}{.32\linewidth}
        \includegraphics[width=\linewidth]{figures/statistics/regularization/himmelblau_time_50_neurons}
        \subcaption{Solution time of problem \eqref{prob:minANN} for Himmelblau function.}
    \end{subfigure}
    \begin{subfigure}{\linewidth}
            \centering
            \definecolor{crimson2143940}{RGB}{214,39,40}
\definecolor{darkorange25512714}{RGB}{255,127,14}
\definecolor{forestgreen4416044}{RGB}{44,160,44}
\definecolor{mediumpurple148103189}{RGB}{148,103,189}
\definecolor{steelblue31119180}{RGB}{31,119,180}
\definecolor{darkgray176}{RGB}{176,176,176}
\begin{tikzpicture} 
    \begin{axis}[%
    hide axis,
    xmin=10,
    xmax=50,
    ymin=0,
    ymax=0.4,
    legend style={
        draw=white!15!black,
        legend cell align=left,
        legend columns=-1, 
        legend style={
            draw=none,
            column sep=1ex,
            line width=0.5pt
        }
    },
    ]
    \addlegendimage{line width=2pt, color=C4}
    \addlegendentry{32};
    \addlegendimage{line width=2pt, color=C0}
    \addlegendentry{16};
    \addlegendimage{line width=2pt, color=C3}
    \addlegendentry{8};
    \addlegendimage{line width=2pt, color=C1}
    \addlegendentry{4};
    \addlegendimage{line width=2pt, color=C5}
    \addlegendentry{2};
    \end{axis}
\end{tikzpicture}
    \end{subfigure}
    \caption{Mean absolute percentage error on test set, number of linear regions, percentage of stable neurons and computational times in problem \eqref{prob:minANN} of trained ANNs with varying number of hidden layers with 50 neurons, trained with different levels of $\ell^1$ regularization.}
    \label{fig:model_statistics}%
    \typeout{LINE WIDTH: \the\linewidth}%
\end{figure}


\begin{figure}
    \centering
    \begin{subfigure}{.32\linewidth}
        \centering
        \includegraphics[width=\linewidth,trim={0 0 0 8mm},clip]{figures/mesh/relu_peaks_5_layer_25_neurons_0e+00_regu_0_dropout_mesh}
        \subcaption{Standard, 11,898 regions.}
        \label{subfig:mesh_no_regu}
    \end{subfigure}
    \begin{subfigure}{.32\linewidth}
        \centering
        \includegraphics[width=\linewidth,trim={0 0 0 8mm},clip]{figures/mesh/relu_peaks_5_layer_25_neurons_1e-05_regu_0_dropout_mesh}
        \subcaption{Regularization, 6,893 regions.}
        \label{subfig:mesh_regu}
    \end{subfigure}
    \begin{subfigure}{.32\linewidth}
        \centering
        \includegraphics[width=\linewidth,trim={0 0 0 8mm},clip]{figures/mesh/relu_peaks_5_layer_25_neurons_0e+00_regu_20_dropout_mesh.png}
        \subcaption{Dropout, 14,674 regions.}
        \label{subfig:mesh_dropout}
    \end{subfigure}
    \caption{Linear regions for ReLU networks approximating the Peaks function \eqref{fun:peaks} with five hidden layers of 25 neurons each. Color-coded in the backgrounds are the outputs of the neural networks. Compared are networks with different training options: \cref{subfig:mesh_no_regu} with no regularization or dropout, \cref{subfig:mesh_regu} with $\ell^1$ regularization and $\lambda=10^{-5}$, \cref{subfig:mesh_dropout} with $20$\% dropout. Regularizing the weights of the ANN during training decreases the number of linear regions, applying dropout increases it and also changes their sizes.}
    \label{fig:linear_regions}
\end{figure}


\subsection{Effect of clipped ReLU}

The effect of the clipping is obvious in the big-M coefficients of formulation \eqref{eq:bigM_clipped}, which are illustrated in \vref{fig:big_Ms_clipped} for a threshold of $M=5.0$. Compared to the big-M coefficients derived for the regular ReLU activation function and depicted in \vref{fig:IA_bounds}, the clipped ReLU formulation yields lower bounds, though this may depend on the particular choice of $M$. This is also obvious from the results in \vref{tab:results}, with clipped ReLU leading to greater reductions in big-M coefficients compared to OBBT. We also observe that LP-based bound tightening for neural networks with clipped ReLU activation does not improve the bounds to the same degree as it did for the regular ReLU activation function as depicted in \vref{fig:LR_bounds}.

As the results in \Cref{tab:results} suggest, using the clipped ReLU  \eqref{eq:bigM_clipped} yields only marginal computational speedup compared to the standard ReLU activation. There seems to be a tradeoff between a higher number of binary variables needed for modeling \eqref{eq:bigM_clipped} and the slightly higher number of linear regions on the one hand, and the reduction of big-M coefficients on the other hand. 

\begin{figure}[ht!]
    \centering
    \begin{subfigure}{0.49\textwidth}
        \includegraphics[width=\textwidth]{figures/clipped_5_big_M_unscaled}
        \caption{Pre-activation bounds $U^{(k)}$ determined via interval arithmetic for clipped ReLU ANN with the big-M formulation \eqref{eq:bigM_clipped} and $M=5.0$.}
        \label{fig:IA_bounds_clipped}
    \end{subfigure}
    \hfill
    \begin{subfigure}{0.49\textwidth}
        \includegraphics[width=\textwidth]{figures/clipped_5_big_M_unscaled_OBBT}
        \caption{Pre-activation bounds $U^{(k)}$ determined via LP-based bound tightening for clipped ReLU ANN with big-M formulation \eqref{eq:bigM_clipped} and $M=5.0$.}
        \label{fig:LR_bounds_clipped}
    \end{subfigure}
   
    \caption{Comparison of pre-activation bounds  $U^{(k)}$ for neural networks with ten hidden layers and clipped ReLU formulation \eqref{eq:bigM_clipped} with $M=5.0$ as activation function. Compared to the bounds derived via interval arithmetic for the regular ReLU activation shown in \vref{fig:IA_bounds}, the bounds for the clipped ReLU are generally lower. Moreover, due to the threshold $M$, the bounds stay approximately constant over the layers. Solving auxiliary LPs only noticeably tightens bounds in the first few layers.}
    \label{fig:big_Ms_clipped}
\end{figure}
 
\subsection{Effect of Dropout}

For the Peaks function only, we trained additional networks with different levels of dropout applied to the hidden layers, namely 10 and 20 percent. In combination with the other hyperparameters (regularization, depth and width of the network) this yields a total of 240 trained neural networks with dropout whose properties we can compare. As illustrated in \Cref{tab:results}, we find that dropout leads to neural networks with three to four times more linear regions on average, confirming the findings of \citet{Zhang2020a}. This is also evident in \Cref{fig:linear_regions}, where the linear regions of three exemplary ANNs are compared for networks with five hidden layers. Another effect of dropout is the percentage of stable neurons, which is reduced by approximately 20~\% on average compared to networks trained without dropout and a drastic increase in the magnitude of the big-M coefficients. In combination, this leads to a reduction in instances that could be solved to global optimality and a simultaneous four to six-fold increase in computational time for those instances that could be solved. 


\section{Conclusions and outlook}\label{sec:discussion}

In this paper, we compared different variations of training and scaling methods for ReLU networks with respect to their effect on the performance of global optimization solvers on problems with these networks  embedded. We divided these methods into those that are applied during training and those that can be used on trained networks. For the latter category, we proposed a scaling method specific to the ReLU activation function, which equivalently transforms a ReLU ANN such that the $\ell^1$ norm of the networks weights and biases is minimized. This has the desired effect of reducing the constant coefficients in big-M formulations of the network's activation functions. In numerical experiments, we demonstrated that this method can be used to reduce the computational effort of solving subsequent optimization problems, when it is used in combination with bound tightening. Although in our study we only investigated the direct minimization of feed-forward neural networks with their big-M formulation of ReLU networks, we believe that the findings are also applicable in other contexts. These might include optimization problems with ReLU networks using different MILP encodings, e.g., the partition-based formulation from \citet{Tsay2021}, or other optimization settings, e.g., more difficult optimization problems from real-world applications. In fact, by employing regularization during training we were able to solve a complex superstructure optimization problem in chemical engineering that had been computationally intractable before \citep{Klimek2024}.

Moreover, to the best of our knowledge, this is the first computational study that links various training methods to both the number of linear regions and the percentage of fixed neurons as well as the computational effort in subsequent optimization problems. Doing so, we were able to provide empirical evidence for several observations from the literature, e.g., an increased number of linear regions for networks trained with dropout, and computational speedup due to higher rates of fixed neurons for networks trained with $\ell^1$ regularization. 

Further research may include a more thorough analysis into how the used training methods and hyperparameter options used when training a neural network impact its number of linear regions and the number of fixed neurons. Also, different objectives in \eqref{prob:scaling} may be conceivable to promote other properties in the transformed networks. It may also be promising to investigate transformations that allow minor perturbations of the functional relationship.

\begin{acknowledgements}
This project has received funding from the European Regional Development Fund (grants timingMatters and IntelAlgen) under the European Union’s Horizon Europe Research and Innovation Program, 
from the research initiative ``SmartProSys: Intelligent Process Systems for the Sustainable Production of Chemicals'' funded by the Ministry for Science, Energy, Climate Protection and the Environment of the State of Saxony-Anhalt, and
from the German Research Foundation DFG within GRK 2297 ’Mathematical Complexity Reduction’ and priority program 2331 ’Machine Learning in Chemical Engineering’ under grant SA 2016/3-1.
\end{acknowledgements}



\bibliographystyle{spbasic}
\bibliography{main}

\section{Appendix}
\subsection{Preliminaries: Missing Proofs}\label{appendix:preliminaries}
We present the formal proof of Lemma \ref{lemma:broadcast-easy}, establishing that whenever $\bb$ can be achieved, the $\byzantineSM$ problem is solvable.
\BroadcastEasy*

\begin{proof}
The parties distribute their input preference lists via $\bb$. This provides the parties with an identical view over the parties' preferences lists. If a party $P$ has not sent a valid preference list, then $P$ is byzantine, and the honest parties may simply assign a pre-defined default preference list to it. 
Afterwards, each party runs $\GaleShapley$ offline with the preference lists obtained and obtains a matching $M$. Each party then outputs its match in $M$.

Termination comes from $\bb$'s termination property. $\bb$'s validity condition ensures that, if the input of an honest party $P$ in our $\byzantineSM$ instance is the preference list $\pi_P$, then party $P$ has preference list $\pi_P$ in each of the honest parties' offline executions of $\GaleShapley$.
$\bb$'s Agreement properties ensures that all honest parties run $\GaleShapley$ with the same input. Since $\GaleShapley$ is deterministic, all honest parties obtain the same output $M$. 

According to Theorem \ref{theorem:gale-shapley}, $M$ is a proper matching (if $u$ is matched with $v$, then $v$ is matched with $u$) satisfying stability (no blocking pair). Therefore, as each honest party outputs its match in $M$, symmetry, non-competition and stability hold. Hence, $\byzantineSM$ is achieved.
\end{proof}


\subsection{Simplified Stable Matching: Missing Proofs} \label{appendix:simplified-stable-matching}

We first include the proof of Lemma \ref{coro:to-simplified}, establishing that $\simplifiedSM$ reduces to $\byzantineSM$.

\SimplifiedReduction*

\begin{proof}
It suffices to show that any protocol solving $\byzantineSM$ also solves $\simplifiedSM$. 
Given a protocol $\Pi$ solving $\byzantineSM$, we construct a protocol $\Pi'$ that solves $\simplifiedSM$, as follows:

Given its favorite as input, each party constructs an arbitrary preference list with the favorite ranked first. Afterward, parties join an invocation of $\Pi$ with the constructed lists as inputs. The output obtained in $\Pi$ for $\byzantineSM$ is used in $\Pi'$ as the output for $\simplifiedSM$.


First, $\Pi'$ maintains the resilience thresholds of $\Pi$. Moreover, the termination, symmetry, and non-competition guarantees of $\Pi'$ follow directly from $\Pi$ achieving termination, symmetry, and non-competition, respectively. Finally, if two honest parties are each other's favorites, they rank each other first in the constructed lists and, consequently, always form a blocking pair if they are not matched. Therefore, the simplified stability property of $\Pi'$ is guaranteed by the stability property of $\Pi$.
\end{proof}

We now present the proof of Lemma \ref{lemma:reduce-number}, allowing us to extend impossibility results from small settings to larger settings.
\ReduceNumberLemma*
\begin{proof}
We partition $L$ into $d$ disjoint sets $L_1$, \dots, $L_d$ such that $1 \leq |L_1|,\dots,|L_d| \leq \lceil |L| / d \rceil = \lceil k/d \rceil$. Similarly, we partition $R$ into $d$ disjoint sets $R_1$, \dots, $R_d$ such that $1 \leq |R_1|,\dots,|R_d| \leq \lceil |R| / d \rceil = \lceil k/d \rceil$. From each of these sets, we pick one representative: $l_1, \dots, l_d, r_1, \dots, r_d$. We build $\Pi'$ solving $\simplifiedSM$ for $2d$ parties: $l_1', \dots, l_d'$ on the left side and $r_1', \dots, r_d'$ on the right side, as follows:
\begin{itemize}[nosep]
\item Each party $l_i'$ in $\Pi'$ simulates all the parties in $L_i$ running $\Pi$. Similarly, each party $r_j'$ in $\Pi'$ simulates all the parties in $R_j$ running $\Pi$.
\item Input: If the input (favorite) of $l_i'$ is $r_j'$, then we assign $r_j$ as the favorite of $l_i$. Similarly, if the input of $r_j'$ is $l_i'$, then we assign $l_i$ as the favorite of $r_j$. For parties that are not representatives of their group, we assign arbitrary favorites.
\item Output: For a given $l_i$, if there is a $r_j$ such that $l_i$ matches $r_j$, then $l_i'$ declares that it matches $r_j'$. Otherwise $l_i'$ declares that it matches nobody.
\end{itemize}

We are basically running the $\simplifiedSM$ algorithm on the whole graph but only looking at the representative of each set and discarding anything unrelated to them. As a consequence, $\Pi'$ achieves termination, symmetry, simplified stability, and non-competition since $\Pi$ achieves termination, symmetry, simplified stability, and non-competition. 

As each party in $\Pi'$ simulates up to $\lceil k / d \rceil$ parties 
from $\Pi$ and $\Pi$ supports up to $t_L$ byzantine parties in $L$ and $t_R$ byzantine parties in $R$, the bound on the number of byzantine parties supported by $\Pi'$ follows immediately.
\end{proof}

\subsection{Byzantine Broadcast with General Adversaries}\label{appendix:general-adversaries}
To achieve the feasibility part of Theorem \ref{theo:pki-complete}, 
we got help from the result below.

\GeneralAdversaries*

As mentioned before, this is a corollary of \cite[{Theorem 2}]{DISC:FitMau98}. We again highlight that \cite{DISC:FitMau98} assumes a \emph{general adversary}, which we briefly introduce next. In this adversarial model, the corruption power of the adversary is specified by a (subset-closed) 
% needs to define an 
% considers an 
\emph{adversarial structure} $\mathcal{Z} \subseteq 2^\mathcal{P}$, where $\mathcal{P}$ denotes the set of parties. In particular, the adversary may choose to corrupt any set of parties in $\mathcal{Z}$.
For instance, if $\mathcal{P} := \{P_1, P_2, \ldots, P_5\}$, a potential adversarial structure $\mathcal{Z}$ is $\{\varnothing, \{P_1\}, \{P_2\}, \{P_1, P_2\}, \{P_4\}\}$, which means that the adversary may choose between corrupting no parties, corrupting parties $P_1, P_2$ (or only one of the two), or corrupting only party $P_4$. In contrast, one often considers a \emph{threshold adversary}, which may corrupt up to $t$ of the $n$ parties (as is the case in most literature): this is a particular case of the general adversary model where $\mathcal{Z}$ is the set of all subsets of at most $t$ parties. The adversary assumed in our work sits in-between these two: we assume that the adversary may corrupt up to $t_L$ parties in $L$ and up to $t_R$ parties in $R$, hence our adversary structure is $\mathcal{Z^\star} := \{S_L \cup S_R \mid S_L \subseteq L, S_R \subseteq R, \abs{S_L} \leq t_L, \abs{S_R} \leq t_R\}$. This can be thought of as the product of two threshold adversary structures.

In the general adversaries model, {\cite[{Theorem 2}]{DISC:FitMau98}} states the following:
\begin{theorem}[\hspace{-1pt}{\cite[{Theorem 2}]{DISC:FitMau98}}] \label{thm:general-adversaries-explicit}
    Assume a fully-connected unauthenticated network, and an adversary structure $\mathcal{Z}$ such that for any three sets $Z_1, Z_2, Z_3 \in \mathcal{Z}$ it holds that $Z_1 \cup Z_2 \cup Z_3 \neq \mathcal{P}$. Then, there is a protocol achieving $\bb$ in this setting.
\end{theorem}

Then, to prove Lemma \ref{lemma:general-adversaries}, we only need to show no three sets in our 
% implicit
adversary structure $\mathcal{Z}^\star$ cover the set of $n$ parties:
\begin{proof}[Proof of Lemma \ref{lemma:general-adversaries}]
Consider our adversarial structure $\mathcal{Z^\star} := \{S_L \cup S_R \mid S_L \subseteq L, S_R \subseteq R, \abs{S_L} \leq t_L, \abs{S_R} \leq t_R\}$, and let $Z_1, Z_2, Z_3 \in \mathcal{Z^\star}$ be arbitrary. We show that $Z_1 \cup Z_2 \cup Z_3 \neq L \cup R$.


Without loss of generality, we may assume that the condition $t_L < k / 3$ holds (the case where $t_R < k / 3$ and $t_L \geq k / 3$ is analogous). As every $Z \in Z^\star$ contains at most $t_L$ parties in $L$, it follows that $Z_1 \cup Z_2 \cup Z_3$ contain at most $3 \cdot t_L < 3 \cdot k / 3 = k$ parties in $L$. Hence, at least one element of $L$ is uncovered by $Z_1 \cup Z_2 \cup Z_3$, from which $Z_1 \cup Z_2 \cup Z_3 \neq L \cup R$.

Then, we may apply Theorem \ref{thm:general-adversaries-explicit} and conclude that there is a protocol achieving $\bb$ in our setting.
\end{proof}

\subsection{Unauthenticated Setting: Missing Proofs}\label{appendix:complete-bipartite-graph-without-pki}
We present the proof of Lemma \ref{lemma:pki-bipartite}. This has provided reductions between the communication models when analyzing $\byzantineSM$ in unauthenticated settings.
\BipartiteCommunicationNoPKI*

\begin{proof}
Let $u,v$ be parties in $S$. We want to simulate an authenticated channel between $u$ and $v$, i.e the receiver knows who the sender is. 

If $u$ wants to send a message $M$ to $v$, it sends the message ($u \rightarrow v$, $M$) to every party in $S'$. Then, if a party in $S'$ receives a message ($u \rightarrow v$, $M$) from $u$, it forwards it to $v$. Finally, if $v$ receives the same message ($u \rightarrow v$, $M$) from a majority (i.e strictly more than $k/2$) of $S'$, it considers it received message $M$ from $u$.

Using this strategy, we can see that sending a message takes a bounded amount of time (at most $2 \Delta$). Moreover, if $u$ is honest and sends a message $M$, at least $k - t_{S'} > k/2$ parties from $S'$ will forward it to $v$ which will therefore accept it.
If $v$ accepts a message $M$ from $u$, this means strictly more than $k/2$ parties from $S'$ forwarded it. Because $t_{S'} < k/2$, at least one honest party forwarded it, meaning $u$ intended to send this message (being honest or not).
\end{proof}

\subsection{Authenticated Setting: Missing Proofs} \label{appendix:bipartite-pki}
We present the proof of Lemma \ref{lemma:with-pki-one-sided}, which has provided us with reductions between the communicated models when analyzing $\byzantineSM$ in authenticated settings.

\PKIOneSided*

\begin{proof}
Let $u,v$ be parties in $S$. We want to simulate an authenticated channel between $u$ and $v$.
% to send a message from $u$ to $v$ in an authenticated way, i.e the receiver knows who the sender is. 

If $u$ wants to send a message $M$ to $v$, it sends the signed message ($u \rightarrow v$, $M$) to every party in $S'$. Then, if a party in $S'$ receives a message ($u \rightarrow v$, $M$) with a valid signature from $u$, it forwards it to $v$. Finally, if $v$ receives a message ($u \rightarrow v$, $M$) from a party in $S'$ with a valid signature, it considers it receives message $M$ from $u$.

Using this strategy, we can see that sending a message takes a bounded amount of time (at most $2 \Delta$). Moreover, if $u$ is honest and sends a message $M$, at least $k - t_{S'} > 0$ parties from $S'$ will forward it to $v$, which will therefore accept it.
If $v$ accepts a message $M$ from $u$, this message is signed by $u$, meaning $u$ intended to send this message (being honest or not).
\end{proof}

\MagicOmissionsNew*
\begin{proof}
Let $(u,v)$ be two parties in $L$ and assume that $u$ wants to send a message $\msg$ to $v$. $u$ sends a signed message $(u \rightarrow v, \timestamp, \msgId, \msg)$ to all parties in $R$, $\tau$ being the current timestamp and $\msgId$ being a message identifier. Parties in $R$ then forward this signed message to $v$. If $v$ receives a message $(u, \rightarrow, v, \timestamp, \msgId, \msg)$ properly signed by $u$ such that $\timestamp$ is at most $2\Delta$ units of time in the past and $u$ has not seen $\msgId$ has not been seen before, it accepts message $\msgId$ from $u$.

With this approach, if at least one party in $R$ is honest, messages always get forwarded and received within $2\Delta$ units of time. Otherwise, note that the byzantine parties cannot forge signatures on the honest parties' behalf. Hence, the byzantine parties may choose whether to forward the message or not, then causing an omission.
\end{proof}



\subsection{Protocols in Settings with or without Omissions} \label{appendix:sync-and-omission}
In order to prove sufficiency when one side may be completely byzantine in Section \ref{subsection:bipartite-pki}, we have considered a setting consisting of a fully-connected synchronous network where \emph{omissions} may occur: if a message is delivered, it is delivered within $\Delta$ time. We have utilized the building blocks described by the theorems below:
\BAWithOmissions*
\BBWithOmissions*

In the following, we describe the constructions behind these theorems.

\paragraph{Byzantine Agreement.} We start by presenting protocol $\Pi_{\ba}$.
We need a \emph{synchronous} $k$-party $\ba$ protocol resilient against $t_L < k / 3$ corruptions. We may use, for instance, the protocol of \cite{King}, presented below.
\begin{protocolbox}{$\Pi_{\king}$}
    \algoHead{Code for party $P \in L$ with input $v_{\inputt}$}
    \begin{algorithmic}[1]
    \State If you have not obtained any output by time $3(t_L + 1) \cdot \Delta$, output $\bot$.
    \State $v := v_{\inputt}$
    \For {$i = 1 \ldots t_L + 1$}
    \State \textbf{(Round 1)}
    \State Send $(\val, v)$ to all parties. 
    \State Wait $\Delta$ time.
    \State \textbf{(Round 2)}
    \State If you have received $(\val, v')$ for the same $v'$ from $k - t_L$ parties in $L$:
    \State \hspace{0.5cm} Send $(\propose, v')$ to all parties
    \State Wait $\Delta$ time.
    \State \textbf{(Round 3})
    \State If you have received some $(\propose, v')$ from more than $t_L$ parties, set $v' = v$.
    \State \textbf{King $P_i$ only}: Send $v_K := v$ to all parties.
    \State Wait $\Delta$ time.
    \State If you have received strictly less than $k - t_L$ messages $(\propose, v')$ for any $v'$:
    \State \hspace{0.5cm} If you have received $v_K$ from king $P_i$: Set $v = v_K$
    \EndFor
    \State Output $v$
\end{algorithmic}
\end{protocolbox}

The next theorem comes directly from \cite{King}.
\begin{theorem}[Theorem 3.1 of \cite{King}]
    Whenever $\Pi_{\king}$ runs in a synchronous network with maximum delay $\Delta$ and at most $t_L < k / 3$ byzantine corruptions, $\Pi_{\king}$ achieves $\ba$ within $\Delta_{\king} := 3(t_L + 1) \cdot \Delta$ time.
\end{theorem}

We note that, due to line 1, termination is guaranteed even when omissions occur.
\begin{remark}
    Whenever $\Pi_{\king}$ runs in a synchronous network with omissions with maximum delay $\Delta$, it achieves termination within $\Delta_{\king}(\Delta) := 3(t_L + 1) \cdot \Delta$ time.
\end{remark}

However, $\Pi_{\king}$ does not achieve weak agreement when omissions occur. To achieve this property, we need one more round of communication, as presented below.

\begin{protocolbox}{$\Pi_{\ba}$}
    \algoHead{Code for party $P \in L$ with input $v_{\inputt}$}
    \begin{algorithmic}[1]
    \State Join $\king$ with input $v_{\inputt}$ and obtain output $y$.
    \State At time $\Delta_{\king}(\Delta)$, send $y$ to every party.
    \State If the same value $z$ is received from $k - t_L$ parties in $L$ by time $\Delta_{\king}(\Delta) + \Delta$, output $z$. Otherwise, output $\bot$.
\end{algorithmic}
\end{protocolbox}


\begin{proof}[Proof of Theorem \ref{thm:ba-omissions}]
Termination  is achieved within $\Delta_\ba(\Delta) = 3(t_L + 1) \cdot \Delta + \Delta$ time even if omissions occur: this follows from $\Pi_{\king}$'s termination guarantees. 

We first show that $\ba$ is achieved when no omissions occur. In this case $\Pi_{\king}$ achieves $\ba$. Hence, all honest parties obtain the same value $y$. Then, at least the $k - t_L$ honest parties send $y$ to all parties. Since no omissions occur, these messages are received within $\Delta$ time, and all parties output $y$. Moreover, due to $\Pi_{\king}$'s validity, if all honest parties had the same input $v$, the honest parties have obtained $y = v$, and therefore all honest parties output $y$. Consequently, $\ba$ is achieved.


We may now discuss weak agreement if omissions occur. In this case $\Pi_{\king}$ only achieves termination (and it is possible that the honest parties obtain $y = \bot$). Assume that an honest party $p$ outputs $z \neq \bot$ in $\Pi_{\ba}$. Then, $p$ has received $z$ from $k - t_L$ parties, hence from at least $n - 2t_L > t_L$ honest parties. As such, every party can receive strictly less than $n - t_L$ values $z' \neq z$, and therefore no honest party outputs $z' \neq z$.  Therefore, weak agreement holds.
\end{proof}

\paragraph{Byzantine Broadcast.}
Protocol $\Pi_{\bb}$ is a simple reduction to $\Pi_{\ba}$. The sender $S$ sends its value to all parties, and afterwards the parties run $\Pi_{\ba}$ to agree on the value received.
\begin{protocolbox}{$\Pi_{\bb}$}
    \algoHead{Code for sender $S \in L$ with input $v_{S}$}
    \begin{algorithmic}[1]
    \State Send $v_{S}$ to all parties.
    \end{algorithmic}

    \algoHead{Code for party $P \in L$}
    \begin{algorithmic}[1]
    \State Receive value $v$ from the sender. If you did not receive a value within $\Delta$ time, set $v :=$ default value (default preference list).
    \State At time $\Delta$, join $\Pi_{\ba}$ with input $v$. Return the output obtained.
\end{algorithmic}
\end{protocolbox}

\begin{proof}[Proof of Theorem \ref{thm:bb-omissions}]
$\Pi_{\ba}$ achieves termination within $\Delta_{\ba}(\Delta)$ time even when omissions occur, hence $\Pi_{\bb}$ achieves termination within $\Delta_{\bb}(\Delta) = \Delta + \Delta_{\ba}(\Delta)$ time even when omissions occur.

If no omissions occur, $\Pi_{\ba}$ achieves $\ba$. Therefore, $\Pi_{\bb}$ achieves agreement. If the sender is honest with input $v_S$, all honest parties join $\Pi_{\ba}$ with input $v_S$, and the validity guarantee of $\Pi_{\ba}$ ensures that the parties output $v_S$. Then, $\Pi_{\bb}$ achieves validity, and consequently $\bb$.

Lastly, if omissions occur, as $\Pi_{\ba}$ achieves weak agreement, $\Pi_{\bb}$ also achieves weak agreement.
\end{proof}




\end{document}
