\documentclass[smallextended]{svjour3}
\listfiles
\makeatletter % https://tex.stackexchange.com/a/499541
\let\cl@chapter\undefined
\makeatletter
% Language setting
% Replace `english' with e.g. `spanish' to change the document language
\usepackage[english]{babel}

% Set page size and margins
% Replace `letterpaper' with `a4paper' for UK/EU standard size
\usepackage[letterpaper,top=2cm,bottom=2cm,left=3cm,right=3cm,marginparwidth=1.75cm]{geometry}

% Useful packages
\usepackage{natbib}
\usepackage{xparse}
\usepackage{amsmath}
\usepackage{amssymb}
\usepackage{mathtools}
\usepackage{graphicx}
\usepackage{subcaption}
\usepackage[colorlinks=true, allcolors=blue]{hyperref}
%\usepackage{authblk}
\usepackage{pgfplots}
\usepackage{import}
\usepackage{color}
\usepackage{xcolor}
\usepackage{csquotes}
\usepackage{multirow}
\usepackage{todonotes}
\usepackage{varioref}
\usepackage[capitalise,noabbrev]{cleveref}
\usepackage[title]{appendix}
\usepackage{float}
\usepackage{booktabs}

\pgfplotsset{compat=1.16}

\title{An analysis of optimization problems involving ReLU neural networks}
\author{Christoph Plate \and 
            Mirko Hahn  \and
            Alexander Klimek \and
            Caroline Ganzer \and
            Kai Sundmacher \and
            Sebastian Sager}
\institute{Christoph Plate, Mirko Hahn, Kai Sundmacher, Sebastian Sager \at
         Otto von Guericke University Magdeburg, Magdeburg, Germany \\
         \email{\{christoph.plate, mirhahn, kai.sundmacher, sager\}@ovgu.de}
         \and
         Christoph Plate, Alexander Klimek, Caroline Ganzer, Kai Sundmacher, Sebastian Sager \at
         Max Planck Institute for Dynamics of Complex Technical Systems, Magdeburg, Germany \\
         \email{\{plate, klimek, cganzer, sundmacher, sager\}@mpi-magdeburg.mpg.de}
         \and
         Corresponding author: Christoph Plate
}

\date{Received: date / Accepted: date}
\newcommand{\myReLU}[1]{\textrm{ReLU}\left(#1\right)}
\newcommand{\myClippedReLU}[2][M]{\textrm{ReLU}_{#1}\left(#2\right)}

% Prohibit line breaking in inline math
\binoppenalty=10000
\relpenalty=10000

% Mark overfull boxes with a black bar
\overfullrule=1ex

% ORCID
% Alex: 
% Caroline: 0000-0002-7081-4523
% Sebastian: 0000-0002-0283-9075
% Sundmacher: 0000-0003-3251-0593
% Christoph: 0000-0003-0354-8904
% Mirko: 0000-0002-2442-3978

\begin{document}
\maketitle

\begin{abstract}
Solving mixed-integer optimization problems with embedded neural networks with ReLU activation functions is challenging.
Big-M coefficients that arise in relaxing binary decisions related to these functions grow exponentially with the number of layers. 
We survey and propose different approaches to analyze and improve the run time behavior of mixed-integer programming solvers in this context.
Among them are clipped variants and regularization techniques applied during training as well as optimization-based bound tightening and a novel scaling for given ReLU networks. %which yields a functionally equivalent neural network with lower big-M coefficients. 
We numerically compare these approaches for three benchmark problems from the literature.
We use the number of linear regions, the percentage of stable neurons, and overall computational effort as indicators. 
As a major takeaway we observe and quantify a trade-off between the often desired redundancy of neural network models versus the computational costs for solving related optimization problems.
\keywords{optimization \and machine learning \and neural network \and integer programming}
% \PACS{PACS code1 \and PACS code2 \and more}
%\subclass{MSC code1 \and MSC code2 \and more}
\end{abstract}


%!TEX root=main.tex

\section{Introduction}
% Decision-makers, analysts, data scientists, and policymakers frequently rely on data to draw conclusions and extract insights. This data-driven approach helps them identify actionable recommendations aimed at influencing an outcome of interest, such as increasing product satisfaction or income levels or decreasing the likelihood of experiencing serious health conditions \cite{galhotra2022hyper,lakkaraju2016interpretable,agrawal1994fast}. 
\revc{Prescriptions, or actionable recommendations, are commonly generated across various fields to influence key outcomes such as improving product satisfaction, enhancing economic policies, or increasing business efficiency. 
%Decision- or policy-makers, analysts, data scientists, and 
Policymakers in government, decision-makers in businesses, and data scientists in various fields, often rely on data-driven approaches to identify 
%actionable recommendations 
potential actions to influence an outcome of interest, such as increasing income levels or loan approval rates}.
% , or decreasing the likelihood of experiencing serious health conditions. 
%
While association or prediction-based methods are extensively used in practice to draw useful insights from data, they typically identify correlations among variables and may fail to reveal the underlying causal factors, i.e., which actions may result in an improved outcome, needed for informed decision-making. 
%For recommendations to be truly impactful, there must be a clear  explanation that justifies why a particular decision is appropriate for a specific subpopulation~\cite{sun2021treatment,plecko2022causal}. 

\emph{Causal analysis} or {\em causal inference}, therefore, is considered one of the most important requirements to generate prescriptions that are {\em actionable} and aligned with human reasoning~\cite{imbens2024causal}. Causal inference, and in particular {\em observational studies} for causal inference on collected data (when controlled trials are impossible due to cost or ethical reasons), have been extensively studied in the statistics and artificial intelligence (AI) literature for several decades \cite{rubin2005causal, pearl2009causal}. Motivated by this foundational work on causal inference, the notion of causality has also influenced the field of database research. The causal models from AI have been extended to relational databases \cite{salimi2020causal},  and causality has been incorporated into various data management tasks such as finding responsibilities of inputs toward query answers ~\cite{meliou2010causality, meliou2009so, meliou2014causality}, explanations for query answers \cite{roy2014formal, DBLP:journals/pacmmod/YoungmannCGR24}, data discovery~\cite{galhotra2023metam,youngmann2023causal}, data cleaning~\cite{pirhadi2024otclean,salimi2019interventional}, hypothetical reasoning \cite{galhotra2022causal}, and large system diagnostics~\cite{markakis2024sawmill,causalsim,sage, gudmundsdottir2017demonstration}. 


\revc{If-then rules are generally considered interpretable by humans~\cite{lakkaraju2016interpretable,guidotti2018local,van2021evaluating,pradhan2022interpretable,chen2018optimization}.
We give a concrete example of the difference between association and causation in generating prescriptions or recommended actions in the form of if-then rules below}:
\begin{example}	%
\label{example:ex1} {\bf Importance of causal prescriptions:}
Consider the Stack Overflow (SO) annual developer survey
\cite{stackoverflowreport}, where respondents from around the world answer
questions about their jobs and demographics. A sample of the dataset \reva{with a subset of the
attributes (there are 20 attributes)} is presented in \cref{tab:data}.
%
Alice, a researcher in the United Nations (UN) finance department, is interested in discovering ways to increase the salaries of high-tech employees worldwide. She is looking for a set of actionable recommendations 
%(that we call a prescription rules) 
to raise the overall average salary.
%
Using association-based approaches~\cite{chen2018optimization,lakkaraju2016interpretable}, she may discover that individuals residing in the US who identify as straight or heterosexual tend to earn higher salaries (see \cref{exp:quality} for full details). However, this observation merely indicates a correlation: people living in the US, for example, generally earn more than those outside the country. Their comparatively higher salaries are primarily attributable to the country's economy and are unrelated to their sexual orientation. Thus, this observation cannot be used as a prescription rule to increase salary. 
Our causal analysis, on the other hand, reveals that individuals aged 25-34 with dependents would benefit from working as front-end developers.
This results in a \$44,009 annual salary increase on average. \reva{Another observation is that students should pursue an
undergraduate major in CS. %Computer Science (CS). 
This can boost their salary by \$22,174 per year} (see details in \cref{sec:casestudy}).
\end{example}

%It has been incorporated into various tasks including . 
%Causal interventions are often more relatable and easier to understand, as they offer insight into the underlying reasons behind the recommendations and allow unraveling complex cause-effect relationships that govern our world~\cite{pearl2009causality}. Furthermore, causal interventions often have long-lasting effects~\cite{imbens2024causal}.

%, making it essential that the prescribed actions are not only actionable but also 

%causally consistent. 

%Decision makings, in particular, high-stak

\cut{
In this work, {we study the problem of generating causal insights (referred to as \emph{prescription rules}), which serve as actionable recommendations} to improve an outcome of interest.
Recent works have introduced causality to the field of database research~\cite{meliou2010causality,  meliou2014causality,salimi2020causal,10.14778/3554821.3554902}. It has been incorporated into various tasks including data discovery~\cite{galhotra2023metam,youngmann2023causal}, data cleaning~\cite{pirhadi2024otclean,salimi2019interventional}, and large system diagnostics~\cite{markakis2024sawmill,causalsim,sage, gudmundsdottir2017demonstration}. 
We propose using causal inference to generate prescription rules that are both actionable and justifiable.
}

While generating prescriptions based on causal inference may help in robust decision-making, causal prescriptions that solely consider the betterment of an outcome (like salary) are not enough in practice. 
It is well-known that decision-making in many high-stake applications (like hiring policy, or policy for approving loans by banks) may lead to disparate societal or economic impact on different sub-populations. 
As a shocking example from a recent work called 
%For example, 
CauSumX~\cite{DBLP:journals/pacmmod/YoungmannCGR24} that generates a set of causal explanations for an aggregated view, the explanations generated %by CauSumX %recommendations which 
suggest that male individuals do a Bachelor's degree to increase their salary while %suggesting that 
being an unmarried woman 
%the recommendation for women includes getting married 
has the most adverse effect on salary
(borrowed directly 
from Fig.~19 in~\cite{youngmann2024summarizedcausalexplanationsaggregate}). 
%We demonstrate the advantage of using causal reasoning to generate actionable recommendations and the limitations of not considering fairness requirements in the following example. 
We explored this further in the context of generating prescriptions and observed that prescriptions that are not fairness-aware can generate unfair outcomes to some subpopulations which we refer to as the {\em protected group}. Examples include women, Black, Latino, or Native Americans, individuals with a disability, countries with a weaker economy, or other protected groups specific to an application. %Here is a concrete example:


% Understanding the causal factors behind these recommendations is crucial to ensuring that decisions lead to fair and equitable outcomes, particularly in sensitive applications where biased decisions can perpetuate or even exacerbate societal inequalities.
% While prior work has extensively explored techniques for association rule mining~\cite{kumbhare2014overview}, recent efforts have focused on deriving causal explanations for individual data points or entire datasets~\cite{salimi2018bias,youngmann2022explaining,ma2023xinsight}. Although some of these methods produce causally consistent insights, the absence of fairness considerations in the process can lead to unfair outcomes, further reinforcing existing biases. For example, CauSumX~\cite{DBLP:journals/pacmmod/YoungmannCGR24} generates causal recommendation which suggest male individuals to do a Bachelor's degree to increase salary while the recommendation for women include getting married (borrowed directly from Figure~19 in the paper~\cite{youngmann2024summarizedcausalexplanationsaggregate}). 





%\emph{Causal inference} has been thoroughly studied in AI and Statistics~\cite{pearl2009causal,rubin2005causal}. Causal analysis is a vital tool in determining the effect of a \emph{treatment} on an \emph{outcome}, and has been used in decision-making in medicine \cite{robins2000marginal}, economics \cite{banerjee2011poor}, biology \cite{shipley2016cause}, and in high-stakes areas such as identifying the root causes of failures in critical infrastructure systems to prevent catastrophic outcomes. Recent works have introduced causality to the field of database research~\cite{meliou2010causality,  meliou2014causality,salimi2020causal,10.14778/3554821.3554902}. It has been incorporated into various tasks including data discovery~\cite{galhotra2023metam,youngmann2023causal}, query result explanation~\cite{salimi2018bias,youngmann2022explaining,DBLP:journals/pacmmod/YoungmannCGR24}, and large system diagnostics~\cite{markakis2024sawmill,causalsim,sage, gudmundsdottir2017demonstration}. We propose leveraging causal inference to generate interpretable and justifiable insights (referred to as \emph{prescription rules}), which serve as actionable recommendations to improve an outcome of interest. Causal reasoning is considered one of the most important requirements,  to generate insights that are actionable and aligned with human reasoning.




\begin{table*}[]
\footnotesize
    \centering
    	\caption{\textnormal{A subset of the Stack Overflow dataset.}}
         \label{tab:data}
    	% \vspace{-4mm}
  			\begin{tabular}[b]{|l|l|l|c|l|l|c|l|c|}
  			
				%\multicolumn{9}{c}{\textbf{Users}}\\ 
				\hline

				\textbf{ID}
    
    % \textbf{Country}& \textbf{Continent} 
    
    &\textbf{Gender} &\textbf{Ethnicity}&
				\textbf{Age} &\textbf{Role} &
				\textbf{Education} &\textbf{Country}&\textbf{Undergrad Major}&\textbf{Salary}
				\\ \hline

				1 &Male&White&26&Data Scientist & PhD& US&Computer Science&180k\\
    		2 &Non-binary&White&32&QA developer & Bachelor's degree& US&Mechanical Eng.&83k\\

 3 &Male&South Asian&29&C-suite executive  & Bachelor's degree & India&Computer Science&24k\\

  % 4 &Female&South Asian&25&Back-end developer  & Master's degree & India&Mathematics&7.5k\\

  4 &Female&East Asian&21&Back-end developer & Bachelor's degree & China&Computer Science&19k\\
  

        % $\ldots$ &  $\ldots$&  $\ldots$&  $\ldots$&  $\ldots$&  $\ldots$&  $\ldots$&  $\ldots$&  $\ldots$&  $\ldots$&  $\ldots$\\
    \hline
			\end{tabular}
            \vspace{-5mm}
\end{table*}




\begin{example}	%
\label{example:ex2}
{\bf Importance of fair prescriptions:}
Continuing Example~\ref{example:ex1}, while those causal prescription rules are highly beneficial for the overall population, they are considerably less effective for individuals residing in countries with a low GDP (indicating a weaker economy). For this group, the average expected increase in salary is only approximately \$13,000 per year (in contrast to \$44,009 for the entire group). % \sr{add which rule 44k or 25k} 
Consequently, implementing these rules would exacerbate the disparity between those living in countries with strong economies and those in countries with weaker economies.
\end{example}




% Our objective is to generate a small set of prescription rules aimed at increasing (or decreasing) an outcome of interest. This is framed as an optimization problem where the goal is to select the fewest prescription rules that maximize utility (i.e., the expected increase or decrease in the outcome). However, 

The example above shows that focusing solely on maximizing utility (\revc{i.e., increasing income}) can result in a scenario where only some of the population receive significant improvement, while others experience no benefit (\revc{only a small benefit for individuals from countries with weaker economies in our example}). Additionally, even if a large portion of the population receives recommendations, a protected subpopulation might not share the benefits and, worse, their situation could deteriorate, exacerbating inequalities.

Examples~\ref{example:ex1} and \ref{example:ex2} show that it is crucial to provide recommendations that are (1) {\em causal} for the outcome (beyond associations),  and (2) also {\em fair or equitable} in terms of the outcome for both the protected and non-protected groups. While recent work in database research
has focused on deriving {\em causal explanations} for individual data points, aggregated view, or entire datasets~\cite{salimi2018bias,youngmann2022explaining,ma2023xinsight, DBLP:journals/pacmmod/YoungmannCGR24}, and in particular \cite{DBLP:journals/pacmmod/YoungmannCGR24} has considered generating a set of causal explanations for an aggregated view that resemble a ruleset, 
%Although some of these methods produce causally consistent insights, 
the absence of fairness considerations in generating these causal explanations can lead to unfair outcomes for the protected group.
%further reinforcing existing biases.


%\red{We, therefore, enable users to incorporate various \emph{coverage and fairness constraints} along with the overall objective of improving an outcome of interest. }

\medskip
\noindent
\textbf{Our contributions.~} 
Motivated by the dual goals of generating causal and fair prescriptions for the betterment of an outcome, we introduce a {\em fairness-aware framework leveraging causal reasoning for generating a set of actionable prescription rules (ruleset)} called \sysName\ (\underline{Fair} \underline{CA}usal \underline{P}rescription).
%
Following research on fairness in data management~\cite{stoyanovich2020responsible,galhotra2022causal}, we assume the existence of a \emph{protected subpopulation}, defined by an attribute such as gender or race for people, or GDP of a country. Motivated by the causal explanation rules for an aggregated view \cite{DBLP:journals/pacmmod/YoungmannCGR24}, each prescription rule in our ruleset applies to a sub-population defined by a {\em grouping attribute}, and prescribes a {\em treatment or intervention} to improve the {\em outcome} for this sub-population. Fairness constraints ensure that the expected utility of the protected population is {\em comparable} to the utility of the unprotected individuals. We borrow the notions of \emph{group and individual fairness} from the fairness literature but tailor them for prescription rules. In addition to the fairness constraints, our coverage constraints ensure that a substantial fraction of the population and protected subpopulation receives at least one recommendation. 
%We demonstrate how such constraints ensure that the generated rules apply to a large portion of the population and ensure fairness through the following example.

\begin{example}
\label{ex:intro_example_3}
Continuing Examples~\ref{example:ex1} and \ref{example:ex2}, Alice uses our proposed system, called \sysName, to impose fairness and coverage constraints to discover useful and equitable recommendations for increasing salaries worldwide. In particular,
Alice chooses to implement a coverage constraint to ensure that the selected rules apply to a significant portion of people worldwide, including a sufficiently large number of individuals from countries with low GDP (the protected group). She also imposes a fairness constraint to ensure that the expected gains for both protected and non-protected groups are comparable.
\reva{She discovers, for example, that for individuals with 6-8 years of coding experience (a subpopulation comprising 21\% of the entire dataset and 25\% of the protected group), pursuing a bachelor’s degree in computer science will increase the expected salary by $\$14.9k$ for protected and by $\$17.8k$ for non-protected}. (See \cref{sec:casestudy} for more details.) This prescription rule applies to a large portion of the population and ensures fairness by providing a similar expected gain for both protected and non-protected groups, and the allowed difference of outcomes between these two populations may be adjusted by choosing appropriate thresholds in the fairness definitions. 
\end{example}


\noindent
Our main contributions are as follows. \\
%\begin{itemize}[leftmargin=*,topsep=0pt]
{\bf (1)} We {\bf develop a framework that generates a set of prescription rules to enhance an outcome of interest (Section~\ref{sec:problem})}. A prescription rule consists of a \emph{grouping pattern} and an \emph{intervention pattern}, representing the target subpopulation and the actionable recommendation for that group, respectively. The strength of the {\em conditional causal effect} (Section~\ref{sec:background-causal}) of this intervention on the subgroup is used to measure the expected utility of a rule. Our objective is to identify the smallest set of rules that maximizes overall expected utility. We refer to this problem as the {\em \probName} problem.
We adopt several notions of fairness (individual vs. group, statistical parity vs. bounded group loss) from the literature to define the {\bf fairness constraints} for our problem. In addition, {\bf coverage constraints} (for individual rules or for a group) ensure that the solution for the \probName\ problem is applied to a sufficient number of individuals and to minimize inequalities. We show NP-hardness for different variants of the problems and properties (matroid) useful in our algorithms. 
%We establish several definitions for group and individual fairness constraints tailored for prescription rules.
\smallskip
    \par
    \noindent
{\bf (2)} We {\bf develop a general three-step algorithm named \sysName to solve the optimization problem of selecting a fair prescription ruleset (Section~\ref{sec:algo})}. The first step involves mining frequent grouping patterns using the Apriori algorithm~\cite{agrawal1994fast}. In the second step, we employ a lattice-based algorithm to find high utility and fair intervention patterns for grouping patterns identified in the previous step. Finally, the third step applies a greedy approach to determine a solution. \sysName\ can be easily adapted to accommodate all variants of the \probName\ problem.

\smallskip
\par
\noindent
{\bf (3) We provide a detailed  case study  (Section~\ref{sec:casestudy}) and experimental analysis (Section~\ref{sec:experiments}) to evaluate our framework and algorithms.}
The case study shows the qualitative difference of different variants of our problem for different choices of the fairness and coverage constraints. The experiments include two datasets, three baselines, and 18 variations of our problem with different constraints. Our evaluations suggest that fairness may come at the cost of expected
utility for everyone. However, without fairness constraints, we often observe a significant disparity between the protected and non-protected. We also observe that
achieving individual fairness is harder than group fairness,
as most high-utility or high-coverage rules are unfair. Lastly, we show that \sysName\ can generate  prescription rules over large datasets in a reasonable time. 

%\end{itemize}


%\paragraph*{Paper outline} 
We discuss related work in \cref{sec:related}, review background on causal inference in \Cref{sec:background-causal}, %and our problem formulation can be found in \cref{sec:problem}. Our algorithmic framework is presented in \cref{sec:algo}. A case study demonstrating the impact of different constraint configurations on the solution is given in \cref{exp:problem_variants}, and our experimental evaluation is detailed in \cref{sec:experiments}. Finally, we 
and discuss the limitations of our framework and future work in \cref{sec:conc}.

% \noindent
% \boxed{\parbox{\columnwidth}{$\bullet$ 
% For people with a professional degree, move to the United Kingdom
%  (coverage = 435 (20), coverage-protected = 20 (13), utility = 186855, utility-protected = 0.)\\
% $\bullet$ For graphic developers, move to the	United States
%  (coverage = 116 (29), coverage-protected = 8 (2), utility = 169431, utility-protected = 0).\\
% $\bullet$ For people who have no formal education, move to the United States
%  (coverage = 123 (34), coverage-protected = 7 (2), utility = 206742, utility-protected = 0).\\
% % \textcolor{red}{size = 38, length = 76, overlap = 64029181, utility = 1659307}\\
% \textcolor{blue}{overall coverage =674, expected utility = 187485
% coverage-protected = 35, expected utility-protected = 0}
% \sr{should mention protected group, and possibly not mention coverage in the intro or just intuitively like high coverage}
% }}


% Alice notes that although these rules result in a \$187,485 increase in the overall salary for those to whom they apply, they only affect a small fraction of the population, specifically 674 individuals. Additionally, although the expected salary increase is substantial, there is no expected increase in salary for non-males, a subpopulation of particular interest to Alice. In other words, applying these rules would result in no gain for non-males.
% \end{example}

% \begin{example}[Episode 2 - coverage and fairness constraints]
% Alice introduces coverage and fairness constraints to ensure that enough people will benefit from the rules and that they will be \emph{fair} with respect to non-males. Specifically, she demands that the benefit for a randomly chosen individual to whom one of the rules applies is nearly the same as the benefit for a randomly chosen individual who identifies as non-male and to whom one of the rules applies.

% After adding these constraints, \sysName\ recommends the following set of prescription rules:



% \noindent
% \boxed{\parbox{\columnwidth}{$\bullet$ 
% For people who have no formal education, move to the United States
%  (coverage = 123 (34), coverage-protected = 7 (2), utility = 206742, utility-protected = 0)\\
% $\bullet$ 
% For females, change role to	DevOps specialist (coverage = 2256 (47), coverage-protected = 2256 (47), utility = 90023, utility-protected = 90023).\\
% $\bullet$ For people with a Master's degree, move to the	United States
%  (coverage = 9097 (2222), coverage-protected = 642 (236), utility = 85390, utility-protected = 84201).\\
% % \textcolor{red}{size = 38, length = 76, overlap = 64029181, utility = 1659307}\\
% \textcolor{blue}{overall coverage =11476	
% , expected utility = 87601,
% coverage-protected = 2905, expected utility-protected = 88519}
% }} 







% \begin{figure}[t]
%         \centering
%         \begin{minipage}[b]{1.0\linewidth}
%             \small
%             \begin{tcolorbox}[colback=white]
%             \vspace{-2mm}
% $\bullet$ For backend developers, the treatment with the highest effect on salary is “Country = US” effect size = 78646
% \begin{itemize}
%     \item For non-male the effect is only: 59429
%     \item For male the effect is 80454
% \end{itemize}

% $\bullet$ For frontend developers, the treatment with the highest effect is :Formal Education = Bachelor's degree” effect size: 17340
% \begin{itemize}
%     \item For white the effect is 33464
%     \item For non-white the effect is 15320
% \end{itemize}


% $\bullet$ For people in Europe, the treatment with the highest effect on salary is “DevType = C-suite executive” effect size = 53254
% \begin{itemize}
%     \item For white the effect is 55112
%     \item For non-white 35249
% \end{itemize}



%             \vspace{-2mm}
%             \end{tcolorbox}
%         \end{minipage}%%
%          % \vspace{-4mm}
%         \caption{Set of prescription rules.}
%         \label{fig:so-explanation}
%     \end{figure}


\section{Methods}\label{sec:methods}

In this section we discuss several methods that have an impact on the overall performance of a MINLP solver such as \texttt{Gurobi}, when applied to optimization problems of type \eqref{prob:embedded}.
%
We start by introducing two measures of complexity in this context in Section~\ref{subsec:measures}, namely the number of regions partitioning the input domain in which the function $h(x)$ has identical linear output behavior, and the number of stable ReLU neurons.

Then we examine methods that are applicable to trained ANNs. In Section~\ref{subsec:boundtightening} bound tightening approaches for the optimization problem are presented. In Section~\ref{sec:scaling} we propose a novel scaling method that improves the $\ell^1$ regularization term of a pre-trained network without changing its encoded function. This method can be used after completed training of the ANN (a posteriori) and before the optimization is started (a priori).

In Sections~\ref{sec:regularization}, \ref{sec:clipped}, and \ref{sec:dropout} we investigate modifications to the training of the ANN, in particular regularization of training weights, clipped ReLU formulations, and the use of dropout during training.


%The goal of this paper is to examine and expand upon some of the acceleration methods enumerated in the previous section. 
%As the big-M formulation \eqref{eq:bigM} is used in the majority of publications investigating optimization problems with embedded ReLU neural networks, we restrict ourselves to this formulation. 
%Regarding the methods proposed for accelerating optimization algorithms, we particularly focus on the effects of bound tightening and training with regularization. We also propose a simple scaling method that improves the $\ell^1$ regularization term of a pre-trained network without changing its encoded function. Apart from these acceleration methods, we also consider dropout as it is a popular method applied during training. Before analyzing these methods in more detail, we introduce two important characteristics of ReLU networks which are an indicator of the complexity of optimization problems with embedded neural networks. 

\subsection{Measures of complexity of ReLU ANNs}\label{subsec:measures}

While the solution of mixed-integer optimization problems is difficult ($\mathcal{NP}$-complete) in general, it is well known that the number of optimization variables and the tightness of relaxations of the integer variables have a major impact on computational runtimes. In the context of embedded ANNs, we shall consider two particular indicators of complexity.
%From the point of view of optimization, several aspects complicate optimizing over neural networks. These include large number of variables and weak relaxations, among others. In this section, we introduce properties of neural networks which are indicators of the complexity, which we use later in our numerical study.

\subsubsection{Number of linear regions of ReLU networks}

ReLU ANN describe piecewise affine-linear functions \citep{Grigsby2022}. Therefore, the network partitions the input domain $\mathcal{X} \subseteq \mathbb{R}^{n_x}$ into regions in which $h(x)$ is affine linear. These regions are typically called \textit{linear regions}. The bounds on the number of linear regions of a neural network with given depth and width was investigated in \citet{Montufar2014} and later improved on in \citet{Raghu2017}. In general, the number of linear regions of a neural network corresponds to the number of feasible activation patterns in \eqref{eq:bigM}, i.e., the binary decisions whether a neuron is on or off for all neurons in the neural network. Thus, it is an important statistic when considering the complexity of optimizing over neural networks, e.g., in branch-and-bound frameworks, where the variables to branch on represent active or inactive neurons. 

\subsubsection{Number of stable ReLU neurons}

The number of variables a branch-and-bound method has to branch on is an important statistic for estimating the complexity of the optimization problem. In ReLU networks, the variables to branch on are the binary variables $z$ in \eqref{eq:bigM} of every neuron in the network. However, if a neuron can be identified as stable, no binary variable has to be added to model the neuron. To identify stable neurons, their pre-activation bounds are used. The neuron $i$ in layer $j$ is called stably active if $L^{(j)}_i > 0$ and stably inactive if $U^{(j)}_i < 0$, for $j \in [J]$ and $i \in [n_j]$ for all inputs in the input domain $\mathcal{X} \subseteq \mathbb{R}^{n_x}$. 

A regularization to induce ReLU stability was proposed in \citet{Xiao2019} to speed up verification of ReLU networks, whereas in \citet{Serra2020} stable neurons are used to compress neural networks. 
To enumerate the linear regions of a ReLU ANN, we exploit the fact that, within a given linear region, the input and output of each neuron is an affine linear functional in the ANN's overall input space. We use forward sensitivity propagation to calculate the gradient of each neuron's regional input functional and simultaneously perform a forward evaluation of the linearized ANN at the input space's coordinate origin to determine each affine input functional's output shift. With both gradient and shift, we can determine a hyperplane in input space along which the neuron's ReLU activation would switch. We then construct a linear equation system that describes the intersection of halfspaces within which all neurons would retain their current activation pattern. We add the bounds of the input domain to this equation system to ensure boundedness of the linear region. We then use a variant of the \texttt{QuickHull} algorithm~\citep{Barber1996} via the \texttt{SciPy} library~\citep{SciPy2020} to reduce this equation system into an irredundant one and to determine the vertices of the linear region. This also reveals information on which neurons define the facets of the linear region, which means that we can jump across these facets to adjacent regions by switching the activity of those neuron's activation functions. Assuming that there is no facet along which two neurons switch simultaneously, this allows us to enumerate all linear regions that intersect the input domain. We can detect the edge case of two neurons switching simultaneously because it would cause us to enter a region with an empty interior. We do not observe this behavior. 

\subsection{bound tightening} \label{subsec:boundtightening}

Calibrating the big-M coefficients in MILP formulations is crucial for performance of optimization algorithms. bound tightening plays an important role in this context. With ReLU ANNs, there are different ways to compute the big-M coefficients.
%
\subsubsection{Interval arithmetic}\label{subsec:ia_bounds}
%
In the presence of input bounds $L^{(0)} \leq x^{(0)} \leq U^{(0)}$ with $L^{(0)}, U^{(0)} \in \mathbb{R}^{n_x}$, big-M coefficients of formulation \eqref{eq:bigM} can be computed via interval arithmetic (IA).
%
\begin{align}
    L^{(k)}_i &= \sum_{j=1}^{n_{k-1}} \min \{ W_{i,j}^{(k)} L_j^{(k-1)}, W_{i,j}^{(k)} U_j^{(k-1)} \} + b^{(k)}_i, \quad k \in [J],  i \in [n_k] \\
    U^{(k)}_i &= \sum_{j=1}^{n_{k-1}} \max \{ W_{i,j}^{(k)} L_j^{(k-1)}, W_{i,j}^{(k)} U_j^{(k-1)} \} + b^{(k)}_i, \quad k \in [J],  i \in [n_k] 
\end{align}
%
This forward propagation yields valid bounds. However, it ignores the fact that the activation of neurons, i.e., whether they are on the left or right arm of the ReLU function, is not independent between neurons. This results in overly relaxed approximations of the actual bounds. As a result, there is typically an exponential increase of big-M coefficients with increasing depth. This behaviour is exemplified in \vref{fig:IA_bounds}.


\subsubsection{LP-based bound tightening}\label{subsec:lr_bounds}

The bounds from \cref{subsec:ia_bounds} can be tightened by taking advantage of dependencies between the neurons as well as potentially existing bounds on the output of the neural network $L^{(J)} \leq x^{(J)} \leq U^{(J)}$. This is achieved by solving two auxiliary optimization problems per neuron, minimizing and maximizing, respectively, the pre-activation value of each neuron. The optimization problem for computing tighter bounds for neuron $k$ in layer $j$, with $j \in [J],\,k \in [n_k]$ in its general form as an MILP reads

\begin{equation}\label{prob:obbt}
    \begin{aligned}
    \min_{x,z}\ & W^{(j)}_k x^{(j-1)} + b^{(j)}_k \\ 
    \textrm{s.t.}\ & \begin{alignedat}[t]{3}
            x^{(j)}_i & \geq 0, \quad && j \in [J],\, i \in [n_j] \\
            x^{(j)}_i & \geq W^{(j)}_i x^{(j-1)} + b^{(j)}_i, \quad && j \in [J],\, i \in [n_j] \\
            x^{(j)}_i & \leq W^{(j)}_i x^{(j-1)} + b^{(j)}_i - L^{(j)}_i (1-z^{(j)}_i), \quad && j \in [J],\, i \in [n_j] \\
            x^{(j)}_i & \leq U^{(j)}_i z^{(j)}_i, \quad && j \in [J],\, i \in [n_j] \\
            x^{(0)}_i  & \leq U^{(0)}_i,        \quad && i \in [n_x]\\
            x^{(0)}_i  & \geq L^{(0)}_i,        \quad && i \in [n_x]\\
            z^{(j)}_i & \in \{0,1\}. \quad && j \in [J],\, i \in [n_j]
        \end{alignedat}
    \end{aligned}
\end{equation}
Solving \eqref{prob:obbt} yields a valid lower bound $L^{(j)}_k$, while the corresponding upper bound $U^{(j)}_k$ is computed by maximizing instead of minimizing in \eqref{prob:obbt}. In order to reduce the computational effort, typically the LP relaxation of formulation \eqref{prob:obbt} is considered. Hence, the auxiliary problems are linear programs (LPs) and can be solved efficiently. Solving the MILP directly is considered in \citet{Badilla2023,Grimstad2019}. However, the reduction in computational effort in subsequent optimization is quickly outweighed by the effort spent on solving the bound tightening MILPs. Therefore, we only consider the LP-based bound tightening procedure in this paper. One degree of freedom when performing bound tightening is the ordering of variables for which bounds are tightened. As the direction of bound propagation is from the input to the output layer, this is also the natural order to perform the tightening. However, within each layer the order may be chosen arbitrarily. Different methods to choose this order are discussed in \citet{Rossig2021}. However, they do not find any advantage of more advanced methods over a simple, fixed ordering of variables. Therefore, in this contribution, we apply bound tightening in a fixed ordering of variables.% The effects of LP-based bound tightening on a ReLU network with ten hidden layers is illustrated in \vref{fig:big_Ms}.

\subsection{A posteriori scaling of ReLU ANNs} \label{sec:scaling}
Weights of neural networks are not uniquely determined by the training process and the training data, i.e., there are different realizations of weights and biases that define the same functional relationship of input and output. This observation can be exploited to design algorithms that transform a trained neural network into a functionally identical network with some desired property. This could be, e.g., a lower norm of the weight matrices. With the input bounds remaining unchanged, this would lead to a reduction of big-M coefficients, which could be beneficial in subsequent optimization problems.

In case of the ReLU activation function, one can exploit its positive homogeneity. For a single neuron $i$ in layer $k$, with $k \in [J], \ i \in [n_k]$ and a scalar $c^{(k)}_i > 0$ it holds, that
\begin{align}
    \myReLU{c^{(k)}_i \left( W^{(k)}_i x^{(k-1)} + b_i \right)} = c^{(k)}_i \cdot \myReLU{ W^{(k)}_i x^{(k-1)} + b_i},
\end{align}
with $W^{(k)}_i \in \mathbb{R}^{1 \times n_{k-1}}$ being the $i$-th row of the weight matrix in layer $k$. 
%
To ensure the functional equivalence of the neural network, the $i$-th column of the weight matrix of layer $k+1$, corresponding to the scaled neuron $i$ in layer $k$, needs to be multiplied with the reciprocal of $c^{(k)}_i$. As all neurons of the neural network may be scaled, all weight matrices except the first and the last are scaled with the ratio of the two scaling factors of their surrounding layers. As the bias is not multiplied with the output from the previous layer, no multiplication with the reciprocal is needed. In the final layer $J$, no more new scaling factors may be introduced as they can no longer be compensated in subsequent layers. Therefore, only the scaling of layer $J-1$ is compensated by multiplying $W^{(J)}$ with the reciprocals of the scaling factors of the penultimate layer. For any set of scaling factors $c^{(k)}_i > 0,\,k \in [J], i \in [n_k]$, scaled weights and biases  $\tilde{W}$ and $\tilde b$, computed as 
\begin{equation}
    \begin{alignedat}{3}
        \tilde{W}_{i,j}^{(1)} &= W_{i,j}^{(1)} \cdot c_i^{(1)}, \quad && i \in [n_1],\, j \in [n_x] \\       
        \tilde{W}_{i,j}^{(k)} &= W_{i,j}^{(k)} \cdot \frac{c_i^{(k)}}{c_j^{(k-1)}}, \quad && k \in \{2, \ldots, J-1\},\, i \in [n_k],\, j \in [n_{k-1}],\\
        \tilde{W}_{i,j}^{(J)} &= W_{i,j}^{(J)} \cdot \frac{1}{c_j^{(J-1)}}, \quad && i \in [n_J],\, j \in [n_{J-1}] \\
        \tilde{b}_i^{(k)} &= b_i^{(k)} \cdot c_i^{(k)}, \quad && k \in [J-1], \, i \in [n_k]
    \end{alignedat}
\end{equation}
define functionally equivalent neural networks.
This basic idea of an equivalent transformation of ReLU networks via scaling one layer and compensating the effect of scaling in the next layer is illustrated in \vref{fig:rescale}. 

\begin{figure}
    \centering
    \includegraphics[width=.6\linewidth]{figures/hahnRescale.png}
    \caption{Equivalent scaling of ReLU ANNs. Scalar factor $c$ is multiplied row-wise to weight matrix and corresponding bias of current layer, resulting in a scaling of the output of the neuron by a factor of $c$. To compensate this, the weight matrix in the subsequent layer needs to be multiplied column-wise with the reciprocal of $c$.}
    \label{fig:rescale}
\end{figure}
%
The scaling factors $c_i^{(k)}$ can be chosen arbitrarily. However, we can specifically choose them such that the resulting network has favorable properties. We propose formulating an optimization problem to obtain scaling factors that minimize the absolute value of the scaled weights $\tilde{W}$ and biases $\tilde{b}$. This lower norm of the weights then yields lower big-M coefficients. As noted in \cref{subsec:ia_bounds}, these are determined solely\footnote{While bounds on the output of the network can be propagated backwards through the network and thus influence the big-M coefficients \citep{Grimstad2019}, we refer only to the big-M coefficients derived via interval arithmetic and forward propagation of input bounds as explained in \cref{subsec:ia_bounds}.} by the input bounds and the magnitude of weights and biases. Hence, this approach can be applied to networks that were not initially trained with regularization in order to generate an equivalent neural network with lower big-M coefficients. Of course, other effects of regularization, e.g., weight sparsity, cannot be obtained by this method. The proposed optimization problem is
%
\begin{equation}
    \begin{aligned}
        \min_{c}\ & \sum_{i=1}^{n_1} \sum_{j=1}^{n_x} \lvert W_{i,j}^{(1)} \rvert \cdot c_i^{(1)} +  \sum_{k=2}^{J-1} \sum_{i=1}^{n_k} \sum_{j=1}^{n_{k-1}}  \lvert W_{i,j}^{(k)} \rvert \cdot \frac{c_i^{(k)}}{c_j^{(k-1)}} \\*
        & + \sum_{k=1}^{J-1} \sum_{i=1}^{n_k} \lvert b_i^{(k)} \rvert \cdot c_i^{(k)} + \sum_{i=1}^{n_J} \sum_{j=1}^{n_{J-1}} \lvert W_{i,j}^{(J)} \rvert \cdot \frac{1}{c_i^{(J)}} \\
        \textrm{s.t.}\ & \begin{alignedat}[t]{3}
            c_i^{(k)} & > 0, \quad && k \in [J],\, i \in [n_k] \\
            c & \in \bigtimes_{k=1}^J \mathbb{R}^{n_k} \quad
        \end{alignedat}
    \end{aligned}
\end{equation}
%
This optimization problem is not trivial to solve directly because it involves fractions and strict inequality constraints. However, because all $c_i^{(k)}$ have to be strictly positive, we can convert it into a convex optimization problem on a closed set by replacing each $c_i^{(k)}$ with its logarithm. Each summand in the objective function then becomes an evaluation of the exponential function, multiplication becomes addition, and division becomes subtraction. With the logarithm of $c^{(k)}_i$  referred to as $\tilde{c}^{(k)}_i$, the transformed optimization problem reads
%
\begin{equation}\label{prob:scaling}
    \begin{aligned}
        \min_{\tilde{c}}\ & \sum_{i=1}^{n_1} \sum_{j=1}^{n_x} \exp \left( \log \left( \lvert W_{i,j}^{(1)} \rvert\right) + \tilde{c}_i^{(1)} \right)   + \sum_{k=2}^{J} \sum_{i=1}^{n_k} \sum_{j=1}^{n_{k-1}} \exp \left( \log \left( \lvert W_{i,j}^{(k)} \rvert \right)+ \tilde{c}_i^{(k)} - \tilde{c}_j^{(k-1)} \right)\\*
        & + \displaystyle \sum_{k=1}^{J} \sum_{i=1}^{n_k} \exp \left( \log \left( \lvert b_i^{(k)} \rvert \right) +  \tilde{c}_i^{(k)} \right) + \sum_{i=1}^{n_J} \sum_{j=1}^{n_{J-1}}  \exp \left( \log\left(\lvert W_{i,j}^{(J)} \rvert \right) - \tilde{c}_i^{(J)} \right) \\
        \textrm{s.t.}\ & \tilde{c} \in \bigtimes_{k=1}^J \mathbb{R}^{n_k}
    \end{aligned}
\end{equation}
%


\subsection{Regularization} \label{sec:regularization}

The objective function for training neural networks typically consists of two terms. The first 
accounts for the mismatch between prediction and data, while the second term aims at preventing overfitting and thus allowing for a better generalization of the model to unseen data. With $W \in \mathbb{R}^d$ denoting the vector of all weights and biases and $N \in  \mathbb{N}$ representing the number of training samples of inputs and outputs $(x_i, y_i), \, i \in [N]$, the objective reads
%
\begin{equation}\label{prob:training}
    \begin{aligned}
        \min_{W}\ & \frac{1}{N} \sum_{i=1}^{N} \left( h(x_i) -y_i\right)^2 + \lambda \Omega(W)
    \end{aligned}
\end{equation}
%
Popular choices for the regularization term $\Omega : \mathbb{R}^d \mapsto \mathbb{R}$ are the penalization of large magnitudes of weights and biases by using some vector norm, e.g., $\Omega(W) = \| W \|_p$, with typically $p=1$ and $p=2$. Typical ways to measure the generalization performance of a model is to compute the mean absolute percentage error (MAPE) defined as 
\begin{align}
    \text{MAPE}\left(\hat{y}, y\right) = \frac{1}{n} \sum_{i=1}^{n} \frac{|\hat{y}_i - y_i|}{\max\{\varepsilon, |y_i|\}}
\end{align}
for predictions $\hat{y}$ on the test dataset.

While it is known that $\ell^1$ regularization leads to sparser regression models \citep{Tibshirani1996}, \citet{Xiao2019,Serra2020} found that applying $\ell^1$ regularization also increased ReLU stability, i.e., the percentage of stable neurons. The authors of \citet{Xiao2019} also propose a dedicated ReLU stability regularization \eqref{eq:stability_regularization}, which penalizes the sign differences in the pre-activation bounds of each neuron, thus encouraging stability.
\begin{align}\label{eq:stability_regularization}
    \Omega_{\text{RS}}(W) = - \sum_{i=1}^{J} \sum_{j=1}^{n_i} \text{sign}(U_j^{(i)}) \cdot \text{sign}(L_j^{(i)})
\end{align}
For practical purposes, a smooth reformulation of \eqref{eq:stability_regularization} is used, and \citet{Xiao2019} show that verification problems of neural networks trained using this regularization can be solved faster than with $\ell^1$ regularization due to a higher number of stable neurons. In this paper, we will however focus on investigating the effect of varying levels of $\ell^1$ regularization on the performance of optimization algorithms, as it is one of the most commonly used types of regularization.

\subsection{Clipped ReLU} \label{sec:clipped}
%
One of the reasons why big-M coefficients in ReLU networks increase quickly with increasing network depth is that the ReLU activation function is unbounded. A variation of the ReLU function is the clipped ReLU function proposed in \citet{Hannun2014}. In the clipped ReLU function, the output of the function is bounded by an upper value $M \in \mathbb{R}$, i.e.,
\begin{align}\label{eq:relu_m}
    \myClippedReLU[M]{x} = \max\bigl\{0,\, \min\{M,\, x\}\bigr\}
\end{align}
%
Using standard disjunctive programming notation, the feasible set of $x_i^{(j)} = \myClippedReLU[M]{ W^{(j)}_i x^{(j-1)} + b_i}$ can be written as
%
\[
\begin{bmatrix}
    x_i^{(j)} = 0 \\
    W^{(j)}_i x^{(j-1)} + b_i \leq 0
\end{bmatrix}
\vee 
\begin{bmatrix}
    x_i^{(j)} = W^{(j)}_i x^{(j-1)} + b_i\\
    0 < W^{(j)}_i x^{(j-1)} + b_i < M 
\end{bmatrix}
\vee
\begin{bmatrix}
    x_i^{(j)} = M \\
    W^{(j)}_i x^{(j-1)} + b_i \geq M 
\end{bmatrix}
\]
%
We formulate a big-M relaxation of this feasible set as
\begin{align}
    \begin{split}
        x_i^{(j)} & \geq 0, \\
        x_i^{(j)} & \leq M z_{1_i}^{(j)}, \\
        x_i^{(j)} & \leq U_i^{(j)} z_{1_i}^{(j)}, \\
        x_i^{(j)} & \leq W^{(j)}_i x^{(j-1)} + b_i - L_i^{(j)} \cdot (1-z_{1_i}^{(j)}),\\
        x_i^{(j)} & \geq M z_{2_i}^{(j)}, \\
        x_i^{(j)} & \geq W^{(j)}_i x^{(j-1)} + b_i - (U_i^{(j)}-M) z_{2_i}^{(j)}, \\
        z_{1_i}^{(j)}, z_{2_i}^{(j)} & \in \{0,1\},\\
        z_{1_i}^{(j)} & \geq z_{2_i}^{(j)},
    \end{split}
    \label{eq:bigM_clipped}
\end{align}
similar to the formulation suggested in a preprint version of \citet{Anderson2020}.
This formulation comes at the cost of an additional binary variable compared to the standard big-M formulation \eqref{eq:bigM}. If both binary variables are zero, the neuron is inactive and $x_i^{(j)}=0$. In the case $z_{1_i}^{(j)}=1, z_{2_i}^{(j)}=0$, the neuron is active and $0 \leq x_i^{(j)} =  W^{(j)}_i x^{(j-1)} + b_i \leq M$. If both binary variables are non-zero, the neuron's output is limited by the threshold $M$. 

\subsection{Dropout} \label{sec:dropout}

Dropout is a technique applied during training proposed in \citet{Srivastava2014} to prevent overfitting the data by randomly turning off a percentage of the neurons in some or all layers. Therefore, redundancies have to be established in the neural network to achieve an adequate accuracy. There is empirical evidence that neural networks trained with dropout have more linear regions \citep{Zhang2020a} than those trained without. Hence, in contrast to the aforementioned methods, it is expected that applying dropout during training leads to more complex neural networks which makes optimizing over them more difficult.
We will thus apply dropout as an antithesis to validate our conjecture that the runtime of MINLP solvers increases for more redundant and decreases for less redundant ANN models.


\section{Numerical results}\label{sec:results}

In the \cref{sec:methods}, we have enumerated some methods to formulate, train, and scale feed-forward neural networks with ReLU activation (or variations thereof), as well as to tighten their relaxation prior to optimization through bound tightening. In this section, we evaluate how these methods affect global optimization performance. In order to do so, we train neural networks as surrogates for several non-convex benchmark functions and compare solver performance with various post-processing steps.

We first present numerical results on relevant characteristics of ReLU ANNs in the context of optimization. These include their expressive power as measured by the number of linear regions they define and the percentage of stable neurons that can be determined from the pre-activation bounds, introduced in the beginning of \cref{sec:methods}.  We count only those linear regions that intersect the relevant input domain of each function.

We show how the methods presented in \cref{sec:methods} impact these quantities and improve the performance of optimization algorithms. For this, we restrict ourselves to minimizing the output of feed-forward ReLU ANNs, i.e., the optimization problem we solve reads
\begin{equation}\label{prob:minANN}
    \min_{x} \ h(x)
\end{equation}
where $h \colon \mathbb{R}^{{n_x}} \mapsto \mathbb{R}$ is the trained neural network. 
The benchmark functions we consider for approximation and subsequent minimization are:

\begin{figure}[t]
    \centering
    \begin{subfigure}{.32\linewidth}
        \includegraphics[width=\linewidth]{figures/surf_peak.png}
        \subcaption{Peaks function \eqref{fun:peaks}}\label{fig:peaks-function}
    \end{subfigure}
    \begin{subfigure}{.32\linewidth}
        \includegraphics[width=\linewidth]{figures/surf_ack.png}
        \subcaption{Ackley's function \eqref{fun:ack}}\label{fig:ack-function}
    \end{subfigure}
    \begin{subfigure}{.32\linewidth}
        \includegraphics[width=\linewidth]{figures/surf_him.png}
        \subcaption{Himmelblau's function \eqref{fun:him}}\label{fig:him-function}
    \end{subfigure}
    \caption{Surface plots of the benchmark functions for surrogate model training and optimization.}
    \label{fig:different_functions}
\end{figure}


\begin{enumerate}
    \item The Peaks function $f_{\text{peaks}} \colon \mathbb{R}^2 \mapsto \mathbb{R}$ is given by
    \begin{align}
        \begin{split}
        f_{\text{peaks}}(x,y) &= - 3 (1-x)^2 \exp\bigl(-x^2 - (y+1)^2\bigr)   - 10 \Bigl( \frac{x}{5} - x^3 - y^5\Bigr) \exp(-x^2 - y^2) \\
            & \quad - \frac{1}{3}  \exp\bigl(-(x+1)^2 - y^2\bigr).
        \end{split}
        \label{fun:peaks}
    \end{align}
    It is commonly used as a benchmark function, e.g., in \citet{Schweidtmann2019a} and has multiple local minima and maxima on the domain $x,y \in [-2,2]$. The global minimum is $(0.228, -1.626)$ with objective value $-6.551$. The function is depicted in \vref{fig:peaks-function}.

    \item Ackley's function $f_{\text{ackley}} \colon \mathbb{R}^2 \mapsto \mathbb{R}$ is defined by
    \begin{align}
        \begin{split}
            f_{\text{ackley}}(x,y) & = -20 \cdot \exp \left(-\frac{1}{5}\sqrt{\frac{1}{2}(x^2+y^2)}\right) \\
                            & \quad  - \exp \left(\frac{1}{2} \bigl(\cos(2\pi x) + \cos(2\pi y)\bigr) \right) + \exp(1) + 20    
        \end{split}
        \label{fun:ack}
    \end{align}
    and is often used as a benchmark function for optimization algorithms. For instance, it is used in \citet{Tsay2021}. It is considered on the domain $x,y \in [-3.5,3.5]$. It is non-convex, has several local minima and one global minimum at $x=y=0$ with objective value $0$. A surface plot is depicted in \vref{fig:ack-function}.

    \item Himmelblau's function $f_{\text{himmelblau}} \colon \mathbb{R}^2 \mapsto \mathbb{R}$ with
    \begin{align}\label{fun:him}
        f_{\text{himmelblau}}(x,y) =  (x^2 + y - 11)^2 + (x + y^2 - 7)^2
    \end{align}
    is considered on the domain $x,y \in [-5,5]$, where it has four equivalent local (and global) minima: $(3.0,2.0)$, $(-2.805, 3.131)$, $(-3.779, -3.283)$ and $(3.584,-1.848)$. All have objective function value $0$. A surface plot is depicted in \vref{fig:him-function}.
\end{enumerate}

In our numerical study, we consider a total of 1080 different neural networks. This number of instances stems from considering the three benchmark functions used for the approximation of \cref{fun:peaks,fun:ack,fun:him} and the specific choices for the hyperparameters of the trained neural networks. These differ both in their width and depth, as well as the activation function and the level of $\ell^1$ regularization applied during training. The specific options for these hyperparameters are given in \Cref{table:hyperparameter}. Using Latin Hypercube sampling, we generated training data of 100,000 samples for the Peaks function \eqref{fun:peaks} and Himmelblau's function \eqref{fun:him}, and 150,000 samples for Ackley's function \eqref{fun:ack}, to account for its higher nonconvexity. For training, we first normalize both input and output data, and reserve 30\% of the data as a test set to evaluate the generalization of the networks. All networks are then trained for 300 epochs using the Adam algorithm \citep{Kingma2017}. To study the effect of scaling and bound tightening on each of the trained networks, we solve problems \eqref{prob:scaling} and \eqref{prob:obbt}, where applicable. As the scaling method is not designed for the clipped ReLU, we can only solve \eqref{prob:scaling} for the 360 instances with standard ReLU activation. We use \texttt{OMLT} \citep{Ceccon2022} to set up the constraints for the ReLU ANNs via \texttt{Pyomo}~\citep{Bynum2021,Hart2011}, and \texttt{Gurobi}~\citep{gurobi}~{v11.0.1} with default options and a time limit of 300 seconds to solve the resulting optimization problems. 
%
\begin{figure}
    \centering
    \begin{subfigure}{0.49\textwidth}
        \includegraphics[width=\textwidth]{figures/big_M_unscaled}
        \caption{Big-M coefficients $U^{(k)}$ determined via IA for standard ReLU ANN.}
        \label{fig:IA_bounds}
    \end{subfigure}
    \hfill
    \begin{subfigure}{0.49\textwidth}
        \includegraphics[width=\textwidth]{figures/big_M_unscaled_OBBT}
        \caption{Big-M coefficients $U^{(k)}$ determined via OBBT for standard ReLU ANN.}
        \label{fig:LR_bounds}
    \end{subfigure}
    
    \begin{subfigure}{0.49\textwidth}
        \includegraphics[width=\textwidth]{figures/big_M_scaled}
        \caption{Big-M coefficients $U^{(k)}$ determined via IA for ReLU ANN after ReLU scaling.}
        \label{fig:scaled_IA_bounds}
    \end{subfigure}
    \hfill
    \begin{subfigure}{0.49\textwidth}
        \includegraphics[width=\textwidth]{figures/big_M_scaled_OBBT.pdf}
        \caption{Big-M coefficients $U^{(k)}$ determined via OBBT for ReLU ANN after ReLU scaling.}
        \label{fig:scaled_LR_bounds}
    \end{subfigure}
            
    \caption{Comparison of pre-activation bounds  $U^{(k)}$ for functionally equivalent neural networks with ten hidden layers. The original bounds derived via interval arithmetic shown in \ref{fig:IA_bounds} are characterized by the typical exponential increase due to forward propagation of the input bounds. Solving auxiliary LPs yields tighter bounds, although the exponential increase is still present, as shown in \ref{fig:LR_bounds}. Comparable bounds can be computed via solving the scaling problem \eqref{prob:scaling}, with the distinction that the bounds on the output of the network are equivalent to those derived from interval arithmetic. For the scaled neural network, solving the bound tightening problem \eqref{prob:obbt} in addition yields even tighter bounds on the big-M coefficients in the hidden layers with ReLU activation, as can be seen in \ref{fig:scaled_LR_bounds}, while the output bounds are equivalent to those in \ref{fig:LR_bounds}.}
    \label{fig:big_Ms}
\end{figure}


\begin{table}[ht]
    \centering
    \caption{Hyperparameter options for training of neural networks. Besides varying the depth and width of the networks, we investigate two variants of the clipped ReLU activation \eqref{eq:relu_m} and five levels of $\ell^1$ regularization. All hidden layers have the same dimension.}
    \label{table:hyperparameter}
    \begin{tabular}{l c } 
        \toprule 
        Hyperparameter & Options \\ \midrule
        Hidden Layers & $1,\ldots,10$  \\
        Layer Width &  $25,50$ \\
        Activation & ReLU, ReLU$_2$, ReLU$_5$ \\ 
        $\lambda$ &  $0.0, 10^{-7}, 10^{-6},10^{-5}, 10^{-4}, 10^{-3}$\\ \bottomrule 
    \end{tabular}
\end{table}

\subsection{Effect of OBBT}

For the 1080 trained neural networks, we solve the LP-relaxation of \eqref{prob:obbt} to compute tighter big-M coefficients for formulation \eqref{eq:bigM}, and use them in the optimization problem \eqref{prob:minANN}. The effect on a network with ten hidden layers is illustrated in \vref{fig:big_Ms}. Compared to the IA bounds, there is a reduction in big-M coefficients of the last layer by roughly two orders of magnitude. As \vref{tab:results} shows, OBBT is effective for all trained networks. We assess the reduction in big-M coefficients across all networks by comparing the averaged distances between upper and lower bound $U^{(j)}_k -L^{(j)}_k$ for bounds based on OBBT and IA. Then, over all networks, we calculate the geometric mean over the ratio of these averages. The resulting geometric mean of 0.54 suggests, that, as a rough estimate, OBBT is reducing the big-M coefficients by half. 
As a side effect of these tighter bounds, the percentage of stable neurons increases by 5.5 percent on average. We assess the resulting improvement in computational times by calculating the ratios of the measured  computational times with tightened bounds and those with the original bounds, restricted to instances that were solved to global optimality in both cases. Over these ratios, we again form the geometric mean. With a geometric mean of 0.57, bound tightening brings a significant computational speedup, though it does not substantially increase the number of instances that are solved within the time limit. \vref{fig:improvement_obbt} illustrates the parities of percentage of stable neurons and computational time.


\subsection{Effect of ReLU scaling}

\Vref{fig:big_Ms} illustrates the effects of solving the scaling problem \eqref{prob:scaling} on the big-M coefficients of a neural network with ten hidden layers. The first observation is that the output bounds remain unchanged compared to the original neural network, which is expected as the functional relationship is equivalent. However, the lower $\ell^1$ norm of the weights leads to a reduction in the big-M coefficients for the hidden layers. They are roughly on the same order of magnitude as those obtained via LP-based bound tightening. When both scaling and bound tightening are applied sequentially, the bounds for the hidden layers are tighter than those achieved by OBBT on its own. Also, with the sequential application of scaling and tightening we do not observe any clear sign of an exponential increase in bounds with increasing depth.

Using the big-M formulation with standard bounds obtained via IA as a baseline, we compare the following options:
\begin{enumerate}
    \item ReLU scaling only: We solve Problem~\eqref{prob:scaling} to obtain equivalent weights and biases with lower $\ell^1$ norm;
    \item ReLU scaling and subsequent LP-based bound tightening: a combination of the two methods.
\end{enumerate}
As shown in \vref{tab:results}, ReLU scaling on its own, as well as combined with OBBT, is able to reduce the big-M coefficients more than applying OBBT on an unscaled network. This is clearly illustrated by the geometric means over the ratios of averaged distances of upper and lower bounds $L$ and $U$ of 0.388 and 0.16 for ReLU scaling and ReLU scaling combined with OBBT compared to unscaled networks, respectively. 
Again, we compute the improvement in computational times as a geometric mean over the ratios of computational times with improved bounds and those with interval arithmetic bounds. We observe that scaling the neural network weights by solving \eqref{prob:scaling} yields only a marginal improvement with a geometric mean of {0.936}. However, combining this scaling with subsequent bound tightening yields a more substantial computational speedup as indicated by a geometric mean of {0.467}. This seems to stem from the tighter big-M coefficients, but also from an increased percentage of stable neurons. Compared to the default bounds, there is an average increase by {7.2} percent. In \vref{fig:improvement_scaling_obbt}, the parities of  computational times for the two comparisons are shown. We note that the parity plot in \vref{subfig:IA_vs_scaler_and_OBBT} suggests that the average speedup may be driven by a few outlier instances in which in the combined method performs exceptionally well.

Overall, with the scaled ReLU networks and their default bounds from interval arithmetic, 307 instances can be solved within the time limit. With tightened bounds, there is a slight reduction to 299 instances.


\begin{figure}
    \centering
    \begin{subfigure}{.47\linewidth}
        \centering
        Percentage of stable neurons
        \includegraphics[width=\linewidth]{figures/parities/obbt/parity_percentage_fixed}
        \subcaption{Parity plot for percentage of stable neurons compared for bounds from IA and LP-based OBBT.}
        \label{subfig:percentage_fixed_IA_vs_OBBT}
    \end{subfigure}
    \begin{subfigure}{.47\linewidth}
        \centering
        Computational time
        \includegraphics[width=\linewidth]{figures/parities/obbt/parity_time}
        \subcaption{Parity plot for computational time compared for bounds from IA and LP-based OBBT.}
        \label{subfig:time_IA_vs_OBBT}
    \end{subfigure}

    \caption{Parity plots comparing percentage of stable neurons and computational times of optimally solved instances of \eqref{prob:minANN} for bounds derived from IA and LP-based OBBT. Solving \eqref{prob:obbt} leads to an increase of 5.5 percentage points in stable neurons on average. This carries over to a reduction in computational time shown in \subref{subfig:time_IA_vs_OBBT}. The ratios of times with OBBT and IA bounds have a geometric mean of 0.57.}
    \label{fig:improvement_obbt}
\end{figure}


\begin{figure}
    \centering
    \begin{subfigure}{.47\linewidth}
        \centering
        Computational time
        \includegraphics[width=\linewidth]{figures/parities/scaled/parity_time}
        \subcaption{Runtime comparison between IA bounds for the baseline network (\enquote{Default}) and IA for the scaled ANN (\enquote{ReLU scaling}).}
        \label{subfig:IA_vs_scaler}
    \end{subfigure}
    \begin{subfigure}{.47\linewidth}
        \centering
        Computational time
        \includegraphics[width=\linewidth]{figures/parities/scaled_obbt/parity_time}
        \subcaption{Runtime comparison between IA bounds for the baseline network (\enquote{Default}) and OBBT for the scaled ANN (\enquote{ReLU scaling + OBBT}).}
        \label{subfig:IA_vs_scaler_and_OBBT}
    \end{subfigure}

    \caption{Parity plots comparing computational times for optimally solved instances of \eqref{prob:minANN} in different versions: \subref{subfig:IA_vs_scaler}: IA bounds for baseline vs. scaled ReLU network with a geometric mean ratio of {0.936}; \subref{subfig:IA_vs_scaler_and_OBBT}: IA bounds for baseline network vs. OBBT bounds for scaled network with a geometric mean ratio of {0.467}.}
    \label{fig:improvement_scaling_obbt}
\end{figure}

\subsection{Effect of regularization}

\begin{table}
    \addtolength{\tabcolsep}{-0.2em}
    \centering
    \caption{Influence of training options, bound tightening and ReLU scaling on all trained neural networks and their optimization problems \eqref{prob:minANN}. In each row, the effect of the listed method is evaluated by comparing it to similar networks that differ only in this particular method, e.g., for $\ell^1$ regularization we compare neural networks that were trained with the specified level of regularization to those that were trained without regularization. The first and second column show the number of solved instances without and with the applied technique and the number of instances in total in this comparison. The third column lists the reduction of big-M coefficients as measured by the geometric mean of the ratio of averaged distances of pre-activation bounds $U- L$ of the adapted network and that of the baseline network. The fourth column shows the arithmetic mean of the increase in percentage points of stable neurons due to the applied method. The fifth column shows the geometric mean of the ratio between the number of linear regions of the adapted network and that of the baseline network. The last column shows the geometric mean of the ratio between the computational time with the adapted network and that observed with the baseline network, but is limited to instances in which the optimization problems for both networks are solved within the time limit. We observe a computational speedup with regularization, bound tightening and ReLU-scaling, while dropout leads to a deterioration in performance.}
    \label{tab:results}
    \begin{tabular}{lc|cccccc}
    \multicolumn{1}{c}{}    &  & \begin{tabular}[c]{@{}c@{}}Solved instances\\ (adapted vs. baseline)\end{tabular} & \begin{tabular}[c]{@{}c@{}}Instances\\  total\end{tabular} & 
    \begin{tabular}[c]{@{}c@{}}Geom. mean \\ $\overline{U-L}$ \end{tabular} &
    \begin{tabular}[c]{@{}c@{}}Improvement \\ stable neurons\end{tabular} &    \begin{tabular}[c]{@{}c@{}}Geom. mean \\ lin. regions\end{tabular} & \begin{tabular}[c]{@{}c@{}}Geom. mean \\ time\end{tabular} \\ \hline
    \multirow{5}{*}{$\lambda$} & 1e-3   & 349 vs. 151 & 360 & 0.009  & 0.379    & 0.283  & 0.028 \\
                             & 1e-4   & 358 vs. 151 & 360  & 0.024  & 0.216  & 0.463 & 0.059  \\
                             & 1e-5   & 352 vs. 151 & 360  & 0.052  & 0.122  & 0.817 & 0.109  \\
                             & 1e-6   & 326 vs. 151  & 360 & 0.133  & 0.146  & 1.089 & 0.208  \\
                             & 1e-7   & 287 vs. 151  & 360 & 0.261  & 0.213  & 0.996 & 0.280  \\ \hline
    \multirow{2}{*}{\begin{tabular}[c]{@{}l@{}}Clipped\\ ReLU\end{tabular}} 
                            & M=2    &609 vs. 608 & 720   & 0.415   & 0.029 & 1.094 & 0.931  \\
                            & M=5    & 606 vs. 608 & 720  & 0.560   & 0.011 & 1.055 & 0.974  \\ \hline
    \multirow{2}{*}{Dropout}    & 10\%   & 146 vs. 217 & 240 & 12.761  & -0.204 & 4.098 & 5.825  \\
                                & 20\%   & 152 vs. 217& 240  & 15.403  & -0.210 & 3.513 & 4.845  \\ \hline \hline
    \multicolumn{2}{l|}{OBBT}   & 912 vs. 911  & 1080  & 0.541 & 0.055  & 1.0   & 0.570  \\ \hline
    \multirow{2}{*}{\begin{tabular}[c]{@{}l@{}}ReLU\\ scaling\end{tabular}} &        & 307 vs. 303  & 360   & 0.388 & 0.0    & 1.0   & 0.936  \\
                & OBBT & 299 vs. 303  & 360  & 0.160  & 0.072  & 1.0   & 0.467                                                     
    \end{tabular}
\end{table}

In \vref{fig:model_statistics}, we depict how the mean absolute error on the test set, the number of linear regions, the percentage of neurons of fixed activation, and the solver runtime correlate with the depth of the neural networks for networks with 50 neurons per layer with different regularization parameters. 
In the first row, we see the performance of the neural networks on the test data as measured by the MAPE. We see, that large regularization parameters lead to a degradation of accuracy on the test dataset, especially for Ackley's function. For small regularization parameters there is a high level of agreement between the predictions and the ground truth on the test data. Further, in some instances, training with moderate levels of $\ell^1$ regularization does in fact lead to be better generalization of the neural network. 
%
The second row of \cref{fig:model_statistics} shows the number of linear regions as an indicator of the complexity, or expressive power of the neural network. With increasing levels of regularization, we obtain neural networks with a lower number of linear regions. This is also illustrated in \vref{fig:linear_regions}. Comparing the number of linear regions among the three different functions, the neural networks which approximate Ackley's function have the most linear regions. This is plausible comparing the surface plots in \vref{fig:different_functions}, because Ackley's function exhibits a large number of local oscillations.
%
In the third row of \cref{fig:model_statistics}, we plot the percentage of stable neurons. These are neurons whose input bounds are either non-negative or non-positive, which means that they are in a fixed state of activation regardless of input. No binary variables have to be added to model the activation function of such neurons. Confirming the findings of \citet{Xiao2019,Serra2020}, higher values of $\lambda$ lead to a higher percentage of stable neurons. 
%
The last row shows the computational times in the optimization problem \eqref{prob:minANN}. Comparing the runtimes among the three functions, Ackley's function appears to be the hardest to minimize. Here, we cannot solve unregularized networks with as little as three hidden layers to global optimality within the specified time limit. Based on the observation that ANNs approximating this function have an increased number of linear regions and that several local minima exist in the input domain, this is expected behavior. Increasing the regularization generally lowers the time to compute global minima for all three functions. While the global minima of unregularized networks cannot be determined for any network with more than four layers, applying moderate levels of regularization makes almost all instances tractable. The only exception here is Ackley's function, which remains unsolved for the lowest regularization parameter $\lambda = 10^{-7}$ as well. While \cref{fig:model_statistics} shows the results for all networks with 50 neurons per hidden layer, we obtain similar results for those with 25 neurons (data shown in the appendix).
In combination with the results in \vref{tab:results}, this illustrates that regularization proved the most effective method by improving big-M coefficients, increasing the number of stable neurons and decreasing the number of linear regions, thus enabling the computational speedup.
%
\begin{figure}[H]
    \centering
    \begin{subfigure}{.32\linewidth}
        \centering
        Peaks
        \includegraphics[width=\linewidth]{figures/statistics/regularization/peaks_mape_50_neurons}
        \subcaption{MAPE on test set of ReLU ANNs approximating \eqref{fun:peaks}.}
    \end{subfigure}
    \begin{subfigure}{.32\linewidth}
        \centering
        Ackley
        \includegraphics[width=\linewidth]{figures/statistics/regularization/ackley_mape_50_neurons}
        \subcaption{MAPE on test set of ReLU ANNs approximating \eqref{fun:ack}.}
    \end{subfigure}
    \begin{subfigure}{.32\linewidth}
        \centering
        Himmelblau
        \includegraphics[width=\linewidth]{figures/statistics/regularization/himmelblau_mape_50_neurons}
        \subcaption{MAPE on test set of ReLU ANNs approximating \eqref{fun:him}.}
    \end{subfigure}
    
    %\begin{subfigure}{.32\linewidth}
    %    \centering
    %    \includegraphics[width=\linewidth]{figures/statistics/regularization/peaks_r2_50_neurons}
    %    \subcaption{$R^2$ on test set of ReLU ANNs approximating \eqref{fun:peaks}.}
    %\end{subfigure}
    %\begin{subfigure}{.32\linewidth}
    %    \centering
    %    \includegraphics[width=\linewidth]{figures/statistics/regularization/ackley_r2_50_neurons}
    %    \subcaption{$R^2$ on test set of ReLU ANNs approximating \eqref{fun:ack}.}
    %\end{subfigure}
    %\begin{subfigure}{.32\linewidth}
    %    \centering
    %    \includegraphics[width=\linewidth]{figures/statistics/regularization/himmelblau_r2_50_neurons}
    %    \subcaption{$R^2$ on test set of ReLU ANNs approximating \eqref{fun:him}.}
    %\end{subfigure}

    \begin{subfigure}{.32\linewidth}
        \centering
        \includegraphics[width=\linewidth]{figures/statistics/regularization/peaks_num_regions_50_neurons}
        \subcaption{Number of linear regions of ReLU ANNs approximating \eqref{fun:peaks}.}
    \end{subfigure}
    \begin{subfigure}{.32\linewidth}
        \centering
        \includegraphics[width=\linewidth]{figures/statistics/regularization/ackley_num_regions_50_neurons}
        \subcaption{Number of linear regions of ReLU ANNs approximating \eqref{fun:ack}.}
    \end{subfigure}
    \begin{subfigure}{.32\linewidth}
        \centering
        \includegraphics[width=\linewidth]{figures/statistics/regularization/himmelblau_num_regions_50_neurons}
        \subcaption{Number of linear regions of ReLU ANNs approximating \eqref{fun:him}.}
    \end{subfigure}

    \begin{subfigure}{.32\linewidth}
        \includegraphics[width=\linewidth]{figures/statistics/regularization/peaks_percentage_fixed_50_neurons}
        \subcaption{Percentage of stable neurons for big-M coefficients based on interval arithmetic bounds.}
    \end{subfigure}
    \begin{subfigure}{.32\linewidth}
        \includegraphics[width=\linewidth]{figures/statistics/regularization/ackley_percentage_fixed_50_neurons}
        \subcaption{Percentage of stable neurons for big-M coefficients based on interval arithmetic bounds.}
    \end{subfigure}
    \begin{subfigure}{.32\linewidth}
        \includegraphics[width=\linewidth]{figures/statistics/regularization/himmelblau_percentage_fixed_50_neurons}
        \subcaption{Percentage of stable neurons for big-M coefficients based on interval arithmetic bounds.}
    \end{subfigure}


    \begin{subfigure}{.32\linewidth}
        \includegraphics[width=\linewidth]{figures/statistics/regularization/peaks_time_50_neurons}
        \subcaption{Solution time of problem \eqref{prob:minANN} for Peaks function.}
        \typeout{PLOT LINE WIDTH: \the\linewidth}%
    \end{subfigure}
    \begin{subfigure}{.32\linewidth}
        \includegraphics[width=\linewidth]{figures/statistics/regularization/ackley_time_50_neurons}
        \subcaption{Solution time of problem \eqref{prob:minANN} for Ackley's function.}
    \end{subfigure}
    \begin{subfigure}{.32\linewidth}
        \includegraphics[width=\linewidth]{figures/statistics/regularization/himmelblau_time_50_neurons}
        \subcaption{Solution time of problem \eqref{prob:minANN} for Himmelblau function.}
    \end{subfigure}
    \begin{subfigure}{\linewidth}
            \centering
            \definecolor{crimson2143940}{RGB}{214,39,40}
\definecolor{darkorange25512714}{RGB}{255,127,14}
\definecolor{forestgreen4416044}{RGB}{44,160,44}
\definecolor{mediumpurple148103189}{RGB}{148,103,189}
\definecolor{steelblue31119180}{RGB}{31,119,180}
\definecolor{darkgray176}{RGB}{176,176,176}
\begin{tikzpicture} 
    \begin{axis}[%
    hide axis,
    xmin=10,
    xmax=50,
    ymin=0,
    ymax=0.4,
    legend style={
        draw=white!15!black,
        legend cell align=left,
        legend columns=-1, 
        legend style={
            draw=none,
            column sep=1ex,
            line width=0.5pt
        }
    },
    ]
    \addlegendimage{line width=2pt, color=C6}
    \addlegendentry{$10^{-01}$};
    \addlegendimage{line width=2pt, color=C0}
    \addlegendentry{$10^{-02}$};
    \addlegendimage{line width=2pt, color=C1}
    \addlegendentry{$10^{-03}$};
    \addlegendimage{line width=2pt, color=C3}
    \addlegendentry{$10^{-04}$};
    \addlegendimage{line width=2pt, color=C4}
    \addlegendentry{$10^{-05}$};
    \addlegendimage{line width=2pt, color=C5}
    \addlegendentry{$10^{-08}$};
    \addlegendimage{line width=2pt, color=C8}
    \addlegendentry{$10^{-09}$};
    \addlegendimage{line width=2pt, color=C7}
    \addlegendentry{$10^{-12}$};
    \end{axis}
\end{tikzpicture}
    \end{subfigure}
    \caption{Mean absolute percentage error on test set, number of linear regions, percentage of stable neurons and computational times in problem \eqref{prob:minANN} of trained ANNs with varying number of hidden layers with 50 neurons, trained with different levels of $\ell^1$ regularization.}
    \label{fig:model_statistics}%
    \typeout{LINE WIDTH: \the\linewidth}%
\end{figure}


\begin{figure}
    \centering
    \begin{subfigure}{.32\linewidth}
        \centering
        \includegraphics[width=\linewidth,trim={0 0 0 8mm},clip]{figures/mesh/relu_peaks_5_layer_25_neurons_0e+00_regu_0_dropout_mesh}
        \subcaption{Standard, 11,898 regions.}
        \label{subfig:mesh_no_regu}
    \end{subfigure}
    \begin{subfigure}{.32\linewidth}
        \centering
        \includegraphics[width=\linewidth,trim={0 0 0 8mm},clip]{figures/mesh/relu_peaks_5_layer_25_neurons_1e-05_regu_0_dropout_mesh}
        \subcaption{Regularization, 6,893 regions.}
        \label{subfig:mesh_regu}
    \end{subfigure}
    \begin{subfigure}{.32\linewidth}
        \centering
        \includegraphics[width=\linewidth,trim={0 0 0 8mm},clip]{figures/mesh/relu_peaks_5_layer_25_neurons_0e+00_regu_20_dropout_mesh.png}
        \subcaption{Dropout, 14,674 regions.}
        \label{subfig:mesh_dropout}
    \end{subfigure}
    \caption{Linear regions for ReLU networks approximating the Peaks function \eqref{fun:peaks} with five hidden layers of 25 neurons each. Color-coded in the backgrounds are the outputs of the neural networks. Compared are networks with different training options: \cref{subfig:mesh_no_regu} with no regularization or dropout, \cref{subfig:mesh_regu} with $\ell^1$ regularization and $\lambda=10^{-5}$, \cref{subfig:mesh_dropout} with $20$\% dropout. Regularizing the weights of the ANN during training decreases the number of linear regions, applying dropout increases it and also changes their sizes.}
    \label{fig:linear_regions}
\end{figure}


\subsection{Effect of clipped ReLU}

The effect of the clipping is obvious in the big-M coefficients of formulation \eqref{eq:bigM_clipped}, which are illustrated in \vref{fig:big_Ms_clipped} for a threshold of $M=5.0$. Compared to the big-M coefficients derived for the regular ReLU activation function and depicted in \vref{fig:IA_bounds}, the clipped ReLU formulation yields lower bounds, though this may depend on the particular choice of $M$. This is also obvious from the results in \vref{tab:results}, with clipped ReLU leading to greater reductions in big-M coefficients compared to OBBT. We also observe that LP-based bound tightening for neural networks with clipped ReLU activation does not improve the bounds to the same degree as it did for the regular ReLU activation function as depicted in \vref{fig:LR_bounds}.

As the results in \Cref{tab:results} suggest, using the clipped ReLU  \eqref{eq:bigM_clipped} yields only marginal computational speedup compared to the standard ReLU activation. There seems to be a tradeoff between a higher number of binary variables needed for modeling \eqref{eq:bigM_clipped} and the slightly higher number of linear regions on the one hand, and the reduction of big-M coefficients on the other hand. 

\begin{figure}[ht!]
    \centering
    \begin{subfigure}{0.49\textwidth}
        \includegraphics[width=\textwidth]{figures/clipped_5_big_M_unscaled}
        \caption{Pre-activation bounds $U^{(k)}$ determined via interval arithmetic for clipped ReLU ANN with the big-M formulation \eqref{eq:bigM_clipped} and $M=5.0$.}
        \label{fig:IA_bounds_clipped}
    \end{subfigure}
    \hfill
    \begin{subfigure}{0.49\textwidth}
        \includegraphics[width=\textwidth]{figures/clipped_5_big_M_unscaled_OBBT}
        \caption{Pre-activation bounds $U^{(k)}$ determined via LP-based bound tightening for clipped ReLU ANN with big-M formulation \eqref{eq:bigM_clipped} and $M=5.0$.}
        \label{fig:LR_bounds_clipped}
    \end{subfigure}
   
    \caption{Comparison of pre-activation bounds  $U^{(k)}$ for neural networks with ten hidden layers and clipped ReLU formulation \eqref{eq:bigM_clipped} with $M=5.0$ as activation function. Compared to the bounds derived via interval arithmetic for the regular ReLU activation shown in \vref{fig:IA_bounds}, the bounds for the clipped ReLU are generally lower. Moreover, due to the threshold $M$, the bounds stay approximately constant over the layers. Solving auxiliary LPs only noticeably tightens bounds in the first few layers.}
    \label{fig:big_Ms_clipped}
\end{figure}
 
\subsection{Effect of Dropout}

For the Peaks function only, we trained additional networks with different levels of dropout applied to the hidden layers, namely 10 and 20 percent. In combination with the other hyperparameters (regularization, depth and width of the network) this yields a total of 240 trained neural networks with dropout whose properties we can compare. As illustrated in \Cref{tab:results}, we find that dropout leads to neural networks with three to four times more linear regions on average, confirming the findings of \citet{Zhang2020a}. This is also evident in \Cref{fig:linear_regions}, where the linear regions of three exemplary ANNs are compared for networks with five hidden layers. Another effect of dropout is the percentage of stable neurons, which is reduced by approximately 20~\% on average compared to networks trained without dropout and a drastic increase in the magnitude of the big-M coefficients. In combination, this leads to a reduction in instances that could be solved to global optimality and a simultaneous four to six-fold increase in computational time for those instances that could be solved. 


\section{Conclusions and outlook}\label{sec:discussion}

In this paper, we compared different variations of training and scaling methods for ReLU networks with respect to their effect on the performance of global optimization solvers on problems with these networks  embedded. We divided these methods into those that are applied during training and those that can be used on trained networks. For the latter category, we proposed a scaling method specific to the ReLU activation function, which equivalently transforms a ReLU ANN such that the $\ell^1$ norm of the networks weights and biases is minimized. This has the desired effect of reducing the constant coefficients in big-M formulations of the network's activation functions. In numerical experiments, we demonstrated that this method can be used to reduce the computational effort of solving subsequent optimization problems, when it is used in combination with bound tightening. Although in our study we only investigated the direct minimization of feed-forward neural networks with their big-M formulation of ReLU networks, we believe that the findings are also applicable in other contexts. These might include optimization problems with ReLU networks using different MILP encodings, e.g., the partition-based formulation from \citet{Tsay2021}, or other optimization settings, e.g., more difficult optimization problems from real-world applications. In fact, by employing regularization during training we were able to solve a complex superstructure optimization problem in chemical engineering that had been computationally intractable before \citep{Klimek2024}.

Moreover, to the best of our knowledge, this is the first computational study that links various training methods to both the number of linear regions and the percentage of fixed neurons as well as the computational effort in subsequent optimization problems. Doing so, we were able to provide empirical evidence for several observations from the literature, e.g., an increased number of linear regions for networks trained with dropout, and computational speedup due to higher rates of fixed neurons for networks trained with $\ell^1$ regularization. 

Further research may include a more thorough analysis into how the used training methods and hyperparameter options used when training a neural network impact its number of linear regions and the number of fixed neurons. Also, different objectives in \eqref{prob:scaling} may be conceivable to promote other properties in the transformed networks. It may also be promising to investigate transformations that allow minor perturbations of the functional relationship.

\begin{acknowledgements}
This project has received funding from the European Regional Development Fund (grants timingMatters and IntelAlgen) under the European Union’s Horizon Europe Research and Innovation Program, 
from the research initiative ``SmartProSys: Intelligent Process Systems for the Sustainable Production of Chemicals'' funded by the Ministry for Science, Energy, Climate Protection and the Environment of the State of Saxony-Anhalt, and
from the German Research Foundation DFG within GRK 2297 ’Mathematical Complexity Reduction’ and priority program 2331 ’Machine Learning in Chemical Engineering’ under grant SA 2016/3-1.
\end{acknowledgements}



\bibliographystyle{spbasic}
\bibliography{main}

\newpage 
\appendix

\begin{center}
\LARGE
\textbf{Appendix for ``Neural Flow Samplers with \\Shortcut Models''}
\end{center}

\etocdepthtag.toc{mtappendix}
\etocsettagdepth{mtchapter}{none}
\etocsettagdepth{mtappendix}{subsection}
{\small \tableofcontents}

\section{Importance Sampling and Sequential Monte Carlo}

This section reviews the basic Sequential Monte Carlo (SMC) algorithm. We begin by introducing importance sampling and its application to estimating the intractable time derivative $\partial_t \log Z_t$, as presented in \cite{tian2024liouville}. We then proceed with an introduction to Sequential Monte Carlo, which is employed in our methods to estimate $\partial_t \log Z_t$.

\subsection{Importance Sampling} \label{sec:appendix_is}
Consider a target distribution $\pi (x) = \frac{\rho(x)}{Z}$, where $\rho(x) \geq 0$ is the unnormalised probability density and $Z = \int \rho(x) \dif x$ denotes the normalising constant, which is typically intractable. For a test function $\phi(x)$ of interest, estimating its expectation under $\pi$ through direct sampling can be challenging.
Importance sampling (IS) \citep{kahn1950random} instead introduces a proposal distribution $q$, which is easy to sample from, and proposes an expectation estimator as follows
\begin{align}
    \E_{\pi(x)} [\phi(x)] = \frac{1}{Z} \E_{q(x)} \left[ \frac{\rho(x)}{q(x)} \phi(x) \right] = \frac{\E_{q(x)}\left[ \frac{\rho(x)}{q(x)} \phi(x) \right]}{\E_{q(x)}\left[ \frac{\rho(x)}{q(x)} \right]}.
\end{align}
Thus, the expectation can be estimated via the Monte Carlo method
\begin{align}
    \E_{\pi(x)} [\phi(x)] \approx \sum_{k=1}^K \frac{w^{(k)}}{\sum_{=1}^N w^{(j)}} \phi(x^{(k)}), \quad x^{(k)} \sim q(x),
\end{align}
where $w^{(k)} = \frac{\rho(x^{(k)})}{q(x^{(k)})}$ denotes the importance weight. While importance sampling yields a consistent estimator as $N \rightarrow \infty$, it typically suffers from high variance and low effective sample size \citep{thiebaux1984interpretation} when the proposal deviates from the target distribution. 
In theory, a zero-variance estimator can be achieved if $q(x) \propto \rho(x) \phi(x)$; however, this condition is rarely satisfied in practice. This limitation renders importance sampling inefficient in high-dimensional spaces, as a large number of Monte Carlo samples are required to mitigate the variance.

\textbf{Approximating $\partial_t \log Z_t$ with Importance Sampling.}
\cite{tian2024liouville} propose approximating $\partial_t \log Z_t$ using importance sampling, where they express $\partial_t \log Z_t \approx \sum_k \frac{w_t^{(k)}}{\sum_k w_t^{(k)}} \partial_t \log \tilde{p}_t (x_t^{(k)})$. Here $x_t^{(k)} \sim p_t (x;\theta)$ denotes the sample generated by the velocity $v_t (x;\theta)$ at time $t$, and $\log w_t^{(k)} = \int_0^t \delta_\tau (x_\tau;v_t(\cdot;\theta)) \dif \tau$. For completeness, we provide a step-by-step recall of the proof of the correctness of this estimator from \cite{tian2024liouville}.

First, we show that $\partial_t \log Z_t$ is given by the expectation over $p_t$:
\begin{align}
    \partial_t \log Z_t = \partial_t \log \int \tilde{p}_t (x) \dif x = \frac{1}{Z_t} \int \tilde{p}_t(x) \partial_t \log \tilde{p}_t(x) \dif x = \E_{p_t(x)} [\partial_t \log \tilde{p}_t (x)].
\end{align}
$\partial_t \log Z_t$ therefore can be estimated via importance sampling
\begin{align}
    \partial_t \log Z_t \approx \sum_{k=1}^K \frac{w_t^{(k)}}{\sum_j w_t^{(j)}} \partial_t \log \tilde{p}_t(x^{(k)}), \quad x^{(k)} \sim p_t (x;\theta),
\end{align}
where $w_t^{(k)} = \frac{p_t(x^{(k)})}{p_t(x^{(k)}; \theta)}$. Next, we show that the weight $\log w_t^{(k)}$ is given by integrating $\delta_t(x;v_t(\cdot;\theta))$ on $[0,t]$, where $\delta_t$ is defined in \cref{eq:nfs_loss} i.e. $\log \frac{p_t(x_t)}{p_t(x_t;\theta)} = \int_0^t \delta_s(x;v_s(\cdot;\theta)) \dif s$.

We begin by computing the instantaneous rate of change of the log-densities of the model $p_t(x_t;\theta)$ along the trajectories generated by $v_t(x_t; \theta)$, as in \cref{eq:change-of-variable}
\begin{align}
    \partial_t [\log p_t(x_t; \theta)] 
    &= \partial_t \log p_t(x_t;\theta) + \nabla_{x_t} \log p_t(x_t;\theta) \cdot \frac{\dif x_t}{\dif t} \nonumber \\
    &= (-\nabla_{x_t} \cdot v_t(x_t;\theta) - v_t(x_t;\theta) \cdot \nabla_{x_t} \log p_t(x_t;\theta)) + \nabla_{x_t} \log p_t(x_t;\theta) \cdot v_t(x_t;\theta) \nonumber \\
    &= -\nabla_{x_t} \cdot v_t(x_t; \theta).
\end{align}
Similarly, the instantaneous rate of change of the log-densities of the target $p_t(x_t)$ along the trajectories generated by $v_t(x_t; \theta)$ is
\begin{align}
    \partial_t [\log p_t(x_t)] 
    &= \partial_t \log p_t(x_t) + \nabla_{x_t} \log p_t(x_t) \cdot \frac{\dif x_t}{\dif t} \nonumber \\
    &= \partial_t \log p_t(x_t) + \nabla_{x_t} \log p_t(x_t) \cdot v_t(x_t; \theta).
\end{align}
Note that the score of the target is tractable $\nabla_x \log p_t(x) = \nabla_x \log \tilde{p}_t(x)$. Therefore, the log densities along the trajectories can be computed via
\begin{align}
    \log p_t(x_t; \theta) &= \log p_0 (x_0;\theta) - \int_0^t  \nabla_{x_s} \cdot v_s(x_s; \theta) \dif s\\
    \log p_t(x_t) &=  \log p_0 (x_0) + \int_0^t [\partial_s \log p_s(x_s) + \nabla_{x_s} \log p_s(x_s) \cdot v_s(x_s; \theta)] \dif s.
\end{align}
Since $p_0(x; \theta) = p_0(x) = \eta (x)$ due to the annealing path construction $p_t \propto \rho^t \eta^{1-t}$, we have
\begin{align}
    \log \frac{p_t(x_t)}{p_t(x_t;\theta)} 
    &= \int_0^t [ \nabla_{x_s} \cdot v_s(x_s; \theta) + \partial_s \log p_s(x_s) + \nabla_{x_s} \log p_s(x_s) \cdot v_s(x_s; \theta) ] \dif s \nonumber \\
    &= \int_0^t \delta_s(x;v_s(\cdot;\theta)) \dif s,
\end{align}
which completes the proof.

\textbf{Remark.} Although importance sampling offers an elegant method for estimating the intractable time derivatives $\partial_t \log Z_t$, it faces two main challenges. First, as previously discussed, importance sampling typically suffers from high variance and requires large sample sizes to improve the effective sample size. More critically, the computation of the weight involves the intractable term $\partial_t \log p_s(x_t)$, which in turn depends on $\partial_t \log Z_t = \E_{p_t(x)}[\partial_t \log \tilde{p}_t(x)]$. \cite{tian2024liouville} approximate this expectation naively by averaging $\partial_t \log \tilde{p}_t(x)$ over the mini-batch during training, which introduces both approximation errors and bias in the importance weights.

In the following section, we introduce Sequential Monte Carlo, which is employed in our methodology to estimate $\partial_t \log Z_t$, balancing the efficiency of the short-run MCMC driven by the velocity with the effectiveness of low variance.

\subsection{Sequential Monte Carlo} \label{sec:appendix-smc}
Sequential Monte Carlo (SMC) provides an alternative method to estimate the intractable expectation $\E_{p_t (x)} [\phi(x)]$.
Specifically, SMC decomposes the task into easier subproblems involving a set of unnormalised intermediate target distributions $\{\tilde{p}_{t_m} (x_{t_m})\}_{m=0}^M$.\footnote{We consider a discrete-time schedule that satisfies $0=t_0 < t_1 < \dots < t_m < \dots < t_{M-1} < t_M = 1$.}
We begin by introducing sequential importance sampling (SIS):
\begin{align}
    \E_{p_{t_m} (x)} [\phi(x)] 
    &= \int q(x_{t_0:t_m}) \frac{p(x_{t_0:t_m})}{q(x_{t_0:t_m})} \phi(x_{t_m}) \dif x_{t_0:t_m} \nonumber \\
    &\approx \frac{1}{K} \sum_{k=1}^K \frac{p(x_{t_0:t_m}^{(k)})}{q(x_{t_0:t_m}^{(k)})} \phi(x_{t_m}^{(k)}), \quad \text{where} \quad x_{t_0:t_m}^{(k)} \sim q(x_{t_0:t_m}^{(k)}).
\end{align}
The importance weights are $w_{t_m}^{(k)} = \frac{p(x_{t_0:t_m}^{(k)})}{q(x_{t_0:t_m}^{(k)})}$.
The key ingredients of SMC are the proposal distributions $q(x_{t_0:t_m})$ and the target distributions $p(x_{t_0:t_m})$.
Here we consider a general form associated with a sequence of forward kernels $q(x_{t_0:t_m}) = q(x_{t_0})\prod_{s=0}^{m-1} \mathcal{F}_{t_s} (x_{t_{s+1}}|x_{t_s})$, and the target distribution is defined by a sequence of backward kernels $p(x_{t_0:t_m}) = p(x_{t_m}) \prod_{s=0}^{m-1}\mathcal{B}_{t_s} (x_{t_s} | x_{t_{s+1}})$. Substituting this into the expression for the importance weights gives
\begin{align}
    w_{t_m}^{(k)} = \frac{p(x_{t_0:t_m}^{(k)})}{q(x_{t_0:t_m}^{(k)})} = \frac{p(x_{t_m})\prod_{s=0}^{m-1}\mathcal{B}_{t_s} (x_{t_s} | x_{t_{s+1}})}{q(x_{t_0})\prod_{s=0}^{m-1} \mathcal{F}_{t_s} (x_{t_{s+1}}|x_{t_s})} = w_{t_{m-1}}^{(k)} W_{t_m}^{(k)},
\end{align}
where $W_{t_m}^{(k)}$, termed the incremental weights, are calculated as,
\begin{align}
    W_{t_m}^{(k)} = \frac{p_{t_m}(x_{t_m})}{p_{t_{m-1}}(x_{t_{m-1}})} \frac{\mathcal{B}_{t_m}(x_{t_{m-1}}|x_{t_m})}{\mathcal{F}_{t_m}(x_{t_m} | x_{t_{m-1}})}.
\end{align}
By defining the backward kernel as $\mathcal{B}_{t_m} (x_{t_{m-1}}|x_{t_m}) = \frac{p_{t_{m-1}}(x_{t_{m-1}})\mathcal{F}_{t_m}(x_{t_m} | x_{t_{m-1}})}{p_{t_{m-1}} (x_{t_m})}$, the incremental weight is tractable and becomes
\begin{align}
    W_{t_m}^{(k)} = \frac{p_{t_m}(x_{{t_m}})}{p_{t_{m-1}}(x_{{t_m}})}.
\end{align}
Therefore, the expectation can be approximated as
\begin{align}
    \E_{p_{t_m}(x)} [\phi(x)] \approx \sum_k \tilde{w}^{(k)}_{t_m} \phi(x_{t_m}), \quad  \tilde{w}^{(k)}_{t_m} = \frac{{w}^{(k)}_{t_m}}{\sum_j {w}^{(j)}_{t_m}}, \quad w_{t_m}^{(k)} = w_{t_{m-1}}^{(k)} W_{t_m}^{(k)} \propto w_{t_{m-1}}^{(k)} \frac{\tilde{p}_{t_m}(x_t)}{\tilde{p}_{t_{m-1}}(x_{t_m})}. \nonumber
\end{align}
The SIS method is elegant, as the weights can be computed on the fly. However, with a straightforward application of SIS, the distribution of importance weights typically becomes increasingly skewed as $t$ progresses, resulting in many samples having negligible weights. This imbalance reduces the effective sample size and the overall efficiency of the algorithm.
To alleviate this issue, a common approach used in SMC is to introduce a resampling step. At each time step $t$, the samples $x_t^{(k)}$ are resampled using systematic resampling method based on the normalized weights $\tilde{w}_t^{(k)}$\footnote{Rather than resampling at every time step $t$, a more advanced resampling method involves making the resampling decision adaptively, based on criteria such as the Effective Sample Size \citep{doucet2001introduction}.}. 
\begin{wrapfigure}{r}{0.57\linewidth}
\vspace{-4mm}
\centering
% UGLY USE OF \vspace & \hspace follows
    \begin{minipage}[t]{\linewidth}
    \centering
        \begin{algorithm}[H]
        \setstretch{1.33}
        % \renewcommand{\thealgorithm}{} % unenables algorithm number
        \caption{Velocity-Driven SMC} \small
        \label{alg:SMC} 
        \textbf{Input}: velocity $v_t(\cdot;\theta)$; sample size $K$; time steps $\{t_m\}_{m=0}^M$ \\
        \textbf{Output}: samples and weights $\{ x_{t_m}^{(k)}, \tilde{w}_{t_m}^{(k)} \}_{k=1, m=0}^{K, M}$
        \begin{algorithmic}[1] %[1] enables line numbers
        \Procedure{VD-SMC}{$v_t, K, \{t_m\}_{m=0}^M$}
            \For{$k = 1, \dots, K$}
                \State $x_0^{(k)} \sim p_0 (x_0), \quad w_0^{(k)} = p_0 (x_0^{(k)})$
            \EndFor
            \State $\tilde{w}_0^{(k)} = w_0^{(k)} / \sum_{i=1}^K w_0^{(i)}$
            \For{$m = 1, \dots, M$}
                \For{$k = 1, \dots, K$}
                    \If{ESS($\tilde{w}_{t_{m-1}}^{1:K}$) $< \mathrm{ESS}_{\text{min}}$}
                        \State $\!\!\!a_{t_m}^{(k)} \!\sim\! \mathrm{Systematic}(\tilde{w}_{t_{m-1}}^{1:K}), \!\!\!\!\!\quad \hat{w}_{t_{m-1}}^{(k)} \!=\! 1$
                    \Else
                        \State $a_{t_m}^{(k)} = k, \quad \hat{w}_{t_{m-1}}^{(k)} = w_{t_{m-1}}^{(k)}$
                    \EndIf
                    \State $\dif t \leftarrow t_m - t_{m-1}$
                    \State $x_{t_m}^{(k)} \!=\! \mathrm{HMC}(x_{t_{m-1}}^{(a_{t_m}^{(k)})} \!\!\!+\! v_{t_{m\!-\!1}}(x_{t_{m-1}}^{(a_{t_m}^{(k)})};\theta) \dif t)$
                    \State $w_{t_m}^{(k)} = \hat{w}_{t_{m-1}}^{(k)} \frac{\tilde{p}_{t_m} (x_{t_m}^{(k)})}{\tilde{p}_{t_{m-1}} (x_{t_m}^{(k)})}$
                \EndFor
                \State $\tilde{w}_{t_m} = w_{t_m}^{(k)} / \sum_{i=1}^K w_{t_m}^{(i)}$
            \EndFor
        \EndProcedure
        \end{algorithmic}
        \end{algorithm}
    \end{minipage}
\vspace{-5mm}
\end{wrapfigure}
The resampled particles are then assigned equal weights to mitigate the bias introduced by the skewness in the weight distribution.
This resampling trick prevents the sample set from degenerating, where only a few particles carry significant weight while others contribute minimally. By periodically resampling, the algorithm maintains diversity in the particle set. It ensures that the estimation is focused on regions of high probability density, leading to less skewed importance weight distributions.
To encourage the convergence of MCMC transition kernels, we also introduce a velocity-driven step.
The implementation of the proposed velocity-driven sequential Monte Carlo (VD-SMC) is given by \cref{alg:SMC}.
Given the sample size $K$ and the time schedule $\{t_m\}_{m=1}^M$ with $t_0=0, t_M=1$, the algorithm VD-SMC returns the samples and importance weights $\{ x_{t_m}^{(k)}, \tilde{w}_{t_m}^{(k)} \}_{k=1, m=0}^{K, M}$.
Therefore, the intractable time derivative can be approximated as $\partial_t \log Z_t = \E_{p_t} \partial_t \log \tilde{p}_t(x) \approx \sum_k \tilde{w}_t^{(k)} \partial_t \log \tilde{p}_t(x_t^{(k)})$. However, as illustrated in \cref{fig:std_dt_logZt}, the estimation of $\E_{p_t} \partial_t \log \tilde{p}_t(x)$ exhibits higher variance compared to using $\E_{p_t} \xi_t(x;v_t)$. Therefore, in practice, we approximate the time derivative as $\partial_t \log Z_t = \E_{p_t} \xi_t(x;v_t(\cdot;\theta_{\mathrm{sg}})) \approx \sum_k \tilde{w}_t^{(k)} \xi_t(x_t^{(k)};v_t(\cdot;\theta_{\mathrm{sg}}))$, where $\theta_{\mathrm{sg}}$ denotes the parameters with gradients detached.


\section{Variance Reduction with Control Variates}

\subsection{Proof of \cref{eq:partial_logZ_fpi}} \label{sec:appendix-proof_partial_logZ}
Recall that in \cref{eq:partial_logZ_fpi}, we show that the following equation holds:
\begin{align}
    \!\!\partial_t \log Z_t \!\!=\!\! \argmin_{c_t} \!\E_{p_t} (\xi_t(x;v_t) \!-\! c_t)^2, \xi_t(x;v_t) \!\triangleq\! \partial_t \log \tilde{p}_t (x) \!+\! \nabla_x \!\cdot\! v_t (x) \!+\! v_t (x) \!\cdot\! \nabla_x \log p_t (x). \nonumber
\end{align}
We provide a detailed proof of this result here. First, we have the following Lemmas.
\begin{lemma}[Stein's Identity \citep{stein2004use}] \label{lemma:divergence}
    Assuming that the target density $p_t$ vanishes at infinity, i.e., $p_t(x) = 0$, whenever $\exists d$ such that $x[d] = \infty$, where $x[d]$ denotes the $d$-th element of the vector $x$. Under this assumption, we have the result: $\int [\nabla_x \!\cdot\! v_t (x) \!+\! v_t (x) \!\cdot\! \nabla_x \log p_t (x)]\tilde{p}_t (x) \dif x = 0$.  
\end{lemma}
\begin{proof}
    To prove the result, notice that
    \begin{align}
        \int [\nabla_x \!\cdot\! v_t (x) \!+\! v_t (x) \!\cdot\! \nabla_x \log p_t (x)]\tilde{p}_t (x) \dif x
        &= \int  \tilde{p}_t (x) \nabla_x \!\cdot\! v_t (x) \!+\! v_t (x) \!\cdot\! \nabla_x \tilde{p}_t (x)  \dif x \nonumber \\
        &= \int \nabla_x \cdot [v_t(x) \tilde{p}_t (x)] \dif x \nonumber \\
        &= \sum_d \int \frac{\dif}{\dif x_d} [v_t(x) \tilde{p}_t (x)][d] \dif x_d \nonumber \\
        &= \sum_d \left. [v_t(x) \tilde{p}_t (x)][d] \right|_{x_d = -infty}^{x_d = +\infty} = 0, \nonumber
    \end{align}
    where the last row applies the divergence theorem $\int_a^b f^\prime (t) \dif t = f(b) - f(a)$.
\end{proof}
\begin{lemma} \label{lemma:l2}
    Let $c_t^* = \argmin_{c_t} \E_{p_t} (\xi_t(x;v_t) \!-\! c_t)^2$, then $c_t^* = \E_{p_t}\xi_t(x;v_t)$.
\end{lemma}
\begin{proof}
    To see this, we can expand the objective
    \begin{align}
        \mathcal{L}(c_t) 
        = \E_{p_t} (\xi_t(x;v_t) \!-\! c_t)^2 
        = c_t^2 - 2c_t\E_{p_t} \xi_t(x;v_t) + \mathrm{c}
        = (c_t - \E_{p_t} \xi_t(x;v_t))^2 + \mathrm{c}', \nonumber
    \end{align}
    where $c, c'$ are constants w.r.t. $c_t$. Therefore $c_t^* \!=\! \argmin_{c_t}\! \E_{p_t} (\xi_t(x;v_t) \!-\! c_t)^2 \!=\! \E_{p_t}\xi_t(x;v_t)$.
\end{proof}
Now, it is ready to prove \cref{eq:partial_logZ_fpi}. Specifically,
\begin{align}
    c_t^* 
    &= \E_{p_t}\xi_t(x;v_t) \nonumber \\
    &= \frac{1}{\int \tilde{p}_t(x) \dif x} \int \tilde{p}_t(x) [\partial_t \log \tilde{p}_t (x) \!+\! \nabla_x \!\cdot\! v_t (x) \!+\! v_t (x) \!\cdot\! \nabla_x \log p_t (x)] \dif x \nonumber \\
    &= \frac{1}{\int \tilde{p}_t(x) \dif x} \int \partial_t \tilde{p}_t (x) + [\nabla_x \!\cdot\! v_t (x) \!+\! v_t (x) \!\cdot\! \nabla_x \log p_t (x)]\tilde{p}_t (x) \dif x \nonumber \\
    &= \frac{1}{\int \tilde{p}_t(x) \dif x}  \int \partial_t \tilde{p}_t (x) \dif x  \nonumber \\
    &= \partial_t \log Z_t, \nonumber
\end{align}
where the first and fourth equations follow \cref{lemma:divergence,lemma:l2}, respectively, which completes the proof.

\begin{wrapfigure}{r}{0.42\linewidth}
    \centering
    % \vspace{-4mm}
    \includegraphics[width=.39\textwidth]{figures/training/loss_plot.png}
    \vspace{-3mm}
    \caption{Training loss of using different estimators of $\partial_t \log Z_t$.}
    \label{fig:loss}
    \vspace{-4mm}
\end{wrapfigure}
\textbf{Remark.} \cref{eq:partial_logZ_fpi} provides an alternative approach to estimate $\partial_t \log Z_t$. As illustrated in \cref{fig:std_dt_logZt}, this estimation exhibits lower variance compared to using $\E_{p_t} \partial_t \log \tilde{p}_t (x)$. This reduction in variance can potentially lead to better optimisation. To evaluate this, we conducted experiments on GMM datasets by minimizing the loss in \cref{eq:nfs_loss}, employing two different methods to estimate $\partial_t \log Z_t$: $\E_{p_t} \partial_t \log \tilde{p}_t (x)$ and $\E_{p_t} \xi_t (x;v_t)$. The loss values during training are plotted against the training steps in \cref{fig:loss}. The results show that the estimator of $\E_{p_t} \xi_t (x;v_t)$ achieves lower loss values, highlighting the superior training effects achieved with the lower variance estimation of $\partial_t \log Z_t$.



\subsection{Stein Control Variates} \label{sec:appendix-control-variates}
In this section, we provide a perspective from control variates to explain the observation of variance reduction in \cref{fig:std_dt_logZt}.
In particular, consider a Monte Carlo integration problem $\mu = \E_\pi [f (x)]$, which can be estimated as $\hat{\mu} = \frac{1}{K} \sum_{k=1}^K f(x^{(k)}), x^{(k)} \sim \pi$.
Assuming another function exists with a known mean $\gamma = \E_\pi [g(x)]$, we call $g$ the control variate. We then can construct another estimator $\check{\mu} = \frac{1}{K} \sum_{k=1}^K (f(x^{(k)}) - \beta g(x^{(k)})) + \beta \gamma$, where $\beta$ is a scalar coefficient and controls the scale of
the control variate. It is obvious that $\E [\check{\mu}] = \E [\hat{\mu}] = \mu, \forall \beta \in \mathbb{R}$. Moreover,  we can choose a $\beta$ to minimize the variance of $\check{\mu}$. To obtain it, we first derive the variance of $\check{\mu}$
\begin{align} \label{eq:variance_mu_cv}
    \mathbb{V}[\check{\mu}] = \frac{1}{K} (\mathbb{V} [f] - 2\beta \mathrm{Cov}(f, g) + \beta^2\mathbb{V} [g]).
\end{align}
Since $ \mathbb{V}[\check{\mu}]$ is convex w.r.t. $\beta$, by differentiating it w.r.t. $\beta$ and zeroing it, we find the optimal value, $\beta^* = \mathrm{Cov}(f, g) / \mathbb{V}[g]$. Substituting it into \cref{eq:variance_mu_cv}, we get the minimal variance 
\begin{align}
    \mathbb{V}[\check{\mu}] = \frac{1}{K} \mathbb{V}[\hat{\mu}](1 - \mathrm{Corr}(f, g)^2).
\end{align}
This shows that, given the optimal value $\beta^*$, any function $g$ that correlates to $f$, whether positively or negatively, reduces the variance of the estimator, i.e., $\mathbb{V}[\check{\mu}] < \mathbb{V}[\hat{\mu}]$. 
In practice, the optimal $\beta^*$ can be estimated from a small number of samples \citep{ranganath2014black}. However, the primary challenge lies in finding an appropriate function $g$. For a detailed discussion on control variates, see \cite{geffner2018using}.

Fortunately, \cref{lemma:divergence}  offers a systematic way to construct a control variate for $\E_{p_t} [f(x)] \triangleq \E_{p_t}[\partial_t \log \tilde{p}_t(x)] \approx \frac{1}{K} \sum_{k=1}^K \partial_t \log \tilde{p}_t(x^{(k)})$, where $x^{(k)} \sim p_t$. Specifically, we define $g(x) = \nabla_x \cdot v_t(x;\theta) + v_t(x;\theta) \cdot \nabla_x \log p_t (x)$, from which we have $\gamma = \E_{p_t}[g(x)] = 0$. Using this, we construct a new estimator:
\begin{align}
    \check{\mu} \!=\! \frac{1}{K} \sum_{k=1}^K \partial_t \log \tilde{p}_t(x^{(k)}) \!+\! \beta^* (\nabla_x \cdot v_t(x^{(k)};\theta) \!+\! v_t(x^{(k)};\theta) \cdot \nabla_x \log p_t (x^{(k)})), \quad x^{(k)} \!\sim\! p_t.
\end{align}
Moreover, when $\theta$ is optimal, \cref{eq:nfs_loss} equals zero, implying $g(x) = -f(x) + c$, where $c$ is a constant independent of the sample $x$. In this case, $\mathrm{Corr}(f, g) = -1$, and $\check{\mu}$ becomes a zero-variance estimator.
As an additional clarification, Stein's identity from \cref{lemma:divergence} is also employed as a control variate in \cite{liu2017action}, where it is utilized to optimise the policy in reinforcement learning.


\begin{algorithm}[!t]
\caption{Training Procedure of \ours (one training epoch only for illustration)}
\label{alg:nfs_training}
\textbf{Input}: initial shortcut model $s_t (\cdot,\cdot;\theta)$, time spans $\{t_m\}_{m=0}^M$, shortcut distances $\{2^{-e}\}_{e=0}^E$ \\
\textbf{Output}: trained shortcut model $s_t (\cdot,\cdot;\theta)$
\begin{algorithmic}[1]
    \State $\tilde{t}_0 \leftarrow 0, \tilde{t}_m \sim \mathcal{U}([t_{m-1}, t_{m}]), m = 1,\dots,M$   \textcolor{gray}{\Comment{Sample time steps}}
    \State \!\!$\{ x_{\tilde{t}_m}^{(k)}\!, \tilde{w}_{\tilde{t}_m}^{(k)} \}_{k=1, m=0}^{K, M} \!\!\leftarrow\!\! \mathrm{VD-SMC}(s_t (\cdot,0;\theta), K,  \{\tilde{t}_m\}_{m=0}^M)$  \textcolor{gray}{\Comment{Generate samples using \cref{alg:SMC}}}
    \For{$t, \{x_t, \tilde{w}_t\} \sim \{ x_{\tilde{t}_m}^{(k)}, \tilde{w}_{\tilde{t}_m}^{(k)} \}_{k=1, m=0}^{K, M}$} \textcolor{gray}{\Comment{Executed with mini-batch in parallel practically}}
        \State \!\!\!$\xi_t \!\leftarrow\! \partial_t \log \tilde{p}_t (x_t) \!+\! \nabla_x \!\cdot\! s_t (x_t,0;\theta) \!+\! s_t (x_t,0;\theta) \!\cdot\! \nabla_x \log p_t (x_t)$
        \State \!\!\!$c_t \!\leftarrow \!\!\sum_{k} \! \tilde{w}_t^{(k)} \!\!\left[ \!\partial_t \log \tilde{p}_t (x_t^{(k)}\!) \!+\! \nabla_x \!\cdot\! s_t (x_t^{(k)}\!\!\!,0;\theta_{\mathrm{sg}}) \!+\! s_t (x_t^{(k)}\!\!\!,0;\theta_{\mathrm{sg}}) \!\cdot\! \nabla_x \log p_t (x_t^{(k\!)}) \!\right]$ \textcolor{gray}{\Comment{ $\partial_t \log Z_t$}}
        \State \!\!\!$d \sim \mathcal{U}( \{2^{-e}\}_{e=0}^E )$ \textcolor{gray}{\Comment{Sample shortcut distance}}
        \State \!\!\!$s_{\text{target}} \leftarrow s_t (x_t, d;\theta)/2 + s_{t+d} (x_{t+d}, d;\theta)/2$ \textcolor{gray}{\Comment{Compute shortcut target}}
        \State \!\!\!$\mathcal{L}(\theta) \leftarrow (\xi_t - c_t)^2 + \lVert s_t(x_t. 2d;\theta) - s_{\text{target}} \rVert_2^2$   \textcolor{gray}{\Comment{Compute training loss}}
        \State \!\!\!$\theta \leftarrow \theta - \eta \nabla_\theta \mathcal{L}(\theta)$    \textcolor{gray}{\Comment{Perform gradient update}}
    \EndFor
\end{algorithmic}
\end{algorithm}

\begin{algorithm}[!t]
\caption{Sampling Procedure of \ours}
\label{alg:nfs_sampling}
\textbf{Input}: trained shortcut model $s_t (\cdot,\cdot;\theta)$, initial density $p_0$, \# steps $M$\\
\textbf{Output}: generated samples $x$
\begin{algorithmic}[1]
\State $x_0 \sim p_0, \quad d \leftarrow \frac{1}{M}, \quad t \leftarrow 0$ \textcolor{gray}{\Comment{Initialisation}}
\For{$m = 0, \dots, M-1$}
    \State $x \leftarrow x + s_t (x,d;\theta) d$
    \State $t \leftarrow t + d$
\EndFor
\end{algorithmic}
\end{algorithm}

\section{Training and Sampling Algorithms}
The training and sampling algorithms are detailed in \cref{alg:nfs_training,alg:nfs_sampling}, respectively. For clarity, \cref{alg:nfs_training} illustrates a single training epoch.
In particular, we parameterise the model with a single neural network $s_t (x, d;\theta)$ that takes the sample $x$, time step $t$, and shortcut distance $d$ as input to anticipate the shortcut direction. This design enables \ours to model in continuous time, unlike the baseline LFIS \citep{tian2024liouville}, which trains separate neural networks for each time step --- a memory-intensive and inefficient approach.
To train the model, we define time spans $\{t_m\}_{m=0}^M$ that are evenly distributed over $[0,1]$, satisfying $0=t_0<\dots<t_M=1$ and $2t_m  = t_{m+1} + t_{m-1}, \forall m$.
In each epoch, we uniformly sample time steps from the time spans $\tilde{t}_m \sim \mathcal{U}([t_{m-1}, t_m])$ and ensure that $\tilde{t}_0 = 0$.\footnote{More advanced schedule beyond uniform sampling remain important future works.}
Subsequently, \cref{alg:SMC} is invoked to generate training samples, which resembles distribution $q$ as defined in \cref{eq:nfs_loss}. Notably, any $q$ distribution can be used to generate training samples. The choice of the proposed velocity-driven SMC is motivated by two key reasons:  
\begin{itemize}
    \item[i)]At the beginning of training, the generated samples are far from the mode, encouraging the model to focus on mode-covering. As training progresses, the generated samples become more accurate, gradually shifting toward mode-seeking, ultimately balancing exploration and exploitation for improved learning efficiency.
    \item[ii)] Improved $\partial_t \log Z_t$ estimation efficiency. SMC returns the samples and importance weights for each time step simultaneously, streamlining the estimation of $\partial_t \log Z_t$.
\end{itemize}
After generating training samples, we compute the loss in \cref{eq:loss-nfs2} and update the model using gradient descent, as outlined in steps 4–9 of \cref{alg:nfs_training}.


\section{Related Work}

Sampling from given probability distributions has been a longstanding research challenge. Monte Carlo methods, such as Annealed Importance Sampling \citep{neal2001annealed} and Sequential Monte Carlo \citep{del2006sequential}, are considered the gold standards for sampling, but they tend to be computationally expensive and often suffer from slow convergence \citep{roberts2001optimal}.
Amortised variational methods like normalizing flows \citep{rezende2015variational} and latent variable models \citep{he2024training} provide appealing alternatives to matching the target distribution, offering faster inference but often at the cost of approximation errors and limited expressiveness.
Hybrid approaches \citep{wu2020stochastic,zhang2021differentiable,geffner2021mcmc,matthews2022continual,midgley2022flow} that combine MCMC and variational inference have shown promising potential by leveraging the strengths of both methods.

Building upon the success of generative modelling, diffusion models \citep{ho2020denoising} and flow matching \citep{lipman2022flow} have been applied to sampling tasks. Specifically, \cite{vargas2023denoising,nusken2024transport} exploit diffusion processes for learning to sample. However, these approaches require simulation to compute the objective. 
To resolve this issue, \cite{AkhoundSadegh2024IteratedDE} propose to use a bi-level training scheme that iteratively generates samples and performs score matching with Monte Carlo estimates of the target, resembling training diffusion models, and does not require simulation. \cite{ouyang2024bnem} introduce a variant that replaces score matching with direct regression on the energy, which is shown to reduce variance. Similarly, \cite{woo2024iterated} present another variant that targets on the MC-estimated vector fields in a flow matching framework.
Other approaches also focus on learning the velocity field; for instance,
\cite{tian2024liouville,mate2023learning} learn the velocity field to satisfy the continuity equation of the given probability path.

Beyond the above methods, stochastic control \citep{pavon1989stochastic,tzen2019theoretical} has also been applied to the sampling. For example, \cite{zhang2021path} propose path integral sampler (PIS) based on the connections between sampling and optimal control \citep{chen2016relation}. \cite{berner2022optimal} also establish the connection between optimal control and generative modelling based on stochastic differential equations \citep{kloeden1992stochastic}, which can be applied in sampling.
Generative flow networks (GFlowNets) \citep{lahlou2023theory} are appealing alternatives for sampling from unnormalised densities.
\cite{zhang2022unifying} establishes a connection between diffusion models and GFlowNets, leveraging this relationship to enhance learning-based sampling \citep{zhang2023diffusion}, which only requires simulating partial trajectories, improving the efficiency compared to PIS.

\section{Experimental Details} \label{sec:sppendix_exp_details}
\subsection{Datasets}
\textbf{Gaussian Mixture Model (GMM-40).} We use a 40 Gaussian mixture density in 2 dimensions as proposed by \cite{midgley2022flow}. This density consists of a mixture of 40 evenly weighted Gaussians with identical covariances
\[
\Sigma = 
\begin{bmatrix}
40 & 0 \\
0 & 40
\end{bmatrix}
\]
and \( \mu_i \) are uniformly distributed over the \([-40, 40]\) box, i.e., \( \mu_i \sim U(-40, 40)^2 \).
\[
p(x) = \frac{1}{40} \sum_{i=1}^{40} \mathcal{N}(x; \mu_i, \Sigma)
\]
\textbf{Many Well 32 (MW-32).} We use a 32-dimensional Many Well density, as proposed by \cite{midgley2022flow}. This density consists of a mixture of \( n_{\text{wells}} = 16 \) independent double-well potentials:
\[
E(x) = \sum_{i=1}^{n_{\text{wells}}} E_{\text{DW}}(x_i)
\]
where each \( x_i \) corresponds to a pair of variables in a 2-dimensional space. The energy function for a single double-well potential is defined as:
\[
E_{\text{DW}}(x) = \frac{1}{2} \left( (x_1 - \mu_1)^2 + (x_2 - \mu_2)^2 \right)
\]
Here, the wells are symmetrically distributed across a grid in the 32-dimensional space, where each pair of dimensions corresponds to a well, and \( \mu_i \) is uniformly distributed over the space. The total log probability is proportional to the sum of energies from all wells:
$\log p(x) \propto E(x) = \sum_{i=1}^{n_{\text{wells}}} E_{\text{DW}}(x_i)$.

\textbf{Double Well 4.} The energy function for the DW-4 dataset was introduced in \cite{pmlr-v119-kohler20a} and corresponds to a system of 4 particles in a 2-dimensional space. The system is governed by a double-well potential based on the pairwise distances of the particles. For a system of 4 particles, \( x = \{x_1, \dots, x_4\} \), the energy is given by:
\[
E(x) = \frac{1}{2\tau} \sum_{i,j} \left[ a(d_{ij} - d_0) + b(d_{ij} - d_0)^2 + c(d_{ij} - d_0)^4 \right]
\]
where \( d_{ij} = \| x_i - x_j \|_2 \) is the Euclidean distance between particles \( i \) and \( j \). Following previous work, we set \( a = 0 \), \( b = -4 \), \( c = 0.9 \), and the temperature parameter \( \tau = 1 \).
To evaluate the efficacy of our samples, we use a validation and test set from the MCMC samples in \cite{klein2024equivariant} as the ground truth samples following the practice of previous works \citep{AkhoundSadegh2024IteratedDE}.

\subsection{Metrics}

We evaluate the methods using the Wasserstein-2 ($\mathcal{W}_2$) distance and the Total Variation (TV), both computed with 1,000 ground truth and generated samples. To compute TV, the support is divided into 200 bins along each dimension, and the empirical distribution over 1,000 samples is used.
For GMM-40, we report the metrics$\mathcal{W}_2$ on energy space $\mathcal{E}$ and TV on the data space $\mathcal{X}$. 
For MW-32, we find that $\mathcal{E}-\mathcal{W}_2$ is unstable and thus report $\mathcal{E}$-TV instead. Given the 32-dimensional nature of MW-32, computing TV is impractical; therefore, we report the $\mathcal{W}_2$ metric on the data space rather than TV.
For the $n$-body system DW-4, we do not report any metrics in the data space due to its equivariance. Instead, we assess performance using metrics in the energy space ($\mathcal{E}$-$\mathcal{W}_2$ and $\mathcal{E}$-TV) and the interatomic coordinates $\mathcal{D}$ ($\mathcal{D}$-TV) to account for invariance.


\subsection{Training Details}
\textbf{Gaussian Mixture Model (GMM-40).}
We evaluate our method on a 40-mode Gaussian mixture in $\mathbb{R}^2$ to test multi-modal exploration. The velocity field is parameterized by a 4-layer MLP ($128$-dimensional hidden layers, Layer Norm, and GeLU activations) trained using velocity-guided sequential Monte Carlo with Hamiltonian kernels (3 HMC steps, 5 leapfrog steps, step size $\eta=0.1$). Initial particles are sampled from $\mathcal{N}(\mathbf{0}, 25\mathbf{I})$, optimized via AdamW ($\beta_1=0.9$, $\beta_2=0.999$) with learning rate $4 \times 10^{-4}$, weight decay $10^{-4}$, and gradient clipping at $\ell_2$-norm $1.0$. Training uses $128$-particle batches for $10^4$ epochs (500 steps/epoch) with early stopping, converging significantly before the epoch limit.


\textbf{Many Well 32 (MW-32).}
We assess scalability in high dimensions using a $2^{32}$-mode Many Well potential on $\mathbb{R}^{32}$, exhibiting exponential mode growth with dimension. The velocity field employs a 4-layer MLP ($128$-dimensional hidden layers, Layer Norm, and GeLU activations) trained via velocity-guided SMC with enhanced Hamiltonian kernels (6 HMC steps, 10 leapfrog steps, step size $\eta=0.1$). Initialized from $\mathcal{N}(\mathbf{0}, 2\mathbf{I})$, optimization uses AdamW ($\beta_1=0.9$, $\beta_2=0.999$) with learning rate $1e^{-3}$, weight decay $10^{-4}$, and $\ell_2$-gradient clipping at $1.0$. Training maintains $128$-particle batches across $10^4$ epochs (500 steps/epoch) with early stopping.

\textbf{Double Well 4 (DW-4).}
We assess performance in particle-like system using a DW-4 potential on euclidean space. The velocity field employs a 4-layer MLP ($512$-dimensional hidden layers, Layer Norm, and GeLU activations) trained via velocity-guided SMC with enhanced Hamiltonian kernels (10 HMC steps, 10 leapfrog steps, step size $\eta=0.01$). Initialized from $\mathcal{N}(\mathbf{0}, 2\mathbf{I})$, optimization uses AdamW ($\beta_1=0.9$, $\beta_2=0.999$) with learning rate $4e^{-3}$, weight decay $10^{-4}$, and $\ell_2$-gradient clipping at $1.0$. Training maintains $128$-particle batches across $10^4$ epochs (500 steps/epoch) with early stopping.

\section{Additional Experimental Results} \label{sec:appendix_add_exp_results}


\begin{figure}[!t]
    \centering
    \begin{minipage}[t]{0.195\linewidth}
        \centering
        \includegraphics[width=1.\linewidth]{figures/samples/mw-32/data_samples_2plots.png}
        Ground Truth
    \end{minipage}
    \begin{minipage}[t]{0.195\linewidth}
        \centering
        \includegraphics[width=1.\linewidth]{figures/samples/mw-32/fab_samples_2plots.png}
        FAB
    \end{minipage}
    \begin{minipage}[t]{0.195\linewidth}
        \centering
        \includegraphics[width=1.\linewidth]{figures/samples/mw-32/idem_samples_2plots.png}
        iDEM
    \end{minipage}
    \begin{minipage}[t]{0.195\linewidth}
        \centering
        \includegraphics[width=1.\linewidth]{figures/samples/mw-32/lfis_samples_2plots.png}
        LFIS
    \end{minipage}
    \begin{minipage}[t]{0.195\linewidth}
        \centering
        \includegraphics[width=1.\linewidth]{figures/samples/mw-32/nfs_samples_2plots.png}
        NFS (ours)
    \end{minipage}
    \caption{Samples on MW-32. First row: 2D marginal samples from the 1st and 4th dimensions; Second row: 2D marginal samples from the 2rd and 3rd dimensions.}
    \label{fig:mw32-visualised-2plots}
\end{figure}

\subsection{Visualisation of MW-32}

\begin{wrapfigure}{r}{0.5\linewidth}
    \centering
    \vspace{-4mm}
    \includegraphics[width=.47\textwidth]{figures/samples/mw-32/mw32_hist.png}
    \vspace{-2mm}
    \caption{Histogram of sample energy on MW-32.}
    \label{fig:mw32-energy-hist}
    \vspace{-4mm}
\end{wrapfigure}
This section presents additional visualizations of generated samples on MW-32. As shown in \cref{fig:mw32-visualised-2plots}, only FAB and \ours accurately capture the modes. While iDEM locates the modes, it struggles to identify their correct weights. Additionally, LFIS, another flow-based sampler similar to \ours, produces noisy samples, highlighting the high variance issue associated with importance sampling.
We further illustrate the histogram of sample energy on MW-32, where we draw the empirical energy distribution using 5,000 samples.
It shows that \ours achieves competitive performance with FAB, and notably outperforms iDEM and LFIS.




\subsection{Comparisons with Different Sampling Steps}
One key advantage of \ours is its ability to achieve high-quality results with fewer sampling steps. In this section, we compare \ours to the SOTA diffusion-based sampler iDEM and the flow-based sampler LFIS, using varying numbers of sampling steps. As demonstrated in \cref{fig:gmm-diff-steps,fig:mw32-diff-steps}, \ours produces better samples compared to both iDEM and LFIS, when using fewer sampling steps.

\section{Limitations and Future Work} \label{sec:appendix-limit-future-work}


A key challenge of flow-based samplers is the computation of the divergence (see \cref{eq:nfs_loss}), which becomes prohibitive in high-dimensional settings. While the Hutchinson estimator \citep{hutchinson1989stochastic} can be used in practice, it introduces both variance and bias. Alternatively, more advanced architectures can be employed where the divergence is computed analytically \citep{gerdes2023learning}. By adopting such architectures, we expect our approach to be scalable to more complex applications, such as molecular simulation \citep{frenkel2023understanding}, Lennard-Jones potential \citep{klein2024equivariant}, and Bayesian inference \citep{neal1993probabilistic}.

Rather than building off of the square error, the objective in \cref{eq:nfs_loss} can also be formulated using the Bregman divergence, which provides a more general framework for measuring discrepancies and can potentially lead to improved optimization properties, such as better convergence and robustness to outliers.
Moreover, while the shortcut model reduces the number of sampling steps required, achieving exact likelihood estimation within this framework remains unclear, presenting a promising direction for future research.

\begin{figure}[!t]
    \centering
    \begin{minipage}[t]{1.\linewidth}
        \centering
        \rotatebox{90}{\makebox[60pt]{iDEM}}
        \includegraphics[width=.97\linewidth]{figures/samples/gmm/idem_gmm_diff_stpes.jpg}
    \end{minipage}
    \begin{minipage}[t]{1.\linewidth}
        \centering
        \rotatebox{90}{\makebox[60pt]{LFIS}}
        \includegraphics[width=.97\linewidth]{figures/samples/gmm/lfis_gmm_diff_stpes.jpg}
    \end{minipage}
    \begin{minipage}[t]{1.\linewidth}
        \centering
        \hspace{-1.5mm} \rotatebox{90}{\makebox[70pt]{\ours}}
        \includegraphics[width=.97\linewidth]{figures/samples/gmm/nfs_gmm_diff_stpes.jpg}
    \end{minipage}
    \caption{Illustration of the generated samples using different sampling steps on GMM-40.}
    \label{fig:gmm-diff-steps}
\end{figure}

\begin{figure}[!t]
    \centering
    \begin{minipage}[t]{1.\linewidth}
        \centering
        \rotatebox{90}{\makebox[60pt]{iDEM}} 
        \includegraphics[width=.97\linewidth]{figures/samples/mw-32/idem_mw32_diff_stpes.jpg}
    \end{minipage}
    \begin{minipage}[t]{1.\linewidth}
        \centering
        \rotatebox{90}{\makebox[60pt]{LFIS}}
        \includegraphics[width=.97\linewidth]{figures/samples/mw-32/lfis_mw32_diff_stpes.jpg}
    \end{minipage}
    \begin{minipage}[t]{1.\linewidth}
        \centering
        \hspace{-1.5mm} \rotatebox{90}{\makebox[70pt]{\ours}}
        \includegraphics[width=.97\linewidth]{figures/samples/mw-32/nfs_mw32_diff_stpes.jpg}
    \end{minipage}
    \caption{Illustration of the generated samples using different sampling steps on MW-32.}
    \label{fig:mw32-diff-steps}
\end{figure}


\end{document}
