\section{Literature Review}
\label{literature_review}

% Introduction and background
Assessing the probability of a ball possession leading to a goal is fundamental in football analytics. Several methods have been suggested to measure this potential, each with distinct advantages and disadvantages.

Possession-based metrics such as VAEP ____ and xThreat (xT) ____ rely solely on \textit{event data}, which pertain to on-ball actions, to quantify the added value of actions in terms of goal-scoring probability.

% Existing metrics overview
Alternative metrics, such as Dangerousity ____, the value of passes model ____, and expected pass ____, utilize \textit{tracking data}, capturing the positions of all players on the pitch multiple times per second. This provides a deeper insight into game dynamics, yet often lacks the necessary interpretability for practical use. Additionally, these models often rely on simplified or partial representations of the current game state rather than integrating the full game state information, including each player's instantaneous position and velocity.

% EPV model description
____ introduce the Expected Possession Value (EPV) model to football as a framework for estimating the probability of a team scoring or conceding the next goal at any given moment during a possession. The model decomposes this value into the expected outcomes of three main actions: passes, ball drives, and shots. Using spatio-temporal tracking data, the model provides visually interpretable surfaces for assessing the expected value of passes, defined as the probability a goal is scored or conceded within the next 15 seconds of play, and the pass success probabilities, thereby allowing fine-grained evaluations of game states. ____ extend this framework by refining the granularity of analysis. This refinement involves explicit modeling of additional components like pass likelihood (for every location the likelihood it is the destination of a pass), dribble success evaluation, and action selection probabilities. ____ describe the results for each model, focusing on model calibration to improve interpretability and accuracy. Together, these models offer a comprehensive and interpretable tool for analyzing football dynamics, leveraging the spatial and contextual nuances captured in tracking data. Hereafter, we refer to this model as F21-EPV.

%and assessing the value of the three resulting categories of actions against the likelihood of those particular actions chosen by the ball carrier.

% \begin{itemize}
%     \item \textbf{Pass likelihood}: Calculates the probability of pass destinations.
%     \item \textbf{Dribble Evaluation}: Assesses dribble value, considering both the probability of success and its impact on goal probabilities.
%     \item \textbf{Action Selection}: Weighs the ball carrier's probability of choosing to pass, dribble, or shoot.
% \end{itemize}


% Areas for improvement
While F21-EPV represents a significant advance in football analytics, our analysis identifies and addresses the following areas of potential improvement.

\begin{enumerate}
    \item \textbf{Comparative Analysis:} 
        To our knowledge, there is currently no published method for assessing the relative value of game states to determine the more valuable state and comparing this with the output of an EPV model. We present a first step by focusing on the pass EPV model and by utilizing domain experts (see Section~\ref{benchmark_creation}).
    \item \textbf{Ball Height:}
        Previous models, including F21-EPV, do not account for the vertical dimension (height) of the ball. Our model incorporates the ball's z-axis (height), recognizing the distinct dynamics of aerial versus ground passes to improve accuracy. This addition is important, as the ball is often not at ground level during small portions of a football match. This property has received relatively little attention in football research, with the notable exception of ____, who demonstrated that aerial passes have a lower success rate than ground passes.

    \item \textbf{Cross-Competition Training:}
        Most existing models are trained on data from a single competition. We train our model on multiple competitions, specifically the Dutch Eredivisie and the 2022 World Cup.
    
    \item \textbf{Model Replicability:}  
        We encountered difficulties replicating the results reported by ____ when applying their approach to our dataset. To overcome this, we adopt an architecture inspired by the conventional U-Net framework ____. 

    \item \textbf{Risk-Reward Decomposition:}
        F21-EPV provides a single EPV value for each potential pass destination. In contrast, we decompose the value of passes into separate risk and reward components, providing insights into the potential benefits and drawbacks of each pass destination.
\end{enumerate}

% Conclusion
This research aims to improve the accuracy and interpretability of EPV models in football and to contribute to a deeper understanding of game dynamics and decision making. The OJN-Pass-EPV benchmark provides an evaluation framework for assessing the performance of pass models.