\section{Introduction}

% Background and Context
In recent years, football analytics has become increasingly important for gaining a competitive edge. This paper focuses on a specific metric in the expanding realm of football data analysis, \textit{Expected Possession Value} (EPV) (\cite{fernandez2019decomposing}), which quantifies the probability of scoring or conceding a goal given the current \textit{game state}.

% First Research Question
The first research question we address is how to evaluate the quality of any EPV model. This is a general question for the research field and to start answering it we introduce the OJN-Pass-EPV benchmark consisting of pairs of game states, where an expert panel determines the game state that they deem to be more dangerous. The OJN-Pass-EPV benchmark focuses on the relative \textit{pass} value of game states and as such can be used to evaluate \textit{pass models}.

% Second Research Question
Our second research question is whether we can replicate the results of \cite{Fernández2021} using a dataset from the Dutch Eredivisie and the 2022 World Cup before attempting to improve upon their model. Our attempts reveal discrepancies in model parameters and issues related to the vanishing gradient problem when employing a similar network structure for the pass model. To overcome this, we introduce a more classical U-net-type convolutional neural network structure. We test our model on the two datasets, delving into a relatively unexplored area of assessing the adaptability and robustness of a model across different levels of football competition.

% Pass Analysis Innovation
Finally, we focus on \textit{pass EPV}, splitting the \textit{pass value} into two distinct components: \textit{reward} and \textit{risk}, for both successful and unsuccessful passes. This approach allows for a more detailed assessment of each pass.
By separately quantifying the reward (the potential positive impact of a pass) and the risk (the potential negative impact of a pass), our model offers an improved view of player decision making. We furthermore include ball height as a new feature, as such adding the crucial vertical dimension of play and allowing to distinguish between aerial and ground passes.

% Main Contributions
Our main contributions are:
\begin{itemize}
    \item The OJN-Pass-EPV benchmark, consisting of pairs of game states with given relative EPVs, can be used to assess the quality of any pass EPV model.
    \item A new U-net-type convolutional neural network architecture for the evaluation of pass EPV.
    \item Assessment of the adaptability and robustness of our model in different levels of football competition.
    \item The splitting of the pass value into reward and risk for both successful and unsuccessful passes.
    \item The inclusion of ball height as a new feature.
\end{itemize}

% Paper Structure
The following sections review the relevant literature, detail our methodology, present the results, and discuss the implications of our findings for football analytics and strategy.
\newpage