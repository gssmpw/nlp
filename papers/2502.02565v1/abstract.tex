\abstract{
% Introduction
This paper introduces the first Expected Possession Value (EPV) benchmark and a new and improved EPV model for football. %
% First Research Question
Through the introduction of the OJN-Pass-EPV benchmark, we present a novel method to quantitatively assess the quality of EPV models by using pairs of game states with given relative EPVs. %
% Second Research Question
Next, we attempt to replicate the results of \cite{Fernández2021} using a dataset containing Dutch Eredivisie and World Cup matches. Following our failure to do so, we propose a new architecture based on U-net-type convolutional neural networks, achieving good results in model loss and Expected Calibration Error. %
% Third Research Question
Finally, we present an improved pass model that incorporates ball height and contains a new dual-component pass value model that analyzes reward and risk. %
% Results and Contributions
The resulting EPV model correctly identifies the higher value state in 78\% of the game state pairs in the OJN-Pass-EPV benchmark, demonstrating its ability to accurately assess goal-scoring potential. %
% Conclusion and Implications
Our findings can help assess the quality of EPV models, improve EPV predictions, help assess potential reward and risk of passing decisions, and improve player and team performance.}