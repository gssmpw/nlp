\section{Discussion}

This research contributes to the advancement of EPV models for football by introducing the first EPV benchmark dataset and by validating an enhanced EPV model across distinct competitions, namely the Dutch Eredivisie and the 2022 World Cup. The novel integration of ball height as a feature and the segmentation of pass value into distinct reward and risk components refine the model's predictive capabilities and augment its interpretability. These enhancements enable a more nuanced analysis of pass decisions, facilitating a deeper understanding of their potential outcomes.

In application, the model could serve as an aid in optimizing pass strategies by contrasting players' actual choices with theoretically optimal ones, as illustrated in Figure \ref{fig:optimal_value_location}. This comparison not only identifies opportunities for improvement but also highlights the importance of strategic pass selection in influencing game outcomes. 

\begin{figure}[h]
\centering
\includegraphics[width=1.0\linewidth]{Export_decision_making.png}
\caption{Analysis of pass decisions with OJN-EPV (based on the output in Definition~\ref{def:output}). This figure shows a player's actual pass (marked by "+") against the optimal pass location (circled) as identified by OJN-EPV. The comparison highlights potential areas for decision-making refinement, illustrating how OJN-EPV can assist in identifying more dangerous pass options.}
\label{fig:optimal_value_location}
\end{figure}

\newpage

\newtheorem{definition}{Definition}
\small
\begin{definition}\label{def:output}
\begin{gather*}
\mathrm{Output}(x, y) = \begin{cases} V(x, y) & \text{if } L(x, y) > 0.001 \\ 0 & \text{otherwise} \end{cases} \\
\text{where } V(x, y) = S(x, y) V_{s}(x, y) + (1 - S(x, y)) V_{u}(x, y) \\
V_s(x, y) = P_{\text{score}}(x, y | \text{success}) - P_{\text{concede}}(x, y | \text{success}) \\
V_u(x, y) = P_{\text{score}}(x, y | \text{no success}) - P_{\text{concede}}(x, y | \text{no success}) \\
\begin{aligned}
\text{and } & V(x,y) : \text{estimated value of a pass that ends up at location } (x, y),\\
& L(x, y) : \text{likelihood that a pass ends up at location } (x, y), \\
& S(x, y) : \text{probability of a successful pass to} (x, y), \\
& P_{\text{score}}(x, y | \text{success}) : \text{probability of scoring after a successful pass to } (x, y), \\
& P_{\text{concede}}(x, y | \text{success}) : \text{probability of conceding after a successful pass to } (x, y), \\
& P_{\text{score}}(x, y | \text{no success}) : \text{probability of scoring after an unsuccessful pass to } (x, y), \\
& P_{\text{concede}}(x, y | \text{no success}) : \text{probability of conceding after an unsuccessful pass to } (x, y), \\
\end{aligned}
\end{gather*}
\end{definition}

Definition~\ref{def:output} underscores how OJN-EPV can be applied to focus on likely pass destinations (those with $L(x, y) > 0.001$) before computing the value of a potential pass. By emphasizing a threshold that captures sufficiently probable pass locations, we concentrate our attention on meaningful pass options. 

While this work advances the state of EPV modeling, additional steps are required for broad practical adoption. For instance, OJN-EPV does not explicitly capture the intended recipient of a pass, and integrating that intention would help differentiate the quality of a decision from the quality of its execution \citep{Power2017, peralta2020seeing, dick2022can, spearman2017physics}. Including more extensive datasets, adding player-specific properties, and assessing EPV throughout entire matches rather than discrete events would further enrich the model. These efforts would facilitate an even deeper understanding of both individual and team performance, reinforcing the utility of EPV in guiding tactical strategies and player evaluations.

