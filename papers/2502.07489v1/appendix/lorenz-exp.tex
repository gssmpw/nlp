\section{Lorenz}\label{app:lorenz}
We conducted a forecasting experiment on the chaotic Lorenz Attractor: 
\begin{align}
    \frac{dx}{dt} &= \sigma (y - x) \\
    \frac{dy}{dt} &= x (\rho - z) - y \\
    \frac{dz}{dt} &= xy - \beta z
\end{align}

Instead of varying the constants as we do in \Bench, we set the parameters to 
\begin{itemize}
    \item $\rho = 28$
    \item $\sigma = 10$
    \item $\beta=\frac{8}{3}$
\end{itemize}
We vary the sample the initial states from $x \sim [1,3], y\sim[0,2] z\sim[0,2]$.
Following \citet{Gilpin2021.Chaosa}, the task is to forecast the final $\frac{1}{6}$ of the IMTS based on the initial $\frac{5}{6}$.
Since we always use the exact same constants, we create 200 instead of 2000 time series instances. 
Outside the mentioned changes, the experimental protocol is the same as the one used in our experiments with the ODEs from \Bench.

As shown in \Cref{tab:lorenz}, there is no model which significantly outperforms the constant baseline  GraFITi-C.

\begin{table}[!h]
    \scriptsize
    \caption{%
        Test MSE IMTS created on the chaotic Lorenz-Attractor.
    }\label{tab:lorenz}
    \centering
    \begin{tabular}{l cccccc}
        \toprule
    \\	Dataset & GRU-ODE & LinODEnet & CRU & Neural Flow & GraFITi & GraFITi-C
    \\	\midrule
        Lorenz & 1.334±0.097 & 1.313±0.123 & 1.292±0.087 & 1.345±0.105 & 1.294±0.090 & 1.303±0.079
    \\	\bottomrule
    \end{tabular}
\end{table}