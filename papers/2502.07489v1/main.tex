\documentclass{article}

\usepackage{iclr2025_conference}

\iclrfinalcopy

\usepackage[utf8]{inputenc} % allow utf-8 input
\usepackage[T1]{fontenc}    % use 8-bit T1 fonts
\usepackage{hyperref}       % hyperlinks
\usepackage{url}            % simple URL typesetting
\usepackage{booktabs}       % professional-quality tables
\usepackage{amsfonts}       % blackboard math symbols
\usepackage{nicefrac}       % compact symbols for 1/2, etc.
\usepackage{microtype}      % microtypography
\usepackage{xcolor}         % colors
\usepackage{natbib}
\usepackage{longtable}
\usepackage{amsmath}
\usepackage{amssymb}
\usepackage{caption}
\usepackage{graphicx}
\usepackage{cleveref}
\usepackage{mathtools}
\usepackage{textcomp}
\usepackage{todonotes}
\usepackage{amsthm}
\usepackage{float}
\usepackage{import}% better than \input, helps with relative paths
\usepackage{tikz}
\usepackage{pgfplots}
\usepackage{pgfplotstable}
\pgfplotsset{compat=1.18}
\usepackage{threeparttable}


%\title{Biological Differential Equations as a semisynthetic benchmark for Irregularly sampled multivariate Time Series Forecasting}

%Max proposal:

\title{\Bench: A Benchmark for Irregularly Sampled Multivariate Time Series Forecasting
Based on Biological ODEs}
% The \author macro works with any number of authors. There are two commands
% used to separate the names and addresses of multiple authors: \And and \AND.
%
% Using \And between authors leaves it to LaTeX to determine where to break the
% lines. Using \AND forces a line break at that point. So, if LaTeX puts 3 of 4
% authors names on the first line, and the last on the second line, try using
% \AND instead of \And before the third author name.


\author{%
  Christian Klötergens \\
  ISMLL \& VWFS DARC\\
  University of Hildesheim\\
  Hildesheim, Germany\\
  \texttt{kloetergens@ismll.de}\\
  \And
  Vijaya Krishna Yalavarthi\\
  ISMLL\\
  University of Hildesheim\\
  Hildesheim, Germany\\
  \texttt{yalavarthi@ismll.de}
  \And
  Randolf Scholz\\
  ISMLL\\
  University of Hildesheim\\
  Hildesheim, Germany\\
  \texttt{scholz@ismll.de}
  \And
  Maximilian Stubbemann\\
  ISMLL \& VWFS DARC\\
  University of Hildesheim\\
  Hildesheim, Germany\\
  \texttt{stubbemann@ismll.de}
  \And
  Stefan Born\\
  Institute of Mathematics\\
  TU Berlin\\
  Berlin, Germany\\
  \texttt{born@math.tu-berlin.de}
  \And
  Lars Schmidt-Thieme\\
  ISMLL \&VWFS DARC\\
  University of Hildesheim\\
  Hildesheim, Germany\\
  \texttt{schmidt-thieme@ismll.de}
  % examples of more authors
  % \And
  % Coauthor \\
  % Affiliation \\
  % Address \\
  % \texttt{email} \\
  % \AND
  % Coauthor \\
  % Affiliation \\
  % Address \\
  % \texttt{email} \\
  % \And
  % Coauthor \\
  % Affiliation \\
  % Address \\
  % \texttt{email} \\
  % \And
  % Coauthor \\
  % Affiliation \\
  % Address \\
  % \texttt{email} \\
}


\theoremstyle{plain}
\newtheorem{lemma}{Lemma}
\theoremstyle{definition}
\newtheorem{definition}{Definition}


%
\setlength\unitlength{1mm}
\newcommand{\twodots}{\mathinner {\ldotp \ldotp}}
% bb font symbols
\newcommand{\Rho}{\mathrm{P}}
\newcommand{\Tau}{\mathrm{T}}

\newfont{\bbb}{msbm10 scaled 700}
\newcommand{\CCC}{\mbox{\bbb C}}

\newfont{\bb}{msbm10 scaled 1100}
\newcommand{\CC}{\mbox{\bb C}}
\newcommand{\PP}{\mbox{\bb P}}
\newcommand{\RR}{\mbox{\bb R}}
\newcommand{\QQ}{\mbox{\bb Q}}
\newcommand{\ZZ}{\mbox{\bb Z}}
\newcommand{\FF}{\mbox{\bb F}}
\newcommand{\GG}{\mbox{\bb G}}
\newcommand{\EE}{\mbox{\bb E}}
\newcommand{\NN}{\mbox{\bb N}}
\newcommand{\KK}{\mbox{\bb K}}
\newcommand{\HH}{\mbox{\bb H}}
\newcommand{\SSS}{\mbox{\bb S}}
\newcommand{\UU}{\mbox{\bb U}}
\newcommand{\VV}{\mbox{\bb V}}


\newcommand{\yy}{\mathbbm{y}}
\newcommand{\xx}{\mathbbm{x}}
\newcommand{\zz}{\mathbbm{z}}
\newcommand{\sss}{\mathbbm{s}}
\newcommand{\rr}{\mathbbm{r}}
\newcommand{\pp}{\mathbbm{p}}
\newcommand{\qq}{\mathbbm{q}}
\newcommand{\ww}{\mathbbm{w}}
\newcommand{\hh}{\mathbbm{h}}
\newcommand{\vvv}{\mathbbm{v}}

% Vectors

\newcommand{\av}{{\bf a}}
\newcommand{\bv}{{\bf b}}
\newcommand{\cv}{{\bf c}}
\newcommand{\dv}{{\bf d}}
\newcommand{\ev}{{\bf e}}
\newcommand{\fv}{{\bf f}}
\newcommand{\gv}{{\bf g}}
\newcommand{\hv}{{\bf h}}
\newcommand{\iv}{{\bf i}}
\newcommand{\jv}{{\bf j}}
\newcommand{\kv}{{\bf k}}
\newcommand{\lv}{{\bf l}}
\newcommand{\mv}{{\bf m}}
\newcommand{\nv}{{\bf n}}
\newcommand{\ov}{{\bf o}}
\newcommand{\pv}{{\bf p}}
\newcommand{\qv}{{\bf q}}
\newcommand{\rv}{{\bf r}}
\newcommand{\sv}{{\bf s}}
\newcommand{\tv}{{\bf t}}
\newcommand{\uv}{{\bf u}}
\newcommand{\wv}{{\bf w}}
\newcommand{\vv}{{\bf v}}
\newcommand{\xv}{{\bf x}}
\newcommand{\yv}{{\bf y}}
\newcommand{\zv}{{\bf z}}
\newcommand{\zerov}{{\bf 0}}
\newcommand{\onev}{{\bf 1}}

% Matrices

\newcommand{\Am}{{\bf A}}
\newcommand{\Bm}{{\bf B}}
\newcommand{\Cm}{{\bf C}}
\newcommand{\Dm}{{\bf D}}
\newcommand{\Em}{{\bf E}}
\newcommand{\Fm}{{\bf F}}
\newcommand{\Gm}{{\bf G}}
\newcommand{\Hm}{{\bf H}}
\newcommand{\Id}{{\bf I}}
\newcommand{\Jm}{{\bf J}}
\newcommand{\Km}{{\bf K}}
\newcommand{\Lm}{{\bf L}}
\newcommand{\Mm}{{\bf M}}
\newcommand{\Nm}{{\bf N}}
\newcommand{\Om}{{\bf O}}
\newcommand{\Pm}{{\bf P}}
\newcommand{\Qm}{{\bf Q}}
\newcommand{\Rm}{{\bf R}}
\newcommand{\Sm}{{\bf S}}
\newcommand{\Tm}{{\bf T}}
\newcommand{\Um}{{\bf U}}
\newcommand{\Wm}{{\bf W}}
\newcommand{\Vm}{{\bf V}}
\newcommand{\Xm}{{\bf X}}
\newcommand{\Ym}{{\bf Y}}
\newcommand{\Zm}{{\bf Z}}

% Calligraphic

\newcommand{\Ac}{{\cal A}}
\newcommand{\Bc}{{\cal B}}
\newcommand{\Cc}{{\cal C}}
\newcommand{\Dc}{{\cal D}}
\newcommand{\Ec}{{\cal E}}
\newcommand{\Fc}{{\cal F}}
\newcommand{\Gc}{{\cal G}}
\newcommand{\Hc}{{\cal H}}
\newcommand{\Ic}{{\cal I}}
\newcommand{\Jc}{{\cal J}}
\newcommand{\Kc}{{\cal K}}
\newcommand{\Lc}{{\cal L}}
\newcommand{\Mc}{{\cal M}}
\newcommand{\Nc}{{\cal N}}
\newcommand{\nc}{{\cal n}}
\newcommand{\Oc}{{\cal O}}
\newcommand{\Pc}{{\cal P}}
\newcommand{\Qc}{{\cal Q}}
\newcommand{\Rc}{{\cal R}}
\newcommand{\Sc}{{\cal S}}
\newcommand{\Tc}{{\cal T}}
\newcommand{\Uc}{{\cal U}}
\newcommand{\Wc}{{\cal W}}
\newcommand{\Vc}{{\cal V}}
\newcommand{\Xc}{{\cal X}}
\newcommand{\Yc}{{\cal Y}}
\newcommand{\Zc}{{\cal Z}}

% Bold greek letters

\newcommand{\alphav}{\hbox{\boldmath$\alpha$}}
\newcommand{\betav}{\hbox{\boldmath$\beta$}}
\newcommand{\gammav}{\hbox{\boldmath$\gamma$}}
\newcommand{\deltav}{\hbox{\boldmath$\delta$}}
\newcommand{\etav}{\hbox{\boldmath$\eta$}}
\newcommand{\lambdav}{\hbox{\boldmath$\lambda$}}
\newcommand{\epsilonv}{\hbox{\boldmath$\epsilon$}}
\newcommand{\nuv}{\hbox{\boldmath$\nu$}}
\newcommand{\muv}{\hbox{\boldmath$\mu$}}
\newcommand{\zetav}{\hbox{\boldmath$\zeta$}}
\newcommand{\phiv}{\hbox{\boldmath$\phi$}}
\newcommand{\psiv}{\hbox{\boldmath$\psi$}}
\newcommand{\thetav}{\hbox{\boldmath$\theta$}}
\newcommand{\tauv}{\hbox{\boldmath$\tau$}}
\newcommand{\omegav}{\hbox{\boldmath$\omega$}}
\newcommand{\xiv}{\hbox{\boldmath$\xi$}}
\newcommand{\sigmav}{\hbox{\boldmath$\sigma$}}
\newcommand{\piv}{\hbox{\boldmath$\pi$}}
\newcommand{\rhov}{\hbox{\boldmath$\rho$}}
\newcommand{\upsilonv}{\hbox{\boldmath$\upsilon$}}

\newcommand{\Gammam}{\hbox{\boldmath$\Gamma$}}
\newcommand{\Lambdam}{\hbox{\boldmath$\Lambda$}}
\newcommand{\Deltam}{\hbox{\boldmath$\Delta$}}
\newcommand{\Sigmam}{\hbox{\boldmath$\Sigma$}}
\newcommand{\Phim}{\hbox{\boldmath$\Phi$}}
\newcommand{\Pim}{\hbox{\boldmath$\Pi$}}
\newcommand{\Psim}{\hbox{\boldmath$\Psi$}}
\newcommand{\Thetam}{\hbox{\boldmath$\Theta$}}
\newcommand{\Omegam}{\hbox{\boldmath$\Omega$}}
\newcommand{\Xim}{\hbox{\boldmath$\Xi$}}


% Sans Serif small case

\newcommand{\Gsf}{{\sf G}}

\newcommand{\asf}{{\sf a}}
\newcommand{\bsf}{{\sf b}}
\newcommand{\csf}{{\sf c}}
\newcommand{\dsf}{{\sf d}}
\newcommand{\esf}{{\sf e}}
\newcommand{\fsf}{{\sf f}}
\newcommand{\gsf}{{\sf g}}
\newcommand{\hsf}{{\sf h}}
\newcommand{\isf}{{\sf i}}
\newcommand{\jsf}{{\sf j}}
\newcommand{\ksf}{{\sf k}}
\newcommand{\lsf}{{\sf l}}
\newcommand{\msf}{{\sf m}}
\newcommand{\nsf}{{\sf n}}
\newcommand{\osf}{{\sf o}}
\newcommand{\psf}{{\sf p}}
\newcommand{\qsf}{{\sf q}}
\newcommand{\rsf}{{\sf r}}
\newcommand{\ssf}{{\sf s}}
\newcommand{\tsf}{{\sf t}}
\newcommand{\usf}{{\sf u}}
\newcommand{\wsf}{{\sf w}}
\newcommand{\vsf}{{\sf v}}
\newcommand{\xsf}{{\sf x}}
\newcommand{\ysf}{{\sf y}}
\newcommand{\zsf}{{\sf z}}


% mixed symbols

\newcommand{\sinc}{{\hbox{sinc}}}
\newcommand{\diag}{{\hbox{diag}}}
\renewcommand{\det}{{\hbox{det}}}
\newcommand{\trace}{{\hbox{tr}}}
\newcommand{\sign}{{\hbox{sign}}}
\renewcommand{\arg}{{\hbox{arg}}}
\newcommand{\var}{{\hbox{var}}}
\newcommand{\cov}{{\hbox{cov}}}
\newcommand{\Ei}{{\rm E}_{\rm i}}
\renewcommand{\Re}{{\rm Re}}
\renewcommand{\Im}{{\rm Im}}
\newcommand{\eqdef}{\stackrel{\Delta}{=}}
\newcommand{\defines}{{\,\,\stackrel{\scriptscriptstyle \bigtriangleup}{=}\,\,}}
\newcommand{\<}{\left\langle}
\renewcommand{\>}{\right\rangle}
\newcommand{\herm}{{\sf H}}
\newcommand{\trasp}{{\sf T}}
\newcommand{\transp}{{\sf T}}
\renewcommand{\vec}{{\rm vec}}
\newcommand{\Psf}{{\sf P}}
\newcommand{\SINR}{{\sf SINR}}
\newcommand{\SNR}{{\sf SNR}}
\newcommand{\MMSE}{{\sf MMSE}}
\newcommand{\REF}{{\RED [REF]}}

% Markov chain
\usepackage{stmaryrd} % for \mkv 
\newcommand{\mkv}{-\!\!\!\!\minuso\!\!\!\!-}

% Colors

\newcommand{\RED}{\color[rgb]{1.00,0.10,0.10}}
\newcommand{\BLUE}{\color[rgb]{0,0,0.90}}
\newcommand{\GREEN}{\color[rgb]{0,0.80,0.20}}

%%%%%%%%%%%%%%%%%%%%%%%%%%%%%%%%%%%%%%%%%%
\usepackage{hyperref}
\hypersetup{
    bookmarks=true,         % show bookmarks bar?
    unicode=false,          % non-Latin characters in AcrobatÕs bookmarks
    pdftoolbar=true,        % show AcrobatÕs toolbar?
    pdfmenubar=true,        % show AcrobatÕs menu?
    pdffitwindow=false,     % window fit to page when opened
    pdfstartview={FitH},    % fits the width of the page to the window
%    pdftitle={My title},    % title
%    pdfauthor={Author},     % author
%    pdfsubject={Subject},   % subject of the document
%    pdfcreator={Creator},   % creator of the document
%    pdfproducer={Producer}, % producer of the document
%    pdfkeywords={keyword1} {key2} {key3}, % list of keywords
    pdfnewwindow=true,      % links in new window
    colorlinks=true,       % false: boxed links; true: colored links
    linkcolor=red,          % color of internal links (change box color with linkbordercolor)
    citecolor=green,        % color of links to bibliography
    filecolor=blue,      % color of file links
    urlcolor=blue           % color of external links
}
%%%%%%%%%%%%%%%%%%%%%%%%%%%%%%%%%%%%%%%%%%%


\input{physics_patch.sty}
\usepackage{tikz}
\usepackage{pgfplots}
\usepackage{pgfplotstable}
\usepackage{wrapfig}
\usepackage{subcaption}
\pgfplotsset{compat=1.18}

\begin{document}

\maketitle
\begin{abstract}
%Research on forecasting irregularly sampled multivariate time series (IMTS)
%with missing values predominantly relies on just four datasets and a few small toy examples for evaluation.
%While ordinary differential equations (ODE) are the prevalent models in science and engineering, a baseline model that forecasts a constant value outperforms
%sophisticated ODE-based models from the last five years on three of these existing datasets.

%This unintuitive finding hampers further research on ODE-based models, a more plausible model
%family. In this paper, we develop a methodology to generate irregularly sampled
%multivariate time series (IMTS) datasets from ordinary differential equations
%and to select challenging instances via rejection sampling. 
%To address this limitation, we introduce the \textbf{P}hysiome \textbf{M}odel \textbf{R}epository-\textbf{F}orecasting-(\Bench),~a wide benchmark of IMTS
%datasets consisting of 50 individual datasets, derived from ordinary
%differential equations from biological research, that are stored in the Physiome Model Repository (PMR).
%
%Biological processes are well-suited for generating IMTS datasets, as they are inherently multivariate and irregularly measured in real-world experiments.
%Additionally, the PMR provides Python implementations of many of these models, allowing us to create \Bench~in an automated manner.
%While Biology researchers create their models based on very few and non-published observations,
%they enable us to create an arbitrary number of time series which relate to possible measurements of a real-world phenomenon. 
%\Bench is the first benchmark for IMTS forecasting that we are aware of and an order of magnitude larger
%than the current evaluation setting of four datasets. 

%The datasets included in \Bench~are diverse enough to highlight different strengths of competing models. 
%This results in different performance rankings of competing models for each dataset and the absence of a single best model. 
%In that, \Bench~differs from the currently used IMTS forecasting datasets, which is expected to provide a fresh impulse to research
%on IMTS forecasting models.

%on \Bench ODE-based models can play to their
%strength and our benchmark can differentiate in a meaningful way between
%different ODE-based models.
%This way, we hope to give a new impulse to research
%on ODE-based forecasting models.
State-of-the-art methods for forecasting irregularly sampled time series
with missing values predominantly rely on just four datasets and a few small toy examples for evaluation.
While ordinary differential equations (ODE) are the prevalent models in science and engineering, a baseline model that forecasts a constant value outperforms
ODE-based models from the last five years on three of these existing datasets.
This unintuitive finding hampers further research on ODE-based models, a more plausible model
family. In this paper, we develop a methodology to generate irregularly sampled
multivariate time series (IMTS) datasets from ordinary differential equations
and to select challenging instances via rejection sampling. Using this
methodology, we create~\Bench, a large and sophisticated benchmark of IMTS
datasets consisting of 50 individual datasets, derived from ODE models developed by 
research in Biology. \Bench~is the first
benchmark for IMTS forecasting that we are aware of and an order of magnitude larger
than the current evaluation setting. 
Using \Bench, we show qualitatively completely different results than those derived
from the current four datasets: on \Bench~deep learning methods based on ODEs can play to their
strength and our benchmark can differentiate in a meaningful way between
different IMTS forecasting models. This way, we expect to give a new impulse to research
on irregular time series modeling.

\end{abstract}

\import{content/}{01_intro_mst}
\import{content/}{02_problem_formulation}
\import{content/}{03_models}
\import{content/}{05_grafiti_const}
\import{content/}{06_dataset_creation}
\import{content/}{07_Experiments}
\import{content/}{09_extensions_and_limitations}
\import{content/}{04_related_work}
\import{content/}{08_conclusion}
\section*{Reproducibility Statement}

Our experiments can be reproduced by following the instructions provided in our Git repository:
\url{https://anonymous.4open.science/r/Phyisiome-ODE-E53D}. There are two options to obtain the datasets from \Bench. 
The first is to regenerate everything using the code and instructions in the repository. The second is to download the data from Zenodo. 
However, the second option will only be available after the double-blind review phase, as we could not find a way to publish data on Zenodo anonymously.

\bibliographystyle{abbrvnat}
\bibliography{references,references_extra}

\clearpage\appendix
\clearpage\import{appendix/}{rw_data_set_description}
\clearpage\import{appendix/}{proof}
\clearpage\import{appendix/}{lipschitz}
\clearpage\import{appendix/}{dataset_infos}
\clearpage\import{appendix/}{ode_links}
\clearpage\import{appendix/}{hyperparameters}
\clearpage\import{appendix/}{regularly-sampled-MTS-experiments}
\clearpage\import{appendix/}{lorenz-exp}
%\import{appendix}{ethics_license}
% \import{appendix/}{results.tex}

\end{document}
