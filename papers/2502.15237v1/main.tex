%%
%% This is file `sample-sigconf-authordraft.tex',
%% generated with the docstrip utility.
%%
%% The original source files were:
%%
%% samples.dtx  (with options: `all,proceedings,bibtex,authordraft')
%% 
%% IMPORTANT NOTICE:
%% 
%% For the copyright see the source file.
%% 
%% Any modified versions of this file must be renamed
%% with new filenames distinct from sample-sigconf-authordraft.tex.
%% 
%% For distribution of the original source see the terms
%% for copying and modification in the file samples.dtx.
%% 
%% This generated file may be distributed as long as the
%% original source files, as listed above, are part of the
%% same distribution. (The sources need not necessarily be
%% in the same archive or directory.)
%%
%%
%% Commands for TeXCount
%TC:macro \cite [option:text,text]
%TC:macro \citep [option:text,text]
%TC:macro \citet [option:text,text]
%TC:envir table 0 1
%TC:envir table* 0 1
%TC:envir tabular [ignore] word
%TC:envir displaymath 0 word
%TC:envir math 0 word
%TC:envir comment 0 0
%%
%%
%% The first command in your LaTeX source must be the \documentclass
%% command.
%%
%% For submission and review of your manuscript please change the
%% command to \documentclass[manuscript, screen, review]{acmart}.
%%
%% When submitting camera ready or to TAPS, please change the command
%% to \documentclass[sigconf]{acmart} or whichever template is required
%% for your publication.
%%
%%

\documentclass[sigconf,screen]{acmart}

%%
%% \BibTeX command to typeset BibTeX logo in the docs
\AtBeginDocument{%
  \providecommand\BibTeX{{%
    Bib\TeX}}}

%% Rights management information.  This information is sent to you
%% when you complete the rights form.  These commands have SAMPLE
%% values in them; it is your responsibility as an author to replace
%% the commands and values with those provided to you when you
%% complete the rights form.
\setcopyright{acmlicensed}
\copyrightyear{2024}
\acmYear{2024}
\acmDOI{XXXXXXX.XXXXXXX}

%% These commands are for a PROCEEDINGS abstract or paper.

\acmBooktitle{Companion Proceedings of the 33rd ACM Symposium on the Foundations of Software Engineering (FSE '25), June 23--27, 2025, Trondheim, Norway}
%%
%%  Uncomment \acmBooktitle if the title of the proceedings is different
%%  from ``Proceedings of ...''!
%%
%%\acmBooktitle{Woodstock '18: ACM Symposium on Neural Gaze Detection,
%%  June 03--05, 2018, Woodstock, NY}
\acmISBN{978-1-4503-XXXX-X/18/06}


%%
%% Submission ID.
%% Use this when submitting an article to a sponsored event. You'll
%% receive a unique submission ID from the organizers
%% of the event, and this ID should be used as the parameter to this command.
%%\acmSubmissionID{123-A56-BU3}

%%
%% For managing citations, it is recommended to use bibliography
%% files in BibTeX format.
%%
%% You can then either use BibTeX with the ACM-Reference-Format style,
%% or BibLaTeX with the acmnumeric or acmauthoryear sytles, that include
%% support for advanced citation of software artefact from the
%% biblatex-software package, also separately available on CTAN.
%%
%% Look at the sample-*-biblatex.tex files for templates showcasing
%% the biblatex styles.
%%

%%
%% The majority of ACM publications use numbered citations and
%% references.  The command \citestyle{authoryear} switches to the
%% "author year" style.
%%
%% If you are preparing content for an event
%% sponsored by ACM SIGGRAPH, you must use the "author year" style of
%% citations and references.
%% Uncommenting
%% the next command will enable that style.
%%\citestyle{acmauthoryear}


%%
%% end of the preamble, start of the body of the document source.
\usepackage{graphicx}
\usepackage{listings}
\usepackage{wrapfig}
\usepackage{caption}
\usepackage{booktabs}
\usepackage{tabularx}
\usepackage{float}
\usepackage{subcaption}
\usepackage{enumitem}
\usepackage{lscape}
\renewcommand{\theequation}{\arabic{equation}}
\usepackage{multirow}
\usepackage{multicol}
\usepackage{url}
\usepackage{amsmath}
\usepackage{ragged2e}
\usepackage{color}
\usepackage{tabularray}
\usepackage{xcolor}
\usepackage{subfiles}
%\usepackage[numbers]{natbib}

\begin{document}

%%
%% The "title" command has an optional parameter,
%% allowing the author to define a "short title" to be used in page headers.


% some options:
 
% \title{Building Better Enterprise AI Assistants: A Knowledge Graph Approach to Response Generation}
% \title{Knowledge Graph Enhanced RAG for Enterprise AI Assistants}
 % \title{Organized Retrieval Knowledge-base for Enterprise AI Assistants}
 \title{From Documents to Dialogue: Building KG-RAG Enhanced AI Assistants}

%%
%% The "author" command and its associated commands are used to define
%% the authors and their affiliations.
%% Of note is the shared affiliation of the first two authors, and the
%% "authornote" and "authornotemark" commands
%% used to denote shared contribution to the research.
\author{Manisha Mukherjee\textsuperscript{1}, Sungchul Kim\textsuperscript{2}, Xiang Chen\textsuperscript{2}, Dan Luo\textsuperscript{2}, Tong Yu\textsuperscript{2}, Tung Mai\textsuperscript{2}}

\affiliation{%
  \textsuperscript{1}Carnegie Mellon University, Pittsburgh, Pennsylvania, USA \\
  \textsuperscript{2}Adobe Research, San Jose, California \country{USA}
} 


% \author{
%   \begin{tabular}[t]{c c}
%     Manisha Mukherjee & Sungchul Kim, Xiang Chen, Dan Luo, Tong Yu, Tung Mai \\
%     \textit{Carnegie Mellon University} & \textit{Adobe Research} \\
%     Pittsburgh, Pennsylvania, USA & San Jose, California, USA
%   \end{tabular}
% }



%%
%% By default, the full list of authors will be used in the page
%% headers. Often, this list is too long, and will overlap
%% other information printed in the page headers. This command allows
%% the author to define a more concise list
%% of authors' names for this purpose.
\renewcommand{\shortauthors}{Mukherjee et al.}

%%
%% The abstract is a short summary of the work to be presented in the
%% article.
\begin{abstract}
The Adobe Experience Platform AI Assistant is a conversational tool that enables organizations to interact seamlessly with proprietary enterprise data through a chatbot. However, due to access restrictions, Large Language Models (LLMs) cannot retrieve these internal documents, limiting their ability to generate accurate zero-shot responses. To overcome this limitation, we use a Retrieval-Augmented Generation (RAG) framework powered by a Knowledge Graph (KG) to retrieve relevant information from external knowledge sources, enabling LLMs to answer questions over private or previously unseen document collections. In this paper, we propose a novel approach for building a high-quality, low-noise KG. We apply several techniques, including incremental entity resolution using seed concepts, similarity-based filtering to deduplicate entries, assigning confidence scores to entity-relation pairs to filter for high-confidence pairs, and linking facts to source documents for provenance. Our KG-RAG system retrieves relevant tuples, which are added to the user prompts context before being sent to the LLM generating the response. Our evaluation demonstrates that this approach significantly enhances response relevance, reducing \textit{irrelevant} answers by over \textbf{50\% } and increasing \textit{fully relevant} answers by \textbf{88\% } compared to the existing production system.



% AI assistants often generate inaccurate and contextually incorrect responses when retrieving information from unstructured sources. This paper explores the use of Knowledge Graphs (KGs) to improve AI assistants' ability to generate accurate and context-aware responses. By incrementally constructing KGs from seed concepts and assigning confidence levels to the relations between entities, we provide provenance and improved context understanding. Our method demonstrates an improved accuracy assessed through cosine similarity using LLM as a judge for the assessment of the relevance of answers.
\end{abstract}



%%
%% The code below is generated by the tool at http://dl.acm.org/ccs.cfm.
%% Please copy and paste the code instead of the example below.
%%



%%
%% Keywords. The author(s) should pick words that accurately describe
%% the work being presented. Separate the keywords with commas.
\keywords{Generative AI, LLM, KG, RAG, Knowledge Management}
%% A "teaser" image appears between the author and affiliation
%% information and the body of the document, and typically spans the
%% page.


\received{January 2025}


%%
%% This command processes the author and affiliation and title
%% information and builds the first part of the formatted document.
\maketitle


\section{Introduction}

\subfile{intro}

\section{Implementation}

\subfile{implementation}
\section{Evaluation}

\subfile{eval}
\section{Results and Discussion}
\subfile{results}

\section{Related Work}

\subfile{relatedWork}
\section{Conclusion}
\subfile{conclusion}

\bibliographystyle{ACM-Reference-Format}
\bibliography{sample-base}
\end{document}