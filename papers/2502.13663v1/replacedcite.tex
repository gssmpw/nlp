\section{Related Work}
\label{secRelatedWork}
% (3D)cellular-connected 
Interference management for the AUs, e.g., UAVs, has been extensively studied ____.
% To achieve interference mitigation, i
In ____, the three-dimensional maximal-ratio transmission beamforming is adopted by BSs to serve UAVs and ground users on the shared channel.
% In ____, by exploiting the sensing ability of UAV and idle BSs in the network, aerial-ground interference mitigation solutions for both uplink and downlink are introduced.
The work in ____ reviews traditional aerial-ground interference mitigation solutions, including inter-cell cooperation, dynamic frequency reuse, coordinated multipoint (CoMP), non-orthogonal multiple access (NOMA), beamforming, along with their respective drawbacks.
Then, solutions utilizing the sensing ability of UAVs and idle BSs in the network are proposed.
% Then, new solutions for both uplink and downlink are introduced by exploiting the sensing ability of UAV and idle BSs in the network.
In ____, downlink cooperative beamforming with interference transmission and cancellation is proposed to mitigate the interference to the UAV caused by the co-channel terrestrial transmissions.
% maximize the UAV's receive signal-to-interference-plus-noise ratio (SINR) by jointly optimizing the power allocations at all of its serving BSs for transmitting the UAV's and co-channel terrestrial users' messages.
% with UAVs and BSs collaboration are studied.
% 。
% With UAVs and BSs collaboration, aerial-ground interference mitigation solutions are studied in ____.
% To mitigate this interference, 
% several optimization-based methods, e.g., 


% In the cellular-connected UAV scenario, the downlink interference management has been studied ____.
% The DRL-based downlink interference management for the cellular-connected UAV has been studied ____.
% In aerial-terrestrial networks where the cellular-connected UAVs and terrestrial UEs coexist, the DRL-based downlink interference management has been studied ____.
The DRL-based downlink interference management in aerial-terrestrial networks has been explored, where the cellular-connected UAVs and terrestrial UEs coexist ____.
% In ____, to minimize the ergodic outage duration of the UAV, resource block allocation and downlink beamforming for UAV are determined in large-scale and in small-scale respectively.
In ____, to minimize the ergodic outage duration of the UAV, resource block allocation and downlink beamforming for UAV are determined by the D3QN agent in large-scale and an agent in small-scale respectively.
% the twin delayed DDPG (TD3)
% Each UAV can be associated with more than one BS. The power of all BSs is considered to be a fixed constant.
% In the cellular-connected UAV scenario, to minimize the ergodic outage duration of the UAV, DRL-based resource block (RB) allocation and downlink beamforming are proposed, where D3QN is used to determine RB allocation and twin delayed DDPG (TD3) is used to determine beamforming ____.
In ____, to maximize the throughput of terrestrial UEs under the transmission rate threshold for UAVs and terrestrial UEs, the number of muting cells and the slice time allocation are determined by a DQN agent.

% to achieve load balancing and interference management
% In this article, we focus on addressing the interference management in CATN through the multi-agent DRL (MADRL) based joint UA and beamforming, whose degraded version, 
The joint UA and power control based on multi-agent DRL (MADRL) has been extensively studied ____.
To address the challenge of environmental non-stationary posed by independent agents in ____, the handover and power allocation problem is modeled as a fully cooperative multi-agent task in ____, which is solved by a multi-agent proximal policy optimization (PPO) algorithm based on the centralized training with decentralized execution (CTDE) framework where each UE acts as an agent.
In ____, a MADRL framework with scalable observation and action spaces is developed, where each access point (AP) acts as an agent.
In ____, a hierarchical MADRL-based framework is proposed, where an edge server agent determines the UA, while multiple AP agents manage power control.
In ____, device association, spectrum allocation, and power allocation in heterogeneous networks are optimized, where each BS acts as a D3QN agent.
%  by a distributed coordinated multi-agent framework


% bai2022distributed, bai2024distributed
% In the joint-transmission scenario, where one UT can be served by a set of APs and each AP can only serve one UT at the same time, a joint UA and beamforming scheme based on multi-agent DRL has been studied ____, where a two-timescale framework is proposed.
A joint UA and beamforming scheme based on MADRL is proposed in ____, which adopts a two-timescale framework, i.e., UA and beamforming are determined on large and small time scales respectively.
This scheme is tailored for the joint transmission scenario, where one UE can be served by multiple APs.
%  and each AP can only serve one UE at the same time.
% small cell 
Besides, MADRL-based beamforming for noncoherent joint transmission is studied, where multi-antenna BSs jointly serve single-antenna UEs ____.
However, these schemes can not be directly applied to the non-joint transmission scenario, where one UE can be served by only one BS.
%  and each BS can serve multiple UEs at the same time.
To the best of our knowledge, research on DRL-based UA and beamforming scheme remains limited in the non-joint transmission scenario.
% DRL-based user association and beamforming is still an open research problem.


% based on safe DRL
Safe DRL has been studied for various applications due to its ability to deal with decision problems with safety constraints ____.
In ____, the safe DRL-based electric vehicles charging scheduling is studied, where the cost measures the deviation of battery energy from the charging target.
In ____, safe DRL is applied to the real-time optimal power flow problem.
In ____, safe DRL is applied to the UAV-aided task offloading with the energy consumption constraint of the UAV.
In ____, safe DRL is exploited to maximize the sum rate under the average rate per user constraint in the UAV-enabled wireless network.
% In ____, safe DRL is applied to the UAV-enabled wireless network to maximize the sum-rate under the average rate per user constraint.
Nonetheless, the application of safe DRL in wireless communication is still rare, and the aforementioned studies are limited to a single agent.
To this end, in this study, we design a safe DRL-based DDUACBF framework for the CATNs to maximize the sum rate of the TUs while ensuring the protection of the AUs.