\subsection{Superpixel Segmentation} \label{sec:segmentation}

Superpixels are perceptually meaningful, connected regions that group pixels based on similarities in color or other features, first introduced by \citet{renmalik2003}. Since then, various algorithmic approaches have been developed \citep{achanta2012, felzenszwalb2004}. Defining appropriate local regions is crucial for extracting spatial features in spectral-spatial models. While fixed-size windows (e.g., \citet{ertem2020}) have shown promising results, they constrain the ability to fully capture spatial context. In contrast, superpixels provide adaptive regions that enhance discriminative information, as demonstrated by \citet{fang2015kernals}. \citet{cui2018} further highlighted this by employing a superpixel-based random walker to refine an SVM probability map with significant success. Additionally, Cui et al. showed that superpixel spectra are more stable and less sensitive to noise than individual pixel spectra, making superpixel-based approaches more robust to image noise.


% The most common algorithm used in clustering based superpixel methods is Lloyd's algorithm \citep{lloyd2006}, a modified version of the popular k-means clustering algorithm. In the context of Lloyd's algorithm, let us first formalise the definition of a superpixel segmentation.

\begin{definition}[Superpixel Segmentation]  
Given an image \( I : \Lambda \to \mathbb{R}^d \), where \( \Lambda \subset \mathbb{Z}^2 \) represents the image domain, superpixel segmentation partitions \( \Lambda \) into a set of regions \(\{S_i\}_{i=1}^n\). Each superpixel \( S_i \) is defined as \( S_i = \{x \in \Lambda : f(x) = i\} \), where \( f: \Lambda \to \{1, \ldots, n\} \) is a labeling function that assigns each pixel \( x \) to one of the \( n \) superpixels based on a feature function.  
\end{definition}


We propose using the Felzenszwalb segmentation algorithm \citep{felzenszwalb2004} as an enhancement to superpixel-based methods, which predominantly rely on SLIC \citep{achanta2010slic} and its variants.

\begin{figure}[h]
    \centering
    \includegraphics[width=\linewidth]{method/images/felz_salinas.pdf}
    \caption{The Salinas HSI segmented using Felzenszwalb segmentation algorithm. \citep{felzenszwalb2004} The first figure shows a false-colored RGB image and other 3 shows the image segmented using a minimum size of 50, 100, 200 pixels respectively.}
    \label{fig:felz_salinas}
\end{figure}

% \begin{table}[h!]
%     \centering
%     \begin{tabular}{|c|c|c|c|c|}
%         \hline
%         Method & OA & AA & KA & No. \\
%         \hline
%         SLIC & 0.9969 & 0.9946 & 0.9966 & 2210 \\
%         \hline
%         Felzenszwalb & \textbf{0.9987} & \textbf{0.9955} & \textbf{0.9976} & 517 \\
%         \hline
%         Quickshift & 0.9978 & 0.9955 & 0.9976 & 274 \\
%         \hline
%     \end{tabular}
%     \caption{Segmentation performance if each superpixel was assigned by its most common class among its pixels, and then compared against ground truth labels. This test was performed on the SALINAS dataset. All parameters are set by default according to the skimage library \citep{skimage2014}, except for \texttt{n\_segments = 2000} for SLIC, and \texttt{min\_size = 100} for Felzenszwalb.}
%     \label{tab:my_label}
% \end{table}