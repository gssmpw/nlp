\section{Method} \label{sec:method}

\begin{figure*}[h!]
    \centering
    \includegraphics[width=\linewidth]{method/images/MOB-GCN.pdf}
    \caption{The proposed framework follows a structured pipeline for hyperspectral image (HSI) classification. First, the HSI is read and undergoes dimensionality reduction before applying superpixel segmentation. Features are then extracted from each superpixel and, along with the initial labeling, are used to construct a superpixel-based graph. This graph is first processed by a bottom encoder before undergoing recursive pooling and encoding at multiple resolutions. The latent representations from all resolutions, including the bottom encoding, are concatenated and passed into a final classifier. The predicted labels are then mapped back to the superpixel regions, producing the final classification of the HSI.}
    \label{fig:overall_architecture}
\end{figure*}

In this study, we aim to achieve precise classification predictions for a vast amount of unlabeled data while relying on a minimal set of labeled samples. We formulate the classification task within the framework of semi-supervised learning (SSL).

\begin{definition}[Semi-Supervised Classification Task]  
Given a labeled dataset \(\{(x_i, y_i)\}_{i=1}^l\) and a label set \(\mathcal{L} = \{1, \ldots, c\}\) where \(\{y_i\}_{i=1}^l \in \mathcal{L}\), the goal is to learn a function \(f: \mathbb{R}^d \rightarrow \mathbb{R}^{c+u}\) that leverages the unlabeled data \(\{x_k\}_{k=l+1}^{l+u}\) to enhance prediction accuracy for \(\{x_k\}_{k=l+1}^{l+u}\).  
\end{definition}

\subsection{Superpixel Segmentation} \label{sec:segmentation}

Superpixels are perceptually meaningful, connected regions that group pixels based on similarities in color or other features, first introduced by \citet{renmalik2003}. Since then, various algorithmic approaches have been developed \citep{achanta2012, felzenszwalb2004}. Defining appropriate local regions is crucial for extracting spatial features in spectral-spatial models. While fixed-size windows (e.g., \citet{ertem2020}) have shown promising results, they constrain the ability to fully capture spatial context. In contrast, superpixels provide adaptive regions that enhance discriminative information, as demonstrated by \citet{fang2015kernals}. \citet{cui2018} further highlighted this by employing a superpixel-based random walker to refine an SVM probability map with significant success. Additionally, Cui et al. showed that superpixel spectra are more stable and less sensitive to noise than individual pixel spectra, making superpixel-based approaches more robust to image noise.


% The most common algorithm used in clustering based superpixel methods is Lloyd's algorithm \citep{lloyd2006}, a modified version of the popular k-means clustering algorithm. In the context of Lloyd's algorithm, let us first formalise the definition of a superpixel segmentation.

\begin{definition}[Superpixel Segmentation]  
Given an image \( I : \Lambda \to \mathbb{R}^d \), where \( \Lambda \subset \mathbb{Z}^2 \) represents the image domain, superpixel segmentation partitions \( \Lambda \) into a set of regions \(\{S_i\}_{i=1}^n\). Each superpixel \( S_i \) is defined as \( S_i = \{x \in \Lambda : f(x) = i\} \), where \( f: \Lambda \to \{1, \ldots, n\} \) is a labeling function that assigns each pixel \( x \) to one of the \( n \) superpixels based on a feature function.  
\end{definition}


We propose using the Felzenszwalb segmentation algorithm \citep{felzenszwalb2004} as an enhancement to superpixel-based methods, which predominantly rely on SLIC \citep{achanta2010slic} and its variants.

\begin{figure}[h]
    \centering
    \includegraphics[width=\linewidth]{method/images/felz_salinas.pdf}
    \caption{The Salinas HSI segmented using Felzenszwalb segmentation algorithm. \citep{felzenszwalb2004} The first figure shows a false-colored RGB image and other 3 shows the image segmented using a minimum size of 50, 100, 200 pixels respectively.}
    \label{fig:felz_salinas}
\end{figure}

% \begin{table}[h!]
%     \centering
%     \begin{tabular}{|c|c|c|c|c|}
%         \hline
%         Method & OA & AA & KA & No. \\
%         \hline
%         SLIC & 0.9969 & 0.9946 & 0.9966 & 2210 \\
%         \hline
%         Felzenszwalb & \textbf{0.9987} & \textbf{0.9955} & \textbf{0.9976} & 517 \\
%         \hline
%         Quickshift & 0.9978 & 0.9955 & 0.9976 & 274 \\
%         \hline
%     \end{tabular}
%     \caption{Segmentation performance if each superpixel was assigned by its most common class among its pixels, and then compared against ground truth labels. This test was performed on the SALINAS dataset. All parameters are set by default according to the skimage library \citep{skimage2014}, except for \texttt{n\_segments = 2000} for SLIC, and \texttt{min\_size = 100} for Felzenszwalb.}
%     \label{tab:my_label}
% \end{table}

\subsection{Feature Extraction} \label{sec:feature}

The next step involves extracting relevant features from the superpixels, which will be used in the subsequent graph construction step. In this work, we adopt the feature extraction approach outlined by \citet{sellars2020} for superpixels.

For each superpixel \( S_i \), we extract three distinct types of features to enhance spatial and contextual information.  

1. \textbf{Mean Feature Vector (\(\vec{S}_i^m\))}:  
   To capture localized spatial information, we apply a mean filter to each superpixel, computing the mean feature vector as:  
   \begin{equation} \label{equ:mean_filter}
       \vec{S}_i^m = \frac{\sum_{j=1}^{n_i} \hat{I}(p_{i, j})}{n_i}.
   \end{equation}  

2. \textbf{Weighted Feature Vector (\(\vec{S}_i^w\))}:  
   To incorporate spatial relationships between neighboring superpixels, we compute a weighted combination of the mean feature vectors of adjacent superpixels. Adjacency is determined using 4-connectivity (left, right, up, and down) within the image grid. For each superpixel \( S_i \), let \( \zeta_i = \{z_1, z_2, \dots, z_J\} \) denote the set of indices of its \( J \) adjacent superpixels. The weighted feature vector is then given by:  
   \begin{equation} \label{equ:weighted_filter}
       \vec{S}_i^w = \sum_{j=1}^{J} w_{i, z_j} \vec{S}_{z_j}^m,
   \end{equation}  
   where the weight \( w_{i, z_j} \) between adjacent superpixels is computed using a softmax function:  
   \begin{equation} \label{equ:weight_softmax}
       w_{i, z_j} = \frac{\exp\left(-\|\vec{S}_{z_j}^m - \vec{S}_i^m\|_2^2 / h\right)}{\sum_{j=1}^{J} \exp\left(-\|\vec{S}_{z_j}^m - \vec{S}_i^m\|_2^2 / h\right)},
   \end{equation}  
   where \( h \) is a predefined scalar parameter.  

3. \textbf{Centroid Location (\(\vec{S}_i^p\))}:  
   Finally, to encode spatial positioning, we compute the centroid location of each superpixel as:  
   \begin{equation} \label{equ:centroid}
       \vec{S}_i^p = \frac{\sum_{j=1}^{n_i} p_{i, j}}{n_i}.
   \end{equation}

\subsection{Graph-based Classification} \label{sec:graphclass}

As noted by \citet{campsvalls2007}, many graph-based algorithms involve computing and manipulating large kernel matrices that include both labeled and unlabeled data. For an image with \( n \) pixels, the corresponding graph Laplacian matrix has a size of \( n \times n \), and its inversion via singular value decomposition has a computational complexity of \( O(n^3) \), making it impractical for large-scale applications. To mitigate this issue, instead of representing each pixel as a graph node, we use superpixels as nodes, significantly reducing the node count since \( K \ll n \). This approach allows efficient matrix operations without requiring approximations while also improving classification accuracy by defining meaningful local regions within the data.

Using the extracted features and the superpixel-based node set, we construct a weighted, undirected graph \( G = (V, E, W) \). The edge weight between adjacent superpixels \( S_i \) and \( S_j \) is defined using two Gaussian kernels:  
\begin{equation} \label{equ:graph_weight}
    w_{ij} = s_{ij}l_{ij},
\end{equation}  
where the individual components are given by:  
\begin{equation} \label{equ:sigma_s}
    s_{ij} = \exp\left(\frac{(\beta-1)\|\overline{S}_i^w - \overline{S}_j^w\|_2^2 - \beta\|\overline{S}_i^m - \overline{S}_j^m\|_2^2}{\sigma_s^2}\right),
\end{equation}  
\begin{equation} \label{equ:sigma_l}
    l_{ij} = \exp\left(\frac{-\|\overline{S}_i^p - \overline{S}_j^p\|_2^2}{\sigma_l^2}\right),
\end{equation}  
where \( \beta \) controls the balance between mean and weighted feature contributions, while \( \sigma_s \) and \( \sigma_l \) define the widths of the Gaussian kernels. The resulting weights range between 0 and 1, where a value of 1 indicates maximum similarity. The graph edges are determined using a \( k \)-nearest neighbors (KNN) approach, with edge weights defined as:  
\begin{equation} \label{equ:knn_graph}
    W_{ij} = \begin{cases}
    w_{ij}, & \text{if } i \text{ is one of the } k \text{ nearest neighbors of } j, \\
    & \text{or vice versa}, \\
    0,       & \text{otherwise}.
    \end{cases}
\end{equation}  

During training, a subset of labeled spectral pixels is randomly selected from the original hyperspectral image. The initial label of each superpixel is assigned as the average of the labels of its constituent pixels. If no labeled pixels exist within a superpixel, it remains unassigned initially. The label information is stored in a matrix \( Y \in \mathbb{R}^{K \times c} \), where \( c \) is the number of classes and \( K \) is the total number of superpixels. The entry \( Y_{vl} \) represents the seed label \( l \) for node \( v \). 

The weight matrix and initial labels are then processed using the Local and Global Consistency (LGC) algorithm \citep{zhou2004lgc}, a graph-based semi-supervised learning method that enforces smoothness over the graph structure by minimizing a cost function. The final label matrix \( F \in \mathbb{R}^{K \times c} \) is obtained by minimizing:  
\begin{equation} \label{equ:lcg_loss}
    Q(F) = \frac{1}{2} \sum_{i,j=1}^n W_{ij} \left\|\frac{F_i}{\sqrt{D_{ii}}} - \frac{F_j}{\sqrt{D_{jj}}}\right\|^2 + \frac{\mu}{2} \sum_{i=1}^n \sum_{c=1}^C -y_{ic} \log f_{ic},
\end{equation}  
where \( F^* = \arg\min Q(F) \) represents the optimal label assignment.

\begin{figure*}[h]
    \centering
    \includegraphics[width=\linewidth]{method/images/mgn_embeddings_with_edges.pdf}
    \caption{Graph construction visualization for the INDIAN dataset, with node placement based on TSNE embeddings of node features and node labels from GCN inference.}
    \label{fig:indian_graph}
\end{figure*}



\subsection{Multiresolution Graph Learning}

\subsubsection{General Construction}

Multiresolution Graph Networks (MGN), introduced by \citet{hy2019covariant}, offer a framework for analyzing graphs across multiple resolutions. Given an undirected, weighted graph \( \mathcal{G} = (V, E, \mathbf{A}, \mathbf{F}_v) \), where \( V \) and \( E \) denote the sets of nodes and edges, respectively, the graph structure is defined by the adjacency matrix \( \mathbf{A} \in \mathbb{R}^{|V| \times |V|} \). Each node is associated with a feature vector, represented as \( \mathbf{F}_v \in \mathbb{R}^{|V| \times d_v} \), capturing relevant attributes for downstream processing.

\paragraph{Graph Coarsening}
A \emph{K-cluster partition} of a graph divides its nodes into \( K \) mutually exclusive clusters, \( V_1, \ldots, V_K \), where each cluster forms a subgraph. The process of \emph{coarsening} involves constructing a reduced graph \( \widetilde{\mathcal{G}} \), in which each node represents an entire cluster, and edges between these nodes are weighted based on inter-cluster connections in the original graph. This procedure is applied iteratively across multiple levels, resulting in an \emph{L-level coarsening}, where the topmost level condenses the graph into a single node.

\paragraph{Multiresolution Graph Network (MGN)}
MGN operates by iteratively transforming a graph into coarser representations through three key components:
\begin{enumerate}
    \item \textbf{Clustering}: This step partitions the graph into clusters.
    \item \textbf{Encoder}: A graph neural network encodes each cluster into latent node features.
    \item \textbf{Pooling}: Latent features from each cluster are combined into a single vector, which is used to represent the coarser graph at the next level.
\end{enumerate}
These steps are repeated across all levels of resolution, with learnable parameters governing each component. The goal is to predict properties of the original graph by leveraging hierarchical structures.

\subsubsection{Learning to Cluster}

The clustering operation in MGN is differentiable and uses a soft assignment of nodes to clusters, optimized during training via a Gumbel-softmax approximation. This ensures that the clustering procedure can be incorporated into backpropagation for efficient learning.

In summary, MGN provides a scalable way to learn hierarchical representations of graphs by iteratively coarsening them while preserving node and edge information through learnable neural network layers.


% \subsection{Overall architecture} \label{sec:overall}

\begin{algorithm}
\caption{MultiscaleHSI}\label{alg:multiscalehsi}
\begin{algorithmic}
    \Procedure{MULTISCALEHSI}{Input Image}
        \State Segment superpixels using the Felzenszwalb algorithm
        \State Create mean vector features of the pixels inside superpixels using \ref{equ:mean_filter}
        \State Create weighted vector features using \ref{equ:weighted_filter} and \ref{equ:weight_softmax}
        \State Create centroidal features using \ref{equ:centroid}
        \State Construct K-nearest neighbor graph, where $w_{ij}$ is found by \ref{equ:sigma_s} and \ref{equ:sigma_l}
        \If {$i$ is one of the $k$ nearest neighbors of $j$}
        \State $\mathbf{W}_{ij} = w_{ij}$
        \Else
        \State $\mathbf{W}_{ij} = 0$
        \EndIf
        \State Stratified train-test split (5\% or 10\%) on ground truth pixels
        \If {a superpixel contains a labelled pixel}
        \State Label by their most common training pixel and assign for training
        \Else 
        \State Label randomly as pseudolabel.
        \EndIf
        \State Train MOB-GCN using Local and Global Consistency (LGC), given by \ref{equ:lcg_loss}
    \EndProcedure
\end{algorithmic}
\end{algorithm}



\subsection{Optimal scale selection} \label{sec:optimal}

Selecting the optimal scale for image segmentation is a critical aspect of Object-Based Image Analysis (OBIA) and interpretation. The optimal segmentation scale is defined as the scale at which image objects most accurately correspond to real-world ground features across the entire image. We based on \citet{20-photo-j}’s method for optimal scale selection and applied it for multiple optimal scale selections by selecting ``peaks'' from the relative changes and sort them by value - representing descending impact from number of scales. The optimal scale selection for HrHS image segmentation proceeds as follows:
\begin{enumerate}
    \item Segmenting the HrHS image at different scales and calculating CV;
    \item Detecting and removing outliers using IF algorithm;
    \item Calculating and constructing NN-nCV and NN-nRoC graphs;
    \item Inspecting NN-nRoC graphs and selecting an optimal scale.
\end{enumerate}
The coefficient of variation (CV) for a segment in a single-band image is computed as:  

\[
CV_{p} = \sqrt{\frac{\sum_{i=1}^{N} (x_i - \mu_p)^2}{N}},
\]

where \(CV_{p}\) represents the CV of segment \(p\), \(N\) is the total number of pixels in the segment, \(x_i\) is the intensity value of the \(i\)-th pixel, and \(\mu_p\) is the mean intensity of all pixels within the segment.  

For hyperspectral images, the mean CV across all spectral bands is given by:  

\[
CV_{p} = \frac{\sum_{j=1}^{B} CV_{p,j}}{B},
\]

where \(B\) is the total number of spectral bands, and \(CV_{p,j}\) is the CV of segment \(p\) in band \(j\). The overall average CV across all segments \(P\) in the image is computed as:  

\[
CV_{avg} = \frac{\sum_{p=1}^{P} CV_{p}}{P}.
\]

After computing \(CV_{avg}\), segments with extreme values are filtered out using the Isolation Forest (IF) algorithm \citep{liu2012}, reducing computational cost and memory usage. Once outliers are removed, the NN-nRoC (nearest neighbor-normalized rate of change) is computed using:  

\[
\text{NN-nRoC} = \frac{\left| \left(\frac{\text{NN-nCV}_{n} - \text{NN-nCV}_{n-1}}{\text{NN-nCV}_{n-1}} \right)\right|}{P},
\]

where \(\text{NN-nCV}_{n}\) and \(\text{NN-nCV}_{n-1}\) denote the averaged NN-normalized CV across all segments at scales \(n\) and \(n-1\), respectively.  

Finally, NN-nCV and NN-nRoC graphs are generated for each segmentation method, and the NN-nRoC graph is analyzed to determine the optimal segmentation scale. Peaks in the NN-nRoC graph, where NN-nCV changes significantly, indicate abrupt shifts in intra-segment homogeneity, which correspond to key segmentation scales.

\begin{table}
\centering
\caption{Optimal Scale Selection.}
\label{tab:optimal_scale_selection}
\begin{tabular}{|l|c|}
\hline
\textbf{Dataset} & \textbf{Optimal Scales} \\
\hline
\textbf{INDIAN}   & [42, 24, 17, 8, 4] \\
\hline 
\textbf{SALINAS}  & [55, 31, 23, 14, 10, 4] \\
\hline 
\textbf{PAVIA}    & [71, 17, 14, 8, 5] \\
\hline 
\textbf{BOTSWANA} & [9, 7, 5] \\
\hline 
\textbf{KENNEDY}  & [55, 17, 12, 6] \\
\hline 
\textbf{TORONTO}  & [55, 24, 22, 18] \\
\hline
\end{tabular}
\end{table}

