\section{Method} \label{sec:method}

\begin{figure*}[h!]
    \centering
    \includegraphics[width=\linewidth]{method/images/MOB-GCN.pdf}
    \caption{The proposed framework follows a structured pipeline for hyperspectral image (HSI) classification. First, the HSI is read and undergoes dimensionality reduction before applying superpixel segmentation. Features are then extracted from each superpixel and, along with the initial labeling, are used to construct a superpixel-based graph. This graph is first processed by a bottom encoder before undergoing recursive pooling and encoding at multiple resolutions. The latent representations from all resolutions, including the bottom encoding, are concatenated and passed into a final classifier. The predicted labels are then mapped back to the superpixel regions, producing the final classification of the HSI.}
    \label{fig:overall_architecture}
\end{figure*}

In this study, we aim to achieve precise classification predictions for a vast amount of unlabeled data while relying on a minimal set of labeled samples. We formulate the classification task within the framework of semi-supervised learning (SSL).

\begin{definition}[Semi-Supervised Classification Task]  
Given a labeled dataset \(\{(x_i, y_i)\}_{i=1}^l\) and a label set \(\mathcal{L} = \{1, \ldots, c\}\) where \(\{y_i\}_{i=1}^l \in \mathcal{L}\), the goal is to learn a function \(f: \mathbb{R}^d \rightarrow \mathbb{R}^{c+u}\) that leverages the unlabeled data \(\{x_k\}_{k=l+1}^{l+u}\) to enhance prediction accuracy for \(\{x_k\}_{k=l+1}^{l+u}\).  
\end{definition}

\begin{figure}[ht!]
\centering
\begin{tabular}{c@{}c@{}c@{}c}
Input & Ours & DFF & DFF+Ref  \\
\includegraphics[width=0.24\linewidth]{figures/segmentation/sedan/ours/no_seg/2.png} &
\includegraphics[width=0.24\linewidth]
{figures/segmentation/merged/ours/sedan/2.png} &
\includegraphics[width=0.24\linewidth]
{figures/segmentation/merged/dff/sedan/2.png} &
\includegraphics[width=0.24\linewidth]
{figures/segmentation/merged/ours_no_dep/sedan/2.png} \\ 
\includegraphics[width=0.24\linewidth]{figures/segmentation/sedan/ours/no_seg/8.png} &
\includegraphics[width=0.24\linewidth]
{figures/segmentation/merged/ours/sedan/8.png} &
\includegraphics[width=0.24\linewidth]
{figures/segmentation/merged/dff/sedan/8.png} &
\includegraphics[width=0.24\linewidth]
{figures/segmentation/merged/ours_no_dep/sedan/8.png} \\ 
\includegraphics[width=0.24\linewidth]{figures/segmentation/sedan/ours/no_seg/10.png} &
\includegraphics[width=0.24\linewidth]
{figures/segmentation/merged/ours/sedan/10.png} &
\includegraphics[width=0.24\linewidth]
{figures/segmentation/merged/dff/sedan/10.png} & 
\includegraphics[width=0.24\linewidth]
{figures/segmentation/merged/ours_no_dep/sedan/10.png} \\ 

\includegraphics[trim={0cm 0.6cm 0cm 2cm},clip, width=0.24\linewidth]
{figures/segmentation/car/ours/no_seg/8.png} &
\includegraphics[trim={0cm 0.6cm 0cm 2cm},clip, width=0.24\linewidth]
{figures/segmentation/merged/ours/car/8.png} &
\includegraphics[trim={0cm 0.6cm 0cm 2cm},clip, width=0.24\linewidth]
{figures/segmentation/merged/dff/car/8.png} &
\includegraphics[trim={0cm 0.6cm 0cm 2cm},clip, width=0.24\linewidth]
{figures/segmentation/merged/ours_no_dep/car/8.png} \\
\includegraphics[trim={0cm 0.6cm 0cm 2cm},clip, width=0.24\linewidth]
{figures/segmentation/car/ours/no_seg/56.png} &
\includegraphics[trim={0cm 0.6cm 0cm 2cm},clip, width=0.24\linewidth]
{figures/segmentation/merged/ours/car/56.png} &
\includegraphics[trim={0cm 0.6cm 0cm 2cm},clip, width=0.24\linewidth]
{figures/segmentation/merged/dff/car/56.png} &
\includegraphics[trim={0cm 0.6cm 0cm 2cm},clip, width=0.24\linewidth]
{figures/segmentation/merged/ours_no_dep/car/56.png} \\
\includegraphics[trim={0cm 0.6cm 0cm 2cm},clip, width=0.24\linewidth]
{figures/segmentation/car/ours/no_seg/192.png} &
\includegraphics[trim={0cm 0.6cm 0cm 2cm},clip, width=0.24\linewidth]
{figures/segmentation/merged/ours/car/192.png} &
\includegraphics[trim={0cm 0.6cm 0cm 2cm},clip, width=0.24\linewidth]
{figures/segmentation/merged/dff/car/192.png} &
\includegraphics[trim={0cm 0.6cm 0cm 2cm},clip, width=0.24\linewidth]
{figures/segmentation/merged/ours_no_dep/car/192.png} \\

\end{tabular}
\vspace{-0.3cm}
\caption{3D objects segmentation from three novel views, for the Sedan scene from real-world RefNeRF \cite{verbin2022refnerf} dataset for the objects of Bonet-top, Windshield, Hubcups and Wheels, and for the Car scene from synthetic Shiny Blender \cite{verbin2022refnerf} dataset for the objects of Windshield and Wheels. We compare our result to DFF~\cite{kobayashi2022decomposing} and to a baseline where DFF is optimized for features while RefNeRF is optimized for appearance (see \cref{sec:semantic_segmentation}).} 

\label{fig:segmentation}
\vspace{-0.3cm}
\end{figure}










\subsection{Feature Extraction} \label{sec:feature}

The next step involves extracting relevant features from the superpixels, which will be used in the subsequent graph construction step. In this work, we adopt the feature extraction approach outlined by \citet{sellars2020} for superpixels.

For each superpixel \( S_i \), we extract three distinct types of features to enhance spatial and contextual information.  

1. \textbf{Mean Feature Vector (\(\vec{S}_i^m\))}:  
   To capture localized spatial information, we apply a mean filter to each superpixel, computing the mean feature vector as:  
   \begin{equation} \label{equ:mean_filter}
       \vec{S}_i^m = \frac{\sum_{j=1}^{n_i} \hat{I}(p_{i, j})}{n_i}.
   \end{equation}  

2. \textbf{Weighted Feature Vector (\(\vec{S}_i^w\))}:  
   To incorporate spatial relationships between neighboring superpixels, we compute a weighted combination of the mean feature vectors of adjacent superpixels. Adjacency is determined using 4-connectivity (left, right, up, and down) within the image grid. For each superpixel \( S_i \), let \( \zeta_i = \{z_1, z_2, \dots, z_J\} \) denote the set of indices of its \( J \) adjacent superpixels. The weighted feature vector is then given by:  
   \begin{equation} \label{equ:weighted_filter}
       \vec{S}_i^w = \sum_{j=1}^{J} w_{i, z_j} \vec{S}_{z_j}^m,
   \end{equation}  
   where the weight \( w_{i, z_j} \) between adjacent superpixels is computed using a softmax function:  
   \begin{equation} \label{equ:weight_softmax}
       w_{i, z_j} = \frac{\exp\left(-\|\vec{S}_{z_j}^m - \vec{S}_i^m\|_2^2 / h\right)}{\sum_{j=1}^{J} \exp\left(-\|\vec{S}_{z_j}^m - \vec{S}_i^m\|_2^2 / h\right)},
   \end{equation}  
   where \( h \) is a predefined scalar parameter.  

3. \textbf{Centroid Location (\(\vec{S}_i^p\))}:  
   Finally, to encode spatial positioning, we compute the centroid location of each superpixel as:  
   \begin{equation} \label{equ:centroid}
       \vec{S}_i^p = \frac{\sum_{j=1}^{n_i} p_{i, j}}{n_i}.
   \end{equation}

\subsection{Graph-based Classification} \label{sec:graphclass}

As noted by \citet{campsvalls2007}, many graph-based algorithms involve computing and manipulating large kernel matrices that include both labeled and unlabeled data. For an image with \( n \) pixels, the corresponding graph Laplacian matrix has a size of \( n \times n \), and its inversion via singular value decomposition has a computational complexity of \( O(n^3) \), making it impractical for large-scale applications. To mitigate this issue, instead of representing each pixel as a graph node, we use superpixels as nodes, significantly reducing the node count since \( K \ll n \). This approach allows efficient matrix operations without requiring approximations while also improving classification accuracy by defining meaningful local regions within the data.

Using the extracted features and the superpixel-based node set, we construct a weighted, undirected graph \( G = (V, E, W) \). The edge weight between adjacent superpixels \( S_i \) and \( S_j \) is defined using two Gaussian kernels:  
\begin{equation} \label{equ:graph_weight}
    w_{ij} = s_{ij}l_{ij},
\end{equation}  
where the individual components are given by:  
\begin{equation} \label{equ:sigma_s}
    s_{ij} = \exp\left(\frac{(\beta-1)\|\overline{S}_i^w - \overline{S}_j^w\|_2^2 - \beta\|\overline{S}_i^m - \overline{S}_j^m\|_2^2}{\sigma_s^2}\right),
\end{equation}  
\begin{equation} \label{equ:sigma_l}
    l_{ij} = \exp\left(\frac{-\|\overline{S}_i^p - \overline{S}_j^p\|_2^2}{\sigma_l^2}\right),
\end{equation}  
where \( \beta \) controls the balance between mean and weighted feature contributions, while \( \sigma_s \) and \( \sigma_l \) define the widths of the Gaussian kernels. The resulting weights range between 0 and 1, where a value of 1 indicates maximum similarity. The graph edges are determined using a \( k \)-nearest neighbors (KNN) approach, with edge weights defined as:  
\begin{equation} \label{equ:knn_graph}
    W_{ij} = \begin{cases}
    w_{ij}, & \text{if } i \text{ is one of the } k \text{ nearest neighbors of } j, \\
    & \text{or vice versa}, \\
    0,       & \text{otherwise}.
    \end{cases}
\end{equation}  

During training, a subset of labeled spectral pixels is randomly selected from the original hyperspectral image. The initial label of each superpixel is assigned as the average of the labels of its constituent pixels. If no labeled pixels exist within a superpixel, it remains unassigned initially. The label information is stored in a matrix \( Y \in \mathbb{R}^{K \times c} \), where \( c \) is the number of classes and \( K \) is the total number of superpixels. The entry \( Y_{vl} \) represents the seed label \( l \) for node \( v \). 

The weight matrix and initial labels are then processed using the Local and Global Consistency (LGC) algorithm \citep{zhou2004lgc}, a graph-based semi-supervised learning method that enforces smoothness over the graph structure by minimizing a cost function. The final label matrix \( F \in \mathbb{R}^{K \times c} \) is obtained by minimizing:  
\begin{equation} \label{equ:lcg_loss}
    Q(F) = \frac{1}{2} \sum_{i,j=1}^n W_{ij} \left\|\frac{F_i}{\sqrt{D_{ii}}} - \frac{F_j}{\sqrt{D_{jj}}}\right\|^2 + \frac{\mu}{2} \sum_{i=1}^n \sum_{c=1}^C -y_{ic} \log f_{ic},
\end{equation}  
where \( F^* = \arg\min Q(F) \) represents the optimal label assignment.

\begin{figure*}[h]
    \centering
    \includegraphics[width=\linewidth]{method/images/mgn_embeddings_with_edges.pdf}
    \caption{Graph construction visualization for the INDIAN dataset, with node placement based on TSNE embeddings of node features and node labels from GCN inference.}
    \label{fig:indian_graph}
\end{figure*}



\subsection{Multiresolution Graph Learning}

\subsubsection{General Construction}

Multiresolution Graph Networks (MGN), introduced by \citet{hy2019covariant}, offer a framework for analyzing graphs across multiple resolutions. Given an undirected, weighted graph \( \mathcal{G} = (V, E, \mathbf{A}, \mathbf{F}_v) \), where \( V \) and \( E \) denote the sets of nodes and edges, respectively, the graph structure is defined by the adjacency matrix \( \mathbf{A} \in \mathbb{R}^{|V| \times |V|} \). Each node is associated with a feature vector, represented as \( \mathbf{F}_v \in \mathbb{R}^{|V| \times d_v} \), capturing relevant attributes for downstream processing.

\paragraph{Graph Coarsening}
A \emph{K-cluster partition} of a graph divides its nodes into \( K \) mutually exclusive clusters, \( V_1, \ldots, V_K \), where each cluster forms a subgraph. The process of \emph{coarsening} involves constructing a reduced graph \( \widetilde{\mathcal{G}} \), in which each node represents an entire cluster, and edges between these nodes are weighted based on inter-cluster connections in the original graph. This procedure is applied iteratively across multiple levels, resulting in an \emph{L-level coarsening}, where the topmost level condenses the graph into a single node.

\paragraph{Multiresolution Graph Network (MGN)}
MGN operates by iteratively transforming a graph into coarser representations through three key components:
\begin{enumerate}
    \item \textbf{Clustering}: This step partitions the graph into clusters.
    \item \textbf{Encoder}: A graph neural network encodes each cluster into latent node features.
    \item \textbf{Pooling}: Latent features from each cluster are combined into a single vector, which is used to represent the coarser graph at the next level.
\end{enumerate}
These steps are repeated across all levels of resolution, with learnable parameters governing each component. The goal is to predict properties of the original graph by leveraging hierarchical structures.

\subsubsection{Learning to Cluster}

The clustering operation in MGN is differentiable and uses a soft assignment of nodes to clusters, optimized during training via a Gumbel-softmax approximation. This ensures that the clustering procedure can be incorporated into backpropagation for efficient learning.

In summary, MGN provides a scalable way to learn hierarchical representations of graphs by iteratively coarsening them while preserving node and edge information through learnable neural network layers.


% \subsection{Overall architecture} \label{sec:overall}

\begin{algorithm}
\caption{MultiscaleHSI}\label{alg:multiscalehsi}
\begin{algorithmic}
    \Procedure{MULTISCALEHSI}{Input Image}
        \State Segment superpixels using the Felzenszwalb algorithm
        \State Create mean vector features of the pixels inside superpixels using \ref{equ:mean_filter}
        \State Create weighted vector features using \ref{equ:weighted_filter} and \ref{equ:weight_softmax}
        \State Create centroidal features using \ref{equ:centroid}
        \State Construct K-nearest neighbor graph, where $w_{ij}$ is found by \ref{equ:sigma_s} and \ref{equ:sigma_l}
        \If {$i$ is one of the $k$ nearest neighbors of $j$}
        \State $\mathbf{W}_{ij} = w_{ij}$
        \Else
        \State $\mathbf{W}_{ij} = 0$
        \EndIf
        \State Stratified train-test split (5\% or 10\%) on ground truth pixels
        \If {a superpixel contains a labelled pixel}
        \State Label by their most common training pixel and assign for training
        \Else 
        \State Label randomly as pseudolabel.
        \EndIf
        \State Train MOB-GCN using Local and Global Consistency (LGC), given by \ref{equ:lcg_loss}
    \EndProcedure
\end{algorithmic}
\end{algorithm}



\subsection{Classification Results of MOB-GCN} 

The MOB-GCN model, especially in its optimized form (MOB-GCN (Optimal)), consistently achieved the highest OA compared to single-scale GCNs and the non-optimized MOB-GCN. This indicates that incorporating multiscale information and selecting optimal scales significantly enhances classification performance.

See table \ref{table:5_percent_results}, with only 5\% of the data used for training , the MOB-GCN (Optimal) achieved an OA of 94.28\% on the Indian Pines dataset, while the single-scale GCN only achieved 92.85\%. For the Salinas dataset with 5\% training data, the MOB-GCN (Optimal) showed a substantial improvement with an OA of 98.85\%, compared to the GCN's 89.35\%. Similar trends were observed across other datasets like Pavia, Kennedy Space Center, Botswana and University of Toronto, with the optimized MOB-GCN consistently outperforming the other models.

The superior performance of the MOB-GCN (Optimal) is not limited to small training datasets, with the optimized model performing the best across all sample sizes (5\%, 10\%, and 20\%). This shows the model's robustness and adaptability to varying amounts of training data. The MOB-GCN models show marked improvement over the GCN, especially with smaller sample sizes.

The key to the MOB-GCN's performance is its ability to integrate features extracted from multiple segmentation scales, enabling it to capture both fine-grained details and broader contextual information. The multiscale approach allows for a more comprehensive understanding of complex structures within hyperspectral images, which leads to more accurate classification outcomes. By using superpixels as nodes in the graph, the MOB-GCN reduces the computational overhead associated with processing large hyperspectral images\cite{sellars2020}. 

\subsection{Comparing MOB-GCN with Single-Scale Methods}

The MOB-GCN consistently surpassed the single-scale GCN in performance across all datasets. The single-scale GCN often exhibited lower classification accuracy and a higher prevalence of speckle noise in the output maps, highlighting its limitations in capturing the hierarchical
spatial-spectral relationships within HSI data. In contrast, the MOB-GCN, which integrates information from multiple scales, generated smoother and more accurate classification maps.

\subsection{Impact of scale on MOB-GCN Performance}

The automatic optimal scale selection method, which identifies the most informative segmentation scales, is crucial to the success of the MOB-GCN. This method determines optimal scales based on the relative changes in the coefficient of variation (CV) across different segmentation scales. It selects the "peaks" of these changes, which indicate significant variations in image object heterogeneity. Our experiments demonstrated that combining features from 4–6 optimal segmentation scales was generally sufficient to achieve the desired classification performance across most datasets. The specific optimal scales for each dataset are provided in Table VI. For instance, the optimal scales for the Indian Pines dataset were 42, 24, 17, 8, and 4, while for the Salinas dataset, they were 55, 31, 23, 14, 10, and 4.

\begin{figure}[h!]
    \centering
    \includegraphics[width=\linewidth]{discussions/images/merged_inertia_std.pdf}
    \caption{Inertia and Superpixel standard deviation for K-means clustering assessment on the SALINAS dataset.}
    \label{fig:inertia_standard_deviation}
\end{figure}

\begin{figure}[h!]
    \centering
    \includegraphics[width=\linewidth]{discussions/images/NN-nROC.pdf}
    \caption{NN-nRoC on every number of clusters on the SALINAS Dataset.
}
    \label{fig:nn_nRoC}
\end{figure}

\begin{figure}[h!]
    \centering
    \includegraphics[width=\linewidth]{discussions/images/candidate_clusters.pdf}
    \caption{MGN Performance on training with 10 sizes of candidate clusters, each are chosen by descending NN-nROC value. Each test is repeated 5 times, and the last 3 are omitted due to poor performance. The test was performed on the SALINAS dataset.}
    \label{fig:candidate_size}
\end{figure}



