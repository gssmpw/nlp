% \subsection{Overall architecture} \label{sec:overall}

\begin{algorithm}
\caption{MultiscaleHSI}\label{alg:multiscalehsi}
\begin{algorithmic}
    \Procedure{MULTISCALEHSI}{Input Image}
        \State Segment superpixels using the Felzenszwalb algorithm
        \State Create mean vector features of the pixels inside superpixels using \ref{equ:mean_filter}
        \State Create weighted vector features using \ref{equ:weighted_filter} and \ref{equ:weight_softmax}
        \State Create centroidal features using \ref{equ:centroid}
        \State Construct K-nearest neighbor graph, where $w_{ij}$ is found by \ref{equ:sigma_s} and \ref{equ:sigma_l}
        \If {$i$ is one of the $k$ nearest neighbors of $j$}
        \State $\mathbf{W}_{ij} = w_{ij}$
        \Else
        \State $\mathbf{W}_{ij} = 0$
        \EndIf
        \State Stratified train-test split (5\% or 10\%) on ground truth pixels
        \If {a superpixel contains a labelled pixel}
        \State Label by their most common training pixel and assign for training
        \Else 
        \State Label randomly as pseudolabel.
        \EndIf
        \State Train MOB-GCN using Local and Global Consistency (LGC), given by \ref{equ:lcg_loss}
    \EndProcedure
\end{algorithmic}
\end{algorithm}

