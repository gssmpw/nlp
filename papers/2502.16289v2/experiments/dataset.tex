\subsection{Datasets} \label{sec:datasets}

We use six benchmark HSI datasets to evaluate our approach, which have the following characteristics. The first five can be found at \url{https://www.ehu.eus/ccwintco/index.php/Hyperspectral_Remote_Sensing_Scenes} and UT-HSI-301 can be found at \url{http://vclab.science.uoit.ca/datasets/ut-hsi301/}.

\begin{table*}[h!]
\centering
\caption{Comparison of Hyperspectral Datasets}
\begin{tabular}{|c|c|c|c|c|c|c|}
\hline
\textbf{Data} & \textbf{Location} & \textbf{Spatial resolution (m)} & \textbf{No. of bands} & \textbf{Range (nm)} & \textbf{Labelled data (class)} \\
\hline
Indian Pines Dataset & Indiana, USA & 20 & 200 & 400-2500 & 16  \\
\hline
Salinas & Salinas Valley, California, USA & 3.7 & 200 & 400-2500 & 16  \\
\hline
University of Pavia & Pavia, Italy & 1.3 & 115 & 430-860 & 9  \\
\hline
Botswana & Okavango Delta, Botswana & 30 & 242 & 400-2500 & 14  \\
\hline
Kennedy Space Center & Florida, USA & 18 & 224 & 400-2500 & 13  \\
\hline
University of Toronto & Bolton, Ontario, Canada & 0.3 & 301 & 400-1000 & 4 \\
\hline
\end{tabular}
\label{tab:hyperspectral-comparison}
\end{table*}

% \begin{itemize}
%     \item \textbf{Indian Pines Dataset.} The dataset was collected by an airborne visible/infrared imaging spectrometer (AVIRIS) sensor over an agricultural site in Indiana and has 16 classes. The data set consists of 145 $\times$ 145 pixels, 200 spectral bands rangeing from 400 nm to 2500 nm, and a spatial resolution of 20 m \citep{baumgardner2015220}.
    
%     \item \textbf{Salinas.} This image was also collected by the AVIRIS sensor over Salinas Valley, California, and contains 16 classes. The data set size is 512 $\times$ 217 pixels. The image has 200 spectral bands between 400 - 2500 nm, and a spatial resolution of 3.7 m.
    
%     \item \textbf{University of Pavia.} This image was acquired by the reflective optics system imaging spectrometer (ROSIS). The image (610 $\times$ 340 pixels) covers the Engineering School at the University of Pavia and has 9 classes. The image contains 115 spectral bands between 430 - 860 nm and has a spatial resolution of 1.3 m.
    
%     \item \textbf{Botswana.} The image was acquired in Okavango Delta, Botswana by the Hyperion sensor. The image (1476 $\times$ 256 pixels) has a spatial resolution of 30 m, covering a 7.7 km strip. The image has 242 bands between 400 - 2500 nm, and the labelled data has 14 classes.
    
%     \item \textbf{Kennedy Space Center.} The image was acquired over Kennedy Space Center (KSC) in Florida, on March 23, 1996, by the NASA AVIRIS sensor. AVIRIS acquires data at 224 bands between 400 - 2500 nm. The KSC image (512 $\times$ 614)  has a spatial resolution of 18 m. For classification purposes, 13 classes representing the various land cover types were labelled for model training.
    
%     \item \textbf{University of Toronto.} Hyperspectral-301 (UT-HSI-301) dataset consists of three hyperspectral images. The Hyperspectral image (724 $\times$ 632 pixels) collected in an urban-rural transitional area. We refer to this image as “Suburban” dataset. It was captured around Bolton area in southern Ontario, Canada. The image has 301 bands from 400 nm to 1000 nm. The image resolution is 0.3 m and the covered area is around 41182 square meters. \citep{20-photo-j}
% \end{itemize}

% We further examine MGN’s performance with a dataset that has not been extensively used in academia. University of Toronto Hyperspectral-301 is distinguished by its significantly higher resolution compared to most academic hyperspectral images, allowing us to evaluate the model's ability to handle detailed spectral information. 


% The increased resolution provides an opportunity to assess MOB-GCN’s effectiveness in extracting fine-grained features and its robustness in dealing with high-dimensional data. By leveraging this unique dataset, we aim to uncover new insights into the model’s strengths and limitations, potentially contributing valuable knowledge to the field of hyperspectral image analysis.
