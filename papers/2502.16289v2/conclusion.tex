\section{Conclusion} \label{sec:conclusion}

This novel MOB-GCN leverages a multiresolution approach, inspired by Multiresolution Graph Networks (MGN), to capture both fine-grained details and global context in HSI data. By integrating features from multiple segmentation scales, MOB-GCN achieves higher classification accuracy than single-scale GCN models. The MOB-GCN (Optimal) model, which incorporates an automatic optimal scale selection, consistently achieves the highest Overall Accuracy (OA) across various datasets and sample sizes. For instance, with only 5\% training data, the optimized MOB-GCN achieved significantly higher OA on the Salinas (98.85\% vs. 89.35\%) and Kennedy (93.69\% vs. 80.10\%) datasets, compared to the single-scale GCN model.

MOB-GCN is particularly advantageous when labeled data is limited, a common challenge in remote sensing image classification due to the time-consuming and labor-intensive nature of field data collection. By leveraging information from multiple resolutions, MOB-GCN enhances robustness and adaptability, achieving superior classification performance even with scarce training data.

MOB-GCN is optimized for computational efficiency by representing superpixels as graph nodes, significantly reducing the number of nodes compared to pixel-based methods. This smaller graph size allows for faster matrix inversions and other computations. Additionally, the multiresolution process, which includes graph coarsening at higher levels, further accelerates processing, enhancing the scalability of the approach for HSI analysis.

The automatic optimal scale selection method, based on analyzing the coefficient of variation (CV) across different segmentation scales, is essential to MOB-GCN's superior performance. By identifying the most significant scales, the model effectively captures important spatial-spectral features. Experiments show that combining features from 4–6 optimal segmentation scales was generally sufficient to achieve the desired classification performance across most datasets.

It is important to highlight that the benefits of multiresolution networks like MOB-GCN tend to diminish as the size and complexity of hyperspectral images increase. For very large HSI datasets, conventional GNNs may provide a more efficient alternative, as they can handle large graphs without the overhead of constructing multiscale hierarchies.  

Overall, MOB-GCN presents a practical approach for hyperspectral image analysis, particularly in applications that demand both high accuracy and computational efficiency. Its strong performance with limited training data further enhances its applicability, addressing the challenges associated with acquiring labeled hyperspectral data in remote sensing.
