\section{Related Work} \label{sec:related_work}

\paragraph{Graph Representation Learning} Graph representation learning is essential for leveraging structural information in graph-structured data by embedding it in low-dimensional spaces. \cite{deepwalk, node2vec, network_embedding, pmlr-v196-hy22a} Methods have evolved from early spectral techniques to sophisticated approaches such as graph neural networks (GNNs). Recent advances, such as graph transformers \cite{NEURIPS2019_9d63484a, kim2022pure, pmlr-v202-cai23b, 10.1063/5.0152833} and graph attention networks (GATs) \cite{velickovic2018graph}, adapt transformers and attention mechanisms to graph domains to capture long-range node interactions. These innovations enhance the processing and analysis of graph-structured data.

% \paragraph{Temporal Graph Learning} Temporal graph learning has become an increasingly prominent area of research, driven by the need to model dynamically changing relationships over time. It has been applied to various fields, such as traffic prediction \cite{10.5555/3304222.3304273, li2018diffusion, pmlr-v228-nguyen24a} and pandemic modeling \cite{pmlr-v184-hy22a, Tran2023MGLEPMG, nguyen2023predicting}. Traditional Graph Neural Networks (GNNs) are designed for static graph structures, limiting their effectiveness in dynamic scenarios where nodes and edges continuously evolve, such as video analysis or multi-view scene understanding. Temporal Graph Neural Networks (Temporal GNNs) \cite{rossi2020temporal, 10.1145/3543873.3587301} overcome these limitations by incorporating temporal dependencies and leveraging time series learning. This allows them to capture evolving patterns within dynamic graphs, significantly improving their predictive capabilities in applications requiring temporal modeling.

\paragraph{Multiscale Graph Methods} Multiscale graph methods effectively capture hierarchical structures and integrate local and global information. They are beneficial in domains with multiresolution characteristics like hyperspectral imaging. Multiresolution graph learning dynamically constructs hierarchical graph representations through graph coarsening. Previous works have proposed multiresolution Graph Neural Networks \cite{Hy_2023, pmlr-v184-hy22a, 10.1063/5.0152833, trang2024scalable} and Graph Transformers that use data-driven clustering to partition graphs into multiple levels. In hyperspectral imaging, multiscale methods are relatively underexplored, despite using superpixels for classification has shown promise. However, these methods often rely on a single resolution or scale, which may overlook hierarchical relationships crucial for robust classification.

\paragraph{Hyperspectral Image Classification} %The semi-supervised classification of hyperspectral images (HSIs) has been extensively studied in the remote sensing community. Numerous previous works have employed graph-based learning techniques, which are particularly relevant to our approach. In graph-based methods, data points are represented as nodes, with edges and weights reflecting the similarity between them. 
The semi-supervised classification of hyperspectral images (HSIs) has been a focal point of research within the remote sensing community, with graph-based learning techniques emerging as a prominent approach. In these methods, data points are represented as nodes, while edges and weights encode the similarity between them, enabling effective spatial-spectral modeling.
One of the pioneering methods in this area was introduced by \citet{campsvalls2007}, where a combination of spectral and spatial kernels, along with the Nystr\"om extension for matrix approximation, was used for HSI classification. However, this method exhibited relatively low accuracy compared to more recent techniques. Later, \citet{gao2014} improved the performance by introducing a bilayer graph-based learning algorithm. Their approach combined a pixel-based graph, similar to \cite{campsvalls2007}, with a hypergraph constructed from grouping relations derived through unsupervised learning. 
% Additionally, \citet{cui2018} employed an extended random walker (ERW) on a superpixel-based graph to refine classification results obtained from a support vector machine (SVM), significantly enhancing the SVM's accuracy by leveraging graph-based information.

% In graph theory, several multiresolution and multiscale algorithms have been proposed, such as spectral graph wavelets \citep{hammond2009}, diffusion wavelets \citep{coifmanmaggioni2006}, graph wavelet transform via multiresolution matrix factorization \citep{hykondor2021}, and multilevel graph clustering \citep{dhillon2005, dhillon2007}. With the rise of deep learning, graph neural networks (GNNs), which generalize conventional convolutional neural networks to graphs, have become increasingly popular and are replacing traditional graph algorithms \citep{scarselli2009, duvenaud2015, niepert2016, gilmer2017, hy2019covariant, kondor2018, maron2018, thiede2020}.


% Old
% Our work is distinctive in that prior research has not fully addressed the challenge of simultaneously capturing both local and global information in graph-based HSI classification. By considering information at multiple resolutions, models can achieve a deeper understanding of both fine-grained details and the broader context of an image, leading to more accurate and robust predictions. This approach is inspired by the human visual system's hierarchical processing, where information is analyzed from coarse to fine scales. In this paper, we introduce an extension of multiresolution graph neural networks (MGNNs) \cite{Hy_2023} to construct a hierarchy of resolutions, with an attention mechanism to select the appropriate resolution at each stage of the process, improving overall classification performance.

\paragraph{Object-Based Image Analysis}
OBIA techniques cluster pixels with similar spectral and textural characteristics to create image objects that more accurately represent real-world surface features. By classifying a smaller number of image objects, the OBIA approaches are more computationally efficient compared to pixel-based methods. Previous studies have proposed methods for determining optimal segmentation scales but are limited to utilizing features from a single segmentation scale in classification. In addition to the pixel's grey value, more features can be extracted and be included in the classification step to improve accuracy in the OBIA approaches \citep{20-photo-j, dao2019, dao2015}. By grouping pixels to form image objects, the method reduces the speckle noise effect in the classified image and generates more meaningful classification results \citep{he2015}. To achieve accurate and robust classification results in OBIA approaches, it is critical to determine optimal segmentation scales and extract neccessary features for classification models. Previous studies \citep{20-photo-j, draguct2010esp, sellars2020, wan2019multiscaledynamicgraphconvolutional, zhang2021hyperspectralimagesegmentationbased} have proposed several optimal scale selection methods and achieved promissing results. However, these studies are limited to utilizing features from a single segmentation scale in classification, and no study has incorporated information extracted from multiple optimal scales to improve classification outcomes.

% New
Our work addresses these issues by integrating multiresolution graph learning with object-based image analysis into a unified framework. Unlike traditional single-scale GNNs or static superpixel-based methods, we employ Felzenszwalb’s superpixel segmentation and construct a dynamic multiscale graph hierarchy to model both fine-grained and global spatial patterns.
