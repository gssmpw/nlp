\section{Introduction} \label{sec:introduction}

Hyperspectral images (HSIs) provide rich spectral information, making them valuable for various applications. \citep{lu2020} This unique characteristic has led to the widespread use of HSI in various applications, such as classification \citep{melgani2004, fang2015, fang2018, DAO2021102542}, object tracking \citep{wang2010, uzkent2016, uzkent2017} and object detection \citep{pan2003, liu2016, zhang2016}, environmental monitoring \citep{ellis2004, manfreda2018}, and vegetation health assessment \citep{DAO2021102364}. However, classification of HSIs remains challenging due to the limited availability of labeled training data, and the high dimensionality of the data, significant spatial variability. Traditional pixel-based classification methods often suffer from low accuracy and speckle noise. To tackle these issues, object-based image analysis (OBIA) has emerged as a promising image interpretation approach as it reduces noise in the classified map and improves classification accuracy and computational efficiency \citep{20-photo-j, blaschke2010object}. Object-Based Image Analysis (OBIA) involves two main steps: image segmentation, which clusters pixels into meaningful objects, and image classification, which categorizes these objects into specific classes. Between the two steps, determining the optimal segmentation scales and extracting object features are critical to achieving high-quality classification outcomes \citep{20-photo-j, draguct2010esp}. Previous studies are limited to utilizing features from a single segmentation scale \citep{20-photo-j}, which can overlook important hierarchical relationships within the data.
 

% Recent studies have focused heavily on the pixel-based classification of hyperspectral data, where each pixel in an image is assigned to a class. Key challenges include the high dimensionality of the spectral data, significant spatial variability, and the limited availability of labeled training data due to the cost and complexity of data annotation. The pixel-based approach also suffers from low classification accuracy and speckle noise effect on the output classified map. Various solutions to these challenges have been explored, primarily through supervised learning and object-based image analysis (OBIA) approaches. 



% Kernel-based methods, such as support vector machines (SVM) \citep{melgani2004, mercier2003}, have been extensively applied to HSI classification. While early kernel methods utilized only spectral features, later methods, such as the multiple kernel learning (MKL) approach of \citet{fang2015kernals}, combined spatial and spectral features to improve performance.

% To address the high dimensionality of HSIs, several feature extraction techniques have been developed. These methods aim to reduce dimensionality while preserving separability between classes. For example, \citet{kang2014} employed image fusion and recursive filtering, \citet{li2015} leveraged local binary patterns for texture extraction, and \citet{fang2018} used covariance matrix representations to capture spectral-spatial correlations.

% Deep learning has significantly advanced HSI classification, with convolutional neural networks (CNNs) being widely applied to extract spatial and spectral features \citep{makantasis2015, zhao2016}. \citet{makantasis2015} used CNNs to extract high-level features for a multilayer perceptron, while \citet{lin2018} demonstrated promising results using generative adversarial networks (GANs). Graph neural networks (GNNs), which extend the concept of convolution to graph structures, have been successfully applied to a wide range of tasks, including physical system modeling \citep{battaglia2016}, molecular representation learning \citep{kearnes2016}, citation network analysis \citep{kipf2016}, and community detection in social networks \citep{su2022}. The Message Passing Neural Network (MPNN) \citep{gilmer2017} is a widely adopted GNN framework, where nodes exchange information through localized message-passing. However, MPNNs lack the ability to capture both local and global graph features simultaneously. Furthermore, the method is computationally expensive and impractical when applied to a hyperspectral image at the pixel level, as it requires one to construct and process a graph from every pixel with a large number of bands.

% Supervised methods achieve strong performance, but require extensive labeled datasets, which are costly to acquire. Unsupervised learning, though free from labeled data, faces challenges like ill-posed clustering and bridging clusters to known classes \citep{barlow1989, zhu2016}. Semi-supervised learning (SSL) \citep{chapelle2006} balances these approaches by leveraging both labeled and unlabeled data to improve classification. For instance, the Superpixel Graph Learning (SGL) framework \citep{sellars2020} uses superpixels to build a graph representation of HSIs, classified with the Local and Global Consistency (LGC) method \citep{zhou2004lgc}.

To overcome this limitation, \citet{Hy_2023} introduced Multiresolution Graph Networks (MGN), which dynamically construct multiple resolutions of the input graph using data-driven clustering. MGN employs two GNN modules at each resolution: one for graph representation learning and another for graph coarsening. This adaptive clustering process is crucial to capture both local and global features, allowing the model to gradually focus on broader structures as necessary.

Motivated by the flexibility and efficiency of MGN, we propose a novel Multiscale Object-Based Graph Convolutional Network (MOB-GCN) that incorporates a multiresolution mechanism for robust HSI classification. Our approach utilizes the MGN architecture and provides an ablation study on benchmark datasets. In summary, our contributions are:
\begin{enumerate} 
    \item Developing an automatic optimal segmentation scale determination method for effective extraction of image object's spatial and spectral features from hyperspectral images at at different scales;
    \item Developing a novel multiscale object-based classification method that integrates features extracted from multiple segmentation scales to improve the overall classification accuracy;
    \item Comparing the performance of our proposed MOB-GCN model with the single-scale GCN model to demonstrate the advantages of our multiscale method.
\end{enumerate}


This approach integrates multiresolution graph learning into a unified framework, providing a more comprehensive understanding of hyperspectral images compared to single-scale methods.

% The proposed framework includes an automatic optimal segmentation scale determination method for effective feature extraction. The method also incorporates multiple segmentation scales to improve overall classification accuracy, and then compares the performance of the proposed model with a single-scale GCN model to demonstrate its advantages.
