\section{PAVIA at 5\% Sample Data} \label{sec:apx_pavia}



\begin{figure}[h!]
\centering
\begin{tabular}{cc}
\includegraphics[width=0.5\textwidth]{appendix/images/PAVIA/metric_comparison.pdf} &
\includegraphics[width=0.5\textwidth]{appendix/images/PAVIA/MOB-GCN_Optimal_confusion_matrix.pdf} \\
\end{tabular}
\caption{PAVIA on 5\% Sample Data. Model comparison and Average Confusion Matrix from MOB-GCN (Optimal).}
\label{fig:pavia_metrics}
\end{figure}

\begin{figure}[h!]
\centering
\begin{tabular}{cc}
\includegraphics[width=0.5\textwidth]{appendix/images/PAVIA/NN-nROC.pdf} &
\includegraphics[width=0.5\textwidth]{appendix/images/PAVIA/candidate_clusters.pdf} \\
\end{tabular}
\caption{PAVIA on 5\% Sample Data. NN-nRoC on every number of clusters on the PAVIA Dataset. Peaks on PAVIA’ NN-nROC are placed at 5, 8, 17, 14, 71, 56, 50, 58, 40, 25 with descending value.}
\label{fig:pavia_scales}
\end{figure}