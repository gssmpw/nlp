\section{BOTSWANA at 5\% Sample Data} \label{sec:apx_botswana}




\begin{figure}[h!]
\centering
\begin{tabular}{cc}
\includegraphics[width=0.5\textwidth]{appendix/images/BOTSWANA/metric_comparison.pdf} &
\includegraphics[width=0.5\textwidth]{appendix/images/BOTSWANA/MOB-GCN_Optimal_confusion_matrix.pdf} \\
\end{tabular}
\caption{BOTSWANA on 5\% Sample Data. Model comparison and Average Confusion Matrix from MOB-GCN (Optimal).}
\label{fig:botswana_metrics}
\end{figure}

\begin{figure}[h!]
\centering
\begin{tabular}{cc}
\includegraphics[width=0.5\textwidth]{appendix/images/BOTSWANA/NN-nROC.pdf} &
\includegraphics[width=0.5\textwidth]{appendix/images/BOTSWANA/candidate_clusters.pdf} \\
\end{tabular}
\caption{BOTSWANA on 5\% Sample Data. NN-nRoC on every number of clusters on the BOTSWANA Dataset. Peaks on PAVIA’ NN-nROC are placed at 5, 7, 9, 33, 13, 20, 30, 15, 58, 42 with descending value.}
\label{fig:botswana_scales}
\end{figure}