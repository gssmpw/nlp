\section{INDIAN at 5\% Sample Data} \label{sec:apx_indian}

\begin{figure}[h!]
\centering
\begin{tabular}{cc}
\includegraphics[width=0.5\textwidth]{appendix/images/INDIAN/metric_comparison.pdf} &
\includegraphics[width=0.5\textwidth]{appendix/images/INDIAN/MOB-GCN_Optimal_confusion_matrix.pdf} \\
\end{tabular}
\caption{INDIAN on 5\% Sample Data. Model comparison and Average Confusion Matrix from MOB-GCN (Optimal).}
\label{fig:indian_metrics}
\end{figure}

\begin{figure}[h!]
\centering
\begin{tabular}{cc}
\includegraphics[width=0.5\textwidth]{appendix/images/INDIAN/NN-nROC.pdf} &
\includegraphics[width=0.5\textwidth]{appendix/images/INDIAN/candidate_clusters.pdf} \\
\end{tabular}
\caption{INDIAN on 5\% Sample Data. NN-nRoC on every number of clusters on the INDIAN Dataset. Peaks on INDIAN’ NN-nROC are placed at 4, 17, 8, 24, 42, 84, 102, 47, 75, 55 with descending value.}
\label{fig:indian_scales}
\end{figure}
