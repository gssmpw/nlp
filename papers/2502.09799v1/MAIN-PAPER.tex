
%%
%% This is file `sample-manuscript.tex',
%% generated with the docstrip utility.
%%
%% The original source files were:
%%
%% samples.dtx  (with options: `all,proceedings,bibtex,manuscript')
%% 
%% IMPORTANT NOTICE:
%% 
%% For the copyright see the source file.
%% 
%% Any modified versions of this file must be renamed
%% with new filenames distinct from sample-manuscript.tex.
%% 
%% For distribution of the original source see the terms
%% for copying and modification in the file samples.dtx.
%% 
%% This generated file may be distributed as long as the
%% original source files, as listed above, are part of the
%% same distribution. (The sources need not necessarily be
%% in the same archive or directory.)
%%
%%
%% Commands for TeXCount
%TC:macro \cite [option:text,text]
%TC:macro \citep [option:text,text]
%TC:macro \citet [option:text,text]
%TC:envir table 0 1
%TC:envir table* 0 1
%TC:envir tabular [ignore] word
%TC:envir displaymath 0 word
%TC:envir math 0 word
%TC:envir comment 0 0
%%
%%
%% The first command in your LaTeX source must be the \documentclass
%% command.
%%
%% For submission and review of your manuscript please change the
%% command to \documentclass[manuscript, screen, review]{acmart}.
%%
%% When submitting camera ready or to TAPS, please change the command
%% to \documentclass[sigconf]{acmart} or whichever template is required
%% for your publication.
%%
%%
% \documentclass[manuscript,review,anonymous]{acmart}
\documentclass[sigconf]{acmart}
% \documentclass{acmengage}
\usepackage{soul}
\usepackage{multirow}

%%
%% \BibTeX command to typeset BibTeX logo in the docs
\AtBeginDocument{%
  \providecommand\BibTeX{{%
    Bib\TeX}}}

%% Rights management information.  This information is sent to you
%% when you complete the rights form.  These commands have SAMPLE
%% values in them; it is your responsibility as an author to replace
%% the commands and values with those provided to you when you
%% complete the rights form.
\copyrightyear{2025} 
\acmYear{2025} 
\setcopyright{cc}
\setcctype{by}
\acmConference[CHI '25]{CHI Conference on Human Factors in Computing Systems}{April 26-May 1, 2025}{Yokohama, Japan}
\acmBooktitle{CHI Conference on Human Factors in Computing Systems (CHI '25), April 26-May 1, 2025, Yokohama, Japan}
\acmDOI{10.1145/3706598.3713971} 
\acmISBN{979-8-4007-1394-1/25/04}


%% These commands are for a PROCEEDINGS abstract or paper.
% \acmConference[Conference acronym 'XX]{Make sure to enter the correct
%   conference title from your rights confirmation emai}{June 03--05,
%   2025}{Woodstock, NY}
%%
%%  Uncomment \acmBooktitle if the title of the proceedings is different
%%  from ``Proceedings of ...''!
%%
%%\acmBooktitle{Woodstock '18: ACM Symposium on Neural Gaze Detection,
%%  June 03--05, 2018, Woodstock, NY}
% \acmISBN{978-1-4503-XXXX-X/18/06}


%%
%% Submission ID.
%% Use this when submitting an article to a sponsored event. You'll
%% receive a unique submission ID from the organizers
%% of the event, and this ID should be used as the parameter to this command.
%%\acmSubmissionID{123-A56-BU3}

%%
%% For managing citations, it is recommended to use bibliography
%% files in BibTeX format.
%%
%% You can then either use BibTeX with the ACM-Reference-Format style,
%% or BibLaTeX with the acmnumeric or acmauthoryear sytles, that include
%% support for advanced citation of software artefact from the
%% biblatex-software package, also separately available on CTAN.
%%
%% Look at the sample-*-biblatex.tex files for templates showcasing
%% the biblatex styles.
%%

%%
%% The majority of ACM publications use numbered citations and
%% references.  The command \citestyle{authoryear} switches to the
%% "author year" style.
%%
%% If you are preparing content for an event
%% sponsored by ACM SIGGRAPH, you must use the "author year" style of
%% citations and references.
%% Uncommenting
%% the next command will enable that style.
%%\citestyle{acmauthoryear}


%%
%% end of the preamble, start of the body of the document source.
\begin{document}

%%
%% The "title" command has an optional parameter,
%% allowing the author to define a "short title" to be used in page headers.
\title{Co-designing Large Language Model Tools for Project-Based Learning with K-12 Educators}

%%
%% The "author" command and its associated commands are used to define
%% the authors and their affiliations.
%% Of note is the shared affiliation of the first two authors, and the
%% "authornote" and "authornotemark" commands
%% used to denote shared contribution to the research.
% \author{ANONYMOUS AUTHOR(S)}
% \author{Ben Trovato}
% \authornote{Both authors contributed equally to this research.}
% \email{trovato@corporation.com}
% \orcid{1234-5678-9012}
% \author{G.K.M. Tobin}
% \authornotemark[1]
% \email{webmaster@marysville-ohio.com}
% \affiliation{%
%   \institution{Institute for Clarity in Documentation}
%   \city{Dublin}
%   \state{Ohio}
%   \country{USA}
% }

\author{Prerna Ravi}
\affiliation{%
  \institution{Massachusetts Institute of Technology}
  \city{Cambridge, MA}
  \country{USA}}
\email{prernar@mit.edu}

\author{John Masla}
\affiliation{%
  \institution{Massachusetts Institute of Technology}
  \city{Cambridge, MA}
  \country{USA}}
\email{j_masla@mit.edu}

\author{Gisella Kakoti}
\affiliation{%
  \institution{Massachusetts Institute of Technology}
  \city{Cambridge, MA}
  \country{USA}}
\email{gkakoti@alum.mit.edu}

\author{Grace C. Lin}
\affiliation{%
  \institution{Massachusetts Institute of Technology}
  \city{Cambridge, MA}
  \country{USA}}
\email{gcl@mit.edu}

\author{Emma Anderson}
\affiliation{%
  \institution{Massachusetts Institute of Technology}
  \city{Cambridge, MA}
  \country{USA}}
\email{eanderso@mit.edu}

\author{Matt Taylor}
\affiliation{%
  \institution{Massachusetts Institute of Technology}
  \city{Cambridge, MA}
  \country{USA}}
\email{mewtaylor@gmail.com}

\author{Anastasia K. Ostrowski}
\affiliation{%
  \institution{Purdue University}
  \city{West Lafayette, IN}
  \country{USA}}
\email{akostrow@purdue.edu}

\author{Cynthia Breazeal}
\affiliation{%
  \institution{Massachusetts Institute of Technology}
  \city{Cambridge MA}
  \country{USA}}
\email{cynthiab@media.mit.edu}

\author{Eric Klopfer}
\affiliation{%
  \institution{Massachusetts Institute of Technology}
  \city{Cambridge MA}
  \country{USA}}
\email{klopfer@mit.edu}

\author{Hal Abelson}
\affiliation{%
  \institution{Massachusetts Institute of Technology}
  \city{Cambridge MA}
  \country{USA}}
\email{hal@mit.edu}

% \author{Valerie B\'eranger}
% \affiliation{%
%   \institution{Inria Paris-Rocquencourt}
%   \city{Rocquencourt}
%   \country{France}
% }

% \author{Aparna Patel}
% \affiliation{%
%  \institution{Rajiv Gandhi University}
%  \city{Doimukh}
%  \state{Arunachal Pradesh}
%  \country{India}}

% \author{Huifen Chan}
% \affiliation{%
%   \institution{Tsinghua University}
%   \city{Haidian Qu}
%   \state{Beijing Shi}
%   \country{China}}

% \author{Charles Palmer}
% \affiliation{%
%   \institution{Palmer Research Laboratories}
%   \city{San Antonio}
%   \state{Texas}
%   \country{USA}}
% \email{cpalmer@prl.com}

% \author{John Smith}
% \affiliation{%
%   \institution{The Th{\o}rv{\"a}ld Group}
%   \city{Hekla}
%   \country{Iceland}}
% \email{jsmith@affiliation.org}

% \author{Julius P. Kumquat}
% \affiliation{%
%   \institution{The Kumquat Consortium}
%   \city{New York}
%   \country{USA}}
% \email{jpkumquat@consortium.net}

%%
%% By default, the full list of authors will be used in the page
%% headers. Often, this list is too long, and will overlap
%% other information printed in the page headers. This command allows
%% the author to define a more concise list
%% of authors' names for this purpose.
\renewcommand{\shortauthors}{Ravi, et al.}

%%
%% The abstract is a short summary of the work to be presented in the
%% article.
\begin{abstract}
  The emergence of generative AI, particularly large language models (LLMs), has opened the door for student-centered and active learning methods like project-based learning (PBL). However, PBL poses practical implementation challenges for educators around project design and management, assessment, and balancing student guidance with student autonomy. The following research documents a co-design process with interdisciplinary K-12 teachers to explore and address the current PBL challenges they face. Through teacher-driven interviews, collaborative workshops, and iterative design of wireframes, we gathered evidence for ways LLMs can support teachers in implementing high-quality PBL pedagogy by automating routine tasks and enhancing personalized learning. Teachers in the study advocated for supporting their professional growth and augmenting their current roles without replacing them. They also identified affordances and challenges around classroom integration, including resource requirements and constraints, ethical concerns, and potential immediate and long-term impacts. Drawing on these, we propose design guidelines for future deployment of LLM tools in PBL.
\end{abstract}

%%
%% The code below is generated by the tool at http://dl.acm.org/ccs.cfm.
%% Please copy and paste the code instead of the example below.
%%
\begin{CCSXML}
<ccs2012>
   <concept>
       <concept_id>10003120.10003123.10010860.10010911</concept_id>
       <concept_desc>Human-centered computing~Participatory design</concept_desc>
       <concept_significance>500</concept_significance>
       </concept>
   <concept>
       <concept_id>10010147.10010178</concept_id>
       <concept_desc>Computing methodologies~Artificial intelligence</concept_desc>
       <concept_significance>300</concept_significance>
       </concept>
   <concept>
       <concept_id>10010405.10010489</concept_id>
       <concept_desc>Applied computing~Education</concept_desc>
       <concept_significance>500</concept_significance>
       </concept>
    <concept>
       <concept_id>10003120.10003121.10011748</concept_id>
       <concept_desc>Human-centered computing~Empirical studies in HCI</concept_desc>
       <concept_significance>300</concept_significance>
       </concept>
 </ccs2012>
\end{CCSXML}

\ccsdesc[500]{Human-centered computing~Participatory design}
\ccsdesc[300]{Computing methodologies~Artificial intelligence}
\ccsdesc[500]{Applied computing~Education}
\ccsdesc[300]{Human-centered computing~Empirical studies in HCI}

%%
%% Keywords. The author(s) should pick words that accurately describe
%% the work being presented. Separate the keywords with commas.
\keywords{Generative AI, {LLMs}, AI for education, project-based learning, co-design, teachers, interviews}

% \received{20 February 2007}
% \received[revised]{12 March 2009}
% \received[accepted]{5 June 2009}

%%
%% This command processes the author and affiliation and title
%% information and builds the first part of the formatted document.
\maketitle

\section{INTRODUCTION}
Project-based learning (PBL) has gained prominence as a K-12 educational approach that immerses students in meaningful, real-world tasks, fostering deeper learning experiences \cite{boaler1998open, panasan2010learning, schneider2002performance, chen2019revisiting, PBLWorks}. 
% According to the Buck Institute for Education, \textbf{“Project Based Learning (PBL) is a teaching method in which students learn by actively engaging in real-world and personally meaningful projects”}\cite{PBLWorks}. 
Unlike traditional instructional methods, PBL emphasizes student-centered pedagogy, where learners actively construct knowledge through exploration, collaboration, and reflection \cite{condliffe2017project, noguera2015equal, peterson2012uncovering}. 
% This approach not only nurtures a love of learning but also encourages students to form personal connections to their academic experiences, making education more relevant and impactful \cite{PBLWorks}.
% PBL is grounded in several key educational theories. Constructivism, as articulated by theorists like Piaget and Vygotsky, suggests that deep learning occurs when students actively engage with and reflect on their experiences \cite{piaget1971biology, vygotsky2011interaction, resnick2018knowing}. PBL brings this theory to life by involving students in projects that require the application of knowledge in authentic contexts and fostering meaningful understanding \cite{condliffe2017project}. Similarly, expeditionary learning promotes hands-on, experiential activities, enabling students to collaboratively tackle real-world challenges \cite{blumenfeld1991motivating}. Additionally, PBL embodies principles of constructionism, introduced by Seymour Papert, which emphasize the creation of tangible artifacts in the learning process, thereby empowering them to be creators and innovators in their learning journeys \cite{krajcik2006project}.
% Despite initial resistance from those favoring traditional content knowledge instruction, PBL has gained traction due to its alignment with the deeper learning goals emphasized by educational standards like the Common Core and the Next Generation Science Standards \cite{scardamalia2011new, condliffe2017project}. 
% It not only enhances cognitive and interpersonal skills but also boosts motivation, fosters a strong academic and growth mindset, and develops crucial metacognitive abilities like self-regulation and continuous improvement, preparing students for lifelong learning and future challenges \cite{hernandez2009learning, kaldi2011project, bender2012project, tseng2013attitudes, thomas1998project, thomas1999project, english2013supporting}. 
% PBL also aligns closely with efforts to promote equity in education. 
By providing students with opportunities to explore topics of personal and cultural relevance, PBL supports diverse learning needs and backgrounds, making education more inclusive and accessible.
% This focus on equity is crucial in addressing the systemic disparities that exist in traditional educational settings, where marginalized students often face barriers to meaningful engagement and achievement.

While PBL offers significant benefits to students, it also necessitates a fundamental shift in educators' roles from knowledge providers to learning facilitators \cite{ertmer2006jumping, savery2015overview}. This requires educators to adopt new pedagogical strategies, manage complex classroom dynamics and provide ongoing, formative assessments that support student learning \cite{sahoo2013formative, albanese2019types}.
Challenges to PBL implementation include designing standards-aligned projects that meet diverse student needs, managing planning time, transitioning students to active roles \cite{simons2004instructional, chen2019revisiting, thomas1998project, zheng2024charting, viro2020teachers, ertmer2006jumping, gallagher1997problem},  creating authentic driving questions and assessing interpersonal skills, which traditional assessment methods often fail to capture \cite{markula2022key, blumenfeld1991motivating, mentzer2017examination, zheng2024charting, aksela2019project, colley2006understanding, brinkerhoff2004support}. Such challenges emphasize the need for nuanced instructional strategies for PBL that better capture the scope of student learning \cite{zheng2024charting}.

% Several scholars have also discussed the challenges around generating authentic and relevant driving questions to guide student work in PBL \cite{markula2022key, blumenfeld1991motivating, mentzer2017examination, zheng2024charting}. While these are student-driven, the responsibility and success of implementation falls on the teacher \cite{aksela2019project, colley2006understanding}.  Assessing student learning in PBL contexts adds another layer of complexity, especially when it comes to evaluating soft skills such as teamwork, creativity, and problem-solving, which traditional assessment methods often fail to capture \cite{brinkerhoff2004support}. Furthermore, balancing instructor guidance with student autonomy are persistent challenges \cite{zheng2024charting}. These issues highlight the need for more nuanced and formative assessment strategies that go beyond standardized tests to capture the full scope of student learning in PBL .

To address these challenges, the Human-Computer Interaction (HCI) community has shown a rising interest in generative artificial intelligence (GenAI) systems for education \cite{muller2022genaichi, sun2024generative, weisz2024design, han2024teachers, zheng2024charting, cai2024advancing, shaer2024ai}. 
Large language models (LLMs) in particular %have had a significant impact on the field of education \cite{han2024teachers, zheng2024charting, cai2024advancing, shaer2024ai}, and 
are uniquely suited to
address the inherent complexities of PBL {\cite{gustafson2025enhancing, zheng2024selfgauge, nikolicsupporting}} due to their capacity to personalize learning and offer real-time custom feedback with the potential to support differentiated instruction in PBL contexts \cite{asrifan2024integrating, kong2024developing, alam2023intelligence}. 
By automating routine tasks and augmenting project workflows, these tools could help streamline classroom management, thereby supporting teachers in delivering more engaging learning experiences \cite{zheng2024charting, krajcik2006project}.
% AI-powered tools can provide personalized learning experiences, automate routine administrative tasks, and offer real-time feedback to both students and teachers. 
% In PBL, these tools can streamline the management of projects, track student progress, and provide differentiated feedback that caters to individual learning needs \cite{zheng2024charting}. Teachers can also leverage these tools to evaluate student work for coherence, creativity, and critical thinking, and track student participation and collaboration to highlight areas for improvement \cite{krajcik2006project}.
Given the potential of LLMs to transform PBL, it is essential that these tools are designed prioritizing educators' needs and experiences, with teachers actively involved in the design process \cite{penuel2007designing, bakah2012updating, wan2021exploratory, handelzalts2019collaborative, mckenney2016collaborative}. 
%A collaborative co-design process with teachers can ensure that these technologies are effective and align with the realities of classroom practice \cite{penuel2007designing, bakah2012updating, wan2021exploratory, handelzalts2019collaborative, mckenney2016collaborative}. 
% Our research is motivated by the need to create AI tools that support, rather than replace, the teacher’s role, enhancing their ability to facilitate PBL and promote student-centered learning.


\subsection{Our Contribution}

To the best of our knowledge, this is the first co-design study with K-12 teachers aimed specifically at incorporating LLMs into PBL pedagogy.  Given the unprecedented and rapidly advancing nature of GenAI and its ethical considerations, our research is particularly timely. Our research objectives (RO) include:

\begin{enumerate}
    \item \textbf{Explore Demands and Challenges in PBL:} We investigated specific challenges educators encounter when implementing PBL in their classrooms, focusing on how they assess student learning outcomes in areas such as teamwork, creativity, and problem-solving. 
    \item \textbf{Co-design GenAI Tools for PBL:} Next, we explored how GenAI-powered tools, co-designed with interdisciplinary educators, could address diverse needs and challenges of PBL environments. Our co-design workshops examined how these tools could alleviate teachers' workloads while supporting their current practices, pedagogical goals, and professional growth. We constructed wireframes embodying teachers' values %that emerged from the co-design process with teachers 
    and received iterative feedback, {from teachers with diverse expertise,} on their potential integration into PBL environments.
    \item \textbf{Outline Design Considerations for LLM Tools in PBL:} Finally, we propose a set of design guidelines for LLM tools that educational technology designers and educators may find valuable for assessing student progress, setting learning goals, providing iterative and personalized feedback, and managing resources, timelines, and artifacts across different student projects and milestones.
\end{enumerate}

Our findings provide valuable insights into shaping the future use of LLMs in PBL, \textit{guiding and encouraging} its responsible implementation in classrooms. 
{It should be noted that although we used the term GenAI for the study framing with our participants, they focused on a particular type of GenAI: LLMs for their proposed solutions. Because of this duality, both these terms are used in this paper.}
% {It should be noted that although we use the terms GenAI, LLMs, and AI interchangeably throughout this paper, the majority of our use cases and discussions center around applications of text generation.}

\section{RELATED WORK}

% Our research draws on literature in three key areas. First, we build on the extensive background of Project-Based Learning (PBL), exploring both the benefits it offers and the challenges it presents for teachers. Second, we engage with the expanding body of research on AI-based tools in education, particularly focusing on large language model (LLM)-based tools that bolster and enhance the pedagogical practices commonly employed in PBL. Finally, we build on prior research in the co-design of educational tools, emphasizing the collaborative development process to create AI solutions that meet the unique needs of educators. 

\subsection{Project-Based Learning}


Rooted in constructivist learning theories, project-based learning (PBL) places students at the center of the learning process, challenging conventional education practices and redefining the roles of teaching, learning, and school organization \cite{fitzgerald2020overlapping, guo2023effects, kokotsaki2016project, woods2024}.  
Unlike problem-based learning, which promotes promotes deductive reasoning \cite{condliffe2017project, thomas1999project}, and involves structured evaluation based on how well a student’s solution addresses a defined problem (e.g., designing a more effective trash can to improve adoption or conducting participatory action research on a social issue), project-based learning engages students in complex, real-world tasks to create a broader range of artifacts (e.g. reports, models, and presentations) that demonstrate learning \cite{krajcik2006project, oguz2014comparison, thomas2010review, chen2019revisiting, barron2008teaching, thomas1999project} (e.g., using photojournalism to explore local flora and fauna or writing a comic book to analyze tropes in the hero’s journey). 
This fosters creativity and critical thinking through open-ended driving questions (e.g. “what is the proper role of government in a democracy?”) \cite{parker2011rethinking, blumenfeld1991motivating, diehl1999project, svihla2020facilitating}.
%These foster creativity and critical thinking through open-ended driving questions, leading to multiple learning pathways \cite{blumenfeld1991motivating, diehl1999project, svihla2020facilitating}.
% Project-Based Learning (PBL) has established itself as a powerful and transformative approach to education, offering an alternative to traditional instructional methods by placing students at the center of the learning process. Since its inception, PBL has provided a pedagogical framework that reimagines the roles of teaching, learning, and school organization, challenging conventional educational practices \cite{fitzgerald2020overlapping, guo2023effects, kokotsaki2016project, woods2024}.
% \textbf{Core Concepts and Practices in PBL: }PBL is deeply rooted in constructivist learning theories, which suggest that students achieve a deeper understanding of content when they actively "construct and reconstruct" knowledge through direct experience and interaction with the world \cite{krajcik2006project, oguz2014comparison, thomas2010review, chen2019revisiting}.  At its core, PBL involves students engaging in complex, real-world tasks over an extended period, resulting in the creation of artifacts such as reports, models, and presentations \cite{barron2008teaching, thomas1999project}. These artifacts serve as tangible representations of students' learning and are often shared with an audience, further enhancing the relevance and authenticity of the learning experience \cite{blumenfeld1991motivating}.
% The structure of PBL is typically characterized by several key steps: students and/or teachers identify an open-ended question or problem to address, students conduct research to explore potential solutions, they create artifacts that encapsulate their new ideas, and finally, they present these artifacts to an audience \cite{bell2010project, kokotsaki2016project} . Crucially, the driving questions in PBL are often situated within ill-defined problem spaces, allowing for multiple solutions and diverse learning pathways \cite{diehl1999project, svihla2020facilitating}. This approach not only fosters creativity and critical thinking but also places control of the learning process in the hands of students, encouraging them to take ownership of their projects.
Research has consistently demonstrated the numerous benefits of PBL across various educational contexts. Unlike traditional instruction, PBL (1) encourages students to delve deeply into the subject matter, fostering a more profound understanding of the material \cite{boaler1998open, panasan2010learning, schneider2002performance, chen2019revisiting}; (2) enhances mastery of subjects and supports the development of critical 21st-century skills like collaboration and communication \cite{condliffe2017project, noguera2015equal, peterson2012uncovering, chen2019revisiting}; (3) boosts motivation, and engagement \cite{hernandez2009learning,kaldi2011project, blumenfeld1991motivating, holm2011project, bender2012project, intel2007designing}; %by making learning relevant to students' lives and encouraging responsibility for their learning \cite{hernandez2009learning,kaldi2011project, blumenfeld1991motivating, holm2011project, bender2012project, intel2007designing}. It 
and (4) develops metacognitive skills by helping students assess their progress and continuously improve \cite{thomas1998project, thomas1999project, english2013supporting, bender2012project, tseng2013attitudes}. %preparing them for future challenges \cite{thomas1998project, thomas1999project, english2013supporting, bender2012project, tseng2013attitudes}.

% PBL has been shown to enhance students' mastery of subject matter by providing opportunities for active, hands-on learning \cite{chen2019revisiting}. This approach not only helps students retain knowledge but also enables them to apply what they have learned to real-world situations. Furthermore, PBL supports the development of essential 21st-century skills, such as collaboration, communication, and critical thinking
% \cite{condliffe2017project, noguera2015equal, peterson2012uncovering}. As students work together to solve complex problems, they learn how to collaborate effectively with others, share ideas, and provide constructive feedback.

% In addition to cognitive and interpersonal benefits, PBL is associated with positive intrapersonal outcomes, such as increased motivation and a stronger academic mindset \cite{hernandez2009learning,kaldi2011project}. The relevance of PBL projects to students' lives often sparks greater interest in the content, leading to higher levels of engagement and persistence \cite{blumenfeld1991motivating, holm2011project}. Students are also more likely to pursue their interests and take responsibility for their learning when they see the direct application of their work to real-world challenges \cite{bender2012project, intel2007designing}. 

% PBL has also been linked to the development of metacognitive skills, such as self-regulation and self-monitoring, which are crucial for lifelong learning \cite{thomas1998project, thomas1999project, english2013supporting} Through the iterative process of project work, students learn to assess their progress, identify areas for improvement, and adapt their strategies accordingly. This emphasis on self-assessment and continuous improvement not only enhances students' academic performance but also fosters a growth mindset, preparing them for future challenges \cite{bender2012project, tseng2013attitudes}.

While PBL is inherently student-centered, the success of this approach is heavily influenced by the role of teachers as facilitators, guiding and scaffolding student learning
\cite{barron2008teaching}. This shift in roles and professional identity requires teachers to adapt their pedagogical approaches and develop new skills, often leading to greater ownership, self-efficacy, and confidence in their teaching \cite{choi2019does, havice2018evaluating, potvin2021consequential}. However, transitioning to PBL often requires targeted professional development (PD) that provides teachers with peer collaboration, reflection, and ongoing support during PBL implementation \cite{Aitken2019, dunbar2022shifting, blumenfeld1991motivating, quint2018project, park2018equip}. %Educative features within PBL materials, such as guidance on adapting content for diverse learners, help teachers tailor projects to student needs \cite{balldeveloping, davis2005designing}.
Given the pivotal role of teachers in harnessing PBL benefits, our research reveals their needs and explores how LLMs could support them in successfully implementing PBL and foster continued professional growth. 

% This shift from traditional content delivery to a more facilitative role requires teachers to adapt their pedagogical approaches and develop new skills.
% As teachers become more experienced with PBL, they often develop a greater sense of ownership over their practice and classroom curricula \cite{potvin2021consequential}. This shift in professional identity is accompanied by increased self-efficacy, as teachers gain confidence in their ability to support student learning through PBL \cite{choi2019does, havice2018evaluating}. 
% However, transitioning to PBL-based instruction often requires targeted professional development (PD) experiences that specifically address the unique demands of this pedagogy
% \cite{Aitken2019, dunbar2022shifting, blumenfeld1991motivating, quint2018project, park2018equip}. Effective PD for PBL should include opportunities for teachers to collaborate with peers, reflect on their practice, and receive ongoing support as they implement PBL in their classrooms. 
% Educative features within PBL materials, such as guidance on adapting content for diverse learners, can also help teachers make informed decisions about how to tailor projects to meet the needs of their students \cite{balldeveloping, davis2005designing}.

To maximize the effectiveness of these tools, it is essential to consider challenges teachers encounter during PBL implementation, especially around generating authentic driving questions, 
managing time and group work, and balancing instructor-led guidance with student-directed learning \cite{zheng2024charting, thomas2010review, mergendoller2005managing}. 
%Classroom management can be difficult, in maintaining student engagement during self-directed learning and group work \cite{thomas2010review, mergendoller2005managing}. 
%^Additionally, 
Students’ discomfort with the cognitive and social demands of PBL can lead to frustration, particularly among high-achieving students accustomed to traditional instruction \cite{condliffe2017project}. In addition, teachers face challenges initiating inquiry, facilitating dialogue, and scaffolding learning \cite{reiser2006making, kali2008technology, krajcik2014promises, quintana2018scaffolding, reiser2018scaffolding}. 
Assessing student learning in PBL is also notably difficult, as traditional tests often fail to capture the depth of understanding that PBL aims to develop.  %Aligning assessments with PBL's deeper learning outcomes can be a challenge, as standardized tests are inadequate, and 
Performance-based assessments are hard to implement reliably \cite{hertzog2007transporting, mergendoller2005managing, darling2010beyond, aslan2015examining}, student artifacts can be difficult to score consistently, and teachers lack the time to provide personalized student feedback \cite{hattie2011instruction}. %In particular, teachers grapple with consistently scoring student artifacts and lack of time for personalized student feedback \cite{hattie2011instruction}.
Integrating technology is also challenging due to broader institutional factors such as limited resources, district mandates, and lack of school tech maintenance and support \cite{zheng2024charting}. Our study unpacks these PBL challenges in the literature through the nuanced experiences of teachers in interdisciplinary settings and explores how LLMs impact their roles and instructional practices. 

% Although student artifacts, such as reports and presentations, are commonly used, they are critiqued for neglecting the learning process. To address this, educators often use in-class presentations, learning journals, portfolios, and self-reflection to provide a more comprehensive evaluation \cite{zheng2024charting}.

% Classroom management is another significant concern, especially in maintaining student engagement during self-regulated learning and group work \cite{thomas2010review}. Issues like student misbehavior, lack of motivation, and intragroup conflicts are commonly reported, although experienced PBL teachers may encounter fewer of these problems \cite{mergendoller2005managing}.
% Establishing and maintaining classroom norms for effective group work is essential but requires further research to identify best practices \cite{darling2008creating}. Additionally, students’ discomfort with the cognitive and social demands of PBL can lead to frustration, particularly among high-achieving students accustomed to traditional instruction \cite{condliffe2017project}.

% Teachers face unique pedagogical challenges in PBL, including initiating student inquiry, facilitating dialogue, and providing the time and resources for in-depth investigations \cite{zheng2024charting}. Scaffolding student learning is crucial, yet teachers need more guidance on how to effectively implement and gradually reduce these supports \cite{reiser2006making, kali2008technology, krajcik2014promises, quintana2018scaffolding, reiser2018scaffolding}. 
% Promoting rigor in PBL also requires careful planning and collaborative project reviews to ensure that students engage in deep learning \cite{darling2008creating}.

% Technological integration poses additional challenges. Effective PBL requires access to quality technological resources and support for teachers in using technology, yet many schools lack these resources \cite{zheng2024charting}. 

% While technology can enhance learning, particularly for students with special needs, its impact on English Language Learners (ELLs) and students needing differentiation remains underexplored and warrants further investigation \cite{zheng2024charting}.
% Assessments in PBL pose another significant challenge, especially when aligning them with the deeper learning outcomes that PBL promotes. Standardized tests often do not capture these outcomes, making performance-based assessments more suitable but difficult to implement reliably \cite{hertzog2007transporting, mergendoller2005managing, darling2010beyond, aslan2015examining}. 
% While student artifacts are valuable assessment tools, consistent scoring remains a challenge, highlighting the need for more reliable strategies \cite{grant2005project, krajcik2014promises}. Additionally, teachers often lack the time or resources to provide the quality feedback necessary for guiding student learning \cite{hattie2011instruction}. 
% Rubrics linked to curricular units and descriptions of quality student work may help address this issue \cite{krajcik2014promises}. 
% Finally, broader institutional and contextual factors, such as teacher mobility, technology maintenance issues, and district mandates, can hinder PBL implementation. Support from school leadership and collaboration with other teachers can help mitigate these challenges \cite{zheng2024charting}.
% In our work, we unpack the PBL challenges highlighted in the literature by drawing on the specific and nuanced experiences of teachers operating within unique and often interdisciplinary environments. With the advent of GenAI, our work also investigates how these tools influence their roles, identities, and instructional practices. 
% Finally, we believe engaging teachers in conversations about their challenges is an essential part of the co-design process itself. It empowers them to voice their needs, fostering a sense of ownership and investment in the tools we develop. 
% We now engage with the expanding body of research on AI-based tools in education that bolster and enhance the pedagogical practices commonly employed in PBL as well.

\subsection{AI Tools to Support PBL Pedagogical Needs}

%We also contribute to an expanding body of AI in education research.
AI's capacity to personalize learning is beneficial in PBL settings where differentiated instruction is crucial. AI tools can support targeted instruction by tailoring lessons for diverse learners, encouraging student iteration on artifacts, and adapting materials to individual student strengths and weaknesses \cite{asrifan2024integrating, kong2024developing, alam2023intelligence}. 
Helping students manage time and materials efficiently, tools like Trello and Cronofy, integrated with AI plug-ins, can predict resource needs  \cite{tanga2024exploration}. 
AI-assisted design tools like Autodesk Dreamcatcher and intelligent project management software such as Asana and Monday.com can facilitate student project management, allowing them to focus on creative and critical thinking \cite{tanga2024exploration}.

%AI has shown potential in improving the quality of collaboration for project completion and knowledge construction \cite{tanga2024exploration, dutta2024enhancing, asrifan2024integrating}. 
Considering project collaboration and knowledge construction\cite{tanga2024exploration, dutta2024enhancing, asrifan2024integrating}, research scholars have discussed how AI can evaluate group performance by analyzing interaction patterns and predicting outcomes based on academic and behavioral data, enabling more effective grouping strategies \cite{cen2016quantitative, schneider2014toward}. AI systems can also guide students toward productive problem-solving, suggesting activities on collaborative styles, and pedagogical interventions to improve group work \cite{adamson2014towards, dyke2013enhancing, kumar2010architecture}. 
% Tan, Lee, and Lee (2022) discuss how AI techniques can evaluate group performance by analyzing metrics such as tone, relevance, and interaction patterns \cite{tan2022systematic}. This analysis helps teachers and students understand how group dynamics influence learning outcomes and how they can be improved. 
% AI can also predict group performance based on various factors, including students’ academic performance, interests, and behavioral data, allowing for more effective grouping strategies \cite{cen2016quantitative, schneider2014toward}. 
% AI can also provide real-time feedback to improve interaction patterns during collaborative activities. For instance, AI agents can guide students towards more productive problem-solving behaviors, recommend specific activities based on collaborative styles, and even suggest pedagogical interventions to tutors for improving group work \cite{adamson2014towards, dyke2013enhancing, kumar2010architecture}. 
% These capabilities are particularly valuable in PBL settings, where collaboration and teamwork are essential for project completion and knowledge construction.
Considering grading in PBL, AI-powered assessment systems, such as Automated Essay Scoring (AES) and Automated Written Corrective Feedback (AWCF), can offer real-time, continuous feedback, helping students to refine their work iteratively \cite{rudolph2023chatgpt, cope2021artificial}. These systems not only reduce the teacher's workload but also enhance the accuracy and efficiency of grading, allowing teachers to focus on more meaningful interactions with students \cite{owan2023exploring}. AI can also support formative assessment practices in PBL by providing insights into students' progress, helping teachers guide and support students more effectively \cite{lan2024teachers}.

Despite potential benefits, integrating LLMs into PBL presents challenges, such as biased algorithms, that necessitates careful design and human oversight to ensure fairness and accuracy in assessment and feedback \cite{schneider2023towards}. Comprehensive PD is needed for teachers to use LLM tools responsibly and equitably \cite{lan2024teachers, askarbekuly2024llm}, understand
%. Recent work has also underscored the need for future research exploring 
their long-term effects across age groups and subject areas, and address ethical concerns like data privacy \cite{asrifan2024integrating, wang2024artificial, zha2024designing}. In our current work, we aim to overcome some of these challenges by co-designing GenAI-powered support systems with \textit{teachers} in PBL settings. {
This is unlike recent studies that have focused on co-designing these tools with \textit{students} in the higher education contexts \cite{zheng2024selfgauge, zheng2024charting, nikolicsupporting, gustafson2025enhancing}. We also examine challenges more unique to K-12 PBL, that emphasizes scaffolding and skill-building \cite{condliffe2017project}.} 

\subsection{Co-designing Educational Tools with Teachers}

Co-design is a collaborative, iterative process where teachers, researchers, and developers jointly design, prototype, and evaluate educational tools %each with clearly defined roles
\cite{roschelle2006co}. This approach leverages stakeholders' expertise to address concrete educational needs, making it well-suited for creating technology-enhanced learning environments that are contextually relevant and practical for real-world classrooms \cite{cober2015teachers, brown1992design, lingnau2007empowering}. Studies have documented the co-design of mobile science applications, scripted wiki environments, and other educational technologies that support authentic scientific inquiry and other pedagogical practices \cite{spikol2009integrating, zhang2010deconstructing, peters2009co, cober2015teachers}.

Involving teachers in the co-design process enhances development efficiency, increases teacher agency and ownership, and promotes higher adoption rates and sustained use of educational tools beyond the initial study \cite{penuel2007designing, bakah2012updating, wan2021exploratory, handelzalts2019collaborative, mckenney2016collaborative, lin2021engaging}. %Participation fosters ownership, leading to higher adoption rates and integration into teaching practices \cite{lin2021engaging}. 
Co-design can also serve as a form of professional development, enabling teachers to deepen their understanding of new technologies and explore ways to incorporate them into instructional strategies \cite{wan2021exploratory, bakah2012updating, voogt2015collaborative}. Because co-design fosters a more reciprocal and participatory approach {\cite{alfredo2024human}}, it has the potential to upend the traditional, unidirectional process of educational technology  \cite{teeters2016challenge} and enables teachers to engage with complex topics beyond their expertise \cite{disalvo2017participatory}. %contributing to tool development while learning from the process \cite{disalvo2017participatory}. 
Furthermore, co-design has the potential to influence the broader sociocultural context of schools and drive systemic change in educational technology design and implementation by reshaping power dynamics and promoting an ethics of care within educational research and design \cite{wake2013developing, higgins2019power, matuk2021students}. {Scholars have however noted that involving teachers in the human centered design process happens less often in K-12 settings compared to higher education \cite{topali2024designing}.}


%This can drive systemic changes in how educational technologies are developed, implemented, and sustained within schools.
% In addition to its impact on teacher learning, past research has considered how teachers participate in the design of technology-enhanced learning environments \cite{cober2015teachers}. For example, studies have documented the co-design of mobile science applications, scripted wiki environments, and other educational technologies that support authentic scientific inquiry and other pedagogical practices \cite{spikol2009integrating, zhang2010deconstructing, peters2009co}. These efforts underscore the importance of involving teachers in the design process to ensure that the resulting tools are functional and aligned with their students' educational goals and needs.

% \textbf{To the best of our knowledge, no existing studies have documented a co-design process with K-12 teachers aimed at integrating GenAI into project-based learning pedagogy. Given the unprecedented and rapidly advancing nature of GenAI, along with its potential to address the unique challenges of PBL and the ethical considerations it entails, our research is particularly timely. Our findings and empirical evidence also provide valuable insights into shaping the future use of GenAI in PBL, }\textit{\textbf{guiding}}\textbf{ and }\textit{\textbf{encouraging}}\textbf{ its responsible and effective implementation.}

\section{METHODS}

We conducted two studies: the first with expert PBL teachers and the second with teachers {who had varying levels of PBL expertise. We explain our definition for expertise in section 3.1.1.} 
% The first study engaged with K-12 teachers who had extensive experience with project-based learning (PBL). 
% This study involved need-finding interviews, two co-design workshops, and the development and testing of wireframes. 
% The second study involved a workshop with novice PBL teachers to gather their perspectives on AI and PBL and to collect feedback on the same set of wireframes used in the first study. 
By involving both groups, we sought to 
address the varied challenges and opportunities in PBL classrooms.  {Our studies were approved by the Institutional Review Board (IRB) at Massachusetts Institute of Technology. We obtained informed consent from all participants.}

\subsection{STUDY 1} 
In this study, we engaged interdisciplinary {teachers with high levels of PBL experience and comfort. We worked with experts to leverage their deep pedagogical knowledge and refined understanding of effective PBL strategies, accumulated through years of trial, reflection, and adaptation.
They participated} in semi-structured interviews, a two-step co-design process, and feedback sessions on wireframes. We followed guidelines put forth in the literature \cite{ostrowski2021long, liu2003towards} by incorporating a mix of both divergent and convergent design thinking opportunities for participants. The divergent stages facilitated the generation of numerous ideas and concepts, while convergent stages focused on refining and narrowing down these ideas \cite{ostrowski2021long}. By offering diverse tools and activities, we enabled participants to discover the most effective methods for articulating their thoughts and generating new ideas \cite{ostrowski2021long}.

\subsubsection{\textbf{Participants and Recruitment}}

Eleven participants were recruited via mailing lists, direct emails, collaborations, and social media across the United States. {We specifically recruited participants with high levels of experience and comfort with PBL. This was the only precondition to participate in the study.}
% This approach ensured all participants had some exposure to and experience with project-based learning pedagogy, aligning with our goal of understanding current pedagogical methods and challenges in PBL. 
Teachers were compensated \$200 for participation (of approximately 10 hours) on this study.

Teachers completed a pre-survey covering demographic information and details regarding their past PBL and GenAI experience as part of the recruitment process. {We received over 50 responses from interested teachers.}
We {narrowed this list down to select our final 11 participants for the study \cite{caine2016local}. We aimed to have a diverse sample by selecting teachers} from multiple U.S. states, school systems (rural/urban/suburban areas as well as public/private schools with varied resource access \cite{goh2016urban}), subject backgrounds, varying levels of GenAI experience, and diverse student age groups.
We also sought teachers from both STEM and non-STEM classrooms, those who work with historically marginalized students, and those who have taught PBL online \cite{eubanks2018automating, vanderlinde2010gap, ravi2021pandemic, trust2021emergency}.
% We also looked for teachers who engage with historically marginalized students and areas, those from STEM and non-STEM classrooms, and teachers who have taught PBL online. We aimed to address the diverse realities of classrooms, including the unique needs of marginalized students and the logistical challenges of remote learning, often overlooked by current educational technologies \cite{eubanks2018automating, vanderlinde2010gap, ravi2021pandemic, trust2021emergency}. 
We wanted to intentionally design GenAI PBL tools with {the above factors in mind from the beginning to ensure inclusive educational practices and} avoid exacerbating existing disparities \cite{eubanks2018automating}.


\begin{table*}[ht]
    \centering
    \begin{tabular}{|p{0.1\linewidth}|l |p{0.12\linewidth}|p{0.1\linewidth}|p{0.12\linewidth}|p{0.1\linewidth}|p{0.1\linewidth}|p{0.1\linewidth}|}
    % \begin{tabular}{|>{\raggedright\arraybackslash}p{0.13\linewidth}|>{\raggedright\arraybackslash}p{0.13\linewidth}|>{\raggedright\arraybackslash}p{0.2\linewidth}|>{\raggedright\arraybackslash}p{0.15\linewidth}|>{\raggedright\arraybackslash}p{0.15\linewidth}|>{\raggedright\arraybackslash}p{0.15\linewidth}|} 
    \hline 
         \textbf{Participant}  & \textbf{Gender}   & \textbf{{Available School Details}}&  \textbf{Subjects} &  \textbf{Student Age-groups} &  \textbf{Years of experience with PBL} &  \textbf{Comfort levels with PBL} & \textbf{GenAI Tool Experience}\\ \hline \hline 
         P01 &F  &public, urban, socioeconomically disadvantaged students&  English&  9th-12th grade, higher ed, adults&  More than 5 years&  5& 5\\ \hline 
         P02 &M  &public, suburban&  Computer Science&  9th-12th &  More than 5 years&  5& 3\\ \hline 
         P03 &M  &public, experiential learning school&  History&  9th-12th &  1-2 years&  4& 2\\ \hline 
         P04 &M  &public, urban, socioeconomically disadvantaged students&  Mathematics&  9th-12th &  3-5 years&  4& 3\\ \hline 
         P05 &F  &public, online teaching&  Design, Tech, and Eng&  9th-12th, higher ed, adults&  More than 5 years&  5& 4\\ \hline 
         P06 &F  &public, rural&  Design, Tech, and Eng&  3rd-5th, 6th-8th&  More than 5 years&  4& 5\\ \hline 
         P07 &M  &online teaching&  Mathematics&  3rd-5th grade, 6th-8th, 9th-12th&  More than 5 years&  5& 4\\ \hline 
         P08 &F  &private&  Computer Science&  3rd-5th, 6th-8th, 9th-12th, higher ed, adults&  3-5 years&  4& 3\\ \hline 
         P09 &M  &private&  History&  kindergarten, 1st-2nd, 3rd-5th, 6th-8th&  More than 5 years&  5& 4\\ \hline
 P10 &F  &private, urban& Design, Tech, and Eng& 9th-12th& More than 5 years& 5&4\\\hline
 P11 &F  &public, suburban& Computer Science& 9th-12th& More than 5 years& 5&2\\\hline
    \end{tabular}
    \caption{Study 1 participants' details, N=11}
    \label{tab:participants}
    \Description{The table provides details of the participants from Study 1, which includes teachers with high levels of expertise in project-based learning (PBL). The table lists 11 participants (P01 to P11) and includes information on their gender, school details, subjects taught, student age groups, years of experience with PBL, comfort levels with PBL, and experience with generative AI (GenAI) tools.
    Among the participants, there is a mix of male (M) and female (F) teachers across different subjects such as English, Computer Science, History, Math, and Design, Technology, and Engineering. They teach in a mix of private/ public and rural/urban/suburban schools. P05 and P07 teach in online settings, P01 and P04 work with socio-economically disadvantaged students, and P03 is a part of an experiential learning school.
    The teachers work with a wide range of student groups, including kindergarten, elementary (3rd-5th grade), middle school (6th-8th grade), high school (9th-12th grade), higher education, and adults. Most participants have more than five years of experience with PBL, with a few having between one to five years. Comfort levels with PBL range from 4 to 5 on a scale where 5 indicates the highest comfort. Similarly, their experience with GenAI tools varies, with scores ranging from 2 to 5, reflecting varying levels of familiarity and usage.}
\end{table*}

Table \ref{tab:participants} breaks down the participants recruited (from the pre-survey responses). The teachers instructed various subjects, including computer science (n = 3), history (n = 2), English (n = 1), math (n = 2), and design, tech, and engineering (n = 3). Additionally, P06 served as a tech integration specialist, collaborating with other teachers to design and implement technology-enhanced projects. P05 and P07 taught PBL in online settings.
% just add this to the table
% Of the participants, 6 identified as female and 5 identified as male. In terms of ethnicity, 8 participants identified as white and 3 identified as African-American. 
% The distribution of experience added diversity to our data, capturing a range of challenges and successes that might emerge at different stages of PBL adoption. This mix allowed for a comprehensive understanding of both long-term and newer PBL implementations in classrooms.
{Teachers,  on average, had 5+ years of experience implementing PBL and reported an average comfort level of 4.6/5. In this study, we define PBL expertise as a combination of years of experience and comfort levels (showing self-efficacy) resulting from focused PBL implementation \cite{rand}. Comfort, in particular, is influenced not only by prior experience but also by the school systems teachers operate within. For instance, P03 described themselves as a PBL expert despite having relatively few years of experience, attributing their confidence to the rigorous training and practice received from working at a school strongly focused on experiential learning \cite{lam2010school}.}

\subsubsection{\textbf{Data Collection}}
\paragraph{\textbf{Interviews}}

Between the months of March and April 2024, we conducted 11 semi-structured interviews \cite{kvale2009interviews}. All interviews were audio recorded with participant consent, ranged from 60-90 minutes each, and were conducted remotely via Zoom. We used these interviews as an opportunity to build rapport with our participants prior to co-design. The narrative style of the interview encouraged participants to engage in divergent thinking, allowing them to respond to open-ended questions freely with minimal redirection from the interviewers \cite{ostrowski2021long}. 


% During the interviews, we asked participants a series of questions to understand their teaching contexts and experiences with project-based learning (PBL). We started with broader questions on the \textbf{integration of PBL} into their classroom curriculum, including its fit with other instructional methods, curriculum design approaches,  technology use and classroom norms for success, and the influence of contextual factors like teacher-student ratios on PBL effectiveness. 
% We then discussed strategies for ensuring coherence across projects, clarifying instructions, using \textbf{driving questions for maintaining focus}, prioritizing student voice and choice, resolving conflicts, and tracking progress iteratively. 
% We also spent a substantial portion of the interviews on \textbf{grading and assessments}, exploring teachers' approaches to grading weekly project deliverables and formative assessments, and creating and using rubrics. Teachers shared challenges with portfolio grading, differentiation, tracking progress, and managing PBL components like resources, deadlines, and timelines for different groups.
% Teachers also spoke about student transition challenges from traditional learning to PBL, addressing frustrations, preferences for standardized tests, and strategies to ease discomfort. We also discussed scaffolding techniques and teachers' views on PBL's open-ended nature. These insights helped shape our co-design workshops.
During the interviews, we asked participants about their teaching contexts and experiences with PBL, starting with its integration into the curriculum, alignment with other methods, technology use, and contextual factors like teacher-student ratios. We then explored strategies for maintaining coherence across projects, fostering student engagement, and tracking progress. A significant portion of the interviews focused on assessments, including approaches to grading, rubric use, and challenges with differentiation and progress tracking. We also asked teachers about helping students transition from traditional learning to PBL, and their views on scaffolding techniques and the open-ended nature of PBL.
The recruitment survey responses from our 11 participants informed these interview questions (see Table \ref{tab:participants}). In addition, the broader topics in the protocol were decided based on the discussions outlined in the literature review by Condliffe et al. \cite{condliffe2017project}. 

% We also delved into \textbf{student transition challenges} from traditional learning to PBL, exploring initial frustrations, preferences for standardized tests, and strategies to reduce discomfort. We asked how they support and scaffold student learning, and decide on the amount of scaffolding needed at different project stages. 
% We concluded the interviews by inquiring about teachers' perceptions of the open-ended nature of PBL. 

% \begin{itemize}
%     \item How is project-based learning a part of your curriculum?
%     \begin{itemize}
%         \item How does PBL fit in with other instructional methods?
%         \item How long does a PBL unit last?
%     \end{itemize}
%     \item Can you describe your overall experience with implementing project-based learning in your classroom?
%     \begin{itemize}
%         \item Do you entirely design your own curriculum? Or do you adapt an existing one?
%         \item How did you transition from director to facilitator of learning?
%         \item When do you know using technology is appropriate?
%         \item Do you have any classroom norms to ensure PBL success?
%     \end{itemize}
%     \item What are some contextual factors in your classroom or school that have influenced your implementation of PBL?
%     \item What is the teacher-student ratio in your classroom?
% How do you perceive the impact of student-teacher ratios on the effectiveness of PBL?

% \end{itemize}


% We then transitioned into discussing strategies for establishing coherence across multiple student projects, addressing ambiguity in instructions given to students, and the critical role of \textbf{driving questions in maintaining focus} and direction. Teachers also discussed their strategies for prioritizing student voice and choice in project planning, their approaches to resolving conflicts, and their methods for iteratively tracking both individual and groups’ progress:


% \begin{itemize}
%     \item Driving question: How do you address any ambiguity in PBL instructions and guide students towards the overall goal or question?
%     \begin{itemize}
%         \item Is there one major question spanning multiple projects? How is it structured?
% Do you assign this or do students make their own?

%         \item Are students generally able to understand the main driving question?
%         \item Do students loop back to the driving question or master question periodically?
%         \item Do you employ any strategies to ensure this? 
%     \end{itemize}
%     \item Student voice and groups: 
%     \begin{itemize}
%         \item Do students explore solutions related to their own questions?
%         \item Does every student get the same autonomy? 
%         \item Do roles vary within student groups? How do you keep track of this if they do vary?
%         \item How do you resolve conflicts? 
%     \end{itemize}
% \end{itemize}


% We also spent a substantial portion of the interviews on \textbf{grading and assessments}. We examined teachers’ approaches to grading weekly project deliverables and formative assessments, measuring knowledge and skills development, and creating and using rubrics. Teachers shared challenges with portfolio-oriented grading, differentiation, and tracking progress across milestones. We discussed managing the various moving components of PBL, such as resource management, deadlines, and balancing timelines with every group’s projects:


% \begin{itemize}
%     \item How do you approach grading in a project-based learning environment?
%     \begin{itemize}
%         \item Are projects graded for completion?
%         \item How do you measure development and application of new knowledge and skills?
%         \item What are some student artifacts you include in your assessment? 
% How do you score these artifacts in a valid and reliable manner?

%     \end{itemize}
%     \item Do you use any rubrics? 
%     \begin{itemize}
%         \item How are these designed wrt learning goals and master questions?
%         \item How are these linked to curricular units?
%     \end{itemize}
%     \item Do you use self-reflection assessments? If so, what are they? 
%     \begin{itemize}
%         \item How do you combine these with your own assessment of the same questions? 
%         \item How does this inform instruction? 
%     \end{itemize}
% \end{itemize}


% We also delved into \textbf{student transition challenges} from traditional learning to PBL, exploring initial frustrations, preferences for standardized tests, and strategies to reduce discomfort. We asked how they support and scaffold student learning, and decide on the amount of scaffolding needed at different project stages. 


% \begin{itemize}
%     \item What challenges do students face when transitioning from traditional learning styles to project-based learning?
%     \item How do you reduce students’ discomfort with the new cognitive and social demands that PBL places on them?
%     \item How do you address situations where students seem unsure of the next steps or encounter obstacles in their projects?
%     \item Do you provide scaffolding? How do you decide how much scaffolding is needed and when? 

% \end{itemize}
% The interview discussions helped frame the structure and focus areas of our co-design workshops. We discuss that in the next section.

\paragraph{\textbf{Co-design Workshops}}

We conducted two co-design workshops (each lasting 2.5-3 hours) with teachers to explore how GenAI tools could support diverse PBL classroom needs without disrupting current practices. 
% Following the interviews, we conducted a series of co-design workshops with teachers to collaboratively investigate the role of AI tools in addressing diverse needs of PBL classrooms without disrupting current pedagogical practices. The co-design process consisted of two workshops, each lasting between two and a half to three hours. The aim of the first workshop was to come up with concrete scenarios in which teachers experienced difficulties with effectively implementing project-based learning pedagogy in their classrooms and to brainstorm potential ways future GenAI tools could assist them in navigating these challenging scenarios. Based on findings from the first workshop, the second workshop challenged participants to delve deeper into conceptualizing and developing GenAI tools to address the implementation challenges identified in the first workshop. During the second workshop, participants were also asked to discuss ethical and societal impacts of the proposed AI solutions on the various stakeholders in the education system. Overall, the co-design workshops were meant to serve as a starting place for thinking about how AI could improve the experiences of teachers, with differing levels of experiences and training in PBL, when assessing student learning outcomes, regulating teamwork, creativity and collaborative problem solving.
Due to teachers' limited availability and varying schedules, each workshop was divided into three smaller group sessions with three to four teachers, enabling focused, interactive discussions in a manageable online setting. The sessions, which included teachers from different subject areas to ensure diverse perspectives, were conducted remotely via Zoom and recorded with participants' consent. Below, we outline the structure and activities of each workshop:
% Given the limited availability and varying schedules of teachers in our study, we split each workshop up into three group sessions of three to four teachers each. This approach paralleled small group discussions typically found in large co-design workshops, allowing for more focused and interactive exchanges, where participants engaged deeply and collaboratively in a more manageable online setting. 
% These sessions were all conducted remotely via Zoom and were audio recorded with consent from the participants. We tried to strike a good balance of teachers from different subject areas for each group session to account for diverse perspectives and needs in the design process. Below we describe the structure and activities completed in each workshop:


\paragraph{\textbf{Workshop 1}}

The first co-design workshop aimed to identify key PBL challenges and explore how GenAI could help alleviate teacher workload. Each session included three to four teachers, a facilitator, and a notetaker. The make-up of each group was:

\begin{itemize}
    \item Group 1: P11 (Computer Science), P06 (Design, Tech, Engineering), P03 (History), P07 (Mathematics)
    \item Group 2: P05 (Design, Tech, Engineering), P02 (Computer Science), P04 (Mathematics)
    \item Group 3: P01 (English), P08 (Computer Science), P10 (Design, Tech, Engineering), P09 (History)
\end{itemize}

Workshop participants engaged in hands-on tasks below to foster creative thinking and group collaboration, with discussions throughout to explore their ideas:

\begin{enumerate}
    \item After establishing workshop norms \cite{ostrowski2021long}, participants used \href{https://miro.com/}{Miro boards} for three five-minute rounds of \textbf{open brainstorming}. They filled post-it notes identifying (1) key challenges in PBL, shared (2) past strategies and tools they had used to address those challenges, and (3) envisioned dream tool features. After brainstorming rounds, participants reviewed the post-it notes, labeled promising ideas, and linked them to past experiences and potential concerns. They then grouped these ideas into categories on the Miro boards, which helped reinforce the ideas that emerged from our interviews.

    \item We then gave a \textbf{presentation about GenAI} with examples of the technology’s use cases in K-12 education. We explained how ChatGPT works and encouraged participants to use this knowledge to explore its potential and limitations in classrooms. Teachers also shared their experiences using these tools in their teaching. {Our teachers gravitated towards using LLMs during subsequent stages of the study, even though they may have referred to GenAI during discussions as it was a more commonly recognized term for them.}
    
    \item Finally, participants expanded on the initial problem scoping to ideate where GenAI {(particularly LLMs)} could be incorporated into their selected category from the board. They created \textbf{storyboards} \cite{buxton2010sketching} of concrete scenarios depicting PBL implementation challenges and a concept of how a fictional LLM tool could serve as a solution in their everyday teaching context. These storyboards were then shared and discussed within the group. 
\end{enumerate}
 
Written material from the participants and notetakers, and audio recordings of the small group discussions were collected online for analysis. 
% The full protocol for this workshop can be found in Appendix \ref{app3}.


\paragraph{\textbf{Workshop 2}}

We summarized the brainstorming boards and storyboards from Workshop 1 to guide activities for Workshop 2 that developed conceptual design prototypes for PBL GenAI tools. Each group session consisted of three to four teachers, a primary facilitator, and a notetaker (similar to Workshop 1). We tried to reorganize participant groups to encourage a wider range of discussions across the two workshops: 

\begin{itemize}
    \item Group 1: P11 (Computer Science), P05 (Design, Tech, Engineering), P09 (History)
    \item Group 2: P01 (English), P08 (Computer Science), P10 (Design, Tech, Engineering)
    \item Group 3: P03 (History), P06 (Tech Integration: all subjects), P02 (Computer Science), P04 (Mathematics)
\end{itemize}

\begin{figure*}
    \centering
    \includegraphics[width=1\linewidth]{samples/blank_worksheets.pdf}
    \caption{Examples of {blank worksheets distributed to participants in workshop 2} outlining stakeholder impacts and ethical implications of PBL-LLM tools}
    \label{fig:worksheets456}
    \Description{The figure includes two worksheets to guide discussions on developing and integrating a generative AI tool in education, focusing on stakeholders and ethical impacts. Step 4 asks participants to identify values, benefits, and harms for stakeholders. Step 5 explores short- and long-term ethical impacts, including transparency in tool use.}
\end{figure*}

In this workshop, participants developed conceptual prototypes based on the storyboards from Workshop 1. They discussed their tools' LLM roles, its ethical impact on stakeholders, and classroom integration constraints by collaboratively completing worksheets with specific prompts using \href{https://slides.google.com/}{Google Slides}, suitable for the remote setting. An example of {blank} worksheets from some of the steps below is shown in Figure \ref{fig:worksheets456}. We summarize this process below:


\begin{enumerate}
    \item \textbf{Define Final Tool Idea:} Participant groups first selected 1-2 storyboards with overlapping themes for further exploration, wrote a detailed description, and ensured all members agreed on the concept, which became the basis for their final prototype.
    \item \textbf{Identify Stakeholders:} Participants then identified direct and indirect stakeholders involved in their chosen scenarios and their specific roles and interaction with the tool. 
    % They described what was happening in the scenario, who was using the technology, whether they were alone or with others, and the setting of the scene (e.g., classroom or home).
% \begin{figure}
%     \centering
%     \includegraphics[width=1\linewidth]{Worksheets1_2.png}
%     \caption{Examples of workshop 2 worksheets ideating PBL AI tools and their stakeholders}
%     \label{fig:worksheets1_2}
% \end{figure}
    \item \textbf{Identify How the GenAI Prototype Is Used: } Next, they detailed the role of the GenAI in the selected scenarios. 
    Groups created a list of envisioned features, considered their data-sharing requirements, preferred devices for deployment (phone, tablet, or laptop), and potential integration with existing Learning Management Systems (LMS) like Canvas, Google Classroom, etc.
    % An example of step 3’s worksheet is shown in figure \ref{fig:worksheets3}.

% \begin{figure}
%     \centering
%     \includegraphics[width=1\linewidth]{Worksheets3.png}
%     \caption{Examples of workshop 2 worksheets identifying roles of GenAI}
%     \label{fig:worksheets3}
% \end{figure}
    \item \textbf{Ethical Implications: }We then led participants through discussions of their designs' societal impacts. Participants received an ethical implications handout to identify values, benefits, and potential harms for each stakeholder group from Step 2 \cite{ali2023ai, arnold2019factsheets, liao2020questioning}. They also investigated the immediate short-term and long-term consequences of using this tool in their classrooms and the PBL community. 
    
    \item \textbf{Classroom Constraints:} Finally, participants considered practical aspects of integrating their prototype in their schools, including resource requirements, tool training and maintenance, anticipated challenges (technical, logistical, pedagogical), and strategies to address these challenges effectively.
\end{enumerate}
% Example worksheets from steps 4, 5, and 6 are shown in figure \ref{fig:worksheets456}.

% By following this structured approach, Workshop 2 facilitated the development of concrete, actionable GenAI tool ideas tailored to the needs and challenges of PBL teaching. 
Prototype worksheets, facilitator notes, and audio recordings of discussions were collected for analysis. 
% The full protocol for this workshop can be found in Appendix \ref{app3}.


\paragraph{\textbf{PBL GenAI Wireframes Walkthrough}}
\label{testing}

Using data from interviews and co-design sessions, we created wireframes outlining valuable design features for teachers using PBL, focusing on curriculum support (project brainstorming and lesson planning), assessment support (rubric creation with differentiation and grading), and progress tracking (monitoring progress at the individual student, group, and class levels). 
These wireframes and support areas, described further in our results section, were reviewed in one-on-one sessions with teachers, who assessed the usability of the tool's wireframes within their classroom contexts and unique pedagogical practices.
Sessions were conducted remotely via Zoom and recorded with consent. We expanded on this feedback by including other teachers outside Study 1 (Section \ref{study2methods}).
We administered a post-survey at the end to understand participants' perceptions of GenAI and PBL and gather their reflections on Study 1's activities.

% Using the data collected from the interviews and co-design sessions, we curated a list of potential design features that teachers implementing project-based learning would find valuable. This comprehensive list was used to create wireframes that outline pedagogical supports for teachers, focusing on three key areas: curriculum support (including project brainstorming and lesson planning), assessment support (including rubric creation with differentiation and grading), and progress tracking support (including monitoring progress at the individual student, group, and class levels). 
% Figure \ref{fig:wireframes-methods} shows an example page from each of these supports. 

% \begin{figure}
%     \centering
%     \includegraphics[width=1\linewidth]{wireframes-methods.png}
%     \caption{Example wireframe screens for curriculum planning (left), assessments (center), and progress tracking (right)}
%     \label{fig:wireframes-methods}
% \end{figure}
% We describe these areas of support further in our co-design and wireframes analysis in Chapter \ref{chap:5}. These wireframes are detailed in the results section [insert \#], contextualized by the findings from the interviews and co-design sessions.

% \begin{figure}
%     \centering
%     \includegraphics[width=0.75\linewidth]{wireframes-comments.png}
%     \caption{Example of study 1 participants interacting with the wireframes}
%     \label{fig:wireframes-comments}
% \end{figure}

% We then sought feedback on these wireframes through one-on-one sessions with the participating teachers. These were informal conversations where we walked participants through the wireframes and gave them opportunities to explore the different supports, leave comments on the design features and decisions, and provide constructive feedback for improvement. An example of these comments is shown in figure \ref{fig:wireframes-comments}. Teachers specifically assessed the usability of the tool's wireframes within their classroom contexts and unique pedagogical practices. These sessions were conducted remotely via Zoom and were audio recorded with consent from the participants for the purposes of transcription. We expanded our sample for evaluating the effectiveness of these wireframes by also conducting testing sessions with other teachers from outside the study. We elaborate on the second study in the section \ref{study2methods}.

% \subsubsection{Post survey}

% , receiving 9 responses from 11 participants. The survey covered feedback on their experiences from the interviews, co-design workshops, wireframes' testing sessions, including ratings on their confidence, excitement, and concerns on integrating AI into their teaching practices.
% To wrap up the study, we distributed a post-survey to gauge participants' perceptions of AI and PBL and to gather reflections on their experiences from engaging with the various activities in the study. We received 9 responses (out of the 11 participants) on our post-survey. We covered the following areas of evaluation in our post survey:

% \begin{itemize}
%     \item \textbf{\textbf{Part 1: Interviews}}
%     \begin{itemize}
%         \item Participants’ rating on their interview experience (5-point likert scale)
%         \item Elements that participants really liked about the interviews
%         \item Elements participants would change about the interviews
%     \end{itemize}
%     \item \textbf{\textbf{Part 2: Co-design }}
%     \begin{itemize}
%         \item Participants’ rating on their co-design experience (5-point likert scale)
%         \item Elements that participants really liked about the co-design process in the context of the subjects they teach
%         \item Any effects/influence of collaboration with other teachers and researchers on participants' understanding and approach to PBL
%         \item Specific tools or strategies introduced during the co-design sessions that participants found challenging or useful
%         \item Any new skills/insights gained that contributed to participants’ professional growth 
%         \item Areas for improvement in the activities conducted during the co-design sessions
%     \end{itemize}
%     \item \textbf{\textbf{Part 3: Wireframes Testing }}
%     \begin{itemize}
%         \item Participants’ rating on their experience giving feedback on the wireframes (5-point likert scale)
%         \item Unique elements in the wireframes compared to existing AI tools
%         \item Onboarding resources needed to use the tool in their classrooms
%         \item Participants’ constraints and concerns integrating the tool in their classrooms
%     \end{itemize}
%     \item \textbf{Part 4: AI Perceptions }
%     \begin{itemize}
%         \item Participants’s confidence in designing and implementing AI tools for their classroom (5-point likert scale)
%         \item Participants’ excitement around using AI within PBL (5-point likert scale)
%         \item Participants’ feelings about GenAI tools, more generally (where 1 was super pessimistic and 5 was super optimistic)
%         \item Participants’ reflections on their evolving identities as teacher with the advent of GenAI and its impact on teaching creativity 
%         \item Participants’ reflections on aspects of teaching that should never be automated by AI
%     \end{itemize}
% \end{itemize}



\subsubsection{\textbf{Data Analysis}}

We transcribed all interviews and took an inductive, iterative approach to thematic analysis \cite{braun2019reflecting}. Initial codes were identified from the transcripts and refined into broader themes, which were continuously reviewed to accurately capture participants' insights. Two researchers then coded the transcripts, achieving a Cohen’s kappa of 0.906 on 40\% of the data, indicating substantial interrater reliability \cite{stemler2019comparison, mchugh2012interrater}. We then deductively applied these themes to the rest of our co-design and wireframe testing data \cite{azungah2018qualitative}. \\
% We transcribed all interviews and co-design sessions in order to initiate the coding process. We took an inductive and iterative approach to thematic analysis \cite{braun2019reflecting}. We first identified themes inductively from the interview transcripts. This was undertaken primarily by me. We identified initial codes and patterns emerging from the data, which were then refined and categorized into broader themes. These themes were iteratively reviewed and adjusted to ensure they accurately captured the participants' experiences and insights. The iterative process allowed for continuous refinement of the themes and their definitions. 
% Once the initial themes and definitions were established, two researchers coded the interviews by assigning themes to the transcripts. To determine the interrater reliability among researchers, a \href{https://en.wikipedia.org/wiki/Cohen%27s_kappa}{Cohen’s kappa} was calculated for the interviews for 40\% of the coded transcripts. We achieved a final Cohen’s kappa of 0.906 which was within the range for the substantial agreement considered acceptable for interrater reliability \cite{ostrowski2021long, steen2013co}. The final set of themes from the interviews were also used to code the transcripts and artifacts from the other parts of our study (co-design and testing). These themes not only informed the development of the GenAI tool wireframes but also offered valuable insights into the specific needs and challenges teachers face in implementing project-based learning.
% We saw a mix of themes outlining various pedagogical aspects of PBL, specific opportunities and challenges identified for both teachers and students, as well as student and teachers’ perceptions on the use of AI tools in the classroom. \\

\begin{table*}[ht]
    \centering
    {
    \begin{tabular}{|l|p{0.5\linewidth}|}
    \hline 
         \textbf{Pre-survey metric}& \textbf{Collated particpant responses}\\
         \hline \hline 
         Participants' gender&14 female and 16 male\\\hline
         Subjects taught (multi-select)& Science (15), Mathematics (7), Computer Science (6), Engineering and Technology (4), STEAM (3), English (1), Business (1), and Teacher Education (1)\\ \hline 
         Students' age groups (multi-select)& Kindergarten (2), 1st-5th grade (7), 6th-8th grade (14), 9th grade and above (21)\\ \hline 
         Experience with PBL& 46\% for more than 5 years, 23\% for 3-5 years, and 31\% for a year or less.\\ \hline 
         Comfort levels with PBL (scale 1-5)& 5: 40\%, 4: 26.7\%, 3: 16.7\%, 2: 10\%, 1: 6.7\%\\ \hline
 Describe past PBL experience&Difficult (Significant challenges, poor engagement, and outcomes): 23\% \newline
Neutral (mix of challenges and successes): 46\% \newline
Very successful (Highly positive, engaged students, and excellent outcomes): 33\%\\\hline
 Experience with GenAI tools like ChatGPT&1: 20\%, 2: 26.7\%, 3: 26.7\%, 4: 20\%, 5: 6.7\%\\\hline 
    \end{tabular}
    }
    \caption{Study 2 participants' details, N=30}
    \label{tab:study2_participants}
    \Description{The table provides details of the participants (N=30) from Study 2, which includes teachers with varying levels of expertise in project-based learning (PBL). The table presents collated participant responses on gender, subjects taught, student age groups, experience with PBL, comfort with PBL, description of past PBL experiences and experience with GenAI tools like ChatGPT. There is an equal mix of male (M) and female (F) teachers across different subjects such as Science, Math, Computer Science, Technology, and Engineering, STEAM, English, Business, and Teacher Education. Most participants teach middle and high school. There is a mix on experience and comfort levels. Participants are equally divided between those who had found success with PBL and those who had not. Experience with GenAI tools also varies all across.}
\end{table*}

\subsection{STUDY 2}
{The bulk of the co-design process was conducted with the expert teachers in Study 1. Following this, we conducted a second study involving teachers with varying levels of experience and comfort with PBL (and hence varying expertise as per our definition in 3.1.1). This workshop aimed to gather a wider range of perspectives from a diverse group of teachers, ensuring the tool wireframes could adapt to different teaching styles and classroom environments (as per RO2).}
This approach also evaluated the tool's potential to reduce barriers {of entry} to PBL and allow educators to focus more on experiential learning and holistic development.

% We aimed to establish new, generalizable methods to determine the requirements for tools that educators would want to use for assessing student progress, setting learning goals, providing iterative and personalized feedback, and managing resources, timelines, and artifacts across various student projects and milestones. To achieve this, we included perspectives from teachers who do not regularly implement project-based learning (PBL) pedagogy in their classrooms. This approach was intended to ensure that the GenAI tool wireframes were adaptable to diverse teaching methods and environments.

% By involving these teachers, we sought to identify any potential limitations or areas for improvement in the wireframes that may not have been evident from the PBL-focused feedback. Additionally, this inclusion helped us evaluate the tool’s potential to promote a more experiential learning approach by addressing the systemic barriers to incorporating PBL in different school settings. By making time-consuming tasks such as grading and project management more manageable, we explored whether the tool design could enable educators to focus more on facilitating action learning and holistic development within PBL environments.

\subsubsection{\textbf{Participants and Recruitment}}
\label{study2methods}

Thirty STEM teachers from diverse U.S. states and school contexts were recruited in collaboration with the MIT Scheller Teacher Education Program and MIT Alumni Clubs. These teachers, admitted to the Science and Engineering Program for Teachers (SEPT) program, participated in a 2-hour workshop on GenAI for project-based and collaborative learning. 
We distributed a pre-survey to explore the backgrounds, {subjects,} and experiences of workshop participants {(see Table \ref{tab:study2_participants}). Teachers primarily taught middle and high school students and had different levels of PBL experience (average of 3.4 years) and comfort (average of 3.8/5) resulting in varying expertise.}
Participants were equally divided between those who had found success with PBL and those who had not.



% 30 STEM teachers were recruited from across the United States ensuring a diverse pool from various states and school contexts. Staff members in the MIT Scheller Teacher Education Program (STEP) collaborated with the MIT Alumni Clubs to review teacher applications for the MIT Science and Engineering Program for Teachers (SEPT). Teachers admitted to the program participated in our 2-hour workshop on GenAI support for project-based and collaborative learning as part of their week-long program. Here is a breakdown of subjects/disciplines participants in this workshop taught in decreasing order: Science: 15, Mathematics: 7, Computer Science: 6, Engineering and Technology: 4, STEAM (Science, technology, engineering, art and math): 3, English: 1, Business: 1, and Teacher Education: 1. Participants also taught across a broad range of age groups, with the majority working in middle and high school settings (see figure \ref{fig:age-groups}).

% \begin{figure}
%     \centering
%     \includegraphics[width=1\linewidth]{age-groups-study2.png}
%     \caption{Study 2 participants' student age groups}
%     \label{fig:age-groups}
% \end{figure}

\subsubsection{\textbf{Data Collection}}

% \subsubsection{Background Survey}
% [REMOVE THIS SECTION AND PUT IT UNDER PARTICIPANTS]
% We distributed a pre-survey to explore the backgrounds and experiences of workshop participants, covering experience and comfort levels with PBL, ratings on implementing assessment methods, experience with GenAI tools, and interest in collaborative/project-based learning. 

% We distributed a pre-survey to further investigate the backgrounds and prior experiences of teachers attending this workshop. Here were the following areas we covered: 

% \begin{enumerate}
%     \item Age groups of students taught 
%     \item Years of experience with project-based learning
%     \item Comfort levels with project-based learning (on a five-point likert scale)
%     \item Description of prior experiences with project-based learning where participants selected one of the five options: 
%     \begin{enumerate}
%         \item Difficult Experience: Significant challenges, poor engagement, and outcomes.
%         \item Mostly Challenging: Many difficulties with occasional successes.
%         \item Neutral: Equal balance of challenges and successes.
%         \item Mostly Successful: Positive experience with manageable challenges.
%         \item Very Successful: Highly positive, engaged students, and excellent outcomes
%     \end{enumerate}
%     \item Ratings on past experience implementing assessment/grading methods for evaluating student projects and learning outcomes in PBL
%     \item Experience with GenAI tools like ChatGPT
%     \item Levels of interest in collaborative/project based learning
% \end{enumerate}
% We present these responses in chapter \ref{chap:5}. 

% \subsubsection{Polls on AI perceptions and collaborative learning}

% We kickstarted the workshop with polls using \href{https://www.mentimeter.com}{Mentimeter} to gauge participants' perceptions of AI and collaborative learning. Questions covered included their feelings about AI, familiarity with GenAI (from novice to experienced), experiences using AI for creating curriculum materials and rubrics, and views on how their teaching roles might change with AI.

% We kickstarted our workshop with a few polls to gauge all 30 participants’ perceptions of AI and collaborative learning. We used \href{https://www.mentimeter.com}{Mentimeter} as our polling platform. These were the following questions we covered: 

% \begin{enumerate}
%     \item What feelings do you have about AI? Describe in one or two words
% Examples of participant responses: exciting, limitations, resourceful, nervous, potential dangerous, etc.

%     \item What is your familiarity with GenAI?
%     \begin{enumerate}
%         \item NOVICE - What's GenAI? 
%         \item BEGINNER - What can I do with GenAI? 
%         \item INTERMEDIATE - How does GenAI work? EXPERIENCED - How can I design GenAI tools? 
%         \item EXPERIENCED - How can I design GenAI tools?
%     \end{enumerate}
%     \item Have you used AI for creating curriculum materials/activities before? Describe your experience
%     \item Have you used AI for creating rubrics/grading? Describe your experience
%     \item How do you see your role as a teacher changing with the introduction of AI tools?
% \end{enumerate}
% We present the responses to these reflections Chapter \ref{chap:5}.

% \subsubsection{Discussions on the evolving role of teachers}

% After the polls, participants were divided into three groups based on their interests: curriculum support, assessment support, and progress tracking support. In the first 15-20 minutes of these smaller sessions, participants discussed defining their identity with AI, identifying tasks that should remain human-driven, and the impact of AI on teaching creativity. We also examined how AI tools affect teacher well-being, focusing on their impact on workload and stress, the parts of teaching that educators enjoy the most, and the adaptability of AI-generated lesson plans and assignments to support these aspects.

% Following the group discussions with polls, we split participants up into three groups corresponding to the three key areas: curriculum support (including project brainstorming and lesson planning), assessment support (including rubric creation with differentiation and grading), and progress tracking support (including monitoring progress at the individual student, group, and class levels) [from section \ref{testing}]. Participants self-selected one of the three groups depending on their pedagogical area of interest. We spent the first 15-20 minutes of this smaller one-hour group session discussing the following:

% \begin{itemize}
%     \item \textbf{\textbf{Identity and Role of Teachers}}
%     \begin{itemize}
%         \item How do you define your identity as a teacher in the context of using AI tools?
% What are the core aspects of teaching that you believe should remain human-driven?

%         \item Are there specific tasks or roles you believe should never be automated by AI?
% Why do you feel these should remain within the teacher's control?

%         \item What impact do you think GenAI has on your teaching creativity? 
% Can you provide examples of how you incorporate your own ideas into AI-generated content?

%     \end{itemize}
%     \item \textbf{\textbf{Teacher well-being}}
%     \begin{itemize}
%         \item How do you think using GenAI tools affects your workload and stress levels? 
% Do you find that it helps reduce burnout or does it add new challenges? 

%         \item What aspects of teaching do you enjoy the most? 
% How do you think GenAI tools could enhance or detract from these aspects? 

%         \item In your experience, how flexible are AI-generated lesson plans and assignments? 
% How often do you need to modify or adapt them to fit your classroom needs?

%     \end{itemize}
% \end{itemize}

% \subsubsection{PBL GenAI Wireframes Walkthrough}

% \begin{figure}
%     \centering
%     \includegraphics[width=0.75\linewidth]{wireframes-study2-comments.png}
%     \caption{Example of study 2 participants interacting with the wireframes}
%     \label{fig:wireframes-study2-comments}
% \end{figure}

Study 2's workshop started with brief discussions on participants' perceptions of GenAI and collaborative learning. Following these, participants were divided into three groups based on their pedagogical interests: curriculum planning, assessment, and progress monitoring.
Similar to our procedure at the end of section \ref{testing} of Study 1, we walked Study 2 participants in our small group sessions through the same set of wireframes, which focused on curriculum support, assessment support, and progress tracking. After a brief review, discussions centered on the specific support area participants in the small group had selected. Participants explored the wireframes, provided feedback, and assessed usability in their classroom contexts. They also identified tool tasks best kept human-driven, and their impact on teaching creativity, well-being, workload, and stress.
% Similar to our procedure in section \ref{testing}, we also walked participants in our small group sessions through the same set of wireframes. To remind the reader, our wireframes focussed on three key areas: curriculum support (including project brainstorming and lesson planning), assessment support (including rubric creation with differentiation and grading), and progress tracking support (including monitoring progress at the individual student, group, and class levels). After briefly reviewing the wireframe pages, we focussed our small group discussions on the support that participants had originally picked when self-selecting a group (check previous section). We gave them opportunities to explore the different supports, leave comments on the design features and decisions, and provide constructive feedback for improvement through informal discussions [see figure \ref{fig:wireframes-study2-comments} with names redacted]. Teachers specifically assessed the usability of the tool's wireframes within their classroom contexts and unique pedagogical practices. We elaborate on these wireframes’ feedback from both studies combined in Chapter \ref{chap:5}.
We followed this with a post-survey to gauge participants’ excitement and concerns on GenAI, interest in PBL, desired additional features for the wireframes, and onboarding resources needed for classroom use.
% \subsubsection{Post-survey}
% After the small group sessions, we distributed a post-survey to gauge participants’ perceptions of AI and PBL after interacting with the wireframes. We received 22 responses from 30 participants, covering excitement and concerns on GenAI, interest in collaborative/project-based learning, desired additional features for AI PBL tools, and onboarding resources needed for their classroom use.

% Following the small group sessions, we distributed a post-survey to understand participants’ perceptions of AI and PBL after interacting with the wireframes and associated discussions. We received 22 responses to the post survey out of the 30 consented participants. We covered the following areas of evaluation in our post survey: 

% \begin{itemize}
%     \item Participants’ excitement about the potential of using GenAI solutions (5-point likert scale)
%     \item Participants’ feelings about GenAI tools, more generally (where 1 was super pessimistic and 5 was super optimistic)
%     \item Participants’ interest in collaborative and project-based learning in classrooms
%     \item Additional features participants wanted to see in the tool (that were not already in the wireframes they engaged with)
%     \item Onboarding resources needed to use the tool in their classrooms
% \end{itemize}

\subsubsection{\textbf{Data Analysis}}

Discussions from all three groups were audio recorded with consent from participants for the purposes of transcription. We then deductively applied the same themes from Study 1 to code our data from Study 2 \cite{azungah2018qualitative}. \textbf{For the purposes of this paper, we focus on Study 1's interviews and co-design workshops, as well as the PBL GenAI wireframes and their feedback from Study 1 and Study 2.
Note that the feedback on the wireframes has been analyzed and discussed by combining the results from both studies, as they together shaped the design guidelines and future directions for the tool.}


\begin{table*}[ht]
    \centering
    \begin{tabular}{|p{0.2\linewidth}||p{0.2\linewidth}|p{0.5\linewidth}|}\hline
  {\textbf{Category}}&\textbf{Themes} & \textbf{Definitions}\\\hline \hline 
  {\multirow{15}{*}{(A) Set stage for PBL needs }}& school context& details on subjects taught, grades, public/private school, student backgrounds, teacher experience, the way PBL is structured in the school\\ \cline{2-3}
          & specific project examples& specific examples that teachers gave on projects they have implemented in their classrooms. Include details about the project sp we know what is, not just in reference to other parts pedagogy.\\ 
          \cline{2-3}& student agency& how teachers give students autonomy, choice, freedom on projects, including group formation\\ 
                    \cline{2-3}& student differentiation & how material/grading/instruction is adapted for different groups/needs of students, including personalized support. Supports for some students.\\ 
                    \cline{2-3}&student scaffolding& any and all tools/strategies for scaffolding different parts of PBL for students. Supports all students get.\\  
                    \cline{2-3}&teacher role& role of teacher in PBL setting\\ 
                    \cline{2-3}&personalized feedback& how individual and personal feedback is given to every student\\ 
                    \cline{2-3}&goals, checklists, progress tracking& anything around setting goals, milestones, checklists, deadlines, monitoring regular progress etc. (exit tickets)\\  
                    \cline{2-3}&online learning platforms and tools& types of online learning platforms and grading platforms used (e.g. Canvas, Google Classroom)\\  
                    \cline{2-3}&rubrics/grading scales& details on rubrics and grading schemes/strategies used for evaluating different parts of projects \\ 
            \cline{2-3}&self and peer reflections&any details on self and peer reflections/feedback as part of projects\\
             \cline{2-3}&managing student groups&details on managing student groups and dynamics, interpersonal skills and interactions. conflict resolution, assigned roles, leveraging different strengths etc.\\ 
            \cline{2-3}& remote PBL&experiences and challenges of PBL in online settings. Includes structure and procedures for conducting class\\
           \cline{2-3}& PBL inclusion&details on gender gaps, STEM, equity, etc.\\ 
           \hline \hline
{\multirow{4}{*}{(B) Challenging aspects}}&PBL challenges for students&specific details on challenges encountered by students in PBL\\ \cline{2-3}
  &PBL challenges for teachers&specific details on PBL challenges encountered by teachers\\ \cline{2-3} 
  &PBL teacher perceptions&teacher perceptions of PBL\\ \cline{2-3}
  &PBL student perceptions&student perceptions of PBL\\ \hline \hline
{\multirow{3}{*}{(C) AI + PBL }}&AI perceptions&any details on teacher's perceptions/details of AI, including GenAI \\ \cline{2-3} 
  &current AI uses/challenges in classroom&details on use cases and challenges from using AI in classrooms\\ \cline{2-3} 
  &PBL wish list&details on things teachers wish for in the future in their classrooms\\ \hline 
    \end{tabular}
    \caption{Final set of themes and their definitions from our data analysis. {These span three main types: (A) those that set the stage for PBL needs, (B) those that describe challenging aspects of PBL implementation, and (C) those that show AI integration in PBL}}
    \label{tab:themes}
    \Description{The table outlines key themes and their definitions related to project-based learning (PBL) and AI in education. It includes themes like school context, student agency, teacher roles, and challenges faced by both students and teachers in PBL. It also addresses the use of online learning platforms, AI perceptions, and inclusion issues such as gender gaps and equity in PBL settings. Themes have been classified by category: (A) those that set the stage for PBL needs, (B) those that describe challenging aspects of PBL implementation, and (C) those that show AI integration in PBL.}
\end{table*}

\section{RESULTS}

\begin{figure*}[ht]
    \centering  \includegraphics[width=0.9\linewidth]{samples/timeline.png}
    \caption{A timeline illustrating the various stages of our study alongside the corresponding paper section numbers where their findings are discussed. For Study 2, note that we present findings from the final wireframe testing phase only, as mentioned in Section 3.2.3}
    \label{fig:methods-overview}
    \Description{The figure provides a timeline of two studies in our paper. Study 1 involves expert PBL teachers and begins with surveys and interviews to gather insights into their demands and challenges in implementing PBL. This is followed by two co-design workshops: the first workshop centers around problem space brainstorming, an introduction to generative AI (GenAI), and the storyboarding of GenAI concepts, while the second workshop focuses on conceptual prototyping of PBL GenAI tools, with discussions on ethical impacts and classroom integration. Study 2 engages PBL teachers with varying levels of expertise through workshop surveys and polls to collect their experiences with AI and PBL. Additionally, small group discussions explore their perceptions of AI in collaborative learning and how it influences their roles and identities as teachers. The findings from both studies are integrated into the iterative development and testing of wireframe prototypes for a new LLM tool designed to support PBL. Teacher feedback from both study participants is used to refine the tool's design around curriculum, assessments, and progress tracking supports.}
\end{figure*}
\begin{figure*}[ht]
    \centering    \includegraphics[width=0.6\linewidth]{samples/mapping.png}
    \caption{A figure displaying the mapping of research objectives to the different stages of the study and their results' coded themes}
    \label{fig:mapping}
    \Description{This figure displays the mapping of research objectives to the different stages of the study and their results' coded themes. RO1 is explored through study 1 surveys and interviews with expert teachers. These give rise to themes in categories A and B. RO2 is explored through study 1’s co-design workshops with expert teachers, study 2’s surveys, polls and small discussions with teachers of varying levels of PBL expertise. This leads to the emergence of category C themes. Additionally, RO2 is also examined using the PBL-LLM wireframes evaluated with both study 1 and study 2 teachers. These wireframes see themes from all three categories A, B and C. Finally, RO3 is explored in the design recommendations subsection of the paper under Discussion.}
\end{figure*}

Table \ref{tab:themes} shows themes and their corresponding definitions that emerged from our data. {Section 4.1 delves into Category A themes that set the stage for foundational PBL needs and Category B themes regarding challenging aspects associated with addressing these needs. Section 4.2 highlights Category C themes that suggest pathways for integrating GenAI (specifically LLMs) into PBL, drawing on key insights from our two co-design workshops. Finally, Section 4.3 presents wireframes for our proposed PBL LLM tools, synthesizing themes from all three categories. This mapping of research objectives, study stages, and themes is shown in Figures \ref{fig:methods-overview} and \ref{fig:mapping}.}

\subsection{Exploring Demands and Challenges in Project-Based Learning {(Category A and B themes)}}
We present findings from our interviews in Study 1, focusing on themes highlighting teachers' core beliefs about the demands of high-quality PBL implementation {(Category A in 4.1.1)} and the challenges for them and their students {(Category B in 4.1.2). These answer RO1}. { We recognize that some of the findings from these interviews align with themes already discussed in the literature. However, our primary aim was to build relationships with our teachers by gaining a nuanced, in-depth understanding of the types of projects they implement within their distinct subject areas, along with the specific challenges they face. By centering teachers' voices and lived experiences, we capture and communicate the authentic context of their pedagogy, enabling readers to better appreciate how the subsequent co-design artifacts are shaped by and respond to their unique classroom demands.}

% \subsubsection{Understanding PBL School Contexts}

% The diverse contexts in which project-based learning was implemented by teachers in our study—from public to private schools and across a range of subjects and grade levels—highlighted its versatility and adaptability. Teachers taught subjects from design and engineering to math, computer science, social studies, business, and life skills across middle and high school levels. Many implemented interdisciplinary projects in partnership with other teachers at their schools, combining subjects such as art, history, and engineering, to create a collaborative learning experience. 
% % Additionally, specialized courses like artificial intelligence and vocational training highlight the breadth of PBL applications.
% Both public and private school settings in the US were represented in our participant sample. Private schools offered flexibility and resources for self-directed learning, while public schools, sometimes restricted by standardized curricula, still implemented PBL through programs in business and STEM. Our teachers worked with students from diverse socioeconomic backgrounds and reading levels, requiring adaptable and inclusive project designs to meet diverse needs. 

% Private schools often offered more flexibility and resources for self-determined learning, allowing students to design their own curricula and pursue individual projects. Public schools, while sometimes constrained by standardized state-wide curricula, also implemented PBL through structured programs focusing on business, entrepreneurship, and STEM education. These settings provided a diverse context for PBL, with each school type contributing unique advantages and challenges within our sample.

% \textbf{Student Backgrounds:} Teachers in our study taught students from varied backgrounds, influencing how PBL was implemented and received. In private schools, the student body was extremely diverse, with a mix of privileged students and those from less advantaged backgrounds on scholarships. Public schools often served a broader demographic, including first-generation college students and those from lower-income families. The range of academic levels within classrooms, from students reading below grade level to those ahead of the curve, required teachers to formulate adaptable and inclusive project designs to meet diverse needs.

% Our teachers brought a wealth of experience to their roles, shaping how PBL was structured within their classrooms. Many educators oversaw specialized labs or collaborative spaces where students independently pursued diverse projects, receiving targeted feedback and resources. They adopted innovative models like flipped classrooms and mastery learning to enhance engagement and understanding. PBL classroom pedagogy often included a mix of traditional lectures and hands-on activities, ensuring students developed the foundational knowledge with multiple avenues to apply those skills to real-world scenarios. We also saw themes related to inclusion in PBL in our data, particularly addressing gender gaps, equity, and support for students with diverse abilities. Teachers highlighted challenges in maintaining female participation in STEM and emphasized using culturally relevant projects to empower marginalized students. Strategies like social contracts promoted collaboration and inclusion for students with varying abilities, ensuring a supportive environment for all learners.

% Teachers noted the challenge of maintaining female participation in STEM fields as students progress into high school, highlighting the need for sustained efforts to support young women in STEM through PBL initiatives. They also emphasized the importance of culturally relevant projects that empower marginalized students to address issues pertinent to their communities. Additionally, strategies such as social contracts were used to foster collaboration and inclusion among students with varying physical and cognitive abilities, ensuring a supportive environment for all learners.

% \subsubsection{\textbf{Differentiated Instruction and Student-centered Learning}}

% \paragraph{\textbf{Student Scaffolding}} Our participants employed a variety of scaffolding techniques to support all students in PBL, helping them develop skills progressively and build confidence in their abilities. Some teachers in our study began by integrating essential skills and tools through foundational exercises, gradually scaling them up to more complex projects as students' capabilities grew. 
% They encouraged critical thinking by asking guiding questions to promote independent problem-solving:

% \begin{quote}
%   \textit{  "I ask them questions back. If they say the glue (project materials) isn't working, I'll ask, ‘Why do you think the glue isn't working? What glue did you use?’ This usually helps them think through what they did and solve the problem themselves." -- P10}
% \end{quote}

% Teachers also mentioned using design thinking as a scaffold:
% \begin{quote}[r]{-- P10}
% \textit{“We work in design thinking circles, starting with paper, then writing, ideation, and prototyping. The first prototype never works, and by the third one, maybe it does. They understand this cycle by now. Earlier in the year, it's always a fight, but they learn the cycle of how to build something. By 11th and 12th grade, they understand the process and use their time wisely. If we're learning software, we'll do a 10-15 minute lesson, and then they can work on something, chunking it up because they can't sit for an hour listening to me. I'll show them the software, give them a tutorial as homework, and then have a deliverable to ensure they did it. }
% \end{quote}
% Scaffolding also included personalized support through one-on-one sessions, office hours, and peer collaboration. For example, students with more experience were paired with those who were less familiar with tools or concepts, fostering a collaborative learning environment. Teachers often provided high-level examples to set standards and expectations, and conducted small group sessions to ensure personalized instruction. To aid in scaffolding, teachers also leveraged various technological tools and resources. Projects often began with clear instructions, rubrics, and structured templates to guide students through each phase. For example, in technology-based projects, students started with guided tutorials and detailed documentation and gradually took on more complex tasks, ensuring they were not overwhelmed by technical challenges.
% Scaffolding also included personalized support through one-on-one sessions, office hours, and peer collaboration, pairing experienced students with beginners. Our teachers provided high-level examples to set standards and expectations, and used templates to guide students through each project phase. In technology-based projects, students began with tutorials and gradually moved to more complex tasks to prevent being overwhelmed. Continuous assessment and feedback were crucial components of scaffolding, allowing teachers to adjust support based on student progress. Examples of student work were shared to provide peer benchmarks and inspiration, helping students see different approaches and improve their own projects. These activities built resilience and adaptability, fostering student autonomy.

% Teachers used formative assessments to gauge student progress and adjust support accordingly. Examples of student work were shared to provide peer benchmarks and inspiration, helping students see different approaches and improve their own projects. Scaffolding was enhanced through engaging activities that promote critical thinking and problem-solving. Teachers designed scenarios and challenges that required students to think creatively and apply their knowledge in new ways.

% \begin{quote}[r]{-- P02}
% \textit{"Sometimes projects don't work...Students are often too averse to failure. Scaffolding and helping students set reasonable expectations are key." }
% \end{quote}

% These activities not only made learning more interactive but also helped students develop resilience and adaptability. The ultimate goal of scaffolding was to build student autonomy. Our teachers gradually reduced support as students demonstrated proficiency. This transition was based on ongoing assessments, with scaffolding removed when students consistently showed they could manage tasks independently. Effective scaffolding in PBL thus involved a mix of skill-building exercises, critical questioning, iterative design processes, personalized and group support, technology integration, continuous assessment, and engaging activities. These strategies collectively ensured that all students received the necessary support to succeed in project-based learning environments, fostering both independence and collaborative skills.

% \paragraph{\textbf{Student Differentiation}} Our teachers employed various strategies to differentiate materials, grading, and instruction to meet diverse student needs. 
% They allowed flexibility in projects aligned with students' interests and career goals.
% offered targeted feedback, and provided individualized support through one-on-one sessions and check-ins. 
% In courses like digital marketing and media, students created projects that reflected their interests and potential career paths, such as designing a museum or developing a business plan.
% Projects were broken down into manageable tasks with clear instructions, enabling students to work at their own pace, with faster students moving on to advanced tasks while others received additional exercises to catch up. 
% In project groups, roles were assigned to cater to students' strengths and needs. For instance, in a business class, roles like chief executive officer or chief financial officer were designated based on students' skills and interests.
% \begin{quote}
% \textit{"We also ensure groups are balanced so that students who need more help don't always rely on the same peers, fostering empowerment rather than burden." -- P06}
% \end{quote}

% \paragraph{\textbf{Student Agency}} Our teachers fostered student agency by giving students autonomy, choice, and freedom in their projects, allowing them to take ownership of their learning and develop their individual interests and skills. Projects often started with broad questions or themes, allowing students to choose specific problems to investigate:

% \begin{quote}
% \textit{“The most important thing is that the driving question of the project is being addressed. The content standards should be hit. Students can choose different types of artifacts to express their answers, such as a podcast, an essay, a play, a children's book, etc. The type of deliverable isn't as important as how well it addresses the question and incorporates the story elements—networks, communities, production, and distribution—that were impacted by the artifact. The central question should guide their work. After presenting all these stories, we can have a discussion to draw conclusions and reflect on the central question.” --P03}
% \end{quote}

% \begin{figure}
%     \centering
%     \includegraphics[width=0.75\linewidth]{choice-boards.png}
%     \caption{Examples of choice boards used by our teachers to promote student agency}
%     \label{fig:choice-boards}
% \end{figure}

% Teachers like P06 often used choice boards to offer students various options for completing projects. For example, students could choose to create infographics, slide presentations, or digital drawings, each with clear instructions and extension activities. A few examples of these choice boards for different projects are shown in figure \ref{fig:choice-boards}.


% \subsubsection{Project Management Strategies and Assessments}
% \paragraph{\textbf{Goals, Checklists, and Progress Tracking}} Teachers mentioned regularly reinforcing deadlines to help maintain student engagement and ensure timely task completion. 

% \begin{quote}
% \textit{“I feel like a coach with these students. I start with a brief instructional section since my classroom is a lab with tools all around, which can be distracting. Most of the reading is assigned as homework. In class, we discuss our goals for 5 to 10 minutes, then they work while I go table to table, supporting them…. I'll set up in front of the class what to do and what I expect them to learn each day. This becomes a classroom norm, especially with the tenth graders who need more support. I'll have it up on the whiteboard and reinforce it verbally.” -- P10}
% \end{quote}

% Teachers supported students in setting regular goals, with students submitting a weekly deliverable for their projects. P06 used Padlets for students to submit their daily work (e.g., snapshots, videos, responses) and as a reflective tool for exit tickets, where students documented their learning journey, challenges, and their solutions (see figure XXX for example). These methods fostered metacognitive skills and informed instructional strategies, providing a centralized platform for tracking progress and enabling timely feedback and interventions. 


% This iterative process of documenting, reflecting, and evidencing helped students internalize the cycle of thinking, doing, and reflecting as projects grew in complexity.

% \begin{figure}
%     \centering
%     \includegraphics[width=0.75\linewidth]{padlet.png}
%     \caption{Examples of Padlets used by our teachers to document student progress and reflections}
%     \label{fig:padlet}
% \end{figure}

% \paragraph{\textbf{Rubrics/Grading Scales}} 
% The rubric allowed nuanced evaluations with increments of 0.25 points to differentiate levels.
% Standards-based grading was a common practice, with P02 using a three-tier system (developing, beginner, mastery) to track progress and support differentiated instruction. Some teachers, like P9, involved students in creating rubrics for greater ownership, using their informal feedback to refine criteria. In contrast, P5 preferred rubrics with more general objectives (instead of specifying exact requirements) that outlined key components needed for success but allowed for flexibility in how students met the criteria. Interpersonal skills and social-emotional learning (SEL) were also integrated into grading rubrics. For example, P03 and P06 combined students' critical thinking, collaboration, and empathy with content standards, within their four-point rubric. This approach ensured a comprehensive assessment of both content and process across the various rubric formats.

% \paragraph{\textbf{Managing Student Groups}} 
% Our teachers employed strategies to manage group dynamics, foster interpersonal skills, and leverage individual strengths, including conflict resolution, role assignments, and regular check-ins. P02 promoted frequent group rotations to expose students to working with a variety of peers . Additionally, P02 utilized a "\textit{Three, then me" }strategy, encouraging students to seek help from three peers before approaching the teacher, thereby fostering peer support and collaboration. P04 mediated conflicts and ensured equity by assigning clear roles and rotating them across projects. P10 emphasized the importance of designing teams with balanced skills and personalities. By pairing students with complementary strengths, students could learn from each other and enhance their overall performance.

% \paragraph{\textbf{Self and Peer Reflections}} 
% In project-based learning environments, self and peer reflections played a pivotal role in enhancing student engagement, fostering critical thinking, and promoting collaborative learning. 
% Teachers employed strategies to incorporate self-assessment and peer feedback, ensuring that students had opportunities to reflect on their learning processes, understand their strengths and weaknesses, and receive constructive feedback from their peers.

% \begin{quote}
%     \textit{“The big self-reflection is at the end when we do a \href{https://en.wikipedia.org/wiki/Harkness_table}{Harkness discussion} and go back to the driving question. This is a chance to verbalize what the learning was...What did you learn from it as far as the driving question? What did you learn about your own process, and what would you do differently?...It becomes a conversation about what collaboration means and how well they think they did.” -- P03}
% \end{quote}

% P04 utilized exit tickets, reflective slides in project portfolios, and KWL sheets (What do I know? What do I wonder? What do I want to learn?) at the beginning of projects. 

% Peer feedback was another critical aspect of the reflective process, providing students with diverse perspectives on their work and fostering a collaborative learning environment. P10 had students critique peer presentations using post-it notes with one positive comment and one suggestion for improvement. P8 implemented the "Glows and Grows" method during presentations, where students shared what excited them about their own projects (glows) and identified areas for improvement (grows), fostering a growth mindset. Other teachers used anonymous peer evaluation forms for constructive feedback on work and group dynamics, promoting accountability and support among students. 

\subsubsection{\textbf{Demands in Project-Based Learning}}
% \paragraph{\textbf{Teachers’ Perspectives on PBL}} 
Teachers in our study emphasized the distinct elements of PBL that had to be addressed to foster engagement and promote deeper learning.
% Many preferred PBL over traditional testing, as it allows students to demonstrate learning through projects, with ongoing assessments to monitor progress. 
P10 stressed the importance of focusing on the big picture while managing granular project details, helping students see the connections between project components.
Teachers underscored the interdisciplinary nature of PBL, requiring them to develop the expertise to integrate subjects like physics, engineering, and design, to foster holistic learning. 
% P02 highlighted that minimal direct instruction encouraged independent and collaborative work, enhancing critical thinking and problem-solving. 
% Teachers also valued the social-emotional benefits of PBL, particularly for middle school students, and noted the importance of understanding students' backgrounds to create personalized learning experiences.
% They highlighted the need for openness and flexibility in adapting to PBL's unstructured nature. 
The \textbf{adaptability and improvisational aspects of PBL} were recurrent themes, as illustrated by P06:

\begin{quote}
    \textit{“Last year, I was ready to throw my 3D printer out the window because students couldn't even name their projects properly despite detailed instructions. I had to rethink my approach and make tweaks...Flexibility and recognizing that you don't have everything figured out is key to successful PBL.” -- P06} 
\end{quote}

Teachers shared how \textbf{prior PD experiences}, including workshops and training, were crucial for implementing PBL effectively and staying current with evolving educational practices and technologies.
They also spoke about their roles and responsibilities as \textit{\textbf{``expert facilitators and coaches''}}, helping students navigate projects while promoting independence.
P02 noted that much of the time was spent observing, asking questions, and offering support as needed, allowing students to explore and experiment. Handling multiple projects required effective \textbf{context-switching and organization.} P07 stressed building student confidence, particularly in challenging subjects like math, and creating a supportive classroom where mistakes were part of learning. 
Our teachers relied on standardized rubrics and project descriptions to ensure fairness, but they had to \textbf{adapt the content to different abilities}, necessitating the creation of multiple versions of resources.
% \begin{quote}
%     \textit{“If I have a student with a different reading ability, I might provide them with sources that read aloud or adjust the reading level. Their checklist might include fewer words or simpler tasks. If the rubric asks for five examples and I know a student can only manage three, I adjust their rubric accordingly. We try to set realistic expectations for each student and hold them accountable for those.” --P06}
% \end{quote}
Teachers also understood that some students preferred hands-off approaches while others needed frequent support, and they balanced these needs through annotated monitoring and \textbf{personalized interactions}:

\begin{quote}
\textit{"Some students embrace it and love the challenge and freedom to explore in their own way. Others resist and rebel against it, preferring a more traditional approach with worksheets, lectures, and tests. And then there's everything in between. Some are willing but don't know how to do it." --P03}
\end{quote}

% \paragraph{\textbf{Students’ Perspectives on PBL}} 
% Students saw PBL as a way to engage in meaningful, hands-on projects beyond rote memorization and standardized testing, fostering a deeper understanding of the material. 
% Teachers used scaffolding, like checklists and clearly defined assignments, to help students transition to PBL's self-directed nature. Teachers mapped rubric items to specific content standards (from state or nation-wide frameworks), as well as other dimensions like design, aesthetics, creativity, and sustainability, fine-tuning categories yearly to align with evolving goals. These tools helped students stay organized and focused, preventing them from becoming overwhelmed by the scope of their projects. Teachers tackled managing group projects by defining tasks and deadlines for each member's role from the outset, ensuring equitable contributions and smooth project progress. Regular check-ins, documentation, and reflection established a structured yet flexible environment, keeping students engaged while building skills in time management, self-assessment, and collaborative problem-solving.
% These tools allowed students to outline their existing knowledge, curiosity, learning goals, and progress, enhancing their metacognitive skills. 
Teachers used \textbf{scaffolding tools like checklists} to help students adapt to PBL's self-directed approach, keeping them organized and focused. They mapped rubric items to content standards and other criteria, such as creativity and sustainability, refining these annually to match evolving goals. To manage group projects, teachers proactively defined tasks and deadlines for individual members, conducted \textbf{regular check-ins}, and \textbf{encouraged reflection}, ensuring equitable contributions and progress. These strategies fostered a structured yet flexible learning environment, enhancing students' time management, self-assessment, problem-solving, and metacognitive skills.

\subsubsection{\textbf{Challenges in Project-Based Learning}}
Implementing PBL presented organizational difficulties for teachers, balancing content depth and breadth, and managing diverse student needs.
The shift from traditional assessments to more dynamic, project-based evaluations also demanded innovative assessment methods. P02 highlighted the challenges of \textbf{evaluating open-ended projects without additional support} staff in schools. 
% Personalizing learning within PBL was difficult, as some students thrived on open-ended exploration while others needed more structure:
% \begin{shadequote}[r]{-- P04} \textit{"The transition has its challenges. Building the process of independent thinking is a big one because students are used to being directed by the teacher."} \end{shadequote}
Teachers grappled with balancing the depth of knowledge required for specific projects with the breadth of content mandated by the curriculum. This often meant that students developed an in-depth understanding of their chosen project topic but lacked comprehensive knowledge of other areas, impacting their performance on traditional assessments.  
Integrating technology also added complexity, with students' enthusiasm for new tools sometimes \textbf{diverting focus from project goals}. Additionally, the availability of resources and the need to create or adapt materials presented ongoing struggles. Teachers often felt constrained by their curriculum's scope and sequence, which limited their ability to fully embrace PBL's potential.
\textbf{Fair grading} was particularly challenging due to the subjective nature of projects and the need for continuous feedback. P10 highlighted the \textbf{workload} involved:

\begin{quote}
\textit{"It's a lot of work. I'm constantly grading. I don't get a break because it's a continuous cycle of work. I give iterative feedback throughout, and my turnaround is as quick as possible, but it means grading all weekend." -- P10}
\end{quote}

\textbf{Maintaining student engagement} was another recurring challenge in PBL. P01 noted that some students (e.g. struggling readers) found it hard to stay on task, requiring additional prompts and support to remain focused. 
\textbf{The transition from lecture-based learning to PBL} was difficult, with students initially resisting the shift from teacher-directed instruction. P05 recounted how some students found projects daunting and needed the \textbf{tasks broken into smaller, manageable steps}. This step-by-step approach helped students gradually build confidence and see their progress, ultimately leading to successful project completion. Teachers also stressed the importance of social-emotional development, noting the pandemic's negative impact on students' ability to collaborate and support each other.  Flexibility and empathy were essential in helping students manage PBL challenges, particularly when overwhelmed or facing mental health issues. P09 observed that students often struggled to apply memorized concepts to projects, requiring a structured approach with front-loaded skills before diving into projects. The hands-on nature of PBL demanded continuous support to bridge the gap between conceptual knowledge and practice. P11 noted challenges with \textbf{unequal team participation}, affecting grades and contributions. Ensuring fairness required careful monitoring and regular teacher intervention. 
% \paragraph{\textbf{Remote Project-based Learning}}
% Two teachers in our study also taught project-based learning in entirely remote settings. P05 used a mix of synchronous and asynchronous activities in an online Honors Introduction to Artificial Intelligence course, providing structured guidance through shared Google Docs and discussion forums on Canvas. This approach allowed for real-time tracking of student progress and facilitated peer learning, despite the limitations of remote group work. 

These challenges in student engagement compounded in \textbf{remote settings}, with many students preferring chat over cameras or microphones. P05 used breakout rooms for small group discussions and maintained engagement through Zoom polls and exit tickets, despite inconsistent Wi-Fi and varying school district permissions. \textbf{Logistical barriers, like inconsistent access to resources}, further complicated remote PBL. 

% thoughts on commenting this theme below since it shows up in section 4.2 anyway?
% \subsubsection{AI Integration in Project-Based Learning}
% Teachers had mixed feelings about the role of AI in education. While some were optimistic about its potential, others expressed caution. Teachers noted that AI tools like Claude and Magic School were preferred over ChatGPT due to their ease of use, especially for creating sentence starters and instructional materials. These tools required less prompting and were more intuitive, aligning better with the needs of educators who might not have strong coding skills. Teachers agreed that AI should benefit both students and educators. While AI could assist in lesson planning, its use in student assessment was limited by policies and technology. Some emphasized AI's role in helping students explore complex concepts and encouraged them to critically evaluate AI-generated ideas:

% \begin{quote}
% \textit{"Students need to challenge the claims that AI makes and dig deeper. AI is a good place to start, like Wikipedia, but it’s not the final source. Use it to generate ideas and questions, but then go further." -- P03}
% \end{quote}

% AI's use in classrooms was mainly for administrative tasks like generating worksheets, rubrics, and lesson plans rather than direct teaching or assessment. For example, P04 used AI to aid project planning in a computer science club by providing a blueprint to help students brainstorm ideas. Teachers saw potential in AI for differentiated instruction, providing resources at various reading levels to meet diverse student needs.
% However, AI's use for grading was often limited by district policies, and teachers found AI-generated rubrics lacked necessary nuance. AI's challenges in understanding context and providing accurate insights were also noted:

% \begin{quote}
% \textit{"Our projects are too complex and dynamic. It would require a lot of context for the AI to understand and evaluate accurately." -- P10} 
% \end{quote}

% Teachers also faced technical and policy barriers, such as ensuring AI tool security and adapting to evolving state policies on AI use in education. 
% The co-design sessions further explored these issues, where teachers collaboratively developed scenarios for using GenAI to address concerns raised in the interviews.

\subsection{Integrating AI in Project-Based
Learning {(Category C themes)}}

Building on interviews, we present findings from the co-design process, where teachers explored customizing LLM tools to meet PBL's specific demands {(Category C). These help answer RO2.}
% We developed and refined wireframes to align with teachers' core values for seamless integration and included insights from novice PBL teachers in Study 2 on the same.

% \begin{figure}
%     \centering
%     \includegraphics[width=0.75\linewidth]{miro-all3.png}
%     \caption{Workshop 1 brainstorming boards from all three groups}
%     \label{fig:groupsall-miro}
% \end{figure}

\subsubsection{\textbf{Co-design Workshop 1: }}
The first workshop aimed to identify challenges teachers faced with implementing PBL and map scenarios in which LLMs tools could help. 
% Below, we summarize the key themes and discussions from workshop activities.

\paragraph{\textbf{Brainstorming- Challenges, Current Strategies, and “Magic wand” features:}} Across our three groups and rounds of brainstorming, we saw similarities in the categories that emerged. 
% The discussions, even when focused on assessment, covered interconnected elements like student engagement, progress tracking, differentiation, and group dynamics. This highlighted the holistic nature of PBL, where all components work together to enhance classroom effectiveness.

Participants discussed strategies for \textbf{setting goals and tracking progress} in PBL. One suggestion was a \textit{"dashboard to monitor progress with a traffic light system (red, yellow, green)" (Group 1)}, for visual tracking. Another idea involved using a checklist in the form of a \textit{"bingo board for final projects" (Group 3)}. However, these tools sometimes failed when items were checked off without actual completion. Frequent check-ins and breaking projects into manageable tasks were also mentioned, though this process was noted to be labor-intensive due to the need for continuous tracking, timely feedback, and adjustments to lesson plans.
Participants promoted \textbf{student autonomy} by \textit{"maintaining momentum” (Group 3)} during classroom time, \textit{“keeping project ideas fresh" (Group 1)}, and organizing \textit{“fun, creative tasks within projects to reduce cognitive and social discomfort” (Group 1)}. Allowing students to \textit{"pick goals with teacher feedback" (Group 1)}, and “\textit{outline the top three criteria they wish to be graded on” (Group 3)} gave them more control over their learning. However, balancing this autonomy with meeting curriculum standards remained a challenge.

Participants emphasized the need for \textbf{differentiation and personalization} by \textit{"challenging all students where they are" (Group 1)} and providing \textit{"feedback targeted at their current level of maturity, technical skill, and aspirations" (Group 1)}. There was a call for tools that offer \textit{"additional resources tailored to students' strengths and weaknesses" (Group 3)} and help navigating complex materials. Ensuring \textit{"equitable [opportunities] for students of diverse backgrounds and learning levels." (Group 3)}, along with having alternate plans if needed, was seen as crucial for student success.

\textbf{Managing group dynamics} was another area of focus within the brainstorming boards, such as dealing with \textit{"students who give minimal effort to the team" (Group 1)} and \textit{"ensuring that everyone in the group understands the tasks" (Group 2).} \textit{"Assessing [group] projects for the group as well as individuals" (Group 2)} was noted as a difficulty, highlighting the need for fair rubrics. \textbf{Time management} also surfaced as a significant concern. While self-reflecting on incremental progress was seen as valuable for students, it was also time-consuming. Integrating technology required time to train students, and providing meaningful feedback posed a challenge when balancing thoroughness with efficiency. 

Participants discussed various \textbf{grading strategies and challenges}, such as the use of external automated tools for scaffolded templates like \textit{"Repl.it, Codecheck.it, CodingRooms, Google Colab Notebooks" (Group 2)}, which were helpful but time-consuming to set up. There was debate over balancing and \textit{"assessing [final] product vs. [learning] process"} \textit{(Group 2)} within projects. 
Other concerns included integrating standards-based grading into rubrics while managing the pressure from parents to convert these into traditional letter grades for transcripts. Fairness in grading was also highlighted, given the complexity of creating rubrics for different project parts, student needs, and interdisciplinary topics. Some educators suggested involving students in the evaluation process through \textit{"self-grading along with teacher grading" (Group 3)}, which promoted student self-awareness and provided deeper insights to educators for personalized feedback. 

\paragraph{\textbf{Storyboarding Areas of Opportunity for GenAI Tools in PBL:}} Participants expanded on initial problem scoping by creating storyboards illustrating scenarios in which a hypothetical LLM tool could help address challenges in their daily teaching practices. Our analysis revealed three major areas of support: \#1: curriculum and lesson planning, \#2: assessments and grading, and \#3: managing group dynamics and progress tracking. \\ 
% We present a few examples of storyboards embodying each of these areas and discuss the specific role teachers wanted the AI system to be playing.

\textbf{[Support area \#1]- Curriculum \& Lesson Planning + [Support area \#2]- Assessments \& Grading}:
Our participants created storyboards where teachers and students used the LLM system to collaboratively brainstorm project ideas, generate lesson materials, and customize these for diverse student needs. They also envisioned the LLM aiding in creating assessments (rubrics), and automating grading with teacher input. The themes from support areas \#1 and \#2 frequently co-occurred in these scenarios. For example: 

\begin{figure*}
    \centering
    \includegraphics[width=1\linewidth]{samples/storyboard-P11.png}
    \caption{Storyboard (P11) mapping a step-by-step teacher-LLM interaction for project and assessment ideation}
    \label{fig:storyboard-P11}
    \Description{The figure illustrates a storyboard by P11 outlining the process of an LLM tool assisting teachers in planning and implementing a PBL activity. It shows steps where the teacher inputs project details, and the LLM provides suggestions for project design, student tasks, assessments, and accommodations for special needs. The LLM generates various resources, such as an introductory slideshow, exam, and necessary documents, and also autogrades student results, offering formative feedback and recommendations.}
\end{figure*}
\begin{figure*}
    \centering
    \includegraphics[width=1\linewidth]{samples/P05_new.jpg}
    \caption{Storyboard (P05) showing an LLM tool for alternative assignments and lesson plans}
    \label{fig:storyboard-P05}
    \Description{The figure showcases a storyboard from P05 with LLMs for creating alternative lesson plans and assessments. It highlights challenges, such as accommodating different learning styles and maintaining fairness in assessments, and suggests that LLMs can help quickly generate multiple prompts, brainstorm activities, and save teachers time. It also emphasizes that LLM tools can support teachers in applying Universal Design Principles and meeting regulatory requirements.}
\end{figure*}

% \textbf{[EXAMPLE 1] P11: AI Chatbot for Project Definition and Assessment Design}
% Role of AI: Chatbot generates ideas for curriculum development and assessments. Reduces administrative tasks while leaving cognitively demanding work to teachers.
P11 highlighted the challenges of defining projects and designing assessments in PBL, which can be time-consuming (upwards of 40 hours). They noted difficulty in finding comprehensive resources online suited to various instructional needs. To address these issues, P11 envisioned a LLM-powered chatbot that assists in \textbf{project creation} by generating tailored project plans based on class information and goals provided by the teacher (Figure \ref{fig:storyboard-P11}). The tool would offer suggestions for student tasks, sentence starters, and \textbf{help design rubrics and checklists}, accommodating \textbf{differentiated instruction} by considering students' physical, academic, and behavioral needs. Additionally, it could generate documents like timelines, visual progress indicators, and follow-up exams. Other teachers suggested integrating an open-source database with both tested and untested unit plans for cross-referencing into this design. This storyboard streamlines labor-intensive tasks in PBL, allowing teachers to adapt projects instead of starting from scratch.

\begin{quote}
\textit{“For formative assessments, we need status checks to ensure students are on track during a project. For example, in a 3-day project, students might stay focused because the deadline is near. But in a longer project, they might get off track. The chatbot could provide feedback on their progress, motivation, and any issues they're facing, helping them stay on course... For summative assessments, I would want the chatbot to suggest feedback and explain its reasoning, allowing me to modify it as needed. Feedback should be age-appropriate and level-appropriate. For instance, some students need positive feedback, while others might be motivated by challenges. The chatbot should offer tailored suggestions and allow me to adjust them before finalizing.” -- P11}
\end{quote}

% \begin{figure}
%     \centering
%     \includegraphics[width=0.75\linewidth]{storyboard-P10.png}
%     \caption{Storyboard (P10) depicting an AI assistant to monitor at-risk students}
%     \label{fig:storyboard-P10}
% \end{figure}

% \textbf{[EXAMPLE 2] P05: AI Tool for Alternative Assignments and Lesson Plans} 
P05’s storyboard (Figure \ref{fig:storyboard-P05}) envisioned LLMs helping \textbf{create} \textbf{alternative or customized assessments} to accommodate different learning styles. They highlighted the challenge of creating varied assessment formats, such as multiple versions of vocabulary tests, to meet diverse needs and prevent sharing of test content between different class periods. P05 suggested LLMs could quickly generate similar essay prompts or multiple-choice questions, \textbf{maintaining fairness} across classes and compiling these into a centralized question bank. They also saw LLMs as a \textbf{brainstorming partner for generating activity ideas} when time is limited, allowing teachers to focus more on individualized instruction and feedback. However, they noted concerns about accessibility, particularly in online settings (pertinent to P05) , where account restrictions limit access to GenAI tools like Google’s Gemini or Microsoft’s Copilot. Other participants discussed the learning curve in using GenAI tools like ChatGPT, noting the difficulty of recalling effective prompts and the need for prompt-sharing among educators for P05's tool. \\

\textbf{[Support area \#3]- Managing Group Dynamics and Progress Tracking}:
% We also observed several storyboards focused on utilizing AI systems to monitor progress at both individual and group levels, as well as simplifying the process of communicating with students and parents when necessary. Teachers expressed a desire for the AI system to help with documenting group project artifacts, tracking individual contributions within groups, ensuring equitable participation, and resolving conflicts among group members. We present one example storyboard from this theme below:
We observed storyboards illustrating LLM systems to monitor individual and group progress, streamline communication with students and parents, document project artifacts, track individual contributions within groups, ensure equitable participation, and resolve group conflicts.

\begin{figure}
    \centering
    \includegraphics[width=1\linewidth]{samples/storyboard-P10.png}
    \caption{Storyboard (P10) depicting an LLM assistant to monitor at-risk students}
    \label{fig:storyboard-P10}
    \Description{The figure is a storyboard by P10 depicting a process for tracking student progress using LLMs. It shows a teacher grading projects and noticing a missing assignment, prompting a red alert button. The teacher clicks to view a student's real-time progress and options to email the student or other team members. The teacher uses LLMs to schedule a check-in, track meetings, send notifications, and maintain a log of actions, ensuring consistent communication.}
\end{figure}

For example, P10 envisioned an LLM system to \textbf{support teachers monitoring at-risk students} (Figure \ref{fig:storyboard-P10}) by offering a comprehensive visual representation of a student’s academic performance, highlighting key issues like missed assignments. The system would generate detailed reports and \textbf{automate communication with colleagues, parents, and administrators}. The LLM could also help schedule meetings by coordinating calendars and setting reminders for follow-ups to facilitate timely interventions. 
% \begin{quote}
% \textit{"I wanted to do all of the assistant work that takes so, so much time, and then maybe it prompts you with a one-week check-in, a two-week check-in, or no check-in option. You can set up future meetings if you see the train wreck is going to happen, which a lot of times the teachers can see in advance." -- P10}
% \end{quote}
P10 emphasized the value of LLMs in maintaining continuity of support as students progress through different grade levels, analyzing patterns such as chronic lateness or incomplete projects to enable proactive support. Other participants suggested that the tool could also be adapted for student leaders to \textbf{track team submissions }and manage project delays, to avoid adding to the teacher's workload.

\subsubsection{\textbf{Co-design Workshop 2: }}

% In this workshop, participants developed a conceptual prototype based on the storyboard ideas and supports from Workshop 1. All groups converged on using GenAI for brainstorming new project ideas, creating detailed lesson plans, designing fair rubrics with differentiation, and tracking individual and group progress. These aligned well with the three support types identified in the Workshop 1 storyboards. Below, we present participant discussions surrounding tool stakeholders, AI data sharing requirements, classroom integration and constraints, and ethical impacts of these prototypes.
In this workshop, participants built a conceptual prototype based on storyboard ideas from Workshop 1. 
These prototypes aligned with the three support areas from the storyboards above. We present the discussions on tool stakeholders, LLM data sharing, classroom integration, and ethical impacts below.

\paragraph{\textbf{Tool Stakeholders:}}
Participants identified key groups involved in implementing LLM tools in PBL, outlining their values, potential benefits and harms: \textbf{Classroom teachers, co-teachers, }and\textbf{ new/pre-service teachers} could benefit from reduced administrative workload and support for lesson planning but would be concerned about over-reliance on LLMs and loss of control. \textbf{Special education teachers} emphasize meeting diverse student needs and Individualized Education Program (IEP) goals, with worries about data privacy and maintaining personal connections with students. Participants also discussed involving \textbf{IT teams, the Board of Education, students, }and\textbf{ parents} as stakeholders, each valuing student progress, data security, and educational equity, but wary of privacy concerns, technological divides, and the potential impact on teacher-student interactions.

\paragraph{\textbf{Data sharing requirements of LLM prototypes:}}
Participants suggested that the LLM tool should \textbf{access teacher-specific information}, such as state/local standards, project goals, past assignments, rubrics, guiding questions, and preferred instructional methods (e.g., Socratic seminars, Process Oriented Guided Inquiry Learning (POGIL)). This data would help the LLM align its outputs with educational objectives and teaching styles. To personalize content, educators also stipulated that the system have \textbf{access to comprehensive student profiles}, including standardized test scores, reading and math levels, and accommodations from IEPs. This would enable the LLM to tailor materials to match students' learning levels and accommodations, such as adjusted deadlines for students who require extra time.

Participants also emphasized the importance of \textbf{integrating the LLM tool with existing Learning Management Systems (LMS)} to allow access to student demographic data, schedules, performance records, and adaptation of outputs between teachers. This would also facilitate communication with students and families regarding academic progress and interventions. They acknowledged the need for data to enhance LLM functionality but raised concerns about privacy, especially regarding sensitive data like student learning needs or disabilities. Authorization and \textbf{compliance with data privacy laws} (Ed Law 2d, COPPA, FERPA, HIPAA) was also emphasized.
They also discussed challenges in ensuring the LLM's ability to respect privacy during data access and \textbf{teacher training to use the tool ethically} and responsibly.

% \begin{table}
%     \centering
%     \resizebox{!}{7cm}{
%     \begin{tabular}{|p{0.14\linewidth}|p{0.24\linewidth}|p{0.28\linewidth}|p{0.28\linewidth}|} \hline 
%          \textbf{Stakeholders} &  \textbf{Values} &  \textbf{Benefits} & \textbf{Harms}\\ \hline \hline
%          Classroom teachers and co-teachers&  - Creating lessons that meet educational standards
% - High student engagement
% - Fun and interactive classroom environment&  - Ability to quickly document comprehensive coverage of standards
% - Saves time
% - Reduces administrative workload
% - Encourages experimentation with new teaching methods 
% - Facilitates instructional collaboration
% & -  Hallucinations leading to bad content; - Bias in data
% - Loss of control over educational processes
% - Risk of plagiarism
% - Potential loss of originality in project ideas
% - Risk of misjudging material quality
% - Potential over-reliance on AI-generated content
% \\ \hline 
%          New/Pre-service Teachers&  Support in lesson planning and teaching methods&  -  Helps implement lessons effectively
% - Facilitates quick adaptation to teaching
% & - Over-reliance on AI 
% - May not develop essential teaching skills\\ \hline 
%          Teaching Mentors/Curriculum Coaches&  - Practical implementation of teaching methods
% - Supporting new teachers&  - Facilitates communication with new or struggling teachers
% - Aids in lesson planning& - Risk of lazy/irregular mentorship
% - Potential forced compliance with AI-driven performance assessments\\ \hline 
%          Resource Room Teachers/ Counselors/ Advisors&  - Meeting students' diverse needs
% - Ensuring student progress&  - Provides more individualized attention to students
% - Helps track and support students’ progress& - Concerns over informed consent for data use
% - Potential loss of personal connection with students\\ \hline 
%          Committee on Special Education&  - Meeting IEP goals 
% - Program modifications&  Supports classroom teachers in meeting program recommendations and accommodations& - Increased caseload pressure
% - Accessibility challenges for AI-generated content and assessments
% \\ \hline 
%          Board of Education (BOE)&  - Ensuring students meet standards
% - Graduation rates 
% - Equity among schools&  Improved student learning may enhance standardized test scores and graduation rates& - Technological divide could exacerbate inequalities
% - Budgetary concerns\\ \hline 
%          Information Technology (IT) Teams&  - Data security 
% - Ease of maintenance 
% - Reliable technology 
% - Smooth classroom integration&  - Streamlines classroom technology use
% - Provides troubleshooting support
% - Detailed analytics& - Potential interference with existing filters and privacy settings 
% - Increased workload from troubleshooting and integration challenges\\ \hline 
%          Students&  -  Preparation for college/career
% - Personalized learning
% - Academic success and good grades
% - Positive classroom experience&  - Provides timely feedback
% - Delivers high-quality, engaging material tailored to individual needs
% - Supports personalized learning& - Ethical concerns around biased outputs and information privacy
% - Risk of reduced teacher-student interaction\\ \hline 
%          Parents/Families&  - Student safety 
% - Successful educational outcomes
% - College preparation&  - Receives timely feedback on student progress
% - Supports student engagement& - AI may not fully grasp emotional and social aspects of learning 
% - Privacy concerns regarding student data
% - Risk of relying too much on AI for educational decisions\\ \hline
%     \end{tabular}}
%     \caption{Participant-curated stakeholder matrix showing values, benefits, and harms for proposed PBL AI tools}
%     \label{tab:stakeholders}
% \end{table}

\begin{figure}
    \centering
    \includegraphics[width=1\linewidth]{samples/Project_Ideation.pdf}
    \caption{PBL LLM tool supports for brainstorming project ideas}
    \label{fig:project-wireframe}
    \Description{The figure shows a wireframe for a tool to brainstorm project ideas with LLMs. On the left, it allows teachers to input details such as project topics, learning goals, deliverables, group sizes, and educational standards to generate project ideas. On the right, the generated project ideas are displayed for review and editing, with options to select, modify, or regenerate ideas and proceed with planning the project implementation.}
\end{figure}

\paragraph{\textbf{Classroom Integration and Constraints}} Teachers also explored the practical considerations for implementing their prototypes in their specific school and classroom contexts. 
They emphasized needing \textbf{reliable infrastructure}, including strong internet and compatible devices. 
\textbf{High-quality PD and ongoing support} were deemed essential for ethical and practical tool use. A continuous feedback mechanism for user experience and feature requests was also highlighted, along with clear expectations and accountability for its implementation, potentially overseen by tech committees or school boards.

However, teachers anticipated several challenges accompanying this integration, including \textbf{obtaining approval and managing costs}, navigating budget constraints, and ensuring affordability and accessibility. There were concerns about \textbf{resistance to change from educators accustomed to traditional teaching methods}, especially if the tool changes established workflows. \textbf{Time constraints} were also noted, as teachers already face significant demands and may struggle to find time for training. \textbf{Communicating effectively with parents} and addressing their concerns about LLM tools was deemed crucial, with an emphasis on building trust and educating them on benefits and potential harms.



\paragraph{\textbf{Ethical Impact of GenAI Prototypes: Benefits and Harms}}
Participants were enthusiastic about the tool's potential to support teachers by streamlining the creation of projects and rubrics based on educational standards, goals, and \textbf{differentiated practices}, thereby improving lesson planning flexibility and fostering equity in the classroom. Additionally, the tool could enhance student learning by making lessons more interactive, personalized, and connected to real-world problems, potentially \textbf{increasing motivation} and \textbf{participation}. Participants, however, expressed concerns about the tool potentially reducing teacher and student autonomy through \textbf{over-reliance} on LLM-generated recommendations, which could undermine human judgment and diminish valuable teacher-student interactions crucial for trust and understanding. They also worried about the tool \textbf{disrupting existing workflows} during its initial integration and \textbf{exacerbating digital inequalities} due to varying student access to reliable internet.

% \textbf{Long-Term Benefits and Harms in the Project-Based Learning (PBL) Community}
Participants saw the tool as a way to make \textbf{PBL more approachable} for teachers, encouraging continued use and innovation by simplifying the process. They also noted its potential to maintain consistency across classes and \textbf{align educational materials with state standards}, potentially enhancing outcomes. On the other hand, they were concerned that over-reliance on LLMs could stifle creativity, resulting in generic responses and reducing originality in classroom activities. They also raised issues about potential \textbf{copyright violations}, loss of creative ownership in education, and the invasiveness of the technology, especially in material sourcing.
% \textbf{Ensuring Transparency in the Tool’s Use} 
Participants emphasized the need for transparency in LLM tool use, including clear communication with parents, administrators, and other stakeholders. They stressed understanding the LLM's data sources and algorithms to avoid issues surrounding its \textit{\textbf{"black box"}} nature and suggested incorporating warnings into interfaces to \textbf{verify LLM-generated content} before usage. 
% To mitigate potential harms, comprehensive teacher training was also recommended to use AI tools responsibly, ensuring they support rather than replace human judgment and creativity.

% Successful implementation of the proposed prototypes thus required careful planning around necessary resources, overcoming challenges, and ensuring transparency. Participants emphasized these factors as crucial for maximizing the tool's benefits while addressing ethical, logistical, and privacy concerns.

% \subsubsection{[Study 2] Perspectives from Novice PBL Teachers} Study 2 aimed to gather perspectives from teachers who do not regularly use project-based learning (PBL) to ensure the adaptability of GenAI tool wireframes across diverse teaching methods and environments. We present participants' responses the AI perception polls, and discussions on the evolving role of teachers. Then, in section XXX we combine feedback on the PBL AI tool wireframes from both studies 1 and 2.


% We kickstarted our workshop with a few polls to gauge participants’ perceptions of AI and
% collaborative learning. 

% \begin{figure}
%     \centering
%     \includegraphics[width=0.7\linewidth]{menti-ai-cloud.png}
%     \caption{Word cloud showing study 2 participants' feelings about AI}
%     \label{fig:ai-cloud}
% \end{figure}

% % thoughts on omitting the paragraph section below?
% \paragraph{\textbf{Polls on AI Perceptions and Collaborative Learning}}
% The word cloud in Figure \ref{fig
% } shows a mix of participant' positive and cautious sentiments toward AI, with terms like "interesting," "exciting," and "powerful" indicating optimism, and "unknown," "limitations," and "dangerous" reflecting concerns about risks and uncertainties. Participants' responses on using AI for curriculum planning and grading varied. Many found AI useful for simplifying text, creating content, and automating tasks, though some raised concerns about its accuracy and the need for refinement. While AI-generated rubrics were seen as helpful starting points, they often required adjustments. Several teachers had not used AI at all, citing lack of interest, uncertainty, or the need for more tailored prompts. These insights suggest a need for more support and guidance in effectively applying AI in teaching.

% \paragraph{\textbf{Discussions on the Evolving Role of Teachers}}
% Following the group discussions with polls, we split participants up into three groups based on their pedagogical areas of interest (curriculum, assessments, or progress tracking). In these smaller groups, teachers reflected deeply on how AI tools intersect with their identities and roles as educators. While AI could streamline tasks like idea generation and drafting, teachers emphasized that the core of teaching—problem-solving, student assessment, and building relationships with students—must remain human-driven. They saw themselves as facilitators guiding students in using AI effectively, ensuring students maintain their voice and critical thinking, and remain active participants in their learning process rather than passive con-
% sumers of AI-generated content. 

\begin{figure*}
    \centering
    \includegraphics[width=1\linewidth]{samples/Lessons.pdf}
    \caption{PBL LLM tool supports for lesson planning}
    \label{fig:lessons-wireframes}
    \Description{The figure shows wireframes for planning and organizing a project over multiple weeks. The "Project 1 Dashboard" provides an overview and weekly previews where teachers can plan lessons, assessments, scaffolding, and deliverables. Each week includes options to edit lesson plans, create rubrics, and submit details to an LLM tool (CAIL). Scaffolded LLM-powered lesson planning templates allow generating exit tickets, brainstorming activities, and creating project milestones with personalized documents, resources, and checklists to support learning objectives.}
\end{figure*}

\subsection{PBL GenAI Tool Wireframe Prototypes and Testing {(all Category themes)}}
From these discussions, we created wireframes for a teacher dashboard spanning the three categories of support (curriculum, assessments, and progress tracking). These wireframes explictly link their features to the core PBL challenges identified (from category B themes), and translate the identified needs (from category A themes) into actionable LLM-PBL solutions (in category C themes). \textbf{Below, we present these wireframes and the combined feedback received from teachers in both Study 1 (section 3.1.1) and Study 2 (section 3.2.1). 
By gathering perspectives from teachers with varying levels of PBL {expertise}, we wanted to ensure adaptability of LLM tool wireframes across diverse teaching methods and environments (addressing RO2).}
For brevity, we use the term CAIL (Collaborative AI for Learning) for the system and will refer to it as such throughout this section. 

\subsubsection{\textbf{Curriculum Supports}}
We focused on two key needs: brainstorming new project ideas with CAIL and co-implementing weekly or unit lesson plans based on these finalized ideas.


% Our key design focus here was to build on established PBL practices, reducing the need for teachers to create resources from scratch. 
% Figure \ref{fig:project-wireframe} shows our wireframes for ideating and developing classroom projects in collaboration with CAIL. 
The project brainstorming wireframes (Figure \ref{fig:project-wireframe}) allow educators to expand on initial ideas gathered from classroom experiences, augmenting teacher creativity from the outset. Teachers input their learning goals and standards, ensuring \textbf{alignment with curricular objectives}. The tool integrates a curated list of U.S. national and state standards and allows teachers to \textbf{define expected project outputs} (e.g., slide decks, podcasts, portfolios), group size, project duration, and upload preferred lesson planning templates. CAIL offers multiple LLM-generated ideas, enabling teachers to refine and expand on them and provide students with \textbf{varied project options}.

\begin{figure*}
    \centering
    \includegraphics[width=1\linewidth]{samples/Rubrics.pdf}
    \caption{PBL LLM tool supports for rubric creation and differentiation}
    \label{fig:rubrics-wireframes}
    \Description{The figure shows wireframes for creating and differentiating rubrics for student assessments. The "Create Rubrics" screen allows teachers to select knowledge checks or project milestones, choose rubric types, and pre-populate competencies. The "Differentiation for Rubrics" section provides options to modify rubrics for different student needs, including those with IEP/504 requirements, and personalize rubrics for each group. The "All Rubric Versions at a Glance" screen lets teachers view, edit, and approve all rubric versions created, ensuring they meet diverse learning needs.}
\end{figure*}

% TODO bold key words corresponding to bullet points from thesis
% Teachers generally appreciated the project brainstorming tool for its potential to streamline project development but suggested several enhancements to make it more adaptable and intuitive. 
Our participants suggested adding adjustable project durations, with the ability to \textbf{accommodate different class schedules}. Live collaboration features were also requested, enabling interdisciplinary teamwork among educators for formulating ideas. Teachers expressed interest in uploading custom standards and explored the idea of the LLM crosswalking standards to identify and address gaps. Combining multiple LLM-generated project ideas into a \textbf{choice board} for students was highlighted as a potential feature that could enhance differentiation. 

Teachers can then use CAIL for lesson planning (Figure \ref{fig:lessons-wireframes}) to implement selected project ideas. This includes creating formative assessments ("knowledge checks"), exit tickets, hooks, and scaffolded activities like games, discussions, peer learning, and problem-solving tasks. 
CAIL offers teachers a \textbf{structured lesson planning template} aligned with the standards they selected for their project ideas. 
For this wireframe, we picked an \href{https://www.nextgenscience.org/}{NGSS (Next Generation Science Standards)} example template that guides teachers through performance expectations, learning outcomes, and reflection elements. 
It provides targeted lesson plan suggestions, reducing planning time and offering guidance for both new and experienced PBL teachers. It maintains \textbf{institutional memory} by gathering data to inform future practices. Additionally, CAIL assists teachers in \textbf{creating milestone documents} and tailoring resources, such as checklists and schedules, to guide student groups through project phases.
% Teachers emphasized the importance of understanding the tool's long-term time-saving benefits, placing value on including in-template AI features for efficiently generating engaging content aligned with PBL principles.
Participants recognized the tool's long-term time-saving potential:

\begin{quote}
\textit{ “These are the 'money' tools for me. These buttons that help with generating these smaller activities that support the bigger project would be a huge time and brain saver.” --P01}
\end{quote}

% They also spoke about the tool’s potential in addressing professional development needs:
% \begin{quote}
% \textit{“Professional development is such an important part of this, so many people miss that when they launch a new initiative...this tool has already got that in mind.”}
% \end{quote}

However, they stressed the need for \textbf{thorough PD and microlearning modules} during rollout to help educators fully understand the tool's benefits, suggesting tutorials no longer than 20 minutes to avoid overwhelming them. 
Participants highlighted the importance of clear learning outcomes for each activity, aligning with competency-based teaching, and appreciated the iterative refinement of lesson plans. They also suggested that LLMs estimate activity time requirements while acknowledging their current limitations in precise time predictions.
% Participants emphasized the importance of including clear learning outcomes for each activity to align with competency-based teaching. Teachers appreciated the iterative process of refining lesson plans and wanted the AI to estimate time requirements for various activities, acknowledging the current limitations in precise time predictions.

\begin{figure}[ht]
    \centering
    \includegraphics[width=1\linewidth]{samples/Grading.pdf}
    \caption{PBL LLM tool supports for grading}
    \label{fig:grading-wireframes}
    \Description{The figure shows wireframes for managing and grading formative assessments and project milestones. It displays submission statuses for assignments and exit tickets, with options to view responses, summarize feedback, remind or follow up with students, and autograde submissions. Teachers can preview rubrics, upload scored examples, provide feedback, and view final scores, with options to send live scoring updates.}
\end{figure}

\begin{figure*}[ht]
    \centering
    \includegraphics[width=1\linewidth]{samples/Progress_tracking.pdf}
    \caption{PBL LLM tool supports for progress tracking}
    \label{fig:progress-wireframes}
    \Description{The figure shows wireframes for tracking student progress at both class and group levels for Week 1. The "Progress Tracking: Class Level" screen provides summaries of exit ticket feedback, knowledge check performance, and group project tracking, highlighting areas where students excel or need improvement. The "Progress Tracking: Group 1" screen offers a detailed view of individual group performance, including contributions and areas needing support. There are options to generate new activities based on identified needs, follow up on competencies, and adjust lesson plans for the next week.}
\end{figure*}

\subsubsection{\textbf{Assessment Supports}} 
The assessment support section of the wireframes received mixed reactions from teachers, as discussed below. 

Teachers can use CAIL to \textbf{develop rubrics} (Figure \ref{fig:rubrics-wireframes}) for formative assignments and project milestones. They can input specific student artifacts for assessment and automatically populate rubric rows with competencies from the selected curriculum standards. Educators can further refine the criteria by editing acceptable standards for each rubric item, and \textbf{selecting a rubric type} (e.g., 1-point, 3-point, 4-point, or 5-point) to suit their classroom.
The tool also supports differentiation by generating multiple rubric versions tailored to individual student needs. Teachers can input \textbf{IEP/504 requirements} and pick competencies for specific students and groups, ensuring fair assessment by maintaining consistent standards while accommodating individual learning trajectories.

Participants suggested including examples to explain the different rubric types to help new teachers, \textbf{student work samples to set expectations}, and using visual highlights on rubrics to make feedback more accessible, noting that students often overlooked detailed feedback. 
Participants raised \textbf{privacy concerns} about using sensitive student IEP data, suggesting anonymized profiles and LMS integration for secure access. 
Participants emphasized the need for both \textbf{teacher- and student-facing rubrics}, with LLM assistance in simplifying language for students and using "I can" (\textit{e.g.,} "I can analyze data from a chart to make informed conclusions.") statements, to make these more empowering.

Figure \ref{fig:grading-wireframes} shows the CAIL grading page, which simplifies evaluating formative assessments and project milestones by tracking submissions, generating response summaries, and sending reminders for pending work. It allows teachers to upload scored examples as benchmarks, helping LLMs contextualize assessments. CAIL maps rubrics to relevant tasks, with LLM-generated scores complemented by required teacher feedback. An optional “live scoring” feature provides instant feedback to students if enabled by teachers.



Teachers were cautious about relying solely on LLMs for grading, particularly for subjective or creative work, emphasizing the \textbf{need for human oversight} to ensure feedback is personalized. Concerns were raised about biases in LLM grading, as well as its ability to handle atypical responses and physical artifacts that require nuanced understanding. Some participants like P05 (who taught PBL in online settings) did not want grading to be handled by the LLM at all:

\begin{quote}
\textit{“I understand why people want auto grading...but I think you learn so much about the individual students by interacting with their work. How else are you interacting with them? By seeing what they did and reading it. I wouldn't feel good about having AI grade it.” -- P05}
\end{quote}

Some teachers suggested using live scoring features to boost engagement and prompt revisions, \textbf{minimizing grading iterations}. However, others preferred the option to review and adjust LLM-generated comments before students received them and supported integrating custom feedback based on \textbf{classroom observations to complement LLM scores}.

\subsubsection{\textbf{Progress Tracking Supports}} 
The progress tracking section of the wireframes (Figure \ref{fig:progress-wireframes}) provides educators with an overview of student performance across individual, group, and classroom levels. The \textbf{Exit Tickets Summary} highlights topics students enjoy, areas needing help, and interests for future exploration, enabling targeted interventions; the Knowledge Check Performance Summary displays \textbf{class strengths and areas of concern} on competencies, allowing follow-up with struggling students; and the \textbf{Groups Project and Contributions} Tracking monitors group members' progress and individual contributions for effective interventions.

Participants recommended making progress data accessible to students individually and anonymously at the class level to enhance motivation and self-awareness. They emphasized \textbf{cross-disciplinary tracking for a holistic view of performance} and coordinated support across subjects. Although teachers found the progress data valuable for differentiation and scaffolding, they were concerned \textbf{about administrators using the data} \textbf{to evaluate their own performance.} They suggested creating an opt-in database to share projects and best practices, searchable using LLMs, based on student progress.

% In summary, participants emphasized the value of the proposed tool wireframes in breaking down and structuring the PBL process, highlighting its potential to better support both students and teachers compared to other tools. 


% I did not add the following section in cuz I am tempted to not have it in the paper, altogether. The survey results do not show anything new that the results above dont already cover. Thoughts?
% \subsection{[Study 1 and 2] Post Surveys}
% TBD

\section{DISCUSSION}

\subsection{Tensions in Designing PBL LLM Tools}
Our findings suggest that LLMs hold significant potential to enhance PBL practices by alleviating several teacher concerns. However, their use involves trade-offs that must be carefully managed for long-term effectiveness and sustainable use in the classroom. We present three tensions critical for the success of PBL-LLM tools: (1) technology choice and feasibility, (2) teacher agency, creativity, PD, and learning, and (3) balancing educational depth with reducing administrative workload.

\subsubsection{\textbf{Technology Choice and Feasibility:}} The selective use of LLMs is crucial to their effectiveness and sustainability in PBL. While LLMs can offer powerful capabilities in generating educational content and personalized learning materials \cite{owan2023exploring, lan2024teachers, zheng2024charting, weisz2024design}, teachers in our co-design workshops raised issues related to the accuracy and bias inherent in LLM-generated content, as seen in prior work \cite{zha2024designing, schneider2023towards, owan2023exploring}. LLMs could also %perpetuate or even exacerbate biases present in their training data, 
undermine the fairness and inclusivity of educational practices due to this bias \cite{fang2024bias, rudolph2023chatgpt, schneider2023towards}. %particularly when these tools are used for high-stakes tasks like grading creative work or interpreting student reflections \cite{fang2024bias, rudolph2023chatgpt, schneider2023towards}. 
Our teachers expressed doubts about the reliability of LLMs in handling tasks that require a deep understanding of context or the subtle nuances involved in student expressions and creative multimodal artifacts \cite{rudolph2023chatgpt, schneider2023towards}. This skepticism reflects broader concerns about the limits of GenAI's interpretive abilities and risks associated with delegating too much of the evaluative process to technology \cite{zheng2024charting, rudolph2023chatgpt}. %Teachers were worried that over-reliance on AI might lead to shallow assessments that miss the underlying cognitive and emotional dimensions of student work, potentially impacting the quality and equity of provided feedback. 

The feasibility of implementing LLMs in diverse educational settings was another significant issue. Our study highlighted concerns around logistical constraints, such as limited access to high-speed internet, adequate digital infrastructure, and computational resources, which can vary greatly across schools and districts \cite{zha2024designing}. These disparities raise concerns about digital equity, as schools with fewer resources may find themselves unable to leverage advanced LLM tools, potentially widening the gap in educational opportunities \cite{owan2023exploring, lan2024teachers}. Despite these challenges, teachers in our study were open to strategic and scoped integration of PBL-LLM tools by focusing on PBL-specific tasks best suited for LLM optimization to ensure their effectiveness in diverse contexts.

\subsubsection{\textbf{Teacher Agency, Creativity, Professional Development, and Learning: }}
Maintaining teacher agency and fostering creativity when integrating LLM tools were recurring areas of discussion in our work. 
Teachers expressed a strong preference for tools that align with their pedagogical goals, support creativity, and promote critical thinking among students \cite{lan2024teachers}. However, they also voiced concerns that excessive automation could diminish their role \cite{cope2021artificial}, particularly for novice PBL educators still learning the nuances of implementing high-quality PBL pedagogy. %This risk of reducing teachers to mere operators of AI was a significant concern, underscoring the need for tools that empower teachers as active participants and learners. 

To address these concerns, our findings presented opportunities for LLM tools to offer teachers real-time guidance during resource creation while allowing for their active input.  
Rather than replacing teachers, the LLM could act as a co-educator and brainstorming partner, supporting their ability to design and implement activities \cite{nagy2023gen}. Our wireframes suggest opportunities for integrating LLM-driven PD modules into PBL tools, helping teachers build new skills and strategies.
Drawing on perspectives of teachers in our study, this training could also focus on helping PBL educators understand the tool's long-term benefits and customization options, ensuring it integrates smoothly into existing workflows. The in-tool LLM support could serve as an additional learning resource for novice PBL teachers by providing scaffolded guidance during lesson planning, aiding them in developing the skills needed to design and implement effective PBL lessons.
Teachers in our study were particularly enthusiastic about the prospect of learning prompt engineering and other LLM-specific skills, seeing these as new ways to enhance their creativity and adapt technology to their classroom needs \cite{so2024enhancing}. 
To realize these benefits, the design of LLM tools must involve ongoing collaboration among developers, teachers, and PD designers. 

\subsubsection{\textbf{Balancing Educational Depth with Reducing Administrative Workload:}}
LLM tools have the potential to significantly reduce administrative burdens in PBL settings, such as managing group projects, tracking student progress, and providing individualized feedback \cite{chan2023ai} that  
%By automating routine tasks like grading, lesson planning, and progress monitoring, AI tools 
can free up teachers to focus on more meaningful instructional activities \cite{samala2024unveiling}. However, our results caution against allowing efficiency gains to oversimplify the complex aspects of PBL, such as developing students' interpersonal skills, cultivating creativity, and supporting critical thinking—areas that require nuanced, human-centered guidance 
\cite{lan2024teachers}. Similar to previous work, our teachers also feared that excessive automation could diminish opportunities for spontaneous, in-the-moment learning experiences and the development of soft skills that are crucial for student growth \cite{chen2024artificial, hutson2023rethinking}. There is thus a pressing need to identify which administrative tasks can be effectively automated without compromising the integrity of the learning experience. %Overemphasizing technological capabilities could undermine teacher agency or result in tools that are too complex or impractical for everyday use. Conversely, focusing solely on teacher agency without considering efficiency could lead to tools that are perceived as adding to, rather than alleviating, the teacher's workload. 
% Keeping these trade-offs in mind, we propose a set of design recommendations below to enhance the teacher’s role, support their professional growth, and optimize the AI PBL process without compromising its core educational values. 

\subsection{LLM Design Recommendations for Project-Based Learning}

Keeping the above tensions in mind, we %now leverage empirical evidence from our study to 
present design recommendations for educational technology designers and educators when integrating LLMs into PBL (addressing RO3).
We draw on the \textbf{“Gold Standard PBL: Project Based Teaching Practices” framework from the Buck Institute for Education} that guides teachers in implementing high-quality PBL \cite{PBLWorks}. 
% by blending traditional instructional practices within a project context. 
We briefly describe the framework's seven PBL teaching practices %from the Gold Standard framework 
(Figure \ref{fig:gold-pbl}) and unpack potential ways LLM systems can mitigate implementation barriers. %that arise when implementing each practice.
\textbf{We focus on recommendations that alleviate teachers' administrative workload in implementing high-quality PBL, allowing them to dedicate more time to the creative and fulfilling aspects of teaching.}
We advocate for these discussions to actively include new teachers, curriculum coaches, school boards, admins, information technology (IT) and special education committees — stakeholders our teachers emphasized as vital but often underrepresented in education tool design \cite{pnevmatikos2020stakeholders}.
We structure the two sub-sections below by mapping a combination of these practices to the wireframe supports we developed and the design recommendations they informed.

\begin{figure}
    \centering
    \includegraphics[width=1\linewidth]{samples/gold-pbl.png}
    \caption{Gold Standard Framework: Project Based Teaching Practices}
    \label{fig:gold-pbl}
    \Description{The figure illustrates the "Gold Standard PBL" framework, which outlines seven project-based teaching practices. At the center are the core learning goals: key knowledge, understanding, and success skills. The surrounding practices include: "Design & Plan," "Align to Standards," "Build the Culture," "Manage Activities," "Scaffold Student Learning," "Assess Student Learning," and "Engage & Coach." Each practice contributes to achieving the central learning goals in a project-based learning environment.}
\end{figure}

\subsubsection{\textbf{Recommendations for Project Ideation, Curriculum Planning, and Management}}- \\

\noindent\fbox{
%
    \parbox{\columnwidth}{%
\textbf{Gold Standard PBL Teaching Practice \#1: Design and Plan: }\textbf{“}\textit{Teachers create or adapt a project for their context and students, and plan its implementation from launch to culmination while allowing for some degree of student voice and choice.”}}%
}
\noindent\fbox{
%
    \parbox{\columnwidth}{%
\textbf{Gold Standard PBL Teaching Practice \#2: Align to Standards:} \textit{\textbf{“}}\textit{Teachers use standards to plan the project and make sure it addresses key knowledge and understanding from subject areas to be included.” }
}%
}
\noindent\fbox{
%
    \parbox{\columnwidth}{%
\textbf{Gold Standard PBL Teaching Practice \#3: Manage Activities:} \textit{\textbf{“}}\textit{Teachers work with students to organize tasks and schedules, set checkpoints and deadlines, find and use resources, create products and make them public.”}}%
}
\noindent\fbox{
%
    \parbox{\columnwidth}{%
\textbf{Gold Standard PBL Teaching Practice \#4: Scaffold Student Learning:}\textit{ “Teachers employ a variety of lessons, tools, and instructional strategies to support all students in reaching project goals.”}
}%
}

Based on the feedback we received on project ideation and curriculum planning supports, we outline the following design considerations for Practices 1, 2, 3, and 4: \\

\textbf{Recommendation 1: Augment Teacher Creativity with LLM Tools}

\textit{Design LLM tools to augment and elevate teacher creativity by empowering educators to lead the project brainstorming process from the outset while leveraging LLMs to support mundane, peripheral tasks. Build on established pedagogical practices of PBL educators, reducing the need for creating resources from scratch.}

Our results showed evidence of PBL teachers grappling with balancing the depth of knowledge required for specific projects with the breadth of content mandated by the curriculum. Teachers also raised concerns about the excessive contextual information required by current LLM systems and their ability to effectively prompt the LLM for specific tasks. A proposed solution is structured input: teachers specify contextual details such as specific learning goals, standards, desired final project artifacts, and duration of implementation, and LLM tools generate ideas aligned with curricular objectives and tailored to classroom needs. This structured input approach not only supports experienced educators but also provides novice PBL teachers with foundational knowledge for constructing high-quality PBL units. \\

\textbf{Recommendation 2: Provide Teacher-Directed Scaffolding with Flexibility}

\textit{Design LLM tools for PBL that intentionally prioritize and mandate core teacher inputs, while offering optional customization features.}

This recommendation is rooted in two ideas from our research:
(1) the need to promote high-quality pedagogy by ensuring that key LLM functionalities are integrated into teaching practices, while also providing flexibility for educators to choose supplementary features that support their unique classroom needs,
(2) ensuring the LLM system has sufficient contextual information from the required inputs to generate high-quality resources.
For example, requiring teachers to input or select educational standards for their project ideas should be mandatory, ensuring alignment with curriculum requirements and learning outcomes \cite{kurtz2024strategies}. In contrast, optional features can accommodate varying school contexts, as our results showed that teachers sometimes chose not to use extended features depending on their specific needs. \\

\textbf{Recommendation 3: Ensure Flexibility in Scheduling for Diverse Classroom Needs}

\textit{Design LLM tools to accommodate specific temporal structures of diverse educational environments.}

This includes accounting for various class schedules, including block scheduling (a method of dividing time into distinct blocks and assigning each block to a specific task or activity) \cite{chen2019revisiting, savery2015overview}. Since methods for chunking projects varied vastly across teachers of different grade levels and disciplines in our workshops’ data \cite{kokotsaki2016project}, features to parameterize project duration based on hours, days, or weeks are necessary to provide the flexibility teachers require when brainstorming ideas for projects.\\

\textbf{Recommendation 4: Promote Differentiation with LLM-Generated Project Ideas}

\textit{Design LLM tools to generate multiple project ideas that teachers can combine into choice boards, thereby enhancing differentiation and student agency.}

Implementing choice boards with multiple project options was a common practice among our teachers (Figure \ref{fig:choice-boards}), but proved challenging due to time constraints and difficulties with inventing novel, creative ideas each time. To address this, the tool should allow for continuous modification and regeneration of LLM-generated content, enabling teachers to refine and align projects with their educational goals and classroom needs.\\

\begin{figure*}
    \centering
    \includegraphics[width=0.9\linewidth]{samples/choice-boards.png}
    \caption{Examples of choice boards used by our teachers to promote student agency}
    \label{fig:choice-boards}
    \Description{The figure features three example choice boards for projects. The "Ancient Civilizations Project Menu" allows students to choose projects such as creating a book, writing a news script, recording a podcast, or designing a newspaper front page to showcase the legacy of ancient civilizations. The "Colonial Advertisement Project Menu" offers similar creative options like developing posters, journal entries, interviews, and plays to engage students in learning about colonial history. The "Reading Response Choice Board" provides diverse activities for students to demonstrate their understanding of a book, including creating videos, timelines, book trailers, and interactive documents using various digital tools.}
\end{figure*}


\textbf{Recommendation 5: Streamline Lesson Planning with LLM-powered Standards Templates}

\textit{Embed LLM supports within existing lesson planning templates from common educational standards to streamline the process of generating engaging instructional content.} 

This can reduce the time and cognitive load for teachers when aligning PBL activities with academic standards. Consistent with other scholars, our research revealed that teachers often struggled with the complexity of designing knowledge-check activities, driving questions, and other project components while ensuring alignment with standards \cite{savery2015overview, ertmer2006jumping}. The LLM system could provide immediate, context-sensitive suggestions and resources, tailored to specific lesson goals and standards, allowing teachers to focus on customizing their lesson plans rather than building them from scratch. This feature is especially beneficial for less experienced teachers, easing their adoption of PBL principles, while also providing experienced educators with tools to document and share their expertise.

\subsubsection{\textbf{Recommendations for Progress Tracking and Assessments}}- \\

\noindent\fbox{
%
    \parbox{\columnwidth}{%
\textbf{Gold Standard PBL Teaching Practice \#5: Build the Culture: }\textit{“Teachers explicitly and implicitly promote student independence and growth, open-ended inquiry, team spirit, and attention to quality.”}
}%
}

\noindent\fbox{
%
    \parbox{\columnwidth}{%
\textbf{Gold Standard PBL Teaching Practice \#6: Assess Student Learning: }\textit{“Teachers use formative and summative assessments of knowledge, understanding, and success skills, and include self and peer assessment of team and individual work.”}
}%
}

\noindent\fbox{
%
    \parbox{\columnwidth}{%
\textbf{Gold Standard PBL Teaching Practice \#7: Engage and Coach: }“\textit{Teachers engage students in their learning and work alongside them to identify when they need skill-building, redirection, encouragement, and celebration.”}
}%
}
Based on the feedback we received for our assessments and progress tracking supports, we outline the following design considerations for Practices 5, 6, and 7:\\

\textbf{Recommendation 6: Develop Equitable and Actionable Rubrics}

\textit{Incorporate a default rubric format within the LLM tool that promotes fair and equitable grading practices, aligning with high-quality PBL pedagogy.}

Our research data revealed the importance of having educators collaborate with LLMs to set actionable and easily understandable expectations within rubrics that meet required competencies. The LLM tool should also facilitate the creation of two separate rubric sets: one for teachers and administrators, focusing on competencies from established standards, and another for students, emphasizing clear, actionable expectations tied to project outputs.
\begin{itemize}
    \item \textbf{Recommendation 6.1: }\textbf{Implement single-point rubrics as the default rubric type within LLM tools for PBL.}
Single-point rubrics use a single column for feedback instead of multiple performance levels. Research shows that such rubrics are quicker and easier for teachers to create since they don't need to anticipate all possible ways students might \textit{“fail expectations”} \cite{gonzalez_single-point_2015}. For students, these rubrics are simpler to understand, focusing on clear target expectations. They also promote higher-quality feedback, as teachers highlight specific strengths and areas for improvement, fostering a growth mindset and iterative learning.

    \item \textbf{Recommendation 6.2: }\textbf{Augment rubric items with detailed teacher feedback.}
Rubric items must be augmented with detailed feedback from teachers. Designers of the LLM tools should explore the possibility of providing a starting point for feedback from student submissions, allowing teachers to elaborate and personalize their notes \cite{takale2024assessing, han2024teachers}.

    \item \textbf{Recommendation 6.3: }\textbf{Integrate LLM-assisted positive feedback to complement constructive criticism alongside rubrics.}
Our teachers mentioned struggling to provide granular positive feedback due to time constraints. Praise and celebration, bolstered by relevant specifics, can instill student confidence and promote balanced assessments.\\
\end{itemize}

\textbf{Recommendation 7: Ensure Privacy in LLM-Driven Differentiation}

\textit{Ensure that data processing and LLM interactions for differentiating student resources occur locally on the teacher's device or within a secure school network.}

Teachers in our study expressed significant concerns about privacy when using LLMs to differentiate rubrics for students with IEPs, particularly regarding the sharing of sensitive student information. To address these concerns, LLM tools should either operate locally or be integrated securely within existing Learning Management Systems (LMS) as an app or extension \cite{wang2024artificial, zha2024designing}. This approach ensures alignment with privacy regulations such as COPPA (Children's Online Privacy Protection Rule). Pseudonyms can also automatically replace identifying student information before being used for differentiation. \\

\begin{figure*}
    \centering
    \includegraphics[width=0.79\linewidth]{samples/padlet.png}
    \caption{Examples of Padlets used by our teachers to document student progress and self-reflections}
    \label{fig:padlet}
    \Description{The figure shows a Padlet board featuring various student posts showcasing their projects and activities completed in the makerspace. Each post includes a photo or video with a caption describing the project, such as building structures, playing with educational robots, and creating sets. There are options for the teacher to approve or reject each submission, encouraging students to share their learning experiences.}
\end{figure*}

\textbf{Recommendation 8: Integrate Project Management Scaffolds and Supports}

\begin{itemize}
    \item \textbf{Recommendation 8.1: PBL LLM tools should include student-facing project management features that complement teacher supports}. Teachers in our study found balancing open-ended problem-solving with personalized guidance challenging, especially in managing group projects. The LLM can serve as a brainstorming partner and offer scaffolding to guide students, providing support when teachers are unavailable \cite{kim2020bot, kim2020bot, kim2024engaged, chase2009teachable, tan2022systematic, shaer2024ai}. LLM conversational agents could potentially capture key discussion points and synthesize them for review, acting as self-check tools to ensure groups complete daily tasks and take appropriate next steps \cite{lan2024teachers, tanga2024exploration, cai2024advancing}. Additionally, these agents can analyze and monitor participation frequency, and assess the quality of contributions to ensure equitable participation within groups \cite{ouyang2023integration, sekeroglu2019student, tan2022systematic}.

    \item \textbf{Recommendation 8.2: Design LLM
    tools that assist teachers in providing personalized feedback during active class time.} By alerting teachers to groups needing immediate attention and tracking group progress, LLMs can help prioritize the teacher’s time and provide notes for asynchronous feedback \cite{dutta2024enhancing, kanchon2024enhancing}. This support allows teachers to efficiently manage multiple groups without overwhelming their workload, ensuring that all student groups receive necessary guidance.

    \item \textbf{Recommendation 8.3: Implement robust LLM-powered portfolio supports that enable students to submit evidence of their processes and reflections, documenting their learning journey from ideation to project completion}. 
    Prior research has underscored the importance of processes (\textit{how} students did things throughout the project, from ideation to prototyping, for example, including challenges faced and \textit{what} they did about them), and reflection (growth or trajectory over time) \cite{fields2023communicating}. 
    Our teachers used Padlets for students to submit their daily work (e.g., snapshots, videos, responses) and as a reflective tool for exit tickets (Figure \ref{fig:padlet}).
    LLMs can support the creation of individual portfolios to showcase personal skills and group portfolios to document project development \cite{lin2023portfolio}. Additionally, LLMs should incorporate self and peer reflection activities, as well as conflict resolution tools, drawn from toolkits like \href{https://makered.org/beyondrubrics/toolkit/}{“Beyond Rubrics”}, to ensure meaningful, ongoing assessment \cite{mlpblbrief, kokotsaki2016project}. By embedding these elements into portfolios, LLM can facilitate both formative and summative assessments, offering a comprehensive view of student learning.

    \item \textbf{Recommendation 8.4: Design LLM tools to support rather than replace teacher involvement in the grading process by organizing and presenting data that tracks student progress}. Our study revealed that teachers highly value their agency in the grading process as it enables them to deeply understand student thinking, provide personalized feedback, and make informed decisions. Teachers also expressed concerns about the quality of LLM-generated feedback and the subjectivity involved in grading PBL student multimodal artifacts. By organizing data on milestones, monitoring group performance, and curating exit ticket responses, LLM tools can alleviate administrative burdens and provide a structured starting point for teachers. This ensures that teachers remain at the center of the grading process, using LLMs as a means to enhance their ability to deliver nuanced and targeted feedback, ultimately leading to more effective and personalized student assessments.
\end{itemize}

\section{LIMITATIONS AND FUTURE WORK}

Our study has limitations that suggest avenues for future research. First, our focus on PBL in the context of the United States limits the generalizability of the findings; future research should explore the applicability of LLM-driven PBL tools in culturally diverse educational settings globally. {We also acknowledge the small sample size and absence of collected racial demographics information as limitations for generalizability.} Additionally, our study involved a self-selected group of teachers who were likely more interested in GenAI and PBL, potentially biasing the findings toward a more positive outlook. Future work could include longer-term studies that investigate how such tools might engage teachers who are more resistant to PBL or GenAI integration. The wireframes developed in this study are only a starting point;  building and testing a working product through iterative design cycles is crucial to evaluate the feasibility of GenAI for the proposed features. Future research should focus on co-designing student-facing supports that align with teacher tools, fostering an integrated flow of information and feedback to create a balanced ecosystem within the LLM-driven PBL experience. 

\begin{acks}
We would like to thank Robert Parks and Raechel Walker for their invaluable assistance with the workshops. We also sincerely appreciate pivotal input received from Christina Bosch during key stages of the study. Finally, this study would have been impossible without the 41 remarkable educators, all deeply passionate about project-based learning, who brought their enthusiasm
to every step of the process.
\end{acks}

% Future research should develop student-facing supports that align seamlessly with teacher-facing tools. This alignment would foster an integrated flow of information and feedback between students and teachers, creating a balanced ecosystem within the AI-driven PBL experience. Such efforts would enhance the overall effectiveness of these tools, ensuring that they serve as meaningful extensions of the teacher’s capabilities and support a cohesive, collaborative learning environment.
% notes: 
% PBL in the united states context only-- pull paper that says we need to look at it elsewhere too 
% longer term studies on investigating how the tool could bring in more teachers who are resistant to using PBL.
% Our participants
% were a self-selected group who were most likely more interested/excited than average about the topic.
% online settings -- diff levels of interaction that in person setting workshops?
% wireframes just starting point for this work-- more iterative design is needed with working prototypes to see the feasibility of GenAI for the features mentioned
% particularly seeing how it adapts to various classrooms -- collecting observational data 




% There are many opportunities to extend this work, guided by our design recommendations. However, these recommendations come with their own set of tradeoffs and decisions that require careful consideration to ensure the effective and sustainable use of AI PBL tools in the future. We present these considerations as three key tool principles—or more accurately, tensions—that emerged during our discussions:

% \begin{itemize}
%     \item \textbf{Technology Choice and Feasibility:} The selection of technology, particularly the choice between Large Language Models (LLMs) and other AI technologies, plays a crucial role in the feasibility and functionality of the tool. While LLMs are powerful for generating content and providing personalized learning materials, they face challenges related to accuracy, bias, and computational resource requirements. They also may be less reliable for tasks requiring deep contextual understanding or nuanced interpretation, such as grading and assessments. Designers must weigh these factors against classroom needs and logistical constraints. Future research should identify PBL-specific tasks best suited for LLM optimization to ensure that tools deployed are both effective and appropriate.

%     \item \textbf{Teacher Agency, Creativity, Professional Development, and Learning:} AI tools must maintain teacher agency and creativity, allowing teachers to align tools with their pedagogical goals and foster creativity and critical thinking among students. This includes providing opportunities for teachers to learn and grow professionally through the use of these tools, ensuring that they are not simply passive users but active participants (and learners) in the pedagogical process. Too much automation could diminish teachers' roles, particularly for novice PBL educators still developing essential skills. Future research could explore integrating AI-driven professional development modules directly into PBL tools, offering real-time guidance and feedback as teachers design and implement activities, thus enhancing their skills while allowing for personalized input.

%     \item \textbf{Optimizing for Efficiency and Reducing Administrative Workload:} AI tools have the potential to significantly reduce the administrative burden on teachers by automating tasks like grading, lesson planning, and progress tracking. This can free up teachers to focus on more meaningful and joyous instructional activities.
%     However, this efficiency should not oversimplify the complex aspects of PBL, such as development of interpersonal skills and the cultivation of student creativity. Future studies should investigate which routine tasks can be effectively automated without compromising educational quality.  
% \end{itemize}

% Designing AI tools for PBL requires balancing potential trade-offs: overemphasizing technological capability may compromise teacher agency, while prioritizing efficiency could diminish the PBL experience or alienate teachers. Conversely, focusing solely on teacher agency without efficiency might result in tools that add to, rather than ease, the workload. The goal is to create tools that enhance the teacher’s role, support professional growth, and optimize PBL without sacrificing core educational values. Achieving this balance ensures AI tools extend teacher capabilities rather than replace them, fostering a thriving learning environment. Further research is needed to refine these dynamics early in the design process to avoid unintended consequences. Future research should focus on developing student-facing supports that align with the teacher-facing tools proposed in this paper, fostering an integrated flow of information and feedback between students and teachers. This approach aims to create a balanced ecosystem within the tool, enhancing the effectiveness of the AI-driven PBL experience through cohesive and collaborative learning.







% %% the bibliography file.
\bibliographystyle{ACM-Reference-Format}
\bibliography{sample-base}


% %%
% %% If your work has an appendix, this is the place to put it.
% \appendix

% \section{Research Methods}

% \subsection{Part One}

% Lorem ipsum dolor sit amet, consectetur adipiscing elit. Morbi
% malesuada, quam in pulvinar varius, metus nunc fermentum urna, id
% sollicitudin purus odio sit amet enim. Aliquam ullamcorper eu ipsum
% vel mollis. Curabitur quis dictum nisl. Phasellus vel semper risus, et
% lacinia dolor. Integer ultricies commodo sem nec semper.

% \subsection{Part Two}

% Etiam commodo feugiat nisl pulvinar pellentesque. Etiam auctor sodales
% ligula, non varius nibh pulvinar semper. Suspendisse nec lectus non
% ipsum convallis congue hendrerit vitae sapien. Donec at laoreet
% eros. Vivamus non purus placerat, scelerisque diam eu, cursus
% ante. Etiam aliquam tortor auctor efficitur mattis.

% \section{Online Resources}

% Nam id fermentum dui. Suspendisse sagittis tortor a nulla mollis, in
% pulvinar ex pretium. Sed interdum orci quis metus euismod, et sagittis
% enim maximus. Vestibulum gravida massa ut felis suscipit
% congue. Quisque mattis elit a risus ultrices commodo venenatis eget
% dui. Etiam sagittis eleifend elementum.

% Nam interdum magna at lectus dignissim, ac dignissim lorem
% rhoncus. Maecenas eu arcu ac neque placerat aliquam. Nunc pulvinar
% massa et mattis lacinia.

\end{document}
\endinput
%%
%% End of file `sample-manuscript.tex'.
