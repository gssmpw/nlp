\section{RELATED WORK}
% Our research draws on literature in three key areas. First, we build on the extensive background of Project-Based Learning (PBL), exploring both the benefits it offers and the challenges it presents for teachers. Second, we engage with the expanding body of research on AI-based tools in education, particularly focusing on large language model (LLM)-based tools that bolster and enhance the pedagogical practices commonly employed in PBL. Finally, we build on prior research in the co-design of educational tools, emphasizing the collaborative development process to create AI solutions that meet the unique needs of educators. 

\subsection{Project-Based Learning}


Rooted in constructivist learning theories, project-based learning (PBL) places students at the center of the learning process, challenging conventional education practices and redefining the roles of teaching, learning, and school organization \cite{fitzgerald2020overlapping, guo2023effects, kokotsaki2016project, woods2024}.  
Unlike problem-based learning, which promotes promotes deductive reasoning \cite{condliffe2017project, thomas1999project}, and involves structured evaluation based on how well a student’s solution addresses a defined problem (e.g., designing a more effective trash can to improve adoption or conducting participatory action research on a social issue), project-based learning engages students in complex, real-world tasks to create a broader range of artifacts (e.g. reports, models, and presentations) that demonstrate learning \cite{krajcik2006project, oguz2014comparison, thomas2010review, chen2019revisiting, barron2008teaching, thomas1999project} (e.g., using photojournalism to explore local flora and fauna or writing a comic book to analyze tropes in the hero’s journey). 
This fosters creativity and critical thinking through open-ended driving questions (e.g. “what is the proper role of government in a democracy?”) \cite{parker2011rethinking, blumenfeld1991motivating, diehl1999project, svihla2020facilitating}.
%These foster creativity and critical thinking through open-ended driving questions, leading to multiple learning pathways \cite{blumenfeld1991motivating, diehl1999project, svihla2020facilitating}.
% Project-Based Learning (PBL) has established itself as a powerful and transformative approach to education, offering an alternative to traditional instructional methods by placing students at the center of the learning process. Since its inception, PBL has provided a pedagogical framework that reimagines the roles of teaching, learning, and school organization, challenging conventional educational practices \cite{fitzgerald2020overlapping, guo2023effects, kokotsaki2016project, woods2024}.
% \textbf{Core Concepts and Practices in PBL: }PBL is deeply rooted in constructivist learning theories, which suggest that students achieve a deeper understanding of content when they actively "construct and reconstruct" knowledge through direct experience and interaction with the world \cite{krajcik2006project, oguz2014comparison, thomas2010review, chen2019revisiting}.  At its core, PBL involves students engaging in complex, real-world tasks over an extended period, resulting in the creation of artifacts such as reports, models, and presentations \cite{barron2008teaching, thomas1999project}. These artifacts serve as tangible representations of students' learning and are often shared with an audience, further enhancing the relevance and authenticity of the learning experience \cite{blumenfeld1991motivating}.
% The structure of PBL is typically characterized by several key steps: students and/or teachers identify an open-ended question or problem to address, students conduct research to explore potential solutions, they create artifacts that encapsulate their new ideas, and finally, they present these artifacts to an audience \cite{bell2010project, kokotsaki2016project} . Crucially, the driving questions in PBL are often situated within ill-defined problem spaces, allowing for multiple solutions and diverse learning pathways \cite{diehl1999project, svihla2020facilitating}. This approach not only fosters creativity and critical thinking but also places control of the learning process in the hands of students, encouraging them to take ownership of their projects.
Research has consistently demonstrated the numerous benefits of PBL across various educational contexts. Unlike traditional instruction, PBL (1) encourages students to delve deeply into the subject matter, fostering a more profound understanding of the material \cite{boaler1998open, panasan2010learning, schneider2002performance, chen2019revisiting}; (2) enhances mastery of subjects and supports the development of critical 21st-century skills like collaboration and communication \cite{condliffe2017project, noguera2015equal, peterson2012uncovering, chen2019revisiting}; (3) boosts motivation, and engagement \cite{hernandez2009learning,kaldi2011project, blumenfeld1991motivating, holm2011project, bender2012project, intel2007designing}; %by making learning relevant to students' lives and encouraging responsibility for their learning \cite{hernandez2009learning,kaldi2011project, blumenfeld1991motivating, holm2011project, bender2012project, intel2007designing}. It 
and (4) develops metacognitive skills by helping students assess their progress and continuously improve \cite{thomas1998project, thomas1999project, english2013supporting, bender2012project, tseng2013attitudes}. %preparing them for future challenges \cite{thomas1998project, thomas1999project, english2013supporting, bender2012project, tseng2013attitudes}.

% PBL has been shown to enhance students' mastery of subject matter by providing opportunities for active, hands-on learning \cite{chen2019revisiting}. This approach not only helps students retain knowledge but also enables them to apply what they have learned to real-world situations. Furthermore, PBL supports the development of essential 21st-century skills, such as collaboration, communication, and critical thinking
% \cite{condliffe2017project, noguera2015equal, peterson2012uncovering}. As students work together to solve complex problems, they learn how to collaborate effectively with others, share ideas, and provide constructive feedback.

% In addition to cognitive and interpersonal benefits, PBL is associated with positive intrapersonal outcomes, such as increased motivation and a stronger academic mindset \cite{hernandez2009learning,kaldi2011project}. The relevance of PBL projects to students' lives often sparks greater interest in the content, leading to higher levels of engagement and persistence \cite{blumenfeld1991motivating, holm2011project}. Students are also more likely to pursue their interests and take responsibility for their learning when they see the direct application of their work to real-world challenges \cite{bender2012project, intel2007designing}. 

% PBL has also been linked to the development of metacognitive skills, such as self-regulation and self-monitoring, which are crucial for lifelong learning \cite{thomas1998project, thomas1999project, english2013supporting} Through the iterative process of project work, students learn to assess their progress, identify areas for improvement, and adapt their strategies accordingly. This emphasis on self-assessment and continuous improvement not only enhances students' academic performance but also fosters a growth mindset, preparing them for future challenges \cite{bender2012project, tseng2013attitudes}.

While PBL is inherently student-centered, the success of this approach is heavily influenced by the role of teachers as facilitators, guiding and scaffolding student learning
\cite{barron2008teaching}. This shift in roles and professional identity requires teachers to adapt their pedagogical approaches and develop new skills, often leading to greater ownership, self-efficacy, and confidence in their teaching \cite{choi2019does, havice2018evaluating, potvin2021consequential}. However, transitioning to PBL often requires targeted professional development (PD) that provides teachers with peer collaboration, reflection, and ongoing support during PBL implementation \cite{Aitken2019, dunbar2022shifting, blumenfeld1991motivating, quint2018project, park2018equip}. %Educative features within PBL materials, such as guidance on adapting content for diverse learners, help teachers tailor projects to student needs \cite{balldeveloping, davis2005designing}.
Given the pivotal role of teachers in harnessing PBL benefits, our research reveals their needs and explores how LLMs could support them in successfully implementing PBL and foster continued professional growth. 

% This shift from traditional content delivery to a more facilitative role requires teachers to adapt their pedagogical approaches and develop new skills.
% As teachers become more experienced with PBL, they often develop a greater sense of ownership over their practice and classroom curricula \cite{potvin2021consequential}. This shift in professional identity is accompanied by increased self-efficacy, as teachers gain confidence in their ability to support student learning through PBL \cite{choi2019does, havice2018evaluating}. 
% However, transitioning to PBL-based instruction often requires targeted professional development (PD) experiences that specifically address the unique demands of this pedagogy
% \cite{Aitken2019, dunbar2022shifting, blumenfeld1991motivating, quint2018project, park2018equip}. Effective PD for PBL should include opportunities for teachers to collaborate with peers, reflect on their practice, and receive ongoing support as they implement PBL in their classrooms. 
% Educative features within PBL materials, such as guidance on adapting content for diverse learners, can also help teachers make informed decisions about how to tailor projects to meet the needs of their students \cite{balldeveloping, davis2005designing}.

To maximize the effectiveness of these tools, it is essential to consider challenges teachers encounter during PBL implementation, especially around generating authentic driving questions, 
managing time and group work, and balancing instructor-led guidance with student-directed learning \cite{zheng2024charting, thomas2010review, mergendoller2005managing}. 
%Classroom management can be difficult, in maintaining student engagement during self-directed learning and group work \cite{thomas2010review, mergendoller2005managing}. 
%^Additionally, 
Students’ discomfort with the cognitive and social demands of PBL can lead to frustration, particularly among high-achieving students accustomed to traditional instruction \cite{condliffe2017project}. In addition, teachers face challenges initiating inquiry, facilitating dialogue, and scaffolding learning \cite{reiser2006making, kali2008technology, krajcik2014promises, quintana2018scaffolding, reiser2018scaffolding}. 
Assessing student learning in PBL is also notably difficult, as traditional tests often fail to capture the depth of understanding that PBL aims to develop.  %Aligning assessments with PBL's deeper learning outcomes can be a challenge, as standardized tests are inadequate, and 
Performance-based assessments are hard to implement reliably \cite{hertzog2007transporting, mergendoller2005managing, darling2010beyond, aslan2015examining}, student artifacts can be difficult to score consistently, and teachers lack the time to provide personalized student feedback \cite{hattie2011instruction}. %In particular, teachers grapple with consistently scoring student artifacts and lack of time for personalized student feedback \cite{hattie2011instruction}.
Integrating technology is also challenging due to broader institutional factors such as limited resources, district mandates, and lack of school tech maintenance and support \cite{zheng2024charting}. Our study unpacks these PBL challenges in the literature through the nuanced experiences of teachers in interdisciplinary settings and explores how LLMs impact their roles and instructional practices. 

% Although student artifacts, such as reports and presentations, are commonly used, they are critiqued for neglecting the learning process. To address this, educators often use in-class presentations, learning journals, portfolios, and self-reflection to provide a more comprehensive evaluation \cite{zheng2024charting}.

% Classroom management is another significant concern, especially in maintaining student engagement during self-regulated learning and group work \cite{thomas2010review}. Issues like student misbehavior, lack of motivation, and intragroup conflicts are commonly reported, although experienced PBL teachers may encounter fewer of these problems \cite{mergendoller2005managing}.
% Establishing and maintaining classroom norms for effective group work is essential but requires further research to identify best practices \cite{darling2008creating}. Additionally, students’ discomfort with the cognitive and social demands of PBL can lead to frustration, particularly among high-achieving students accustomed to traditional instruction \cite{condliffe2017project}.

% Teachers face unique pedagogical challenges in PBL, including initiating student inquiry, facilitating dialogue, and providing the time and resources for in-depth investigations \cite{zheng2024charting}. Scaffolding student learning is crucial, yet teachers need more guidance on how to effectively implement and gradually reduce these supports \cite{reiser2006making, kali2008technology, krajcik2014promises, quintana2018scaffolding, reiser2018scaffolding}. 
% Promoting rigor in PBL also requires careful planning and collaborative project reviews to ensure that students engage in deep learning \cite{darling2008creating}.

% Technological integration poses additional challenges. Effective PBL requires access to quality technological resources and support for teachers in using technology, yet many schools lack these resources \cite{zheng2024charting}. 

% While technology can enhance learning, particularly for students with special needs, its impact on English Language Learners (ELLs) and students needing differentiation remains underexplored and warrants further investigation \cite{zheng2024charting}.
% Assessments in PBL pose another significant challenge, especially when aligning them with the deeper learning outcomes that PBL promotes. Standardized tests often do not capture these outcomes, making performance-based assessments more suitable but difficult to implement reliably \cite{hertzog2007transporting, mergendoller2005managing, darling2010beyond, aslan2015examining}. 
% While student artifacts are valuable assessment tools, consistent scoring remains a challenge, highlighting the need for more reliable strategies \cite{grant2005project, krajcik2014promises}. Additionally, teachers often lack the time or resources to provide the quality feedback necessary for guiding student learning \cite{hattie2011instruction}. 
% Rubrics linked to curricular units and descriptions of quality student work may help address this issue \cite{krajcik2014promises}. 
% Finally, broader institutional and contextual factors, such as teacher mobility, technology maintenance issues, and district mandates, can hinder PBL implementation. Support from school leadership and collaboration with other teachers can help mitigate these challenges \cite{zheng2024charting}.
% In our work, we unpack the PBL challenges highlighted in the literature by drawing on the specific and nuanced experiences of teachers operating within unique and often interdisciplinary environments. With the advent of GenAI, our work also investigates how these tools influence their roles, identities, and instructional practices. 
% Finally, we believe engaging teachers in conversations about their challenges is an essential part of the co-design process itself. It empowers them to voice their needs, fostering a sense of ownership and investment in the tools we develop. 
% We now engage with the expanding body of research on AI-based tools in education that bolster and enhance the pedagogical practices commonly employed in PBL as well.

\subsection{AI Tools to Support PBL Pedagogical Needs}

%We also contribute to an expanding body of AI in education research.
AI's capacity to personalize learning is beneficial in PBL settings where differentiated instruction is crucial. AI tools can support targeted instruction by tailoring lessons for diverse learners, encouraging student iteration on artifacts, and adapting materials to individual student strengths and weaknesses \cite{asrifan2024integrating, kong2024developing, alam2023intelligence}. 
Helping students manage time and materials efficiently, tools like Trello and Cronofy, integrated with AI plug-ins, can predict resource needs  \cite{tanga2024exploration}. 
AI-assisted design tools like Autodesk Dreamcatcher and intelligent project management software such as Asana and Monday.com can facilitate student project management, allowing them to focus on creative and critical thinking \cite{tanga2024exploration}.

%AI has shown potential in improving the quality of collaboration for project completion and knowledge construction \cite{tanga2024exploration, dutta2024enhancing, asrifan2024integrating}. 
Considering project collaboration and knowledge construction\cite{tanga2024exploration, dutta2024enhancing, asrifan2024integrating}, research scholars have discussed how AI can evaluate group performance by analyzing interaction patterns and predicting outcomes based on academic and behavioral data, enabling more effective grouping strategies \cite{cen2016quantitative, schneider2014toward}. AI systems can also guide students toward productive problem-solving, suggesting activities on collaborative styles, and pedagogical interventions to improve group work \cite{adamson2014towards, dyke2013enhancing, kumar2010architecture}. 
% Tan, Lee, and Lee (2022) discuss how AI techniques can evaluate group performance by analyzing metrics such as tone, relevance, and interaction patterns \cite{tan2022systematic}. This analysis helps teachers and students understand how group dynamics influence learning outcomes and how they can be improved. 
% AI can also predict group performance based on various factors, including students’ academic performance, interests, and behavioral data, allowing for more effective grouping strategies \cite{cen2016quantitative, schneider2014toward}. 
% AI can also provide real-time feedback to improve interaction patterns during collaborative activities. For instance, AI agents can guide students towards more productive problem-solving behaviors, recommend specific activities based on collaborative styles, and even suggest pedagogical interventions to tutors for improving group work \cite{adamson2014towards, dyke2013enhancing, kumar2010architecture}. 
% These capabilities are particularly valuable in PBL settings, where collaboration and teamwork are essential for project completion and knowledge construction.
Considering grading in PBL, AI-powered assessment systems, such as Automated Essay Scoring (AES) and Automated Written Corrective Feedback (AWCF), can offer real-time, continuous feedback, helping students to refine their work iteratively \cite{rudolph2023chatgpt, cope2021artificial}. These systems not only reduce the teacher's workload but also enhance the accuracy and efficiency of grading, allowing teachers to focus on more meaningful interactions with students \cite{owan2023exploring}. AI can also support formative assessment practices in PBL by providing insights into students' progress, helping teachers guide and support students more effectively \cite{lan2024teachers}.

Despite potential benefits, integrating LLMs into PBL presents challenges, such as biased algorithms, that necessitates careful design and human oversight to ensure fairness and accuracy in assessment and feedback \cite{schneider2023towards}. Comprehensive PD is needed for teachers to use LLM tools responsibly and equitably \cite{lan2024teachers, askarbekuly2024llm}, understand
%. Recent work has also underscored the need for future research exploring 
their long-term effects across age groups and subject areas, and address ethical concerns like data privacy \cite{asrifan2024integrating, wang2024artificial, zha2024designing}. In our current work, we aim to overcome some of these challenges by co-designing GenAI-powered support systems with \textit{teachers} in PBL settings. {
This is unlike recent studies that have focused on co-designing these tools with \textit{students} in the higher education contexts \cite{zheng2024selfgauge, zheng2024charting, nikolicsupporting, gustafson2025enhancing}. We also examine challenges more unique to K-12 PBL, that emphasizes scaffolding and skill-building \cite{condliffe2017project}.} 

\subsection{Co-designing Educational Tools with Teachers}

Co-design is a collaborative, iterative process where teachers, researchers, and developers jointly design, prototype, and evaluate educational tools %each with clearly defined roles
\cite{roschelle2006co}. This approach leverages stakeholders' expertise to address concrete educational needs, making it well-suited for creating technology-enhanced learning environments that are contextually relevant and practical for real-world classrooms \cite{cober2015teachers, brown1992design, lingnau2007empowering}. Studies have documented the co-design of mobile science applications, scripted wiki environments, and other educational technologies that support authentic scientific inquiry and other pedagogical practices \cite{spikol2009integrating, zhang2010deconstructing, peters2009co, cober2015teachers}.

Involving teachers in the co-design process enhances development efficiency, increases teacher agency and ownership, and promotes higher adoption rates and sustained use of educational tools beyond the initial study \cite{penuel2007designing, bakah2012updating, wan2021exploratory, handelzalts2019collaborative, mckenney2016collaborative, lin2021engaging}. %Participation fosters ownership, leading to higher adoption rates and integration into teaching practices \cite{lin2021engaging}. 
Co-design can also serve as a form of professional development, enabling teachers to deepen their understanding of new technologies and explore ways to incorporate them into instructional strategies \cite{wan2021exploratory, bakah2012updating, voogt2015collaborative}. Because co-design fosters a more reciprocal and participatory approach {\cite{alfredo2024human}}, it has the potential to upend the traditional, unidirectional process of educational technology  \cite{teeters2016challenge} and enables teachers to engage with complex topics beyond their expertise \cite{disalvo2017participatory}. %contributing to tool development while learning from the process \cite{disalvo2017participatory}. 
Furthermore, co-design has the potential to influence the broader sociocultural context of schools and drive systemic change in educational technology design and implementation by reshaping power dynamics and promoting an ethics of care within educational research and design \cite{wake2013developing, higgins2019power, matuk2021students}. {Scholars have however noted that involving teachers in the human centered design process happens less often in K-12 settings compared to higher education \cite{topali2024designing}.}


%This can drive systemic changes in how educational technologies are developed, implemented, and sustained within schools.
% In addition to its impact on teacher learning, past research has considered how teachers participate in the design of technology-enhanced learning environments \cite{cober2015teachers}. For example, studies have documented the co-design of mobile science applications, scripted wiki environments, and other educational technologies that support authentic scientific inquiry and other pedagogical practices \cite{spikol2009integrating, zhang2010deconstructing, peters2009co}. These efforts underscore the importance of involving teachers in the design process to ensure that the resulting tools are functional and aligned with their students' educational goals and needs.

% \textbf{To the best of our knowledge, no existing studies have documented a co-design process with K-12 teachers aimed at integrating GenAI into project-based learning pedagogy. Given the unprecedented and rapidly advancing nature of GenAI, along with its potential to address the unique challenges of PBL and the ethical considerations it entails, our research is particularly timely. Our findings and empirical evidence also provide valuable insights into shaping the future use of GenAI in PBL, }\textit{\textbf{guiding}}\textbf{ and }\textit{\textbf{encouraging}}\textbf{ its responsible and effective implementation.}