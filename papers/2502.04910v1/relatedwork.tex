\section{Related Work}
The effect of recency and popularity patterns has been extensively studied in the recommender system literature where it is typically attributed to selection, exposure, presentation, and other biases in interaction data~\citep{chen2023bias,wang2023survey,klimashevskaia2024survey}. Prior research on temporal patterns in dynamic graph datasets has focused on three main directions.

\textbf{Summary metrics.} A range of metrics has been developed to characterize the presence of various temporal patterns. For instance, \citet{poursafaei2022towards} characterized \textit{novelty} (new edges per timestamp), \textit{reoccurrence} (fraction of transductive edges), and \textit{surprise} (test-only edges), demonstrating the challenge of predicting entirely new connections. Similarly, \citet{daniluk2023temporal} proposed statistical distance-based measures to capture both short- and long-term global popularity dynamics, exposing weaknesses in existing evaluation protocols and negative sampling strategies.

\textbf{Tools for interpretation and visualization.} Complementing these metrics are tools designed to make temporal patterns more interpretable. \citet{poursafaei2022towards} introduced Temporal Edge Appearance (TEA) and Temporal Edge Traffic (TET) plots, which reveal when memorization-based approaches may fail -- particularly in sparse graphs or when reoccurrence is low and the surprise index is high. \citet{shirzadkhani2024temporal} later built on this work to provide deeper insights into data characteristics.

\textbf{Leveraging temporal heuristics for prediction.} Beyond measurement and visualization, researchers have proposed models and heuristics to exploit temporal information for prediction tasks. \citet{poursafaei2022towards} presented the EdgeBank heuristic, which achieves strong performance in transductive settings, while \citet{daniluk2023temporal} introduced PopTrack, a simple popularity-based heuristic that outperformed state-of-the-art methods on multiple benchmarks which was then used to create harder negative samples. In a related vein, \citet{poursafaei2022strong} demonstrated that combining structural, interaction-based, and temporal features can produce expressive node representations for accurate classification in both static and dynamic scenarios.% \citet{cong2023we} identified a similar pattern and introduced simple neural network components to account for different aspects of the signal.