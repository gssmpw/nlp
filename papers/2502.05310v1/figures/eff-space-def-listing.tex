\begin{figure}[t]
\begin{lncodebox}
\begin{lstlisting}[style=haskell,numbers=left,numbersep=12pt]
class Effect e where
  spaces :: (Space t, Typeable v) =>
    e p t n a -> [(SpaceName, SomeParametricSpace n)]   (@\label{line:spaces}@)
  hasPrimarySpace :: e p t n a -> Bool  (@\label{line:has-primary-space}@)
  mapEmbedded :: (forall v. t n v -> t' n v) -> e p t n a -> e p t' n a
  nodeTags :: e p t n a -> [Tag]  (@\label{line:node-tags}@)
  navigate :: (Space t, Monad m) =>  (@\label{line:navigate:start}@)
    e p t n a -> Maybe (ChoiceFun m n -> m (Tracked n a))   (@\label{line:navigate:end}@)

class Space sp where  -- Instances: `Tree s p` and `OpaqueSpace p` (@\label{line:space:start}@)  (@\label{line:space:end}@)
  source :: sp n v -> SpaceSource n v
  spaceTags :: sp n v -> [Tag]  (@\label{line:space:end}@) (@\label{line:space-tags}@)

data SpaceSource n v  (@\label{line:space-source:start}@)
  = forall s p. SourceTree (Tree s p n v)
  | forall q. (Query q, QResult q ~ v) => SourceQuery (AttachedQuery n q)  (@\label{line:space-source:end}@)

data OpaqueSpace p n v  -- `OpaqueSpace p` is an instance of `Space`   (@\label{line:opaque-space:start}@)
getStream :: OpaqueSpace p n v -> SearchStream (ExecMonad p) (Tracked n v)
type ExecMonad p = ReaderT (p, [Demonstration]) (WriterT Trace IO)  (@\label{line:opaque-space:end}@)

data SomeParametricSpace n =  (@\label{line:some-parametric-space:start}@)
  forall sp i v. (Space sp, Typeable i, Typeable v) =>
  SomeParametricSpace (Tracked n i -> sp n v)  (@\label{line:some-parametric-space:end}@)

type ChoiceFun m n = forall sp v. (Space sp) => sp n v -> m (Tracked n v)   (@\label{line:choice-fun}@)
\end{lstlisting}
\end{lncodebox}
\vspace{-0.2cm}
\caption{Type Definitions for \emph{Effects} and \emph{Spaces}. The type variable naming convention from Figure~\ref{fig:tree-def} applies. An \code{AttachedQuery} is a wrapper around a \emph{query} (Section~\ref{sec:queries}), which further ensures that query answers can be turned into proper \emph{tracked values}. More details on the definition of \code{OpaqueSpace} (Lines~\ref{line:opaque-space:start}-\ref{line:opaque-space:end}) are provided in Section~\ref{sec:policy-lang}. For now, it is enough to notice that the \code{getStream} function can be used to obtain a \emph{search stream} of tracked values given an \emph{inner policy} (\code{p} argument of the \code{ReaderT} monad transformer).}\label{fig:eff-space-def}
\end{figure}
