\begin{figure}
\begin{lcodebox}
\begin{lstlisting}[style=haskell]
mzero :: SearchStream m a
yield :: a -> SearchStream m a ; spent, barrier :: Budget -> SearchStream m a
(<|>) :: SearchStream m a -> SearchStream m a -> SearchStream m a
withBudget :: Budget -> SearchStream m a -> SearchStream m a
streamTake :: Int -> SearchStream m a -> SearchStream m a
streamBind :: SearchStream m a -> (a -> SearchStream m b) -> SearchStream m b
\end{lstlisting}
\end{lcodebox}
\vspace{-0.3cm}
\caption{Examples of Stream Combinators. The \code{(Monad m)} constraint applies implicitly everywhere. Streams can be concatenated via \code{<|>}, for which \code{mzero} is the neutral element. The \code{withBudget} and \code{streamTake} functions take a stream and return a prefix of it, as defined by a budget limit or a number of desired elements. The \code{streamBind} function is akin to \code{concatMap} for lists and is used in defining \code{dfs} in Figure~\ref{fig:dfs}.}\label{fig:stream-combinators}
\end{figure}
