\begin{figure}
\begin{ccodebox}
\begin{lstlisting}[style=haskell]
branch :: (Branch `In` s) => OpaqueSpaceBuilder p a -> Strategy s p a
query :: (Query q) => (p -> PromptingPolicy) -> q -> OpaqueSpaceBuilder p (QResult q)
search :: (p -> (SearchPolicy s', p')) -> Strategy s' p' a -> OpaqueSpaceBuilder p a
\end{lstlisting}
\end{ccodebox}
\vspace{-0.3cm}
\caption{Using and Building Opaque Spaces. The \code{branch} function takes an \code{OpaqueSpaceBuilder} value as an argument and returns a monadic value wrapping an element from this space (\code{Strategy s p} has a monadic structure). \code{OpaqueSpaceBuilder} can be understood as a synonym for \code{OpaqueSpace} on first reading (the latter captures additional information obtained during the reification process, which is needed for properly tracking value provenance, as explained in Section~\ref{sec:ext-comp-trees}). The \code{OpaqueSpaceBuilder} type takes two type arguments: one for the type of the associated inner policy and one for the type of space elements. The \code{In} constraint expresses an effect's membership in a strategy signature. The \code{query} function takes as its first argument a function that maps the surrounding inner policy to a prompting policy (see Figure~\ref{fig:strategy-overview} for an example). A query type \code{q} implements the \code{Query} type class and is associated an answer type \code{Query q} (Section~\ref{sec:queries}). Finally, the \code{search} function takes as its first argument a function that maps the surrounding inner policy to a pair of a search policy and of an inner policy with suitable type.}\label{fig:branch-query-search}
\end{figure}
