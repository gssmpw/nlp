\begin{figure}
\begin{subfigure}[b]{\textwidth}
\begin{lcodebox}
\begin{lstlisting}[style=haskell]
data Fail :: EffectKind where Fail :: Fail p t n Void

fail :: (Fail `In` s) => Strategy s p a
ensure :: (Fail `In` s) => Bool -> Strategy s p ()
\end{lstlisting}
\end{lcodebox}
\vspace{-0.15cm}
\caption{The \code{Fail} effect. A failure node admits no children and so its action type is the empty type \code{Void}.}\label{fig:fail-effect}
\vspace{0.3cm}
\end{subfigure}

\begin{subfigure}[b]{\textwidth}
\begin{lcodebox}
\begin{lstlisting}[style=haskell]
data CompareAndBranch :: EffectKind where
  CompareAndBranch :: (Typeable a) =>
    { cands :: OpaqueSpace p n a
    , compare :: [Tracked n a] -> OpaqueSpace p n [Double]
    } -> CompareAndBranch p t n a

compareAndBranch :: (CompareAndBranch `In` s, Typeable a)
  => OpaqueSpaceBuilder p a -> ([a] -> OpaqueSpaceBuilder p [Double])
  -> Strategy s p a
\end{lstlisting}
\end{lcodebox}
\vspace{-0.15cm}
\caption{The \code{CompareAndBranch} effect, which illustrates the ability for nodes to contain \emph{parametric} spaces.}\label{fig:compare-and-branch-effect}
\vspace{0.3cm}
\end{subfigure}

\begin{subfigure}[b]{\textwidth}
\begin{lcodebox}
\begin{lstlisting}[style=haskell]
data Join :: EffectKind where
  Join :: (Typeable a, Typeable b) =>
    { left :: t n a, right :: t n b} -> Join p t n (a, b)

join :: (Join `In` s, Typeable a, Typeable b)
     => Strategy s p a -> Strategy s p b -> Strategy s p (a, b)
\end{lstlisting}
\end{lcodebox}
\vspace{-0.15cm}
\caption{The \code{Join} effect, which illustrates the ability for nodes to contain \emph{embedded trees}.}\label{fig:join-effect}
\end{subfigure}
\caption{Some Examples of Effects. Defining a new effect requires defining a type that implements the \code{Effect} class (Figure~\ref{fig:eff-space-def}). Opaque spaces have a phantom type argument \code{n} identifying the tree node that they belong to. Elements of those spaces as similarly marked with the \code{Tracked} wrapper (Section~\ref{sec:ext-comp-trees}).}\label{fig:effect-examples-short}
\end{figure}


