\begin{figure}
\centering
\includegraphics[width=\textwidth]{figures/img/demo-screenshot.png}
\vspace{-1cm}
\caption{Using the Delphyne VSCode Extension to Write a Demonstration. Passing tests are underlined in blue while failing tests are underlined in red. Code actions allow inspecting the destination node of each test. Doing so for the failing test reveals that an \code{IsProposalNovel} query needs to be answered. This query can be added to the demonstration body by clicking on the \code{+} icon next to the query name. In general, the three views in the left pane allow inspecting arbitrary traces and describe a single selected node at any  moment in time. The \code{Tree} view shows the position of this node in the trace, the \code{Node} view shows all spaces attached to it, and the \code{Actions} view shows all associated actions. The trace can be navigated by clicking on actions to access children, on spaces to access nested trees, or on ancestor nodes from the \code{Tree} view. }\label{fig:demo-screenshot}
\end{figure}
