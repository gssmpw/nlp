\definecolor{query-bg-color}{RGB}{212, 230, 241}
\definecolor{query-answer-color}{RGB}{84, 153, 199}
\definecolor{success-color}{RGB}{20, 90, 50}
\definecolor{success-bg-color}{RGB}{212, 239, 223}
\definecolor{failure-bg-color}{RGB}{253, 237, 236}
\definecolor{node-bg-color}{RGB}{242, 244, 244 }


\begin{figure}
  \def\botArr{0.7cm}
  \def\qdist{0.25cm}
  \def\adist{-0.04cm}
  \def\mainNodeType{B}
  \def\auxNodeType{C}
  \def\attachAnswers#1#2#3{\node[answers] [#1 = \adist of #2] {#3};}
  \def\attachSuccessVal#1#2#3{\node[successVal] [#1 = \adist of #2] {#3};}
  \def\attachQuery#1#2#3#4#5{%
    \node[query] (#1) [#2 = \qdist of #3] {};%
    \node[answers] [#4 = \adist of #1] {#5};%
    \draw[choice] (#1) -- (#3);}
  \def\attachChild#1#2#3#4#5#6#7{
    \node[#5] (#1) at ($ (#3) + (#4) $) {#2};
    \draw[child] (#3) -- node[lbl,#6]{#7} (#1);}
  \begin{tikzpicture}[
    node/.style = {draw,circle,minimum size=0.4cm, inner sep=0pt, font=\ttfamily\footnotesize, fill=node-bg-color},
    failureNode/.style = {node, fill=failure-bg-color},
    success/.style = {font=\footnotesize, inner sep=0pt},
    successNode/.style = {node, double, minimum size=0.35cm, fill=success-bg-color},
    query/.style = {draw, diamond, inner sep=0pt, minimum size=0.33cm, fill=query-bg-color},
    child/.style = {->, >={Classical TikZ Rightarrow[width=2mm]}},
    choice/.style = {dash pattern=on 1.5pt off 1pt, -},
    lbl/.style = {midway, font=\ttfamily\footnotesize},
    choiceLbl/.style = {midway, font=\ttfamily\footnotesize},
    answers/.style = {font=\ttfamily\footnotesize, color = query-answer-color},
    successVal/.style = {font=\ttfamily\footnotesize, color = success-color},
    every path/.style = {line width=0.7pt}
  ]
    \node (bot) {};
    \node[node] (root) at ($ (bot) + (90:\botArr) $) {\mainNodeType};
    \draw[child,line width=0.4mm] (bot) -- (root);
    \node[node] (auxRoot) at ($ (root) + (0:2.3cm) $) {\auxNodeType};
    \draw[choice] (root) -- node[choiceLbl,above] {$\refcname{cands}$} (auxRoot);
    \attachQuery{q0}{above}{auxRoot}{left}{$1, 2$}
    \attachChild{aux1}{\auxNodeType}{auxRoot}{25:1cm}{node}{above}{$1$}
    \attachQuery{q1}{above}{aux1}{left}{$3, 4$}
    \attachChild{aux2}{\auxNodeType}{auxRoot}{-25:1cm}{node}{below}{$2$}
    \attachQuery{q2}{below}{aux2}{left}{$5$}
    \attachChild{auxC1}{F}{aux1}{25:1cm}{failureNode}{above}{$3$}
    \attachChild{auxC2}{}{aux1}{-25:1cm}{successNode}{below}{$4$}
    \attachSuccessVal{right}{auxC2}{$\refpath{1, 4}$}
    \attachChild{auxC3}{}{aux2}{-25:1cm}{successNode}{below}{$5$}
    \attachSuccessVal{right}{auxC3}{$\refpath{2, 5}$}
    \node[query] (compare) at ($ (auxRoot.south |- q2) $) {};
    \draw[choice] (auxRoot) -- node[choiceLbl,left,xshift=0.05cm] {$\refcname{compare} \, \reflist{1,2}$} (compare);
    \attachAnswers{left}{compare}{$6$}

    \attachChild{main1}{\mainNodeType}{root}{115:1.3cm}{node}{right,xshift=0.53cm}{$\refpath{2, 5}$}
      \attachQuery{main1Q}{left}{main1}{left}{$7$}
      \attachChild{main1Fail}{F}{main1}{115:1cm}{failureNode}{left}{$7$}
    \attachChild{main2}{\mainNodeType}{root}{65:1.3cm}{node}{left, xshift=-0.53cm}{$\refpath{1, 4}$}
      \attachQuery{main2Q}{right}{main2}{right}{$8, 9$}
      \attachChild{main2Fail}{F}{main2}{115:1cm}{failureNode}{left}{$8$}
      \attachChild{main2Success}{}{main2}{65:1cm}{successNode}{right}{$9$}
        \attachSuccessVal{right}{main2Success}{$\refpath{\refpath{2, 5}, 9}$}
  \end{tikzpicture}
  \vspace{-0.2cm}
  \caption{A Strategy Tree. Only a subset of spaces and subtrees is shown for each node (many nodes are infinitely branching). The root is indicated by a thick incoming arrow. Numbers denote answer identifiers. Diamonds denote queries (labeled with answer identifiers) while circles denote nodes (labeled \texttt{B} for \code{Branch}, \texttt{F} for \code{Fail}, and \texttt{C} for \code{CompareAndBranch}). Double circles denote success leaves (labeled with the reference of the associated value). Children are attached to their parents via solid arrows (labeled with value references) while spaces are attached with dotted edges (labeled with space references, defaulting to \code{cands}). }\label{fig:tree-example}
\end{figure}
