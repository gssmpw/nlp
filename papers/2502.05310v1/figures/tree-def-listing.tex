\begin{figure}[t]
\begin{lncodebox}
\hspace{17pt}
\begin{lstlisting}[style=haskell,numbers=left,numbersep=12pt]
data Tree s p n v  (@\label{line:tree:start}@)
  = Success GlobalNodeRef (Tracked n v)  (@\label{line:success}@)
  | forall n'. SomeNode GlobalNodeRef (NodeOf s p (Tree s p) n' (Tree s p n v))  (@\label{line:some-node}\label{line:tree:end}@)

data NodeOf s p t n k where  (@\label{line:node-of:start}@)
  Z :: Node e p t n k -> NodeOf (e ': s) p t n k
  S :: NodeOf s p t n k -> NodeOf (e ': s) p t n k  (@\label{line:node-of:end}@)

data Node e p t n k =
  forall a. (Effect e) =>
  Node { effect :: e p t n a, child :: Tracked n a -> k }

reify :: Strategy s p v -> SomeTree s p v (@\label{line:make-tree}@)
data SomeTree s p v = forall n. Tree s p n v
\end{lstlisting}
\end{lncodebox}
\vspace{-0.2cm}
\caption{Type Definition for a \emph{Strategy Tree}. We use the following convention to name single-letter type variables: \code{e} for \emph{effect} (such as \code{Branch}), \code{s} for \emph{signature} (i.e. a list of effects such as \code{[Branch, Fail]}), \code{p} for \emph{inner policy type} (such as \code{GPIP}), \code{v} for \emph{value} (the return type of a \emph{tree} or \emph{space}), \code{n} for \emph{node} (a phantom type that uniquely identifies a tree node), \code{a} for \emph{action type} (the type of value expected by the \code{child} function), and \code{t} for \emph{tree} (\code{Tree s p} for some \code{s} and \code{p}, used to restrict the type of \emph{embedded trees}). The \code{NodeOf} type models an extensible sum type, with constructors that encode a Peano integer pointing to an element of the signature parameter \code{s}. See Figure~\ref{fig:refs-def} for the definition of \code{Tracked} and \code{GlobalValueRef} and see Section~\ref{sec:ext-comp-trees} for more explanations. }\label{fig:tree-def}
\end{figure}
