\section{Related Work}
\label{sec:related-work}

Many frameworks exist to simplify the development of LLM-enabled programs~\cite{liu2023prompting}. However, our framework is unique in harnessing the full power and generality of nondeterministic programming to robustly integrate prompting into modular, first-class programs. To the best of our knowledge, it is also the first to identify and address the challenge of treating few-shot examples as \emph{maintainable} program components. Finally, while existing frameworks offer a restricted set of search primitives, such as repeated trials and majority voting, our framework allows writing and composing arbitrary, \emph{modular} policies. While this level of power and generality may not be necessary for all applications, we expect Delphyne to be particularly useful in scenarios where reliable external feedback enables the full exploitation of search. This is notably the case in areas such as program synthesis, program verification and theorem proving, where high-quality feedback is available through interpreters, static analyzers, model-checkers and proof assistants.

The DSPy framework allows building LLM pipelines by combining declarative modules that carry natural-language typed signatures~\cite{khattab2024dspy}. Notably, it allows automatically optimizing prompts and examples based on an objective function, drawing similarity to backpropagation-based learning frameworks such as Pytorch~\cite{paszke2019pytorch}. Future work will explore integrating such functionality into oracular programming frameworks. The LMQL language~\cite{beurer2023prompting} enables guiding LLMs by enforcing precise answer templates, making it a natural complement to our framework, which orchestrates multiple requests. Future work will integrate an LMQL-based prompting policy into Delphyne.