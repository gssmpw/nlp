\section{Preliminary notation and problem definition}\label{sec:prel}

We begin by providing preliminary notation and formally defining our problem.

Our input is an unweighted graph  $G = (V, E)$, and we denote the number of nodes and edges by $n = \abs{V}$ and  $m = \abs{E}$. 
Given a graph $G = (V, E)$ and a set of nodes $U \subseteq V$, we define $E(U) \subseteq E$ to be the subset of edges having both endpoints in $U$. We denote the degree of vertex $v$ as $\deg(v)$. We denote the set of adjacent edges of vertex $v$ as $N(v)$.  % and neighboring vertices of vertex $v$ as $B(v)$.

We want to label the set of edges $E$ as either {\em strong} or {\em weak}. 
To perform the labeling,  we use the \emph{strong triadic closure} ({\stc}) property~\cite{sintos2014using}.  We say that the graph is \stc satisfied if for any given triplet of vertices, $(x, y, z)$ the following holds: if  $(x, y)$ and $(y, z)$ are connected and are labeled as strongly connected, then the edge $(x, z)$  always exists at least as a  weak edge. 

We call a triplet of vertices $(x, y, z)$  a {\em wedge}, if $(x, y) \in E$, $(y, z) \in E$, and $(x, z) \notin E$.
%We often denote a wedge as $y, (x, z)$ when the edges $(x, y)$ and $(y, z)$ form a wedge. 
A \emph{wedge graph} $Z(G)$ consists of the edges of the graph $G$ that contribute to at least one wedge as its vertices. If the two edges $e_1$ and $e_2$ of $G$ form a wedge, we add an edge between the two nodes in $Z(G)$ that corresponds to $e_1$ and $e_2$, that is, each edge of $Z(G)$ corresponds to a wedge of $G$.
 

Given a graph $G = (V, E)$ and a labeling $L$ of the edges as {strong} or  {weak}, we write $E_s(U, L)$ and  $E_w(U, L)$ to be the set of {strong} and {weak} edges of the graph induced by a set of vertices $U$.
We also write $m_s(U, L) = \abs{E_s(U, L)}$ and $m_w(U, L) = \abs{E_w(U, L)}$.
Finally, for a vertex $u \in U$, we define a strong and weak degree, $\deg_s(u, U, L)$ and $\deg_w(u, U, L)$ to be the number of strong or weak edges in $U$ adjacent to $u$. We may omit $L$ or $U$ from the notation when it is clear from the context.
 

Given a weight parameter $\lambda$ where $\lambda \in [0,\;1]$ and a label assignment $L$, we define a score
\[
	\score{U, L; \lambda} =\frac{m_s(U, L) + \lambda m_w(U, L)} {\abs{U}} \quad .
\]
We may omit $L$ or $\lambda$ when it is clear from the context.

We consider the following optimization problem. 

\begin{problem}[\prbstrwk]
\label{pr:label-subgrap-str-wk}
Given a graph $G = (V,E)$ and a weight parameter $\lambda$, find a subset of vertices $U \subseteq V$ and a labeling $L$ of the edges as {strong} or {weak}  such that the \stc property is satisfied in $(U, E(U))$ and  $\score{U, L; \lambda}$ is maximized.
\end{problem}



Note that when $\lambda = 1$ then the labeling does not matter, and \prbstrwk reduces to dense subgraph discovery, that is, finding $U$ with the largest ratio $\abs{E(U)}/\abs{U}$. On the other hand, if $\lambda = 0$, then we only take into account strong edges; we will show in Section~\ref{sec:compl} that in this case, \prbstrwk is equal to finding a maximum clique.


Note also that the labeling depends on the underlying subgraph $U$, that is, we need to find $U$ and $L$ simultaneously.