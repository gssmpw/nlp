\section{Related work}\label{sec:related}

\textbf{The strong triadic closure (\stc) property:}
As an early work in this line of research, \citet{sintos2014using} considered the problem of minimizing the number of weak edges (analogously maximizing the number of strong edges) while labeling the edges compliant with the \stc property.
We will refer to this optimization problem as \prbminSTC.

\citet{sintos2014using} showed
that \prbminSTC is equivalent to solving a minimum vertex cover, which we denote by \prbcovermin, in a wedge graph $Z(G)$.
In \prbcovermin, we search for a minimum number of nodes such that at least one endpoint of each edge is in the set.

\citet{sintos2014using} proposed the following algorithm for \prbminSTC, which we denote by \algminstc.
Given a graph $G$ they first construct the wedge graph $Z(G)$ and find its vertex cover. Next, they label the set of edges of $G$ that corresponds to the set of vertices in the vertex cover as weak, and the remaining edges as strong. Since \prbcovermin is \np-hard, they approximate it with a maximal matching algorithm~\citep{clarkson1983modification}. The algorithm picks an arbitrary edge and adds both endpoints of the edge to the cover, and the edges incident to both endpoints are deleted. It continues until no edges are left. This algorithm outputs a {\em maximal matching} which is known to yield a $2$-approximation since at least one endpoint should be in the cover.

The problem of finding a labeling of edges that maximizes the number of strong edges while satisfying the \stc property is \np-hard for general graphs \cite{sintos2014using} and split graphs \cite{konstantinidis2020maximizing}. 
Nevertheless, it becomes polynomial-time solvable for proper interval
graphs~\cite{konstantinidis2020maximizing}, cographs~\cite{konstantinidis2018strong}, and trivially perfect graphs~\cite{konstantinidis2020maximizing}.
Given communities, \citet{rozenshtein2017inferring} considered the problem of inferring strengths while minimizing \stc violations with additional connectivity constraints.
\citet{oettershagen2022inferring} extended the idea of inferring tie strength for temporal networks.
\citet{matakos2022strengthening} considered the problem of strengthening edges to maximize \stc violations, which they consider as opportunities to build new connections.
\citet{adriaens2020relaxing} formulated both minimization and maximization versions of \stc problems as linear programs.

\textbf{The dense subgraph discovery:}
Finding dense subgraphs is a core problem in social network analysis. 
Given a graph, the densest subgraph problem is defined as finding a subgraph with the highest average degree density~(twice the number of edges divided by the number of nodes).
Finding the densest subgraph for a single graph has been extensively studied~\cite{goldberg1984finding,charikar2000greedy,khuller2009finding}.
\citet{goldberg1984finding} proposed an exact,
polynomial time algorithm that solves a sequence of min-cut instances. We will refer to this algorithm as \algdg. \citet{asahiro2000greedily} provided a greedy algorithm and 
 \citet{charikar2000greedy} proved that their greedy algorithm gives a $1/2$-approximation, showed how to implement the algorithm using Fibonacci heaps, and devised a linear-programming formulation of the problem. The idea of the greedy algorithm is that at each iteration, a vertex with the minimum degree is removed, and then the densest
subgraph among all the produced subgraphs is returned as the solution. We will refer to this algorithm as \algdc. It has also been extended for multiple graph snapshots~\cite{jethava2015finding,semertzidis2019finding,arachchi2023jaccard}. The problem has been studied in a streaming setting~\cite{bhattacharya2015space}. 
To the best of our knowledge, this is the first attempt to consider the notion of density together with the \stc property.

In addition to degree density (a.k.a average degree), alternative types of density measures have also been considered previously such as triangle density and $k$-clique density~\cite{tsourakakis2015k}.
Triangle density is defined as the ratio between the number of triangles and the number of vertices of the subgraph.
The definition that will be used in this paper is the ratio between the number of induced edges and nodes.
Adopting our problem to other density measures is left open as future work.










