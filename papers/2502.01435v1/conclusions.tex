\section{Concluding remarks}\label{sec:conclusions}

We introduced a novel dense subgraph discovery problem that takes into account the strength of ties within the subgraph.
Here we label each edge either as {strong} or {weak} based on the strong triadic closure principle (\stc). The \stc property requires that if one node {strongly} connects with two other nodes, then those two nodes should at least have a  {weak} connection between them. 
Our goal was to maximize a density-like measure defined as the sum of the number of strong edges and the number of weak edges weighted by a weight parameter, divided by the number of nodes within the subgraph.
We showed that our optimization problem is \np-hard, and connects the two well-known problems of finding dense subgraphs and maximum cliques. To solve the problem, we presented an exact algorithm based on integer linear programming. In addition, we presented a polynomial-time algorithm based on linear programming, a greedy heuristic algorithm, and two other straightforward algorithms based on the algorithms for the densest subgraph discovery.

The experiments with synthetic data showed that our approach recovers the
latent dense components. The experiments on real-world networks confirmed that our
proposed algorithms discovered the subgraphs reasonably fast in practice.
Finally, we presented a case study where our algorithm produced interpretable results suggesting the practical usefulness of our problem setting and algorithms.

The idea of combining the dense subgraph problem
with the \stc property opens up several lines of work. For example, 
instead of using the ratio between the number of edges and the number of nodes as the density measure, we can incorporate other density measures.


