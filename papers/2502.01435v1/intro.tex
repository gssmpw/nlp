\section{Introduction}

Many social networks naturally contain both {\em strongly} connected and {\em weakly} connected interactions among the entities of the network.
A question of particular interest is that given a set of pairwise user interactions, how to infer the strength of the social ties within the network? In other words, how to label the edges of an undirected graph as either strong or weak?

The problem of inferring the strength of social ties based on {\em strong triadic closure} principle~(\stc) has drawn attention over the past decade within the data mining community~\cite{sintos2014using,rozenshtein2017inferring,adriaens2020relaxing,oettershagen2022inferring,konstantinidis2018strong,matakos2022strengthening}. The \stc property assumes that there exist two types of social ties in the network: either {\em strong} or {\em weak}.
Let $A, B$, and $C$ be three entities in the network.
If the entities $A$ and $B$ are strongly connected with the entity $C$, then there should be at least a weak connection between $A$ and $B$. In other words, if both $A$ and $B$ are strong friends of $C$, then some kind of connection between $A$ and $B$ should also exist.
Note that these labels are not known and the goal is to infer the labels from the given unlabeled graph.

In this paper, we incorporate the \stc property into the problem of dense subgraph discovery~\cite{sintos2014using, goldberg1984finding}. More formally, given a subgraph and a weight parameter $\lambda$, we define a score as the ratio between the sum of the number of {strong} and {weak} edges weighted by $\lambda$ and the number of nodes within the subgraph.
Our objective is to find a subgraph \emph{and} a labeling that maximize our score while satisfying the \stc property within the subgraph. 

We will see that when $\lambda = 0$ finding an optimal subgraph is equal to finding a maximum clique. On the other hand, for $\lambda = 1$, finding an optimal subgraph is equal to finding the densest subgraph, that is, a subgraph $U$ maximizing the ratio of edges and nodes, $\abs{E(U)}/\abs{U}$. Both of these problems are well-studied. Optimizing the score for $0 < \lambda < 1$, yields a problem that is between these two cases. We expect that for small $\lambda$s the returned subgraph resembles a clique whereas large $\lambda$s yield a subgraph similar to the densest subgraph.

\begin{figure}
\begin{center}
\begin{tikzpicture}[scale = .6]
\tikzstyle{lm1} = [fill = yafcolor5!50!black, inner sep = 1.2pt]
\tikzstyle{lm2} = [circle, fill = yafcolor4!50!black, inner sep = 1pt]
\tikzstyle{lm3} = [circle, fill = gray!50, inner sep = 1pt]
\tikzstyle{ll} = []
%\tikzstyle{le1} = [yafcolor2, thick, bend left = 10, opacity = 0.5]
%\tikzstyle{le2} = [gray, bend left = 10, opacity = 0.5]
\tikzstyle{le1} = [yafcolor5, thick, bend left = 10]
\tikzstyle{le2} = [yafcolor4, thick, bend left = 10]
\tikzstyle{le3} = [gray!50, bend left = 10]
\input{karate_graph/karate_sintos}
\node[anchor=south west] at (-0.5, -3) {(a)};
\end{tikzpicture}
\hspace{0.5cm}
\begin{tikzpicture}[scale = .6]
\tikzstyle{lm1} = [fill = yafcolor3!50!black, inner sep = 1.2pt]
\tikzstyle{lm2} = [circle, fill = yafcolor4!50!black, inner sep = 1pt]
\tikzstyle{lm3} = [circle, fill = gray!50, inner sep = 1pt]
\tikzstyle{ll} = []
%\tikzstyle{le1} = [yafcolor2, thick, bend left = 10, opacity = 0.5]
%\tikzstyle{le2} = [gray, bend left = 10, opacity = 0.5]
\tikzstyle{le1} = [yafcolor4, thick, bend left = 10]
\tikzstyle{le2} = [yafcolor5, thick, bend left = 10]
\tikzstyle{le3} = [gray!50, bend left = 10]
\input{karate_graph/karate_greedy}
\node[anchor=south west] at (-0.5, -3) {(b)};
\end{tikzpicture}
\hspace{0.5cm}
\begin{tikzpicture}[scale = .6]
\tikzstyle{lm1} = [fill = yafcolor3!50!black, inner sep = 1.2pt]
\tikzstyle{lm2} = [circle, fill = yafcolor4!50!black, inner sep = 1pt]
\tikzstyle{lm3} = [circle, fill = gray!50, inner sep = 1pt]
\tikzstyle{ll} = []
%\tikzstyle{le1} = [yafcolor2, thick, bend left = 10, opacity = 0.5]
%\tikzstyle{le2} = [gray, bend left = 10, opacity = 0.5]
\tikzstyle{le1} = [yafcolor4, thick, bend left = 10]
\tikzstyle{le2} = [yafcolor5, thick, bend left = 10]
\tikzstyle{le3} = [gray!50, bend left = 10]

\node[lm2] (n0) at (-0.39521, -0.90369) {};
\node[lm2] (n1) at (-0.1494, -0.63131) {};
\node[lm2] (n2) at (0.089756, -0.058231) {};
\node[lm2] (n3) at (0.32331, -1.0532) {};
\node[lm3] (n4) at (0.023263, -1.9859) {};
\node[lm3] (n5) at (-1.2108, -1.6373) {};
\node[lm3] (n6) at (-0.73841, -1.9544) {};
\node[lm3] (n7) at (0.69671, -1.0172) {};
\node[lm2] (n8) at (0.45414, -0.064462) {};
\node[lm3] (n9) at (1.35, 0.053898) {};
\node[lm3] (n10) at (-0.47034, -1.7978) {};
\node[lm3] (n11) at (-1.6432, -0.89285) {};
\node[lm3] (n12) at (0.52145, -1.7831) {};
\node[lm2] (n13) at (0.59789, -0.39239) {};
\node[lm3] (n14) at (1.15, 1.6278) {};
\node[lm3] (n15) at (1.644, 0.73643) {};
\node[lm3] (n16) at (-1.4464, -2.6101) {};
\node[lm3] (n17) at (-1.278, -0.60231) {};
\node[lm3] (n18) at (0.51654, 1.9636) {};
\node[lm3] (n19) at (-0.30691, 0.035621) {};
\node[lm3] (n20) at (0.10109, 1.891) {};
\node[lm3] (n21) at (-1.1086, -1.0869) {};
\node[lm3] (n22) at (1.4907, 1.075) {};
\node[lm3] (n23) at (-0.2767, 1.6423) {};
\node[lm3] (n24) at (-1.5541, 0.77017) {};
\node[lm3] (n25) at (-1.2836, 1.3377) {};
\node[lm3] (n26) at (1.5531, 1.5267) {};
\node[lm3] (n27) at (-0.5343, 0.9102) {};
\node[lm3] (n28) at (-0.16321, 0.83704) {};
\node[lm3] (n29) at (0.70218, 1.7305) {};
\node[lm2] (n30) at (0.93024, 0.14795) {};
\node[lm3] (n31) at (-0.5944, 0.37166) {};
\node[lm2] (n32) at (0.42538, 0.99008) {};
\node[lm2] (n33) at (0.58393, 0.82341) {};
\begin{pgfonlayer}{background}

% strong edges

\draw (n0) edge[le1] (n1);
\draw (n0) edge[le1] (n2);
\draw (n0) edge[le1] (n3);
\draw (n0) edge[le1] (n13);

\draw (n1) edge[le1] (n2);
\draw (n1) edge[le1] (n3);
\draw (n1) edge[le1] (n13);

\draw (n2) edge[le1] (n3);
\draw (n2) edge[le1] (n13);

\draw (n3) edge[le1] (n13);

\draw (n8) edge[le1] (n32);
\draw (n8) edge[le1] (n30);
\draw (n8) edge[le1] (n33);



\draw (n30) edge[le1] (n33);
\draw (n30) edge[le1] (n32);
\draw (n32) edge[le1] (n33);


% weak edges

\draw (n0) edge[le2] (n8);
\draw (n1) edge[le2] (n30);
\draw (n2) edge[le2] (n8);
\draw (n2) edge[le2] (n32);
\draw (n13) edge[le2] (n33);





%  other edges


\draw (n0) edge[le3] (n31);
\draw (n2) edge[le3] (n28);
\draw (n28) edge[le3] (n33);
\draw (n31) edge[le3] (n32);
\draw (n31) edge[le3] (n33);
\draw (n28) edge[le3] (n31);

\draw (n0) edge[le3] (n4);
\draw (n0) edge[le3] (n5);
\draw (n0) edge[le3] (n6);
\draw (n0) edge[le3] (n7);

\draw (n0) edge[le3] (n10);
\draw (n0) edge[le3] (n11);
\draw (n0) edge[le3] (n12);
\draw (n0) edge[le3] (n17);
\draw (n0) edge[le3] (n19);
\draw (n0) edge[le3] (n21);
\draw (n1) edge[le3] (n7);
\draw (n1) edge[le3] (n17);
\draw (n1) edge[le3] (n19);
\draw (n1) edge[le3] (n21);



\draw (n2) edge[le3] (n7);

\draw (n2) edge[le3] (n9);

\draw (n2) edge[le3] (n27);

\draw (n3) edge[le3] (n7);
\draw (n3) edge[le3] (n12);
\draw (n4) edge[le3] (n6);
\draw (n4) edge[le3] (n10);
\draw (n5) edge[le3] (n6);
\draw (n5) edge[le3] (n10);
\draw (n5) edge[le3] (n16);
\draw (n6) edge[le3] (n16);

\draw (n9) edge[le3] (n33);

\draw (n14) edge[le3] (n32);
\draw (n14) edge[le3] (n33);
\draw (n15) edge[le3] (n32);
\draw (n15) edge[le3] (n33);
\draw (n18) edge[le3] (n32);
\draw (n18) edge[le3] (n33);
\draw (n19) edge[le3] (n33);
\draw (n20) edge[le3] (n32);
\draw (n20) edge[le3] (n33);
\draw (n22) edge[le3] (n32);
\draw (n22) edge[le3] (n33);
\draw (n23) edge[le3] (n25);
\draw (n23) edge[le3] (n27);
\draw (n23) edge[le3] (n29);
\draw (n23) edge[le3] (n32);
\draw (n23) edge[le3] (n33);
\draw (n24) edge[le3] (n25);
\draw (n24) edge[le3] (n27);
\draw (n24) edge[le3] (n31);
\draw (n25) edge[le3] (n31);
\draw (n26) edge[le3] (n29);
\draw (n26) edge[le3] (n33);
\draw (n27) edge[le3] (n33);


\draw (n29) edge[le3] (n32);
\draw (n29) edge[le3] (n33);




\end{pgfonlayer}





%%%%%%%%%%%%%%%%%%
% NEW IP implemenation 
% 0.7

% strong 
% [(0, 1), (0, 2), (0, 3), (0, 7), (1, 2), (1, 3), (1, 7), (2, 3), (2, 7), (3, 7), (8, 30), (8, 32), (8, 33), (28, 31), (30, 32), (30, 33), (32, 33)]
% obj:  2.2333333333333325 subgraph :  12, st 17.0 wk 14.0, itr 10
% IP based : lam: 0.7 density : 2.2333333333333325 size: 12


% [0, 1, 2, 3, 7, 8, 13, 28, 30, 31, 32, 33]

% weak
% [(0, 8), (0, 13), (0, 31), (1, 13), (1, 30), (2, 8), (2, 13), (2, 28), (2, 32), (3, 13), (13, 33), (28, 33), (31, 32), (31, 33)]


%%%%%%%%%%%%%%%%%%
%0.5

% [(0, 1), (0, 2), (0, 3), (0, 13), (1, 2), (1, 3), (1, 13), (2, 3), (2, 13), (3, 13), (8, 30), (8, 32), (8, 33), (30, 32), (30, 33), (32, 33)]

% [(0, 8), (1, 30), (2, 8), (2, 32), (13, 33)]

% [0, 1, 2, 3, 8, 13, 30, 32, 33]
% obj:  2.0555555555555554 subgraph :  9, st 16.0 wk 5.0, itr 10
% IP based : lam: 0.5 density : 2.0555555555555554 size: 9
% 0.22390294075012207 seconds ---


%%%%%%%%%%%%%%%%%%Full-graph
% [(0, 1), (0, 2), (0, 3), (0, 7), (1, 2), (1, 3), (1, 7), (2, 3), (2, 7), (3, 7), (4, 10), (5, 6), (5, 16), (6, 16), (8, 30), (8, 32), (8, 33), (23, 27), (24, 25), (24, 31), (25, 31), (26, 29), (30, 32), (30, 33), (32, 33)]

% SUBGRPAH OF [0, 1, 2, 3, 8, 13, 30, 32, 33]
% [(0, 1), (0, 2), (0, 3), (1, 2), (1, 3), (2, 3), , (8, 30), (8, 32), (8, 33),, (30, 32), (30, 33), (32, 33)]

% (0,13), (1, 13), (2, 13), (3,13) : w to s
\node[anchor=south west] at (-0.5, -3) {(c)};
\end{tikzpicture}


\caption{
Strong~(Red) and weak~(Blue) edges of the Karate club dataset maximizing the number of strong edges (a), $\lambda = 0$ (b), and $\lambda = 0.5$ (c) using our integer linear program based algorithm~(\algip). 
% Strong edge density is given by $0.74$ for (a). alg:greedy
We define our score as the sum of the number of strong and weak edges weighted by a parameter $\lambda$, divided by the size of the subgraph.
The scores are $2.0$ and $2.06$ for (b) and (c), respectively. We see that (b) is a clique of size $5$.}
\label{fig:karate}
\end{center}
\end{figure}


\textbf{Example:} 
We illustrate the difference between our problem and the original STC problem considered by \citet{sintos2014using} in Figure~\ref{fig:karate}.
The goal of this paper is to find a subgraph that maximizes our score while satisfying the \stc property. In contrast, \citet{sintos2014using} aims to label \emph{all} the edges in the graph such that the number of weak edges is minimized.
Figure~\ref{fig:karate} shows the results obtained with the Karate club dataset with our exact algorithm. We should stress that the labeling of the discovered \emph{subgraph} might be different from the labeling that maximizes strong edges for the whole graph. In Figure~\ref{fig:karate}~(c), we see that $4$ weak edges have been turned into strong while the labeling of the remaining edges is unchanged.

We show that our problem is \np-hard when $\lambda < 1$, and even inapproximable when $\lambda = 0$ since our problem then reduces to the \prbmaxclique problem. However, the $\lambda =  1$ case is solvable in polynomial time. To solve the problem,
we first propose an exact algorithm based on integer linear programming which runs in exponential time.  We consider four other heuristics that run in polynomial time in the size of the input graph:
We propose a linear linear programming based heuristic in Section~\ref{sec:lp}
and a greedy algorithm in Section~\ref{sec:greedy}.
We also propose two straightforward algorithms that combine the existing algorithms for solving the densest subgraph problem and finding STC-compliant labeling in an entire graph.

The remainder of the paper is organized as follows.
First, we introduce preliminary notation and our problem in Section~\ref{sec:prel}.
Next, we present the related work in Section~\ref{sec:related}.
Next, we show that our problem is \np-hard in Section~\ref{sec:compl} and then explain our algorithms in Section~\ref{sec:algorithm}.
Finally, we present our experimental evaluation in Section~\ref{sec:exp} and conclude the paper with a discussion in Section~\ref{sec:conclusions}.