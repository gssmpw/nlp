\section{Computational complexity}\label{sec:compl}
In this section, we analyze the computational complexity of the \prbstrwk problem by showing its \np-hardness when $\lambda < 1$ and its connection to the \prbmaxclique problem, when $\lambda = 0$. 
At the other extreme, when $\lambda = 1$, \prbstrwk is equivalent to the problem of finding the densest subgraph, which can be solved in polynomial time using the algorithm presented by~\citet{goldberg1984finding}.
 
\begin{proposition}
\label{prop:np_hardness}
For $\lambda=0$, \prbstrwk is \np-hard.
\end{proposition}

The proof (given in Appendix~\ref{appendix:complexity}) also shows that the maximum clique 
%$C$
yields precisely the optimal score 
%$q(C, L') = q(U, L)$ 
while satisfying the \stc property. As a consequence, combining this with the inapproximability results for \prbmaxclique~\cite{zuckerman2006linear} gives the next result (see Appendix~\ref{appendix:complexity} for the proof).

\begin{proposition}
\label{prop:inapproximability}
For $\lambda=0$, \prbstrwk does not have any polynomial time approximation algorithm with an approximation ratio better than $n^{1-\epsilon}$ for any constant $\epsilon > 0$, unless $\p = \np$.
\end{proposition}

Note that when $\lambda > 0$, we can obtain a $\frac{1}{\lambda}$-approximation by finding the densest subgraph with nodes $U$ and using a labeling $L$ that labels each edge as weak. Compared to an optimal solution $U^*$ with labeling $L^*$, we then obtain the score 
\begin{equation*}
\begin{split}
q(U, L; \lambda) & = \frac{\lambda\abs{E(U)}}{\abs{U}} \geq \frac{\lambda\abs{E(U^*)}}{\abs{U^*}} = \lambda\frac{m_s(U^*, L^*)+m_w(U^*, L^*)}{\abs{U^*}} \\
& \geq \lambda\frac{m_s(U^*, L^*)+\lambda m_w(U^*, L^*)}{\abs{U^*}} = \lambda q(U^*, L^*; \lambda).
\end{split}
\end{equation*}

In summary, \prbstrwk is inapproximable when $\lambda = 0$ but solvable in polynomial time when $\lambda = 1$. Finally, we state that \prbstrwk is also \np-hard for $0 < \lambda < 1$.

\begin{proposition}
\label{prop:nphard2}
\prbstrwk is \np-hard for $0 < \lambda < 1$.
\end{proposition}

The proof of Proposition~\ref{prop:nphard2} is in Appendix~\ref{appendix:complexity}.

