\begin{table}[t]
\centering
\small
% \resizebox{\columnwidth}{!}{%
\caption{Comparison of different \emph{``shades''} of fine-tuning methods. Each approach conceals different pieces of information regarding the backbone model and has varying requirements. The \mmark symbol indicates partial requirements or information that may vary depending on usage and often involves trade-offs. For instance, while LoRA may not require multiple backbone copies, it leads to multiple computational flows during inference. Although the zero-shot Black-box approach benefits from the most \cmark marks, \ours significantly improves zero-shot results by exposing only the gradient flow within the model.}

\label{tab:boxes}
\resizebox{0.95\columnwidth}{!}{%
\begin{tabular}{@{}lcccccc@{}}
\toprule
 Approach & \multicolumn{3}{c}{Hidden Information} & \multicolumn{3}{c}{Requirements} \\ \midrule
 & \begin{tabular}[c]{@{}c@{}}Gradients\\ Flow\end{tabular} & \begin{tabular}[c]{@{}c@{}}Backbone\\ Weights\end{tabular} & \multicolumn{1}{c|}{\begin{tabular}[c]{@{}c@{}}Layers\\ Sizes\end{tabular}} & \begin{tabular}[c]{@{}c@{}}No Layer\\ Choice\end{tabular} & \begin{tabular}[c]{@{}c@{}}Single\\ Backbone Copy\end{tabular} & \begin{tabular}[c]{@{}c@{}}Single Flow\\ Computation\end{tabular} \\ \midrule
\whitebox Full Finetune & \xmark & \xmark & \multicolumn{1}{c|}{\xmark} & \xmark & \xmark & \xmark \\
\whitebox LoRA & \xmark & \cmark & \multicolumn{1}{c|}{\xmark} & \xmark & \mmark & \xmark \\
\lightgraybox \oursp (ours) & \xmark & \cmark & \multicolumn{1}{c|}{\mmark} & \xmark & \mmark & \mmark \\
\darkgraybox \ours (ours) & \xmark & \cmark & \multicolumn{1}{c|}{\cmark} & \cmark & \cmark & \cmark \\
\blackbox Original {\small(zero-shot)} & \cmark & \cmark & \multicolumn{1}{c|}{\cmark} & \cmark & \cmark & \cmark \\ \bottomrule
\end{tabular}
}
  % }
\end{table}
