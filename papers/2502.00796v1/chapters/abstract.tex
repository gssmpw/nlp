The emergence of foundational models has greatly improved performance across various downstream tasks, with fine-tuning often yielding even better results. However, existing fine-tuning approaches typically require access to model weights and layers, leading to challenges such as managing multiple model copies or inference pipelines, inefficiencies in edge device optimization, and concerns over proprietary rights, privacy, and exposure to unsafe model variants. In this paper, we address these challenges by exploring ``Gray-box'' fine-tuning approaches, where the model's architecture and weights remain hidden, allowing only gradient propagation. We introduce a novel yet simple and effective framework that adapts to new tasks using two lightweight learnable modules at the model's input and output. Additionally, we present a less restrictive variant that offers more entry points into the model, balancing performance with model exposure. We evaluate our approaches across several backbones on benchmarks such as text-image alignment, text-video alignment, and sketch-image alignment. Results show that our Gray-box approaches are competitive with full-access fine-tuning methods, despite having limited access to the model.