% \documentclass[sigconf,review,anonymous]{acmart}
% \usepackage[a-1b]{pdfx}
\documentclass[sigconf]{acmart}
\copyrightyear{2025}
\acmYear{2025}
\setcopyright{rightsretained}

%% These commands are specific for your submission.
\acmConference[SIGIR '25]{Proceedings of the 48th International ACM SIGIR Conference on Research and Development in Information Retrieval}{July 13--18, 2024}{Padua, Italy.}
\acmBooktitle{Proceedings of the 48th Int'l ACM SIGIR Conference on Research and Development in Information Retrieval (SIGIR '25), July 13--18, 2024, Padua, Italy}




\makeatother


%
% inline lists.  usage,
%    \begin{inlinelist}
%       \item first item,
%       \item second item, and
%       \item last item.
%    \end{inlinelist}
%
\usepackage{enumitem}
\newlist{inlinelist}{enumerate*}{1}
\setlist*[inlinelist,1]{%
  label=(\roman*),
}
\usepackage{subfigure}
\usepackage{booktabs}
\usepackage{multirow}
\usepackage{array}
\usepackage{makecell}
\usepackage{colortbl}
\usepackage{xcolor}
\usepackage{bbm}
\usepackage{arydshln}

\newcommand{\movie}{Movie\xspace}
\newcommand{\landmark}{Landmark\xspace}
\newcommand{\person}{Person\xspace}


\newcommand{\gptfouro}{GPT-4o\xspace}
\newcommand{\gptfouromini}{GPT-4o-mini\xspace}

\settopmatter{printacmref=false}



\title{Open-Ended and Knowledge-Intensive Video Question Answering}

% \title{Multi-Modal Retrieval Augmentation for Open-Ended and Knowledge-Intensive Video Question Answering}



\author{Md Zarif Ul Alam}
\affiliation{\institution{University of Massachusetts Amherst}
\country{United States}}
\email{zarifalam@cs.umass.edu}

\author{Hamed Zamani}
\affiliation{\institution{University of Massachusetts Amherst}
\country{United States}}
\email{zamani@cs.umass.edu}




\begin{document}



% \fancyhead{}

% \begin{abstract}
% While current video question answering systems perform well on some tasks requiring only direct visual understanding, they struggle with questions demanding knowledge beyond what is immediately observable in the video content. We refer to this challenging scenario as knowledge-intensive video question answering (KI-VideoQA), where models must retrieve and integrate external information with visual understanding to generate accurate responses. This work presents the first attempt to (1) study multi-modal retrieval-augmented generation for KI-VideoQA, and (2) go beyond multi-choice questions by studying open-ended questions in this task. Through an extensive empirical study of state-of-the-art retrieval and vision language models in both zero-shot and fine-tuned settings, we explore how different retrieval augmentation strategies can enhance knowledge integration in KI-VideoQA. We analyze three key aspects: (1) model's effectiveness across different information sources and modalities, (2) the impact of heterogeneous multi-modal context integration, and (3) model's effectiveness across different query formulation and retrieval result consumption. Our results suggest that while retrieval augmentation generally improves performance, its effectiveness varies significantly based on modality choice and retrieval strategy. Additionally, we find that successful knowledge integration often requires careful consideration of query formulation and optimal retrieval depth. Our exploration advances state-of-the-art accuracy for multiple choice questions by over 17.5\% on the KnowIT VQA dataset.
% % As a result of our analysis, we believe that KI-VideoQA systems require specialized approaches to multi-modal retrieval and context integration.
% % Finally, our evaluation in open-ended settings reveals important insights about Vision Language Model's ability to generate faithful and coherent responses while drawing on external knowledge.
% \end{abstract}


\begin{abstract}
Video question answering that requires external knowledge beyond the visual content remains a significant challenge in AI systems. While models can effectively answer questions based on direct visual observations, they often falter when faced with questions requiring broader contextual knowledge. To address this limitation, we investigate knowledge-intensive video question answering (KI-VideoQA) through the lens of multi-modal retrieval-augmented generation, with a particular focus on handling open-ended questions rather than just multiple-choice formats. Our comprehensive analysis examines various retrieval augmentation approaches using cutting-edge retrieval and vision language models, testing both zero-shot and fine-tuned configurations. We investigate several critical dimensions: the interplay between different information sources and modalities, strategies for integrating diverse multi-modal contexts, and the dynamics between query formulation and retrieval result utilization. Our findings reveal that while retrieval augmentation shows promise in improving model performance, its success is heavily dependent on the chosen modality and retrieval methodology. The study also highlights the critical role of query construction and retrieval depth optimization in effective knowledge integration. Through our proposed approach, we achieve a substantial 17.5\% improvement in accuracy on multiple choice questions in the KnowIT VQA dataset, establishing new state-of-the-art performance levels.
\end{abstract}

% \keywords{Video Question Answering; Retrieval Augmented Generation; Vision Language Models; Multi-Modal Question Answering}

% \begin{CCSXML}
% <ccs2012>
% <concept>
% <concept_id>10002951.10003317</concept_id>
% <concept_desc>Information systems~Information retrieval</concept_desc>
% <concept_significance>500</concept_significance>
% </concept>
% <concept>
% <concept_id>10010147.10010257</concept_id>
% <concept_desc>Computing methodologies~Machine learning</concept_desc>
% <concept_significance>500</concept_significance>
% </concept>
% </ccs2012>
% \end{CCSXML}

% \ccsdesc[500]{Information systems~Information retrieval}
% \ccsdesc[500]{Computing methodologies~Machine learning}

\maketitle

%!TEX root=main.tex

\section{Introduction}
% Decision-makers, analysts, data scientists, and policymakers frequently rely on data to draw conclusions and extract insights. This data-driven approach helps them identify actionable recommendations aimed at influencing an outcome of interest, such as increasing product satisfaction or income levels or decreasing the likelihood of experiencing serious health conditions \cite{galhotra2022hyper,lakkaraju2016interpretable,agrawal1994fast}. 
\revc{Prescriptions, or actionable recommendations, are commonly generated across various fields to influence key outcomes such as improving product satisfaction, enhancing economic policies, or increasing business efficiency. 
%Decision- or policy-makers, analysts, data scientists, and 
Policymakers in government, decision-makers in businesses, and data scientists in various fields, often rely on data-driven approaches to identify 
%actionable recommendations 
potential actions to influence an outcome of interest, such as increasing income levels or loan approval rates}.
% , or decreasing the likelihood of experiencing serious health conditions. 
%
While association or prediction-based methods are extensively used in practice to draw useful insights from data, they typically identify correlations among variables and may fail to reveal the underlying causal factors, i.e., which actions may result in an improved outcome, needed for informed decision-making. 
%For recommendations to be truly impactful, there must be a clear  explanation that justifies why a particular decision is appropriate for a specific subpopulation~\cite{sun2021treatment,plecko2022causal}. 

\emph{Causal analysis} or {\em causal inference}, therefore, is considered one of the most important requirements to generate prescriptions that are {\em actionable} and aligned with human reasoning~\cite{imbens2024causal}. Causal inference, and in particular {\em observational studies} for causal inference on collected data (when controlled trials are impossible due to cost or ethical reasons), have been extensively studied in the statistics and artificial intelligence (AI) literature for several decades \cite{rubin2005causal, pearl2009causal}. Motivated by this foundational work on causal inference, the notion of causality has also influenced the field of database research. The causal models from AI have been extended to relational databases \cite{salimi2020causal},  and causality has been incorporated into various data management tasks such as finding responsibilities of inputs toward query answers ~\cite{meliou2010causality, meliou2009so, meliou2014causality}, explanations for query answers \cite{roy2014formal, DBLP:journals/pacmmod/YoungmannCGR24}, data discovery~\cite{galhotra2023metam,youngmann2023causal}, data cleaning~\cite{pirhadi2024otclean,salimi2019interventional}, hypothetical reasoning \cite{galhotra2022causal}, and large system diagnostics~\cite{markakis2024sawmill,causalsim,sage, gudmundsdottir2017demonstration}. 


\revc{If-then rules are generally considered interpretable by humans~\cite{lakkaraju2016interpretable,guidotti2018local,van2021evaluating,pradhan2022interpretable,chen2018optimization}.
We give a concrete example of the difference between association and causation in generating prescriptions or recommended actions in the form of if-then rules below}:
\begin{example}	%
\label{example:ex1} {\bf Importance of causal prescriptions:}
Consider the Stack Overflow (SO) annual developer survey
\cite{stackoverflowreport}, where respondents from around the world answer
questions about their jobs and demographics. A sample of the dataset \reva{with a subset of the
attributes (there are 20 attributes)} is presented in \cref{tab:data}.
%
Alice, a researcher in the United Nations (UN) finance department, is interested in discovering ways to increase the salaries of high-tech employees worldwide. She is looking for a set of actionable recommendations 
%(that we call a prescription rules) 
to raise the overall average salary.
%
Using association-based approaches~\cite{chen2018optimization,lakkaraju2016interpretable}, she may discover that individuals residing in the US who identify as straight or heterosexual tend to earn higher salaries (see \cref{exp:quality} for full details). However, this observation merely indicates a correlation: people living in the US, for example, generally earn more than those outside the country. Their comparatively higher salaries are primarily attributable to the country's economy and are unrelated to their sexual orientation. Thus, this observation cannot be used as a prescription rule to increase salary. 
Our causal analysis, on the other hand, reveals that individuals aged 25-34 with dependents would benefit from working as front-end developers.
This results in a \$44,009 annual salary increase on average. \reva{Another observation is that students should pursue an
undergraduate major in CS. %Computer Science (CS). 
This can boost their salary by \$22,174 per year} (see details in \cref{sec:casestudy}).
\end{example}

%It has been incorporated into various tasks including . 
%Causal interventions are often more relatable and easier to understand, as they offer insight into the underlying reasons behind the recommendations and allow unraveling complex cause-effect relationships that govern our world~\cite{pearl2009causality}. Furthermore, causal interventions often have long-lasting effects~\cite{imbens2024causal}.

%, making it essential that the prescribed actions are not only actionable but also 

%causally consistent. 

%Decision makings, in particular, high-stak

\cut{
In this work, {we study the problem of generating causal insights (referred to as \emph{prescription rules}), which serve as actionable recommendations} to improve an outcome of interest.
Recent works have introduced causality to the field of database research~\cite{meliou2010causality,  meliou2014causality,salimi2020causal,10.14778/3554821.3554902}. It has been incorporated into various tasks including data discovery~\cite{galhotra2023metam,youngmann2023causal}, data cleaning~\cite{pirhadi2024otclean,salimi2019interventional}, and large system diagnostics~\cite{markakis2024sawmill,causalsim,sage, gudmundsdottir2017demonstration}. 
We propose using causal inference to generate prescription rules that are both actionable and justifiable.
}

While generating prescriptions based on causal inference may help in robust decision-making, causal prescriptions that solely consider the betterment of an outcome (like salary) are not enough in practice. 
It is well-known that decision-making in many high-stake applications (like hiring policy, or policy for approving loans by banks) may lead to disparate societal or economic impact on different sub-populations. 
As a shocking example from a recent work called 
%For example, 
CauSumX~\cite{DBLP:journals/pacmmod/YoungmannCGR24} that generates a set of causal explanations for an aggregated view, the explanations generated %by CauSumX %recommendations which 
suggest that male individuals do a Bachelor's degree to increase their salary while %suggesting that 
being an unmarried woman 
%the recommendation for women includes getting married 
has the most adverse effect on salary
(borrowed directly 
from Fig.~19 in~\cite{youngmann2024summarizedcausalexplanationsaggregate}). 
%We demonstrate the advantage of using causal reasoning to generate actionable recommendations and the limitations of not considering fairness requirements in the following example. 
We explored this further in the context of generating prescriptions and observed that prescriptions that are not fairness-aware can generate unfair outcomes to some subpopulations which we refer to as the {\em protected group}. Examples include women, Black, Latino, or Native Americans, individuals with a disability, countries with a weaker economy, or other protected groups specific to an application. %Here is a concrete example:


% Understanding the causal factors behind these recommendations is crucial to ensuring that decisions lead to fair and equitable outcomes, particularly in sensitive applications where biased decisions can perpetuate or even exacerbate societal inequalities.
% While prior work has extensively explored techniques for association rule mining~\cite{kumbhare2014overview}, recent efforts have focused on deriving causal explanations for individual data points or entire datasets~\cite{salimi2018bias,youngmann2022explaining,ma2023xinsight}. Although some of these methods produce causally consistent insights, the absence of fairness considerations in the process can lead to unfair outcomes, further reinforcing existing biases. For example, CauSumX~\cite{DBLP:journals/pacmmod/YoungmannCGR24} generates causal recommendation which suggest male individuals to do a Bachelor's degree to increase salary while the recommendation for women include getting married (borrowed directly from Figure~19 in the paper~\cite{youngmann2024summarizedcausalexplanationsaggregate}). 





%\emph{Causal inference} has been thoroughly studied in AI and Statistics~\cite{pearl2009causal,rubin2005causal}. Causal analysis is a vital tool in determining the effect of a \emph{treatment} on an \emph{outcome}, and has been used in decision-making in medicine \cite{robins2000marginal}, economics \cite{banerjee2011poor}, biology \cite{shipley2016cause}, and in high-stakes areas such as identifying the root causes of failures in critical infrastructure systems to prevent catastrophic outcomes. Recent works have introduced causality to the field of database research~\cite{meliou2010causality,  meliou2014causality,salimi2020causal,10.14778/3554821.3554902}. It has been incorporated into various tasks including data discovery~\cite{galhotra2023metam,youngmann2023causal}, query result explanation~\cite{salimi2018bias,youngmann2022explaining,DBLP:journals/pacmmod/YoungmannCGR24}, and large system diagnostics~\cite{markakis2024sawmill,causalsim,sage, gudmundsdottir2017demonstration}. We propose leveraging causal inference to generate interpretable and justifiable insights (referred to as \emph{prescription rules}), which serve as actionable recommendations to improve an outcome of interest. Causal reasoning is considered one of the most important requirements,  to generate insights that are actionable and aligned with human reasoning.




\begin{table*}[]
\footnotesize
    \centering
    	\caption{\textnormal{A subset of the Stack Overflow dataset.}}
         \label{tab:data}
    	% \vspace{-4mm}
  			\begin{tabular}[b]{|l|l|l|c|l|l|c|l|c|}
  			
				%\multicolumn{9}{c}{\textbf{Users}}\\ 
				\hline

				\textbf{ID}
    
    % \textbf{Country}& \textbf{Continent} 
    
    &\textbf{Gender} &\textbf{Ethnicity}&
				\textbf{Age} &\textbf{Role} &
				\textbf{Education} &\textbf{Country}&\textbf{Undergrad Major}&\textbf{Salary}
				\\ \hline

				1 &Male&White&26&Data Scientist & PhD& US&Computer Science&180k\\
    		2 &Non-binary&White&32&QA developer & Bachelor's degree& US&Mechanical Eng.&83k\\

 3 &Male&South Asian&29&C-suite executive  & Bachelor's degree & India&Computer Science&24k\\

  % 4 &Female&South Asian&25&Back-end developer  & Master's degree & India&Mathematics&7.5k\\

  4 &Female&East Asian&21&Back-end developer & Bachelor's degree & China&Computer Science&19k\\
  

        % $\ldots$ &  $\ldots$&  $\ldots$&  $\ldots$&  $\ldots$&  $\ldots$&  $\ldots$&  $\ldots$&  $\ldots$&  $\ldots$&  $\ldots$\\
    \hline
			\end{tabular}
            \vspace{-5mm}
\end{table*}




\begin{example}	%
\label{example:ex2}
{\bf Importance of fair prescriptions:}
Continuing Example~\ref{example:ex1}, while those causal prescription rules are highly beneficial for the overall population, they are considerably less effective for individuals residing in countries with a low GDP (indicating a weaker economy). For this group, the average expected increase in salary is only approximately \$13,000 per year (in contrast to \$44,009 for the entire group). % \sr{add which rule 44k or 25k} 
Consequently, implementing these rules would exacerbate the disparity between those living in countries with strong economies and those in countries with weaker economies.
\end{example}




% Our objective is to generate a small set of prescription rules aimed at increasing (or decreasing) an outcome of interest. This is framed as an optimization problem where the goal is to select the fewest prescription rules that maximize utility (i.e., the expected increase or decrease in the outcome). However, 

The example above shows that focusing solely on maximizing utility (\revc{i.e., increasing income}) can result in a scenario where only some of the population receive significant improvement, while others experience no benefit (\revc{only a small benefit for individuals from countries with weaker economies in our example}). Additionally, even if a large portion of the population receives recommendations, a protected subpopulation might not share the benefits and, worse, their situation could deteriorate, exacerbating inequalities.

Examples~\ref{example:ex1} and \ref{example:ex2} show that it is crucial to provide recommendations that are (1) {\em causal} for the outcome (beyond associations),  and (2) also {\em fair or equitable} in terms of the outcome for both the protected and non-protected groups. While recent work in database research
has focused on deriving {\em causal explanations} for individual data points, aggregated view, or entire datasets~\cite{salimi2018bias,youngmann2022explaining,ma2023xinsight, DBLP:journals/pacmmod/YoungmannCGR24}, and in particular \cite{DBLP:journals/pacmmod/YoungmannCGR24} has considered generating a set of causal explanations for an aggregated view that resemble a ruleset, 
%Although some of these methods produce causally consistent insights, 
the absence of fairness considerations in generating these causal explanations can lead to unfair outcomes for the protected group.
%further reinforcing existing biases.


%\red{We, therefore, enable users to incorporate various \emph{coverage and fairness constraints} along with the overall objective of improving an outcome of interest. }

\medskip
\noindent
\textbf{Our contributions.~} 
Motivated by the dual goals of generating causal and fair prescriptions for the betterment of an outcome, we introduce a {\em fairness-aware framework leveraging causal reasoning for generating a set of actionable prescription rules (ruleset)} called \sysName\ (\underline{Fair} \underline{CA}usal \underline{P}rescription).
%
Following research on fairness in data management~\cite{stoyanovich2020responsible,galhotra2022causal}, we assume the existence of a \emph{protected subpopulation}, defined by an attribute such as gender or race for people, or GDP of a country. Motivated by the causal explanation rules for an aggregated view \cite{DBLP:journals/pacmmod/YoungmannCGR24}, each prescription rule in our ruleset applies to a sub-population defined by a {\em grouping attribute}, and prescribes a {\em treatment or intervention} to improve the {\em outcome} for this sub-population. Fairness constraints ensure that the expected utility of the protected population is {\em comparable} to the utility of the unprotected individuals. We borrow the notions of \emph{group and individual fairness} from the fairness literature but tailor them for prescription rules. In addition to the fairness constraints, our coverage constraints ensure that a substantial fraction of the population and protected subpopulation receives at least one recommendation. 
%We demonstrate how such constraints ensure that the generated rules apply to a large portion of the population and ensure fairness through the following example.

\begin{example}
\label{ex:intro_example_3}
Continuing Examples~\ref{example:ex1} and \ref{example:ex2}, Alice uses our proposed system, called \sysName, to impose fairness and coverage constraints to discover useful and equitable recommendations for increasing salaries worldwide. In particular,
Alice chooses to implement a coverage constraint to ensure that the selected rules apply to a significant portion of people worldwide, including a sufficiently large number of individuals from countries with low GDP (the protected group). She also imposes a fairness constraint to ensure that the expected gains for both protected and non-protected groups are comparable.
\reva{She discovers, for example, that for individuals with 6-8 years of coding experience (a subpopulation comprising 21\% of the entire dataset and 25\% of the protected group), pursuing a bachelor’s degree in computer science will increase the expected salary by $\$14.9k$ for protected and by $\$17.8k$ for non-protected}. (See \cref{sec:casestudy} for more details.) This prescription rule applies to a large portion of the population and ensures fairness by providing a similar expected gain for both protected and non-protected groups, and the allowed difference of outcomes between these two populations may be adjusted by choosing appropriate thresholds in the fairness definitions. 
\end{example}


\noindent
Our main contributions are as follows. \\
%\begin{itemize}[leftmargin=*,topsep=0pt]
{\bf (1)} We {\bf develop a framework that generates a set of prescription rules to enhance an outcome of interest (Section~\ref{sec:problem})}. A prescription rule consists of a \emph{grouping pattern} and an \emph{intervention pattern}, representing the target subpopulation and the actionable recommendation for that group, respectively. The strength of the {\em conditional causal effect} (Section~\ref{sec:background-causal}) of this intervention on the subgroup is used to measure the expected utility of a rule. Our objective is to identify the smallest set of rules that maximizes overall expected utility. We refer to this problem as the {\em \probName} problem.
We adopt several notions of fairness (individual vs. group, statistical parity vs. bounded group loss) from the literature to define the {\bf fairness constraints} for our problem. In addition, {\bf coverage constraints} (for individual rules or for a group) ensure that the solution for the \probName\ problem is applied to a sufficient number of individuals and to minimize inequalities. We show NP-hardness for different variants of the problems and properties (matroid) useful in our algorithms. 
%We establish several definitions for group and individual fairness constraints tailored for prescription rules.
\smallskip
    \par
    \noindent
{\bf (2)} We {\bf develop a general three-step algorithm named \sysName to solve the optimization problem of selecting a fair prescription ruleset (Section~\ref{sec:algo})}. The first step involves mining frequent grouping patterns using the Apriori algorithm~\cite{agrawal1994fast}. In the second step, we employ a lattice-based algorithm to find high utility and fair intervention patterns for grouping patterns identified in the previous step. Finally, the third step applies a greedy approach to determine a solution. \sysName\ can be easily adapted to accommodate all variants of the \probName\ problem.

\smallskip
\par
\noindent
{\bf (3) We provide a detailed  case study  (Section~\ref{sec:casestudy}) and experimental analysis (Section~\ref{sec:experiments}) to evaluate our framework and algorithms.}
The case study shows the qualitative difference of different variants of our problem for different choices of the fairness and coverage constraints. The experiments include two datasets, three baselines, and 18 variations of our problem with different constraints. Our evaluations suggest that fairness may come at the cost of expected
utility for everyone. However, without fairness constraints, we often observe a significant disparity between the protected and non-protected. We also observe that
achieving individual fairness is harder than group fairness,
as most high-utility or high-coverage rules are unfair. Lastly, we show that \sysName\ can generate  prescription rules over large datasets in a reasonable time. 

%\end{itemize}


%\paragraph*{Paper outline} 
We discuss related work in \cref{sec:related}, review background on causal inference in \Cref{sec:background-causal}, %and our problem formulation can be found in \cref{sec:problem}. Our algorithmic framework is presented in \cref{sec:algo}. A case study demonstrating the impact of different constraint configurations on the solution is given in \cref{exp:problem_variants}, and our experimental evaluation is detailed in \cref{sec:experiments}. Finally, we 
and discuss the limitations of our framework and future work in \cref{sec:conc}.

% \noindent
% \boxed{\parbox{\columnwidth}{$\bullet$ 
% For people with a professional degree, move to the United Kingdom
%  (coverage = 435 (20), coverage-protected = 20 (13), utility = 186855, utility-protected = 0.)\\
% $\bullet$ For graphic developers, move to the	United States
%  (coverage = 116 (29), coverage-protected = 8 (2), utility = 169431, utility-protected = 0).\\
% $\bullet$ For people who have no formal education, move to the United States
%  (coverage = 123 (34), coverage-protected = 7 (2), utility = 206742, utility-protected = 0).\\
% % \textcolor{red}{size = 38, length = 76, overlap = 64029181, utility = 1659307}\\
% \textcolor{blue}{overall coverage =674, expected utility = 187485
% coverage-protected = 35, expected utility-protected = 0}
% \sr{should mention protected group, and possibly not mention coverage in the intro or just intuitively like high coverage}
% }}


% Alice notes that although these rules result in a \$187,485 increase in the overall salary for those to whom they apply, they only affect a small fraction of the population, specifically 674 individuals. Additionally, although the expected salary increase is substantial, there is no expected increase in salary for non-males, a subpopulation of particular interest to Alice. In other words, applying these rules would result in no gain for non-males.
% \end{example}

% \begin{example}[Episode 2 - coverage and fairness constraints]
% Alice introduces coverage and fairness constraints to ensure that enough people will benefit from the rules and that they will be \emph{fair} with respect to non-males. Specifically, she demands that the benefit for a randomly chosen individual to whom one of the rules applies is nearly the same as the benefit for a randomly chosen individual who identifies as non-male and to whom one of the rules applies.

% After adding these constraints, \sysName\ recommends the following set of prescription rules:



% \noindent
% \boxed{\parbox{\columnwidth}{$\bullet$ 
% For people who have no formal education, move to the United States
%  (coverage = 123 (34), coverage-protected = 7 (2), utility = 206742, utility-protected = 0)\\
% $\bullet$ 
% For females, change role to	DevOps specialist (coverage = 2256 (47), coverage-protected = 2256 (47), utility = 90023, utility-protected = 90023).\\
% $\bullet$ For people with a Master's degree, move to the	United States
%  (coverage = 9097 (2222), coverage-protected = 642 (236), utility = 85390, utility-protected = 84201).\\
% % \textcolor{red}{size = 38, length = 76, overlap = 64029181, utility = 1659307}\\
% \textcolor{blue}{overall coverage =11476	
% , expected utility = 87601,
% coverage-protected = 2905, expected utility-protected = 88519}
% }} 







% \begin{figure}[t]
%         \centering
%         \begin{minipage}[b]{1.0\linewidth}
%             \small
%             \begin{tcolorbox}[colback=white]
%             \vspace{-2mm}
% $\bullet$ For backend developers, the treatment with the highest effect on salary is “Country = US” effect size = 78646
% \begin{itemize}
%     \item For non-male the effect is only: 59429
%     \item For male the effect is 80454
% \end{itemize}

% $\bullet$ For frontend developers, the treatment with the highest effect is :Formal Education = Bachelor's degree” effect size: 17340
% \begin{itemize}
%     \item For white the effect is 33464
%     \item For non-white the effect is 15320
% \end{itemize}


% $\bullet$ For people in Europe, the treatment with the highest effect on salary is “DevType = C-suite executive” effect size = 53254
% \begin{itemize}
%     \item For white the effect is 55112
%     \item For non-white 35249
% \end{itemize}



%             \vspace{-2mm}
%             \end{tcolorbox}
%         \end{minipage}%%
%          % \vspace{-4mm}
%         \caption{Set of prescription rules.}
%         \label{fig:so-explanation}
%     \end{figure}




\section{Related Work}

Our work is informed by foundational paradigms in visual analytics including exploratory data analysis and exploratory search (Sec. \ref{sec:related-eda}). 
We also build on many prior methods for subgroup analysis from data mining and machine learning, the design space of which we describe in Sec. \ref{sec:related-subgroup-analysis}.

\subsection{Exploratory Data Analysis and Search}
\label{sec:related-eda}

\citeauthor{tukey_exploratory_1970} describes \textit{exploratory data analysis} (EDA) as ``looking at data to see what it seems to say''~\cite{tukey_exploratory_1970}.
EDA is therefore distinct from hypothesis testing, or confirmatory data analysis, in its emphasis on generating insight from the \textit{data} rather than prior knowledge and expectations.
Many systems for EDA are informed by interaction techniques for \textit{exploratory search}, in which people navigate through and query information resources to build understanding about some latent concept of interest~\cite{white_exploratory_2009}.
In these interactive search settings, features such as sorting, filtering, and faceted searches~\cite{yee_2003_faceted} play a key role in helping users uncover useful information.
Applied to EDA, these techniques can enable steerable recommendations of how to visualize data features~\cite{wongsuphasawat_voyager_2016,lee_2021_lux} or efficient overviews of text data~\cite{felix_texttile_2017}.
We draw inspiration from these search techniques in the design of Divisi.

A wide variety of EDA techniques have been developed for different types of data, including small-scale tabular settings~\cite{wongsuphasawat_voyager_2016,lee_2021_lux}, high-dimensional data~\cite{Liu2017}, text data~\cite{felix_texttile_2017}, and general unstructured data~\cite{Smilkov2016}.
It is often easiest to find useful insights in EDA on tabular data because the features are generally intrinsically interpretable. 
In contrast, for text or image data the ``features'' (words or pixels) may not have any meaning on their own, making it difficult to interpret what instances have in common.
As datasets grow larger, there may also be many different subtypes within the dataset, limiting the insight provided by top-level metrics and distributions.
For this reason, many prior works aim to mitigate the complexity of large, high-dimensional datasets by automatically deriving semantically meaningful features or ``concepts'' to bootstrap the analysis process~\cite{suresh_kaleidoscope_2023,kim_interpretability_2018}.
Alternatively, some systems allow the user to define their own features of interest~\cite{wu_errudite_2020,cabrera_zeno_2023}.
However, these methods require the user to already know roughly what concept they are looking for, limiting their opportunities to explore and find unexpected patterns.
Our work relies on the presence of interpretable tabular features for every instance; however, we design for use cases in which the data scientist wants to find the relevant features out of a large set of potentially-meaningful set of descriptors.
This can afford the simplicity of working with tabular data while not restricting the analysis to the user's prior hypotheses.

% \begin{enumerate}
%     \item EDA \cite{tukey_exploratory_1970}, exploratory search \cite{white_exploratory_2009,marchionini_exploratory_2006} - what are the activities involved in each?
%     \item Faceted browsing~\cite{yee_2003_faceted}, sort and filter
%     \item More modern notions of EDA: text exploration, image exploration, embedding analysis
%     \item Benefits of traditional EDA
%     \item Challenges in extending the traditional notions of EDA to modern, large-scale datasets: multiple driving phenomena or subtypes, many variables (possibly more than can be reasoned about), uninterpretable variables
% \end{enumerate}

\subsection{Tools for Subgroup Analysis}
\label{sec:related-subgroup-analysis}

Sometimes called slice discovery, cluster analysis, or rule mining, subgroup analysis is an important part of data science that can help people understand phenomena in a dataset~\cite{liu_exploratory_2020,gamberger_active_2003}, help model builders diagnose and fix issues~\cite{piorkowski_aimee_2023,zhang_drml_2022,cabrera2021deblinder,robertson_angler_2023,zhang_sliceteller_2022, jain_distilling_2022}, explain model predictions~\cite{ribeiro_anchors_2018}, or even be used in place of a model~\cite{lavrac_decision_2004}.
However, it is usually all but impossible to define clear-cut, interpretable subgroups that exactly capture the outcome of interest (e.g., model errors), creating a design space of trade-offs for how to produce useful insights.
A wide array of subgroup analysis techniques have been developed, varying across several dimensions:

\textit{Conceptualization of a subgroup.} Differences in data types, user needs, and algorithm formulations give rise to different definitions of what a subgroup is. 
At the most subjective level, subgroups can be any semantic human-readable description of instances, regardless of whether it is encoded in the data, such as ``images of people with glasses''~\cite{cabrera2021deblinder}. 
They can also be defined by numerical proximity to some conceptual entity, such as a direction or neighborhood around an instance in an embedding space~\cite{eyuboglu_domino_2022,kim_interpretability_2018,ahn_escape_2023}. 
Finally, subgroups can be defined more precisely by constructing rules for membership, such as textual patterns~\cite{wu_errudite_2020,robertson_angler_2023} or predicates on tabular features~\cite{kwon_rmexplorer_2022,hurley_interactive_2022}. 
While the latter results in the clearest subgroup definitions, it also requires crafting or mining high-quality rules.

\textit{Source of initiative.} Many subgroup discovery methods require the data scientist to define subgroups themselves, using the affordances of the various subgroup concepts described above~\cite{cabrera_zeno_2023,wu_errudite_2020,kwon_rmexplorer_2022}. 
These methods are flexible and often provide useful insights on known areas of interest, but it can be difficult to find \textit{new} subgroups without spending time perusing individual instances. 
Algorithm-initiated approaches can provide a strong initial set of subgroups to explore~\cite{chung_slice_2020,zhang_sliceteller_2022}; however, these techniques heavily focus on producing the most relevant set of subgroups in the initial query.
There is currently a lack of \textit{mixed-initiative} subgroup analysis approaches that allow the user to interactively steer the algorithm's output to produce more relevant slices.
When subgroup analysis tools do offer mixed-initiative interactions, it is typically to \textit{refine} the subgroup definitions~\cite{slyman_vlslice_2023} or to characterize and assess their validity~\cite{hurley_interactive_2022}, both of which are supported in Divisi within our broader mixed-initiative workflow.

\textit{Visualization.} Designs to visualize and compare subgroup-level data characteristics are largely dependent on the way the subgroups are conceptualized.
For example, most clustering-based tools use dimensionality reduction scatter plots, which provide a valuable overview of the dataset but are difficult to map to data features~\cite{Liu2019,slyman_vlslice_2023,xuan_attributionscanner_2024,suresh_kaleidoscope_2023,sivaraman_emblaze_2022}.
For handcrafted subgroups on tabular data, brushable histograms can serve as controls to define predicates that are then visualized in strip plots~\cite{cabrera_fairvis_2019} or domain-specific visualizations~\cite{kwon_rmexplorer_2022}.
To visualize rule-based subgroups generated by an algorithm, table representations with sparkline charts or glyphs are often preferred as they can efficiently present summary statistics over many subgroups~\cite{kahng_visual_2016,kerrigan_slicelens_2023,zhang_sliceteller_2022}.
Similarly, UpSet plots~\cite{2014_infovis_upset} provide a dense visual representation of metrics within multiple set intersections.
Divisi combines several of these elements, including the scatter plot and the subgroup table with sparklines, with novel adaptations for tasks such as assessing overlap and coverage.

\textit{Algorithmic approach.} We can divide prior algorithms for subgroup discovery into four broad classes: lattice search, frequent itemsets, classification, and clustering.
Lattice search methods, such as Slice Finder~\cite{chung_slice_2020,sagadeeva_sliceline_2021}, \textsc{Premise}~\cite{hedderich_label-descriptive_2022}, and the Nugget Browser~\cite{guo_nugget_2011}, perform combinatorial search of a space of discrete rules to find those that most satisfy the algorithm's desirability criteria.
These methods can result in easily-interpretable subgroups, but they tend to scale poorly to datasets with hundreds or thousands of features due to combinatorial explosion.
Frequent itemset-based methods, such as DivExplorer~\cite{pastor_looking_2021} and the method developed by \citeauthor{suzuki_rule_2023}~\cite{suzuki_rule_2023}, draw on efficient algorithms from data mining such as FPgrowth, then score and rank the returned subgroups.
Similarly, these methods work best with a relatively small number of possible feature combinations.
Classification-based methods can overcome some of the performance considerations of lattice search and frequent itemset approaches \cite{yuan_isea_2022,yuan_visual_2022}, but their results often require significant work to interpret.
Finally, clustering-based methods aim to group together instances by similarity in a high-dimensional space such as a learned embedding~\cite{zhang_manifold_2019,eyuboglu_domino_2022,kim_interpretability_2018}.
Though these methods can provide insight into unstructured data, they often require a trained model, sometimes one that is jointly trained with natural-language representations, limiting their applicability.
Moreover, like classification methods, the resulting clusters and concepts are not always straightforward to interpret because of their reliance on learned embeddings.
Divisi builds on this extensive space of previous algorithms, adopting a modified lattice search approach that addresses scalability issues using approximation.
While it is most directly applicable to tabular datasets as a result, we propose ways to use it in unstructured data contexts in Sec. \ref{sec:use-case}.

Because there are so many alternative techniques for subgroup analysis, each with their own specific associated data types and challenges, there is not a clear consensus of what approach should be applied to a given problem.
As a result, data scientists may not typically include subgroup analysis in the exploratory phase of their workflows.
Our work aims to make it easier to perform subgroup analyses interactively within a typical programming environment, and we assess in our study whether they might find such capabilities useful in their daily work.

% \begin{enumerate}
%     \item Why subgroup analysis? It can be used in place of classifiers ~\cite{lavrac_decision_2004}, it is useful for experts to understand phenomena in a dataset \cite{liu_exploratory_2020,gamberger_active_2003} or explain predictions~\cite{ribeiro_anchors_2018}, and it can help model builders diagnose and fix issues in their models~\cite{piorkowski_aimee_2023,zhang_drml_2022,cabrera2021deblinder,robertson_angler_2023,zhang_sliceteller_2022, jain_distilling_2022}.
%     \item There are many different subgroup analysis approaches, which vary in how the subgroup is conceptualized:
%     \begin{itemize}
%         \item Conceptualization of a subgroup: can be defined by a semantic human-readable description~\cite{cabrera2021deblinder}, defined by a pattern for text data~\cite{wu_errudite_2020,robertson_angler_2023,hedderich_label-descriptive_2022}, or a rule based on tabular features~\cite{kwon_rmexplorer_2022,hurley_interactive_2022}, based on proximity to some concept or a direction in an embedding space~\cite{suresh_kaleidoscope_nodate}, or based on clusters~\cite{Cavallo2019}
        
%     \end{itemize}
%     \item Source of initiative: often entirely human-initiated~\cite{cabrera_zeno_2023,wu_errudite_2020,kwon_rmexplorer_2022}, or algorithm-initiated. Slice discovery involves approaches to automatically generate the subgroups of interest by mining them from patterns in the data. Algorithmic approaches vary:
%     \begin{itemize}
%         \item rule mining - enumerate possible combinations of features and score them~\cite{chung_slice_2020,sagadeeva_sliceline_2021}
%         \item frequent itemsets~\cite{pastor_looking_2021,zhang_sliceteller_2022}
%         \item embedding-representation approaches~\cite{eyuboglu_domino_2022,kim_interpretability_2018}. For unstructured data we can also use cross-modal representation spaces to label clusters~\cite{slyman_vlslice_2023,eyuboglu_domino_2022}
%     \end{itemize}
%     \item Literature gap: mixed-initiative systems for subgroup discovery. Some interactive systems incorporating subgroup discovery allow users to refine the subgroup definitions~\cite{slyman_vlslice_2023,} or to investigate the characteristics of the subgroups and assess their validity~\cite{hurley_interactive_2022}. few systems have been developed that allow 
% \end{enumerate}
% rule-based explanations

% \gabis{Where do we define our notion of framing and relevant terms? Will that happen in the intro? For example here we use the term ``sentiment shifts'', which I think requires defintion.}\gili{done in intro}

% \gabis{Recurring comment - we should change tense to present, while most of the paper is currently in the past tense, I indicated this in some places, but should verify throughout.}

% \begin{figure*}[htbp]
    \centering
    % First Subfigure
    \begin{subfigure}{0.49\textwidth} % Adjust width as needed
        \centering
        \includegraphics[width=\textwidth]{images/orig_negative_models_distribution.png} % Replace with your image path
        \caption{Sentences that are \textbf{negative} in their original form.}
        \label{fig:negative-flip}
    \end{subfigure}
    % \hfill % Adds horizontal space between subfigures
    % Second Subfigure
    \begin{subfigure}{0.49\textwidth}
        \centering
        \includegraphics[width=\textwidth]{images/orig_positive_models_distribution.png} % Replace with your image path
        \caption{Sentences that are \textbf{positive} in their original form.}
        \label{fig:positive-flip}
    \end{subfigure}
    \caption{Proportion of sentences for which LLMs flipped sentiment, became neutral, or retained the original sentiment when presented with opposite sentiment framing. For example, this measures the percentage of sentences originally labeled as positive, that were labeled as negative after applying negative framing (and vice versa).
    }
    \label{fig:flip-proportion}
\end{figure*}


Our dataset curation consists of three steps, as depicted in Figure~\ref{fig:fig1}. First, we collect natural, real-world statements, with some clear sentiment, either positive or negative (\S\ref{sec:base-statements}; e.g., ``I won the highest prize'' as positive). Next, 
we reframe each statement by adding a prefix or suffix conveying the opposite sentiment
% for each statement, we add a framing that conveys the opposite sentiment to the base statement 
(\S\ref{sec:adding-framing}; e.g., ``I won the highest prize, although I lost all my friends on the way''). Finally, we collect large-scale human annotations via crowdsourcing, to label the sentiment shifts when wrapping the statements with the opposite framing (\S\ref{sec:human-annotations}; e.g., labeling ``negative'' the statement ``I won the highest prize, although I lost all my friends on the way''). 
%\gabis{I think we can remove the textual examples here to save space}

The complete dataset consists of 1000 statements, in which 500 are statements that their base form has positive sentiment, and 500 are base negative statements. 




\subsection{Collecting Base Statements}\label{sec:base-statements}
First, we collect base statements, which convey a clear sentiment, either clearly positive or clearly negative statements. We use \spike{} -- an extractive search system, which allows to extract statements from real-world datasets~\cite{taub-tabib-etal-2020-interactive}.
%\gabis{there's also a citation for spike}.\footnote{~\url{https://spike.apps.allenai.org}} 
Specifically, we collect statements from Amazon Reviews dataset, which are naturally occurring, sentiment-rich, texts but are less likely to trigger strong preexisting biases or emotional reactions, which may be a confound for our experiment.\footnote{~\url{https://spike.apps.allenai.org/datasets/reviews}} 
% \gabis{Why did we use this specifically? I think once we write the intro it would be good to relate to what we wrote there and how this domain is relevant.}
\begin{figure}[tb!]
    \centering
    \includegraphics[width=\linewidth]{images/roberta_score_before_after_framing.png}
    \caption{Distribution of sentiment scores before and after applying opposite-sentiment framing, as detailed in Section~\ref{sec:adding-framing}. Prior to framing, base sentences exhibit a clear polarity (positive or negative), whereas the application of opposite framing introduces ambiguity, shifting the sentiment scores toward a less distinct polarity.}
    \label{fig:pos-score-dist}
\end{figure}


Using \spike, we extract ${\sim}6k$ statements that fulfilled our designated queries, which we found correlated with clear sentiment. We designed the queries to capture positive or negative verbs that describe actions with some clear sentiment (e.g., ``enjoy'' or ``waste''), or statements with positive or negative adjective, describing an outcome with a clear sentiment (e.g., ``good'' or ``nasty''). The patterns and queries used for extraction are detailed in Appendix~\ref{sec:appendix-spike}.
% \gabis{needs more details, what are our queries? What were we aiming for? I understand that at a high level we're looking for clear sentiment, but how do we achieve this via lexical-syntactic queries?}. 
Next, we run in-house annotations to label and filter the extracted statements, to handle negations or other cases where the statement does not convey a clear sentiment. 
The filtering process results in $1,301$ positive statements, and $1,229$ negative statements.


\subsection{Adding Framing}\label{sec:adding-framing}

To reframe the statements in our dataset, we use GPT-4~\cite{achiam2023gpt}.\footnote{We used the gpt-4-0613 version.} 
% \gabis{do we have more details about which GPT4? what date?}
% The model was asked to keep he base statement unchanged, and add some prefix or suffix, that can be either positive or negative, oppositely to the base statement sentiment (e.g., I won the highest proze, althoug I lost all my friends on the way). 
The input prompt includes a 1-shot example, followed by a task description ``Add a <SENTIMENT> suffix or prefix to the given statement. Don't change the original statement.'', where SENTIMENT is either ``positive'' or ``negative'', opposite to the base statement sentiment (i.e., positive framing for negative base statement, and vice versa).

Unlike the base statement, the conveying sentiment of reframed statements is more ambiguous and there is no one clear label, as shown in Figure~\ref{fig:pos-score-dist}.\footnote{Scores in Figure~\ref{fig:pos-score-dist} are given by a fine-tuned sentiment analysis model ~\url{https://huggingface.co/cardiffnlp/twitter-roberta-base-sentiment-latest}}
%as we present the sentiment scores assigned by a fine-tuned sentiment analysis model,\footnote{~\url{https://huggingface.co/cardiffnlp/twitter-roberta-base-sentiment-latest}} %that was shown to be state-of-the-art when fine-tuned on sentiment analysis~\cite{csanady2024llambert}. 
% We present the sentiment scores 
% before and after reframing. It shows that wrapping the statement with the opposite sentiment injects ambiguity to the overall sentiment, as the sentiment scores become more dispersed. 
The exhibeted ambiguity in sentiment allows us to measure to what extent LLMs' shifting sentiment after framing, and how correlated it is to human behavior.



% In Figure~\ref{fig:pos-score-dist}, \gabis{Is roberta SOTA? it's a bit old by now. Do we have a reference to back this up?}\footnote{RoBERTa, fine-tuned for sentiment analysis~\url{https://huggingface.co/cardiffnlp/twitter-roberta-base-sentiment-latest}} The base statement scores are predominantly centered around binary values, either strongly positive or strongly negative. In contrast, the sentiment scores after opposite framing are more dispersed, reflecting increased ambiguity in sentiment. 
% \gabis{I'm not sure if this paragraph belongs here, maybe should be a subsection on its own at the end of the section?}


\subsection{Collecting Human Annotations}\label{sec:human-annotations}

In the final step, we collect human annotations through Amazon Mechanical Turk to evaluate the framing effect in \name{} over human participants, providing a reference for comparison with LLMs.\footnote{\url{https://www.mturk.com}} 
Details about the annotation platform are elaborated in Appendix~\ref{sec:mturk-appendix}.

The complete dataset includes 1K statements, each annotated by five different annotators. Given our budget, we preferred to collect five annotations per statement, resulting in less statements, but providing a more robust scoring for the ambiguity of a statement.

% We select a pool of 10 qualified workers who successfully passed our qualification test, which consisted of 20 base statements (unframed), for which annotators were expected to achieve perfect accuracy. The estimated hourly wage for the entire experiment was approximately 14USD per hour. More details about the annotation platform can be found in Appendix~\ref{sec:mturk-appendix}. Given our budget, we preferred to collect five annotations per statement, resulting in less statements, but providing a more robust scoring for the ambiguity of a statement.

For the annotation process, each statement in our dataset is presented in its reframed version (i.e., positive base statements with negative framing and vice versa), to five different annotators. This setup generates, for each dataset instance, a score ranging from 0 to 5, representing the number of annotators that votes for the sentiment that aligns with the opposite framing, which means that the overall sentiment of the reframed statement has shifted from its base sentiment. For example, in Figure~\ref{fig:fig1}, the statement ``I won the highest prize, although I lost all my friends on the way'' is shown to have two annotators voting ``negative'', which aligns with the sentiment of the framing and not the base statement, so the label for that instance in \name{} would be 2 (sentiment shifts).

% \gabist{It is important to note that there is no definitive ``right'' or ``wrong'' label for these statements, as the opposite sentiment framing often renders the sentiment conveyed highly ambiguous.}
Instances with score near 0 indicate that annotators agree that the overall sentiment remains unchanged despite the opposite framing. Score closer to 5 indicates that annotators agree that reframing shifted the perceived sentiment, while score around 2-3 suggests that the opposite framing makes the sentiment ambiguous.



\section{Multi-Modal RAG for KI-VideoQA}
\label{sec:method}
In this work, we address the task of knowledge-intensive open-ended video question answering. Formally, given an input video $V$, a question $q$ about the video, and a corpus or knowledge source $\mathcal{K}$, the task is to answer the question. Note that in this task, questions cannot be merely answered by the given video; answering the question requires external knowledge. An example of this task is presented in Figure~\ref{fig:method_overview} (top left corner).

\paragraph{\textbf{Inputs}}
Let $V = [f_1, f_2, ..., f_t]$ be a video sequence of $t$ frames, where each frame $f_t \in \mathbb{R}^{h \times w \times 3}$ represents an RGB image with the height of $h$ and width of $w$ pixels. Each video is accompanied by subtitles ($S$) between the start and end timestamp for that video. Each question $q$ is a natural language sequence of $l$ tokens ${w_1, w_2, \cdots, w_l}$.

\paragraph{\textbf{Knowledge Sources}}
A knowledge source $\mathcal{K}$ is a collection of $m$ information items: $\mathcal{K} = \{k_1, k_2, ..., k_m\}$, where each information item may contain information needed to answer some questions. Each information item can be either textual (e.g., unstructured text documents) or visual (e.g., videos). This paper studies these and even multi-modal knowledge source situations which is a mixture of both textual and visual information items. 

\paragraph{\textbf{Output}}
Our work encompasses two distinct question answering tasks: (1) \textbf{Multiple Choice Questions}: Given a set of candidate answers $\mathcal{A} = {a^1, a^2, ..., a^K}$ where $K$ is typically equal to 4, the task is to select one correct answer $a^* \in \mathcal{A}$. (2) \textbf{Open-Ended Questions}: The expected output is a free-form text with the length of $p$ as the answer: $a = {y_1, y_2, \cdots, y_p}$, where each $y_i \in \mathcal{V}$ is a token from the vocabulary $\mathcal{V}$.


% \begin{enumerate}
%     \item \textbf{Multiple Choice Questions}: Given a set of candidate answers $\mathcal{A} = {a^1, a^2, ..., a^K}$ where $K$ is typically equal to 4, the task is to select one correct answer $a^* \in \mathcal{A}$. 
% %     The model outputs a probability distribution over the candidates:
% % \begin{equation}
% % P(a^k|\mathcal{V}, q, \mathcal{K}) \quad \text{for } k \in {1,...,K}
% % \end{equation}

%     \item \textbf{Open-Ended Questions}: The expected output is a free-form text with the length of $p$ as the answer: $a = {y_1, y_2, \cdots, y_p}$, where each $y_i \in \mathcal{V}$ is a token from the vocabulary $\mathcal{V}$.
% \end{enumerate}

Even though multiple choice questions for KI-VideoQA has been briefly explored in \citet{garcia2020knowit}, prior work has not studied retrieval-augmented solutions. Moreover, to the best of our knowledge, no prior work explored open-ended question answering in KI-VideoQA. 





% \subsection{Multi-Modal RAG for KI-VideoQA}
One approach to answer questions is to feed the given question and video (or its subtitle) to a VLM for answer generation. This approach has been studied in \cite{garcia2020knowit,wu2021transferring}. However, we believe that this approach cannot perform well for many knowledge-intensive questions. If the LLM has observed and learned the ``knowledge'' required for answering the question, this approach is likely to generate a correct answer, however, this is not always possible. For instance, this approach would not perform well in non-stationary data situations where new videos and knowledge sources have been produced after the VLM training. Therefore, this work presents the first attempt to apply ideas from the retrieval-augmented text generation (RAG) literature \cite{reml,rag} to KI-VideoQA.

In the following, we introduce a generic framework and discuss various implementations of it in each experiment. The framework, as depicted in Figure~\ref{fig:method_overview}, consists of three main components: query formulation, multi-modal knowledge retrieval, and a VLM for information synthesis and answer generation. The framework processes input in the form of a text question, video frames, and additional information about the video if available such as associated subtitles, utilizing external knowledge bases to generate accurate answers in both multiple-choice and open-ended settings. In the following, we describe various implementations of each of these components studied in this paper.


\subsection{Query Formulation}
The effectiveness of retrieval heavily depends on query formulation. We explore several query formulation strategies:
\begin{enumerate}[leftmargin=*]
\item \textbf{Question-only}: Using the question text $q$ as the query:
% \begin{equation}
% q_{raw} = q
% \end{equation}

\item \textbf{Question + Options}: For multiple choice questions, we can concatenate the question with all provided options as the query. This query formulation cannot be applied to open-ended answer generation, as options are not available to the model in this setting.
% concatenating the question with all possible options:
% \begin{equation}
%     q_{opt} = q \oplus \{a^1, a^2, ..., a^K\}
% \end{equation}
% where $\oplus$ denotes concatenation and $\{a^1, ..., a^K\}$ are the candidate answers.

\item \textbf{Question + Subtitle}: Enriching the query with the video subtitle, by concatenating the question with the video subtitle.
% \hamed{In your equation you mentioned subtitle at time $t$. What is $t$ here?} 
% relevant subtitle context:
% \begin{equation}
%     q_{sub} = q \oplus s_t
% \end{equation}
% where $s_t$ represents the subtitle at timestamp $t$.

\item \textbf{LLM-Based Query Rewriting}: Rewriting the question using a VLM that takes the video, subtitle and question, and is prompted to rewrite the question for higher quality retrieval. \footnote{Detailed prompts available in source code}
% Applying transformations to the base query using the inputs with a VLM:
% \begin{equation}
%     q_{trans} = T(q)
% \end{equation}
% where $T(\cdot)$ is the transformation that reformulates the query for better retrieval.

\end{enumerate}

\subsection{Multi-Modal Knowledge Retrieval}

Given a formulated query $q^*$ obtained from the last component, we perform retrieval over a diverse set of multi-modal knowledge sources as follows:
\begin{enumerate}[leftmargin=*]
\item \textbf{Subtitle Retrieval}: This retrieval model takes a collection of videos as a knowledge source and uses their subtitles to construct a text document for every video in the collection. Therefore, this knowledge source basically contains historical dialogues in the video collection and enables the system to leverage conversational context and spoken information that may not be visually apparent.

% \begin{equation}
% \mathcal{R}s = R_s(q*, \mathcal{K}_s)
% \end{equation}

% \begin{equation}
%     \mathcal{R}_c = R_c(q_*, \mathcal{K}_c)
% \end{equation}

\item \textbf{Video Caption Retrieval}: Developing effective video retrieval models is challenging and one approach is to turn the video into a textual description through video captioning. Therefore, each video will be represented by a single text document, automatically generated using a large vision-language model. We use Qwen2-VL-2B \cite{wang2024qwen2vl} in zero-shot setting for this purpose. These captions serve as an intermediate representation bridging visual and textual modalities. To provide an insight into what a video caption may contain, we provide one example in Figure~\ref{fig:method_overview} (bottom right corner).

% This corpus provides dense semantic descriptions of video content.

\item \textbf{Video Retrieval}: Knowledge retrieval can be done directly to the collection of videos. We do video retrieval using the generated captions and finding the corresponding video of the retrieved captions. This component helps in understanding visual context, actions, and temporal relationships that may be crucial for answering some question.


% \begin{equation}
%     \mathcal{R}_c = R_c(q_*, \mathcal{K}_c)
% \end{equation}

\end{enumerate}

Each retriever component operates independently and can be implemented using various retrieval architectures, such as sparse or dense retrieval models. Each returns a ranked list of information items with the highest relevance score, given the formulated query $q^*$. We implement and evaluate three distinct retrieval models:
\begin{itemize}[leftmargin=*]
    \item \textbf{BM25 \cite{robertson1994some}}: A sparse retrieval model rooted in classical probabilistic models. We use Elasticsearch's implementation of BM25. We use the default BM25 parameters (i.e., $b=0.75$ and $k_1=1.2$). This approach can only be employed for subtitle and video caption retrieval.

     \item \textbf{NV-Embed-v2 \cite{lee2024nv}}: State-of-the-art dense retrieval model which was ranked No. 1 on the Massive Text Embedding Benchmark \cite{muennighoff-etal-2023-mteb} as of January 5, 2025 with an impressive score of 72.31 across 56 text embedding tasks. The model is built on the Mistral-7B-v0.1 architecture and produces embeddings with a dimension of 4096. 
     % They used several new ideas: a two-staged instruction tuning method to enhance the accuracy of both retrieval and non-retrieval tasks, used Latent-attention mechanism to allows the LLM to attend to latent vectors, resulting in improved pooled embedding output. 
     % \hamed{it would be nice if you include more information about this. how many parameters? what is especial about it? for instance, a model that is trained using knowledge distillation on the x and y dataset.}
    
    \item \textbf{Stella \cite{zhang2024jasper}}: We use another state-of-the-art dense retrieval model, stella\_en\_400M\_v5.\footnote{Available at \url{https://huggingface.co/dunzhang/stella_en_400M_v5}.} This model only has 400M parameters and encodes queries and documents into 1024-dimensional dense vectors. Compared to other similar performing embedding models, both the number of parameters and encoded vector dimension are very small; for example, NV-Embed-v2 \cite{lee2024nv}, bge-en-icl \cite{li2024making}, and e5-mistral-7b-instruct \cite{wang2022text, wang2023improving, wang2024multilingual} have 7B parameters, and their vector dimensions are 4096. The deployment and application of these larger models in industry were hampered by their vector dimensions and numerous parameters, making it too slow for practical use. Stella uses an innovative distillation technique to achieve high performance while maintaining a smaller footprint. In our experiments we saw 11x speed improvement for a fixed batch size during encoding, compared to NV-Embed-v2.

% This makes it a compact yet powerful dense retrieval model that addresses the industry deployment challenges of large vector dimensions.
    
    % The model encodes queries and documents into 1024-dimensional dense vectors. Note that Stella has shown improved performance compared to commonly used dense retrieval models, such as DPR, ANCE, and TAS-B \cite{}. \hamed{I added the last sentence. Please find a good citation for it, so people don't question your choice.}
    
   
\end{itemize}

% For all retrieval models, we maintain a consistent evaluation setup. 


% \hamed{maybe better to highlight retrieval models and also the fact that we consider heterogenuous information at some point.}

\subsection{Augmented Answer Generation}

The final component is a VLM that generates the answer. It takes the original inputs (video, question, subtitle and options for the MCQ setting) and retrieved knowledge from each of the retrievers. The model utilizes heterogenuous information both from the input and retrieved knowledge. We utilize Qwen2-VL with two billion parameters \cite{wang2024qwen2vl} as our primary reader model, which serves as the foundation for answering both multiple choice and open-ended questions. Qwen2-VL processes the input video frames, question, and retrieved knowledge simultaneously to generate answers. For video frame processing, we sample frames uniformly at 1 FPS and encode them using Qwen2VL's native visual encoder. Each frame height and width is resized into 224 pixels before feeding to the VLM.
% \hamed{more information. what LLMs? any specific prompt? zero-shot or fine-tuning.}

\subsection{Fine-Tuning}
This section describes the method used for fine-tuning the Qwen2VL 2B model. The model is initialized using the pre-trained weights available on HuggingFace.\footnote{Available at \url{https://huggingface.co/Qwen/Qwen2-VL-2B-Instruct}.} We use the Adam optimizer with weight decay (AdamW) \cite{loshchilov2017decoupled} for fine-tuning with a learning rate of 1e-5. We use the cross-entropy loss function with a batch size of 1 due to the high memory requirement of processing videos by VLMs. To minimize the impact of gradient fluctuation, we update model parameters with 50 gradient accumulation step, resulting in an effective batch size of 50. The model uses Flash Attention 2 \cite{dao2023flashattention2fasterattentionbetter} for better acceleration and memory efficiency, especially when processing multiple videos. We use the training portion of each dataset for fine-tuning the model.


% \hamed{here describe your model training. loss funciton. optimization. etc.}

% Model: Qwen2VL
% Model Size: 2B parameters
% Pre-trained Model: The model is initialized with the pre-trained weights from Qwen/Qwen2-VL-2B-Instruct
% Optimizer: AdamW (Adam with Weight Decay)
% Learning Rate: 1e-5
% Loss Function: Cross-Entropy Loss
% Batch Size: 1 (due to the high memory requirements of processing images and videos)
% Gradient Accumulation Steps: 50 (Gradient accumulation is used to simulate a larger batch size by accumulating gradients over multiple forward passes before performing a backward pass and updating the model parameters. This helps in managing memory constraints while still allowing for effective training.)
% Flash Attention: The model uses flash\_attention\_2 for better acceleration and memory efficiency, especially when processing multiple images or videos.
% Training data: train partition of the datasets

% \section{Experimental Setup}
% \hamed{this can be a subsection if it's too brief.}

\section{Results}
\label{sec:results}
% In this section, we answer the research questions formulated in Section \ref{experiments}.

\subsection{Effectiveness of indexing succinct facts to improve information retrieval efficiency} To answer \textbf{RQ1}, we measure the memory and computational costs of fact-checking using full-Wikipedia compared to the pruned version proposed in this work. We first measure the index size on disk, measuring the raw JSON file size containing the article titles and texts, for each experiment setting. In \autoref{fig:disk_size}, we observe a significant reduction in disk space usage with HoVer having a reduction ranging from \textbf{44-55\%}, and WiCE from \textbf{44-57\%}. Additionally, the number of sentences stored in the index also decreases, with HoVer showing a reduction from 52-61\% and WiCE 52-59\%, indicating that at least half of the sentences are not helpful in claim verification.

\begin{table}[htb!]
\small
\centering
\footnotesize
\begin{tabular}{c c c c c}
\toprule
\small
\textbf{Method} & \textbf{Disk Size} & \textbf{Size reduction} & \textbf{\#Sentences} & \textbf{\% decrease}\\
\hline \hline
\multicolumn{1}{l}{\colorg\textbf{HoVer}} & \colorg& \colorg & \colorg & \colorg  \\
Full-Wiki  &  11.28 GiB & - & 94,914,378 & -   \\
Fact Extraction & 6.19 GiB & \down{45}& 45,894,704 & \down{52} \\
Citation Extraction & \textbf{5.07 GiB} & \down{\textbf{55} } & \textbf{36,886,889} & \down{\textbf{61}} \\
Fusion & 5.45 GiB & \down{52} & 39,842,574 & \down{{58}} \\

\hline
\multicolumn{1}{l}{\colorg \textbf{WiCE}} & \colorg &  \colorg & \colorg & \colorg \\
Full-Wiki & 15.28 GiB & - &  126,533,841 & -  \\
Fact Extraction & 8.56 GiB & \down{44} & 61,040,380 & \down{52}\\
Citation Extraction & \textbf{6.56 GiB} & \down{\textbf{57}} & \textbf{51,735,961} & \down{\textbf{59}}  \\
Fusion & 6.85 GiB & \down{55}& 54,070,295 & \down{57} \\

\hline
\end{tabular}
\caption{Comparison sizes for the corpora per experiment setting, consisting of English Wikipedia articles 2017 (HoVer) and 2024 (WiCE). Reduction is measured compared to the  Full-Wiki data setting. \down{} denotes a reduction in corpus size and number of sentence compared to Full-Wiki setting.}
\label{fig:disk_size}
\end{table}
\vspace{-2em}

 Following the reduction in disk size, a notable improvement in retrieval latency is evident, as demonstrated in \autoref{tab:bm25_latency}.
Regarding document retrieval latency (which encompasses both column values), there's an observed speedup ranging from approximately 1.5x (334 ms) to 1.6x (316 ms) compared to the original experimental setting for HoVer (495 ms). Similarly, in WiCE experiments, we witness a comparable speedup rate ranging from 1.4x (446 ms) to 1.6x (399 ms) compared to the original experimental setting (636 ms). This observation suggests that while the reduced text size contributes to efficient retrieval, it could further be improved.

\begin{table}[htb!]
\centering
\small
\footnotesize
% \vspace{-1cm}
\begin{tabular}{l c c c c}
\multirow{2}{*}{\makecell{\textbf{Methods}}} & \multirow{2}{*}{\textbf{Retrieval}} & \multirow{2}{*}{\makecell{\textbf{Total Latency}}} & \multirow{2}{*}{\makecell{\textbf{Speedup}}} \\
& \\
\hline
\multicolumn{1}{l}{\colorg\textit{HoVer}} & \colorg & \colorg & \colorg \\
 Full-Wiki & 426  ms & 659 ms & - \\
Fact Extraction & 257 ms  & 338  ms & 1.9x \\
Citation Extraction & 246 ms  &  327  ms & \speedup{2.0x}  \\
Fusion & 265 ms  & 345  ms & 1.9x  \\

\multicolumn{1}{l}{\colorg\textit{WiCE}} & \colorg & \colorg & \colorg \\
  Full-Wiki &  559 ms  &  831  ms & - \\
Fact Extraction & 372 ms  & 468   ms & 1.8x \\
Citation Extraction & 330 ms   & 419  ms & \speedup{2.0x} \\
Fusion & 347 ms & 436  ms & 1.9x \\
\hline
\end{tabular}
\caption{Retrieval and total latency for Sparse retrieval with Re-ranking. Speedup is compared with respect to the total latency of the Full-Wiki setup.}
\label{tab:bm25_latency}
\vspace{-2em}
\end{table}

% \begin{table}[htpb!]
\centering
\footnotesize
\begin{tabular}{l c c c c c c c c}
\multirow{2}{*}{} & \multicolumn{2}{c}{\makecell{Document retrieval}} & \multirow{2}{*}{\makecell{Sentence \\ Retrieval}} & \multirow{2}{*}{\makecell{Claim \\ Verification}} & \multicolumn{2}{c}{Total Latency} & \multicolumn{2}{c}{Speedup} \\
\cline{2-3}\cline{6-7}\cline{8-9}
& CPU & GPU &  & &  CPU & GPU & CPU & GPU \\ 
\hline
\multicolumn{1}{l}{\textit{HoVer}} &  \\
 \textbf{Full-Wiki (S+R) } &  \multicolumn{2}{c}{\textbf{491  ms}} & \textbf{157  ms} & \textbf{7 ms} &  \multicolumn{2}{c}{\textbf{659 ms}} & - & - \\

Full-Wiki & 515 ms & 31 ms & 153 ms & 8 ms & 676 ms & 192  ms & 1.0x & 3.4x \\
% Original  &  523 ms & 30 ms  & - & 9 ms & 532 ms & 39 ms & 1.2x & 16.9x \\
Claim detection & 513 ms & 23 ms & - & 8 ms & 521 ms & 31 ms & 1.3x & 21.3x  \\
Citation Extraction & 479 ms & 23 ms & - & 9 ms &  488 ms & 32 ms & 1.4x & 20.6x \\
Fusion & 500 ms  & 23 ms & - & 9 ms & 509 ms & 32  ms & 1.3x & 20.6x \\

\multicolumn{1}{l}{\textit{WiCE}} & \\
 \textbf{Full-Wiki (S+R)} & \multicolumn{2}{c}{\textbf{636 ms}} & \textbf{186 ms} & \textbf{9  ms}  & \multicolumn{2}{c}{\textbf{831  ms}} & - & - \\
Full-Wiki & 685 ms & 34 ms & 184 ms & 9 ms & 878 ms & 227  ms & 1.0x & 3.7x \\
% Original  & 610  ms & 34 ms  & - & 9 ms & 619  ms & 43 ms & 1.3x & 19.3x \\
Claim detection &  622 ms & 31 ms  & - & 9 ms & 631 ms & 40 ms & 1.3x & 20.8x \\
Citation Extraction &  610 ms & 31 ms  & - & 9 ms & 619 ms & 40 ms & 1.3x & 20.8x  \\
Fusion & 619  ms & 31 ms  & - & 9 ms & 628 ms & 40 ms & 1.3x & 20.8x  \\[5mm]
\hline
\end{tabular}
\caption{Retrieval and inference latency for Dense retrieval setup on data settings. Speedup is compared with respect to the total latency of the Sparse Retrieval setup with original data setting (bold font).}
\label{tab:faiss_latency}
\end{table}




% However, sparse retrieval does not capture semantic information and requires a costly re-ranking stage. While transitioning from Sparse to Dense retrieval may help improve the performance dense retrieval introduces additional computational costs as seen in Table \ref{tab:faiss_latency}. This is due to our use of FAISS utilising fixed-dimensionality vectors where despite varying article text lengths, the constant number of text embeddings minimizes impact on retrieval speed. However, dense retrieval libraries offer GPU support, which can yield substantial speedups compared to the CPU-based retrieval of BM25 and FAISS. GPU retrieval shows substantial speedups: 16.6-22.3x for HoVer and 17.9-20.2x for WiCE compared to CPU retrieval. Compared to BM25, GPU retrieval offers 16.0-21.5x speedup for HoVer and 18.7-20.5x speedup for WiCE. Thus, making Dense Retrieval a highly efficient and viable option over standard Sparse Retrieval with re-ranking.
%\vspace{-0.5em}

\noindent \textbf{Insight 1}: \textit{
Extraction of succinct facts reduces storage requirements and improves latency for Sparse retrieval while only leading to a minor loss in task performance.}
\vspace{-2em}
% %%%%%%%%%%%%%%%%%%%%%%%%%%%%%%%%%%%%%%%%%%%%%%%%%%%%%%%%%%%%%%%%%%%%%%%%%%%%%%%%%%%

\subsection{Effectiveness of pruned knowledge sources on overall pipeline efficiency and downstream fact-checking performance?}
To answer \textbf{RQ2}, we now shift focus to analyzing the inference time throughout the entire pipeline instead of solely the retrieval part. Extraction of just supporting facts not only has a improvement in the retrieval stage but also on the overall inference latency across the pipeline. For HoVer this being a 1.9-2.0x speedup and 1.8-2.0x for WiCE experiments. This improvement can be attributed to not only faster retrieval times but also the elimination of the Sentence Retrieval stage, which previously imposed significant latency overhead. 

%\begin{table}[htb!]
\small
\begin{tabular}{c c c c c c c c}
\multirow{2}{*}{Experiment setting} & \multirow{2}{*}{Accuracy} & \multicolumn{2}{c}{F1} & \multicolumn{2}{c}{Precision} & \multicolumn{2}{c}{Recall}  \\ 
\cline{3-8}
  & &  Weighted  & Macro & Weighted & Macro & Weighted & Macro      \\
\hline
\multicolumn{1}{l}{\textit{Sparse + Re-ranking}} & & & & \\
Full-Wiki & \textbf{67.79} & \textbf{67.59} & \textbf{67.63} & \textbf{68.45} & \textbf{68.39} & \textbf{67.79}  & \textbf{67.93}\\
Claim detection & \underline{62.33} & 62.02 & 62.08 & \underline{62.98} & \underline{62.92} & \underline{62.33} & \underline{62.50} \\
Citation Extraction & 60.91 & 60.61 & 60.66 & 61.47 & 61.42 & 60.91 & 61.07 \\
Fusion & 62.28 & \underline{62.15} & \underline{62.18} & 62.60 & 62.56 & 62.28 & 62.39  \\[5mm]

\hline
\multicolumn{1}{l}{\textit{Dense Retrieval}} & & & & \\
Full-Wiki & 64.60 & 64.45 & 64.45 & 64.86 & 64.86 & 64.60 & 64.60 \\
% Original & 62.90 & 62.72 & 62.76 & 63.33 & 63.28 & 62.90 & 63.02 \\
Claim detection & \underline{61.50} & \underline{60.94} & \underline{60.94} & \underline{62.20} & \underline{62.20} & \underline{61.50} & \underline{61.50} \\
Citation Extraction & 59.67 & 59.40 & 59.46 & 60.13 & 60.09 & 59.67 & 59.82 \\
Fusion & 59.51 & 59.32 & 59.37 & 59.85 & 59.81 & 59.51 & 59.64  \\[5mm]

\hline
\multicolumn{1}{l}{\textit{Index Compression}} & & & & & & &  \\
Full-Wiki & 63.30 & 62.54 & 62.54 & 64.48 & 64.48 & 63.30 & 63.30 \\
% Original & 63.02 & 62.08 & 62.08 & 64.46 & 64.46 & 63.02 & 63.02  \\
Claim detection & \underline{61.92} & \underline{61.71} & \underline{61.71} & \underline{62.19} & \underline{62.19} & \underline{61.92} & \underline{61.93}  \\
Citation Extraction & 59.98 & 59.12 & 59.12 & 60.89 & 60.89 & 59.98 & 59.98   \\
Fusion & 61.58 & 61.43 & 61.43 & 61.75 & 61.75 & 61.58 & 61.58   \\[5mm]

\hline
\end{tabular}
\caption{Performance experiments on HoVer data and adjustments using full document text of English Wikipedia. The underlined-styled values represent the second best  within each retrieval setup.}
\label{tab:hover_performance_metrics}
\end{table}
% Full document text 
% 
% \begin{figure}
%     \centering
%     \includegraphics[width=0.82\linewidth]{figs/graphs/hover_perf.png}
%     \caption{HoVer performance comparison}
%     \label{fig:enter-label}
% \end{figure}
% %\begin{table}[htb!]
\centering
\footnotesize
\begin{tabular}{c c c c c c c c}
\multirow{2}{*}{Experiment setting} & \multirow{2}{*}{Accuracy} & \multicolumn{2}{c}{F1} & \multicolumn{2}{c}{Precision} & \multicolumn{2}{c}{Recall}  \\ 
\cline{3-8}
  & &  Weighted  & Macro & Weighted & Macro & Weighted & Macro      \\
\hline
\multicolumn{1}{l}{\textit{Sparse + Re-ranking}} & & & & \\
Full-Wiki & \textbf{63.69} &  \textbf{61.84} &  \textbf{55.24} &  \textbf{61.12} &  \textbf{56.54} &  \textbf{63.69} &  \textbf{55.32 } \\
Claim detection & 61.90 & 60.12 & \underline{53.33} & 59.27 & 54.26 & 61.90 & \underline{53.53} \\
Citation Extraction & 61.01 & 59.56 & 52.96 & 58.75 & 53.59 & 61.01 & 53.09 \\
Fusion & \underline{63.39} & \underline{60.21} & 52.48 & \underline{59.46} & \underline{54.69} & \underline{63.39} & 53.27  \\[5mm]

\hline
\multicolumn{1}{l}{\textit{Dense Retrieval}} & & & & \\
Full-Wiki &  61.61 & 60.95 & 55.21 & 60.47 & 55.49 & 61.61 & 55.13 \\
% Full-Wiki  & 60.42 & 58.80 & 51.96 & 57.90 & 52.58 & 60.42 & 52.19 \\
Claim detection & 61.01 & 58.94 & 51.78 & 57.96 & 52.70 & 61.01 & 52.17 \\
Citation Extraction & 58.63 & 58.48 & \underline{52.92} & 58.35 & 52.95 & 58.63 & \underline{52.91} \\
Fusion & \underline{61.31} & \underline{59.34} & 52.30 & 58.40 & \underline{53.23} & \underline{61.31} & 52.62  \\[5mm]

\hline
\multicolumn{1}{l}{\textit{Index Compression}} & & & & & & &  \\
Full-Wiki & 62.46 & 61.38 & 55.27 & 60.74 & 55.84 & 62.46 & 55.20  \\
% Original  & 60.46 & 60.63 & 55.64 & 60.81 & 55.60 & 60.46 & 55.70  \\
Claim detection & 59.31 & 59.02 & \underline{53.32} & 58.77 & 53.39 & 59.31 & \underline{53.30}  \\
Citation Extraction & 60.74 & 59.21 & 52.42 & 58.34 & 53.04 & 60.74 & 52.60  \\
Fusion & \underline{63.04} & \underline{59.79} & 51.89 & 58.94 & \underline{53.97} & \underline{63.04} & 52.76 \\[5mm]

\hline
\end{tabular}
\caption{Performance experiments on WiCE data and adjustments using full document text of English Wikipedia. The bold-styled values represent the baseline while the underlined-styled values represent the highest scores of the re-ranked data within a retrieval setup category.}
\label{tab:wice_performance_metrics}
\end{table}
% Full document text 
% 

% \begin{figure}[hbt!]
%     \centering
%     \includegraphics[width=0.85\linewidth]{figs/graphs/wice_perf.png}
%     \caption{WiCe performance comparison}
%     \label{fig:enter-label}
% \end{figure}

\begin{figure*}[hbt!]
    \begin{subfigure}{.5\textwidth}
        

\begin{tikzpicture}
\edef\mylst{"67.59","64.45","62.54"}
\edef\explora{"62.15","59.32","61.43"}

    \begin{axis}[
            ybar=1.5pt,
            width=6.7cm,
            bar width=0.35,
            every axis plot/.append style={fill},
            grid=major,
            xtick={1, 4, 8,9,11},
            xticklabels={Sparse + re-rank, Dense, IC},
            ylabel style = {font=\tiny},
        yticklabel style = {font=\boldmath \tiny,xshift=0.05ex},
        xticklabel style ={font=\tiny,yshift=0.5ex},
            ylabel={Performance (F1 weighted)},
            enlarge x limits=0.15,
            ymin=0,
            ymax=86,
            legend style ={font=\tiny,yshift=0.05ex},
            area legend,
            nodes near coords style={font=\tiny,align=center,text width=1em},
            legend entries={FW, FE, CE, Fu},
            legend cell align={left},
            legend pos=north west,
            legend columns=-1,
            legend style={/tikz/every even column/.append style={column sep=0.06cm}},
        ]
        \addplot+[
            ybar,
            plotColor1*,
            postaction={
                    pattern=north east lines
                },
                    nodes near coords=\pgfmathsetmacro{\mystring}{{\mylst}[\coordindex]}\textbf{\mystring},
        ] plot coordinates {
                (1,67.59)
                (4,64.45)
                (8,62.54)
            };
        \addplot+[
            ybar,
            plotColor2*,
        ] plot coordinates {
                (1,62.02)
                (4,60.94)
                (8,61.71)
                (9,0)
            };

                    \addplot+[
            ybar,
            plotColor3*,
            draw=black,
    nodes near coords align={vertical},
            postaction={
                    pattern=north west lines
                },
        ] plot coordinates {
                (1,60.61)
                (4,59.40)
                (8,59.12)
                (9,0)
            };
             \addplot+[
            ybar,
            plotColor4*,
            draw=black,
            postaction={
                    pattern=north east lines
                },
            nodes near coords=\pgfmathsetmacro{\mystring}{{\explora}[\coordindex]}\textbf{\mystring},
        ] plot coordinates {
                (1,62.15)
                (4,59.32)
                (8,61.43)
            };
    \end{axis}
\end{tikzpicture}

\subcaption{HoVer}
    \end{subfigure}
        \begin{subfigure}{.5\textwidth}
    

\begin{tikzpicture}
\edef\mylst{"61.84","60.95","61.38"}
\edef\explora{"60.21","59.34","59.79"}

    \begin{axis}[
            ybar=1.5pt,
            width=6.7cm,
            bar width=0.35,
            every axis plot/.append style={fill},
            grid=major,
            xtick={1, 4, 8,9,11},
            xticklabels={Sparse + re-rank, Dense, IC},
            ylabel style = {font=\tiny},
        yticklabel style = {font=\boldmath \tiny,xshift=0.05ex},
        xticklabel style ={font=\tiny,yshift=0.5ex},
            ylabel={Performance (F1 weighted)},
            enlarge x limits=0.15,
            ymin=0,
            ymax=86,
            legend style ={font=\tiny,yshift=0.05ex},
            area legend,
            nodes near coords style={font=\tiny,align=center,text width=1em},
            legend entries={FW, FE, CE, Fu},
            legend cell align={left},
            legend pos=north west,
            legend columns=-1,
            legend style={/tikz/every even column/.append style={column sep=0.06cm}},
        ]
        \addplot+[
            ybar,
            plotColor1*,
            postaction={
                    pattern=north east lines
                },
                    nodes near coords=\pgfmathsetmacro{\mystring}{{\mylst}[\coordindex]}\textbf{\mystring},
        ] plot coordinates {
                (1,61.84)
                (4,60.95)
                (8,61.38)
            };
        \addplot+[
            ybar,
            plotColor2*,
        ] plot coordinates {
                (1,60.12)
                (4,58.94)
                (8,59.02)
                (9,0)
            };

                    \addplot+[
            ybar,
            plotColor3*,
            draw=black,
    nodes near coords align={vertical},
            postaction={
                    pattern=north west lines
                },
        ] plot coordinates {
                (1,59.56)
                (4,58.48)
                (8,59.21)
                (9,0)
            };
             \addplot+[
            ybar,
            plotColor4*,
            draw=black,
            postaction={
                    pattern=north east lines
                },
            nodes near coords=\pgfmathsetmacro{\mystring}{{\explora}[\coordindex]}\textbf{\mystring},
        ] plot coordinates {
                (1,60.21)
                (4,59.34)
                (8,59.79)
            };
    \end{axis}
\end{tikzpicture}

    \subcaption{Wice}

    \end{subfigure}
    \caption{HoVer and WiCe task performance (FW- Full-Wiki, FE - Fact Extraction, IC- Index Compression, CE - Citation Extraction, Fu - Fusion)}
    \label{fig:performance_plot}
    \end{figure*}

\begin{figure*}[hbt!]
    \begin{subfigure}{.5\textwidth}
        

\begin{tikzpicture}
\edef\mylst{"67.59","64.45","62.54"}
\edef\explora{"62.15","59.32","61.43"}

    \begin{axis}[
            ybar=1.5pt,
            width=6.2cm,
            bar width=0.35,
            every axis plot/.append style={fill},
            grid=major,
            xtick={1, 4, 8,9,11},
            xticklabels={Sparse + re-rank, Dense, IC},
            ylabel style = {font=\tiny},
        yticklabel style = {font=\boldmath \tiny,xshift=0.05ex},
        xticklabel style ={font=\tiny,yshift=0.5ex},
            ylabel={Recall@10},
            enlarge x limits=0.15,
            ymin=0,
            ymax=0.5,
            legend style ={font=\tiny,yshift=0.05ex},
            area legend,
            nodes near coords style={font=\tiny,align=center,text width=1em},
            legend entries={FW, FE, CE, Fu},
            legend cell align={left},
            legend pos=north west,
            legend columns=-1,
            legend style={/tikz/every even column/.append style={column sep=0.06cm}},
        ]
        \addplot+[
            ybar,
            plotColor1*,
            postaction={
                    pattern=north east lines
                },
        ] plot coordinates {
                (1,0.136)
                (4,0.123)
                (8,0.098)
            };
        \addplot+[
            ybar,
            plotColor2*,
        ] plot coordinates {
                (1,0.105)
                (4,0.094)
                (8,0.098)
                (9,0)
            };

                    \addplot+[
            ybar,
            plotColor3*,
            draw=black,
    nodes near coords align={vertical},
            postaction={
                    pattern=north west lines
                },
        ] plot coordinates {
                (1,0.126)
                (4,0.143)
                (8,0.096)
                (9,0)
            };
             \addplot+[
            ybar,
            plotColor4*,
            draw=black,
        ] plot coordinates {
                (1,0.124)
                (4,0.141)
                (8,0.097)
            };
    \end{axis}
\end{tikzpicture}

\subcaption{WiCE (nDCG@10)}
    \end{subfigure}
        \begin{subfigure}{.5\textwidth}
    

\begin{tikzpicture}
\edef\mylst{"67.59","64.45","62.54"}
\edef\explora{"62.15","59.32","61.43"}

    \begin{axis}[
            ybar=1.5pt,
            width=6.4cm,
            bar width=0.35,
            every axis plot/.append style={fill},
            grid=major,
            xtick={1, 4, 8,9,11},
            xticklabels={Sparse + re-rank, Dense, IC},
            ylabel style = {font=\tiny},
        yticklabel style = {font=\boldmath \tiny,xshift=0.05ex},
        xticklabel style ={font=\tiny,yshift=0.5ex},
            ylabel={Recall@10},
            enlarge x limits=0.15,
            ymin=0,
            ymax=0.5,
            legend style ={font=\tiny,yshift=0.05ex},
            area legend,
            nodes near coords style={font=\tiny,align=center,text width=1em},
            legend entries={FW, FE, CE, Fu},
            legend cell align={left},
            legend pos=north west,
            legend columns=-1,
            legend style={/tikz/every even column/.append style={column sep=0.06cm}},
        ]
        \addplot+[
            ybar,
            plotColor1*,
        ] plot coordinates {
                (1,0.309)
                (4,0.195)
                (8,0.160)
            };
        \addplot+[
            ybar,
            plotColor2*,
        ] plot coordinates {
                (1,0.241)
                (4,0.166)
                (8,0.163)
                (9,0)
            };

                    \addplot+[
            ybar,
            plotColor3*,
            draw=black,
    nodes near coords align={vertical},
            postaction={
                    pattern=north west lines
                },
        ] plot coordinates {
                (1,0.295)
                (4,0.226)
                (8,0.174)
                (9,0)
            };
             \addplot+[
            ybar,
            plotColor4*,
            draw=black,
        ] plot coordinates {
                (1,0.286)
                (4,0.223)
                (8,0.160)
            };
    \end{axis}
\end{tikzpicture}

    \subcaption{WiCE (Recall@10)}

    \end{subfigure}
    \caption{Retrieval performance comparison}
    \label{fig:retrieval_perf}
    \end{figure*}
The evaluation of Sparse and Dense Retrieval setups in HoVer and WiCE experiments reveals that Sparse Retrieval, particularly fact extraction (FE) and Fusion approaches, maintains performance closest to the Full-Wiki setup as measured by weighted F1 in Figure \ref{fig:performance_plot}, while citation extraction has a larger drop in performance. Most notably, the Fusion method compared to the other methods has relatively high scores, underscoring the importance of combining supporting facts extraction methods for optimal results. We also report retrieval performance for WiCE Figure \ref{fig:retrieval_perf} using measures nDCG@10 and Recall@10 using annotated documents provided for WiCE. We observe trends similar to overall task performance demonstrating that efficient retrieval approaches explored do not negatively impact task performance to a significant extent.

\noindent\textbf{Insight 2}: \textit{We find that extracting supporting facts improves efficiency across the entire pipeline, with Sparse setups achieving up to 2.0x speedups with only a minimal performance decline.}
\vspace{-1em}

\begin{table}[htb!]
\centering
\footnotesize
\begin{tabular}{l  c c c c }
\hline
\multirow{2}{*}{Method}  & \multicolumn{2}{c}{Total Latency}  & \multicolumn{2}{c}{Speedup} \\
\cline{2-5}
& CPU & GPU  &  CPU & GPU  \\ 
\hline \hline
\multicolumn{1}{l}{\colorg \textit{HoVer}} & \colorg & \colorg & \colorg & \colorg \\
 Full-Wiki (S+R) &  \multicolumn{2}{c}{659 ms} & - & - \\
Full-Wiki & 214  ms & 174 ms & 3.1x &  3.8x \\
% Original  &  55 ms & 13 ms  & - & 12 ms & 67 ms & 25 ms & 9.8x &  26.4x \\
Fact Extraction  & 60 ms & 21 ms & 11.0x & 31.4x  \\
Citation Extraction  & \textbf{51 ms} & \textbf{20 ms} & \textbf{\speedup{12.9x}} & \textbf{\speedup{33.0x}} \\
Fusion  & 63 ms & 24 ms &  10.5x & 27.5x \\
\hline
\multicolumn{1}{l}{\colorg\textit{WiCE}} & \colorg & \colorg & \colorg & \colorg   \\
 Full-Wiki (S+R) & \multicolumn{2}{c}{831  ms} & - & - \\
Full-Wiki &  292 ms & 238 ms & 2.8x  &  3.5x \\
% Original  &  95 ms & 43  ms  & - & 11 ms & 106 ms & 54 ms & 7.8x &  15.4x \\
Fact Extraction  & 103 ms & 48 ms & 8.1x & 17.3x \\
Citation Extraction  & 98 ms & 46 ms & 8.5x & 18.1x \\
Fusion  &\textbf{98 ms} & \textbf{46  ms} & \speedup{8.5x} &  \speedup{18.1x} \\
\hline
\end{tabular}
\caption{Latency and speedup measurements for Index compression setup. Speedup is compared with respect to the total latency of Sparse-retrieval + Re-rank (S+R) pipeline with the Full-Wiki setup. (S+R) runs on both CPU and GPU, sparse retrieval running on CPU and rest of components running on GPU}
%\vspace{-1cm}
\label{tab:jpq_latency}
\end{table}

%\begin{table}[htb!]
\centering
\footnotesize
\begin{tabular}{l c c c c c c c c}
\hline
\multirow{2}{*}{Method} & \multicolumn{2}{c}{\makecell{Term-based \\ document retrieval}} & \multirow{2}{*}{\makecell{Sentence \\ Retrieval}} & \multirow{2}{*}{\makecell{Claim \\ Verification}} & \multicolumn{2}{c}{Total Latency}  & \multicolumn{2}{c}{Speedup} \\
\cline{2-3}\cline{6-7}\cline{8-9}
& CPU & GPU &  & &  CPU & GPU & CPU & GPU \\ 
\hline \hline
 \multicolumn{1}{l}{\colorg\textit{HoVer}} & \colorg & \colorg & \colorg & \colorg & \colorg & \colorg & \colorg & \colorg \\
 \textbf{Full-Wiki (S+R)} &  \multicolumn{2}{c}{\textbf{491  ms}} & \textbf{157  ms} & \textbf{7 ms} &  \multicolumn{2}{c}{\textbf{659 ms}} & - & - \\
Full-Wiki &  53 ms & 13 ms & 153 ms & 8 ms & 214  ms & 174 ms & 3.1x &  3.8x \\
% Original  &  55 ms & 13 ms  & - & 12 ms & 67 ms & 25 ms & 9.8x &  26.4x \\
Claim detection  &  51 ms & 12  ms  & - &  9 ms & 60 ms & 21 ms & 11.0x & 31.4x  \\
Citation Extraction  & 46  ms & 11 ms  & - & 9 ms & 51 ms & 20 ms & 12.9x & 33.0x \\
Fusion  & 51  ms & 12 ms  & - & 12 ms & 63 ms & 24 ms &  10.5x & 27.5x \\
\hline
\multicolumn{1}{l}{\colorg\textit{WiCE}} & \colorg & \colorg & \colorg & \colorg & \colorg & \colorg & \colorg & \colorg  \\
 \textbf{Full-Wiki (S+R)} & \multicolumn{2}{c}{\textbf{636 ms}} & \textbf{186 ms} & \textbf{9  ms}  & \multicolumn{2}{c}{\textbf{831  ms}} & - & - \\
Full-Wiki & 97 ms & 43 ms & 186 ms & 9 ms & 292 ms & 238 ms & 2.8x  &  3.5x \\
% Original  &  95 ms & 43  ms  & - & 11 ms & 106 ms & 54 ms & 7.8x &  15.4x \\
Claim detection  &  92 ms & 37 ms  & - & 11 ms & 103 ms & 48 ms & 8.1x & 17.3x \\
Citation Extraction  & 89  ms & 37 ms  & - & 9 ms & 98 ms & 46 ms & 8.5x & 18.1x \\
Fusion  & 89  ms & 37  ms  & - &  9 ms & 98 ms & 46  ms & 8.5x &  18.1x \\
\hline
\end{tabular}
\caption{Retrieval and inference latency for Index compression setup. Speedup is compared to the total latency of (S+R) pipeline with Full-Wiki setup.}
\label{tab:jpq_latency}
\vspace{-2em}
\end{table}

% %%%%%%%%%%%%%%%%%%%%%%%%%%%%%%%%%%%%%%%%%%%%%%%%%%%%%%%%%%%%%%%%%%%%%%%%%%%%%%%%%%%
\vspace{-2em}
\subsection{Effectiveness of index compression on enhancing the efficiency of dense retrieval and fact-checking systems?}

To answer \textbf{RQ3}, we make use of index compression to further improve upon Dense Retrieval setups, not only with respect to memory requirements but also enhancing total inference latency compared to the sparse retrieve + re-rank setups in classical pipelines.  The index sizes of Wikipedia collection for standard dense retrieval are 7.51 GiB for HoVer and 9.70 GiB for WiCE. Using the JPQ index compression model with M=96 subvectors, we significantly reduced the storage space for vector embeddings from 1.5 KiB to 104.12 B. This reduced the HoVer index size to 544.89 MiB and the WiCE index to \textbf{672.95 MiB}, achieving a \textbf{93\% reduction (14.4:1 compression ratio)}. Further reducing subvectors could decrease the index size but may impact performance.


The utilization of JPQ index compression leads to significant reductions in retrieval latency compared to dense Retrieval and sparse retrieval, as demonstrated in \autoref{tab:jpq_latency}. CPU retrieval gains notable speedups of approximately 10.0x for HoVer experiments and 7.0x for WiCE experiments, while GPU retrieval shows about 2.0x and 0.8x speedups, respectively. These improvements are attributed to learned compression in JPQ, enhancing computational efficiency. 
Significant improvements are also observed when examining the inference latency of the whole pipeline. The CPU-based approaches shows impressive speedups (upto \textbf{12.9x} for HoVer and \textbf{8.5x} for WiCE), and GPU-based approaches achieve even higher gains (\textbf{33.0x} for HoVer and \textbf{18.1x} for WiCE). 

Surprisingly, in our experiments we observe that JPQ yields better results than standard Dense Retrieval as shown in Figure \ref{fig:performance_plot}. This is particularly due to joint training of the query encoder and index compression. In addition, JPQ employs end-end negative sampling, which further improves retrieval performance despite significant compression of embeddings.

\mpara{Insight 3}: \textit{We find that index compression reduces index size by \textbf{93\%} resulting in speedups for CPU-based setups up to 10x and GPU-based setups more than 20x compared to classical fact-checking pipeline.}

\subsection{Discussion of Live Fact-checking results}
We employ the pruned index (2024 Wiki collection) using our Fusion approach followed by compression of the index for live Fact-checking of 2024 presidential debate. The pipeline comprises a dense retrieval using compressed index followed by claim verification. We use the query encoder and NLI models trained on HoVer for this application. We compare this approach to also the classical sparse-retrieval+re-rank fact-checking pipeline over the Full-Wiki collection (without pruning). The task performance is shown in Figure \ref{fig:livefc} and the corresponding pipeline efficiency is shown in Table \ref{tab:livefc}. We observe that the pruned collection coupled with retrieval using index compression leads to impressive speedups (\textbf{3.4x}) over classical pipeline over the full collection without significant drop in task performance (Figure \ref{fig:livefc}). The results demonstrate that efficient retrieval is critical for live fact-checking. Our experiments demonstrate that our approach for efficient retrieval provides significant speedups on CPUs further making the technology accessible even in low-resource scenarios which has significant implications in aiding detection of misinformation and disinformation at scale.
\begin{figure}
\centering
 \hspace{6em}     \begin{subfigure}{.8\textwidth}
        

\begin{tikzpicture}
\edef\mylst{"56.95","56.66","56.94",""}
\edef\explora{"55.92","57.82","52.93",""}

    \begin{axis}[
            ybar=14pt,
            width=6cm,
            bar width=0.35,
            every axis plot/.append style={fill},
            grid=major,
            xtick={1, 4, 8,9,11},
            xticklabels={Sparse + re-rank, Dense, IC},
            ylabel style = {font=\small},
        yticklabel style = {font=\boldmath \tiny,xshift=0.05ex},
        xticklabel style ={font=\tiny,yshift=0.5ex},
            ylabel={Performance (F1 weighted)},
            enlarge x limits=0.15,
            ymin=0,
            ymax=86,
            legend style ={font=\tiny,yshift=0.05ex},
            area legend,
            nodes near coords style={font=\tiny,align=center,text width=1em},
            legend entries={Full-Wiki, Fusion},
            legend cell align={left},
            legend pos=north west,
            legend columns=-1,
            legend style={/tikz/every even column/.append style={column sep=0.06cm}},
        ]
        \addplot+[
            ybar,
            plotColor1*,
            postaction={
                    pattern=north east lines
                },
                    nodes near coords=\pgfmathsetmacro{\mystring}{{\mylst}[\coordindex]}\textbf{\mystring},
        ] plot coordinates {
                (1,56.95)
                (4,56.66)
                (8,56.94)
            };
        \addplot+[
            ybar,
            plotColor2*,
            postaction={
                    pattern=north east lines
                },
            nodes near coords=\pgfmathsetmacro{\mystring}{{\explora}[\coordindex]}\textbf{\mystring},
        ] plot coordinates {
                (1,55.92)
                (4,57.82)
                (8,52.83)
                (9,0)
            };

    \end{axis}
\end{tikzpicture}


    \end{subfigure}
    \caption{Live fact-checking performance across different corpus setups}
    \label{fig:livefc}
\end{figure}
\vspace{-2em}
\begin{table}[htb!]
\centering
\footnotesize % Reduced font size
\setlength{\tabcolsep}{3pt} % Reduce space between columns
\renewcommand{\arraystretch}{0.9} % Reduce space between rows
\begin{tabular}{l c c c c c c}
\multirow{2}{*} & \multicolumn{2}{c}{\makecell{ Retrieval}}   & \multicolumn{2}{c}{Total Latency} & \multicolumn{2}{c}{Speedup} \\
\cline{2-3}\cline{4-5}\cline{5-7} \\[-1mm]
& CPU & GPU & CPU& GPU& CPU & GPU \\
\hline \\

% Term-based Document Retrieval
\colorg\textit{Sparse + Re-ranking} & \colorg & \colorg & \colorg  & \colorg & \colorg & \colorg\\ 
Full-Wiki & 463  & -   & 695  & - & \multicolumn{2}{c}{-} \\
Fusion & 274  & -   & 479   & - & \multicolumn{2}{c}{1.5x} \\
\hline \\
% Dense Retrieval setup
\colorg\textit{Dense Retrieval} & \colorg & \colorg & \colorg & \colorg & \colorg  & \colorg\\ 
Full-Wiki &  433   & 32  & 553   &  152  & 1.3x & 4.6x \\
Fusion & 412   & 32 & 511  & 131  & 1.4x & 5.3x \\
\hline \\
% Index Compression setup
\colorg\textit{Index Compression} & \colorg & \colorg & \colorg & \colorg & \colorg & \colorg\\  
Full-Wiki & 100  & 50   & 228  & 178  & 3.0x & 3.9x \\
\textbf{Fusion (ours)} & \textbf{89 }   & \textbf{43}  & \textbf{203}  & 157  & \speedup{3.4x} & 4.4x \\

\hline
\end{tabular}
\caption{Latency Comparisons for Live Fact-checking (in milliseconds (ms))}

\label{tab:livefc}
\end{table}


% \section{RQ 1: How does indexing supporting facts improve information retrieval efficiency?}
% In this section, we investigate the impact of indexing supporting facts on information retrieval efficiency by comparing the disk space utilization and retrieval latency across different experiment settings. Here we aim to discern the benefits of storing only supporting facts in the index as opposed to the entire corpus. 

% \subsection{Corpus Size}
% \begin{table}[htb!]
\small
\centering
\footnotesize
\begin{tabular}{c c c c c}
\toprule
\small
\textbf{Method} & \textbf{Disk Size} & \textbf{Size reduction} & \textbf{\#Sentences} & \textbf{\% decrease}\\
\hline \hline
\multicolumn{1}{l}{\colorg\textbf{HoVer}} & \colorg& \colorg & \colorg & \colorg  \\
Full-Wiki  &  11.28 GiB & - & 94,914,378 & -   \\
Fact Extraction & 6.19 GiB & \down{45}& 45,894,704 & \down{52} \\
Citation Extraction & \textbf{5.07 GiB} & \down{\textbf{55} } & \textbf{36,886,889} & \down{\textbf{61}} \\
Fusion & 5.45 GiB & \down{52} & 39,842,574 & \down{{58}} \\

\hline
\multicolumn{1}{l}{\colorg \textbf{WiCE}} & \colorg &  \colorg & \colorg & \colorg \\
Full-Wiki & 15.28 GiB & - &  126,533,841 & -  \\
Fact Extraction & 8.56 GiB & \down{44} & 61,040,380 & \down{52}\\
Citation Extraction & \textbf{6.56 GiB} & \down{\textbf{57}} & \textbf{51,735,961} & \down{\textbf{59}}  \\
Fusion & 6.85 GiB & \down{55}& 54,070,295 & \down{57} \\

\hline
\end{tabular}
\caption{Comparison sizes for the corpora per experiment setting, consisting of English Wikipedia articles 2017 (HoVer) and 2024 (WiCE). Reduction is measured compared to the  Full-Wiki data setting. \down{} denotes a reduction in corpus size and number of sentence compared to Full-Wiki setting.}
\label{fig:disk_size}
\end{table}
\vspace{-2em}
% To get an idea of how storing just the supporting facts data in the index improves efficiency compared to storing the entire corpus, a comparison can be made on how much these different settings occupy disk space. As mentioned in \autoref{sec:metrics}, to get an accurate estimate, only the dictionaries containing the article's title and document text are saved to raw JSON files. Across all experiment settings as seen in \autoref{fig:disk_size},  a notable reduction in disk space usage is observed compared to the original Wikipedia document corpus. This reduction ranges from approximately 45\% (claim detection) to 55\% depending on the setting for the HoVer corpus data. Likewise, for the WiCE corpus data, we can observe approximately 44\% to 57\% reduction. Moreover, in correlation with the reduced disk size, it is evident that the number of sentences stored in the index also decreases across each experiment setting compared to the original corpus data. For HoVer this ranges from 52\%  (claim detection) to 61\% (citation extraction) and WiCE ranges from 52\% to 59\%. This indicates that at least half of the sentences are considered as not claim-worthy across the different re-ranking methods.

% \subsection{Retrieval Latency}\label{ssec:retrieval_latency}
% \paragraph{Sparse retrieval} Following the reduction in disk size, a notable enhancement in retrieval latency is evident, as demonstrated in both the Term-based and Neural-based document retrieval columns of \autoref{tab:bm25_latency}. To avoid any ambiguity, it's crucial to clarify that the speedup listed in the table pertains to the total latency, which is relevant for addressing RQ2, rather than solely focusing on document retrieval.
% Regarding document retrieval latency (which encompasses both column values), there's an observed speedup ranging from approximately 1.5x (334 ms) to 1.6x (316 ms) compared to the original experimental setting for HoVer (495 ms). Similarly, in WiCE experiments, we witness a comparable speedup rate ranging from 1.4x (446 ms) to 1.6x (399 ms) compared to the original experimental setting (636 ms). This observation suggests that while the reduced text size contributes to expedited retrieval, the enhancement is only somewhat proportional.

% \begin{table}[htb!]
\centering
\small
\footnotesize
% \vspace{-1cm}
\begin{tabular}{l c c c c}
\multirow{2}{*}{\makecell{\textbf{Methods}}} & \multirow{2}{*}{\textbf{Retrieval}} & \multirow{2}{*}{\makecell{\textbf{Total Latency}}} & \multirow{2}{*}{\makecell{\textbf{Speedup}}} \\
& \\
\hline
\multicolumn{1}{l}{\colorg\textit{HoVer}} & \colorg & \colorg & \colorg \\
 Full-Wiki & 426  ms & 659 ms & - \\
Fact Extraction & 257 ms  & 338  ms & 1.9x \\
Citation Extraction & 246 ms  &  327  ms & \speedup{2.0x}  \\
Fusion & 265 ms  & 345  ms & 1.9x  \\

\multicolumn{1}{l}{\colorg\textit{WiCE}} & \colorg & \colorg & \colorg \\
  Full-Wiki &  559 ms  &  831  ms & - \\
Fact Extraction & 372 ms  & 468   ms & 1.8x \\
Citation Extraction & 330 ms   & 419  ms & \speedup{2.0x} \\
Fusion & 347 ms & 436  ms & 1.9x \\
\hline
\end{tabular}
\caption{Retrieval and total latency for Sparse retrieval with Re-ranking. Speedup is compared with respect to the total latency of the Full-Wiki setup.}
\label{tab:bm25_latency}
\vspace{-2em}
\end{table}
% \paragraph{CPU-based Dense Retrieval} One might typically anticipate a more pronounced disparity between the original data and the reranked data in the document retrieval phase. However when transitioning from the Sparse retrieval setup to the Dense retrieval setup, as depicted in the first column of \autoref{tab:faiss_latency}, only negligible differences between the different settings are observed. This is attributed to FAISS utilizing vectors instead of computing the relevance ranking of documents to the query, as is the case with BM25. Despite variations in the length of each article across settings, the number of text embeddings (with fixed dimensionality size) created remains constant, corresponding to the number of encoded text spans, which is consistent across settings. Thus minimizing the impact of extracting supporting facts on document retrieval latency when using Dense Retrieval. Comparing the Dense document retrieval (CPU) column in \autoref{tab:faiss_latency} to the baselines listed in \autoref{tab:bm25_latency}, it is observed to be of a similar latency or even slightly slower. For HoVer, we can observe a 0.9x (523 ms) to 1.0x (479 ms) compared to the baseline (495 ms). Likewise, for WiCE we can observe a similar latency speedup of 0.9x (685 ms) to 1.0x (610 ms) speedup compared to its baseline (636 ms). This suggests that the indexing of supporting facts would not significantly impact information retrieval efficiency in such scenarios. 

% \begin{table}[htpb!]
\centering
\footnotesize
\begin{tabular}{l c c c c c c c c}
\multirow{2}{*}{} & \multicolumn{2}{c}{\makecell{Document retrieval}} & \multirow{2}{*}{\makecell{Sentence \\ Retrieval}} & \multirow{2}{*}{\makecell{Claim \\ Verification}} & \multicolumn{2}{c}{Total Latency} & \multicolumn{2}{c}{Speedup} \\
\cline{2-3}\cline{6-7}\cline{8-9}
& CPU & GPU &  & &  CPU & GPU & CPU & GPU \\ 
\hline
\multicolumn{1}{l}{\textit{HoVer}} &  \\
 \textbf{Full-Wiki (S+R) } &  \multicolumn{2}{c}{\textbf{491  ms}} & \textbf{157  ms} & \textbf{7 ms} &  \multicolumn{2}{c}{\textbf{659 ms}} & - & - \\

Full-Wiki & 515 ms & 31 ms & 153 ms & 8 ms & 676 ms & 192  ms & 1.0x & 3.4x \\
% Original  &  523 ms & 30 ms  & - & 9 ms & 532 ms & 39 ms & 1.2x & 16.9x \\
Claim detection & 513 ms & 23 ms & - & 8 ms & 521 ms & 31 ms & 1.3x & 21.3x  \\
Citation Extraction & 479 ms & 23 ms & - & 9 ms &  488 ms & 32 ms & 1.4x & 20.6x \\
Fusion & 500 ms  & 23 ms & - & 9 ms & 509 ms & 32  ms & 1.3x & 20.6x \\

\multicolumn{1}{l}{\textit{WiCE}} & \\
 \textbf{Full-Wiki (S+R)} & \multicolumn{2}{c}{\textbf{636 ms}} & \textbf{186 ms} & \textbf{9  ms}  & \multicolumn{2}{c}{\textbf{831  ms}} & - & - \\
Full-Wiki & 685 ms & 34 ms & 184 ms & 9 ms & 878 ms & 227  ms & 1.0x & 3.7x \\
% Original  & 610  ms & 34 ms  & - & 9 ms & 619  ms & 43 ms & 1.3x & 19.3x \\
Claim detection &  622 ms & 31 ms  & - & 9 ms & 631 ms & 40 ms & 1.3x & 20.8x \\
Citation Extraction &  610 ms & 31 ms  & - & 9 ms & 619 ms & 40 ms & 1.3x & 20.8x  \\
Fusion & 619  ms & 31 ms  & - & 9 ms & 628 ms & 40 ms & 1.3x & 20.8x  \\[5mm]
\hline
\end{tabular}
\caption{Retrieval and inference latency for Dense retrieval setup on data settings. Speedup is compared with respect to the total latency of the Sparse Retrieval setup with original data setting (bold font).}
\label{tab:faiss_latency}
\end{table}




% \paragraph{GPU-based Dense Retrieval} However, it is worth noting that Dense retrieval can still be faster, particularly with dense retrieval libraries such as FAISS offering GPU support, which can yield substantial speedups compared to both CPU retrieval of BM25 and FAISS. This advantage is evident in the data, showcasing notable speedups ranging from 16.6x to 22.3x speedup for HoVer GPU retrieval over CPU retrieval, and 17.9x to 20.2x speedup for WiCE. Furthermore, when comparing FAISS GPU retrieval to the BM25 retrieval, we can see an approximate 16.0x (31 ms) to 21.5x (23 ms) speedup for HoVer and 18.7x (34 ms) to 20.5x (31 ms) speedup for WiCE. Therefore the GPU-based approach makes Dense Retrieval a viable option, unlike the CPU-based variant. 

% \subsection{Key Takeaways} 
% Extracting supporting facts from the data corpus can lead to only requiring to store at least half of the data. Although this has a positive effect on the latency for Sparse retrieval, with Dense document retrieval this is not the case due to how the vector embeddings are constructed (being per article rather than per sentence). Furthermore, while CPU-based Dense retrieval may not necessarily outperform Sparse retrieval methods in terms of latency, thereby presenting less immediate appeal, the incorporation of GPU support leads to significant speed enhancements. Thus, although extracting supporting facts does not help much in Dense document retrieval unlike Sparse retrieval in terms of retrieval latency, the incorporation of the GPU-based Dense retrieval renders it a much more compelling option for achieving efficiency. 

% %%%%%%%%%%%%%%%%%%%%%%%%%%%%%%%%%%%%%%%%%%%%%%%%%%%%%%%%%%%%%%%%%%%%%%%%%%%%%%%%%%%

% \section{RQ 2: How does indexing supporting facts affect overall pipeline efficiency and downstream fact-checking performance?}
% In continuation of the previous research inquiry concerning retrieval latency and disk size, this section delves into an analysis of the overall inference time across the entire pipeline. Additionally, recognizing that faster processing times do not necessarily equate to better performance a further analysis will be done on the performance metrics.

% \subsection{Inference Latency}
% \paragraph{Sparse Retrieval Setup:} The enhancement in retrieval latency, as evidenced in \autoref{tab:bm25_latency}, mirrors a noticeable improvement in the overall inference latency across the pipeline. This improvement spans approximately 1.9x to 2.0x for the HoVer experiments and 1.8x to 2.0x for WiCE experiments. However, the reduction in total latency cannot be solely ascribed to faster retrieval times. It also arises from the elimination of the Sentence Retrieval stage, which previously imposed significant latency overhead. Upon closer inspection of \autoref{tab:bm25_latency}, it becomes apparent that the absence of the Sentence Retrieval stage impacts the Claim Verification stage. Notably, experiments conducted on the original corpus data exhibit much lower inference latency compared to supporting facts data. Nevertheless, the variance between these experiment settings is minimal, and the impact on total latency results is insignificant. This overall trend indicates that indexing supporting facts for the BM25 retrieval setup predominantly benefits inference times for the Rule-based document retrieval and Sentence Retrieval stages. Furthermore, it reveals that the Claim Verification stage is slightly, yet negligibly, affected when considering the entire pipeline inference.

% \paragraph{Dense Retrieval Setup:} In a similar vein as the document retrieval comparisons of RQ1 (see \autoref{ssec:retrieval_latency}), the total inference of the Dense retrieval setup presents notable differences in results between CPU- and GPU-based Dense retrieval compared to Sparse retrieval. This divergence is evident in \autoref{tab:faiss_latency}, where for HoVer experiments, the CPU-based approach exhibits a 1.2x to 1.4x speedup, while the GPU-based approach demonstrates a 16.9x to 21.3x speedup compared to the baseline. Similarly, WiCE experiments show approximately a 1.3x speedup for the CPU-based approach and 19.3x to 20.8x speedup for the GPU-based approach. The key distinction lies in the influence of omitting the Sentence Retrieval stage for the original corpus data. Its omission introduces significant overhead to the total latency. For the CPU-based approach, this translates to a 1.3x speedup for HoVer (676 ms vs. 532 ms) and a 1.4x speedup for WiCE (878 ms vs. 619 ms). Conversely, the GPU-based approach experiences a 4.9x speedup for HoVer (192 ms vs. 39 ms) and a 5.3x speedup for WiCE (227 ms vs. 43 ms). Overall, this underscores that including Sentence Retrieval adds substantial overhead, especially for GPU-based approaches operating with lower latency magnitudes. Therefore, the supporting facts data for Dense Retrieval, while not significantly impacting document retrieval, offers significant speedup for total inference latency, allowing for the effective omission of the Sentence Retrieval stage and its associated latency overhead.

% \subsection{Performance Metrics Evaluation}

% \begin{table}[htb!]
\small
\begin{tabular}{c c c c c c c c}
\multirow{2}{*}{Experiment setting} & \multirow{2}{*}{Accuracy} & \multicolumn{2}{c}{F1} & \multicolumn{2}{c}{Precision} & \multicolumn{2}{c}{Recall}  \\ 
\cline{3-8}
  & &  Weighted  & Macro & Weighted & Macro & Weighted & Macro      \\
\hline
\multicolumn{1}{l}{\textit{Sparse + Re-ranking}} & & & & \\
Full-Wiki & \textbf{67.79} & \textbf{67.59} & \textbf{67.63} & \textbf{68.45} & \textbf{68.39} & \textbf{67.79}  & \textbf{67.93}\\
Claim detection & \underline{62.33} & 62.02 & 62.08 & \underline{62.98} & \underline{62.92} & \underline{62.33} & \underline{62.50} \\
Citation Extraction & 60.91 & 60.61 & 60.66 & 61.47 & 61.42 & 60.91 & 61.07 \\
Fusion & 62.28 & \underline{62.15} & \underline{62.18} & 62.60 & 62.56 & 62.28 & 62.39  \\[5mm]

\hline
\multicolumn{1}{l}{\textit{Dense Retrieval}} & & & & \\
Full-Wiki & 64.60 & 64.45 & 64.45 & 64.86 & 64.86 & 64.60 & 64.60 \\
% Original & 62.90 & 62.72 & 62.76 & 63.33 & 63.28 & 62.90 & 63.02 \\
Claim detection & \underline{61.50} & \underline{60.94} & \underline{60.94} & \underline{62.20} & \underline{62.20} & \underline{61.50} & \underline{61.50} \\
Citation Extraction & 59.67 & 59.40 & 59.46 & 60.13 & 60.09 & 59.67 & 59.82 \\
Fusion & 59.51 & 59.32 & 59.37 & 59.85 & 59.81 & 59.51 & 59.64  \\[5mm]

\hline
\multicolumn{1}{l}{\textit{Index Compression}} & & & & & & &  \\
Full-Wiki & 63.30 & 62.54 & 62.54 & 64.48 & 64.48 & 63.30 & 63.30 \\
% Original & 63.02 & 62.08 & 62.08 & 64.46 & 64.46 & 63.02 & 63.02  \\
Claim detection & \underline{61.92} & \underline{61.71} & \underline{61.71} & \underline{62.19} & \underline{62.19} & \underline{61.92} & \underline{61.93}  \\
Citation Extraction & 59.98 & 59.12 & 59.12 & 60.89 & 60.89 & 59.98 & 59.98   \\
Fusion & 61.58 & 61.43 & 61.43 & 61.75 & 61.75 & 61.58 & 61.58   \\[5mm]

\hline
\end{tabular}
\caption{Performance experiments on HoVer data and adjustments using full document text of English Wikipedia. The underlined-styled values represent the second best  within each retrieval setup.}
\label{tab:hover_performance_metrics}
\end{table}
% Full document text 
% 

% \begin{table}[htb!]
\centering
\footnotesize
\begin{tabular}{c c c c c c c c}
\multirow{2}{*}{Experiment setting} & \multirow{2}{*}{Accuracy} & \multicolumn{2}{c}{F1} & \multicolumn{2}{c}{Precision} & \multicolumn{2}{c}{Recall}  \\ 
\cline{3-8}
  & &  Weighted  & Macro & Weighted & Macro & Weighted & Macro      \\
\hline
\multicolumn{1}{l}{\textit{Sparse + Re-ranking}} & & & & \\
Full-Wiki & \textbf{63.69} &  \textbf{61.84} &  \textbf{55.24} &  \textbf{61.12} &  \textbf{56.54} &  \textbf{63.69} &  \textbf{55.32 } \\
Claim detection & 61.90 & 60.12 & \underline{53.33} & 59.27 & 54.26 & 61.90 & \underline{53.53} \\
Citation Extraction & 61.01 & 59.56 & 52.96 & 58.75 & 53.59 & 61.01 & 53.09 \\
Fusion & \underline{63.39} & \underline{60.21} & 52.48 & \underline{59.46} & \underline{54.69} & \underline{63.39} & 53.27  \\[5mm]

\hline
\multicolumn{1}{l}{\textit{Dense Retrieval}} & & & & \\
Full-Wiki &  61.61 & 60.95 & 55.21 & 60.47 & 55.49 & 61.61 & 55.13 \\
% Full-Wiki  & 60.42 & 58.80 & 51.96 & 57.90 & 52.58 & 60.42 & 52.19 \\
Claim detection & 61.01 & 58.94 & 51.78 & 57.96 & 52.70 & 61.01 & 52.17 \\
Citation Extraction & 58.63 & 58.48 & \underline{52.92} & 58.35 & 52.95 & 58.63 & \underline{52.91} \\
Fusion & \underline{61.31} & \underline{59.34} & 52.30 & 58.40 & \underline{53.23} & \underline{61.31} & 52.62  \\[5mm]

\hline
\multicolumn{1}{l}{\textit{Index Compression}} & & & & & & &  \\
Full-Wiki & 62.46 & 61.38 & 55.27 & 60.74 & 55.84 & 62.46 & 55.20  \\
% Original  & 60.46 & 60.63 & 55.64 & 60.81 & 55.60 & 60.46 & 55.70  \\
Claim detection & 59.31 & 59.02 & \underline{53.32} & 58.77 & 53.39 & 59.31 & \underline{53.30}  \\
Citation Extraction & 60.74 & 59.21 & 52.42 & 58.34 & 53.04 & 60.74 & 52.60  \\
Fusion & \underline{63.04} & \underline{59.79} & 51.89 & 58.94 & \underline{53.97} & \underline{63.04} & 52.76 \\[5mm]

\hline
\end{tabular}
\caption{Performance experiments on WiCE data and adjustments using full document text of English Wikipedia. The bold-styled values represent the baseline while the underlined-styled values represent the highest scores of the re-ranked data within a retrieval setup category.}
\label{tab:wice_performance_metrics}
\end{table}
% Full document text 
% 

% \paragraph{Sparse Retrieval performance} Utilising the metrics laid out in \autoref{sec:metrics}, the pipeline results have been evaluated for the different settings and laid out in \autoref{tab:hover_performance_metrics} for the HoVer experiments and \autoref{tab:wice_performance_metrics} for WiCE experiments. When comparing the different HoVer experiment settings within the Sparse Retrieval setup, Claim detection comes the closest to the baseline with close to 5.5 points difference across the metrics for the HoVer experiments. Important to note is that Fusion follows close with less than a point difference. For the WiCE Sparse retrieval setup, the opposite occurred with the Fusion data being the closest with a marginal 0.3 point difference followed by Claim detection with a 1.5 points difference. In both datasets, the Citation extraction takes the biggest loss in accuracy that being 6.9 points for HoVer and 2.7 points difference for WiCE. We can reason the fact that citation extraction takes the biggest performance degradation to the fact that not all claim-worthy sentences contain citations, therefore missing out on crucial evidence sentences. Unlike the other settings which consider the complete text instead of only the cited sentences and determine claim-worthiness on what the claim-detection model selects. Overall, relating to the inference time, we can see that for HoVer with a speedup of at least 1.5x to 1.6x, we only lose 6.9 to 5.5 points in performance across various metrics for the best re-ranking setup. Likewise, for WiCE, with a speedup of 1.4x to 1.6x we only lose 2.7 to 0.3 points. This positively demonstrates that indexing just the supporting facts does show meaningful results in terms of overall pipeline efficiency, while maintaining roughly the same performance. Additionally, this also indicates we can achieve good results by using a combination of citation extraction together with another supporting facts extraction method such as Claim detection.

% \paragraph{Dense Retrieval performance} When examining the performance results of Dense retrieval compared to Sparse Retrieval, it becomes evident that there is a slight decline across all experiments. For HoVer, this decline ranges from a modest 0.8 point difference in Claim detection to a more substantial 2.9 points in Fusion data. Similarly, WiCE experiences a loss ranging from a 0.9 difference in accuracy between Claim detection settings to approximately 2.4 points in Citation extraction. Crucially, it is to assess how these performances compare against the baselines. In HoVer, the accuracy loss ranges from 8.3 points for Fusion data to 6.3 points for Claim detection. WiCE experiences a loss ranging from 5.1 points in Citation extraction to 2.4 points in Fusion. These findings suggest that while transitioning from Sparse retrieval to just a Dense retrieval component incurs some loss, it's not substantial across various experiments involving supporting facts data. Moreover, the performance is notably strong in claim detection, while citation extraction lags behind by only a few points. Interestingly, while Fusion performs as well as Citation extraction in HoVer experiments, Fusion data outperforms Claim detection in WiCE. This highlights the significance of combining citation extraction with another supporting facts extraction method to achieve optimal results, similar to the Sparse retrieval setup.

% \paragraph{Sentence Retrieval stage ablation} Comparing experiments on the original data between the two retrieval methods reveals a more significant decline for HoVer, with a loss of 3.2 points with Sentence Selection and 4.9 points without it. For WiCE, the difference is 2.1 points with Sentence Selection and 3.3 points without it. When assessing these losses against the baselines, it becomes evident that both methods generally outperform the supporting facts data experiments by a few points. This suggests that the contribution of the Sentence Retrieval stage in the pipeline to performance improvement is marginal. With the supporting facts extraction thus becomes quite effective in achieving nearly the same performance. Consequently, to enhance efficiency, eliminating this Sentence Retrieval stage would result in only a loss of less than a few points.

% \subsection{Key Takeaways} 
% Incorporating supporting facts into both Sparse and Dense retrieval setups yields noteworthy enhancements in overall pipeline efficiency. Sparse retrieval setups demonstrate speedups ranging from up to around 1.5x, while Dense retrieval setups exhibit even more substantial improvements, achieving up to approximately 20.0x with GPU-based approaches. These notable speedups are primarily attributed to the removal of the Sentence Retrieval stage, which incurs considerable latency overhead. Further evaluation indicates a minor decline in performance when transitioning from Sparse to Dense retrieval, though the loss is not substantial. Specifically, claim detection remains robust, while citation extraction may lag behind by a few points. However, Fusion data yields promising results, often comparable to or outperforming other extraction methods, emphasizing the significance of amalgamating various extraction techniques for supporting facts. Moreover, ablation experiments on the Sentence Retrieval stage reveal its marginal contribution to performance improvement. Comparisons between original data and supporting facts data show only a slight decline in performance, showcasing that utilising only the supporting facts only incurs a modest loss in performance (around 6 points for HoVer and 3 points for WiCE). This suggests that although supporting facts do not affect document retrieval latency in the Dense Retrieval setup, it does help with overall pipeline latency due to avoiding the latency overhead of Sentence Selection. In conclusion, these results underscore the meaningful impact of indexing supporting facts on the overall pipeline efficiency, with only minimal losses in downstream fact-checking performance.

% %%%%%%%%%%%%%%%%%%%%%%%%%%%%%%%%%%%%%%%%%%%%%%%%%%%%%%%%%%%%%%%%%%%%%%%%%%%%%%%%%%%

% \section{RQ 3: In what ways does index compression enhance the efficiency of dense retrieval and fact-checking systems?}

% In this final research inquiry concerning the addition of index compression, this section explores how index compression improves upon Dense Retrieval in not only the constructed index size, but also document retrieval and total inference latency. Additionally, a final comparison will be made on the overall performance against Sparse retrieval and standard Dense Retrieval.

% \subsection{Compressed Index Size}
% In our FAISS experiments, we consistently observe an index size of approximately 7.51 GiB across all HoVer settings and 9.70 GiB across all WiCE settings. While one might anticipate that re-ranking would influence the amount of text utilized for generating vector embeddings, it's crucial to note that the index size remains unchanged. This is due to the fact that we generate vector embeddings on a per-article basis with only the text itself being altered. To address this issue, we employed JPQ, an index compression model. Despite using a relatively high number of subvectors for the JPQ model (M=96), we observed a significant reduction in the total index size. Specifically, the individual vector embeddings now occupy only 104.12 B in storage space, down from 1.5 KiB previously. This reduction is remarkable. For the HoVer experiments, the index size decreased from 7.51 GiB to 544.89 MiB, and for the WiCE experiments, we observed a decrease from 9.70 GiB to 672.95 MiB. Overall, this constitutes an impressive reduction of nearly 93\% or a compression ratio of 14.4:1 in index size for both experiment setups. It's worth noting that employing fewer sub-vectors could potentially lead to an even more substantial reduction in index size; however, this would come at the cost of decreased performance.

% \begin{table}[htb!]
\centering
\footnotesize
\begin{tabular}{l c c c c c c c c}
\hline
\multirow{2}{*}{Method} & \multicolumn{2}{c}{\makecell{Term-based \\ document retrieval}} & \multirow{2}{*}{\makecell{Sentence \\ Retrieval}} & \multirow{2}{*}{\makecell{Claim \\ Verification}} & \multicolumn{2}{c}{Total Latency}  & \multicolumn{2}{c}{Speedup} \\
\cline{2-3}\cline{6-7}\cline{8-9}
& CPU & GPU &  & &  CPU & GPU & CPU & GPU \\ 
\hline \hline
 \multicolumn{1}{l}{\colorg\textit{HoVer}} & \colorg & \colorg & \colorg & \colorg & \colorg & \colorg & \colorg & \colorg \\
 \textbf{Full-Wiki (S+R)} &  \multicolumn{2}{c}{\textbf{491  ms}} & \textbf{157  ms} & \textbf{7 ms} &  \multicolumn{2}{c}{\textbf{659 ms}} & - & - \\
Full-Wiki &  53 ms & 13 ms & 153 ms & 8 ms & 214  ms & 174 ms & 3.1x &  3.8x \\
% Original  &  55 ms & 13 ms  & - & 12 ms & 67 ms & 25 ms & 9.8x &  26.4x \\
Claim detection  &  51 ms & 12  ms  & - &  9 ms & 60 ms & 21 ms & 11.0x & 31.4x  \\
Citation Extraction  & 46  ms & 11 ms  & - & 9 ms & 51 ms & 20 ms & 12.9x & 33.0x \\
Fusion  & 51  ms & 12 ms  & - & 12 ms & 63 ms & 24 ms &  10.5x & 27.5x \\
\hline
\multicolumn{1}{l}{\colorg\textit{WiCE}} & \colorg & \colorg & \colorg & \colorg & \colorg & \colorg & \colorg & \colorg  \\
 \textbf{Full-Wiki (S+R)} & \multicolumn{2}{c}{\textbf{636 ms}} & \textbf{186 ms} & \textbf{9  ms}  & \multicolumn{2}{c}{\textbf{831  ms}} & - & - \\
Full-Wiki & 97 ms & 43 ms & 186 ms & 9 ms & 292 ms & 238 ms & 2.8x  &  3.5x \\
% Original  &  95 ms & 43  ms  & - & 11 ms & 106 ms & 54 ms & 7.8x &  15.4x \\
Claim detection  &  92 ms & 37 ms  & - & 11 ms & 103 ms & 48 ms & 8.1x & 17.3x \\
Citation Extraction  & 89  ms & 37 ms  & - & 9 ms & 98 ms & 46 ms & 8.5x & 18.1x \\
Fusion  & 89  ms & 37  ms  & - &  9 ms & 98 ms & 46  ms & 8.5x &  18.1x \\
\hline
\end{tabular}
\caption{Retrieval and inference latency for Index compression setup. Speedup is compared to the total latency of (S+R) pipeline with Full-Wiki setup.}
\label{tab:jpq_latency}
\vspace{-2em}
\end{table}



% \subsection{Pipeline Efficiency}
%  \paragraph{Document Retrieval Latency} When examining the retrieval latency outlined in \autoref{tab:jpq_latency}, a notable observation can be made towards the Dense document retrieval compared to the Dense Retrieval results outlined in \autoref{tab:faiss_latency}. This significant enhancement can be primarily attributed to the utilization of the index compression model, which effectively reduces the index size. As a result, retrieval latency experiences a considerable improvement due to the smaller vector embeddings, facilitating faster similarity computation. Here one can observe a substantial speedup achieved in CPU retrieval of approximately 10.0x across the HoVer experiment settings and 7.0x for WiCE experiments. Similarly, GPU retrieval exhibits a speedup of approximately 2.0x for HoVer experiments and 0.8x for WiCE experiments. This is generally in line with the reported results in the original JPQ paper \cite{zhan2021jointly}. Although the measurements for HoVer fall in line with these reported results, one may notice that the WiCE retrieval speedup is lower than that of HoVer. This is even worse for the GPU-based retrieval latency instead of being better than the standard GPU-based Dense retrieval. We reason this to the fact that the WiCE claim dataset is a lot more complex. In the original WiCE paper, the results that were reported already indicate a not so particularly high performance being achieved. This coupled with the use of a different model for creating the embeddings results in marginally worse performance instead of a speedup such as the case with HoVer. 
 
%  \paragraph{Pipeline Inference Latency} In examining the total inference latency, as further detailed in \autoref{tab:jpq_latency}, the utilization of compressed indexing and the ensuing document retrieval speed enhancements result in a notable boost across the board. The advancements brought about by JPQ, which further build upon the foundations of Dense Retrieval, are particularly significant. Notably, CPU latency has seen a substantial improvement compared to previous benchmarks on the supporting fact data, exhibiting a noteworthy speedup ranging from 10.5x to 12.9x for HoVer experiments, and 8.1x to 8.5x for WiCE experiments relative to their respective baselines. Meanwhile, the GPU-based approach, especially in the case of HoVer experiments, has yielded even more impressive results, achieving speedups ranging from 27.5x to 33.0x. While WiCE experiments on the GPU may not experience such dramatic speedups, they still showcase marked enhancements over their original baselines that range from 17.3x to 18.1x speedups. When assessing the impact of the Sentence Selection stage on the original corpus data settings, the findings reinforce the observations made in the standard Dense Retrieval setup. Furthermore, the disparity in the reported speedups between the tables underscores the significance of incorporating index compression. 
  
% \subsection{Performance Metrics Evaluation}
% When comparing the performance of JPQ in the HoVer experiments (as shown in \autoref{tab:hover_performance_metrics}) as well as the performance of the WiCE experiments (presented in \autoref{tab:wice_performance_metrics}), a notable trend emerges. The index compression brought by JPQ generally yields higher scores compared to the standard Dense retrieval experiments. This improvement is particularly striking as the gap between the JPQ experiments and the baseline performances is further narrowed. In the HoVer experiments, this enhancement ranges from marginal increases of less than a point in Claim detection and Citation extraction to a significant 2-point boost in the Fusion data. Conversely, in the WiCE experiments, while Claim detection experiences a slight decline of almost 2 points, Citation extraction and Fusion demonstrate the opposite trend.
% Typically, one might expect index compression techniques to yield inferior results compared to the standard Dense retrieval setup due to the lossy nature of compressing embeddings. However, a straightforward explanation for this unexpected improvement lies in the utilization of different pre-trained models for generating the embeddings. In the standard Dense retrieval, we rely on the all-MiniLM-L6-v2 model, which maps sentences and paragraphs to a 384-dimensional dense vector space. In contrast, the JPQ model employed for index compression initially generates embeddings of size 768 and subsequently reduces the embedding size using PQ centroids to achieve smaller vector sizes. Furthermore, it's worth noting that JPQ learns the index for the query vectors, unlike the approach in standard Dense retrieval where the index is kept separate. The latter essentially operates in a zero-shot inference manner, as we do not fine-tune the encoders on specific datasets but instead store and retrieve the created embeddings directly in our FAISS setup.


% \subsection{Key Takeaways} 
% Enhancing Dense retrieval through the use of index compression via the JPQ model has remarkably reduced the index size for Dense retrieval by a substantial 93\%. Further analysis indicates significant speedups of up to 10.0x for the CPU-based approach, while the GPU-based approach achieves a modest speedup of up to 2.0x in the HoVer experiments. However, it experiences a slight slowdown in the WiCE experiments. A huge emphasis on achieving efficiency is particularly pertinent in the context of CPU-based Dense Retrieval with index compression. Here the latency times of the CPU-based approach come in close to the GPU-based approach. These findings not only signify efficiency gains concerning resource utilization for index storage, but also pave the way for experiments on lower-end machines especially ones lacking GPU capabilities. Thereby maximizing the benefits of CPU-based methodologies. Regarding performance, experiments involving index compression generally outperform standard Dense retrieval. This superiority can be attributed to the utilization of different pre-trained models and learned index techniques, resulting in slightly enhanced outcomes.


% \section{Key Findings and Conclusions}

\section{Conclusion}\label{sec:con}

Our work contributes empirical insights on the photorealism of AI-generated images and a taxonomy of artifacts commonly found in AI-generated images, organized into five categories: anatomical implausibilities, stylistic artifacts, functional implausibilities, violations of physics, and sociocultural implausibilities. We find that the photorealism of AI-generated images depends on the scene complexity of the image, the kind of artifacts and implausibilities, if any, detectable in an image, the duration of visual attention to an image, and the quality of human effort to select appropriate prompts and curate images. A question such as ``How photorealistic are state-of-the-art diffusion models'' may sound simple, but the answer is more complex and depends on many details, including what images are generated and selected, how photorealism is measured, what real images are included in the experiment, and how much time, skill, and effort a human participant has and willing to offer. This paper offers an initial exploration into how we can address this question and develops a practical taxonomy that offers scaffolding for building AI--literacy interventions to help people navigate the capabilities and limitations of diffusion models and whether an image is AI-generated or not. 

\begin{acks}

This material is based upon work supported by Robert Pozen, and in part with funding from the Department of Defense (DoD). Any opinions, findings, conclusions, or recommendations expressed in this material are those of the authors and do not necessarily reflect the views of the DoD or any agency or entity of the United States Government. We thank Will Thompson from Kellogg Research Support for performing a replication check.
\end{acks}

\bibliographystyle{ACM-Reference-Format}
\bibliography{XX-references}

\end{document}
