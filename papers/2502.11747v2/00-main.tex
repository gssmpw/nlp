% \documentclass[sigconf,review,anonymous]{acmart}
% \usepackage[a-1b]{pdfx}
\documentclass[sigconf]{acmart}
\copyrightyear{2025}
\acmYear{2025}
\setcopyright{rightsretained}

%% These commands are specific for your submission.
\acmConference[SIGIR '25]{Proceedings of the 48th International ACM SIGIR Conference on Research and Development in Information Retrieval}{July 13--18, 2024}{Padua, Italy.}
\acmBooktitle{Proceedings of the 48th Int'l ACM SIGIR Conference on Research and Development in Information Retrieval (SIGIR '25), July 13--18, 2024, Padua, Italy}




\makeatother


%
% inline lists.  usage,
%    \begin{inlinelist}
%       \item first item,
%       \item second item, and
%       \item last item.
%    \end{inlinelist}
%
\usepackage{enumitem}
\newlist{inlinelist}{enumerate*}{1}
\setlist*[inlinelist,1]{%
  label=(\roman*),
}
\usepackage{subfigure}
\usepackage{booktabs}
\usepackage{multirow}
\usepackage{array}
\usepackage{makecell}
\usepackage{colortbl}
\usepackage{xcolor}
\usepackage{bbm}
\usepackage{arydshln}

\newcommand{\movie}{Movie\xspace}
\newcommand{\landmark}{Landmark\xspace}
\newcommand{\person}{Person\xspace}


\newcommand{\gptfouro}{GPT-4o\xspace}
\newcommand{\gptfouromini}{GPT-4o-mini\xspace}

\settopmatter{printacmref=false}



\title{Open-Ended and Knowledge-Intensive Video Question Answering}

% \title{Multi-Modal Retrieval Augmentation for Open-Ended and Knowledge-Intensive Video Question Answering}



\author{Md Zarif Ul Alam}
\affiliation{\institution{University of Massachusetts Amherst}
\country{United States}}
\email{zarifalam@cs.umass.edu}

\author{Hamed Zamani}
\affiliation{\institution{University of Massachusetts Amherst}
\country{United States}}
\email{zamani@cs.umass.edu}




\begin{document}



% \fancyhead{}

% \begin{abstract}
% While current video question answering systems perform well on some tasks requiring only direct visual understanding, they struggle with questions demanding knowledge beyond what is immediately observable in the video content. We refer to this challenging scenario as knowledge-intensive video question answering (KI-VideoQA), where models must retrieve and integrate external information with visual understanding to generate accurate responses. This work presents the first attempt to (1) study multi-modal retrieval-augmented generation for KI-VideoQA, and (2) go beyond multi-choice questions by studying open-ended questions in this task. Through an extensive empirical study of state-of-the-art retrieval and vision language models in both zero-shot and fine-tuned settings, we explore how different retrieval augmentation strategies can enhance knowledge integration in KI-VideoQA. We analyze three key aspects: (1) model's effectiveness across different information sources and modalities, (2) the impact of heterogeneous multi-modal context integration, and (3) model's effectiveness across different query formulation and retrieval result consumption. Our results suggest that while retrieval augmentation generally improves performance, its effectiveness varies significantly based on modality choice and retrieval strategy. Additionally, we find that successful knowledge integration often requires careful consideration of query formulation and optimal retrieval depth. Our exploration advances state-of-the-art accuracy for multiple choice questions by over 17.5\% on the KnowIT VQA dataset.
% % As a result of our analysis, we believe that KI-VideoQA systems require specialized approaches to multi-modal retrieval and context integration.
% % Finally, our evaluation in open-ended settings reveals important insights about Vision Language Model's ability to generate faithful and coherent responses while drawing on external knowledge.
% \end{abstract}


\begin{abstract}
Video question answering that requires external knowledge beyond the visual content remains a significant challenge in AI systems. While models can effectively answer questions based on direct visual observations, they often falter when faced with questions requiring broader contextual knowledge. To address this limitation, we investigate knowledge-intensive video question answering (KI-VideoQA) through the lens of multi-modal retrieval-augmented generation, with a particular focus on handling open-ended questions rather than just multiple-choice formats. Our comprehensive analysis examines various retrieval augmentation approaches using cutting-edge retrieval and vision language models, testing both zero-shot and fine-tuned configurations. We investigate several critical dimensions: the interplay between different information sources and modalities, strategies for integrating diverse multi-modal contexts, and the dynamics between query formulation and retrieval result utilization. Our findings reveal that while retrieval augmentation shows promise in improving model performance, its success is heavily dependent on the chosen modality and retrieval methodology. The study also highlights the critical role of query construction and retrieval depth optimization in effective knowledge integration. Through our proposed approach, we achieve a substantial 17.5\% improvement in accuracy on multiple choice questions in the KnowIT VQA dataset, establishing new state-of-the-art performance levels.
\end{abstract}

% \keywords{Video Question Answering; Retrieval Augmented Generation; Vision Language Models; Multi-Modal Question Answering}

% \begin{CCSXML}
% <ccs2012>
% <concept>
% <concept_id>10002951.10003317</concept_id>
% <concept_desc>Information systems~Information retrieval</concept_desc>
% <concept_significance>500</concept_significance>
% </concept>
% <concept>
% <concept_id>10010147.10010257</concept_id>
% <concept_desc>Computing methodologies~Machine learning</concept_desc>
% <concept_significance>500</concept_significance>
% </concept>
% </ccs2012>
% \end{CCSXML}

% \ccsdesc[500]{Information systems~Information retrieval}
% \ccsdesc[500]{Computing methodologies~Machine learning}

\maketitle

\section{Introduction}

Tutoring has long been recognized as one of the most effective methods for enhancing human learning outcomes and addressing educational disparities~\citep{hill2005effects}. 
By providing personalized guidance to students, intelligent tutoring systems (ITS) have proven to be nearly as effective as human tutors in fostering deep understanding and skill acquisition, with research showing comparable learning gains~\citep{vanlehn2011relative,rus2013recent}.
More recently, the advancement of large language models (LLMs) has offered unprecedented opportunities to replicate these benefits in tutoring agents~\citep{dan2023educhat,jin2024teach,chen2024empowering}, unlocking the enormous potential to solve knowledge-intensive tasks such as answering complex questions or clarifying concepts.


\begin{figure}[t!]
\centering
\includegraphics[width=1.0\linewidth]{Figs/Fig.intro.pdf}
\caption{An illustration of coding tutoring, where a tutor aims to proactively guide students toward completing a target coding task while adapting to students' varying levels of background knowledge. \vspace{-5pt}}
\label{fig:example}
\end{figure}

\begin{figure}[t!]
\centering
\includegraphics[width=1.0\linewidth]{Figs/Fig.scaling.pdf}
\caption{\textsc{Traver} with the trained verifier shows inference-time scaling for coding tutoring (detailed in \S\ref{sec:scaling_analysis}). \textbf{Left}: Performance vs. sampled candidate utterances per turn. \textbf{Right}: Performance vs. total tokens consumed per tutoring session. \vspace{-15pt}}
\label{fig:scale}
\end{figure}


Previous research has extensively explored tutoring in educational fields, including language learning~\cite{swartz2012intelligent,stasaski-etal-2020-cima}, math reasoning~\cite{demszky-hill-2023-ncte,macina-etal-2023-mathdial}, and scientific concept education~\cite{yuan-etal-2024-boosting,yang2024leveraging}. 
Most aim to enhance students' understanding of target knowledge by employing pedagogical strategies such as recommending exercises~\cite{deng2023towards} or selecting teaching examples~\cite{ross-andreas-2024-toward}. 
However, these approaches fall short in broader situations requiring both understanding and practical application of specific pieces of knowledge to solve real-world, goal-driven problems. 
Such scenarios demand tutors to proactively guide people toward completing targeted tasks (e.g., coding).
Furthermore, the tutoring outcomes are challenging to assess since targeted tasks can often be completed by open-ended solutions.



To bridge this gap, we introduce \textbf{coding tutoring}, a promising yet underexplored task for LLM agents.
As illustrated in Figure~\ref{fig:example}, the tutor is provided with a target coding task and task-specific knowledge (e.g., cross-file dependencies and reference solutions), while the student is given only the coding task. The tutor does not know the student's prior knowledge about the task.
Coding tutoring requires the tutor to proactively guide the student toward completing the target task through dialogue.
This is inherently a goal-oriented process where tutors guide students using task-specific knowledge to achieve predefined objectives. 
Effective tutoring requires personalization, as tutors must adapt their guidance and communication style to students with varying levels of prior knowledge. 


Developing effective tutoring agents is challenging because off-the-shelf LLMs lack grounding to task-specific knowledge and interaction context.
Specifically, tutoring requires \textit{epistemic grounding}~\citep{tsai2016concept}, where domain expertise and assessment can vary significantly, and \textit{communicative grounding}~\citep{chai2018language}, necessary for proactively adapting communications to students' current knowledge.
To address these challenges, we propose the \textbf{Tra}ce-and-\textbf{Ver}ify (\textbf{\model}) agent workflow for building effective LLM-powered coding tutors. 
Leveraging knowledge tracing (KT)~\citep{corbett1994knowledge,scarlatos2024exploring}, \model explicitly estimates a student's knowledge state at each turn, which drives the tutor agents to adapt their language to fill the gaps in task-specific knowledge during utterance generation. 
Drawing inspiration from value-guided search mechanisms~\citep{lightman2023let,wang2024math,zhang2024rest}, \model incorporates a turn-by-turn reward model as a verifier to rank candidate utterances. 
By sampling more candidate tutor utterances during inference (see Figure~\ref{fig:scale}), \model ensures the selection of optimal utterances that prioritize goal-driven guidance and advance the tutoring progression effectively. 
Furthermore, we present \textbf{Di}alogue for \textbf{C}oding \textbf{T}utoring (\textbf{\eval}), an automatic protocol designed to assess the performance of tutoring agents. 
\eval employs code generation tests and simulated students with varying levels of programming expertise for evaluation. While human evaluation remains the gold standard for assessing tutoring agents, its reliance on time-intensive and costly processes often hinders rapid iteration during development. 
By leveraging simulated students, \eval serves as an efficient and scalable proxy, enabling reproducible assessments and accelerated agent improvement prior to final human validation. 



Through extensive experiments, we show that agents developed by \model consistently demonstrate higher success rates in guiding students to complete target coding tasks compared to baseline methods. We present detailed ablation studies, human evaluations, and an inference time scaling analysis, highlighting the transferability and scalability of our tutoring agent workflow.




\section{Related Work}
Our work builds on several lines of research: exploring the role of text with visualizations, visualization and text systems, and image and text authoring interfaces.

\subsection{The role of text with visualizations}
The interplay between text and visual elements in data visualization has been a significant area of interest with increased advocacy for treating text as co-equal to visualization~\cite{stokesgive, lundgard2021accessible}. Kim et al.~\cite{kim2021towards} conducted a study to understand how readers integrate charts and captions in line charts. The study findings indicated that when both the chart and text emphasize the same prominent features, readers take away insights from both modalities. Their research underscores the importance of coherence between visual and textual elements and how external context provided by captions can enhance the reader's comprehension of the chart's message. Building on these insights, Lundgard and Satyanarayan~\cite{lundgard2021accessible} proposed a four-level model for content conveyed by natural language descriptions of visualizations. Their model delineates semantic content into four distinct levels: elemental and encoded properties (Level 1), statistical concepts (Level 2), perceptual and cognitive phenomena (Level 3), and contextual insights (Level 4).

Focusing on the role of textual annotations in visualization, Stokes et al.~\cite{stokes2022striking} observed that readers favored heavily annotated charts over less annotated charts or text alone. This preference highlights the added value of textual annotations in aiding data interpretation, with specific emphasis on how different types of semantic content impact the takeaways drawn by readers. Further contributions by Quadri et al.~\cite{quadri2024you} and Fan et al.~\cite{fan2024understanding} explored high-level visualization comprehension and the impact of text details and spatial autocorrelation on reader takeaways in thematic maps. These studies collectively underline the critical role of textual elements in shaping viewer perceptions, understanding, and accessibility of visual data. Ottley et al.~\cite{ottley2019curious} and Stokes et al.~\cite{stokes2023role} have also contributed to this body of research, focusing on how annotations influence perceptions of bias and predictions, reinforcing the multifaceted impact of text on visual data interpretation.

Our work further explores how text and charts can be better aligned with one another by offering a mixed-initiative authoring interface. Specifically, \pluto~allows leveraging both direct manipulation interactions and user-drafted text to generate recommendations for communicative text and chart design. Furthermore, \pluto's text recommendations explicitly incorporate Lundgard and Satyanarayan's model~\cite{lundgard2021accessible} for semantic information conveyed by visualization descriptions.
In doing so, the system ensures that the generated text has good semantic coverage and structure (e.g., generated descriptions start by conveying the chart's encodings and then list high-level trends) and is appropriate for the intended communicative use (e.g., the semantic information conveyed by titles is different from descriptions accompanying a chart or annotations on the chart).


\subsection{Visualization and text systems}

The integration of visualization and text has led to the development of various systems designed to facilitate the creation, interpretation, and enhancement of data visualizations with textual elements. He et al.~\cite{he2024leveraging} surveyed the leveraging of large models for crafting narrative visualizations, highlighting the potential of AI in supporting the narrative aspect of data visualization. This is complemented by AutoTitle, an interactive title generator for visualizations~\cite{liu2023autotitle}, and Vistext, a benchmark for semantically rich chart captioning~\cite{tang2023vistext}. VizFlow demonstrates the effectiveness of facilitating author-reader interaction by dynamically connecting text segments to corresponding chart elements to help enrich the storytelling experience~\cite{sultanum2021}. This body of research highlights the need for tools to support more nuanced integration of text and visualization.

Supporting the co-authoring of text and charts, Latif et al. introduced Kori~\cite{latif2021kori}, an interactive system for synthesizing text and charts in data documents, emphasizing the seamless integration of visual and textual data for enhanced communication.
\new{CrossData~\cite{chen2022crossdata} presents an interactive coupling between text and data in documents, enabling actions based on the document text and adjusting data values in the text through direct manipulation on the chart.
Such systems illustrate the potential for the bidirectional linking between text and charts to assist rich authoring of data-driven narratives.
}
Furthermore, systems like EmphasisChecker~\cite{kim2023emphasischecker}, Intentable~\cite{choi2022intentable}, Chart-to-text~\cite{obeid2020chart}, DataDive~\cite{kim2024datadive},
\new{InkSight~\cite{lin2023inksight}},
and FigurA11y~\cite{singh2024figura11y} focus on guiding chart and caption creation, supporting readers' contextualization of statistical statements, and assisting in writing scientific alt text. Recent work like SciCapenter supports the composition of scientific figure captions using AI-generated content and quality ratings \cite{hsu2024scicapenter}.
DataTales~\cite{sultanum2023datatales} is another example of a recent system using a large language model for authoring data-driven articles, indicating the growing interest in AI-assisted data storytelling.
These systems collectively illustrate the expanding scope of text integration into visualization, from enhancing data document creation to improving accessibility and data-driven communication. Reviewing the aforementioned tools and the use of generative AI for visualization more broadly, Basole and Major~\cite{basole2024generative} discuss how generative AI methods and tools offer creativity assistance and automation within the visualization workflow, specifically highlighting a shift towards ``human-led AI-assisted'' paradigms, where generative AI not only augments the creative process but also becomes a co-creator.

Aligned with this paradigm shift, \pluto~adopts a mixed-initiative approach that leverages the capabilities of generative AI to help create semantic alignment between the chart and its corresponding text for effective data-driven communication.
However, \pluto~differs from existing chart-and-text authoring tools in three significant ways.
First, going beyond existing systems that primarily leverage unimodal information from the chart to generate text, \pluto~supports multimodal authoring combining information from both the chart (including any direct interactions with marks) and user-drafted text.
Furthermore, unlike prior tools that focus exclusively on generating complete descriptions/captions or titles, \pluto's recommendations can be leveraged in flexible ways to author not only titles and descriptions but also more fine-grained annotations and sentence completions. Second, while existing tools primarily recommend text for a given chart, \pluto's recommendations are bidirectional.
Specifically, the system suggests chart design changes like sorting or adding embellishments based on the authored text, resulting in artifacts that more clearly communicate takeaways via a combination of text and charts. Lastly, unlike existing tools that primarily rely on pre-trained knowledge in generative AI models, \pluto's recommendations are grounded in a theoretical research-based model of semantic information conveyed in visualization text~\cite{lundgard2021accessible}, ensuring the generated text covers the appropriate level of detail and is effective for communication \emph{alongside} the chart.

% \gabis{Where do we define our notion of framing and relevant terms? Will that happen in the intro? For example here we use the term ``sentiment shifts'', which I think requires defintion.}\gili{done in intro}

% \gabis{Recurring comment - we should change tense to present, while most of the paper is currently in the past tense, I indicated this in some places, but should verify throughout.}

% \begin{figure*}[htbp]
    \centering
    % First Subfigure
    \begin{subfigure}{0.49\textwidth} % Adjust width as needed
        \centering
        \includegraphics[width=\textwidth]{images/orig_negative_models_distribution.png} % Replace with your image path
        \caption{Sentences that are \textbf{negative} in their original form.}
        \label{fig:negative-flip}
    \end{subfigure}
    % \hfill % Adds horizontal space between subfigures
    % Second Subfigure
    \begin{subfigure}{0.49\textwidth}
        \centering
        \includegraphics[width=\textwidth]{images/orig_positive_models_distribution.png} % Replace with your image path
        \caption{Sentences that are \textbf{positive} in their original form.}
        \label{fig:positive-flip}
    \end{subfigure}
    \caption{Proportion of sentences for which LLMs flipped sentiment, became neutral, or retained the original sentiment when presented with opposite sentiment framing. For example, this measures the percentage of sentences originally labeled as positive, that were labeled as negative after applying negative framing (and vice versa).
    }
    \label{fig:flip-proportion}
\end{figure*}


Our dataset curation consists of three steps, as depicted in Figure~\ref{fig:fig1}. First, we collect natural, real-world statements, with some clear sentiment, either positive or negative (\S\ref{sec:base-statements}; e.g., ``I won the highest prize'' as positive). Next, 
we reframe each statement by adding a prefix or suffix conveying the opposite sentiment
% for each statement, we add a framing that conveys the opposite sentiment to the base statement 
(\S\ref{sec:adding-framing}; e.g., ``I won the highest prize, although I lost all my friends on the way''). Finally, we collect large-scale human annotations via crowdsourcing, to label the sentiment shifts when wrapping the statements with the opposite framing (\S\ref{sec:human-annotations}; e.g., labeling ``negative'' the statement ``I won the highest prize, although I lost all my friends on the way''). 
%\gabis{I think we can remove the textual examples here to save space}

The complete dataset consists of 1000 statements, in which 500 are statements that their base form has positive sentiment, and 500 are base negative statements. 




\subsection{Collecting Base Statements}\label{sec:base-statements}
First, we collect base statements, which convey a clear sentiment, either clearly positive or clearly negative statements. We use \spike{} -- an extractive search system, which allows to extract statements from real-world datasets~\cite{taub-tabib-etal-2020-interactive}.
%\gabis{there's also a citation for spike}.\footnote{~\url{https://spike.apps.allenai.org}} 
Specifically, we collect statements from Amazon Reviews dataset, which are naturally occurring, sentiment-rich, texts but are less likely to trigger strong preexisting biases or emotional reactions, which may be a confound for our experiment.\footnote{~\url{https://spike.apps.allenai.org/datasets/reviews}} 
% \gabis{Why did we use this specifically? I think once we write the intro it would be good to relate to what we wrote there and how this domain is relevant.}
\begin{figure}[tb!]
    \centering
    \includegraphics[width=\linewidth]{images/roberta_score_before_after_framing.png}
    \caption{Distribution of sentiment scores before and after applying opposite-sentiment framing, as detailed in Section~\ref{sec:adding-framing}. Prior to framing, base sentences exhibit a clear polarity (positive or negative), whereas the application of opposite framing introduces ambiguity, shifting the sentiment scores toward a less distinct polarity.}
    \label{fig:pos-score-dist}
\end{figure}


Using \spike, we extract ${\sim}6k$ statements that fulfilled our designated queries, which we found correlated with clear sentiment. We designed the queries to capture positive or negative verbs that describe actions with some clear sentiment (e.g., ``enjoy'' or ``waste''), or statements with positive or negative adjective, describing an outcome with a clear sentiment (e.g., ``good'' or ``nasty''). The patterns and queries used for extraction are detailed in Appendix~\ref{sec:appendix-spike}.
% \gabis{needs more details, what are our queries? What were we aiming for? I understand that at a high level we're looking for clear sentiment, but how do we achieve this via lexical-syntactic queries?}. 
Next, we run in-house annotations to label and filter the extracted statements, to handle negations or other cases where the statement does not convey a clear sentiment. 
The filtering process results in $1,301$ positive statements, and $1,229$ negative statements.


\subsection{Adding Framing}\label{sec:adding-framing}

To reframe the statements in our dataset, we use GPT-4~\cite{achiam2023gpt}.\footnote{We used the gpt-4-0613 version.} 
% \gabis{do we have more details about which GPT4? what date?}
% The model was asked to keep he base statement unchanged, and add some prefix or suffix, that can be either positive or negative, oppositely to the base statement sentiment (e.g., I won the highest proze, althoug I lost all my friends on the way). 
The input prompt includes a 1-shot example, followed by a task description ``Add a <SENTIMENT> suffix or prefix to the given statement. Don't change the original statement.'', where SENTIMENT is either ``positive'' or ``negative'', opposite to the base statement sentiment (i.e., positive framing for negative base statement, and vice versa).

Unlike the base statement, the conveying sentiment of reframed statements is more ambiguous and there is no one clear label, as shown in Figure~\ref{fig:pos-score-dist}.\footnote{Scores in Figure~\ref{fig:pos-score-dist} are given by a fine-tuned sentiment analysis model ~\url{https://huggingface.co/cardiffnlp/twitter-roberta-base-sentiment-latest}}
%as we present the sentiment scores assigned by a fine-tuned sentiment analysis model,\footnote{~\url{https://huggingface.co/cardiffnlp/twitter-roberta-base-sentiment-latest}} %that was shown to be state-of-the-art when fine-tuned on sentiment analysis~\cite{csanady2024llambert}. 
% We present the sentiment scores 
% before and after reframing. It shows that wrapping the statement with the opposite sentiment injects ambiguity to the overall sentiment, as the sentiment scores become more dispersed. 
The exhibeted ambiguity in sentiment allows us to measure to what extent LLMs' shifting sentiment after framing, and how correlated it is to human behavior.



% In Figure~\ref{fig:pos-score-dist}, \gabis{Is roberta SOTA? it's a bit old by now. Do we have a reference to back this up?}\footnote{RoBERTa, fine-tuned for sentiment analysis~\url{https://huggingface.co/cardiffnlp/twitter-roberta-base-sentiment-latest}} The base statement scores are predominantly centered around binary values, either strongly positive or strongly negative. In contrast, the sentiment scores after opposite framing are more dispersed, reflecting increased ambiguity in sentiment. 
% \gabis{I'm not sure if this paragraph belongs here, maybe should be a subsection on its own at the end of the section?}


\subsection{Collecting Human Annotations}\label{sec:human-annotations}

In the final step, we collect human annotations through Amazon Mechanical Turk to evaluate the framing effect in \name{} over human participants, providing a reference for comparison with LLMs.\footnote{\url{https://www.mturk.com}} 
Details about the annotation platform are elaborated in Appendix~\ref{sec:mturk-appendix}.

The complete dataset includes 1K statements, each annotated by five different annotators. Given our budget, we preferred to collect five annotations per statement, resulting in less statements, but providing a more robust scoring for the ambiguity of a statement.

% We select a pool of 10 qualified workers who successfully passed our qualification test, which consisted of 20 base statements (unframed), for which annotators were expected to achieve perfect accuracy. The estimated hourly wage for the entire experiment was approximately 14USD per hour. More details about the annotation platform can be found in Appendix~\ref{sec:mturk-appendix}. Given our budget, we preferred to collect five annotations per statement, resulting in less statements, but providing a more robust scoring for the ambiguity of a statement.

For the annotation process, each statement in our dataset is presented in its reframed version (i.e., positive base statements with negative framing and vice versa), to five different annotators. This setup generates, for each dataset instance, a score ranging from 0 to 5, representing the number of annotators that votes for the sentiment that aligns with the opposite framing, which means that the overall sentiment of the reframed statement has shifted from its base sentiment. For example, in Figure~\ref{fig:fig1}, the statement ``I won the highest prize, although I lost all my friends on the way'' is shown to have two annotators voting ``negative'', which aligns with the sentiment of the framing and not the base statement, so the label for that instance in \name{} would be 2 (sentiment shifts).

% \gabist{It is important to note that there is no definitive ``right'' or ``wrong'' label for these statements, as the opposite sentiment framing often renders the sentiment conveyed highly ambiguous.}
Instances with score near 0 indicate that annotators agree that the overall sentiment remains unchanged despite the opposite framing. Score closer to 5 indicates that annotators agree that reframing shifted the perceived sentiment, while score around 2-3 suggests that the opposite framing makes the sentiment ambiguous.



\section{Multi-Modal RAG for KI-VideoQA}
\label{sec:method}
In this work, we address the task of knowledge-intensive open-ended video question answering. Formally, given an input video $V$, a question $q$ about the video, and a corpus or knowledge source $\mathcal{K}$, the task is to answer the question. Note that in this task, questions cannot be merely answered by the given video; answering the question requires external knowledge. An example of this task is presented in Figure~\ref{fig:method_overview} (top left corner).

\paragraph{\textbf{Inputs}}
Let $V = [f_1, f_2, ..., f_t]$ be a video sequence of $t$ frames, where each frame $f_t \in \mathbb{R}^{h \times w \times 3}$ represents an RGB image with the height of $h$ and width of $w$ pixels. Each video is accompanied by subtitles ($S$) between the start and end timestamp for that video. Each question $q$ is a natural language sequence of $l$ tokens ${w_1, w_2, \cdots, w_l}$.

\paragraph{\textbf{Knowledge Sources}}
A knowledge source $\mathcal{K}$ is a collection of $m$ information items: $\mathcal{K} = \{k_1, k_2, ..., k_m\}$, where each information item may contain information needed to answer some questions. Each information item can be either textual (e.g., unstructured text documents) or visual (e.g., videos). This paper studies these and even multi-modal knowledge source situations which is a mixture of both textual and visual information items. 

\paragraph{\textbf{Output}}
Our work encompasses two distinct question answering tasks: (1) \textbf{Multiple Choice Questions}: Given a set of candidate answers $\mathcal{A} = {a^1, a^2, ..., a^K}$ where $K$ is typically equal to 4, the task is to select one correct answer $a^* \in \mathcal{A}$. (2) \textbf{Open-Ended Questions}: The expected output is a free-form text with the length of $p$ as the answer: $a = {y_1, y_2, \cdots, y_p}$, where each $y_i \in \mathcal{V}$ is a token from the vocabulary $\mathcal{V}$.


% \begin{enumerate}
%     \item \textbf{Multiple Choice Questions}: Given a set of candidate answers $\mathcal{A} = {a^1, a^2, ..., a^K}$ where $K$ is typically equal to 4, the task is to select one correct answer $a^* \in \mathcal{A}$. 
% %     The model outputs a probability distribution over the candidates:
% % \begin{equation}
% % P(a^k|\mathcal{V}, q, \mathcal{K}) \quad \text{for } k \in {1,...,K}
% % \end{equation}

%     \item \textbf{Open-Ended Questions}: The expected output is a free-form text with the length of $p$ as the answer: $a = {y_1, y_2, \cdots, y_p}$, where each $y_i \in \mathcal{V}$ is a token from the vocabulary $\mathcal{V}$.
% \end{enumerate}

Even though multiple choice questions for KI-VideoQA has been briefly explored in \citet{garcia2020knowit}, prior work has not studied retrieval-augmented solutions. Moreover, to the best of our knowledge, no prior work explored open-ended question answering in KI-VideoQA. 





% \subsection{Multi-Modal RAG for KI-VideoQA}
One approach to answer questions is to feed the given question and video (or its subtitle) to a VLM for answer generation. This approach has been studied in \cite{garcia2020knowit,wu2021transferring}. However, we believe that this approach cannot perform well for many knowledge-intensive questions. If the LLM has observed and learned the ``knowledge'' required for answering the question, this approach is likely to generate a correct answer, however, this is not always possible. For instance, this approach would not perform well in non-stationary data situations where new videos and knowledge sources have been produced after the VLM training. Therefore, this work presents the first attempt to apply ideas from the retrieval-augmented text generation (RAG) literature \cite{reml,rag} to KI-VideoQA.

In the following, we introduce a generic framework and discuss various implementations of it in each experiment. The framework, as depicted in Figure~\ref{fig:method_overview}, consists of three main components: query formulation, multi-modal knowledge retrieval, and a VLM for information synthesis and answer generation. The framework processes input in the form of a text question, video frames, and additional information about the video if available such as associated subtitles, utilizing external knowledge bases to generate accurate answers in both multiple-choice and open-ended settings. In the following, we describe various implementations of each of these components studied in this paper.


\subsection{Query Formulation}
The effectiveness of retrieval heavily depends on query formulation. We explore several query formulation strategies:
\begin{enumerate}[leftmargin=*]
\item \textbf{Question-only}: Using the question text $q$ as the query:
% \begin{equation}
% q_{raw} = q
% \end{equation}

\item \textbf{Question + Options}: For multiple choice questions, we can concatenate the question with all provided options as the query. This query formulation cannot be applied to open-ended answer generation, as options are not available to the model in this setting.
% concatenating the question with all possible options:
% \begin{equation}
%     q_{opt} = q \oplus \{a^1, a^2, ..., a^K\}
% \end{equation}
% where $\oplus$ denotes concatenation and $\{a^1, ..., a^K\}$ are the candidate answers.

\item \textbf{Question + Subtitle}: Enriching the query with the video subtitle, by concatenating the question with the video subtitle.
% \hamed{In your equation you mentioned subtitle at time $t$. What is $t$ here?} 
% relevant subtitle context:
% \begin{equation}
%     q_{sub} = q \oplus s_t
% \end{equation}
% where $s_t$ represents the subtitle at timestamp $t$.

\item \textbf{LLM-Based Query Rewriting}: Rewriting the question using a VLM that takes the video, subtitle and question, and is prompted to rewrite the question for higher quality retrieval. \footnote{Detailed prompts available in source code}
% Applying transformations to the base query using the inputs with a VLM:
% \begin{equation}
%     q_{trans} = T(q)
% \end{equation}
% where $T(\cdot)$ is the transformation that reformulates the query for better retrieval.

\end{enumerate}

\subsection{Multi-Modal Knowledge Retrieval}

Given a formulated query $q^*$ obtained from the last component, we perform retrieval over a diverse set of multi-modal knowledge sources as follows:
\begin{enumerate}[leftmargin=*]
\item \textbf{Subtitle Retrieval}: This retrieval model takes a collection of videos as a knowledge source and uses their subtitles to construct a text document for every video in the collection. Therefore, this knowledge source basically contains historical dialogues in the video collection and enables the system to leverage conversational context and spoken information that may not be visually apparent.

% \begin{equation}
% \mathcal{R}s = R_s(q*, \mathcal{K}_s)
% \end{equation}

% \begin{equation}
%     \mathcal{R}_c = R_c(q_*, \mathcal{K}_c)
% \end{equation}

\item \textbf{Video Caption Retrieval}: Developing effective video retrieval models is challenging and one approach is to turn the video into a textual description through video captioning. Therefore, each video will be represented by a single text document, automatically generated using a large vision-language model. We use Qwen2-VL-2B \cite{wang2024qwen2vl} in zero-shot setting for this purpose. These captions serve as an intermediate representation bridging visual and textual modalities. To provide an insight into what a video caption may contain, we provide one example in Figure~\ref{fig:method_overview} (bottom right corner).

% This corpus provides dense semantic descriptions of video content.

\item \textbf{Video Retrieval}: Knowledge retrieval can be done directly to the collection of videos. We do video retrieval using the generated captions and finding the corresponding video of the retrieved captions. This component helps in understanding visual context, actions, and temporal relationships that may be crucial for answering some question.


% \begin{equation}
%     \mathcal{R}_c = R_c(q_*, \mathcal{K}_c)
% \end{equation}

\end{enumerate}

Each retriever component operates independently and can be implemented using various retrieval architectures, such as sparse or dense retrieval models. Each returns a ranked list of information items with the highest relevance score, given the formulated query $q^*$. We implement and evaluate three distinct retrieval models:
\begin{itemize}[leftmargin=*]
    \item \textbf{BM25 \cite{robertson1994some}}: A sparse retrieval model rooted in classical probabilistic models. We use Elasticsearch's implementation of BM25. We use the default BM25 parameters (i.e., $b=0.75$ and $k_1=1.2$). This approach can only be employed for subtitle and video caption retrieval.

     \item \textbf{NV-Embed-v2 \cite{lee2024nv}}: State-of-the-art dense retrieval model which was ranked No. 1 on the Massive Text Embedding Benchmark \cite{muennighoff-etal-2023-mteb} as of January 5, 2025 with an impressive score of 72.31 across 56 text embedding tasks. The model is built on the Mistral-7B-v0.1 architecture and produces embeddings with a dimension of 4096. 
     % They used several new ideas: a two-staged instruction tuning method to enhance the accuracy of both retrieval and non-retrieval tasks, used Latent-attention mechanism to allows the LLM to attend to latent vectors, resulting in improved pooled embedding output. 
     % \hamed{it would be nice if you include more information about this. how many parameters? what is especial about it? for instance, a model that is trained using knowledge distillation on the x and y dataset.}
    
    \item \textbf{Stella \cite{zhang2024jasper}}: We use another state-of-the-art dense retrieval model, stella\_en\_400M\_v5.\footnote{Available at \url{https://huggingface.co/dunzhang/stella_en_400M_v5}.} This model only has 400M parameters and encodes queries and documents into 1024-dimensional dense vectors. Compared to other similar performing embedding models, both the number of parameters and encoded vector dimension are very small; for example, NV-Embed-v2 \cite{lee2024nv}, bge-en-icl \cite{li2024making}, and e5-mistral-7b-instruct \cite{wang2022text, wang2023improving, wang2024multilingual} have 7B parameters, and their vector dimensions are 4096. The deployment and application of these larger models in industry were hampered by their vector dimensions and numerous parameters, making it too slow for practical use. Stella uses an innovative distillation technique to achieve high performance while maintaining a smaller footprint. In our experiments we saw 11x speed improvement for a fixed batch size during encoding, compared to NV-Embed-v2.

% This makes it a compact yet powerful dense retrieval model that addresses the industry deployment challenges of large vector dimensions.
    
    % The model encodes queries and documents into 1024-dimensional dense vectors. Note that Stella has shown improved performance compared to commonly used dense retrieval models, such as DPR, ANCE, and TAS-B \cite{}. \hamed{I added the last sentence. Please find a good citation for it, so people don't question your choice.}
    
   
\end{itemize}

% For all retrieval models, we maintain a consistent evaluation setup. 


% \hamed{maybe better to highlight retrieval models and also the fact that we consider heterogenuous information at some point.}

\subsection{Augmented Answer Generation}

The final component is a VLM that generates the answer. It takes the original inputs (video, question, subtitle and options for the MCQ setting) and retrieved knowledge from each of the retrievers. The model utilizes heterogenuous information both from the input and retrieved knowledge. We utilize Qwen2-VL with two billion parameters \cite{wang2024qwen2vl} as our primary reader model, which serves as the foundation for answering both multiple choice and open-ended questions. Qwen2-VL processes the input video frames, question, and retrieved knowledge simultaneously to generate answers. For video frame processing, we sample frames uniformly at 1 FPS and encode them using Qwen2VL's native visual encoder. Each frame height and width is resized into 224 pixels before feeding to the VLM.
% \hamed{more information. what LLMs? any specific prompt? zero-shot or fine-tuning.}

\subsection{Fine-Tuning}
This section describes the method used for fine-tuning the Qwen2VL 2B model. The model is initialized using the pre-trained weights available on HuggingFace.\footnote{Available at \url{https://huggingface.co/Qwen/Qwen2-VL-2B-Instruct}.} We use the Adam optimizer with weight decay (AdamW) \cite{loshchilov2017decoupled} for fine-tuning with a learning rate of 1e-5. We use the cross-entropy loss function with a batch size of 1 due to the high memory requirement of processing videos by VLMs. To minimize the impact of gradient fluctuation, we update model parameters with 50 gradient accumulation step, resulting in an effective batch size of 50. The model uses Flash Attention 2 \cite{dao2023flashattention2fasterattentionbetter} for better acceleration and memory efficiency, especially when processing multiple videos. We use the training portion of each dataset for fine-tuning the model.


% \hamed{here describe your model training. loss funciton. optimization. etc.}

% Model: Qwen2VL
% Model Size: 2B parameters
% Pre-trained Model: The model is initialized with the pre-trained weights from Qwen/Qwen2-VL-2B-Instruct
% Optimizer: AdamW (Adam with Weight Decay)
% Learning Rate: 1e-5
% Loss Function: Cross-Entropy Loss
% Batch Size: 1 (due to the high memory requirements of processing images and videos)
% Gradient Accumulation Steps: 50 (Gradient accumulation is used to simulate a larger batch size by accumulating gradients over multiple forward passes before performing a backward pass and updating the model parameters. This helps in managing memory constraints while still allowing for effective training.)
% Flash Attention: The model uses flash\_attention\_2 for better acceleration and memory efficiency, especially when processing multiple images or videos.
% Training data: train partition of the datasets

% \section{Experimental Setup}
% \hamed{this can be a subsection if it's too brief.}

\section{Experiments}
\subsection{Implementation Details}
\label{sec:implemenet}
\paragraph{Datasets} We use the Multiview Rendering Dataset \cite{qiu2023richdreamer,zuo2024sparse3d} based on Objaverse \cite{objaverse} for training. The dataset includes 260K objects, with 38 views rendered for each object, with a resolution of $512\times512$. To obtain the surface point clouds, we transform the 3D models according to the rendering settings, filter out those that are not aligned with the rendered images, and use Poisson sampling method\cite{poisson} to sample the surface.  We randomly split the final processed data into training and testing sets, with the training dataset consisting of 200K objects.
% We take our in-domain evaluation using the test set from Objaverse, including 2000 objects. To evaluate our model's cross-domain 能力, We 在Google Scanned Objects(GSO) dataset进行评估,which 包含1030个真实的扫描3D model, and we take 32 views renderd for each model in 球面. 
We conduct our in-domain evaluation using the test set from Objaverse, which includes 2,000 objects. To assess our model's cross-domain capabilities, we evaluate it on the Google Scanned Objects (GSO) \cite{downs2022google}dataset, which contains 1,030 real-world scanned 3D models, with 32 views rendered for each model on a spherical surface.

% 我们使用单图作为输入,以所有available的views作为noval views 来评测我们和所有比较方法的单图生成质量. And take the peak-signal-to-noise ratio (PSNR),
% perceptual quality measure LPIPS, structural similarity index (SSIM) 作为evaluation metrics, which is same to previous work\cite{zou2024triplane, chen2025lara}


% 在我们的实现中,anchor latents的fix length是2048,维度是8. the model dim of Anchor-GS VAE是512, with 两个transformer block in encoder and eight transformer block in decoder. For training the Anchor-GS VAE, we set the weights of the losses with $\lambda_s=1, \lambda_l=1, \lambda_c=1, \lambda_e=1 $ and $\lambda_{KL}=0.001$.
%
% For the Seed-Anchor Mapping Module, we use 24 transformer bolck to implement with a model dim 512, 其中 4 bolcks for downsample and 4 blocks for upsample. For the Seed Points Generation, we use 24 transfomrmer blocks to implement with model dim 512. And thanks to the seed points 的sparse 特性, we can directly learn the distribution of seed points 而不需要去train 一个 VAE. 
%
% We train the Anchor-GS VAE use only a subset of our collected datasets, with a batchsize of 128 on 8 40G A100 with 24K steps. For trainging the Seed-Anchor Mapping Module, we use our full collected datasets, with a batchsize 0f 1280 on 64 32G V100 with 20K steps. For training the Seed Points Generation Module, We training on 48 32G V100 with 54K steps.
% In our implementation, the anchor latents have a fixed length of 2048 and a dimension of 8, and the model dimension in our transformer blocks is 512, each transformer block has two attention layers and a feed-forward layer, similar to \cite{zou2024triplane}. The Anchor-GS VAE consists of two transformer blocks in the encoder and eight transformer blocks in the decoder. For training the Anchor-GS VAE, we random select 8 views,  one view as input and all 8 views as the ground truth images for supervision, and we set the loss weights as follows: \(\lambda_s = 1\), \(\lambda_l = 1\), \(\lambda_c = 1\), \(\lambda_e = 1\), and \(\lambda_{KL} = 0.001\).  


\paragraph{Network}
In our implementation, the anchor latents have a fixed length of 2048 and a dimension of 8. The model dimension in our transformer blocks is 512, with each transformer block comprising two attention layers and a feed-forward layer, following the design in \cite{zou2024triplane}. The Anchor-GS VAE consists of two transformer blocks in the encoder and eight transformer blocks in the decoder. 
%
The Seed-Anchor Mapping Module is implemented using 24 transformer blocks, with 4 blocks for downsampling and 4 blocks for upsampling. Similarly, the Seed Points Generation Module is implemented with 24 transformer blocks. Leveraging the sparsity of seed points, we directly learn their distribution without requiring a VAE. The image conditioning in our model is extracted using DINOv2\cite{oquab2023dinov2}.


\paragraph{Training Details}
For training the Anchor-GS VAE, we randomly select 8 views per object, using one view as the input and all 8 views as ground truth images for supervision. The loss weights are set as \(\lambda_s = 1\), \(\lambda_l = 1\), \(\lambda_c = 1\), \(\lambda_e = 1\), and \(\lambda_{KL} = 0.001\). We train the Anchor-GS VAE on a subset of our collected dataset containing approximately 40K objects, using a batch size of 128 on 8 A100 GPUs (40GB) for 24K steps. The Seed-Anchor Mapping Module is trained on the full dataset with a batch size of 1280 on 64 V100 GPUs(32GB) for 20K steps. The Seed Points Generation Module is trained on 48 V100 GPUs (32GB) for 54K steps.  We use the AdamW optimizer with an initial learning rate of \(4 \times 10^{-4}\), which is gradually reduced to zero using cosine annealing during training. The sampling steps for both the Seed-Anchor Mapping Module and the Seed Points Generation Module are set to 50 during inference.
% For training the Anchor-GS VAE, we randomly select 8 views per object, using one view as the input and all 8 views as ground truth images for supervision. The loss weights are set as follows: \(\lambda_s = 1\), \(\lambda_l = 1\), \(\lambda_c = 1\), \(\lambda_e = 1\), and \(\lambda_{KL} = 0.001\).
% We train the Anchor-GS VAE using a subset of our collected dataset around 40K objects with a batch size of 128 on 8 A100 GPUs (40GB) for 24K steps. The Seed-Anchor Mapping Module is trained on the full dataset with a batch size of 1280 on 64 V100 GPUs (32GB) for 20K steps. For the Seed Points Generation Module, we train on 48 V100 GPUs (32GB) for 54K steps. We use the AdamW optimizer with a learning rate of 4e-4, and the learning rate is cosine anneled to zero during training.


\paragraph{Baseline}
% We compared our methods with 之前的SOTA的3D生成模型 in one image input setting. LGM and LaRa use one image as input, then use multi-view diffusion models to get four views of the object, then get corresponding 3DGS from the multiview images in a feed-forard mamner. TriplaneGS first get dense point clouds from the single input image, then use a triplane to 聚合特征 then get the corresponding attributes of 3DGS, achieving SOTA performances.
We compared our method with previous SOTA 3DGS generation models in the single-image input setting. LGM and LaRA rely on 2D multi-view diffusion priors to obtain multi-view images, which are then used to generate the output 3DGS in a feed-forward manner, as described in ~\ref{sec:related-2d-diffusion}. TriplaneGS~\cite{zou2024triplane} does not require a 2D diffusion prior, directly generating 3DGS from a single input image, as outlined in ~\ref{sec:related_3d}. Both of them achieving SOTA performance. For each compared method, we use the official models and provided weights and ensure careful alignment of the camera parameters.

% Both  LGM~\cite{tang2025lgm} and LaRa~\cite{chen2025lara} take one image as input and then use multi-view diffusion models\cite{shi2023mvdream} to generate four views of the object. These multi-view images are subsequently converted into corresponding 3D Gaussian Splatting (3DGS) representations in a feed-forward manner. TriplaneGS~\cite{zou2024triplane}, on the other hand, first generates dense point clouds from the single input image and then aggregates features using a triplane representation to infer the corresponding attributes of 3DGS, also achieving SOTA performance. For each compared method, we use the official models and provided weights and ensure careful alignment of the camera parameters.\todo{remove duplicate description, mention LaRA is designed for four views input}
% compared methods






\subsection{Results of VAE Reconstruction}
In Fig. \ref{fig:vae}, we present the results of our Anchor-GS VAE. Given point clouds and a single image, our Anchor-GS VAE achieves high-quality reconstructions with detailed geometry and textures.



\subsection{Results of 3D Generation }
\label{sec:comparison}
% 1. 首先讲在哪些数据上进行评估。然后逐个分析结果的值,最后说我们的效果达到了SOTA的效果
% 2. 展示可视化的结果,再逐个分析。表明我们的方法相比于没有用diffusion的方法能更好的学习三维物体的分布。
% Table 1展示了我们的方法和previous SOTA methods在Objaverse和GSO上的评测结果。As described in \ref{sec:implemenet},评测在一个dense viewpoint settings下进行,the results are average use all available objects and viewpoints in the testing datasets. The 多视角不一致 in the multiview diffusion model used by LGM 和 LARA 会导致生成几何不一致的3DGS,特别是导致在新视角下的伪影,So 他们会产生相对较低的值在我们的dense viewpoint评估settings下。相比于他们,我们的方法不借助于2D diffusion先验,可以直接从单张图像中得到理想的3D表示。TriplaneGS使用通过tansformer block一次forward facing得到point clouds,但是这种方式往往不能准确学习到3D点云的分布,always failed in 输入图像中没有的区域。与之相比,我们的方法使用diffusion-based methods, 首先学习一个sparse 的seed points, which is easy to learn and can学习到3D的distribution,then mapping from seed points to anchor latents.SO get more 鲁邦的结果。
\paragraph{Metrics} 
Following previous works \cite{zou2024triplane, chen2025lara}, we use peak signal-to-noise ratio (PSNR), perceptual quality measure LPIPS, and structural similarity index (SSIM) as evaluation metrics to assess different aspects of image similarity between the predicted and ground truth. Additionally, we report the time required to infer a single 3DGS. We use a single image as input and evaluate the 3D generation quality using all available views as testing views to compare our method with others, all renderings are performed at a resolution of 512.

Tab. \ref{tab:quantitative comparison} presents the quantitative evaluation results of our method compared to previous SOTA methods on the Objaverse and GSO datasets, along with qualitative results shown in Fig. \ref{fig:image-3d}. The multi-view diffusion model used in LGM tend to produce more diverse but uncontrollable results, and lacks precise camera pose control. As a result, it fails in our dense viewpoints evaluation, achieving PSNR scores of 12.76 and 13.81 on the Objaverse and GSO test sets, respectively.

As shown in Tab. \ref{tab:quantitative comparison}, LGM and LaRa, influenced by the multi-view inconsistency of 2D diffusion models, achieve relatively lower scores in our dense viewpoint evaluation. In contrast, our method achieves the best results across both datasets, with only a slight overhead in inference time.

Fig. \ref{fig:image-3d} presents the first six rows from the Objaverse dataset and the last three rows from the GSO dataset. All methods are compared using the same camera viewpoints. For the Objaverse dataset, the rendering viewpoints are the left and rear views relative to the input viewpoint, while for the GSO dataset, the views are selected to showcase the object as completely as possible. Compared to methods using 2D diffusion priors, such as LGM and LaRa, our method demonstrates better multi-view geometric consistency, while the former tends to generate artifacts or unrealistic results in our displayed views. Compared to TGS, our method learns the 3D object distribution more effectively, resulting in more geometrically consistent multi-view results, such as the sharp feature in the left view in the first knife case.
% Compared to these methods, 我们的方法能取得更multiview geometry consistent的结果. 例如所有的方法都在第一个case中不能正确的表示输入图像中的sharp feature,导致在所有视图中都是近似的结果.

% As described in Sec. \ref{sec:implemenet}, the evaluations are conducted under a dense viewpoint setting, with the results averaged over all available objects and viewpoints in the testing datasets. The multiview diffusion models used by LGM \todo{add lgm fail reason: Image dream can't control viewpoint}and LaRa exhibit inconsistencies across viewpoints, resulting in geometrically inconsistent 3D Gaussian Splatting (3DGS) representations. This inconsistency particularly manifests as artifacts when rendering from novel viewpoints, leading to relatively lower performance under our dense viewpoint evaluation setting. In contrast, our method does not rely on 2D diffusion priors and directly generates a high-quality 3D representation from a single input image. 

% TriplaneGS employs a Transformer-based approach to predict dense point clouds in a single forward pass. However, this approach can face challenges in accurately capturing the 3D distribution of points, particularly in regions not visible in the input image, which may lead to less optimal performance in some cases. In comparison, our method adopts a diffusion-based strategy, first learning a sparse set of seed points. This approach simplifies the learning process, allowing the model to better capture the underlying 3D distribution. The seed points are then mapped to anchor latents, resulting in more robust and consistent outcomes.

\subsection{Editing Results Based on Drag}
% The drag results are presented in Fig. \ref{fig:edit}.
As shown in Fig. \ref{fig:edit}, our method enables Seed-Points-Driven Deformation. Starting with generated seed points from the input image, the sparse nature of the seed points allows for easy editing using 3D tools (e.g., Blender\cite{blender}) with a few drag operations. The edited 3DGS can then be obtained within 2 seconds.
% 如Fig. \ref{fig:edit}.所示,我们的方法可以进行Seed-Points-Driven Deformation. 对于一个generated seed points from input image,
% 由于seed points稀疏的特性,我们可以很方便借助3D编辑工具(blender)使用有限的几次Drag操作对seed points进行编辑,and以2s时间得到编辑后的3DGS.
\subsection{Ablation Study}
% \paragraph{coarse2fine in vae}
% \paragraph{two-stage generation}
% \todo{not finish here}
\paragraph{Seed Points Generation}
% We use the Recitified flow model to learn the generation of seed points with the conditon of single input image. Due to the sparse of seed points, the flow model is easy to learn and 可以很好的学到seed points' distribution. We also 实现this module using transformer block 使用一次feed forward的方法从learnable embeddings 中得到point cloud,like \cite{zou2024triplane}. As shown in Fig. \ref{fig:ablation-seed-gen}, the Feed-forward method failed to learn the distribution of the seed points, 在图片上不可见的区域无法生成理想的结果。
We employ a Rectified Flow model to generate seed points conditioned on a single input image. Owing to the sparsity of the seed points, the flow model is easier to train and effectively learns the distribution of the seed points. However, we also explored an alternative implementation using a transformer-based feed-forward approach, where point clouds are generated from learnable embeddings in a single pass, as in \cite{zou2024triplane}. As demonstrated in Fig. \ref{fig:ablation-seed-gen}, the feed-forward approach struggles to capture the true distribution of seed points and fails to produce satisfactory results in regions not visible in the input image.


\paragraph{Dimension Alignment}
% 为了让Seed-Anchor Mapping Module 的起点和target具有相同的维度,我们将Seed points 通过VAE的encoder based on Eq. \ref{eq:encode_seed_latents}. 这可以保证Mapping的起点的分布和终点的分布更接近,从而降低了Seed-Anchor Mapping Module的学习难度并且避免了Mapping时对image condition的过度依赖. And the alignment bewteen points and image achieved by projection in encoder is vivtal in the Seed-Points-Driven Deformation, as we can change the position of draged seed points while preserve it's correponding projed feats.
To match the dimension of the starting and target points in the Seed-Anchor Mapping Module, we encode the seed points using the Anchor-GS VAE encoder (Eq. \ref{eq:encode_seed_latents}). This process brings their distributions closer, reducing learning difficulty and reliance on image conditions. 
% Additionally, the projection-based alignment between points and the image in the encoder is critical for Seed-Points-Driven Deformation, enabling position adjustments of dragged seed points while preserving their projected features, as shown in Eq.\ref{eq:edit_seed_encode}. 
To validate this method, we conducted experiments by replacing the encoding approach with positional encoding .  When using positional encoding, the Seed-Anchor Mapping overly relied on the image condition, neglecting the contribution of the seed points and failing to enable seed-driven 3DGS deformation, as shown in Fig. \ref{fig:ablation-seed-enc}. 

% When positional encoding is used, the Seed-Anchor Mapping overly relies on the image condition, neglecting the true geometric state of the seed points and failing to achieve seed-driven 3DGS deformation.
% When using positional encoding, the Seed-Anchor Mapping overly relied on the image condition, 忽略了seed points真实的几何状态 and failing to enable seed-driven 3DGS deformation. 
% To validate this method, we conducted experiments by independently testing two variations: replacing the encoding approach with positional encoding and removing the projection of seed points onto the input image during encoding (Fig. \ref{fig:ablation-seed-enc}).  When using positional encoding, the Seed-Anchor Mapping overly relied on the image condition, neglecting the contribution of the seed points and failing to enable seed-driven 3DGS deformation. Separately, without projection-based alignment, the Mapping Module failed when the seed points and the input image were misaligned under the given viewpoint.
% 为了验证这个方法的有效性,我们测试了将基于encoder of anchor-GS VAE的方法换成positional encoding 和 在encode时不采用seed points与输入图像的投影。结果如图所示。在采用positional encoding时,the Seed-Anchor Mapping 将会过度依赖于image condition,导致忽略了从start points本身出发,从而无法实现基于seed points 的3DGS deformation. And lack of the projection-based alignment, the Mapping Module 会在seed points与input image 在给定视角下不一致的情况下fail.


\paragraph{Token Alignment}
We ensure token alignment in Flow Matching by organizing tokens around seed points, followed by  cluster-based partition and repetition. To evaluate its effectiveness, we conducted two ablation experiments, as shown in Tab. \ref{tab:ablation-tokenalign}. In the \textit{No-cluster+No-repetition} setting, we omitted the clustering step, aligning only the corresponding seed and anchor latents while filling unmatched portions with noise. This also prevented cluster-based downsampling in the Flow Model, resulting in doubled memory consumption. In the \textit{No-cluster} setting, we repeated the seed latents to match the number of anchor latents but left them unordered, leading to disorganized token matching. As shown in Tab. \ref{tab:ablation-tokenalign}, on a 40K subset with the same number of training steps, the absence of token alignment significantly degraded Flow Matching performance, resulting in inaccurate correspondences.
% We ensure token alignment in Flow Matching by organizing tokens around seed points, followed by repetition and cluster-based rearrangement. To validate its effectiveness, we conducted two ablation experiments. In the first, we repeated seed latents to match the number of anchor latents but left them unordered, leading to disordered token matching. In the second, we aligned only the corresponding seed and anchor latents, filling the unmatched portions with noise. Without cluster-based rearrangement, downsampling in the Rectified Flow Model became impossible, doubling memory consumption. Tab. \ref{tab:ablation-tokenalign} shows that on a 40K subset, with the same number of training steps, flow matching performance is significantly degraded without token alignment, failing to produce accurate correspondences.

% We ensure token alignment in Flow Matching by organizing tokens around seed points, followed by repetition and cluster-based rearrangement. To validate its effectiveness, we conducted two ablation experiments, as shown in Tab. \ref{tab:ablation-tokenalign}. In the No-cluster, 我们不再进行分cluster,aligned only the corresponding seed and anchor latents, filling the unmatched portions with noise. And we can't do cluster-based downsample in the Flow Model, 这会导致 memory consumption double. In the No-rearrange, we repeated seed latents to match the number of anchor latents but left them unordered, leading to disordered token matching. Tab. \ref{tab:ablation-tokenalign} shows that on a 40K subset, with the same number of training steps, flow matching performance is significantly degraded without token alignment, failing to produce accurate correspondences.


% we repeated seed latents to match the number of anchor latents but left them unordered, leading to disordered token matching. In the second, we aligned only the corresponding seed and anchor latents, filling the unmatched portions with noise. Without cluster-based rearrangement, downsampling in the Rectified Flow Model became impossible, doubling memory consumption. Tab. \ref{tab:ablation-tokenalign} shows that on a 40K subset, with the same number of training steps, flow matching performance is significantly degraded without token alignment, failing to produce accurate correspondences.
% Token alignment 保证了Flow matching 中start point和end point中token数目和对应位置上的语义是匹配的,使用以Seed points为center的方式来组织token,并进行repeat和rearrange。我们设计了消融实验来验证这个模块的必要性和有效性,第一个对照实验是在对seed latets repeat到和anchor latents相同的个数后,我们不对anchor latents进行重排序, so the token之间的匹配关系是杂乱的。第二个对照实验是我们仅仅将对应的seed latents和anchor latents进行对齐,其余无法对齐的部分使用noise进行填充。并且由于没有进行cluster based rearrange,我们无法在Rectified flow Modle中进行downsaple, which 增加了两倍的计算时的显存消耗。在一个40K subset上 经过同样的training steps后,结果如表所示。Without the token alignment, the Flow matching的效果大打折扣,无法得到理想的对应的结果。
% We ensure token alignment in Flow Matching by organizing the tokens around seed points and performing repeating and rearranging operations. This guarantees that the number of tokens and their semantic correspondence between the start and end points are aligned. To validate the necessity and effectiveness of this module, we designed ablation experiments. The first experiment involved repeating the seed latents to match the number of anchor latents but without reordering the anchor latents, resulting in a disordered matching relationship between tokens. The second experiment aligned only the corresponding seed latents and anchor latents, filling the unmatched portions with noise. Without the cluster-based rearrangement, we were unable to downsample in the Rectified Flow Model, which increased the memory consumption during computation by a factor of two. After training on a 40K subset for the same number of steps, the results, shown in the table, indicate that without token alignment, the performance of flow matching is significantly degraded, failing to achieve the desired correspondence.


% Table
\begin{table}%
\caption{ Quantitative evaluation of our method compared to previous work. $\dagger$ achieves relatively lower PSNR values in the evaluation, so we display the results in Sec. \ref{sec:comparison}.}
\label{tab:quantitative comparison}
% \begin{minipage}{\columnwidth}
\resizebox{0.5\textwidth}{!}{
% \begin{center}
\begin{tabular}{llllllll}
  \toprule
  \multirow{2}{*}{Method}  & \multicolumn{3}{c}{Objaverse\cite{objaverse}}   & \multicolumn{3}{c}{GSO\cite{downs2022google}}& \multirow{2}{*}{Time(s)}\\
% \cline{2-4}   \cline{5-7} \cline{8-10} \cline{11-13}
\cmidrule(r){2-4}  \cmidrule(r){5-7} 
   & PSNR$\uparrow$& SSIM$\uparrow$& LPIPS$\downarrow$ & PSNR$\uparrow$& SSIM$\uparrow$& LPIPS $\downarrow$
   \\ \midrule
  LGM$\dagger$\cite{tang2025lgm}     & -&0.836&0.211&-&0.833&0.21&4.82\\
  LaRa\cite{chen2025lara}  & 16.57&0.860&0.174&15.98&9.869&0.162&9.50\\
  TriplaneGS\cite{zou2024triplane}  &18.80 &0.883&0.143&19.84&0.900&0.120&0.70\\
  Ours &20.92&0.896&0.120&20.52&0.904&0.1122&4.71\\
  \bottomrule
\end{tabular}
% \end{center}
}
\end{table}%


\begin{table}%
\caption{Ablation about token alignment}
\label{tab:ablation-tokenalign}
\begin{minipage}{\columnwidth}
\begin{center}
\begin{tabular}{llll}
  \toprule
   & PSNR$\uparrow$& SSIM$\uparrow$& LPIPS$\downarrow$ 
   \\ \midrule
  No-cluster+No-repetition  & 18.84&0.877&0.141\\
  No-cluster     & 19.20 &0.876&0.142\\
  ours-full  &19.94 &0.881&0.134\\
  \bottomrule
\end{tabular}
\end{center}
\bigskip\centering


\end{minipage}
\end{table}%

%figure
\begin{figure}
  \includegraphics[width=\linewidth]{figs/ablation_seed.pdf}
  \caption{Ablation study about different seed points geneartion methods: (a) using our method, (b) using Transformer.}
  \label{fig:ablation-seed-gen}
\end{figure}

\begin{figure}
  \includegraphics[width=\linewidth]{figs/ablation_enc.pdf}
  \caption{Without Dimension Alignment, seed-points-driven deformation fails}
  \label{fig:ablation-seed-enc}
\end{figure}




% \section{Key Findings and Conclusions}

\section{Conclusion and Suggestions}

Our work, including the creation of \texttt{ScholarLens} and the proposal of \texttt{LLMetrica}, provides methods for assessing LLM penetration in scholarly writing and peer review. By incorporating diverse data types and a range of evaluation techniques, we consistently observe the growing influence of LLMs across various scholarly processes, raising concerns about the credibility of academic research. As LLMs become more integrated into scholarly workflows, it is crucial to establish strategies that ensure their responsible and ethical use, addressing both content creation and the peer review process. 

Despite existing guidelines restricting LLM-generated content in scholarly writing and peer review,\footnote{\href{https://aclrollingreview.org/acguidelines\#-task-3-checking-review-quality-and-chasing-missing-reviewers}{Area Chair} \&  \href{https://aclrollingreview.org/reviewerguidelines\#q-can-i-use-generative-ai}{Reviewer} \& \href{https://www.aclweb.org/adminwiki/index.php/ACL_Policy_on_Publication_Ethics\#Guidelines_for_Generative_Assistance_in_Authorship}{Author} guidelines.} challenges still remain. 
To address these, we propose the following based on our work and findings: 
(i) \textbf{Increase transparency in LLM usage within scholarly processes} by incorporating LLM assistance into review checklists, encouraging explicit acknowledgment of LLM support in paper acknowledgments, and 
reporting LLM usage patterns across diverse demographic groups;
% reporting LLM penetration based on social demographic features;
(ii) \textbf{Adopt policies to prevent irresponsible LLM reviewers} by establishing feedback channels for authors on LLM-generated reviews and developing fine-grained LLM detection models~\cite{abassy-etal-2024-llm, cheng2024beyond, artemova2025beemobenchmarkexperteditedmachinegenerated} to distinguish acceptable LLM roles (e.g., language improvement vs. content creation);
(iii) \textbf{Promote data-driven research in scholarly processes} by supporting the collection of review data for further robust analysis~\cite{dycke-etal-2022-yes}.\footnote{\url{https://arr-data.aclweb.org/}}

% make LLM usage transparent in scholarly processes: such as incorporating LLM usage into review checklists, encouraging explicit acknowledgment of LLM assistance in paper acknowledgments, and reporting LLM penetration based on social demographic features; (ii) Adopt policies to prevent irresponsible LLM reviewers: such as providing authors feedback on LLM-assisted reviews, and developing fine-grained LLM detection models~\cite{cheng2024beyond} to distinguish acceptable LLM roles (e.g., language improvement vs. content creation); (iii) Encourage data-driven research in scholarly processes: such as supporting review data collection for further research.

 



\bibliographystyle{ACM-Reference-Format}
\bibliography{XX-references}

\end{document}
