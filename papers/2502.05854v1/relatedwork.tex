\section{Related Work}
\label{section2}
In this section, we will summarize the related work concerning two aspects: SPM with gap constraints and negative SPM. 

\subsection{SPM with gap constraints}
Sequential pattern mining (SPM)  is an important knowledge discovery method \cite{pmdb2022,wu2022tmis}. A variety of SPM methods have been proposed to meet different requirements, such as rare pattern mining \cite {rarepattern,rarepattern2}, utility pattern mining \cite{hanpminer, unil-utility1, 6_Gan2021}, episode pattern mining \cite{13episodepattern, 14episoderule}, closed or maximal SPM  \cite{29_Wu2020, 30_Li2021}, process pattern mining \cite {processmining1, processmining2}, distributed pattern mining \cite {distributedmining}, spatial co-location pattern mining \cite{Wang2022colocation, Wang2023}, negative SPM \cite {dong2023, dong2018}, and SPM with gap constraints \cite{tkdd2012}. A disadvantage of traditional SPM methods is that these methods only pay attention to whether a pattern occurs in a sequence but disregard that a pattern may occur multiple times in it. Therefore, some important patterns may be missed using traditional SPM.  To overcome this issue, SPM with gap constraints was proposed \cite{33_Zhang2007}. 

SPM with gap constraints has many different names, such as repetitive SPM \cite {yongxintong} or tandem repeat discovery in bioinformatics \cite {tandemrepeat}. More importantly, SPM with gap constraints is more difficult than classical SPM, since the calculation of the pattern support in sequences becomes a pattern matching issue \cite{scis2017}. Moreover, SPM with gap constraints has four different methods to calculate the number of occurrences: no-condition  \cite{12_Min2020}, one-off condition (or one-off SPM) \cite{45_Wu2021,oneoff2}, nonoverlapping condition (or nonoverlapping SPM) \cite{14_Wu2018}, and disjoint condition \cite{disjoint2021}. The case of no-condition means that each item can be reused. The one-off condition means that each item can be used at most once \cite{28_Wu2021}. The nonoverlapping condition means that each item cannot be reused by the same $p_j$, but can be reused by different $p_j$ \cite{gengtkde2024}. The disjoint condition means that the maximum position of an occurrence should be less than the minimum position of the next occurrence. 

Among the above four methods, the no-condition was proposed first \cite {33_Zhang2007}. More importantly, the no-condition is arguably the most popular method \cite{nocondition2021}, since the support calculation of the no-condition can be regarded as calculating the number of all occurrences and it is easy to be understood, while the support calculations of other three methods can be regarded as only counting some specific occurrences.



Although the above SPM with gap constraints methods can find interesting patterns in various situations, they ignore the missing items in sequences. Therefore, it is easy to overlook potentially important information.

\subsection{Negative SPM}
{To discover missing items, Ouyang and Huang \cite {ouyang2007} first proposed three forms of negative SPM in transaction databases: $<$$\overline {X}$, Y$>$, $<$X, $\overline {Y}$$>$, and $<$$\overline {X}$, $\overline {Y}$$>$. One of the drawbacks of this study is that it limits the position of negative items in negative sequence pattern mining. To address this problem, Hsueh et al. \cite{39_Hsueh2008} designed a PNSP algorithm to mine negative sequential patterns in which the location of negative items is more flexible. Based on PNSP, Zheng et al. \cite{40_Zheng2009} introduced a GSP-like method, called neg-GSP, to mine negative sequential patterns. Both PNSP and neg-GSP do not satisfy the Apriori property, and they need to mine positive sequential patterns before mining negative sequential patterns. Therefore, the running performances of PNSP and neg-GSP need to be improved.} 

Note that in NSP mining, there is currently no unified definition of negative containment due to the concealment of non-occurrence items \cite{dong2019}. Therefore, various negative SPM were proposed \cite {hannegative, 18_Wang2021}. Among them,  e-NSP \cite{41_Cao2016},  e-RNSP \cite{43_Dong2020}, and ONP-Miner \cite{onpminer} are three state-of-the-art methods. e-NSP \cite{41_Cao2016} and e-RNSP  \cite{43_Dong2020} are two very similar methods, since both e-NSP and e-RNSP discover frequent positive patterns at first. Then, they generate negative candidate patterns based on frequent positive patterns and mine frequent negative patterns. The difference between e-NSP and e-RNSP is that the former does not consider the repetition of patterns in the sequence, while the latter does.  For example, the support of pattern ``abc'' in sequence  $ \mathbf s $ =acabcabcac is one according to e-NSP, since e-NSP does not consider the repetition of patterns in the sequence, while the support of pattern ``abc''  in sequence  $ \mathbf s $ =acabcabcac is two according to e-RNSP, since e-RNSP considers the repetition. 

However, ONP-Miner is far different from e-NSP and e-RNSP, since ONP-Miner \cite {onpminer} has gap constraints, while e-NSP \cite{41_Cao2016} and e-RNSP  \cite{43_Dong2020} do not have gap constraints. Due to the consideration of gap constraints, ONP-Miner can discover more valuable negative patterns, and the experimental results showed that ONP-Miner can mine more negative patterns than e-NSP and e-RNSP.

ONP-Miner is a negative SPM with gap constraints under the one-off condition and adopts one-off pattern matching to calculate the support of a pattern. Unfortunately, it was proven that one-off pattern matching with gap constraints is an NP-Hard problem \cite {28_Wu2021, oneoff2}, which means that the support cannot be accurately calculated and can only be approximated calculation using a heuristic strategy. Thus, ONP-Miner is an approximate mining algorithm that may lose some feasible patterns. 

%To overcome the shortages of ONP-Miner, this paper investigates repetitive negative SPM with gap constraints under the no condition that can mine all feasible positive and negative patterns. 

{To tackle the above issues, compared with the previous research, the novelties of this paper are three aspects. }

\begin {enumerate}
\item { To overcome the shortages of ONP-Miner \cite {onpminer}, this paper investigates repetitive negative SPM with gap constraints under no condition which can mine all feasible positive and negative patterns.}

\item {Classical negative SPM methods, such as e-NSP \cite{41_Cao2016} and e-RNSP \cite{43_Dong2020}, discover frequent positive patterns and then generate negative candidate patterns, which generate redundant candidate negative patterns. To effectively reduce the number of candidate patterns, we propose a pattern join strategy with negative patterns to generate candidate patterns based on Theorems 1 and 2, which can simultaneously generate both positive and negative candidate patterns.}

\item { Classical SPM with gap constraints method \cite{32_Wu2014} adopts Incomplete-Nettree structure to calculate the supports of candidate patterns with the same prefix pattern, which requires scanning the whole sequence for each prefix pattern. To improve the efficiency, we adopt the key-value arrays of prefix and suffix subpatterns to calculate the supports of both positive and negative candidate patterns with gap constraints that can avoid redundant rescanning of the original sequence dataset, thus improving the calculation efficiency of the support of candidate patterns.}

\end {enumerate}