%%%%%%%%%%%%%%%%%%%%%%%%%%%%%%%%%%%%%%%%%%%%%%%%%%%%%%%%%%%%%%%%%%%%%%%%%%%
\section{Proofs of \S\ref{sec:compl} (\nameref{sec:compl})}
\label{sec:proof:compl}
%%%%%%%%%%%%%%%%%%%%%%%%%%%%%%%%%%%%%%%%%%%%%%%%%%%%%%%%%%%%%%%%%%%%%%%%%%%


%%%%%%%%%%%%%%%%%%%%%%%%%%%%%%%%%%%%%%%%%%%%%%%%%%%%%%%%%%%%%%%%%%%%%%%%%%%
\subsection{\nameref{sec:compl:fin}}
\label{sec:proof:compl:fin}
%%%%%%%%%%%%%%%%%%%%%%%%%%%%%%%%%%%%%%%%%%%%%%%%%%%%%%%%%%%%%%%%%%%%%%%%%%%

We begin with Proposition~\ref{prop:compl:fin:ded}.
We first recall the content of \eqref{eq:compl:cc}:
\begin{equation}
\label{eq:proof:compl:cc}
\begin{array}{c}

\dfrac{}{\C(\True)}

\quad

\dfrac
  {\text{$\BT \in \Base$ and $a \in \BT$}}
  {\C(\form a)}

\quad

\dfrac{\C(\varphi)}
  {\C(\form\fold \varphi)}

\quad

\dfrac{\C(\varphi) 
  \quad
  \C(\psi)}
  {\C(\pair{\varphi,\psi})}

\quad

\dfrac{\C(\psi)
  \quad
  \psi \thesis \varphi}
  {\C(\varphi)}
\\\\

\ax{C}
\dfrac{\C(\psi)}
  {(\psi \realto \False) \,\thesis\, \False}

\quad

\dfrac{\begin{array}{l}
  \text{$I$ finite and $\forall i \in I$,}~
  \C(\psi_i) 
  ~\text{and}~
  \C(\varphi_i) ;
  \\
  \text{$\forall J \sle I$,}~
  \bigwedge_{j \in J} \psi_j \thesis \False
  ~~\text{or}~~
  \C\left( \bigwedge_{j \in J} \varphi_j \right)
  \end{array}}
  {\C\left( \bigwedge_{i \in I}(\psi_i \realto \varphi_i) \right)}
%~(\text{$I$ finite})
\end{array}
\end{equation}

We now recall the statement of Proposition~\ref{prop:compl:fin:ded}.

%%%%%%%%%%%%%%%%%%%%%%%%%%%%%%%%%%%%%%%%%%%%%%%%%%%%%%%%%%%%%%%%%%%%%%%%%%%
\begin{proposition}[Proposition \ref{prop:compl:fin:ded}]
\label{prop:proof:compl:fin:ded}
%%%%%%%%%%%%%%%%%%%%%%%%%%%%%%%%%%%%%%%%%%%%%%%%%%%%%%%%%%%%%%%%%%%%%%%%%%%
In the extension of Figure~\ref{fig:log:ded} (\S\ref{sec:log})
with \eqref{eq:proof:compl:cc}:
\begin{enumerate}[(1)]
\item
for all $\varphi,\psi \in \Lang_\land(\PT)$,
we have
$\psi \thesis_\PT \varphi$
if, and only if,
$\I\psi \sle \I\varphi$;

\item
for all $\varphi \in \Lang_\land$,
we have
$\C(\varphi)$
if, and only if,
$\I\varphi \neq \emptyset$.
\end{enumerate}
%%%%%%%%%%%%%%%%%%%%%%%%%%%%%%%%%%%%%%%%%%%%%%%%%%%%%%%%%%%%%%%%%%%%%%%%%%%
\end{proposition}
%%%%%%%%%%%%%%%%%%%%%%%%%%%%%%%%%%%%%%%%%%%%%%%%%%%%%%%%%%%%%%%%%%%%%%%%%%%

The proof of Proposition~\ref{prop:proof:compl:fin:ded}
is split into Lemmas \ref{lem:proof:compl:fin:ded:cor},
\ref{lem:proof:compl:fin:ded:c-false}
and
\ref{lem:proof:compl:fin:ded:compl}.
Namely:
\begin{itemize}
\item
Lemma~\ref{lem:proof:compl:fin:ded:cor} is the soundness of
the system made of Figure~\ref{fig:log:ded} (\S\ref{sec:log})
and \eqref{eq:proof:compl:cc} for formulae in $\Lang_\land$.

\item
Lemma~\ref{lem:proof:compl:fin:ded:c-false} is a form of dichotomy:
for every $\psi \in \Lang_\land$,
either $\I\psi = \emptyset$ and $\psi \thesis \False$ is derivable,
or $\I\psi \neq \emptyset$ and $\C(\psi)$ is derivable.

\item
Lemma~\ref{lem:proof:compl:fin:ded:compl} is the completeness
of the deduction relation for formulae in $\Lang_\land$.
\end{itemize}

%%%%%%%%%%%%%%%%%%%%%%%%%%%%%%%%%%%%%%%%%%%%%%%%%%%%%%%%%%%%%%%%%%%%%%%%%%%
\begin{lemma}
\label{lem:proof:compl:fin:ded:cor}
%%%%%%%%%%%%%%%%%%%%%%%%%%%%%%%%%%%%%%%%%%%%%%%%%%%%%%%%%%%%%%%%%%%%%%%%%%%
In the extension of Figure~\ref{fig:log:ded} (\S\ref{sec:log})
with \eqref{eq:proof:compl:cc}:
\begin{enumerate}[(1)]
\item
for all $\varphi,\psi \in \Lang_\land(\PT)$,
we have
$\I\psi \sle \I\varphi$
if
$\psi \thesis_\PT \varphi$;

\item
for all $\varphi \in \Lang_\land$,
we have
$\I\varphi \neq \emptyset$
if
$\C(\varphi)$.
\end{enumerate}
%%%%%%%%%%%%%%%%%%%%%%%%%%%%%%%%%%%%%%%%%%%%%%%%%%%%%%%%%%%%%%%%%%%%%%%%%%%
\end{lemma}
%%%%%%%%%%%%%%%%%%%%%%%%%%%%%%%%%%%%%%%%%%%%%%%%%%%%%%%%%%%%%%%%%%%%%%%%%%%





%%%%%%%%%%%%%%%%%%%%%%%%%%%%%%%%%%%%%%%%%%%%%%%%%%%%%%%%%%%%%%%%%%%%%%%%%%%
\begin{proof}
%%%%%%%%%%%%%%%%%%%%%%%%%%%%%%%%%%%%%%%%%%%%%%%%%%%%%%%%%%%%%%%%%%%%%%%%%%%
We reason by mutual induction the definition of $\thesis$ and $\C$.
Thanks to Proposition~\ref{prop:sem:sound:ded},
we do not have to consider the rules in Figure~\ref{fig:log:ded}.
We reason by cases on the last applied rule.
\begin{description}
\item[Cases of]
\[
\begin{array}{c}

\dfrac{}{\C(\True)}

\quad

\dfrac
  {\text{$\BT \in \Base$ and $a \in \BT$}}
  {\C(\form a)}

\quad

\dfrac{\C(\varphi)}
  {\C(\form\fold \varphi)}

\quad

\dfrac{\C(\psi)
  \quad
  \psi \thesis \varphi}
  {\C(\varphi)}

\end{array}
\]

\noindent
Trivial.

\item[Case of]
\[
\dfrac{\C(\varphi) 
  \quad
  \C(\psi)}
  {\C(\pair{\varphi,\psi})}
\]

\noindent
Recall that $\pair{\varphi,\psi} = \form{\pi_1}\varphi \land \form{\pi_2}\psi$.
Hence, if $\I{\pair{\varphi,\psi}} = \emptyset$ then we must have
either $\I\varphi = \emptyset$ or $\I\psi = \emptyset$,
and we conclude by induction hypothesis.

\item[Case of]
\[
\ax{C}
\dfrac{\C(\psi)}
  {(\psi \realto \False) \,\thesis\, \False}
\]

By induction hypothesis, we have $\I\psi \neq \emptyset$.
Hence $\I{\psi \realto \False} = \emptyset$.

\item[Case of]
\[
\dfrac{\begin{array}{l}
  \text{$I$ finite and $\forall i \in I$,}~
  \C(\psi_i) 
  ~\text{and}~
  \C(\varphi_i) ;
  \\
  \text{$\forall J \sle I$,}~
  \bigwedge_{j \in J} \psi_j \thesis \False
  ~\text{or}~
  \C\left( \bigwedge_{j \in J} \varphi_j \right)
  \end{array}}
  {\C\left( \bigwedge_{i \in I}(\psi_i \realto \varphi_i) \right)}
\]

Let $\PT,\PTbis$ such that
$\varphi_i \in \Lang_\land(\PT)$ and $\psi_i \in \Lang_\land(\PTbis)$
for all $i \in I$.

First, by induction hypothesis 
we have
$\I{\psi_i} \neq \emptyset$
and
$\I{\varphi_i} \neq \emptyset$
for all $i \in I$.
Hence, it follows from Lemma~\ref{lem:top:char:fin}
that for each $i \in I$,
there are finite $d_i \in \I\PT$
and $e_i \in \I\PTbis$
such that
$\up d_i = \I{\varphi_i}$
and
$\up e_i = \I{\psi_i}$.
We thus have
$\I{\psi_i \realto \varphi_i} = \up(e_i \step d_i)$
for each $i \in I$,
so that
\[
\begin{array}{l l l}
  \I{\bigwedge_{i \in I} \left( \psi_i \realto \varphi_i \right)}
& =
& \bigcap_{i \in I} \up(e_i \step d_i)
\end{array}
\]


Assume 
$\I{\bigwedge_{i \in I} \left( \psi_i \realto \varphi_i \right)} = \emptyset$.
As recalled in \S\ref{sec:proof:sem:pure}
(see also \cite[Theorem 1.4.12]{ac98book}),
there is some $J \sle I$
such that
\[
\begin{array}{l l l !{\quad\text{and}\quad} l l l}
  \bigcap_{i \in I} \up e_i
& \neq
& \emptyset

& \bigcap_{i \in I} \up d_i
& =
& \emptyset
\end{array}
\]

But the induction hypothesis
yields either $\bigcap_{i \in I} \up e_i = \emptyset$
or $\bigcap_{i \in I} \up d_i \neq \emptyset$,
a contradiction.
\qed
\end{description}
%%%%%%%%%%%%%%%%%%%%%%%%%%%%%%%%%%%%%%%%%%%%%%%%%%%%%%%%%%%%%%%%%%%%%%%%%%%
\end{proof}
%%%%%%%%%%%%%%%%%%%%%%%%%%%%%%%%%%%%%%%%%%%%%%%%%%%%%%%%%%%%%%%%%%%%%%%%%%%

%%%%%%%%%%%%%%%%%%%%%%%%%%%%%%%%%%%%%%%%%%%%%%%%%%%%%%%%%%%%%%%%%%%%%%%%%%%
\begin{lemma}
\label{lem:proof:compl:fin:ded:c-false}
%%%%%%%%%%%%%%%%%%%%%%%%%%%%%%%%%%%%%%%%%%%%%%%%%%%%%%%%%%%%%%%%%%%%%%%%%%%
For all $\psi \in \Lang_\land(\PT)$,
\begin{enumerate}[(1)]
\item
\label{item:proof:compl:fin:ded:c-false:c}
if $\I\psi \neq \emptyset$,
then $\C(\psi)$ is derivable;


\item
\label{item:proof:compl:fin:ded:c-false:ded}
if $\I\psi = \emptyset$,
then $\psi \thesis \False$ is derivable.
\end{enumerate}
%%%%%%%%%%%%%%%%%%%%%%%%%%%%%%%%%%%%%%%%%%%%%%%%%%%%%%%%%%%%%%%%%%%%%%%%%%%
\end{lemma}
%%%%%%%%%%%%%%%%%%%%%%%%%%%%%%%%%%%%%%%%%%%%%%%%%%%%%%%%%%%%%%%%%%%%%%%%%%%

%%%%%%%%%%%%%%%%%%%%%%%%%%%%%%%%%%%%%%%%%%%%%%%%%%%%%%%%%%%%%%%%%%%%%%%%%%%
\begin{proof}
%%%%%%%%%%%%%%%%%%%%%%%%%%%%%%%%%%%%%%%%%%%%%%%%%%%%%%%%%%%%%%%%%%%%%%%%%%%
Both statements are proven by a simultaneous induction on the (finite!)
size of $\psi \in \Lang_\land$.

Note that $\C(\True)$ and $\False \thesis \False$
are always derivable,
so that we may always assume $\psi \neq \True$ in
item \eqref{item:proof:compl:fin:ded:c-false:c}
and $\psi \neq \False$ in
item \eqref{item:proof:compl:fin:ded:c-false:ded}.

We now reason by cases on $\PT$.
\begin{description}
\item[Case of $\BT$ with $\BT \in \Base$.]
We begin with item \eqref{item:proof:compl:fin:ded:c-false:c}.
If $\I\psi \neq \emptyset$,
then we must have $\I\psi = \{a\}$ for some $a \in \BT$.
Hence $\psi$ is $\thesisiff$-equivalent to $\form a$,
and we get $\C(\psi)$ since $\C(\form a)$ and $\form a \thesis \psi$.

We now turn to item \eqref{item:proof:compl:fin:ded:c-false:ded}.
If $\psi \neq \False$, then it must be the case that
$\psi$ is a finite conjunction containing (at least)
$\form a$ and $\form b$ for some $a \neq b$ in $\BT$.
We can thus conclude using the rule $\form a \land \form b \thesis \False$.

\item[Case of $\rec\TV.\PT$.]
If $\psi \neq \False$,
then $\psi$ is $\thesisiff$-equivalent to a formula of the form
$\bigwedge_{j \in J}\form\fold \psi_j$ for some finite $J$,
where each $\psi_j$ is smaller than $\psi$.
Let $\psi'$ be the formula
$\bigwedge_{j \in J} \psi_j$.
Note that $\psi \thesisiff \form\fold\psi'$ by Example~\ref{ex:log:modalnf},
so that
$\I\psi = \I{\form\fold}(\I{\psi'})$.

We first consider item \eqref{item:proof:compl:fin:ded:c-false:c}.
If $\I\psi \neq \emptyset$,
then $\I{\psi'} \neq \emptyset$.
If moreover $J \neq \emptyset$ (otherwise $\psi = \True$),
then
$\psi'$ is smaller than $\psi$ and the induction hypothesis
yields $\C(\psi')$.
Hence $\C(\form\fold \psi')$
and we get $\C(\psi)$ since $\form\fold \psi' \thesis \psi$.

We now turn to item \eqref{item:proof:compl:fin:ded:c-false:ded}.
If $\I\psi = \emptyset$ then
$\I{\psi'} = \emptyset$.
In this case, $J$ must be non-empty
(since otherwise $\I{\psi'} = \I{\True}$).
So $\psi'$ is smaller than $\psi$
and the induction hypothesis yields
$\psi' \thesis_{\PT[\rec\TV.\PT/\TV]} \False$,
so that
$\form\fold\psi' \thesis_{\rec\TV.\PT} \form\fold \False$.
Then we are done since $\form\fold \False \thesis \False$
(take $I = \emptyset$ in the rule
$\form\triangle \bigvee (-) \thesis \bigvee \form\triangle(-)$).

\item[Case of $\PT_1 \times \PT_2$.]
If $\psi \neq \False$,
%
then $\psi$ is $\thesisiff$-equivalent to a formula of the form
\( 
  (\bigwedge_{j \in J} \form{\pi_1}\psi_j)
  \land
  (\bigwedge_{k \in K} \form{\pi_2}\psi_k)
\)
where $\psi_j,\psi_k$ are smaller than $\psi$,
and
where we can assume w.l.o.g.\ $J \cap K = \emptyset$.
Let $\psi' = \bigwedge_{j \in J} \psi_j$
and $\psi'' = \bigwedge_{k \in K} \psi_k$,
so that $\psi \thesisiff \form{\pi_1}\psi' \land \form{\pi_2}\psi''$.

We first consider item \eqref{item:proof:compl:fin:ded:c-false:c}.
If $\I\psi \neq \emptyset$,
then $\I{\psi'} \neq \emptyset$
and $\I{\psi''} \neq \emptyset$.
If $J$ (resp.\@ $K$) is non-empty,
then $\psi'$ (resp. $\psi''$) is smaller than $\psi$
and the induction hypothesis applies to yield
$\C(\psi')$ (resp.\ $\C(\psi'')$).
If $J$ (resp.\ $K$) is empty, then
$\psi' = \True$ (resp.\ $\psi'' = \True$),
so that $\C(\psi')$ (resp.\ $\C(\psi'')$).
%
Hence, in any case we get $\C(\psi')$ and $\C(\psi'')$,
so that $\C(\pair{\psi',\psi''})$ and we are done.

We now turn to item \eqref{item:proof:compl:fin:ded:c-false:ded}.
If  $\I\psi = \emptyset$,
then we must have either $\I{\psi'} = \emptyset$
or $\I{\psi''} = \emptyset$,
say  $\I{\psi'} = \emptyset$ (the other case is symmetric).
Reasoning similarly as above
yields $\psi' \thesis_{\PT_1} \False$ by induction hypothesis,
and we conclude using
$\form{\pi_1}\False \thesis_{\PT_1 \times \PT_2} \False$.

\item[Case of $\PTbis \arrow \PT$.]
If $\psi \neq \False$,
then $\psi$ is $\thesisiff$-equivalent to a formula of the form
$\bigwedge_{i \in I} (\psi''_i \realto \psi'_i)$,
where $\psi'_i, \psi''_i$ are smaller than $\psi$.

Assume first that for some $i \in I$,
we have $\I{\psi'_i} = \emptyset$
with $\I{\psi''_i} \neq \emptyset$.
Then $\I{\psi''_i \realto \psi'_i} = \emptyset$
and $\I\psi = \emptyset$.
Hence we must be in the case of item \eqref{item:proof:compl:fin:ded:c-false:ded}.
Moreover, by induction hypothesis we have $\C(\psi''_i)$
and $\psi' \thesis \False$,
so that we can derive $\psi \thesis \False$ using the rule $\ax{C}$.

Otherwise, we have $\I{\psi'_i} \neq \emptyset$
for all $i \in I$ such that $\I{\psi''_i} \neq \emptyset$.

Given $i \in I$ such that $\I{\psi''_i} = \emptyset$,
by induction hypothesis we have $\psi''_i \thesis \False$,
and since $\True \thesis \left( \False \realto \varphi \right)$
for any formula $\varphi$ (Remark~\ref{rem:log:realto}),
we get $\True \thesis \left(\psi''_i \realto \psi'_i \right)$.
Hence
$\bigwedge_{j \neq i}(\psi''_j \realto \psi'_j) \thesisiff \psi$.


We can therefore reduce the case of 
$\bigwedge_{i \in I} (\psi''_i \realto \psi'_i)$
where 
$\I{\psi'_i} \neq \emptyset$ and $\I{\psi''_i} \neq \emptyset$
for all $i \in I$.
In, particular, the induction hypothesis yields $\C(\psi'_i)$
and $\C(\psi''_i)$ for all $i \in I$.

Regarding item \eqref{item:proof:compl:fin:ded:c-false:c},
if $\I\psi \neq \emptyset$,
then for all $J \sle I$ we have either
$\SI{\bigwedge_{j \in J} \psi''_j} = \emptyset$
or
$\SI{\bigwedge_{j \in J} \psi'_j} \neq \emptyset$,
so that either
$\bigwedge_{j \in J} \psi''_j \thesis \False$
or
$\C(\bigwedge_{j \in J} \psi'_j)$
by induction hypothesis.
We can thus obtain $\C(\psi)$ by using the last rule in \eqref{eq:proof:compl:cc}.

Concerning item \eqref{item:proof:compl:fin:ded:c-false:ded},
if $\I\psi = \emptyset$,
then there is some $J \sle I$
such that
$\SI{\bigwedge_{j \in J} \psi''_j} \neq \emptyset$
while
$\SI{\bigwedge_{j \in J} \psi'_j} = \emptyset$.
Hence
$\C(\bigwedge_{j \in J} \psi''_j)$
and
$\bigwedge_{j \in J} \psi'_j \thesis \False$
by induction hypothesis.
%
We have
\[
\begin{array}{l l l}
  \psi
& \thesis
& \bigwedge_{i \in I}
  \left(
  \left( \bigwedge_{j \in J} \psi''_j \right)
  \realto
  \psi'_i
  \right)
\end{array}
\]

\noindent
and thus
\[
\begin{array}{l l l}
  \psi
& \thesis
& \left( \bigwedge_{j \in J} \psi''_j \right)
  \realto
  \bigwedge_{j \in J} \psi'_j
\end{array}
\]

\noindent
and we obtain $\psi \thesis \False$ using the rule $\ax{C}$.
\qed
\end{description}
%%%%%%%%%%%%%%%%%%%%%%%%%%%%%%%%%%%%%%%%%%%%%%%%%%%%%%%%%%%%%%%%%%%%%%%%%%%
\end{proof}
%%%%%%%%%%%%%%%%%%%%%%%%%%%%%%%%%%%%%%%%%%%%%%%%%%%%%%%%%%%%%%%%%%%%%%%%%%%

%%%%%%%%%%%%%%%%%%%%%%%%%%%%%%%%%%%%%%%%%%%%%%%%%%%%%%%%%%%%%%%%%%%%%%%%%%%
\begin{lemma}
\label{lem:proof:compl:fin:ded:compl}
%%%%%%%%%%%%%%%%%%%%%%%%%%%%%%%%%%%%%%%%%%%%%%%%%%%%%%%%%%%%%%%%%%%%%%%%%%%
For all $\varphi,\psi \in \Lang_\land(\PT)$,
if $\I\psi \sle \I\varphi$,
then $\psi \thesis \varphi$ is derivable;
%%%%%%%%%%%%%%%%%%%%%%%%%%%%%%%%%%%%%%%%%%%%%%%%%%%%%%%%%%%%%%%%%%%%%%%%%%%
\end{lemma}
%%%%%%%%%%%%%%%%%%%%%%%%%%%%%%%%%%%%%%%%%%%%%%%%%%%%%%%%%%%%%%%%%%%%%%%%%%%

%%%%%%%%%%%%%%%%%%%%%%%%%%%%%%%%%%%%%%%%%%%%%%%%%%%%%%%%%%%%%%%%%%%%%%%%%%%
\begin{proof}
%%%%%%%%%%%%%%%%%%%%%%%%%%%%%%%%%%%%%%%%%%%%%%%%%%%%%%%%%%%%%%%%%%%%%%%%%%%
The proof is by induction on sum of the (finite!) 
sizes of $\psi$ and $\varphi$.

First, note that if $\varphi = \bigwedge_{i \in I}\varphi_i$
for some finite set $I$,
then $\I\psi \sle \I\varphi$ implies
$\I\psi \sle \I{\varphi_i}$ for all $i \in I$,
and we can obtain $\psi \thesis \varphi$ from the induction hypotheses.

Also, if $\I\psi = \emptyset$, then Lemma~\ref{lem:proof:compl:fin:ded:c-false}
yields that $\psi \thesis \False$,
so that $\psi \thesis \varphi$.
This in particular applies when $\I\varphi = \emptyset$,
since we must then have $\I\psi = \emptyset$ as well.

We can thus assume that $\varphi$ is not a conjunction
and that both $\I\varphi$ and $\I\psi$ are not empty.
We now reason by cases on $\PT$.
\begin{description}
\item[Case of $\BT$ with $\BT \in \Base$.]
In this case, we must have $\I\varphi = \{a\}$ for some $a \in \BT$,
and thus $\I\psi = \{a\}$ as well.
We can then obtain $\psi \thesis \varphi$ using the rule $\form a \thesis \form a$.

\item[Case of $\rec\TV.\PT$.]
Reasoning as in Lemma~\ref{lem:proof:compl:fin:ded:c-false},
we can assume that $\psi$ is of the form
$\bigwedge_{j \in J}\form\fold \psi_j$ for some finite $J$.
Let $\psi'$ be the formula
$\bigwedge_{j \in J} \psi_j$.
Note that $\psi \thesisiff \form\fold\psi'$ by Example~\ref{ex:log:modalnf},
so that
$\I\psi = \I{\form\fold}(\I{\psi'})$.

Moreover, we must have $\varphi = \form\fold \varphi'$.
Hence $\I\psi \sle \I\varphi$
implies $\I{\psi'} \sle \I{\varphi'}$.
Note that $\varphi'$ is smaller than $\varphi$
while $\psi'$ is not greater than $\psi$.
Hence the induction hypothesis yields
$\psi' \thesis \varphi'$, and we are done.

\item[Case of $\PT_1 \times \PT_2$.]
Reasoning as in Lemma~\ref{lem:proof:compl:fin:ded:c-false},
we can assume that $\psi$ is of the form
\( 
  (\bigwedge_{j \in J} \form{\pi_1}\psi_j)
  \land
  (\bigwedge_{k \in K} \form{\pi_2}\psi_k)
\)
with $J \cap K = \emptyset$.
Let $\psi' = \bigwedge_{j \in J} \psi_j$
and $\psi'' = \bigwedge_{k \in K} \psi_k$,
so that $\psi \thesisiff \form{\pi_1}\psi' \land \form{\pi_2}\psi''$.

Moreover, we have $\varphi = \form{\pi_i}\varphi'$
say $i = 1$ (the case $i =2$ is symmetric).
But then we must have $\I{\psi'} \sle \I{\varphi'}$
and similarly as above (again), the induction hypothesis
yields $\psi' \thesis_{\PT_1} \varphi'$.
It is then easy to conclude.

\item[Case of $\PTbis \arrow \PT$.]
This is the most important case.
The proof is an adaptation to our setting
of the proof of~\cite[Proposition 10.5.2]{ac98book}.

First, note that $\varphi$ must be of the form $\varphi'' \realto \varphi'$.
If $\I{\varphi''} = \emptyset$,
then by Lemma~\ref{lem:proof:compl:fin:ded:c-false}
we obtain $\varphi'' \thesis \False$,
so that $\True \thesis (\varphi'' \realto \varphi')$
by Remark~\ref{rem:log:realto}.
Hence $\psi \thesis \varphi$ in this case.
We can thus assume $\I{\varphi''} \neq \emptyset$.
Since $\I\varphi \neq \emptyset$,
this implies that $\I{\varphi'} \neq \emptyset$ as well.
Hence, by Lemma~\ref{lem:top:char:fin}
there are finite $d''_i,d'_i$
such that $\I{\varphi''_i} = \up d''_i$
and $\I{\varphi'_i} = \up d'_i$.

On the other hand,
reasoning similarly as in Lemma~\ref{lem:proof:compl:fin:ded:c-false},
we can assume that $\psi$ is of the form
$\bigwedge_{i \in I} (\psi''_i \realto \psi'_i)$
for some finite set $I$, with $\I{\psi''_i} \neq \emptyset$
and $\I{\psi'_i} \neq \emptyset$ for all $i \in I$.
Hence, by Lemma~\ref{lem:top:char:fin},
for each $i \in I$ there are finite $e''_i,e'_i$
such that $\I{\psi''_i} = \up e''_i$
and $\I{\psi'_i} = \up e'_i$.
%
Moreover, $\bigvee_{i \in I}(e''_i \step e'_i)$
exists since $\I\psi \neq \emptyset$.

Hence $\I\psi \sle \I\varphi$ means
\[
\begin{array}{l l l}
  \up \bigvee_{i \in I} \left(e''_i \step e'_i \right)
& \sle
& \up \left( d'' \step d' \right)
\end{array}
\]

\noindent
which implies
\[
\begin{array}{l l l}
  d'' \step d'
& \leq
& \bigvee_{i \in I} \left(e''_i \step e'_i \right)
\end{array}
\]

\noindent
We thus have
\[
\begin{array}{l l l}
  d'
& \leq
& \bigvee_{d'' \geq e''_i} e_i
\end{array}
\]

\noindent
that is
\[
\begin{array}{l l l}
  \up \bigvee_{\up d'' \sle \up e''_i} e'_i
& \sle
& \up d'
\end{array}
\]

\noindent
In other words,
\[
\begin{array}{l l l}
  \I{\bigwedge_{\I{\varphi''} \sle \I{\psi ''_i}} \psi'_i}
& \sle
& \I{\varphi'}
\end{array}
\]

\noindent
and by induction hypothesis
\[
\begin{array}{l l l}
  \bigwedge_{\varphi'' \thesis \psi ''_i} \psi'_i
& \thesis
& \varphi'
\end{array}
\]

Hence we are done since
\[
\begin{array}{l l l}
  \psi
& \thesis
& \bigwedge_{i \in I} \left(
  \left( \bigwedge_{\varphi'' \thesis \psi ''_i} \psi''_i \right)
  \realto
  \psi'_i
  \right)
\end{array}
\]

\noindent
and thus
\[
\begin{array}{l l l}
  \psi
& \thesis
& \left( \bigwedge_{\varphi'' \thesis \psi ''_i} \psi''_i \right)
  \realto
  \bigwedge_{\varphi'' \thesis \psi ''_i} \psi'_i
\end{array}
\]

\noindent
while
\(
\varphi'' \thesis \bigwedge_{\varphi'' \thesis \psi ''_i} \psi''_i
\).
\qed
\end{description}
%%%%%%%%%%%%%%%%%%%%%%%%%%%%%%%%%%%%%%%%%%%%%%%%%%%%%%%%%%%%%%%%%%%%%%%%%%%
\end{proof}
%%%%%%%%%%%%%%%%%%%%%%%%%%%%%%%%%%%%%%%%%%%%%%%%%%%%%%%%%%%%%%%%%%%%%%%%%%%


We now turn to Theorem~\ref{thm:compl:fin},
namely the completeness for finite judgments.
While this result is essentially due to Abramsky \cite{abramsky91apal},
we nevertheless offer a proof since our system formally
differs from that of \cite{abramsky91apal}.
Let us recall the statement of Theorem~\ref{thm:compl:fin}.

%%%%%%%%%%%%%%%%%%%%%%%%%%%%%%%%%%%%%%%%%%%%%%%%%%%%%%%%%%%%%%%%%%%%%%%%%%%
\begin{theorem}[Theorem \ref{thm:compl:fin}]
\label{thm:proof:compl:fin}
%%%%%%%%%%%%%%%%%%%%%%%%%%%%%%%%%%%%%%%%%%%%%%%%%%%%%%%%%%%%%%%%%%%%%%%%%%%
Assume $\Env$ and  $\RT$ are finite.
%and $\UPT\Env \thesis M : \UPT\RT$ is derivable.
If $\Env \thesis M : \RT$ is sound,
then $\Env \thesis M : \RT$ is derivable in the system of \S\ref{sec:reft}
extended with \eqref{eq:proof:compl:cc}.
%%%%%%%%%%%%%%%%%%%%%%%%%%%%%%%%%%%%%%%%%%%%%%%%%%%%%%%%%%%%%%%%%%%%%%%%%%%
\end{theorem}
%%%%%%%%%%%%%%%%%%%%%%%%%%%%%%%%%%%%%%%%%%%%%%%%%%%%%%%%%%%%%%%%%%%%%%%%%%%

%%%%%%%%%%%%%%%%%%%%%%%%%%%%%%%%%%%%%%%%%%%%%%%%%%%%%%%%%%%%%%%%%%%%%%%%%%%
\begin{proof}
%%%%%%%%%%%%%%%%%%%%%%%%%%%%%%%%%%%%%%%%%%%%%%%%%%%%%%%%%%%%%%%%%%%%%%%%%%%
First, note that Lemma~\ref{lem:reft} (\S\ref{sec:reft})
restricts to finite types, in the sense that if a type
$\RTter$ is finite, then there is some $\varphi \in \Lang_\land(\UPT\RTter)$
such that $\RTter \eqtype \reft{\UPT\RTter \mid \varphi}$.

Consider first the case of a sound judgment
$\Env \thesis M : \RT$
where 
$\Env = x_1:\RTbis_1,\dots,x_n:\RTbis_n$
is such that $\I{\RTbis_i} = \emptyset$ for some $i \in \{1,\dots,n\}$.
Since $\I{\UPT{\RTbis_i}}$ is not empty (as it is a Scott domain),
taking $\psi \in \Lang_\land(\UPT{\RTbis_i})$
such that $\RTbis_i \eqtype \reft{\UPT{\RTbis_i} \mid \psi}$,
we must have $\I\psi = \emptyset$
and thus $\psi \thesis \False$ by Lemma~\ref{lem:proof:compl:fin:ded:c-false}.
We can thus conclude by taking $I = \emptyset$ in the rule
\[
\dfrac{
  \begin{array}{l}
  \UPT{\Env'}, x:\PTbis, \UPT{\Env''} \thesis M : \UPT\RT
  \\
  \text{for each $i \in I$,}\quad
  \Env', x:\reft{\PTbis \mid \psi_i},\Env' \thesis M : \RT
  \end{array}}
  {\Env', x : \reft{\PTbis \mid \bigvee_{i \in I} \psi_i} , \Env'' \thesis M : \RT}
\]




Hence we can reduce the case of a sound judgment
$\Env \thesis M : \RT$ with
$\Env = x_1:\RTbis_1,\dots,x_n:\RTbis_n$
such that $\I{\RTbis_i} \neq \emptyset$ for all $i = 1,\dots,n$.
We now reason by induction on the typing derivation of $\UPT\Env \thesis M : \UPT\RT$.
\begin{description}
\item[Case of]
\[
\dfrac{(x:\UPT\RT) \in \UPT\Env}
  {\UPT\Env \thesis x:\UPT\RT}
\]

We have $(x : \RTbis) \in \Env$ for some type $\RTbis$ with $\UPT\RTbis = \UPT\RT$.
Let $\varphi,\psi \in \Lang_\land(\UPT\RT)$
such that $\RT \eqtype \reft{\UPT\RT \mid \varphi}$
and $\RTbis \eqtype \reft{\UPT\RT \mid \psi}$.
By assumption on $\Env \thesis M :\RT$,
we have $\I\psi \sle \I\varphi$.
Hence $\psi \thesis \varphi$
by Proposition~\ref{prop:proof:compl:fin:ded}.
We then conclude by subtyping.

\item[Case of]
\[
\dfrac{\UPT\Env \thesis N_1 : \PT_1
  \qquad
  \UPT\Env \thesis N_2 : \PT_2}
  {\UPT\Env \thesis \pair{N_1,N_2} : \PT_1 \times \PT_2}
\]

\noindent
where $\UPT\RT = \PT_1 \times \PT_2$
and $M = \pair{N_1,N_2}$.

Let $\varphi \in \Lang_\land(\PT_1 \times \PT_2)$
such that $\RT \eqtype \reft{\PT_1 \times \PT_2 \mid \varphi}$.
Our assumption on $\Env \thesis M : \RT$
implies that $\I\varphi \neq \emptyset$.
Hence,
reasoning as in the proof of Lemma~\ref{lem:proof:compl:fin:ded:c-false}
yields that $\varphi \thesisiff \pair{\psi_1,\psi_2}$
for some $\psi_1 \in \Lang_\land(\PT_1)$
and some $\psi_2 \in \Lang_\land(\PT_2)$.

Since $\Env \thesis M :\RT$ is sound,
so are
$\Env \thesis N_1 : \reft{\PT_1 \mid \psi_1}$
and
$\Env \thesis N_1 : \reft{\PT_2 \mid \psi_2}$.
The induction hypotheses yield that
$\Env \thesis N_1 : \reft{\PT_1 \mid \psi_1}$
and
$\Env \thesis N_1 : \reft{\PT_2 \mid \psi_2}$
are derivable.
We can then conclude using the rules
\[
\dfrac{\Env \thesis N_i : \reft{\PT_i \mid \psi_i}
  \qquad
  \Env \thesis N_{3-i} : \PT_{3-i}}
  {\Env \thesis \pair{N_1,N_2} : \reft{\PT_1 \times \PT_2 \mid \form{\pi_i} \psi_i}}
\]

\noindent
for $i = 1$ and $i = 2$.

\item[Case of]
\[
\begin{array}{c}
\dfrac{\UPT\Env \thesis N : \PT \times \PTbis}
  {\UPT\Env \thesis \pi_i(N) : \UPT\RT}
\end{array}
\]

\noindent
where $i = 1,2$ and $M = \pi_1(N)$.
Assume w.l.o.g.\ $i = 1$ (so that $\UPT\RT = \PT$).
Let $\varphi \in \Lang_\land(\UPT\RT)$ such that
$\RT \eqtype \reft{\UPT\RT \mid \varphi}$.
Our assumption on $\Env \thesis M : \RT$
implies that $\I\varphi \neq \emptyset$,
and
since $\Env \thesis M : \RT$ is sound,
we get that
$\Env \thesis N : \reft{\PT \times \PTbis \mid \form{\pi_1}\varphi}$
is sound, and thus derivable by induction hypothesis.
We then conclude with the rule
\[
\dfrac{\Env \thesis N : \reft{\PT \times \PTbis \mid \form{\pi_1} \varphi}}
  {\Env \thesis \pi_1(N) : \reft{\PT \mid \varphi}}
\]

\item[Case of]
\[
\dfrac{\UPT\Env \thesis N : \PT[\rec\TV.\PT/\TV]}
  {\UPT\Env \thesis \fold(N) : \rec\TV.\PT}
\]

\noindent
where $\UPT\RT = \rec\TV.\PT$ and $M = \fold(M)$.

Let $\varphi \in \Lang_\land(\rec\TV.\PT)$
such that $\RT \eqtype \reft{ \rec\TV.\PT \mid \varphi}$.
Our assumption on $\Env \thesis M : \RT$
implies that $\I\varphi \neq \emptyset$.
Hence,
reasoning as in the proof of Lemma~\ref{lem:proof:compl:fin:ded:c-false}
yields that $\varphi \thesisiff \form{\fold}\psi$
for some $\psi \in \Lang_\land(\PT[\rec\TV.\PT/\TV])$.
Moreover, since $\Env \thesis M : \RT$ is sound,
so is
$\Env \thesis N : \reft{\PT[\rec\TV.\PT/\TV] \mid \psi}$.
We can thus conclude using the induction hypothesis and the rule
\[
\dfrac{\Env \thesis N : \reft{\PT[\rec\TV.\PT/\TV] \mid \psi}}
  {\Env \thesis \fold(N) : \reft{\rec\TV.\PT \mid \form\fold \psi}}
\]


\item[Case of]
\[
\dfrac{\UPT\Env \thesis N : \rec\TV.\PT}
  {\UPT\Env \thesis \unfold(N) : \PT[\rec\TV.\PT/\TV]}
\]

\noindent
where $\UPT\RT = \PT[\rec\TV.\PT/\TV]$
and $N = \unfold(M)$.

Let $\varphi \in \Lang_\land(\PT[\rec\TV.\PT/\TV])$
such that
$\RT \eqtype \reft{\PT[\rec\TV.\PT/\TV] \mid \varphi}$.
Our assumption on $\Env \thesis M : \RT$
implies that 
$\Env \thesis N : \reft{\rec\TV.\PT \mid \form\fold \varphi}$
is sound,
and we conclude using the induction hypothesis and the rule
\[
\dfrac{\Env \thesis N : \reft{\rec\TV.\PT \mid \form\fold \varphi}}
  {\Env \thesis \unfold(N) : \reft{\PT[\rec\TV.\PT/\TV] \mid \varphi}}
\]

\item[Case of]
\[
\dfrac{\UPT\Env,x:\PTbis \thesis N : \PT}
  {\UPT\Env \thesis \lambda x.N : \PTbis \arrow \PT}
\]

\noindent
where $\UPT\RT = \PTbis \arrow \PT$,
and where $M = \lambda x.N$.

Let $\varphi \in \Lang_\land(\PTbis \arrow \PT)$
such that
$\RT \eqtype \reft{\PTbis \arrow \PT \mid \varphi}$.
Our assumption on $\Env \thesis M : \RT$
implies $\I\varphi \neq \emptyset$.
Reasoning as in the proof of Lemma~\ref{lem:proof:compl:fin:ded:c-false}
yields that $\varphi \thesisiff \bigwedge_{i \in I}(\varphi''_i \realto \varphi'_i)$
for some finite set $I$.
Let $i \in I$.
The judgment
\(
  \Env
  \thesis
  \lambda x.N
  :
  \reft{\PTbis \arrow \PT \mid \varphi''_i \realto \varphi'_i}
\)
is sound,
and so is
\(
  \Env, x : \reft{\PTbis \mid \varphi''_i}
  \thesis
  N
  :
  \reft{\PT \mid \varphi'_i}
\).
Using the induction hypothesis, we derive
\(
  \Env
  \thesis
  \lambda x.N
  :
  \reft{\PTbis \arrow \PT \mid \varphi''_i \realto \varphi'_i}
\).
We can then derive
\(
  \Env
  \thesis
  \lambda x.N
  :
  \reft{\PTbis \arrow \PT \mid \varphi}
\).


\item[Case of]
\[
\dfrac{\UPT\Env \thesis N : \PTbis \arrow \PT
  \qquad
  \UPT\Env \thesis V : \PTbis}
  {\UPT\Env \thesis N V : \PT}
\]

\noindent
where $\UPT\RT = \PT$ and where $M = N V$.

Write $\Env = x_1:\RTbis_1,\dots,x_n:\RTbis_n$.
Given $i \in \{1,\dots,n\}$,
let $\psi_i \in \Lang_\land(\UPT{\RTbis_i})$
such that $\RTbis_i \eqtype \reft{\UPT{\RTbis_i} \mid \psi_i}$.
Moreover, by assumption we have $\I{\psi_i} \neq \emptyset$,
hence by Lemma~\ref{lem:top:char:fin}
there is a finite $e_i \in \I{\UPT{\RTbis_i}}$
such that $\I{\psi_i} = \up e_i$.

Similarly, let $\varphi \in \Lang_\land(\PT)$
such that $\RT \eqtype \reft{\PT \mid \varphi}$.
Our assumption on $\Env \thesis M : \RT$
implies that $\I\varphi \neq \emptyset$.
Hence, again by Lemma~\ref{lem:top:char:fin}
there is a finite $d \in \I{\PT}$
such that $\I{\varphi} = \up d$.


Since $\Env \thesis M : \RT$ is sound,
we have
$\I M(\vec e) \in \varphi$.
But note that
$\I M(\vec e) = \I N(\vec e)\left(\I V(\vec e) \right)$.

Now, since $\I\PTbis$ is a Scott domain, it is algebraic,
and $\I V(\vec e)$ is the directed l.u.b.\ of the finite $e \leq \I V(\vec e)$.
Since $\I N(\vec e)$ is Scott-continuous, we thus get
that $\I M(\vec e)$ is the l.u.b.\ of the directed set
\[
\left\{
  \I N(\vec e)(e)
  \mid
  \text{$e$ finite and $\leq \I V(\vec e)$}
\right\}
\]

\noindent
Since $d \leq \I M(\vec e)$ and since $d$ is finite,
it follows that we have
$d \leq \I N(\vec e)(e)$ for some finite $e \leq \I V(\vec e)$.
By Lemma~\ref{lem:top:char:fin},
there is a formula $\psi \in \Lang_\land(\PTbis)$
such that $\I\psi = \up e$.

Since $d \leq \I N(\vec e)(e)$, we have
$(e \step d) \leq \I N(\vec e)$,
so that
$\I N(\vec e) \in \I{\psi \realto \varphi}$.
Since $\I N$ is monotone, it follows that 
$\Env \thesis N : \reft{\PTbis \arrow \PT \mid \psi \realto \varphi}$
is sound.
Hence, this judgment is derivable by induction hypothesis.

Similarly, since $e \leq \I V(\vec e)$,
we obtain that the judgment
$\Env \thesis V : \reft{\PTbis \mid \psi}$
is sound and thus derivable.

We can then easily derive $\Env \thesis M : \reft{\PT \mid \varphi}$
and $\Env \thesis M : \RT$.

\item[Case of]
\[
\dfrac{\UPT\Env,x:\UPT\RT \thesis N : \UPT\RT}
  {\UPT\Env \thesis \fix x.N : \UPT\RT}
\]

\noindent
where $M = \fix x.N$.

Write $\Env = x_1:\RTbis_1,\dots,x_n:\RTbis_n$.
Similarly as above, for each $i \in \{1,\dots,n\}$
there is a finite $e_i \in \I{\UPT{\RTbis_i}}$
such that $\I{\RTbis_i} = \up e_i$.
Similarly, there are $\varphi \in \Lang_\land(\UPT\RT)$
such that $\RT \eqtype \reft{\UPT\RT \mid \varphi}$,
and a finite $d \in \I{\UPT\RT}$ such that $\I\varphi = \up d$.

Let $f \colon \I{\UPT\RT} \to \I{\UPT\RT}$
be the Scott-continuous function which takes
$a \in \I{\UPT\RT}$ to $\I N(\vec e,a)$.
We have
\[
\begin{array}{l l l}
  \I{\fix x.N}(\vec e)
& =
& \bigvee_{k \in \NN}
  f^k(\bot)
\end{array}
\]

Since $d \leq \I{\fix x.N}(\vec e)$ with $d$ finite,
there is some $k \in \NN$
such that
$d \leq f^k(\bot)$.
Write $d_k$ for $d$.
By induction, for each $j = k-1,\dots,0$,
there is some finite $d_j$ such that 
$d_{j+1} \leq f(d_j)$
and
$d_j \leq f^{j}(\bot)$.
In particular, $d_0 = \bot$.
For each $j = 0,\dots,k$,
let $\varphi_j$ such that $\I{\varphi_j} = \up d_j$.
Note that $\varphi_k = \varphi$.
Moreover, since $d_0 = \bot$,
we can take $\varphi_0 = \True$.

Again reasoning similarly as above,
we obtain that 
$\Env, x : \reft{\UPT\RT \mid \varphi_j} \thesis N : \reft{\UPT\RT \mid \varphi_{j+1}}$
is sound and thus derivable for each $j = 0,\dots,k-1$.
Moreover,
$\Env \thesis \fix x.N : \reft{\UPT\RT \mid \varphi_0}$
is derivable.
We can then derive $\Env \thesis \fix x.N : \reft{\UPT\RT \mid \varphi}$
by iterated applications of the rule
\[
\dfrac{\Env \thesis \fix x.N : \reft{\PT \mid \psi}
  \quad
  \Env, x: \reft{\PT \mid \psi} \thesis N : \reft{\PT \mid \psi'}}
  {\Env \thesis \fix x.N : \reft{\PT \mid \psi'}}
~(\psi,\psi' \in \Lang_\land(\PT))
\]

\item[Case of]
\[
\dfrac{}
  {\UPT\Env \thesis a : \BT}
\]

Let $\varphi \in \Lang_\land(\BT)$ such that $\RT \eqtype \reft{\BT \mid \varphi}$.
By assumption on $\Env \thesis M :\RT$,
we have $a \in \I\varphi$,
so that $\I{\form a} \sle \I\varphi$.
Hence $\form a \thesis \varphi$
by Proposition~\ref{prop:proof:compl:fin:ded}.
We can then conclude by subtyping and
\[
\dfrac{}
  {\Env \thesis a : \reft{\BT \mid \form a}}
\]

\item[Case of]
\[
\dfrac{ \UPT\Env \thesis N : \BT
  \qquad\text{for each $a \in \BT$,\quad} \UPT\Env \thesis N_a : \UPT\RT}
  {\UPT\Env \thesis \cse\ N\ \copair{a \mapsto N_a \mid a \in \BT} : \UPT\RT}
\]

We reason similarly as in the cases of $\fix x.N$ and $N V$ above.

Write $\Env = x_1:\RTbis_1,\dots,x_n:\RTbis_n$.
For each $i \in I$
there is a finite $e_i \in \I{\UPT{\RTbis_i}}$
such that $\I{\RTbis_i} = \up e_i$.
Also, there is $\varphi \in \Lang_\land(\UPT\RT)$
such that $\RT \eqtype \reft{\UPT\RT \mid \varphi}$.

Assume first that $\I\varphi = \I\True$,
so that $\varphi \thesisiff \True$
by Proposition~\ref{prop:proof:compl:fin:ded}.
Then we have $\RT \eqtype \UPT\RT$ and we easily
derive $\Env \thesis M : \RT$.

Otherwise, we must have $\bot \notin \I\varphi$,
so that $\I M(\vec e) \neq \bot$
and thus $\I N(\vec e) \neq \bot$.
Hence $\I N(\vec e) = b$ for some $b \in \BT$.
Since $\I M(\vec e) = \I{N_b}(\vec e)$,
we obtain that the judgment
$\Env \thesis N_b : \RT$ is sound and thus derivable.
Moreover,
$\Env \thesis N : \reft{\BT \mid \form b}$ is sound and thus
derivable.
We can then conclude using the rule
\[
\dfrac{
  \Env \thesis N : \reft{\BT \mid \form b}
  \qquad
  \Env \thesis N_b : \RT
  \qquad
  \text{for each $a \in A$,\quad} \UPT\Env \thesis N_a : \UPT\RT}
  {\Env \thesis \cse\ N\ \copair{a \mapsto N_a \mid a \in \BT} : \RT}
\]
\qed
\end{description}
%%%%%%%%%%%%%%%%%%%%%%%%%%%%%%%%%%%%%%%%%%%%%%%%%%%%%%%%%%%%%%%%%%%%%%%%%%%
\end{proof}
%%%%%%%%%%%%%%%%%%%%%%%%%%%%%%%%%%%%%%%%%%%%%%%%%%%%%%%%%%%%%%%%%%%%%%%%%%%







%%%%%%%%%%%%%%%%%%%%%%%%%%%%%%%%%%%%%%%%%%%%%%%%%%%%%%%%%%%%%%%%%%%%%%%%%%%
\subsection{\nameref{sec:main}}
\label{sec:proof:main}
%%%%%%%%%%%%%%%%%%%%%%%%%%%%%%%%%%%%%%%%%%%%%%%%%%%%%%%%%%%%%%%%%%%%%%%%%%%

Note that Lemma~\ref{lem:compl:nf} and Theorem~\ref{thm:main}
are proven in \S\ref{sec:app:main}.
We prove Proposition~\ref{prop:main:eta}.

%%%%%%%%%%%%%%%%%%%%%%%%%%%%%%%%%%%%%%%%%%%%%%%%%%%%%%%%%%%%%%%%%%%%%%%%%%%
\begin{proposition}[Proposition \ref{prop:main:eta}]
\label{prop:proof:main:eta}
%%%%%%%%%%%%%%%%%%%%%%%%%%%%%%%%%%%%%%%%%%%%%%%%%%%%%%%%%%%%%%%%%%%%%%%%%%%
A normal judgment $\Env \thesis M : \RT$ is sound (resp.\ derivable)
if, and only if, so are all $(\Env' \thesis M' : \RT') \in \eta(\Env \thesis M : \RT)$.
%%%%%%%%%%%%%%%%%%%%%%%%%%%%%%%%%%%%%%%%%%%%%%%%%%%%%%%%%%%%%%%%%%%%%%%%%%%
\end{proposition}
%%%%%%%%%%%%%%%%%%%%%%%%%%%%%%%%%%%%%%%%%%%%%%%%%%%%%%%%%%%%%%%%%%%%%%%%%%%

The proof of Proposition~\ref{prop:proof:main:eta}
relies on the following
Lemmas~\ref{lem:proof:main:eta:prod} and \ref{lem:proof:main:eta:fun}.

%%%%%%%%%%%%%%%%%%%%%%%%%%%%%%%%%%%%%%%%%%%%%%%%%%%%%%%%%%%%%%%%%%%%%%%%%%%
\begin{lemma}
\label{lem:proof:main:eta:prod}
%%%%%%%%%%%%%%%%%%%%%%%%%%%%%%%%%%%%%%%%%%%%%%%%%%%%%%%%%%%%%%%%%%%%%%%%%%%
A (not necessarily normal) judgment
$\Env \thesis M : \RT_1 \times \RT_2$
is sound (resp.\ derivable)
if, and only if,
so are $\Env \thesis \pi_1 M : \RT_1$
and $\Env \thesis \pi_2 M : \RT_2$.
%%%%%%%%%%%%%%%%%%%%%%%%%%%%%%%%%%%%%%%%%%%%%%%%%%%%%%%%%%%%%%%%%%%%%%%%%%%
\end{lemma}
%%%%%%%%%%%%%%%%%%%%%%%%%%%%%%%%%%%%%%%%%%%%%%%%%%%%%%%%%%%%%%%%%%%%%%%%%%%

%%%%%%%%%%%%%%%%%%%%%%%%%%%%%%%%%%%%%%%%%%%%%%%%%%%%%%%%%%%%%%%%%%%%%%%%%%%
\begin{proof}
%%%%%%%%%%%%%%%%%%%%%%%%%%%%%%%%%%%%%%%%%%%%%%%%%%%%%%%%%%%%%%%%%%%%%%%%%%%
By Lemma~\ref{lem:reft},
there are formulae
$\varphi_1 \in \Lang(\UPT{\RT_1})$
and
$\varphi_2 \in \Lang(\UPT{\RT_2})$
such that
$\RT_1 \eqtype \reft{\UPT{\RT_1} \mid \varphi_1}$
and
$\RT_2 \eqtype \reft{\UPT{\RT_2} \mid \varphi_2}$.
Hence
\(
  \RT_1 \times \RT_2
  \eqtype
  \reft{\UPT{\RT_1} \times \UPT{\RT_2} \mid \pair{\varphi_1,\varphi_2}}
\).


It follows that
$\Env \thesis M : \RT_1 \times \RT_2$
is sound if, and only if,
so are $\Env \thesis \pi_1 M : \RT_1$
and $\Env \thesis \pi_2 M : \RT_2$.

It is clear that $\Env \thesis \pi_1 M : \RT_1$
and $\Env \thesis \pi_2 M : \RT_2$
are derivable whenever so is
$\Env \thesis M : \RT_1 \times \RT_2$.

For the converse, assume that
$\Env \thesis \pi_1 M : \RT_1$
and $\Env \thesis \pi_2 M : \RT_2$
are derivable.
We first show that
$\Env \thesis M : \RT_1 \times \UPT{\RT_2}$.
We reason by induction on the derivation of $\Env \thesis \pi_1 M : \RT_1$
and by cases on the last possible rule.
\begin{description}
\item[Case of]
\[
\dfrac{
  \begin{array}{l}
  \UPT\Env \thesis \pi_1 M : \UPT{\RT_1}
  \\
  \text{for each $i \in I$,}\quad \Env \thesis \pi_1 M : \reft{\UPT{\RT_1} \mid \psi_i}
  \end{array}}
  {\Env \thesis \pi_1 M : \reft{\UPT{\RT_1} \mid \bigwedge_{i \in I} \psi_i}}
\]

\noindent
where $\varphi_1 = \bigwedge_{i \in I}\psi_i$.
Then by induction hypothesis and subtyping,
for all $i \in I$ we can derive
\[
\begin{array}{*{5}{l}}
  \Env
& \thesis
& M
& :
& \reft{\UPT{\RT_1} \times \UPT{\RT_2} \mid \form{\pi_1}\psi_i}
\end{array}
\]

\noindent
and thus
\[
\begin{array}{*{5}{l}}
  \Env
& \thesis
& M
& :
& \reft{\UPT{\RT_1} \times \UPT{\RT_2} \mid \bigwedge_{i \in I} \form{\pi_1}\psi_i}
\end{array}
\]

\noindent
We can then conclude using subtyping and Example~\ref{ex:log:modalnf}.

\item[Case of]
\[
\dfrac{
  \begin{array}{l}
  \UPT\Env, x:\PTbis, \UPT{\Env'} \thesis \pi_1 M : \UPT{\RT_1}
  \\
  \text{for each $i \in I$,}\quad
  \Env, x:\reft{\PTbis \mid \psi_i},\Env' \thesis \pi_1 M : \RT_1
  \end{array}}
  {\Env, x : \reft{\PTbis \mid \bigvee_{i \in I} \psi_i} , \Env' \thesis \pi_1 M : \RT_1}
\]

By induction hypothesis.

\item[Case of]
\[
\dfrac{
  \Env \subtype \Env'
  \quad 
  \RT'_1 \subtype \RT_1
  \quad
  \Env' \thesis \pi_1 M : \RT'_1}
  {\Env \thesis \pi_1 M : \RT_1}
\]


By subtyping we obtain
$\Env' \thesis \pi_2 M : \RT_2$
and the induction hypothesis yields
$\Env' \thesis M : \RT'_1 \times \UPT{\RT_2}$.
Then we are done since
$\RT'_1 \times \UPT{\RT_2} \subtype \RT_1 \times \UPT{\RT_2}$
and $\Env \subtype \Env'$.


\item[Case of]
\[
\dfrac{\Env \thesis M : \reft{\UPT{\RT_1} \times \UPT{\RT_2} \mid \form{\pi_1} \varphi_1}}
  {\Env \thesis \pi_1 M : \reft{\UPT{\RT_1} \mid \varphi_1}}
\]

Since
\(
  \reft{\UPT{\RT_1} \times \UPT{\RT_2} \mid \form{\pi_1} \varphi_1}
  \eqtype
  \RT_1 \times \UPT{\RT_2}
\).

\item[Case of]
\[
\dfrac{\Env \thesis M : \RT_1 \times \RT_2}
  {\Env \thesis \pi_1 M : \RT_1}
\]

Since
$\RT_1 \times \RT_2 \subtype \RT_1 \times \UPT{\RT_2}$.
\end{description}

\noindent
We similarly obtain
$\Env \thesis M : \UPT{\RT_1} \times \RT_2$.
Using subtyping, we then get
\[
\begin{array}{*{5}{l}}
  \Env
& \thesis
& M
& :
& \reft{\UPT{\RT_1} \times \UPT{\RT_2} \mid \form{\pi_1}\varphi_1}
\\

  \Env
& \thesis
& M
& :
& \reft{\UPT{\RT_1} \times \UPT{\RT_2} \mid  \form{\pi_2}\varphi_2}
\end{array}
\]

\noindent
from which we get
\[
\begin{array}{*{5}{l}}
  \Env
& \thesis
& M
& :
& \reft{\UPT{\RT_1} \times \UPT{\RT_2} \mid \pair{\varphi_1,\varphi_2}}
\end{array}
\]

\noindent
and thus
$\Env \thesis M : \RT_1 \times \RT_2$.
\qed
%%%%%%%%%%%%%%%%%%%%%%%%%%%%%%%%%%%%%%%%%%%%%%%%%%%%%%%%%%%%%%%%%%%%%%%%%%%
\end{proof}
%%%%%%%%%%%%%%%%%%%%%%%%%%%%%%%%%%%%%%%%%%%%%%%%%%%%%%%%%%%%%%%%%%%%%%%%%%%

%%%%%%%%%%%%%%%%%%%%%%%%%%%%%%%%%%%%%%%%%%%%%%%%%%%%%%%%%%%%%%%%%%%%%%%%%%%
\begin{lemma}
\label{lem:proof:main:eta:fun}
%%%%%%%%%%%%%%%%%%%%%%%%%%%%%%%%%%%%%%%%%%%%%%%%%%%%%%%%%%%%%%%%%%%%%%%%%%%
A (not necessarily normal) judgment
$\Env \thesis M : \RTbis \arrow \RT$
is sound (resp.\ derivable)
if, and only if,
so is
$\Env, x:\RTbis \thesis M x : \RT$.
%%%%%%%%%%%%%%%%%%%%%%%%%%%%%%%%%%%%%%%%%%%%%%%%%%%%%%%%%%%%%%%%%%%%%%%%%%%
\end{lemma}
%%%%%%%%%%%%%%%%%%%%%%%%%%%%%%%%%%%%%%%%%%%%%%%%%%%%%%%%%%%%%%%%%%%%%%%%%%%

%%%%%%%%%%%%%%%%%%%%%%%%%%%%%%%%%%%%%%%%%%%%%%%%%%%%%%%%%%%%%%%%%%%%%%%%%%%
\begin{proof}
%%%%%%%%%%%%%%%%%%%%%%%%%%%%%%%%%%%%%%%%%%%%%%%%%%%%%%%%%%%%%%%%%%%%%%%%%%%
By Lemma~\ref{lem:reft},
there are formulae
$\varphi \in \Lang(\UPT{\RT})$
and
$\psi \in \Lang(\UPT{\RTbis})$
such that
$\RT \eqtype \reft{\UPT{\RT} \mid \varphi}$
and
$\RTbis \eqtype \reft{\UPT{\RTbis} \mid \psi}$.
Hence
\(
  \RTbis \arrow \RT
  \eqtype
  \reft{\UPT{\RTbis} \arrow \UPT{\RT} \mid \psi \realto \varphi}
\).


It follows that
$\Env \thesis M : \RTbis \arrow \RT$
is sound if, and only if,
so is $\Env,x:\RTbis \thesis M x : \RT$.

It is clear that $\Env, x:\RTbis \thesis M x : \RT$
is derivable whenever so is
$\Env \thesis M : \RTbis \arrow \RT$.

For the converse, assume that
$\Env,x:\RTbis \thesis M x : \RT$
is derivable.
We show that
$\Env \thesis M : \RTbis \arrow \RT$
is derivable by induction on the derivation of
$\Env,x:\RTbis \thesis M x : \RT$.
We reason by cases on the last possible rule.
\begin{description}
\item[Case of]
\[
\dfrac{
  \begin{array}{l}
  \UPT\Env, x : \UPT\RTbis \thesis M x : \UPT{\RT}
  \\
  \text{for each $i \in I$,}\quad
  \Env,x:\RTbis \thesis M x : \reft{\UPT{\RT} \mid \varphi_i}
  \end{array}}
  {\Env,x:\RTbis \thesis M x : \reft{\UPT{\RT} \mid \bigwedge_{i \in I} \varphi_i}}
\]

\noindent
where $\varphi = \bigwedge_{i \in I}\varphi_i$.
Then by induction hypothesis and subtyping,
for all $i \in I$ we can derive
\[
\begin{array}{*{5}{l}}
  \Env
& \thesis
& M
& :
& \reft{\UPT{\RTbis} \arrow \UPT{\RT} \mid \psi \realto \varphi_i}
\end{array}
\]

\noindent
and thus
\[
\begin{array}{*{5}{l}}
  \Env
& \thesis
& M
& :
& \reft{\UPT{\RTbis} \arrow \UPT{\RT} \mid \bigwedge_{i \in I}(\psi \realto \varphi_i)}
\end{array}
\]

\noindent
We can then conclude by subtyping since
\[
\begin{array}{l !{\quad\thesis\quad} l}
  \bigwedge_{i \in I}(\psi \realto \varphi_i)
& \left( \psi \realto \bigwedge_{i \in I}\varphi_i \right)
\end{array}
\]

\item[Case of]
\[
\dfrac{
  \begin{array}{l}
  \UPT\Env, y:\PTbis, \UPT{\Env'},x : \UPT\RTbis \thesis M x : \UPT{\RT}
  \\
  \text{for each $i \in I$,}\quad
  \Env, y:\reft{\PTbis \mid \psi_i},\Env',x : \RTbis \thesis M x : \RT
  \end{array}}
  {\Env, y : \reft{\PTbis \mid \bigvee_{i \in I} \psi_i} , \Env', x : \RTbis
  \thesis M x : \RT}
\]

By induction hypothesis.

\item[Case of]
\[
\dfrac{
  \begin{array}{l}
  \UPT\Env, x:\UPT\RTbis \thesis M x  : \UPT{\RT}
  \\
  \text{for each $i \in I$,}\quad
  \Env, x:\reft{\UPT\RTbis \mid \psi_i} \thesis M x : \RT
  \end{array}}
  {\Env, x : \reft{\UPT\RTbis \mid \bigvee_{i \in I} \psi_i} \thesis M x : \RT}
\]

\noindent
where $\psi = \bigvee_{i \in I}\psi_i$.
By induction hypothesis and subtyping,
for all $i \in I$ we can derive
\[
\begin{array}{*{5}{l}}
  \Env
& \thesis
& M
& :
& \reft{\UPT{\RTbis} \arrow \UPT{\RT} \mid \psi_i \realto \varphi}
\end{array}
\]

\noindent
and thus
\[
\begin{array}{*{5}{l}}
  \Env
& \thesis
& M
& :
& \reft{\UPT{\RTbis} \arrow \UPT{\RT} \mid \bigwedge_{i \in I}(\psi_i \realto \varphi)}
\end{array}
\]

We can then conclude by subtyping since
\[
\begin{array}{l !{\quad\thesis\quad} l}
  \bigwedge_{i \in I}(\psi_i \realto \varphi)
& \left( \bigvee_{i \in I} \psi_i \right) \realto \varphi
\end{array}
\]

\item[Case of]
\[
\dfrac{
  \Env \subtype \Env'
  \quad
  \RTbis \subtype \RTbis'
  \quad 
  \RT' \subtype \RT
  \quad
  \Env', x:\RTbis' \thesis M x : \RT'}
  {\Env,x:\RTbis \thesis M x : \RT}
\]

By induction hypothesis we obtain
$\Env' \thesis M : \RTbis' \arrow \RT'$.
Then we are done since
$\RTbis' \arrow \RT' \subtype \RTbis \arrow \RT$
and $\Env \subtype \Env'$.

\item[Case of]
\[
\dfrac{\Env \thesis M : \RTbis \arrow \RT
  \qquad
  \Env, x: \RTbis \thesis x : \RTbis}
  {\Env, x: \RTbis \thesis M x : \RT}
\]

Trivial.
\qed
\end{description}
%%%%%%%%%%%%%%%%%%%%%%%%%%%%%%%%%%%%%%%%%%%%%%%%%%%%%%%%%%%%%%%%%%%%%%%%%%%
\end{proof}
%%%%%%%%%%%%%%%%%%%%%%%%%%%%%%%%%%%%%%%%%%%%%%%%%%%%%%%%%%%%%%%%%%%%%%%%%%%

We can now prove Proposition~\ref{prop:proof:main:eta}.

%%%%%%%%%%%%%%%%%%%%%%%%%%%%%%%%%%%%%%%%%%%%%%%%%%%%%%%%%%%%%%%%%%%%%%%%%%%
\begin{proof}[of Proposition~\ref{prop:proof:main:eta}]
%%%%%%%%%%%%%%%%%%%%%%%%%%%%%%%%%%%%%%%%%%%%%%%%%%%%%%%%%%%%%%%%%%%%%%%%%%%
We reason by induction on the fonf type $\RT$.
If $\RT$ is normal, then the result is trivial
since 
$\eta(\Env \thesis M : \RT) = \left\{ \Env \thesis M : \RT \right\}$.
In the cases of $\RT_1 \times \RT_2$
and $\RTbis \arrow \RT$ (with $\RTbis$ normal)
we conclude by induction hypothesis
and Lemmas~\ref{lem:proof:main:eta:prod} and \ref{lem:proof:main:eta:fun}, respectively.
\qed
%%%%%%%%%%%%%%%%%%%%%%%%%%%%%%%%%%%%%%%%%%%%%%%%%%%%%%%%%%%%%%%%%%%%%%%%%%%
\end{proof}
%%%%%%%%%%%%%%%%%%%%%%%%%%%%%%%%%%%%%%%%%%%%%%%%%%%%%%%%%%%%%%%%%%%%%%%%%%%





%%%%%%%%%%%%%%%%%%%%%%%%%%%%%%%%%%%%%%%%%%%%%%%%%%%%%%%%%%%%%%%%%%%%%%%%%%%
\subsection{\nameref{sec:compl:general}}
\label{sec:proof:compl:general}
%%%%%%%%%%%%%%%%%%%%%%%%%%%%%%%%%%%%%%%%%%%%%%%%%%%%%%%%%%%%%%%%%%%%%%%%%%%

We prove Lemma~\ref{lem:compl:nf:wf}.

%%%%%%%%%%%%%%%%%%%%%%%%%%%%%%%%%%%%%%%%%%%%%%%%%%%%%%%%%%%%%%%%%%%%%%%%%%%
\begin{lemma}[Lemma \ref{lem:compl:nf:wf}]
\label{lem:proof:compl:nf:wf}
%%%%%%%%%%%%%%%%%%%%%%%%%%%%%%%%%%%%%%%%%%%%%%%%%%%%%%%%%%%%%%%%%%%%%%%%%%%
For each $\varphi \in \Lang(\PT)$, there is a $\psi \in \Norm(\PT)$
such that $\varphi \thesisiff \psi$
in the extension of Figure~\ref{fig:log:ded} (\S\ref{sec:log})
with \eqref{eq:proof:compl:cc} and $\ax{WF}$.
%%%%%%%%%%%%%%%%%%%%%%%%%%%%%%%%%%%%%%%%%%%%%%%%%%%%%%%%%%%%%%%%%%%%%%%%%%%
\end{lemma}
%%%%%%%%%%%%%%%%%%%%%%%%%%%%%%%%%%%%%%%%%%%%%%%%%%%%%%%%%%%%%%%%%%%%%%%%%%%


%%%%%%%%%%%%%%%%%%%%%%%%%%%%%%%%%%%%%%%%%%%%%%%%%%%%%%%%%%%%%%%%%%%%%%%%%%%
\begin{proof}
%%%%%%%%%%%%%%%%%%%%%%%%%%%%%%%%%%%%%%%%%%%%%%%%%%%%%%%%%%%%%%%%%%%%%%%%%%%
The proof is by induction on $\varphi$.
In the case of $\bigwedge$ and $\bigvee$,
we conclude by induction hypothesis and Example~\ref{ex:log:distr}.
In the case of $\form\triangle\varphi$ ($\triangle$ either $\pi_1$, $\pi_2$ or $\fold$),
we conclude by induction hypothesis and Example~\ref{ex:log:modalnf}.

Consider now the case of $\psi \realto \varphi$.
By induction hypothesis we can assume $\varphi \in \Norm$.
By combining the induction hypothesis with 
Example~\ref{ex:log:distr},
we can assume that $\psi$
is a $\bigvee$ of $\bigwedge$'s of formulae in $\Lang_\land$.
Since
\[
\begin{array}{r !{\quad\thesisiff\quad} l}
  \bigwedge_{i \in I}\left(\psi \realto \varphi_i \right)
& \psi \realto \left(\bigwedge_{i \in I} \varphi_i\right)
\\

  \bigwedge_{i \in I}\left( \psi_i \realto \varphi \right)
& \left(\bigvee_{i \in I} \psi_i \right) \realto \varphi
\end{array}
\]

\noindent
we can reduce to the case of
$\psi = \bigwedge_{i \in I}\psi_i$
and
$\varphi = \bigvee_{k \in K}\varphi_k$
with $\varphi_k,\psi_i \in \Lang_\land$.

Now, note that we can derive
\[
\begin{array}{r !{\quad\thesisiff\quad} l}
  \left( \bigwedge_{i \in I} \psi_i \right)
  \realto
  \varphi
& \bigvee_{\text{$J \sle I$, $J$ finite}}
  \left(
  \left( \bigwedge_{j \in J} \psi_j \right)
  \realto
  \varphi
  \right)
\end{array}
\]

\noindent
Indeed, the $\thesis$ direction is given by the rule $\ax{WF}$.
The converse is derivable using the left-rule for $\bigvee$,
since $\bigwedge_{i \in I}\psi_i \thesis \bigwedge_{j \in J}\psi_j$
whenever $J \sle I$.

It follows that we can actually assume $\psi \in \Lang_\land$
(still with $\varphi = \bigvee_{k \in K}\varphi_k$
where $\varphi_k \in \Lang_\land$).
If $K \neq \emptyset$, then we can conclude using the rule
$\ax{F}$ in Figure~\ref{fig:log:ded}.

Otherwise, $K = \emptyset$ and $\varphi = \False$.
If $\C(\psi)$ then we conclude using the rule $\ax{C}$
in \eqref{eq:proof:compl:cc}.
Otherwise, by Proposition~\ref{prop:proof:compl:fin:ded}
we have $\psi \thesis \False$,
and we are done since
$\True \thesis (\False \realto \False)$
by Remark~\ref{rem:log:realto}.
\qed
%%%%%%%%%%%%%%%%%%%%%%%%%%%%%%%%%%%%%%%%%%%%%%%%%%%%%%%%%%%%%%%%%%%%%%%%%%%
\end{proof}
%%%%%%%%%%%%%%%%%%%%%%%%%%%%%%%%%%%%%%%%%%%%%%%%%%%%%%%%%%%%%%%%%%%%%%%%%%%


