%%%%%%%%%%%%%%%%%%%%%%%%%%%%%%%%%%%%%%%%%%%%%%%%%%%%%%%%%%%%%%%%%%%%%%%%%%%
\section{Semantics}
\label{sec:sem}
%%%%%%%%%%%%%%%%%%%%%%%%%%%%%%%%%%%%%%%%%%%%%%%%%%%%%%%%%%%%%%%%%%%%%%%%%%%


%%%%%%%%%%%%%%%%%%%%%%%%%%%%%%%%%%%%%%%%%%%%%%%%%%%%%%%%%%%%%%%%%%%%%%%%%%%
\subsubsection{Scott Domains.}
\label{sec:scott}
%%%%%%%%%%%%%%%%%%%%%%%%%%%%%%%%%%%%%%%%%%%%%%%%%%%%%%%%%%%%%%%%%%%%%%%%%%%
We shall interpret pure types as Scott domains and terms as Scott-continuous functions.
We mostly use the terminology of~\cite[\S 1]{ac98book}.
A \emph{dcpo} is a poset with all directed suprema.
A \emph{cpo} is dcpo with a least element (often denoted $\bot$).
A function between dcpos is \emph{Scott-continuous}
if it preserves the order (i.e.\ is monotone) as well as directed suprema.
A Scott-continuous function is \emph{strict} if it preserves
least elements.

%%%%%%%%%%%%%%%%%%%%%%%%%%%%%%%%%%%%%%%%%%%%%%%%%%%%%%%%%%%%%%%%%%%%%%%%%%%
\begin{definition}[Scott Domain]
\label{def:scott}
%%%%%%%%%%%%%%%%%%%%%%%%%%%%%%%%%%%%%%%%%%%%%%%%%%%%%%%%%%%%%%%%%%%%%%%%%%%
A \emph{Scott domain} is a bounded-complete algebraic cpo.
%
$\Scott$ is the category of Scott domains and Scott-continuous functions.
%%%%%%%%%%%%%%%%%%%%%%%%%%%%%%%%%%%%%%%%%%%%%%%%%%%%%%%%%%%%%%%%%%%%%%%%%%%
\end{definition}
%%%%%%%%%%%%%%%%%%%%%%%%%%%%%%%%%%%%%%%%%%%%%%%%%%%%%%%%%%%%%%%%%%%%%%%%%%%

Recall that a cpo $X$ is bounded-complete if
any two $x,y \in X$ have a sup (or \emph{least} upper bound)
$x \vee y \in X$ whenever they have an upper bound in $X$.

An element $d$ of a dcpo $X$ is \emph{finite}
if $d \in D$ for all directed $D \sle X$ such that
$d \leq \bigvee D$.%
\footnote{Finite elements are called \emph{compact}
in~\cite{ac98book}.}
Note that $\bot$ is always finite,
and that if $d,d' \in X$ are finite,
then $d \vee d'$ is finite whenever it exists.
%
The cpo $X$ is \emph{algebraic} if for each $x \in X$,
the set $\{ d\in X \mid \text{$d$ finite and $\leq x$} \}$
is directed and has sup $x$.

The category $\Scott$ is Cartesian-closed
(see e.g.\ \cite[Corollary 4.1.6]{aj95chapter}).

%%%%%%%%%%%%%%%%%%%%%%%%%%%%%%%%%%%%%%%%%%%%%%%%%%%%%%%%%%%%%%%%%%%%%%%%%%%
\subsubsection{Semantics of the Pure System.}
\label{sec:sem:pure}
%%%%%%%%%%%%%%%%%%%%%%%%%%%%%%%%%%%%%%%%%%%%%%%%%%%%%%%%%%%%%%%%%%%%%%%%%%%
Typed terms $\Env \thesis M : \PT$ of the pure system (\S\ref{sec:pure})
are interpreted as morphisms $\I M \colon \I\Env \to \I\PT$ in $\Scott$,
where $\I{\Env} = \prod_{i=1}^n\I{\PTbis_i}$ when
$\Env = x_1:\PTbis_1,\dots,x_n:\PTbis_n$.
This is well-known.


Base types $\BT \in \Base$ are interpreted as \emph{flat domains}
$\I\BT \deq \BT_\bot$,
where
$\BT_\bot$ is $\BT+\{\bot\}$ with $\BT$ discrete.
For each $a \in \BT$, we let $\I{a} \colon \one \to \I{\BT}$
be the constant map of value $a$.
The term %construct 
$\cse\ M\ \copair{a \mapsto N_a \mid a \in \BT}$
is interpreted using the strict Scott-continuous function
which takes $b \in \BT$ and $(y_a)_a \in X^{\BT}$ to $y_b$.


\begin{full}
Product types $\PT \times \PTbis$ are interpreted using
the Cartesian product of $\Scott$, i.e.\ the Cartesian product
of sets equipped with component-wise order.
%
Arrow types $\PTbis \to \PT$ are interpreted using the
closed structure of $\Scott$,
given by equipping each homset $\Scott(X,Y)$ with the pointwise order.
\end{full}

We refer to~\cite{ac98book,aj95chapter,streicher06book}
for the interpretation of recursive types $\rec\TV.\PT$.%
\opt{full}{\footnote{See Appendix~\ref{sec:proof:sem:pure} for details.}}


Term-level fixpoints $\fix x.M$ are interpreted
using the usual fixpoint combinators $\term Y \colon (X \to X) \to X$
taking $f \colon X \to X$ to $\term Y(f) \deq \bigvee_{n \in \NN} f^n(\bot)$.

%%%%%%%%%%%%%%%%%%%%%%%%%%%%%%%%%%%%%%%%%%%%%%%%%%%%%%%%%%%%%%%%%%%%%%%%%%%
\begin{example}
\label{ex:scott:stream-tree}
%%%%%%%%%%%%%%%%%%%%%%%%%%%%%%%%%%%%%%%%%%%%%%%%%%%%%%%%%%%%%%%%%%%%%%%%%%%
The domain $\I{\Stream\PTbis}$
of streams 
(resp.\ $\I{\Tree\PTbis}$ of trees)
is $\I\PTbis^K$ equipped with the pointwise order,
where $K = \NN$ (resp.\ $K = \two^*$).
The finite elements are those of finite support,
where the support of $z \in \I\PTbis^K$ is the set of all positions $p \in K$
such that $z(p) \neq \bot$.

Given $x \in \I{\Stream\PTbis}$,
we have $\I\hd(x) = x(0)$
while $\I\tl(x)$ is the stream taking $n \in \NN$ to $x(n+1) \in \I\PTbis$.
Moreover, $x = \I\Cons(\I\hd(x),\I\tl(x))$.%
\footnote{Note that $\I{\Stream\BT}$ differs from the usual \emph{Kahn domain}
$\BT^* \cup \BT^\omega$
(see e.g.~\cite[Definition 3.7.5 and Example 5.4.4]{vickers89book}
or~\cite[\S 7.4]{dst19book}, see also~\cite{vvk05concur}).}

Similarly, if $y \in \I{\Tree\PTbis}$
then $\I\lbl(y) = y(\es)$ is the root label of $y$,
while $\I\lft(y)$ and $\I\rght(y)$ are the left- and right-subtrees
of $y$, respectively.
%%%%%%%%%%%%%%%%%%%%%%%%%%%%%%%%%%%%%%%%%%%%%%%%%%%%%%%%%%%%%%%%%%%%%%%%%%%
\end{example}
%%%%%%%%%%%%%%%%%%%%%%%%%%%%%%%%%%%%%%%%%%%%%%%%%%%%%%%%%%%%%%%%%%%%%%%%%%%

%%%%%%%%%%%%%%%%%%%%%%%%%%%%%%%%%%%%%%%%%%%%%%%%%%%%%%%%%%%%%%%%%%%%%%%%%%%
\subsubsection{Scott Topology.}
\label{sec:top}
%%%%%%%%%%%%%%%%%%%%%%%%%%%%%%%%%%%%%%%%%%%%%%%%%%%%%%%%%%%%%%%%%%%%%%%%%%%
The semantics of refinement types involves some topology.
We refer to e.g.\ \cite[\S 1.2]{ac98book},
\cite[\S 2.3]{aj95chapter} or \cite[\S 7.1]{gg24book}.
See also~\cite{rs24jfla}.

Let $(X,\leq)$ be a dcpo.
%
A set $\SP \sle X$ is \emph{Scott-open}
if $\SP$ is upward-closed
(if $x \in \SP$ and $x \leq y$ in $X$, then $y \in \SP$),
and if moreover $\SP$ is inaccessible by directed sups,
in the sense that if $\bigvee D \in \SP$
with $D \sle X$ directed, then $D \cap \SP \neq \emptyset$.
This equips $X$ with a topology, called the \emph{Scott topology}.%
\footnote{Moreover, we have $x \leq y$ if, and only if,
$x \in \SP$ implies $y \in \SP$ for every Scott-open $\SP$.}
A function between dcpos is Scott-continuous
precisely when it is continuous for the Scott topology.

If $X$ is algebraic, then the Scott-opens are exactly the unions of sets of the form
$\up d = \{ x \in X \mid d \leq x\}$, with $d$ finite in $X$.
%
Note that $\up d$ is a compact subset of $X$ when $d$ is finite in $X$.
If $X$ is a Scott domain, then $\up d \cap \up d'$
is compact for all finite $d,d' \in X$
(by bounded-completeness,
if $\up d \cap \up d'$ is non-empty, then $d \vee d'$
is defined, finite and such that $\up (d \vee d') = \up d \cap \up d'$).%
\footnote{It is well-known that Scott domains are \emph{coherent} topological spaces
(see \cite[Proposition 4.2.17, \S 4.2.3]{aj95chapter},
and also \cite[Definition 5.2.21]{goubault13book}
and \cite[\S 2.3]{gg24book}).}

A set $\SP \sle X$ is \emph{saturated}
if $\SP$ is upward-closed,
or equivalently if $\SP$ is an intersection of Scott-open sets
(see e.g.\ \cite[Proposition 4.2.9]{goubault13book}).


%%%%%%%%%%%%%%%%%%%%%%%%%%%%%%%%%%%%%%%%%%%%%%%%%%%%%%%%%%%%%%%%%%%%%%%%%%%
\subsubsection{Semantics of Formulae.}
\label{sec:sem:log}
%%%%%%%%%%%%%%%%%%%%%%%%%%%%%%%%%%%%%%%%%%%%%%%%%%%%%%%%%%%%%%%%%%%%%%%%%%%
For each $\varphi \in \Lang(\PT)$ we define a set $\I\varphi \sle \I\PT$
using the following \emph{semantic modalities}:
$\I{\form a} \deq \{a\} \sle \I\BT$
for $\BT \in \Base$ and $a \in \BT$,
and
\begin{equation*}
\begin{array}{r !{~}c!{~} l l l}
  \SP \in \Po(\I{\PT_i})
& \longmapsto
& \I{\form{\pi_i}}(\SP)
& \deq
& \left\{
  x \in \I{\PT_1 \times \PT_2}
  \mid
  \pi_i(x) \in \SP
  \right\}
\\

  \SP \in \Po(\I{\PT[\rec\TV.\PT/\TV]})
& \longmapsto
& \I{\form\fold}(\SP)
& \deq
& \left\{
  x \in \I{\rec\TV.\PT}
  \mid
  \I\unfold(x) \in \SP
  \right\}
\\

  \SP \in \Po(\I\PTbis)
  \,,\,
  \SPbis \in \Po(\I\PT)
& \longmapsto
& (\SP \realto \SPbis)
& \deq
& \left\{
  f \in \I{\PTbis \arrow \PT}
  \mid
  \forall x \in \SP,~ f(x) \in \SPbis
  \right\}
\end{array}
\end{equation*}


\noindent
We let
$\I{\form{\pi_i}\varphi} \deq \I{\form{\pi_i}}(\I\varphi)$,
$\I{\form{\fold}\varphi} \deq \I{\form{\fold}}(\I\varphi)$,
and
$\I{\psi \realto \varphi} \deq \I\psi \realto \I\varphi$.
Conjunctions and disjunctions are interpreted as intersections
and unions. 

%%%%%%%%%%%%%%%%%%%%%%%%%%%%%%%%%%%%%%%%%%%%%%%%%%%%%%%%%%%%%%%%%%%%%%%%%%%
\begin{example}
\label{ex:sem:modalmu}
%%%%%%%%%%%%%%%%%%%%%%%%%%%%%%%%%%%%%%%%%%%%%%%%%%%%%%%%%%%%%%%%%%%%%%%%%%%
Assume given \emph{propositional variables}
$p^\PT,\dots$ for each pure type $\PT$.
If a formula $\varphi(p^\PT)$ of type $\PT$ is positive in $p^\PT$,
then it induces a monotone function on $(\Po(\I\PT),\sle)$
with least and greatest fixpoints
\(
  \I{\mu p.\varphi}
  =
  \llbracket \bigvee_{\alpha \leq |\Po(\I\PT)|} \varphi^\alpha(\False) \rrbracket
\)
and
\(
  \I{\nu p.\varphi}
  =
  \llbracket\bigwedge_{\alpha \leq |\Po(\I\PT)|} \varphi^\alpha(\True) \rrbracket
\)
\cite[\S 20]{gtw02alig}.
This generalizes Examples~\ref{ex:form:stream}, \ref{ex:form:tree}.
%%%%%%%%%%%%%%%%%%%%%%%%%%%%%%%%%%%%%%%%%%%%%%%%%%%%%%%%%%%%%%%%%%%%%%%%%%%
\end{example}
%%%%%%%%%%%%%%%%%%%%%%%%%%%%%%%%%%%%%%%%%%%%%%%%%%%%%%%%%%%%%%%%%%%%%%%%%%%


Lemma~\ref{lem:top:char:fin} and~\ref{lem:top:char}
below are semantic characterizations of 
the classes of formulae in Definition~\ref{def:form} (\S\ref{sec:log}).
This yields the soundness of the rule $\ax{F}$
in Figure~\ref{fig:log:ded} (\S\ref{sec:log}).

%%%%%%%%%%%%%%%%%%%%%%%%%%%%%%%%%%%%%%%%%%%%%%%%%%%%%%%%%%%%%%%%%%%%%%%%%%%
\begin{lemma}
\label{lem:top:char:fin}
%%%%%%%%%%%%%%%%%%%%%%%%%%%%%%%%%%%%%%%%%%%%%%%%%%%%%%%%%%%%%%%%%%%%%%%%%%%
Given $\varphi \in \Lang_\land(\PT)$,
if $\I\varphi \neq \emptyset$ then
$\I\varphi = \up d$ for some finite $d \in \I\PT$.
Conversely, if $d \in \I\PT$ is finite, then $\up d = \I\varphi$ for some
$\varphi \in \Lang_{\land}(\PT)$.
%%%%%%%%%%%%%%%%%%%%%%%%%%%%%%%%%%%%%%%%%%%%%%%%%%%%%%%%%%%%%%%%%%%%%%%%%%%
\end{lemma}
%%%%%%%%%%%%%%%%%%%%%%%%%%%%%%%%%%%%%%%%%%%%%%%%%%%%%%%%%%%%%%%%%%%%%%%%%%%



%%%%%%%%%%%%%%%%%%%%%%%%%%%%%%%%%%%%%%%%%%%%%%%%%%%%%%%%%%%%%%%%%%%%%%%%%%%
\begin{lemma}
\label{lem:top:char}
%%%%%%%%%%%%%%%%%%%%%%%%%%%%%%%%%%%%%%%%%%%%%%%%%%%%%%%%%%%%%%%%%%%%%%%%%%%
A set $\SP \sle \I\PT$ is saturated (resp.\ Scott-open)
if, and only if, there is a formula $\varphi \in \Lang(\PT)$
(resp.\@ $\varphi \in \Lang_\Open(\PT)$)
such that $\SP = \I\varphi$.

In particular, for each $\varphi \in \Lang(\PT)$
we have $\I\varphi = \I\psi$ for some $\psi \in \Norm(\PT)$.
%%%%%%%%%%%%%%%%%%%%%%%%%%%%%%%%%%%%%%%%%%%%%%%%%%%%%%%%%%%%%%%%%%%%%%%%%%%
\end{lemma}
%%%%%%%%%%%%%%%%%%%%%%%%%%%%%%%%%%%%%%%%%%%%%%%%%%%%%%%%%%%%%%%%%%%%%%%%%%%


%%%%%%%%%%%%%%%%%%%%%%%%%%%%%%%%%%%%%%%%%%%%%%%%%%%%%%%%%%%%%%%%%%%%%%%%%%%
\begin{proposition}[Soundness of Deduction]
\label{prop:sem:sound:ded}
%%%%%%%%%%%%%%%%%%%%%%%%%%%%%%%%%%%%%%%%%%%%%%%%%%%%%%%%%%%%%%%%%%%%%%%%%%%
If $\psi \thesis \varphi$ is derivable in the basic deduction system 
in Figure~\ref{fig:log:ded} (\S\ref{sec:log}),
then $\I\psi \sle \I\varphi$.
%%%%%%%%%%%%%%%%%%%%%%%%%%%%%%%%%%%%%%%%%%%%%%%%%%%%%%%%%%%%%%%%%%%%%%%%%%%
\end{proposition}
%%%%%%%%%%%%%%%%%%%%%%%%%%%%%%%%%%%%%%%%%%%%%%%%%%%%%%%%%%%%%%%%%%%%%%%%%%%

%%%%%%%%%%%%%%%%%%%%%%%%%%%%%%%%%%%%%%%%%%%%%%%%%%%%%%%%%%%%%%%%%%%%%%%%%%%
\begin{proof}
%%%%%%%%%%%%%%%%%%%%%%%%%%%%%%%%%%%%%%%%%%%%%%%%%%%%%%%%%%%%%%%%%%%%%%%%%%%
We only detail the case of $\ax{F}$.
If $\I\psi = \emptyset$, then for all $\SP \sle \I\PT$
we have $\I\psi \realto \SP = \I{\PTbis \arrow \PT}$,
and we are done since $I$ is assumed to be non-empty.

Otherwise, we have $\I\psi = \up d$ by Lemma~\ref{lem:top:char:fin}.
Let $f \in \I{\PTbis \arrow \PT}$.
If $\up d$ is included in
$f^{-1}(\I{\bigvee_i \varphi_i}) = \bigcup_{i} f^{-1}(\I{\varphi_i})$,
then $d \in f^{-1}(\I{\varphi_i})$ for some $i$.
Hence $\up d \sle f^{-1}(\I{\varphi_i})$
as $f^{-1}(\I{\varphi_i})$ is saturated
(since $f$ is monotone and since $\I{\varphi_i}$ is saturated
by Lemma~\ref{lem:top:char}).
\qed
%%%%%%%%%%%%%%%%%%%%%%%%%%%%%%%%%%%%%%%%%%%%%%%%%%%%%%%%%%%%%%%%%%%%%%%%%%%
\end{proof}
%%%%%%%%%%%%%%%%%%%%%%%%%%%%%%%%%%%%%%%%%%%%%%%%%%%%%%%%%%%%%%%%%%%%%%%%%%%

%%%%%%%%%%%%%%%%%%%%%%%%%%%%%%%%%%%%%%%%%%%%%%%%%%%%%%%%%%%%%%%%%%%%%%%%%%%
\subsubsection{Semantics of Refinement Types.}
\label{sec:sem:reft}
%%%%%%%%%%%%%%%%%%%%%%%%%%%%%%%%%%%%%%%%%%%%%%%%%%%%%%%%%%%%%%%%%%%%%%%%%%%
The interpretation $\I\RT \sle \I{\UPT\RT}$
of a type $\RT$ is defined as
$\I{\reft{\PT \mid \varphi}} \deq \I\varphi$,
$\I{\RT \times \RTbis} \deq \I\RT \times \I\RTbis$
and
$\I{\RTbis \arrow \RT} \deq \I\RTbis \realto \I\RT$.

%%%%%%%%%%%%%%%%%%%%%%%%%%%%%%%%%%%%%%%%%%%%%%%%%%%%%%%%%%%%%%%%%%%%%%%%%%%
\begin{definition}[Sound Typing Judgement]
\label{def:sound:typing}
%%%%%%%%%%%%%%%%%%%%%%%%%%%%%%%%%%%%%%%%%%%%%%%%%%%%%%%%%%%%%%%%%%%%%%%%%%%
A judgment $\Env \thesis M : \RT$ with
$\Env = x_1:\RTbis_1,\dots,x_n:\RTbis_n$
is \emph{sound} %(notation $\Env \models M : \RT$)
if $\UPT\Env \thesis M :\UPT\RT$ is derivable and
if moreover $\I M(u_1,\dots,u_n) \in \I\RT$
whenever $u_i \in \I{\RTbis_i}$ for all $i =1,\dots,n$.
%%%%%%%%%%%%%%%%%%%%%%%%%%%%%%%%%%%%%%%%%%%%%%%%%%%%%%%%%%%%%%%%%%%%%%%%%%%
\end{definition}
%%%%%%%%%%%%%%%%%%%%%%%%%%%%%%%%%%%%%%%%%%%%%%%%%%%%%%%%%%%%%%%%%%%%%%%%%%%

\noindent
The judgments in
Tab.~\ref{tab:reft} (Ex.~\ref{ex:reft:fun}) are sound.
Also, derivable judgments are sound.

%%%%%%%%%%%%%%%%%%%%%%%%%%%%%%%%%%%%%%%%%%%%%%%%%%%%%%%%%%%%%%%%%%%%%%%%%%%
\begin{theorem}[Soundness of Typing]
\label{thm:sem:sound:reft}
%%%%%%%%%%%%%%%%%%%%%%%%%%%%%%%%%%%%%%%%%%%%%%%%%%%%%%%%%%%%%%%%%%%%%%%%%%%
If $\Env \thesis M : \RT$ is derivable in the system of~\S\ref{sec:reft},
then $\Env \thesis M : \RT$ is sound.
%%%%%%%%%%%%%%%%%%%%%%%%%%%%%%%%%%%%%%%%%%%%%%%%%%%%%%%%%%%%%%%%%%%%%%%%%%%
\end{theorem}
%%%%%%%%%%%%%%%%%%%%%%%%%%%%%%%%%%%%%%%%%%%%%%%%%%%%%%%%%%%%%%%%%%%%%%%%%%%






