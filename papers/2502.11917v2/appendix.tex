%%%%%%%%%%%%%%%%%%%%%%%%%%%%%%%%%%%%%%%%%%%%%%%%%%%%%%%%%%%%%%%%%%%%%%%%%%%
\section{Additional Material}
\label{sec:app}
%%%%%%%%%%%%%%%%%%%%%%%%%%%%%%%%%%%%%%%%%%%%%%%%%%%%%%%%%%%%%%%%%%%%%%%%%%%

%%%%%%%%%%%%%%%%%%%%%%%%%%%%%%%%%%%%%%%%%%%%%%%%%%%%%%%%%%%%%%%%%%%%%%%%%%%
\begin{figure}[t!]
%%%%%%%%%%%%%%%%%%%%%%%%%%%%%%%%%%%%%%%%%%%%%%%%%%%%%%%%%%%%%%%%%%%%%%%%%%%
\[
\begin{array}{c}
\dfrac{(x:\PT) \in \Env}
  {\Env \thesis x:\PT}

\qquad\quad

\dfrac{\Env,x:\PTbis \thesis M : \PT}
  {\Env \thesis \lambda x.M : \PTbis \arrow \PT}

\qquad\quad

\dfrac{\Env \thesis M : \PTbis \arrow \PT
  \qquad
  \Env \thesis N : \PTbis}
  {\Env \thesis M N : \PT}

\\\\

\dfrac{\Env \thesis M : \PT
  \qquad
  \Env \thesis N : \PTbis}
  {\Env \thesis \pair{M,N} : \PT \times \PTbis}

\qquad\quad

\dfrac{\Env \thesis M : \PT \times \PTbis}
  {\Env \thesis \pi_1(M) : \PT}

\qquad\quad

\dfrac{\Env \thesis M : \PT \times \PTbis}
  {\Env \thesis \pi_2(M) : \PTbis}

\\\\

\dfrac{\Env,x:\PT \thesis M : \PT}
  {\Env \thesis \fix x.M : \PT}

\qquad\quad

\dfrac{\Env \thesis M : \PT[\rec\TV.\PT/\TV]}
  {\Env \thesis \fold(M) : \rec\TV.\PT}

\qquad\quad

\dfrac{\Env \thesis M : \rec\TV.\PT}
  {\Env \thesis \unfold(M) : \PT[\rec\TV.\PT/\TV]}

\\\\

\dfrac{}
  {\Env \thesis a : \BT}
~\text{($\BT \in \Base$ and $a \in \BT$)}

\qquad\quad

\dfrac{ \Env \thesis M : \BT
  \qquad\text{for each $a \in \BT$,\quad} \Env \thesis N_a : \PT}
  {\Env \thesis \cse\ M\ \copair{a \mapsto N_a \mid a \in \BT} : \PT}

\end{array}
\]
\caption{Typing Rules of the Pure Calculus.%
\label{fig:app:puretyping}}
%%%%%%%%%%%%%%%%%%%%%%%%%%%%%%%%%%%%%%%%%%%%%%%%%%%%%%%%%%%%%%%%%%%%%%%%%%%
\end{figure}
%%%%%%%%%%%%%%%%%%%%%%%%%%%%%%%%%%%%%%%%%%%%%%%%%%%%%%%%%%%%%%%%%%%%%%%%%%%

\noindent
Figure~\ref{fig:app:puretyping} gathers all rules of the pure calculus.

%%%%%%%%%%%%%%%%%%%%%%%%%%%%%%%%%%%%%%%%%%%%%%%%%%%%%%%%%%%%%%%%%%%%%%%%%%%
\subsection{\nameref{sec:wf}}
\label{sec:app:wf}
%%%%%%%%%%%%%%%%%%%%%%%%%%%%%%%%%%%%%%%%%%%%%%%%%%%%%%%%%%%%%%%%%%%%%%%%%%%

A give a proof of Lemma~\ref{lem:wf} 
in order to illustrate Proposition~\ref{prop:wf} in a simple case.



%%%%%%%%%%%%%%%%%%%%%%%%%%%%%%%%%%%%%%%%%%%%%%%%%%%%%%%%%%%%%%%%%%%%%%%%%%%
\begin{lemma}[Lemma \ref{lem:wf}]
\label{lem:app:wf}
%%%%%%%%%%%%%%%%%%%%%%%%%%%%%%%%%%%%%%%%%%%%%%%%%%%%%%%%%%%%%%%%%%%%%%%%%%%
The rule $\ax{WF}$ is sound.
%%%%%%%%%%%%%%%%%%%%%%%%%%%%%%%%%%%%%%%%%%%%%%%%%%%%%%%%%%%%%%%%%%%%%%%%%%%
\end{lemma}
%%%%%%%%%%%%%%%%%%%%%%%%%%%%%%%%%%%%%%%%%%%%%%%%%%%%%%%%%%%%%%%%%%%%%%%%%%%

%%%%%%%%%%%%%%%%%%%%%%%%%%%%%%%%%%%%%%%%%%%%%%%%%%%%%%%%%%%%%%%%%%%%%%%%%%%
\begin{proof}
%%%%%%%%%%%%%%%%%%%%%%%%%%%%%%%%%%%%%%%%%%%%%%%%%%%%%%%%%%%%%%%%%%%%%%%%%%%
\begin{full}
Recall that the rule $\ax{WF}$ is
\[
\dfrac{\text{for all $i\in I$, $\psi_i \in \Lang_\land(\PTbis)$}
  \qquad
  \varphi \in \Lang_\Open(\PT)}
  {
  \left( \bigwedge_{i \in I} \psi_i \right)
  \realto
  \varphi
  \,\thesis\,
  \bigvee_{\text{$J \sle I$, $J$ finite}}
  \left(
  \left( \bigwedge_{j \in J} \psi_j \right)
  \realto
  \varphi
  \right)
  }
\]
\end{full}

We shall apply Proposition~\ref{prop:wf} to the Scott domain $\SI\PTbis$.
%
Let $f \in \SI{\PTbis \arrow \PT}$
such that $\SI{\bigwedge_{i \in I} \psi_i} \sle f^{-1}(\SI\varphi)$.
%
Note that $f^{-1}(\SI\varphi)$ is Scott-open since $f$ is Scott-continuous
while $\SI\varphi$ is Scott-open by Lemma~\ref{lem:top:char}.

Let $\Filt$ be the set of all
$\SI{\bigwedge_{j \in J} \psi_j}$, where $J$ ranges over all finite subsets of $I$.
We check the assumptions of Proposition~\ref{prop:wf}.
\begin{itemize}
\item
First, $\Filt$ is filtering since it is non-empty
(as $\emptyset$ is a finite subset of $I$)
and since given 
$\SI{\bigwedge_{j \in J} \psi_j}$
and
$\SI{\bigwedge_{k \in K} \psi_k}$
in $\Filt$,
we have
$\SI{\bigwedge_{\ell \in J \cup K} \psi_\ell} \in \Filt$
with
\(
  \SI{\bigwedge_{\ell \in J \cup K} \psi_\ell}
  \sle
  \SI{\bigwedge_{j \in J} \psi_j}
  ,
  \SI{\bigwedge_{k \in K} \psi_k}
\).

\item
Second, it follows from 
Lemmas~\ref{lem:top:char:fin} and \ref{lem:top:char}
that $\Filt$ consists of compacts saturated subsets of $\SI\PTbis$.

\item
Third, we have
\[
\begin{array}{l l l}
  \I{\bigwedge_{i \in I} \psi_i}
& =
& \bigcap_{i \in I} \I{\psi_i}
\\

& =
& \bigcap_{J \sle_\fin I} \bigcap_{j \in J} \I{\psi_j}
\\

& =
& \bigcap_{J \sle_\fin I} \I{\bigwedge_{j \in J} \psi_j}
\\

& =
& \bigcap \Filt
\end{array}
\]
\end{itemize}

\noindent
Now we are done since by Proposition~\ref{prop:wf} there is some
$\SI{\bigwedge_{j \in J} \psi_j} \in \Filt$
such that 
$\SI{\bigwedge_{j \in J} \psi_j} \sle f^{-1}(\I\varphi)$.
\qed
%%%%%%%%%%%%%%%%%%%%%%%%%%%%%%%%%%%%%%%%%%%%%%%%%%%%%%%%%%%%%%%%%%%%%%%%%%%
\end{proof}
%%%%%%%%%%%%%%%%%%%%%%%%%%%%%%%%%%%%%%%%%%%%%%%%%%%%%%%%%%%%%%%%%%%%%%%%%%%


%%%%%%%%%%%%%%%%%%%%%%%%%%%%%%%%%%%%%%%%%%%%%%%%%%%%%%%%%%%%%%%%%%%%%%%%%%%
\subsection{\nameref{sec:main}}
\label{sec:app:main}
%%%%%%%%%%%%%%%%%%%%%%%%%%%%%%%%%%%%%%%%%%%%%%%%%%%%%%%%%%%%%%%%%%%%%%%%%%%

We prove Lemma~\ref{lem:compl:nf} and our Main Theorem~\ref{thm:main}.


%%%%%%%%%%%%%%%%%%%%%%%%%%%%%%%%%%%%%%%%%%%%%%%%%%%%%%%%%%%%%%%%%%%%%%%%%%%
\begin{lemma}[Lemma \ref{lem:compl:nf}]
\label{lem:app:compl:nf}
%%%%%%%%%%%%%%%%%%%%%%%%%%%%%%%%%%%%%%%%%%%%%%%%%%%%%%%%%%%%%%%%%%%%%%%%%%%
Given $\varphi,\psi \in \Norm(\PT)$,
if $\I\psi \sle \I\varphi$,
then $\psi \thesis_\PT \varphi$
is derivable in the extension of Figure~\ref{fig:log:ded} (\S\ref{sec:log})
with \eqref{eq:compl:cc}.
%%%%%%%%%%%%%%%%%%%%%%%%%%%%%%%%%%%%%%%%%%%%%%%%%%%%%%%%%%%%%%%%%%%%%%%%%%%
\end{lemma}
%%%%%%%%%%%%%%%%%%%%%%%%%%%%%%%%%%%%%%%%%%%%%%%%%%%%%%%%%%%%%%%%%%%%%%%%%%%

%%%%%%%%%%%%%%%%%%%%%%%%%%%%%%%%%%%%%%%%%%%%%%%%%%%%%%%%%%%%%%%%%%%%%%%%%%%
\begin{proof}
%%%%%%%%%%%%%%%%%%%%%%%%%%%%%%%%%%%%%%%%%%%%%%%%%%%%%%%%%%%%%%%%%%%%%%%%%%%
The general strategy is to reduce to the finite case
(Proposition~\ref{prop:compl:fin:ded}),
by using Proposition~\ref{prop:wf}, but without using the rule $\ax{WF}$.

Let $\varphi,\psi \in \Norm(\PT)$ such that $\I\psi \sle \I\varphi$.
Since $\varphi \in \Norm(\PT)$, we have
$\varphi = \bigwedge_{k \in K} \varphi_k$
with $\varphi_k \in \Lang_\Open(\PT)$.
Hence, for each $k \in K$ we have
$\I\psi \sle \I{\varphi_k}$.
Thanks to the right-rule for $\bigwedge$ in Figure~\ref{fig:log:ded},
we can therefore reduce to the case of $\varphi \in \Lang_{\Open}(\PT)$.

We now assume $\varphi \in \Lang_\Open(\PT)$,
with $\varphi = \bigvee_{k \in K} \varphi_k$
and $\varphi_k \in \Lang_\land(\PT)$.
Since $\psi \in \Norm(\PT)$, using Example~\ref{ex:log:distr}
we can actually put $\psi$ in $\bigvee\bigwedge$-form:
we have
$\psi \thesisiff \bigvee_{i \in I} \bigwedge_{j \in J_i} \psi_{i,j}$
with $\psi_{i,j} \in \Lang_\land(\PT)$.
If $\I\psi \sle \I\varphi$,
then for all $i \in I$ we have
$\I{\psi_i} \sle \I\varphi$.
%
Thanks to the left-rule for $\bigvee$ in Figure~\ref{fig:log:ded},
we can therefore reduce to the case where $\psi$ is
of the form $\bigwedge_{i \in I} \psi_i$ with $\psi_i \in \Lang_\land(\PT)$.

Assume $\SI{\bigwedge_{i \in I} \psi_i} \sle \I\varphi$
with $\psi_i \in \Lang_\land(\PT)$,
and with $\varphi \in \Lang_\Open(\PT)$ as above.
We use Proposition~\ref{prop:wf}.
Similarly as in the proof of Lemma~\ref{lem:app:wf},
let $\Filt$ be the set of all
$\SI{\bigwedge_{j \in J} \psi_j}$, where $J$ ranges over all finite subsets of $I$.
The assumptions of Proposition~\ref{prop:wf} are checked similarly as in the proof of
Lemma~\ref{lem:app:wf}.
Again similarly as in the proof of Lemma~\ref{lem:app:wf},
there is some finite $J \sle I$ such that
$\SI{\bigwedge_{j \in J} \psi_j} \sle \I\varphi$.

Since $\bigwedge_{i \in I}\psi_i \thesis \bigwedge_{j \in J}\psi_j$,
we are done if we show that 
$\bigwedge_{j \in J}\psi_i \thesis \varphi$
is derivable.
Note that $\bigwedge_{j \in J}\psi_j \in \Lang_\land(\PT)$
since $J$ is finite.

Assume that $\SI{\bigwedge_{j \in J}\psi_j} = \emptyset$.
By Proposition~\ref{prop:compl:fin:ded}
we have $\bigwedge_{j \in J}\psi_j \thesis \False$,
from which we get
$\bigwedge_{j \in J}\psi_i \thesis \varphi$.

Otherwise, by Lemma~\ref{lem:top:char:fin}
there is some finite $d \in \I\PT$ such that
$\up d = \SI{\bigwedge_{j \in J}\psi_j}$.
Hence $d \in \I\varphi$,
and there is some $k \in K$ such that $d \in \I{\varphi_k}$.
But this implies
$\SI{\bigwedge_{j \in J}\psi_j} \sle \I{\varphi_k}$,
and
$\bigwedge_{j \in J}\psi_j \thesis \varphi_k$
is derivable by
Proposition~\ref{prop:compl:fin:ded}.
We then obtain
$\bigwedge_{j \in J}\psi_i \thesis \varphi$
using the right-rule for $\bigvee$.
\qed
%%%%%%%%%%%%%%%%%%%%%%%%%%%%%%%%%%%%%%%%%%%%%%%%%%%%%%%%%%%%%%%%%%%%%%%%%%%
\end{proof}
%%%%%%%%%%%%%%%%%%%%%%%%%%%%%%%%%%%%%%%%%%%%%%%%%%%%%%%%%%%%%%%%%%%%%%%%%%%

We now turn to our main result (Theorem~\ref{thm:main}).
Given a typing context $\Env = x_1:\RTbis_1,\dots,x_n:\RTbis_n$,
we write $\I\Env$ for $\I{\RTbis_1} \times \dots \times \I{\RTbis_n}$.

%%%%%%%%%%%%%%%%%%%%%%%%%%%%%%%%%%%%%%%%%%%%%%%%%%%%%%%%%%%%%%%%%%%%%%%%%%%
\begin{theorem}[Theorem \ref{thm:main}]
\label{thm:app:main}
%%%%%%%%%%%%%%%%%%%%%%%%%%%%%%%%%%%%%%%%%%%%%%%%%%%%%%%%%%%%%%%%%%%%%%%%%%%
If $\Env \thesis M : \RT$ is sound and normal
then $\Env \thesis M : \RT$ is derivable in the system of \S\ref{sec:reft}
extended with \eqref{eq:compl:cc}.
%%%%%%%%%%%%%%%%%%%%%%%%%%%%%%%%%%%%%%%%%%%%%%%%%%%%%%%%%%%%%%%%%%%%%%%%%%%
\end{theorem}
%%%%%%%%%%%%%%%%%%%%%%%%%%%%%%%%%%%%%%%%%%%%%%%%%%%%%%%%%%%%%%%%%%%%%%%%%%%

%%%%%%%%%%%%%%%%%%%%%%%%%%%%%%%%%%%%%%%%%%%%%%%%%%%%%%%%%%%%%%%%%%%%%%%%%%%
\begin{proof}
%%%%%%%%%%%%%%%%%%%%%%%%%%%%%%%%%%%%%%%%%%%%%%%%%%%%%%%%%%%%%%%%%%%%%%%%%%%
Thanks to Proposition~\ref{prop:main:eta},
we only have to consider the case of a normal judgment $\Env \thesis M : \RT$
in which the type $\RT$ is normal.
The general idea of the proof is somehow similar to that of
Lemma~\ref{lem:app:compl:nf}:
we reduce to the finite case (Theorem~\ref{thm:compl:fin}),
by using Proposition~\ref{prop:wf}, but without using the rule $\ax{WF}$.

Since $\RT$ is normal, it is of the form $\reft{\PT \mid \varphi}$
with $\varphi \in \Norm(\PT)$.
Similarly as in Lemma~\ref{lem:app:compl:nf}
(but using the right-rule for $\bigwedge$ in Figure~\ref{fig:reft:reftyping}),
we can reduce to the case of $\varphi \in \Lang_\Open(\PT)$,
with $\varphi$ of the form $\bigvee_{k \in K} \varphi_k$,
where $\varphi_k \in \Lang_\land(\PT)$.

Assume $\Env = \Env', x : \RTbis$.
Since $\RTbis$ is normal, it is of the form
$\reft{\PTbis \mid \psi}$, with $\psi \in \Norm(\PTbis)$.
Again similarly as in Lemma~\ref{lem:app:compl:nf},
using Example~\ref{ex:log:distr}
we can actually put $\psi$ in $\bigvee\bigwedge$-form:
we have
$\psi \thesisiff \bigvee_{i \in I} \bigwedge_{j \in J_i} \psi_{i,j}$
with $\psi_{i,j} \in \Lang_\land(\PTbis)$.
For each $i \in I$,
the judgment
$\Env', x : \reft{\PTbis \mid \psi_i} \thesis M : \reft{\PT \mid \varphi}$
is sound (since so is $\Env \thesis M : \RT$).
Hence, using the left-rule for $\bigvee$
in Figure~\ref{fig:reft:reftyping},
we can reduce to the case where $\psi$ is a $\bigwedge$
of formulae in $\Lang_\land(\PTbis)$.

Repeating the above for each declaration $(x:\RTbis) \in \Env$,
we can assume that $\Env$ is of the form
$x_1:\RTbis_1,\dots,x_n:\RTbis_n$,
where $\RTbis_i = \reft{\PTbis_i \mid \psi_i}$
with $\psi_i = \bigwedge_{j \in J_i} \psi_{i,j}$
and $\psi_{i,j} \in \Lang_\land(\PTbis_i)$.

We shall now apply Proposition~\ref{prop:wf}
to the Scott domain $\I{\UPT\Env}$.
Note that $\I M$ is a Scott-continuous function $\I{\UPT\Env} \to \I\PT$,
so that $\SP \deq \I M^{-1}(\I\varphi)$
is open in $\I{\UPT\Env}$ by Lemma~\ref{lem:top:char}.
Let $\Filt$ consist of all the
\[
\begin{array}{l l l}
  \I{\bigwedge_{\ell \in L_1} \psi_{1,\ell}}
& \times \dots \times
& \I{\bigwedge_{\ell \in L_n} \psi_{n,\ell}}
\end{array}
\]

\noindent
where $L_1,\dots,L_n$ range over all finite subsets
of $J_1,\dots,J_n$, respectively.
It is easy to see
that $\Filt$ and $\SP$ meet the assumptions of Proposition~\ref{prop:wf},
namely that $\Filt$ is a filtering family of compact saturated
subsets of $\I{\UPT\Env}$ such that $\bigcap \Filt \sle \SP$.
\begin{full}
\begin{description} 
\item[$\Filt$ is filtering.]
Indeed, $\Filt$ is non-empty.
Moreover, given
\[
\begin{array}{l !{\quad\text{and}\quad} l}
  \I{\bigwedge_{\ell \in L_1} \psi_{1,\ell}}
  \times \dots \times
  \I{\bigwedge_{\ell \in L_n} \psi_{n,\ell}}
%\quad\text{and}\quad

& \I{\bigwedge_{\ell \in L'_1} \psi_{1,\ell}}
  \times \dots \times
  \I{\bigwedge_{\ell \in L'_n} \psi_{n,\ell}}
\end{array}
\]

\noindent
in $\Filt$,
we have
\[
\begin{array}{l l l}
  \I{\bigwedge_{\ell \in L_1 \cup L'_1} \psi_{1,\ell}}
  \times \dots \times
  \I{\bigwedge_{\ell \in L_n \cup L'_n} \psi_{n,\ell}}
& \in
& \Filt
\end{array}
\]

\noindent
with
\[
\begin{array}{l l l}
  \I{\bigwedge_{\ell \in L_i \cup L'_i} \psi_{i,\ell}}
& \sle
& \I{\bigwedge_{\ell \in L_i} \psi_{i,\ell}}
  \,,\, 
  \I{\bigwedge_{\ell \in L'_i} \psi_{i,\ell}}
\end{array}
\]

\noindent
for all $i = 1,\dots,n$.

\item[$\Filt$ consists of compact saturated subsets.]

First, it is clear that $\Filt$ consists of saturated sets
since each $\SI{\bigwedge_{\ell \in L_i} \psi_{i,\ell}}$
is saturated by Lemma~\ref{lem:top:char},
while $\I{\UPT\Env}$ is equipped with the pointwise order.

Moreover, $\Filt$ consists of compact sets since
by Lemma~\ref{lem:top:char:fin}
each element of $\Filt$ is a (finite) product of
compact sets.

\item[We have $\bigcap \Filt \sle \SP$.]
Indeed, by assumption we have
$\prod_i \I{\psi_i} \sle \SP$,
while
\[
\begin{array}{l l l}
  \prod_i \I{\psi_i}
& =
& \bigcap_{j_1 \in J_1} \dots \bigcap_{j_n \in J_n}
  \prod_i \I{\psi_{i,j_i}}
\\

& =
& \bigcap_{L_1 \sle_\fin J_1} \dots \bigcap_{L_n \sle_\fin J_n}
  \prod_i \I{\bigwedge_{j \in L_i}\psi_{i,j}}
\\

& =
& \bigcap \Filt
\end{array}
\]
\end{description}
\end{full}

%\noindent
Hence Proposition~\ref{prop:wf} applies,
and there are 
finite
$L_1 \sle J_1,\dots,L_n \sle J_n$ s.t.\
\[
\begin{array}{l l l}
  \I{\bigwedge_{j \in L_1}\psi_{1,j}}
  \times \dots \times
  \I{\bigwedge_{j \in L_n}\psi_{n,j}}
& \sle
& \SP
\end{array}
\]

Using subtyping, we can therefore reduce to the
sound judgment
\[
\begin{array}{c}
  x_1 : \reft{\PTbis_1 ~\middle|~ \bigwedge_{j \in L_1}\psi_{1,j}}
  ,\dots,
  x_n : \reft{\PTbis_n ~\middle|~ \bigwedge_{j \in L_n}\psi_{n,j}}
  \thesis
  M : \reft{\PT \mid \varphi}
\end{array}
\]

Assume that for some $i$ we have
$\SI{\bigwedge_{j \in L_i}\psi_{i,j}} = \emptyset$.
Then Proposition~\ref{prop:compl:fin:ded}
yields
$\bigwedge_{j \in L_i}\psi_{i,j} \thesis \False$
and we can conclude using the left-rule for $\bigvee$
in Figure~\ref{fig:reft:reftyping}.

Otherwise, by Lemma~\ref{lem:top:char:fin}
for each $i$ there is some finite $e_i \in \I{\PTbis_i}$
such that
$\up e_i = \SI{\bigwedge_{j \in L_i}\psi_{i,j}}$.
Recall that $\varphi = \bigvee_{k \in K}\varphi_k$
with $\varphi_k \in \Lang_\land(\PT)$.
We have
\[
\begin{array}{*{5}{l}}
  \up \vec e
& =
& \I{\bigwedge_{j \in L_1}\psi_{1,j}}
  \times \dots \times
  \I{\bigwedge_{j \in L_n}\psi_{n,j}}
& \sle
& \bigcup_{k \in K} \I M^{-1}(\I{\varphi_k})
\end{array}
\]

\noindent
Hence, for some $k \in K$ the judgment
\[
\begin{array}{c}
  x_1 : \reft{\PTbis_1 ~\middle|~ \bigwedge_{j \in L_1}\psi_{1,j}}
  ,\dots,
  x_n : \reft{\PTbis_n ~\middle|~ \bigwedge_{j \in L_n}\psi_{n,j}}
  \thesis
  M : \reft{\PT \mid \varphi_k}
\end{array}
\]

\noindent
is sound.
We can now conclude by Theorem~\ref{thm:compl:fin} and subtyping.
\qed
%%%%%%%%%%%%%%%%%%%%%%%%%%%%%%%%%%%%%%%%%%%%%%%%%%%%%%%%%%%%%%%%%%%%%%%%%%%
\end{proof}
%%%%%%%%%%%%%%%%%%%%%%%%%%%%%%%%%%%%%%%%%%%%%%%%%%%%%%%%%%%%%%%%%%%%%%%%%%%






