%%%%%%%%%%%%%%%%%%%%%%%%%%%%%%%%%%%%%%%%%%%%%%%%%%%%%%%%%%%%%%%%%%%%%%%%%%%
\section{Future Work}
\label{sec:conc}
%%%%%%%%%%%%%%%%%%%%%%%%%%%%%%%%%%%%%%%%%%%%%%%%%%%%%%%%%%%%%%%%%%%%%%%%%%%

We think of the present infinitary system as an intermediary
between denotational semantics and finitary type systems
in the style of \cite{jr21esop}.
In the later,
the logic uses fixpoints in the spirit of the modal $\mu$-calculus
(cf.\ Example~\ref{ex:sem:modalmu}).
When fixpoints are \emph{alternation-free}%
\footnote{This corresponds to
``alternation depth $1$'' in~\cite[\S 2.2 \& \S 4.1]{bw18chapter}.
See also~\cite[\S 7]{bs07chapter} and~\cite{sv10apal}.}
(which includes $\LTL$ on $\Stream\BT$ and $\CTL$ on $\Tree \BT$),
their semantics can computed by iteration up to $\omega$.
In order to reason syntactically over (finite) unfoldings of alternation-free fixpoints,
the system of \cite{jr21esop} uses a term language over natural
numbers (with quantifications over these).

We target a similar finitarization of our system,
in which alternation-free fixpoints $\mu p.\varphi(p)$
and $\nu p.\varphi(p)$ would be seen as
$(\exists k)\varphi^k(\False)$ and $(\forall k)\varphi^k(\True)$.
Rules $\ax{WF}$ and $\ax{D}$ may turn out to be problematic.
Our Main Theorem~\ref{thm:main} shows that $\ax{WF}$
is not needed in an interesting range of cases.
On the other hand, in view of Example~\ref{ex:log:distr}
we think rule $\ax{D}$ could be handled (under appropriate assumptions)
using enough fresh Skolem symbols, as in
\[
\tag{$f$ fresh function symbol}
\begin{array}{c}
\dfrac{(\forall k)\psi(k,f(k)) ~\thesis~ \varphi}
  {(\forall k)(\exists \ell)\psi(k,\ell) ~\thesis~ \varphi}

\qquad

\dfrac{\psi ~\thesis~ (\exists k)\varphi(k,f(k))}
  {\psi ~\thesis~ (\exists k)(\forall \ell)\varphi(k,\ell)}

\end{array}
\]


Further, we expect to handle alternation-free modal $\mu$-properties
on (finitary) polynomial types, thus targeting a system which as a whole
would be based on $\FPC$.
But polynomial types involve sums,
and sums are not universal in $\Scott$.%
\footnote{See e.g. \cite[Exercise 6.1.10]{ac98book}.}
We think of working with Call-By-Push-Value (CBPV)~\cite{levy03book,levy22siglog}
for the usual adjunction between dcpos and cpos with strict functions.
On the long run, it would be nice if this basis could extend to
enriched models of CBPV,
so as to handle further computational effects.
Print and global store are particularly relevant,
as an important trend in proving temporal properties
considers programs generating streams of events.
Major works in this line include
\cite{ssv08jfp,hc14lics,hl17lics,nukt18lics,kt14lics,ust17popl,nukt18lics,%
su23popl}.
In contrast with ours, these approaches are based on trace semantics
of syntactic expressions rather than denotational domains.%
%%%%%%%%%%%%%%%%%%%%%%%%%%%%%%%%%%%%%%%%%%%%%%%%%%%%%%%%%%%%%%%%%%%%%%%%%%%
\footnote{See e.g.~\cite[Theorem 4.1 (and Figure 6)]{nukt18lics}
or~\cite[Theorem 1 (and Definition 20 from the full version)]{su23popl}.}
%%%%%%%%%%%%%%%%%%%%%%%%%%%%%%%%%%%%%%%%%%%%%%%%%%%%%%%%%%%%%%%%%%%%%%%%%%%


In a different direction, we think the approach of this paper
could extend to linear types~\cite{hjk00mscs,nw03concur,winskel04llcs},
possibly relying on the categorical study of~\cite{bf06book}.





