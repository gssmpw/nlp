% This is samplepaper.tex, a sample chapter demonstrating the
% LLNCS macro package for Springer Computer Science proceedings;
% Version 2.21 of 2022/01/12
\documentclass[runningheads]{llncs}

\usepackage[utf8]{inputenc}
\usepackage[T1]{fontenc}
% T1 fonts will be used to generate the final print and online PDFs,
% so please use T1 fonts in your manuscript whenever possible.
% Other font encondings may result in incorrect characters.

% \usepackage{graphicx}
% Used for displaying a sample figure. If possible, figure files should
% be included in EPS format.

% francais
%\usepackage[T1]{fontenc}
%\usepackage[francais]{babel}

% mathematical typesetting
\usepackage{enumerate}
\usepackage{array}
%\usepackage[all]{xy}
\usepackage{tikz-cd}

\usepackage{booktabs}   %% For formal tables:
                        %% http://ctan.org/pkg/booktabs

% mathfonts
\usepackage{mathtools}
%\usepackage{latexsym}
%\usepackage{amsmath}
\usepackage{amsfonts}
\usepackage{amssymb}
%\usepackage{amsthm}
\usepackage{stmaryrd}
\usepackage{bussproofs}  \EnableBpAbbreviations
\usepackage{cmll}
%\usepackage{mathabx}
%\usepackage{euler}
\usepackage{bm} % Bold mathematical symbols

\DeclareSymbolFont{symbolsC}{U}{txsyc}{m}{n}
%\DeclareMathSymbol{\boxright}{\mathrel}{symbolsC}{128}
\DeclareMathSymbol{\circledwedge}{\mathrel}{symbolsC}{84}
\DeclareMathSymbol{\circledvee}{\mathrel}{symbolsC}{85}


% Versions : short or full
\usepackage[%
  none,
  full,		% full (online) version
]{optional}



% Hyperref
%\usepackage[pagebackref]{hyperref}
%\usepackage{hyperref}
\opt{full}{\usepackage[pagebackref,linktoc=all]{hyperref}}
\opt{full}{\renewcommand*{\backref}[1]{}}
\opt{full}{\renewcommand*{\backrefalt}[4]{{
    \ifcase #1 Not cited.%
          \or Cited on page~#2.%
          \else Cited on pages #2.%
    \fi%
}}}

\hypersetup{
  breaklinks=true,
  colorlinks = true,
  urlcolor = black,
  linkcolor= black,
  citecolor= black,
  filecolor= black,
}

\opt{full}{\setcounter{tocdepth}{3}}

% If you use the hyperref package, please uncomment the following two lines
% to display URLs in blue roman font according to Springer's eBook style:
\usepackage{color}
\renewcommand\UrlFont{\color{blue}\rmfamily}
\urlstyle{rm}


% macros
\usepackage{macros}

%%%%%% CONDITIONAL DISPLAY OF SOME PROOFS %%%%%%
\usepackage{comment}


\opt{full}{\newcommand\FLAGFULL{}}
\ifdefined\FLAGFULL
  \newenvironment{full}{}{}
\else
  \excludecomment{full}
\fi



% Unnumbered environnments
\spnewtheorem*{remark*}{Remark}{\itshape}{\upshape}





% Titre et auteur
\newcommand{\me}{Colin \textsc{Riba}}
\newcommand{\myinstitute}{ENS de Lyon, CNRS, Université Claude Bernard Lyon 1, LIP,
UMR 5668, 69342, Lyon cedex 07, France}
\newcommand\colinemail{\texttt{colin.riba@ens-lyon.fr}}

\author{Colin Riba \and Alexandre Kejikian}
% \author{First Author\inst{1}\orcidID{0000-1111-2222-3333} \and
% Second Author\inst{2,3}\orcidID{1111-2222-3333-4444} \and
% Third Author\inst{3}\orcidID{2222--3333-4444-5555}}

% First names are abbreviated in the running head.
% If there are more than two authors, 'et al.' is used.
\authorrunning{C. Riba and A. Kejikian}
%\authorrunning{F. Author et al.}

\institute{ENS de Lyon, CNRS, Université Claude Bernard Lyon 1, LIP, UMR 5668,
69342, Lyon cedex 07, France}
%\email{\{colin.riba,alexandre.kejikian\}@ens-lyon.fr}}

% \institute{Princeton University, Princeton NJ 08544, USA \and
% Springer Heidelberg, Tiergartenstr. 17, 69121 Heidelberg, Germany
% \email{lncs@springer.com}\\
% \url{http://www.springer.com/gp/computer-science/lncs} \and
% ABC Institute, Rupert-Karls-University Heidelberg, Heidelberg, Germany\\
% \email{\{abc,lncs\}@uni-heidelberg.de}}

%\newcommand\mytitle{Contribution Title}
\newcommand\mytitle{Infinitary Refinement Types for Temporal Properties in Scott Domains}
%\newcommand\mytitlerunning{Abbreviated paper title}


\title{\mytitle}
%\titlerunning{\mytitlerunning}
% If the paper title is too long for the running head, you can set
% an abbreviated paper title here



%%%%%%%%%%%%%%%%%%%%%%%%%%%%%%%%%%%%%%%%%%%%%%%%%%%%%%%%%%%%%%%%%%%%%%%%%%%
\begin{document}
%%%%%%%%%%%%%%%%%%%%%%%%%%%%%%%%%%%%%%%%%%%%%%%%%%%%%%%%%%%%%%%%%%%%%%%%%%%

\maketitle              % typeset the header of the contribution


\begin{abstract}  
Test time scaling is currently one of the most active research areas that shows promise after training time scaling has reached its limits.
Deep-thinking (DT) models are a class of recurrent models that can perform easy-to-hard generalization by assigning more compute to harder test samples.
However, due to their inability to determine the complexity of a test sample, DT models have to use a large amount of computation for both easy and hard test samples.
Excessive test time computation is wasteful and can cause the ``overthinking'' problem where more test time computation leads to worse results.
In this paper, we introduce a test time training method for determining the optimal amount of computation needed for each sample during test time.
We also propose Conv-LiGRU, a novel recurrent architecture for efficient and robust visual reasoning. 
Extensive experiments demonstrate that Conv-LiGRU is more stable than DT, effectively mitigates the ``overthinking'' phenomenon, and achieves superior accuracy.
\end{abstract}  

\section{Introduction}


\begin{figure}[t]
\centering
\includegraphics[width=0.6\columnwidth]{figures/evaluation_desiderata_V5.pdf}
\vspace{-0.5cm}
\caption{\systemName is a platform for conducting realistic evaluations of code LLMs, collecting human preferences of coding models with real users, real tasks, and in realistic environments, aimed at addressing the limitations of existing evaluations.
}
\label{fig:motivation}
\end{figure}

\begin{figure*}[t]
\centering
\includegraphics[width=\textwidth]{figures/system_design_v2.png}
\caption{We introduce \systemName, a VSCode extension to collect human preferences of code directly in a developer's IDE. \systemName enables developers to use code completions from various models. The system comprises a) the interface in the user's IDE which presents paired completions to users (left), b) a sampling strategy that picks model pairs to reduce latency (right, top), and c) a prompting scheme that allows diverse LLMs to perform code completions with high fidelity.
Users can select between the top completion (green box) using \texttt{tab} or the bottom completion (blue box) using \texttt{shift+tab}.}
\label{fig:overview}
\end{figure*}

As model capabilities improve, large language models (LLMs) are increasingly integrated into user environments and workflows.
For example, software developers code with AI in integrated developer environments (IDEs)~\citep{peng2023impact}, doctors rely on notes generated through ambient listening~\citep{oberst2024science}, and lawyers consider case evidence identified by electronic discovery systems~\citep{yang2024beyond}.
Increasing deployment of models in productivity tools demands evaluation that more closely reflects real-world circumstances~\citep{hutchinson2022evaluation, saxon2024benchmarks, kapoor2024ai}.
While newer benchmarks and live platforms incorporate human feedback to capture real-world usage, they almost exclusively focus on evaluating LLMs in chat conversations~\citep{zheng2023judging,dubois2023alpacafarm,chiang2024chatbot, kirk2024the}.
Model evaluation must move beyond chat-based interactions and into specialized user environments.



 

In this work, we focus on evaluating LLM-based coding assistants. 
Despite the popularity of these tools---millions of developers use Github Copilot~\citep{Copilot}---existing
evaluations of the coding capabilities of new models exhibit multiple limitations (Figure~\ref{fig:motivation}, bottom).
Traditional ML benchmarks evaluate LLM capabilities by measuring how well a model can complete static, interview-style coding tasks~\citep{chen2021evaluating,austin2021program,jain2024livecodebench, white2024livebench} and lack \emph{real users}. 
User studies recruit real users to evaluate the effectiveness of LLMs as coding assistants, but are often limited to simple programming tasks as opposed to \emph{real tasks}~\citep{vaithilingam2022expectation,ross2023programmer, mozannar2024realhumaneval}.
Recent efforts to collect human feedback such as Chatbot Arena~\citep{chiang2024chatbot} are still removed from a \emph{realistic environment}, resulting in users and data that deviate from typical software development processes.
We introduce \systemName to address these limitations (Figure~\ref{fig:motivation}, top), and we describe our three main contributions below.


\textbf{We deploy \systemName in-the-wild to collect human preferences on code.} 
\systemName is a Visual Studio Code extension, collecting preferences directly in a developer's IDE within their actual workflow (Figure~\ref{fig:overview}).
\systemName provides developers with code completions, akin to the type of support provided by Github Copilot~\citep{Copilot}. 
Over the past 3 months, \systemName has served over~\completions suggestions from 10 state-of-the-art LLMs, 
gathering \sampleCount~votes from \userCount~users.
To collect user preferences,
\systemName presents a novel interface that shows users paired code completions from two different LLMs, which are determined based on a sampling strategy that aims to 
mitigate latency while preserving coverage across model comparisons.
Additionally, we devise a prompting scheme that allows a diverse set of models to perform code completions with high fidelity.
See Section~\ref{sec:system} and Section~\ref{sec:deployment} for details about system design and deployment respectively.



\textbf{We construct a leaderboard of user preferences and find notable differences from existing static benchmarks and human preference leaderboards.}
In general, we observe that smaller models seem to overperform in static benchmarks compared to our leaderboard, while performance among larger models is mixed (Section~\ref{sec:leaderboard_calculation}).
We attribute these differences to the fact that \systemName is exposed to users and tasks that differ drastically from code evaluations in the past. 
Our data spans 103 programming languages and 24 natural languages as well as a variety of real-world applications and code structures, while static benchmarks tend to focus on a specific programming and natural language and task (e.g. coding competition problems).
Additionally, while all of \systemName interactions contain code contexts and the majority involve infilling tasks, a much smaller fraction of Chatbot Arena's coding tasks contain code context, with infilling tasks appearing even more rarely. 
We analyze our data in depth in Section~\ref{subsec:comparison}.



\textbf{We derive new insights into user preferences of code by analyzing \systemName's diverse and distinct data distribution.}
We compare user preferences across different stratifications of input data (e.g., common versus rare languages) and observe which affect observed preferences most (Section~\ref{sec:analysis}).
For example, while user preferences stay relatively consistent across various programming languages, they differ drastically between different task categories (e.g. frontend/backend versus algorithm design).
We also observe variations in user preference due to different features related to code structure 
(e.g., context length and completion patterns).
We open-source \systemName and release a curated subset of code contexts.
Altogether, our results highlight the necessity of model evaluation in realistic and domain-specific settings.






%%%%%%%%%%%%%%%%%%%%%%%%%%%%%%%%%%%%%%%%%%%%%%%%%%%%%%%%%%%%%%%%%%%%%%%%%%%
\section{A Refinement Type System}
\label{sec:system}
%%%%%%%%%%%%%%%%%%%%%%%%%%%%%%%%%%%%%%%%%%%%%%%%%%%%%%%%%%%%%%%%%%%%%%%%%%%


\noindent
We assume given a collection of sets $\Base$,
which will play the role of \emph{base types}.

%%%%%%%%%%%%%%%%%%%%%%%%%%%%%%%%%%%%%%%%%%%%%%%%%%%%%%%%%%%%%%%%%%%%%%%%%%%
\subsection{The Pure System}
\label{sec:pure}
%%%%%%%%%%%%%%%%%%%%%%%%%%%%%%%%%%%%%%%%%%%%%%%%%%%%%%%%%%%%%%%%%%%%%%%%%%%

The \emph{pure types}
(notation $\PT,\PTbis,\dots$) are
the closed types over the grammar
\begin{equation*}
\begin{array}{r @{\ \ }c@{\ \ } l}
     \PT
&    \bnf
&    \BT
\gss \PT \times \PT
\gss \PT \arrow \PT
\gss \TV
\gss \rec \TV.\PT 
\end{array}
\end{equation*}

\noindent
where $\BT \in \Base$,
where $\TV$ ranges over an infinite supply of \emph{type variables},
and where $\rec\TV.\PT$ binds $\TV$ in $\PT$.
%
We consider terms from the grammar
\[
\begin{array}{r !{~}r!{~} l}
    M,N 
&   \bnf
&   x
\gs \lambda x.M
\gs M N
\gs \fix x.M
\gs \fold(M)
\gs \unfold(M)
\\

&   \mid
&   \pair{M,N}
\gs \pi_1(M)
\gs \pi_2(M)
\gs a
\gs \cse\ M\ \copair{a \mapsto N_a \mid a \in \BT}
\end{array}
\]

\noindent
where $\BT \in \Base$ and $a \in \BT$.
The term constructors $\fold, \unfold, \pi_1, \pi_2$
are often written in curried form,
so that e.g.\ $(\fold M)$ stands for $\fold(M)$.


Terms are typed as usual, with judgments of the form
$\Env \thesis M : \PT$, where $\Env$ is a list
$x_1:\PTbis_1,\dots,x_n:\PTbis_n$ with $x_i \neq x_j$
if $i \neq j$.
Some typing rules are presented in Figure~\ref{fig:reft:puretyping}.%
\footnote{The set of all typing rules of the pure system is in
Figure~\ref{fig:app:puretyping}, \S\ref{sec:app}.}
Of course, each type $\PT$ is inhabited
by the term $\Omega_\PT \deq \fix x.x : \PT$.


%%%%%%%%%%%%%%%%%%%%%%%%%%%%%%%%%%%%%%%%%%%%%%%%%%%%%%%%%%%%%%%%%%%%%%%%%%%
\begin{figure}[t!]
%%%%%%%%%%%%%%%%%%%%%%%%%%%%%%%%%%%%%%%%%%%%%%%%%%%%%%%%%%%%%%%%%%%%%%%%%%%
\[
\begin{array}{c}
\dfrac{\Env,x:\PT \thesis M : \PT}
  {\Env \thesis \fix x.M : \PT}
\qquad\quad

\dfrac{\Env \thesis M : \PT[\rec\TV.\PT/\TV]}
  {\Env \thesis \fold(M) : \rec\TV.\PT}

\qquad\quad

\dfrac{\Env \thesis M : \rec\TV.\PT}
  {\Env \thesis \unfold(M) : \PT[\rec\TV.\PT/\TV]}

\\\\

\dfrac{}
  {\Env \thesis a : \BT}

\qquad\quad

\dfrac{ \Env \thesis M : \BT
  \qquad\text{for each $a \in \BT$,\quad} \Env \thesis N_a : \PT}
  {\Env \thesis \cse\ M\ \copair{a \mapsto N_a \mid a \in \BT} : \PT}

\end{array}
\]
\caption{Typing rules of the pure calculus (excerpt).%
\label{fig:reft:puretyping}}
%%%%%%%%%%%%%%%%%%%%%%%%%%%%%%%%%%%%%%%%%%%%%%%%%%%%%%%%%%%%%%%%%%%%%%%%%%%
\end{figure}
%%%%%%%%%%%%%%%%%%%%%%%%%%%%%%%%%%%%%%%%%%%%%%%%%%%%%%%%%%%%%%%%%%%%%%%%%%%

%%%%%%%%%%%%%%%%%%%%%%%%%%%%%%%%%%%%%%%%%%%%%%%%%%%%%%%%%%%%%%%%%%%%%%%%%%%
% BEGIN FULL
\begin{full}
\begin{remark*} %[Operational Semantics]
%%%%%%%%%%%%%%%%%%%%%%%%%%%%%%%%%%%%%%%%%%%%%%%%%%%%%%%%%%%%%%%%%%%%%%%%%%%
This paper concerns the denotational semantics
of the above fragment of $\FPC$ in %usual
Scott domains.
This denotational semantics, to be discussed in~\S\ref{sec:sem},
is compatible with the contextual closure of the %following
usual evaluation rules.
\[
\begin{array}{r c l !{\qquad} r c l}
  (\lambda x.M)N
& \rhd
& M[N/x]

& \unfold(\fold M)
& \rhd
& M
\\

  \pi_i \pair{M_1,M_2}
& \rhd
& M_i

& \fix x.M
& \rhd
& M[\fix x.M/x]
\\

  \cse\ a\ \copair{a \mapsto N_a \mid a \in \BT}
& \rhd
&  N_a
\end{array}
\]
%%%%%%%%%%%%%%%%%%%%%%%%%%%%%%%%%%%%%%%%%%%%%%%%%%%%%%%%%%%%%%%%%%%%%%%%%%%
\end{remark*}
\end{full}
% END FULL
%%%%%%%%%%%%%%%%%%%%%%%%%%%%%%%%%%%%%%%%%%%%%%%%%%%%%%%%%%%%%%%%%%%%%%%%%%%

%%%%%%%%%%%%%%%%%%%%%%%%%%%%%%%%%%%%%%%%%%%%%%%%%%%%%%%%%%%%%%%%%%%%%%%%%%%
\begin{example} %[Streams and Trees]
\label{ex:pure}
%%%%%%%%%%%%%%%%%%%%%%%%%%%%%%%%%%%%%%%%%%%%%%%%%%%%%%%%%%%%%%%%%%%%%%%%%%%
The type of \emph{streams over $\PTbis$} is
$\Stream\PTbis \deq \rec\TV.\, \PTbis \times \TV$.
It is equipped with the constructor
\(
  \Cons
  \deq
  \lambda h .\lambda t.\fold \pair{h,t}
  :
  \PTbis \arrow \Stream\PTbis \arrow \Stream \PTbis
\).
We use the infix notation $(M \Colon N)$ for $(\Cons M\, N)$.
The usual \emph{head} and \emph{tail} functions
are $\hd \deq \lambda s.\, \pi_1 (\unfold s) : \Stream\PTbis \arrow \PTbis$
and $\tl \deq \lambda s.\, \pi_2 (\unfold s) : \Stream\PTbis \arrow \Stream\PTbis$.

The type of \emph{binary trees over $\PTbis$} is
$\Tree\PTbis \deq \rec\TV.\, \PTbis \times (\TV \times \TV)$.
The constructor
$\Node : \PTbis \arrow \Tree\PTbis \arrow \Tree\PTbis \arrow \Tree\PTbis$
and the destructors
$\lbl : \Tree\PTbis \arrow \PTbis$
and
$\lft, \rght : \Tree\PTbis \arrow \Tree\PTbis$
are defined similarly as resp.\ $\Cons, \hd, \tl$ on streams.
%%%%%%%%%%%%%%%%%%%%%%%%%%%%%%%%%%%%%%%%%%%%%%%%%%%%%%%%%%%%%%%%%%%%%%%%%%%
\end{example}
%%%%%%%%%%%%%%%%%%%%%%%%%%%%%%%%%%%%%%%%%%%%%%%%%%%%%%%%%%%%%%%%%%%%%%%%%%%

%%%%%%%%%%%%%%%%%%%%%%%%%%%%%%%%%%%%%%%%%%%%%%%%%%%%%%%%%%%%%%%%%%%%%%%%%%%
\begin{table}[t!]
%%%%%%%%%%%%%%%%%%%%%%%%%%%%%%%%%%%%%%%%%%%%%%%%%%%%%%%%%%%%%%%%%%%%%%%%%%%
\[
\begin{array}{c}
\toprule

\begin{array}{r !{~}c!{~} l !{~}c!{~} l}
  \bm{\map}
& \deq
& \lambda f. \fix g. \lambda x.~
  (f\ (\hd x)) \Colon (g\ (\tl x))

& :
& (\PT \arrow \PTbis)
  \longarrow
  \Stream\PT
  \longarrow
  \Stream\PTbis

\end{array}


\\\midrule

\begin{array}{r !{~}l!{~} l}
  \bm{\filter}
& :
& (\PTbis \arrow \Bool)
  \longarrow
  \Stream\PTbis
  \longarrow
  \Stream\PTbis
\\


& \deq
& \lambda p. \fix g.\lambda x.~
  \term{if}~ (p\ (\hd x))
  ~\term{then}~ (\hd x) \Colon (g\ (\tl x))
  ~\term{else}~ (g\ (\tl x))
\end{array}


\\\midrule

\begin{array}{r !{~}l!{~} l !{~}c!{~} l}
  \bm{\diag}
& \deq
& \diagaux (\lambda x.x)
& :
& \Stream(\Stream \PTbis) \longarrow \Stream \PTbis
\\[0.5em]

  \bm{\diagaux}
& :
& \multicolumn{3}{l}{
  (\Stream\PTbis \arrow \Stream\PTbis)
  \longarrow
  \Stream (\Stream \PTbis)
  \longarrow
  \Stream \PTbis
  }
\\

& \deq
& \multicolumn{3}{l}{
  \fix g. \lambda k. \lambda x.~
  \big( (\hd \comp\, k) (\hd x) \big)
  \Colon
  \big( g\ (k \comp \tl)\ (\tl x) \big)
  }
\end{array}

\\\midrule

\begin{array}{c !{\qquad} c}

\begin{array}[t]{l !{~}l!{~} l}

  \bm{\extract}
& :
& \Rou \PTbis \longarrow \PTbis
\\
& \deq 
& \fix e.\lambda c. \unfold\, c\ e
\end{array}

&

\begin{array}[t]{l !{~}l!{~} l}
  \bm{\Over}
& :
& \Rou \PTbis
\\
& \deq
& \fix c. \fold(\lambda k. k\ c)
\end{array}

\end{array}

\\\\

\begin{array}{r !{~}l!{~} l}
  \bm{\bft}
& \deq
& \begin{array}{l !{~}c!{~} l}
    \lambda t.~ \extract (\bftaux\ t\ \Over)
  & :
  & \Tree \PTbis
    \longarrow
    \Stream \PTbis
  \end{array}
\\[0.5em]

  \bm{\bftaux}
& :
& \Tree\PTbis
  \longarrow
  \Rou (\Stream\PTbis)
  \longarrow
  \Rou (\Stream\PTbis)
\\

& \deq
& \fix g.\lambda t.\lambda c.
  \fold \left(
    \lambda k.~
    (\lbl t) \Colon
    \left( \unfold\, c \ \big( k \comp (g (\lft t)) \comp (g (\rght t)) \big) \right)
  \right)

\end{array}

\\\bottomrule

\end{array}
\]
\caption{Functions on Streams and Trees.\label{tab:ex}}
%%%%%%%%%%%%%%%%%%%%%%%%%%%%%%%%%%%%%%%%%%%%%%%%%%%%%%%%%%%%%%%%%%%%%%%%%%%
\end{table}
%%%%%%%%%%%%%%%%%%%%%%%%%%%%%%%%%%%%%%%%%%%%%%%%%%%%%%%%%%%%%%%%%%%%%%%%%%%






%%%%%%%%%%%%%%%%%%%%%%%%%%%%%%%%%%%%%%%%%%%%%%%%%%%%%%%%%%%%%%%%%%%%%%%%%%%
\begin{example}
\label{ex:pure:fun}
%%%%%%%%%%%%%%%%%%%%%%%%%%%%%%%%%%%%%%%%%%%%%%%%%%%%%%%%%%%%%%%%%%%%%%%%%%%
Table~\ref{tab:ex} defines some functions on streams and trees.

On streams, besides the usual $\map$ function,
we consider the $\filter$ function from \S\ref{sec:intro}.
This assumes that $\Base$ contains a set $\Bool = \{\term{tt},\term{ff} \}$
of \emph{Booleans}.
The notation
$\term{if}~ M ~\term{then}~ N_{\term{tt}} ~\term{else}~ N_{\term{ff}}$
stands for the term
$\cse\ M\ \copair{a \mapsto N_a \mid a \in \Bool}$.
Finally, the function $\diag$ computes the diagonal of a stream of streams.
We refer to~\cite[Example 8.3]{jr21esop} for explanations.
Just note that
$\comp$ denotes composition of functions,
so that $M \comp N$ stands for $\lambda x. M(N\, x)$.

On trees, the function $\bft$ implements Martin Hofmann's breadth-first traversal
(see e.g.~\cite{bms19types,jr21esop}).
It uses the recursive type
$\Rou\PTbis \deq \rec\TV.\, (\TV \arrow \PTbis) \arrow \PTbis$.
%%%%%%%%%%%%%%%%%%%%%%%%%%%%%%%%%%%%%%%%%%%%%%%%%%%%%%%%%%%%%%%%%%%%%%%%%%%
\end{example}
%%%%%%%%%%%%%%%%%%%%%%%%%%%%%%%%%%%%%%%%%%%%%%%%%%%%%%%%%%%%%%%%%%%%%%%%%%%


%%%%%%%%%%%%%%%%%%%%%%%%%%%%%%%%%%%%%%%%%%%%%%%%%%%%%%%%%%%%%%%%%%%%%%%%%%%
% FULL
\begin{full}
\begin{remark*} %[Fixpoints]
%%%%%%%%%%%%%%%%%%%%%%%%%%%%%%%%%%%%%%%%%%%%%%%%%%%%%%%%%%%%%%%%%%%%%%%%%%%
We assumed a term former $\fix x.M$ for term-level fixpoints,
but it is well known that $\fix$ is definable in presence of recursive types
(cf.\ e.g.\ \cite[\S 20.1]{pierce02book}).
%%%%%%%%%%%%%%%%%%%%%%%%%%%%%%%%%%%%%%%%%%%%%%%%%%%%%%%%%%%%%%%%%%%%%%%%%%%
\end{remark*}
\end{full}
% FULL
%%%%%%%%%%%%%%%%%%%%%%%%%%%%%%%%%%%%%%%%%%%%%%%%%%%%%%%%%%%%%%%%%%%%%%%%%%%

%%%%%%%%%%%%%%%%%%%%%%%%%%%%%%%%%%%%%%%%%%%%%%%%%%%%%%%%%%%%%%%%%%%%%%%%%%%
\subsection{Negation-Free Infinitary Modal Logics}
\label{sec:log}
%%%%%%%%%%%%%%%%%%%%%%%%%%%%%%%%%%%%%%%%%%%%%%%%%%%%%%%%%%%%%%%%%%%%%%%%%%%


%%%%%%%%%%%%%%%%%%%%%%%%%%%%%%%%%%%%%%%%%%%%%%%%%%%%%%%%%%%%%%%%%%%%%%%%%%%
\begin{figure}[t!]
%%%%%%%%%%%%%%%%%%%%%%%%%%%%%%%%%%%%%%%%%%%%%%%%%%%%%%%%%%%%%%%%%%%%%%%%%%%
\[
\begin{array}{c}

\dfrac{\varphi \in \Lang(\PT_1)}
  {\form{\pi_1} \in \Lang(\PT_1 \times \PT_2)}

\qquad\qquad

\dfrac{\varphi \in \Lang(\PT_2)}
  {\form{\pi_2} \in \Lang(\PT_1 \times \PT_2)}

\qquad\qquad

\dfrac{\psi \in \Lang(\PTbis)
  \qquad
  \varphi \in \Lang(\PT)}
  {\psi \realto \varphi \in \Lang(\PTbis \arrow \PT)}

\\\\

\dfrac{\varphi \in \Lang(\PT[\rec \TV.\PT/\TV])}
  {\form\fold \varphi \in \Lang(\rec \TV.\PT)}

\qquad\qquad

\dfrac
  {\text{$\BT \in \Base$ and $a \in \BT$}}
  {\form a \in \Lang(\BT)}

\end{array}
\]
\caption{Modalities.%
\label{fig:modal}}
%%%%%%%%%%%%%%%%%%%%%%%%%%%%%%%%%%%%%%%%%%%%%%%%%%%%%%%%%%%%%%%%%%%%%%%%%%%
\end{figure}
%%%%%%%%%%%%%%%%%%%%%%%%%%%%%%%%%%%%%%%%%%%%%%%%%%%%%%%%%%%%%%%%%%%%%%%%%%%


\noindent
We consider negation-free infinitary formulae with modalities
as in~\cite{abramsky91apal,bk03ic,jr21esop}.



%%%%%%%%%%%%%%%%%%%%%%%%%%%%%%%%%%%%%%%%%%%%%%%%%%%%%%%%%%%%%%%%%%%%%%%%%%%
\begin{definition}[Formulae]
\label{def:form}
%%%%%%%%%%%%%%%%%%%%%%%%%%%%%%%%%%%%%%%%%%%%%%%%%%%%%%%%%%%%%%%%%%%%%%%%%%%
Let $\PT$ be a pure type.

The formulae $\varphi \in \Lang(\PT)$
are formed using the modalities in Figure~\ref{fig:modal}
together with arbitrary set-indexed %(resp.\ finite)
conjunctions $\bigwedge_{i \in I} \varphi_i$ and
disjunctions $\bigvee_{i \in I} \varphi_i$.
We write $\True$ (resp.\ $\False$) for the empty conjunction (resp.\ disjunction).

We let $\Lang_\land(\PT)$
consist of those $\varphi \in \Lang(\PT)$
in which all conjunctions are finite and all disjunctions are empty
($\False$ is the only disjunction allowed in $\Lang_\land(\PT)$).

The formulae $\varphi \in \Lang_\Open(\PT)$
are formed from formulae in $\Lang_\land(\PT)$
using arbitrary disjunctions and finite conjunctions.

The \emph{normal forms} $\varphi \in \Norm(\PT)$
are the $\varphi = \bigwedge_{i \in I} \bigvee_{j \in J_i}\psi_{i,j}$
with $\psi_{i,j} \in \Lang_\land(\PT)$.
%%%%%%%%%%%%%%%%%%%%%%%%%%%%%%%%%%%%%%%%%%%%%%%%%%%%%%%%%%%%%%%%%%%%%%%%%%%
\end{definition}
%%%%%%%%%%%%%%%%%%%%%%%%%%%%%%%%%%%%%%%%%%%%%%%%%%%%%%%%%%%%%%%%%%%%%%%%%%%

\noindent
Note that $\Lang(\PT)$ and $\Lang_{\Open}(\PT)$ are proper classes,
while $\Lang_{\land}(\PT)$ and $\Norm(\PT)$ are sets.


The semantics of formulae is defined in~\S\ref{sec:sem:log}.
%The intented meaning of the modalities in Figure~\ref{fig:modal} is as follows.
Their intended meaning is as follows.
The formula $\psi \realto \varphi \in \Lang(\PTbis \arrow \PT)$
is intended to select those $M : \PTbis \arrow \PT$
such that $\varphi$ holds on $M N : \PT$ whenever $\psi$ holds on $N : \PTbis$.
%
Similarly, $\form\fold \varphi$ holds on $M$ whenever
$\varphi$ holds on $\unfold M$.
%
%
For $i = 1,2$,
the formula $\form{\pi_i} \varphi$
selects those $M : \PT_1 \times \PT_2$
such that $\varphi$ holds on $\pi_i M$.
%
With
$\pair{\varphi_1,\varphi_2} \deq \form{\pi_1}\varphi_1 \land \form{\pi_2} \varphi_2$,
we have a formula which holds on those $M$ such that
$\varphi_i$ holds on $\pi_i M$ for $i = 1,2$.



%%%%%%%%%%%%%%%%%%%%%%%%%%%%%%%%%%%%%%%%%%%%%%%%%%%%%%%%%%%%%%%%%%%%%%%%%%%
\begin{example}
\label{ex:form:base}
%%%%%%%%%%%%%%%%%%%%%%%%%%%%%%%%%%%%%%%%%%%%%%%%%%%%%%%%%%%%%%%%%%%%%%%%%%%
Given $\BT \in \Base$ and $a \in \BT$,
the formula $\form a$ is intended to hold
on $a$ but not on the $b \in \BT \setminus \{a\}$.
For instance, given $\SP \sle \BT$, the formula
$\bigwedge_{a \in \SP}(\form a \realto \form{\term{tt}})$
is intended to select the $p : \BT \arrow \Bool$
such that $(p\, a)$ is $\term{tt}$ for all $a \in \SP$.
%%%%%%%%%%%%%%%%%%%%%%%%%%%%%%%%%%%%%%%%%%%%%%%%%%%%%%%%%%%%%%%%%%%%%%%%%%%
\end{example}
%%%%%%%%%%%%%%%%%%%%%%%%%%%%%%%%%%%%%%%%%%%%%%%%%%%%%%%%%%%%%%%%%%%%%%%%%%%



%%%%%%%%%%%%%%%%%%%%%%%%%%%%%%%%%%%%%%%%%%%%%%%%%%%%%%%%%%%%%%%%%%%%%%%%%%%
\begin{example}
\label{ex:form:stream}
%%%%%%%%%%%%%%%%%%%%%%%%%%%%%%%%%%%%%%%%%%%%%%%%%%%%%%%%%%%%%%%%%%%%%%%%%%%
On streams $\Stream\PTbis$, the composite modalities $\form\hd$ and $\form\tl$
are defined as $\form\hd \psi \deq \form\fold \form{\pi_1} \psi$
and $\form\tl \varphi \deq \form\fold \form{\pi_2} \varphi$.
Given $\psi \in \Lang(\PTbis)$ and $\varphi \in \Lang(\Stream\PTbis)$,
the formulae $\form\hd\psi \in \Lang(\Stream\PTbis)$
and $\form\tl\varphi \in \Lang(\Stream\PTbis)$
select those streams $M$ such that $\psi$ holds on $(\hd M)$
and such that $\varphi$ holds on $(\tl M)$, respectively.
In the following, we write $\Next\varphi$ for $\form\tl\varphi$.

Using $\NN$-indexed connectives,
we can define
the usual $\LTL$ modalities $\Box$ and $\Diam$ as 
$\Box \varphi \deq \bigwedge_{n \in \NN} \bigwedge_{0\leq k \leq n} \Next^k \varphi$
and
$\Diam \varphi \deq \bigvee_{n \in \NN} \bigvee_{0\leq k \leq n} \Next^k \varphi$.
Hence, $\Box \varphi$ (resp.\ $\Diam \varphi$)
is intended to hold on those $M : \Stream\PTbis$
such that $\varphi$ holds on $\tl^n M$ for all $n \in \NN$
(resp.\ for some $n \in \NN$).
In particular, $\Box\Diam \form\hd \psi$ (resp.\ $\Diam\Box\form\hd\psi$)
selects those streams with infinitely many (resp.\ ultimately all) elements
satisfying $\psi$.
%%%%%%%%%%%%%%%%%%%%%%%%%%%%%%%%%%%%%%%%%%%%%%%%%%%%%%%%%%%%%%%%%%%%%%%%%%%
\end{example}
%%%%%%%%%%%%%%%%%%%%%%%%%%%%%%%%%%%%%%%%%%%%%%%%%%%%%%%%%%%%%%%%%%%%%%%%%%%


%%%%%%%%%%%%%%%%%%%%%%%%%%%%%%%%%%%%%%%%%%%%%%%%%%%%%%%%%%%%%%%%%%%%%%%%%%%
\begin{example}
\label{ex:form:tree}
%%%%%%%%%%%%%%%%%%%%%%%%%%%%%%%%%%%%%%%%%%%%%%%%%%%%%%%%%%%%%%%%%%%%%%%%%%%
Similarly, on trees $\Tree\PTbis$ 
one can define composite modalities $\form\lbl$, $\form\lft$ and $\form\rght$
such that
$\form\lbl\psi, \form\lft\varphi, \form\rght\varphi \in \Lang(\Tree\PTbis)$
whenever $\psi \in \Lang(\PTbis)$ and $\varphi \in \Lang(\Tree\PTbis)$.

Moreover, the $\LTL$ stream modalities $\Box,\Diam$ have their 
usual $\CTL$ counterparts 
$\forall\Box$, $\exists\Box$,
$\forall\Diam$ and $\exists\Diam$.
Namely, given $\varphi \in \Lang(\Tree\PTbis)$,
\[
\begin{array}{r l l !{\qquad\quad} r l l}
  \forall \Box \varphi
& \deq
& \bigwedge_{n \in \NN} 
  (\varphi \land \Land(\pl))^n(\True)

& \forall\Diam \varphi
& \deq
& \bigvee_{n \in \NN} 
  (\varphi \lor \Land(\pl))^n(\False)
\\

  \exists \Box \varphi
& \deq
& \bigwedge_{n \in \NN} 
  (\varphi \land \Lor(\pl))^n(\True)

& \exists\Diam \varphi
& \deq
& \bigvee_{n \in \NN} 
  (\varphi \lor \Lor(\pl))^n(\False)
\end{array}
\]


\noindent
where
$\Land\theta \deq \form\lft \theta \land \form\rght \theta$
and
$\Lor\theta \deq \form\lft \theta \lor \form\rght \theta$.

The intended meaning of $\forall\Box \form\lbl\psi$ is to select
those trees whose node labels all satisfy $\psi$,
while $\exists\Box \form\lbl\psi$ asks $\psi$ to hold on all
labels in some infinite path.
The formula $\exists\Diam \form\lbl\psi$
holds if there is a node whose label satisfies $\psi$,
and $\forall\Diam \form\lbl\psi$ requires that every infinite path
has a node label on which $\psi$ holds.
%%%%%%%%%%%%%%%%%%%%%%%%%%%%%%%%%%%%%%%%%%%%%%%%%%%%%%%%%%%%%%%%%%%%%%%%%%%
\end{example}
%%%%%%%%%%%%%%%%%%%%%%%%%%%%%%%%%%%%%%%%%%%%%%%%%%%%%%%%%%%%%%%%%%%%%%%%%%%

Examples~\ref{ex:form:stream} and~\ref{ex:form:tree}
are generalized in Example~\ref{ex:sem:modalmu} (\S\ref{sec:sem:reft})
to (negation-free) least and greatest fixpoints 
in the style of the modal $\mu$-calculus
(see e.g. \cite{bs07chapter,bw18chapter}).



%%%%%%%%%%%%%%%%%%%%%%%%%%%%%%%%%%%%%%%%%%%%%%%%%%%%%%%%%%%%%%%%%%%%%%%%%%%
\begin{figure}[t!]
%%%%%%%%%%%%%%%%%%%%%%%%%%%%%%%%%%%%%%%%%%%%%%%%%%%%%%%%%%%%%%%%%%%%%%%%%%%
\[
\begin{array}{c}

\dfrac{\psi \thesis \theta
  \quad
  \theta \thesis \varphi}
  {\psi \thesis \varphi}

\quad

\dfrac{a \neq b}
  {\form a \land \form b \thesis_{\BT} \False}

\quad

\ax{D}
\dfrac{}
  {\bigwedge_{i \in I}\bigvee_{j \in J_i} \varphi_{i,j}
  \,\thesis\,
  \bigvee_{f \in \prod_{i \in I} J_i}\bigwedge_{i \in I} \varphi_{i,f(i)}}

\\\\

\dfrac{}
  {\varphi \thesis \varphi}

\quad

\dfrac{\text{for each $i \in I$, $\psi \thesis \varphi_i$}}
  {\psi \thesis \bigwedge_{i \in I} \varphi_i}

\quad

\dfrac{\psi_i \thesis \varphi}
  {\bigwedge_{i \in I} \psi_i \thesis \varphi}
~(i \in I)

\quad

\dfrac{}
  {\bigwedge_{i \in I} \form\triangle \varphi_i
  \thesis
  \form\triangle \bigwedge_{i \in I} \varphi_i}

\\\\

\dfrac{\psi \thesis \varphi_i}
  {\psi \thesis \bigvee_{i \in I} \varphi_i}
~(i \in I)

\qquad\qquad

\dfrac{\text{for each $i \in I$, $\psi_i \thesis \varphi$}}
  {\bigvee_{i \in I}\psi_i \thesis \varphi}

\qquad\qquad

\dfrac{}
  {\form\triangle \bigvee_{i \in I}\varphi_i
  \thesis
  \bigvee_{i \in I} \form\triangle\varphi_i}

\\\\

\ax{F}
\dfrac{\psi \in \Lang_\land(\PTbis)
  \quad~~
  \varphi_i \in \Lang(\PT)
  \quad~~
  I \neq \emptyset}
  {\psi \realto \left( \bigvee_{i \in I}\varphi_i \right)
  \,\thesis\,
  \bigvee_{i \in I} \left( \psi \realto \varphi_i \right) }

\qquad

\dfrac{\psi' \thesis_{\PTbis} \psi
  \qquad
  \varphi \thesis_{\PT} \varphi'}
  {\psi \realto \varphi \,\thesis_{\PTbis \arrow \PT}\, \psi' \realto \varphi'}

\qquad

\dfrac{\psi \thesis \varphi}
  {\form\triangle \psi \thesis \form\triangle \varphi}

\\\\

\dfrac{}
  {\bigwedge_{i \in I}\left(\psi \realto \varphi_i \right)
  \,\thesis\,
  \psi \realto \left(\bigwedge_{i \in I} \varphi_i\right)}

\qquad\qquad

\dfrac{}
  {\bigwedge_{i \in I}\left( \psi_i \realto \varphi \right)
  \,\thesis\,
  \left(\bigvee_{i \in I} \psi_i \right) \realto \varphi}

\end{array}
\]
\caption{Basic deduction rules, where $\triangle$ is either $\pi_1$, $\pi_2$ or $\fold$.%
\label{fig:log:ded}}
%%%%%%%%%%%%%%%%%%%%%%%%%%%%%%%%%%%%%%%%%%%%%%%%%%%%%%%%%%%%%%%%%%%%%%%%%%%
\end{figure}
%%%%%%%%%%%%%%%%%%%%%%%%%%%%%%%%%%%%%%%%%%%%%%%%%%%%%%%%%%%%%%%%%%%%%%%%%%%



%%%%%%%%%%%%%%%%%%%%%%%%%%%%%%%%%%%%%%%%%%%%%%%%%%%%%%%%%%%%%%%%%%%%%%%%%%%
\begin{definition}[Deduction]
%%%%%%%%%%%%%%%%%%%%%%%%%%%%%%%%%%%%%%%%%%%%%%%%%%%%%%%%%%%%%%%%%%%%%%%%%%%
A \emph{sequent} has the form $\psi \thesis_{\PT} \varphi$
where $\varphi,\psi \in \Lang(\PT)$.
We often write $\psi \thesis \varphi$ for $\psi \thesis_{\PT} \varphi$.
%
\emph{Basic deduction} is defined by the rules in
Fig.~\ref{fig:log:ded}.

We write $\psi \thesisiff \varphi$ when the sequents
$\psi \thesis \varphi$ and $\varphi \thesis \psi$
are both derivable.
%%%%%%%%%%%%%%%%%%%%%%%%%%%%%%%%%%%%%%%%%%%%%%%%%%%%%%%%%%%%%%%%%%%%%%%%%%%
\end{definition}
%%%%%%%%%%%%%%%%%%%%%%%%%%%%%%%%%%%%%%%%%%%%%%%%%%%%%%%%%%%%%%%%%%%%%%%%%%%

Note that $\varphi \thesis \True$ and $\False \thesis \varphi$
by definition of $\True$ and $\False$.
%
One can derive that $\thesis$ preserves conjunctions
and disjunctions:
if $\psi_i \thesis \varphi_i$ for all $i \in I$,
then
$\bigwedge_{i \in I} \psi_i \thesis \bigwedge_{i \in I} \varphi_i$
and
$\bigvee_{i \in I} \psi_i \thesis \bigvee_{i \in I} \varphi_i$.

%%%%%%%%%%%%%%%%%%%%%%%%%%%%%%%%%%%%%%%%%%%%%%%%%%%%%%%%%%%%%%%%%%%%%%%%%%%
\begin{example}
\label{ex:log:modalnf}
%%%%%%%%%%%%%%%%%%%%%%%%%%%%%%%%%%%%%%%%%%%%%%%%%%%%%%%%%%%%%%%%%%%%%%%%%%%
Let $\triangle$ be either $\pi_1,\pi_2$ or $\fold$.
The modality $\form\triangle$ commutes over conjunctions and disjunctions
($\bigwedge_i \form\triangle \varphi_i \thesisiff \form\triangle\bigwedge_i \varphi_i$,
and similarly for $\bigvee$).
In particular, for each normal form $\varphi$
there is a normal form $\psi$ such that $\form\triangle \varphi \thesisiff \psi$.
%%%%%%%%%%%%%%%%%%%%%%%%%%%%%%%%%%%%%%%%%%%%%%%%%%%%%%%%%%%%%%%%%%%%%%%%%%%
\end{example}
%%%%%%%%%%%%%%%%%%%%%%%%%%%%%%%%%%%%%%%%%%%%%%%%%%%%%%%%%%%%%%%%%%%%%%%%%%%

%%%%%%%%%%%%%%%%%%%%%%%%%%%%%%%%%%%%%%%%%%%%%%%%%%%%%%%%%%%%%%%%%%%%%%%%%%%
\begin{example}
\label{ex:log:distr}
%%%%%%%%%%%%%%%%%%%%%%%%%%%%%%%%%%%%%%%%%%%%%%%%%%%%%%%%%%%%%%%%%%%%%%%%%%%
As usual, the converse of $\ax{D}$ is derivable, and so is the dual
law
\(
  \bigwedge_{f \in \prod_{i \in I}J_i}\bigvee_{i \in I} \varphi_{i,f(i)}
  \,\thesisiff\,
  \bigvee_{i \in I}\bigwedge_{j \in J_i}\varphi_{i,j}
\)
(see e.g. \cite[Lemma VII.1.10]{johnstone82book}).
%%%%%%%%%%%%%%%%%%%%%%%%%%%%%%%%%%%%%%%%%%%%%%%%%%%%%%%%%%%%%%%%%%%%%%%%%%%
\end{example}
%%%%%%%%%%%%%%%%%%%%%%%%%%%%%%%%%%%%%%%%%%%%%%%%%%%%%%%%%%%%%%%%%%%%%%%%%%%

%%%%%%%%%%%%%%%%%%%%%%%%%%%%%%%%%%%%%%%%%%%%%%%%%%%%%%%%%%%%%%%%%%%%%%%%%%%
\begin{remark}
\label{rem:log:realto}
%%%%%%%%%%%%%%%%%%%%%%%%%%%%%%%%%%%%%%%%%%%%%%%%%%%%%%%%%%%%%%%%%%%%%%%%%%%
Taking $I = \emptyset$ in the last two rules of Fig.~\ref{fig:log:ded}
yields $\True \thesis \left(\psi \realto \True \right)$
and $\True \thesis \left(\False \realto \varphi \right)$.
%Hence rule $\ax{F}$ would be unsound with $I = \emptyset$ and $\psi = \False$.
The rule $\ax{F}$ would thus be unsound with $I = \emptyset$ and $\psi = \False$.
Rule $\ax{F}$ differs from usual systems for DTLF
(cf.\ \cite[\S 4.2]{abramsky91apal} \cite[Figure 5]{bk03ic}
and \cite[Figure 10.3]{ac98book}).
The case of $I = \emptyset$
will be handled by rule $\ax{C}$ in \eqref{eq:compl:cc}, \S\ref{sec:compl:fin}.
%%%%%%%%%%%%%%%%%%%%%%%%%%%%%%%%%%%%%%%%%%%%%%%%%%%%%%%%%%%%%%%%%%%%%%%%%%%
\end{remark}
%%%%%%%%%%%%%%%%%%%%%%%%%%%%%%%%%%%%%%%%%%%%%%%%%%%%%%%%%%%%%%%%%%%%%%%%%%%



%%%%%%%%%%%%%%%%%%%%%%%%%%%%%%%%%%%%%%%%%%%%%%%%%%%%%%%%%%%%%%%%%%%%%%%%%%%
\subsection{Refinement Types}
\label{sec:reft}
%%%%%%%%%%%%%%%%%%%%%%%%%%%%%%%%%%%%%%%%%%%%%%%%%%%%%%%%%%%%%%%%%%%%%%%%%%%


\noindent
\emph{Refinement types} (or \emph{types}), notation $\RT,\RTbis,\dots$,
are given by the grammar
\begin{equation*}
\begin{array}{r @{\ \ }c@{\ \ } l}
    \RT
&   \bnf
&   \PT
\gs \reft{\PT \mid \varphi}
\gs \RT \times \RT
\gs \RT \arrow \RT
\end{array}
\end{equation*}

\noindent
where $\PT$ is a pure type and $\varphi \in \Lang(\PT)$.
%
We shall consider typing judgments of the form
$\Env \thesis M : \RT$,
where $\Env$ is allowed to mention refinement types.
A judgment
$M : \reft{\PT \mid \varphi}$
is intended to mean that $M$ is of pure type $\PT$ and satisfies $\varphi$.


%%%%%%%%%%%%%%%%%%%%%%%%%%%%%%%%%%%%%%%%%%%%%%%%%%%%%%%%%%%%%%%%%%%%%%%%%%%
\begin{example}
\label{ex:reft:base}
%%%%%%%%%%%%%%%%%%%%%%%%%%%%%%%%%%%%%%%%%%%%%%%%%%%%%%%%%%%%%%%%%%%%%%%%%%%
Given a base type $\BT \in \Base$ and $\SP \sle \BT$,
a judgment of the form
\(
  p :
  \reft{\BT \arrow \Bool
  \mid
  \bigwedge_{a \in \SP}\left(\form a \realto \form{\term{tt}} \right)}
\)
expresses that 
$(p\, a)$ yields $\term{tt}$ for all $a \in \SP$.
%%%%%%%%%%%%%%%%%%%%%%%%%%%%%%%%%%%%%%%%%%%%%%%%%%%%%%%%%%%%%%%%%%%%%%%%%%%
\end{example}
%%%%%%%%%%%%%%%%%%%%%%%%%%%%%%%%%%%%%%%%%%%%%%%%%%%%%%%%%%%%%%%%%%%%%%%%%%%




%%%%%%%%%%%%%%%%%%%%%%%%%%%%%%%%%%%%%%%%%%%%%%%%%%%%%%%%%%%%%%%%%%%%%%%%%%%
\begin{table}[t!]
%%%%%%%%%%%%%%%%%%%%%%%%%%%%%%%%%%%%%%%%%%%%%%%%%%%%%%%%%%%%%%%%%%%%%%%%%%%
\begin{center}
\scalebox{0.9}{\(
\begin{array}{c}
\toprule

\multicolumn{1}{l}{\text{\textbf{Map on streams}
(with $\triangle$ either $\Box$, $\Diam$, $\Diam\Box$ or $\Box\Diam$)}}
\\
\begin{array}{*{7}{l}}
  \map
& :
& \reft{\PT \arrow \PTbis \mid \psi \realto \varphi}
  %(\reft{\PT \mid \psi} \arrow \reft{\PTbis \mid \varphi})
& \longarrow
& \reft{\Stream \PT \mid \triangle \form{\hd}\psi}
& \longarrow
& \reft{\Stream \PTbis \mid \triangle \form{\hd}\varphi}
\end{array}

\\\midrule

\multicolumn{1}{l}{\text{\textbf{Filter on streams}
(with $\triangle$ either $\Box$ or $\Box\Diam$)}}
\\
\begin{array}{*{7}{l}}
  \filter
& :
& \reft{\BT \arrow \Bool \mid \bigwedge_{a \in \SP}(\form a \realto \form{\term{tt}})}
& \longarrow
& \reft{\Stream \PTbis \mid \triangle \form{\hd}\bigvee_{a \in \SP}\form a}
& \longarrow
& \reft{\Stream \PTbis \mid \Box \form{\hd}\bigvee_{a \in \SP}\form a}
\end{array}

\\\midrule

\multicolumn{1}{l}{\text{\textbf{Diagonal of streams of streams}
(with $\triangle$ either $\Box$ or $\Diam\Box$)}}
\\
\begin{array}{l l l l l}
  \diag
& :
& \reft{\Stream (\Stream \PTbis) \mid \triangle \form\hd \Box \form{\hd}\varphi}
& \longto 
& \reft{\Stream \PTbis \mid \triangle \form{\hd}\varphi}
\\
\end{array}

\\\midrule


\multicolumn{1}{l}{\text{\textbf{Breadth-first tree traversal}
(see Example~\ref{ex:reft:fun} for $\triangle$ and $\overline\triangle$)}}
\\

\begin{array}{l l r c l}
  \bft
& :
& \reft{\Tree\PTbis \mid \triangle \form\lbl \varphi}
& \longto
& \reft{\Stream\PTbis \mid \overline\triangle \form\hd \varphi}
\end{array}

\\\bottomrule

\end{array}\)}
\end{center}
\caption{Some judgments with refinement types
(functions defined in Table~\ref{tab:ex}).%
\label{tab:reft}}
%%%%%%%%%%%%%%%%%%%%%%%%%%%%%%%%%%%%%%%%%%%%%%%%%%%%%%%%%%%%%%%%%%%%%%%%%%%
\end{table}
%%%%%%%%%%%%%%%%%%%%%%%%%%%%%%%%%%%%%%%%%%%%%%%%%%%%%%%%%%%%%%%%%%%%%%%%%%%

%%%%%%%%%%%%%%%%%%%%%%%%%%%%%%%%%%%%%%%%%%%%%%%%%%%%%%%%%%%%%%%%%%%%%%%%%%%
\begin{example}
\label{ex:reft:fun}
%%%%%%%%%%%%%%%%%%%%%%%%%%%%%%%%%%%%%%%%%%%%%%%%%%%%%%%%%%%%%%%%%%%%%%%%%%%
Table~\ref{tab:reft} presents some specifications,
expressed as refinement types, for functions defined in Table~\ref{tab:ex}
(see Example~\ref{ex:pure:fun}).

For the $\map$ function,
assuming $f : \reft{\PTbis \arrow \PT \mid \psi \realto \varphi}$,
if $\triangle$ is $\Box$ (resp.\ $\Diam, \Box\Diam, \Diam\Box$),
then the judgment expresses
that $(\map f)$
takes a stream with all (resp.\ some, infinitely many, ultimately all)
elements satisfying $\psi$ to a stream with all
(resp.\ some, infinitely many, ultimately all)
elements satisfying~$\varphi$.

The specifications for $\filter$ are the expected ones.
Let $p : \BT \to \Bool$ such that
$(p\, a)$ yields $\term{tt}$ for all $a \in \SP$.
If $\triangle$ is $\Box$ (resp.\ $\Box\Diam$).
then the judgment means that
$(\filter p)$ takes a stream with all (resp.\@ infinitely many) elements in $\SP$
to a stream with all elements in $\SP$.
Recalling that the stream formula $\Box \form\hd \form a$ amounts to
$\bigwedge_{n \in \NN} \Next^n \form\hd \form a$,
note that none of the formulae $\Next^n \form\hd \form a$ hold on
$\Omega_{\Stream\BT} : \Stream\BT$.

Concerning the diagonal,
%As for the diagonal function,
if $\triangle$ is $\Box$ (resp.\ $\Diam\Box$),
then
the judgment
expresses that $\diag$ takes a stream whose component streams
all (resp.\@ ultimately all) satisfy $\Box \form\hd \varphi$
to a stream whose elements all (resp.\ ultimately all)
satisfy $\varphi$.

For the tree traversal $\bft$ we can allow for any sound
combination of $\triangle$ and $\overline\triangle$.
This includes all pairs $(\triangle,\overline\triangle)$
among
$(\forall \Box, \Box)$,
$(\exists \Box, \Box\Diam)$,
$(\exists \Diam, \Diam)$,
$(\forall \Diam, \Diam)$
and
$(\forall \Box \exists \Diam, \Box\Diam)$.
For instance,
if $\triangle$ is $\forall\Box$
(resp.\ $\exists\Diam, \forall\Box\exists\Diam$),
then
the judgment says that $\bft$
takes a tree with all (resp.\ some, infinitely many) node
labels satisfying $\varphi$ to a stream with all (resp.\ some, infinitely may)
elements satisfying~$\varphi$.
%%%%%%%%%%%%%%%%%%%%%%%%%%%%%%%%%%%%%%%%%%%%%%%%%%%%%%%%%%%%%%%%%%%%%%%%%%%
\end{example}
%%%%%%%%%%%%%%%%%%%%%%%%%%%%%%%%%%%%%%%%%%%%%%%%%%%%%%%%%%%%%%%%%%%%%%%%%%%

%%%%%%%%%%%%%%%%%%%%%%%%%%%%%%%%%%%%%%%%%%%%%%%%%%%%%%%%%%%%%%%%%%%%%%%%%%%
\begin{figure}[t!]
%%%%%%%%%%%%%%%%%%%%%%%%%%%%%%%%%%%%%%%%%%%%%%%%%%%%%%%%%%%%%%%%%%%%%%%%%%%
\[
\begin{array}{c}

\dfrac{}
  {\RT \subtype \UPT\RT}

\qquad

\dfrac{}
  {\PT \subtype \reft{\PT \mid \True}}

\qquad

\dfrac{\psi \thesis_{\PT} \varphi}
  {\reft{\PT \mid \psi} \subtype \reft{\PT \mid \varphi}}

\qquad

\dfrac{\RT \subtype \RTbis
  \qquad
  \RTbis \subtype \RTter}
  {\RT \subtype \RTter}

\\\\

\dfrac{\RT \subtype \RT'
  \qquad
  \RTbis \subtype \RTbis'}
  {\RT \times \RTbis \subtype \RT' \times \RTbis'}

\qquad

\dfrac{}{
  \reft{\PT \mid \varphi}
  \times
  \reft{\PTbis \mid \psi}
  \eqtype
  \reft{\PT \times \PTbis \mid \pair{\varphi,\psi}}}

\qquad

\dfrac{}
  {\RT \subtype \RT}

\\\\

\dfrac{\RTbis' \subtype \RTbis
  \qquad
  \RT \subtype \RT'}
  {\RTbis \arrow \RT \subtype \RTbis' \arrow \RT'}


\qquad

\dfrac{}
  {\reft{\PTbis \mid \psi} \arrow \reft{\PT \mid \varphi}
  \eqtype
  \reft{\PTbis \arrow \PT \mid \psi \realto \varphi}}

\end{array}
\]

\caption{Subtyping.%
\label{fig:reft:subtyping}}
%%%%%%%%%%%%%%%%%%%%%%%%%%%%%%%%%%%%%%%%%%%%%%%%%%%%%%%%%%%%%%%%%%%%%%%%%%%
\end{figure}
%%%%%%%%%%%%%%%%%%%%%%%%%%%%%%%%%%%%%%%%%%%%%%%%%%%%%%%%%%%%%%%%%%%%%%%%%%%

%%%%%%%%%%%%%%%%%%%%%%%%%%%%%%%%%%%%%%%%%%%%%%%%%%%%%%%%%%%%%%%%%%%%%%%%%%%
\begin{figure}[t!]
%%%%%%%%%%%%%%%%%%%%%%%%%%%%%%%%%%%%%%%%%%%%%%%%%%%%%%%%%%%%%%%%%%%%%%%%%%%
\begin{center}
\scalebox{0.89}{\(
\begin{array}{c}
\dfrac{
  \begin{array}{l}
  \UPT\Env \thesis M : \PT
  \\
  \text{for each $i \in I$,}\quad \Env \thesis M : \reft{\PT \mid \varphi_i}
  \end{array}}
  {\Env \thesis M : \reft{\PT \mid \bigwedge_{i \in I} \varphi_i}}

\qquad\quad

\dfrac{
  \begin{array}{l}
  \UPT\Env, x:\PTbis, \UPT{\Env'} \thesis M : \UPT\RT
  \\
  \text{for each $i \in I$,}\quad
  \Env, x:\reft{\PTbis \mid \psi_i},\Env' \thesis M : \RT
  \end{array}}
  {\Env, x : \reft{\PTbis \mid \bigvee_{i \in I} \psi_i} , \Env' \thesis M : \RT}

\\\\


\dfrac{
  \Env' \subtype \Env
  \quad 
  \RT \subtype \RT'
  \quad
  \Env \thesis M : \RT}
  {\Env' \thesis M : \RT'}

\qquad

\dfrac{\Env \thesis \fix x.M : \reft{\PT \mid \psi}
  \quad
  \Env, x: \reft{\PT \mid \psi} \thesis M : \reft{\PT \mid \varphi}}
  {\Env \thesis \fix x.M : \reft{\PT \mid \varphi}}
%~\text{\scalebox{0.875}{$(\varphi,\psi \in \Lang_\land)$}}
~(\varphi,\psi \in \Lang_\land)

\\\\

\dfrac{\Env \thesis M : \reft{\PT_1 \times \PT_2 \mid \form{\pi_i} \varphi}}
  {\Env \thesis \pi_i(M) : \reft{\PT_i \mid \varphi}}
~(i=1,2)

\qquad\quad

\dfrac{\Env \thesis M_i : \reft{\PT_i \mid \varphi}
  \qquad
  \Env \thesis M_{3-i} : \PT_{3-i}}
  {\Env \thesis \pair{M_1,M_2} : \reft{\PT_1 \times \PT_2 \mid \form{\pi_i} \varphi}}
~(i=1,2)
\\\\


\dfrac{}
  {\Env \thesis a : \reft{\BT \mid \form a}}

\qquad\quad


\dfrac{
  \Env \thesis M : \reft{\BT \mid \form b}
  \qquad
  \Env \thesis N_b : \RT
  \qquad
  \text{for each $a \in A$,\quad} \UPT\Env \thesis N_a : \UPT\RT}
  {\Env \thesis \cse\ M\ \copair{a \mapsto N_a \mid a \in \BT} : \RT}


\\\\


\dfrac{\Env \thesis M : \reft{\PT[\rec\TV.\PT/\TV] \mid \varphi}}
  {\Env \thesis \fold(M) : \reft{\rec\TV.\PT \mid \form\fold \varphi}}

\qquad\quad

\dfrac{\Env \thesis M : \reft{\rec\TV.\PT \mid \form\fold \varphi}}
  {\Env \thesis \unfold(M) : \reft{\PT[\rec\TV.\PT/\TV] \mid \varphi}}

\end{array}\)}
\end{center}
\caption{Typing with refinement types.%
\label{fig:reft:reftyping}}
%%%%%%%%%%%%%%%%%%%%%%%%%%%%%%%%%%%%%%%%%%%%%%%%%%%%%%%%%%%%%%%%%%%%%%%%%%%
\end{figure}
%%%%%%%%%%%%%%%%%%%%%%%%%%%%%%%%%%%%%%%%%%%%%%%%%%%%%%%%%%%%%%%%%%%%%%%%%%%

Each refinement type $\RT$ has an \emph{underlying pure type} $\UPT\RT$
defined by induction from $\UPT\PT \deq \PT$
and $\UPT{\reft{\PT \mid \varphi}} \deq \PT$.
We write $\UPT\Env$ for the extension of $\UPT{\pl}$ to $\Env$.

We derive typing judgments $\Env \thesis M : \RT$
using the rules in Figure~\ref{fig:reft:reftyping}
augmented with \emph{all} the typing rules of the pure system (\S\ref{sec:pure})
extended to refinement types.
Deduction on formulae (\S\ref{sec:log})
enters the type system via a subtyping relation $\RTbis \subtype \RT$.
Subtyping rules are presented in Figure~\ref{fig:reft:subtyping},
where $\RTbis \eqtype \RT$
stands for the conjunction of $\RTbis \subtype \RT$ and $\RT \subtype \RTbis$.
Subtyping is extended to typing contexts:
given $\Env = x_1:\RTbis_1,\dots,x_n:\RTbis_n$
and $\Env' = x_1:\RTbis'_1,\dots,x_n:\RTbis'_n$,
we let $\Env \subtype \Env'$ when $\RTbis_i \subtype \RTbis'_i$
for all $i =1,\dots,n$.
Note that if $\Env \thesis M :\RT$ is derivable
then so is $\UPT\Env \thesis M : \UPT\RT$.

The rules in Figures~\ref{fig:reft:reftyping} and~\ref{fig:reft:subtyping}
are direct adaptations of those in~\cite{abramsky91apal,bk03ic,jr21esop}.
In particular, the rule for $\fix$
(in which $\varphi,\psi \in \Lang_\land(\PT)$)
comes from~\cite{abramsky91apal}.


%%%%%%%%%%%%%%%%%%%%%%%%%%%%%%%%%%%%%%%%%%%%%%%%%%%%%%%%%%%%%%%%%%%%%%%%%%%
\begin{example}
%%%%%%%%%%%%%%%%%%%%%%%%%%%%%%%%%%%%%%%%%%%%%%%%%%%%%%%%%%%%%%%%%%%%%%%%%%%
The following rules are derived using the 
last rule in Figure~\ref{fig:reft:subtyping}.
\[
\begin{array}{c}
\dfrac{\Env,x : \reft{\PTbis \mid \psi} \thesis M : \reft{\PT \mid \varphi}}
  {\Env \thesis \lambda x.M : \reft{\PTbis \arrow \PT \mid \psi \realto \varphi}}

\quad

\dfrac{\Env \thesis M : \reft{\PTbis \arrow \PT \mid \psi \realto \varphi}
  \quad
  \Env \thesis N : \reft{\PTbis \mid \psi}}
  {\Env \thesis M N : \reft{\PT \mid \varphi}}
\end{array}
\]
%%%%%%%%%%%%%%%%%%%%%%%%%%%%%%%%%%%%%%%%%%%%%%%%%%%%%%%%%%%%%%%%%%%%%%%%%%%
\end{example}
%%%%%%%%%%%%%%%%%%%%%%%%%%%%%%%%%%%%%%%%%%%%%%%%%%%%%%%%%%%%%%%%%%%%%%%%%%%


%%%%%%%%%%%%%%%%%%%%%%%%%%%%%%%%%%%%%%%%%%%%%%%%%%%%%%%%%%%%%%%%%%%%%%%%%%%
\begin{lemma}
\label{lem:reft}
%%%%%%%%%%%%%%%%%%%%%%%%%%%%%%%%%%%%%%%%%%%%%%%%%%%%%%%%%%%%%%%%%%%%%%%%%%%
For each type $\RT$, there is $\varphi \in \Lang(\UPT\RT)$
such that $\RT \eqtype \reft{\UPT\RT \mid \varphi}$.
%%%%%%%%%%%%%%%%%%%%%%%%%%%%%%%%%%%%%%%%%%%%%%%%%%%%%%%%%%%%%%%%%%%%%%%%%%%
\end{lemma}
%%%%%%%%%%%%%%%%%%%%%%%%%%%%%%%%%%%%%%%%%%%%%%%%%%%%%%%%%%%%%%%%%%%%%%%%%%%


Our goal is to devise extensions of this type system which
are sound and complete w.r.t.\ the usual Scott semantics,
the sense that given $\thesis M : \PT$,
\[
\begin{array}{c !{\quad}c!{\quad} c}
  \thesis M : \reft{\PT \mid \varphi}
& \text{if, and only if,}
& \text{$\varphi$ holds on $\I M$ in the Scott semantics.}
\end{array}
\]

\noindent
The Scott semantics is recalled in~\S\ref{sec:sem},
while \S\ref{sec:compl} discusses completeness.
In particular, all 
typing judgments in Table~\ref{tab:reft} (Example~\ref{ex:reft:fun})
will be derivable.






\begin{figure}[htbp]
    \centering
    \includegraphics[width=\textwidth]{figures/sem.png}
    \caption{Structural equation model showing the effects of political information polarization and recommendation system exposure on perceived polarization and opinion change. Path coefficients represent standardized regression weights. Solid lines indicate direct effects, dashed lines represent covariances. $^*p < .05$, $^{**}p < .01$, $^{***}p < .001$.}
    \label{fig:path-model}
\end{figure}

%%%%%%%%%%%%%%%%%%%%%%%%%%%%%%%%%%%%%%%%%%%%%%%%%%%%%%%%%%%%%%%%%%%%%%%%%%%
\section{Completeness}
\label{sec:compl}
%%%%%%%%%%%%%%%%%%%%%%%%%%%%%%%%%%%%%%%%%%%%%%%%%%%%%%%%%%%%%%%%%%%%%%%%%%%




%%%%%%%%%%%%%%%%%%%%%%%%%%%%%%%%%%%%%%%%%%%%%%%%%%%%%%%%%%%%%%%%%%%%%%%%%%%
\subsubsection{The Finite Case.}
\label{sec:compl:fin}
%%%%%%%%%%%%%%%%%%%%%%%%%%%%%%%%%%%%%%%%%%%%%%%%%%%%%%%%%%%%%%%%%%%%%%%%%%%
Since the rule $\ax{F}$ assumes $I\neq \emptyset$,
it does not allow us to derive 
$(\psi \realto \False) \thesis \False$.
This sequent is sound only when $\I\psi \neq \emptyset$.
In \cite{abramsky91apal},
Abramsky introduced \emph{coprimeness predicates} %$\C$
which select those finite $\varphi$ with $\I\varphi \neq \emptyset$.
We extend our basic deduction system (Figure \ref{fig:log:ded} in \S\ref{sec:log})
with the predicate $\C$ and the rules in \eqref{eq:compl:cc} below.
Recall that $\pair{\varphi,\psi} = \form{\pi_1}\varphi \land \form{\pi_2}\psi$.
\begin{equation}
\label{eq:compl:cc}
\scalebox{.9125}{\text{$\begin{array}{c}

\dfrac{}{\C(\True)}

\quad

\dfrac
  {\text{$\BT \in \Base$ and $a \in \BT$}}
  {\C(\form a)}

\quad

\dfrac{\C(\varphi)}
  {\C(\form\fold \varphi)}

\quad

\dfrac{\C(\varphi) 
  \quad
  \C(\psi)}
  {\C(\pair{\varphi,\psi})}

\quad

\dfrac{\C(\psi)
  \quad
  \psi \thesis \varphi
  \quad
  \varphi \in \Lang_\land}
  {\C(\varphi)}
%~(\varphi \in \Lang_\land)
\\\\

\ax{C}
\dfrac{\C(\psi)}
  {(\psi \realto \False) \,\thesis\, \False}

\quad

\dfrac{\begin{array}{l}
  \text{$I$ finite and $\forall i \in I$,}~
  \C(\psi_i) 
  ~\text{and}~
  \C(\varphi_i) ;
  \\
  \text{$\forall J \sle I$,}~
  \bigwedge_{j \in J} \psi_j \thesis \False
  ~~\text{or}~~
  \C\left( \bigwedge_{j \in J} \varphi_j \right)
  \end{array}}
  {\C\left( \bigwedge_{i \in I}(\psi_i \realto \varphi_i) \right)}
%~(\text{$I$ finite})
\end{array}$}}
\end{equation}


\noindent
In contrast with \cite{abramsky91apal,bk03ic,ac98book},
our $\C$ is a consistency predicate rather than a coprimeness predicate.
Note that the clauses defining $\C$ are positive.%
\footnote{Compare with \cite[Figure 3]{bk03ic} and \cite[Figure 10.3]{ac98book}.}



%%%%%%%%%%%%%%%%%%%%%%%%%%%%%%%%%%%%%%%%%%%%%%%%%%%%%%%%%%%%%%%%%%%%%%%%%%%
\begin{proposition}
\label{prop:compl:fin:ded}
%%%%%%%%%%%%%%%%%%%%%%%%%%%%%%%%%%%%%%%%%%%%%%%%%%%%%%%%%%%%%%%%%%%%%%%%%%%
In the extension of Figure~\ref{fig:log:ded} (\S\ref{sec:log})
with \eqref{eq:compl:cc}:
\begin{enumerate}[(1)]
\item
for all $\varphi,\psi \in \Lang_\land(\PT)$,
we have
$\psi \thesis_\PT \varphi$
if, and only if,
$\I\psi \sle \I\varphi$;

\item
for all $\varphi \in \Lang_\land$,
we have
$\C(\varphi)$
if, and only if,
$\I\varphi \neq \emptyset$.
\end{enumerate}
%%%%%%%%%%%%%%%%%%%%%%%%%%%%%%%%%%%%%%%%%%%%%%%%%%%%%%%%%%%%%%%%%%%%%%%%%%%
\end{proposition}
%%%%%%%%%%%%%%%%%%%%%%%%%%%%%%%%%%%%%%%%%%%%%%%%%%%%%%%%%%%%%%%%%%%%%%%%%%%

\noindent
In particular, for each $\varphi \in \Lang_\land$,
either $\C(\varphi)$ or $\varphi \thesis \False$ is derivable.

A type is \emph{finite} if it only contains formulae $\varphi \in \Lang_\land$.
A typing context $x_1:\RTbis_1,\dots,x_n : \RTbis_n$
is finite if so are all $\RTbis_i$'s.
Completeness for finite types can be obtained
from minor adaptations to \cite{abramsky91apal}.

%%%%%%%%%%%%%%%%%%%%%%%%%%%%%%%%%%%%%%%%%%%%%%%%%%%%%%%%%%%%%%%%%%%%%%%%%%%
\begin{theorem}[Abramsky \cite{abramsky91apal}]
\label{thm:compl:fin}
%%%%%%%%%%%%%%%%%%%%%%%%%%%%%%%%%%%%%%%%%%%%%%%%%%%%%%%%%%%%%%%%%%%%%%%%%%%
Assume $\Env$ and $\RT$ are finite.
If $\Env \thesis M : \RT$ is sound,
then $\Env \thesis M : \RT$ is derivable in the system of \S\ref{sec:reft}
extended with \eqref{eq:compl:cc}.
%%%%%%%%%%%%%%%%%%%%%%%%%%%%%%%%%%%%%%%%%%%%%%%%%%%%%%%%%%%%%%%%%%%%%%%%%%%
\end{theorem}
%%%%%%%%%%%%%%%%%%%%%%%%%%%%%%%%%%%%%%%%%%%%%%%%%%%%%%%%%%%%%%%%%%%%%%%%%%%



%%%%%%%%%%%%%%%%%%%%%%%%%%%%%%%%%%%%%%%%%%%%%%%%%%%%%%%%%%%%%%%%%%%%%%%%%%%
\subsubsection{Well-Filteredness.}
\label{sec:wf}
%%%%%%%%%%%%%%%%%%%%%%%%%%%%%%%%%%%%%%%%%%%%%%%%%%%%%%%%%%%%%%%%%%%%%%%%%%%
Following~\cite{bk03ic},
completeness for types with infinitary formulae relies on the fact that
Scott domains are \emph{well-filtered} spaces.
The latter is stated in \cite[Corollary 7.1.11]{aj95chapter}
and \cite[Proposition 8.3.5]{goubault13book} as a consequence
of the Hofmann-Mislove (or Scott-open filter) Theorem.
It can also be obtained from \cite[Theorem 7.38]{gg24book}.
%
A subset $F$ of a poset $P$ is \emph{filtering} if $F$ is directed in $P^\op$.


%%%%%%%%%%%%%%%%%%%%%%%%%%%%%%%%%%%%%%%%%%%%%%%%%%%%%%%%%%%%%%%%%%%%%%%%%%%
\begin{proposition}[Well-Filteredness]
\label{prop:wf}
%%%%%%%%%%%%%%%%%%%%%%%%%%%%%%%%%%%%%%%%%%%%%%%%%%%%%%%%%%%%%%%%%%%%%%%%%%%
Let $X$ be an algebraic dcpo,%
\footnote{More generally, this result holds for any sober space $X$
(with $\SP$ open in $X$).}
and let $\Filt$ be a set of compact saturated subsets of $X$.
If $\Filt$ is filtering in $\Po(X)$
and $\bigcap \Filt \sle \SP$ for some Scott-open $\SP$,
then $Q \sle \SP$ for some $Q \in \Filt$.
%%%%%%%%%%%%%%%%%%%%%%%%%%%%%%%%%%%%%%%%%%%%%%%%%%%%%%%%%%%%%%%%%%%%%%%%%%%
\end{proposition}
%%%%%%%%%%%%%%%%%%%%%%%%%%%%%%%%%%%%%%%%%%%%%%%%%%%%%%%%%%%%%%%%%%%%%%%%%%%

Proposition~\ref{prop:wf} yields the soundness of the following deduction rule.
\[
\ax{WF}~
\dfrac{\text{for all $i\in I$, $\psi_i \in \Lang_\land(\PTbis)$}
  \qquad
  \varphi \in \Lang_\Open(\PT)}
  {
  \left( \bigwedge_{i \in I} \psi_i \right)
  \realto
  \varphi
  \,\thesis\,
  \bigvee_{\text{$J \sle I$, $J$ finite}}
  \left(
  \left( \bigwedge_{j \in J} \psi_j \right)
  \realto
  \varphi
  \right)
  }
\]

%%%%%%%%%%%%%%%%%%%%%%%%%%%%%%%%%%%%%%%%%%%%%%%%%%%%%%%%%%%%%%%%%%%%%%%%%%%
\begin{lemma}
\label{lem:wf}
%%%%%%%%%%%%%%%%%%%%%%%%%%%%%%%%%%%%%%%%%%%%%%%%%%%%%%%%%%%%%%%%%%%%%%%%%%%
The rule $\ax{WF}$ is sound.
%%%%%%%%%%%%%%%%%%%%%%%%%%%%%%%%%%%%%%%%%%%%%%%%%%%%%%%%%%%%%%%%%%%%%%%%%%%
\end{lemma}
%%%%%%%%%%%%%%%%%%%%%%%%%%%%%%%%%%%%%%%%%%%%%%%%%%%%%%%%%%%%%%%%%%%%%%%%%%%


%%%%%%%%%%%%%%%%%%%%%%%%%%%%%%%%%%%%%%%%%%%%%%%%%%%%%%%%%%%%%%%%%%%%%%%%%%%
\subsubsection{Main Results.}
\label{sec:main}
%%%%%%%%%%%%%%%%%%%%%%%%%%%%%%%%%%%%%%%%%%%%%%%%%%%%%%%%%%%%%%%%%%%%%%%%%%%
Theorem~\ref{thm:main} below
gives sufficient conditions for the completeness of the system in \S\ref{sec:reft}
extended with \eqref{eq:compl:cc}.
%
This relies on Well-Filteredness (Proposition~\ref{prop:wf}),
but avoids the rule $\ax{WF}$.
Proofs of Lemma~\ref{lem:compl:nf} and Theorem~\ref{thm:main}
are given in Appendix \ref{sec:app:main}.
Motivations are discussed in \S\ref{sec:conc}.

%%%%%%%%%%%%%%%%%%%%%%%%%%%%%%%%%%%%%%%%%%%%%%%%%%%%%%%%%%%%%%%%%%%%%%%%%%%
\begin{lemma}
\label{lem:compl:nf}
%%%%%%%%%%%%%%%%%%%%%%%%%%%%%%%%%%%%%%%%%%%%%%%%%%%%%%%%%%%%%%%%%%%%%%%%%%%
Given $\varphi,\psi \in \Norm(\PT)$,
if $\I\psi \sle \I\varphi$,
then $\psi \thesis_\PT \varphi$
is derivable in the extension of Figure~\ref{fig:log:ded} (\S\ref{sec:log})
with \eqref{eq:compl:cc}.
%%%%%%%%%%%%%%%%%%%%%%%%%%%%%%%%%%%%%%%%%%%%%%%%%%%%%%%%%%%%%%%%%%%%%%%%%%%
\end{lemma}
%%%%%%%%%%%%%%%%%%%%%%%%%%%%%%%%%%%%%%%%%%%%%%%%%%%%%%%%%%%%%%%%%%%%%%%%%%%


%%%%%%%%%%%%%%%%%%%%%%%%%%%%%%%%%%%%%%%%%%%%%%%%%%%%%%%%%%%%%%%%%%%%%%%%%%%
\begin{definition}
%%%%%%%%%%%%%%%%%%%%%%%%%%%%%%%%%%%%%%%%%%%%%%%%%%%%%%%%%%%%%%%%%%%%%%%%%%%
A type is \emph{normal} if it is pure or 
$\reft{\PT \mid \varphi}$ with $\varphi \in \Norm(\PT)$.
A typing context $x_1:\RTbis_1,\dots,x_n : \RTbis_n$
is \emph{normal} if so are all $\RTbis_i$'s.

The \emph{first-order over normal forms} (\emph{fonf}) types
are generated by the grammar
\[
\begin{array}{r @{\ \ }c@{\ \ } l}
    \RT
&   \bnf
&   \RTbis
\gs \RT \times \RT
\gs \RTbis \arrow \RT
\end{array}
\]

\noindent
with $\RTbis$ normal.
%
A judgment $\Env \thesis M : \RT$ is \emph{normal}
if $\Env$ is normal and $\RT$ is fonf.
%%%%%%%%%%%%%%%%%%%%%%%%%%%%%%%%%%%%%%%%%%%%%%%%%%%%%%%%%%%%%%%%%%%%%%%%%%%
\end{definition}
%%%%%%%%%%%%%%%%%%%%%%%%%%%%%%%%%%%%%%%%%%%%%%%%%%%%%%%%%%%%%%%%%%%%%%%%%%%

We shall see that if $\Env \thesis M : \RT$ is sound and normal,
then it is derivable.
The idea is to reduce to the finite case (Theorem~\ref{thm:compl:fin})
by using Proposition~\ref{prop:wf},
but without using the rule $\ax{WF}$.
We first show that $\RT$ can be assumed to be normal.
To each normal judgment $\Env \thesis M : \RT$
we associate a set of normal judgments $\eta(\Env \thesis M : \RT)$.
We let
\(
  \eta\left(\Env \thesis M : \RT \right)
  \deq
  \left\{\Env \thesis M : \RT \right\}
\)
if $\RT$ is normal, and
\[
\begin{array}{r !{\quad\deq\quad} l}

  \eta\left(\Env \thesis M : \RT_1 \times \RT_2 \right)
& \eta\left(\Env \thesis \pi_1 M : \RT_1 \right)
  \cup
  \eta\left(\Env \thesis \pi_2 M : \RT_2 \right)
\\

  \eta\left(\Env \thesis M : \RTbis \arrow \RT \right)
& \eta\left( \Env, x: \RTbis \thesis M x : \RT \right)

\end{array}
\]

\noindent
Note that for each $(\Env' \thesis M' : \RT') \in \eta(\Env \thesis M : \RT)$,
the type $\RT'$ is normal.

%%%%%%%%%%%%%%%%%%%%%%%%%%%%%%%%%%%%%%%%%%%%%%%%%%%%%%%%%%%%%%%%%%%%%%%%%%%
\begin{proposition}
\label{prop:main:eta}
%%%%%%%%%%%%%%%%%%%%%%%%%%%%%%%%%%%%%%%%%%%%%%%%%%%%%%%%%%%%%%%%%%%%%%%%%%%
A normal judgment $\Env \thesis M : \RT$ is sound (resp.\ derivable)
if, and only if, so are all $(\Env' \thesis M' : \RT') \in \eta(\Env \thesis M : \RT)$.
%%%%%%%%%%%%%%%%%%%%%%%%%%%%%%%%%%%%%%%%%%%%%%%%%%%%%%%%%%%%%%%%%%%%%%%%%%%
\end{proposition}
%%%%%%%%%%%%%%%%%%%%%%%%%%%%%%%%%%%%%%%%%%%%%%%%%%%%%%%%%%%%%%%%%%%%%%%%%%%



%%%%%%%%%%%%%%%%%%%%%%%%%%%%%%%%%%%%%%%%%%%%%%%%%%%%%%%%%%%%%%%%%%%%%%%%%%%
\begin{theorem}[Main Result]
\label{thm:main}
%%%%%%%%%%%%%%%%%%%%%%%%%%%%%%%%%%%%%%%%%%%%%%%%%%%%%%%%%%%%%%%%%%%%%%%%%%%
If $\Env \thesis M : \RT$ is sound and normal
then $\Env \thesis M : \RT$ is derivable in the system of \S\ref{sec:reft}
extended with \eqref{eq:compl:cc}.
%%%%%%%%%%%%%%%%%%%%%%%%%%%%%%%%%%%%%%%%%%%%%%%%%%%%%%%%%%%%%%%%%%%%%%%%%%%
\end{theorem}
%%%%%%%%%%%%%%%%%%%%%%%%%%%%%%%%%%%%%%%%%%%%%%%%%%%%%%%%%%%%%%%%%%%%%%%%%%%


With the help of Examples \ref{ex:log:modalnf} and \ref{ex:log:distr},
the judgments for $\filter$, $\diag$
and $\bft$ in Table~\ref{tab:reft} (Example~\ref{ex:reft:fun})
can be assumed to be normal whenever so is $\varphi$.
Hence our Main Theorem~\ref{thm:main} applies and these
judgments are derivable 
in the system of \S\ref{sec:reft} extended with \eqref{eq:compl:cc},
but without the rule $\ax{WF}$.
This improves on \cite{jr21esop},
which does not handle $\filter$,
and which handles $\bft$ only when $\triangle$ is $\forall\Box$.

As for $\map$,
one has to assume that $\psi \in \Lang_\land$ (in addition to $\varphi \in \Norm$).
Recall from Lemma~\ref{lem:top:char} that
any formula is \emph{semantically} equivalent to a normal form.

%%%%%%%%%%%%%%%%%%%%%%%%%%%%%%%%%%%%%%%%%%%%%%%%%%%%%%%%%%%%%%%%%%%%%%%%%%%
\subsubsection{The General Case.}
\label{sec:compl:general}
%%%%%%%%%%%%%%%%%%%%%%%%%%%%%%%%%%%%%%%%%%%%%%%%%%%%%%%%%%%%%%%%%%%%%%%%%%%
Using $\ax{WF}$ and Example~\ref{ex:log:modalnf},
any formula is \emph{provably} equivalent to a $\psi \in \Norm$.
This yields the completeness result of Bonsangue \& Kok \cite{bk03ic}.

%%%%%%%%%%%%%%%%%%%%%%%%%%%%%%%%%%%%%%%%%%%%%%%%%%%%%%%%%%%%%%%%%%%%%%%%%%%
\begin{lemma}
\label{lem:compl:nf:wf}
%%%%%%%%%%%%%%%%%%%%%%%%%%%%%%%%%%%%%%%%%%%%%%%%%%%%%%%%%%%%%%%%%%%%%%%%%%%
For each $\varphi \in \Lang(\PT)$, there is a $\psi \in \Norm(\PT)$
such that $\varphi \thesisiff \psi$
in the extension of Figure~\ref{fig:log:ded} (\S\ref{sec:log})
with \eqref{eq:compl:cc} and $\ax{WF}$.
%%%%%%%%%%%%%%%%%%%%%%%%%%%%%%%%%%%%%%%%%%%%%%%%%%%%%%%%%%%%%%%%%%%%%%%%%%%
\end{lemma}
%%%%%%%%%%%%%%%%%%%%%%%%%%%%%%%%%%%%%%%%%%%%%%%%%%%%%%%%%%%%%%%%%%%%%%%%%%%


%%%%%%%%%%%%%%%%%%%%%%%%%%%%%%%%%%%%%%%%%%%%%%%%%%%%%%%%%%%%%%%%%%%%%%%%%%%
\begin{corollary}[Bonsangue \& Kok \cite{bk03ic}]
%%%%%%%%%%%%%%%%%%%%%%%%%%%%%%%%%%%%%%%%%%%%%%%%%%%%%%%%%%%%%%%%%%%%%%%%%%%
If $\Env \thesis M : \RT$ is sound 
then $\Env \thesis M : \RT$ is derivable in the system of \S\ref{sec:reft}
extended with \eqref{eq:compl:cc} and $\ax{WF}$.
%%%%%%%%%%%%%%%%%%%%%%%%%%%%%%%%%%%%%%%%%%%%%%%%%%%%%%%%%%%%%%%%%%%%%%%%%%%
\end{corollary}
%%%%%%%%%%%%%%%%%%%%%%%%%%%%%%%%%%%%%%%%%%%%%%%%%%%%%%%%%%%%%%%%%%%%%%%%%%%






\section{Conclusion}


\sdeni{}{We introduced convex-concave generative-adversarial characterization of inverse Nash equilibria for a large class of games, including normal-form, finite state and action Markov games, and a number of continuous state and action Markov games. This novel formulation then allowed us to obtain polynomial-time computation guarantees for inverse equilibria in these games, a rather surprising result since the computation of a Nash equilibrium is in general PPAD-complete. Our result can be thus seen as a positive computation result for game theory. We then extended our characterization to a multiagent apprenticeship learning setting, where we souught to not only rationalize the observed behavior as an inverse Nash equilibrium but also make predictions based off the inverse Nash equilibrium, and have shown in experiments on prices in Spanish electricity markets that our approach to solving multiagent apprenticship learning can be effective at predicting behavior in multiagent systems. The approach to inverse game theory that we provided in this paper is a highly flexible one and thus can be used to solve inverse equilibrium beyond inverse Nash equilibria and future work could explore ways to extend our approach to other game-theoretic settings and equilibrium concepts.}


\amy{discuss extenstion to other eqm concepts? CE, CCE, etc.}

\amy{and other idea from yesterday. check text thread?}




\appendix
\subsection{Lloyd-Max Algorithm}
\label{subsec:Lloyd-Max}
For a given quantization bitwidth $B$ and an operand $\bm{X}$, the Lloyd-Max algorithm finds $2^B$ quantization levels $\{\hat{x}_i\}_{i=1}^{2^B}$ such that quantizing $\bm{X}$ by rounding each scalar in $\bm{X}$ to the nearest quantization level minimizes the quantization MSE. 

The algorithm starts with an initial guess of quantization levels and then iteratively computes quantization thresholds $\{\tau_i\}_{i=1}^{2^B-1}$ and updates quantization levels $\{\hat{x}_i\}_{i=1}^{2^B}$. Specifically, at iteration $n$, thresholds are set to the midpoints of the previous iteration's levels:
\begin{align*}
    \tau_i^{(n)}=\frac{\hat{x}_i^{(n-1)}+\hat{x}_{i+1}^{(n-1)}}2 \text{ for } i=1\ldots 2^B-1
\end{align*}
Subsequently, the quantization levels are re-computed as conditional means of the data regions defined by the new thresholds:
\begin{align*}
    \hat{x}_i^{(n)}=\mathbb{E}\left[ \bm{X} \big| \bm{X}\in [\tau_{i-1}^{(n)},\tau_i^{(n)}] \right] \text{ for } i=1\ldots 2^B
\end{align*}
where to satisfy boundary conditions we have $\tau_0=-\infty$ and $\tau_{2^B}=\infty$. The algorithm iterates the above steps until convergence.

Figure \ref{fig:lm_quant} compares the quantization levels of a $7$-bit floating point (E3M3) quantizer (left) to a $7$-bit Lloyd-Max quantizer (right) when quantizing a layer of weights from the GPT3-126M model at a per-tensor granularity. As shown, the Lloyd-Max quantizer achieves substantially lower quantization MSE. Further, Table \ref{tab:FP7_vs_LM7} shows the superior perplexity achieved by Lloyd-Max quantizers for bitwidths of $7$, $6$ and $5$. The difference between the quantizers is clear at 5 bits, where per-tensor FP quantization incurs a drastic and unacceptable increase in perplexity, while Lloyd-Max quantization incurs a much smaller increase. Nevertheless, we note that even the optimal Lloyd-Max quantizer incurs a notable ($\sim 1.5$) increase in perplexity due to the coarse granularity of quantization. 

\begin{figure}[h]
  \centering
  \includegraphics[width=0.7\linewidth]{sections/figures/LM7_FP7.pdf}
  \caption{\small Quantization levels and the corresponding quantization MSE of Floating Point (left) vs Lloyd-Max (right) Quantizers for a layer of weights in the GPT3-126M model.}
  \label{fig:lm_quant}
\end{figure}

\begin{table}[h]\scriptsize
\begin{center}
\caption{\label{tab:FP7_vs_LM7} \small Comparing perplexity (lower is better) achieved by floating point quantizers and Lloyd-Max quantizers on a GPT3-126M model for the Wikitext-103 dataset.}
\begin{tabular}{c|cc|c}
\hline
 \multirow{2}{*}{\textbf{Bitwidth}} & \multicolumn{2}{|c|}{\textbf{Floating-Point Quantizer}} & \textbf{Lloyd-Max Quantizer} \\
 & Best Format & Wikitext-103 Perplexity & Wikitext-103 Perplexity \\
\hline
7 & E3M3 & 18.32 & 18.27 \\
6 & E3M2 & 19.07 & 18.51 \\
5 & E4M0 & 43.89 & 19.71 \\
\hline
\end{tabular}
\end{center}
\end{table}

\subsection{Proof of Local Optimality of LO-BCQ}
\label{subsec:lobcq_opt_proof}
For a given block $\bm{b}_j$, the quantization MSE during LO-BCQ can be empirically evaluated as $\frac{1}{L_b}\lVert \bm{b}_j- \bm{\hat{b}}_j\rVert^2_2$ where $\bm{\hat{b}}_j$ is computed from equation (\ref{eq:clustered_quantization_definition}) as $C_{f(\bm{b}_j)}(\bm{b}_j)$. Further, for a given block cluster $\mathcal{B}_i$, we compute the quantization MSE as $\frac{1}{|\mathcal{B}_{i}|}\sum_{\bm{b} \in \mathcal{B}_{i}} \frac{1}{L_b}\lVert \bm{b}- C_i^{(n)}(\bm{b})\rVert^2_2$. Therefore, at the end of iteration $n$, we evaluate the overall quantization MSE $J^{(n)}$ for a given operand $\bm{X}$ composed of $N_c$ block clusters as:
\begin{align*}
    \label{eq:mse_iter_n}
    J^{(n)} = \frac{1}{N_c} \sum_{i=1}^{N_c} \frac{1}{|\mathcal{B}_{i}^{(n)}|}\sum_{\bm{v} \in \mathcal{B}_{i}^{(n)}} \frac{1}{L_b}\lVert \bm{b}- B_i^{(n)}(\bm{b})\rVert^2_2
\end{align*}

At the end of iteration $n$, the codebooks are updated from $\mathcal{C}^{(n-1)}$ to $\mathcal{C}^{(n)}$. However, the mapping of a given vector $\bm{b}_j$ to quantizers $\mathcal{C}^{(n)}$ remains as  $f^{(n)}(\bm{b}_j)$. At the next iteration, during the vector clustering step, $f^{(n+1)}(\bm{b}_j)$ finds new mapping of $\bm{b}_j$ to updated codebooks $\mathcal{C}^{(n)}$ such that the quantization MSE over the candidate codebooks is minimized. Therefore, we obtain the following result for $\bm{b}_j$:
\begin{align*}
\frac{1}{L_b}\lVert \bm{b}_j - C_{f^{(n+1)}(\bm{b}_j)}^{(n)}(\bm{b}_j)\rVert^2_2 \le \frac{1}{L_b}\lVert \bm{b}_j - C_{f^{(n)}(\bm{b}_j)}^{(n)}(\bm{b}_j)\rVert^2_2
\end{align*}

That is, quantizing $\bm{b}_j$ at the end of the block clustering step of iteration $n+1$ results in lower quantization MSE compared to quantizing at the end of iteration $n$. Since this is true for all $\bm{b} \in \bm{X}$, we assert the following:
\begin{equation}
\begin{split}
\label{eq:mse_ineq_1}
    \tilde{J}^{(n+1)} &= \frac{1}{N_c} \sum_{i=1}^{N_c} \frac{1}{|\mathcal{B}_{i}^{(n+1)}|}\sum_{\bm{b} \in \mathcal{B}_{i}^{(n+1)}} \frac{1}{L_b}\lVert \bm{b} - C_i^{(n)}(b)\rVert^2_2 \le J^{(n)}
\end{split}
\end{equation}
where $\tilde{J}^{(n+1)}$ is the the quantization MSE after the vector clustering step at iteration $n+1$.

Next, during the codebook update step (\ref{eq:quantizers_update}) at iteration $n+1$, the per-cluster codebooks $\mathcal{C}^{(n)}$ are updated to $\mathcal{C}^{(n+1)}$ by invoking the Lloyd-Max algorithm \citep{Lloyd}. We know that for any given value distribution, the Lloyd-Max algorithm minimizes the quantization MSE. Therefore, for a given vector cluster $\mathcal{B}_i$ we obtain the following result:

\begin{equation}
    \frac{1}{|\mathcal{B}_{i}^{(n+1)}|}\sum_{\bm{b} \in \mathcal{B}_{i}^{(n+1)}} \frac{1}{L_b}\lVert \bm{b}- C_i^{(n+1)}(\bm{b})\rVert^2_2 \le \frac{1}{|\mathcal{B}_{i}^{(n+1)}|}\sum_{\bm{b} \in \mathcal{B}_{i}^{(n+1)}} \frac{1}{L_b}\lVert \bm{b}- C_i^{(n)}(\bm{b})\rVert^2_2
\end{equation}

The above equation states that quantizing the given block cluster $\mathcal{B}_i$ after updating the associated codebook from $C_i^{(n)}$ to $C_i^{(n+1)}$ results in lower quantization MSE. Since this is true for all the block clusters, we derive the following result: 
\begin{equation}
\begin{split}
\label{eq:mse_ineq_2}
     J^{(n+1)} &= \frac{1}{N_c} \sum_{i=1}^{N_c} \frac{1}{|\mathcal{B}_{i}^{(n+1)}|}\sum_{\bm{b} \in \mathcal{B}_{i}^{(n+1)}} \frac{1}{L_b}\lVert \bm{b}- C_i^{(n+1)}(\bm{b})\rVert^2_2  \le \tilde{J}^{(n+1)}   
\end{split}
\end{equation}

Following (\ref{eq:mse_ineq_1}) and (\ref{eq:mse_ineq_2}), we find that the quantization MSE is non-increasing for each iteration, that is, $J^{(1)} \ge J^{(2)} \ge J^{(3)} \ge \ldots \ge J^{(M)}$ where $M$ is the maximum number of iterations. 
%Therefore, we can say that if the algorithm converges, then it must be that it has converged to a local minimum. 
\hfill $\blacksquare$


\begin{figure}
    \begin{center}
    \includegraphics[width=0.5\textwidth]{sections//figures/mse_vs_iter.pdf}
    \end{center}
    \caption{\small NMSE vs iterations during LO-BCQ compared to other block quantization proposals}
    \label{fig:nmse_vs_iter}
\end{figure}

Figure \ref{fig:nmse_vs_iter} shows the empirical convergence of LO-BCQ across several block lengths and number of codebooks. Also, the MSE achieved by LO-BCQ is compared to baselines such as MXFP and VSQ. As shown, LO-BCQ converges to a lower MSE than the baselines. Further, we achieve better convergence for larger number of codebooks ($N_c$) and for a smaller block length ($L_b$), both of which increase the bitwidth of BCQ (see Eq \ref{eq:bitwidth_bcq}).


\subsection{Additional Accuracy Results}
%Table \ref{tab:lobcq_config} lists the various LOBCQ configurations and their corresponding bitwidths.
\begin{table}
\setlength{\tabcolsep}{4.75pt}
\begin{center}
\caption{\label{tab:lobcq_config} Various LO-BCQ configurations and their bitwidths.}
\begin{tabular}{|c||c|c|c|c||c|c||c|} 
\hline
 & \multicolumn{4}{|c||}{$L_b=8$} & \multicolumn{2}{|c||}{$L_b=4$} & $L_b=2$ \\
 \hline
 \backslashbox{$L_A$\kern-1em}{\kern-1em$N_c$} & 2 & 4 & 8 & 16 & 2 & 4 & 2 \\
 \hline
 64 & 4.25 & 4.375 & 4.5 & 4.625 & 4.375 & 4.625 & 4.625\\
 \hline
 32 & 4.375 & 4.5 & 4.625& 4.75 & 4.5 & 4.75 & 4.75 \\
 \hline
 16 & 4.625 & 4.75& 4.875 & 5 & 4.75 & 5 & 5 \\
 \hline
\end{tabular}
\end{center}
\end{table}

%\subsection{Perplexity achieved by various LO-BCQ configurations on Wikitext-103 dataset}

\begin{table} \centering
\begin{tabular}{|c||c|c|c|c||c|c||c|} 
\hline
 $L_b \rightarrow$& \multicolumn{4}{c||}{8} & \multicolumn{2}{c||}{4} & 2\\
 \hline
 \backslashbox{$L_A$\kern-1em}{\kern-1em$N_c$} & 2 & 4 & 8 & 16 & 2 & 4 & 2  \\
 %$N_c \rightarrow$ & 2 & 4 & 8 & 16 & 2 & 4 & 2 \\
 \hline
 \hline
 \multicolumn{8}{c}{GPT3-1.3B (FP32 PPL = 9.98)} \\ 
 \hline
 \hline
 64 & 10.40 & 10.23 & 10.17 & 10.15 &  10.28 & 10.18 & 10.19 \\
 \hline
 32 & 10.25 & 10.20 & 10.15 & 10.12 &  10.23 & 10.17 & 10.17 \\
 \hline
 16 & 10.22 & 10.16 & 10.10 & 10.09 &  10.21 & 10.14 & 10.16 \\
 \hline
  \hline
 \multicolumn{8}{c}{GPT3-8B (FP32 PPL = 7.38)} \\ 
 \hline
 \hline
 64 & 7.61 & 7.52 & 7.48 &  7.47 &  7.55 &  7.49 & 7.50 \\
 \hline
 32 & 7.52 & 7.50 & 7.46 &  7.45 &  7.52 &  7.48 & 7.48  \\
 \hline
 16 & 7.51 & 7.48 & 7.44 &  7.44 &  7.51 &  7.49 & 7.47  \\
 \hline
\end{tabular}
\caption{\label{tab:ppl_gpt3_abalation} Wikitext-103 perplexity across GPT3-1.3B and 8B models.}
\end{table}

\begin{table} \centering
\begin{tabular}{|c||c|c|c|c||} 
\hline
 $L_b \rightarrow$& \multicolumn{4}{c||}{8}\\
 \hline
 \backslashbox{$L_A$\kern-1em}{\kern-1em$N_c$} & 2 & 4 & 8 & 16 \\
 %$N_c \rightarrow$ & 2 & 4 & 8 & 16 & 2 & 4 & 2 \\
 \hline
 \hline
 \multicolumn{5}{|c|}{Llama2-7B (FP32 PPL = 5.06)} \\ 
 \hline
 \hline
 64 & 5.31 & 5.26 & 5.19 & 5.18  \\
 \hline
 32 & 5.23 & 5.25 & 5.18 & 5.15  \\
 \hline
 16 & 5.23 & 5.19 & 5.16 & 5.14  \\
 \hline
 \multicolumn{5}{|c|}{Nemotron4-15B (FP32 PPL = 5.87)} \\ 
 \hline
 \hline
 64  & 6.3 & 6.20 & 6.13 & 6.08  \\
 \hline
 32  & 6.24 & 6.12 & 6.07 & 6.03  \\
 \hline
 16  & 6.12 & 6.14 & 6.04 & 6.02  \\
 \hline
 \multicolumn{5}{|c|}{Nemotron4-340B (FP32 PPL = 3.48)} \\ 
 \hline
 \hline
 64 & 3.67 & 3.62 & 3.60 & 3.59 \\
 \hline
 32 & 3.63 & 3.61 & 3.59 & 3.56 \\
 \hline
 16 & 3.61 & 3.58 & 3.57 & 3.55 \\
 \hline
\end{tabular}
\caption{\label{tab:ppl_llama7B_nemo15B} Wikitext-103 perplexity compared to FP32 baseline in Llama2-7B and Nemotron4-15B, 340B models}
\end{table}

%\subsection{Perplexity achieved by various LO-BCQ configurations on MMLU dataset}


\begin{table} \centering
\begin{tabular}{|c||c|c|c|c||c|c|c|c|} 
\hline
 $L_b \rightarrow$& \multicolumn{4}{c||}{8} & \multicolumn{4}{c||}{8}\\
 \hline
 \backslashbox{$L_A$\kern-1em}{\kern-1em$N_c$} & 2 & 4 & 8 & 16 & 2 & 4 & 8 & 16  \\
 %$N_c \rightarrow$ & 2 & 4 & 8 & 16 & 2 & 4 & 2 \\
 \hline
 \hline
 \multicolumn{5}{|c|}{Llama2-7B (FP32 Accuracy = 45.8\%)} & \multicolumn{4}{|c|}{Llama2-70B (FP32 Accuracy = 69.12\%)} \\ 
 \hline
 \hline
 64 & 43.9 & 43.4 & 43.9 & 44.9 & 68.07 & 68.27 & 68.17 & 68.75 \\
 \hline
 32 & 44.5 & 43.8 & 44.9 & 44.5 & 68.37 & 68.51 & 68.35 & 68.27  \\
 \hline
 16 & 43.9 & 42.7 & 44.9 & 45 & 68.12 & 68.77 & 68.31 & 68.59  \\
 \hline
 \hline
 \multicolumn{5}{|c|}{GPT3-22B (FP32 Accuracy = 38.75\%)} & \multicolumn{4}{|c|}{Nemotron4-15B (FP32 Accuracy = 64.3\%)} \\ 
 \hline
 \hline
 64 & 36.71 & 38.85 & 38.13 & 38.92 & 63.17 & 62.36 & 63.72 & 64.09 \\
 \hline
 32 & 37.95 & 38.69 & 39.45 & 38.34 & 64.05 & 62.30 & 63.8 & 64.33  \\
 \hline
 16 & 38.88 & 38.80 & 38.31 & 38.92 & 63.22 & 63.51 & 63.93 & 64.43  \\
 \hline
\end{tabular}
\caption{\label{tab:mmlu_abalation} Accuracy on MMLU dataset across GPT3-22B, Llama2-7B, 70B and Nemotron4-15B models.}
\end{table}


%\subsection{Perplexity achieved by various LO-BCQ configurations on LM evaluation harness}

\begin{table} \centering
\begin{tabular}{|c||c|c|c|c||c|c|c|c|} 
\hline
 $L_b \rightarrow$& \multicolumn{4}{c||}{8} & \multicolumn{4}{c||}{8}\\
 \hline
 \backslashbox{$L_A$\kern-1em}{\kern-1em$N_c$} & 2 & 4 & 8 & 16 & 2 & 4 & 8 & 16  \\
 %$N_c \rightarrow$ & 2 & 4 & 8 & 16 & 2 & 4 & 2 \\
 \hline
 \hline
 \multicolumn{5}{|c|}{Race (FP32 Accuracy = 37.51\%)} & \multicolumn{4}{|c|}{Boolq (FP32 Accuracy = 64.62\%)} \\ 
 \hline
 \hline
 64 & 36.94 & 37.13 & 36.27 & 37.13 & 63.73 & 62.26 & 63.49 & 63.36 \\
 \hline
 32 & 37.03 & 36.36 & 36.08 & 37.03 & 62.54 & 63.51 & 63.49 & 63.55  \\
 \hline
 16 & 37.03 & 37.03 & 36.46 & 37.03 & 61.1 & 63.79 & 63.58 & 63.33  \\
 \hline
 \hline
 \multicolumn{5}{|c|}{Winogrande (FP32 Accuracy = 58.01\%)} & \multicolumn{4}{|c|}{Piqa (FP32 Accuracy = 74.21\%)} \\ 
 \hline
 \hline
 64 & 58.17 & 57.22 & 57.85 & 58.33 & 73.01 & 73.07 & 73.07 & 72.80 \\
 \hline
 32 & 59.12 & 58.09 & 57.85 & 58.41 & 73.01 & 73.94 & 72.74 & 73.18  \\
 \hline
 16 & 57.93 & 58.88 & 57.93 & 58.56 & 73.94 & 72.80 & 73.01 & 73.94  \\
 \hline
\end{tabular}
\caption{\label{tab:mmlu_abalation} Accuracy on LM evaluation harness tasks on GPT3-1.3B model.}
\end{table}

\begin{table} \centering
\begin{tabular}{|c||c|c|c|c||c|c|c|c|} 
\hline
 $L_b \rightarrow$& \multicolumn{4}{c||}{8} & \multicolumn{4}{c||}{8}\\
 \hline
 \backslashbox{$L_A$\kern-1em}{\kern-1em$N_c$} & 2 & 4 & 8 & 16 & 2 & 4 & 8 & 16  \\
 %$N_c \rightarrow$ & 2 & 4 & 8 & 16 & 2 & 4 & 2 \\
 \hline
 \hline
 \multicolumn{5}{|c|}{Race (FP32 Accuracy = 41.34\%)} & \multicolumn{4}{|c|}{Boolq (FP32 Accuracy = 68.32\%)} \\ 
 \hline
 \hline
 64 & 40.48 & 40.10 & 39.43 & 39.90 & 69.20 & 68.41 & 69.45 & 68.56 \\
 \hline
 32 & 39.52 & 39.52 & 40.77 & 39.62 & 68.32 & 67.43 & 68.17 & 69.30  \\
 \hline
 16 & 39.81 & 39.71 & 39.90 & 40.38 & 68.10 & 66.33 & 69.51 & 69.42  \\
 \hline
 \hline
 \multicolumn{5}{|c|}{Winogrande (FP32 Accuracy = 67.88\%)} & \multicolumn{4}{|c|}{Piqa (FP32 Accuracy = 78.78\%)} \\ 
 \hline
 \hline
 64 & 66.85 & 66.61 & 67.72 & 67.88 & 77.31 & 77.42 & 77.75 & 77.64 \\
 \hline
 32 & 67.25 & 67.72 & 67.72 & 67.00 & 77.31 & 77.04 & 77.80 & 77.37  \\
 \hline
 16 & 68.11 & 68.90 & 67.88 & 67.48 & 77.37 & 78.13 & 78.13 & 77.69  \\
 \hline
\end{tabular}
\caption{\label{tab:mmlu_abalation} Accuracy on LM evaluation harness tasks on GPT3-8B model.}
\end{table}

\begin{table} \centering
\begin{tabular}{|c||c|c|c|c||c|c|c|c|} 
\hline
 $L_b \rightarrow$& \multicolumn{4}{c||}{8} & \multicolumn{4}{c||}{8}\\
 \hline
 \backslashbox{$L_A$\kern-1em}{\kern-1em$N_c$} & 2 & 4 & 8 & 16 & 2 & 4 & 8 & 16  \\
 %$N_c \rightarrow$ & 2 & 4 & 8 & 16 & 2 & 4 & 2 \\
 \hline
 \hline
 \multicolumn{5}{|c|}{Race (FP32 Accuracy = 40.67\%)} & \multicolumn{4}{|c|}{Boolq (FP32 Accuracy = 76.54\%)} \\ 
 \hline
 \hline
 64 & 40.48 & 40.10 & 39.43 & 39.90 & 75.41 & 75.11 & 77.09 & 75.66 \\
 \hline
 32 & 39.52 & 39.52 & 40.77 & 39.62 & 76.02 & 76.02 & 75.96 & 75.35  \\
 \hline
 16 & 39.81 & 39.71 & 39.90 & 40.38 & 75.05 & 73.82 & 75.72 & 76.09  \\
 \hline
 \hline
 \multicolumn{5}{|c|}{Winogrande (FP32 Accuracy = 70.64\%)} & \multicolumn{4}{|c|}{Piqa (FP32 Accuracy = 79.16\%)} \\ 
 \hline
 \hline
 64 & 69.14 & 70.17 & 70.17 & 70.56 & 78.24 & 79.00 & 78.62 & 78.73 \\
 \hline
 32 & 70.96 & 69.69 & 71.27 & 69.30 & 78.56 & 79.49 & 79.16 & 78.89  \\
 \hline
 16 & 71.03 & 69.53 & 69.69 & 70.40 & 78.13 & 79.16 & 79.00 & 79.00  \\
 \hline
\end{tabular}
\caption{\label{tab:mmlu_abalation} Accuracy on LM evaluation harness tasks on GPT3-22B model.}
\end{table}

\begin{table} \centering
\begin{tabular}{|c||c|c|c|c||c|c|c|c|} 
\hline
 $L_b \rightarrow$& \multicolumn{4}{c||}{8} & \multicolumn{4}{c||}{8}\\
 \hline
 \backslashbox{$L_A$\kern-1em}{\kern-1em$N_c$} & 2 & 4 & 8 & 16 & 2 & 4 & 8 & 16  \\
 %$N_c \rightarrow$ & 2 & 4 & 8 & 16 & 2 & 4 & 2 \\
 \hline
 \hline
 \multicolumn{5}{|c|}{Race (FP32 Accuracy = 44.4\%)} & \multicolumn{4}{|c|}{Boolq (FP32 Accuracy = 79.29\%)} \\ 
 \hline
 \hline
 64 & 42.49 & 42.51 & 42.58 & 43.45 & 77.58 & 77.37 & 77.43 & 78.1 \\
 \hline
 32 & 43.35 & 42.49 & 43.64 & 43.73 & 77.86 & 75.32 & 77.28 & 77.86  \\
 \hline
 16 & 44.21 & 44.21 & 43.64 & 42.97 & 78.65 & 77 & 76.94 & 77.98  \\
 \hline
 \hline
 \multicolumn{5}{|c|}{Winogrande (FP32 Accuracy = 69.38\%)} & \multicolumn{4}{|c|}{Piqa (FP32 Accuracy = 78.07\%)} \\ 
 \hline
 \hline
 64 & 68.9 & 68.43 & 69.77 & 68.19 & 77.09 & 76.82 & 77.09 & 77.86 \\
 \hline
 32 & 69.38 & 68.51 & 68.82 & 68.90 & 78.07 & 76.71 & 78.07 & 77.86  \\
 \hline
 16 & 69.53 & 67.09 & 69.38 & 68.90 & 77.37 & 77.8 & 77.91 & 77.69  \\
 \hline
\end{tabular}
\caption{\label{tab:mmlu_abalation} Accuracy on LM evaluation harness tasks on Llama2-7B model.}
\end{table}

\begin{table} \centering
\begin{tabular}{|c||c|c|c|c||c|c|c|c|} 
\hline
 $L_b \rightarrow$& \multicolumn{4}{c||}{8} & \multicolumn{4}{c||}{8}\\
 \hline
 \backslashbox{$L_A$\kern-1em}{\kern-1em$N_c$} & 2 & 4 & 8 & 16 & 2 & 4 & 8 & 16  \\
 %$N_c \rightarrow$ & 2 & 4 & 8 & 16 & 2 & 4 & 2 \\
 \hline
 \hline
 \multicolumn{5}{|c|}{Race (FP32 Accuracy = 48.8\%)} & \multicolumn{4}{|c|}{Boolq (FP32 Accuracy = 85.23\%)} \\ 
 \hline
 \hline
 64 & 49.00 & 49.00 & 49.28 & 48.71 & 82.82 & 84.28 & 84.03 & 84.25 \\
 \hline
 32 & 49.57 & 48.52 & 48.33 & 49.28 & 83.85 & 84.46 & 84.31 & 84.93  \\
 \hline
 16 & 49.85 & 49.09 & 49.28 & 48.99 & 85.11 & 84.46 & 84.61 & 83.94  \\
 \hline
 \hline
 \multicolumn{5}{|c|}{Winogrande (FP32 Accuracy = 79.95\%)} & \multicolumn{4}{|c|}{Piqa (FP32 Accuracy = 81.56\%)} \\ 
 \hline
 \hline
 64 & 78.77 & 78.45 & 78.37 & 79.16 & 81.45 & 80.69 & 81.45 & 81.5 \\
 \hline
 32 & 78.45 & 79.01 & 78.69 & 80.66 & 81.56 & 80.58 & 81.18 & 81.34  \\
 \hline
 16 & 79.95 & 79.56 & 79.79 & 79.72 & 81.28 & 81.66 & 81.28 & 80.96  \\
 \hline
\end{tabular}
\caption{\label{tab:mmlu_abalation} Accuracy on LM evaluation harness tasks on Llama2-70B model.}
\end{table}

%\section{MSE Studies}
%\textcolor{red}{TODO}


\subsection{Number Formats and Quantization Method}
\label{subsec:numFormats_quantMethod}
\subsubsection{Integer Format}
An $n$-bit signed integer (INT) is typically represented with a 2s-complement format \citep{yao2022zeroquant,xiao2023smoothquant,dai2021vsq}, where the most significant bit denotes the sign.

\subsubsection{Floating Point Format}
An $n$-bit signed floating point (FP) number $x$ comprises of a 1-bit sign ($x_{\mathrm{sign}}$), $B_m$-bit mantissa ($x_{\mathrm{mant}}$) and $B_e$-bit exponent ($x_{\mathrm{exp}}$) such that $B_m+B_e=n-1$. The associated constant exponent bias ($E_{\mathrm{bias}}$) is computed as $(2^{{B_e}-1}-1)$. We denote this format as $E_{B_e}M_{B_m}$.  

\subsubsection{Quantization Scheme}
\label{subsec:quant_method}
A quantization scheme dictates how a given unquantized tensor is converted to its quantized representation. We consider FP formats for the purpose of illustration. Given an unquantized tensor $\bm{X}$ and an FP format $E_{B_e}M_{B_m}$, we first, we compute the quantization scale factor $s_X$ that maps the maximum absolute value of $\bm{X}$ to the maximum quantization level of the $E_{B_e}M_{B_m}$ format as follows:
\begin{align}
\label{eq:sf}
    s_X = \frac{\mathrm{max}(|\bm{X}|)}{\mathrm{max}(E_{B_e}M_{B_m})}
\end{align}
In the above equation, $|\cdot|$ denotes the absolute value function.

Next, we scale $\bm{X}$ by $s_X$ and quantize it to $\hat{\bm{X}}$ by rounding it to the nearest quantization level of $E_{B_e}M_{B_m}$ as:

\begin{align}
\label{eq:tensor_quant}
    \hat{\bm{X}} = \text{round-to-nearest}\left(\frac{\bm{X}}{s_X}, E_{B_e}M_{B_m}\right)
\end{align}

We perform dynamic max-scaled quantization \citep{wu2020integer}, where the scale factor $s$ for activations is dynamically computed during runtime.

\subsection{Vector Scaled Quantization}
\begin{wrapfigure}{r}{0.35\linewidth}
  \centering
  \includegraphics[width=\linewidth]{sections/figures/vsquant.jpg}
  \caption{\small Vectorwise decomposition for per-vector scaled quantization (VSQ \citep{dai2021vsq}).}
  \label{fig:vsquant}
\end{wrapfigure}
During VSQ \citep{dai2021vsq}, the operand tensors are decomposed into 1D vectors in a hardware friendly manner as shown in Figure \ref{fig:vsquant}. Since the decomposed tensors are used as operands in matrix multiplications during inference, it is beneficial to perform this decomposition along the reduction dimension of the multiplication. The vectorwise quantization is performed similar to tensorwise quantization described in Equations \ref{eq:sf} and \ref{eq:tensor_quant}, where a scale factor $s_v$ is required for each vector $\bm{v}$ that maps the maximum absolute value of that vector to the maximum quantization level. While smaller vector lengths can lead to larger accuracy gains, the associated memory and computational overheads due to the per-vector scale factors increases. To alleviate these overheads, VSQ \citep{dai2021vsq} proposed a second level quantization of the per-vector scale factors to unsigned integers, while MX \citep{rouhani2023shared} quantizes them to integer powers of 2 (denoted as $2^{INT}$).

\subsubsection{MX Format}
The MX format proposed in \citep{rouhani2023microscaling} introduces the concept of sub-block shifting. For every two scalar elements of $b$-bits each, there is a shared exponent bit. The value of this exponent bit is determined through an empirical analysis that targets minimizing quantization MSE. We note that the FP format $E_{1}M_{b}$ is strictly better than MX from an accuracy perspective since it allocates a dedicated exponent bit to each scalar as opposed to sharing it across two scalars. Therefore, we conservatively bound the accuracy of a $b+2$-bit signed MX format with that of a $E_{1}M_{b}$ format in our comparisons. For instance, we use E1M2 format as a proxy for MX4.

\begin{figure}
    \centering
    \includegraphics[width=1\linewidth]{sections//figures/BlockFormats.pdf}
    \caption{\small Comparing LO-BCQ to MX format.}
    \label{fig:block_formats}
\end{figure}

Figure \ref{fig:block_formats} compares our $4$-bit LO-BCQ block format to MX \citep{rouhani2023microscaling}. As shown, both LO-BCQ and MX decompose a given operand tensor into block arrays and each block array into blocks. Similar to MX, we find that per-block quantization ($L_b < L_A$) leads to better accuracy due to increased flexibility. While MX achieves this through per-block $1$-bit micro-scales, we associate a dedicated codebook to each block through a per-block codebook selector. Further, MX quantizes the per-block array scale-factor to E8M0 format without per-tensor scaling. In contrast during LO-BCQ, we find that per-tensor scaling combined with quantization of per-block array scale-factor to E4M3 format results in superior inference accuracy across models. 


% ---- Bibliography ----
%
% BibTeX users should specify bibliography style 'splncs04'.
% References will then be sorted and formatted in the correct style.
\bibliographystyle{splncs04}
\bibliography{bibliographie}


\opt{full}{\newpage}

\opt{full}{%%%%%%%%%%%%%%%%%%%%%%%%%%%%%%%%%%%%%%%%%%%%%%%%%%%%%%%%%%%%%%%%%%%%%%%%%%%
\section{Proofs of \S\ref{sec:system} (\nameref{sec:system})}
%%%%%%%%%%%%%%%%%%%%%%%%%%%%%%%%%%%%%%%%%%%%%%%%%%%%%%%%%%%%%%%%%%%%%%%%%%%

%%%%%%%%%%%%%%%%%%%%%%%%%%%%%%%%%%%%%%%%%%%%%%%%%%%%%%%%%%%%%%%%%%%%%%%%%%%
\subsection{Proofs of \S\ref{sec:log} (\nameref{sec:log})}
\label{sec:proof:log}
%%%%%%%%%%%%%%%%%%%%%%%%%%%%%%%%%%%%%%%%%%%%%%%%%%%%%%%%%%%%%%%%%%%%%%%%%%%

%%%%%%%%%%%%%%%%%%%%%%%%%%%%%%%%%%%%%%%%%%%%%%%%%%%%%%%%%%%%%%%%%%%%%%%%%%%
\begin{figure}[t!]
%%%%%%%%%%%%%%%%%%%%%%%%%%%%%%%%%%%%%%%%%%%%%%%%%%%%%%%%%%%%%%%%%%%%%%%%%%%
\[
\begin{array}{c}

\dfrac{\text{for each $i \in I$, $\psi_i \,\thesis\, \varphi_i$}}
  {\bigwedge_{i \in I} \psi_i \,\thesis\, \bigwedge_{i \in I}\varphi_i}

\qquad\qquad

\dfrac{\text{for each $i \in I$, $\psi_i \thesis \varphi_i$}}
  {\bigvee_{i \in I} \psi_i \thesis \bigvee_{i \in I}\varphi_i}

\\\\

  \form\triangle \bigwedge_{i \in I} \varphi_i
  \,\thesisiff\,
  \bigwedge_{i \in I} \form\triangle \varphi_i

\qquad\qquad

  \bigvee_{i \in I} \form\triangle\varphi_i
  \,\thesisiff\,
  \form\triangle \bigvee_{i \in I}\varphi_i

\\\\

  \bigwedge_{i \in I} \bigvee_{j \in J_i} \varphi_{i,j}
  \,\thesisiff\,
  \bigvee_{f \in \prod_{i \in I}J_i}\bigwedge_{i \in I} \varphi_{i,f(i)}

\\\\

  \bigwedge_{f \in \prod_{i \in I}J_i}\bigvee_{i \in I} \varphi_{i,f(i)}
  \,\thesisiff\,
  \bigvee_{i \in I}\bigwedge_{j \in J_i}\varphi_{i,j}
\end{array}
\]
\caption{Some derivable rules and sequents,
where $\triangle$ is either $\pi_1$, $\pi_2$ or $\fold$.%
\label{fig:proof:log:derivable}}
%%%%%%%%%%%%%%%%%%%%%%%%%%%%%%%%%%%%%%%%%%%%%%%%%%%%%%%%%%%%%%%%%%%%%%%%%%%
\end{figure}
%%%%%%%%%%%%%%%%%%%%%%%%%%%%%%%%%%%%%%%%%%%%%%%%%%%%%%%%%%%%%%%%%%%%%%%%%%%

\noindent
In this Appendix~\ref{sec:proof:log},
we give details on Figure~\ref{fig:proof:log:derivable},
which gathers some derivable rule and sequents
(including those of Examples~\ref{ex:log:modalnf} and~\ref{ex:log:distr}).


%%%%%%%%%%%%%%%%%%%%%%%%%%%%%%%%%%%%%%%%%%%%%%%%%%%%%%%%%%%%%%%%%%%%%%%%%%%
\begin{lemma}
\label{lem:proof:log:functprop}
%%%%%%%%%%%%%%%%%%%%%%%%%%%%%%%%%%%%%%%%%%%%%%%%%%%%%%%%%%%%%%%%%%%%%%%%%%%
The following rules are derivable
\[
\begin{array}{c}

\dfrac{\text{for each $i \in I$, $\psi_i \thesis \varphi_i$}}
  {\bigwedge_{i \in I} \psi_i \thesis \bigwedge_{i \in I}\varphi_i}

\qquad\qquad

\dfrac{\text{for each $i \in I$, $\psi_i \thesis \varphi_i$}}
  {\bigvee_{i \in I} \psi_i \thesis \bigvee_{i \in I}\varphi_i}

\end{array}
\]
%%%%%%%%%%%%%%%%%%%%%%%%%%%%%%%%%%%%%%%%%%%%%%%%%%%%%%%%%%%%%%%%%%%%%%%%%%%
\end{lemma}
%%%%%%%%%%%%%%%%%%%%%%%%%%%%%%%%%%%%%%%%%%%%%%%%%%%%%%%%%%%%%%%%%%%%%%%%%%%

%%%%%%%%%%%%%%%%%%%%%%%%%%%%%%%%%%%%%%%%%%%%%%%%%%%%%%%%%%%%%%%%%%%%%%%%%%%
\begin{proof}
%%%%%%%%%%%%%%%%%%%%%%%%%%%%%%%%%%%%%%%%%%%%%%%%%%%%%%%%%%%%%%%%%%%%%%%%%%%
The premise of the first rule
yields
$\bigwedge_{i \in I} \psi_i \thesis \varphi_i$
for all $i \in I$, from which we obtain
$\bigwedge_{i \in I} \psi_i \thesis \bigwedge_{i \in I} \varphi_i$.
The second rule is handled similarly.
%%%%%%%%%%%%%%%%%%%%%%%%%%%%%%%%%%%%%%%%%%%%%%%%%%%%%%%%%%%%%%%%%%%%%%%%%%%
\end{proof}
%%%%%%%%%%%%%%%%%%%%%%%%%%%%%%%%%%%%%%%%%%%%%%%%%%%%%%%%%%%%%%%%%%%%%%%%%%%




%%%%%%%%%%%%%%%%%%%%%%%%%%%%%%%%%%%%%%%%%%%%%%%%%%%%%%%%%%%%%%%%%%%%%%%%%%%
\begin{lemma}
%%%%%%%%%%%%%%%%%%%%%%%%%%%%%%%%%%%%%%%%%%%%%%%%%%%%%%%%%%%%%%%%%%%%%%%%%%%
The following sequents are derivable,
where $\triangle$ is either $\pi_1$, $\pi_2$ or $\fold$:
\[
\begin{array}{c}

  \form\triangle \bigwedge_{i \in I} \varphi_i
  \,\thesis\,
  \bigwedge_{i \in I} \form\triangle \varphi_i

\qquad\text{and}\qquad

  \bigvee_{i \in I} \form\triangle\varphi_i
  \,\thesis\,
  \form\triangle \bigvee_{i \in I}\varphi_i

\end{array}
\]
%%%%%%%%%%%%%%%%%%%%%%%%%%%%%%%%%%%%%%%%%%%%%%%%%%%%%%%%%%%%%%%%%%%%%%%%%%%
\end{lemma}
%%%%%%%%%%%%%%%%%%%%%%%%%%%%%%%%%%%%%%%%%%%%%%%%%%%%%%%%%%%%%%%%%%%%%%%%%%%

%%%%%%%%%%%%%%%%%%%%%%%%%%%%%%%%%%%%%%%%%%%%%%%%%%%%%%%%%%%%%%%%%%%%%%%%%%%
\begin{proof}
%%%%%%%%%%%%%%%%%%%%%%%%%%%%%%%%%%%%%%%%%%%%%%%%%%%%%%%%%%%%%%%%%%%%%%%%%%%
For each $i \in I$ we have
\begin{center}
\AXC{}
\UIC{$\varphi_i \,\thesis\, \varphi_i$}
\UIC{$\bigwedge_{i \in I} \varphi_i \,\thesis\, \varphi_i$}
\UIC{$\form\triangle \bigwedge_{i \in I} \varphi_i \,\thesis\, \form\triangle \varphi_i$}
\DisplayProof
\end{center}

\noindent
from which we obtain the first sequent.
The other one is derived similarly.
\qed
%%%%%%%%%%%%%%%%%%%%%%%%%%%%%%%%%%%%%%%%%%%%%%%%%%%%%%%%%%%%%%%%%%%%%%%%%%%
\end{proof}
%%%%%%%%%%%%%%%%%%%%%%%%%%%%%%%%%%%%%%%%%%%%%%%%%%%%%%%%%%%%%%%%%%%%%%%%%%%



%%%%%%%%%%%%%%%%%%%%%%%%%%%%%%%%%%%%%%%%%%%%%%%%%%%%%%%%%%%%%%%%%%%%%%%%%%%
\begin{lemma}
%%%%%%%%%%%%%%%%%%%%%%%%%%%%%%%%%%%%%%%%%%%%%%%%%%%%%%%%%%%%%%%%%%%%%%%%%%%
The following sequents are derivable
\[
\begin{array}{r !{~}l!{~} l}
  \bigvee_{f \in \prod_{i \in I}J_i}\bigwedge_{i \in I} \varphi_{i,f(i)}
& \thesis
& \bigwedge_{i \in I} \bigvee_{j \in J_i} \varphi_{i,j}
\\

  \bigvee_{i \in I}\bigwedge_{j \in J_i}\varphi_{i,j}
& \thesis
& \bigwedge_{f \in \prod_{i \in I}J_i} \bigvee_{i \in I} \varphi_{i,f(i)}
\end{array}
\]
%%%%%%%%%%%%%%%%%%%%%%%%%%%%%%%%%%%%%%%%%%%%%%%%%%%%%%%%%%%%%%%%%%%%%%%%%%%
\end{lemma}
%%%%%%%%%%%%%%%%%%%%%%%%%%%%%%%%%%%%%%%%%%%%%%%%%%%%%%%%%%%%%%%%%%%%%%%%%%%

%%%%%%%%%%%%%%%%%%%%%%%%%%%%%%%%%%%%%%%%%%%%%%%%%%%%%%%%%%%%%%%%%%%%%%%%%%%
\begin{proof}
%%%%%%%%%%%%%%%%%%%%%%%%%%%%%%%%%%%%%%%%%%%%%%%%%%%%%%%%%%%%%%%%%%%%%%%%%%%
We only discuss the first, as the other one can be dealt-with similarly.
Let $f \in \prod_{i \in I}J_i$.
For each $i \in I$, derive
\begin{center}
\AXC{}
\UIC{$\varphi_{i,f(i)} \,\thesis\, \varphi_{i,f(i)}$}
\UIC{$\varphi_{i,f(i)} \,\thesis\, \bigvee_{j \in J_i} \varphi_{i,j}$}
\DisplayProof
\end{center}

\noindent
Hence Lemma~\ref{lem:proof:log:functprop}
gives
\[
\begin{array}{l l l}
  \bigwedge_{i \in I} \varphi_{i,f(i)}
& \thesis
& \bigwedge_{i \in I} \bigvee_{j \in J_i} \varphi_{i,j}
\end{array}
\]

\noindent
Then we are done since this holds for each $f \in \prod_{i \in I}J_i$.
\qed
%%%%%%%%%%%%%%%%%%%%%%%%%%%%%%%%%%%%%%%%%%%%%%%%%%%%%%%%%%%%%%%%%%%%%%%%%%%
\end{proof}
%%%%%%%%%%%%%%%%%%%%%%%%%%%%%%%%%%%%%%%%%%%%%%%%%%%%%%%%%%%%%%%%%%%%%%%%%%%




%%%%%%%%%%%%%%%%%%%%%%%%%%%%%%%%%%%%%%%%%%%%%%%%%%%%%%%%%%%%%%%%%%%%%%%%%%%
\begin{lemma}
\label{lem:proof:log:distr}
%%%%%%%%%%%%%%%%%%%%%%%%%%%%%%%%%%%%%%%%%%%%%%%%%%%%%%%%%%%%%%%%%%%%%%%%%%%
The following sequent is derivable
\[
\dfrac{}
  {\bigwedge_{f \in \prod_{i \in I}J_i}\bigvee_{i \in I} \varphi_{i,f(i)}
  \thesis
  \bigvee_{i \in I}\bigwedge_{j \in J_i}\varphi_{i,j}}
\]
%%%%%%%%%%%%%%%%%%%%%%%%%%%%%%%%%%%%%%%%%%%%%%%%%%%%%%%%%%%%%%%%%%%%%%%%%%%
\end{lemma}
%%%%%%%%%%%%%%%%%%%%%%%%%%%%%%%%%%%%%%%%%%%%%%%%%%%%%%%%%%%%%%%%%%%%%%%%%%%

%%%%%%%%%%%%%%%%%%%%%%%%%%%%%%%%%%%%%%%%%%%%%%%%%%%%%%%%%%%%%%%%%%%%%%%%%%%
\begin{proof}
%%%%%%%%%%%%%%%%%%%%%%%%%%%%%%%%%%%%%%%%%%%%%%%%%%%%%%%%%%%%%%%%%%%%%%%%%%%
This sequent amounts a well-known fact on completely distributive
complete lattices,
see e.g.~\cite[Lemma VII.1.10]{johnstone82book}.
We nevertheless offer a detailed proof.
Using the distributive law $\ax{D}$, we have
\begin{equation*}
\begin{array}{l !{~}l!{~} l}
  \bigwedge_{f \in \prod_{i \in I}J_i}\bigvee_{i \in I} \varphi_{i,f(i)}
& \thesis
& \bigvee_{F\colon (\prod_{i \in I}J_i) \to I}
  \bigwedge_{f \in \prod_{i \in I}J_i}
  \varphi_{F(f),f(F(f))}
\end{array}
\end{equation*}

\noindent
Hence we are done if we show
\begin{equation*}
\begin{array}{l !{~}l!{~} l}
  \bigvee_{F\colon (\prod_{i \in I}J_i) \to I}
  \bigwedge_{f \in \prod_{i \in I}J_i}
  \varphi_{F(f),f(F(f))}
& \thesis
& \bigvee_{i \in I}\bigwedge_{j \in J_i}\varphi_{i,j}
\end{array}
\end{equation*}

\noindent
So let $F \colon \left(\prod_{i \in I}J_i\right)  \to I$
and assume toward a contradiction that 
\begin{equation*}
\begin{array}{l !{~}l!{~} l}
  \bigwedge_{f \in \prod_{i \in I}J_i}
  \varphi_{F(f),f(F(f))}
& \not\thesis
& \bigvee_{i \in I}\bigwedge_{j \in J_i}\varphi_{i,j}
\end{array}
\end{equation*}

\noindent
It follows that for each $i \in I$, there is some $j \in J_i$ such that
\begin{equation*}
\begin{array}{l !{~}l!{~} l}
  \bigwedge_{f \in \prod_{i \in I}J_i}
  \varphi_{F(f),f(F(f))}
& \not\thesis
& \varphi_{i,j}
\end{array}
\end{equation*}

\noindent
Using the Axiom of Choice, we get a function $g \in \prod_{i \in I}J_i$
such that for all $i \in I$,
\begin{equation*}
\begin{array}{l !{~}l!{~} l}
  \bigwedge_{f \in \prod_{i \in I}J_i}
  \varphi_{F(f),f(F(f))}
& \not\thesis
& \varphi_{i,g(i)}
\end{array}
\end{equation*}

\noindent
In particular,
\begin{equation*}
\begin{array}{l !{~}l!{~} l}
  \bigwedge_{f \in \prod_{i \in I}J_i}
  \varphi_{F(f),f(F(f))}
& \not\thesis
& \varphi_{F(g),g(F(g))}
\end{array}
\end{equation*}

\noindent
a contradiction.
\qed
%%%%%%%%%%%%%%%%%%%%%%%%%%%%%%%%%%%%%%%%%%%%%%%%%%%%%%%%%%%%%%%%%%%%%%%%%%%
\end{proof}
%%%%%%%%%%%%%%%%%%%%%%%%%%%%%%%%%%%%%%%%%%%%%%%%%%%%%%%%%%%%%%%%%%%%%%%%%%%


%%%%%%%%%%%%%%%%%%%%%%%%%%%%%%%%%%%%%%%%%%%%%%%%%%%%%%%%%%%%%%%%%%%%%%%%%%%
\subsection{Proofs of \S\ref{sec:reft} (\nameref{sec:reft})}
\label{sec:proof:reft}
%%%%%%%%%%%%%%%%%%%%%%%%%%%%%%%%%%%%%%%%%%%%%%%%%%%%%%%%%%%%%%%%%%%%%%%%%%%

Lemma~\ref{lem:reft}
will be useful for completeness (\S\ref{sec:compl} and \S\ref{sec:proof:compl}).

%%%%%%%%%%%%%%%%%%%%%%%%%%%%%%%%%%%%%%%%%%%%%%%%%%%%%%%%%%%%%%%%%%%%%%%%%%%
\begin{lemma}[Lemma \ref{lem:reft}]
\label{lem:proof:reft}
%%%%%%%%%%%%%%%%%%%%%%%%%%%%%%%%%%%%%%%%%%%%%%%%%%%%%%%%%%%%%%%%%%%%%%%%%%%
For each $\RT$, there is $\varphi \in \Lang(\UPT\RT)$
such that $\RT \eqtype \reft{\UPT\RT \mid \varphi}$.
%%%%%%%%%%%%%%%%%%%%%%%%%%%%%%%%%%%%%%%%%%%%%%%%%%%%%%%%%%%%%%%%%%%%%%%%%%%
\end{lemma}
%%%%%%%%%%%%%%%%%%%%%%%%%%%%%%%%%%%%%%%%%%%%%%%%%%%%%%%%%%%%%%%%%%%%%%%%%%%

%%%%%%%%%%%%%%%%%%%%%%%%%%%%%%%%%%%%%%%%%%%%%%%%%%%%%%%%%%%%%%%%%%%%%%%%%%%
\begin{proof}
%%%%%%%%%%%%%%%%%%%%%%%%%%%%%%%%%%%%%%%%%%%%%%%%%%%%%%%%%%%%%%%%%%%%%%%%%%%
The proof is by induction on $\RT$.
The base case of $\reft{\PT \mid \varphi}$ is trivial.
In the base case of $\PT$, one can take $\varphi = \True$.
In the cases of $\RT \times \RTbis$ and $\RTbis \arrow \RT$,
by induction hypotheses we get $\varphi \in \Lang(\UPT\RT)$
and $\psi \in \Lang(\UPT\RTbis)$ such that
$\RT \eqtype \reft{\UPT\RT \mid \varphi}$
and
$\RTbis \eqtype \reft{\UPT\RTbis \mid \varphi}$.
We then conclude with
\[
\begin{array}{r c l}
  \RT \times \RTbis
& \eqtype
& \reft{\UPT\RT \times \UPT\RTbis \mid \pair{\varphi,\psi}}
\\

  \RTbis \arrow \RT
& \eqtype
& \reft{\UPT\RTbis \arrow \UPT\RT \mid \psi \realto \varphi}
\end{array}
\]
\qed
%%%%%%%%%%%%%%%%%%%%%%%%%%%%%%%%%%%%%%%%%%%%%%%%%%%%%%%%%%%%%%%%%%%%%%%%%%%
\end{proof}
%%%%%%%%%%%%%%%%%%%%%%%%%%%%%%%%%%%%%%%%%%%%%%%%%%%%%%%%%%%%%%%%%%%%%%%%%%%

}
\opt{full}{%%%%%%%%%%%%%%%%%%%%%%%%%%%%%%%%%%%%%%%%%%%%%%%%%%%%%%%%%%%%%%%%%%%%%%%%%%%
\section{Proofs of \S\ref{sec:sem} (\nameref{sec:sem})}
\label{sec:proof:sem}
%%%%%%%%%%%%%%%%%%%%%%%%%%%%%%%%%%%%%%%%%%%%%%%%%%%%%%%%%%%%%%%%%%%%%%%%%%%

%%%%%%%%%%%%%%%%%%%%%%%%%%%%%%%%%%%%%%%%%%%%%%%%%%%%%%%%%%%%%%%%%%%%%%%%%%%
\subsection{\nameref{sec:sem:pure}}
\label{sec:proof:sem:pure}
%%%%%%%%%%%%%%%%%%%%%%%%%%%%%%%%%%%%%%%%%%%%%%%%%%%%%%%%%%%%%%%%%%%%%%%%%%%


%%%%%%%%%%%%%%%%%%%%%%%%%%%%%%%%%%%%%%%%%%%%%%%%%%%%%%%%%%%%%%%%%%%%%%%%%%%
\subsubsection{Solutions of Recursive Domain Equations.}
%%%%%%%%%%%%%%%%%%%%%%%%%%%%%%%%%%%%%%%%%%%%%%%%%%%%%%%%%%%%%%%%%%%%%%%%%%%
We review the usual solution of recursive domain equations.
We refer to~\cite{ac98book,aj95chapter,streicher06book}.


%%%%%%%%%%%%%%%%%%%%%%%%%%%%%%%%%%%%%%%%%%%%%%%%%%%%%%%%%%%%%%%%%%%%%%%%%%%
\paragraph{Categories of Domains.}
%%%%%%%%%%%%%%%%%%%%%%%%%%%%%%%%%%%%%%%%%%%%%%%%%%%%%%%%%%%%%%%%%%%%%%%%%%%
In the following, $\DCPO$ is the category of those
posets with all directed suprema, and with Scott-continuous
functions as morphisms.
$\CPO$ is the full subcategory of $\DCPO$
on posets with a least element,
and $\Scott$ is a full subcategory of $\CPO$.


Recall that $\DCPO$, $\CPO$ and $\Scott$
have finite products
(equipped with the component-wise order).
See \cite[Theorem 3.3.3, Theorem 3.3.5 and Corollary 4.1.6]{aj95chapter}.
Hence for each $n \in \NN$, the categories
$\Scott^n$, $\CPO^n$ and $\DCPO^n$
are (not full) subcategories of $\Scott$, $\CPO$ and $\DCPO$
respectively.

%%%%%%%%%%%%%%%%%%%%%%%%%%%%%%%%%%%%%%%%%%%%%%%%%%%%%%%%%%%%%%%%%%%%%%%%%%%
\begin{lemma}
\label{lem:proof:scott:enrich}
%%%%%%%%%%%%%%%%%%%%%%%%%%%%%%%%%%%%%%%%%%%%%%%%%%%%%%%%%%%%%%%%%%%%%%%%%%%
If $n \in \NN$ then
$\DCPO^n$,
$\CPO^n$, $\Scott^n$ are enriched in $\DCPO$.
%%%%%%%%%%%%%%%%%%%%%%%%%%%%%%%%%%%%%%%%%%%%%%%%%%%%%%%%%%%%%%%%%%%%%%%%%%%
\end{lemma}
%%%%%%%%%%%%%%%%%%%%%%%%%%%%%%%%%%%%%%%%%%%%%%%%%%%%%%%%%%%%%%%%%%%%%%%%%%%

%%%%%%%%%%%%%%%%%%%%%%%%%%%%%%%%%%%%%%%%%%%%%%%%%%%%%%%%%%%%%%%%%%%%%%%%%%%
\begin{proof}
%%%%%%%%%%%%%%%%%%%%%%%%%%%%%%%%%%%%%%%%%%%%%%%%%%%%%%%%%%%%%%%%%%%%%%%%%%%
The result for $n=1$ follows from the
Cartesian-closure of $\DCPO$, $\CPO$ and $\Scott$
(\cite[Theorem 3.3.3, Theorem 3.3.5 and Corollary 4.1.6]{aj95chapter}).
In the cases of $n \neq 1$, the result follows from the fact that
in $\DCPO, \CPO, \Scott$, finite products are Cartesian products
of sets equipped with the component-wise order.
\qed
%%%%%%%%%%%%%%%%%%%%%%%%%%%%%%%%%%%%%%%%%%%%%%%%%%%%%%%%%%%%%%%%%%%%%%%%%%%
\end{proof}
%%%%%%%%%%%%%%%%%%%%%%%%%%%%%%%%%%%%%%%%%%%%%%%%%%%%%%%%%%%%%%%%%%%%%%%%%%%

Let $\cat C$ be a category enriched over $\DCPO$.
Given objects $X,Y \in \cat C$,
an \emph{embedding-projection} pair $X \to Y$
is a pair of morphisms $\ladj f: X\rightleftarrows Y:\radj f$
where $\radj f \comp \ladj f = \id_X$ and $\ladj f \comp \radj f \leq \id_Y$.
The morphism $\ladj f$ is an \emph{embedding}
(it reflects (as well as preserves) the order),
while $\radj f$ is a \emph{projection}.
%Note that $\radj e$ and $\ladj e$ are strict.
Note that if $X$ (resp.\ $Y$) has a least element,
then so does $Y$ (resp.\ $X$) and $\ladj f$ (resp $\radj f$)
is strict.
It is well-known that $\ladj f$ completely determines $\radj f$
and reciprocally, see~\cite[\S 7.1]{ac98book}
(cf.\ also~\cite[\S 3.1.4]{aj95chapter} and~\cite[\S 9]{streicher06book}).
Given an embedding $e$ (resp.\ a projection $p$),
we write $\radj e$ (resp.\ $\ladj p$)
for the corresponding projection (resp.\ embedding).

We write $\cat C^\ep$ for the category with the same objects as $\cat C$,
and with embedding-projection pairs as morphisms.
Note that we have faithful functors
$\ladj{(\pl)} \colon \cat C^\ep \to \cat C$
and
$\radj{(\pl)} \colon \cat C^\ep \to \cat C^\op$
(taking $(\ladj f,\radj f)$ to $\ladj f$ and to $\radj f$,
respectively).
Given a functor $H$ of codomain $\cat C^\ep$,
we write $\radj H$ for $\radj{(\pl)} \comp H$,
and similarly for $\ladj H$.


%%%%%%%%%%%%%%%%%%%%%%%%%%%%%%%%%%%%%%%%%%%%%%%%%%%%%%%%%%%%%%%%%%%%%%%%%%%
\paragraph{The Limit-Colimit Coincidence.}
%%%%%%%%%%%%%%%%%%%%%%%%%%%%%%%%%%%%%%%%%%%%%%%%%%%%%%%%%%%%%%%%%%%%%%%%%%%
The following (crucial and) well-known fact
is \cite[Theorem 7.1.10]{ac98book}
(see also \cite[Theorem 3.3.7]{aj95chapter}).

%%%%%%%%%%%%%%%%%%%%%%%%%%%%%%%%%%%%%%%%%%%%%%%%%%%%%%%%%%%%%%%%%%%%%%%%%%%
\begin{theorem}
\label{thm:proof:scott:limcolim}
%%%%%%%%%%%%%%%%%%%%%%%%%%%%%%%%%%%%%%%%%%%%%%%%%%%%%%%%%%%%%%%%%%%%%%%%%%%
Let $K \colon \omega \to \cat C^\ep$ be a functor
where $\cat C$ is enriched over $\DCPO$.
Each limiting cone $\varpi \colon \Lim \radj K \to \radj K$
for $\radj K \colon \omega^\op \to \cat C$ consists of projections,
and the $(\ladj{(\varpi_n)},\varpi_n)_n$
form a colimiting cocone
$K \to \Colim K$ in $\cat C^\ep$.
%%%%%%%%%%%%%%%%%%%%%%%%%%%%%%%%%%%%%%%%%%%%%%%%%%%%%%%%%%%%%%%%%%%%%%%%%%%
\end{theorem}
%%%%%%%%%%%%%%%%%%%%%%%%%%%%%%%%%%%%%%%%%%%%%%%%%%%%%%%%%%%%%%%%%%%%%%%%%%%

%%%%%%%%%%%%%%%%%%%%%%%%%%%%%%%%%%%%%%%%%%%%%%%%%%%%%%%%%%%%%%%%%%%%%%%%%%%
\begin{proof}
%%%%%%%%%%%%%%%%%%%%%%%%%%%%%%%%%%%%%%%%%%%%%%%%%%%%%%%%%%%%%%%%%%%%%%%%%%%
Let
$K \colon \omega \to \cat C^\ep$
and consider a limiting cone
\begin{equation}
\label{diag:proof:scott:lim}
\begin{array}{c}
\begin{tikzcd}[column sep=2em] % normal=2.4em
&
& \Lim \radj K
  \arrow{dll}[above]{\varpi_0}
  \arrow{dl}{\varpi_1}
  \arrow{d}{\varpi_2}
  \arrow{dr}[below]{\varpi_n}
  \arrow{drr}{\varpi_{n+1}}

\\

  \radj K(0)
& \radj K(1)
  \arrow{l}[below]{\radj{(k_0)}}
& \radj K(2)
  \arrow{l}[below]{\radj{(k_1)}}
& \radj K(n)
  \arrow[dashed]{l}
& \radj K(n+1)
  \arrow{l}[below]{\radj{(k_n)}}
& \phantom{F}
  \arrow[dashed]{l}
\end{tikzcd}
\end{array}
\end{equation}


\noindent
in $\cat C$.
The components of the colimiting cocone
\begin{equation}
\label{diag:proof:scott:colim}
\begin{array}{c}
\begin{tikzcd}[column sep=2em]
&
& \Colim K

\\

  K(0)
  \arrow{r}[below]{k_0}
  \arrow{urr}[above]{\gamma_0}
& K(1)
  \arrow{r}[below]{k_1}
  \arrow{ur}[below]{\gamma_1}
& K(2)
  \arrow[dashed]{r}
  \arrow{u}{\gamma_2}
& K(n)
  \arrow{r}[below]{k_n}
  \arrow{ul}[below]{\gamma_n}
& K(n+1)
  \arrow[dashed]{r}
  \arrow{ull}[above]{\gamma_{n+1}}
& \phantom{F}
\end{tikzcd}
\end{array}
\end{equation}

\noindent
in $\cat C^\ep$ are given by $\radj{(\gamma_n)} = \varpi_n$
for projections.

Concerning embeddings,
for each $n \in \NN$ we build a cone with vertex $K(n) = \radj K(n)$
as follows.
Given $m \in \NN$, we have a morphism $h_{n,m} \colon K(n) \to K(m)$
obtained by composing $\radj{(k_i)}$'s or $\ladj{(k_i)}$'s
according to whether $m \leq n$ or $n \leq m$.
The $h_{n,m}$'s can be made so that $h_{n,m} = \radj{(k_m)} \comp h_{n,m+1}$.
The universal property of limits in $\cat C$ then yields
a unique morphism $c_n$ from $K(n) = \radj K(n)$ to $\Lim \radj K(n)$
such that $\varpi_m \comp c_n = h_{n,m}$ for all $m \in \NN$.

We are going to show that $c_n = \ladj{(\varpi_n)}$.
Note that $\varpi_n \comp c_n$ is the identity by definition of $c_n$.
It remains to show that $c_n \comp \varpi_n \leq \id_{\Lim \radj K}$.
We first show that
$(c_n \comp \varpi_n)_n$ forms an increasing sequence
in $\cat C(\Lim \radj K, \Lim \radj K)$.
To this end, note that $\varpi_n = \radj{(k_n)} \comp \varpi_{n+1}$ 
(since $\varpi$ is a cone).
We moreover have
$c_n = c_{n+1} \comp \ladj{(k_n)}$
since
\(
  \varpi_m \comp c_{n+1} \comp \ladj{(k_n)}
  =
  h_{n+1,m} \comp \ladj{(k_n)}
  =
  h_{n,m}
\)
for all $m \in \NN$.
We compute
\[
\begin{array}{*{5}{l}}
  c_n \comp \varpi_n
& =
& c_{n+1} \comp \ladj{(k_n)} \comp \radj{(k_n)} \comp \varpi_{n+1}
& \leq
& c_{n+1} \comp \varpi_{n+1}
\end{array}
\]

Let $\ell = \bigvee_{n}(c_n \comp \varpi_n)$.
We now claim that $\ell$ is the identity.
This will yield that $c_n \comp \varpi_n \leq \id_{\Lim \radj K}$.
In order to show that $\ell = \id_{\Lim \radj K}$,
we show that $\varpi_m \comp \ell = \varpi_m$ for all $m \in \NN$,
and use the universal property of limits in $\cat C$.
We have
\[
\begin{array}{l l l}
  \varpi_m \comp \ell
& =
& \varpi_m \comp \bigvee_{n}(c_n \comp \varpi_n)
\\

& =
& \bigvee_n \left(
  \varpi_m \comp c_n \comp \varpi_n
  \right)
\\

& =
& \bigvee_n \left(
  h_{n,m} \comp \varpi_n
  \right)
\\

& =
& \bigvee_{n\geq m} \left(
  h_{n,m} \comp \varpi_n
  \right)
\end{array}
\]

\noindent
But by definition of $h_{n,m}$, we have
$h_{n,m} \comp \varpi_n = \varpi_m$ when $m \leq n$.

We can thus set $\gamma_n = (c_n, \varpi_n)$.
Moreover, $\gamma = (\gamma_n)_n$ is indeed a cocone since
$c_n = c_{n+1} \comp \ladj{(k_n)}$ (see above).

We now claim that $\gamma \colon K \to \Lim \radj K$ is colimiting.
To this end, consider a cocone $\tau \colon K \to C$.
We thus get a cone $\radj\tau \colon \radj K \to \radj C$ in $\cat C$,
and the universal property of limits yields a unique
$p \colon \radj C \to \Lim \radj K$ such that
$\varpi_n \comp p = \radj{(\tau_n)}$ for all $n \in \NN$.
We show that $p$ is a projection.
We define a morphism $e \colon \Lim \radj K \to C$
as $e = \bigvee_{n}(\ladj{(\tau_n)} \comp \varpi_n)$.
We have
\[
\begin{array}{l l l}
  e \comp p
& =
& \left( \bigvee_{n} \ladj{(\tau_n)} \comp \varpi_n \right) \comp p
\\

& =
& \bigvee_n \ladj{(\tau_n)} \comp \radj{(\tau_n)}
\\

& \leq
& \id_{C}
\end{array}
\]

\noindent
On the other hand, given $m \in \NN$ we have
\[
\begin{array}{l l l}
  \varpi_m \comp p \comp e
& =
& \bigvee_{n} \radj{(\tau_m)} \comp \ladj{(\tau_n)} \comp \varpi_n
\\

& =
& \bigvee_{n \geq m} \radj{(\tau_m)} \comp \ladj{(\tau_n)} \comp \varpi_n
\\

& =
& \bigvee_{n \geq m} h_{n,m} \comp \radj{(\tau_n)} \comp \ladj{(\tau_n)} \comp \varpi_n
\\

& =
& \bigvee_{n \geq m} h_{n,m} \comp \varpi_n
\\

& =
& \bigvee_{n \geq m} \varpi_m
\\

& =
& \varpi_m
\end{array}
\]

\noindent
so that $p \comp e = \id_{\Lim \radj K}$
by the universal property of limits in $\cat C$.

Moreover, for all $n \in \NN$ we have
\[
\begin{array}{l l l}
  e \comp c_n
& =
& \bigvee_m \ladj{(\tau_m)} \comp \varpi_m \comp c_n
\\

& =
& \bigvee_m \ladj{(\tau_m)} \comp h_{n,m}
\\

& =
& \bigvee_{m\geq n} \ladj{(\tau_m)} \comp h_{n,m}
\\

& =
& \bigvee_{m\geq n} \ladj{(\tau_n)}
\\

& =
& \ladj{(\tau_n)}
\end{array}
\]

Consider now a morphism $\ell \colon \Lim \radj K \to C$
in $\cat C^\ep$
such that 
$\varpi_n \comp \radj\ell = \radj{(\tau_n)}$
and
$\ladj\ell \comp c_n = \ladj{(\tau_n)}$
for all $n \in \NN$.
The universal property of limits in $\cat C$
yields $\radj\ell = p$, so that $\ladj\ell = e$
since $e$ is uniquely determined from $p$.
\qed
%%%%%%%%%%%%%%%%%%%%%%%%%%%%%%%%%%%%%%%%%%%%%%%%%%%%%%%%%%%%%%%%%%%%%%%%%%%
\end{proof}
%%%%%%%%%%%%%%%%%%%%%%%%%%%%%%%%%%%%%%%%%%%%%%%%%%%%%%%%%%%%%%%%%%%%%%%%%%%


%%%%%%%%%%%%%%%%%%%%%%%%%%%%%%%%%%%%%%%%%%%%%%%%%%%%%%%%%%%%%%%%%%%%%%%%%%%
\paragraph{Solutions of Domain Equations.}
%%%%%%%%%%%%%%%%%%%%%%%%%%%%%%%%%%%%%%%%%%%%%%%%%%%%%%%%%%%%%%%%%%%%%%%%%%%
We shall use Theorem~\ref{thm:proof:scott:limcolim}
in the following situation.
Consider a functor
\[
\begin{array}{*{5}{l}}
  G
& :
& \cat D^\ep \times \cat C^\ep
  % \cat C^\ep
& \longto
& \cat C^\ep
\end{array}
\]

\noindent
where $\cat C$ and $\cat D$ are enriched over $\DCPO$.
We moreover assume that $\cat C$ has a terminal object $\one$
which is initial in $\cat C^\ep$.
We are going to define a functor
\[
\begin{array}{*{5}{l}}
  K
& \colon
& \cat D^\ep \times \omega
& \longto
& \cat C^\ep
\end{array}
\]

\noindent
Given an object $B$ of $\cat D^\ep$,
$K(B,\pl)$ is the $\omega$-chain in $\cat C^\ep$
obtained by iterating $G_B = G(B,\pl)$ from the initial object $\one$ of $\cat C^\ep$:
\begin{equation}
\label{diag:proof:scott:chain}
\begin{tikzcd} %[column sep=large]
  \one
  \arrow{r}[above]{\one}
& G_B(\one)
   \arrow{r}[above]{G_B(\one)}
 & G^2_B(\one)
  \arrow[dashed]{r}
& G^n_B(\one)
  \arrow{r}[above]{G_B^n(\one)}
& G^{n+1}_B(\one)
  \arrow[dashed]{r}
& \phantom{F}
\end{tikzcd}
\end{equation}


Given a morphism $f \colon B \to B'$ in $\cat D^\ep$,
$K(f,\pl)$ is obtained by commutativity of the following.
\begin{equation}
\label{diag:proof:scott:natdiag}
\begin{array}{c}
\begin{tikzcd} %[column sep=large]

  \one
  \arrow{r}[above]{\one}
  \arrow{d}{\one}
& G_B(\one)
   \arrow{r}[above]{G_B(\one)}
   \arrow{d}{G_f(\one)}
 & G^2_B(\one)
   \arrow{d}{G_f^2(\one)}
  \arrow[dashed]{r}
& G^n_B(\one)
  \arrow{r}[above]{G_B^n(\one)}
  \arrow{d}{G_f^n(\one)}
& G^{n+1}_B(\one)
  \arrow{d}{G_f^{n+1}(\one)}
  \arrow[dashed]{r}
& \phantom{F}

\\

  \one
  \arrow{r}[below]{\one}
& G_{B'}(\one)
   \arrow{r}[below]{G_{B'}(\one)}
 & G^2_{B'}(\one)
  \arrow[dashed]{r}
& G^n_{B'}(\one)
  \arrow{r}[below]{G_{B'}^n(\one)}
& G^{n+1}_{B'}(\one)
  \arrow[dashed]{r}
& \phantom{F}
\end{tikzcd}
\end{array}
\end{equation}



Assume now that $\cat C$ has limits of $\omega^\op$-chains
of projections.
Then Theorem~\ref{thm:proof:scott:limcolim}
yields that each $K(B,\pl)$ has a colimit in $\cat C^\ep$.
Since $K$ is a functor $\cat D^\ep \times \omega \to \cat C^\ep$,
it follows from \cite[Theorem V.3.1]{maclane98book}
that these colimits assemble into a functor
\[
\begin{array}{l l r c l}
  \Fix G
& :
& \cat D^\ep
& \longto
& \cat C^\ep
\\

&
& B
& \longmapsto
& \Colim_{n \in \omega} K(B,n)
\end{array}
\]

If $G(B,\pl)$ preserves colimits of $\omega$-chains,
then the universal property of colimits gives an isomorphism
\(
  \fold^\ep
  :
  G(B,\Fix G(B))
  \rightleftarrows
  \Fix G(B)
  :
  \unfold^\ep
\)
in $\cat C^\ep$.

We are going to prove the following.

%%%%%%%%%%%%%%%%%%%%%%%%%%%%%%%%%%%%%%%%%%%%%%%%%%%%%%%%%%%%%%%%%%%%%%%%%%%
\begin{proposition}
\label{prop:proof:scott:contfunct}
%%%%%%%%%%%%%%%%%%%%%%%%%%%%%%%%%%%%%%%%%%%%%%%%%%%%%%%%%%%%%%%%%%%%%%%%%%%
If $G \colon \cat D^\ep \times \cat C^\ep \to \cat C^\ep$
preserves colimits of $\omega$-chains,
then so do $\Fix G \colon \cat D^\ep \to \cat C^\ep$.
%%%%%%%%%%%%%%%%%%%%%%%%%%%%%%%%%%%%%%%%%%%%%%%%%%%%%%%%%%%%%%%%%%%%%%%%%%%
\end{proposition}
%%%%%%%%%%%%%%%%%%%%%%%%%%%%%%%%%%%%%%%%%%%%%%%%%%%%%%%%%%%%%%%%%%%%%%%%%%%

The proof of Proposition~\ref{prop:proof:scott:contfunct}
is split into the following lemmas.
Fix a functor $G \colon \cat D^\ep \times \cat C^\ep \to \cat C^\ep$
which preserves colimits of $\omega$-chains.

%%%%%%%%%%%%%%%%%%%%%%%%%%%%%%%%%%%%%%%%%%%%%%%%%%%%%%%%%%%%%%%%%%%%%%%%%%%
\begin{lemma}
\label{lem:proof:scott:contdiag}
%%%%%%%%%%%%%%%%%%%%%%%%%%%%%%%%%%%%%%%%%%%%%%%%%%%%%%%%%%%%%%%%%%%%%%%%%%%
The diagonal functor $\Delta \colon \cat D^\ep \to \cat D^\ep \times \cat D^\ep$
preserves colimits of $\omega$-chains.
%%%%%%%%%%%%%%%%%%%%%%%%%%%%%%%%%%%%%%%%%%%%%%%%%%%%%%%%%%%%%%%%%%%%%%%%%%%
\end{lemma}
%%%%%%%%%%%%%%%%%%%%%%%%%%%%%%%%%%%%%%%%%%%%%%%%%%%%%%%%%%%%%%%%%%%%%%%%%%%

%%%%%%%%%%%%%%%%%%%%%%%%%%%%%%%%%%%%%%%%%%%%%%%%%%%%%%%%%%%%%%%%%%%%%%%%%%%
\begin{proof}
%%%%%%%%%%%%%%%%%%%%%%%%%%%%%%%%%%%%%%%%%%%%%%%%%%%%%%%%%%%%%%%%%%%%%%%%%%%
Since colimits are pointwise in functor categories
(\cite[Corollary V.3]{maclane98book}).%
\footnote{Note that \cite[Corollary V.3]{maclane98book} only gives the result for limits.
But recall that the opposite of a functor category $[\cat C,\cat D]$
is the functor category $[\cat C^\op, \cat D^\op]$.}
%%%%%%%%%%%%%%%%%%%%%%%%%%%%%%%%%%%%%%%%%%%%%%%%%%%%%%%%%%%%%%%%%%%%%%%%%%%
\end{proof}
%%%%%%%%%%%%%%%%%%%%%%%%%%%%%%%%%%%%%%%%%%%%%%%%%%%%%%%%%%%%%%%%%%%%%%%%%%%

Lemma~\ref{lem:proof:scott:contdiag}
entails in particular that each functor
$G^n_{(\pl)}(\one) \colon \cat D^\ep \to \cat C^\ep$
preserves colimits of $\omega$-chains
($G^{n+1}_{(\pl)}(\one)$ is
$G(\pl,G^{n}_{(\pl)}(\one)) \comp \Delta$).

Proposition~\ref{prop:proof:scott:contfunct}
relies on the fact that the functor
$K \colon \cat D^\ep \to \funct{\omega,\cat C^\ep}$
preserves colimits of $\omega$-chains.
This involves some notation.

Let $W \colon \omega \to \cat D^\ep$ be an $\omega$-chain,
with colimiting cocone $\gamma \colon W \to \Colim W$.
In the following, we write $w_m \colon W(m) \to W(m+1)$
for the connecting morphisms of $W$.
The cocone $K \gamma \colon K(W) \to K(\Colim W)$
has component at $m \in \NN$
the commutative diagram in \eqref{diag:proof:scott:natdiag}
where on takes $\gamma_m \colon W(m) \to \Colim W$
for $f \colon B \to B'$.


%%%%%%%%%%%%%%%%%%%%%%%%%%%%%%%%%%%%%%%%%%%%%%%%%%%%%%%%%%%%%%%%%%%%%%%%%%%
\begin{lemma}
\label{lem:proof:scott:colimiting}
%%%%%%%%%%%%%%%%%%%%%%%%%%%%%%%%%%%%%%%%%%%%%%%%%%%%%%%%%%%%%%%%%%%%%%%%%%%
The cocone $K\gamma \colon K(W) \to K(\Colim W)$
is colimiting.
%%%%%%%%%%%%%%%%%%%%%%%%%%%%%%%%%%%%%%%%%%%%%%%%%%%%%%%%%%%%%%%%%%%%%%%%%%%
\end{lemma}
%%%%%%%%%%%%%%%%%%%%%%%%%%%%%%%%%%%%%%%%%%%%%%%%%%%%%%%%%%%%%%%%%%%%%%%%%%%

%%%%%%%%%%%%%%%%%%%%%%%%%%%%%%%%%%%%%%%%%%%%%%%%%%%%%%%%%%%%%%%%%%%%%%%%%%%
\begin{proof}
%%%%%%%%%%%%%%%%%%%%%%%%%%%%%%%%%%%%%%%%%%%%%%%%%%%%%%%%%%%%%%%%%%%%%%%%%%%
First, it follows from the above that each
$G_\gamma^n(\one) \colon G_W^n(\one) \to G_{\Colim W}^n(\one)$
is colimiting.

Consider now a cocone
$\tau \colon K(W) \to H$ in $\funct{\omega,\cat C^\ep}$.
For each $m \in \NN$, we have
$\tau_m = \tau_{m+1} \comp K(w_m)$,
that is
\begin{equation}
\label{diag:proof:scott:taucocone}
\begin{array}{c}
\begin{tikzcd}[column sep=2.14em, row sep=large]
% column sep normal=2.4em
% row sep normal=1.8em

  \one
  \arrow{r}[above]{\one}
  \arrow{d}{\one}
& G_{B}(\one)
   \arrow{r}[above]{G_{B}(\one)}
   \arrow{d}{G_{w_m}(\one)}
& G^2_{B}(\one)
   \arrow{d}{G_{w_m}^2(\one)}
  \arrow[dashed]{r}
& G^n_{B}(\one)
  \arrow{r}[above]{G_{B}^n(\one)}
  \arrow{d}{G_{w_m}^n(\one)}
& G^{n+1}_{B}(\one)
  \arrow{d}{G_{w_m}^{n+1}(\one)}
  \arrow[dashed]{r}
& \phantom{F}

\\

  \one
  \arrow{r}[above]{\one}
  \arrow{d}{(\tau_{m+1})_0}
& G_{B'}(\one)
   \arrow{r}[above]{G_{B'}(\one)}
   \arrow{d}{(\tau_{m+1})_1}
& G^2_{B'}(\one)
   \arrow{d}{(\tau_{m+1})_2}
  \arrow[dashed]{r}
& G^n_{B'}(\one)
  \arrow{r}[above]{G_{B'}^n(\one)}
  \arrow{d}{(\tau_{m+1})_n}
& G^{n+1}_{B'}(\one)
  \arrow{d}{(\tau_{m+1})_{n+1}}
  \arrow[dashed]{r}
& \phantom{F}

\\

  H(0)
  \arrow{r}[below]{h(0)}
& H(1)
   \arrow{r}[below]{h(1)}
& H(2)
  \arrow[dashed]{r}
& H(n)
  \arrow{r}[below]{h(n)}
& H(n+1)
  \arrow[dashed]{r}
& \phantom{F}
\end{tikzcd}
\end{array}
\end{equation}

\noindent
where $B$ is $W(m)$, $B'$ is $W(m+1)$
and the $h(n) \colon H(n) \to H(n+1)$ are the connective morphisms of $H$.
In particular, for each $m \in \NN$ and each $n \in \NN$,
we have $(\tau_m)_n = (\tau_{m+1})_n \comp G_{w_m}^n(\one)$.
Hence, for each $n \in \NN$ we have a cocone
$((\tau_m)_n)_m \colon G_W^n(\one) \to H(n)$,
and the universal property of $G_\gamma^n(\one)$
gives a unique morphism $\ell_n \colon G_{\Colim W}^n(\one) \to H(n)$
such that $(\tau_m)_n = \ell_n \comp G_{\gamma_m}^n(\one)$ for all $m \in \NN$.

We show that the $\ell_n$'s assemble into a morphism
$\ell \colon K(\Colim W) \to H$ in $\funct{\omega,\cat C^\ep}$.
We thus have to show that the following commutes
\begin{equation*}
\begin{array}{c}
\begin{tikzcd} %[column sep=large]

  \one
  \arrow{r}[above]{\one}
  \arrow{d}{\ell_0}
& G_{B'}(\one)
   \arrow{r}[above]{G_{B'}(\one)}
   \arrow{d}{\ell_1}
 & G^2_{B'}(\one)
   \arrow{d}{\ell_2}
  \arrow[dashed]{r}
& G^n_{B'}(\one)
  \arrow{r}[above]{G_{B'}^n(\one)}
  \arrow{d}{\ell_n}
& G^{n+1}_{B'}(\one)
  \arrow{d}{\ell_{n+1}}
  \arrow[dashed]{r}
& \phantom{F}

\\

  H(0)
  \arrow{r}[below]{h(0)}
& H(1)
   \arrow{r}[below]{h(1)}
& H(2)
  \arrow[dashed]{r}
& H(n)
  \arrow{r}[below]{h(n)}
& H(n+1)
  \arrow[dashed]{r}
& \phantom{F}

\end{tikzcd}
\end{array}
\end{equation*}

\noindent
where $B'$ is $\Colim W$.
We show that $\ell_{n+1} \comp G_{B'}^n(\one) = h(n) \comp \ell_n$
for all $n \in \NN$.
For each $m \in \NN$, 
by commutativity of \eqref{diag:proof:scott:natdiag}
and \eqref{diag:proof:scott:taucocone}
%and functoriality of $G_{(\pl)}^{n+}$
we have
\[
\begin{array}{l l l}
  \ell_{n+1} \comp G_{B'}^{n}(\one) \comp G_{\gamma_m}^{n}(\one)
& =
& \ell_{n+1} \comp G_{\gamma_m}^{n+1}(\one) \comp G_B^{n}(\one)
\\

& =
& (\tau_m)_{n+1} \comp G_B^{n}(\one)
\\

& =
& h(n) \comp (\tau_m)_n
\\

& =
& h(n) \comp \ell_n \comp G_{\gamma_m}^n(\one)
\end{array}
\]

\noindent
where $B$ is $W(m)$.
Then we are done by the universal property of $G_{\gamma_m}^n(\one)$.

Consider finally a morphism $f \colon K(\Colim W) \to H$ in
$\funct{\omega,\cat C^\ep}$
such that $f \comp K(\gamma) = \tau$.
Then for all $m \in \NN$ we have
$f \comp K(\gamma_m) = \tau_m$,
and thus
$f_n \comp G_{\gamma_m}^n(\one) = (\tau_m)_n$
for all $n \in \NN$.
It follows that $f_n = \ell_n$, so that $f = \ell$.
\qed
%%%%%%%%%%%%%%%%%%%%%%%%%%%%%%%%%%%%%%%%%%%%%%%%%%%%%%%%%%%%%%%%%%%%%%%%%%%
\end{proof}
%%%%%%%%%%%%%%%%%%%%%%%%%%%%%%%%%%%%%%%%%%%%%%%%%%%%%%%%%%%%%%%%%%%%%%%%%%%

We can now prove Proposition~\ref{prop:proof:scott:contfunct}.

%%%%%%%%%%%%%%%%%%%%%%%%%%%%%%%%%%%%%%%%%%%%%%%%%%%%%%%%%%%%%%%%%%%%%%%%%%%
\begin{proof}[of Proposition~\ref{prop:proof:scott:contfunct}]
%%%%%%%%%%%%%%%%%%%%%%%%%%%%%%%%%%%%%%%%%%%%%%%%%%%%%%%%%%%%%%%%%%%%%%%%%%%
Let $W \colon \omega \to \cat D^\ep$ be an $\omega$-chain.
By Lemma~\ref{lem:proof:scott:colimiting},
and since colimits always commute over colimits,
we have
\[
\begin{array}{l l l}
  \Fix G(\Colim W)
& =
& \Colim_{n \in \omega} K(\Colim W,n)
\\

& \cong
& \Colim_{n \in \omega} \Colim_{m \in \omega} K(W(m),n)
\\

& \cong
& \Colim_{m \in \omega} \Colim_{n \in \omega} K(W(m),n)
\\

& \cong
& \Colim_{m \in \omega} \Fix G(W(m))
\end{array}
\]
\qed
%%%%%%%%%%%%%%%%%%%%%%%%%%%%%%%%%%%%%%%%%%%%%%%%%%%%%%%%%%%%%%%%%%%%%%%%%%%
\end{proof}
%%%%%%%%%%%%%%%%%%%%%%%%%%%%%%%%%%%%%%%%%%%%%%%%%%%%%%%%%%%%%%%%%%%%%%%%%%%


%%%%%%%%%%%%%%%%%%%%%%%%%%%%%%%%%%%%%%%%%%%%%%%%%%%%%%%%%%%%%%%%%%%%%%%%%%%
\paragraph{Local Continuity.}
%%%%%%%%%%%%%%%%%%%%%%%%%%%%%%%%%%%%%%%%%%%%%%%%%%%%%%%%%%%%%%%%%%%%%%%%%%%
Functors $G \colon \cat D^\ep \times \cat C^\ep \to \cat E^\ep$
will be obtained from ``mixed-variance'' functors
\[
\begin{array}{*{5}{l}}
  F
& :
& \cat D^\op \times \cat C
& \longto
& \cat E
\end{array}
\]

\noindent
where
$\cat D,\cat C, \cat E$ are enriched over $\DCPO$.

%%%%%%%%%%%%%%%%%%%%%%%%%%%%%%%%%%%%%%%%%%%%%%%%%%%%%%%%%%%%%%%%%%%%%%%%%%%
\begin{definition}
\label{def:proof:scott:loc}
%%%%%%%%%%%%%%%%%%%%%%%%%%%%%%%%%%%%%%%%%%%%%%%%%%%%%%%%%%%%%%%%%%%%%%%%%%%
We say that $F$
is \emph{locally} \emph{monotone} (resp.\ \emph{continuous})
if each hom-function
\[
\begin{array}{r c l}
  \cat D(B',B) \times \cat C(A,A')
& \longto
& \cat C(F(B,A), F(B',A'))
\\

  (g,f)
& \longmapsto
& F(g,f)
\end{array}
\]

\noindent
is monotone (resp.\ Scott-continuous).
%%%%%%%%%%%%%%%%%%%%%%%%%%%%%%%%%%%%%%%%%%%%%%%%%%%%%%%%%%%%%%%%%%%%%%%%%%%
\end{definition}
%%%%%%%%%%%%%%%%%%%%%%%%%%%%%%%%%%%%%%%%%%%%%%%%%%%%%%%%%%%%%%%%%%%%%%%%%%%

\noindent
We refer to \cite[Definition 5.2.5]{aj95chapter},
\cite[Definition 7.1.15]{ac98book}
and \cite[Definition 9.1]{streicher06book}.
%
The following is a straightforward adaptation of \cite[Proposition 7.1.19]{ac98book}
(see also \cite[Proposition 5.2.6]{aj95chapter}).

%%%%%%%%%%%%%%%%%%%%%%%%%%%%%%%%%%%%%%%%%%%%%%%%%%%%%%%%%%%%%%%%%%%%%%%%%%%
\begin{lemma}
\label{lem:proof:scott:lift}
%%%%%%%%%%%%%%%%%%%%%%%%%%%%%%%%%%%%%%%%%%%%%%%%%%%%%%%%%%%%%%%%%%%%%%%%%%%
Let
$F \colon \cat D^\op \times \cat C \to \cat E$
be locally monotone.
Then $F$ lifts to a covariant functor
\[
\begin{array}{*{5}{l}}
  F^\ep
& :
& \cat D^\ep \times \cat C^\ep
& \longto
& \cat E^\ep
\end{array}
\]

\noindent
with $F^\ep(B,A) = F(B,A)$ on objects and
$F^\ep(g,f) = (F(\radj g,\ladj f) \,,\, F(\ladj g,\radj f))$
on morphisms.

If moreover $F$ is locally continuous, then $F^\ep$
preserves colimits of $\omega$-chains.
%%%%%%%%%%%%%%%%%%%%%%%%%%%%%%%%%%%%%%%%%%%%%%%%%%%%%%%%%%%%%%%%%%%%%%%%%%%
\end{lemma}
%%%%%%%%%%%%%%%%%%%%%%%%%%%%%%%%%%%%%%%%%%%%%%%%%%%%%%%%%%%%%%%%%%%%%%%%%%%



%%%%%%%%%%%%%%%%%%%%%%%%%%%%%%%%%%%%%%%%%%%%%%%%%%%%%%%%%%%%%%%%%%%%%%%%%%%
\subsubsection{Interpretation of Pure Types.}
%%%%%%%%%%%%%%%%%%%%%%%%%%%%%%%%%%%%%%%%%%%%%%%%%%%%%%%%%%%%%%%%%%%%%%%%%%%
A \emph{pure type expression} is a possibly open production of the
grammar of pure types (\S\ref{sec:pure}), namely
\[
\begin{array}{r @{\ \ }c@{\ \ } l}
     \PT
&    \bnf
&    \BT
\gss \PT \times \PT
\gss \PT \arrow \PT
\gss \TV
\gss \rec \TV.\PT 
\end{array}
\]

\noindent
where $\BT \in \Base$,
where $\TV$ is a type variable,
and where $\rec\TV.\PT$ binds $\TV$ in $\PT$.

Consider a pure type expression $\PT$ with free
type variables $\vec \TV = \TV_1,\dots,\TV_n$.
We are going to interpret $\PT$ as a functor
\[
\begin{array}{*{5}{l}}
  \I\PT
& :
& \left( \Scott^\ep \right)^{n}
& \longto
& \Scott^\ep
\end{array}
\]

\noindent
which preserves colimits of $\omega$-chains.


%%%%%%%%%%%%%%%%%%%%%%%%%%%%%%%%%%%%%%%%%%%%%%%%%%%%%%%%%%%%%%%%%%%%%%%%%%%
\paragraph{Preliminaries.}
%%%%%%%%%%%%%%%%%%%%%%%%%%%%%%%%%%%%%%%%%%%%%%%%%%%%%%%%%%%%%%%%%%%%%%%%%%%
Recall that the category $\Scott$ is Cartesian-closed
(products and homsets are equipped with pointwise orders),
see \cite[Corollary 4.1.6]{aj95chapter} or \cite[\S 1.4]{ac98book}.
This yields functors
$\Scott(\pl,\pl) \colon \Scott^\op \times \Scott \to \Scott$
and
$(\pl) \times (\pl) \colon \Scott \times \Scott \to \Scott$.
These functors are locally continuous
(\cite[Example 7.1.16]{ac98book}).
By combining Lemma~\ref{lem:proof:scott:enrich},
Lemma~\ref{lem:proof:scott:lift} and Lemma~\ref{lem:proof:scott:contdiag},
we obtain functors
\[
\begin{array}{*{5}{l}}
  \left(\Scott(\pl,\pl) \right)^\ep
  ,\,
  \left( (\pl) \times (\pl) \right)^\ep
& :
& \Scott^\ep \times \Scott^\ep
& \longto
& \Scott^\ep
\end{array}
\]

\noindent
which preserve colimits of $\omega$-chains.

Moreover, $\Scott$ has limits of $\omega^\op$-chains
of projections (in the embedding-projection sense),
see \cite[Theorem 3.3.7, Theorem 3.3.11 and Proposition 4.1.3]{aj95chapter}.
%
More precisely, the full inclusion $\Scott \emb \DCPO$
creates limits for $\omega^\op$-chains of projections.%
\footnote{The notion of creation of limits has to be understood in the usual
sense of \cite[Definition V.1]{maclane98book}.}
%
In particular, $\Scott$
is closed in $\DCPO$ under limits of $\omega^\op$-chains of projections.
%
Note that the category $\DCPO$ has all limits,
and that they are created by the forgetful functor to the category
of posets (and monotone functions),
see \cite[Theorem 3.3.1]{aj95chapter}.
%
It follows that given $K \colon \omega \to \cat \Scott^\ep$,
the limit of $\radj K \colon \omega^\op \to \Scott$
is
\[
  \left\{
  (x_i)_i \in \prod_{i \in \NN} K(i)
  \mathrel{\Big|}
  \radj K(i \leq j)(x_j) = x_i
  \right\}
\]

\noindent
equipped with the pointwise order.
Moreover, the limiting
cone $\Lim \radj K \to \radj K$ 
consists in set-theoretic projections.%
\footnote{These are also projections in the embedding-projection sense
by Theorem~\ref{thm:proof:scott:limcolim}.}
In view of Theorem~\ref{thm:proof:scott:limcolim},
we also get that $\Scott$ is closed in $\DCPO$
under colimits of $\omega$-chains of embeddings.

The terminal object $\one$ of $\Scott$ is initial in $\Scott^\ep$
(\cite[Proposition 7.1.9]{ac98book}).


%%%%%%%%%%%%%%%%%%%%%%%%%%%%%%%%%%%%%%%%%%%%%%%%%%%%%%%%%%%%%%%%%%%%%%%%%%%
\paragraph{Definition of the Interpretation.}
%%%%%%%%%%%%%%%%%%%%%%%%%%%%%%%%%%%%%%%%%%%%%%%%%%%%%%%%%%%%%%%%%%%%%%%%%%%
Let $\PT$ be a (pure) type expression with free
type variables $\vec \TV = \TV_1,\dots,\TV_n$.
The interpretation $\I\PT \colon \left(\Scott^\ep\right)^n \to \Scott^\ep$
is defined by induction on $\PT$.
\begin{itemize}
\item
In the case of $\PT = \TV_i$,
we let $\I\PT$ take $\vec X = X_1,\dots,X_n$ to $X_i$.

\item
In the case of $\BT \in \Base$,
we let $\I\PT(\vec X)$ be the flat domain $\BT_\bot$,
where $\BT_\bot$ is $\BT + \{\bot\}$ with $\BT$ discrete.


\item
In the cases of $\PT \times \PTbis$
and $\PTbis \arrow \PT$,
the induction hypotheses give us
functors
\[
\begin{array}{*{5}{l}}
  \I\PT, \I\PTbis
& :
& \left( \Scott^\ep \right)^{n}
& \longto
& \Scott^\ep
\end{array}
\]

\noindent
which preserve colimits of $\omega$-chains.
We can thus set
\[
\begin{array}{r c l}
  \I{\PTbis \arrow \PT}(\vec X)
& =
& \left( \Scott \left( \I{\PTbis}(\vec X) ,\, \I{\PT}(\vec X) \right) \right)^\ep
\\

  \I{\PT \times \PTbis}(\vec X)
& =
& \left(
  \I\PT(\vec X) \times \I\PTbis(\vec X)
  \right)^\ep
\end{array}
\]


\item
In the case of $\rec \TV.\PT$,
the induction hypothesis gives a functor
\[
\begin{array}{*{5}{l}}
  \I\PT
& :
& \left( \Scott^\ep \right)^{n} \times \Scott^\ep
& \longto
& \Scott^\ep
\end{array}
\]

\noindent
which preserves colimits of $\omega$-chains.
Theorem~\ref{thm:proof:scott:limcolim}
gives a functor
\[
\begin{array}{l l r c l}
  \I{\rec \TV.\PT}
& :
& \left( \Scott^\ep \right)^{n} 
& \longto
& \Scott^\ep
\\

&
& \vec X
& \longmapsto
& \Fix (\I{\PT}(\vec X))
\end{array}
\]

\noindent
This functor preserves colimits of $\omega$-chains by
Proposition~\ref{prop:proof:scott:contfunct}.
%
Moreover, since $\I\PT$ preserves
colimits of $\omega$-chains,
we obtain canonical isomorphisms
\(
  \I\fold
  :
  \I{\PT[\rec \TV.\PT/\TV]}(\vec X)
  \rightleftarrows
  \I{\rec \TV.\PT}(\vec X)
  :
  \I\unfold
\)
by taking
$\I\fold = \ladj{(\fold^\ep)}$
and
$\I\unfold = \ladj{(\unfold^\ep)}$.
\end{itemize}



%%%%%%%%%%%%%%%%%%%%%%%%%%%%%%%%%%%%%%%%%%%%%%%%%%%%%%%%%%%%%%%%%%%%%%%%%%%
\begin{figure}[t!]
%%%%%%%%%%%%%%%%%%%%%%%%%%%%%%%%%%%%%%%%%%%%%%%%%%%%%%%%%%%%%%%%%%%%%%%%%%%
\[
\begin{array}{c}

\dfrac{}
  {\bot \in \Fin(\I\PT(\vec X))}  


\qquad\qquad

\dfrac{\text{$\BT \in \Base$ and $a \in \BT$}}
  {a \in \Fin(\I\BT(\vec X))}

\qquad\qquad

\dfrac{\text{$d$ finite in $X_i$}}
  {d \in \Fin(\I{\TV_i}(\vec X))}

\\\\

\dfrac{d \in \Fin(\I\PT(\vec X))
  \qquad
  e \in \Fin(\I\PTbis(\vec X))}
  {(d,e) \in \Fin(\I{\PT \times \PTbis}(\vec X))}

\qquad\qquad

\dfrac{d \in \Fin(\I{\PT[\rec\TV.\PT/\TV]}(\vec X))}
  {\I\fold(d) \in \Fin(\I{\rec\TV.\PT}(\vec X))}

\\\\

\dfrac{\begin{array}{l}
  \text{for all $i \in I$,~
  $d_i \in \Fin(\I{\PT})(\vec X)$
  ~and~
  $e_i \in \Fin(\I{\PTbis})(\vec X)$ \@;}
  \\
  \text{for all $J \sle I$,~
  $\bigvee_{j \in J} d_j$ defined in $\I{\PT}(\vec X)$
  ~$\imp$~
  $\bigvee_{j \in J} e_j$ defined in $\I{\PTbis}(\vec X)$}
  \end{array}}
  {\bigvee_{i \in I}(d_i \step e_i) \in \Fin(\I{\PT \arrow \PTbis}(\vec X))}
~(\text{$I$ finite})

\end{array}
\]
\caption{Inductive description of the finite elements of $\I\PT(\vec X)$.%
\label{fig:proof:sem:finelt}}
%%%%%%%%%%%%%%%%%%%%%%%%%%%%%%%%%%%%%%%%%%%%%%%%%%%%%%%%%%%%%%%%%%%%%%%%%%%
\end{figure}
%%%%%%%%%%%%%%%%%%%%%%%%%%%%%%%%%%%%%%%%%%%%%%%%%%%%%%%%%%%%%%%%%%%%%%%%%%%

%%%%%%%%%%%%%%%%%%%%%%%%%%%%%%%%%%%%%%%%%%%%%%%%%%%%%%%%%%%%%%%%%%%%%%%%%%%
\paragraph{Description of the Finite Elements.}
%%%%%%%%%%%%%%%%%%%%%%%%%%%%%%%%%%%%%%%%%%%%%%%%%%%%%%%%%%%%%%%%%%%%%%%%%%%
For each (pure) type expression $\PT$
with free variables $\vec\PT = \PT_1,\dots,\PT_n$,
we define a set $\Fin(\I{\PT}(\vec X))$.
The definition is by induction on derivations with
the rules in Figure~\ref{fig:proof:sem:finelt}.
The set $\Fin(\I{\PT}(\vec X))$
describes the finite elements of $\I{\PT}(\vec X)$.
This relies on the following.

Given $\BT \in \Base$,
the finite elements of the flat domain $\I\BT$
are exactly the elements of $\BT$.

Let $X,Y \in \Scott$.
The finite elements in the product $X \times Y$
are exactly the pairs of finite elements.
The finite elements of $\Scott(X,Y)$ are exactly the finite sups of step functions.
Given finite $d \in X$ and $e \in Y$,
the \emph{step function} $(d \step e) \colon X \to Y$
is defined as $(d \step e)(x) = e$ if $x \geq d$ and
$(d \step e)(x)= \bot$ otherwise.
Recall that the sup $\bigvee_{i \in I}(d_i \step e_i)$ of 
a finite family of step functions exists
%precisely when
if, and only if,
for every $J \sle I$, the set $\{e_j \mid j \in J\}$ has an upper bound
whenever so does $\{d_j \mid j \in J\}$.
See \cite[Theorem 1.4.12]{ac98book}.


Concerning recursive types, let 
\[
\begin{array}{*{5}{l}}
  G
& :
& \Scott^\ep
& \longto
& \Scott^\ep
\end{array}
\]

\noindent
be a functor which preserves colimits of $\omega$-chains.
Recall that $\Fix G$ is the colimit in \eqref{diag:proof:scott:colim}
where $K \colon \omega \to \Scott^\ep$
takes $n$ to $G^n(\one)$
(similarly as in \eqref{diag:proof:scott:chain}).
We have seen that $\Scott$ is closed in $\DCPO$
under colimits of $\omega$-chains of embeddings.
Hence it follows from \cite[Theorem 3.3.11]{aj95chapter}
that the finite elements of $\Fix G$
are the images of the finite elements of the $G^n(\one)$'s
under the components of the colimiting cocone
$\gamma \colon K \to \Fix G$.

We thus have the following.

%%%%%%%%%%%%%%%%%%%%%%%%%%%%%%%%%%%%%%%%%%%%%%%%%%%%%%%%%%%%%%%%%%%%%%%%%%%
\begin{proposition}
\label{prop:proof:scott:fin}
%%%%%%%%%%%%%%%%%%%%%%%%%%%%%%%%%%%%%%%%%%%%%%%%%%%%%%%%%%%%%%%%%%%%%%%%%%%
$\Fin(\I\PT(\vec X))$ is the set of finite elements of $\I\PT(\vec X)$.
%%%%%%%%%%%%%%%%%%%%%%%%%%%%%%%%%%%%%%%%%%%%%%%%%%%%%%%%%%%%%%%%%%%%%%%%%%%
\end{proposition}
%%%%%%%%%%%%%%%%%%%%%%%%%%%%%%%%%%%%%%%%%%%%%%%%%%%%%%%%%%%%%%%%%%%%%%%%%%%





%%%%%%%%%%%%%%%%%%%%%%%%%%%%%%%%%%%%%%%%%%%%%%%%%%%%%%%%%%%%%%%%%%%%%%%%%%%
\paragraph{Example.}
%%%%%%%%%%%%%%%%%%%%%%%%%%%%%%%%%%%%%%%%%%%%%%%%%%%%%%%%%%%%%%%%%%%%%%%%%%%
We now provide some details on Example~\ref{ex:scott:stream-tree}
on $\I{\Stream\PTbis}$ and $\I{\Tree\PTbis}$,
where $\PTbis$ is a pure type.
We handle streams and binary trees
uniformly by considering the covariant functor
\[
\begin{array}{l l r c l}
  F
& :
& \Scott
& \longto
& \Scott
\\

&
& X
& \longmapsto
& \I\PTbis \times X^\Dir
\end{array}
\]

\noindent
where $\Dir$ is a finite set.
In view of Theorem~\ref{thm:proof:scott:limcolim},
$\Fix F$ is the limit of the $\omega^\op$-chain
\[
\begin{tikzcd} %[column sep=large]
  \one
& F(\one)
  \arrow{l}[above]{\one}
& F^2(\one)
  \arrow{l}[above]{F(\one)}
& F^n(\one)
  \arrow[dashed]{l}
& F^{n+1}(\one)
  \arrow{l}[above]{F^n(\one)}
& \phantom{F}
  \arrow[dashed]{l}
\end{tikzcd}
\]

\noindent
where $\one$ is the terminal Scott domain $\{\bot\}$.
Hence, $\Fix F$ is
\[
\begin{array}{c}
  \left\{
  x \in \prod_{n \in \NN} F^n(\one)
  \mathrel{\Big|}
  x(n) = F^n(\one)(x(n+1))
  \right\}
\end{array}
\]

\noindent
equipped with the pointwise order.
We show that $\Fix F$ is isomorphic to $\I\PTbis^{\Dir^*}$.
Define for each $n \in \NN$ an
isomorphism $\iota_n \colon \I\PTbis^{\Dir^n} \to F^n(\one)$
as $\iota_0 = \one \colon \one \to \one$
and
\[
\begin{array}{l l r c l}
  \iota_{n+1}
& :
& \I\PTbis^{\Dir^{n+1}}
& \longto
& F^{n+1}(\one) = \I\PTbis \times \left( F^n(\one) \right)^\Dir
\\

&
& T
& \longmapsto
& \left( T(\es), (\iota_n(u \mapsto T(d \cdot u)))_{d \in \Dir}  \right)
\end{array}
\]

\noindent
and to observe that the following commutes
\[
\begin{tikzcd}
  \I\PTbis^{\Dir^{n+1}}
  \arrow{d}[left]{T \mapsto T\restr \Dir^n}
  \arrow{r}{\iota_{n+1}}
& F^{n+1}(\one)
  \arrow{d}{F^n(\one)}
\\
  \I\PTbis^{\Dir^{n}}
  \arrow{r}[below]{\iota_{n}}
& F^n(\one)
\end{tikzcd}
\]

The characterization of the finite elements then follows from
Proposition~\ref{prop:proof:scott:fin}.





%%%%%%%%%%%%%%%%%%%%%%%%%%%%%%%%%%%%%%%%%%%%%%%%%%%%%%%%%%%%%%%%%%%%%%%%%%%
\subsection{\nameref{sec:sem:log}}
\label{sec:proof:sem:log}
%%%%%%%%%%%%%%%%%%%%%%%%%%%%%%%%%%%%%%%%%%%%%%%%%%%%%%%%%%%%%%%%%%%%%%%%%%%

First note that if $\triangle$ is either $\pi_1$, $\pi_2$ or $\fold$,
then
since $\I{\form\triangle}$ acts by inverse image
(of resp.\ $\pi_1$, $\pi_2$ and $\I\unfold$), we directly
have that $\I{\form\triangle}$
is monotone (w.r.t.\ inclusion) and preserves all unions and all intersections.

We now consider Lemma~\ref{lem:top:char:fin}.

%%%%%%%%%%%%%%%%%%%%%%%%%%%%%%%%%%%%%%%%%%%%%%%%%%%%%%%%%%%%%%%%%%%%%%%%%%%
\begin{lemma}[Lemma~\ref{lem:top:char:fin}]
\label{lem:proof:top:char:fin}
%%%%%%%%%%%%%%%%%%%%%%%%%%%%%%%%%%%%%%%%%%%%%%%%%%%%%%%%%%%%%%%%%%%%%%%%%%%
Given $\varphi \in \Lang_\land(\PT)$, if $\I\varphi \neq \emptyset$ then
$\I\varphi = \up d$ for some finite $d \in \I\PT$.
Conversely, if $d \in \I\PT$ is finite, then $\up d = \I\varphi$ for some
$\varphi \in \Lang_{\land}(\PT)$.
%%%%%%%%%%%%%%%%%%%%%%%%%%%%%%%%%%%%%%%%%%%%%%%%%%%%%%%%%%%%%%%%%%%%%%%%%%%
\end{lemma}
%%%%%%%%%%%%%%%%%%%%%%%%%%%%%%%%%%%%%%%%%%%%%%%%%%%%%%%%%%%%%%%%%%%%%%%%%%%

The proof of Lemma~\ref{lem:proof:top:char:fin} is split into the next
two lemmas.

%%%%%%%%%%%%%%%%%%%%%%%%%%%%%%%%%%%%%%%%%%%%%%%%%%%%%%%%%%%%%%%%%%%%%%%%%%%
\begin{lemma}
\label{lem:proof:top:fin:compact-open}
%%%%%%%%%%%%%%%%%%%%%%%%%%%%%%%%%%%%%%%%%%%%%%%%%%%%%%%%%%%%%%%%%%%%%%%%%%%
Given $\varphi \in \Lang_\land(\PT)$, if $\I\varphi \neq \emptyset$ then
$\I\varphi = \up d$ for some finite $d \in \I\PT$.
%%%%%%%%%%%%%%%%%%%%%%%%%%%%%%%%%%%%%%%%%%%%%%%%%%%%%%%%%%%%%%%%%%%%%%%%%%%
\end{lemma}
%%%%%%%%%%%%%%%%%%%%%%%%%%%%%%%%%%%%%%%%%%%%%%%%%%%%%%%%%%%%%%%%%%%%%%%%%%%

%%%%%%%%%%%%%%%%%%%%%%%%%%%%%%%%%%%%%%%%%%%%%%%%%%%%%%%%%%%%%%%%%%%%%%%%%%%
\begin{proof}
%%%%%%%%%%%%%%%%%%%%%%%%%%%%%%%%%%%%%%%%%%%%%%%%%%%%%%%%%%%%%%%%%%%%%%%%%%%
The proof is by induction on $\varphi \in \Lang_\land(\PT)$.
We rely on the description of finite elements given
by Proposition~\ref{prop:proof:scott:fin}
(see Figure~\ref{fig:proof:sem:finelt}).
\begin{description}
\item[Case of $\True$.]
In this case, we have $\I\varphi = \up \bot$.

\item[Case of $\varphi \land \psi$.]
First, note that $\I\varphi, \I\psi$ are non-empty since so is their intersection.
By induction hypothesis, there are finite $d,e \in \I\PT$
such that $\I\varphi = \up d$ and $\I\psi = \up e$.
Since $\up d \cap \up e$ is non-empty, and since $\I\PT$ is a Scott domain,
we get that $d \vee e$ is defined, finite, and such that
$\up(d\vee e) = \up d \cap \up e$.
Hence $\I{\varphi \land \psi} = \up(d \vee e)$.

\item[Case of $\form a$ (with $a \in \BT$ for $\BT \in \Base$).]
Since $\I{\form a} = \up a$.

\item[Case of $\form{\triangle}\varphi$
with $\triangle$ either $\pi_1$, $\pi_2$ or $\fold$.]
Note that $\I\varphi$ is non-empty since so is
$\I{\form\triangle\varphi} = \I{\form\triangle}(\I\varphi)$.
Hence by induction hypothesis, there is some finite $d$
such that $\I\varphi = \up d$.

Consider first the case of $\triangle = \fold$.
Then, since $\I\unfold$ is an isomorphism with inverse $\I\fold$
we have
\[
\begin{array}{l l l}
  \I{\form\triangle\varphi}
& =
& \I{\form\triangle}(\up d)
\\

& =
& \left\{
  x \in \I{\rec\TV.\PT} \mid \I\unfold(x) \geq d
  \right\}
\\

& =
& \up \I\fold(d)
\end{array}
\]

\noindent
The result then follows from
Proposition~\ref{prop:proof:scott:fin}.

Consider now the case of $\triangle = \pi_i$,
say $\triangle = \pi_1$ (the other case is symmetric).
Since the order in $\I{\PT_1 \times \PT_2}$ is pointwise,
we have
\[
\begin{array}{l l l}
  \I{\form{\pi_1}\varphi}
& =
& \I{\form{\pi_1}}(\up d)

\\
& =
& \left\{
  x \in \I{\PT_1 \times \PT_2} \mid \pi_1(x) \geq d
  \right\}
\\

& =
& \up(d,\bot)
\end{array}
\]

\noindent
and the result again follows from
Proposition~\ref{prop:proof:scott:fin}.

\item[Case of $\psi \realto \varphi$.]
First, if $\I\psi = \emptyset$,
then $\I{\psi \realto \varphi} = \up \bot$.

Assume now that $\I\psi \neq \emptyset$.
In this case, we must also have $\I\varphi \neq \emptyset$.
Hence by induction hypothesis there are $d,e$ finite
such that
$\up e = \I\psi$ and $\up d = \I\varphi$.
Then we are done since
\[
\begin{array}{l l l}
  \I{\psi \realto \varphi}
& =
& \left\{ f \mid \forall x \in \I\psi,~ f(x) \in \I\varphi\right\}
\\

& =
& \left\{ f \mid \forall x \geq e,~ f(x) \geq d\right\}
\\

& =
& \up \left(e \step d \right)
\end{array}
\]
\qed
\end{description}
%%%%%%%%%%%%%%%%%%%%%%%%%%%%%%%%%%%%%%%%%%%%%%%%%%%%%%%%%%%%%%%%%%%%%%%%%%%
\end{proof}
%%%%%%%%%%%%%%%%%%%%%%%%%%%%%%%%%%%%%%%%%%%%%%%%%%%%%%%%%%%%%%%%%%%%%%%%%%%

%%%%%%%%%%%%%%%%%%%%%%%%%%%%%%%%%%%%%%%%%%%%%%%%%%%%%%%%%%%%%%%%%%%%%%%%%%%
\begin{lemma}
\label{lem:proof:top:compact-open:fin}
%%%%%%%%%%%%%%%%%%%%%%%%%%%%%%%%%%%%%%%%%%%%%%%%%%%%%%%%%%%%%%%%%%%%%%%%%%%
If $d \in \I\PT$ is finite,
then there is
$\varphi \in \Lang_\land(\PT)$
such that $\up d = \I\varphi$.
%%%%%%%%%%%%%%%%%%%%%%%%%%%%%%%%%%%%%%%%%%%%%%%%%%%%%%%%%%%%%%%%%%%%%%%%%%%
\end{lemma}
%%%%%%%%%%%%%%%%%%%%%%%%%%%%%%%%%%%%%%%%%%%%%%%%%%%%%%%%%%%%%%%%%%%%%%%%%%%

%%%%%%%%%%%%%%%%%%%%%%%%%%%%%%%%%%%%%%%%%%%%%%%%%%%%%%%%%%%%%%%%%%%%%%%%%%%
\begin{proof}
%%%%%%%%%%%%%%%%%%%%%%%%%%%%%%%%%%%%%%%%%%%%%%%%%%%%%%%%%%%%%%%%%%%%%%%%%%%
We rely on Proposition~\ref{prop:proof:scott:fin}
and on the inductive definition of $\Fin(\I\PT)$ in Figure~\ref{fig:proof:sem:finelt}.
We reason by cases on the rules 
in Figure~\ref{fig:proof:sem:finelt}.
\begin{description}
\item[Case of]
\[
\dfrac{}
  {\bot \in \Fin(\I\PT)}  
\]

\noindent
Since $\up \bot = \I{\True}$.

\item[Case of]
\[
\dfrac{\text{$\BT \in \Base$ and $a \in \BT$}}
  {a \in \Fin(\I\BT)}
\]

\noindent
Since $\up a = \I{\form a}$.

\item[Case of]
\[
\dfrac{d \in \Fin(\I\PT)
  \qquad
  e \in \Fin(\I\PTbis)}
  {(d,e) \in \Fin(\I{\PT \times \PTbis})}
\]

\noindent
By induction hypothesis, we
have $\varphi \in \Lang_\land(\PT)$
and $\psi \in \Lang_\land(\PTbis)$
such that
$\I\varphi = \up d$
and
$\I\psi = \up e$.
Since the order in $\I{\PT \times \PTbis}$
is pointwise, we get
\[
\begin{array}{l l l}
  \up(d,e)
& =
& \up d \times \up e
\\

& =
& (\up d \times \I\PTbis)
  \cap
  (\I\PT \times \up e)
\\

& =
& \I{\form{\pi_1}\varphi \land \form{\pi_2}\psi}
\end{array}
\]

\item[Case of]
\[
\dfrac{d \in \Fin(\I{\PT[\rec\TV.\PT/\TV]})}
  {\I\fold(d) \in \Fin(\I{\rec\TV.\PT})}
\]

\noindent
By induction hypothesis, there is
$\varphi \in \Lang_\land(\PT[\rec\TV.\PT/\TV])$
such that $\I\varphi = \up d$.
We thus have $\up \I\fold(d) = \I{\form\fold \varphi}$.

\item[Case of]
\[
\dfrac{\begin{array}{l}
  \text{for all $i \in I$,~
  $d_i \in \Fin(\I{\PT})$
  ~and~
  $e_i \in \Fin(\I{\PTbis})$ \@;}
  \\
  \text{for all $J \sle I$,~
  $\bigvee_{j \in J} d_j$ defined in $\I{\PT}$
  ~$\imp$~
  $\bigvee_{j \in J} e_j$ defined in $\I{\PTbis}$}
  \end{array}}
  {\bigvee_{i \in I}(d_i \step e_i) \in \Fin(\I{\PT \arrow \PTbis})}
\]

\noindent
where $I$ is a finite set.

By induction hypothesis, for each $i \in I$
there are $\varphi_i \in \Lang_\land(\PTbis)$
and $\psi_i \in \Lang_\land(\PT)$
such that $\up e_i = \I{\varphi_i}$
and $\up d_i = \I{\psi_i}$.
Note that
\[
\begin{array}{l l l}
  \up(d_i \step e_i)
& =
& \left\{
  f \colon \I\PT \to \I\PTbis \mid
  \forall x \geq d_i,~
  f(x) \geq e_i
  \right\}
\\

& =
& \left\{
  f \colon \I\PT \to \I\PTbis \mid
  \forall x \in \I{\psi_i},~
  f(x) \in \I{\varphi_i}
  \right\}
\\

& =
& \I{\psi_i \realto \varphi_i}
\end{array}
\]

\noindent
The result then follows from the fact that
\[
\begin{array}{l l l}
  \up \left(
  \bigvee_{i \in I} d_i \step e_i
  \right)
& =
& \bigcap_{i \in I} \up(d_i \step e_i)
\end{array}
\]
\qed
\end{description}
%%%%%%%%%%%%%%%%%%%%%%%%%%%%%%%%%%%%%%%%%%%%%%%%%%%%%%%%%%%%%%%%%%%%%%%%%%%
\end{proof}
%%%%%%%%%%%%%%%%%%%%%%%%%%%%%%%%%%%%%%%%%%%%%%%%%%%%%%%%%%%%%%%%%%%%%%%%%%%


We now turn to Lemma~\ref{lem:top:char}.
We first recall its statement.


%%%%%%%%%%%%%%%%%%%%%%%%%%%%%%%%%%%%%%%%%%%%%%%%%%%%%%%%%%%%%%%%%%%%%%%%%%%
\begin{lemma}[Lemma~\ref{lem:top:char}]
\label{lem:proof:top:char}
%%%%%%%%%%%%%%%%%%%%%%%%%%%%%%%%%%%%%%%%%%%%%%%%%%%%%%%%%%%%%%%%%%%%%%%%%%%
A set $\SP \sle \I\PT$ is saturated (resp.\ Scott-open)
if, and only if, there is a formula $\varphi \in \Lang(\PT)$
(resp.\@ $\varphi \in \Lang_\Open(\PT)$)
such that $\SP = \I\varphi$.

In particular, for each $\varphi \in \Lang(\PT)$
we have $\I\varphi = \I\psi$ for some $\psi \in \Norm(\PT)$.
%%%%%%%%%%%%%%%%%%%%%%%%%%%%%%%%%%%%%%%%%%%%%%%%%%%%%%%%%%%%%%%%%%%%%%%%%%%
\end{lemma}
%%%%%%%%%%%%%%%%%%%%%%%%%%%%%%%%%%%%%%%%%%%%%%%%%%%%%%%%%%%%%%%%%%%%%%%%%%%

%%%%%%%%%%%%%%%%%%%%%%%%%%%%%%%%%%%%%%%%%%%%%%%%%%%%%%%%%%%%%%%%%%%%%%%%%%%
\begin{proof}
%%%%%%%%%%%%%%%%%%%%%%%%%%%%%%%%%%%%%%%%%%%%%%%%%%%%%%%%%%%%%%%%%%%%%%%%%%%
Consider first the case of a set $\SP \sle \I\PT$
which is open (resp.\ saturated).
Then $\SP$ is a union (resp.\ an intersection of unions) of
sets of the form $\up d$ with $d \in \I\PT$ finite.
Then result then follows from Lemma~\ref{lem:proof:top:compact-open:fin}
using the closure of $\Lang_\Open(\PT)$ under arbitrary disjunctions
(resp.\ the closure of $\Lang(\PT)$ under arbitrary disjunctions
and conjunctions).

The converse is proven by induction on formulae.
Since opens are stable under unions and finite intersections,
the case of $\varphi \in \Lang_\Open(\PT)$
directly follows from Lemma~\ref{lem:proof:top:fin:compact-open}.
As for $\varphi \in \Lang(\PT)$,
since saturated sets are stable under all unions and intersections,
we only have to consider the cases of modalities.
\begin{description}
\item[Case of $\form a$ (with $a \in \BT$ for $\BT \in \Base$).]
Trivial, since $\I{\form a}$ is compact open.

\item[Case of $\form{\triangle}\varphi$
with $\triangle$ either $\pi_1$, $\pi_2$ or $\fold$.]
Similarly as in Lemma~\ref{lem:proof:top:fin:compact-open},
one can apply the induction hypothesis and use the fact
that $\I{\form\triangle}$ preserves all unions and all intersections.

\item[Case of $\psi \realto \varphi$.]
By induction hypothesis, $\I\varphi$ is upward-closed.
Hence so is $\I{\psi \realto \varphi}$.
\end{description}

For the last part of the statement,
let $\varphi \in \Lang(\PT)$.
Since $\I\varphi$ is saturated,
it is an intersection of unions of sets
of the form $\up d$ with $d \in \I\PT$ finite.
Lemma~\ref{lem:proof:top:compact-open:fin} yields
that such $\up d$'s are definable in $\Lang_\land(\PT)$,
whence the result.
\qed
%%%%%%%%%%%%%%%%%%%%%%%%%%%%%%%%%%%%%%%%%%%%%%%%%%%%%%%%%%%%%%%%%%%%%%%%%%%
\end{proof}
%%%%%%%%%%%%%%%%%%%%%%%%%%%%%%%%%%%%%%%%%%%%%%%%%%%%%%%%%%%%%%%%%%%%%%%%%%%



We finally discuss Proposition~\ref{prop:sem:sound:ded}.

%%%%%%%%%%%%%%%%%%%%%%%%%%%%%%%%%%%%%%%%%%%%%%%%%%%%%%%%%%%%%%%%%%%%%%%%%%%
\begin{proposition}[Soundness of Deduction (Proposition \ref{prop:sem:sound:ded})]
\label{prop:proof:sem:sound:ded}
%%%%%%%%%%%%%%%%%%%%%%%%%%%%%%%%%%%%%%%%%%%%%%%%%%%%%%%%%%%%%%%%%%%%%%%%%%%
If $\psi \thesis \varphi$ is derivable in the basic deduction system 
in Figure~\ref{fig:log:ded} (\S\ref{sec:log}),
then $\I\psi \sle \I\varphi$.
%%%%%%%%%%%%%%%%%%%%%%%%%%%%%%%%%%%%%%%%%%%%%%%%%%%%%%%%%%%%%%%%%%%%%%%%%%%
\end{proposition}
%%%%%%%%%%%%%%%%%%%%%%%%%%%%%%%%%%%%%%%%%%%%%%%%%%%%%%%%%%%%%%%%%%%%%%%%%%%

%%%%%%%%%%%%%%%%%%%%%%%%%%%%%%%%%%%%%%%%%%%%%%%%%%%%%%%%%%%%%%%%%%%%%%%%%%%
\begin{proof}
%%%%%%%%%%%%%%%%%%%%%%%%%%%%%%%%%%%%%%%%%%%%%%%%%%%%%%%%%%%%%%%%%%%%%%%%%%%
The proof is by induction on $\psi \thesis \varphi$,
and by cases on the rules in Figure~\ref{fig:log:ded}.
The cases of the rules for (infinitary) propositional
logic directly follow from the definition of the interpretation.
So we just have to discuss modalities.

Let $\triangle$ be either $\pi_1$, $\pi_2$ or $\fold$.
Since $\I{\form\triangle}$ acts by inverse image
(of resp.\ $\pi_1$, $\pi_2$ and $\I\unfold$), we directly
have that $\I{\form\triangle}$
is monotone (w.r.t.\ inclusion) and preserves all unions and all intersections.
This handles all the rules for $\form\triangle$.

The rule $\ax{F}$ has already been discussed,
and the other rules for $\realto$ are straightforward to check.
\qed
%%%%%%%%%%%%%%%%%%%%%%%%%%%%%%%%%%%%%%%%%%%%%%%%%%%%%%%%%%%%%%%%%%%%%%%%%%%
\end{proof}
%%%%%%%%%%%%%%%%%%%%%%%%%%%%%%%%%%%%%%%%%%%%%%%%%%%%%%%%%%%%%%%%%%%%%%%%%%%




}
\opt{full}{%%%%%%%%%%%%%%%%%%%%%%%%%%%%%%%%%%%%%%%%%%%%%%%%%%%%%%%%%%%%%%%%%%%%%%%%%%%
\section{Proofs of \S\ref{sec:compl} (\nameref{sec:compl})}
\label{sec:proof:compl}
%%%%%%%%%%%%%%%%%%%%%%%%%%%%%%%%%%%%%%%%%%%%%%%%%%%%%%%%%%%%%%%%%%%%%%%%%%%


%%%%%%%%%%%%%%%%%%%%%%%%%%%%%%%%%%%%%%%%%%%%%%%%%%%%%%%%%%%%%%%%%%%%%%%%%%%
\subsection{\nameref{sec:compl:fin}}
\label{sec:proof:compl:fin}
%%%%%%%%%%%%%%%%%%%%%%%%%%%%%%%%%%%%%%%%%%%%%%%%%%%%%%%%%%%%%%%%%%%%%%%%%%%

We begin with Proposition~\ref{prop:compl:fin:ded}.
We first recall the content of \eqref{eq:compl:cc}:
\begin{equation}
\label{eq:proof:compl:cc}
\begin{array}{c}

\dfrac{}{\C(\True)}

\quad

\dfrac
  {\text{$\BT \in \Base$ and $a \in \BT$}}
  {\C(\form a)}

\quad

\dfrac{\C(\varphi)}
  {\C(\form\fold \varphi)}

\quad

\dfrac{\C(\varphi) 
  \quad
  \C(\psi)}
  {\C(\pair{\varphi,\psi})}

\quad

\dfrac{\C(\psi)
  \quad
  \psi \thesis \varphi}
  {\C(\varphi)}
\\\\

\ax{C}
\dfrac{\C(\psi)}
  {(\psi \realto \False) \,\thesis\, \False}

\quad

\dfrac{\begin{array}{l}
  \text{$I$ finite and $\forall i \in I$,}~
  \C(\psi_i) 
  ~\text{and}~
  \C(\varphi_i) ;
  \\
  \text{$\forall J \sle I$,}~
  \bigwedge_{j \in J} \psi_j \thesis \False
  ~~\text{or}~~
  \C\left( \bigwedge_{j \in J} \varphi_j \right)
  \end{array}}
  {\C\left( \bigwedge_{i \in I}(\psi_i \realto \varphi_i) \right)}
%~(\text{$I$ finite})
\end{array}
\end{equation}

We now recall the statement of Proposition~\ref{prop:compl:fin:ded}.

%%%%%%%%%%%%%%%%%%%%%%%%%%%%%%%%%%%%%%%%%%%%%%%%%%%%%%%%%%%%%%%%%%%%%%%%%%%
\begin{proposition}[Proposition \ref{prop:compl:fin:ded}]
\label{prop:proof:compl:fin:ded}
%%%%%%%%%%%%%%%%%%%%%%%%%%%%%%%%%%%%%%%%%%%%%%%%%%%%%%%%%%%%%%%%%%%%%%%%%%%
In the extension of Figure~\ref{fig:log:ded} (\S\ref{sec:log})
with \eqref{eq:proof:compl:cc}:
\begin{enumerate}[(1)]
\item
for all $\varphi,\psi \in \Lang_\land(\PT)$,
we have
$\psi \thesis_\PT \varphi$
if, and only if,
$\I\psi \sle \I\varphi$;

\item
for all $\varphi \in \Lang_\land$,
we have
$\C(\varphi)$
if, and only if,
$\I\varphi \neq \emptyset$.
\end{enumerate}
%%%%%%%%%%%%%%%%%%%%%%%%%%%%%%%%%%%%%%%%%%%%%%%%%%%%%%%%%%%%%%%%%%%%%%%%%%%
\end{proposition}
%%%%%%%%%%%%%%%%%%%%%%%%%%%%%%%%%%%%%%%%%%%%%%%%%%%%%%%%%%%%%%%%%%%%%%%%%%%

The proof of Proposition~\ref{prop:proof:compl:fin:ded}
is split into Lemmas \ref{lem:proof:compl:fin:ded:cor},
\ref{lem:proof:compl:fin:ded:c-false}
and
\ref{lem:proof:compl:fin:ded:compl}.
Namely:
\begin{itemize}
\item
Lemma~\ref{lem:proof:compl:fin:ded:cor} is the soundness of
the system made of Figure~\ref{fig:log:ded} (\S\ref{sec:log})
and \eqref{eq:proof:compl:cc} for formulae in $\Lang_\land$.

\item
Lemma~\ref{lem:proof:compl:fin:ded:c-false} is a form of dichotomy:
for every $\psi \in \Lang_\land$,
either $\I\psi = \emptyset$ and $\psi \thesis \False$ is derivable,
or $\I\psi \neq \emptyset$ and $\C(\psi)$ is derivable.

\item
Lemma~\ref{lem:proof:compl:fin:ded:compl} is the completeness
of the deduction relation for formulae in $\Lang_\land$.
\end{itemize}

%%%%%%%%%%%%%%%%%%%%%%%%%%%%%%%%%%%%%%%%%%%%%%%%%%%%%%%%%%%%%%%%%%%%%%%%%%%
\begin{lemma}
\label{lem:proof:compl:fin:ded:cor}
%%%%%%%%%%%%%%%%%%%%%%%%%%%%%%%%%%%%%%%%%%%%%%%%%%%%%%%%%%%%%%%%%%%%%%%%%%%
In the extension of Figure~\ref{fig:log:ded} (\S\ref{sec:log})
with \eqref{eq:proof:compl:cc}:
\begin{enumerate}[(1)]
\item
for all $\varphi,\psi \in \Lang_\land(\PT)$,
we have
$\I\psi \sle \I\varphi$
if
$\psi \thesis_\PT \varphi$;

\item
for all $\varphi \in \Lang_\land$,
we have
$\I\varphi \neq \emptyset$
if
$\C(\varphi)$.
\end{enumerate}
%%%%%%%%%%%%%%%%%%%%%%%%%%%%%%%%%%%%%%%%%%%%%%%%%%%%%%%%%%%%%%%%%%%%%%%%%%%
\end{lemma}
%%%%%%%%%%%%%%%%%%%%%%%%%%%%%%%%%%%%%%%%%%%%%%%%%%%%%%%%%%%%%%%%%%%%%%%%%%%





%%%%%%%%%%%%%%%%%%%%%%%%%%%%%%%%%%%%%%%%%%%%%%%%%%%%%%%%%%%%%%%%%%%%%%%%%%%
\begin{proof}
%%%%%%%%%%%%%%%%%%%%%%%%%%%%%%%%%%%%%%%%%%%%%%%%%%%%%%%%%%%%%%%%%%%%%%%%%%%
We reason by mutual induction the definition of $\thesis$ and $\C$.
Thanks to Proposition~\ref{prop:sem:sound:ded},
we do not have to consider the rules in Figure~\ref{fig:log:ded}.
We reason by cases on the last applied rule.
\begin{description}
\item[Cases of]
\[
\begin{array}{c}

\dfrac{}{\C(\True)}

\quad

\dfrac
  {\text{$\BT \in \Base$ and $a \in \BT$}}
  {\C(\form a)}

\quad

\dfrac{\C(\varphi)}
  {\C(\form\fold \varphi)}

\quad

\dfrac{\C(\psi)
  \quad
  \psi \thesis \varphi}
  {\C(\varphi)}

\end{array}
\]

\noindent
Trivial.

\item[Case of]
\[
\dfrac{\C(\varphi) 
  \quad
  \C(\psi)}
  {\C(\pair{\varphi,\psi})}
\]

\noindent
Recall that $\pair{\varphi,\psi} = \form{\pi_1}\varphi \land \form{\pi_2}\psi$.
Hence, if $\I{\pair{\varphi,\psi}} = \emptyset$ then we must have
either $\I\varphi = \emptyset$ or $\I\psi = \emptyset$,
and we conclude by induction hypothesis.

\item[Case of]
\[
\ax{C}
\dfrac{\C(\psi)}
  {(\psi \realto \False) \,\thesis\, \False}
\]

By induction hypothesis, we have $\I\psi \neq \emptyset$.
Hence $\I{\psi \realto \False} = \emptyset$.

\item[Case of]
\[
\dfrac{\begin{array}{l}
  \text{$I$ finite and $\forall i \in I$,}~
  \C(\psi_i) 
  ~\text{and}~
  \C(\varphi_i) ;
  \\
  \text{$\forall J \sle I$,}~
  \bigwedge_{j \in J} \psi_j \thesis \False
  ~\text{or}~
  \C\left( \bigwedge_{j \in J} \varphi_j \right)
  \end{array}}
  {\C\left( \bigwedge_{i \in I}(\psi_i \realto \varphi_i) \right)}
\]

Let $\PT,\PTbis$ such that
$\varphi_i \in \Lang_\land(\PT)$ and $\psi_i \in \Lang_\land(\PTbis)$
for all $i \in I$.

First, by induction hypothesis 
we have
$\I{\psi_i} \neq \emptyset$
and
$\I{\varphi_i} \neq \emptyset$
for all $i \in I$.
Hence, it follows from Lemma~\ref{lem:top:char:fin}
that for each $i \in I$,
there are finite $d_i \in \I\PT$
and $e_i \in \I\PTbis$
such that
$\up d_i = \I{\varphi_i}$
and
$\up e_i = \I{\psi_i}$.
We thus have
$\I{\psi_i \realto \varphi_i} = \up(e_i \step d_i)$
for each $i \in I$,
so that
\[
\begin{array}{l l l}
  \I{\bigwedge_{i \in I} \left( \psi_i \realto \varphi_i \right)}
& =
& \bigcap_{i \in I} \up(e_i \step d_i)
\end{array}
\]


Assume 
$\I{\bigwedge_{i \in I} \left( \psi_i \realto \varphi_i \right)} = \emptyset$.
As recalled in \S\ref{sec:proof:sem:pure}
(see also \cite[Theorem 1.4.12]{ac98book}),
there is some $J \sle I$
such that
\[
\begin{array}{l l l !{\quad\text{and}\quad} l l l}
  \bigcap_{i \in I} \up e_i
& \neq
& \emptyset

& \bigcap_{i \in I} \up d_i
& =
& \emptyset
\end{array}
\]

But the induction hypothesis
yields either $\bigcap_{i \in I} \up e_i = \emptyset$
or $\bigcap_{i \in I} \up d_i \neq \emptyset$,
a contradiction.
\qed
\end{description}
%%%%%%%%%%%%%%%%%%%%%%%%%%%%%%%%%%%%%%%%%%%%%%%%%%%%%%%%%%%%%%%%%%%%%%%%%%%
\end{proof}
%%%%%%%%%%%%%%%%%%%%%%%%%%%%%%%%%%%%%%%%%%%%%%%%%%%%%%%%%%%%%%%%%%%%%%%%%%%

%%%%%%%%%%%%%%%%%%%%%%%%%%%%%%%%%%%%%%%%%%%%%%%%%%%%%%%%%%%%%%%%%%%%%%%%%%%
\begin{lemma}
\label{lem:proof:compl:fin:ded:c-false}
%%%%%%%%%%%%%%%%%%%%%%%%%%%%%%%%%%%%%%%%%%%%%%%%%%%%%%%%%%%%%%%%%%%%%%%%%%%
For all $\psi \in \Lang_\land(\PT)$,
\begin{enumerate}[(1)]
\item
\label{item:proof:compl:fin:ded:c-false:c}
if $\I\psi \neq \emptyset$,
then $\C(\psi)$ is derivable;


\item
\label{item:proof:compl:fin:ded:c-false:ded}
if $\I\psi = \emptyset$,
then $\psi \thesis \False$ is derivable.
\end{enumerate}
%%%%%%%%%%%%%%%%%%%%%%%%%%%%%%%%%%%%%%%%%%%%%%%%%%%%%%%%%%%%%%%%%%%%%%%%%%%
\end{lemma}
%%%%%%%%%%%%%%%%%%%%%%%%%%%%%%%%%%%%%%%%%%%%%%%%%%%%%%%%%%%%%%%%%%%%%%%%%%%

%%%%%%%%%%%%%%%%%%%%%%%%%%%%%%%%%%%%%%%%%%%%%%%%%%%%%%%%%%%%%%%%%%%%%%%%%%%
\begin{proof}
%%%%%%%%%%%%%%%%%%%%%%%%%%%%%%%%%%%%%%%%%%%%%%%%%%%%%%%%%%%%%%%%%%%%%%%%%%%
Both statements are proven by a simultaneous induction on the (finite!)
size of $\psi \in \Lang_\land$.

Note that $\C(\True)$ and $\False \thesis \False$
are always derivable,
so that we may always assume $\psi \neq \True$ in
item \eqref{item:proof:compl:fin:ded:c-false:c}
and $\psi \neq \False$ in
item \eqref{item:proof:compl:fin:ded:c-false:ded}.

We now reason by cases on $\PT$.
\begin{description}
\item[Case of $\BT$ with $\BT \in \Base$.]
We begin with item \eqref{item:proof:compl:fin:ded:c-false:c}.
If $\I\psi \neq \emptyset$,
then we must have $\I\psi = \{a\}$ for some $a \in \BT$.
Hence $\psi$ is $\thesisiff$-equivalent to $\form a$,
and we get $\C(\psi)$ since $\C(\form a)$ and $\form a \thesis \psi$.

We now turn to item \eqref{item:proof:compl:fin:ded:c-false:ded}.
If $\psi \neq \False$, then it must be the case that
$\psi$ is a finite conjunction containing (at least)
$\form a$ and $\form b$ for some $a \neq b$ in $\BT$.
We can thus conclude using the rule $\form a \land \form b \thesis \False$.

\item[Case of $\rec\TV.\PT$.]
If $\psi \neq \False$,
then $\psi$ is $\thesisiff$-equivalent to a formula of the form
$\bigwedge_{j \in J}\form\fold \psi_j$ for some finite $J$,
where each $\psi_j$ is smaller than $\psi$.
Let $\psi'$ be the formula
$\bigwedge_{j \in J} \psi_j$.
Note that $\psi \thesisiff \form\fold\psi'$ by Example~\ref{ex:log:modalnf},
so that
$\I\psi = \I{\form\fold}(\I{\psi'})$.

We first consider item \eqref{item:proof:compl:fin:ded:c-false:c}.
If $\I\psi \neq \emptyset$,
then $\I{\psi'} \neq \emptyset$.
If moreover $J \neq \emptyset$ (otherwise $\psi = \True$),
then
$\psi'$ is smaller than $\psi$ and the induction hypothesis
yields $\C(\psi')$.
Hence $\C(\form\fold \psi')$
and we get $\C(\psi)$ since $\form\fold \psi' \thesis \psi$.

We now turn to item \eqref{item:proof:compl:fin:ded:c-false:ded}.
If $\I\psi = \emptyset$ then
$\I{\psi'} = \emptyset$.
In this case, $J$ must be non-empty
(since otherwise $\I{\psi'} = \I{\True}$).
So $\psi'$ is smaller than $\psi$
and the induction hypothesis yields
$\psi' \thesis_{\PT[\rec\TV.\PT/\TV]} \False$,
so that
$\form\fold\psi' \thesis_{\rec\TV.\PT} \form\fold \False$.
Then we are done since $\form\fold \False \thesis \False$
(take $I = \emptyset$ in the rule
$\form\triangle \bigvee (-) \thesis \bigvee \form\triangle(-)$).

\item[Case of $\PT_1 \times \PT_2$.]
If $\psi \neq \False$,
%
then $\psi$ is $\thesisiff$-equivalent to a formula of the form
\( 
  (\bigwedge_{j \in J} \form{\pi_1}\psi_j)
  \land
  (\bigwedge_{k \in K} \form{\pi_2}\psi_k)
\)
where $\psi_j,\psi_k$ are smaller than $\psi$,
and
where we can assume w.l.o.g.\ $J \cap K = \emptyset$.
Let $\psi' = \bigwedge_{j \in J} \psi_j$
and $\psi'' = \bigwedge_{k \in K} \psi_k$,
so that $\psi \thesisiff \form{\pi_1}\psi' \land \form{\pi_2}\psi''$.

We first consider item \eqref{item:proof:compl:fin:ded:c-false:c}.
If $\I\psi \neq \emptyset$,
then $\I{\psi'} \neq \emptyset$
and $\I{\psi''} \neq \emptyset$.
If $J$ (resp.\@ $K$) is non-empty,
then $\psi'$ (resp. $\psi''$) is smaller than $\psi$
and the induction hypothesis applies to yield
$\C(\psi')$ (resp.\ $\C(\psi'')$).
If $J$ (resp.\ $K$) is empty, then
$\psi' = \True$ (resp.\ $\psi'' = \True$),
so that $\C(\psi')$ (resp.\ $\C(\psi'')$).
%
Hence, in any case we get $\C(\psi')$ and $\C(\psi'')$,
so that $\C(\pair{\psi',\psi''})$ and we are done.

We now turn to item \eqref{item:proof:compl:fin:ded:c-false:ded}.
If  $\I\psi = \emptyset$,
then we must have either $\I{\psi'} = \emptyset$
or $\I{\psi''} = \emptyset$,
say  $\I{\psi'} = \emptyset$ (the other case is symmetric).
Reasoning similarly as above
yields $\psi' \thesis_{\PT_1} \False$ by induction hypothesis,
and we conclude using
$\form{\pi_1}\False \thesis_{\PT_1 \times \PT_2} \False$.

\item[Case of $\PTbis \arrow \PT$.]
If $\psi \neq \False$,
then $\psi$ is $\thesisiff$-equivalent to a formula of the form
$\bigwedge_{i \in I} (\psi''_i \realto \psi'_i)$,
where $\psi'_i, \psi''_i$ are smaller than $\psi$.

Assume first that for some $i \in I$,
we have $\I{\psi'_i} = \emptyset$
with $\I{\psi''_i} \neq \emptyset$.
Then $\I{\psi''_i \realto \psi'_i} = \emptyset$
and $\I\psi = \emptyset$.
Hence we must be in the case of item \eqref{item:proof:compl:fin:ded:c-false:ded}.
Moreover, by induction hypothesis we have $\C(\psi''_i)$
and $\psi' \thesis \False$,
so that we can derive $\psi \thesis \False$ using the rule $\ax{C}$.

Otherwise, we have $\I{\psi'_i} \neq \emptyset$
for all $i \in I$ such that $\I{\psi''_i} \neq \emptyset$.

Given $i \in I$ such that $\I{\psi''_i} = \emptyset$,
by induction hypothesis we have $\psi''_i \thesis \False$,
and since $\True \thesis \left( \False \realto \varphi \right)$
for any formula $\varphi$ (Remark~\ref{rem:log:realto}),
we get $\True \thesis \left(\psi''_i \realto \psi'_i \right)$.
Hence
$\bigwedge_{j \neq i}(\psi''_j \realto \psi'_j) \thesisiff \psi$.


We can therefore reduce the case of 
$\bigwedge_{i \in I} (\psi''_i \realto \psi'_i)$
where 
$\I{\psi'_i} \neq \emptyset$ and $\I{\psi''_i} \neq \emptyset$
for all $i \in I$.
In, particular, the induction hypothesis yields $\C(\psi'_i)$
and $\C(\psi''_i)$ for all $i \in I$.

Regarding item \eqref{item:proof:compl:fin:ded:c-false:c},
if $\I\psi \neq \emptyset$,
then for all $J \sle I$ we have either
$\SI{\bigwedge_{j \in J} \psi''_j} = \emptyset$
or
$\SI{\bigwedge_{j \in J} \psi'_j} \neq \emptyset$,
so that either
$\bigwedge_{j \in J} \psi''_j \thesis \False$
or
$\C(\bigwedge_{j \in J} \psi'_j)$
by induction hypothesis.
We can thus obtain $\C(\psi)$ by using the last rule in \eqref{eq:proof:compl:cc}.

Concerning item \eqref{item:proof:compl:fin:ded:c-false:ded},
if $\I\psi = \emptyset$,
then there is some $J \sle I$
such that
$\SI{\bigwedge_{j \in J} \psi''_j} \neq \emptyset$
while
$\SI{\bigwedge_{j \in J} \psi'_j} = \emptyset$.
Hence
$\C(\bigwedge_{j \in J} \psi''_j)$
and
$\bigwedge_{j \in J} \psi'_j \thesis \False$
by induction hypothesis.
%
We have
\[
\begin{array}{l l l}
  \psi
& \thesis
& \bigwedge_{i \in I}
  \left(
  \left( \bigwedge_{j \in J} \psi''_j \right)
  \realto
  \psi'_i
  \right)
\end{array}
\]

\noindent
and thus
\[
\begin{array}{l l l}
  \psi
& \thesis
& \left( \bigwedge_{j \in J} \psi''_j \right)
  \realto
  \bigwedge_{j \in J} \psi'_j
\end{array}
\]

\noindent
and we obtain $\psi \thesis \False$ using the rule $\ax{C}$.
\qed
\end{description}
%%%%%%%%%%%%%%%%%%%%%%%%%%%%%%%%%%%%%%%%%%%%%%%%%%%%%%%%%%%%%%%%%%%%%%%%%%%
\end{proof}
%%%%%%%%%%%%%%%%%%%%%%%%%%%%%%%%%%%%%%%%%%%%%%%%%%%%%%%%%%%%%%%%%%%%%%%%%%%

%%%%%%%%%%%%%%%%%%%%%%%%%%%%%%%%%%%%%%%%%%%%%%%%%%%%%%%%%%%%%%%%%%%%%%%%%%%
\begin{lemma}
\label{lem:proof:compl:fin:ded:compl}
%%%%%%%%%%%%%%%%%%%%%%%%%%%%%%%%%%%%%%%%%%%%%%%%%%%%%%%%%%%%%%%%%%%%%%%%%%%
For all $\varphi,\psi \in \Lang_\land(\PT)$,
if $\I\psi \sle \I\varphi$,
then $\psi \thesis \varphi$ is derivable;
%%%%%%%%%%%%%%%%%%%%%%%%%%%%%%%%%%%%%%%%%%%%%%%%%%%%%%%%%%%%%%%%%%%%%%%%%%%
\end{lemma}
%%%%%%%%%%%%%%%%%%%%%%%%%%%%%%%%%%%%%%%%%%%%%%%%%%%%%%%%%%%%%%%%%%%%%%%%%%%

%%%%%%%%%%%%%%%%%%%%%%%%%%%%%%%%%%%%%%%%%%%%%%%%%%%%%%%%%%%%%%%%%%%%%%%%%%%
\begin{proof}
%%%%%%%%%%%%%%%%%%%%%%%%%%%%%%%%%%%%%%%%%%%%%%%%%%%%%%%%%%%%%%%%%%%%%%%%%%%
The proof is by induction on sum of the (finite!) 
sizes of $\psi$ and $\varphi$.

First, note that if $\varphi = \bigwedge_{i \in I}\varphi_i$
for some finite set $I$,
then $\I\psi \sle \I\varphi$ implies
$\I\psi \sle \I{\varphi_i}$ for all $i \in I$,
and we can obtain $\psi \thesis \varphi$ from the induction hypotheses.

Also, if $\I\psi = \emptyset$, then Lemma~\ref{lem:proof:compl:fin:ded:c-false}
yields that $\psi \thesis \False$,
so that $\psi \thesis \varphi$.
This in particular applies when $\I\varphi = \emptyset$,
since we must then have $\I\psi = \emptyset$ as well.

We can thus assume that $\varphi$ is not a conjunction
and that both $\I\varphi$ and $\I\psi$ are not empty.
We now reason by cases on $\PT$.
\begin{description}
\item[Case of $\BT$ with $\BT \in \Base$.]
In this case, we must have $\I\varphi = \{a\}$ for some $a \in \BT$,
and thus $\I\psi = \{a\}$ as well.
We can then obtain $\psi \thesis \varphi$ using the rule $\form a \thesis \form a$.

\item[Case of $\rec\TV.\PT$.]
Reasoning as in Lemma~\ref{lem:proof:compl:fin:ded:c-false},
we can assume that $\psi$ is of the form
$\bigwedge_{j \in J}\form\fold \psi_j$ for some finite $J$.
Let $\psi'$ be the formula
$\bigwedge_{j \in J} \psi_j$.
Note that $\psi \thesisiff \form\fold\psi'$ by Example~\ref{ex:log:modalnf},
so that
$\I\psi = \I{\form\fold}(\I{\psi'})$.

Moreover, we must have $\varphi = \form\fold \varphi'$.
Hence $\I\psi \sle \I\varphi$
implies $\I{\psi'} \sle \I{\varphi'}$.
Note that $\varphi'$ is smaller than $\varphi$
while $\psi'$ is not greater than $\psi$.
Hence the induction hypothesis yields
$\psi' \thesis \varphi'$, and we are done.

\item[Case of $\PT_1 \times \PT_2$.]
Reasoning as in Lemma~\ref{lem:proof:compl:fin:ded:c-false},
we can assume that $\psi$ is of the form
\( 
  (\bigwedge_{j \in J} \form{\pi_1}\psi_j)
  \land
  (\bigwedge_{k \in K} \form{\pi_2}\psi_k)
\)
with $J \cap K = \emptyset$.
Let $\psi' = \bigwedge_{j \in J} \psi_j$
and $\psi'' = \bigwedge_{k \in K} \psi_k$,
so that $\psi \thesisiff \form{\pi_1}\psi' \land \form{\pi_2}\psi''$.

Moreover, we have $\varphi = \form{\pi_i}\varphi'$
say $i = 1$ (the case $i =2$ is symmetric).
But then we must have $\I{\psi'} \sle \I{\varphi'}$
and similarly as above (again), the induction hypothesis
yields $\psi' \thesis_{\PT_1} \varphi'$.
It is then easy to conclude.

\item[Case of $\PTbis \arrow \PT$.]
This is the most important case.
The proof is an adaptation to our setting
of the proof of~\cite[Proposition 10.5.2]{ac98book}.

First, note that $\varphi$ must be of the form $\varphi'' \realto \varphi'$.
If $\I{\varphi''} = \emptyset$,
then by Lemma~\ref{lem:proof:compl:fin:ded:c-false}
we obtain $\varphi'' \thesis \False$,
so that $\True \thesis (\varphi'' \realto \varphi')$
by Remark~\ref{rem:log:realto}.
Hence $\psi \thesis \varphi$ in this case.
We can thus assume $\I{\varphi''} \neq \emptyset$.
Since $\I\varphi \neq \emptyset$,
this implies that $\I{\varphi'} \neq \emptyset$ as well.
Hence, by Lemma~\ref{lem:top:char:fin}
there are finite $d''_i,d'_i$
such that $\I{\varphi''_i} = \up d''_i$
and $\I{\varphi'_i} = \up d'_i$.

On the other hand,
reasoning similarly as in Lemma~\ref{lem:proof:compl:fin:ded:c-false},
we can assume that $\psi$ is of the form
$\bigwedge_{i \in I} (\psi''_i \realto \psi'_i)$
for some finite set $I$, with $\I{\psi''_i} \neq \emptyset$
and $\I{\psi'_i} \neq \emptyset$ for all $i \in I$.
Hence, by Lemma~\ref{lem:top:char:fin},
for each $i \in I$ there are finite $e''_i,e'_i$
such that $\I{\psi''_i} = \up e''_i$
and $\I{\psi'_i} = \up e'_i$.
%
Moreover, $\bigvee_{i \in I}(e''_i \step e'_i)$
exists since $\I\psi \neq \emptyset$.

Hence $\I\psi \sle \I\varphi$ means
\[
\begin{array}{l l l}
  \up \bigvee_{i \in I} \left(e''_i \step e'_i \right)
& \sle
& \up \left( d'' \step d' \right)
\end{array}
\]

\noindent
which implies
\[
\begin{array}{l l l}
  d'' \step d'
& \leq
& \bigvee_{i \in I} \left(e''_i \step e'_i \right)
\end{array}
\]

\noindent
We thus have
\[
\begin{array}{l l l}
  d'
& \leq
& \bigvee_{d'' \geq e''_i} e_i
\end{array}
\]

\noindent
that is
\[
\begin{array}{l l l}
  \up \bigvee_{\up d'' \sle \up e''_i} e'_i
& \sle
& \up d'
\end{array}
\]

\noindent
In other words,
\[
\begin{array}{l l l}
  \I{\bigwedge_{\I{\varphi''} \sle \I{\psi ''_i}} \psi'_i}
& \sle
& \I{\varphi'}
\end{array}
\]

\noindent
and by induction hypothesis
\[
\begin{array}{l l l}
  \bigwedge_{\varphi'' \thesis \psi ''_i} \psi'_i
& \thesis
& \varphi'
\end{array}
\]

Hence we are done since
\[
\begin{array}{l l l}
  \psi
& \thesis
& \bigwedge_{i \in I} \left(
  \left( \bigwedge_{\varphi'' \thesis \psi ''_i} \psi''_i \right)
  \realto
  \psi'_i
  \right)
\end{array}
\]

\noindent
and thus
\[
\begin{array}{l l l}
  \psi
& \thesis
& \left( \bigwedge_{\varphi'' \thesis \psi ''_i} \psi''_i \right)
  \realto
  \bigwedge_{\varphi'' \thesis \psi ''_i} \psi'_i
\end{array}
\]

\noindent
while
\(
\varphi'' \thesis \bigwedge_{\varphi'' \thesis \psi ''_i} \psi''_i
\).
\qed
\end{description}
%%%%%%%%%%%%%%%%%%%%%%%%%%%%%%%%%%%%%%%%%%%%%%%%%%%%%%%%%%%%%%%%%%%%%%%%%%%
\end{proof}
%%%%%%%%%%%%%%%%%%%%%%%%%%%%%%%%%%%%%%%%%%%%%%%%%%%%%%%%%%%%%%%%%%%%%%%%%%%


We now turn to Theorem~\ref{thm:compl:fin},
namely the completeness for finite judgments.
While this result is essentially due to Abramsky \cite{abramsky91apal},
we nevertheless offer a proof since our system formally
differs from that of \cite{abramsky91apal}.
Let us recall the statement of Theorem~\ref{thm:compl:fin}.

%%%%%%%%%%%%%%%%%%%%%%%%%%%%%%%%%%%%%%%%%%%%%%%%%%%%%%%%%%%%%%%%%%%%%%%%%%%
\begin{theorem}[Theorem \ref{thm:compl:fin}]
\label{thm:proof:compl:fin}
%%%%%%%%%%%%%%%%%%%%%%%%%%%%%%%%%%%%%%%%%%%%%%%%%%%%%%%%%%%%%%%%%%%%%%%%%%%
Assume $\Env$ and  $\RT$ are finite.
%and $\UPT\Env \thesis M : \UPT\RT$ is derivable.
If $\Env \thesis M : \RT$ is sound,
then $\Env \thesis M : \RT$ is derivable in the system of \S\ref{sec:reft}
extended with \eqref{eq:proof:compl:cc}.
%%%%%%%%%%%%%%%%%%%%%%%%%%%%%%%%%%%%%%%%%%%%%%%%%%%%%%%%%%%%%%%%%%%%%%%%%%%
\end{theorem}
%%%%%%%%%%%%%%%%%%%%%%%%%%%%%%%%%%%%%%%%%%%%%%%%%%%%%%%%%%%%%%%%%%%%%%%%%%%

%%%%%%%%%%%%%%%%%%%%%%%%%%%%%%%%%%%%%%%%%%%%%%%%%%%%%%%%%%%%%%%%%%%%%%%%%%%
\begin{proof}
%%%%%%%%%%%%%%%%%%%%%%%%%%%%%%%%%%%%%%%%%%%%%%%%%%%%%%%%%%%%%%%%%%%%%%%%%%%
First, note that Lemma~\ref{lem:reft} (\S\ref{sec:reft})
restricts to finite types, in the sense that if a type
$\RTter$ is finite, then there is some $\varphi \in \Lang_\land(\UPT\RTter)$
such that $\RTter \eqtype \reft{\UPT\RTter \mid \varphi}$.

Consider first the case of a sound judgment
$\Env \thesis M : \RT$
where 
$\Env = x_1:\RTbis_1,\dots,x_n:\RTbis_n$
is such that $\I{\RTbis_i} = \emptyset$ for some $i \in \{1,\dots,n\}$.
Since $\I{\UPT{\RTbis_i}}$ is not empty (as it is a Scott domain),
taking $\psi \in \Lang_\land(\UPT{\RTbis_i})$
such that $\RTbis_i \eqtype \reft{\UPT{\RTbis_i} \mid \psi}$,
we must have $\I\psi = \emptyset$
and thus $\psi \thesis \False$ by Lemma~\ref{lem:proof:compl:fin:ded:c-false}.
We can thus conclude by taking $I = \emptyset$ in the rule
\[
\dfrac{
  \begin{array}{l}
  \UPT{\Env'}, x:\PTbis, \UPT{\Env''} \thesis M : \UPT\RT
  \\
  \text{for each $i \in I$,}\quad
  \Env', x:\reft{\PTbis \mid \psi_i},\Env' \thesis M : \RT
  \end{array}}
  {\Env', x : \reft{\PTbis \mid \bigvee_{i \in I} \psi_i} , \Env'' \thesis M : \RT}
\]




Hence we can reduce the case of a sound judgment
$\Env \thesis M : \RT$ with
$\Env = x_1:\RTbis_1,\dots,x_n:\RTbis_n$
such that $\I{\RTbis_i} \neq \emptyset$ for all $i = 1,\dots,n$.
We now reason by induction on the typing derivation of $\UPT\Env \thesis M : \UPT\RT$.
\begin{description}
\item[Case of]
\[
\dfrac{(x:\UPT\RT) \in \UPT\Env}
  {\UPT\Env \thesis x:\UPT\RT}
\]

We have $(x : \RTbis) \in \Env$ for some type $\RTbis$ with $\UPT\RTbis = \UPT\RT$.
Let $\varphi,\psi \in \Lang_\land(\UPT\RT)$
such that $\RT \eqtype \reft{\UPT\RT \mid \varphi}$
and $\RTbis \eqtype \reft{\UPT\RT \mid \psi}$.
By assumption on $\Env \thesis M :\RT$,
we have $\I\psi \sle \I\varphi$.
Hence $\psi \thesis \varphi$
by Proposition~\ref{prop:proof:compl:fin:ded}.
We then conclude by subtyping.

\item[Case of]
\[
\dfrac{\UPT\Env \thesis N_1 : \PT_1
  \qquad
  \UPT\Env \thesis N_2 : \PT_2}
  {\UPT\Env \thesis \pair{N_1,N_2} : \PT_1 \times \PT_2}
\]

\noindent
where $\UPT\RT = \PT_1 \times \PT_2$
and $M = \pair{N_1,N_2}$.

Let $\varphi \in \Lang_\land(\PT_1 \times \PT_2)$
such that $\RT \eqtype \reft{\PT_1 \times \PT_2 \mid \varphi}$.
Our assumption on $\Env \thesis M : \RT$
implies that $\I\varphi \neq \emptyset$.
Hence,
reasoning as in the proof of Lemma~\ref{lem:proof:compl:fin:ded:c-false}
yields that $\varphi \thesisiff \pair{\psi_1,\psi_2}$
for some $\psi_1 \in \Lang_\land(\PT_1)$
and some $\psi_2 \in \Lang_\land(\PT_2)$.

Since $\Env \thesis M :\RT$ is sound,
so are
$\Env \thesis N_1 : \reft{\PT_1 \mid \psi_1}$
and
$\Env \thesis N_1 : \reft{\PT_2 \mid \psi_2}$.
The induction hypotheses yield that
$\Env \thesis N_1 : \reft{\PT_1 \mid \psi_1}$
and
$\Env \thesis N_1 : \reft{\PT_2 \mid \psi_2}$
are derivable.
We can then conclude using the rules
\[
\dfrac{\Env \thesis N_i : \reft{\PT_i \mid \psi_i}
  \qquad
  \Env \thesis N_{3-i} : \PT_{3-i}}
  {\Env \thesis \pair{N_1,N_2} : \reft{\PT_1 \times \PT_2 \mid \form{\pi_i} \psi_i}}
\]

\noindent
for $i = 1$ and $i = 2$.

\item[Case of]
\[
\begin{array}{c}
\dfrac{\UPT\Env \thesis N : \PT \times \PTbis}
  {\UPT\Env \thesis \pi_i(N) : \UPT\RT}
\end{array}
\]

\noindent
where $i = 1,2$ and $M = \pi_1(N)$.
Assume w.l.o.g.\ $i = 1$ (so that $\UPT\RT = \PT$).
Let $\varphi \in \Lang_\land(\UPT\RT)$ such that
$\RT \eqtype \reft{\UPT\RT \mid \varphi}$.
Our assumption on $\Env \thesis M : \RT$
implies that $\I\varphi \neq \emptyset$,
and
since $\Env \thesis M : \RT$ is sound,
we get that
$\Env \thesis N : \reft{\PT \times \PTbis \mid \form{\pi_1}\varphi}$
is sound, and thus derivable by induction hypothesis.
We then conclude with the rule
\[
\dfrac{\Env \thesis N : \reft{\PT \times \PTbis \mid \form{\pi_1} \varphi}}
  {\Env \thesis \pi_1(N) : \reft{\PT \mid \varphi}}
\]

\item[Case of]
\[
\dfrac{\UPT\Env \thesis N : \PT[\rec\TV.\PT/\TV]}
  {\UPT\Env \thesis \fold(N) : \rec\TV.\PT}
\]

\noindent
where $\UPT\RT = \rec\TV.\PT$ and $M = \fold(M)$.

Let $\varphi \in \Lang_\land(\rec\TV.\PT)$
such that $\RT \eqtype \reft{ \rec\TV.\PT \mid \varphi}$.
Our assumption on $\Env \thesis M : \RT$
implies that $\I\varphi \neq \emptyset$.
Hence,
reasoning as in the proof of Lemma~\ref{lem:proof:compl:fin:ded:c-false}
yields that $\varphi \thesisiff \form{\fold}\psi$
for some $\psi \in \Lang_\land(\PT[\rec\TV.\PT/\TV])$.
Moreover, since $\Env \thesis M : \RT$ is sound,
so is
$\Env \thesis N : \reft{\PT[\rec\TV.\PT/\TV] \mid \psi}$.
We can thus conclude using the induction hypothesis and the rule
\[
\dfrac{\Env \thesis N : \reft{\PT[\rec\TV.\PT/\TV] \mid \psi}}
  {\Env \thesis \fold(N) : \reft{\rec\TV.\PT \mid \form\fold \psi}}
\]


\item[Case of]
\[
\dfrac{\UPT\Env \thesis N : \rec\TV.\PT}
  {\UPT\Env \thesis \unfold(N) : \PT[\rec\TV.\PT/\TV]}
\]

\noindent
where $\UPT\RT = \PT[\rec\TV.\PT/\TV]$
and $N = \unfold(M)$.

Let $\varphi \in \Lang_\land(\PT[\rec\TV.\PT/\TV])$
such that
$\RT \eqtype \reft{\PT[\rec\TV.\PT/\TV] \mid \varphi}$.
Our assumption on $\Env \thesis M : \RT$
implies that 
$\Env \thesis N : \reft{\rec\TV.\PT \mid \form\fold \varphi}$
is sound,
and we conclude using the induction hypothesis and the rule
\[
\dfrac{\Env \thesis N : \reft{\rec\TV.\PT \mid \form\fold \varphi}}
  {\Env \thesis \unfold(N) : \reft{\PT[\rec\TV.\PT/\TV] \mid \varphi}}
\]

\item[Case of]
\[
\dfrac{\UPT\Env,x:\PTbis \thesis N : \PT}
  {\UPT\Env \thesis \lambda x.N : \PTbis \arrow \PT}
\]

\noindent
where $\UPT\RT = \PTbis \arrow \PT$,
and where $M = \lambda x.N$.

Let $\varphi \in \Lang_\land(\PTbis \arrow \PT)$
such that
$\RT \eqtype \reft{\PTbis \arrow \PT \mid \varphi}$.
Our assumption on $\Env \thesis M : \RT$
implies $\I\varphi \neq \emptyset$.
Reasoning as in the proof of Lemma~\ref{lem:proof:compl:fin:ded:c-false}
yields that $\varphi \thesisiff \bigwedge_{i \in I}(\varphi''_i \realto \varphi'_i)$
for some finite set $I$.
Let $i \in I$.
The judgment
\(
  \Env
  \thesis
  \lambda x.N
  :
  \reft{\PTbis \arrow \PT \mid \varphi''_i \realto \varphi'_i}
\)
is sound,
and so is
\(
  \Env, x : \reft{\PTbis \mid \varphi''_i}
  \thesis
  N
  :
  \reft{\PT \mid \varphi'_i}
\).
Using the induction hypothesis, we derive
\(
  \Env
  \thesis
  \lambda x.N
  :
  \reft{\PTbis \arrow \PT \mid \varphi''_i \realto \varphi'_i}
\).
We can then derive
\(
  \Env
  \thesis
  \lambda x.N
  :
  \reft{\PTbis \arrow \PT \mid \varphi}
\).


\item[Case of]
\[
\dfrac{\UPT\Env \thesis N : \PTbis \arrow \PT
  \qquad
  \UPT\Env \thesis V : \PTbis}
  {\UPT\Env \thesis N V : \PT}
\]

\noindent
where $\UPT\RT = \PT$ and where $M = N V$.

Write $\Env = x_1:\RTbis_1,\dots,x_n:\RTbis_n$.
Given $i \in \{1,\dots,n\}$,
let $\psi_i \in \Lang_\land(\UPT{\RTbis_i})$
such that $\RTbis_i \eqtype \reft{\UPT{\RTbis_i} \mid \psi_i}$.
Moreover, by assumption we have $\I{\psi_i} \neq \emptyset$,
hence by Lemma~\ref{lem:top:char:fin}
there is a finite $e_i \in \I{\UPT{\RTbis_i}}$
such that $\I{\psi_i} = \up e_i$.

Similarly, let $\varphi \in \Lang_\land(\PT)$
such that $\RT \eqtype \reft{\PT \mid \varphi}$.
Our assumption on $\Env \thesis M : \RT$
implies that $\I\varphi \neq \emptyset$.
Hence, again by Lemma~\ref{lem:top:char:fin}
there is a finite $d \in \I{\PT}$
such that $\I{\varphi} = \up d$.


Since $\Env \thesis M : \RT$ is sound,
we have
$\I M(\vec e) \in \varphi$.
But note that
$\I M(\vec e) = \I N(\vec e)\left(\I V(\vec e) \right)$.

Now, since $\I\PTbis$ is a Scott domain, it is algebraic,
and $\I V(\vec e)$ is the directed l.u.b.\ of the finite $e \leq \I V(\vec e)$.
Since $\I N(\vec e)$ is Scott-continuous, we thus get
that $\I M(\vec e)$ is the l.u.b.\ of the directed set
\[
\left\{
  \I N(\vec e)(e)
  \mid
  \text{$e$ finite and $\leq \I V(\vec e)$}
\right\}
\]

\noindent
Since $d \leq \I M(\vec e)$ and since $d$ is finite,
it follows that we have
$d \leq \I N(\vec e)(e)$ for some finite $e \leq \I V(\vec e)$.
By Lemma~\ref{lem:top:char:fin},
there is a formula $\psi \in \Lang_\land(\PTbis)$
such that $\I\psi = \up e$.

Since $d \leq \I N(\vec e)(e)$, we have
$(e \step d) \leq \I N(\vec e)$,
so that
$\I N(\vec e) \in \I{\psi \realto \varphi}$.
Since $\I N$ is monotone, it follows that 
$\Env \thesis N : \reft{\PTbis \arrow \PT \mid \psi \realto \varphi}$
is sound.
Hence, this judgment is derivable by induction hypothesis.

Similarly, since $e \leq \I V(\vec e)$,
we obtain that the judgment
$\Env \thesis V : \reft{\PTbis \mid \psi}$
is sound and thus derivable.

We can then easily derive $\Env \thesis M : \reft{\PT \mid \varphi}$
and $\Env \thesis M : \RT$.

\item[Case of]
\[
\dfrac{\UPT\Env,x:\UPT\RT \thesis N : \UPT\RT}
  {\UPT\Env \thesis \fix x.N : \UPT\RT}
\]

\noindent
where $M = \fix x.N$.

Write $\Env = x_1:\RTbis_1,\dots,x_n:\RTbis_n$.
Similarly as above, for each $i \in \{1,\dots,n\}$
there is a finite $e_i \in \I{\UPT{\RTbis_i}}$
such that $\I{\RTbis_i} = \up e_i$.
Similarly, there are $\varphi \in \Lang_\land(\UPT\RT)$
such that $\RT \eqtype \reft{\UPT\RT \mid \varphi}$,
and a finite $d \in \I{\UPT\RT}$ such that $\I\varphi = \up d$.

Let $f \colon \I{\UPT\RT} \to \I{\UPT\RT}$
be the Scott-continuous function which takes
$a \in \I{\UPT\RT}$ to $\I N(\vec e,a)$.
We have
\[
\begin{array}{l l l}
  \I{\fix x.N}(\vec e)
& =
& \bigvee_{k \in \NN}
  f^k(\bot)
\end{array}
\]

Since $d \leq \I{\fix x.N}(\vec e)$ with $d$ finite,
there is some $k \in \NN$
such that
$d \leq f^k(\bot)$.
Write $d_k$ for $d$.
By induction, for each $j = k-1,\dots,0$,
there is some finite $d_j$ such that 
$d_{j+1} \leq f(d_j)$
and
$d_j \leq f^{j}(\bot)$.
In particular, $d_0 = \bot$.
For each $j = 0,\dots,k$,
let $\varphi_j$ such that $\I{\varphi_j} = \up d_j$.
Note that $\varphi_k = \varphi$.
Moreover, since $d_0 = \bot$,
we can take $\varphi_0 = \True$.

Again reasoning similarly as above,
we obtain that 
$\Env, x : \reft{\UPT\RT \mid \varphi_j} \thesis N : \reft{\UPT\RT \mid \varphi_{j+1}}$
is sound and thus derivable for each $j = 0,\dots,k-1$.
Moreover,
$\Env \thesis \fix x.N : \reft{\UPT\RT \mid \varphi_0}$
is derivable.
We can then derive $\Env \thesis \fix x.N : \reft{\UPT\RT \mid \varphi}$
by iterated applications of the rule
\[
\dfrac{\Env \thesis \fix x.N : \reft{\PT \mid \psi}
  \quad
  \Env, x: \reft{\PT \mid \psi} \thesis N : \reft{\PT \mid \psi'}}
  {\Env \thesis \fix x.N : \reft{\PT \mid \psi'}}
~(\psi,\psi' \in \Lang_\land(\PT))
\]

\item[Case of]
\[
\dfrac{}
  {\UPT\Env \thesis a : \BT}
\]

Let $\varphi \in \Lang_\land(\BT)$ such that $\RT \eqtype \reft{\BT \mid \varphi}$.
By assumption on $\Env \thesis M :\RT$,
we have $a \in \I\varphi$,
so that $\I{\form a} \sle \I\varphi$.
Hence $\form a \thesis \varphi$
by Proposition~\ref{prop:proof:compl:fin:ded}.
We can then conclude by subtyping and
\[
\dfrac{}
  {\Env \thesis a : \reft{\BT \mid \form a}}
\]

\item[Case of]
\[
\dfrac{ \UPT\Env \thesis N : \BT
  \qquad\text{for each $a \in \BT$,\quad} \UPT\Env \thesis N_a : \UPT\RT}
  {\UPT\Env \thesis \cse\ N\ \copair{a \mapsto N_a \mid a \in \BT} : \UPT\RT}
\]

We reason similarly as in the cases of $\fix x.N$ and $N V$ above.

Write $\Env = x_1:\RTbis_1,\dots,x_n:\RTbis_n$.
For each $i \in I$
there is a finite $e_i \in \I{\UPT{\RTbis_i}}$
such that $\I{\RTbis_i} = \up e_i$.
Also, there is $\varphi \in \Lang_\land(\UPT\RT)$
such that $\RT \eqtype \reft{\UPT\RT \mid \varphi}$.

Assume first that $\I\varphi = \I\True$,
so that $\varphi \thesisiff \True$
by Proposition~\ref{prop:proof:compl:fin:ded}.
Then we have $\RT \eqtype \UPT\RT$ and we easily
derive $\Env \thesis M : \RT$.

Otherwise, we must have $\bot \notin \I\varphi$,
so that $\I M(\vec e) \neq \bot$
and thus $\I N(\vec e) \neq \bot$.
Hence $\I N(\vec e) = b$ for some $b \in \BT$.
Since $\I M(\vec e) = \I{N_b}(\vec e)$,
we obtain that the judgment
$\Env \thesis N_b : \RT$ is sound and thus derivable.
Moreover,
$\Env \thesis N : \reft{\BT \mid \form b}$ is sound and thus
derivable.
We can then conclude using the rule
\[
\dfrac{
  \Env \thesis N : \reft{\BT \mid \form b}
  \qquad
  \Env \thesis N_b : \RT
  \qquad
  \text{for each $a \in A$,\quad} \UPT\Env \thesis N_a : \UPT\RT}
  {\Env \thesis \cse\ N\ \copair{a \mapsto N_a \mid a \in \BT} : \RT}
\]
\qed
\end{description}
%%%%%%%%%%%%%%%%%%%%%%%%%%%%%%%%%%%%%%%%%%%%%%%%%%%%%%%%%%%%%%%%%%%%%%%%%%%
\end{proof}
%%%%%%%%%%%%%%%%%%%%%%%%%%%%%%%%%%%%%%%%%%%%%%%%%%%%%%%%%%%%%%%%%%%%%%%%%%%







%%%%%%%%%%%%%%%%%%%%%%%%%%%%%%%%%%%%%%%%%%%%%%%%%%%%%%%%%%%%%%%%%%%%%%%%%%%
\subsection{\nameref{sec:main}}
\label{sec:proof:main}
%%%%%%%%%%%%%%%%%%%%%%%%%%%%%%%%%%%%%%%%%%%%%%%%%%%%%%%%%%%%%%%%%%%%%%%%%%%

Note that Lemma~\ref{lem:compl:nf} and Theorem~\ref{thm:main}
are proven in \S\ref{sec:app:main}.
We prove Proposition~\ref{prop:main:eta}.

%%%%%%%%%%%%%%%%%%%%%%%%%%%%%%%%%%%%%%%%%%%%%%%%%%%%%%%%%%%%%%%%%%%%%%%%%%%
\begin{proposition}[Proposition \ref{prop:main:eta}]
\label{prop:proof:main:eta}
%%%%%%%%%%%%%%%%%%%%%%%%%%%%%%%%%%%%%%%%%%%%%%%%%%%%%%%%%%%%%%%%%%%%%%%%%%%
A normal judgment $\Env \thesis M : \RT$ is sound (resp.\ derivable)
if, and only if, so are all $(\Env' \thesis M' : \RT') \in \eta(\Env \thesis M : \RT)$.
%%%%%%%%%%%%%%%%%%%%%%%%%%%%%%%%%%%%%%%%%%%%%%%%%%%%%%%%%%%%%%%%%%%%%%%%%%%
\end{proposition}
%%%%%%%%%%%%%%%%%%%%%%%%%%%%%%%%%%%%%%%%%%%%%%%%%%%%%%%%%%%%%%%%%%%%%%%%%%%

The proof of Proposition~\ref{prop:proof:main:eta}
relies on the following
Lemmas~\ref{lem:proof:main:eta:prod} and \ref{lem:proof:main:eta:fun}.

%%%%%%%%%%%%%%%%%%%%%%%%%%%%%%%%%%%%%%%%%%%%%%%%%%%%%%%%%%%%%%%%%%%%%%%%%%%
\begin{lemma}
\label{lem:proof:main:eta:prod}
%%%%%%%%%%%%%%%%%%%%%%%%%%%%%%%%%%%%%%%%%%%%%%%%%%%%%%%%%%%%%%%%%%%%%%%%%%%
A (not necessarily normal) judgment
$\Env \thesis M : \RT_1 \times \RT_2$
is sound (resp.\ derivable)
if, and only if,
so are $\Env \thesis \pi_1 M : \RT_1$
and $\Env \thesis \pi_2 M : \RT_2$.
%%%%%%%%%%%%%%%%%%%%%%%%%%%%%%%%%%%%%%%%%%%%%%%%%%%%%%%%%%%%%%%%%%%%%%%%%%%
\end{lemma}
%%%%%%%%%%%%%%%%%%%%%%%%%%%%%%%%%%%%%%%%%%%%%%%%%%%%%%%%%%%%%%%%%%%%%%%%%%%

%%%%%%%%%%%%%%%%%%%%%%%%%%%%%%%%%%%%%%%%%%%%%%%%%%%%%%%%%%%%%%%%%%%%%%%%%%%
\begin{proof}
%%%%%%%%%%%%%%%%%%%%%%%%%%%%%%%%%%%%%%%%%%%%%%%%%%%%%%%%%%%%%%%%%%%%%%%%%%%
By Lemma~\ref{lem:reft},
there are formulae
$\varphi_1 \in \Lang(\UPT{\RT_1})$
and
$\varphi_2 \in \Lang(\UPT{\RT_2})$
such that
$\RT_1 \eqtype \reft{\UPT{\RT_1} \mid \varphi_1}$
and
$\RT_2 \eqtype \reft{\UPT{\RT_2} \mid \varphi_2}$.
Hence
\(
  \RT_1 \times \RT_2
  \eqtype
  \reft{\UPT{\RT_1} \times \UPT{\RT_2} \mid \pair{\varphi_1,\varphi_2}}
\).


It follows that
$\Env \thesis M : \RT_1 \times \RT_2$
is sound if, and only if,
so are $\Env \thesis \pi_1 M : \RT_1$
and $\Env \thesis \pi_2 M : \RT_2$.

It is clear that $\Env \thesis \pi_1 M : \RT_1$
and $\Env \thesis \pi_2 M : \RT_2$
are derivable whenever so is
$\Env \thesis M : \RT_1 \times \RT_2$.

For the converse, assume that
$\Env \thesis \pi_1 M : \RT_1$
and $\Env \thesis \pi_2 M : \RT_2$
are derivable.
We first show that
$\Env \thesis M : \RT_1 \times \UPT{\RT_2}$.
We reason by induction on the derivation of $\Env \thesis \pi_1 M : \RT_1$
and by cases on the last possible rule.
\begin{description}
\item[Case of]
\[
\dfrac{
  \begin{array}{l}
  \UPT\Env \thesis \pi_1 M : \UPT{\RT_1}
  \\
  \text{for each $i \in I$,}\quad \Env \thesis \pi_1 M : \reft{\UPT{\RT_1} \mid \psi_i}
  \end{array}}
  {\Env \thesis \pi_1 M : \reft{\UPT{\RT_1} \mid \bigwedge_{i \in I} \psi_i}}
\]

\noindent
where $\varphi_1 = \bigwedge_{i \in I}\psi_i$.
Then by induction hypothesis and subtyping,
for all $i \in I$ we can derive
\[
\begin{array}{*{5}{l}}
  \Env
& \thesis
& M
& :
& \reft{\UPT{\RT_1} \times \UPT{\RT_2} \mid \form{\pi_1}\psi_i}
\end{array}
\]

\noindent
and thus
\[
\begin{array}{*{5}{l}}
  \Env
& \thesis
& M
& :
& \reft{\UPT{\RT_1} \times \UPT{\RT_2} \mid \bigwedge_{i \in I} \form{\pi_1}\psi_i}
\end{array}
\]

\noindent
We can then conclude using subtyping and Example~\ref{ex:log:modalnf}.

\item[Case of]
\[
\dfrac{
  \begin{array}{l}
  \UPT\Env, x:\PTbis, \UPT{\Env'} \thesis \pi_1 M : \UPT{\RT_1}
  \\
  \text{for each $i \in I$,}\quad
  \Env, x:\reft{\PTbis \mid \psi_i},\Env' \thesis \pi_1 M : \RT_1
  \end{array}}
  {\Env, x : \reft{\PTbis \mid \bigvee_{i \in I} \psi_i} , \Env' \thesis \pi_1 M : \RT_1}
\]

By induction hypothesis.

\item[Case of]
\[
\dfrac{
  \Env \subtype \Env'
  \quad 
  \RT'_1 \subtype \RT_1
  \quad
  \Env' \thesis \pi_1 M : \RT'_1}
  {\Env \thesis \pi_1 M : \RT_1}
\]


By subtyping we obtain
$\Env' \thesis \pi_2 M : \RT_2$
and the induction hypothesis yields
$\Env' \thesis M : \RT'_1 \times \UPT{\RT_2}$.
Then we are done since
$\RT'_1 \times \UPT{\RT_2} \subtype \RT_1 \times \UPT{\RT_2}$
and $\Env \subtype \Env'$.


\item[Case of]
\[
\dfrac{\Env \thesis M : \reft{\UPT{\RT_1} \times \UPT{\RT_2} \mid \form{\pi_1} \varphi_1}}
  {\Env \thesis \pi_1 M : \reft{\UPT{\RT_1} \mid \varphi_1}}
\]

Since
\(
  \reft{\UPT{\RT_1} \times \UPT{\RT_2} \mid \form{\pi_1} \varphi_1}
  \eqtype
  \RT_1 \times \UPT{\RT_2}
\).

\item[Case of]
\[
\dfrac{\Env \thesis M : \RT_1 \times \RT_2}
  {\Env \thesis \pi_1 M : \RT_1}
\]

Since
$\RT_1 \times \RT_2 \subtype \RT_1 \times \UPT{\RT_2}$.
\end{description}

\noindent
We similarly obtain
$\Env \thesis M : \UPT{\RT_1} \times \RT_2$.
Using subtyping, we then get
\[
\begin{array}{*{5}{l}}
  \Env
& \thesis
& M
& :
& \reft{\UPT{\RT_1} \times \UPT{\RT_2} \mid \form{\pi_1}\varphi_1}
\\

  \Env
& \thesis
& M
& :
& \reft{\UPT{\RT_1} \times \UPT{\RT_2} \mid  \form{\pi_2}\varphi_2}
\end{array}
\]

\noindent
from which we get
\[
\begin{array}{*{5}{l}}
  \Env
& \thesis
& M
& :
& \reft{\UPT{\RT_1} \times \UPT{\RT_2} \mid \pair{\varphi_1,\varphi_2}}
\end{array}
\]

\noindent
and thus
$\Env \thesis M : \RT_1 \times \RT_2$.
\qed
%%%%%%%%%%%%%%%%%%%%%%%%%%%%%%%%%%%%%%%%%%%%%%%%%%%%%%%%%%%%%%%%%%%%%%%%%%%
\end{proof}
%%%%%%%%%%%%%%%%%%%%%%%%%%%%%%%%%%%%%%%%%%%%%%%%%%%%%%%%%%%%%%%%%%%%%%%%%%%

%%%%%%%%%%%%%%%%%%%%%%%%%%%%%%%%%%%%%%%%%%%%%%%%%%%%%%%%%%%%%%%%%%%%%%%%%%%
\begin{lemma}
\label{lem:proof:main:eta:fun}
%%%%%%%%%%%%%%%%%%%%%%%%%%%%%%%%%%%%%%%%%%%%%%%%%%%%%%%%%%%%%%%%%%%%%%%%%%%
A (not necessarily normal) judgment
$\Env \thesis M : \RTbis \arrow \RT$
is sound (resp.\ derivable)
if, and only if,
so is
$\Env, x:\RTbis \thesis M x : \RT$.
%%%%%%%%%%%%%%%%%%%%%%%%%%%%%%%%%%%%%%%%%%%%%%%%%%%%%%%%%%%%%%%%%%%%%%%%%%%
\end{lemma}
%%%%%%%%%%%%%%%%%%%%%%%%%%%%%%%%%%%%%%%%%%%%%%%%%%%%%%%%%%%%%%%%%%%%%%%%%%%

%%%%%%%%%%%%%%%%%%%%%%%%%%%%%%%%%%%%%%%%%%%%%%%%%%%%%%%%%%%%%%%%%%%%%%%%%%%
\begin{proof}
%%%%%%%%%%%%%%%%%%%%%%%%%%%%%%%%%%%%%%%%%%%%%%%%%%%%%%%%%%%%%%%%%%%%%%%%%%%
By Lemma~\ref{lem:reft},
there are formulae
$\varphi \in \Lang(\UPT{\RT})$
and
$\psi \in \Lang(\UPT{\RTbis})$
such that
$\RT \eqtype \reft{\UPT{\RT} \mid \varphi}$
and
$\RTbis \eqtype \reft{\UPT{\RTbis} \mid \psi}$.
Hence
\(
  \RTbis \arrow \RT
  \eqtype
  \reft{\UPT{\RTbis} \arrow \UPT{\RT} \mid \psi \realto \varphi}
\).


It follows that
$\Env \thesis M : \RTbis \arrow \RT$
is sound if, and only if,
so is $\Env,x:\RTbis \thesis M x : \RT$.

It is clear that $\Env, x:\RTbis \thesis M x : \RT$
is derivable whenever so is
$\Env \thesis M : \RTbis \arrow \RT$.

For the converse, assume that
$\Env,x:\RTbis \thesis M x : \RT$
is derivable.
We show that
$\Env \thesis M : \RTbis \arrow \RT$
is derivable by induction on the derivation of
$\Env,x:\RTbis \thesis M x : \RT$.
We reason by cases on the last possible rule.
\begin{description}
\item[Case of]
\[
\dfrac{
  \begin{array}{l}
  \UPT\Env, x : \UPT\RTbis \thesis M x : \UPT{\RT}
  \\
  \text{for each $i \in I$,}\quad
  \Env,x:\RTbis \thesis M x : \reft{\UPT{\RT} \mid \varphi_i}
  \end{array}}
  {\Env,x:\RTbis \thesis M x : \reft{\UPT{\RT} \mid \bigwedge_{i \in I} \varphi_i}}
\]

\noindent
where $\varphi = \bigwedge_{i \in I}\varphi_i$.
Then by induction hypothesis and subtyping,
for all $i \in I$ we can derive
\[
\begin{array}{*{5}{l}}
  \Env
& \thesis
& M
& :
& \reft{\UPT{\RTbis} \arrow \UPT{\RT} \mid \psi \realto \varphi_i}
\end{array}
\]

\noindent
and thus
\[
\begin{array}{*{5}{l}}
  \Env
& \thesis
& M
& :
& \reft{\UPT{\RTbis} \arrow \UPT{\RT} \mid \bigwedge_{i \in I}(\psi \realto \varphi_i)}
\end{array}
\]

\noindent
We can then conclude by subtyping since
\[
\begin{array}{l !{\quad\thesis\quad} l}
  \bigwedge_{i \in I}(\psi \realto \varphi_i)
& \left( \psi \realto \bigwedge_{i \in I}\varphi_i \right)
\end{array}
\]

\item[Case of]
\[
\dfrac{
  \begin{array}{l}
  \UPT\Env, y:\PTbis, \UPT{\Env'},x : \UPT\RTbis \thesis M x : \UPT{\RT}
  \\
  \text{for each $i \in I$,}\quad
  \Env, y:\reft{\PTbis \mid \psi_i},\Env',x : \RTbis \thesis M x : \RT
  \end{array}}
  {\Env, y : \reft{\PTbis \mid \bigvee_{i \in I} \psi_i} , \Env', x : \RTbis
  \thesis M x : \RT}
\]

By induction hypothesis.

\item[Case of]
\[
\dfrac{
  \begin{array}{l}
  \UPT\Env, x:\UPT\RTbis \thesis M x  : \UPT{\RT}
  \\
  \text{for each $i \in I$,}\quad
  \Env, x:\reft{\UPT\RTbis \mid \psi_i} \thesis M x : \RT
  \end{array}}
  {\Env, x : \reft{\UPT\RTbis \mid \bigvee_{i \in I} \psi_i} \thesis M x : \RT}
\]

\noindent
where $\psi = \bigvee_{i \in I}\psi_i$.
By induction hypothesis and subtyping,
for all $i \in I$ we can derive
\[
\begin{array}{*{5}{l}}
  \Env
& \thesis
& M
& :
& \reft{\UPT{\RTbis} \arrow \UPT{\RT} \mid \psi_i \realto \varphi}
\end{array}
\]

\noindent
and thus
\[
\begin{array}{*{5}{l}}
  \Env
& \thesis
& M
& :
& \reft{\UPT{\RTbis} \arrow \UPT{\RT} \mid \bigwedge_{i \in I}(\psi_i \realto \varphi)}
\end{array}
\]

We can then conclude by subtyping since
\[
\begin{array}{l !{\quad\thesis\quad} l}
  \bigwedge_{i \in I}(\psi_i \realto \varphi)
& \left( \bigvee_{i \in I} \psi_i \right) \realto \varphi
\end{array}
\]

\item[Case of]
\[
\dfrac{
  \Env \subtype \Env'
  \quad
  \RTbis \subtype \RTbis'
  \quad 
  \RT' \subtype \RT
  \quad
  \Env', x:\RTbis' \thesis M x : \RT'}
  {\Env,x:\RTbis \thesis M x : \RT}
\]

By induction hypothesis we obtain
$\Env' \thesis M : \RTbis' \arrow \RT'$.
Then we are done since
$\RTbis' \arrow \RT' \subtype \RTbis \arrow \RT$
and $\Env \subtype \Env'$.

\item[Case of]
\[
\dfrac{\Env \thesis M : \RTbis \arrow \RT
  \qquad
  \Env, x: \RTbis \thesis x : \RTbis}
  {\Env, x: \RTbis \thesis M x : \RT}
\]

Trivial.
\qed
\end{description}
%%%%%%%%%%%%%%%%%%%%%%%%%%%%%%%%%%%%%%%%%%%%%%%%%%%%%%%%%%%%%%%%%%%%%%%%%%%
\end{proof}
%%%%%%%%%%%%%%%%%%%%%%%%%%%%%%%%%%%%%%%%%%%%%%%%%%%%%%%%%%%%%%%%%%%%%%%%%%%

We can now prove Proposition~\ref{prop:proof:main:eta}.

%%%%%%%%%%%%%%%%%%%%%%%%%%%%%%%%%%%%%%%%%%%%%%%%%%%%%%%%%%%%%%%%%%%%%%%%%%%
\begin{proof}[of Proposition~\ref{prop:proof:main:eta}]
%%%%%%%%%%%%%%%%%%%%%%%%%%%%%%%%%%%%%%%%%%%%%%%%%%%%%%%%%%%%%%%%%%%%%%%%%%%
We reason by induction on the fonf type $\RT$.
If $\RT$ is normal, then the result is trivial
since 
$\eta(\Env \thesis M : \RT) = \left\{ \Env \thesis M : \RT \right\}$.
In the cases of $\RT_1 \times \RT_2$
and $\RTbis \arrow \RT$ (with $\RTbis$ normal)
we conclude by induction hypothesis
and Lemmas~\ref{lem:proof:main:eta:prod} and \ref{lem:proof:main:eta:fun}, respectively.
\qed
%%%%%%%%%%%%%%%%%%%%%%%%%%%%%%%%%%%%%%%%%%%%%%%%%%%%%%%%%%%%%%%%%%%%%%%%%%%
\end{proof}
%%%%%%%%%%%%%%%%%%%%%%%%%%%%%%%%%%%%%%%%%%%%%%%%%%%%%%%%%%%%%%%%%%%%%%%%%%%





%%%%%%%%%%%%%%%%%%%%%%%%%%%%%%%%%%%%%%%%%%%%%%%%%%%%%%%%%%%%%%%%%%%%%%%%%%%
\subsection{\nameref{sec:compl:general}}
\label{sec:proof:compl:general}
%%%%%%%%%%%%%%%%%%%%%%%%%%%%%%%%%%%%%%%%%%%%%%%%%%%%%%%%%%%%%%%%%%%%%%%%%%%

We prove Lemma~\ref{lem:compl:nf:wf}.

%%%%%%%%%%%%%%%%%%%%%%%%%%%%%%%%%%%%%%%%%%%%%%%%%%%%%%%%%%%%%%%%%%%%%%%%%%%
\begin{lemma}[Lemma \ref{lem:compl:nf:wf}]
\label{lem:proof:compl:nf:wf}
%%%%%%%%%%%%%%%%%%%%%%%%%%%%%%%%%%%%%%%%%%%%%%%%%%%%%%%%%%%%%%%%%%%%%%%%%%%
For each $\varphi \in \Lang(\PT)$, there is a $\psi \in \Norm(\PT)$
such that $\varphi \thesisiff \psi$
in the extension of Figure~\ref{fig:log:ded} (\S\ref{sec:log})
with \eqref{eq:proof:compl:cc} and $\ax{WF}$.
%%%%%%%%%%%%%%%%%%%%%%%%%%%%%%%%%%%%%%%%%%%%%%%%%%%%%%%%%%%%%%%%%%%%%%%%%%%
\end{lemma}
%%%%%%%%%%%%%%%%%%%%%%%%%%%%%%%%%%%%%%%%%%%%%%%%%%%%%%%%%%%%%%%%%%%%%%%%%%%


%%%%%%%%%%%%%%%%%%%%%%%%%%%%%%%%%%%%%%%%%%%%%%%%%%%%%%%%%%%%%%%%%%%%%%%%%%%
\begin{proof}
%%%%%%%%%%%%%%%%%%%%%%%%%%%%%%%%%%%%%%%%%%%%%%%%%%%%%%%%%%%%%%%%%%%%%%%%%%%
The proof is by induction on $\varphi$.
In the case of $\bigwedge$ and $\bigvee$,
we conclude by induction hypothesis and Example~\ref{ex:log:distr}.
In the case of $\form\triangle\varphi$ ($\triangle$ either $\pi_1$, $\pi_2$ or $\fold$),
we conclude by induction hypothesis and Example~\ref{ex:log:modalnf}.

Consider now the case of $\psi \realto \varphi$.
By induction hypothesis we can assume $\varphi \in \Norm$.
By combining the induction hypothesis with 
Example~\ref{ex:log:distr},
we can assume that $\psi$
is a $\bigvee$ of $\bigwedge$'s of formulae in $\Lang_\land$.
Since
\[
\begin{array}{r !{\quad\thesisiff\quad} l}
  \bigwedge_{i \in I}\left(\psi \realto \varphi_i \right)
& \psi \realto \left(\bigwedge_{i \in I} \varphi_i\right)
\\

  \bigwedge_{i \in I}\left( \psi_i \realto \varphi \right)
& \left(\bigvee_{i \in I} \psi_i \right) \realto \varphi
\end{array}
\]

\noindent
we can reduce to the case of
$\psi = \bigwedge_{i \in I}\psi_i$
and
$\varphi = \bigvee_{k \in K}\varphi_k$
with $\varphi_k,\psi_i \in \Lang_\land$.

Now, note that we can derive
\[
\begin{array}{r !{\quad\thesisiff\quad} l}
  \left( \bigwedge_{i \in I} \psi_i \right)
  \realto
  \varphi
& \bigvee_{\text{$J \sle I$, $J$ finite}}
  \left(
  \left( \bigwedge_{j \in J} \psi_j \right)
  \realto
  \varphi
  \right)
\end{array}
\]

\noindent
Indeed, the $\thesis$ direction is given by the rule $\ax{WF}$.
The converse is derivable using the left-rule for $\bigvee$,
since $\bigwedge_{i \in I}\psi_i \thesis \bigwedge_{j \in J}\psi_j$
whenever $J \sle I$.

It follows that we can actually assume $\psi \in \Lang_\land$
(still with $\varphi = \bigvee_{k \in K}\varphi_k$
where $\varphi_k \in \Lang_\land$).
If $K \neq \emptyset$, then we can conclude using the rule
$\ax{F}$ in Figure~\ref{fig:log:ded}.

Otherwise, $K = \emptyset$ and $\varphi = \False$.
If $\C(\psi)$ then we conclude using the rule $\ax{C}$
in \eqref{eq:proof:compl:cc}.
Otherwise, by Proposition~\ref{prop:proof:compl:fin:ded}
we have $\psi \thesis \False$,
and we are done since
$\True \thesis (\False \realto \False)$
by Remark~\ref{rem:log:realto}.
\qed
%%%%%%%%%%%%%%%%%%%%%%%%%%%%%%%%%%%%%%%%%%%%%%%%%%%%%%%%%%%%%%%%%%%%%%%%%%%
\end{proof}
%%%%%%%%%%%%%%%%%%%%%%%%%%%%%%%%%%%%%%%%%%%%%%%%%%%%%%%%%%%%%%%%%%%%%%%%%%%


}

\opt{full}{\newpage}
\opt{full}{\tableofcontents}

%%%%%%%%%%%%%%%%%%%%%%%%%%%%%%%%%%%%%%%%%%%%%%%%%%%%%%%%%%%%%%%%%%%%%%%%%%%
\end{document}
%%%%%%%%%%%%%%%%%%%%%%%%%%%%%%%%%%%%%%%%%%%%%%%%%%%%%%%%%%%%%%%%%%%%%%%%%%%


