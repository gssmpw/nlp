%%%%%%%%%%%%%%%%%%%%%%%%%%%%%%%%%%%%%%%%%%%%%%%%%%%%%%%%%%%%%%%%%%%%%%%%%%%
\section{Proofs of \S\ref{sec:sem} (\nameref{sec:sem})}
\label{sec:proof:sem}
%%%%%%%%%%%%%%%%%%%%%%%%%%%%%%%%%%%%%%%%%%%%%%%%%%%%%%%%%%%%%%%%%%%%%%%%%%%

%%%%%%%%%%%%%%%%%%%%%%%%%%%%%%%%%%%%%%%%%%%%%%%%%%%%%%%%%%%%%%%%%%%%%%%%%%%
\subsection{\nameref{sec:sem:pure}}
\label{sec:proof:sem:pure}
%%%%%%%%%%%%%%%%%%%%%%%%%%%%%%%%%%%%%%%%%%%%%%%%%%%%%%%%%%%%%%%%%%%%%%%%%%%


%%%%%%%%%%%%%%%%%%%%%%%%%%%%%%%%%%%%%%%%%%%%%%%%%%%%%%%%%%%%%%%%%%%%%%%%%%%
\subsubsection{Solutions of Recursive Domain Equations.}
%%%%%%%%%%%%%%%%%%%%%%%%%%%%%%%%%%%%%%%%%%%%%%%%%%%%%%%%%%%%%%%%%%%%%%%%%%%
We review the usual solution of recursive domain equations.
We refer to~\cite{ac98book,aj95chapter,streicher06book}.


%%%%%%%%%%%%%%%%%%%%%%%%%%%%%%%%%%%%%%%%%%%%%%%%%%%%%%%%%%%%%%%%%%%%%%%%%%%
\paragraph{Categories of Domains.}
%%%%%%%%%%%%%%%%%%%%%%%%%%%%%%%%%%%%%%%%%%%%%%%%%%%%%%%%%%%%%%%%%%%%%%%%%%%
In the following, $\DCPO$ is the category of those
posets with all directed suprema, and with Scott-continuous
functions as morphisms.
$\CPO$ is the full subcategory of $\DCPO$
on posets with a least element,
and $\Scott$ is a full subcategory of $\CPO$.


Recall that $\DCPO$, $\CPO$ and $\Scott$
have finite products
(equipped with the component-wise order).
See \cite[Theorem 3.3.3, Theorem 3.3.5 and Corollary 4.1.6]{aj95chapter}.
Hence for each $n \in \NN$, the categories
$\Scott^n$, $\CPO^n$ and $\DCPO^n$
are (not full) subcategories of $\Scott$, $\CPO$ and $\DCPO$
respectively.

%%%%%%%%%%%%%%%%%%%%%%%%%%%%%%%%%%%%%%%%%%%%%%%%%%%%%%%%%%%%%%%%%%%%%%%%%%%
\begin{lemma}
\label{lem:proof:scott:enrich}
%%%%%%%%%%%%%%%%%%%%%%%%%%%%%%%%%%%%%%%%%%%%%%%%%%%%%%%%%%%%%%%%%%%%%%%%%%%
If $n \in \NN$ then
$\DCPO^n$,
$\CPO^n$, $\Scott^n$ are enriched in $\DCPO$.
%%%%%%%%%%%%%%%%%%%%%%%%%%%%%%%%%%%%%%%%%%%%%%%%%%%%%%%%%%%%%%%%%%%%%%%%%%%
\end{lemma}
%%%%%%%%%%%%%%%%%%%%%%%%%%%%%%%%%%%%%%%%%%%%%%%%%%%%%%%%%%%%%%%%%%%%%%%%%%%

%%%%%%%%%%%%%%%%%%%%%%%%%%%%%%%%%%%%%%%%%%%%%%%%%%%%%%%%%%%%%%%%%%%%%%%%%%%
\begin{proof}
%%%%%%%%%%%%%%%%%%%%%%%%%%%%%%%%%%%%%%%%%%%%%%%%%%%%%%%%%%%%%%%%%%%%%%%%%%%
The result for $n=1$ follows from the
Cartesian-closure of $\DCPO$, $\CPO$ and $\Scott$
(\cite[Theorem 3.3.3, Theorem 3.3.5 and Corollary 4.1.6]{aj95chapter}).
In the cases of $n \neq 1$, the result follows from the fact that
in $\DCPO, \CPO, \Scott$, finite products are Cartesian products
of sets equipped with the component-wise order.
\qed
%%%%%%%%%%%%%%%%%%%%%%%%%%%%%%%%%%%%%%%%%%%%%%%%%%%%%%%%%%%%%%%%%%%%%%%%%%%
\end{proof}
%%%%%%%%%%%%%%%%%%%%%%%%%%%%%%%%%%%%%%%%%%%%%%%%%%%%%%%%%%%%%%%%%%%%%%%%%%%

Let $\cat C$ be a category enriched over $\DCPO$.
Given objects $X,Y \in \cat C$,
an \emph{embedding-projection} pair $X \to Y$
is a pair of morphisms $\ladj f: X\rightleftarrows Y:\radj f$
where $\radj f \comp \ladj f = \id_X$ and $\ladj f \comp \radj f \leq \id_Y$.
The morphism $\ladj f$ is an \emph{embedding}
(it reflects (as well as preserves) the order),
while $\radj f$ is a \emph{projection}.
%Note that $\radj e$ and $\ladj e$ are strict.
Note that if $X$ (resp.\ $Y$) has a least element,
then so does $Y$ (resp.\ $X$) and $\ladj f$ (resp $\radj f$)
is strict.
It is well-known that $\ladj f$ completely determines $\radj f$
and reciprocally, see~\cite[\S 7.1]{ac98book}
(cf.\ also~\cite[\S 3.1.4]{aj95chapter} and~\cite[\S 9]{streicher06book}).
Given an embedding $e$ (resp.\ a projection $p$),
we write $\radj e$ (resp.\ $\ladj p$)
for the corresponding projection (resp.\ embedding).

We write $\cat C^\ep$ for the category with the same objects as $\cat C$,
and with embedding-projection pairs as morphisms.
Note that we have faithful functors
$\ladj{(\pl)} \colon \cat C^\ep \to \cat C$
and
$\radj{(\pl)} \colon \cat C^\ep \to \cat C^\op$
(taking $(\ladj f,\radj f)$ to $\ladj f$ and to $\radj f$,
respectively).
Given a functor $H$ of codomain $\cat C^\ep$,
we write $\radj H$ for $\radj{(\pl)} \comp H$,
and similarly for $\ladj H$.


%%%%%%%%%%%%%%%%%%%%%%%%%%%%%%%%%%%%%%%%%%%%%%%%%%%%%%%%%%%%%%%%%%%%%%%%%%%
\paragraph{The Limit-Colimit Coincidence.}
%%%%%%%%%%%%%%%%%%%%%%%%%%%%%%%%%%%%%%%%%%%%%%%%%%%%%%%%%%%%%%%%%%%%%%%%%%%
The following (crucial and) well-known fact
is \cite[Theorem 7.1.10]{ac98book}
(see also \cite[Theorem 3.3.7]{aj95chapter}).

%%%%%%%%%%%%%%%%%%%%%%%%%%%%%%%%%%%%%%%%%%%%%%%%%%%%%%%%%%%%%%%%%%%%%%%%%%%
\begin{theorem}
\label{thm:proof:scott:limcolim}
%%%%%%%%%%%%%%%%%%%%%%%%%%%%%%%%%%%%%%%%%%%%%%%%%%%%%%%%%%%%%%%%%%%%%%%%%%%
Let $K \colon \omega \to \cat C^\ep$ be a functor
where $\cat C$ is enriched over $\DCPO$.
Each limiting cone $\varpi \colon \Lim \radj K \to \radj K$
for $\radj K \colon \omega^\op \to \cat C$ consists of projections,
and the $(\ladj{(\varpi_n)},\varpi_n)_n$
form a colimiting cocone
$K \to \Colim K$ in $\cat C^\ep$.
%%%%%%%%%%%%%%%%%%%%%%%%%%%%%%%%%%%%%%%%%%%%%%%%%%%%%%%%%%%%%%%%%%%%%%%%%%%
\end{theorem}
%%%%%%%%%%%%%%%%%%%%%%%%%%%%%%%%%%%%%%%%%%%%%%%%%%%%%%%%%%%%%%%%%%%%%%%%%%%

%%%%%%%%%%%%%%%%%%%%%%%%%%%%%%%%%%%%%%%%%%%%%%%%%%%%%%%%%%%%%%%%%%%%%%%%%%%
\begin{proof}
%%%%%%%%%%%%%%%%%%%%%%%%%%%%%%%%%%%%%%%%%%%%%%%%%%%%%%%%%%%%%%%%%%%%%%%%%%%
Let
$K \colon \omega \to \cat C^\ep$
and consider a limiting cone
\begin{equation}
\label{diag:proof:scott:lim}
\begin{array}{c}
\begin{tikzcd}[column sep=2em] % normal=2.4em
&
& \Lim \radj K
  \arrow{dll}[above]{\varpi_0}
  \arrow{dl}{\varpi_1}
  \arrow{d}{\varpi_2}
  \arrow{dr}[below]{\varpi_n}
  \arrow{drr}{\varpi_{n+1}}

\\

  \radj K(0)
& \radj K(1)
  \arrow{l}[below]{\radj{(k_0)}}
& \radj K(2)
  \arrow{l}[below]{\radj{(k_1)}}
& \radj K(n)
  \arrow[dashed]{l}
& \radj K(n+1)
  \arrow{l}[below]{\radj{(k_n)}}
& \phantom{F}
  \arrow[dashed]{l}
\end{tikzcd}
\end{array}
\end{equation}


\noindent
in $\cat C$.
The components of the colimiting cocone
\begin{equation}
\label{diag:proof:scott:colim}
\begin{array}{c}
\begin{tikzcd}[column sep=2em]
&
& \Colim K

\\

  K(0)
  \arrow{r}[below]{k_0}
  \arrow{urr}[above]{\gamma_0}
& K(1)
  \arrow{r}[below]{k_1}
  \arrow{ur}[below]{\gamma_1}
& K(2)
  \arrow[dashed]{r}
  \arrow{u}{\gamma_2}
& K(n)
  \arrow{r}[below]{k_n}
  \arrow{ul}[below]{\gamma_n}
& K(n+1)
  \arrow[dashed]{r}
  \arrow{ull}[above]{\gamma_{n+1}}
& \phantom{F}
\end{tikzcd}
\end{array}
\end{equation}

\noindent
in $\cat C^\ep$ are given by $\radj{(\gamma_n)} = \varpi_n$
for projections.

Concerning embeddings,
for each $n \in \NN$ we build a cone with vertex $K(n) = \radj K(n)$
as follows.
Given $m \in \NN$, we have a morphism $h_{n,m} \colon K(n) \to K(m)$
obtained by composing $\radj{(k_i)}$'s or $\ladj{(k_i)}$'s
according to whether $m \leq n$ or $n \leq m$.
The $h_{n,m}$'s can be made so that $h_{n,m} = \radj{(k_m)} \comp h_{n,m+1}$.
The universal property of limits in $\cat C$ then yields
a unique morphism $c_n$ from $K(n) = \radj K(n)$ to $\Lim \radj K(n)$
such that $\varpi_m \comp c_n = h_{n,m}$ for all $m \in \NN$.

We are going to show that $c_n = \ladj{(\varpi_n)}$.
Note that $\varpi_n \comp c_n$ is the identity by definition of $c_n$.
It remains to show that $c_n \comp \varpi_n \leq \id_{\Lim \radj K}$.
We first show that
$(c_n \comp \varpi_n)_n$ forms an increasing sequence
in $\cat C(\Lim \radj K, \Lim \radj K)$.
To this end, note that $\varpi_n = \radj{(k_n)} \comp \varpi_{n+1}$ 
(since $\varpi$ is a cone).
We moreover have
$c_n = c_{n+1} \comp \ladj{(k_n)}$
since
\(
  \varpi_m \comp c_{n+1} \comp \ladj{(k_n)}
  =
  h_{n+1,m} \comp \ladj{(k_n)}
  =
  h_{n,m}
\)
for all $m \in \NN$.
We compute
\[
\begin{array}{*{5}{l}}
  c_n \comp \varpi_n
& =
& c_{n+1} \comp \ladj{(k_n)} \comp \radj{(k_n)} \comp \varpi_{n+1}
& \leq
& c_{n+1} \comp \varpi_{n+1}
\end{array}
\]

Let $\ell = \bigvee_{n}(c_n \comp \varpi_n)$.
We now claim that $\ell$ is the identity.
This will yield that $c_n \comp \varpi_n \leq \id_{\Lim \radj K}$.
In order to show that $\ell = \id_{\Lim \radj K}$,
we show that $\varpi_m \comp \ell = \varpi_m$ for all $m \in \NN$,
and use the universal property of limits in $\cat C$.
We have
\[
\begin{array}{l l l}
  \varpi_m \comp \ell
& =
& \varpi_m \comp \bigvee_{n}(c_n \comp \varpi_n)
\\

& =
& \bigvee_n \left(
  \varpi_m \comp c_n \comp \varpi_n
  \right)
\\

& =
& \bigvee_n \left(
  h_{n,m} \comp \varpi_n
  \right)
\\

& =
& \bigvee_{n\geq m} \left(
  h_{n,m} \comp \varpi_n
  \right)
\end{array}
\]

\noindent
But by definition of $h_{n,m}$, we have
$h_{n,m} \comp \varpi_n = \varpi_m$ when $m \leq n$.

We can thus set $\gamma_n = (c_n, \varpi_n)$.
Moreover, $\gamma = (\gamma_n)_n$ is indeed a cocone since
$c_n = c_{n+1} \comp \ladj{(k_n)}$ (see above).

We now claim that $\gamma \colon K \to \Lim \radj K$ is colimiting.
To this end, consider a cocone $\tau \colon K \to C$.
We thus get a cone $\radj\tau \colon \radj K \to \radj C$ in $\cat C$,
and the universal property of limits yields a unique
$p \colon \radj C \to \Lim \radj K$ such that
$\varpi_n \comp p = \radj{(\tau_n)}$ for all $n \in \NN$.
We show that $p$ is a projection.
We define a morphism $e \colon \Lim \radj K \to C$
as $e = \bigvee_{n}(\ladj{(\tau_n)} \comp \varpi_n)$.
We have
\[
\begin{array}{l l l}
  e \comp p
& =
& \left( \bigvee_{n} \ladj{(\tau_n)} \comp \varpi_n \right) \comp p
\\

& =
& \bigvee_n \ladj{(\tau_n)} \comp \radj{(\tau_n)}
\\

& \leq
& \id_{C}
\end{array}
\]

\noindent
On the other hand, given $m \in \NN$ we have
\[
\begin{array}{l l l}
  \varpi_m \comp p \comp e
& =
& \bigvee_{n} \radj{(\tau_m)} \comp \ladj{(\tau_n)} \comp \varpi_n
\\

& =
& \bigvee_{n \geq m} \radj{(\tau_m)} \comp \ladj{(\tau_n)} \comp \varpi_n
\\

& =
& \bigvee_{n \geq m} h_{n,m} \comp \radj{(\tau_n)} \comp \ladj{(\tau_n)} \comp \varpi_n
\\

& =
& \bigvee_{n \geq m} h_{n,m} \comp \varpi_n
\\

& =
& \bigvee_{n \geq m} \varpi_m
\\

& =
& \varpi_m
\end{array}
\]

\noindent
so that $p \comp e = \id_{\Lim \radj K}$
by the universal property of limits in $\cat C$.

Moreover, for all $n \in \NN$ we have
\[
\begin{array}{l l l}
  e \comp c_n
& =
& \bigvee_m \ladj{(\tau_m)} \comp \varpi_m \comp c_n
\\

& =
& \bigvee_m \ladj{(\tau_m)} \comp h_{n,m}
\\

& =
& \bigvee_{m\geq n} \ladj{(\tau_m)} \comp h_{n,m}
\\

& =
& \bigvee_{m\geq n} \ladj{(\tau_n)}
\\

& =
& \ladj{(\tau_n)}
\end{array}
\]

Consider now a morphism $\ell \colon \Lim \radj K \to C$
in $\cat C^\ep$
such that 
$\varpi_n \comp \radj\ell = \radj{(\tau_n)}$
and
$\ladj\ell \comp c_n = \ladj{(\tau_n)}$
for all $n \in \NN$.
The universal property of limits in $\cat C$
yields $\radj\ell = p$, so that $\ladj\ell = e$
since $e$ is uniquely determined from $p$.
\qed
%%%%%%%%%%%%%%%%%%%%%%%%%%%%%%%%%%%%%%%%%%%%%%%%%%%%%%%%%%%%%%%%%%%%%%%%%%%
\end{proof}
%%%%%%%%%%%%%%%%%%%%%%%%%%%%%%%%%%%%%%%%%%%%%%%%%%%%%%%%%%%%%%%%%%%%%%%%%%%


%%%%%%%%%%%%%%%%%%%%%%%%%%%%%%%%%%%%%%%%%%%%%%%%%%%%%%%%%%%%%%%%%%%%%%%%%%%
\paragraph{Solutions of Domain Equations.}
%%%%%%%%%%%%%%%%%%%%%%%%%%%%%%%%%%%%%%%%%%%%%%%%%%%%%%%%%%%%%%%%%%%%%%%%%%%
We shall use Theorem~\ref{thm:proof:scott:limcolim}
in the following situation.
Consider a functor
\[
\begin{array}{*{5}{l}}
  G
& :
& \cat D^\ep \times \cat C^\ep
  % \cat C^\ep
& \longto
& \cat C^\ep
\end{array}
\]

\noindent
where $\cat C$ and $\cat D$ are enriched over $\DCPO$.
We moreover assume that $\cat C$ has a terminal object $\one$
which is initial in $\cat C^\ep$.
We are going to define a functor
\[
\begin{array}{*{5}{l}}
  K
& \colon
& \cat D^\ep \times \omega
& \longto
& \cat C^\ep
\end{array}
\]

\noindent
Given an object $B$ of $\cat D^\ep$,
$K(B,\pl)$ is the $\omega$-chain in $\cat C^\ep$
obtained by iterating $G_B = G(B,\pl)$ from the initial object $\one$ of $\cat C^\ep$:
\begin{equation}
\label{diag:proof:scott:chain}
\begin{tikzcd} %[column sep=large]
  \one
  \arrow{r}[above]{\one}
& G_B(\one)
   \arrow{r}[above]{G_B(\one)}
 & G^2_B(\one)
  \arrow[dashed]{r}
& G^n_B(\one)
  \arrow{r}[above]{G_B^n(\one)}
& G^{n+1}_B(\one)
  \arrow[dashed]{r}
& \phantom{F}
\end{tikzcd}
\end{equation}


Given a morphism $f \colon B \to B'$ in $\cat D^\ep$,
$K(f,\pl)$ is obtained by commutativity of the following.
\begin{equation}
\label{diag:proof:scott:natdiag}
\begin{array}{c}
\begin{tikzcd} %[column sep=large]

  \one
  \arrow{r}[above]{\one}
  \arrow{d}{\one}
& G_B(\one)
   \arrow{r}[above]{G_B(\one)}
   \arrow{d}{G_f(\one)}
 & G^2_B(\one)
   \arrow{d}{G_f^2(\one)}
  \arrow[dashed]{r}
& G^n_B(\one)
  \arrow{r}[above]{G_B^n(\one)}
  \arrow{d}{G_f^n(\one)}
& G^{n+1}_B(\one)
  \arrow{d}{G_f^{n+1}(\one)}
  \arrow[dashed]{r}
& \phantom{F}

\\

  \one
  \arrow{r}[below]{\one}
& G_{B'}(\one)
   \arrow{r}[below]{G_{B'}(\one)}
 & G^2_{B'}(\one)
  \arrow[dashed]{r}
& G^n_{B'}(\one)
  \arrow{r}[below]{G_{B'}^n(\one)}
& G^{n+1}_{B'}(\one)
  \arrow[dashed]{r}
& \phantom{F}
\end{tikzcd}
\end{array}
\end{equation}



Assume now that $\cat C$ has limits of $\omega^\op$-chains
of projections.
Then Theorem~\ref{thm:proof:scott:limcolim}
yields that each $K(B,\pl)$ has a colimit in $\cat C^\ep$.
Since $K$ is a functor $\cat D^\ep \times \omega \to \cat C^\ep$,
it follows from \cite[Theorem V.3.1]{maclane98book}
that these colimits assemble into a functor
\[
\begin{array}{l l r c l}
  \Fix G
& :
& \cat D^\ep
& \longto
& \cat C^\ep
\\

&
& B
& \longmapsto
& \Colim_{n \in \omega} K(B,n)
\end{array}
\]

If $G(B,\pl)$ preserves colimits of $\omega$-chains,
then the universal property of colimits gives an isomorphism
\(
  \fold^\ep
  :
  G(B,\Fix G(B))
  \rightleftarrows
  \Fix G(B)
  :
  \unfold^\ep
\)
in $\cat C^\ep$.

We are going to prove the following.

%%%%%%%%%%%%%%%%%%%%%%%%%%%%%%%%%%%%%%%%%%%%%%%%%%%%%%%%%%%%%%%%%%%%%%%%%%%
\begin{proposition}
\label{prop:proof:scott:contfunct}
%%%%%%%%%%%%%%%%%%%%%%%%%%%%%%%%%%%%%%%%%%%%%%%%%%%%%%%%%%%%%%%%%%%%%%%%%%%
If $G \colon \cat D^\ep \times \cat C^\ep \to \cat C^\ep$
preserves colimits of $\omega$-chains,
then so do $\Fix G \colon \cat D^\ep \to \cat C^\ep$.
%%%%%%%%%%%%%%%%%%%%%%%%%%%%%%%%%%%%%%%%%%%%%%%%%%%%%%%%%%%%%%%%%%%%%%%%%%%
\end{proposition}
%%%%%%%%%%%%%%%%%%%%%%%%%%%%%%%%%%%%%%%%%%%%%%%%%%%%%%%%%%%%%%%%%%%%%%%%%%%

The proof of Proposition~\ref{prop:proof:scott:contfunct}
is split into the following lemmas.
Fix a functor $G \colon \cat D^\ep \times \cat C^\ep \to \cat C^\ep$
which preserves colimits of $\omega$-chains.

%%%%%%%%%%%%%%%%%%%%%%%%%%%%%%%%%%%%%%%%%%%%%%%%%%%%%%%%%%%%%%%%%%%%%%%%%%%
\begin{lemma}
\label{lem:proof:scott:contdiag}
%%%%%%%%%%%%%%%%%%%%%%%%%%%%%%%%%%%%%%%%%%%%%%%%%%%%%%%%%%%%%%%%%%%%%%%%%%%
The diagonal functor $\Delta \colon \cat D^\ep \to \cat D^\ep \times \cat D^\ep$
preserves colimits of $\omega$-chains.
%%%%%%%%%%%%%%%%%%%%%%%%%%%%%%%%%%%%%%%%%%%%%%%%%%%%%%%%%%%%%%%%%%%%%%%%%%%
\end{lemma}
%%%%%%%%%%%%%%%%%%%%%%%%%%%%%%%%%%%%%%%%%%%%%%%%%%%%%%%%%%%%%%%%%%%%%%%%%%%

%%%%%%%%%%%%%%%%%%%%%%%%%%%%%%%%%%%%%%%%%%%%%%%%%%%%%%%%%%%%%%%%%%%%%%%%%%%
\begin{proof}
%%%%%%%%%%%%%%%%%%%%%%%%%%%%%%%%%%%%%%%%%%%%%%%%%%%%%%%%%%%%%%%%%%%%%%%%%%%
Since colimits are pointwise in functor categories
(\cite[Corollary V.3]{maclane98book}).%
\footnote{Note that \cite[Corollary V.3]{maclane98book} only gives the result for limits.
But recall that the opposite of a functor category $[\cat C,\cat D]$
is the functor category $[\cat C^\op, \cat D^\op]$.}
%%%%%%%%%%%%%%%%%%%%%%%%%%%%%%%%%%%%%%%%%%%%%%%%%%%%%%%%%%%%%%%%%%%%%%%%%%%
\end{proof}
%%%%%%%%%%%%%%%%%%%%%%%%%%%%%%%%%%%%%%%%%%%%%%%%%%%%%%%%%%%%%%%%%%%%%%%%%%%

Lemma~\ref{lem:proof:scott:contdiag}
entails in particular that each functor
$G^n_{(\pl)}(\one) \colon \cat D^\ep \to \cat C^\ep$
preserves colimits of $\omega$-chains
($G^{n+1}_{(\pl)}(\one)$ is
$G(\pl,G^{n}_{(\pl)}(\one)) \comp \Delta$).

Proposition~\ref{prop:proof:scott:contfunct}
relies on the fact that the functor
$K \colon \cat D^\ep \to \funct{\omega,\cat C^\ep}$
preserves colimits of $\omega$-chains.
This involves some notation.

Let $W \colon \omega \to \cat D^\ep$ be an $\omega$-chain,
with colimiting cocone $\gamma \colon W \to \Colim W$.
In the following, we write $w_m \colon W(m) \to W(m+1)$
for the connecting morphisms of $W$.
The cocone $K \gamma \colon K(W) \to K(\Colim W)$
has component at $m \in \NN$
the commutative diagram in \eqref{diag:proof:scott:natdiag}
where on takes $\gamma_m \colon W(m) \to \Colim W$
for $f \colon B \to B'$.


%%%%%%%%%%%%%%%%%%%%%%%%%%%%%%%%%%%%%%%%%%%%%%%%%%%%%%%%%%%%%%%%%%%%%%%%%%%
\begin{lemma}
\label{lem:proof:scott:colimiting}
%%%%%%%%%%%%%%%%%%%%%%%%%%%%%%%%%%%%%%%%%%%%%%%%%%%%%%%%%%%%%%%%%%%%%%%%%%%
The cocone $K\gamma \colon K(W) \to K(\Colim W)$
is colimiting.
%%%%%%%%%%%%%%%%%%%%%%%%%%%%%%%%%%%%%%%%%%%%%%%%%%%%%%%%%%%%%%%%%%%%%%%%%%%
\end{lemma}
%%%%%%%%%%%%%%%%%%%%%%%%%%%%%%%%%%%%%%%%%%%%%%%%%%%%%%%%%%%%%%%%%%%%%%%%%%%

%%%%%%%%%%%%%%%%%%%%%%%%%%%%%%%%%%%%%%%%%%%%%%%%%%%%%%%%%%%%%%%%%%%%%%%%%%%
\begin{proof}
%%%%%%%%%%%%%%%%%%%%%%%%%%%%%%%%%%%%%%%%%%%%%%%%%%%%%%%%%%%%%%%%%%%%%%%%%%%
First, it follows from the above that each
$G_\gamma^n(\one) \colon G_W^n(\one) \to G_{\Colim W}^n(\one)$
is colimiting.

Consider now a cocone
$\tau \colon K(W) \to H$ in $\funct{\omega,\cat C^\ep}$.
For each $m \in \NN$, we have
$\tau_m = \tau_{m+1} \comp K(w_m)$,
that is
\begin{equation}
\label{diag:proof:scott:taucocone}
\begin{array}{c}
\begin{tikzcd}[column sep=2.14em, row sep=large]
% column sep normal=2.4em
% row sep normal=1.8em

  \one
  \arrow{r}[above]{\one}
  \arrow{d}{\one}
& G_{B}(\one)
   \arrow{r}[above]{G_{B}(\one)}
   \arrow{d}{G_{w_m}(\one)}
& G^2_{B}(\one)
   \arrow{d}{G_{w_m}^2(\one)}
  \arrow[dashed]{r}
& G^n_{B}(\one)
  \arrow{r}[above]{G_{B}^n(\one)}
  \arrow{d}{G_{w_m}^n(\one)}
& G^{n+1}_{B}(\one)
  \arrow{d}{G_{w_m}^{n+1}(\one)}
  \arrow[dashed]{r}
& \phantom{F}

\\

  \one
  \arrow{r}[above]{\one}
  \arrow{d}{(\tau_{m+1})_0}
& G_{B'}(\one)
   \arrow{r}[above]{G_{B'}(\one)}
   \arrow{d}{(\tau_{m+1})_1}
& G^2_{B'}(\one)
   \arrow{d}{(\tau_{m+1})_2}
  \arrow[dashed]{r}
& G^n_{B'}(\one)
  \arrow{r}[above]{G_{B'}^n(\one)}
  \arrow{d}{(\tau_{m+1})_n}
& G^{n+1}_{B'}(\one)
  \arrow{d}{(\tau_{m+1})_{n+1}}
  \arrow[dashed]{r}
& \phantom{F}

\\

  H(0)
  \arrow{r}[below]{h(0)}
& H(1)
   \arrow{r}[below]{h(1)}
& H(2)
  \arrow[dashed]{r}
& H(n)
  \arrow{r}[below]{h(n)}
& H(n+1)
  \arrow[dashed]{r}
& \phantom{F}
\end{tikzcd}
\end{array}
\end{equation}

\noindent
where $B$ is $W(m)$, $B'$ is $W(m+1)$
and the $h(n) \colon H(n) \to H(n+1)$ are the connective morphisms of $H$.
In particular, for each $m \in \NN$ and each $n \in \NN$,
we have $(\tau_m)_n = (\tau_{m+1})_n \comp G_{w_m}^n(\one)$.
Hence, for each $n \in \NN$ we have a cocone
$((\tau_m)_n)_m \colon G_W^n(\one) \to H(n)$,
and the universal property of $G_\gamma^n(\one)$
gives a unique morphism $\ell_n \colon G_{\Colim W}^n(\one) \to H(n)$
such that $(\tau_m)_n = \ell_n \comp G_{\gamma_m}^n(\one)$ for all $m \in \NN$.

We show that the $\ell_n$'s assemble into a morphism
$\ell \colon K(\Colim W) \to H$ in $\funct{\omega,\cat C^\ep}$.
We thus have to show that the following commutes
\begin{equation*}
\begin{array}{c}
\begin{tikzcd} %[column sep=large]

  \one
  \arrow{r}[above]{\one}
  \arrow{d}{\ell_0}
& G_{B'}(\one)
   \arrow{r}[above]{G_{B'}(\one)}
   \arrow{d}{\ell_1}
 & G^2_{B'}(\one)
   \arrow{d}{\ell_2}
  \arrow[dashed]{r}
& G^n_{B'}(\one)
  \arrow{r}[above]{G_{B'}^n(\one)}
  \arrow{d}{\ell_n}
& G^{n+1}_{B'}(\one)
  \arrow{d}{\ell_{n+1}}
  \arrow[dashed]{r}
& \phantom{F}

\\

  H(0)
  \arrow{r}[below]{h(0)}
& H(1)
   \arrow{r}[below]{h(1)}
& H(2)
  \arrow[dashed]{r}
& H(n)
  \arrow{r}[below]{h(n)}
& H(n+1)
  \arrow[dashed]{r}
& \phantom{F}

\end{tikzcd}
\end{array}
\end{equation*}

\noindent
where $B'$ is $\Colim W$.
We show that $\ell_{n+1} \comp G_{B'}^n(\one) = h(n) \comp \ell_n$
for all $n \in \NN$.
For each $m \in \NN$, 
by commutativity of \eqref{diag:proof:scott:natdiag}
and \eqref{diag:proof:scott:taucocone}
%and functoriality of $G_{(\pl)}^{n+}$
we have
\[
\begin{array}{l l l}
  \ell_{n+1} \comp G_{B'}^{n}(\one) \comp G_{\gamma_m}^{n}(\one)
& =
& \ell_{n+1} \comp G_{\gamma_m}^{n+1}(\one) \comp G_B^{n}(\one)
\\

& =
& (\tau_m)_{n+1} \comp G_B^{n}(\one)
\\

& =
& h(n) \comp (\tau_m)_n
\\

& =
& h(n) \comp \ell_n \comp G_{\gamma_m}^n(\one)
\end{array}
\]

\noindent
where $B$ is $W(m)$.
Then we are done by the universal property of $G_{\gamma_m}^n(\one)$.

Consider finally a morphism $f \colon K(\Colim W) \to H$ in
$\funct{\omega,\cat C^\ep}$
such that $f \comp K(\gamma) = \tau$.
Then for all $m \in \NN$ we have
$f \comp K(\gamma_m) = \tau_m$,
and thus
$f_n \comp G_{\gamma_m}^n(\one) = (\tau_m)_n$
for all $n \in \NN$.
It follows that $f_n = \ell_n$, so that $f = \ell$.
\qed
%%%%%%%%%%%%%%%%%%%%%%%%%%%%%%%%%%%%%%%%%%%%%%%%%%%%%%%%%%%%%%%%%%%%%%%%%%%
\end{proof}
%%%%%%%%%%%%%%%%%%%%%%%%%%%%%%%%%%%%%%%%%%%%%%%%%%%%%%%%%%%%%%%%%%%%%%%%%%%

We can now prove Proposition~\ref{prop:proof:scott:contfunct}.

%%%%%%%%%%%%%%%%%%%%%%%%%%%%%%%%%%%%%%%%%%%%%%%%%%%%%%%%%%%%%%%%%%%%%%%%%%%
\begin{proof}[of Proposition~\ref{prop:proof:scott:contfunct}]
%%%%%%%%%%%%%%%%%%%%%%%%%%%%%%%%%%%%%%%%%%%%%%%%%%%%%%%%%%%%%%%%%%%%%%%%%%%
Let $W \colon \omega \to \cat D^\ep$ be an $\omega$-chain.
By Lemma~\ref{lem:proof:scott:colimiting},
and since colimits always commute over colimits,
we have
\[
\begin{array}{l l l}
  \Fix G(\Colim W)
& =
& \Colim_{n \in \omega} K(\Colim W,n)
\\

& \cong
& \Colim_{n \in \omega} \Colim_{m \in \omega} K(W(m),n)
\\

& \cong
& \Colim_{m \in \omega} \Colim_{n \in \omega} K(W(m),n)
\\

& \cong
& \Colim_{m \in \omega} \Fix G(W(m))
\end{array}
\]
\qed
%%%%%%%%%%%%%%%%%%%%%%%%%%%%%%%%%%%%%%%%%%%%%%%%%%%%%%%%%%%%%%%%%%%%%%%%%%%
\end{proof}
%%%%%%%%%%%%%%%%%%%%%%%%%%%%%%%%%%%%%%%%%%%%%%%%%%%%%%%%%%%%%%%%%%%%%%%%%%%


%%%%%%%%%%%%%%%%%%%%%%%%%%%%%%%%%%%%%%%%%%%%%%%%%%%%%%%%%%%%%%%%%%%%%%%%%%%
\paragraph{Local Continuity.}
%%%%%%%%%%%%%%%%%%%%%%%%%%%%%%%%%%%%%%%%%%%%%%%%%%%%%%%%%%%%%%%%%%%%%%%%%%%
Functors $G \colon \cat D^\ep \times \cat C^\ep \to \cat E^\ep$
will be obtained from ``mixed-variance'' functors
\[
\begin{array}{*{5}{l}}
  F
& :
& \cat D^\op \times \cat C
& \longto
& \cat E
\end{array}
\]

\noindent
where
$\cat D,\cat C, \cat E$ are enriched over $\DCPO$.

%%%%%%%%%%%%%%%%%%%%%%%%%%%%%%%%%%%%%%%%%%%%%%%%%%%%%%%%%%%%%%%%%%%%%%%%%%%
\begin{definition}
\label{def:proof:scott:loc}
%%%%%%%%%%%%%%%%%%%%%%%%%%%%%%%%%%%%%%%%%%%%%%%%%%%%%%%%%%%%%%%%%%%%%%%%%%%
We say that $F$
is \emph{locally} \emph{monotone} (resp.\ \emph{continuous})
if each hom-function
\[
\begin{array}{r c l}
  \cat D(B',B) \times \cat C(A,A')
& \longto
& \cat C(F(B,A), F(B',A'))
\\

  (g,f)
& \longmapsto
& F(g,f)
\end{array}
\]

\noindent
is monotone (resp.\ Scott-continuous).
%%%%%%%%%%%%%%%%%%%%%%%%%%%%%%%%%%%%%%%%%%%%%%%%%%%%%%%%%%%%%%%%%%%%%%%%%%%
\end{definition}
%%%%%%%%%%%%%%%%%%%%%%%%%%%%%%%%%%%%%%%%%%%%%%%%%%%%%%%%%%%%%%%%%%%%%%%%%%%

\noindent
We refer to \cite[Definition 5.2.5]{aj95chapter},
\cite[Definition 7.1.15]{ac98book}
and \cite[Definition 9.1]{streicher06book}.
%
The following is a straightforward adaptation of \cite[Proposition 7.1.19]{ac98book}
(see also \cite[Proposition 5.2.6]{aj95chapter}).

%%%%%%%%%%%%%%%%%%%%%%%%%%%%%%%%%%%%%%%%%%%%%%%%%%%%%%%%%%%%%%%%%%%%%%%%%%%
\begin{lemma}
\label{lem:proof:scott:lift}
%%%%%%%%%%%%%%%%%%%%%%%%%%%%%%%%%%%%%%%%%%%%%%%%%%%%%%%%%%%%%%%%%%%%%%%%%%%
Let
$F \colon \cat D^\op \times \cat C \to \cat E$
be locally monotone.
Then $F$ lifts to a covariant functor
\[
\begin{array}{*{5}{l}}
  F^\ep
& :
& \cat D^\ep \times \cat C^\ep
& \longto
& \cat E^\ep
\end{array}
\]

\noindent
with $F^\ep(B,A) = F(B,A)$ on objects and
$F^\ep(g,f) = (F(\radj g,\ladj f) \,,\, F(\ladj g,\radj f))$
on morphisms.

If moreover $F$ is locally continuous, then $F^\ep$
preserves colimits of $\omega$-chains.
%%%%%%%%%%%%%%%%%%%%%%%%%%%%%%%%%%%%%%%%%%%%%%%%%%%%%%%%%%%%%%%%%%%%%%%%%%%
\end{lemma}
%%%%%%%%%%%%%%%%%%%%%%%%%%%%%%%%%%%%%%%%%%%%%%%%%%%%%%%%%%%%%%%%%%%%%%%%%%%



%%%%%%%%%%%%%%%%%%%%%%%%%%%%%%%%%%%%%%%%%%%%%%%%%%%%%%%%%%%%%%%%%%%%%%%%%%%
\subsubsection{Interpretation of Pure Types.}
%%%%%%%%%%%%%%%%%%%%%%%%%%%%%%%%%%%%%%%%%%%%%%%%%%%%%%%%%%%%%%%%%%%%%%%%%%%
A \emph{pure type expression} is a possibly open production of the
grammar of pure types (\S\ref{sec:pure}), namely
\[
\begin{array}{r @{\ \ }c@{\ \ } l}
     \PT
&    \bnf
&    \BT
\gss \PT \times \PT
\gss \PT \arrow \PT
\gss \TV
\gss \rec \TV.\PT 
\end{array}
\]

\noindent
where $\BT \in \Base$,
where $\TV$ is a type variable,
and where $\rec\TV.\PT$ binds $\TV$ in $\PT$.

Consider a pure type expression $\PT$ with free
type variables $\vec \TV = \TV_1,\dots,\TV_n$.
We are going to interpret $\PT$ as a functor
\[
\begin{array}{*{5}{l}}
  \I\PT
& :
& \left( \Scott^\ep \right)^{n}
& \longto
& \Scott^\ep
\end{array}
\]

\noindent
which preserves colimits of $\omega$-chains.


%%%%%%%%%%%%%%%%%%%%%%%%%%%%%%%%%%%%%%%%%%%%%%%%%%%%%%%%%%%%%%%%%%%%%%%%%%%
\paragraph{Preliminaries.}
%%%%%%%%%%%%%%%%%%%%%%%%%%%%%%%%%%%%%%%%%%%%%%%%%%%%%%%%%%%%%%%%%%%%%%%%%%%
Recall that the category $\Scott$ is Cartesian-closed
(products and homsets are equipped with pointwise orders),
see \cite[Corollary 4.1.6]{aj95chapter} or \cite[\S 1.4]{ac98book}.
This yields functors
$\Scott(\pl,\pl) \colon \Scott^\op \times \Scott \to \Scott$
and
$(\pl) \times (\pl) \colon \Scott \times \Scott \to \Scott$.
These functors are locally continuous
(\cite[Example 7.1.16]{ac98book}).
By combining Lemma~\ref{lem:proof:scott:enrich},
Lemma~\ref{lem:proof:scott:lift} and Lemma~\ref{lem:proof:scott:contdiag},
we obtain functors
\[
\begin{array}{*{5}{l}}
  \left(\Scott(\pl,\pl) \right)^\ep
  ,\,
  \left( (\pl) \times (\pl) \right)^\ep
& :
& \Scott^\ep \times \Scott^\ep
& \longto
& \Scott^\ep
\end{array}
\]

\noindent
which preserve colimits of $\omega$-chains.

Moreover, $\Scott$ has limits of $\omega^\op$-chains
of projections (in the embedding-projection sense),
see \cite[Theorem 3.3.7, Theorem 3.3.11 and Proposition 4.1.3]{aj95chapter}.
%
More precisely, the full inclusion $\Scott \emb \DCPO$
creates limits for $\omega^\op$-chains of projections.%
\footnote{The notion of creation of limits has to be understood in the usual
sense of \cite[Definition V.1]{maclane98book}.}
%
In particular, $\Scott$
is closed in $\DCPO$ under limits of $\omega^\op$-chains of projections.
%
Note that the category $\DCPO$ has all limits,
and that they are created by the forgetful functor to the category
of posets (and monotone functions),
see \cite[Theorem 3.3.1]{aj95chapter}.
%
It follows that given $K \colon \omega \to \cat \Scott^\ep$,
the limit of $\radj K \colon \omega^\op \to \Scott$
is
\[
  \left\{
  (x_i)_i \in \prod_{i \in \NN} K(i)
  \mathrel{\Big|}
  \radj K(i \leq j)(x_j) = x_i
  \right\}
\]

\noindent
equipped with the pointwise order.
Moreover, the limiting
cone $\Lim \radj K \to \radj K$ 
consists in set-theoretic projections.%
\footnote{These are also projections in the embedding-projection sense
by Theorem~\ref{thm:proof:scott:limcolim}.}
In view of Theorem~\ref{thm:proof:scott:limcolim},
we also get that $\Scott$ is closed in $\DCPO$
under colimits of $\omega$-chains of embeddings.

The terminal object $\one$ of $\Scott$ is initial in $\Scott^\ep$
(\cite[Proposition 7.1.9]{ac98book}).


%%%%%%%%%%%%%%%%%%%%%%%%%%%%%%%%%%%%%%%%%%%%%%%%%%%%%%%%%%%%%%%%%%%%%%%%%%%
\paragraph{Definition of the Interpretation.}
%%%%%%%%%%%%%%%%%%%%%%%%%%%%%%%%%%%%%%%%%%%%%%%%%%%%%%%%%%%%%%%%%%%%%%%%%%%
Let $\PT$ be a (pure) type expression with free
type variables $\vec \TV = \TV_1,\dots,\TV_n$.
The interpretation $\I\PT \colon \left(\Scott^\ep\right)^n \to \Scott^\ep$
is defined by induction on $\PT$.
\begin{itemize}
\item
In the case of $\PT = \TV_i$,
we let $\I\PT$ take $\vec X = X_1,\dots,X_n$ to $X_i$.

\item
In the case of $\BT \in \Base$,
we let $\I\PT(\vec X)$ be the flat domain $\BT_\bot$,
where $\BT_\bot$ is $\BT + \{\bot\}$ with $\BT$ discrete.


\item
In the cases of $\PT \times \PTbis$
and $\PTbis \arrow \PT$,
the induction hypotheses give us
functors
\[
\begin{array}{*{5}{l}}
  \I\PT, \I\PTbis
& :
& \left( \Scott^\ep \right)^{n}
& \longto
& \Scott^\ep
\end{array}
\]

\noindent
which preserve colimits of $\omega$-chains.
We can thus set
\[
\begin{array}{r c l}
  \I{\PTbis \arrow \PT}(\vec X)
& =
& \left( \Scott \left( \I{\PTbis}(\vec X) ,\, \I{\PT}(\vec X) \right) \right)^\ep
\\

  \I{\PT \times \PTbis}(\vec X)
& =
& \left(
  \I\PT(\vec X) \times \I\PTbis(\vec X)
  \right)^\ep
\end{array}
\]


\item
In the case of $\rec \TV.\PT$,
the induction hypothesis gives a functor
\[
\begin{array}{*{5}{l}}
  \I\PT
& :
& \left( \Scott^\ep \right)^{n} \times \Scott^\ep
& \longto
& \Scott^\ep
\end{array}
\]

\noindent
which preserves colimits of $\omega$-chains.
Theorem~\ref{thm:proof:scott:limcolim}
gives a functor
\[
\begin{array}{l l r c l}
  \I{\rec \TV.\PT}
& :
& \left( \Scott^\ep \right)^{n} 
& \longto
& \Scott^\ep
\\

&
& \vec X
& \longmapsto
& \Fix (\I{\PT}(\vec X))
\end{array}
\]

\noindent
This functor preserves colimits of $\omega$-chains by
Proposition~\ref{prop:proof:scott:contfunct}.
%
Moreover, since $\I\PT$ preserves
colimits of $\omega$-chains,
we obtain canonical isomorphisms
\(
  \I\fold
  :
  \I{\PT[\rec \TV.\PT/\TV]}(\vec X)
  \rightleftarrows
  \I{\rec \TV.\PT}(\vec X)
  :
  \I\unfold
\)
by taking
$\I\fold = \ladj{(\fold^\ep)}$
and
$\I\unfold = \ladj{(\unfold^\ep)}$.
\end{itemize}



%%%%%%%%%%%%%%%%%%%%%%%%%%%%%%%%%%%%%%%%%%%%%%%%%%%%%%%%%%%%%%%%%%%%%%%%%%%
\begin{figure}[t!]
%%%%%%%%%%%%%%%%%%%%%%%%%%%%%%%%%%%%%%%%%%%%%%%%%%%%%%%%%%%%%%%%%%%%%%%%%%%
\[
\begin{array}{c}

\dfrac{}
  {\bot \in \Fin(\I\PT(\vec X))}  


\qquad\qquad

\dfrac{\text{$\BT \in \Base$ and $a \in \BT$}}
  {a \in \Fin(\I\BT(\vec X))}

\qquad\qquad

\dfrac{\text{$d$ finite in $X_i$}}
  {d \in \Fin(\I{\TV_i}(\vec X))}

\\\\

\dfrac{d \in \Fin(\I\PT(\vec X))
  \qquad
  e \in \Fin(\I\PTbis(\vec X))}
  {(d,e) \in \Fin(\I{\PT \times \PTbis}(\vec X))}

\qquad\qquad

\dfrac{d \in \Fin(\I{\PT[\rec\TV.\PT/\TV]}(\vec X))}
  {\I\fold(d) \in \Fin(\I{\rec\TV.\PT}(\vec X))}

\\\\

\dfrac{\begin{array}{l}
  \text{for all $i \in I$,~
  $d_i \in \Fin(\I{\PT})(\vec X)$
  ~and~
  $e_i \in \Fin(\I{\PTbis})(\vec X)$ \@;}
  \\
  \text{for all $J \sle I$,~
  $\bigvee_{j \in J} d_j$ defined in $\I{\PT}(\vec X)$
  ~$\imp$~
  $\bigvee_{j \in J} e_j$ defined in $\I{\PTbis}(\vec X)$}
  \end{array}}
  {\bigvee_{i \in I}(d_i \step e_i) \in \Fin(\I{\PT \arrow \PTbis}(\vec X))}
~(\text{$I$ finite})

\end{array}
\]
\caption{Inductive description of the finite elements of $\I\PT(\vec X)$.%
\label{fig:proof:sem:finelt}}
%%%%%%%%%%%%%%%%%%%%%%%%%%%%%%%%%%%%%%%%%%%%%%%%%%%%%%%%%%%%%%%%%%%%%%%%%%%
\end{figure}
%%%%%%%%%%%%%%%%%%%%%%%%%%%%%%%%%%%%%%%%%%%%%%%%%%%%%%%%%%%%%%%%%%%%%%%%%%%

%%%%%%%%%%%%%%%%%%%%%%%%%%%%%%%%%%%%%%%%%%%%%%%%%%%%%%%%%%%%%%%%%%%%%%%%%%%
\paragraph{Description of the Finite Elements.}
%%%%%%%%%%%%%%%%%%%%%%%%%%%%%%%%%%%%%%%%%%%%%%%%%%%%%%%%%%%%%%%%%%%%%%%%%%%
For each (pure) type expression $\PT$
with free variables $\vec\PT = \PT_1,\dots,\PT_n$,
we define a set $\Fin(\I{\PT}(\vec X))$.
The definition is by induction on derivations with
the rules in Figure~\ref{fig:proof:sem:finelt}.
The set $\Fin(\I{\PT}(\vec X))$
describes the finite elements of $\I{\PT}(\vec X)$.
This relies on the following.

Given $\BT \in \Base$,
the finite elements of the flat domain $\I\BT$
are exactly the elements of $\BT$.

Let $X,Y \in \Scott$.
The finite elements in the product $X \times Y$
are exactly the pairs of finite elements.
The finite elements of $\Scott(X,Y)$ are exactly the finite sups of step functions.
Given finite $d \in X$ and $e \in Y$,
the \emph{step function} $(d \step e) \colon X \to Y$
is defined as $(d \step e)(x) = e$ if $x \geq d$ and
$(d \step e)(x)= \bot$ otherwise.
Recall that the sup $\bigvee_{i \in I}(d_i \step e_i)$ of 
a finite family of step functions exists
%precisely when
if, and only if,
for every $J \sle I$, the set $\{e_j \mid j \in J\}$ has an upper bound
whenever so does $\{d_j \mid j \in J\}$.
See \cite[Theorem 1.4.12]{ac98book}.


Concerning recursive types, let 
\[
\begin{array}{*{5}{l}}
  G
& :
& \Scott^\ep
& \longto
& \Scott^\ep
\end{array}
\]

\noindent
be a functor which preserves colimits of $\omega$-chains.
Recall that $\Fix G$ is the colimit in \eqref{diag:proof:scott:colim}
where $K \colon \omega \to \Scott^\ep$
takes $n$ to $G^n(\one)$
(similarly as in \eqref{diag:proof:scott:chain}).
We have seen that $\Scott$ is closed in $\DCPO$
under colimits of $\omega$-chains of embeddings.
Hence it follows from \cite[Theorem 3.3.11]{aj95chapter}
that the finite elements of $\Fix G$
are the images of the finite elements of the $G^n(\one)$'s
under the components of the colimiting cocone
$\gamma \colon K \to \Fix G$.

We thus have the following.

%%%%%%%%%%%%%%%%%%%%%%%%%%%%%%%%%%%%%%%%%%%%%%%%%%%%%%%%%%%%%%%%%%%%%%%%%%%
\begin{proposition}
\label{prop:proof:scott:fin}
%%%%%%%%%%%%%%%%%%%%%%%%%%%%%%%%%%%%%%%%%%%%%%%%%%%%%%%%%%%%%%%%%%%%%%%%%%%
$\Fin(\I\PT(\vec X))$ is the set of finite elements of $\I\PT(\vec X)$.
%%%%%%%%%%%%%%%%%%%%%%%%%%%%%%%%%%%%%%%%%%%%%%%%%%%%%%%%%%%%%%%%%%%%%%%%%%%
\end{proposition}
%%%%%%%%%%%%%%%%%%%%%%%%%%%%%%%%%%%%%%%%%%%%%%%%%%%%%%%%%%%%%%%%%%%%%%%%%%%





%%%%%%%%%%%%%%%%%%%%%%%%%%%%%%%%%%%%%%%%%%%%%%%%%%%%%%%%%%%%%%%%%%%%%%%%%%%
\paragraph{Example.}
%%%%%%%%%%%%%%%%%%%%%%%%%%%%%%%%%%%%%%%%%%%%%%%%%%%%%%%%%%%%%%%%%%%%%%%%%%%
We now provide some details on Example~\ref{ex:scott:stream-tree}
on $\I{\Stream\PTbis}$ and $\I{\Tree\PTbis}$,
where $\PTbis$ is a pure type.
We handle streams and binary trees
uniformly by considering the covariant functor
\[
\begin{array}{l l r c l}
  F
& :
& \Scott
& \longto
& \Scott
\\

&
& X
& \longmapsto
& \I\PTbis \times X^\Dir
\end{array}
\]

\noindent
where $\Dir$ is a finite set.
In view of Theorem~\ref{thm:proof:scott:limcolim},
$\Fix F$ is the limit of the $\omega^\op$-chain
\[
\begin{tikzcd} %[column sep=large]
  \one
& F(\one)
  \arrow{l}[above]{\one}
& F^2(\one)
  \arrow{l}[above]{F(\one)}
& F^n(\one)
  \arrow[dashed]{l}
& F^{n+1}(\one)
  \arrow{l}[above]{F^n(\one)}
& \phantom{F}
  \arrow[dashed]{l}
\end{tikzcd}
\]

\noindent
where $\one$ is the terminal Scott domain $\{\bot\}$.
Hence, $\Fix F$ is
\[
\begin{array}{c}
  \left\{
  x \in \prod_{n \in \NN} F^n(\one)
  \mathrel{\Big|}
  x(n) = F^n(\one)(x(n+1))
  \right\}
\end{array}
\]

\noindent
equipped with the pointwise order.
We show that $\Fix F$ is isomorphic to $\I\PTbis^{\Dir^*}$.
Define for each $n \in \NN$ an
isomorphism $\iota_n \colon \I\PTbis^{\Dir^n} \to F^n(\one)$
as $\iota_0 = \one \colon \one \to \one$
and
\[
\begin{array}{l l r c l}
  \iota_{n+1}
& :
& \I\PTbis^{\Dir^{n+1}}
& \longto
& F^{n+1}(\one) = \I\PTbis \times \left( F^n(\one) \right)^\Dir
\\

&
& T
& \longmapsto
& \left( T(\es), (\iota_n(u \mapsto T(d \cdot u)))_{d \in \Dir}  \right)
\end{array}
\]

\noindent
and to observe that the following commutes
\[
\begin{tikzcd}
  \I\PTbis^{\Dir^{n+1}}
  \arrow{d}[left]{T \mapsto T\restr \Dir^n}
  \arrow{r}{\iota_{n+1}}
& F^{n+1}(\one)
  \arrow{d}{F^n(\one)}
\\
  \I\PTbis^{\Dir^{n}}
  \arrow{r}[below]{\iota_{n}}
& F^n(\one)
\end{tikzcd}
\]

The characterization of the finite elements then follows from
Proposition~\ref{prop:proof:scott:fin}.





%%%%%%%%%%%%%%%%%%%%%%%%%%%%%%%%%%%%%%%%%%%%%%%%%%%%%%%%%%%%%%%%%%%%%%%%%%%
\subsection{\nameref{sec:sem:log}}
\label{sec:proof:sem:log}
%%%%%%%%%%%%%%%%%%%%%%%%%%%%%%%%%%%%%%%%%%%%%%%%%%%%%%%%%%%%%%%%%%%%%%%%%%%

First note that if $\triangle$ is either $\pi_1$, $\pi_2$ or $\fold$,
then
since $\I{\form\triangle}$ acts by inverse image
(of resp.\ $\pi_1$, $\pi_2$ and $\I\unfold$), we directly
have that $\I{\form\triangle}$
is monotone (w.r.t.\ inclusion) and preserves all unions and all intersections.

We now consider Lemma~\ref{lem:top:char:fin}.

%%%%%%%%%%%%%%%%%%%%%%%%%%%%%%%%%%%%%%%%%%%%%%%%%%%%%%%%%%%%%%%%%%%%%%%%%%%
\begin{lemma}[Lemma~\ref{lem:top:char:fin}]
\label{lem:proof:top:char:fin}
%%%%%%%%%%%%%%%%%%%%%%%%%%%%%%%%%%%%%%%%%%%%%%%%%%%%%%%%%%%%%%%%%%%%%%%%%%%
Given $\varphi \in \Lang_\land(\PT)$, if $\I\varphi \neq \emptyset$ then
$\I\varphi = \up d$ for some finite $d \in \I\PT$.
Conversely, if $d \in \I\PT$ is finite, then $\up d = \I\varphi$ for some
$\varphi \in \Lang_{\land}(\PT)$.
%%%%%%%%%%%%%%%%%%%%%%%%%%%%%%%%%%%%%%%%%%%%%%%%%%%%%%%%%%%%%%%%%%%%%%%%%%%
\end{lemma}
%%%%%%%%%%%%%%%%%%%%%%%%%%%%%%%%%%%%%%%%%%%%%%%%%%%%%%%%%%%%%%%%%%%%%%%%%%%

The proof of Lemma~\ref{lem:proof:top:char:fin} is split into the next
two lemmas.

%%%%%%%%%%%%%%%%%%%%%%%%%%%%%%%%%%%%%%%%%%%%%%%%%%%%%%%%%%%%%%%%%%%%%%%%%%%
\begin{lemma}
\label{lem:proof:top:fin:compact-open}
%%%%%%%%%%%%%%%%%%%%%%%%%%%%%%%%%%%%%%%%%%%%%%%%%%%%%%%%%%%%%%%%%%%%%%%%%%%
Given $\varphi \in \Lang_\land(\PT)$, if $\I\varphi \neq \emptyset$ then
$\I\varphi = \up d$ for some finite $d \in \I\PT$.
%%%%%%%%%%%%%%%%%%%%%%%%%%%%%%%%%%%%%%%%%%%%%%%%%%%%%%%%%%%%%%%%%%%%%%%%%%%
\end{lemma}
%%%%%%%%%%%%%%%%%%%%%%%%%%%%%%%%%%%%%%%%%%%%%%%%%%%%%%%%%%%%%%%%%%%%%%%%%%%

%%%%%%%%%%%%%%%%%%%%%%%%%%%%%%%%%%%%%%%%%%%%%%%%%%%%%%%%%%%%%%%%%%%%%%%%%%%
\begin{proof}
%%%%%%%%%%%%%%%%%%%%%%%%%%%%%%%%%%%%%%%%%%%%%%%%%%%%%%%%%%%%%%%%%%%%%%%%%%%
The proof is by induction on $\varphi \in \Lang_\land(\PT)$.
We rely on the description of finite elements given
by Proposition~\ref{prop:proof:scott:fin}
(see Figure~\ref{fig:proof:sem:finelt}).
\begin{description}
\item[Case of $\True$.]
In this case, we have $\I\varphi = \up \bot$.

\item[Case of $\varphi \land \psi$.]
First, note that $\I\varphi, \I\psi$ are non-empty since so is their intersection.
By induction hypothesis, there are finite $d,e \in \I\PT$
such that $\I\varphi = \up d$ and $\I\psi = \up e$.
Since $\up d \cap \up e$ is non-empty, and since $\I\PT$ is a Scott domain,
we get that $d \vee e$ is defined, finite, and such that
$\up(d\vee e) = \up d \cap \up e$.
Hence $\I{\varphi \land \psi} = \up(d \vee e)$.

\item[Case of $\form a$ (with $a \in \BT$ for $\BT \in \Base$).]
Since $\I{\form a} = \up a$.

\item[Case of $\form{\triangle}\varphi$
with $\triangle$ either $\pi_1$, $\pi_2$ or $\fold$.]
Note that $\I\varphi$ is non-empty since so is
$\I{\form\triangle\varphi} = \I{\form\triangle}(\I\varphi)$.
Hence by induction hypothesis, there is some finite $d$
such that $\I\varphi = \up d$.

Consider first the case of $\triangle = \fold$.
Then, since $\I\unfold$ is an isomorphism with inverse $\I\fold$
we have
\[
\begin{array}{l l l}
  \I{\form\triangle\varphi}
& =
& \I{\form\triangle}(\up d)
\\

& =
& \left\{
  x \in \I{\rec\TV.\PT} \mid \I\unfold(x) \geq d
  \right\}
\\

& =
& \up \I\fold(d)
\end{array}
\]

\noindent
The result then follows from
Proposition~\ref{prop:proof:scott:fin}.

Consider now the case of $\triangle = \pi_i$,
say $\triangle = \pi_1$ (the other case is symmetric).
Since the order in $\I{\PT_1 \times \PT_2}$ is pointwise,
we have
\[
\begin{array}{l l l}
  \I{\form{\pi_1}\varphi}
& =
& \I{\form{\pi_1}}(\up d)

\\
& =
& \left\{
  x \in \I{\PT_1 \times \PT_2} \mid \pi_1(x) \geq d
  \right\}
\\

& =
& \up(d,\bot)
\end{array}
\]

\noindent
and the result again follows from
Proposition~\ref{prop:proof:scott:fin}.

\item[Case of $\psi \realto \varphi$.]
First, if $\I\psi = \emptyset$,
then $\I{\psi \realto \varphi} = \up \bot$.

Assume now that $\I\psi \neq \emptyset$.
In this case, we must also have $\I\varphi \neq \emptyset$.
Hence by induction hypothesis there are $d,e$ finite
such that
$\up e = \I\psi$ and $\up d = \I\varphi$.
Then we are done since
\[
\begin{array}{l l l}
  \I{\psi \realto \varphi}
& =
& \left\{ f \mid \forall x \in \I\psi,~ f(x) \in \I\varphi\right\}
\\

& =
& \left\{ f \mid \forall x \geq e,~ f(x) \geq d\right\}
\\

& =
& \up \left(e \step d \right)
\end{array}
\]
\qed
\end{description}
%%%%%%%%%%%%%%%%%%%%%%%%%%%%%%%%%%%%%%%%%%%%%%%%%%%%%%%%%%%%%%%%%%%%%%%%%%%
\end{proof}
%%%%%%%%%%%%%%%%%%%%%%%%%%%%%%%%%%%%%%%%%%%%%%%%%%%%%%%%%%%%%%%%%%%%%%%%%%%

%%%%%%%%%%%%%%%%%%%%%%%%%%%%%%%%%%%%%%%%%%%%%%%%%%%%%%%%%%%%%%%%%%%%%%%%%%%
\begin{lemma}
\label{lem:proof:top:compact-open:fin}
%%%%%%%%%%%%%%%%%%%%%%%%%%%%%%%%%%%%%%%%%%%%%%%%%%%%%%%%%%%%%%%%%%%%%%%%%%%
If $d \in \I\PT$ is finite,
then there is
$\varphi \in \Lang_\land(\PT)$
such that $\up d = \I\varphi$.
%%%%%%%%%%%%%%%%%%%%%%%%%%%%%%%%%%%%%%%%%%%%%%%%%%%%%%%%%%%%%%%%%%%%%%%%%%%
\end{lemma}
%%%%%%%%%%%%%%%%%%%%%%%%%%%%%%%%%%%%%%%%%%%%%%%%%%%%%%%%%%%%%%%%%%%%%%%%%%%

%%%%%%%%%%%%%%%%%%%%%%%%%%%%%%%%%%%%%%%%%%%%%%%%%%%%%%%%%%%%%%%%%%%%%%%%%%%
\begin{proof}
%%%%%%%%%%%%%%%%%%%%%%%%%%%%%%%%%%%%%%%%%%%%%%%%%%%%%%%%%%%%%%%%%%%%%%%%%%%
We rely on Proposition~\ref{prop:proof:scott:fin}
and on the inductive definition of $\Fin(\I\PT)$ in Figure~\ref{fig:proof:sem:finelt}.
We reason by cases on the rules 
in Figure~\ref{fig:proof:sem:finelt}.
\begin{description}
\item[Case of]
\[
\dfrac{}
  {\bot \in \Fin(\I\PT)}  
\]

\noindent
Since $\up \bot = \I{\True}$.

\item[Case of]
\[
\dfrac{\text{$\BT \in \Base$ and $a \in \BT$}}
  {a \in \Fin(\I\BT)}
\]

\noindent
Since $\up a = \I{\form a}$.

\item[Case of]
\[
\dfrac{d \in \Fin(\I\PT)
  \qquad
  e \in \Fin(\I\PTbis)}
  {(d,e) \in \Fin(\I{\PT \times \PTbis})}
\]

\noindent
By induction hypothesis, we
have $\varphi \in \Lang_\land(\PT)$
and $\psi \in \Lang_\land(\PTbis)$
such that
$\I\varphi = \up d$
and
$\I\psi = \up e$.
Since the order in $\I{\PT \times \PTbis}$
is pointwise, we get
\[
\begin{array}{l l l}
  \up(d,e)
& =
& \up d \times \up e
\\

& =
& (\up d \times \I\PTbis)
  \cap
  (\I\PT \times \up e)
\\

& =
& \I{\form{\pi_1}\varphi \land \form{\pi_2}\psi}
\end{array}
\]

\item[Case of]
\[
\dfrac{d \in \Fin(\I{\PT[\rec\TV.\PT/\TV]})}
  {\I\fold(d) \in \Fin(\I{\rec\TV.\PT})}
\]

\noindent
By induction hypothesis, there is
$\varphi \in \Lang_\land(\PT[\rec\TV.\PT/\TV])$
such that $\I\varphi = \up d$.
We thus have $\up \I\fold(d) = \I{\form\fold \varphi}$.

\item[Case of]
\[
\dfrac{\begin{array}{l}
  \text{for all $i \in I$,~
  $d_i \in \Fin(\I{\PT})$
  ~and~
  $e_i \in \Fin(\I{\PTbis})$ \@;}
  \\
  \text{for all $J \sle I$,~
  $\bigvee_{j \in J} d_j$ defined in $\I{\PT}$
  ~$\imp$~
  $\bigvee_{j \in J} e_j$ defined in $\I{\PTbis}$}
  \end{array}}
  {\bigvee_{i \in I}(d_i \step e_i) \in \Fin(\I{\PT \arrow \PTbis})}
\]

\noindent
where $I$ is a finite set.

By induction hypothesis, for each $i \in I$
there are $\varphi_i \in \Lang_\land(\PTbis)$
and $\psi_i \in \Lang_\land(\PT)$
such that $\up e_i = \I{\varphi_i}$
and $\up d_i = \I{\psi_i}$.
Note that
\[
\begin{array}{l l l}
  \up(d_i \step e_i)
& =
& \left\{
  f \colon \I\PT \to \I\PTbis \mid
  \forall x \geq d_i,~
  f(x) \geq e_i
  \right\}
\\

& =
& \left\{
  f \colon \I\PT \to \I\PTbis \mid
  \forall x \in \I{\psi_i},~
  f(x) \in \I{\varphi_i}
  \right\}
\\

& =
& \I{\psi_i \realto \varphi_i}
\end{array}
\]

\noindent
The result then follows from the fact that
\[
\begin{array}{l l l}
  \up \left(
  \bigvee_{i \in I} d_i \step e_i
  \right)
& =
& \bigcap_{i \in I} \up(d_i \step e_i)
\end{array}
\]
\qed
\end{description}
%%%%%%%%%%%%%%%%%%%%%%%%%%%%%%%%%%%%%%%%%%%%%%%%%%%%%%%%%%%%%%%%%%%%%%%%%%%
\end{proof}
%%%%%%%%%%%%%%%%%%%%%%%%%%%%%%%%%%%%%%%%%%%%%%%%%%%%%%%%%%%%%%%%%%%%%%%%%%%


We now turn to Lemma~\ref{lem:top:char}.
We first recall its statement.


%%%%%%%%%%%%%%%%%%%%%%%%%%%%%%%%%%%%%%%%%%%%%%%%%%%%%%%%%%%%%%%%%%%%%%%%%%%
\begin{lemma}[Lemma~\ref{lem:top:char}]
\label{lem:proof:top:char}
%%%%%%%%%%%%%%%%%%%%%%%%%%%%%%%%%%%%%%%%%%%%%%%%%%%%%%%%%%%%%%%%%%%%%%%%%%%
A set $\SP \sle \I\PT$ is saturated (resp.\ Scott-open)
if, and only if, there is a formula $\varphi \in \Lang(\PT)$
(resp.\@ $\varphi \in \Lang_\Open(\PT)$)
such that $\SP = \I\varphi$.

In particular, for each $\varphi \in \Lang(\PT)$
we have $\I\varphi = \I\psi$ for some $\psi \in \Norm(\PT)$.
%%%%%%%%%%%%%%%%%%%%%%%%%%%%%%%%%%%%%%%%%%%%%%%%%%%%%%%%%%%%%%%%%%%%%%%%%%%
\end{lemma}
%%%%%%%%%%%%%%%%%%%%%%%%%%%%%%%%%%%%%%%%%%%%%%%%%%%%%%%%%%%%%%%%%%%%%%%%%%%

%%%%%%%%%%%%%%%%%%%%%%%%%%%%%%%%%%%%%%%%%%%%%%%%%%%%%%%%%%%%%%%%%%%%%%%%%%%
\begin{proof}
%%%%%%%%%%%%%%%%%%%%%%%%%%%%%%%%%%%%%%%%%%%%%%%%%%%%%%%%%%%%%%%%%%%%%%%%%%%
Consider first the case of a set $\SP \sle \I\PT$
which is open (resp.\ saturated).
Then $\SP$ is a union (resp.\ an intersection of unions) of
sets of the form $\up d$ with $d \in \I\PT$ finite.
Then result then follows from Lemma~\ref{lem:proof:top:compact-open:fin}
using the closure of $\Lang_\Open(\PT)$ under arbitrary disjunctions
(resp.\ the closure of $\Lang(\PT)$ under arbitrary disjunctions
and conjunctions).

The converse is proven by induction on formulae.
Since opens are stable under unions and finite intersections,
the case of $\varphi \in \Lang_\Open(\PT)$
directly follows from Lemma~\ref{lem:proof:top:fin:compact-open}.
As for $\varphi \in \Lang(\PT)$,
since saturated sets are stable under all unions and intersections,
we only have to consider the cases of modalities.
\begin{description}
\item[Case of $\form a$ (with $a \in \BT$ for $\BT \in \Base$).]
Trivial, since $\I{\form a}$ is compact open.

\item[Case of $\form{\triangle}\varphi$
with $\triangle$ either $\pi_1$, $\pi_2$ or $\fold$.]
Similarly as in Lemma~\ref{lem:proof:top:fin:compact-open},
one can apply the induction hypothesis and use the fact
that $\I{\form\triangle}$ preserves all unions and all intersections.

\item[Case of $\psi \realto \varphi$.]
By induction hypothesis, $\I\varphi$ is upward-closed.
Hence so is $\I{\psi \realto \varphi}$.
\end{description}

For the last part of the statement,
let $\varphi \in \Lang(\PT)$.
Since $\I\varphi$ is saturated,
it is an intersection of unions of sets
of the form $\up d$ with $d \in \I\PT$ finite.
Lemma~\ref{lem:proof:top:compact-open:fin} yields
that such $\up d$'s are definable in $\Lang_\land(\PT)$,
whence the result.
\qed
%%%%%%%%%%%%%%%%%%%%%%%%%%%%%%%%%%%%%%%%%%%%%%%%%%%%%%%%%%%%%%%%%%%%%%%%%%%
\end{proof}
%%%%%%%%%%%%%%%%%%%%%%%%%%%%%%%%%%%%%%%%%%%%%%%%%%%%%%%%%%%%%%%%%%%%%%%%%%%



We finally discuss Proposition~\ref{prop:sem:sound:ded}.

%%%%%%%%%%%%%%%%%%%%%%%%%%%%%%%%%%%%%%%%%%%%%%%%%%%%%%%%%%%%%%%%%%%%%%%%%%%
\begin{proposition}[Soundness of Deduction (Proposition \ref{prop:sem:sound:ded})]
\label{prop:proof:sem:sound:ded}
%%%%%%%%%%%%%%%%%%%%%%%%%%%%%%%%%%%%%%%%%%%%%%%%%%%%%%%%%%%%%%%%%%%%%%%%%%%
If $\psi \thesis \varphi$ is derivable in the basic deduction system 
in Figure~\ref{fig:log:ded} (\S\ref{sec:log}),
then $\I\psi \sle \I\varphi$.
%%%%%%%%%%%%%%%%%%%%%%%%%%%%%%%%%%%%%%%%%%%%%%%%%%%%%%%%%%%%%%%%%%%%%%%%%%%
\end{proposition}
%%%%%%%%%%%%%%%%%%%%%%%%%%%%%%%%%%%%%%%%%%%%%%%%%%%%%%%%%%%%%%%%%%%%%%%%%%%

%%%%%%%%%%%%%%%%%%%%%%%%%%%%%%%%%%%%%%%%%%%%%%%%%%%%%%%%%%%%%%%%%%%%%%%%%%%
\begin{proof}
%%%%%%%%%%%%%%%%%%%%%%%%%%%%%%%%%%%%%%%%%%%%%%%%%%%%%%%%%%%%%%%%%%%%%%%%%%%
The proof is by induction on $\psi \thesis \varphi$,
and by cases on the rules in Figure~\ref{fig:log:ded}.
The cases of the rules for (infinitary) propositional
logic directly follow from the definition of the interpretation.
So we just have to discuss modalities.

Let $\triangle$ be either $\pi_1$, $\pi_2$ or $\fold$.
Since $\I{\form\triangle}$ acts by inverse image
(of resp.\ $\pi_1$, $\pi_2$ and $\I\unfold$), we directly
have that $\I{\form\triangle}$
is monotone (w.r.t.\ inclusion) and preserves all unions and all intersections.
This handles all the rules for $\form\triangle$.

The rule $\ax{F}$ has already been discussed,
and the other rules for $\realto$ are straightforward to check.
\qed
%%%%%%%%%%%%%%%%%%%%%%%%%%%%%%%%%%%%%%%%%%%%%%%%%%%%%%%%%%%%%%%%%%%%%%%%%%%
\end{proof}
%%%%%%%%%%%%%%%%%%%%%%%%%%%%%%%%%%%%%%%%%%%%%%%%%%%%%%%%%%%%%%%%%%%%%%%%%%%




