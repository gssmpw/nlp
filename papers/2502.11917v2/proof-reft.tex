%%%%%%%%%%%%%%%%%%%%%%%%%%%%%%%%%%%%%%%%%%%%%%%%%%%%%%%%%%%%%%%%%%%%%%%%%%%
\section{Proofs of \S\ref{sec:system} (\nameref{sec:system})}
%%%%%%%%%%%%%%%%%%%%%%%%%%%%%%%%%%%%%%%%%%%%%%%%%%%%%%%%%%%%%%%%%%%%%%%%%%%

%%%%%%%%%%%%%%%%%%%%%%%%%%%%%%%%%%%%%%%%%%%%%%%%%%%%%%%%%%%%%%%%%%%%%%%%%%%
\subsection{Proofs of \S\ref{sec:log} (\nameref{sec:log})}
\label{sec:proof:log}
%%%%%%%%%%%%%%%%%%%%%%%%%%%%%%%%%%%%%%%%%%%%%%%%%%%%%%%%%%%%%%%%%%%%%%%%%%%

%%%%%%%%%%%%%%%%%%%%%%%%%%%%%%%%%%%%%%%%%%%%%%%%%%%%%%%%%%%%%%%%%%%%%%%%%%%
\begin{figure}[t!]
%%%%%%%%%%%%%%%%%%%%%%%%%%%%%%%%%%%%%%%%%%%%%%%%%%%%%%%%%%%%%%%%%%%%%%%%%%%
\[
\begin{array}{c}

\dfrac{\text{for each $i \in I$, $\psi_i \,\thesis\, \varphi_i$}}
  {\bigwedge_{i \in I} \psi_i \,\thesis\, \bigwedge_{i \in I}\varphi_i}

\qquad\qquad

\dfrac{\text{for each $i \in I$, $\psi_i \thesis \varphi_i$}}
  {\bigvee_{i \in I} \psi_i \thesis \bigvee_{i \in I}\varphi_i}

\\\\

  \form\triangle \bigwedge_{i \in I} \varphi_i
  \,\thesisiff\,
  \bigwedge_{i \in I} \form\triangle \varphi_i

\qquad\qquad

  \bigvee_{i \in I} \form\triangle\varphi_i
  \,\thesisiff\,
  \form\triangle \bigvee_{i \in I}\varphi_i

\\\\

  \bigwedge_{i \in I} \bigvee_{j \in J_i} \varphi_{i,j}
  \,\thesisiff\,
  \bigvee_{f \in \prod_{i \in I}J_i}\bigwedge_{i \in I} \varphi_{i,f(i)}

\\\\

  \bigwedge_{f \in \prod_{i \in I}J_i}\bigvee_{i \in I} \varphi_{i,f(i)}
  \,\thesisiff\,
  \bigvee_{i \in I}\bigwedge_{j \in J_i}\varphi_{i,j}
\end{array}
\]
\caption{Some derivable rules and sequents,
where $\triangle$ is either $\pi_1$, $\pi_2$ or $\fold$.%
\label{fig:proof:log:derivable}}
%%%%%%%%%%%%%%%%%%%%%%%%%%%%%%%%%%%%%%%%%%%%%%%%%%%%%%%%%%%%%%%%%%%%%%%%%%%
\end{figure}
%%%%%%%%%%%%%%%%%%%%%%%%%%%%%%%%%%%%%%%%%%%%%%%%%%%%%%%%%%%%%%%%%%%%%%%%%%%

\noindent
In this Appendix~\ref{sec:proof:log},
we give details on Figure~\ref{fig:proof:log:derivable},
which gathers some derivable rule and sequents
(including those of Examples~\ref{ex:log:modalnf} and~\ref{ex:log:distr}).


%%%%%%%%%%%%%%%%%%%%%%%%%%%%%%%%%%%%%%%%%%%%%%%%%%%%%%%%%%%%%%%%%%%%%%%%%%%
\begin{lemma}
\label{lem:proof:log:functprop}
%%%%%%%%%%%%%%%%%%%%%%%%%%%%%%%%%%%%%%%%%%%%%%%%%%%%%%%%%%%%%%%%%%%%%%%%%%%
The following rules are derivable
\[
\begin{array}{c}

\dfrac{\text{for each $i \in I$, $\psi_i \thesis \varphi_i$}}
  {\bigwedge_{i \in I} \psi_i \thesis \bigwedge_{i \in I}\varphi_i}

\qquad\qquad

\dfrac{\text{for each $i \in I$, $\psi_i \thesis \varphi_i$}}
  {\bigvee_{i \in I} \psi_i \thesis \bigvee_{i \in I}\varphi_i}

\end{array}
\]
%%%%%%%%%%%%%%%%%%%%%%%%%%%%%%%%%%%%%%%%%%%%%%%%%%%%%%%%%%%%%%%%%%%%%%%%%%%
\end{lemma}
%%%%%%%%%%%%%%%%%%%%%%%%%%%%%%%%%%%%%%%%%%%%%%%%%%%%%%%%%%%%%%%%%%%%%%%%%%%

%%%%%%%%%%%%%%%%%%%%%%%%%%%%%%%%%%%%%%%%%%%%%%%%%%%%%%%%%%%%%%%%%%%%%%%%%%%
\begin{proof}
%%%%%%%%%%%%%%%%%%%%%%%%%%%%%%%%%%%%%%%%%%%%%%%%%%%%%%%%%%%%%%%%%%%%%%%%%%%
The premise of the first rule
yields
$\bigwedge_{i \in I} \psi_i \thesis \varphi_i$
for all $i \in I$, from which we obtain
$\bigwedge_{i \in I} \psi_i \thesis \bigwedge_{i \in I} \varphi_i$.
The second rule is handled similarly.
%%%%%%%%%%%%%%%%%%%%%%%%%%%%%%%%%%%%%%%%%%%%%%%%%%%%%%%%%%%%%%%%%%%%%%%%%%%
\end{proof}
%%%%%%%%%%%%%%%%%%%%%%%%%%%%%%%%%%%%%%%%%%%%%%%%%%%%%%%%%%%%%%%%%%%%%%%%%%%




%%%%%%%%%%%%%%%%%%%%%%%%%%%%%%%%%%%%%%%%%%%%%%%%%%%%%%%%%%%%%%%%%%%%%%%%%%%
\begin{lemma}
%%%%%%%%%%%%%%%%%%%%%%%%%%%%%%%%%%%%%%%%%%%%%%%%%%%%%%%%%%%%%%%%%%%%%%%%%%%
The following sequents are derivable,
where $\triangle$ is either $\pi_1$, $\pi_2$ or $\fold$:
\[
\begin{array}{c}

  \form\triangle \bigwedge_{i \in I} \varphi_i
  \,\thesis\,
  \bigwedge_{i \in I} \form\triangle \varphi_i

\qquad\text{and}\qquad

  \bigvee_{i \in I} \form\triangle\varphi_i
  \,\thesis\,
  \form\triangle \bigvee_{i \in I}\varphi_i

\end{array}
\]
%%%%%%%%%%%%%%%%%%%%%%%%%%%%%%%%%%%%%%%%%%%%%%%%%%%%%%%%%%%%%%%%%%%%%%%%%%%
\end{lemma}
%%%%%%%%%%%%%%%%%%%%%%%%%%%%%%%%%%%%%%%%%%%%%%%%%%%%%%%%%%%%%%%%%%%%%%%%%%%

%%%%%%%%%%%%%%%%%%%%%%%%%%%%%%%%%%%%%%%%%%%%%%%%%%%%%%%%%%%%%%%%%%%%%%%%%%%
\begin{proof}
%%%%%%%%%%%%%%%%%%%%%%%%%%%%%%%%%%%%%%%%%%%%%%%%%%%%%%%%%%%%%%%%%%%%%%%%%%%
For each $i \in I$ we have
\begin{center}
\AXC{}
\UIC{$\varphi_i \,\thesis\, \varphi_i$}
\UIC{$\bigwedge_{i \in I} \varphi_i \,\thesis\, \varphi_i$}
\UIC{$\form\triangle \bigwedge_{i \in I} \varphi_i \,\thesis\, \form\triangle \varphi_i$}
\DisplayProof
\end{center}

\noindent
from which we obtain the first sequent.
The other one is derived similarly.
\qed
%%%%%%%%%%%%%%%%%%%%%%%%%%%%%%%%%%%%%%%%%%%%%%%%%%%%%%%%%%%%%%%%%%%%%%%%%%%
\end{proof}
%%%%%%%%%%%%%%%%%%%%%%%%%%%%%%%%%%%%%%%%%%%%%%%%%%%%%%%%%%%%%%%%%%%%%%%%%%%



%%%%%%%%%%%%%%%%%%%%%%%%%%%%%%%%%%%%%%%%%%%%%%%%%%%%%%%%%%%%%%%%%%%%%%%%%%%
\begin{lemma}
%%%%%%%%%%%%%%%%%%%%%%%%%%%%%%%%%%%%%%%%%%%%%%%%%%%%%%%%%%%%%%%%%%%%%%%%%%%
The following sequents are derivable
\[
\begin{array}{r !{~}l!{~} l}
  \bigvee_{f \in \prod_{i \in I}J_i}\bigwedge_{i \in I} \varphi_{i,f(i)}
& \thesis
& \bigwedge_{i \in I} \bigvee_{j \in J_i} \varphi_{i,j}
\\

  \bigvee_{i \in I}\bigwedge_{j \in J_i}\varphi_{i,j}
& \thesis
& \bigwedge_{f \in \prod_{i \in I}J_i} \bigvee_{i \in I} \varphi_{i,f(i)}
\end{array}
\]
%%%%%%%%%%%%%%%%%%%%%%%%%%%%%%%%%%%%%%%%%%%%%%%%%%%%%%%%%%%%%%%%%%%%%%%%%%%
\end{lemma}
%%%%%%%%%%%%%%%%%%%%%%%%%%%%%%%%%%%%%%%%%%%%%%%%%%%%%%%%%%%%%%%%%%%%%%%%%%%

%%%%%%%%%%%%%%%%%%%%%%%%%%%%%%%%%%%%%%%%%%%%%%%%%%%%%%%%%%%%%%%%%%%%%%%%%%%
\begin{proof}
%%%%%%%%%%%%%%%%%%%%%%%%%%%%%%%%%%%%%%%%%%%%%%%%%%%%%%%%%%%%%%%%%%%%%%%%%%%
We only discuss the first, as the other one can be dealt-with similarly.
Let $f \in \prod_{i \in I}J_i$.
For each $i \in I$, derive
\begin{center}
\AXC{}
\UIC{$\varphi_{i,f(i)} \,\thesis\, \varphi_{i,f(i)}$}
\UIC{$\varphi_{i,f(i)} \,\thesis\, \bigvee_{j \in J_i} \varphi_{i,j}$}
\DisplayProof
\end{center}

\noindent
Hence Lemma~\ref{lem:proof:log:functprop}
gives
\[
\begin{array}{l l l}
  \bigwedge_{i \in I} \varphi_{i,f(i)}
& \thesis
& \bigwedge_{i \in I} \bigvee_{j \in J_i} \varphi_{i,j}
\end{array}
\]

\noindent
Then we are done since this holds for each $f \in \prod_{i \in I}J_i$.
\qed
%%%%%%%%%%%%%%%%%%%%%%%%%%%%%%%%%%%%%%%%%%%%%%%%%%%%%%%%%%%%%%%%%%%%%%%%%%%
\end{proof}
%%%%%%%%%%%%%%%%%%%%%%%%%%%%%%%%%%%%%%%%%%%%%%%%%%%%%%%%%%%%%%%%%%%%%%%%%%%




%%%%%%%%%%%%%%%%%%%%%%%%%%%%%%%%%%%%%%%%%%%%%%%%%%%%%%%%%%%%%%%%%%%%%%%%%%%
\begin{lemma}
\label{lem:proof:log:distr}
%%%%%%%%%%%%%%%%%%%%%%%%%%%%%%%%%%%%%%%%%%%%%%%%%%%%%%%%%%%%%%%%%%%%%%%%%%%
The following sequent is derivable
\[
\dfrac{}
  {\bigwedge_{f \in \prod_{i \in I}J_i}\bigvee_{i \in I} \varphi_{i,f(i)}
  \thesis
  \bigvee_{i \in I}\bigwedge_{j \in J_i}\varphi_{i,j}}
\]
%%%%%%%%%%%%%%%%%%%%%%%%%%%%%%%%%%%%%%%%%%%%%%%%%%%%%%%%%%%%%%%%%%%%%%%%%%%
\end{lemma}
%%%%%%%%%%%%%%%%%%%%%%%%%%%%%%%%%%%%%%%%%%%%%%%%%%%%%%%%%%%%%%%%%%%%%%%%%%%

%%%%%%%%%%%%%%%%%%%%%%%%%%%%%%%%%%%%%%%%%%%%%%%%%%%%%%%%%%%%%%%%%%%%%%%%%%%
\begin{proof}
%%%%%%%%%%%%%%%%%%%%%%%%%%%%%%%%%%%%%%%%%%%%%%%%%%%%%%%%%%%%%%%%%%%%%%%%%%%
This sequent amounts a well-known fact on completely distributive
complete lattices,
see e.g.~\cite[Lemma VII.1.10]{johnstone82book}.
We nevertheless offer a detailed proof.
Using the distributive law $\ax{D}$, we have
\begin{equation*}
\begin{array}{l !{~}l!{~} l}
  \bigwedge_{f \in \prod_{i \in I}J_i}\bigvee_{i \in I} \varphi_{i,f(i)}
& \thesis
& \bigvee_{F\colon (\prod_{i \in I}J_i) \to I}
  \bigwedge_{f \in \prod_{i \in I}J_i}
  \varphi_{F(f),f(F(f))}
\end{array}
\end{equation*}

\noindent
Hence we are done if we show
\begin{equation*}
\begin{array}{l !{~}l!{~} l}
  \bigvee_{F\colon (\prod_{i \in I}J_i) \to I}
  \bigwedge_{f \in \prod_{i \in I}J_i}
  \varphi_{F(f),f(F(f))}
& \thesis
& \bigvee_{i \in I}\bigwedge_{j \in J_i}\varphi_{i,j}
\end{array}
\end{equation*}

\noindent
So let $F \colon \left(\prod_{i \in I}J_i\right)  \to I$
and assume toward a contradiction that 
\begin{equation*}
\begin{array}{l !{~}l!{~} l}
  \bigwedge_{f \in \prod_{i \in I}J_i}
  \varphi_{F(f),f(F(f))}
& \not\thesis
& \bigvee_{i \in I}\bigwedge_{j \in J_i}\varphi_{i,j}
\end{array}
\end{equation*}

\noindent
It follows that for each $i \in I$, there is some $j \in J_i$ such that
\begin{equation*}
\begin{array}{l !{~}l!{~} l}
  \bigwedge_{f \in \prod_{i \in I}J_i}
  \varphi_{F(f),f(F(f))}
& \not\thesis
& \varphi_{i,j}
\end{array}
\end{equation*}

\noindent
Using the Axiom of Choice, we get a function $g \in \prod_{i \in I}J_i$
such that for all $i \in I$,
\begin{equation*}
\begin{array}{l !{~}l!{~} l}
  \bigwedge_{f \in \prod_{i \in I}J_i}
  \varphi_{F(f),f(F(f))}
& \not\thesis
& \varphi_{i,g(i)}
\end{array}
\end{equation*}

\noindent
In particular,
\begin{equation*}
\begin{array}{l !{~}l!{~} l}
  \bigwedge_{f \in \prod_{i \in I}J_i}
  \varphi_{F(f),f(F(f))}
& \not\thesis
& \varphi_{F(g),g(F(g))}
\end{array}
\end{equation*}

\noindent
a contradiction.
\qed
%%%%%%%%%%%%%%%%%%%%%%%%%%%%%%%%%%%%%%%%%%%%%%%%%%%%%%%%%%%%%%%%%%%%%%%%%%%
\end{proof}
%%%%%%%%%%%%%%%%%%%%%%%%%%%%%%%%%%%%%%%%%%%%%%%%%%%%%%%%%%%%%%%%%%%%%%%%%%%


%%%%%%%%%%%%%%%%%%%%%%%%%%%%%%%%%%%%%%%%%%%%%%%%%%%%%%%%%%%%%%%%%%%%%%%%%%%
\subsection{Proofs of \S\ref{sec:reft} (\nameref{sec:reft})}
\label{sec:proof:reft}
%%%%%%%%%%%%%%%%%%%%%%%%%%%%%%%%%%%%%%%%%%%%%%%%%%%%%%%%%%%%%%%%%%%%%%%%%%%

Lemma~\ref{lem:reft}
will be useful for completeness (\S\ref{sec:compl} and \S\ref{sec:proof:compl}).

%%%%%%%%%%%%%%%%%%%%%%%%%%%%%%%%%%%%%%%%%%%%%%%%%%%%%%%%%%%%%%%%%%%%%%%%%%%
\begin{lemma}[Lemma \ref{lem:reft}]
\label{lem:proof:reft}
%%%%%%%%%%%%%%%%%%%%%%%%%%%%%%%%%%%%%%%%%%%%%%%%%%%%%%%%%%%%%%%%%%%%%%%%%%%
For each $\RT$, there is $\varphi \in \Lang(\UPT\RT)$
such that $\RT \eqtype \reft{\UPT\RT \mid \varphi}$.
%%%%%%%%%%%%%%%%%%%%%%%%%%%%%%%%%%%%%%%%%%%%%%%%%%%%%%%%%%%%%%%%%%%%%%%%%%%
\end{lemma}
%%%%%%%%%%%%%%%%%%%%%%%%%%%%%%%%%%%%%%%%%%%%%%%%%%%%%%%%%%%%%%%%%%%%%%%%%%%

%%%%%%%%%%%%%%%%%%%%%%%%%%%%%%%%%%%%%%%%%%%%%%%%%%%%%%%%%%%%%%%%%%%%%%%%%%%
\begin{proof}
%%%%%%%%%%%%%%%%%%%%%%%%%%%%%%%%%%%%%%%%%%%%%%%%%%%%%%%%%%%%%%%%%%%%%%%%%%%
The proof is by induction on $\RT$.
The base case of $\reft{\PT \mid \varphi}$ is trivial.
In the base case of $\PT$, one can take $\varphi = \True$.
In the cases of $\RT \times \RTbis$ and $\RTbis \arrow \RT$,
by induction hypotheses we get $\varphi \in \Lang(\UPT\RT)$
and $\psi \in \Lang(\UPT\RTbis)$ such that
$\RT \eqtype \reft{\UPT\RT \mid \varphi}$
and
$\RTbis \eqtype \reft{\UPT\RTbis \mid \varphi}$.
We then conclude with
\[
\begin{array}{r c l}
  \RT \times \RTbis
& \eqtype
& \reft{\UPT\RT \times \UPT\RTbis \mid \pair{\varphi,\psi}}
\\

  \RTbis \arrow \RT
& \eqtype
& \reft{\UPT\RTbis \arrow \UPT\RT \mid \psi \realto \varphi}
\end{array}
\]
\qed
%%%%%%%%%%%%%%%%%%%%%%%%%%%%%%%%%%%%%%%%%%%%%%%%%%%%%%%%%%%%%%%%%%%%%%%%%%%
\end{proof}
%%%%%%%%%%%%%%%%%%%%%%%%%%%%%%%%%%%%%%%%%%%%%%%%%%%%%%%%%%%%%%%%%%%%%%%%%%%

