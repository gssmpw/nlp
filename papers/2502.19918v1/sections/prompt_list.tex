\section{Prompt List}
\label{sec:appendix_prompt_list}

In this section, we present all the prompts required for the main experiments. To enhance clarity, we provide only one example in the prompt labeled as \textcolor{b}{\texttt{Example 1}}; the other few-shot examples utilized are labeled as \textcolor{b}{\texttt{Other In-Context Few-shots}} within the prompt.

\subsection{Logical Reasoning using FOL}
\label{prompt:f1}

\textbf{System Instruction:}
"You are a helpful assistant designed to determine the truth value for the conclusion based on the given promises in \textit{first-order-logical format}. Analyze the logical relationship between the premises and the conclusion carefully. The truth value should be determined as True, False, or Unknown solely on the deductions you can make from the premises. Only return the truth value as your final answer."

\textcolor{b}{\texttt{Example 1}}:

\textbf{Premises}:

\text{Has}(\text{mary}, \text{flu}) 

$\forall$ x (\text{Has}(x, \text{flu}) $\rightarrow$ \text{Has}(x, \text{influenza}))

$\neg$ \text{Has}(\text{susan}, \text{influenza})

\textbf{Conclusion}:

\text{Has}(\text{mary}, \text{influenza}) $\oplus$ \text{Has}(\text{susan}, \text{influenza})

\textbf{Truth value}:

True

\textcolor{b}{\texttt{Other In-Context Few-shots}}

\textbf{Premises}: \textcolor{r}{\{Premises\}}

\textbf{Conclusions}: \textcolor{g}{\{Conclusions\}}

\textbf{Truth Value}: 

\subsection{Logical Reasoning using CodeProof}
\label{prompt:f2}

\textbf{System Instruction:}
"You are a helpful assistant designed to determine the truth value for the conclusion based on the \textit{resolution tree in code format}. Analyze the logical relationship between the premises and the conclusion carefully. The truth value should be determined as True, False, or Unknown solely on the deductions you can make from the resolution tree. Only return the truth value as your final answer."

\textcolor{b}{\texttt{Example 1}}:

\textbf{Conclusion}:

\text{Has(mary, influenza)} $\oplus$ \text{Has(susan, influenza)}

\textbf{Proof}:

\text{And(Or(Has(mary,influenza),Has(susan,influenza)),Not(And(Has(mary,influenza),Has(susan,influenza))))}\\
\textbar \textendash \textendash \text{Or(Has(mary,influenza),Has(susan,influenza))}\\
\textbar \quad \textbar \textendash \textendash  \text{Has(mary,influenza)}\\
\textbar \quad \quad \textbar \textendash \textendash \text{Has(mary,flu)}\\
\textbar \quad \quad \textbar \textendash \textendash \text{Or(Has(\$x13,influenza),Not(Has(\$x13,flu)))}\\
\textbar \textendash \textendash \text{Or(Not(Has(mary,influenza)),Not(Has(susan,influenza)))}\\
\textbar \quad \textbar \textendash \textendash \text{Not(Has(susan,influenza))}

\textbf{Truth value}:

True

\textcolor{b}{\texttt{Other In-Context Few-shots}}

\textbf{Premises}: \textcolor{r}{\{Premises\}}

\textbf{Conclusions}: \textcolor{g}{\{Conclusions\}}

\textbf{Truth Value}: 

\subsection{Logical Reasoning using FOL+CodeProof}
\label{prompt:f3}

\textbf{System Instruction:}
"You are a helpful assistant designed to determine the truth value for the conclusion based on the \textit{premises in first-order logic format} and \textit{resolution tree in code format}. Analyze the logical relationship between the premises and the conclusion carefully. The truth value should be determined as True, False, or Unknown solely on the deductions you can make from the resolution tree and the given premises. Only return the truth value as your final answer."

\textcolor{b}{\texttt{Example 1}}:

\textbf{Premises}:

\text{Has}(\text{mary}, \text{flu}) 

$\forall$ x (\text{Has}(x, \text{flu}) $\rightarrow$ \text{Has}(x, \text{influenza}))

$\neg$ \text{Has}(\text{susan}, \text{influenza})

\textbf{Conclusion}:

\text{Has(mary, influenza)} $\oplus$ \text{Has(susan, influenza)}

\textbf{Proof}:

\text{And(Or(Has(mary,influenza),Has(susan,influenza)),Not(And(Has(mary,influenza),Has(susan,influenza))))}\\
\textbar \textendash \textendash \text{Or(Has(mary,influenza),Has(susan,influenza))}\\
\textbar \quad \textbar \textendash \textendash  \text{Has(mary,influenza)}\\
\textbar \quad \quad \textbar \textendash \textendash \text{Has(mary,flu)}\\
\textbar \quad \quad \textbar \textendash \textendash \text{Or(Has(\$x13,influenza),Not(Has(\$x13,flu)))}\\
\textbar \textendash \textendash \text{Or(Not(Has(mary,influenza)),Not(Has(susan,influenza)))}\\
\textbar \quad \textbar \textendash \textendash \text{Not(Has(susan,influenza))}

\textbf{Truth value}:

True

\textcolor{b}{\texttt{Other In-Context Few-shots}}

\textbf{Premises}: \textcolor{r}{\{Premises\}}

\textbf{Conclusions}: \textcolor{g}{\{Conclusions\}}

\textbf{Truth Value}: 

\subsection{Logical Reasoning using FOL+CodeProof+Explanations}

\textbf{System Instruction:}
"You are a helpful assistant designed to determine the truth value for the conclusion based on the \textit{premises in first-order logic format} and \textit{resolution tree in code format}. Analyze the logical relationship between the premises and the conclusion carefully. The truth value should be determined as True, False, or Unknown solely on the deductions you can make from the resolution tree and the given premises. Only return the truth value as your final answer."

\textcolor{b}{\texttt{Example 1}}:

\textbf{Premises}:

\text{Has}(\text{mary}, \text{flu}) 

$\forall$ x (\text{Has}(x, \text{flu}) $\rightarrow$ \text{Has}(x, \text{influenza}))

$\neg$ \text{Has}(\text{susan}, \text{influenza})

\textbf{Conclusion}:

\text{Has(mary, influenza)} $\oplus$ \text{Has(susan, influenza)}

\textbf{Proof}:

\text{And(Or(Has(mary,influenza),Has(susan,influenza)),Not(And(Has(mary,influenza),Has(susan,influenza))))}\\
\textbar \textendash \textendash \text{Or(Has(mary,influenza),Has(susan,influenza))}\\
\textbar \quad \textbar \textendash \textendash  \text{Has(mary,influenza)}\\
\textbar \quad \quad \textbar \textendash \textendash \text{Has(mary,flu)}\\
\textbar \quad \quad \textbar \textendash \textendash \text{Or(Has(\$x13,influenza),Not(Has(\$x13,flu)))}\\
\textbar \textendash \textendash \text{Or(Not(Has(mary,influenza)),Not(Has(susan,influenza)))}\\
\textbar \quad \textbar \textendash \textendash \text{Not(Has(susan,influenza))}

\textbf{Explanation:}

The truth value is True because the proof correctly demonstrates the conclusion given the premises. Here's the explanation:

1. Premise 1: \(\text{Has(mary, flu)}\)

2. Premise 2: \(\forall x (\text{Has}(x, \text{flu}) \rightarrow \text{Has}(x, \text{influenza}))\)

From Premise 2, we can infer that:
\[ \text{Has(mary, flu)} \rightarrow \text{Has(mary, influenza)} \]

Since Premise 1 states \(\text{Has(mary, flu)}\), we can conclude:
\[ \text{Has(mary, influenza)} \]

3. Premise 3: \(\neg \text{Has(susan, influenza)}\)

Now, the conclusion we need to prove is:
\[ \text{Has(mary, influenza)} \oplus \text{Has(susan, influenza)} \]

The symbol \(\oplus\) represents exclusive or (XOR), meaning either one but not both.

4. Proof structure:

The proof shows that \(\text{Has(mary, influenza)}\) is true.
It also confirms that \(\text{Has(susan, influenza)}\) is false (from Premise 3).

Given that:
Has(mary, influenza) is \textit{true}.
Has(susan, influenza) is false.

The conclusion \(\text{Has(mary, influenza)} \oplus \text{Has(susan, influenza)}\) holds true because only one of the two statements is true, satisfying the XOR condition.
Therefore, the truth value of the conclusion is True.

\textbf{Truth value}:

True

\textcolor{b}{\texttt{Other In-Context Few-shots}}

\textbf{Premises}: \textcolor{r}{\{Premises\}}

\textbf{Conclusions}: \textcolor{g}{\{Conclusions\}}

\textbf{Truth Value}: 

\subsection{CodeProof Generation}

\textbf{System Instruction:}
``You are tasked with creating a resolution proof tree that leads to a conclusion based on premises provided in the first-order logic format. Begin by converting all premises into conjunctive normal form (CNF). Then, systematically apply the resolution inference rule to derive the conclusion. Your goal is to organize your derivations in a clear tree structure, starting with the premises at the base and gradually working up to the conclusion at the top. Ensure that the proof tree focuses solely on the logical structure and derivations, avoiding any additional annotations or comments within the tree itself."

\textcolor{b}{\texttt{Example 1}}:

\textbf{Premises}:

\text{Has}(\text{mary}, \text{flu}) 

$\forall$ x (\text{Has}(x, \text{flu}) $\rightarrow$ \text{Has}(x, \text{influenza}))

$\neg$ \text{Has}(\text{susan}, \text{influenza})

\textbf{Conclusion}:

\text{Has(mary, influenza)} $\oplus$ \text{Has(susan, influenza)}

\textbf{Proof}:

\text{And(Or(Has(mary,influenza),Has(susan,influenza)),Not(And(Has(mary,influenza),Has(susan,influenza))))}\\
\textbar \textendash \textendash \text{Or(Has(mary,influenza),Has(susan,influenza))}\\
\textbar \quad \textbar \textendash \textendash  \text{Has(mary,influenza)}\\
\textbar \quad \quad \textbar \textendash \textendash \text{Has(mary,flu)}\\
\textbar \quad \quad \textbar \textendash \textendash \text{Or(Has(\$x13,influenza),Not(Has(\$x13,flu)))}\\
\textbar \textendash \textendash \text{Or(Not(Has(mary,influenza)),Not(Has(susan,influenza)))}\\
\textbar \quad \textbar \textendash \textendash \text{Not(Has(susan,influenza))}

\textcolor{b}{\texttt{Other In-Context Few-shots}}

\textbf{Premises}: \textcolor{r}{\{Premises\}}

\textbf{Conclusions}: \textcolor{g}{\{Conclusions\}}

\textbf{Generated Proofs}: 

\section{Contribution v.s. Tree-of-thoughts (ToT) or Graph-of-thoughts (GoT)}

While ToT and GoT also focused on the challenges of autoregressive LLMs that using CoT is considered as linear reasoning, i.e., the reasoning starts from left to right, and cannot support the property like backtracking.
However, 
ToT and GoT are task specific, which means we have to manually work on heavy prompt engineering designed for each task. In addition, we have to rely on additional breadth-first search (BFS) or depth-first search (DFS) algorithms to programmatically control the entire process of ToT or GoT.

