\documentclass[12pt, letterpaper]{article}
\usepackage[margin=1in]{geometry} % Set 1-inch margins
\usepackage{setspace} % Adjust line spacing
\usepackage{url} % For email formatting
\pagenumbering{gobble} % Remove page numbers


\begin{document}

\title{\textbf{Neurosymbolic AI for Travel Demand Prediction: Integrating Decision Tree Rules into Neural Networks}}

\author{
    Kamal Acharya\textsuperscript{1}, Mehul Lad\textsuperscript{1}, Liang Sun\textsuperscript{2}, Houbing Song\textsuperscript{1} \\ \\
    \textsuperscript{1}Department of Information Systems, \\ University of Maryland Baltimore County\\ MD, USA \\ 
    Email: \url{kamala2@umbc.edu}, \url{du72811@umbc.edu}, \url{songh@umbc.edu} \\ \\
    \textsuperscript{2}Department of Mechanical Engineering\\ Baylor University\\ TX, USA \\
    Email: \url{liang_sun@baylor.edu}
}

\date{} % Remove date

\maketitle

\onehalfspacing % Set 1.5 line spacing for readability

\begin{abstract}
Accurate travel demand prediction is essential for optimizing transportation planning, resource allocation, and infrastructure development. It supports the efficient design of transportation systems, reduces congestion, and enables strategic public investments. However, traditional methods often fail to address the nonlinear and multidimensional relationships inherent in mobility data, and while advanced neural network (NN) approaches improve predictive accuracy, their "black-box" nature limits interpretability. To address these challenges, this research introduces a novel Neurosymbolic Artificial Intelligence (Neurosymbolic AI) framework that combines decision tree (DT)-based symbolic reasoning with the predictive power of NNs. This hybrid approach aims to balance the trade-off between interpretability and predictive accuracy, making it suitable for critical transportation applications.

The primary goal of this work is to develop a transparent and accurate framework for travel demand prediction by integrating symbolic reasoning with neural learning. The proposed methodology involves three key tasks: (1) data integration and preprocessing, (2) rule extraction using DTs, and (3) NN training enhanced by the inclusion of DT-extracted rules as additional features. The study leverages geospatial, economic, and mobility datasets to capture the complex determinants of travel demand between the counties in Tennessee state. Features such as land use, points of interest (POI), transportation infrastructure, and economic variables are carefully selected and processed to ensure relevance and quality. Multicollinearity checks are performed to eliminate redundancy, and the resulting dataset serves as the foundation for DT-based rule extraction. In the second task, DTs are constructed at varying depths to balance model interpretability and complexity. The DTs extract interpretable if-then rules that encapsulate key decision-making patterns. For instance, conditions involving travel distance and the number of POIs in origin and destination regions reveal critical travel dynamics. These rules are further refined using variance thresholds (e.g., 0.01, 0.001, 0.0001) to retain only the most significant and informative patterns. These filtered rules are encoded as binary features indicating whether specific conditions are met and integrated into the NN. The third task involves training the NN using the combined dataset comprising the original features and the DT-derived rules. This integration enhances the NN’s ability to model nonlinear relationships while maintaining an interpretable structure through the explicit incorporation of decision rules. The experimental evaluation assesses model performance across three key metrics: Mean Absolute Error (MAE), R-squared (R²), and the Common Part of Commuters (CPC). These metrics provide a comprehensive measure of predictive accuracy, fit, and alignment with observed mobility patterns.

The results, based on extensive experiments, demonstrate the efficacy of the proposed framework. The hybrid model consistently outperforms traditional standalone models across all evaluation metrics. For instance, the combined dataset achieves significantly lower MAE, indicating higher predictive accuracy. The R² values highlight the framework’s superior ability to capture variance in the data, while the CPC metric underscores its alignment with real-world commuter patterns. Notably, rules selected at finer variance thresholds (e.g., 0.0001) exhibit the most substantial impact, effectively capturing nuanced relationships that enhance the NN's performance. By merging symbolic reasoning with neural learning, the framework achieves both interpretability and predictive robustness.

The potential benefits of this work are substantial. First, the framework offers a transparent approach to travel demand prediction, providing clear insights into the decision-making process through interpretable rules. This transparency is critical for stakeholders in transportation planning, where explainability is often as important as accuracy. Second, the enhanced predictive performance supports more efficient resource allocation, infrastructure planning, and congestion mitigation. For example, urban planners can use the model to optimize transit routes, assess the impact of new developments, and predict travel demand shifts due to policy changes. Third, the approach is scalable and adaptable, with the potential to incorporate real-time data for dynamic demand forecasting, further expanding its applicability.

This research presents a pioneering application of Neurosymbolic AI to travel demand modeling. By integrating symbolic reasoning and neural learning, the proposed framework addresses the limitations of traditional and black-box approaches, offering a solution that is both interpretable and accurate.
\end{abstract}

\vspace{1em}
\noindent\textbf{Keywords:} Decision Tree, Neural Network, Neurosymbolic AI, Travel Demand Prediction

\end{document}
