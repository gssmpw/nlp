\section{Related Work}
\label{sec:related_work}

Traditional methods for predicting travel demand have relied heavily on statistical models, such as gravity models\cite{jung2008gravity} and logistic regression techniques\cite{wei2015logistic}. Regression analysis has also been utilized to predict the demand for Regional Air Mobility (RAM)\cite{acharya2025regional} and Advanced Air Mobility (AAM)\cite{10825121}. These models often assume linear relationships and may not effectively capture the complex, nonlinear interactions present in transportation systems.

With the advent of machine learning, more sophisticated models have been employed to improve prediction accuracy. NNs have been widely used due to their ability to model nonlinear relationships and learn from large datasets \cite{guo2020residual}. In the context of travel demand prediction, NNs have demonstrated superior performance over traditional statistical methods by capturing intricate patterns in mobility data \cite{rajendran2021predicting}. They have also been employed as a supportive tool to enhance genetic algorithms, leading to improved outcomes\cite{acharya2024improving}. However, a significant limitation of NNs is their lack of interpretability. The "black-box" nature of these models makes it challenging for practitioners to understand the underlying decision-making processes, which is crucial in transportation planning where transparency and explainability are essential\cite{golshani2018modeling}. To address this black-box puzzle, two distinct branches of AI come into play: XAI\cite{arrieta2020explainable} and Neurosymbolic AI\cite{garcez2023neurosymbolic}. XAI focuses on providing explanations for AI models, typically after the training process. In contrast, Neurosymbolic AI integrates neural learning with symbolic reasoning directly within the model's architecture, creating a more inherently interpretable framework.


Various recent research has explored the application of XAI in various aspects of trip demand prediction, from forecasting overall travel demand to understanding individual mode choices. Hu et al.\cite{hu2023examining} tackled the challenge of predicting population inflow using mobile device location data, comparing various tree-based models and interpretation techniques. Their study revealed that boosting trees outperformed other models and that while feature importance rankings were consistent across models, the choice of importance measures and hyperparameter settings did influence the results. Moving from a nationwide perspective to a city-wide focus, Kim et al. \cite{kim2020stepwise} investigated the dynamic relationship between taxi and ride-hailing services in New York City. Their two-stage modeling approach combined linear regression with a long short-term memory network to predict taxi demand, effectively capturing the influence of ride-hailing services, day of the week, weather conditions, and holidays. This study demonstrated the potential of interpretable deep learning models for developing active demand management strategies, such as quota control systems, to balance demand between different modes and mitigate congestion. Kim\cite{kim2021analysis} examined travel mode choice in Seoul, employing extreme gradient boosting (XGB) \cite{chen2015xgboost} and interpretation techniques like variable importance, interaction analysis, and accumulated local effects (ALE) \cite{danesh2022interpretability} plots. The study not only demonstrated the accuracy of XGB but also revealed the importance of trip- and tour-related variables, age, and number of trips on tour in influencing mode choice decisions. Another study\cite{kashifi2022predicting}, achieves explainable or interpretable AI by employing two key techniques: variable importance analysis and SHapley Additive exPlanations (SHAP) \cite{lundberg2017unified} analysis. Variable importance analysis identifies the input variables, such as trip distance, traveler’s age, and weather conditions, that have the most significant influence on the model's predictions, revealing the strongest predictors of travel mode choice. SHAP analysis then quantifies the impact of each feature, like annual income and car/bicycle ownership, on the prediction for individual instances, providing a deeper understanding of how these factors work together to shape travel mode decisions.

% The integration of symbolic reasoning with neural learning, known as Neurosymbolic AI, has emerged as a promising approach to combine the strengths of both paradigms \cite{garcez2023neurosymbolic}. Neurosymbolic AI aims to enhance the interpretability of neural networks while retaining their powerful predictive capabilities. Previous studies have explored various methods of integrating symbolic knowledge into neural architectures. 

% These studies showcase the diverse applications and growing impact of XAI in trip demand prediction. By shedding light on the decision-making process of machine learning models, XAI not only enhances the accuracy of predictions but also fosters trust and facilitates the translation of these insights into actionable strategies. The ability to understand the factors driving travel demand empowers policymakers, transportation planners, and service providers to make informed decisions that lead to more efficient, sustainable, and user-centric urban transportation systems.








% % ###################################################################################

In the transportation domain, no studies have been found that focused on Neurosymbolic approaches for travel demand prediction. Despite the potential advantages, there is a gap in the literature regarding the application of Neurosymbolic AI to travel demand prediction. Existing studies have only investigated post training accuracy and interpretability of demand prediction models. Our work addresses this gap by proposing a Neurosymbolic framework that leverages DT-extracted rules alongside NNs to predict travel demand between counties.