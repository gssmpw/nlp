\section{Related Works}
% \subsection{Other DNN Accelerators on FPGAs}

% FlexCNN \cite{basalama2023flexcnn} is another framework specific for CNN model accelerator generation on FPGA. FET-OPU \cite{bai2023fet} is an FPGA-based overlay for transformer models, which employs an instruction control unit to dispatch executions of a systolic array and a special function unit. The design is only synthesized. DFX \cite{hong2022dfx} is another overlay design for single and multi-FPGA LLM applications. FlightLLM \cite{zeng2024flightllm} is a vector processor implemented on U280 for Llama model inference. DNNExplorer \cite{zhang2020dnnexplorer} is a hybrid accelerator that applies dataflow designs for the first few layers and executes the rest of the layers on a generic overlay. CHARM \cite{zhuang2023charm} is a framework to generate heterogeneous accelerator on Versal ACAP platform.

% \subsection{Reconfigurable Accelerators}

% Fully pipelined composable architecture (FPCA) \cite{cong2014fully} allows deploying and reconfiguring for multiple DNN tasks and pipelines the computations to increase resource utilization. Overgen \cite{liu2022overgen} is an overlay generator on FPGA for domain-specific applications. The compute engine is implemented in CGRA. Due to design complexity for generalizability, these accelerators have low clock frequency and low single-task resource efficiency. \cite{zhao2023token} deploys a reconfigurable systolic array to pipeline part of the attention layer. 

% SET \cite{cai2023inter} schedules on tiled accelerators using a time-space resource allocation tree. It assumes a very flexible NoC for communication and cannot optimally allocate tasks to every tile at every cycle. FEATHER \cite{tong2024FEATHER} focuses on intra-task reconfiguration for reduction operations in CNN, which is a local optimization.