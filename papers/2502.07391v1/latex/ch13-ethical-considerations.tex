\section{Ethical Considerations} \label{sec:ethical}
In a task such as this, which revolves around explaining the implicit meaning behind an inherently incongruent form of communication, we must be mindful of certain ethical implications. 

It is important to prevent misinterpretations stemming from inaccurate explanations. Considering that sarcasm is usually used to express one's views in an insulting or mocking way, such misinterpretations can lead to excessive harm by highlighting the mockery used in the utterance more than its actual implicit meaning (which can lead to a feeling of alienation or ridicule among the people being mocked) or even in the form of a complete mischaracterization of what the author meant to express. 

Additionally, we must be cognizant of the fact that sarcasm can be interpreted in different ways in the presence of different cultural contexts. For instance, a simple gesture such as a ``thumbs up'' might signify a job well done in one culture. However, it is completely possible that in another culture, the same gesture is seen as one of disrespect. As a result, any sarcastic utterances involving the use of such a gesture will mean different things across the two cultures. Thus, in order to prevent misinterpretations, it is important to take this cultural aspect of sarcasm into account as well.

\section*{Acknowledgement}
The authors acknowledge the partial support of Infosys Foundation through Center of AI (CAI) at IIIT Delhi.