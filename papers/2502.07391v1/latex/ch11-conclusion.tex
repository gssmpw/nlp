\section{Conclusion} \label{sec:con}
In this research, we proposed a novel target-augmented shared fusion-based multimodal sarcasm explanation model, \model. Along with solving the key limitations that we were able to identify in the current state-of-the-art, TEAM, we introduced a novel method of fusing unimodal information into an effective multimodal feature representation. We also manually augmented the MORE dataset by annotating the target of sarcasm. Thorough comparative analyses allowed us to demonstrate the effectiveness of \model\ when compared to existing baselines, including various state-of-the-art MLLMs. Additionally, in order to comprehensively evaluate the generations of the MLLMs and those of \model, we conducted a human evaluation. Even though \model\ did not surpass the scores of two out of the three MLLMs in this evaluation, its results were quite comparable to theirs which is remarkable considering its much smaller size and lesser complexity.  
% comparative analyses of \model\ with existing baselines, including multiple state-of-the-art MLLMs. With these experiments, we were able to prove the effectiveness of \model\ at this task, even when compared to     
% We prove that both of these enhance the model's capability to understand the semantic incongruity in a sarcastic post. 
% Through the experiments we conduct on the MORE dataset, we demonstrate that our model performs better than all existing systems for this task. Additionally, ablation results clearly show the efficacy and necessity of each component of our model.