% =============================================================
\section{Dataset}\label{sec:dataset}
% =============================================================
We use the \dataset\ dataset, an extension of the MORE dataset \citep{Desai_Chakraborty_Akhtar_2022}, for conducting our experiments. The samples of this dataset are sarcastic posts from various online social media platforms such as Twitter, Instagram, and Tumblr. Each sample $i$ consists of an image ($V_i$) and a corresponding text caption ($C_i$). The sarcasm explanation for each sample has been manually annotated. 

We extend the MORE dataset by manually annotating the entity being targeted by sarcasm in each sample as ``target of sarcasm''. To ensure the quality of annotations, the following definitions and guidelines are followed.
% The target of sarcasm is formally defined as any entity, phenomenon, concept or fact that is being ridiculed with the help of sarcasm. We follow strict guidelines for the annotation of the same. These are as follows.
\begin{itemize}[leftmargin=1em, noitemsep, nolistsep]
    \item The target of sarcasm is a short phrase that denotes an \textit{entity, phenomenon, concept, or fact being ridiculed with the use of sarcasm}.
    \item This phrase must not reveal the underlying sarcastic incongruity under any circumstances. 
    \item If two phrases describe the same target, the shorter and more straightforward phrase is preferable. 
    \item Any entity not being directly targeted by sarcasm must not be included. 
\end{itemize}
% We introduce additional information in the dataset in the form of a short, manually annotated phrase that we call `cause of sarcasm'. We define the cause of sarcasm as the entity being targeted by a sarcastic utterance. The intuition behind calling it the `cause' of sarcasm is that the presence of this entity is the primary reason (or `cause') behind someone authoring a sarcastic post in the first place.
A detailed statistical analysis of the \dataset\ dataset is given in Table \ref{table:datset-stats}.