% =============================================================

\subsection{Ablation Study}
\label{sec:ablation}
% =============================================================
% \begin{table*}
  [t]
  \centering
  \resizebox{\textwidth}{!}{%
  \begin{tabular}{cccccccccccc}
    \toprule \multicolumn{2}{c}{Components}                                                             & \multicolumn{5}{c}{Re-executability Rate (\%)} & \multicolumn{5}{c}{Readability (\#)} \\
    \cmidrule(lr){1-2} \cmidrule(lr){3-7} \cmidrule(lr){8-12}        \hspace{8pt}\labelemoji\hspace{8pt}                                                                & \hspace{8pt}\toolemoji\hspace{8pt}                                      & O0                                 & O1             & O2             & O3             & AVG            & O0             & O1             & O2             & O3             & AVG            \\
    \hline
    \rowcolor[rgb]{0.93,0.93,0.93}\multicolumn{12}{c}{\textbf{Initialize with LLM4Decompile-End-6.7B~\citep{llm4decompile}}}   \\
    \xmark                                                                                              & \xmark                                    & 69.51                              & 46.95          & 50.61          & 46.34          & 53.35          & 3.98 & 3.41 & 3.44 & 3.38 & 3.55 \\
    \cmark                                                                                              & \xmark                                    & 75.61                              & 50.61          & 50.00          & 50.00          & 56.55          & 4.01 & 3.44 & 3.39 & \textbf{3.49} & 3.58 \\
    \xmark                                                                                              & \cmark                                    & 83.54                     & \textbf{56.10}          & 51.22          & 50.61 & 60.37 & 4.05 & 3.51 & 3.51 & 3.42 & 3.62 \\
    \cmark                                                                                              & \cmark                                    & \textbf{85.37}                            & \textbf{56.10}                     & \textbf{51.83} & \textbf{52.43}          & \textbf{61.43} & \textbf{4.13} & \textbf{3.60} & \textbf{3.54} & \textbf{3.49} & \textbf{3.69} \\

    \rowcolor[rgb]{0.93,0.93,0.93}\multicolumn{12}{c}{\textbf{Initialize with Deepseek-Coder-6.7B-base~\citep{deepseekcoder}}} \\
    \xmark                                                                                              & \xmark                                    & 59.15                              & 35.98          & 39.02          & 37.80          & 42.99          & 3.71 & 3.05 & 3.16 & 3.05 & 3.24 \\
    \cmark                                                                                              & \xmark                                    & 66.46                              & 41.46          & 38.41          & 36.59          & 45.73          & 3.76 & 3.17 & \textbf{3.21} & 3.08 & 3.31 \\
    \xmark                                                                                              & \cmark                                    & 70.73                              & 39.63          & 39.02          & 40.24          & 47.41          & 3.90 & 3.17 & 3.08 & 3.11 & 3.31 \\
    \cmark                                                                                              & \cmark                                    & \textbf{79.88}                     & \textbf{45.73} & \textbf{43.90} & \textbf{42.68} & \textbf{53.05} & \textbf{3.96} & \textbf{3.21} & 3.18 & \textbf{3.19} & \textbf{3.38} \\
    \bottomrule
  \end{tabular}%
  }
  \caption{The ablation study of different methods across four optimization levels
  (O0, O1, O2, O3), as well as their average scores (AVG). The results in bold represent the optimal performance. The ~\labelemoji~ and ~\toolemoji~ means Relabedling and Function Call. \textbf{Bold} denotes the best performance.}
  \label{tab:ablation}
\end{table*}
% Results of an ablation study on the various components of \model\ are also reported in Table \ref{comparative-analysis}.

We conduct an ablation study on various components of \model\ as \model\ $-$ X, where X corresponds to the Shared Fusion module (\textbf{SF}), the Knowledge Graph (\textbf{KG}), or the Target of Sarcasm (\textbf{TS}). Note that for each variant that does not contain the target of sarcasm, we simply avoid concatenating it with the other text tokens (\S \ref{sec:incorporation-of-target}). The results of the ablation study are given in Table \ref{comparative-analysis}. From the results of the ablation study, we make the following observations.

First, the removal of any component of the model leads to a decrease in its generative performance. This proves that each component of the model contributes significantly to the model's performance. Additionally, we observe that ``\model\ $-$ KG'' performs better than ``\model\ $-$ KG $-$ TS'' and that ``\model\ $-$ SF'' exceeds the performance of ``\model\ $-$ SF $-$ TS''. This demonstrates that the target of sarcasm is useful in facilitating the understanding of the underlying sarcastic incongruity, thus proving our hypothesis as given in Section \ref{sec:intro}.

Second, we compare \model\ with its ``\model\ + TS Concepts'' variant. As opposed to \model, this variant also utilizes external knowledge concepts for the target of sarcasm and helps us observe the effect of the same on the model's performance. For this variant, Equation \ref{eq:target-incorporation} is revised as follows to incorporate the knowledge concepts extracted for the target of sarcasm.
\begin{equation}
\label{eq:TS-concepts-incorporation}
T_{concat}^{(i)} = T_{knowledge}^{(i)} + \verb|</s>| + TS_i + TSC_i
\end{equation}
where $TSC_i$ refers to the external knowledge concepts for the target of sarcasm of the $i^{th}$ sample. Note that while the \model\ + TS Concepts variant was unable to perform as well as \model, it was still able to outperform TEAM on almost all metrics, which further proves the effectiveness of \model's architecture.
