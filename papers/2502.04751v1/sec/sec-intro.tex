\section{Introduction}

In real-world web search systems, addressing an information seeking task often requires retrieving and organizing information from diverse online sources. This task that we term \emph{intricate information seeking} presents a persistent and significant challenge in the field of information retrieval~\cite{strohman2005optimization, talmor2018web}. Unlike conventional single-step search, where a user query typically seeks isolated information, intricate information seeking involves integrating multiple pieces of information across various sources to formulate a comprehensive and accurate final response.
The complexity of this process is further amplified by the necessity to maintain consistency across retrieval steps, especially when user queries encompass multifaceted tasks or require extensive background knowledge. For example, responding to a complex query such as ``\emph{the economic, environmental, and social impacts of the adoption of renewable energy in developing countries}'' entails sourcing multiple relevant documents and synthesizing them into a comprehensive answer, including analyzing economic benefits and costs, assessing environmental sustainability, evaluating social implications.

Typically, driven by the ongoing evolution of large language models~(LLMs)~\cite{zhao2023survey}, a variety of existing methods aim to facilitate multi-step or complex retrieval either by heuristically decomposing the query or by iteratively refining the query through incremental optimization of intermediate outputs~\cite{yao2023react, asaiself}. For instance, some studies adopt planning strategies by decomposing a user query into sub-queries based on surface-level cues with the LLM's internal knowledge~\cite{xu2024search, reddy2024infogent}, while others employ tailored mechanisms~(\eg chain-of-thought reasoning~\cite{wei2022chain} and continuous feedback loops~\cite{shinn2024reflexion}) to progressively align intermediate reasoning steps with the final information seeking goal. Although these methods have shown promise in improving multi-step retrieval quality, they are susceptible to cascading errors, where inaccuracies or omissions in earlier sub-queries can propagate through subsequent steps. 

Inspired by the effectiveness of Monte Carlo Tree Search (MCTS) applied in complex reasoning tasks such as mathematic and code problems~\cite{alphago, YeLKAG21}, we consider incorporating MCTS into intricate information seeking scenarios to help find the optimal retrieval solution. However, two primary challenges emerge when applying MCTS to the task. First, sub-queries generated for expanding the search tree are unbounded at each step: each multifaceted intermediate task can branch into numerous investigative directions, causing the tree search space to grow exponentially. Second, the inherent characteristics of MCTS lead to a focus on local exploration~\cite{browne2012survey}, which may lead to omissions or solecism in the acquired information~\cite{_wiechowski_2022}. Specifically, (1) MCTS node selection relies on local statistics (\eg number of visits and reward accumulation), which are aggregated from limited exploration and lack a holistic understanding of the global information seeking objectives; and (2) MCTS explores the search tree by expanding nodes around the currently selected branch, while its rollout strategies are typically random or heuristic-based, making global optimality difficult to guarantee. Consequently, MCTS risks overlooking pertinent sub-tasks or prematurely converging on suboptimal search paths.


\begin{figure}
    \centering
    \includegraphics[width=0.95\linewidth]{pic/intro.pdf}
    \caption{Illustration of the pitfalls in handling intricate queries. Typical reasoning methods with web search often collect non-comprehensive documents (left), while HG-MCTS can effectively capture all necessary documents (right).}
    \label{fig:intro}
\end{figure}


To address these challenges, we propose a novel framework that incorporates MCTS into intricate information seeking, while simultaneously mitigating its inherent limitations through global guidance and multi-perspective feedback. 
Concretely, we reformulate the task as a progressive information collection process with a knowledge memory. Based on this, we propose \emph{holistically guided MCTS~(HG-MCTS)} that introduces an \emph{adaptive checklist} as a global guidance with a set of designated sub-goals.
This adaptive checklist counters the exponential growth of sub-queries by focusing the MCTS algorithm on only those branches aligned with key facets of the information need, thereby alleviating aimless expansions that could arise and enforcing a more targeted traversal of the search space, which can also be updated during the MCTS process. In parallel, we incorporate \emph{multi-perspective reward modeling} that provides both quantitative and qualitative reward signals with the checklist, allowing MCTS to incorporate a more holistic perspective on exploration. This reward modeling furnishes not just numerical indicators of exploration and retrieval quality but also textual feedback outlining which sub-goals have been addressed and which remain unsolved after node exploration. As a result, MCTS moves beyond its conventional reliance on local statistics, thereby minimizing the risk of prematurely converging on suboptimal paths and broadening its understanding of overarching information seeking objectives. Figure~\ref{fig:intro} illustrates a comparison between our method and the typical information seeking method from the retrieval comprehension perspective. 
Through this synergy, our approach preserves the inherent capability of MCTS for dynamic exploration while strengthening its capacity to incorporate newly acquired knowledge snippets. Our method methodically balances thoroughness and focus, ensuring comprehensive coverage of all relevant information while avoiding redundant or tangential searches. 


Our main contributions are summarized as follows:
\begin{itemize}
    \item We introduce a new information seeking paradigm \emph{HG-MCTS} based on a progressive information collection process with knowledge memory, which integrates an adaptive checklist for holistic sub-goal guidance in MCTS progress, thereby enabling more targeted exploration in multi-step retrieval.
    \item We propose a \emph{multi-perspective reward modeling} strategy that provides both quantitative metrics and qualitative feedback in HG-MCTS, which fosters a richer, step-wise evaluation for the value of new expanded nodes.
    \item We demonstrate how these innovations can be seamlessly integrated to improve both the efficiency and the thoroughness of large-scale web retrieval. Beyond immediate applications in question answering and knowledge-intensive search, our findings offer deeper insights into the design of more interpretable, flexible, and resilient retrieval systems.
\end{itemize}
