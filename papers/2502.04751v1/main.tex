%% This is file `sample-sigconf.tex',
%% generated with the docstrip utility.
%%
%% The original source files were:
%%
%% samples.dtx  (with options: `all,proceedings,bibtex,sigconf')
%% 
%% IMPORTANT NOTICE:
%% 
%% For the copyright see the source file.
%% 
%% Any modified versions of this file must be renamed
%% with new filenames distinct from sample-sigconf.tex.
%% 
%% For distribution of the original source see the terms
%% for copying and modification in the file samples.dtx.
%% 
%% This generated file may be distributed as long as the
%% original source files, as listed above, are part of the
%% same distribution. (The sources need not necessarily be
%% in the same archive or directory.)
%%
%%
%% Commands for TeXCount
%TC:macro \cite [option:text,text]
%TC:macro \citep [option:text,text]
%TC:macro \citet [option:text,text]
%TC:envir table 0 1
%TC:envir table* 0 1
%TC:envir tabular [ignore] word
%TC:envir displaymath 0 word
%TC:envir math 0 word
%TC:envir comment 0 0
%%
%%
%% The first command in your LaTeX source must be the \documentclass
%% command.
%%
%% For submission and review of your manuscript please change the
%% command to \documentclass[manuscript, screen, review]{acmart}.
%%
%% When submitting camera ready or to TAPS, please change the command
%% to \documentclass[sigconf]{acmart} or whichever template is required
%% for your publication.
%%
%%
\documentclass[sigconf]{acmart}

\usepackage{amsmath}
% \usepackage{amssymb}
\usepackage{bm}
\usepackage{algorithm}
\usepackage{algpseudocode}
\usepackage{amsmath}
\usepackage{multirow}
\usepackage{indentfirst}
\usepackage{subfigure}
\usepackage{epsfig}
\usepackage{epstopdf}
\usepackage{graphicx}
\usepackage{url}
\usepackage{xspace}
\usepackage{booktabs}
\usepackage{subeqnarray}
\usepackage{subcaption}
\usepackage{color}
\newtheorem{myDef}{Definition}
\newtheorem{exmp}{Example}

\newcommand{\paratitle}[1]{\vspace{1.5ex}\noindent\textbf{#1}}
\newcommand{\ie}{\emph{i.e.,}\xspace}
\newcommand{\aka}{\emph{a.k.a.,}\xspace}
\newcommand{\eg}{\emph{e.g.,}\xspace}
\newcommand{\etal}{\emph{et al.}\xspace}
\newcommand{\ignore}[1]{}

%%
%% \BibTeX command to typeset BibTeX logo in the docs
\AtBeginDocument{%
  \providecommand\BibTeX{{%
    Bib\TeX}}}

%% Rights management information.  This information is sent to you
%% when you complete the rights form.  These commands have SAMPLE
%% values in them; it is your responsibility as an author to replace
%% the commands and values with those provided to you when you
%% complete the rights form.
% \setcopyright{acmlicensed}
% \copyrightyear{2018}
% \acmYear{2018}
% \acmDOI{XXXXXXX.XXXXXXX}

%% These commands are for a PROCEEDINGS abstract or paper.
% \acmConference[Conference acronym 'XX]{}{June 03--05,
  % 2018}{Woodstock, NY}
%%
%%  Uncomment \acmBooktitle if the title of the proceedings is different
%%  from ``Proceedings of ...''!
%%
%%\acmBooktitle{Woodstock '18: ACM Symposium on Neural Gaze Detection,
%%  June 03--05, 2018, Woodstock, NY}
% \acmISBN{978-1-4503-XXXX-X/18/06}


%%
%% Submission ID.
%% Use this when submitting an article to a sponsored event. You'll
%% receive a unique submission ID from the organizers
%% of the event, and this ID should be used as the parameter to this command.
%%\acmSubmissionID{123-A56-BU3}

%%
%% For managing citations, it is recommended to use bibliography
%% files in BibTeX format.
%%
%% You can then either use BibTeX with the ACM-Reference-Format style,
%% or BibLaTeX with the acmnumeric or acmauthoryear sytles, that include
%% support for advanced citation of software artefact from the
%% biblatex-software package, also separately available on CTAN.
%%
%% Look at the sample-*-biblatex.tex files for templates showcasing
%% the biblatex styles.
%%

%%
%% The majority of ACM publications use numbered citations and
%% references.  The command \citestyle{authoryear} switches to the
%% "author year" style.
%%
%% If you are preparing content for an event
%% sponsored by ACM SIGGRAPH, you must use the "author year" style of
%% citations and references.
%% Uncommenting
%% the next command will enable that style.
%%\citestyle{acmauthoryear}


%%
%% end of the preamble, start of the body of the document source.
\begin{document}

%%
%% The "title" command has an optional parameter,
%% allowing the author to define a "short title" to be used in page headers.
\title{Holistically Guided Monte Carlo Tree Search for Intricate Information Seeking}


%%
%% The "author" command and its associated commands are used to define
%% the authors and their affiliations.
%% Of note is the shared affiliation of the first two authors, and the
%% "authornote" and "authornotemark" commands
%% used to denote shared contribution to the research.
% \author{Ben Trovato}
% \authornote{Both authors contributed equally to this research.}
% \email{trovato@corporation.com}
% \orcid{1234-5678-9012}
% \author{G.K.M. Tobin}
% \authornotemark[1]
% \email{webmaster@marysville-ohio.com}
% \affiliation{%
%   \institution{Institute for Clarity in Documentation}
%   \city{Dublin}
%   \state{Ohio}
%   \country{USA}
% }

\author{Ruiyang Ren$^*$}\thanks{$^*$Equal Contributions.}
\email{reyon_ren@outlook.com}
\affiliation{%
  \institution{Gaoling School of Artificial Intelligence, Renmin University of China}
  \city{Beijing}
  \country{China}}

\author{Yuhao Wang$^*$}
\email{yh.wang500@outlook.com}
\affiliation{%
  \institution{Gaoling School of Artificial Intelligence, Renmin University of China}
  \city{Beijing}
  \country{China}}

\author{Junyi Li$^*$}
\email{junyi_cs@nus.edu.sg}
\affiliation{%
  \institution{National University of Singapore}
  \country{Singapore}}

\author{Jinhao Jiang}
\email{jiangjinhao@ruc.edu.cn}
\affiliation{%
  \institution{Gaoling School of Artificial Intelligence, Renmin University of China}
  \city{Beijing}
  \country{China}}

\author{Wayne Xin Zhao$^\dag$}\thanks{$^\dag$Corresponding Authors.}
\email{batmanfly@gmail.com}
\affiliation{%
  \institution{Gaoling School of Artificial Intelligence, Renmin University of China}
  \city{Beijing}
  \country{China}}

\author{Wenjie Wang$^\dag$}
\email{wenjiewang96@gmail.com}
\affiliation{%
  \institution{University of Science and Technology of China}
  \city{Hefei}
  \country{China}}


\author{Tat-Seng Chua}
\email{dcscts@nus.edu.sg}
\affiliation{%
  \institution{National University of Singapore}
  % \city{Beijing}
  \country{Singapore}}

\renewcommand{\shortauthors}{Ren et al.}

%%
%% The abstract is a short summary of the work to be presented in the
%% article.
\begin{abstract}

In the era of vast digital information, the sheer volume and heterogeneity of available information present significant challenges for intricate information seeking. Users frequently face multi-step web search tasks that involve navigating vast and varied data sources. This complexity demands every step remains comprehensive, accurate, and relevant. However, traditional search methods often struggle to balance the need for localized precision with the broader context required for holistic understanding, leaving critical facets of intricate queries underexplored.
In this paper, we introduce an LLM-based search assistant that adopts a new information seeking paradigm with holistically guided Monte Carlo tree search (HG-MCTS). We reformulate the task as a progressive information collection process with a knowledge memory and unite an adaptive checklist with multi-perspective reward modeling in MCTS. 
The adaptive checklist provides explicit sub-goals to guide the MCTS process toward comprehensive coverage of complex user queries. Simultaneously, our multi-perspective reward modeling offers both exploration and retrieval rewards, along with progress feedback that tracks completed and remaining sub-goals, refining the checklist as the tree search progresses. By striking a balance between localized tree expansion and global guidance, HG-MCTS reduces redundancy in search paths and ensures that all crucial aspects of an intricate query are properly addressed.
Extensive experiments on real-world intricate information seeking tasks demonstrate that HG-MCTS acquires thorough knowledge collections and delivers more accurate final responses compared with existing baselines.

\end{abstract}

%%
%% The code below is generated by the tool at http://dl.acm.org/ccs.cfm.
%% Please copy and paste the code instead of the example below.
%%
% \begin{CCSXML}
% <ccs2012>
%  <concept>
%   <concept_id>00000000.0000000.0000000</concept_id>
%   <concept_desc>Do Not Use This Code, Generate the Correct Terms for Your Paper</concept_desc>
%   <concept_significance>500</concept_significance>
%  </concept>
%  <concept>
%   <concept_id>00000000.00000000.00000000</concept_id>
%   <concept_desc>Do Not Use This Code, Generate the Correct Terms for Your Paper</concept_desc>
%   <concept_significance>300</concept_significance>
%  </concept>
%  <concept>
%   <concept_id>00000000.00000000.00000000</concept_id>
%   <concept_desc>Do Not Use This Code, Generate the Correct Terms for Your Paper</concept_desc>
%   <concept_significance>100</concept_significance>
%  </concept>
%  <concept>
%   <concept_id>00000000.00000000.00000000</concept_id>
%   <concept_desc>Do Not Use This Code, Generate the Correct Terms for Your Paper</concept_desc>
%   <concept_significance>100</concept_significance>
%  </concept>
% </ccs2012>
% \end{CCSXML}

% \ccsdesc[500]{Do Not Use This Code~Generate the Correct Terms for Your Paper}
% \ccsdesc[300]{Do Not Use This Code~Generate the Correct Terms for Your Paper}
% \ccsdesc{Do Not Use This Code~Generate the Correct Terms for Your Paper}
% \ccsdesc[100]{Do Not Use This Code~Generate the Correct Terms for Your Paper}

%%
%% Keywords. The author(s) should pick words that accurately describe
%% the work being presented. Separate the keywords with commas.
% \keywords{Knowledge Boundary; Retrieval Augmentation; Large Language Models}
%% A "teaser" image appears between the author and affiliation
%% information and the body of the document, and typically spans the
%% page.


% \received{20 February 2007}
% \received[revised]{12 March 2009}
% \received[accepted]{5 June 2009}

%%
%% This command processes the author and affiliation and title
%% information and builds the first part of the formatted document.
\maketitle


%=========================================================================
% SparseZipper: Introduction
%=========================================================================

\section{Introduction}
\label{sec-spz-intro}

% *** Hardware trend: matrix extensions on CPUs ***
%   + Matrix extensions for CPUs are coming
%   + Why? --> accelerate dense matrix-matrix multiplications
%   + How? --> in the form of large 2D MAC array and work closely with existing vector architectures
%   + Benefits? --> being close to CPUs

General matrix multiply (GEMM) is a key building block in many different
domains including machine learning, graph analytics, and scientific
computing. Therefore, numerous domain-specific architectures have been
proposed to accelerate dense-dense GEMM (i.e., most values in both input
matrices are non-zeros) with various trade-offs in programmability,
performance, and energy
efficiency~\cite{jouppi-datacenter-isca2017,teich-google-tpu-v2-blog2018,chen-eyeriss-v2-jetcas2019,jouppi-google-tpu-v2-v3-cacm2020,choquette-tensor-core-nvidia-ieeemicro2021}.
In addition to coarse-grain accelerators, CPU vendors have recently
introduced matrix extensions (e.g., Intel's Advanced Matrix Extension
(AMX)~\cite{intel-amx-web,nassif-intel-sapphire-isscc2022,jeong-rasa-dac2021},
Arm's Scalable Matrix Extension (SME)~\cite{arm-sme-web}, RISC-V's matrix
extension proposal~\cite{riscv-mtx-ext-proposal-web}, and IBM's
Matrix-Multiply Assist (MMA)~\cite{ibm-mmx-assist-web}) to their ISAs for
dense-dense GEMM acceleration. Such matrix extensions attempt to strike a
balance between programmability and efficiency, and they are often
implemented using systolic-array-based
micro-architectures~\cite{intel-amx-web,nassif-intel-sapphire-isscc2022}.

% *** Software trend: Sparse computation ***
%   + Data are sparse and in many cases extremely sparse -> motivate the challenge
%   + Unstructured sparsity -> motivate the importance of a general solution
%   + Why dense hardware is no longer enough to do sparse?

However, matrices in workloads are not always dense. In fact, many recent
neural network
models~\cite{reddi-mlperf-isca2020,naumov-dnn-model-arxiv2019,han-deep-compress-arxiv2015,jouppi-datacenter-isca2017,wu-ml-facebook-hpca2019},
real-world graph
analytics~\cite{davis-graphblas-tmos2019,hoefler2011generic,shun-multicore-tc-2015},
and scientific
simulations~\cite{canning-sparse-sim-1996,galli-quantum-sim-1996} operate
on sparse matrices where the majority of values are zeros. In addition,
matrix densities (i.e., the percentage of non-zero values in a matrix)
vary dramatically across domains (e.g., from $10^{-6}\%$ density in
matrices representing social graphs to 50\% density in matrices used in
neural network models~\cite{hegde-extensor-micro2019}). Such low matrix
densities prevent computing GEMM for sparse matrices efficiently on CPUs
using the recently introduced matrix extensions since most
multiplications will involve at least one input value which is zero.
Moreover, sparse matrices are typically stored in compact formats with
metadata indicating positions of non-zero values for space efficiency, so
they are not directly compatible with existing built-in matrix engines
specialized for processing matrices stored in a dense format.

% *** Existing solutions (only closely related work) ***
%   + SparseCore - specialized ISA extension just for sparse computation
%   + VEGETA
%   + SparseTPU? Should we include this? It's not really an ISA extension

In addition to numerous domain-specific sparse-sparse GEMM (SpGEMM)
accelerators~\cite{qin-sigma-hpca2020, hegde-extensor-micro2019,
  zhang-sparch-hpca2020, srivastava-matraptor-micro2020,
  zhang-gamma-asplos2021}, previous work has proposed several ISA
extensions to accelerate sparse computations.
SparseCore~\cite{rao-sparsecore-asplos2022} is a stream-based ISA
extension designed specifically for sparse computations at the cost of
extra hardware for stream registers and stream processing units without
efficiently supporting dense-dense GEMM. VEGETA extends a matrix
extension to accelerate sparse-dense matrix-matrix multiplication (SpMM)
in addition to dense computations~\cite{jeong-vegeta-hpca2023}. However,
VEGETA is limited to SpMM and DNN-specific sparsity structures, so it is
not efficient when multiplying two highly sparse (i.e., less than 1\%
density) matrices with unstructured sparsity structures, which is critical in
various workload domains including graph analytics (e.g., multi-source
breadth-first search, peer pressure clustering, cycle detection, triangle
counting,
etc.)~\cite{dalberto-all-pair-spgemm-2007,shah-graph-spgemm-2007,rabin-maximum-matching-alg-1989,azad-triangle-count-2015},
hybrid linear solvers (e.g., Schur complement method and algebraic multi-grid
methods)~\cite{yamazaki-spgemm-schur-2010}, context-free grammar
parsing~\cite{penn-context-free-grammar-2006}, molecular dynamics
simulatio~\cite{itoh-order-n-spgemm-1995}, and interior point
methods~\cite{karypis-interior-point-alg-1994}.

% *** Our solution ***
% Augment matrix extensions to accelerate SpGEMM

In this work, we propose SparseZipper that minimally extends existing
matrix ISAs and systolic-array-based micro-architecture specialized for
dense-dense GEMM to accelerate SpGEMM operating on highly sparse matrices
with unstructured sparsity structures. SparseZipper targets a conventional
row-wise dataflow SpGEMM algorithm (i.e., Gustavson algorithm) with sparse
matrices represented in commonly used compressed sparse row/column
(CSR/CSC) formats. The abstraction and micro-architecture of SparseZipper
are specialized for accelerating the algorithm's main performance
bottleneck which involves merging multiple sparse vectors represented as
streams of indices (i.e., keys) and data (i.e., values).
By leveraging existing matrix registers for storing key-value streams and a
systolic array for merging multiple streams, SparseZipper incurs minimal area
overhead.
Our performance evaluation shows SparseZipper achieves 5.98$\times$ and
2.61$\times$ speedup over a scalar hash-based implementation of SpGEMM and a
state-of-the-art vectorized SpGEMM version, respectively.
Our component-level area evaluation shows SparseZipper increases the area of a
baseline 16$\times$16 systolic array by only 12.7\%.
This overhead would be much lower when considering an entire processor and its
caches.

%SparseZipper targets the key bottleneck, which is merging partial sparse
%vectors, in a conventional SpGEMM algorithm for data-parallel
%architectures~\cite{li-merge-spmspv-vectorarch-2018,li-spgemm-vector-arch-mchpc2019,fevre-spgemm-rvv-arxiv2023,winter-adaptive-spgemm-gpu-ppopp2019,liu-efficient-spgemm-gpu-ipdps2014,dalton-optimizing-spgemm-gpu-ipdps2015}.
%Each partial sparse vector is considered as a stream of keys (i.e.,
%representing row/column indices of non-zeros in a matrix) and corresponding
%non-zero values.
%At the core of SparseZipper is its ability to efficiently merge such streams in
%parallel by leveraging in-place matrix registers to store parts of concurrent
%streams and built-in systolic array to merge those streams together.
%In order to facilitate that merge operation, we propose a minimal set of
%additional architectural states to keep track of active streams and matrix
%instructions to move streams between matrix registers and memory.
%Our performance evaluations show SparseZipper achieves 5.98$\times$ and
%2.61$\times$ speedup over a scalar hash-based implementation of SpGEMM and a
%vectorized SpGEMM version respectively.
%Our post-synthesis area evaluation shows SparseZipper incurs less than XX\%
%area overhead compared to the baseline matrix engine designed for dense GEMM.

% *** Our contributions ***
\BF{Contributions --} Our key contributions include: (1)~a SparseZipper
ISA extension that enhances an existing matrix ISA to efficiently support
merging multiple key-value streams, the main performance bottleneck in
the conventional row-wise dataflow SpGEMM algorithm; (2)~a minimal set of
micro-architectural changes to a systolic array to support the new
SparseZipper instructions; and (3)~a detailed cycle-level evaluation
demonstrating the performance benefits of SparseZipper and a first-order
area evaluation demonstrating the minimal additional hardware needed for
SparseZipper.

%% Motivate sparse computation
%% + Important in various domains
%% + Example in graph analytics
%
%\paragraph{Motivating sparse matrix computations} Computation on sparse data,
%where a majority of values are zeros, lies at the heart of many important
%application domains such as scientific
%computing~\cite{gilbert-hpc-graphs-2007}, graph
%analytics~\cite{davis-graphblas-tmos2019,hoefler2011generic,shun-multicore-tc-2015},
%machine
%learning~\cite{reddi-mlperf-isca2020,naumov-dnn-model-arxiv2019,han-deep-compress-arxiv2015,jouppi-datacenter-isca2017,wu-ml-facebook-hpca2019},
%and simulation~\cite{canning-sparse-sim-1996,galli-quantum-sim-1996}.
%For instance, in the domain of graph analytics, a graph can be represented as a
%sparse adjacency matrix in which non-zero values in the matrix represent
%connections among vertices.
%Primitive graph operations (e.g., traversal and triangle counting) can be
%expressed in algebraic kernels such as sparse-matrix sparse-vector (SpMSpV) and
%sparse-matrix sparse-matrix (SpMSpM) multiplication~\cite{}.
%Such adjacency matrices are often highly sparse (e.g., only 0.2\% of elements
%in WikiVote network's adjacency matrix are non-zeros).
%
%% Why is it challenging?
%% + Sparse data -> most are zeros -> low arithmetic intensity & memory-bound
%% + Data dependent: matrix sparsity, structure, etc.
%% + Domain dependent
%% + Various data flows
%
%\paragraph{Challenges in sparse matrix computations} The high sparsity of data
%represented in matrix and vector format pose several computational challenges.
%First, using algorithms designed for processing dense data is algorithmically
%inefficient due to wasted multiplications with zero values.
%Therefore, sparse matrices and vectors are often represented in compact formats
%such as compressed sparse row/column (CSR/CSC) so that only non-zero values are
%stored and effectively processed.
%Second, computation on sparse data requires only few arithmetic operations per
%loaded data, which makes it inherently memory-bound.
%Third, the efficiency of sparse computation is largely dependent on input data.
%Previous work has proposed various custom compact matrix formats~\cite{} to
%exploit certain structures of non-zero values and a wide range of data-flows
%(e.g., inner product, outer product, and Gustavsons for SpMSpM) for particular
%levels of matrix sparsity.
%
%% Approach #1: General-purpose compute: GPGPU, vector, and multi-core
%
%\paragraph{General-purpose compute} multi-core CPU~\cite{},
%GPGPU~\cite{merrill-merge-spmv-gpu-sc2016}, vector
%architectures~\cite{nagasaka-intel-knl-icpp2018,li-merge-spmspv-vectorarch-2018}
%
%% Approach #2: Domain-specific accelerator
%
%\paragraph{Domain-specific accelerators} MatRaptor~\cite{}, GraphLily~\cite{},
%OuterSPACE~\cite{}, SpArch~\cite{}, Spada~\cite{}, ExTensor~\cite{},
%GAMMA~\cite{}, SIGMA~\cite{}
%
%% Approach #3: ISA extensions: sparse core
%%   - Middle ground
%%   - Try to argue that this is a right approach but existing work is not good
%%   enough (e.g., SparseCore - substantial architectural changes, VEGETA -
%%   specialized for a certain domain, SAVE - extending vector engine)
%
%\paragraph{ISA extensions} SparseCore~\cite{rao-sparsecore-asplos2022},
%VEGETA~\cite{jeong-vegeta-hpca2023}, SAVE~\cite{gong-save-micro2020}
%
%% Emerging matrix architectures
%%   - Why? and how?
%
%\paragraph{Emerging matrix architectures} Intel
%AMX~\cite{intel-amx-web,nassif-intel-sapphire-isscc2022}, Arm
%SME~\cite{arm-sme-web}, IBM matrix-multiply assist~\cite{ibm-mmx-assist-web},
%and RISC-V matrix extension proposal~\cite{riscv-mtx-ext-proposal-web}
%
%% Our solution
%%
%\paragraph{Our solution}
%
%% Our contributions
%\paragraph{Contributions} Our key contributions include:
%(1) an ISA extension that includes new matrix instructions to efficiently
%support merging streams of keys and values, a key operation in
%merge-based sparse matrix computations,
%(2) an implementation of SpMSpV and SpMSpM using the new matrix instructions
%and existing vector instructions,
%(3) a minimal set of micro-architectural changes to a systolic array-based
%matrix engine to support the new stream merging operation, and
%(4) a detailed cycle-level performance evaluation.

\section{Technical Background}
\label{sec:background}

\subsection{Merge Tree and Contour Tree}
\label{sec:merge-and-contour-tree}
\para{Merge Tree.} 
Let $f:\X \rightarrow \R$ be a continuous scalar field defined on a simply connected domain $\X$. 
The \emph{sublevel set} of $f$ at a threshold $t \in \R$ is defined as 
$\X_t = f^{-1}(-\infty,t] := \{ x \in X \mid f(x) \leq t \}$. 
The \emph{merge tree} of $f$ tracks when (connected) components of $\X_t$ appear and merge as $t$ increases. 
$\X_t$ evolves from being an empty set to contain components surrounding various local minima; these components then merge into one other until eventually there is only a single component.
Leaves of the merge tree correspond to local minima, and interior nodes correspond to saddles where components merge. 
\Cref{fig:contour-tree}(A) and (C) visualize a scalar field and its merge tree.
Formally, we define an equivalence relation $\sim$ on $\X$. We say that $x \sim y$ if and only if $f(x) = f(y)=t$ and $x$ belongs to the same component of $\X_t$ as $y$. The merge tree of $f$ is defined by the quotient space $\X/{\sim}$. 

The merge tree of $f$ defined above is sometimes referred to as the \emph{join tree}, whereas the merge tree of $-f$ is called a \emph{split tree} (see~\cref{fig:contour-tree}(B) for its visualization). The merge tree naturally induces a segmentation of the domain. Let $\phi$ be the canonical map that maps each $x \in \X$ to its equivalence class $[x]$ under $\sim$. Then for each edge $e$ of the contour tree, $\phi^{-1}(e)$ is a monotonic region in $\X$. The inverse image of each edge partitions the domain, which is called the merge-tree-induced segmentation. See \cref{fig:contour-tree}(D) (cf.,~(E)) for a split-tree-induced segmentation.

\para{Contour Tree.} 
The \emph{level set} of $f$ at a threshold $t \in \R$ is $f^{-1}(t):= \{ x \in X \mid f(x) = t \}$.
Each component of $f^{-1}(t)$ is called a \emph{contour}. 
A \emph{contour tree} tracks the relations among contours as $t$ increases. 
Analogous to a merge tree, as $t$ increases, components of $f^{-1}(t)$ appear at local minima, disappear at local maxima, and join or split at saddles. 
Formally, we define another equivalence relation $\sim$ on $\X$, where $x \sim y$ if and only if $f(x) = f(y)$ and $x$ belongs to the same contour as $y$. 
The contour tree is the quotient space $T=T(\X,f):= \X / \sim$. 
There is a 1-1 correspondence between the local extrema of $\X$ and the leaves of $T$. The interior nodes of $T$ correspond to a subset of the saddle points of $\X$ \cite{reeb1946points}. \cref{fig:contour-tree}(F) visualizes the contour tree. Similar to the merge tree, we can define a contour-tree-induced segmentation.

The classic algorithm for computing the contour tree~\cite{carr2003computing} combines join and split trees together to form a contour tree, and comes with a state-of-the-art multi-core implementation~\cite{gueunet2017task}.
In this paper, we compute the contour tree of the input data to determine the initial point-wise error bound. We use the algorithm of Gueunet et al.~\cite{gueunet2017task} (with an in-house implementation) to compute join and split trees, and the algorithm of Carr et al.~\cite{carr2003computing} to combine them into the contour tree. 

\begin{figure}[!ht]
    \centering
    \includegraphics[width=\linewidth]{fig-ct_example.png}
    \vspace{-6mm}
    \caption{(A) Visualizing the graph of a 2D scalar field $f$ defined on a square domain. Local minima are in blue, saddles are in white, and local maxima are in red. (B) Split tree of f.  (C) Merge tree of $f$ (a.k.a.,~join tree). (D) The domain is colored according to split-tree-induced segmentation. (E) Split tree is colored according to the segmentation in (D). (F) Contour tree of $f$. (G) The graph of $f$ after its persistence simplification where a peak is removed. (H) Split tree of $f$ after persistence simplification. Edge $ab$ is removed. (F) Contour tree of $f$ after the merge tree in (B) has been simplified. }
    \label{fig:contour-tree}
    \vspace{-6mm}
\end{figure}

The algorithm by Gueunet et al.~\cite{gueunet2017task} constructs the merge tree one edge at a time. First, starting from each minimum $m$, the algorithm visits points surrounding $m$ in an increasing order until a saddle $s$ is reached. The edge $ms$ is discovered and added to the merge tree; in other words, $m$ is \emph{grown} during this process.  
For each saddle $s$, once all edges that terminate at $s$ have been discovered, $s$ is grown until some new saddle $s'$ is reached, and the edge $ss'$ is discovered and added to the merge tree. This process repeats until the merge tree has been completely discovered. 
To grow from a minimum or a saddle, the points surrounding that minimum or saddle are stored in a heap to ensure that they are processed in an increasing order. The use of heaps presents the main performance bottleneck of the algorithm.

\para{Persistence simplification.} 
In practice, real-world data typically contain noise that creates many small branches in the merge or contour tree, where persistence simplification~\cite{edelsbrunner2001hierarchical} can be used to eliminate these small branches and thereby separate topological features from noise. 
In the context of merge trees, ordinary persistent homology pairs a local extremum (i.e.,~a peak or a valley in \cref{fig:contour-tree}(A)) with a nearby saddle and assigns the pair a value of \emph{persistence}, which describes the scale at which the pair disappears via a perturbation to the function. The persistence is equal to the absolute difference in function value between an extremum and its paired saddle. A function $f$ is simplified by perturbing the function values in order to cancel pairs of critical points below a certain persistence threshold $\varepsilon$. See \cref{fig:contour-tree}(G) for an example. The cancellation of one pair of critical points of $f$ corresponds to a branch being removed from the merge tree, giving rise to its simplification.
The simplified contour tree is more complex \cite{hristov2021w}, but it can be computed by combining the simplified join and split trees. 

In~\cref{fig:contour-tree}, (G) depicts the graph of a function after persistence simplification. For the split tree shown in (B), assuming a persistence threshold of $\varepsilon \geq |f(a)-f(b)|$, the edge $ab$ will be removed after a persistence simplification at $\varepsilon$. As a result, node $c$ is directly connected to node $d$ in the split tree (H), and the join tree in (C) remains unchanged. (I) depicts the contour tree produced by combining the simplified split tree in (H) with the join tree in (C).

\subsection{A Review on TopoSZ}
\label{sec:TopoSZ}

Our framework builds upon a few ingredients from TopoSZ~\cite{yan2023toposz}. 
TopoSZ, in turn, modifies the pipeline from the error-bounded lossy compressor SZ version 1.4 \cite{tao2017significantly}. 

Let $f$ represent the input scalar field, and $f'$ be the reconstructed scalar field (after compression and decompression). 
Let $T$ be the contour tree of $f$ and $T_\varepsilon$ the persistence simplified contour tree at a threshold of $\varepsilon$. Let $T'$ and $T'_\varepsilon$ be defined analogously for $f'$.

SZ1.4 allows the user to specify $\xi$, a pointwise error bound during compression. In turn, there are two user-defined parameters in TopoSZ: a persistence threshold $\varepsilon$, and a pointwise error bound $\xi$. Unlike TopoQZ, TopoSZ does not require that $\varepsilon < \xi$. 
TopoSZ guarantees the preservation of the persistence simplified contour tree during compression while maintaining the pointwise error bound. 
That is, it guarantees that $T_\varepsilon = T'_\varepsilon$, and $|f(x)-f'(x)| \leq \xi$ for each $x \in \X$.

\para{Linear-scaling quantization.} 
SZ version 1.4 introduces a linear-scaling quantization technique to ensure that a strict absolute error bound $\xi$ is maintained. This technique is implemented in TopoSZ.

For each point $x \in \X$ with a ground truth value $f(x)$, an initial guess for its value $g(x)$ (e.g., from a Lorenzo predictor; see below) is shifted by an integer multiple of $2\xi$ to obtain a new value $f'(x)$ such that $|f'(x) - f(x)| \leq \xi$. 

This process can be conceptualized as follows: divide the real line into intervals of length $2\xi$, where one interval is centered on $g(x)$. 
The compressor then calculates how many intervals to shift $g(x)$, so that it can assign a value to $f'(x)$ that is a distance less than $\xi$ from $f(x)$. By construction, if $f(x)$ lies in an interval of length $2\xi$ centered on $f'(x)$, then $|f(x)-f'(x)| \leq \xi$. 
This process is illustrated in \cref{fig:linear-scaling-quantization}.
\begin{figure}[!ht]
  \vspace{-3mm}
  \centering
  \includegraphics[width=\columnwidth]{fig-linear-scaling-quantization.pdf}
  \vspace{-8mm}
  \caption{A standard implementation of a linear-scaling quantization.}
  \label{fig:linear-scaling-quantization}
  \vspace{-4mm}
\end{figure}
\noindent Following this construction, each $x \in \X$ is assigned an integer $n_x$ (corresponding to how many intervals $g(x)$ is shifted) such that $f'(x) = g(x) + 2\xi n_x$. These quantization numbers $\{n_x\}$ are encoded and stored in the compressed file.

If the distribution of $\{n_x\}$ has low entropy (e.g.,~if $\{n_x\}$ are mostly zeros), then the quantization numbers can be compressed to small size using an entropy-based compression algorithm, such as Huffman coding. 
More accurate predictions $g(x)$ generally lead to $\{n_x\}$ with lower entropy.

\para{False Cases.} Yan et al.~\cite{yan2023toposz} introduced three types of false cases to quantify the level of contour tree preservation: false positives, false negatives, and false types, which are illustrated in \cref{fig:false-cases}.
A false positive occurs when a new edge appears in the contour tree of the reconstructed data that does not exist in the same position of the contour tree of the original data. A false negative occurs when an edge of the contour tree from the original data is missing from the contour tree of the reconstructed data. 
A false type occurs when the critical type (maximum, minimum, saddle) of one or both endpoints of an edge of the contour tree does not match between the original and reconstructed data.
TopoSZ focuses on false cases involving extremum-saddle pairs and its algorithm terminates when there are no such false cases. 

\begin{figure}[!ht]
    \vspace{-2mm}
    \centering
    \includegraphics[width=\linewidth]{fig-false-cases.pdf}
    \vspace{-6mm}
    \caption{Three types of false cases. (A) The original contour tree. (B) A false positive: an extra edge is added. (C) A false negative: an edge is missing. (D) A false type: an edge contains a critical point as its endpoint that changes its type.}
    \label{fig:false-cases}
    \vspace{-4mm}
\end{figure}


\para{TopoSZ Pipeline.} The TopoSZ pipeline is as follows:

\para{\underline{Step 1: Upper and lower bound calculation.}}
TopoSZ first applies persistence simplification to $f$ in order to calculate $T_\varepsilon$. For each point $x \in \X$, a lower bound $L(x)$ and an upper bound $U(x)$ are assigned to $x$ according to the contour-tree-induced segmentation.
If $x$ belongs to the segmented region corresponding to edge $ab \in T_\varepsilon$, then $L(x) = \min(f(a),f(b))$ and $U(x) = \max(f(a),f(b))$. The nodes of the contour tree are stored losslessly.

\para{\underline{Step 2: Prediction.}} 
TopoSZ uses a Lorenzo predictor \cite{ibarria2003out} to predict the values of each data point. For each point $x$, the Lorenzo predictor predicts $f(x)$ as a weighted sum of the values from previously predicted points $x'$ satisfying $\|x-x'\|_\infty = 1$. The weights are fixed, and are chosen such that quadratic functions will be perfectly predicted.

\para{\underline{Step 3: Linear-scaling quantization.}} TopoSZ uses linear-scaling quantization with a decreased interval size to ensure that the pointwise upper and lower bounds, as well as the global error bound $\xi$, are maintained for each $x \in \X$. For any $x \in \X$ where no possible quantization code $n_x$ satisfies these conditions, $f(x)$ is stored losslessly.

\para{\underline{Step 4: Iterative upper and lower bound tightening.}} 
If the results from Step 3 do not perfectly preserve the contour tree, that is, if there are false cases presented in the reconstructed data, then the upper and lower bounds are tightened around points corresponding to those false edges, and then Step 3 is repeated. This cycle repeats until there are no false cases. 
%The specifics of this step are reviewed in \cref{sec:topoSZ-detail}.

\para{\underline{Step 5: Lossless compression.}} The numbers from linear-scaling quantization are encoded using Huffman Coding. The relevant information is then stored in a binary file that is further compressed using ZSTD~\cite{collet2018zstandard}.
\section{NumericBench Generation}
In this section, we present our created  NumericBench, which is specifically designed to evaluate fundamental numerical capabilities of LLMs. 
NumericBench consists of diverse datasets and tasks, 
enabling a systematic and comprehensive evaluation.
We discuss the datasets included in NumericBench, the key abilities it evaluates, and the methodology for benchmark generation.

\begin{table*}[t]
	\caption{NumericBench statistics. R: contextual retrieval, C: comparison, S: summary, L: logical reasoning. The token count is calculated based on tiktoken, which is the tokenizer used by Llama3~\cite{grattafiori2024llama3herdmodels}. The sentences used for token calculation include both the context and the question.}
	\centering
	\renewcommand{\arraystretch}{1.15} % 设置行间距为默认的 1.15 倍
	\setlength{\tabcolsep}{1.5pt} % 将列间距设置为 1pt
\resizebox{\textwidth}{!}{
	\begin{tabular}{c|c|c|c|c}
		\toprule
		\textbf{Data} & \textbf{Format} & \textbf{Questions} & \textbf{\# Instance} & \textbf{Avg Token} \\ \midrule
		
		\multirow{3}{*}{} 
		& \multirow{3}{*}{} 
		& \begin{tabular}[c]{@{}c@{}}R: What is the index of the first occurrence\\ of the number -3095 in the list?\end{tabular} 
		& 1000 & 3704.23 \\ \cline{3-5}
		
		\textbf{\begin{tabular}[c]{@{}c@{}}Number\\ List\end{tabular}}
		& $[69, -1, 6.1, \ldots, 5.7]$
		& \begin{tabular}[c]{@{}c@{}}C: Which index holds the smallest number\\
			 in the list between the indices 20 and 80?\end{tabular} 
		& 1000 & 3685.57  \\ \cline{3-5}
		
		& & \begin{tabular}[c]{@{}c@{}}S: What is the average of the index of\\
			 top 30 largest numbers in the list?\end{tabular} 
		& 1000 & 3654.78 \\ \midrule
		
		\multirow{3}{*}{} 
		& \multirow{3}{*}{
		\begin{tabular}[c]{@{}c@{}}
			\{date: 2024-06-19,\\
			close\_price: 9.79, \\
			open\_price: 9.4, \\
			\ldots \\
			PE\_ratio: 4.5416\}
		\end{tabular}
		} 
		& \begin{tabular}[c]{@{}c@{}}
			R: On which date did the close price\\
			 of stock firstly reach 61.76 yuan?
		\end{tabular}
		& 1000 & 27585.35 \\ \cline{3-5}
		
		\textbf{Stock}
		& 
		& \begin{tabular}[c]{@{}c@{}}
			C: Among the top-45 trading value days, which\\
			 date did the stock have the lowest close price?
		\end{tabular}
		 & 1000 & 27595.40 \\ \cline{3-5}
		
		& & \begin{tabular}[c]{@{}c@{}} 
			S: How many days had the close price higher than\\
			 the open price from 2024-07-31 to 2024-12-13?
		\end{tabular}	
		& 1000 & 27561.29 \\ \midrule
		
		\multirow{3}{*}{} 
		& \multirow{3}{*}{
		\begin{tabular}[c]{@{}c@{}}
			\{date: 2024-07-21,\\
			pressure\_msl: 999.96,\\
			dew\_point\_2m: 26.25,\\
			\ldots \\
			cloud\_cover: 61.5\}
		\end{tabular}
		} 
		& \begin{tabular}[c]{@{}c@{}} 
			R: On which date did the dew point temperature\\
			 at two meters firstly drop below 9.15°C?
		\end{tabular}
		& 1000 & 27359.26 \\ \cline{3-5}
		
		\textbf{Weather}
		& & \begin{tabular}[c]{@{}c@{}} 
			C: On which date did the MSL pressure reach its\\
			highest value when the cloud cover was below 9\%?
		\end{tabular}
		& 1000 & 27368.19 \\ \cline{3-5}
		
		& & \begin{tabular}[c]{@{}c@{}} 
			S: What was the average temperature at two meters\\
			when the relative humidity exceeded 78.56\%?
		\end{tabular}
		& 1000 & 27331.21 \\ \midrule
		
		\textbf{Sequence} 
		& $[0.34, 3, 6, \ldots, 111]$ 
		& L: What is the next number in the sequence? & 500 & 677.57 \\ \midrule
		
		\textbf{\begin{tabular}[c]{@{}c@{}}Arithmetic \\Operation\end{tabular}} 
		& \begin{tabular}[c]{@{}c@{}} 
		$a: 6.755,
		b: -1.225$
		\end{tabular}
		& \begin{tabular}[c]{@{}c@{}} 
		 $Q_{oper}$: What is the result of $a + b$?\\
		 $Q_{context}$: What is the result of $a $ plus $b$?
		 
		\end{tabular}
		& 12000 & 112.00 \\ \midrule
		
		\textbf{\begin{tabular}[c]{@{}c@{}}Mixed-number-string\\ Sequence\end{tabular}} 
		& \begin{tabular}[c]{@{}c@{}} 
		$effV2\ldots x98o7Lo$
		\end{tabular}
		& \begin{tabular}[c]{@{}c@{}} 
		How many numbers are there in the string? Note\\
		that a sequence like 'a243b' counts as a single number.
		\end{tabular}
		& 2000 & 196.53 \\ \bottomrule

	\end{tabular}
}
	\label{tab:data_stat}
	
\end{table*}

 

\subsection{Numeric Dataset Collection}
NumericBench offers a diverse collection of numerical datasets and questions designed to reflect real-world scenarios. 
This variety ensures that LLMs are thoroughly tested on their fundamental  abilities on numerical data.

\noindent\textbf{Number List Dataset.}
The synthetic number list dataset consists of simple collections of numerical values (integer and floats) 
presented as ordered or unordered lists.
Numbers in lists are one of the simplest and most fundamental data representations encountered in real-world scenarios.
Despite their simplicity, retrieving, indexing,  comparison, and summary on numbers can verify the fundamental numerical ability of LLMs. 
This dataset serves as a fundamental dataset of how well LLMs understand numerical values as discrete entities.



\noindent\textbf{Stock Dataset.}
The time-series  stock dataset is crawled from Eastmoney website~\cite{eastmoney}, 
which has eighteen attributes, such as stock close prices, open price,  trading volumes, and price-earnings ratio, over time.
Stock  data reflects dynamic, real-world numerical reasoning challenges that involve trend analysis, comparison, and decision-making under uncertainty,  representing real-world financial workflows.
 




\noindent\textbf{Weather Dataset.}
The weather dataset is crawled from Open-Meteo python API~\citep{openmeteo}, which includes data related to weather metrics, such as temperature, precipitation, humidity, and wind speed. 
The data is presented across various longitude and latitude.
 
 




\noindent\textbf{Numeric Sequence  Dataset.}
The synthetic numeric sequence dataset comprises sequences of numbers generated by arithmetic or geometric progression, complex patterns, or noisy inputs. 
Tasks require identifying patterns, predicting the next number, or reasoning about relationships between numbers.
Numerical sequences test the logical reasoning capabilities of LLMs, requiring pattern recognition and multi-step reasoning. This dataset introduces structured challenges that are common in mathematics and algorithmic reasoning.


 
\noindent\textbf{Arithmetic Operation Dataset.}
The dataset comprises 12,000 pairs of simple numbers, each undergoing addition, subtraction, multiplication, and division operations. Each pair of numbers, $a$ and $b$, consists of $k$-digit integers with three decimal places, where $k \in \{1, 2, \cdots, 6\}$. 
For each value of $k$, there are 2,000 pairs, evenly distributed across the four basic operations (i.e, $+, -,  *, /$), with 500 pairs per operation. 
This dataset is to evaluate the fundamental mathematical operation capabilities of LLMs, simulating the majority of mathematical calculation requirements in real-world scenarios.

\noindent\textbf{Mixed-number-string Sequence Dataset.}
The dataset consists of alphanumeric strings of varying lengths $\{50, 100, 150, 200\}$, each containing a randomized mix of letters and digits. For each string length, 500 samples are generated, resulting in a total of 2,000 samples. Each sample includes a query asking for the count of contiguous numeric sequences within the string, where sequences like "a243b" count as a single number. This dataset is designed to assess the ability of LLMs to identify and count numeric sequences.
 







\subsection{Fundamental Numerical Ability}
NumericBench is designed to comprehensively evaluate six fundamental numerical reasoning abilities of LLMs, which is 
%These three fundamental abilities are 
essential for solving real-world numeric-related tasks.
%such as numeric data summary and financial price analysis.


\noindent\textbf{Contextual Retrieval Ability.}
Contextual retrieval ability evaluates how well LLMs can locate, extract, and identify specific numerical values or their positions within structured or unstructured data. 
This includes tasks like finding a specific number in a list, retrieving values , and indexing numbers based on their order.
For example, as shown in Table~\ref{tab:data_stat}, it evaluates LLMs on tasks such as retrieving stock prices and identifying key values within numerical lists or domain-specific data (e.g., stock market and weather-related information).
This ability is fundamental to numerical reasoning because it forms the foundation for higher-order tasks, such as comparison, aggregation, and logical reasoning. 
 
 



\noindent\textbf{Comparison Ability.}
Comparison ability evaluates how well LLMs can compare numerical values to determine relationships such as greater than, less than, or equal to, and identify trends or differences in datasets. 
Comparison is vital for logical reasoning and decision-making, as many real-world tasks depend on accurate numerical evaluation. 
For instance,  as shown in Table~\ref{tab:data_stat},   comparing prices is essential in stock  for assessing performance, while weather forecasting requires analysis of temperature or precipitation trends over time. 
 



\noindent\textbf{Summary Ability.}
Summary ability assesses the LLM’s capacity to aggregate numerical data into concise insights, such as calculating totals, averages, or other statistical metrics. 
Summarization is critical for condensing large datasets into actionable information, enabling decision-making based on aggregated insights rather than raw data. 
This ability is indispensable in domains like electricity usage analysis, where summarizing hourly or daily consumption helps forecast bills, in business reporting for aggregating sales and revenue data to evaluate performance, 
and in healthcare analytics to monitor trends in patient metrics over time.



\noindent\textbf{Logic Reasoning Ability.}
Logical Reasoning Ability measures the LLM’s ability to perform multi-step operations involving numerical data, 
such as recognizing patterns, inferring rules, and applying arithmetic or geometric reasoning to solve complex problems. Logical reasoning extends beyond simple numerical tasks and reflects the LLM’s capacity for deeper, structured thinking. 
This ability is crucial for algorithm design, where solving problems involving numeric sequences or patterns is essential, in scientific research for identifying relationships and correlations in data.

\noindent\textbf{Arithmetic Operation Ability.}
It reflects the LLM's capacity to perform mathematical calculations accurately. Such ability is essential for tasks involving numerical computations, such as  automated machine learning through LLMs.





\noindent\textbf{Number Recognition  Ability.}
This measures the LLM's proficiency in identifying and interpreting numerical information within a given context. It represents a fundamental requirement for handling numeric-based tasks effectively.




\subsection{NumericBench Generation}
We use the number list, stock, and weather datasets to evaluate the contextual retrieval, comparison, and summary abilities of LLMs. 
Specifically, for each ability and each dataset, we prepare a set of questions designed to assess the corresponding target ability.
As shown in Table~\ref{appx:number_question}, Table~\ref{appx:stock_question}, and Table~\ref{appx:weather_question} in Appendix, there are nine question sets in total, covering three abilities across three datasets. 
When evaluating a specific ability (e.g., contextual retrieval) on a specific dataset (e.g., stock data), we randomly select one question from the corresponding question set for each data instance (e.g., a stock instance) 
and manually label the answer. This approach enables us to generate question-answer pairs for each ability on the number list, stock, and weather datasets.

For arithmetic operations and number counting in the strings dataset, the question format is straightforward, as illustrated in Table~\ref{tab:data_stat}. These questions are designed to evaluate the basic arithmetic operation and number recognition abilities of LLMs.




\begin{table*}[t]
% \setlength\tabcolsep{5pt}
\centering
\small
\scalebox{1}{
\begin{tabular}{lccccccccccccccc}
\toprule
\multicolumn{1}{c}{\multirow{2.5}{*}{\textbf{Method}}} & \multicolumn{3}{c}{{HotpotQA}} & \multicolumn{3}{c}{{2WikiMultihopQA}} & \multicolumn{3}{c}{{MuSiQue}} & \multicolumn{3}{c}{{StrategyQA}} & \multicolumn{3}{c}{\textbf{Average}} \\
\cmidrule(r){2-4} \cmidrule(r){5-7} \cmidrule(r){8-10} \cmidrule(r){11-13} \cmidrule(r){14-16}
\multicolumn{1}{c}{} & \textbf{EM} & \textbf{CEM} & \textbf{F1} & \textbf{EM} & \textbf{CEM} & \textbf{F1} & \textbf{EM} & \textbf{CEM} & \textbf{F1} & \textbf{EM} & \textbf{CEM} & \textbf{F1} & \textbf{EM} & \textbf{CEM} & \textbf{F1} \\

\midrule
\multicolumn{16}{c}{\textit{GPT-4o-mini (Closed-source)}} \\
\midrule
% \cmidrule{1-16}
Closed-Book & 28.82 & 39.80 & 34.71 & 24.71 & 30.40 & 24.71 & 7.65 & 15.91 & 9.41 & 73.53 & 73.53 & 73.53 & 33.68 & 39.91 & 35.59 \\
\multicolumn{1}{l}{Chain-of-Tought} & 26.47 & 38.79 & 32.35 & 24.12 & 30.84 & 26.47 & 13.53 & 20.92 & 18.24 & 51.76 & 52.09 & 51.76 & 28.97 & 35.66 & 32.21 \\
\multicolumn{1}{l}{Standard RAG} & 41.18 & 54.36 & 52.94 & 25.88 & 32.09 & 27.65 & 11.76 & 19.91 & 16.47 & 58.24 & 59.53 & 58.24 & 34.27 & 41.47 & 38.83 \\
\cmidrule{2-16}
\multicolumn{1}{l}{ReAct} & 35.88 & 51.08 & 42.35 & 29.41 & 35.46 & 30.00 & 10.00 & 18.05 & 12.35 & 36.47 & 40.46 & 36.47 & 27.94 & 36.26 & 30.29 \\
\multicolumn{1}{l}{Query2doc} & 44.71 & 57.21 & 54.71 & 29.41 & 34.46 & 29.41 & 19.41 & 28.05 & 24.71 & 64.71 & 65.67 & 64.71 & 39.56 & 46.35 & 43.39 \\
\multicolumn{1}{l}{Self-RAG} & 38.82 & 50.32 & 47.65 & 26.47 & 31.87 & 27.65 & 13.53 & 21.29 & 16.47 & 68.82 & 69.10 & 68.82 & 36.91 & 43.15 & 40.15 \\
% \multicolumn{1}{l}{Mind-Search} & 44.0 & 50.0 & 58.6 & 28.0 & 29.0 & 32.2 & 14.0 & 16.0 & 22.8 & \textbf{72.0} & 74.0 & \textbf{72.0} \\
% \multicolumn{1}{l}{Infogent} & 44.0 & 50.0 & 58.6 & 28.0 & 29.0 & 32.2 & 14.0 & 16.0 & 22.8 & \textbf{72.0} & 74.0 & \textbf{72.0} \\
% \multicolumn{1}{l}{RAG-Star}  \\
\cmidrule{2-16}
Ours & 41.76 & 45.88 & 58.69 & 52.94 & 65.75 & 53.53 & 23.67 & 33.21 & 26.04 & 77.67 & 77.67 & 77.67 & \ \ \textbf{49.01}$^{\dagger}$ & \ \ \textbf{55.63}$^{\dagger}$ & \ \ \textbf{53.98}$^{\dagger}$ \\

\midrule
\multicolumn{16}{c}{\textit{Gemini-1.5-flash (Closed-source)}} \\
\midrule
Closed-Book & 19.41 & 31.52 & 24.71 & 24.12 & 30.07 & 24.71 & 3.53 & 10.98 & 6.47 & 32.35 & 32.83 & 32.35 & 19.85 & 26.35 & 22.06 \\
\multicolumn{1}{l}{Chain-of-Tought} & 28.82 & 36.43 & 34.71 & 18.24 & 21.35 & 18.82 & 8.24 & 13.73 & 10.59 & 68.82 & 69.64 & 68.82 & 31.03 & 35.29 & 33.24 \\
\multicolumn{1}{l}{Standard RAG} & 37.65 & 48.95 & 47.06 & 16.47 & 20.54 & 17.65 & 7.06 & 11.54 & 10.00 & 52.94 & 53.89 & 52.94 & 28.53 & 33.73 & 31.91 \\
\cmidrule(lr){2-16}
\multicolumn{1}{l}{ReAct} & 34.71 & 45.44 & 42.35 & 18.24 & 22.33 & 18.82 & 5.88 & 10.38 & 7.06 & 34.91 & 36.03 & 34.91 & 23.44 & 28.55 & 25.79 \\
\multicolumn{1}{l}{Query2doc} & 38.16 & 49.15 & 47.81 & 17.06 & 20.32 & 17.65 & 11.18 & 17.65 & 14.71 & 58.24 & 58.53 & 58.24 & 31.16 & 36.41 & 34.60\\
\multicolumn{1}{l}{Self-RAG} & 39.83 & 49.47 & 47.88 & 15.29 & 19.06 & 17.65 & 7.65 & 12.09 & 10.00 & 54.12 & 54.53 & 54.12 & 29.22 & 33.79 & 32.41 \\
% \multicolumn{1}{l}{Mind-Search} &  &  &  &  &  &  &  &  &  &  &  &  \\
% \multicolumn{1}{l}{Infogent} &  &  &  &  &  &  &  &  &  &  &  &  \\
% \multicolumn{1}{l}{RAG-Star} &  &  &  &  &  &  &  &  &  &  &  &  \\
\cmidrule(lr){2-16}
Ours & 47.65 & 57.65 & 61.98 & 62.94 & 62.94 & 73.94 & 17.75 & 21.30 & 28.31 & 76.19 & 76.19 & 76.19 & \ \ \textbf{51.13}$^{\dagger}$ & \ \ \textbf{54.52}$^{\dagger}$ & \ \ \textbf{60.11}$^{\dagger}$ \\
\midrule
% \multicolumn{13}{c}{\textit{Claude-3.5-sonnet (Closed-source)}} \\
% \midrule
% Closed-Book & 33.82 & 47.01 & 43.53 & 34.71 & 39.70 & 35.88 & 16.76 & 25.91 & 20.00 & 68.82 & 70.27 & 68.82\\
% \multicolumn{1}{l}{Chain-of-Tought} & 34.86 & 48.31 & 42.51 & 40.00 & 49.52 & 40.59 & 16.03 & 26.15 & 19.41 & 19.51 & 27.42 & 19.51 \\
% \multicolumn{1}{l}{Standard RAG} & 40.64 & 53.85 & 51.24 & 17.65 & 21.33 & 18.82 & 12.68 & 17.79 & 15.96 & 57.00 & 60.11 & 57.00 \\
% \cmidrule(lr){2-13}
% \multicolumn{1}{l}{ReAct} & 28.01 & 40.01 & 35.18 & 12.29 & 15.31 & 12.85 & 10.75 & 16.81 & 12.37 & 19.89 & 24.73 & 19.89 \\
% \multicolumn{1}{l}{Query2doc} & 46.50 & 59.86 & 57.61 & 19.81 & 23.46 & 21.70 & 21.97 & 28.88 & 24.28 & 56.00 & 59.14 & 56.00 \\
% \multicolumn{1}{l}{Self-RAG} & 40.23 & 51.66 & 48.28 & 0.00 & 0.00 & 0.00 & 0.00 & 29.17 & 0.00 & 80.00 & 84.44 & 80.00 \\
% % \multicolumn{1}{l}{Mind-Search} &  &  &  &  &  &  &  &  &  &  &  &  \\
% % \multicolumn{1}{l}{Infogent} &  &  &  &  &  &  &  &  &  &  &  &  \\
% \multicolumn{1}{l}{RAG-Star} &  &  &  &  &  &  &  &  &  &  &  &  \\
% \cmidrule(lr){2-13}
% Ours &  &  &  &  &  &  &  &  &  &  &  &  \\
\midrule
\multicolumn{16}{c}{\textit{DeepSeek-V3-chat (Open-source)}} \\
\midrule
Closed-Book & 35.88 & 48.58 & 44.12 & 35.88 & 41.85 & 37.06 & 11.18 & 19.54 & 14.71 & 64.71 & 65.10 & 64.71 & 36.91 & 43.77 & 40.15 \\
\multicolumn{1}{l}{Chain-of-Tought} & 38.82 & 50.59 & 47.06 & 47.06 & 56.14 & 48.24 & 21.18 & 29.99 & 24.12 & 41.76 & 47.46 & 41.76 & 37.21 & 46.12 & 40.30 \\
\multicolumn{1}{l}{Standard RAG} & 40.00 & 55.01 & 50.00 & 30.00 & 34.26 & 31.18 & 15.88 & 24.47 & 18.82 & 64.12 & 65.59 & 64.12 & 37.50 & 44.83 & 41.03 \\
\cmidrule(lr){2-16}
\multicolumn{1}{l}{ReAct} & 40.00 & 54.97 & 48.24 & 32.35 & 36.17 & 32.35 & 16.67 & 34.07 & 20.83 & 23.53 & 28.79 & 23.53 & 28.14 & 38.50 & 31.24 \\
\multicolumn{1}{l}{Query2doc} & 47.19 & 63.11 & 58.05 & 32.35 & 37.14 & 32.35 & 21.18 & 29.89 & 24.71 & 57.65 & 59.40 & 57.65 & 39.59 & 47.39 & 43.19 \\
\multicolumn{1}{l}{Self-RAG} & 43.49 & 56.31 & 51.75 & 27.65 & 32.27 & 28.24 & 16.86 & 25.30 & 19.77 & 52.87 & 53.66 & 52.87 & 35.22 & 41.89 & 38.16 \\
% \multicolumn{1}{l}{Mind-Search} &  &  &  &  &  &  &  &  &  &  &  &  \\
% \multicolumn{1}{l}{Infogent} &  &  &  &  &  &  &  &  &  &  &  &  \\
% \multicolumn{1}{l}{RAG-Star} &  &  &  &  &  &  &  &  &  &  &  &  \\
\cmidrule{2-16}
\textbf{Ours} & 45.88 & 52.94 & 62.99 & 60.00 & 63.53 & 73.15 & 21.30 & 32.60 & 25.44 & 76.47 & 76.47 & 76.47 & \ \ \textbf{50.91}$^{\dagger}$ & \ \ \textbf{56.39}$^{\dagger}$ & \ \ \textbf{59.51}$^{\dagger}$ \\
\bottomrule
\end{tabular}}
\caption{The evaluation results for four representative multi-hop QA datasets are presented, we also report the average results of the four datasets. The symbol ``$^{\dagger}$'' denotes that the performance improvement is statistically significant with p < 0.05 compared against all the baselines.}
\label{tab:main-result}
\end{table*}



\section{Experiments and Analysis}
In this section, we begin by detailing the experimental setup, followed by a comprehensive presentation of the primary results. Subsequently, we conduct an ablation study and provide an in-depth analysis to elucidate our findings further.

\subsection{Experimental Settings}


\subsubsection{Datasets}
% FRAMES~\cite{krishna2024fact}, and
To rigorously assess the efficacy of our proposed methodology, we conducted evaluations using 5 benchmark datasets. All datasets are meticulously designed to challenge models with complex, multi-hop questions that require synthesizing information across multiple documents.
% The detailed statistics of different datasets are shown in Table~\ref{tab:datasets}.

\begin{itemize}
    \item \textbf{FanOutQA}~\cite{zhu2024fanoutqa} is a high-quality dataset comprising complex information-seeking questions and human-written decompositions, which necessitate aggregating information about multiple entities from several sources to formulate a comprehensive answer. 
    % \item \textbf{FRAMES} comprises multi-hop questions requiring integration of information from multiple sources. These questions span diverse topics, and are labeled with reasoning types such as numerical reasoning, tabular reasoning, multiple constraints, temporal reasoning, and post-processing. Following existing study~\cite{reddy2024infogent}, we exclude numerical questions since different LLMs tend to be highly sensitive to them.

    \item \textbf{HotpotQA}~\cite{yang2018hotpotqa} is a widely used dataset for multi-hop question answering, designed to evaluate reason abilities across multiple documents to answer complex questions. The dataset is collected via crowdsourcing based on Wikipedia articles, and annotators are asked to propose questions that require reasoning using the multiple presented Wikipedia articles.

    \item \textbf{2WikiMultihopQA}~\cite{ho2020constructing} is a large-scale multi-hop QA dataset that requires reading multiple paragraphs to answer a given question. The dataset includes four types of questions: comparison, inference, compositional, and bridge-comparison. Each question is accompanied by relevant Wikipedia paragraphs as evidence.
    
    \item \textbf{MuSiQue}~\cite{trivedi2022musique} is created by composing questions from multiple existing single-hop datasets. The dataset is more challenging than previous multi-hop reasoning datasets, with a threefold increase in the human-machine gap and significantly lower disconnected reasoning scores, indicating reduced susceptibility to shortcut reasoning.
    
    \item \textbf{StrategyQA}~\cite{geva2021did} focuses on open-domain questions that require implicit reasoning steps. The dataset consists of 2,780 examples, each comprising a strategy question, its decomposition, and evidence paragraphs. Each question is accompanied by decomposed reasoning steps and relevant Wikipedia paragraphs as evidence.


    
\end{itemize}



\subsubsection{Evaluation Metrics}
We adopted the established evaluation metrics for the adopted datasets to ensure consistency and comparability. For the evaluation of \textit{FanOutQA}, we employed string accuracy, which measures the proportion of exact matches, and ROUGE metrics~\cite{lin2004rouge}, which assess the quality of summarization by comparing multiple features between the generated and reference texts. Specifically, we report ROUGE-1 (R-1), ROUGE-2 (R-2), and ROUGE-L (R-L) scores to comprehensively assess performance.
% For FRAMES, a large language model~(LLM) was utilized to benchmark the alignment of generated answers with ground-truth annotations. We present the evaluation results separately for queries corresponding to each of the four reasoning types.
For \textit{HotpotQA, 2WikiMultihopQA, MuSiQue}, and \textit{StrategyQA}, we adopt Exact Match (EM), F1 score, and Cover Exact Match (CEM) as evaluation metrics. EM measures strict correctness by checking if the predicted answer matches the ground truth. F1 evaluates the overlap between prediction and ground truth, balancing precision and recall. CEM extends EM to multi-hop reasoning, requiring both correct answers and coverage of intermediate reasoning steps.
Similar to the setup in the existing work~\cite{jiang2024rag}, due to the large data scale, we randomly sampled 130 queries from each of the four datasets for evaluation.

\subsubsection{Baselines}
In the evaluation of our proposed method, we compare it against abundantly established baselines to ensure a comprehensive understanding of its performance. These baselines represent a spectrum of approaches commonly employed in the field, ranging from vanilla reasoning strategies to advanced reasoning methods.

\paratitle{Vinilla reasoning.}\quad
The \textit{Closed-Book} method directly prompts the LLM to provide an answer to a question. In contrast, \textit{Chain-of-Thought (CoT)}~\cite{wei2022chain} reasoning involves adding intermediate reasoning steps to facilitate the response. \textit{Standard RAG} first retrieves passages from the Wikipedia corpus using DPR~\cite{dpr2020} and then directly prompts the LLM to refer to these passages in its response. 

\paratitle{Advanced reasoning.}\quad
\textit{ReAct}~\cite{yao2023react} progressively addresses subqueries, ultimately consolidating the intermediate results to form a complete answer.
\textit{Query2doc}~\cite{wang2023query2doc} generates an initial answer using the model and subsequently retrieves relevant information to generate the final answer.
\textit{Self-RAG}~\cite{asaiself} involves first retrieving information and then assessing its relevance before deciding whether to incorporate it into the final answer.
\textit{MindSearch}~\cite{chen2024mindsearch} employs a planner-searcher architecture for searching relevant information.
\textit{Infogent}~\cite{reddy2024infogent} introduces a multi-agent architecture to aggregate multi-source information.
% \textit{RAG-Star}~\cite{jiang2024rag} adopts MCTS with retrieval for deliberate reasoning for multi-hop questions.


To ensure a more comprehensive evaluation of different methods, and to mitigate the influence of any specific model, we employ multiple LLMs as backbone models of different methods, including the closed-source LLM \texttt{GPT-4o-mini}, \texttt{Gemini-1.5-flash-002} and the open-source LLM \texttt{deepseek-v3-chat}.
We evaluated all baseline methods in a \textit{zero-shot} setting, employing them solely for inference without additional training.
Note that certain methods are challenging to replicate across all datasets due to the requirement of dataset-specific refinement. As a result, we are unable to report the results for all baselines on every dataset.


\subsubsection{Implement Details}

For web search, we employed Google Search as the primary search engine, selecting the top-3 web search results as document candidates and adhering to existing methodologies for web crawling and denoising~\cite{reddy2024infogent}. During the MCTS process, we ensured consistency between the policy model and the reward model. The maximum number of simulations was capped at 40, and the search depth was limited to 6 layers. In the upper confidence bound for trees algorithm, the exploration-exploitation balance parameter \( w \) was set to 0.2. Additionally, we generated three sub-queries per iteration (\( m_q = 3 \)).
For all generation tasks, responses were sampled using a temperature of 0.9 and top-\( p \) sampling with \( p = 1.0 \).  All prompts used are shown in the provided anonymous codes.




\subsection{Main Results}
The results of different methods evaluated on five complex reasoning datasets are shown in Table~\ref{tab:main-result} and Table~\ref{tab:fanoutqa}. It can be observed that:

(1) Our proposed method demonstrates significant improvements over all baseline approaches across four multi-hop QA datasets and the FanoutQA dataset that emphasizes the information-gathering task. This performance advantage is consistently observed across multiple popular backbone LLMs, highlighting the general applicability and effectiveness of our HG-MCTS framework. HG-MCTS employs an adaptive checklist to guide the expansion and reward modeling of Monte Carlo Tree Search (MCTS), effectively curtailing the exploration of unproductive pathways while maintaining robust search capabilities. In addition, its emphasis on targeted information gathering minimizes irrelevant content, reducing extraneous noise that could compromise the quality of the generated answers.

(2) We further observe that the MCTS-based approach for complex reasoning outperforms both single-step reasoning and chain-based reasoning methods, indicating that tree search can substantially expand the search space of solution paths, enable the discovery of optimal reasoning paths, and ultimately enhance performance. In addition, incorporating advanced reasoning strategies into RAG proves superior to vanilla reasoning standard RAG or closed-book approaches overall, highlighting the importance of meeting users’ complex information needs through tailored retrieval processes. Notably, on the FanoutQA dataset, our method based on GPT-4o-mini surpasses the baseline built upon the larger and more powerful LLM GPT-4o, offering further evidence for the intrinsic advantages of our approach.

(3) Finally, our method exhibits strong robustness across different backbone LLMs, whereas other methods often experience significant performance fluctuations depending on the underlying model. For example, when using Gemini as the backbone, the performance of baseline methods drops markedly compared with results obtained using the other two LLMs, while our method does not exhibit performance degradation. Additionally, we observe that the open-source model deepseek-v3-chat demonstrates capabilities comparable to its proprietary counterparts on various challenging multi-hop QA tasks, thereby laying a promising foundation for real-world deployment of related methodologies.


\begin{table}[t]
    \centering
    \small
\scalebox{1.015}{
    \setlength{\tabcolsep}{3.3pt} 
    \begin{tabular}{llcccc}
        \toprule
        \textbf{Method} & \textbf{LLM} & \textbf{Acc.} & \textbf{R-1} & \textbf{R-2} & \textbf{R-L} \\
        \midrule
        Closed-Book & LLaMA3 & 46.60 & 46.30 & 26.40 & 38.70 \\
        Closed-Book & GPT-4o & 44.10 & 47.40 & 27.30 & 41.70 \\
        Standard RAG & LLaMA3 & 46.80 & 28.20 & 14.30 & 24.30 \\
        Standard RAG & GPT-4o & 58.00 & 49.40 & 31.00 & 44.30 \\
        MindSearch & GPT-4o-mini & 47.30 & 49.30 & 28.40 & 44.20 \\
        Infogent & GPT-4o-mini & 51.10 & 53.30 & 33.00 & 48.50 \\
        \midrule
        \textbf{Ours} & GPT-4o-mini & \ \ \textbf{58.38}$^{\dagger}$ & \ \ \textbf{55.02}$^{\dagger}$ & \ \ \textbf{35.45}$^{\dagger}$ & \ \ \textbf{49.40}$^{\dagger}$ \\
        \bottomrule
    \end{tabular}
    }
    \caption{The evaluation results on FanoutQA. The symbol ``$^{\dagger}$'' denotes that the performance improvement is statistically significant with p < 0.05 compared against all the baselines. LLaMA3 is the abbreviation of LLaMA3-70B-Instruct.}
    \label{tab:fanoutqa}
\end{table}


\subsection{Ablation Studies}
\label{sec:ablation}

In this section, we conduct an ablation study to validate the effectiveness of key strategies in HG-MCTS comprehensively on FanoutQA. Here, we consider five variants based on HG-MCTS for comparison: (a) \underline{\textit{w/o Exploration Reward}} removes the exploration reward during reward modeling; (b) \underline{\textit{w/o Retrieval Reward}} removes the retrieval reward from the total reward during reward modeling; (c) \underline{\textit{w/o Progress Feedback}} eliminates generating the progress feedback in reward modeling; (d) \underline{\textit{w/o Checklist}} removes checklist for global guidance in the MCTS process; (e) \underline{\textit{w/o HG-MCTS}} removes the entire HG-MCTS strategy, reducing the approach to a linear reasoning strategy.

Table~\ref{tab:ablation} presents the results for the variants of our method, from which we can observe the following findings: 
(a) The performance drops in \underline{\textit{w/o Exploration Reward}}, demonstrating that incorporating exploration rewards facilitates more effective expansions during the tree search process. 
(b) The performance drops in \underline{\textit{w/o Retrieval Reward}}, demonstrating incorporating exploration rewards enables the model to better analyze the benefits of external retrieval information during the MCTS process, thereby achieving improved comprehensive information gathering.
(c) The performance drops in \underline{\textit{w/o Progress Feedback}}, underscoring the necessity of incorporating textual feedback to guide subsequent explorations and dynamically updating the checklist.
(d) The performance drops in \underline{\textit{w/o Checklist}}, demonstrating that incorporating the explicit checklist can effectively guide the expansion and reward modeling in MCTS.
(e) The performance significantly drops in \underline{\textit{w/o HG-MCTS}}, demonstrating that the proposed HG-MCTS plays a pivotal role in enhancing the effectiveness of information seeking.

\begin{table}[t]
    \centering
    \small
\scalebox{1.03}{
    \begin{tabular}{lcccc}
        \toprule
        \textbf{Method} & \textbf{Acc.} & \textbf{R-1} & \textbf{R-2} & \textbf{R-L}  \\
        \midrule
        Ours  & 58.38 & 55.02 & 35.45 & 49.40 \\
        \midrule
        w/o Exploration Reward & 57.23 & 54.20 & 34.91 & 48.87  \\
        w/o Retrieval Reward & 56.39 & 53.68 & 34.06 & 48.15  \\
        w/o Progress Feedback & 54.57 & 52.94 & 33.73 & 47.89  \\
        w/o Checklist & 55.03 & 53.41 & 33.85 & 48.03  \\
        w/o HG-MCTS & 52.55 & 53.25 & 32.74 & 47.82  \\
        \bottomrule
    \end{tabular}}
    \caption{Evaluation results of the proposed method's variants on FanoutQA.}
    \label{tab:ablation}
\end{table}

\begin{figure}[t!]
    \centering
    \subfigure[w/o HG-MCTS]{
        \includegraphics[width=0.38\linewidth]{pic/recall_baseline.pdf}
        \label{fig:subfig1}
    }
    \hspace{0.02\textwidth} 
    \subfigure[HG-MCTS]{
        \includegraphics[width=0.38\linewidth]{pic/recall_ours.pdf}
        \label{fig:subfig2}
    }
    \caption{Information collection evaluation for different methods on Recall rate~(blue part).}
    \label{fig:recall}
\end{figure}

\subsection{Analysis on Information Collection}
To evaluate the comprehensiveness of information collected by the proposed HG-MCTS method during the reasoning process, we conducted a systematic comparison of different methods. Here, we compare the proposed method with the w/o HG-MCTS variant in Section~\ref{sec:ablation}, which also employs an information-gathering process.
Specifically, we evaluate the comprehensiveness by calculating the recall rate of the set of web pages retrieved by different methods for the ground-truth Wikipedia pages annotated as solving each intricate query. 
The ground truth serves as a benchmark, representing the authoritative and comprehensive sources necessary for addressing the query. By analyzing the coverage of the retrieved web pages relative to the ground truth, we aimed to quantify the ability of each method to gather a sufficient and relevant body of information.

As shown in Figure~\ref{fig:recall}, our method demonstrates a higher recall rate compared to the baseline in information collection evaluation, indicating that the proposed targeted MCTS enables a more comprehensive collection of information relevant to user queries. Furthermore, our method avoids the tendency to indiscriminately gather excessive information in pursuit of a higher recall rate. Such an approach, while potentially increasing coverage, introduces substantial noise into the subsequent aggregation task, thereby compromising the overall effectiveness of the information seeking.


\subsection{Scaling Law on Simulation Amount}

Our experiments systematically explore the impact of varying the number of simulation iterations on the performance of our analysis method. We adopt two LLM backbones GPT-4o-mini and DeepSeek-V3-chat on the FanoutQA dataset with various simulation numbers for analysis. 

As shown in Figure~\ref{fig:simulation}, we find that increasing the simulation count initially yields significant gains in the model’s ability to navigate the solution space effectively. With more iterations, the method benefits from a broader exploration of potential reasoning paths, leading to enhanced accuracy and improved recall in downstream tasks. In particular, the enhanced exploration reduces the likelihood of early-stage errors propagating through subsequent steps, thereby reinforcing the overall integrity of the search process.
Moreover, our results also reveal a point of diminishing returns. Beyond a certain number of simulations, additional iterations contribute only marginal improvements to the final performance. This saturation effect can be attributed to the inherent limitations of LLM's internal knowledge and the increased computational overhead, which together may lead to overly complex decision paths without proportional benefits. Thus, while extended simulations promote a more thorough examination of the search space, they also impose a trade-off between improved performance and computational efficiency.



\begin{figure}[t!]
    \centering
    \subfigure[GPT-4o-mini]{
        \includegraphics[width=0.4772\linewidth]{pic/scale_4o.pdf}
        \label{fig:subfig1}
    }
    \subfigure[DeepSeek-V3-Chat]{
        \includegraphics[width=0.4772\linewidth]{pic/scale_ds.pdf}
        \label{fig:subfig2}
    }
    \caption{Evaluation results of HG-MCTS with various simulation numbers employed by different LLMs on FanoutQA.}
    \label{fig:simulation}
\end{figure}
\section{Background and Related Work}
Our work proposes to analyze how user involvement in the planning and execution stages of LLM agents shapes user trust in the LLM agents and the overall task performance of LLM agents. 
Thus, we position our work in \revise{three realms} of related literature: human-AI collaboration (\S~\ref{sec-rel-collaboration-LLM}), \revise{trust and reliance on AI systems (\S~\ref{sec-rel-trust-reliance}),} task support with LLMs and LLM agents (\S~\ref{sec-rel-LLM-agent}). %\glcomment{Be careful with the human-AI collaboration. I find it a bit distracting from our focus}

%\glcomment{the paper did not provide a strong review of literature surrounding AI trust. While I am not an expert in understanding trust in AI, I do know there is rich literature in this area. Because this review was not included, it is difficult to evaluate the originality of this work.}\ujcomment{Improve the section, but also point towards recent reviews for readers to get a more comprehensive view (e.g., Siddharth Mehrotra's recent trust review paper)}

\subsection{Human-AI Collaboration}
\label{sec-rel-collaboration-LLM}
% LLM agent in CHI~\cite{zhang2024s}
% \ujcomment{- General human-AI collaboration; delegation; AI-assisted decision making; trust/reliance; complementary performance; 
% - Metrics introduced in recent years;
% - What has actually worked or shown promise in facilitating optimal human-AI collaboration?
% - How do we position our work in this context?}

% \paratitle{Delegation, algorithm appreciation, algorithm aversion, control}. 
In recent decades, deep learning-based AI systems have shown promising performance across various domains~\cite{yang2022survey,fernando2021deep} and applications~\cite{pouyanfar2018survey,dong2021survey}. 
However, such AI systems are not good at dealing with out-of-distribution data~\cite{jia2017adversarial,mccoy-etal-2019-right}, and their intrinsic probabilistic nature brings much uncertainty in %practical service
practice~\cite{ghahramani2015probabilistic}. 
Such observations raise wide concerns about the accountability and reliability of AI systems~\cite{kaur2022trustworthy}. 
Under such circumstances, human-AI collaboration has been recognized as a well-suited approach %one promising approach 
to taking advantage of their promising predictive power and ensuring trustworthy outcomes~\cite{lai2021towards,jiang2021supporting}. 
While humans can provide more reliable and accountable task outcomes, too much user involvement to check and control AI outcomes is undesirable~\cite{lai2022human}. 
It goes against the premise that AI systems are introduced to reduce human workload. 
In that context, researchers have theorized and empirically analyzed when and where users could and should delegate to AI systems~\cite{lai2022human,lubars2019ask}. 

\paratitle{Task Delegation}. While humans prefer to play the leading role in human-AI collaboration~\cite{lubars2019ask}, delegating to AI systems can bring benefits like cost-saving and higher efficiency. 
Apart from manual delegation decisions, it is common to apply automatic rules for human delegation (\eg heuristics obtained from domain expertise or manually crafted rules~\cite{lai2022human}).
% Humans can delegate to AI decisions based on . 
Many user factors like trust~\cite{lubars2019ask}, human expertise domain~\cite{erlei2024understanding}, and AI knowledge~\cite{pinski2023ai}) have a substantial impact on human delegation behaviors. 
% With an empirical study, Erlei \etal~\cite{erlei2024understanding} found that the human expertise domain impacts human delegation behaviors, and the choice independence will be violated when users consider AI performance in an unrelated task.
% Erlei \etal conducted an empirical study to analyze the impact of choice independence and error type in the appropriate delegation behaviors~\cite{erlei2024understanding}. 
%Besides human delegation to AI systems, 
Another relevant stream of recent research has explored AI delegation to humans~\cite{madras2018predict,fugener2022cognitive,pinski2023ai}. 
Researchers have investigated the conditions under which AI systems should defer to a human decision maker, which may bring benefits of improved fairness~\cite{madras2018predict}, accuracy~\cite{narasimhan2022post}, and complementary teaming~\cite{ijcai2022p344}. 
Compared to human delegation, AI delegation has been observed to achieve more consistent benefits in team performance~\cite{fugener2022cognitive,hemmer2023human}. {In collaboration with LLM agents, users need to determine when they should be involved in high-level planning and real-time execution. Such involvement decisions are similar to the delegation choices made by users. While task delegation is not the focus of our study, future work can explore this further.}% within human-LLM agent collaboration.}


\paratitle{AI-assisted Decision Making} has attracted a lot of research focus in human-AI collaboration literature. 
Most existing work has conducted empirical studies~\cite{lai2021towards} and structured interviews~\cite{jiang2021supporting} to understand how factors surrounding the user, task, and AI systems affect human-AI collaboration. 
User factors like AI literacy~\cite{Chiang-IUI-2022}, cognitive bias~\cite{rastogi2022deciding}, and risk perception~\cite{fogliato2021impact,green2021algorithmic} have been observed to substantially impact user trust and reliance on the AI system. 
Similarly, task characteristics like task complexity and uncertainty~\cite{salimzadeh2023missing,salimzadeh2024dealing} and factors of the AI system (\eg performance feedback~\cite{bansal2019beyond,Lu-CHI-2021}, AI transparency~\cite{vossing2022designing} and confidence of AI advice~\cite{tomsett2020rapid,Zhang-FAT-2020}) also affect user trust and reliance on the AI system. 
For a more comprehensive survey of existing work on AI-assisted decision making, readers can refer to~\cite{lai2021towards}.

% Typically, user trust is operationalized as a subjective attitude toward AI systems/AI advice within the literature on human-AI collaboration. In comparison, user reliance on AI systems is based on user behaviors (\eg adoption of AI advice and modification of AI outcomes). 
% Such formulation can even be dated back to trust and reliance on automation systems~\cite{lee2004trust}.

% \paratitle{Calibrated Trust and Appropriate Reliance}. User trust in the context of human-AI collaboration is typically operationalized as a subjective attitude toward AI systems/AI advice~\cite{lee2004trust}. In comparison, user reliance on AI systems is based on user behaviors (\eg adoption of AI advice and modification of AI outcomes). 
% As pointed out by existing work on trust in algorithmic/automated systems, user trust can substantially affect user reliance~\cite{lee2004trust}. 
% While trust calibration is an important goal in human-AI collaboration, it may be not enough to ensure complementary team performance. 
% Through empirical user studies with different confidence levels of AI predictions, Zhang \etal~\cite{Zhang-FAT-2020} found that ``trust calibration alone is not sufficient to improve AI-assisted decision making''. 
% To achieve optimal human-AI collaboration, humans and AI systems need to play complementary roles~\cite{hemmer2021human,hemmer2024complementarity}, and humans need to know when they should adopt AI assistance. 
% In other words, humans should rely on AI advice when AI systems are correct and outperform them, and override AI advice when AI systems are incorrect or less capable than humans. 
% Such user reliance patterns are denoted as \textit{appropriate reliance}~\cite{schemmer2022should,schemmer2023appropriate}, which is the key to
% achieving complementary team performance. 

% Compared with human assistance, users can easily lose confidence in AI systems after seeing them make the same mistakes~\cite{dietvorst2015algorithm}. 
% Such algorithm aversion can be overcome by enabling users to modify the AI predictions~\cite{dietvorst2018overcoming}. 
% As a result of these 
% under-reliance (disuse AI assistance when AI systems outperform humans) and over-reliance (misuse AI assistance when AI systems are wrong or perform worse than humans).
% , users show contradicting attitudes towards AI assistance: algorithm appreciation~\cite{logg2019algorithm,hou2021expert} and algorithm aversion~\cite{dietvorst2015algorithm,dietvorst2018overcoming}. 

% \paratitle{User Trust}. 
% Most existing work has conducted empirical studies~\cite{lai2021towards} and structured interviews~\cite{jiang2021supporting} to understand user trust in AI systems. 



% As a result of uncalibrated trust, users also show sub-optimal reliance on the AI systems: under-reliance (disuse AI assistance when AI systems outperform humans) and over-reliance (misuse AI assistance when AI systems are wrong or perform worse than humans).\glcomment{Is this claim true: unexpected reliance due to uncalibrated trust?}

% To achieve optimal human-AI collaboration, humans and AI systems are supposed to play complementary roles~\cite{hemmer2021human,hemmer2024complementarity}, and humans know when they should adopt AI assistance. 
% In other words, humans should rely on AI advice when AI systems are correct and outperform them, and humans should override AI advice when AI systems are incorrect or less capable than humans. 
% Such user reliance patterns are denoted as appropriate reliance~\cite{schemmer2022should,schemmer2023appropriate}, which is the key to
% achieving complementary team performance. 
% Many factors like cognitive bias~\cite{he2023knowing}, 
% \paratitle{Interventions to Facilitate Calibrated Trust and Appropriate Reliance} 
% The main issues that lead to sub-optimal human-AI collaboration are: under-reliance (\ie disuse AI assistance when AI systems outperform humans) and over-reliance (\ie misuse AI assistance when AI systems are wrong or perform worse than humans)~\cite{schemmer2022should}. 
% %These reliance behaviors are also highly relevant to uncalibrated user trust. 
% Users with an uncalibrated trust in the AI system can be easily misled to disuse or misuse AI systems~\cite{jacovi2021formalizing}. 
%For example, compared with human assistance, users can easily develop a negative impression of AI systems and lose confidence in AI systems. Such phenomenon is called algorithm aversion~\cite{dietvorst2015algorithm}. 
%By contrast, some users were influenced more by algorithmic decisions instead of human decisions, and they first coined the notion of ``Algorithm Appreciation''~\cite{logg2019algorithm}. 
% Researchers have proposed various interventions to promote appropriate reliance~\cite{he2023knowing,Lu-CHI-2021,lu2024does,chiang2021you,Chiang-IUI-2022} and calibrate user trust in AI systems~\cite{buccinca2021trust,Zhang-FAT-2020}.  
% % We bring some representative interventions here.
% %\glcomment{Here is not good enough. I don't plan to bring too many examples. To check how to improve}
% For example, explainable AI methods have been shown to help reduce over-reliance~\cite{vasconcelos2023explanations} and under-reliance~\cite{wang2021explanations} in different scenarios albeit with little consistency across contexts. 
% Another example is tutorial interventions, which have shown effectiveness in user onboarding~\cite{lai2020chicago}, mitigating cognitive biases~\cite{he2023knowing} and developing AI literacy~\cite{Chiang-IUI-2022}. 
% For a more comprehensive overview of interventions to facilitate trust calibration and appropriate reliance, readers can refer to ~\cite{lai2021towards,eckhardt2024survey}.

% In this work, we analyze how user involvement in the planning and execution stages of LLM agents will shape user trust and affect overall task performance. 
% It is highly relevant to existing studies of human-AI collaboration about user trust and appropriate reliance. 
While machine learning and deep learning methods have been extensively analyzed in existing human-AI collaboration literature, to our knowledge, human-AI collaboration with LLM agents is still under-explored. 
Unlike previous studies where AI systems only follow a fixed mode to generate advice, LLM agents can be equipped with more logical clarity and can provide a step-wise plan and can follow a step-by-step execution. 
With such a plan-then-execute setup, LLM agents can bring high flexibility as well as uncertainty in high-level planning and real-time execution. Little is known about
%Meanwhile, it is unclear 
how well LLM agents can work as daily assistants while handling tasks entailing varying stakes and potential risks. %where wrong actions may cause a loss. 
In our study, we analyzed the impact of user involvement in such AI systems by adjusting their intermediate outcomes (plan and step-by-step execution) to calibrate their trust and improve task outcomes. 
Our findings and implications can help advance the understanding of the effectiveness of LLM agents in human-AI collaboration.
% with humans.

%\glcomment{I find the positioning of our work in the space of human-AI collaboration is a bit challenging. Our major claim is: human-AI collaboration with LLM agents is under-explored}

\subsection{\revise{Trust and Reliance on AI systems}}
\label{sec-rel-trust-reliance}
\revise{Trust and reliance have been important research topics since human adoption of automation systems~\cite{lee2004trust,dzindolet2003role}. Due to the widespread integration of AI systems in nearly all walks of society, %In recent years, 
there has been a growing interest in understanding user %researchers have developed a strong interest in 
trust~\cite{vereschak2021evaluate,mehrotra2024systematic} and reliance~\cite{eckhardt2024survey} on AI systems.}
User trust in the context of human-AI collaboration is typically operationalized as a subjective attitude toward AI systems/AI advice~\cite{lee2004trust}. 
In comparison, user reliance on AI systems is based on user behaviors (\eg adoption of AI advice and modification of AI outcomes). 
\revise{The two constructs have been shown to be highly related~\cite{lee1992trust,lee2004trust}: for example, user trust can substantially affect user reliance~\cite{lee2004trust}. 
However, they are intrinsically different and cannot be viewed as a direct reflection of each other~\cite{kahr2024understanding}. 
Most existing work has, therefore, studied the two constructs separately in terms of subjective trust and objective reliance.}
% As pointed out by existing work on trust in algorithmic/automated systems, user trust can substantially affect user reliance~\cite{lee2004trust}. 

\revise{%In an early analysis of human-AI trust, most literature 
Earlier work exploring human-AI trust primarily focused on the impact of different contextual factors surrounding user (\eg risk perception~\cite{green2021algorithmic}), task (\eg task complexity~\cite{salimzadeh2023missing}), and system (\eg stated accuracy~\cite{yin2019understanding,Zhang-FAT-2020}). 
% Among this literature, performance indicators of the AI system (\eg stated accuracy~\cite{yin2019understanding,Zhang-FAT-2020} and confidence~\cite{rechkemmer2022confidence}) have been extensively studied. 
% For trust calibration, researchers have proposed different interventions like explanations~\cite{Zhang-FAT-2020}, educational tutorials~\cite{Chiang-IUI-2022,chiang2021you,lai2020chicago}, competence comparison~\cite{ma2023should}. 
%According to empirical studies about AI-assisted decision making~\cite{yin2019understanding}, 
Empirical studies have shown that most users tend to trust AI systems that are perceived to be highly accurate~\cite{yin2019understanding}. 
Such trust is vulnerable, as the AI system may provide an illusion of competence with persuasive technology (\eg explanations~\cite{chromik2021think,He-IUI-2025}) or overclaimed performance~\cite{yin2019understanding}. 
Even if the AI systems are accurate on specific datasets, they still suffer from out-of-distribution data~\cite{liu2021understanding,chiang2021you}. 
The misplaced trust in the AI systems can lead to misuse of the systems.
% But such trust can be fragile. 
Several empirical studies~\cite{tolmeijer2021second} have shown that once users realize the AI system errs or performs worse than expected, their trust in the AI system can be violated, %In the extreme case, it can 
even resulting in the disuse of the AI system. 
Both the misuse and disuse of the AI system hinder optimal human-AI collaboration. 
}

\revise{%To calibrate user trust in the AI system, 
To address such concerns, researchers have explored how to help users calibrate their trust in the AI system. %researchers proposed 
Different techniques to help users realize the trustworthiness of the AI system have been proposed~\cite{kaur2022trustworthy,rechkemmer2022confidence,ma2023should}. 
For example, increasing the transparency of AI systems by providing confidence scores~\cite{Zhang-FAT-2020}, explanations~\cite{wang2021explanations}, trustworthiness cues~\cite{liao2022designing}, and uncertainty communication~\cite{Sunnie-FAccT-2024}. 
However, the actual trustworthiness of the AI system does not always align with user perception. 
As found by \citet{banovic2023being}, untrustworthy AI systems can deceive end users to gain their trust. 
Another example is that users can develop an illusion of explanatory depth brought by explainable AI techniques~\cite{chromik2021think}, which leads to uncalibrated trust in the AI system. 
Even if users have indicated trust in the AI system, they may turn to rely more on themselves in final decision-making. 
The reasons are complex, and many factors, such as accountability concerns~\cite{lima2021human,tolmeijer2022capable} and cognitive bias~\cite{he2023knowing}, may affect user reliance behaviors. %Much research effort is required to further our understanding of trust calibration in AI systems.
% On the other hand, research has dived deep into calibrating user trust in AI systems.
}

While trust calibration is an important goal in human-AI collaboration, it may be not enough to ensure complementary team performance. 
Through empirical user studies with different confidence levels of AI predictions, Zhang \etal~\cite{Zhang-FAT-2020} found that ``trust calibration alone is not sufficient to improve AI-assisted decision making''. 
To achieve optimal human-AI collaboration, humans and AI systems need to play complementary roles~\cite{hemmer2021human,hemmer2024complementarity}, and humans need to know when they should adopt AI assistance. 
In other words, humans should rely on AI advice when AI systems are correct and outperform them, and override AI advice when AI systems are incorrect or less capable than humans. 
Such user reliance patterns are denoted as \textit{appropriate reliance}~\cite{schemmer2022should,schemmer2023appropriate}, which is the key to
achieving complementary team performance. 

The main issues that lead to sub-optimal human-AI collaboration are: under-reliance (\ie disuse AI assistance when AI systems outperform humans) and over-reliance (\ie misuse AI assistance when AI systems are wrong or perform worse than humans)~\cite{schemmer2022should}. 
Users with an uncalibrated trust in the AI system can be easily misled to disuse or misuse AI systems~\cite{jacovi2021formalizing}. 
Researchers have proposed various interventions to promote appropriate reliance~\cite{he2023knowing,Lu-CHI-2021,lu2024does,chiang2021you,Chiang-IUI-2022} and calibrate user trust in AI systems~\cite{buccinca2021trust,Zhang-FAT-2020}.  
For example, explainable AI methods have been shown to help reduce over-reliance~\cite{vasconcelos2023explanations} and under-reliance~\cite{wang2021explanations} in different scenarios albeit with little consistency across contexts. 
Another example is tutorial interventions, which have shown effectiveness in user onboarding~\cite{lai2020chicago}, mitigating cognitive biases~\cite{he2023knowing} and developing AI literacy~\cite{Chiang-IUI-2022}. 
For a more comprehensive overview of interventions to facilitate trust calibration and appropriate reliance, readers can refer to ~\cite{lai2021towards,eckhardt2024survey,mehrotra2024systematic,kahr2024understanding}.

\revise{LLM agents~\cite{wang2024survey} have gained much popularity in recent years, distinguishing them from most prior AI systems. 
They can communicate through conversation, plan logically, and can be built to leverage powerful external tools to achieve complex functions.
% The interaction between human and AI systems is relatively limited. 
% Users mostly develop trust and reliance on the AI systems via provided information cues (\eg, explanations and references). 
While trust and reliance have been extensively analyzed in existing human-AI collaboration literature, it is still unclear how users trust and rely on AI systems powered by LLM agents. 
% This work addressed this gap with empirical studies to analyze user trust and reliance on the plan-then-execute LLM agents. 
In our work, calibrated trust is adopted as an important goal for human-AI collaboration in the planning and execution stage. 
Meanwhile, users are expected to fix potential errors in the planning and execution stages, reflecting their reliance on the AI system. 
Our work can substantially advance the understanding of trust and reliance on plan-then-execute LLM agents.
}

% \paratitle{position our work}.

\subsection{Task Support with LLMs and LLM Agents}
\label{sec-rel-LLM-agent}
% \glcomment{Do we need to give more context about LLM Agent? It seems more technical}
LLMs and LLM agents bring new opportunities and challenges to human-AI collaboration~\cite{bommasani2021opportunities}. 
%On the one hand, based on the text generation capability of LLMs, it would be possible for humans to directly give text responses and communicate with any AI systems that take LLMs as their backbone. 
%To this end, 
%LLMs and LLM agents bring new opportunities for more flexible and natural interactions with humans. 
% Meanwhile, the natural language understanding and learning capabilities also enable LLMs to evolve with user feedback. 
It is evident that their generation capabilities can help reduce the cognitive effort from humans. %On the other hand, the LLMs also bring new challenges like dealing with 
But LLMs are also riddled with challenges such as hallucination~\cite{ji2023survey} (\ie generated text seems plausible but is factually incorrect). 
% Fatal errors can be made if users get misled by such persuasive technology, resulting in unaffordable costs (\eg medical diagnosis and financial decisions).
Failure to handle such issues may bring fatal errors with unaffordable costs depending on the context (\eg medical diagnosis). 
%As our work is within the scope of human-AI collaboration, users can refer to corresponding literature reviews to obtain more technical background about LLMs~\cite{zhao2023survey} and LLM agents~\cite{xi2023rise,wang2024survey}.

\begin{figure*}[h]
    \centering
    \includegraphics[width=0.75\textwidth]{figures/Screenshot-planning.png}
    \caption{Screenshot of user-involved planning interface.} 
    \label{fig:planning}
    \Description{Screenshot of user-involved planning interface. At the top, we show the task description along with three buttons: show potential actions, plan edit instruction, and add one step. At the bottom, we show a step-wise plan for setting alarms. Users can click these buttons to achive the function we described in the user-involved planning to edit the plan.}
\end{figure*}

% In recent years, LLMs have gained an explosion of popularity among academia and the industry community. 
Due to the capability of generating coherent, knowledgeable, and high-quality responses to diverse human input~\cite{wei2022emergent}, a wide community of human-computer interaction researchers has paid attention to large language models~\cite{liao2023ai}.
% With large volumes of data, large language models can obtain capabilities to help humans in writing. 
% Ideally, any complex task that can be modularized into a chain of different functions can also be achieved with chaining LLMs.
% LLMs can be used to generate dynamic user interface ~\cite{wang2023enabling}, support scientific writing~\cite{shen2023convxai}, and obtain high-quality data annotation~\cite{gilardi2023chatgpt,wang2024human,he2024if}. 
Researchers have actively explored how LLMs can assist users in various tasks like data annotation~\cite{wang2024human,he2024if}, programming~\cite{omidvar2024evaluating}, scientific writing~\cite{shen2023convxai}, and fact verification~\cite{si2024large}. 
All the above functions can be achieved with elaborate prompt engineering using a single LLM. 
By chaining multiple LLMs with different functions, humans can customize task-specific workflows to solve complex tasks~\cite{wu2022ai}. 
Apart from obtaining answers with a one-shot text generation, LLMs also provide convenient conversational interactions. 
Through empirical studies, such conversational interactions have been shown to be effective in human-AI collaboration with multiple applications, such as decision making~\cite{slack2023explaining,lin2024decision,ma2024towards}, scientific writing~\cite{shen2023convxai}, and mental health support~\cite{sharma2023human}. 
With the growing popularity of LLMs, more and more humans have begun to adopt LLMs (\eg ChatGPT) to boost their work efficiency and productivity %in their everyday work
~\cite{zhao2023survey}.

% Meanwhile, researchers also analyzed different system factors associated with LLMs. 
% For example, Kim \etal~\cite{Sunnie-FAccT-2024} found that the LLM's uncertainty Expression can decrease user trust in wrong AI advice, which helps reduce over-reliance.


LLM agents have been shown to have good planning, memory, and toolkit usage capabilities~\cite{xi2023rise, wang2024survey}. 
% External toolkits greatly increase the impact of human-AI collaboration on the real world. 
When suitable toolkits are provided, LLM agents can readily generate a task-specific plan and solve the tasks using toolkits. 
Attracted by the promise of LLM agents, there have been some early explorations~\cite{geissler2024concept,zheng2023synergizing,zhang2024s} of adopting them in human-AI collaboration contexts. 
%To our knowledge, only a few works~\cite{geissler2024concept,zheng2023synergizing,zhang2024s} have explored human-AI collaboration with LLM agents. 
% These works mainly analyzed how LLM agents can serve specific use cases (\eg design creation~\cite{geissler2024concept}) or conducted structured interviews to obtain expert insights~\cite{zhang2024s,zheng2023synergizing}. 
These works were mostly analyzed in specific use cases (\eg design creation~\cite{geissler2024concept}). 
% There is a lack of empirical studies on user trust and team performance in collaboration with LLM agents.
It is unclear how user trust and team performance are affected by user interactions with LLM agents in a sequential decision making setup (\ie solving a task by executing a sequence of actions) where users can be in control of the execution. %\glcomment{It is the only mention of sequential decision making. In my current framing, I try to give a very specific definition of what type of tasks we are focusing on. I mainly motivate LLM agents to provide daily assistance and facilitate daily life. Do you think we want to highlight sequential decision making?}
To fill this research gap and advance our understanding of user control over LLM agents, we carried out a quantitative empirical study. %to obtain empirical evidence.



% \ujcomment{- How do we position our work in this context?}

% \subsection{Human-agent collaboration}
% \glcomment{More traditional work of human-autonomous agent collaboration? Is it too far from our topic?}

% \subsection{User Trust and Reliance in Human-AI Collaboration}
% \label{sec-rel-trust}

\section{Conclusion and Discussion}
\label{sec:conclusion}

We introduce a novel framework for augmenting \emph{any} lossy compressor to preserve the contour tree of a volumetric dataset while maintaining a user-specified global error bound. 
To do this, our framework first imposes topology-informed upper and lower bounds on each data point. 
It then progressively tightens those bounds until the contour tree is preserved. 
We also introduce a novel encoding scheme that efficiently stores individual points with variable precision and maintains these upper and lower bounds. 
When our framework is used to augment state-of-the-art lossy compressors, it is shown to preserve the contour trees of various scientific datasets.
Our augmented compressors also achieve higher compression ratios and reconstruction quality than those obtained by existing topology-preserving compressors in comparable or faster time.
Our framework will benefit from any advancement with lossy compression since it can be used to augment increasingly effective lossy compressors to achieve better topology-preserving compression. 

Our framework is not without limitations. The compression times are longer than the base compressors. This difference gets worse as the topological complexity of the data increases.
However, in some use-cases, topological preservation is preferable to run time.
Regardless, our framework would benefit from more efficient or parallel implementations for the contour/merge tree computation and the encoding scheme. 


%%
%% The acknowledgments section is defined using the "acks" environment
%% (and NOT an unnumbered section). This ensures the proper
%% identification of the section in the article metadata, and the
%% consistent spelling of the heading.
% \begin{acks}
% To Robert, for the bagels and explaining CMYK and color spaces.
% \end{acks}

%%
%% The next two lines define the bibliography style to be used, and
%% the bibliography file.
\bibliographystyle{ACM-Reference-Format}
\bibliography{sample-base}


%%
%% If your work has an appendix, this is the place to put it.

% \appendix


\end{document}
\endinput
%%
%% End of file `sample-sigconf.tex'.
