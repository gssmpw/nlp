%% This is file `sample-sigconf.tex',
%% generated with the docstrip utility.
%%
%% The original source files were:
%%
%% samples.dtx  (with options: `all,proceedings,bibtex,sigconf')
%% 
%% IMPORTANT NOTICE:
%% 
%% For the copyright see the source file.
%% 
%% Any modified versions of this file must be renamed
%% with new filenames distinct from sample-sigconf.tex.
%% 
%% For distribution of the original source see the terms
%% for copying and modification in the file samples.dtx.
%% 
%% This generated file may be distributed as long as the
%% original source files, as listed above, are part of the
%% same distribution. (The sources need not necessarily be
%% in the same archive or directory.)
%%
%%
%% Commands for TeXCount
%TC:macro \cite [option:text,text]
%TC:macro \citep [option:text,text]
%TC:macro \citet [option:text,text]
%TC:envir table 0 1
%TC:envir table* 0 1
%TC:envir tabular [ignore] word
%TC:envir displaymath 0 word
%TC:envir math 0 word
%TC:envir comment 0 0
%%
%%
%% The first command in your LaTeX source must be the \documentclass
%% command.
%%
%% For submission and review of your manuscript please change the
%% command to \documentclass[manuscript, screen, review]{acmart}.
%%
%% When submitting camera ready or to TAPS, please change the command
%% to \documentclass[sigconf]{acmart} or whichever template is required
%% for your publication.
%%
%%
\documentclass[sigconf]{acmart}

\usepackage{amsmath}
% \usepackage{amssymb}
\usepackage{bm}
\usepackage{algorithm}
\usepackage{algpseudocode}
\usepackage{amsmath}
\usepackage{multirow}
\usepackage{indentfirst}
\usepackage{subfigure}
\usepackage{epsfig}
\usepackage{epstopdf}
\usepackage{graphicx}
\usepackage{url}
\usepackage{xspace}
\usepackage{booktabs}
\usepackage{subeqnarray}
\usepackage{subcaption}
\usepackage{color}
\newtheorem{myDef}{Definition}
\newtheorem{exmp}{Example}

\newcommand{\paratitle}[1]{\vspace{1.5ex}\noindent\textbf{#1}}
\newcommand{\ie}{\emph{i.e.,}\xspace}
\newcommand{\aka}{\emph{a.k.a.,}\xspace}
\newcommand{\eg}{\emph{e.g.,}\xspace}
\newcommand{\etal}{\emph{et al.}\xspace}
\newcommand{\ignore}[1]{}

%%
%% \BibTeX command to typeset BibTeX logo in the docs
\AtBeginDocument{%
  \providecommand\BibTeX{{%
    Bib\TeX}}}

%% Rights management information.  This information is sent to you
%% when you complete the rights form.  These commands have SAMPLE
%% values in them; it is your responsibility as an author to replace
%% the commands and values with those provided to you when you
%% complete the rights form.
% \setcopyright{acmlicensed}
% \copyrightyear{2018}
% \acmYear{2018}
% \acmDOI{XXXXXXX.XXXXXXX}

%% These commands are for a PROCEEDINGS abstract or paper.
% \acmConference[Conference acronym 'XX]{}{June 03--05,
  % 2018}{Woodstock, NY}
%%
%%  Uncomment \acmBooktitle if the title of the proceedings is different
%%  from ``Proceedings of ...''!
%%
%%\acmBooktitle{Woodstock '18: ACM Symposium on Neural Gaze Detection,
%%  June 03--05, 2018, Woodstock, NY}
% \acmISBN{978-1-4503-XXXX-X/18/06}


%%
%% Submission ID.
%% Use this when submitting an article to a sponsored event. You'll
%% receive a unique submission ID from the organizers
%% of the event, and this ID should be used as the parameter to this command.
%%\acmSubmissionID{123-A56-BU3}

%%
%% For managing citations, it is recommended to use bibliography
%% files in BibTeX format.
%%
%% You can then either use BibTeX with the ACM-Reference-Format style,
%% or BibLaTeX with the acmnumeric or acmauthoryear sytles, that include
%% support for advanced citation of software artefact from the
%% biblatex-software package, also separately available on CTAN.
%%
%% Look at the sample-*-biblatex.tex files for templates showcasing
%% the biblatex styles.
%%

%%
%% The majority of ACM publications use numbered citations and
%% references.  The command \citestyle{authoryear} switches to the
%% "author year" style.
%%
%% If you are preparing content for an event
%% sponsored by ACM SIGGRAPH, you must use the "author year" style of
%% citations and references.
%% Uncommenting
%% the next command will enable that style.
%%\citestyle{acmauthoryear}


%%
%% end of the preamble, start of the body of the document source.
\begin{document}

%%
%% The "title" command has an optional parameter,
%% allowing the author to define a "short title" to be used in page headers.
\title{Holistically Guided Monte Carlo Tree Search for Intricate Information Seeking}


%%
%% The "author" command and its associated commands are used to define
%% the authors and their affiliations.
%% Of note is the shared affiliation of the first two authors, and the
%% "authornote" and "authornotemark" commands
%% used to denote shared contribution to the research.
% \author{Ben Trovato}
% \authornote{Both authors contributed equally to this research.}
% \email{trovato@corporation.com}
% \orcid{1234-5678-9012}
% \author{G.K.M. Tobin}
% \authornotemark[1]
% \email{webmaster@marysville-ohio.com}
% \affiliation{%
%   \institution{Institute for Clarity in Documentation}
%   \city{Dublin}
%   \state{Ohio}
%   \country{USA}
% }

\author{Ruiyang Ren$^*$}\thanks{$^*$Equal Contributions.}
\email{reyon_ren@outlook.com}
\affiliation{%
  \institution{Gaoling School of Artificial Intelligence, Renmin University of China}
  \city{Beijing}
  \country{China}}

\author{Yuhao Wang$^*$}
\email{yh.wang500@outlook.com}
\affiliation{%
  \institution{Gaoling School of Artificial Intelligence, Renmin University of China}
  \city{Beijing}
  \country{China}}

\author{Junyi Li$^*$}
\email{junyi_cs@nus.edu.sg}
\affiliation{%
  \institution{National University of Singapore}
  \country{Singapore}}

\author{Jinhao Jiang}
\email{jiangjinhao@ruc.edu.cn}
\affiliation{%
  \institution{Gaoling School of Artificial Intelligence, Renmin University of China}
  \city{Beijing}
  \country{China}}

\author{Wayne Xin Zhao$^\dag$}\thanks{$^\dag$Corresponding Authors.}
\email{batmanfly@gmail.com}
\affiliation{%
  \institution{Gaoling School of Artificial Intelligence, Renmin University of China}
  \city{Beijing}
  \country{China}}

\author{Wenjie Wang$^\dag$}
\email{wenjiewang96@gmail.com}
\affiliation{%
  \institution{University of Science and Technology of China}
  \city{Hefei}
  \country{China}}


\author{Tat-Seng Chua}
\email{dcscts@nus.edu.sg}
\affiliation{%
  \institution{National University of Singapore}
  % \city{Beijing}
  \country{Singapore}}

\renewcommand{\shortauthors}{Ren et al.}

%%
%% The abstract is a short summary of the work to be presented in the
%% article.
\begin{abstract}

In the era of vast digital information, the sheer volume and heterogeneity of available information present significant challenges for intricate information seeking. Users frequently face multi-step web search tasks that involve navigating vast and varied data sources. This complexity demands every step remains comprehensive, accurate, and relevant. However, traditional search methods often struggle to balance the need for localized precision with the broader context required for holistic understanding, leaving critical facets of intricate queries underexplored.
In this paper, we introduce an LLM-based search assistant that adopts a new information seeking paradigm with holistically guided Monte Carlo tree search (HG-MCTS). We reformulate the task as a progressive information collection process with a knowledge memory and unite an adaptive checklist with multi-perspective reward modeling in MCTS. 
The adaptive checklist provides explicit sub-goals to guide the MCTS process toward comprehensive coverage of complex user queries. Simultaneously, our multi-perspective reward modeling offers both exploration and retrieval rewards, along with progress feedback that tracks completed and remaining sub-goals, refining the checklist as the tree search progresses. By striking a balance between localized tree expansion and global guidance, HG-MCTS reduces redundancy in search paths and ensures that all crucial aspects of an intricate query are properly addressed.
Extensive experiments on real-world intricate information seeking tasks demonstrate that HG-MCTS acquires thorough knowledge collections and delivers more accurate final responses compared with existing baselines.

\end{abstract}

%%
%% The code below is generated by the tool at http://dl.acm.org/ccs.cfm.
%% Please copy and paste the code instead of the example below.
%%
% \begin{CCSXML}
% <ccs2012>
%  <concept>
%   <concept_id>00000000.0000000.0000000</concept_id>
%   <concept_desc>Do Not Use This Code, Generate the Correct Terms for Your Paper</concept_desc>
%   <concept_significance>500</concept_significance>
%  </concept>
%  <concept>
%   <concept_id>00000000.00000000.00000000</concept_id>
%   <concept_desc>Do Not Use This Code, Generate the Correct Terms for Your Paper</concept_desc>
%   <concept_significance>300</concept_significance>
%  </concept>
%  <concept>
%   <concept_id>00000000.00000000.00000000</concept_id>
%   <concept_desc>Do Not Use This Code, Generate the Correct Terms for Your Paper</concept_desc>
%   <concept_significance>100</concept_significance>
%  </concept>
%  <concept>
%   <concept_id>00000000.00000000.00000000</concept_id>
%   <concept_desc>Do Not Use This Code, Generate the Correct Terms for Your Paper</concept_desc>
%   <concept_significance>100</concept_significance>
%  </concept>
% </ccs2012>
% \end{CCSXML}

% \ccsdesc[500]{Do Not Use This Code~Generate the Correct Terms for Your Paper}
% \ccsdesc[300]{Do Not Use This Code~Generate the Correct Terms for Your Paper}
% \ccsdesc{Do Not Use This Code~Generate the Correct Terms for Your Paper}
% \ccsdesc[100]{Do Not Use This Code~Generate the Correct Terms for Your Paper}

%%
%% Keywords. The author(s) should pick words that accurately describe
%% the work being presented. Separate the keywords with commas.
% \keywords{Knowledge Boundary; Retrieval Augmentation; Large Language Models}
%% A "teaser" image appears between the author and affiliation
%% information and the body of the document, and typically spans the
%% page.


% \received{20 February 2007}
% \received[revised]{12 March 2009}
% \received[accepted]{5 June 2009}

%%
%% This command processes the author and affiliation and title
%% information and builds the first part of the formatted document.
\maketitle


\section{Introduction}

In real-world web search systems, addressing an information seeking task often requires retrieving and organizing information from diverse online sources. This task that we term \emph{intricate information seeking} presents a persistent and significant challenge in the field of information retrieval~\cite{strohman2005optimization, talmor2018web}. Unlike conventional single-step search, where a user query typically seeks isolated information, intricate information seeking involves integrating multiple pieces of information across various sources to formulate a comprehensive and accurate final response.
The complexity of this process is further amplified by the necessity to maintain consistency across retrieval steps, especially when user queries encompass multifaceted tasks or require extensive background knowledge. For example, responding to a complex query such as ``\emph{the economic, environmental, and social impacts of the adoption of renewable energy in developing countries}'' entails sourcing multiple relevant documents and synthesizing them into a comprehensive answer, including analyzing economic benefits and costs, assessing environmental sustainability, evaluating social implications.

Typically, driven by the ongoing evolution of large language models~(LLMs)~\cite{zhao2023survey}, a variety of existing methods aim to facilitate multi-step or complex retrieval either by heuristically decomposing the query or by iteratively refining the query through incremental optimization of intermediate outputs~\cite{yao2023react, asaiself}. For instance, some studies adopt planning strategies by decomposing a user query into sub-queries based on surface-level cues with the LLM's internal knowledge~\cite{xu2024search, reddy2024infogent}, while others employ tailored mechanisms~(\eg chain-of-thought reasoning~\cite{wei2022chain} and continuous feedback loops~\cite{shinn2024reflexion}) to progressively align intermediate reasoning steps with the final information seeking goal. Although these methods have shown promise in improving multi-step retrieval quality, they are susceptible to cascading errors, where inaccuracies or omissions in earlier sub-queries can propagate through subsequent steps. 

Inspired by the effectiveness of Monte Carlo Tree Search (MCTS) applied in complex reasoning tasks such as mathematic and code problems~\cite{alphago, YeLKAG21}, we consider incorporating MCTS into intricate information seeking scenarios to help find the optimal retrieval solution. However, two primary challenges emerge when applying MCTS to the task. First, sub-queries generated for expanding the search tree are unbounded at each step: each multifaceted intermediate task can branch into numerous investigative directions, causing the tree search space to grow exponentially. Second, the inherent characteristics of MCTS lead to a focus on local exploration~\cite{browne2012survey}, which may lead to omissions or solecism in the acquired information~\cite{_wiechowski_2022}. Specifically, (1) MCTS node selection relies on local statistics (\eg number of visits and reward accumulation), which are aggregated from limited exploration and lack a holistic understanding of the global information seeking objectives; and (2) MCTS explores the search tree by expanding nodes around the currently selected branch, while its rollout strategies are typically random or heuristic-based, making global optimality difficult to guarantee. Consequently, MCTS risks overlooking pertinent sub-tasks or prematurely converging on suboptimal search paths.


\begin{figure}
    \centering
    \includegraphics[width=0.95\linewidth]{pic/intro.pdf}
    \caption{Illustration of the pitfalls in handling intricate queries. Typical reasoning methods with web search often collect non-comprehensive documents (left), while HG-MCTS can effectively capture all necessary documents (right).}
    \label{fig:intro}
\end{figure}


To address these challenges, we propose a novel framework that incorporates MCTS into intricate information seeking, while simultaneously mitigating its inherent limitations through global guidance and multi-perspective feedback. 
Concretely, we reformulate the task as a progressive information collection process with a knowledge memory. Based on this, we propose \emph{holistically guided MCTS~(HG-MCTS)} that introduces an \emph{adaptive checklist} as a global guidance with a set of designated sub-goals.
This adaptive checklist counters the exponential growth of sub-queries by focusing the MCTS algorithm on only those branches aligned with key facets of the information need, thereby alleviating aimless expansions that could arise and enforcing a more targeted traversal of the search space, which can also be updated during the MCTS process. In parallel, we incorporate \emph{multi-perspective reward modeling} that provides both quantitative and qualitative reward signals with the checklist, allowing MCTS to incorporate a more holistic perspective on exploration. This reward modeling furnishes not just numerical indicators of exploration and retrieval quality but also textual feedback outlining which sub-goals have been addressed and which remain unsolved after node exploration. As a result, MCTS moves beyond its conventional reliance on local statistics, thereby minimizing the risk of prematurely converging on suboptimal paths and broadening its understanding of overarching information seeking objectives. Figure~\ref{fig:intro} illustrates a comparison between our method and the typical information seeking method from the retrieval comprehension perspective. 
Through this synergy, our approach preserves the inherent capability of MCTS for dynamic exploration while strengthening its capacity to incorporate newly acquired knowledge snippets. Our method methodically balances thoroughness and focus, ensuring comprehensive coverage of all relevant information while avoiding redundant or tangential searches. 


Our main contributions are summarized as follows:
\begin{itemize}
    \item We introduce a new information seeking paradigm \emph{HG-MCTS} based on a progressive information collection process with knowledge memory, which integrates an adaptive checklist for holistic sub-goal guidance in MCTS progress, thereby enabling more targeted exploration in multi-step retrieval.
    \item We propose a \emph{multi-perspective reward modeling} strategy that provides both quantitative metrics and qualitative feedback in HG-MCTS, which fosters a richer, step-wise evaluation for the value of new expanded nodes.
    \item We demonstrate how these innovations can be seamlessly integrated to improve both the efficiency and the thoroughness of large-scale web retrieval. Beyond immediate applications in question answering and knowledge-intensive search, our findings offer deeper insights into the design of more interpretable, flexible, and resilient retrieval systems.
\end{itemize}

\section{Technical Background}
\label{sec:background}

\subsection{Merge Tree and Contour Tree}
\label{sec:merge-and-contour-tree}
\para{Merge Tree.} 
Let $f:\X \rightarrow \R$ be a continuous scalar field defined on a simply connected domain $\X$. 
The \emph{sublevel set} of $f$ at a threshold $t \in \R$ is defined as 
$\X_t = f^{-1}(-\infty,t] := \{ x \in X \mid f(x) \leq t \}$. 
The \emph{merge tree} of $f$ tracks when (connected) components of $\X_t$ appear and merge as $t$ increases. 
$\X_t$ evolves from being an empty set to contain components surrounding various local minima; these components then merge into one other until eventually there is only a single component.
Leaves of the merge tree correspond to local minima, and interior nodes correspond to saddles where components merge. 
\Cref{fig:contour-tree}(A) and (C) visualize a scalar field and its merge tree.
Formally, we define an equivalence relation $\sim$ on $\X$. We say that $x \sim y$ if and only if $f(x) = f(y)=t$ and $x$ belongs to the same component of $\X_t$ as $y$. The merge tree of $f$ is defined by the quotient space $\X/{\sim}$. 

The merge tree of $f$ defined above is sometimes referred to as the \emph{join tree}, whereas the merge tree of $-f$ is called a \emph{split tree} (see~\cref{fig:contour-tree}(B) for its visualization). The merge tree naturally induces a segmentation of the domain. Let $\phi$ be the canonical map that maps each $x \in \X$ to its equivalence class $[x]$ under $\sim$. Then for each edge $e$ of the contour tree, $\phi^{-1}(e)$ is a monotonic region in $\X$. The inverse image of each edge partitions the domain, which is called the merge-tree-induced segmentation. See \cref{fig:contour-tree}(D) (cf.,~(E)) for a split-tree-induced segmentation.

\para{Contour Tree.} 
The \emph{level set} of $f$ at a threshold $t \in \R$ is $f^{-1}(t):= \{ x \in X \mid f(x) = t \}$.
Each component of $f^{-1}(t)$ is called a \emph{contour}. 
A \emph{contour tree} tracks the relations among contours as $t$ increases. 
Analogous to a merge tree, as $t$ increases, components of $f^{-1}(t)$ appear at local minima, disappear at local maxima, and join or split at saddles. 
Formally, we define another equivalence relation $\sim$ on $\X$, where $x \sim y$ if and only if $f(x) = f(y)$ and $x$ belongs to the same contour as $y$. 
The contour tree is the quotient space $T=T(\X,f):= \X / \sim$. 
There is a 1-1 correspondence between the local extrema of $\X$ and the leaves of $T$. The interior nodes of $T$ correspond to a subset of the saddle points of $\X$ \cite{reeb1946points}. \cref{fig:contour-tree}(F) visualizes the contour tree. Similar to the merge tree, we can define a contour-tree-induced segmentation.

The classic algorithm for computing the contour tree~\cite{carr2003computing} combines join and split trees together to form a contour tree, and comes with a state-of-the-art multi-core implementation~\cite{gueunet2017task}.
In this paper, we compute the contour tree of the input data to determine the initial point-wise error bound. We use the algorithm of Gueunet et al.~\cite{gueunet2017task} (with an in-house implementation) to compute join and split trees, and the algorithm of Carr et al.~\cite{carr2003computing} to combine them into the contour tree. 

\begin{figure}[!ht]
    \centering
    \includegraphics[width=\linewidth]{fig-ct_example.png}
    \vspace{-6mm}
    \caption{(A) Visualizing the graph of a 2D scalar field $f$ defined on a square domain. Local minima are in blue, saddles are in white, and local maxima are in red. (B) Split tree of f.  (C) Merge tree of $f$ (a.k.a.,~join tree). (D) The domain is colored according to split-tree-induced segmentation. (E) Split tree is colored according to the segmentation in (D). (F) Contour tree of $f$. (G) The graph of $f$ after its persistence simplification where a peak is removed. (H) Split tree of $f$ after persistence simplification. Edge $ab$ is removed. (F) Contour tree of $f$ after the merge tree in (B) has been simplified. }
    \label{fig:contour-tree}
    \vspace{-6mm}
\end{figure}

The algorithm by Gueunet et al.~\cite{gueunet2017task} constructs the merge tree one edge at a time. First, starting from each minimum $m$, the algorithm visits points surrounding $m$ in an increasing order until a saddle $s$ is reached. The edge $ms$ is discovered and added to the merge tree; in other words, $m$ is \emph{grown} during this process.  
For each saddle $s$, once all edges that terminate at $s$ have been discovered, $s$ is grown until some new saddle $s'$ is reached, and the edge $ss'$ is discovered and added to the merge tree. This process repeats until the merge tree has been completely discovered. 
To grow from a minimum or a saddle, the points surrounding that minimum or saddle are stored in a heap to ensure that they are processed in an increasing order. The use of heaps presents the main performance bottleneck of the algorithm.

\para{Persistence simplification.} 
In practice, real-world data typically contain noise that creates many small branches in the merge or contour tree, where persistence simplification~\cite{edelsbrunner2001hierarchical} can be used to eliminate these small branches and thereby separate topological features from noise. 
In the context of merge trees, ordinary persistent homology pairs a local extremum (i.e.,~a peak or a valley in \cref{fig:contour-tree}(A)) with a nearby saddle and assigns the pair a value of \emph{persistence}, which describes the scale at which the pair disappears via a perturbation to the function. The persistence is equal to the absolute difference in function value between an extremum and its paired saddle. A function $f$ is simplified by perturbing the function values in order to cancel pairs of critical points below a certain persistence threshold $\varepsilon$. See \cref{fig:contour-tree}(G) for an example. The cancellation of one pair of critical points of $f$ corresponds to a branch being removed from the merge tree, giving rise to its simplification.
The simplified contour tree is more complex \cite{hristov2021w}, but it can be computed by combining the simplified join and split trees. 

In~\cref{fig:contour-tree}, (G) depicts the graph of a function after persistence simplification. For the split tree shown in (B), assuming a persistence threshold of $\varepsilon \geq |f(a)-f(b)|$, the edge $ab$ will be removed after a persistence simplification at $\varepsilon$. As a result, node $c$ is directly connected to node $d$ in the split tree (H), and the join tree in (C) remains unchanged. (I) depicts the contour tree produced by combining the simplified split tree in (H) with the join tree in (C).

\subsection{A Review on TopoSZ}
\label{sec:TopoSZ}

Our framework builds upon a few ingredients from TopoSZ~\cite{yan2023toposz}. 
TopoSZ, in turn, modifies the pipeline from the error-bounded lossy compressor SZ version 1.4 \cite{tao2017significantly}. 

Let $f$ represent the input scalar field, and $f'$ be the reconstructed scalar field (after compression and decompression). 
Let $T$ be the contour tree of $f$ and $T_\varepsilon$ the persistence simplified contour tree at a threshold of $\varepsilon$. Let $T'$ and $T'_\varepsilon$ be defined analogously for $f'$.

SZ1.4 allows the user to specify $\xi$, a pointwise error bound during compression. In turn, there are two user-defined parameters in TopoSZ: a persistence threshold $\varepsilon$, and a pointwise error bound $\xi$. Unlike TopoQZ, TopoSZ does not require that $\varepsilon < \xi$. 
TopoSZ guarantees the preservation of the persistence simplified contour tree during compression while maintaining the pointwise error bound. 
That is, it guarantees that $T_\varepsilon = T'_\varepsilon$, and $|f(x)-f'(x)| \leq \xi$ for each $x \in \X$.

\para{Linear-scaling quantization.} 
SZ version 1.4 introduces a linear-scaling quantization technique to ensure that a strict absolute error bound $\xi$ is maintained. This technique is implemented in TopoSZ.

For each point $x \in \X$ with a ground truth value $f(x)$, an initial guess for its value $g(x)$ (e.g., from a Lorenzo predictor; see below) is shifted by an integer multiple of $2\xi$ to obtain a new value $f'(x)$ such that $|f'(x) - f(x)| \leq \xi$. 

This process can be conceptualized as follows: divide the real line into intervals of length $2\xi$, where one interval is centered on $g(x)$. 
The compressor then calculates how many intervals to shift $g(x)$, so that it can assign a value to $f'(x)$ that is a distance less than $\xi$ from $f(x)$. By construction, if $f(x)$ lies in an interval of length $2\xi$ centered on $f'(x)$, then $|f(x)-f'(x)| \leq \xi$. 
This process is illustrated in \cref{fig:linear-scaling-quantization}.
\begin{figure}[!ht]
  \vspace{-3mm}
  \centering
  \includegraphics[width=\columnwidth]{fig-linear-scaling-quantization.pdf}
  \vspace{-8mm}
  \caption{A standard implementation of a linear-scaling quantization.}
  \label{fig:linear-scaling-quantization}
  \vspace{-4mm}
\end{figure}
\noindent Following this construction, each $x \in \X$ is assigned an integer $n_x$ (corresponding to how many intervals $g(x)$ is shifted) such that $f'(x) = g(x) + 2\xi n_x$. These quantization numbers $\{n_x\}$ are encoded and stored in the compressed file.

If the distribution of $\{n_x\}$ has low entropy (e.g.,~if $\{n_x\}$ are mostly zeros), then the quantization numbers can be compressed to small size using an entropy-based compression algorithm, such as Huffman coding. 
More accurate predictions $g(x)$ generally lead to $\{n_x\}$ with lower entropy.

\para{False Cases.} Yan et al.~\cite{yan2023toposz} introduced three types of false cases to quantify the level of contour tree preservation: false positives, false negatives, and false types, which are illustrated in \cref{fig:false-cases}.
A false positive occurs when a new edge appears in the contour tree of the reconstructed data that does not exist in the same position of the contour tree of the original data. A false negative occurs when an edge of the contour tree from the original data is missing from the contour tree of the reconstructed data. 
A false type occurs when the critical type (maximum, minimum, saddle) of one or both endpoints of an edge of the contour tree does not match between the original and reconstructed data.
TopoSZ focuses on false cases involving extremum-saddle pairs and its algorithm terminates when there are no such false cases. 

\begin{figure}[!ht]
    \vspace{-2mm}
    \centering
    \includegraphics[width=\linewidth]{fig-false-cases.pdf}
    \vspace{-6mm}
    \caption{Three types of false cases. (A) The original contour tree. (B) A false positive: an extra edge is added. (C) A false negative: an edge is missing. (D) A false type: an edge contains a critical point as its endpoint that changes its type.}
    \label{fig:false-cases}
    \vspace{-4mm}
\end{figure}


\para{TopoSZ Pipeline.} The TopoSZ pipeline is as follows:

\para{\underline{Step 1: Upper and lower bound calculation.}}
TopoSZ first applies persistence simplification to $f$ in order to calculate $T_\varepsilon$. For each point $x \in \X$, a lower bound $L(x)$ and an upper bound $U(x)$ are assigned to $x$ according to the contour-tree-induced segmentation.
If $x$ belongs to the segmented region corresponding to edge $ab \in T_\varepsilon$, then $L(x) = \min(f(a),f(b))$ and $U(x) = \max(f(a),f(b))$. The nodes of the contour tree are stored losslessly.

\para{\underline{Step 2: Prediction.}} 
TopoSZ uses a Lorenzo predictor \cite{ibarria2003out} to predict the values of each data point. For each point $x$, the Lorenzo predictor predicts $f(x)$ as a weighted sum of the values from previously predicted points $x'$ satisfying $\|x-x'\|_\infty = 1$. The weights are fixed, and are chosen such that quadratic functions will be perfectly predicted.

\para{\underline{Step 3: Linear-scaling quantization.}} TopoSZ uses linear-scaling quantization with a decreased interval size to ensure that the pointwise upper and lower bounds, as well as the global error bound $\xi$, are maintained for each $x \in \X$. For any $x \in \X$ where no possible quantization code $n_x$ satisfies these conditions, $f(x)$ is stored losslessly.

\para{\underline{Step 4: Iterative upper and lower bound tightening.}} 
If the results from Step 3 do not perfectly preserve the contour tree, that is, if there are false cases presented in the reconstructed data, then the upper and lower bounds are tightened around points corresponding to those false edges, and then Step 3 is repeated. This cycle repeats until there are no false cases. 
%The specifics of this step are reviewed in \cref{sec:topoSZ-detail}.

\para{\underline{Step 5: Lossless compression.}} The numbers from linear-scaling quantization are encoded using Huffman Coding. The relevant information is then stored in a binary file that is further compressed using ZSTD~\cite{collet2018zstandard}.
\section{Method}
\label{sec:method}
We give an overview of our framework in~\cref{sec:method-overview}.
We then describe two novel and technical ingredients in our framework: the logarithmic-scaling quantization (\cref{sec:augment-quantization}) and the progressive upper and lower bound tightening (\cref{sec:augment-tightening}).

\subsection{Overview}
\label{sec:method-overview}

We now describe our framework for augmenting any lossy compressor (called a \emph{base compressor}) to preserve contour trees and maintain strict error bounds. 
Our framework requires two user-specified parameters, a persistence threshold $\varepsilon$ and a pointwise absolute error bound $\xi$. 
It also requires user-specified parameters associated with the specific base compressor being augmented. Our implementation works with rectilinear meshes, and it could easily be modified to work with any simply-connected tetrahedral mesh.

Our framework guarantees that, for any augmented compressor, $T_\varepsilon = T_\varepsilon'$ and $|f(x)-f'(x)| \leq \xi$ for every $x \in \X$. Starting with a standard compressor as the base compressor, we start with a step-by-step overview of our framework. 

\para{\underline{Step 1: Upper and lower bound calculation.}}~We store critical points of the simplified contour tree $T_\varepsilon$ losslessly. We calculate the initial pointwise upper and lower bounds for other point $x \in \X$. The key idea is to locate an edge $ab$ in $T_{\varepsilon}$ whose corresponding range of function values contains $f(x)$. This requires a careful computation using the join and split trees of $T_{\varepsilon}$; see \cref{sec:algorithm-details} for details. 
We let $L(x) = \min(f(a),f(b)) + \zeta$ and $U(x) = \max(f(a),f(b))-\zeta$, where $\zeta = 10^{-5}|f(b)-f(a)|$.
If we allow $x$ to have the same function value as $a$ or $b$, the topology may be altered (e.g., along the boundary of the induced region), resulting in more false cases. Adjusting the error bound by $\zeta$ prevents such issues. We also adjust $L(x)$ and $U(x)$ as needed to ensure that if $L(x) \leq f'(x) \leq U(x)$ then $|f(x)-f'(x)| \leq \xi$. 
 
When computing $T_\varepsilon$, we compute the join and split trees of $f$ and simplify the trees directly with persistence threshold $\varepsilon$. We then combine them to obtain $T_\varepsilon$. During this construction, we track which edge of $T_\varepsilon$ each point $x \in X$ corresponds to. Compared to simplifying the entire scalar field $f$ and then computing the contour tree of the simplified field (like TopoSZ), our strategy leads to equivalent results in less time.

\para{\underline{Step 2: Base compressor.}} 
We apply the base compressor to the input data $f$. 
We compress and then decompress the data to assess changes that need to be made during decompression. 
We refer to the compressed-then-decompressed data as the \emph{intermediate data}.

\para{\underline{Step 3: Logarithmic-scaling quantization.}} 
We introduce a novel quantization technique that respects the pointwise upper and lower bounds imposed in Step 1. 
If possible, the entropy of the quantization numbers $\{n_x\}$ will be identical to that of standard linear-scaling quantization.
However, when linear-scaling quantization cannot produce a prediction for a point $x$ that respects $L(x)$ and $U(x)$, $x$ will be quantized with more precision (i.e.,~$\xi \leftarrow \xi/2$) to satisfy those bounds.

\para{\underline{Step 4: Progressive upper and lower bound tightening.}} 
We introduce a novel technique for calculating adjustments to the intermediate data to guarantee that the contour tree is preserved.
We compute the join and split trees directly. If a false edge is detected during computation, the upper and lower bounds are tightened around points in the segmentation region corresponding to the edge (see \cref{sec:merge-and-contour-tree}). All edges whose growth involved these points are recomputed.
We continue until the join and split trees of the decompressed data match those of the ground truth. We do not compute the contour tree directly as the preservation of the join and split trees guarantees the preservation of the contour tree.

\para{\underline{Step 5: Lossless compression.}} 
We encode the quantization numbers using Huffman coding. The output of the base compressor, the encoded quantization numbers, and any losslessly stored values are written to a binary file which is further losslessly compressed using xz, a general-purpose data compression tool available via {XZ Utils}~\cite{XZUtils}.

\subsection{Logarithmic-Scaling Quantization}
\label{sec:augment-quantization}

We now describe the first novel ingredient in our framework: a variable precision quantization technique that preserves tight pointwise upper and lower bounds. %without significantly compromising the entropy of the overall distribution of quantization numbers. 
For each $x \in \X$, the intermediate data contains an estimated value $g(x)$ for the ground truth value $f(x)$. 
Let $L(x)$ and $U(x)$ denote the lower and upper bounds assigned to $x$.
To ensure that $L(x) \leq f'(x) \leq U(x)$, we assign to each $x \in \X$ a numerator $a_x \in \Z$ and a precision $p_x \in \N$ that indicates the number of iterations. 
Our reconstructed value is 
\begin{equation}
f'(x) = g(x) + \frac{2\xi \cdot a_x}{2^{p_x}}.
\label{eq:fprime-original}
\end{equation}

To calculate each $a_x$ and $p_x$, we first set $p_x=0$. 
We then look for the value of $a_x$ satisfying 
\begin{equation*}
L(x) \leq g(x) + \frac{2\xi \cdot a_x}{2^{p_x}} \leq U(x)
\label{eq:Bounds}
\end{equation*}
such that $|a_x|$ is minimized. If there is no valid value of $a_x$, we increase $p_x$ by $1$ and search again. This process is repeated until a valid $a_x$ is found. If $p_x$ reaches an arbitrary threshold, we stop searching and instead store $f(x)$ losslessly. We set this threshold equal to $11$.

When $p_x = 0$, the above process is the same as the standard linear-scaling quantization, except that we also seek to maintain the upper and lower bounds. 
Each time a linear-scaling quantization fails to identify a valid choice for $a_x$ that yields a value of $f'(x)$ within the upper and lower bounds for $x$, we cut the interval lengths in half by increasing $p_x$ by $1$ and continue searching.
When the interval lengths are smaller, it is more likely that a valid choice of $a_x$ exists. 
It is also possible that during an iteration, multiple valid choices of $a_x$ exist, so we choose the one with the smallest absolute value to minimize the entropy of $\{a_x\}$. 

\begin{figure}[!ht]
    \centering
    \vspace{-2mm}
    \includegraphics[width=\linewidth]{fig-log-scale-quantization.pdf}
    \vspace{-6mm}
    \caption{(A) If $p_x = 0$, there are no valid quantization intervals. (B) Increasing $p_x$ to $1$ allows for a valid quantization interval.}
    \label{fig:log-scale-quantization}
    \vspace{-2mm}
\end{figure}

This process is illustrated in \cref{fig:log-scale-quantization}. 
(A) contains an example where there are no quantization intervals where we can place $f'(x)$ to respect the upper and lower bounds. 
In (B), by raising the precision $p_x$ by 1, the quantization intervals are halved, giving a valid choice for $f'(x)$.

When encoding the data, we store a single quantization number $n_x$ for each $x \in \X$. 
To calculate each $n_x$, we first find the maximum precision $p_m$ used for any single point. The points are assigned the single quantization number $n_x = a_x \cdot 2^{p_m-p_x}$ and the max precision $p_m$ is stored in the compressed output. 
During decompression, the point $x$ is assigned the value 

\begin{equation}
f'(x) = g(x) + \frac{2\xi \cdot n_x}{2^{p_m}}.
\label{eq:fprime}
\end{equation}
Setting $n_x = a_x \cdot 2^{p_m-p_x}$ in Eq.~\eqref{eq:fprime} means that
\begin{equation*}
  g(x) + \frac{2\xi \cdot n_x}{2^{p_m}} = g(x) + \frac{2\xi \cdot a_x \cdot 2^{p_m-p_x}}{2^{p_m}} = g(x) + \frac{2\xi \cdot a_x}{2^{p_x}}.
  \label{eq:logscale}  
\end{equation*}
Therefore, the formulation in Eq.~\eqref{eq:fprime} is equivalent to the original formulation of $f'$ in Eq.~\eqref{eq:fprime-original}.

In comparison with TopoSZ, the above variable precision technique allows us to store fewer points losslessly.
In order to ensure the quantization numbers do not get too large, if any point has a precision greater than $10$ it is stored losslessly. This ensures that $p_m \leq 10$ for all trials.

\subsection{Progressive Upper and Lower Bound Tightening}
\label{sec:augment-tightening}

We now describe the second novel ingredient in our framework, namely, a \emph{progressive error bound tightening} process. 
Specifically, the process computes the join and split trees of the decompressed data. During the computation, it detects false cases, and tightens the upper and lower bounds in the neighborhoods of false cases. The algorithm progresses through merge tree computation, checking the correctness of each edge and tightening when needed, until every edge is correctly preserved.
The process allows us to bypass iteratively recomputing the entire contour tree (in the case of TopoSZ), significantly speeding up the compression process. During the tightening process, we work with merge trees (instead of contour trees), since the persistence of a leaf (local extremum) can be computed from its nearby saddle based on branch decomposition (i.e.,~local information), thereby allowing for our progressive tightening strategy. By contract, computing the persistence of a leaf of a contour tree may require global information from the whole contour tree due to the existence of V and W structures~\cite{hristov2021w}.

We describe this process for the join tree, which works analogously for the split tree. We only consider false cases involving extremum-saddle pairs. 

\para{False case detection}. To detect false cases, we construct $T'$. Doing so allows us to locate mismatches between edges in $T'_\varepsilon$ and those in $T_\varepsilon$.
We construct $T'$ using a modified version of the edge growing procedure from local minima and saddles (see~\cref{sec:merge-and-contour-tree}).
To start, we extract a list of local minima of $f'$ sorted by decreasing function values. Then, proceeding in sorted order, we grow an edge from each local minimum $m$ to a saddle $s$, and check two cases for $s$; see \cref{sec:algorithm-details} for illustrations: 

\underline{Case (I).} If $s$ is unpaired, i.e., $m$ is the first local minimum (among all local minima) whose growth terminates at $s$, then $m$ and $s$ form a persistence pair, with a persistence $p =|f'(s)-f'(m)|$. 
If $p < \varepsilon$, then the edge $ms$ does not belong to $T'_\varepsilon$; otherwise, $ms$ belongs to $T'_\varepsilon$.  

\underline{Case (II).} If $s$ is already paired, then $m$ must pair with some other saddle $s'$, and $s'$ must be an ancestor of $s$ in the join tree. A paired $s$ means that $s$ has been discovered earlier during the growth of another local minimum $m'$ such that $m'$ and $s$ form a persistence pair with persistence $p'$, and the edge $m's$ belongs to $T'$. 

\underline{Case (II.a).} 
Suppose that $p' \geq \varepsilon$. Since $m'$ preceds $m$ in the sorted order, $f'(m') > f'(m)$. Since $s'$ is an ancestor of $s$, $f'(s') > f'(s)$. Therefore $|f'(s') - f'(m)| > |f'(s) - f'(m')| = p' \geq \varepsilon$. 
Thus, the pair $(m,s')$ has a persistence above $\varepsilon$, and $ms$ must be an edge in $T'_\varepsilon$.

\underline{Case (II.b).} 
Now suppose that $p' < \varepsilon$. In this case, we do not have enough information to determine the persistence of $(m,s')$. Therefore, we grow from saddle $s$ to reach a new saddle $s''$. We then check cases (I) and (II) again, using $s''$ in place of $s$. 

Once we are done checking cases (I) and (II), if $m \notin T'_{\varepsilon}$ but $m \in T_{\varepsilon}$, then $m$ is a false negative. 
Likewise, if $ms \in T'_{\varepsilon}$ but $ms \notin T_{\varepsilon}$, then $ms$ is a false positive. 

Growing the global minimum will never produce a false case as long as the rest of $T'_\varepsilon$ is correctly predicted. Thus, we skip the growth at the global minimum, denoted as $\hat{m}$. 
Because $\hat{m}$ is the last growth that remains active, its growth will form the \textit{trunk}, a monotone sequence of edges to the root that links $\hat{m}$ to the remaining saddles~\cite{gueunet2017task}. Since $\hat{m}$ and the remaining saddles are already correctly predicted, so is the trunk, therefore no further false cases are possible, and we skip growing $\hat{m}$. 
This algorithm also admits a number of special cases; see~\cref{sec:algorithm-details}.

\para{Progressive false case correction.} 
If there is a false case, we first tighten the upper and lower bounds of points in some region $R$ to correct it. If $ms$ is a false positive, then $R$ is the region of the merge-tree-induced segmentation of $f'$ corresponding to $ms$. If $m$ is a false negative, and edge $m\hat{s}$ belongs to $T_\varepsilon$ (for some saddle $\hat{s}$), then $R$ is the region of the merge-tree-induced segmentation of $f$ corresponding to $m\hat{s}$. If the same false case occurs multiple times, we grow the region $R$. We tighten the upper and lower bounds of each $x \in R$ similarly to TopoSZ, but we tighten more aggressively to speed up compression. 
We then update the decompressed data $f'$ to respect the new bounds; see~\cref{sec:algorithm-details} for numerical specifics and a comparison with TopoSZ.

Once we update $f'$, these updates may affect parts of the join and split trees beyond the false cases, thus we must recompute those areas to ensure correctness. Specifically, we must check for any extrema bordering $R$ that may have appeared or disappeared as a result of the tightening process and update the trees accordingly. Let $E$ be the set of edges whose segmentation regions border $R$. Then the tightening also may have affected each edge $e \in E$ and every ancestor of $e$ (i.e.,~edges
connecting $e$ to the root of the tree). We recompute all such edges to ensure correctness. As before, we recompute parts of the tree in order of the function values. 



\begin{table*}[t]
% \setlength\tabcolsep{5pt}
\centering
\small
\scalebox{1}{
\begin{tabular}{lccccccccccccccc}
\toprule
\multicolumn{1}{c}{\multirow{2.5}{*}{\textbf{Method}}} & \multicolumn{3}{c}{{HotpotQA}} & \multicolumn{3}{c}{{2WikiMultihopQA}} & \multicolumn{3}{c}{{MuSiQue}} & \multicolumn{3}{c}{{StrategyQA}} & \multicolumn{3}{c}{\textbf{Average}} \\
\cmidrule(r){2-4} \cmidrule(r){5-7} \cmidrule(r){8-10} \cmidrule(r){11-13} \cmidrule(r){14-16}
\multicolumn{1}{c}{} & \textbf{EM} & \textbf{CEM} & \textbf{F1} & \textbf{EM} & \textbf{CEM} & \textbf{F1} & \textbf{EM} & \textbf{CEM} & \textbf{F1} & \textbf{EM} & \textbf{CEM} & \textbf{F1} & \textbf{EM} & \textbf{CEM} & \textbf{F1} \\

\midrule
\multicolumn{16}{c}{\textit{GPT-4o-mini (Closed-source)}} \\
\midrule
% \cmidrule{1-16}
Closed-Book & 28.82 & 39.80 & 34.71 & 24.71 & 30.40 & 24.71 & 7.65 & 15.91 & 9.41 & 73.53 & 73.53 & 73.53 & 33.68 & 39.91 & 35.59 \\
\multicolumn{1}{l}{Chain-of-Tought} & 26.47 & 38.79 & 32.35 & 24.12 & 30.84 & 26.47 & 13.53 & 20.92 & 18.24 & 51.76 & 52.09 & 51.76 & 28.97 & 35.66 & 32.21 \\
\multicolumn{1}{l}{Standard RAG} & 41.18 & 54.36 & 52.94 & 25.88 & 32.09 & 27.65 & 11.76 & 19.91 & 16.47 & 58.24 & 59.53 & 58.24 & 34.27 & 41.47 & 38.83 \\
\cmidrule{2-16}
\multicolumn{1}{l}{ReAct} & 35.88 & 51.08 & 42.35 & 29.41 & 35.46 & 30.00 & 10.00 & 18.05 & 12.35 & 36.47 & 40.46 & 36.47 & 27.94 & 36.26 & 30.29 \\
\multicolumn{1}{l}{Query2doc} & 44.71 & 57.21 & 54.71 & 29.41 & 34.46 & 29.41 & 19.41 & 28.05 & 24.71 & 64.71 & 65.67 & 64.71 & 39.56 & 46.35 & 43.39 \\
\multicolumn{1}{l}{Self-RAG} & 38.82 & 50.32 & 47.65 & 26.47 & 31.87 & 27.65 & 13.53 & 21.29 & 16.47 & 68.82 & 69.10 & 68.82 & 36.91 & 43.15 & 40.15 \\
% \multicolumn{1}{l}{Mind-Search} & 44.0 & 50.0 & 58.6 & 28.0 & 29.0 & 32.2 & 14.0 & 16.0 & 22.8 & \textbf{72.0} & 74.0 & \textbf{72.0} \\
% \multicolumn{1}{l}{Infogent} & 44.0 & 50.0 & 58.6 & 28.0 & 29.0 & 32.2 & 14.0 & 16.0 & 22.8 & \textbf{72.0} & 74.0 & \textbf{72.0} \\
% \multicolumn{1}{l}{RAG-Star}  \\
\cmidrule{2-16}
Ours & 41.76 & 45.88 & 58.69 & 52.94 & 65.75 & 53.53 & 23.67 & 33.21 & 26.04 & 77.67 & 77.67 & 77.67 & \ \ \textbf{49.01}$^{\dagger}$ & \ \ \textbf{55.63}$^{\dagger}$ & \ \ \textbf{53.98}$^{\dagger}$ \\

\midrule
\multicolumn{16}{c}{\textit{Gemini-1.5-flash (Closed-source)}} \\
\midrule
Closed-Book & 19.41 & 31.52 & 24.71 & 24.12 & 30.07 & 24.71 & 3.53 & 10.98 & 6.47 & 32.35 & 32.83 & 32.35 & 19.85 & 26.35 & 22.06 \\
\multicolumn{1}{l}{Chain-of-Tought} & 28.82 & 36.43 & 34.71 & 18.24 & 21.35 & 18.82 & 8.24 & 13.73 & 10.59 & 68.82 & 69.64 & 68.82 & 31.03 & 35.29 & 33.24 \\
\multicolumn{1}{l}{Standard RAG} & 37.65 & 48.95 & 47.06 & 16.47 & 20.54 & 17.65 & 7.06 & 11.54 & 10.00 & 52.94 & 53.89 & 52.94 & 28.53 & 33.73 & 31.91 \\
\cmidrule(lr){2-16}
\multicolumn{1}{l}{ReAct} & 34.71 & 45.44 & 42.35 & 18.24 & 22.33 & 18.82 & 5.88 & 10.38 & 7.06 & 34.91 & 36.03 & 34.91 & 23.44 & 28.55 & 25.79 \\
\multicolumn{1}{l}{Query2doc} & 38.16 & 49.15 & 47.81 & 17.06 & 20.32 & 17.65 & 11.18 & 17.65 & 14.71 & 58.24 & 58.53 & 58.24 & 31.16 & 36.41 & 34.60\\
\multicolumn{1}{l}{Self-RAG} & 39.83 & 49.47 & 47.88 & 15.29 & 19.06 & 17.65 & 7.65 & 12.09 & 10.00 & 54.12 & 54.53 & 54.12 & 29.22 & 33.79 & 32.41 \\
% \multicolumn{1}{l}{Mind-Search} &  &  &  &  &  &  &  &  &  &  &  &  \\
% \multicolumn{1}{l}{Infogent} &  &  &  &  &  &  &  &  &  &  &  &  \\
% \multicolumn{1}{l}{RAG-Star} &  &  &  &  &  &  &  &  &  &  &  &  \\
\cmidrule(lr){2-16}
Ours & 47.65 & 57.65 & 61.98 & 62.94 & 62.94 & 73.94 & 17.75 & 21.30 & 28.31 & 76.19 & 76.19 & 76.19 & \ \ \textbf{51.13}$^{\dagger}$ & \ \ \textbf{54.52}$^{\dagger}$ & \ \ \textbf{60.11}$^{\dagger}$ \\
\midrule
% \multicolumn{13}{c}{\textit{Claude-3.5-sonnet (Closed-source)}} \\
% \midrule
% Closed-Book & 33.82 & 47.01 & 43.53 & 34.71 & 39.70 & 35.88 & 16.76 & 25.91 & 20.00 & 68.82 & 70.27 & 68.82\\
% \multicolumn{1}{l}{Chain-of-Tought} & 34.86 & 48.31 & 42.51 & 40.00 & 49.52 & 40.59 & 16.03 & 26.15 & 19.41 & 19.51 & 27.42 & 19.51 \\
% \multicolumn{1}{l}{Standard RAG} & 40.64 & 53.85 & 51.24 & 17.65 & 21.33 & 18.82 & 12.68 & 17.79 & 15.96 & 57.00 & 60.11 & 57.00 \\
% \cmidrule(lr){2-13}
% \multicolumn{1}{l}{ReAct} & 28.01 & 40.01 & 35.18 & 12.29 & 15.31 & 12.85 & 10.75 & 16.81 & 12.37 & 19.89 & 24.73 & 19.89 \\
% \multicolumn{1}{l}{Query2doc} & 46.50 & 59.86 & 57.61 & 19.81 & 23.46 & 21.70 & 21.97 & 28.88 & 24.28 & 56.00 & 59.14 & 56.00 \\
% \multicolumn{1}{l}{Self-RAG} & 40.23 & 51.66 & 48.28 & 0.00 & 0.00 & 0.00 & 0.00 & 29.17 & 0.00 & 80.00 & 84.44 & 80.00 \\
% % \multicolumn{1}{l}{Mind-Search} &  &  &  &  &  &  &  &  &  &  &  &  \\
% % \multicolumn{1}{l}{Infogent} &  &  &  &  &  &  &  &  &  &  &  &  \\
% \multicolumn{1}{l}{RAG-Star} &  &  &  &  &  &  &  &  &  &  &  &  \\
% \cmidrule(lr){2-13}
% Ours &  &  &  &  &  &  &  &  &  &  &  &  \\
\midrule
\multicolumn{16}{c}{\textit{DeepSeek-V3-chat (Open-source)}} \\
\midrule
Closed-Book & 35.88 & 48.58 & 44.12 & 35.88 & 41.85 & 37.06 & 11.18 & 19.54 & 14.71 & 64.71 & 65.10 & 64.71 & 36.91 & 43.77 & 40.15 \\
\multicolumn{1}{l}{Chain-of-Tought} & 38.82 & 50.59 & 47.06 & 47.06 & 56.14 & 48.24 & 21.18 & 29.99 & 24.12 & 41.76 & 47.46 & 41.76 & 37.21 & 46.12 & 40.30 \\
\multicolumn{1}{l}{Standard RAG} & 40.00 & 55.01 & 50.00 & 30.00 & 34.26 & 31.18 & 15.88 & 24.47 & 18.82 & 64.12 & 65.59 & 64.12 & 37.50 & 44.83 & 41.03 \\
\cmidrule(lr){2-16}
\multicolumn{1}{l}{ReAct} & 40.00 & 54.97 & 48.24 & 32.35 & 36.17 & 32.35 & 16.67 & 34.07 & 20.83 & 23.53 & 28.79 & 23.53 & 28.14 & 38.50 & 31.24 \\
\multicolumn{1}{l}{Query2doc} & 47.19 & 63.11 & 58.05 & 32.35 & 37.14 & 32.35 & 21.18 & 29.89 & 24.71 & 57.65 & 59.40 & 57.65 & 39.59 & 47.39 & 43.19 \\
\multicolumn{1}{l}{Self-RAG} & 43.49 & 56.31 & 51.75 & 27.65 & 32.27 & 28.24 & 16.86 & 25.30 & 19.77 & 52.87 & 53.66 & 52.87 & 35.22 & 41.89 & 38.16 \\
% \multicolumn{1}{l}{Mind-Search} &  &  &  &  &  &  &  &  &  &  &  &  \\
% \multicolumn{1}{l}{Infogent} &  &  &  &  &  &  &  &  &  &  &  &  \\
% \multicolumn{1}{l}{RAG-Star} &  &  &  &  &  &  &  &  &  &  &  &  \\
\cmidrule{2-16}
\textbf{Ours} & 45.88 & 52.94 & 62.99 & 60.00 & 63.53 & 73.15 & 21.30 & 32.60 & 25.44 & 76.47 & 76.47 & 76.47 & \ \ \textbf{50.91}$^{\dagger}$ & \ \ \textbf{56.39}$^{\dagger}$ & \ \ \textbf{59.51}$^{\dagger}$ \\
\bottomrule
\end{tabular}}
\caption{The evaluation results for four representative multi-hop QA datasets are presented, we also report the average results of the four datasets. The symbol ``$^{\dagger}$'' denotes that the performance improvement is statistically significant with p < 0.05 compared against all the baselines.}
\label{tab:main-result}
\end{table*}



\section{Experiments and Analysis}
In this section, we begin by detailing the experimental setup, followed by a comprehensive presentation of the primary results. Subsequently, we conduct an ablation study and provide an in-depth analysis to elucidate our findings further.

\subsection{Experimental Settings}


\subsubsection{Datasets}
% FRAMES~\cite{krishna2024fact}, and
To rigorously assess the efficacy of our proposed methodology, we conducted evaluations using 5 benchmark datasets. All datasets are meticulously designed to challenge models with complex, multi-hop questions that require synthesizing information across multiple documents.
% The detailed statistics of different datasets are shown in Table~\ref{tab:datasets}.

\begin{itemize}
    \item \textbf{FanOutQA}~\cite{zhu2024fanoutqa} is a high-quality dataset comprising complex information-seeking questions and human-written decompositions, which necessitate aggregating information about multiple entities from several sources to formulate a comprehensive answer. 
    % \item \textbf{FRAMES} comprises multi-hop questions requiring integration of information from multiple sources. These questions span diverse topics, and are labeled with reasoning types such as numerical reasoning, tabular reasoning, multiple constraints, temporal reasoning, and post-processing. Following existing study~\cite{reddy2024infogent}, we exclude numerical questions since different LLMs tend to be highly sensitive to them.

    \item \textbf{HotpotQA}~\cite{yang2018hotpotqa} is a widely used dataset for multi-hop question answering, designed to evaluate reason abilities across multiple documents to answer complex questions. The dataset is collected via crowdsourcing based on Wikipedia articles, and annotators are asked to propose questions that require reasoning using the multiple presented Wikipedia articles.

    \item \textbf{2WikiMultihopQA}~\cite{ho2020constructing} is a large-scale multi-hop QA dataset that requires reading multiple paragraphs to answer a given question. The dataset includes four types of questions: comparison, inference, compositional, and bridge-comparison. Each question is accompanied by relevant Wikipedia paragraphs as evidence.
    
    \item \textbf{MuSiQue}~\cite{trivedi2022musique} is created by composing questions from multiple existing single-hop datasets. The dataset is more challenging than previous multi-hop reasoning datasets, with a threefold increase in the human-machine gap and significantly lower disconnected reasoning scores, indicating reduced susceptibility to shortcut reasoning.
    
    \item \textbf{StrategyQA}~\cite{geva2021did} focuses on open-domain questions that require implicit reasoning steps. The dataset consists of 2,780 examples, each comprising a strategy question, its decomposition, and evidence paragraphs. Each question is accompanied by decomposed reasoning steps and relevant Wikipedia paragraphs as evidence.


    
\end{itemize}



\subsubsection{Evaluation Metrics}
We adopted the established evaluation metrics for the adopted datasets to ensure consistency and comparability. For the evaluation of \textit{FanOutQA}, we employed string accuracy, which measures the proportion of exact matches, and ROUGE metrics~\cite{lin2004rouge}, which assess the quality of summarization by comparing multiple features between the generated and reference texts. Specifically, we report ROUGE-1 (R-1), ROUGE-2 (R-2), and ROUGE-L (R-L) scores to comprehensively assess performance.
% For FRAMES, a large language model~(LLM) was utilized to benchmark the alignment of generated answers with ground-truth annotations. We present the evaluation results separately for queries corresponding to each of the four reasoning types.
For \textit{HotpotQA, 2WikiMultihopQA, MuSiQue}, and \textit{StrategyQA}, we adopt Exact Match (EM), F1 score, and Cover Exact Match (CEM) as evaluation metrics. EM measures strict correctness by checking if the predicted answer matches the ground truth. F1 evaluates the overlap between prediction and ground truth, balancing precision and recall. CEM extends EM to multi-hop reasoning, requiring both correct answers and coverage of intermediate reasoning steps.
Similar to the setup in the existing work~\cite{jiang2024rag}, due to the large data scale, we randomly sampled 130 queries from each of the four datasets for evaluation.

\subsubsection{Baselines}
In the evaluation of our proposed method, we compare it against abundantly established baselines to ensure a comprehensive understanding of its performance. These baselines represent a spectrum of approaches commonly employed in the field, ranging from vanilla reasoning strategies to advanced reasoning methods.

\paratitle{Vinilla reasoning.}\quad
The \textit{Closed-Book} method directly prompts the LLM to provide an answer to a question. In contrast, \textit{Chain-of-Thought (CoT)}~\cite{wei2022chain} reasoning involves adding intermediate reasoning steps to facilitate the response. \textit{Standard RAG} first retrieves passages from the Wikipedia corpus using DPR~\cite{dpr2020} and then directly prompts the LLM to refer to these passages in its response. 

\paratitle{Advanced reasoning.}\quad
\textit{ReAct}~\cite{yao2023react} progressively addresses subqueries, ultimately consolidating the intermediate results to form a complete answer.
\textit{Query2doc}~\cite{wang2023query2doc} generates an initial answer using the model and subsequently retrieves relevant information to generate the final answer.
\textit{Self-RAG}~\cite{asaiself} involves first retrieving information and then assessing its relevance before deciding whether to incorporate it into the final answer.
\textit{MindSearch}~\cite{chen2024mindsearch} employs a planner-searcher architecture for searching relevant information.
\textit{Infogent}~\cite{reddy2024infogent} introduces a multi-agent architecture to aggregate multi-source information.
% \textit{RAG-Star}~\cite{jiang2024rag} adopts MCTS with retrieval for deliberate reasoning for multi-hop questions.


To ensure a more comprehensive evaluation of different methods, and to mitigate the influence of any specific model, we employ multiple LLMs as backbone models of different methods, including the closed-source LLM \texttt{GPT-4o-mini}, \texttt{Gemini-1.5-flash-002} and the open-source LLM \texttt{deepseek-v3-chat}.
We evaluated all baseline methods in a \textit{zero-shot} setting, employing them solely for inference without additional training.
Note that certain methods are challenging to replicate across all datasets due to the requirement of dataset-specific refinement. As a result, we are unable to report the results for all baselines on every dataset.


\subsubsection{Implement Details}

For web search, we employed Google Search as the primary search engine, selecting the top-3 web search results as document candidates and adhering to existing methodologies for web crawling and denoising~\cite{reddy2024infogent}. During the MCTS process, we ensured consistency between the policy model and the reward model. The maximum number of simulations was capped at 40, and the search depth was limited to 6 layers. In the upper confidence bound for trees algorithm, the exploration-exploitation balance parameter \( w \) was set to 0.2. Additionally, we generated three sub-queries per iteration (\( m_q = 3 \)).
For all generation tasks, responses were sampled using a temperature of 0.9 and top-\( p \) sampling with \( p = 1.0 \).  All prompts used are shown in the provided anonymous codes.




\subsection{Main Results}
The results of different methods evaluated on five complex reasoning datasets are shown in Table~\ref{tab:main-result} and Table~\ref{tab:fanoutqa}. It can be observed that:

(1) Our proposed method demonstrates significant improvements over all baseline approaches across four multi-hop QA datasets and the FanoutQA dataset that emphasizes the information-gathering task. This performance advantage is consistently observed across multiple popular backbone LLMs, highlighting the general applicability and effectiveness of our HG-MCTS framework. HG-MCTS employs an adaptive checklist to guide the expansion and reward modeling of Monte Carlo Tree Search (MCTS), effectively curtailing the exploration of unproductive pathways while maintaining robust search capabilities. In addition, its emphasis on targeted information gathering minimizes irrelevant content, reducing extraneous noise that could compromise the quality of the generated answers.

(2) We further observe that the MCTS-based approach for complex reasoning outperforms both single-step reasoning and chain-based reasoning methods, indicating that tree search can substantially expand the search space of solution paths, enable the discovery of optimal reasoning paths, and ultimately enhance performance. In addition, incorporating advanced reasoning strategies into RAG proves superior to vanilla reasoning standard RAG or closed-book approaches overall, highlighting the importance of meeting users’ complex information needs through tailored retrieval processes. Notably, on the FanoutQA dataset, our method based on GPT-4o-mini surpasses the baseline built upon the larger and more powerful LLM GPT-4o, offering further evidence for the intrinsic advantages of our approach.

(3) Finally, our method exhibits strong robustness across different backbone LLMs, whereas other methods often experience significant performance fluctuations depending on the underlying model. For example, when using Gemini as the backbone, the performance of baseline methods drops markedly compared with results obtained using the other two LLMs, while our method does not exhibit performance degradation. Additionally, we observe that the open-source model deepseek-v3-chat demonstrates capabilities comparable to its proprietary counterparts on various challenging multi-hop QA tasks, thereby laying a promising foundation for real-world deployment of related methodologies.


\begin{table}[t]
    \centering
    \small
\scalebox{1.015}{
    \setlength{\tabcolsep}{3.3pt} 
    \begin{tabular}{llcccc}
        \toprule
        \textbf{Method} & \textbf{LLM} & \textbf{Acc.} & \textbf{R-1} & \textbf{R-2} & \textbf{R-L} \\
        \midrule
        Closed-Book & LLaMA3 & 46.60 & 46.30 & 26.40 & 38.70 \\
        Closed-Book & GPT-4o & 44.10 & 47.40 & 27.30 & 41.70 \\
        Standard RAG & LLaMA3 & 46.80 & 28.20 & 14.30 & 24.30 \\
        Standard RAG & GPT-4o & 58.00 & 49.40 & 31.00 & 44.30 \\
        MindSearch & GPT-4o-mini & 47.30 & 49.30 & 28.40 & 44.20 \\
        Infogent & GPT-4o-mini & 51.10 & 53.30 & 33.00 & 48.50 \\
        \midrule
        \textbf{Ours} & GPT-4o-mini & \ \ \textbf{58.38}$^{\dagger}$ & \ \ \textbf{55.02}$^{\dagger}$ & \ \ \textbf{35.45}$^{\dagger}$ & \ \ \textbf{49.40}$^{\dagger}$ \\
        \bottomrule
    \end{tabular}
    }
    \caption{The evaluation results on FanoutQA. The symbol ``$^{\dagger}$'' denotes that the performance improvement is statistically significant with p < 0.05 compared against all the baselines. LLaMA3 is the abbreviation of LLaMA3-70B-Instruct.}
    \label{tab:fanoutqa}
\end{table}


\subsection{Ablation Studies}
\label{sec:ablation}

In this section, we conduct an ablation study to validate the effectiveness of key strategies in HG-MCTS comprehensively on FanoutQA. Here, we consider five variants based on HG-MCTS for comparison: (a) \underline{\textit{w/o Exploration Reward}} removes the exploration reward during reward modeling; (b) \underline{\textit{w/o Retrieval Reward}} removes the retrieval reward from the total reward during reward modeling; (c) \underline{\textit{w/o Progress Feedback}} eliminates generating the progress feedback in reward modeling; (d) \underline{\textit{w/o Checklist}} removes checklist for global guidance in the MCTS process; (e) \underline{\textit{w/o HG-MCTS}} removes the entire HG-MCTS strategy, reducing the approach to a linear reasoning strategy.

Table~\ref{tab:ablation} presents the results for the variants of our method, from which we can observe the following findings: 
(a) The performance drops in \underline{\textit{w/o Exploration Reward}}, demonstrating that incorporating exploration rewards facilitates more effective expansions during the tree search process. 
(b) The performance drops in \underline{\textit{w/o Retrieval Reward}}, demonstrating incorporating exploration rewards enables the model to better analyze the benefits of external retrieval information during the MCTS process, thereby achieving improved comprehensive information gathering.
(c) The performance drops in \underline{\textit{w/o Progress Feedback}}, underscoring the necessity of incorporating textual feedback to guide subsequent explorations and dynamically updating the checklist.
(d) The performance drops in \underline{\textit{w/o Checklist}}, demonstrating that incorporating the explicit checklist can effectively guide the expansion and reward modeling in MCTS.
(e) The performance significantly drops in \underline{\textit{w/o HG-MCTS}}, demonstrating that the proposed HG-MCTS plays a pivotal role in enhancing the effectiveness of information seeking.

\begin{table}[t]
    \centering
    \small
\scalebox{1.03}{
    \begin{tabular}{lcccc}
        \toprule
        \textbf{Method} & \textbf{Acc.} & \textbf{R-1} & \textbf{R-2} & \textbf{R-L}  \\
        \midrule
        Ours  & 58.38 & 55.02 & 35.45 & 49.40 \\
        \midrule
        w/o Exploration Reward & 57.23 & 54.20 & 34.91 & 48.87  \\
        w/o Retrieval Reward & 56.39 & 53.68 & 34.06 & 48.15  \\
        w/o Progress Feedback & 54.57 & 52.94 & 33.73 & 47.89  \\
        w/o Checklist & 55.03 & 53.41 & 33.85 & 48.03  \\
        w/o HG-MCTS & 52.55 & 53.25 & 32.74 & 47.82  \\
        \bottomrule
    \end{tabular}}
    \caption{Evaluation results of the proposed method's variants on FanoutQA.}
    \label{tab:ablation}
\end{table}

\begin{figure}[t!]
    \centering
    \subfigure[w/o HG-MCTS]{
        \includegraphics[width=0.38\linewidth]{pic/recall_baseline.pdf}
        \label{fig:subfig1}
    }
    \hspace{0.02\textwidth} 
    \subfigure[HG-MCTS]{
        \includegraphics[width=0.38\linewidth]{pic/recall_ours.pdf}
        \label{fig:subfig2}
    }
    \caption{Information collection evaluation for different methods on Recall rate~(blue part).}
    \label{fig:recall}
\end{figure}

\subsection{Analysis on Information Collection}
To evaluate the comprehensiveness of information collected by the proposed HG-MCTS method during the reasoning process, we conducted a systematic comparison of different methods. Here, we compare the proposed method with the w/o HG-MCTS variant in Section~\ref{sec:ablation}, which also employs an information-gathering process.
Specifically, we evaluate the comprehensiveness by calculating the recall rate of the set of web pages retrieved by different methods for the ground-truth Wikipedia pages annotated as solving each intricate query. 
The ground truth serves as a benchmark, representing the authoritative and comprehensive sources necessary for addressing the query. By analyzing the coverage of the retrieved web pages relative to the ground truth, we aimed to quantify the ability of each method to gather a sufficient and relevant body of information.

As shown in Figure~\ref{fig:recall}, our method demonstrates a higher recall rate compared to the baseline in information collection evaluation, indicating that the proposed targeted MCTS enables a more comprehensive collection of information relevant to user queries. Furthermore, our method avoids the tendency to indiscriminately gather excessive information in pursuit of a higher recall rate. Such an approach, while potentially increasing coverage, introduces substantial noise into the subsequent aggregation task, thereby compromising the overall effectiveness of the information seeking.


\subsection{Scaling Law on Simulation Amount}

Our experiments systematically explore the impact of varying the number of simulation iterations on the performance of our analysis method. We adopt two LLM backbones GPT-4o-mini and DeepSeek-V3-chat on the FanoutQA dataset with various simulation numbers for analysis. 

As shown in Figure~\ref{fig:simulation}, we find that increasing the simulation count initially yields significant gains in the model’s ability to navigate the solution space effectively. With more iterations, the method benefits from a broader exploration of potential reasoning paths, leading to enhanced accuracy and improved recall in downstream tasks. In particular, the enhanced exploration reduces the likelihood of early-stage errors propagating through subsequent steps, thereby reinforcing the overall integrity of the search process.
Moreover, our results also reveal a point of diminishing returns. Beyond a certain number of simulations, additional iterations contribute only marginal improvements to the final performance. This saturation effect can be attributed to the inherent limitations of LLM's internal knowledge and the increased computational overhead, which together may lead to overly complex decision paths without proportional benefits. Thus, while extended simulations promote a more thorough examination of the search space, they also impose a trade-off between improved performance and computational efficiency.



\begin{figure}[t!]
    \centering
    \subfigure[GPT-4o-mini]{
        \includegraphics[width=0.4772\linewidth]{pic/scale_4o.pdf}
        \label{fig:subfig1}
    }
    \subfigure[DeepSeek-V3-Chat]{
        \includegraphics[width=0.4772\linewidth]{pic/scale_ds.pdf}
        \label{fig:subfig2}
    }
    \caption{Evaluation results of HG-MCTS with various simulation numbers employed by different LLMs on FanoutQA.}
    \label{fig:simulation}
\end{figure}
\section{Related Work}
\label{sec:related}

\subsection{Test-Time Slow Thinking with LLMs}

The integration of the slow thinking paradigm~(\aka System 2) inspired reasoning techniques into LLMs has emerged as a pivotal research area~\cite{sutton2019bitter, wang2024q, kahneman2011thinking}, focusing on enhancing the problem-solving ability and the interoperability of LLM. A notable example is OpenAI’s o1 model\footnote{https://openai.com/o1/}, which incorporates extended internal reasoning chains during inference. 
This breakthrough has demonstrated remarkable success on programming and complex scientific benchmarks, which improve the focus of test-time reasoning that leverages extended inference processes to simulate deliberate, stepwise problem-solving akin to human-like cognitive processes without additional training~\cite{zhang2024llama, putta2024agent, luo2024improve}. Techniques such as chain-of-thought~\cite{wei2022chain} and self-consistency decoding~\cite{wangself} exemplify this paradigm, where generating intermediate reasoning steps or exploring multiple solution paths improves reliability and interoperability.
Recent advancements further extend these ideas by integrating search-based algorithms, such as beam search~\cite{kang2024mindstar} and Monte Carlo Tree Search (MCTS)~\cite{zhoulanguage, chen2024alphamath, zhang2024rest}, to systematically explore alternative reasoning trajectories. By exploring multiple outcome branches during inference, search-based methods achieve a favorable exploration-exploitation trade-off and have been widely adopted in reinforcement learning~\cite{hart1968formal, silver2017mastering} and real-world systems such as AlphaGo~\cite{silver2016mastering}. These methods are often guided by reward models, which provide feedback based on procedural or outcome-driven metrics to improve reasoning quality~\cite{snell2024scaling} iteratively. 
These collective efforts underscore a paradigm shift in LLM research, highlighting the complementary relationship between training-time strategies and scalable test-time reasoning mechanisms. 



\subsection{Web Search with Complex Reasoning}

Recent advances in LLMs have shifted web search beyond simple query-response paradigms to sophisticated methods capable of multi-step reasoning for complex information access~\cite{chen2024mindsearch, reddy2024infogent}. These approaches leverage generative models to integrate and interpret diverse information sources in real time. Such capabilities are particularly critical for intricate queries that require synthesizing fragmented or context-sensitive data, where conventional search systems often fail to maintain coherence or overlook essential insights~\cite{hoveyda2024aqa, khotdecomposed}.
Initial efforts, such as WebGPT~\cite{nakano2021webgpt}, follow an iterative process of query generation, information retrieval, summarization, and synthesizing information in response to user queries.
Building on this foundation, subsequent studies adopt chain-of-thought reasoning to decompose complicated tasks into more manageable subqueries, enabling stepwise verification and refinement~\cite{yao2023react, trivedi2023interleaving}. For instance, Search-in-the-Chain~\cite{xu2024search} systematically disaggregates complex information-seeking queries by iteratively generating partial hypotheses and validating them against web-based evidence. 
Additionally, multi-agent collaboration has been explored to further enhance the search process. 
MindSearch~\cite{chen2024mindsearch} employs a planner-searcher architecture to conduct planning based on directed acyclic graphs and to carry out hierarchical information seeking. Similarly, Infogent~\cite{reddy2024infogent} introduces a multi-module collaborative agent framework that orchestrates information aggregation across modular components.
These systems excel in dynamically managing query reformulation, evidence synthesis, and inferential reasoning, seamlessly adapting to emerging information during the search process.
Moreover, some studies incorporate explicitly retrieved external information into the MCTS reasoning process, enhancing the deliberate reasoning capabilities of LLMs in multi-hop problem-solving~\cite{lee2024zero, jiang2024rag}. In contrast to these approaches, our method reformulates the task as an information collection process with the introduction of a novel checklist-based planning mechanism to holistically guide the MCTS reasoning process. Furthermore, we combines quantitative progress reward and qualitative progress feedback in reward modeling, making the MCTS process more intelligent to explore more efficient reasoning paths.

\section{Conclusion}

This work addresses the pressing need for enhanced security in the burgeoning blockchain ecosystem. We investigate the application of Large Language Models (LLMs) to smart contract vulnerability detection and repair, focusing on Solidity and Move. We introduce \textbf{Smartify}, a novel multi-agent framework that significantly improves LLM performance in this critical domain. The contributions of this work are: (1) \textbf{Smartify}, a novel multi-agent framework that enhances LLM-based smart contract vulnerability detection and repair; (2) a method for encoding language-specific knowledge, valuable for low-resource languages like Move; (3) a scalable, adaptable approach applicable to other programming languages and LLMs; (4) a demonstration of Smartify’s efficacy on generalized pre-trained LLMs; and (5) a detailed analysis of the challenges inherent in automated code repair.

\textbf{Smartify} represents a significant advancement in automating smart contract security, a crucial concern in the expanding blockchain landscape. Future work will refine the framework, expand its language coverage, particularly within the blockchain domain, and integrate it into real-world blockchain development workflows. This research lays the foundation for AI-powered tools that can bolster the security and reliability of decentralized applications, fostering a more robust and trustworthy blockchain ecosystem.

%%
%% The acknowledgments section is defined using the "acks" environment
%% (and NOT an unnumbered section). This ensures the proper
%% identification of the section in the article metadata, and the
%% consistent spelling of the heading.
% \begin{acks}
% To Robert, for the bagels and explaining CMYK and color spaces.
% \end{acks}

%%
%% The next two lines define the bibliography style to be used, and
%% the bibliography file.
\bibliographystyle{ACM-Reference-Format}
\bibliography{sample-base}


%%
%% If your work has an appendix, this is the place to put it.

% \appendix


\end{document}
\endinput
%%
%% End of file `sample-sigconf.tex'.
