
\documentclass{article} % For LaTeX2e
\usepackage{times}

% Optional math commands from https://github.com/goodfeli/dlbook_notation.

\usepackage{hyperref}
\usepackage{url}
\usepackage{graphicx}
\usepackage{hyperref}       % hyperlinks
\usepackage{booktabs}       % professional-quality tables
\usepackage{amsfonts}       % blackboard math symbols
\usepackage{nicefrac}       % compact symbols for 1/2, etc.
\usepackage{microtype}      % microtypography
\usepackage{xcolor}         % colors
\usepackage{tcolorbox}
\usepackage{amsmath} 

\usepackage{caption}
\usepackage{subcaption}
\usepackage{multirow}

\definecolor{myblue}{RGB}{64,64,64}
\definecolor{mygray}{gray}{0.9}
\tcbset{
  myboxstyle/.style={
    rounded corners,
    boxrule=1mm,
    colback=mygray,
    colframe=myblue,
    fonttitle=\bfseries,
    coltitle=white,
    boxsep=2mm,
    left=2mm,
    right=2mm,
    top=3mm,
    bottom=3mm,
    fontupper=\large
  }
}

\title{MAGELLAN: Metacognitive Generalization of Learning Progress for Online RL\\ in LLM agents}

%  MAGELLAN: MetAcognitive Generalized Learning progress with LANguage

% Authors must not appear in the submitted version. They should be hidden
% as long as the \iclrfinalcopy macro remains commented out below.
% Non-anonymous submissions will be rejected without review.

\author{Antiquus S.~Hippocampus, Natalia Cerebro \& Amelie P. Amygdale \thanks{ Use footnote for providing further information
about author (webpage, alternative address)---\emph{not} for acknowledging
funding agencies.  Funding acknowledgements go at the end of the paper.} \\
Department of Computer Science\\
Cranberry-Lemon University\\
Pittsburgh, PA 15213, USA \\
\texttt{\{hippo,brain,jen\}@cs.cranberry-lemon.edu} \\
\And
Ji Q. Ren \& Yevgeny LeNet \\
Department of Computational Neuroscience \\
University of the Witwatersrand \\
Joburg, South Africa \\
\texttt{\{robot,net\}@wits.ac.za} \\
\AND
Coauthor \\
Affiliation \\
Address \\
\texttt{email}
}

% The \author macro works with any number of authors. There are two commands
% used to separate the names and addresses of multiple authors: \And and \AND.
%
% Using \And between authors leaves it to \LaTeX{} to determine where to break
% the lines. Using \AND forces a linebreak at that point. So, if \LaTeX{}
% puts 3 of 4 authors names on the first line, and the last on the second
% line, try using \AND instead of \And before the third author name.

\newcommand{\fix}{\marginpar{FIX}}
\newcommand{\new}{\marginpar{NEW}}

%\iclrfinalcopy % Uncomment for camera-ready version, but NOT for submission.



% Notes:
% -Acl is important and having an estimation of LP is hard in natural language goals
% -We introduce a method to estimate and plug it into an already used ACL baseline (MAB bandit)
% -Say we study one form of meta-cognition: estimating the lp (cf. proposition projet ANR + rania’s thesis)
% -experiments for rebuttal: removing HER (and importance on generalization) + using hindsight goals in SR estimators
% -env description is super important (explicitely say why not other envs)

\begin{document}


\maketitle

\begin{abstract}

\end{abstract}

\section{Introduction}
\label{sec:introduction}

\textbf{Figure 1: Schema principal}

\section{Related Work}
\subsection{Autonomous LLM Agents}

\subsection{Automatic Curriculum Learning}

\section{Methods}

\subsection{\textbf{M}et\textbf{A}cognitive \textbf{GE}neralization of \textbf{L}earning progress in \textbf{LAN}guage model agents}

\textbf{Figure 2: Archi si on a la place}

\subsection{Automatic Curriculum Learning using MAGELLAN} \label{sec:acl_magellan}

\subsection{The Little-Zoo environment as testbed} \label{sec:zoo_env}

\textbf{Figure 3: Tech-tree de l'env}



\section{Experiments}
\label{sec:experiments}

\subsection{Q1}

\textbf{Table 1: Comparaison des méthodes}
\textbf{Figure 4: Scaling compute/error}

\subsection{Q2}

\textbf{Figure 5: Plot du SR sur le train (Random vs MALP vs MAGELLAN) (Par but?)}

\subsection{Q3}

\textbf{Table 2: Error du SR sur le test pour MAGELLAN vs MALP}

\subsection{Q4}

\textbf{Figure 6: Test d'adaptation}

\section{Conclusion}
\label{sec:conclusion}

\bibliography{iclr2025_conference}
\bibliographystyle{iclr2025_conference}

\newpage
\appendix

\section*{Figures}

% Figures

\begin{figure}[htbp!]
    \centering
    \caption{Enter Caption}
    \label{fig:enter-label}
\end{figure}

\begin{figure}[htbp!]
    \centering
    \caption{Enter Caption}
    \label{fig:enter-label}
\end{figure}

\begin{figure}[htbp!]
    \centering
    \caption{Enter Caption}
    \label{fig:enter-label}
\end{figure}

\begin{table}[h!] 
\centering 
\begin{tabular}{|l|c|c|c|c|}
\hline 
\textbf{} & \textbf{Oracle} & \textbf{PELP} & \textbf{MALP} & \textbf{MAGELLAN} \\ 
\hline \textbf{Implementable} & $\times$ & $\checkmark$ & $\checkmark$ & $\checkmark$ \\ 
\hline \textbf{Computationally efficient} & $\checkmark$ & $\times$ & $\checkmark$ & $\checkmark$ \\ 
\hline \textbf{Accurate competence prediction} & $\checkmark$ & $\checkmark$ & $\times$ & $\checkmark$ \\ 
\hline 
\end{tabular}
\caption{Caption} 
\end{table} 

\begin{figure}[htbp!]
    \centering
    \includegraphics[width=\linewidth]{Figures//Q1/compute_error_scaling.png}
    \caption{Compute/Error scaling}
    \label{fig:enter-label}
\end{figure}

\begin{figure}[htbp!]
    \centering
    \includegraphics[width=\linewidth]{Figures//Q2/SR_train.png}
    \caption{Evaluation on the train set}
    \label{fig:enter-label}
\end{figure}

\begin{figure}
    \centering
    \includegraphics[width=1\linewidth]{Figures//Q2/SR_train2.png}
    \caption{Enter Caption}
    \label{fig:enter-label}
\end{figure}

\begin{figure}
    \centering
    \includegraphics[width=\linewidth]{Figures//Q4/adaptation_test_shema.png}
    \caption{Enter Caption}
    \label{fig:enter-label}
\end{figure}

\begin{figure}
    \centering
    \includegraphics[width=\linewidth]{Figures//Q4/adaptation_test.png}
    \caption{Enter Caption}
    \label{fig:enter-label}
\end{figure}

\begin{figure}[h]
\centering
\begin{subfigure}{.3\textwidth}
  \centering
  \includegraphics[width=.9\linewidth]{Figures//Q4/adaptation_test_shema.png}
  \caption{}
\end{subfigure}%
\begin{subfigure}{.8\textwidth}
  \centering
  \includegraphics[width=.9\linewidth]{Figures/Q4/adaptation_test.png}
  \caption{}
\end{subfigure}
\caption{}
\label{fig:transition-appearance}
\end{figure}

\end{document}
