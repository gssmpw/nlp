\section{Related Works}
\label{sec:related_works}

\paragraph{LLM Agents for Planning}

Agent-based frameworks for planning have gained interest, focusing on enhancing how LLMs decompose tasks and refine their outputs. The Sibyl framework ____ effectively decomposes tasks into smaller subtasks, assigning each to specialized agents that iteratively collaborate until a solution is reached. OS-Copilot ____ introduces a generalist computer agent that employs self-improvement through modularization and feedback loops. Another approach is KnowAgent ____, which integrates knowledge-augmented planning to enhance the decision-making capabilities of LLM agents. Similarly, Tool-Planner ____ proposed grouping tools based on similar functionalities into toolkits, allowing LLMs to select the best tool for a given task. Many agent-based works focusing on planning have been developed ____. Despite the progress, these methods generally (i) focus on domain-specific tasks or limited benchmarks, reducing generalizability, and (ii) lack or under-explore mechanisms for verifying and refining plans iteratively. While some works explore natural language planning ____, they either single-agent frameworks or evaluate proposed framework on domain-specific benchmarks.  

% Agent-based frameworks for planning have gained interest, focusing on enhancing how LLMs decompose tasks and refine their outputs. The Sibyl framework ____ is one such example that effectively decomposes tasks into smaller subtasks, assigning each to specialized agents. These agents collaborate iteratively in a pipeline, refining their outputs and passing information to one another until a solution is reached. OS-Copilot ____ introduces a generalist computer agent that employs self-improvement techniques. By modularizing tasks and integrating feedback loops, OS-Copilot continuously learns and refines its performance. Another approach is KnowAgent ____, which integrates knowledge-augmented planning to enhance the decision-making capabilities of LLM agents. Similarly, Tool-Planner ____ proposed grouping tools based on similar functionalities into toolkits, allowing LLMs to select the best tool for a given task. Many agent-based works focusing on planning have been developed ____. Despite the progress, these works do not prioritize improving natural language planning for LLMs. Moreover, these approaches often focus on domain-specific tasks or single benchmarks, limiting their generalizability. While some works address natural language planning ____, they either lack agentic frameworks or evaluate on limited, fixed-domain benchmarks.  


\paragraph{Inference-time Algorithms}

Inference-time algorithms have recently shown a significant improvement in LLMs performance during inference. For instance, Best of $\mathcal{N}$ sampling ____ selects the most promising output from multiple generations performed using temperature sampling, while Tree-of-Thought (ToT) ____ models reasoning as an iterative tree search. REBASE ____ optimizes search-space pruning using reward balancing. One very popular approach is Monte Carlo Tree Search (MCTS) ____ which iteratively explores solution paths during inference. Applied to models such as LLaMa-3-8B, it enables self-refinement by revisiting and improving initial solutions. Test-time optimization ____, focuses on dynamically adjusting computational resources during inference ____. Furthermore, ____ uses the inference time algorithms to improve LLMs planning capabilities to solve code synthesis problems. In inference-time algorithms, verification is the key component. In contrast to these past works, here, we enhance performance of inference-time algorithms utilizing constraint-guided verification, and multi-agent collaboration for natural language planning, and its applications in downstream complex reasoning tasks.

% Inference-time algorithms have recently shown a powerful way to optimize LLM output during inference, providing significant improvements in accuracy without scaling the model. One very popular approach is the use of Monte Carlo Tree Search (MCTS) ____, which iteratively explores multiple solution paths during inference. This technique has been successfully applied to models like LLaMa-3-8B, which integrates a self-refinement mechanism that allows the model to revisit and improve its initial solutions over time. Test-time optimization ____, focuses on dynamically adjusting computational resources during inference. Additionally, compute-optimal inference ____ highlights the importance of effectively distributing computational power during problem-solving tasks. Finally, repeated sampling ____ is a technique that uses multiple inference attempts to improve solution quality. \citept{wang2025planning} uses the inference time algorithms to improve LLMs planning capabilities to solve code synthesis problems. In contrast to these past works in specific domains, here, we explore inference-time scaling for natural language planning, and its applications in downstream complex reasoning tasks.


% In contrast to all these approaches which are not focused on improving planning or even on planning with specific domains like code synthesis, our focus is on leveraging inference-time algorithms for improving planning across various domains and applications.

%By optimizing compute resources based on the complexity of a task, this approach strikes a balance between efficiency and accuracy, ensuring that difficult tasks receive more attention while simpler tasks are processed with fewer resources.