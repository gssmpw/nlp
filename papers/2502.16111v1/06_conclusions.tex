\section{Conclusions}
\label{sec:conclusions}

In this work, we proposed \plangen{}, an easily scalable multi-agent approach incorporating three key components: constraint, verification, and selection agents. We leveraged these agents to improve the verification process of existing inference algorithms and proposed three frameworks: Multi-Agent Best of $\mathcal{N}$, ToT, and REBASE. Further, we introduced a Mixture of Algorithms, an iterative framework that integrates the selection agent (Figure \ref{fig:teaser}) to dynamically choose the best algorithm. We evaluated our frameworks on NATURAL PLAN, OlympiadBench, GPQA, and DocFinQA. Experimental results demonstrate that \plangen{} outperforms strong baselines, achieving SOTA results across datasets. Furthermore, our findings suggest that the proposed frameworks are scalable and generalizable to different LLMs, improving their natural language planning ability.


\section*{Limitations}

Despite the strong performance of our frameworks, an area of improvement is the reliance on predefined heuristics for selecting inference-time algorithms, which may not always generalize optimally across all tasks and domains. Additionally, while our frameworks demonstrate strong performance, their computational overhead could be further optimized for efficiency in real-world applications. We believe that our frameworks can be useful in further boosting the planning and reasoning capabilities of existing models such as o1 and Gemini-thinking. In addition, the use of reinforcement learning or meta-learning techniques to dynamically adapt agent strategies based on task complexity could be an interesting area to explore. Moreover, broadening the scope to multi-modal and multi-lingual reasoning would significantly expand the applicability of our approach, and exploring the use of generated planning trajectories for model training offers valuable direction.

\section*{Ethics Statement}

The use of proprietary LLMs such as GPT-4, Gemini, and Claude-3 in this study adheres to their policies of usage. We have used AI assistants (Grammarly and Gemini) to address the grammatical errors and rephrase the sentences.

% \section*{Acknowledgments}