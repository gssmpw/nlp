\begin{table*}
\centering
\footnotesize
% \resizebox{\linewidth}{!}{
\begin{tabular}{p{\linewidth}}
\toprule
\textbf{Question:} Two quantum states with energies E1 and E2 have a lifetime of $10^{-9}$ sec and $10^{-8}$ sec, respectively. We want to clearly distinguish these two energy levels. Which one of the following options could be their energy difference so that they can be clearly resolved?

(A) $10^{-11}$ eV

(B) $10^{-8}$ eV

(C) $10^{-9}$ eV

(D) $10^{-4}$ eV \\ \midrule

\textbf{Constraints for Verifying the Energy Difference Solution for Resolvable Quantum States}
\\ \\
This problem revolves around the energy-time uncertainty principle, which dictates the resolvability of energy levels given their lifetimes. To verify a proposed energy difference ($\Delta E = |E_2 - E_1|$) that allows clear resolution, we need constraints encompassing the following:
\\\\
1. Uncertainty Principle Adherence:

\textbf{Primary Constraint:} The product of the uncertainty in energy ($\Delta E$) and the uncertainty in time ($\Delta t$) must satisfy the energy-time uncertainty relation: $\Delta E \Delta t \geq \frac{\hbar}{2}$, where $\hbar$ is the reduced Planck constant.

\textbf{Specific Constraint:} Since we have two states, we must consider the \textit{combined} uncertainty in time. We can approximate this as the average of the individual lifetimes: $\Delta t \approx \frac{\tau_1 + \tau_2}{2}$, where $\tau_1$ and $\tau_2$ are the lifetimes of the two states.

\textbf{Operational Constraint:} For "clear resolution," we typically require $\Delta E$ to be significantly greater than the uncertainty dictated by the uncertainty principle. A common rule of thumb is $\Delta E \geq \frac{\hbar}{\Delta t}$. A stricter constraint might use a larger factor (e.g., $\Delta E \geq \frac{2\hbar}{\Delta t}$) for even clearer resolution.
\\\\
2. Units Consistency:

\textbf{Unit Constraint:} All calculations must use consistent units. Ensure that energy ($\Delta E$), time ($\Delta t$), and $\hbar$ are expressed in compatible units (e.g., Joules and seconds, or eV and seconds). Clearly state the chosen unit system.
\\\\
3. Numerical Accuracy:

\textbf{Precision Constraint:} Calculations should be performed with sufficient numerical precision to avoid rounding errors. Define the acceptable level of precision.

\textbf{Constant Accuracy:} Use a precise value for the reduced Planck constant ($\hbar$).
\\\\
4. Solution Justification:

\textbf{Explanation Constraint:} The solution must not only provide a numerical value for $\Delta E$ but also clearly explain why this value satisfies the constraints and allows for clear resolution. This explanation should reference the uncertainty principle and the chosen resolution criterion.
\\\\
5. Alternative Solutions Consideration (Optional):

\textbf{Exploration Constraint:} Explore whether multiple values of $\Delta E$ could satisfy the constraints. If so, discuss the trade-offs between different solutions.
\\\\
\textbf{Example Implementation of Constraints:}
\\\\
Let's assume we're working in SI units (Joules and seconds).

1. Uncertainty Principle: $\Delta E \geq \frac{\hbar}{[(10^{-9} \text{ s} + 10^{-8} \text{ s})/2]}$, where $\hbar \approx 1.054 \times 10^{-34} \text{ J s}$.

2. Units: All calculations are in Joules and seconds.

3. Numerical Accuracy: Calculations will be performed to at least 4 significant figures.

4. Justification: The solution will explain how the calculated $\Delta E$ is significantly larger than the uncertainty, ensuring clear resolution.

 \\ \bottomrule
\end{tabular}
% }
\caption{Examples of constraints generated by the constraint agent given reasoning question. The example illustrates the constraint generation for solving physics question from GPQA}
\label{tab:gpqa_constraints_examples}
\end{table*}