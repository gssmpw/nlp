\begin{table*}
\centering
\footnotesize
% \resizebox{\linewidth}{!}{
\begin{tabular}{p{\linewidth}}
\toprule
\textbf{Question:} Two quantum states with energies E1 and E2 have a lifetime of $10^{-9}$ sec and $10^{-8}$ sec, respectively. We want to clearly distinguish these two energy levels. Which one of the following options could be their energy difference so that they can be clearly resolved? \\ \midrule
\textbf{Plan to solve the problem:}\\ \\ 1. Identify the relevant physics principle: This problem involves the energy-time uncertainty principle, which states that the product of the uncertainty in energy ($\Delta E$) and the uncertainty in time ($\Delta t$) is greater than or equal to h-bar/2 ($\hbar/2$). The uncertainty in time is related to the lifetime of the states. To clearly distinguish the energy levels, the energy difference between them must be greater than the uncertainty in their energies.\\ \\ 2. Relate lifetimes to time uncertainty: The lifetimes given ($10^{-9}$ s and $10^{-8}$ s) represent the uncertainties in the time for each state ($\Delta t_1$ and $\Delta t_2$). We need to consider which lifetime is more restrictive for resolving the energy levels. Since we want to *clearly* distinguish them, we should use the *smaller* lifetime as the relevant $\Delta t$. This is because the shorter lifetime leads to a larger energy uncertainty, making the levels harder to resolve.\\ \\ 3. Calculate the minimum energy difference: Use the energy-time uncertainty principle ($\Delta E * \Delta t \geq \hbar/2$) with the smaller lifetime to calculate the minimum resolvable energy difference ($\Delta E$). We can approximate this by setting $\Delta E * \Delta t \approx \hbar/2$. We'll need the value of $\hbar$ (Planck's constant divided by $2\pi$) in eV·s.\\ \\ 4. Compare the calculated $\Delta E$ with the choices: Compare the calculated minimum resolvable energy difference ($\Delta E$) with the given options (A) through (D). The correct answer will be the option that is *greater* than the calculated $\Delta E$, ensuring that the energy difference is large enough to resolve the two levels.\\ \\ 5. Select the correct answer: Based on the comparison in step 4, choose the option that represents an energy difference greater than the minimum resolvable energy difference. \\ \midrule
\textbf{Plan Execution:}\\ \\ 1. Identify the relevant physics principle: The energy-time uncertainty principle ($\Delta E * \Delta t \geq \hbar/2$) governs this situation.\\ \\ 2. Relate lifetimes to time uncertainty: The shorter lifetime, $10^{-9}$ s, is the relevant $\Delta t$ because it leads to a larger energy uncertainty, making resolution more difficult. So, $\Delta t = 10^{-9}$ s.\\ \\ 3. Calculate the minimum energy difference:\\ We have $\Delta E * \Delta t = \hbar/2$. We need $\hbar$ in eV·s. $\hbar = 6.582 \times 10^{-16}$ eV·s.\\ $\Delta E = (\hbar/2) / \Delta t$\\ $\Delta E = (6.582 \times 10^{-16} \text{ eV·s} / 2) / (10^{-9} \text{ s})$\\ $\Delta E = 3.291 \times 10^{-7}$ eV\\ This is approximately $3 \times 10^{-7}$ eV.\\ \\ 4. Compare the calculated $\Delta E$ with the choices: We need an energy difference greater than $3 \times 10^{-7}$ eV to clearly resolve the levels.\\ (A) $10^{-11}$ eV - Too small\\ (B) $10^{-8}$ eV - Too small\\ (C) $10^{-9}$ eV - Too small\\ (D) $10^{-4}$ eV - Large enough\\ \\ 5. Select the correct answer: Option (D) is the only one significantly larger than the calculated minimum energy difference.\\ \\ Answer: D \\ \bottomrule
\end{tabular}
% }
\caption{Examples of generated plans by \plangen{} (Best of $\mathcal{N}$) given physics problem. The example illustrates the plan generation and it's execution for physics problem from GPQA}
\label{tab:gpqa_example}
\end{table*}