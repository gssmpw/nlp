\begin{table*}
\centering
\footnotesize
% \resizebox{\linewidth}{!}{
% \begin{tabularx}{\textwidth}{p{0.98\linewidth}}
\begin{tabular}{p{\linewidth}}
\toprule
\textbf{Question:} For a positive integer $a$, define a sequence of integers $x_{1}, x_{2}, \ldots$ by letting $x_{1}=a$ and $x_{n+1}=2 x_{n}+1$ for $n\geq1$. Let $y_{n}=2^{x_{n}}-1$. Determine the largest possible $k$ such that, for some positive integer $a$, the numbers $y_{1}, \ldots, y_{k}$ are all prime. \\ \midrule

\textbf{Constraints for Verifying the Solution Plan for Prime Sequence Problem}
\\\\
This problem requires a solution plan that likely involves a combination of mathematical reasoning and computational verification. Here's a breakdown of constraints to verify and evaluate such a plan:
\\\\
\textbf{I. Mathematical Reasoning Constraints:}
\\\\
1. \textbf{Correct Recurrence Relation:} The plan must correctly interpret and utilize the given recurrence relations for $x_n$ and $y_n$. Verify that the plan doesn't misinterpret or modify these definitions.

2. \textbf{Mersenne Primes Understanding:} The plan should demonstrate an understanding that $y_n$ are Mersenne numbers (numbers of the form $2^p - 1$). It should acknowledge that for $y_n$ to be prime, $x_n$ must itself be prime (though this is a necessary but not sufficient condition).

3. \textbf{Logical Flow and Justification:} The plan's reasoning should be clear, logical, and well-justified. Each step should follow from the previous ones, with clear explanations for any claims or deductions made. Avoid leaps of logic or unsubstantiated assertions.

4. \textbf{Handling of Base Case ($k=1$):} The plan should explicitly address the trivial case where $k=1$. Since $y_1 = 2^a - 1$, any prime value of 'a' will result in $y_1$ being a potential Mersenne prime. This confirms that $k \ge 1$.

5. \textbf{Strategy for Finding Maximum $k$:} The plan must outline a clear strategy for determining the \textit{largest} possible $k$. This likely involves exploring increasing values of $k$ and checking the primality of the resulting $y_n$ sequence. The plan should explain how it avoids infinite search or unnecessary computations.

6. \textbf{Optimality Argument:} The plan should include an argument (or at least a strategy for developing an argument) for why the found value of $k$ is indeed the \textit{largest} possible. This might involve demonstrating that for $k+1$, no suitable 'a' exists.
\\\\
\textbf{II. Computational Verification Constraints:}
\\\\
1. \textbf{Primality Testing Algorithm:} If the plan involves computational primality testing, it should specify the algorithm used (e.g., probabilistic tests like Miller-Rabin or deterministic tests like AKS). The plan should justify the choice of algorithm based on the size of the numbers being tested and the desired level of certainty.

2. \textbf{Correct Implementation:} If code is involved, it must be correct and free of errors. The plan should include testing procedures to ensure the code accurately implements the chosen primality test and the recurrence relations.

3. \textbf{Computational Resources:} The plan should consider the computational resources required for the primality testing. For large values of $x_n$, testing the primality of $y_n$ can be computationally expensive. The plan should address how it manages these computational challenges.

4. \textbf{Range of 'a' considered:} The plan should specify the range of values for 'a' that will be explored. It should justify this range and explain why exploring values outside this range is unnecessary.

5. \textbf{Termination Condition:} The plan must have a clear termination condition for its computational search. It should explain how it determines when to stop searching for larger values of 'k'.
\\\\
\textbf{III. Presentation and Clarity Constraints:}
\\\\
1. \textbf{Clear and Concise Explanation:} The plan should be presented in a clear, concise, and easy-to-understand manner. Avoid unnecessary jargon or overly complex language.

2. \textbf{Well-Organized Structure:} The plan should be well-organized, with a logical flow of ideas. Use headings, subheadings, and bullet points to improve readability.

3. \textbf{Reproducibility:} The plan should be sufficiently detailed to allow another person to reproduce the results. All assumptions, parameters, and procedures should be clearly stated.

 \\ \bottomrule
\end{tabular}
% \end{tabularx}
% }
\caption{Examples of constraints generated by the constraint agent given MATH reasoning problem. The example illustrates the constraint generation for solving MATH problem from OlympiadBench}
\label{tab:olympiad_math_constraints}
\end{table*}