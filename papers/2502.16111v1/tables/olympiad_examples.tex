\begin{table*}
\centering
\scriptsize
% \resizebox{\linewidth}{!}{
% \begin{tabularx}{\textwidth}{p{0.98\linewidth}}
\begin{tabular}{p{\linewidth}}
\toprule
\textbf{Question:} For a positive integer $a$, define a sequence of integers $x_{1}, x_{2}, \ldots$ by letting $x_{1}=a$ and $x_{n+1}=2 x_{n}+1$ for $n\geq1$. Let $y_{n}=2^{x_{n}}-1$. Determine the largest possible $k$ such that, for some positive integer $a$, the numbers $y_{1}, \ldots, y_{k}$ are all prime. \\ \midrule
\textbf{Plan for Solving the IMO Problem:}\\ 1. Problem Understanding: The problem defines a recursive sequence `x\_n' based on an initial value `a', and a related sequence `y\_n' based on `x\_n'. The goal is to find the largest `k' such that `y\_1', `y\_2', ..., `y\_k' can all be prime numbers for some choice of `a'.\\2. Key Information and Variables:\\ `a': A positive integer, the starting value for the sequence `x\_n'. This is the variable we need to choose optimally.\\ `x\_n': Defined recursively as `x\_1 = a' and `x\_(n+1) = 2x\_n + 1'.\\ `y\_n': Defined as `y\_n = 2\textasciicircum{}(x\_n) - 1'.  This is the sequence we need to check for primality.\\ `k': The length of the subsequence of `y\_n' consisting only of primes. We want to maximize this.\\3. Problem Type and Method: This is a number theory problem involving sequences, primality, and recursion. The method will involve a combination of:\\ Analyzing the structure of `x\_n' and 'y\_n':  Finding a closed-form expression for 'x\_n' in terms of 'a' and `n` will be crucial. This will help us understand the structure of `y\_n'.\\ Primality testing/properties: We'll need to leverage known properties of prime numbers, particularly those related to Mersenne primes (primes of the form 2\textasciicircum{}p - 1, where p is prime).\\ Case analysis/exploration: We might need to explore small values of `a' and `k' to identify patterns and formulate conjectures.\\ Proof by contradiction:  To establish the largest possible `k`, we may need to assume a larger value is possible and derive a contradiction.\\ 4. Solution Process:\\ Step 1: Derive a closed-form expression for `x\_n`.  The recursive definition suggests a linear recurrence relation. We can solve it to express `x\_n` directly in terms of `a` and `n`.\\ Step 2: Substitute `x\_n` into the definition of `y\_n`. This gives us a formula for `y\_n` in terms of `a` and `n`.\\ Step 3: Analyze the primality conditions for `y\_n`. Given that `y\_n` has the form 2\textasciicircum{}m - 1,  we know that for `y\_n` to be prime, `m` (which is `x\_n`) must itself be prime.  This suggests focusing on values of `a` that lead to prime values of `x\_n`.\\ Step 4: Explore small values of `a` and `k`. This helps to identify patterns and limitations on the possible values of `k`.\\ Step 5:  Investigate divisibility properties. Consider the factors of `y\_n` and how they might relate to previous terms in the sequence.  Look for patterns in the factors that might prevent multiple consecutive `y\_n` from being prime.\\ Step 6: Formulate a conjecture for the maximum `k`. Based on the observations from the previous steps, propose a value for the maximum `k`.\\ Step 7: Prove the conjecture. Use a proof by contradiction. Assume a larger `k` is possible and demonstrate that this leads to a contradiction based on the divisibility properties of `y\_n`. This will establish the proposed value of `k` as the maximum. \\ \midrule
\textbf{Plan Execution:}\\ 1. Closed-form for $x_n$: The recurrence relation $x_{n+1} = 2x_n + 1$ with $x_1 = a$ can be solved as follows:\\ $x_2 = 2a + 1$\\ $x_3 = 2(2a+1) + 1 = 4a + 3 = 2^2a + 2^2 - 1$\\ $x_4 = 2(4a+3) + 1 = 8a + 7 = 2^3a + 2^3 - 1$\\ By induction, we can show that $x_n = 2^{n-1}a + 2^{n-1} - 1 = 2^{n-1}(a+1) - 1$.\\ 2. $y_n$ in terms of $a$ and $n$: Substituting the expression for $x_n$ into $y_n = 2^{x_n} - 1$, we get $y_n = 2^{2^{n-1}(a+1) - 1} - 1$.\\ 3. Primality of $y_n$: For $y_n$ to be prime, $x_n$ must be prime.  Let $x_n = p_n$, where $p_n$ is a prime number.\\ 4. Exploring small values:\\ If $a=1$, $x_1 = 1$, $y_1 = 2^1 - 1 = 1$, which is not prime.\\ If $a=2$, $x_1 = 2$, $y_1 = 2^2 - 1 = 3$, which is prime.\\ $x_2 = 2(2) + 1 = 5$, $y_2 = 2^5 - 1 = 31$, which is prime.\\ $x_3 = 2(5) + 1 = 11$, $y_3 = 2^{11} - 1 = 2047 = 23 \cdot 89$, which is not prime.  So, for $a=2$, $k=2$.\\ 5. Divisibility:\\ If $x_1 = a$ is prime, say $a=p$, then $x_2 = 2p+1$.  If $x_2$ is also prime, then $y_1$ and $y_2$ are potentially Mersenne primes.  However, $x_3 = 2(2p+1)+1 = 4p+3$.  If $p=2$, $x_3 = 11$, and $y_3$ is not prime.  If $p=3$, $x_2 = 7$, $x_3 = 15$, which is not prime.  If $p=5$, $x_2 = 11$, $x_3 = 23$, $x_4 = 47$, $x_5 = 95$, which is not prime.\\ \\ If $k=3$, we need $x_1=a$, $x_2=2a+1$, and $x_3=4a+3$ to all be prime.  If $a=2$, $x_1=2$, $x_2=5$, $x_3=11$.  But $y_3 = 2^{11}-1 = 2047 = 23 \cdot 89$, which is not prime.\\ \\ 6. Conjecture: The largest possible $k$ is 2.\\ 7. Proof: We have shown that for $a=2$, $y_1$ and $y_2$ are prime.  We need to show that $y_3$ cannot be prime.  $x_3 = 4a+3$.  If $a \equiv 2 \pmod{3}$, then $4a+3 \equiv 4(2)+3 \equiv 11 \equiv 2 \pmod{3}$.  If $a \equiv 1 \pmod{3}$, then $4a+3 \equiv 4(1)+3 \equiv 7 \equiv 1 \pmod{3}$.  If $a \equiv 0 \pmod{3}$, then $4a+3 \equiv 3 \equiv 0 \pmod{3}$.  So, if $a>3$ and $a \equiv 0 \pmod{3}$, then $x_3$ is divisible by 3 and greater than 3, so it is not prime.  If $a=3$, $x_3 = 15$, not prime.  If $a=2$, $x_3 = 11$, and $y_3 = 2047 = 23 \cdot 89$.\\ \\ So the final answer is $\boxed{2}$. \\ \bottomrule
\end{tabular}
% \end{tabularx}
% }
\caption{Examples of generated plans by \plangen{} (Best of $\mathcal{N}$) given maths problem. The example illustrates the plan generation and it's execution for problem from OlympiadBench (MATH)}
\label{tab:olympiad_math_example}
\end{table*}