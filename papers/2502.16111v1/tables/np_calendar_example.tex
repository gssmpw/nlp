\begin{table*}
\centering
\scriptsize
% \resizebox{\linewidth}{!}{
\begin{tabular}{p{\linewidth}}
\toprule
\textbf{Problem Statement:} You are an expert at scheduling meetings. You are given a few constraints on the existing schedule of each participant, the meeting duration, and possibly some preferences on the meeting time. Note there exists a solution that works with existing schedule of every participant.
\\\\
\textbf{TASK:} You need to schedule a meeting for Michelle, Steven and Jerry for one hour between the work hours of 9:00 to 17:00 on Monday.
\\\\
Here are the existing schedules for everyone during the day:

- Michelle has meetings on Monday during 11:00 to 12:00;

- Steven has blocked their calendar on Monday during 9:00 to 9:30, 11:30 to 12:00, 13:30 to 14:00, 15:30 to 16:00;

 -Jerry has blocked their calendar on Monday during 9:00 to 9:30, 10:00 to 11:00, 11:30 to 12:30, 13:00 to 14:30, 15:30 to 16:00, 16:30 to 17:00;
\\\\
Find a time that works for everyone's schedule and constraints.
\\ \midrule
\textbf{Instance-Specific Constraints and Details:}
\\\\
Participants: Michelle, Steven, Jerry - The meeting must include all three.
\\\\
Meeting Duration: 1 hour - The meeting requires a contiguous 60-minute slot.
\\\\
Work Hours: 9:00 to 17:00 (Monday) -  The meeting must start and end within these working hours.  This acts as a global constraint.
\\\\
Individual Schedules (Existing Meetings/Blocked Times):

    - Michelle: Busy from 11:00 to 12:00 on Monday.
    
    - Steven: Busy from 9:00 to 9:30, 11:30 to 12:00, 13:30 to 14:00, and 15:30 to 16:00 on Monday.
    
    - Jerry: Busy from 9:00 to 9:30, 10:00 to 11:00, 11:30 to 12:30, 13:00 to 14:30, and 15:30 to 17:00 on Monday (note this constraint ends at the end of the workday).
\\\\
While the problem states "Note there exists a solution," this isn't a constraint in itself; it's a guarantee that a feasible solution exists given the constraints. The actual constraints are the schedules, meeting duration, and working hours.
\\ \midrule

\textbf{Generated Plan:}
\\\\
1. Analyze Constraints:

- Participants: Michelle, Steven, Jerry

- Duration: 1 hour

- Work hours: 9:00 - 17:00

- Day: Monday
\\\\
2. Identify Individual Availability:

- Michelle: 9:00-11:00, 12:00-17:00

- Steven: 9:30-11:30, 12:00-13:30, 14:00-15:30, 16:00-17:00

- Jerry: 9:30-10:00, 11:00-11:30, 12:30-13:00, 14:30-15:30, 16:00-16:30
\\\\
3. Determine Common Availability: Looking for 1-hour slots:

- Comparing Michelle and Steven: 9:30-11:00, 12:00-13:30, 14:00-15:30, 16:00-17:00

- Comparing the above with Jerry:  14:30-15:30
\\\\
4. Prioritize Earliest Time Slot: The earliest 1-hour slot is 14:30-15:30
\\\\
5. Ensure Solution Exists:  The slot 14:30-15:30 works for all three.
\\\\
6. Strict Format Requirement:

SOLUTION: Here is the proposed time: Monday, 14:30 - 15:30 \\ \midrule

\textbf{Verification:}
\\\\
The plan demonstrates significant improvement compared to the initial plan.  It correctly identifies individual availabilities and successfully finds the two viable one-hour slots: 9:30-10:30 and 14:30-15:30.  It avoids exploring irrelevant shorter slots.  The plan is well-structured and clearly explains each step.  It's on the verge of providing the correct solution.  However, it has not yet explicitly stated the final answer (the prioritized earliest time slot) which is a key part of the task.  Therefore, while much improved, it doesn't quite merit a top score due to this omission.
\\\\
\textbf{Score: 95}

\\ \bottomrule
\end{tabular}
% }
\caption{Examples of verification generated by the verification agent given generated plan and constraints. This example illustrates verification process for calendar scheduling from NATURAL PLAN}
\label{tab:np_cal_verification_example}
\end{table*}