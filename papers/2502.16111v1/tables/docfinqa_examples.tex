\begin{table*}
\centering
\footnotesize
% \resizebox{\linewidth}{!}{
\begin{tabular}{p{\linewidth}}
\toprule
\textless{}Entire Document\textgreater\\ \\ \textbf{Question:} What is the percentage change in cash flow hedges in 2011 compare to the 2010? \\ \midrule
\textbf{Plan to solve the problem:}\\ \\ 1. Locate the information on cash flow hedges: Search the document for "cash flow hedges" or similar terms related to derivative instruments used for hedging cash flows.  Focus on sections discussing derivative instruments, hedging activities, and related notional amounts.\\ \\ 2. Identify the notional amounts for 2011 and 2010: Once the relevant section is found (Note 2i), extract the notional amounts of the cash flow hedges specifically for fiscal year 2011 and fiscal year 2010.\\ \\ 3. Calculate the percentage change: Use the formula for percentage change: {[}(Notional Amount 2011 - Notional Amount 2010) / Notional Amount 2010{]} * 100.  This will provide the percentage change in cash flow hedges from 2010 to 2011. \\ \midrule
\textbf{Plan Execution:}\\ \\ 1. Locate the information on cash flow hedges: The information on cash flow hedges was located in a section discussing derivative instruments and hedging activities. Specifically, it was found within a sentence detailing the notional amounts of various hedging instruments.\\ \\    Code (if needed):\\    ```python\\    \# No code needed for this step.\\    ```\\    Result: Information on cash flow hedges located successfully.\\ \\ 2. Identify the notional amounts for 2011 and 2010: The notional amounts for cash flow hedges were identified as \$153.7 million for 2011 and \$139.9 million for 2010.\\ \\    Code (if needed):\\    ```python\\    \# No code needed for this step.\\    ```\\    Result: 2011 notional amount: \$153.7 million; 2010 notional amount: \$139.9 million.\\ \\ 3. Calculate the percentage change:\\ \\    Code (if needed):\\    ```python\\    notional\_2011 = 153.7\\    notional\_2010 = 139.9\\    percentage\_change = ((notional\_2011 - notional\_2010) / notional\_2010) * 100\\    rounded\_percentage\_change = round(percentage\_change, 1)\\    print(f"\{rounded\_percentage\_change\}\%")\\    ```\\ \\ Answer: 9.9\% \\ \bottomrule
\end{tabular}
% }
\caption{Examples of generated plans by \plangen{} (Best of $\mathcal{N}$) given financial reasoning problem. The example illustrates the plan generation and it's execution for problem from DocFinQA}
\label{tab:docfinqa_example}
\end{table*}