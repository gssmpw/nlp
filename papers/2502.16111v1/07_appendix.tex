\section{Further Details on LLM Agents}
\label{app:llm_agents}

In this section, we provide additional details about each specialized agent in \plangen{}. We present the prompts used for each agent, highlighting their roles in the framework. The prompt for the constraint agent includes task-specific parameters that can be adjusted to extract relevant constraints for different tasks. In contrast, the prompts for the verification agent and selection agent are entirely task-agnostic, ensuring generalizability and adaptability across various problem domains.

\paragraph{Prompts for Constraint Agent}
The constraint agent is responsible for extracting problem-specific constraints that guide the planning process. To enable systematic extraction of constraints, we design a task-specific prompt for the constraint agent:

% enhanced,
% boxrule=0pt,frame hidden,
% borderline west={4pt}{0pt}{green!75!black},
% top=0pt,bottom=0pt,
% colback=green!10!white,
% sharp corners

\begin{tcolorbox}[boxrule=0pt, frame hidden, title=Prompt, breakable, sharp corners, borderline west={0pt}{0pt}{black!50}, title style={
        colback=black!50, % Example: Light gold title background
        colframe=black!50, % Match title frame to background
        coltitle=black % Black title text
    }]
You are an expert in understanding an input problem and generating set of constraints. Analyze the input problem and extract all relevant instance-specific constraints and contextual details necessary for accurate and feasible planning. 
\\
\\
(\hl{Optional}) These constraints may include:
\\

<You may provide any specific type of constraints>
\\

<You may provide any formatting instruction>
\\

\textbf{Input Problem:} <problem statement>
\end{tcolorbox}

Based on the above prompts, we define the types of constraints used in the NATURAL PLAN benchmark for different planning tasks: calendar scheduling, meeting planning, and trip planning. For DocFinQA, we provide a set of formatting instructions to ensure structured constraint generation. For GPQA and OlympiadBench, the constraint extraction follows the general prompt outlined above.

\paragraph{Prompts for Verification Agent}

The prompt for the verification agent is designed to be task-agnostic, meaning it can be applied across different problem domains without modification. By enforcing strict evaluation criteria, this agent enhances the reliability of \plangen{}, making it robust for various planning and reasoning tasks. In this prompt, list of constraints are generated using constraint agent. Notably, the list of constraints used in the verification prompt is dynamically generated by the constraint agent. This ensures that the verification process is based on instance-specific constraints rather than relying on predefined, static rules.

\begin{tcolorbox}[boxrule=0pt, frame hidden, title=Prompt, breakable, sharp corners, borderline west={2pt}{0pt}{black!50}, title style={
        colback=black!50, % Example: Light gold title background
        colframe=black!50, % Match title frame to background
        coltitle=black % Black title text
    }]

Provide a reward score between -100 and 100 for the quality of the provided plan steps, using strict evaluation standards. Ensure the reward reflects how effectively the plan contributes to progressing toward the correct solution.

\textbf{Problem Statement:}

\{problem\}

\textbf{Plan:}

\{plan\}

\textbf{Consider the following constraints while evaluating:}

- [Constraint 1]

- [Constraint 2]

- [Constraint 3]

\textbf{Provide feedback in the following format:}

[Step-by-step reasoning for the reward score]

\textbf{Score:} [Strictly provide an integer reward score between -100 and 100]
\end{tcolorbox}

\paragraph{Prompts for Selection Agent}

The prompt for the Selection Agent is task-agnostic, allowing it to be applied across various domains without modification. It processes feedback from the verification agent and contextual information from the problem statement to assign suitability scores to different inference-time algorithms.

\begin{tcolorbox}[boxrule=0pt, frame hidden, title=Prompt, breakable, sharp corners, borderline west={0pt}{0pt}{black!50}, title style={
        colback=black!50, % Example: Light gold title background
        colframe=black!50, % Match title frame to background
        coltitle=black % Black title text
    }]

Analyze the following planning problem and explain your reasoning for assigning priority scores to the algorithms based on their suitability. Scores should be between 0 and 1, where 1 represents the most suitable algorithm for the given problem.

\textbf{Problem Statement:} <problem statement>

\textbf{Requirements:} <feedback>

\textbf{Context:} <context if context else `None provided'>

Start by providing a brief reasoning for each algorithm's suitability based on problem complexity. Then, \textbf{ONLY output your response strictly as a list} with the exact format below:

\textbf{Reasoning:}
\begin{itemize}
    \item \textbf{Best of N:} [Explain why this algorithm is or isn’t suitable]
    \item \textbf{Rebase:} [Explain why this algorithm is or isn’t suitable]
    \item \textbf{ToT:} [Explain why this algorithm is or isn’t suitable]
\end{itemize}

\textbf{Scores:}
\begin{equation*}
\begin{aligned}
&[\text{("Best of N", float)}, \\
&\text{("Rebase", float)}, \\
&\text{("ToT", float)}]
\end{aligned}
\end{equation*}

\end{tcolorbox}

\paragraph{Algorithm for Selection using UCB}

The algorithm (Algorithm \ref{algo:selection}) presented is a modified UCB selection strategy that incorporates additional factors for exploration, diversity, and recovery. It initializes each algorithm with basic statistics like reward ($R(a)$), count of trials ($C(a)$), and recovery score ($Rec(a)$). The algorithm computes a normalized reward $\bar{R}_{\text{norm}}(a)$ for each option, balancing the reward with exploration ($E(a)$), which encourages trying less-used algorithms. A diversity bonus $D(a)$ penalizes overused algorithms, while a recovery bonus $RecB(a)$ rewards algorithms that perform well after prior failures. LLM-guided priors ($LLM\_prior$) are used to influence the selection process based on prior knowledge. The final selection is made by maximizing the UCB score, which combines these factors to balance exploitation and exploration.

\paragraph{Ablation Study on UCB Modifications}
\begin{wrapfigure}{r}{0.5\textwidth}
    \centering
    \vspace{-9mm}
    \includegraphics[width=\linewidth]{images/analysis_ucb.pdf}
    % \vspace{-5mm}
    \caption{Ablation Study of UCB Modifications on Selection Agent and its impact on Multi-Agent Mixture of Algorithms framework. div.: diversity bonus, rec: recovery term.}
    \label{fig:ucb_analysis}
\end{wrapfigure}
To design our selection agent, we conducted an ablation study evaluating modifications to the UCB formula, shown in Figure \ref{fig:ucb_analysis}.  Initially, we replaced the selection agent with a simple sequential strategy, termed ``Multi-Agent (Sequential)'', where algorithms execute in sequence, and the verification agent selects the highest-scoring plan.  Next, we implemented a UCB selection agent, but excluded the `diversity bonus' and `recovery term' introduced in our proposed formulation in the main paper, denoted as ``Multi-Agent (UCB w/o div. and rec.)''. Finally, we implemented the complete selection agent incorporating our proposed UCB, labeled ``Multi-Agent (UCB)''.  As shown in Figure \ref{fig:ucb_analysis}, the inclusion of the diversity bonus and recovery terms in the UCB formula ("Multi-Agent (UCB)") resulted in $\sim3.5\%$ performance gain compared to the UCB variant without these terms, further enhancing overall results. Note that the LLM-guided priors are still the part of Multi-Agent (UCB w/o div. and rec.) and Multi-Agent (UCB).

\section{Prompts for Proposed Frameworks}
\label{app:frameworks}

We provide further details in this section regarding the prompts used for \plangen{} (ToT) and \plangen{} (REBASE), as well as the specific algorithms used to execute these inference-time methods.

% \paragraph{Algorithms for Multi-Agent ToT and REBASE} 

% Here, Algorithms 2 and 3 detail the execution pipelines for Multi-Agent ToT and REBASE, respectively.

\paragraph{Prompts used for ToT and REBASE}

\plangen{} (ToT) and \plangen{} (REBASE) employ three prompt types: (1) step prompt, (2) step reward prompt, and (3) completion prompt. Step prompt guide the model to generate subsequent steps based on the problem statement and previously generated steps. Step reward prompt evaluate each intermediate step against the problem statement and constraints, similar to the prompts used by a verification agent. Completion prompt check for a complete solution after each step. If a solution is found, exploration terminates; otherwise, the process continues until a solution is reached.

\begin{tcolorbox}[boxrule=0pt, frame hidden, title=Step Prompt, breakable, sharp corners, borderline west={0pt}{0pt}{black!50}, title style={
        colback=black!50, % Example: Light gold title background
        colframe=black!50, % Match title frame to background
        coltitle=black % Black title text
    }]

You are an expert assistant for generating step-by-step plan to solve a given question using specified tools. Given the problem and any intermediate steps, output only the next step in the plan. Ensure that the next action helps in moving toward the correct plan to solve the given question. Do not provide the full plan. Keep responses concise, focusing solely on the immediate next step that is most effective in progressing toward the correct plan.
\\
\\
<problem>

\{Add a problem statement here\}

</problem>
\\
\\
<intermediate\_step>

\{Append previously generated steps\}

</intermediate\_step>
\end{tcolorbox}

\begin{tcolorbox}[boxrule=0pt, frame hidden, title=Completion Prompt, breakable, sharp corners, borderline west={0pt}{0pt}{black!50}, title style={
        colback=black!50, % Example: Light gold title background
        colframe=black!50, % Match title frame to background
        coltitle=black % Black title text
    }]

You are an assistant tasked with verifying if the final, complete plan to solve the given question has been achieved within the intermediate steps. Output only `1' if the intermediate steps contain the full solution needed to solve the question. If the full plan has not yet been reached, output only `0'. Provide no additional commentary—return exclusively `1' or `0'.
\\
\\
<problem>

\{Add a problem statement here\}

</problem>
\\
\\
<intermediate\_step>

\{Append previously generated steps\}

</intermediate\_step>
\end{tcolorbox}


\section{Details on Benchmarks and Experiments}
\label{app:experiments}

\paragraph{Statistics of Benchmarks}

For evaluation, we utilize evaluation sets of all four benchmarks. For NATURAL PLAN, we employed the provided evaluation sets, consisting of 1000 instances each for Calendar Scheduling and Meeting Planning, and 1600 instances for Trip Planning.  The GPQA evaluation was conducted using the Diamond set, which comprises 198 highly challenging instances.  From OlympiadBench, we selected the text-only problems, excluding those requiring a theorem prover, resulting in 674 instances for the MATH category and 236 for the PHY category. We also used 922 instances from the DocFinQA evaluation set.

\paragraph{Models}
Our primary evaluations use Gemini-1.5-Pro for all the experiments.  We also present a case study with Gemini-2.0-Flash and GPT-4o to showcase the model-agnostic nature of \plangen{}.

\paragraph{Metrics}
We use task-specific metrics for all evaluations. Specifically, we use Exact Match (EM) for NATURAL PLAN similar to \citet{zheng2024natural}, micro-average accuracy for OlympiadBench similar to \citet{he-etal-2024-olympiadbench}, and accuracy for GPQA and DocFinQA (along with F1-Score for DocFinQA).

\paragraph{Feedback prompt for Multi-Agent Baseline} In the multi-agent baseline, we employ a feedback prompt to iteratively generate improved and refined outputs. The prompt is provided below:

\begin{tcolorbox}[boxrule=0pt, frame hidden, title=Feedback Prompt, breakable, sharp corners, borderline west={0pt}{0pt}{black!50}, title style={
        colback=black!50, % Example: Light gold title background
        colframe=black!50, % Match title frame to background
        coltitle=black % Black title text
    }]

Analyze the following planning problem and explain your reasoning for assigning priority scores You are an intelligent assistant capable of self-reflection and refinement. I will provide you with your last response, and your task is to improve it, if necessary.

Here is your previous response:

\{previous\_response\}

Analyze and refine your response step-by-step:

1. Reflect on your reasoning process. Where might it be unclear or incorrect? Improve it.

2. Revise the explanation to address any identified issues and make it more logical and comprehensive.

3. Ensure the final answer is correct, supported by clear reasoning.
\end{tcolorbox}

\paragraph{Hyper-parameters for Experiments}
To ensure deterministic behavior, we set the temperature of all models to 0 for each agent.  For the inference-time algorithms, we used the following settings: \plangen{} (Best of $\mathcal{N}$) with five samples at a temperature of 0.7; Tree of Thoughts (ToT) with three children per root node, generated at a temperature of 0.7; and REBASE, initialized with width $10$ at temperature of 0.7, decremented by 1 after each call to expand.

\section{Additional Analysis}
\label{app:analysis}

\paragraph{Importance of Verification Agent}
The kernel density estimation (KDE) plot visualizes the distribution of reward values assigned to two distinct outcomes: ``Success'' (green) and ``Failure'' (red).  The plot reveals a clear separation between the reward distributions, with ``Success'' outcomes strongly associated with high reward values (around 80-100) and ``Failure'' outcomes primarily associated with low reward values (around 20-40). The sharply peaked green curve suggests consistent and high rewards for successful outcomes, while the broader red curve reflects more variability in rewards assigned to failures.  However, a small bump in the red curve at high reward values (around 80-90) suggests a few instances where failures received unexpectedly high rewards, warranting further investigation. 
\begin{wrapfigure}{r}{0.5\textwidth}
    \centering
    \vspace{2mm}
    \includegraphics[width=\linewidth]{images/kde_plot.pdf}
    % \vspace{2mm}
    \caption{KDE plot illustrating relationship between reward value and outcome (success/failure).}
    \label{fig:kde_plot}
\end{wrapfigure}This observation is further supported by a statistically significant difference between the reward distributions, a Mann-Whitney U test ($U = 116128.0$, $p < 0.0001$). The low p-value (3.42e-09) provides evidence that the difference in reward distributions is statistically significant.

\paragraph{Performance of our frameworks w.r.t. different complexity}

From Figure \ref{fig:np_meet_trip_analysis}, in the meeting planning, \plangen{} (Best of $\mathcal{N}$) excels in both simple and intermediate problems, whereas a \plangen{} (Mixture of Algo.) performs better for complex problems. The trip planning presents a different trend, where \plangen{} (Best of $\mathcal{N}$) and a \plangen{} (Mixture of Algo.) consistently outperform other approaches across all complexity levels. Nonetheless, in very complex problems for both meeting and trip planning, all algorithms exhibit poor performance.

\begin{figure}
    \centering
    \includegraphics[width=\textwidth]{images/analysis_2.pdf}
    \caption{Performance comparison of inference-time algorithms across different complexity levels for meeting and trip planning from NATURAL PLAN}
    \label{fig:np_meet_trip_analysis}
\end{figure}

\section{Hyperparameter Search}\label{app:hype}
\normalsize
We exclusively conduct hyperparameter search on fold 0. 
For \textbf{GraFITi}~\citep{Yalavarthi2024.GraFITi} the hyperparameters for the search are as follows:
\begin{itemize}
    \item The number of layers, with possible values [1, 2, 3, 4].
    \item The number of attention heads, with possible values [1, 2, 4].
    \item The latent dimension, with possible values [16, 32, 64, 128, 256].
\end{itemize}

For the \textbf{LinODEnet} model~\citep{Scholz2022.Latenta} we search the hyperparameters from:
\begin{itemize}
    \item The hidden dimension, with possible values [16, 32, 64, 128].
    \item The latent dimension, with possible values [64, 128, 192, 256].
\end{itemize}

For \textbf{GRU-ODE-Bayes}~\citep{DeBrouwer2019.GRUODEBayesd} we tune the hidden size from [16, 32, 64, 128, 256]

For \textbf{Neural Flows}~\citep{Bilos2021.Neurald} we define the hyperparameter spaces for the search are as follows:
\begin{itemize}
    \item The number of flow layers, with possible values [1, 2, 4].
    \item The hidden dimension, with possible values [16, 32, 64, 128, 256].
    \item The flow model type, with possible values [GRU, ResNet].
\end{itemize}

For the \textbf{CRU}~\citep{Schirmer2022.Modelingb} the hyperparameter space is as follows:
\begin{itemize}
    \item The latent state dimension, with possible values [10, 20, 30].
    \item The number of basis functions, with possible values [10, 20].
    \item The bandwidth with possible values [3, 10].
\end{itemize}


\paragraph{Different hyper-parameters of inference-time algorithms vs. their performance}

We conduct a case study on OlympiadBench, where we analyze the impact of varying hyper-parameters on the performance of different inference-time algorithms. The results (Table \ref{tab:hyperparameters}) indicate that while increasing the number of samples (Best of $\mathcal{N}$), steps (ToT), or refinements (REBASE) lead to marginal improvements, the overall differences remain relatively small. Given this, we opted for lower hyper-parameter values across all inference-time algorithms to balance efficiency and performance.

\paragraph{Frequency of inference-time algorithm selection across datasets}

For the \plangen{} (Mixture of Algo.) method, we analyze how frequently each inference-time algorithm (Best of $\mathcal{N}$, ToT, and REBASE) is selected across different datasets. The results (shown in Table \ref{tab:algo_selection}) show that \plangen{} (ToT) is the most frequently chosen algorithm in NATURAL PLAN, OlympiadBench, and GPQA, indicating its effectiveness in these domains. In contrast, for DocFinQA, \plangen{} (Best of $\mathcal{N}$) is the dominant choice, suggesting that its strategy aligns better with financial reasoning tasks. \plangen{} (REBASE) is selected the least across all datasets, implying that its refinements are less favored by the selection mechanism. These findings highlight the dataset-dependent nature of inference-time algorithm effectiveness and the adaptability of the mixture approach in dynamically choosing the most suitable method.


\begin{algorithm*}
% \footnotesize
\caption{Selection using Modified UCB with LLM-Guided Priors}
\begin{algorithmic}[1]
\State \textbf{Initialize:} $R(a) \gets 0$, $C(a) \gets 1$, $Rec(a) \gets 0$, $F(a) \gets 0$, $D(a) \gets 1$, $T \gets 0$
\State Set $\lambda_{\text{prior}}$, $\alpha_{\text{diversity}}$, $\alpha_{\text{recovery}}$
\State Load LLM-guided priors

\Procedure{SelectAlgorithm}{args}
    \State Compute prior decay: $\lambda_{\text{prior}} \gets \frac{\lambda_{\text{prior}}}{1 + T}$ \Comment{Reduces as trials increase}
    \State Set max exploration term $M \gets 5$
    \State Obtain LLM prior scores: $LLM\_prior \gets \text{LLM\_Guided\_Prior}(args)$
    \State Compute max reward: $R_{\max} \gets \max(R(a))$ (set to 1 if all rewards are 0)

    \For{each algorithm $a$}
        \State Compute normalized reward:
        \[
        \bar{R}_{\text{norm}}(a) \gets \frac{R(a)}{C(a) R_{\max}}
        \]
        \Comment{Scales rewards between 0 and 1 for comparability}

        \State Compute exploration term:
        \[
        E(a) \gets \min\left(\sqrt{\frac{2 \log(T+1)}{C(a)}}, M\right)
        \]
        \Comment{Encourages trying less-used algorithms, capped at $M$}

        \State Compute diversity bonus:
        \[
        D(a) \gets \frac{\alpha_{\text{diversity}}}{C(a) + 1}
        \]
        \Comment{Penalizes frequently used algorithms to encourage variety}

        \State Compute recovery bonus:
        \[
        RecB(a) \gets \alpha_{\text{recovery}} \cdot Rec(a)
        \]
        \Comment{Rewards algorithms that perform well after failures}

        \State Compute final UCB score:
        \[
        UCB(a) \gets \bar{R}_{\text{norm}}(a) + E(a) + \lambda_{\text{prior}} LLM\_prior(a) + D(a) + RecB(a)
        \]
        \Comment{Balances exploitation, exploration, diversity, and recovery}
    \EndFor
    
    \State Select best algorithm:
    \[
    a^* \gets \arg\max_{a} UCB(a)
    \]
    \State \Return $(a^*, UCB(a^*))$
\EndProcedure

\end{algorithmic}
\label{algo:selection}
\end{algorithm*}


\section{Various Examples for Different Components of \plangen{}}
\label{app:examples}

\paragraph{Examples for Constraint Agent}
To illustrate the output of our constraint agent, Table \ref{tab:np_constraints_examples}, Table \ref{tab:gpqa_constraints_examples}, and Table \ref{tab:olympiad_math_constraints} present representative examples of generated constraints. These tables highlight the diverse constraints generated for problem instances of different tasks.

\begin{wraptable}{r}{0.5\textwidth}
\centering
\vspace{-5mm}
\footnotesize
\resizebox{\linewidth}{!}{
\begin{tabular}{l|cccc}
\toprule
Frameworks & NATURAL PLAN & OlympiadBench & GPQA & DocFinQA \\
\midrule
\plangen{} (Best of $\mathcal{N}$) & 19.55\% & 7.09\% & 8.56\% & 81.03\% \\
\plangen{} (ToT) & 68.85\% & 90.09\% & 85.59\% & 12.5\% \\
\plangen{} (REBASE) & 11.6\% & 2.82\% & 5.86\% & 6.47\% \\
\bottomrule
\end{tabular}
}
\caption{Algorithm Selection Frequency by Dataset}
\label{tab:algo_selection}
\end{wraptable}

\paragraph{Example for Verification Agent}
To illustrate the output of our verification agent, Table \ref{tab:np_cal_verification_example} presents representative examples of verification process for NATURAL PLAN (calendar scheduling). This table highlights the how the verification agent verifies the generated plan using constraints.

\paragraph{Examples of Generated Plans}
To demonstrate the plan generation process, Table \ref{tab:np_examples}, Table \ref{tab:gpqa_example}, Table \ref{tab:docfinqa_example}, and Table \ref{tab:olympiad_math_example} present example plans for NATURAL PLAN, GPQA, DocFinQA, and OlympiadBench.  Generated using \plangen{} (Best of $\mathcal{N}$), these tables highlight the varied nature of plans produced across different task types. For GPQA, DocFinQA, and OlympiadBench (i.e., downstream reasoning tasks), the examples additionally illustrate how generated plans are executed to derive final answer. 

%More examples for agents and frameworks within \plangen{} are provided at \url{https://anonymous.4open.science/r/plangen-0C99}

\begin{table*}
\centering
\scriptsize
\renewcommand{\arraystretch}{1.2} % Better spacing

\begin{tabularx}{\textwidth}{>{\raggedright}p{0.1\textwidth}|X|X}
\toprule
\textbf{Benchmark}   & \textbf{Problem Statement}  & \textbf{Generated Constraints} \\ \midrule

Calendar Scheduling & You are an expert at scheduling meetings. You are given a few constraints on the existing schedule of each participant, the meeting duration, and possibly some preferences on the meeting time. Note there exists a solution that works with existing schedule of every participant.

------------

\textbf{TASK:} You need to schedule a meeting for Michelle, Steven and Jerry for one hour between the work hours of 9:00 to 17:00 on Monday.

------------

Here are the existing schedules for everyone during the day:

- Michelle has meetings on Monday during 11:00 to 12:00;

- Steven has blocked their calendar on Monday during 9:00 to 9:30, 11:30 to 12:00, 13:30 to 14:00, 15:30 to 16:00;

 -Jerry has blocked their calendar on Monday during 9:00 to 9:30, 10:00 to 11:00, 11:30 to 12:30, 13:00 to 14:30, 15:30 to 16:00, 16:30 to 17:00;

------------

Find a time that works for everyone's schedule and constraints.

& 

\textbf{Instance-Specific Constraints and Details:}

Participants: Michelle, Steven, Jerry - The meeting must include all three.

------------

Meeting Duration: 1 hour - The meeting requires a contiguous 60-minute slot.

------------

Work Hours: 9:00 to 17:00 (Monday) -  The meeting must start and end within these working hours.  This acts as a global constraint.

------------

Individual Schedules (Existing Meetings/Blocked Times):

- Michelle: Busy from 11:00 to 12:00 on Monday.
    
- Steven: Busy from 9:00 to 9:30, 11:30 to 12:00, 13:30 to 14:00, and 15:30 to 16:00 on Monday.
    
- Jerry: Busy from 9:00 to 9:30, 10:00 to 11:00, 11:30 to 12:30, 13:00 to 14:30, and 15:30 to 17:00 on Monday (note this constraint ends at the end of the workday).
    
------------

While the problem states "Note there exists a solution," this isn't a constraint in itself; it's a guarantee that a feasible solution exists given the constraints. The actual constraints are the schedules, meeting duration, and working hours.

\\ \midrule

Meeting Planning & 
You are visiting San Francisco for the day and want to meet as many friends as possible. Solve the problem by considering various different schedules and picking the best one to optimize your goals.

------------

Travel distances (in minutes):

Bayview to Golden Gate Park: 22.

Golden Gate Park to Bayview: 23.

------------

CONSTRAINTS: You arrive at Bayview at 9:00AM. Barbara will be at Golden Gate Park from 8:00AM to 11:30AM. You'd like to meet Barbara for a minimum of 90 minutes.

------------

Your response should start with 'SOLUTION:'

&  

\textbf{Instance-Specific Constraints and Details:}

Travel Times:
    
    - Bayview to Golden Gate Park: 22 minutes.
    
    - Golden Gate Park to Bayview: 23 minutes (Note: This information is provided but not strictly necessary for this specific problem as the return trip isn't relevant to meeting Barbara).

------------

Locations:
    
    - Starting Location: Bayview
    
    - Meeting Location: Golden Gate Park
    
------------

Scheduled Meeting Times/Participant Availability:
    
    - Barbara is available at Golden Gate Park from 8:00 AM to 11:30 AM.

------------

Arrival Time:
    
    - You arrive at Bayview at 9:00 AM.

------------

Minimum Meeting Duration:
    
    - The meeting with Barbara must be at least 90 minutes long.

------------

Sequence Requirements:
    
    - Travel (Bayview to Golden Gate Park) $\rightarrow$ Wait (if necessary, due to arriving before Barbara's availability ends) $\rightarrow$ Meeting (with Barbara)

------------

Objective: Maximize the number of friends met (in this case, it's simplified to just meeting Barbara, so the objective is to determine IF a meeting is possible given the constraints). 

\\ \bottomrule
\end{tabularx}

\caption{Examples of constraints generated by the constraint agent given problem. These examples illustrate constraint generation for calendar scheduling and meeting planning from NATURAL PLAN}
\label{tab:np_constraints_examples}
\end{table*}

\begin{table*}
\centering
\footnotesize
% \resizebox{\linewidth}{!}{
\begin{tabular}{p{\linewidth}}
\toprule
\textbf{Question:} Two quantum states with energies E1 and E2 have a lifetime of $10^{-9}$ sec and $10^{-8}$ sec, respectively. We want to clearly distinguish these two energy levels. Which one of the following options could be their energy difference so that they can be clearly resolved?

(A) $10^{-11}$ eV

(B) $10^{-8}$ eV

(C) $10^{-9}$ eV

(D) $10^{-4}$ eV \\ \midrule

\textbf{Constraints for Verifying the Energy Difference Solution for Resolvable Quantum States}
\\ \\
This problem revolves around the energy-time uncertainty principle, which dictates the resolvability of energy levels given their lifetimes. To verify a proposed energy difference ($\Delta E = |E_2 - E_1|$) that allows clear resolution, we need constraints encompassing the following:
\\\\
1. Uncertainty Principle Adherence:

\textbf{Primary Constraint:} The product of the uncertainty in energy ($\Delta E$) and the uncertainty in time ($\Delta t$) must satisfy the energy-time uncertainty relation: $\Delta E \Delta t \geq \frac{\hbar}{2}$, where $\hbar$ is the reduced Planck constant.

\textbf{Specific Constraint:} Since we have two states, we must consider the \textit{combined} uncertainty in time. We can approximate this as the average of the individual lifetimes: $\Delta t \approx \frac{\tau_1 + \tau_2}{2}$, where $\tau_1$ and $\tau_2$ are the lifetimes of the two states.

\textbf{Operational Constraint:} For "clear resolution," we typically require $\Delta E$ to be significantly greater than the uncertainty dictated by the uncertainty principle. A common rule of thumb is $\Delta E \geq \frac{\hbar}{\Delta t}$. A stricter constraint might use a larger factor (e.g., $\Delta E \geq \frac{2\hbar}{\Delta t}$) for even clearer resolution.
\\\\
2. Units Consistency:

\textbf{Unit Constraint:} All calculations must use consistent units. Ensure that energy ($\Delta E$), time ($\Delta t$), and $\hbar$ are expressed in compatible units (e.g., Joules and seconds, or eV and seconds). Clearly state the chosen unit system.
\\\\
3. Numerical Accuracy:

\textbf{Precision Constraint:} Calculations should be performed with sufficient numerical precision to avoid rounding errors. Define the acceptable level of precision.

\textbf{Constant Accuracy:} Use a precise value for the reduced Planck constant ($\hbar$).
\\\\
4. Solution Justification:

\textbf{Explanation Constraint:} The solution must not only provide a numerical value for $\Delta E$ but also clearly explain why this value satisfies the constraints and allows for clear resolution. This explanation should reference the uncertainty principle and the chosen resolution criterion.
\\\\
5. Alternative Solutions Consideration (Optional):

\textbf{Exploration Constraint:} Explore whether multiple values of $\Delta E$ could satisfy the constraints. If so, discuss the trade-offs between different solutions.
\\\\
\textbf{Example Implementation of Constraints:}
\\\\
Let's assume we're working in SI units (Joules and seconds).

1. Uncertainty Principle: $\Delta E \geq \frac{\hbar}{[(10^{-9} \text{ s} + 10^{-8} \text{ s})/2]}$, where $\hbar \approx 1.054 \times 10^{-34} \text{ J s}$.

2. Units: All calculations are in Joules and seconds.

3. Numerical Accuracy: Calculations will be performed to at least 4 significant figures.

4. Justification: The solution will explain how the calculated $\Delta E$ is significantly larger than the uncertainty, ensuring clear resolution.

 \\ \bottomrule
\end{tabular}
% }
\caption{Examples of constraints generated by the constraint agent given reasoning question. The example illustrates the constraint generation for solving physics question from GPQA}
\label{tab:gpqa_constraints_examples}
\end{table*}

\begin{table*}
\centering
\footnotesize
% \resizebox{\linewidth}{!}{
% \begin{tabularx}{\textwidth}{p{0.98\linewidth}}
\begin{tabular}{p{\linewidth}}
\toprule
\textbf{Question:} For a positive integer $a$, define a sequence of integers $x_{1}, x_{2}, \ldots$ by letting $x_{1}=a$ and $x_{n+1}=2 x_{n}+1$ for $n\geq1$. Let $y_{n}=2^{x_{n}}-1$. Determine the largest possible $k$ such that, for some positive integer $a$, the numbers $y_{1}, \ldots, y_{k}$ are all prime. \\ \midrule

\textbf{Constraints for Verifying the Solution Plan for Prime Sequence Problem}
\\\\
This problem requires a solution plan that likely involves a combination of mathematical reasoning and computational verification. Here's a breakdown of constraints to verify and evaluate such a plan:
\\\\
\textbf{I. Mathematical Reasoning Constraints:}
\\\\
1. \textbf{Correct Recurrence Relation:} The plan must correctly interpret and utilize the given recurrence relations for $x_n$ and $y_n$. Verify that the plan doesn't misinterpret or modify these definitions.

2. \textbf{Mersenne Primes Understanding:} The plan should demonstrate an understanding that $y_n$ are Mersenne numbers (numbers of the form $2^p - 1$). It should acknowledge that for $y_n$ to be prime, $x_n$ must itself be prime (though this is a necessary but not sufficient condition).

3. \textbf{Logical Flow and Justification:} The plan's reasoning should be clear, logical, and well-justified. Each step should follow from the previous ones, with clear explanations for any claims or deductions made. Avoid leaps of logic or unsubstantiated assertions.

4. \textbf{Handling of Base Case ($k=1$):} The plan should explicitly address the trivial case where $k=1$. Since $y_1 = 2^a - 1$, any prime value of 'a' will result in $y_1$ being a potential Mersenne prime. This confirms that $k \ge 1$.

5. \textbf{Strategy for Finding Maximum $k$:} The plan must outline a clear strategy for determining the \textit{largest} possible $k$. This likely involves exploring increasing values of $k$ and checking the primality of the resulting $y_n$ sequence. The plan should explain how it avoids infinite search or unnecessary computations.

6. \textbf{Optimality Argument:} The plan should include an argument (or at least a strategy for developing an argument) for why the found value of $k$ is indeed the \textit{largest} possible. This might involve demonstrating that for $k+1$, no suitable 'a' exists.
\\\\
\textbf{II. Computational Verification Constraints:}
\\\\
1. \textbf{Primality Testing Algorithm:} If the plan involves computational primality testing, it should specify the algorithm used (e.g., probabilistic tests like Miller-Rabin or deterministic tests like AKS). The plan should justify the choice of algorithm based on the size of the numbers being tested and the desired level of certainty.

2. \textbf{Correct Implementation:} If code is involved, it must be correct and free of errors. The plan should include testing procedures to ensure the code accurately implements the chosen primality test and the recurrence relations.

3. \textbf{Computational Resources:} The plan should consider the computational resources required for the primality testing. For large values of $x_n$, testing the primality of $y_n$ can be computationally expensive. The plan should address how it manages these computational challenges.

4. \textbf{Range of 'a' considered:} The plan should specify the range of values for 'a' that will be explored. It should justify this range and explain why exploring values outside this range is unnecessary.

5. \textbf{Termination Condition:} The plan must have a clear termination condition for its computational search. It should explain how it determines when to stop searching for larger values of 'k'.
\\\\
\textbf{III. Presentation and Clarity Constraints:}
\\\\
1. \textbf{Clear and Concise Explanation:} The plan should be presented in a clear, concise, and easy-to-understand manner. Avoid unnecessary jargon or overly complex language.

2. \textbf{Well-Organized Structure:} The plan should be well-organized, with a logical flow of ideas. Use headings, subheadings, and bullet points to improve readability.

3. \textbf{Reproducibility:} The plan should be sufficiently detailed to allow another person to reproduce the results. All assumptions, parameters, and procedures should be clearly stated.

 \\ \bottomrule
\end{tabular}
% \end{tabularx}
% }
\caption{Examples of constraints generated by the constraint agent given MATH reasoning problem. The example illustrates the constraint generation for solving MATH problem from OlympiadBench}
\label{tab:olympiad_math_constraints}
\end{table*}

\begin{table*}
\centering
\scriptsize
% \resizebox{\linewidth}{!}{
\begin{tabular}{p{\linewidth}}
\toprule
\textbf{Problem Statement:} You are an expert at scheduling meetings. You are given a few constraints on the existing schedule of each participant, the meeting duration, and possibly some preferences on the meeting time. Note there exists a solution that works with existing schedule of every participant.
\\\\
\textbf{TASK:} You need to schedule a meeting for Michelle, Steven and Jerry for one hour between the work hours of 9:00 to 17:00 on Monday.
\\\\
Here are the existing schedules for everyone during the day:

- Michelle has meetings on Monday during 11:00 to 12:00;

- Steven has blocked their calendar on Monday during 9:00 to 9:30, 11:30 to 12:00, 13:30 to 14:00, 15:30 to 16:00;

 -Jerry has blocked their calendar on Monday during 9:00 to 9:30, 10:00 to 11:00, 11:30 to 12:30, 13:00 to 14:30, 15:30 to 16:00, 16:30 to 17:00;
\\\\
Find a time that works for everyone's schedule and constraints.
\\ \midrule
\textbf{Instance-Specific Constraints and Details:}
\\\\
Participants: Michelle, Steven, Jerry - The meeting must include all three.
\\\\
Meeting Duration: 1 hour - The meeting requires a contiguous 60-minute slot.
\\\\
Work Hours: 9:00 to 17:00 (Monday) -  The meeting must start and end within these working hours.  This acts as a global constraint.
\\\\
Individual Schedules (Existing Meetings/Blocked Times):

    - Michelle: Busy from 11:00 to 12:00 on Monday.
    
    - Steven: Busy from 9:00 to 9:30, 11:30 to 12:00, 13:30 to 14:00, and 15:30 to 16:00 on Monday.
    
    - Jerry: Busy from 9:00 to 9:30, 10:00 to 11:00, 11:30 to 12:30, 13:00 to 14:30, and 15:30 to 17:00 on Monday (note this constraint ends at the end of the workday).
\\\\
While the problem states "Note there exists a solution," this isn't a constraint in itself; it's a guarantee that a feasible solution exists given the constraints. The actual constraints are the schedules, meeting duration, and working hours.
\\ \midrule

\textbf{Generated Plan:}
\\\\
1. Analyze Constraints:

- Participants: Michelle, Steven, Jerry

- Duration: 1 hour

- Work hours: 9:00 - 17:00

- Day: Monday
\\\\
2. Identify Individual Availability:

- Michelle: 9:00-11:00, 12:00-17:00

- Steven: 9:30-11:30, 12:00-13:30, 14:00-15:30, 16:00-17:00

- Jerry: 9:30-10:00, 11:00-11:30, 12:30-13:00, 14:30-15:30, 16:00-16:30
\\\\
3. Determine Common Availability: Looking for 1-hour slots:

- Comparing Michelle and Steven: 9:30-11:00, 12:00-13:30, 14:00-15:30, 16:00-17:00

- Comparing the above with Jerry:  14:30-15:30
\\\\
4. Prioritize Earliest Time Slot: The earliest 1-hour slot is 14:30-15:30
\\\\
5. Ensure Solution Exists:  The slot 14:30-15:30 works for all three.
\\\\
6. Strict Format Requirement:

SOLUTION: Here is the proposed time: Monday, 14:30 - 15:30 \\ \midrule

\textbf{Verification:}
\\\\
The plan demonstrates significant improvement compared to the initial plan.  It correctly identifies individual availabilities and successfully finds the two viable one-hour slots: 9:30-10:30 and 14:30-15:30.  It avoids exploring irrelevant shorter slots.  The plan is well-structured and clearly explains each step.  It's on the verge of providing the correct solution.  However, it has not yet explicitly stated the final answer (the prioritized earliest time slot) which is a key part of the task.  Therefore, while much improved, it doesn't quite merit a top score due to this omission.
\\\\
\textbf{Score: 95}

\\ \bottomrule
\end{tabular}
% }
\caption{Examples of verification generated by the verification agent given generated plan and constraints. This example illustrates verification process for calendar scheduling from NATURAL PLAN}
\label{tab:np_cal_verification_example}
\end{table*}

\begin{table*}
\centering
\scriptsize
\renewcommand{\arraystretch}{1.2} % Better spacing

\begin{tabularx}{\textwidth}{>{\raggedright}p{0.1\textwidth}|X|X}
\toprule
\textbf{Category}   & \textbf{Problem Statement}  & \textbf{Generated Plan} \\ \midrule

Calendar Scheduling & You are an expert at scheduling meetings. You are given a few constraints on the existing schedule of each participant, the meeting duration, and possibly some preferences on the meeting time. Note there exists a solution that works with existing schedule of every participant.

------------

\textbf{TASK:} You need to schedule a meeting for Michelle, Steven and Jerry for one hour between the work hours of 9:00 to 17:00 on Monday.

------------

Here are the existing schedules for everyone during the day:

- Michelle has meetings on Monday during 11:00 to 12:00;

- Steven has blocked their calendar on Monday during 9:00 to 9:30, 11:30 to 12:00, 13:30 to 14:00, 15:30 to 16:00;

 -Jerry has blocked their calendar on Monday during 9:00 to 9:30, 10:00 to 11:00, 11:30 to 12:30, 13:00 to 14:30, 15:30 to 16:00, 16:30 to 17:00;
 
------------

Find a time that works for everyone's schedule and constraints.

& 

1. Analyze Constraints:

- Participants: Michelle, Steven, Jerry

- Duration: 1 hour

- Work hours: 9:00 - 17:00

- Day: Monday

------------

2. Identify Individual Availability:

- Michelle: 9:00-11:00, 12:00-17:00

- Steven: 9:30-11:30, 12:00-13:30, 14:00-15:30, 16:00-17:00

- Jerry: 9:30-10:00, 11:00-11:30, 12:30-13:00, 14:30-15:30, 16:00-16:30

------------

3. Determine Common Availability: Looking for 1-hour slots:

- Comparing Michelle and Steven: 9:30-11:00, 12:00-13:30, 14:00-15:30, 16:00-17:00

- Comparing the above with Jerry:  14:30-15:30

------------

4. Prioritize Earliest Time Slot: The earliest 1-hour slot is 14:30-15:30

------------

5. Ensure Solution Exists:  The slot 14:30-15:30 works for all three.

------------

6. Strict Format Requirement:

SOLUTION: Here is the proposed time: Monday, 14:30 - 15:30

\\ \midrule

Meeting Planning & 
You are visiting San Francisco for the day and want to meet as many friends as possible. Solve the problem by considering various different schedules and picking the best one to optimize your goals.

------------

Travel distances (in minutes):

Bayview to Golden Gate Park: 22.

Golden Gate Park to Bayview: 23.

------------

CONSTRAINTS: You arrive at Bayview at 9:00AM. Barbara will be at Golden Gate Park from 8:00AM to 11:30AM. You'd like to meet Barbara for a minimum of 90 minutes.

&  

You leave Bayview at 9:00 AM and travel 22 minutes to Golden Gate Park, arriving at 9:22 AM. You meet Barbara from 9:22 AM – 10:52 AM, fulfilling the 90-minute requirement. \\ \midrule

Trip Planning &
You plan to visit 4 European cities for 12 days in total. You only take direct flights to commute between cities. You plan to stay in Budapest for 6 days. You plan to stay in Reykjavik for 5 days. You would like to meet your friends at Reykjavik between day 6 and day 10 to tour together. You want to spend 2 days in Milan. You are going to attend a wedding in Milan between day 10 and day 11. You would like to visit Porto for 2 days.

------------

Here are the cities that have direct flights: 

Budapest and Reykjavik, Reykjavik and Milan, Budapest and Milan, Milan and Porto.

------------

Find a trip plan of visiting the cities for 12 days by taking direct flights to commute between them. & 
Day 1-6: Arriving in Budapest and visit Budapest for 6 days.

Day 6: Fly from Budapest to Reykjavik.

Day 6-10: Visit Reykjavik for 5 days.

Day 10: Fly from Reykjavik to Milan.

Day 10-11: Visit Milan for 2 days.

Day 11: Fly from Milan to Porto.

Day 11-12: Visit Porto for 2 days.

\\ \bottomrule
\end{tabularx}

\caption{Examples of generated plans by \plangen{} (Best of $\mathcal{N}$) given problem statement. These examples illustrate the plan generation for all three task categories from NATURAL PLAN}
\label{tab:np_examples}
\end{table*}


% \begin{table*}
% \centering
% \footnotesize
% % \setlength\tabcolsep{0.1pt}
% % \footnotesize
% % \resizebox{\linewidth}{!}{
% \begin{tabular}{p{0.2\textwidth}, p{0.4\textwidth}, p{0.4\textwidth}}
% \toprule
% \textbf{Category}   & \multicolumn{1}{c|}{\textbf{Problem Statement}}  & \multicolumn{1}{c}{\textbf{Plan and Solution}} \\ \midrule
% Calendar Scheduling & You are an expert at scheduling meetings. You are given a few constraints on the existing schedule of each participant, the meeting duration, and possibly some preferences on the meeting time. Note there exists a solution that works with existing schedule of every participant. Here are a few example tasks and solutions:\\ \\ TASK: You need to schedule a meeting for Michelle, Steven and Jerry for one hour between the work hours of 9:00 to 17:00 on Monday. \\ \\ Here are the existing schedules for everyone during the day: \\ Michelle has meetings on Monday during 11:00 to 12:00; \\ Steven has blocked their calendar on Monday during 9:00 to 9:30, 11:30 to 12:00, 13:30 to 14:00, 15:30 to 16:00; \\ Jerry has blocked their calendar on Monday during 9:00 to 9:30, 10:00 to 11:00, 11:30 to 12:30, 13:00 to 14:30, 15:30 to 16:00, 16:30 to 17:00; \\ \\ Find a time that works for everyone's schedule and constraints. \\ SOLUTION: & \begin{tabular}[c]{@{}l@{}}Execute the Plan Step-by-Step:\\ \\ 1. Analyze Constraints:\\     - Participants: Michelle, Steven, Jerry\\     - Duration: 1 hour\\     - Work hours: 9:00 - 17:00\\     - Day: Monday\\ \\ 2. Identify Individual Availability:\\     - Michelle: 9:00-11:00, 12:00-17:00\\     - Steven: 9:30-11:30, 12:00-13:30, 14:00-15:30, 16:00-17:00\\     - Jerry: 9:30-10:00, 11:00-11:30, 12:30-13:00, 14:30-15:30, 16:00-16:30\\ \\ \\ 3. Determine Common Availability: Looking for 1-hour slots:\\     - Comparing Michelle and Steven: 9:30-11:00, 12:00-13:30, 14:00-15:30, 16:00-17:00\\     - Comparing the above with Jerry:  14:30-15:30\\ \\ \\ 4. Prioritize Earliest Time Slot: The earliest 1-hour slot is 14:30-15:30\\ \\ 5. Ensure Solution Exists:  The slot 14:30-15:30 works for all three.\\ \\ 6. Strict Format Requirement:\\ \\ SOLUTION: Here is the proposed time: Monday, 14:30 - 15:30\end{tabular} \\ \midrule
% Meeting Planning    & \begin{tabular}[c]{@{}l@{}}You are visiting San Francisco for the day and want to meet as many friends as possible. Solve the problem by considering various different schedules and picking the best one to optimize your goals.\\ \\ Travel distances (in minutes):\\ \\ Bayview to Golden Gate Park: 22.\\ \\ Golden Gate Park to Bayview: 23.\\ \\ CONSTRAINTS: You arrive at Bayview at 9:00AM. Barbara will be at Golden Gate Park from 8:00AM to 11:30AM. You'd like to meet Barbara for a minimum of 90 minutes.\\ \\ Your response should start with 'SOLUTION:'\end{tabular}                                                                                                                                                                                                                                                                                                                                                                                                                    & SOLUTION:You start at Bayview at 9:00AM. You travel to Golden Gate Park in 22 minutes and arrive at 9:22AM. You meet Barbara for 90 minutes from 9:22AM to 10:52AM.                                                                                                                                                                                                                                                                                                                                                                                                                                                                                                                                                                                                                                                                                                                                                                                   \\ \midrule
% Trip Planning       & \begin{tabular}[c]{@{}l@{}}You plan to visit 4 European cities for 12 days in total. You only take direct flights to commute between cities. You plan to stay in Budapest for 6 days. You plan to stay in Reykjavik for 5 days. You would like to meet your friends at Reykjavik between day 6 and day 10 to tour together. You want to spend 2 days in Milan. You are going to attend a wedding in Milan between day 10 and day 11. You would like to visit Porto for 2 days.\\ \\ Here are the cities that have direct flights:\\ Budapest and Reykjavik, Reykjavik and Milan, Budapest and Milan, Milan and Porto.\\ \\ Find a trip plan of visiting the cities for 12 days by taking direct flights to commute between them.\end{tabular}                                                                                                                                                                                                                                                             & \begin{tabular}[c]{@{}l@{}}Day 1-6: Arriving in Budapest and visit Budapest for 6 days.\\ Day 6: Fly from Budapest to Reykjavik.\\ Day 6-10: Visit Reykjavik for 5 days.\\ Day 10: Fly from Reykjavik to Milan.\\ Day 10-11: Visit Milan for 2 days.\\ Day 11: Fly from Milan to Porto.\\ Day 11-12: Visit Porto for 2 days.\end{tabular}                                                                                                                                                                                                                                                                                                                                                                                                                                                                                                                                                                                                             \\ \bottomrule
% \end{tabular}
% % }
% \caption{NATURAL PLAN examples of plan generation}
% \label{tab:np_examples}
% \end{table*}

\begin{table*}
\centering
\footnotesize
% \resizebox{\linewidth}{!}{
\begin{tabular}{p{\linewidth}}
\toprule
\textbf{Question:} Two quantum states with energies E1 and E2 have a lifetime of $10^{-9}$ sec and $10^{-8}$ sec, respectively. We want to clearly distinguish these two energy levels. Which one of the following options could be their energy difference so that they can be clearly resolved? \\ \midrule
\textbf{Plan to solve the problem:}\\ \\ 1. Identify the relevant physics principle: This problem involves the energy-time uncertainty principle, which states that the product of the uncertainty in energy ($\Delta E$) and the uncertainty in time ($\Delta t$) is greater than or equal to h-bar/2 ($\hbar/2$). The uncertainty in time is related to the lifetime of the states. To clearly distinguish the energy levels, the energy difference between them must be greater than the uncertainty in their energies.\\ \\ 2. Relate lifetimes to time uncertainty: The lifetimes given ($10^{-9}$ s and $10^{-8}$ s) represent the uncertainties in the time for each state ($\Delta t_1$ and $\Delta t_2$). We need to consider which lifetime is more restrictive for resolving the energy levels. Since we want to *clearly* distinguish them, we should use the *smaller* lifetime as the relevant $\Delta t$. This is because the shorter lifetime leads to a larger energy uncertainty, making the levels harder to resolve.\\ \\ 3. Calculate the minimum energy difference: Use the energy-time uncertainty principle ($\Delta E * \Delta t \geq \hbar/2$) with the smaller lifetime to calculate the minimum resolvable energy difference ($\Delta E$). We can approximate this by setting $\Delta E * \Delta t \approx \hbar/2$. We'll need the value of $\hbar$ (Planck's constant divided by $2\pi$) in eV·s.\\ \\ 4. Compare the calculated $\Delta E$ with the choices: Compare the calculated minimum resolvable energy difference ($\Delta E$) with the given options (A) through (D). The correct answer will be the option that is *greater* than the calculated $\Delta E$, ensuring that the energy difference is large enough to resolve the two levels.\\ \\ 5. Select the correct answer: Based on the comparison in step 4, choose the option that represents an energy difference greater than the minimum resolvable energy difference. \\ \midrule
\textbf{Plan Execution:}\\ \\ 1. Identify the relevant physics principle: The energy-time uncertainty principle ($\Delta E * \Delta t \geq \hbar/2$) governs this situation.\\ \\ 2. Relate lifetimes to time uncertainty: The shorter lifetime, $10^{-9}$ s, is the relevant $\Delta t$ because it leads to a larger energy uncertainty, making resolution more difficult. So, $\Delta t = 10^{-9}$ s.\\ \\ 3. Calculate the minimum energy difference:\\ We have $\Delta E * \Delta t = \hbar/2$. We need $\hbar$ in eV·s. $\hbar = 6.582 \times 10^{-16}$ eV·s.\\ $\Delta E = (\hbar/2) / \Delta t$\\ $\Delta E = (6.582 \times 10^{-16} \text{ eV·s} / 2) / (10^{-9} \text{ s})$\\ $\Delta E = 3.291 \times 10^{-7}$ eV\\ This is approximately $3 \times 10^{-7}$ eV.\\ \\ 4. Compare the calculated $\Delta E$ with the choices: We need an energy difference greater than $3 \times 10^{-7}$ eV to clearly resolve the levels.\\ (A) $10^{-11}$ eV - Too small\\ (B) $10^{-8}$ eV - Too small\\ (C) $10^{-9}$ eV - Too small\\ (D) $10^{-4}$ eV - Large enough\\ \\ 5. Select the correct answer: Option (D) is the only one significantly larger than the calculated minimum energy difference.\\ \\ Answer: D \\ \bottomrule
\end{tabular}
% }
\caption{Examples of generated plans by \plangen{} (Best of $\mathcal{N}$) given physics problem. The example illustrates the plan generation and it's execution for physics problem from GPQA}
\label{tab:gpqa_example}
\end{table*}

\begin{table*}
\centering
\footnotesize
% \resizebox{\linewidth}{!}{
\begin{tabular}{p{\linewidth}}
\toprule
\textless{}Entire Document\textgreater\\ \\ \textbf{Question:} What is the percentage change in cash flow hedges in 2011 compare to the 2010? \\ \midrule
\textbf{Plan to solve the problem:}\\ \\ 1. Locate the information on cash flow hedges: Search the document for "cash flow hedges" or similar terms related to derivative instruments used for hedging cash flows.  Focus on sections discussing derivative instruments, hedging activities, and related notional amounts.\\ \\ 2. Identify the notional amounts for 2011 and 2010: Once the relevant section is found (Note 2i), extract the notional amounts of the cash flow hedges specifically for fiscal year 2011 and fiscal year 2010.\\ \\ 3. Calculate the percentage change: Use the formula for percentage change: {[}(Notional Amount 2011 - Notional Amount 2010) / Notional Amount 2010{]} * 100.  This will provide the percentage change in cash flow hedges from 2010 to 2011. \\ \midrule
\textbf{Plan Execution:}\\ \\ 1. Locate the information on cash flow hedges: The information on cash flow hedges was located in a section discussing derivative instruments and hedging activities. Specifically, it was found within a sentence detailing the notional amounts of various hedging instruments.\\ \\    Code (if needed):\\    ```python\\    \# No code needed for this step.\\    ```\\    Result: Information on cash flow hedges located successfully.\\ \\ 2. Identify the notional amounts for 2011 and 2010: The notional amounts for cash flow hedges were identified as \$153.7 million for 2011 and \$139.9 million for 2010.\\ \\    Code (if needed):\\    ```python\\    \# No code needed for this step.\\    ```\\    Result: 2011 notional amount: \$153.7 million; 2010 notional amount: \$139.9 million.\\ \\ 3. Calculate the percentage change:\\ \\    Code (if needed):\\    ```python\\    notional\_2011 = 153.7\\    notional\_2010 = 139.9\\    percentage\_change = ((notional\_2011 - notional\_2010) / notional\_2010) * 100\\    rounded\_percentage\_change = round(percentage\_change, 1)\\    print(f"\{rounded\_percentage\_change\}\%")\\    ```\\ \\ Answer: 9.9\% \\ \bottomrule
\end{tabular}
% }
\caption{Examples of generated plans by \plangen{} (Best of $\mathcal{N}$) given financial reasoning problem. The example illustrates the plan generation and it's execution for problem from DocFinQA}
\label{tab:docfinqa_example}
\end{table*}

\begin{table*}
\centering
\scriptsize
% \resizebox{\linewidth}{!}{
% \begin{tabularx}{\textwidth}{p{0.98\linewidth}}
\begin{tabular}{p{\linewidth}}
\toprule
\textbf{Question:} For a positive integer $a$, define a sequence of integers $x_{1}, x_{2}, \ldots$ by letting $x_{1}=a$ and $x_{n+1}=2 x_{n}+1$ for $n\geq1$. Let $y_{n}=2^{x_{n}}-1$. Determine the largest possible $k$ such that, for some positive integer $a$, the numbers $y_{1}, \ldots, y_{k}$ are all prime. \\ \midrule
\textbf{Plan for Solving the IMO Problem:}\\ 1. Problem Understanding: The problem defines a recursive sequence `x\_n' based on an initial value `a', and a related sequence `y\_n' based on `x\_n'. The goal is to find the largest `k' such that `y\_1', `y\_2', ..., `y\_k' can all be prime numbers for some choice of `a'.\\2. Key Information and Variables:\\ `a': A positive integer, the starting value for the sequence `x\_n'. This is the variable we need to choose optimally.\\ `x\_n': Defined recursively as `x\_1 = a' and `x\_(n+1) = 2x\_n + 1'.\\ `y\_n': Defined as `y\_n = 2\textasciicircum{}(x\_n) - 1'.  This is the sequence we need to check for primality.\\ `k': The length of the subsequence of `y\_n' consisting only of primes. We want to maximize this.\\3. Problem Type and Method: This is a number theory problem involving sequences, primality, and recursion. The method will involve a combination of:\\ Analyzing the structure of `x\_n' and 'y\_n':  Finding a closed-form expression for 'x\_n' in terms of 'a' and `n` will be crucial. This will help us understand the structure of `y\_n'.\\ Primality testing/properties: We'll need to leverage known properties of prime numbers, particularly those related to Mersenne primes (primes of the form 2\textasciicircum{}p - 1, where p is prime).\\ Case analysis/exploration: We might need to explore small values of `a' and `k' to identify patterns and formulate conjectures.\\ Proof by contradiction:  To establish the largest possible `k`, we may need to assume a larger value is possible and derive a contradiction.\\ 4. Solution Process:\\ Step 1: Derive a closed-form expression for `x\_n`.  The recursive definition suggests a linear recurrence relation. We can solve it to express `x\_n` directly in terms of `a` and `n`.\\ Step 2: Substitute `x\_n` into the definition of `y\_n`. This gives us a formula for `y\_n` in terms of `a` and `n`.\\ Step 3: Analyze the primality conditions for `y\_n`. Given that `y\_n` has the form 2\textasciicircum{}m - 1,  we know that for `y\_n` to be prime, `m` (which is `x\_n`) must itself be prime.  This suggests focusing on values of `a` that lead to prime values of `x\_n`.\\ Step 4: Explore small values of `a` and `k`. This helps to identify patterns and limitations on the possible values of `k`.\\ Step 5:  Investigate divisibility properties. Consider the factors of `y\_n` and how they might relate to previous terms in the sequence.  Look for patterns in the factors that might prevent multiple consecutive `y\_n` from being prime.\\ Step 6: Formulate a conjecture for the maximum `k`. Based on the observations from the previous steps, propose a value for the maximum `k`.\\ Step 7: Prove the conjecture. Use a proof by contradiction. Assume a larger `k` is possible and demonstrate that this leads to a contradiction based on the divisibility properties of `y\_n`. This will establish the proposed value of `k` as the maximum. \\ \midrule
\textbf{Plan Execution:}\\ 1. Closed-form for $x_n$: The recurrence relation $x_{n+1} = 2x_n + 1$ with $x_1 = a$ can be solved as follows:\\ $x_2 = 2a + 1$\\ $x_3 = 2(2a+1) + 1 = 4a + 3 = 2^2a + 2^2 - 1$\\ $x_4 = 2(4a+3) + 1 = 8a + 7 = 2^3a + 2^3 - 1$\\ By induction, we can show that $x_n = 2^{n-1}a + 2^{n-1} - 1 = 2^{n-1}(a+1) - 1$.\\ 2. $y_n$ in terms of $a$ and $n$: Substituting the expression for $x_n$ into $y_n = 2^{x_n} - 1$, we get $y_n = 2^{2^{n-1}(a+1) - 1} - 1$.\\ 3. Primality of $y_n$: For $y_n$ to be prime, $x_n$ must be prime.  Let $x_n = p_n$, where $p_n$ is a prime number.\\ 4. Exploring small values:\\ If $a=1$, $x_1 = 1$, $y_1 = 2^1 - 1 = 1$, which is not prime.\\ If $a=2$, $x_1 = 2$, $y_1 = 2^2 - 1 = 3$, which is prime.\\ $x_2 = 2(2) + 1 = 5$, $y_2 = 2^5 - 1 = 31$, which is prime.\\ $x_3 = 2(5) + 1 = 11$, $y_3 = 2^{11} - 1 = 2047 = 23 \cdot 89$, which is not prime.  So, for $a=2$, $k=2$.\\ 5. Divisibility:\\ If $x_1 = a$ is prime, say $a=p$, then $x_2 = 2p+1$.  If $x_2$ is also prime, then $y_1$ and $y_2$ are potentially Mersenne primes.  However, $x_3 = 2(2p+1)+1 = 4p+3$.  If $p=2$, $x_3 = 11$, and $y_3$ is not prime.  If $p=3$, $x_2 = 7$, $x_3 = 15$, which is not prime.  If $p=5$, $x_2 = 11$, $x_3 = 23$, $x_4 = 47$, $x_5 = 95$, which is not prime.\\ \\ If $k=3$, we need $x_1=a$, $x_2=2a+1$, and $x_3=4a+3$ to all be prime.  If $a=2$, $x_1=2$, $x_2=5$, $x_3=11$.  But $y_3 = 2^{11}-1 = 2047 = 23 \cdot 89$, which is not prime.\\ \\ 6. Conjecture: The largest possible $k$ is 2.\\ 7. Proof: We have shown that for $a=2$, $y_1$ and $y_2$ are prime.  We need to show that $y_3$ cannot be prime.  $x_3 = 4a+3$.  If $a \equiv 2 \pmod{3}$, then $4a+3 \equiv 4(2)+3 \equiv 11 \equiv 2 \pmod{3}$.  If $a \equiv 1 \pmod{3}$, then $4a+3 \equiv 4(1)+3 \equiv 7 \equiv 1 \pmod{3}$.  If $a \equiv 0 \pmod{3}$, then $4a+3 \equiv 3 \equiv 0 \pmod{3}$.  So, if $a>3$ and $a \equiv 0 \pmod{3}$, then $x_3$ is divisible by 3 and greater than 3, so it is not prime.  If $a=3$, $x_3 = 15$, not prime.  If $a=2$, $x_3 = 11$, and $y_3 = 2047 = 23 \cdot 89$.\\ \\ So the final answer is $\boxed{2}$. \\ \bottomrule
\end{tabular}
% \end{tabularx}
% }
\caption{Examples of generated plans by \plangen{} (Best of $\mathcal{N}$) given maths problem. The example illustrates the plan generation and it's execution for problem from OlympiadBench (MATH)}
\label{tab:olympiad_math_example}
\end{table*}