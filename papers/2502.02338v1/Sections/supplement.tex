
% \section*{Table of Contents}
% \begin{itemize}
%     \item[\textbf{\textcolor{blue}{A}}] \textbf{\textcolor{blue}{Implementation Details}}
%     \begin{itemize}
%         \item[A.1] Gaussian Construction
%         \item[A.2] Neural Radiance Field Rendering
%         \item[A.3] Hierarchical Latent Variables
%         \item[A.4] Modulation
%     \end{itemize}
    
%     \item[\textbf{\textcolor{blue}{B}}] \textbf{\textcolor{blue}{Derivation of Evidence Lower Bound}}
    
%     \item[\textbf{\textcolor{blue}{C}}] \textbf{\textcolor{blue}{Training Details}}
    
%     \item[\textbf{\textcolor{blue}{D}}] \textbf{\textcolor{blue}{More Experimental Results}}
% \end{itemize}

%%%%%%%%%%%%%%%%%%%%%%%%%%%%%%%%%%%%%%%%%%%%
% NerF: $F: (x,y,z,\theta,\phi) \mapsto (c,\sigma)$. 

% Context set (camera pose and image rbg level): 
% \begin{align}
% \tilde{X}_c:&= \{ \tilde{x}_i \}_{i=1}^{N_c}, \tilde{x}_i \in \mathbb{R}^{6} (\text{ray-o}, \text{ray-d})\\
% \tilde{Y}_c:&= \{ \tilde{y}_i \}_{i=1}^{N_c}, \tilde{y}_i \in \mathbb{R}^{3} (\text{RGB})
% \end{align}

% Target set (3d coordinates and 3d rbg with density level):

% $x_i$ is a set of 3D points corresponding to a specific ray direction. 
% \begin{align}
% X_t:&= \{ x_i \}_{i=1}^{N_t}, x_i \in \mathbb{R}^{P\times3} (\text{sampled 3D points})\\
% Y_t:&= \{ y_i \}_{i=1}^{N_t}, y_i \in \mathbb{R}^{P\times4} (c,\sigma)
% \end{align}

% \begin{align}
%     p(Y_t|X_t, \tilde{X}_c, \tilde{Y}_c) & = p(Y_t|X_t,B_c)p(B_c|\tilde{X}_c, \tilde{Y}_c) \\
%     &= \Pi_{i=1}^{N_t} p(y_i|x_i, g, r_i, B_c) p(r_i|g, x_i) p(g|B_c, X_t) p(B_c|\tilde{X}_c, \tilde{Y}_c)
% \end{align}

% \begin{align}
%     p(Y_t|X_t, \tilde{X}_c, \tilde{Y}_c) & \approx p(Y_t|X_t,B_c) \\
%     &= \Pi_{i=1}^{N_t} p(y_i|x_i, g, r_i, B_c) p(r_i|g, x_i) p(g|B_c, X_t)
% \end{align}
%%%%%%%%%%%%%%%%%%%%%%%%%%%%%%%%%%%%%%%%%%%%
\newpage
\section{Neural Radiance Field Rendering}
\label{supp:nerf-render}
In this section, we outline the rendering function of NeRF~\citep{mildenhall2021nerf}. A 5D neural radiance field represents a scene by specifying the volume density and the directional radiance emitted at every point in space. NeRF calculates the color of any ray traversing the scene based on principles from classical volume rendering~\citep{kajiya1984ray}. The volume density $\sigma(\mathbf{x})$ quantifies the differential likelihood of a ray terminating at an infinitesimal particle located at $\mathbf{x}$. The anticipated color $C(\mathbf{r})$ of a camera ray $\mathbf{r}(t) = \mathbf{o} + t\mathbf{d}$, within the bounds $t_n$ and $t_f$, is determined as follows:
\begin{equation}
C(\mathbf{r}) = \int_{t_n}^{t_f} T(t) \sigma(\mathbf{r}(t)) c(\mathbf{r}(t), \mathbf{d}) dt, \quad \text{where} \quad T(t) = \exp \left( - \int_{t_n}^{t} \sigma(\mathbf{r}(s)) ds \right).
\end{equation}

Here, the function $T(t)$ represents the accumulated transmittance along the ray from $t_n$ to $t$, which is the probability that the ray travels from $t_n$ to $t$ without encountering any other particles. To render a view from our continuous neural radiance field, we need to compute this integral $C(\mathbf{r})$ for a camera ray traced through each pixel of the desired virtual camera.


\section{Hierarchical ELBO Derivation}
\label{sec:elbo-general}
Recall the hierarchical predictive distribution:
\begin{equation}
\label{eq:hier-model}
p\bigl(y_T \mid x_T, B_C\bigr)
\,=\, 
\int
\Bigl[
\int 
p\bigl(y_T \mid z_g, z_l, x_T, B_C\bigr)
\,p\bigl(z_l \mid z_g, x_T, B_C\bigr)
\,\mathrm{d}z_l
\Bigr]
p\bigl(z_g \mid x_T, B_C\bigr)
\,\mathrm{d}z_g,
\end{equation}
and its factorized version across $M$ target points:
\[
p(y_T \mid x_T, B_C)
\,=\,
\int
p\bigl(z_g \mid x_T, B_C\bigr)
\Bigl[
\prod_{m=1}^M
\int 
p\bigl(y_{T,m} \mid z_g, z_{l,m}, x_{T,m}, B_C\bigr)
\,p\bigl(z_{l,m} \mid z_g, x_{T,m}, B_C\bigr)
\,\mathrm{d}z_{l,m}
\Bigr]
\,\mathrm{d}z_g.
\]

We introduce a \emph{hierarchical} variational posterior:
\[
q\bigl(z_g, \{z_{l,m}\}\mid x_T, B_T\bigr)
\,=\,
q\bigl(z_g \mid x_T, B_T\bigr)
\,\prod_{m=1}^M
q\bigl(z_{l,m} \mid z_g, x_{T,m}, B_T\bigr),
\]
where $B_T$ are target-derived bases (available only at training). We then write the log-likelihood as

\begin{equation}
\label{eq:outer-inner-begin}
\begin{aligned}
\log p\bigl(y_T \mid x_T, B_C\bigr)
&=\;
\log \int
\int
p\bigl(y_T, z_g, \{z_{l,m}\}\mid x_T, B_C\bigr)
\,\frac{
q\bigl(z_g, \{z_{l,m}\}\mid x_T, B_T\bigr)
}{
q\bigl(z_g, \{z_{l,m}\}\mid x_T, B_T\bigr)
}
\,\mathrm{d}z_l
\,\mathrm{d}z_g
\\[6pt]
&=\;
\log \int p\bigl(z_g\mid x_T, B_C\bigr)
\,\frac{q\bigl(z_g\mid x_T, B_T\bigr)}{q\bigl(z_g\mid x_T, B_T\bigr)}
\Bigl[\!
\int
p\bigl(y_T,\{z_{l,m}\}\mid z_g, x_T, B_C\bigr)
\,\frac{
q\bigl(\{z_{l,m}\}\mid z_g, x_T, B_T\bigr)
}{
q\bigl(\{z_{l,m}\}\mid z_g, x_T, B_T\bigr)
}
\,\mathrm{d}z_l
\Bigr]
\,\mathrm{d}z_g
\,.
\end{aligned}
\end{equation}

\vspace{0.3em}
We first apply Jensen’s inequality w.r.t.\ $q(z_g \mid x_T, B_T)$. This yields:
\begin{equation}
\label{eq:outer-jensen}
\begin{aligned}
\log p\bigl(y_T \mid x_T, B_C\bigr)
&\;\geq\;
\mathbb{E}_{q(z_g \mid x_T, B_T)}
\Bigl[
\log
\int
p\bigl(y_T,\{z_{l,m}\}\mid z_g, x_T, B_C\bigr)
\,\frac{
q\bigl(\{z_{l,m}\}\mid z_g, x_T, B_T\bigr)
}{
q\bigl(\{z_{l,m}\}\mid z_g, x_T, B_T\bigr)
}
\,\mathrm{d}z_l
\Bigr]
\\
&\quad
-\;
D_{\mathrm{KL}}\bigl(
q(z_g \mid x_T, B_T)
\;\|\;
p(z_g \mid x_T, B_C)
\bigr).
\end{aligned}
\end{equation}

\vspace{0.3em}
Inside the expectation over $z_g$, we have
\[
\log
\int
p\bigl(y_T,\{z_{l,m}\}\mid z_g, x_T, B_C\bigr)
\,\frac{
q\bigl(\{z_{l,m}\}\mid z_g, x_T, B_T\bigr)
}{
q\bigl(\{z_{l,m}\}\mid z_g, x_T, B_T\bigr)
}
\,\mathrm{d}z_l
\,.
\]
We again apply Jensen’s inequality, but now w.r.t.\ $q(\{z_{l,m}\}\mid z_g, x_T, B_T)$, factorizing over $m$:
\begin{equation}
\begin{aligned}
\log
\int
&p\bigl(y_T,\{z_{l,m}\}\mid z_g, x_T, B_C\bigr)
\frac{
q\bigl(\{z_{l,m}\}\mid z_g, x_T, B_T\bigr)
}{
q\bigl(\{z_{l,m}\}\mid z_g, x_T, B_T\bigr)
}
\,\mathrm{d}z_l 
\\
&\;\;\geq\;
\mathbb{E}_{q(\{z_{l,m}\}\mid z_g, x_T, B_T)}
\Bigl[
\log p\bigl(y_T\mid z_g, \{z_{l,m}\}, x_T, B_C\bigr)
\Bigr]
\;-\;
\sum_{m=1}^M
D_{\mathrm{KL}}\Bigl(
q\bigl(z_{l,m}\mid z_g, x_{T,m}, B_T\bigr)
\;\big\|\;
p\bigl(z_{l,m}\mid z_g, x_{T,m}, B_C\bigr)
\Bigr).
\end{aligned}
\end{equation}

\vspace{0.3em}

Putting this back into Eq.~\eqref{eq:outer-jensen}, we arrive at the hierarchical ELBO:

\begin{equation}
\label{eq:final-hier-elbo}
\begin{aligned}
\log p\bigl(y_T \mid x_T, B_C\bigr)
&\;\geq\;
\mathbb{E}_{q(z_g \mid x_T, B_T)}
\Biggl[
\sum_{m=1}^M
\mathbb{E}_{q(z_{l,m}\mid z_g, x_{T,m}, B_T)}
\bigl[
\log p\bigl(y_{T,m}\mid z_g, z_{l,m}, x_{T,m}, B_C\bigr)
\bigr]
\\
&\qquad
-\;
\sum_{m=1}^M
D_{\mathrm{KL}}\Bigl(
q\bigl(z_{l,m}\mid z_g, x_{T,m}, B_T\bigr)
\;\|\;
p\bigl(z_{l,m}\mid z_g, x_{T,m}, B_C\bigr)
\Bigr)
\Biggr]
\\
&\quad
-\;
D_{\mathrm{KL}}\Bigl(
q\bigl(z_g \mid x_T, B_T\bigr)
\;\|\;
p\bigl(z_g \mid x_T, B_C\bigr)
\Bigr).
\end{aligned}
\end{equation}


The first expectation over \(q(z_g\mid x_T,B_T)\) enforces global consistency and penalizes deviations from the prior \(p(z_g\mid x_T,B_C)\). The second set of expectations over \(q(z_{l,m}\mid z_g, x_{T,m}, B_T)\) shapes local reconstruction quality (via the log-likelihood) and penalizes deviations from the local prior \(p(z_{l,m}\mid z_g, x_{T,m}, B_C)\).

Hence, the final ELBO (Equation~\ref{eq:final-hier-elbo}) combines these outer and inner regularization terms with the expected log-likelihood of the target data \(y_T\). This allows the model to learn coherent \emph{global} structure as well as \emph{local} (coordinate-specific) details in a principled way.


\section{Implementation Details}
\label{sec:implementation-details}

\subsection{Gaussian Construction}
\label{supp:gaussian}
Since 3D Gaussians represent a special case involving quaternion-based transformations, we use them here as an illustrative example for constructing geometric bases. However, the method remains consistent with the construction of 1D and 2D Gaussian geometric bases.
\paragraph{Geometric Bases with 3D Gaussians.}
To impose geometric structure on the context variables, we encode the context set 
\(\{x_{C}, y_{C}\}\)
into a set of \(M\) \emph{geometric bases}:
\begin{equation}
\label{eq:generation_B1}
{B}_{C} \;=\; \Bigl\{\, {b}_r \Bigr\}_{r=1}^{R}, 
\quad\text{where}\quad 
{b}_r 
=\; 
\Bigl(\,
\mathcal{N}\!\bigl(\mu_r,\;\Sigma_r\bigr)
,\;
\omega_r
\Bigr).
\end{equation}
Each basis \({b}_r\) is thus defined by a 3D Gaussian \(\mathcal{N}\!(\mu_r,\Sigma_r)\) and an associated semantic embedding \(\omega_r\). The center \(\mu_r \in \mathbb{R}^3\) and covariance \(\Sigma_i \in \mathbb{R}^{3\times 3}\) capture location and shape, while \(\omega_r \in \mathbb{R}^{d_B}\) represents additional learned properties (e.g., color or texture). In our implementation, \(d_B = 32\).

\paragraph{Self-Attention Construction.}
We use a self-attention module, denoted \(\texttt{Att}\), to extract these Gaussian parameters from the context data. Concretely,
\begin{equation}
\label{eq:generation_B2}
\mu_i,\;\Sigma_i 
\;=\;
\texttt{Att}\!\bigl(x_{C}, y_{C}\bigr), 
\qquad
\omega_i 
\;=\;
\texttt{Att}\!\bigl(x_{C}, y_{C}\bigr),
\end{equation}
where each call to \(\texttt{Att}\) produces \(M\) \emph{tokens} of hidden dimension \(D\).  An MLP then maps each token into a 10-dimensional vector encoding: 
(i) the 3D center \(\mu_i\), 
(ii) a 3D scaling vector \(\mathbf{s}_i\), and 
(iii) a 4D quaternion \(\mathbf{q}_i\) that, together, define the rotation matrix \(\mathbf{R}_i\). Following \citet{kerbl20233d}, we obtain the covariance \(\Sigma_i\) via
\begin{equation}
\label{eq:cov-matrix}
\Sigma_i 
\;=\;
\mathbf{R}_i\,\bigl(\mathbf{S}_i\mathbf{S}_i^\top\bigr)\,\mathbf{R}_i^\top,
\end{equation}
where \(\mathbf{S}_i = \mathrm{diag}(\mathbf{s}_i)\in\mathbb{R}^{3\times 3}\) is the scaling matrix and \(\mathbf{R}_i\in\mathbb{R}^{3\times 3}\) is derived from \(\mathbf{q}_i\). A separate MLP outputs the \(32\)-dimensional embedding \(\omega_i\). Consequently, each \({b}_i\) is a fully parameterized 3D Gaussian plus a semantic vector, allowing the model to represent rich geometric information inferred from the context set.



% As introduced in Sec.~\ref{sec: geometrybases}, we introduce geometric bases ${\bf{{B}}}_{C}$ to structure the context variables geometrically.
% ${\bf{{B}}}_{C}$ are geometric  bases (Gaussians) inferred from the context views $\{{\bf{\widetilde{X}}}_{C}, {\bf{\widetilde{Y}}}_{C}\}$ with 3D structure information, 
% \textit{i.e.,} ${\bf{b}}_i = \{ \mathcal{N}(\mu_i, \Sigma_i); \omega_i\}$,
% %\textit{i.e., object shape, color and texture.}. $B_C$ is obtained by: 
% % ${\bf{{B}}}_{C}=\texttt{Encoder}\Big({\bf{\widetilde{X}}}_{C}, {\bf{\widetilde{Y}}}_{C}\Big)$. 
% \begin{align}
%     &{\bf{{B}}}_{C} = \{{\bf{b}}_i\}_{i=1}^{M}, {\bf{b}}_i=\{\mathcal{N}(\mu_i, \Sigma_i); \omega_i\},
%     \label{eq: generation_B_1}
%     \\
%     & \mu_i, \Sigma_i = \texttt{Att}({\bf{\widetilde{X}}}_{C}, {\bf{\widetilde{Y}}}_{C}), \texttt{Att}({\bf{\widetilde{X}}}_{C}, {\bf{\widetilde{Y}}}_{C}),
%     \label{eq: generation_B_2}
%     \\
%     & \omega_i = \texttt{Att}({\bf{\widetilde{X}}}_{C}, {\bf{\widetilde{Y}}}_{C}),
%     \label{eq: generation_B_3}
% \end{align}
% where $M$ is the number of the Gaussian bases. $\mu \mathbb \in {R}^3$ is the Gaussian center, $\Sigma \in  \mathbb{R}^{3\times 3}$ is the covariance matrix, and $\omega \in \mathbb{R}^{d_B}$ is the corresponding ${d_B}$-dimension semantic representation. In our implementation, we choose $d_{B}$ as $32$. $\texttt{Att}$ is a self-attention module. Specifically, given the context set $[\widetilde{\mathbf{X}};\widetilde{\mathbf{Y}}] \in \mathbb{R}^{H\times W \times (3+3+3)}$, the visual self-attention module, $\texttt{Att}$, first produces a $M\times D$ tokens with $M$ is the number of visual tokens and $D$ is the hidden dimension. The number of Gaussians we use equals the number of tokens $M$. %Then, we use one MLP to predict centers $\mu$, as well as the rotation $R$ and scaling $S$ matrices parameters for producing covariance matrix $\Sigma$, and one MLP to produce the latent representations $\omega$. 
% Then, we use one MLP with 2 linear layers to map the tokens into a 10-dimensional vector, which includes 3-dimensional Gaussian centers, a 3-dimensional vector for constructing the scaling matrix, and a 4-dimensional vector for quaternion parameters of the rotation matrix. Both the scaling matrix and rotation matrix are used to build the \(3 \times 3\) covariance matrix. This procedure is similar to Gaussian construction in the 3D Gaussian Splatting~\citep{kerbl20233d}.
% Another MLP estimates the latent representation of each Gaussian basis, using a 32-dimensional vector for each Gaussian basis. 

% The covariance matrix is obtained by:
% \begin{equation}
%     \Sigma = RSS^TR^T,
%     \label{eq:cov-matrix}
% \end{equation}
% where $R\in \mathbb{R}^{3\times3}$ is the rotation matrix, and $S \in \mathbb{R}^3$ is the scaling matrix. 











\subsection{Hierarchical Latent Variables}
\label{supp:latent-variables}

\begin{figure}[t]
  \centering
  \includegraphics[width=0.7\textwidth]{Figures/Transformer.pdf} % Adjust the size and filename as needed
  \caption{\textbf{Using transformer encoder to generate ray-specific latent variable $\mathbf{z}_l$.}} % Caption for the figure
  \label{fig:latent-transformer}
  % \vspace{-3mm}
\end{figure}

At the object level, the distribution of the global latent variable \( z_g \) is obtained by aggregating all location representations from \( (B_C, x_T) \). We assume that \( p(z_g \mid B_C, x_T) \) follows a standard Gaussian distribution, and we generate its mean \( \mu_g \) and variance \( \sigma_g \) using MLPs. We then sample a global modulation vector, \( \hat{z}_g \), from its prior distribution \( p(z_g \mid x_T, B_C) \).

Similarly, as shown in Fig.~\ref{fig:latent-transformer}, we aggregate information for each target coordinate \( x_{T,m} \) using \( B_C \), which is then processed through a Transformer along with \( \hat{z}_g \) to predict the local latent variable \( z_{l,m} \) for each target point. The mean \( \mu_{l,m} \) and variance \( \sigma_{l,m} \) of \( z_{l,m} \) are obtained via MLPs.




% At the object level, the distribution of an object-specific latent variable \(\mathbf{z}_o\) is obtained by aggregating all location representations from \((\mathbf{B}_C, \mathbf{X}_T)\). We assume \(p(\mathbf{z}_o | \mathbf{B}_C, \mathbf{X}_T)\) follows a standard Gaussian distribution and generate its mean \(\mu_{o}\) and variance \(\sigma_{o}\) using MLPs. We sample an object-specific modulation vector, \(\hat{\mathbf{z}}_o\), from its prior distribution \(p(\mathbf{z}_o | \mathbf{X}_T, \mathbf{B}_C)\).

% Similarly, as shown in Fig.~\ref{fig:latent-transformer}, we aggregate the information per ray using \(\mathbf{B}_C\), which is then fed into a Transformer along with \(\hat{\mathbf{z}}_o\) to predict the latent variable \(\mathbf{z}_r\) with mean \(\mu_r\) and \(\sigma_r\) for each ray.


 

\subsection{Modulation}
\label{supp:modulate}
We use modulation to The latent variables for modulating the MLP are represented as \([{z}_g; {z}_l]\). Our approach to the modulated MLP layer follows the style modulation techniques described in \citep{karras2020analyzing, guo2023versatile}. Specifically, we consider the weights of an MLP layer (or 1x1 convolution) as \( W \in \mathbb{R}^{d_{\text{in}} \times d_{\text{out}}} \), where \( d_{\text{in}} \) and \( d_{\text{out}} \) are the input and output dimensions respectively, and \( w_{ij} \) is the element at the \(i\)-th row and \(j\)-th column of \( W \).

To generate the style vector \( s \in \mathbb{R}^{d_{\text{in}}} \), we pass the latent variable \( z \) through two MLP layers. Each element \( s_i \) of the style vector \( s \) is then used to modulate the corresponding parameter in \( W \).
\begin{equation}
    w'_{ij} = s_i \cdot w_{ij}, \quad j = 1, \ldots, d_{\text{out}},
\end{equation}
where $w_{ij}$ and $w'_{ij}$ denote the original and modulated weights, respectively.

The modulated weights are normalized to preserve training stability,
\begin{equation}
    w''_{ij} = \frac{w'_{ij}}{\sqrt{\sum_i w'^2_{ij} + \epsilon}}, \quad j = 1, \ldots, d_{\text{out}}.
\end{equation}




% \begin{algorithm}[H]
% \caption{Modulation Layer}
% \begin{algorithmic}[1]
% \REQUIRE Latent variable $z$, weight matrix $W \in \mathbb{R}^{d_{\mathrm{in}} \times d_{\mathrm{out}}}$
% \ENSURE Modulated and normalized weight matrix $W''$
% \STATE \textbf{Compute style vector:}
% \STATE $s \leftarrow \mathrm{MLP}_2\big(\mathrm{MLP}_1(z)\big)$
% \STATE \textbf{Modulate weights:}
% \STATE $W' \leftarrow \operatorname{diag}(s) \times W$
% \STATE \textbf{Normalize modulated weights:}
% \STATE For each column $j$ in $W'$:
% \STATE \hskip1em $\sigma_j \leftarrow \sqrt{\sum_{i=1}^{d_{\mathrm{in}}} (W'_{ij})^2 + \epsilon}$
% \STATE Normalize column $j$ of $W'$: $W''_{:,j} \leftarrow W'_{:,j} / \sigma_j$
% \RETURN $W''$
% \end{algorithmic}
% \end{algorithm}










% \begin{algorithm}[H]
% \caption{Training Procedure}
% \begin{algorithmic}[1]
% \REQUIRE Context set $({\bf{X}}_{C}, {\bf{Y}}_C)$, target set $({\bf{X}}_{T}, {\bf{Y}}_T)$
% \ENSURE Prediction ${\bf{Y}}'_T$
% \vspace{0.5em}
% \STATE Estimate the context bases ${\bf{B}}_C$ and the target bases ${\bf{B}}_T$ (Eq. 12).
% \vspace{0.5em}
% \STATE Estimate the object-specific latent variables:
% \begin{itemize}
%     \item For the context set ${\bf{z}}_o^C$:
%     \[
%     {\bf{z}}_o^C \sim p({\bf{z}}_o \mid {\bf{X}}_C, {\bf{B}}_C)
%     \]
%     \item For the target set ${\bf{z}}_o^T$:
%     \[
%     {\bf{z}}_o^T \sim q({\bf{z}}_o \mid {\bf{X}}_T, {\bf{B}}_T) \quad \text{(Eq. 7)}
%     \]
% \end{itemize}
% \vspace{0.5em}
% \STATE Estimate the ray-specific latent variables:
% \begin{itemize}
%     \item For the context set ${\bf{z}}_r^{C}$:
%     \[
%     {\bf{z}}_r^{C} \sim p({\bf{z}}_r^n \mid {\bf{z}}_o^C, {\bf{x}}_C^{n}, {\bf{B}}_C) \quad \text{(Eq. 8)}
%     \]
%     \item For the target set ${\bf{z}}_r^{T}$:
%     \[
%     {\bf{z}}_r^{T} \sim q({\bf{z}}_r^n \mid {\bf{z}}_o^T, {\bf{x}}_T^{n}, {\bf{B}}_T) \quad \text{(Eq. 8)}
%     \]
% \end{itemize}
% \vspace{0.5em}
% \STATE Modulate MLP $f$ using the target latent variables $\{{\bf{z}}_o^T, {\bf{z}}_r^{T}\}$ (Eqs. 16 \& 17).
% \vspace{0.5em}
% \STATE Render novel views $\hat{\bf{Y}}_T$ using the modulated MLP $f$.
% \vspace{0.5em}
% \STATE \textbf{Compute losses:}
% \begin{itemize}
%     \item Reconstruction loss between predictions and ground truth:
%     \[
%     \mathcal{L}_{\text{recon}} = \text{Loss}(\hat{\bf{Y}}_T, {\bf{Y}}_T)
%     \]
%     \item Latent variable alignment losses (KL divergence) using context and target latent variables (Eq. 10).
% \end{itemize}
% \end{algorithmic}
% \end{algorithm}


% \begin{algorithm}[H]
% \caption{Inference Procedure}
% \begin{algorithmic}[1]
% \REQUIRE Context set $({\bf{X}}_C, {\bf{Y}}_C)$, target input ${\bf{X}}_T$
% \ENSURE Prediction ${\bf{Y}}'_T$
% \vspace{0.5em}
% \STATE Estimate the context bases ${\bf{B}}_C$ (Eq. 12).
% \vspace{0.5em}
% \STATE Estimate the object-specific latent variable ${\bf{z}}_o$ based on the context set:
% \[
% {\bf{z}}_o \sim p({\bf{z}}_o \mid {\bf{X}}_C, {\bf{B}}_C) \quad \text{(Eq. 7)}
% \]
% \vspace{0.5em}
% \STATE Estimate the ray-specific latent variables ${\bf{z}}_r^{T}$:
% \[
% {\bf{z}}_r^{T} \sim p({\bf{z}}_r^n \mid {\bf{z}}_o, {\bf{x}}_T^{n}, {\bf{B}}_C) \quad \text{(Eq. 8)}
% \]
% \vspace{0.5em}
% \STATE Modulate the MLP $f$ using the latent variables $\{{\bf{z}}_o, {\bf{z}}_r^{T}\}$ (Eqs. 16 \& 17).
% \vspace{0.5em}
% \STATE Render novel views $\hat{\bf{Y}}_T$ using the modulated MLP $f$.
% \end{algorithmic}
% \end{algorithm}


% $f_C$ by $\{{\bf{z}}_o, {\bf{z}}_r^n\}_C$, 





\section{Implementation Details}
We train all our models with PyTorch. Adam optimizer is used with a learning rate of $1e-4$. For NeRF-related experiments, we follow the baselines~\citep{chen2022transformers,guo2023versatile} to train the model for 1000 epochs. All experiments are conducted on four NVIDIA A5000 GPUs. For the hyper-parameters $\alpha$ and $\beta$, we simply set them as $0.001$.  


{\paragraph{Model Complexity} The comparison of the number of parameters is presented in Table.~\ref{tab:params_psnr}. Our method, GeomNP, utilizes fewer parameters than the baseline, VNP, while achieving better performance on the ShapeNet Car dataset in terms of PSNR.}

\begin{table}[h!]
\centering
\caption{Comparison of the number of parameters and PSNR on the ShapeNet Car dataset.}
\begin{tabular}{lcc}
\toprule
Method & {\# Parameters} & {PSNR} \\ 
\midrule
VNP     & 34.3M   & 24.21 \\ 
GeomNP  & \textbf{24.0M}   & \textbf{25.13} \\ 
\bottomrule
\end{tabular}
\label{tab:params_psnr}
\end{table}

% \paragraph{Integration with PixelNeRF} 
% \textcolor{blue}{To integrate our method into PixelNeRF, we utilize the same feature extractor and NeRF architecture. Specifically, we employ a pre-trained ResNet to extract features from the observed images. From the latent space of the feature encoder, we predict geometric bases, which are used to re-represent each 3D point in a higher-dimensional space. These re-represented point features are aggregated into latent variables, which are then used to modulate the first two input MLP layers of PixelNeRF's NeRF network. During training, we align the latent variables derived from the context images with those from the target views to ensure consistency.}

% \newpage

\section{More Experimental Results}
\label{supp:more-results}
%%%%%%%%%%%%%%%%%%%%%%%%%%%%%%%%%%%%%%%%%%%%%%%%%%%%%%%%%%%%
%\subsection{Image Regression}

% \begin{figure*}[htbp]
%     \centering
%     \begin{minipage}[b]{0.45\textwidth} 
%         \includegraphics[width=\textwidth]{Figures/image-regression0.pdf} % Adjust filename as needed
%         \caption{CelebA}
%         \label{fig:celeba}
%     \end{minipage}
%     \hfill
%     \begin{minipage}[b]{0.45\textwidth} 
%         \includegraphics[width=\textwidth]{Figures/image-regression1.pdf} % Adjust filename as needed
%         \caption{Imagenette}
%         \label{fig:imagenette}
%     \end{minipage}
%     \caption{\textbf{Visualizations} of image regression results on CelebA (left) and Imagenette (right).}
%     \label{fig:visualization-image-regression}
% \end{figure*}

\begin{figure}[htbp]
  \centering
\includegraphics[width=0.8\textwidth]{Figures/imagenette-more.png} % Adjust the size and filename as needed
  \caption{\textbf{More image regression results on the Imagenette dataset.} Left: ground truth; Right: prediction.} % Caption for the figure
  \label{fig:image-supp-image}
  % \vspace{-3mm}
\end{figure}


\subsection{Image Regression}
\label{supp:image-regression}
We provide more image regression results to demonstrate the effectiveness of our method as shown in Fig.~\ref{fig:image-supp-image}. 

 \paragraph{Image Completion.} In addition, we also conduct experiments of \name{} on image completion (also called image inpainting), which is a more challenging variant of image regression. Essentially, only part of the pixels are given as context, while the INR functions are required to complete the full image. Visualizations in Fig.~\ref{fig:completion} demonstrate the generalization ability of our method to recover realistic images with fine details based on very limited context ($10 \% - 20\%$ pixels).


\begin{figure}[t!]
  \centering
  \includegraphics[width=0.99\textwidth]{Figures/image-completion.pdf} % Adjust the size and filename as needed
  \vspace{-3mm}
  \caption{\textbf{Image completion visualization} on CelebA using $10\%$ (left) and $20\%$ (right) context.}
  \label{fig:completion}
  \vspace{-5mm}
\end{figure}

\subsection{Comparison with GNT.}
\label{sec:compare_gnt}
For fair comparison, we use GNT's image encoder and predict the geometric bases, and GNT's NeRF' network for prediction. Fig.~\ref{fig:1-view-compare} shows that our method is effective when very limited context information is given, while GNT fails. This indicates that our method can sufficiently utilize the given information. 



\begin{figure}[htbp]
  \centering
  \includegraphics[width=0.99\textwidth]{./Figures/nerf-syn-1view.pdf} % Adjust the size and filename as needed
  \vspace{-3mm}
  \caption{\textbf{Qualitative comparison with GNT on 1-view setting.}}
  \label{fig:1-view-compare}
  \vspace{-5mm}
\end{figure}


\paragraph{Cross-Category Example.} Additionally, we perform cross-category evaluation without retraining the model. The model is trained on \texttt{drums} category and evaluated on \texttt{lego}.
As shown in Figure~\ref{fig:cross-category}, \name{} leverages the available context information more effectively, producing higher-quality generations with better color fidelity compared to GNT.
% \label{sec:cross-category}

\begin{figure}[htbp]
  \centering
  \includegraphics[width=0.99\textwidth]{./Figures/cross-category.pdf} % Adjust the size and filename as needed
  \vspace{-3mm}
  \caption{\textbf{Qualitative comparison of cross-category ability.} }
  \label{fig:cross-category}
  \vspace{-5mm}
\end{figure}



% \subsection{Comparison with PixelNeRF}

% \begin{wraptable}{r}{0.42\textwidth}
% \vspace{-4mm}
% \caption{\textbf{Comparison on the DTU MVS dataset.} Training with 1-view context and testing with both 1-view and 3-view context images. Integrating \method{} into the pixelNeRF framework leads to improvement in terms of both PSNR and SSIM.}
% \centering
% \resizebox{0.48\textwidth}{!}{
% \begin{tabular}{llcc}
% \toprule
%  & {Method} & {PSNR} & {SSIM} \\
% \midrule
% \multirow{2}{*}{1-view} 
% & pixelNeRF & 15.51 & 0.51 \\
% \multirow{-1}{*}{} & \cellcolor{lightblue}\textbf{\method{}} (Ours) & \cellcolor{lightblue}\textbf{15.89} & \cellcolor{lightblue}\textbf{0.58} \\
% \midrule
% \multirow{2}{*}{3-view} 
% & pixelNeRF & 15.80 & 0.56 \\
% \multirow{-1}{*}{} & \cellcolor{lightblue}\textbf{\method{}} (Ours) & \cellcolor{lightblue}\textbf{16.99} & \cellcolor{lightblue}\textbf{0.61} \\
% \bottomrule
% \vspace{-3mm}
% \end{tabular}
% }
% \label{tab:dtu-compare}
% \end{wraptable}
% \noindent {\textbf{Comparison on DTU.}}
% %Our method can be flexibly integrated with other approaches. 
% To ensure a fair comparison with pixelNeRF~\citep{yu2021pixelnerf} using the same encoder and NeRF network architecture, we incorporate our probabilistic framework into pixelNeRF. We conducted experiments on real-world scenes from the DTU MVS dataset~\citep{aanaes2016large}. To explore the capability of dealing with extremely limited context information, we 
% train both models with 1-view context and test the 1-view and 3-view results in terms of PSNR and SSIM~\citep{wang2004image} metrics. Both qualitative results in Table~\ref{tab:dtu-compare} and qualitative results in Fig.~\ref{fig:dtu-visualization} demonstrate our probabilistic modeling can improve the existing methods. Notably, even when trained with a 1-view context image and tested with 3-view context images, our method significantly outperforms pixelNeRF, demonstrating that our probabilistic framework effectively utilizes limited observations.


% \begin{figure}[t]
%   \centering
%   \includegraphics[width=1\textwidth]{ICLR2025/Figures/dtu-results.pdf} % Adjust the size and filename as needed
%   \vspace{-6mm}  
%   \caption{\textbf{Novel view synthesis results with 1-view context on the DTU dataset.} \method{} has a more realistic rendering quality than pixelNeRF~\citep{yu2021pixelnerf} for novel views with extremely limited context views (1-view).} % Caption for the figure
%   \label{fig:dtu-visualization}
%   \vspace{-5mm}
% \end{figure}




\subsection{More results on ShapeNet}
In this section, we demonstrate more experimental results on the novel view synthesis task on ShapeNet in Fig~\ref{fig:nerf-supp-shapenet}, comparison with VNP~\cite{guo2023versatile} in Fig.~\ref{fig:nerf-supp-compare}, and image regression on the Imagenette dataset in Fig.~\ref{fig:image-supp-image}. The proposed method is able to generate realistic novel view synthesis and 2D images.


\begin{figure}[htbp]
  \centering
\includegraphics[width=1\textwidth]{Figures/nerf-results-more-supp.pdf} % Adjust the size and filename as needed
  \caption{\textbf{More NeRF results on novel view synthesis task on ShapeNet objects.}} % Caption for the figure
  \label{fig:nerf-supp-shapenet}
\end{figure}


\begin{figure}[htbp]
  \centering
\includegraphics[width=1\textwidth]{Figures/comparsion_vnp.pdf} % Adjust the size and filename as needed
  \caption{\textbf{Comparison between the proposed method and VNP} on novel view synthesis task for ShapeNet objects. Our method has a better rendering quality than VNP for novel views.} % Caption for the figure
  \label{fig:nerf-supp-compare}
\end{figure}


\subsection{Training Time Comparison}

{As illustrated in Fig.\ref{fig:train-time}, with the same training time, our method (GeomNP) demonstrates faster convergence and higher final PSNR compared to the baseline (VNP). }

\begin{figure}[htbp]
  \centering
  \includegraphics[width=0.5\textwidth]{./Figures/train_time_psnr.png} % Adjust the size and filename as needed
  \caption{\textbf{Training time vs. PSNR on the ShapeNet Car dataset.} Our method (GeomNP) demonstrates faster convergence and higher final PSNR compared to the baseline (VNP).} % Caption for the figure
  \label{fig:train-time}
  % \vspace{-3mm}
\end{figure}






\subsection{Qualitative ablation of the hierarchical latent variables}
\label{sec:abl-bases-qua}
{In this section, we perform a qualitative ablation study on the hierarchical latent variables. As illustrated in Fig.~\ref{fig:hier-abl}, the absence of the global variable prevents the model from accurately predicting the object's outline, whereas the local variable captures fine-grained details. When both global and local variables are incorporated, GeomNP successfully estimates the novel view with high accuracy.}



\begin{figure}[htbp]
  \centering
  \includegraphics[width=0.7\textwidth]{./Figures/hierarchical-ablation-new.pdf} % Adjust the size and filename as needed
  \caption{\textbf{Qualitative ablation of the hierarchical latent variables (global and local variables)}. }  % Caption for the figure
  \label{fig:hier-abl}
  % \vspace{-3mm}
\end{figure}



\subsection{More multi-view reconstruction results}
{We integrate our method into GNT~\citep{wang2022attention} framework and perform experiments on the Drums class of the NeRF synthetic dataset. Qualitative comparisons of multi-view results are presented in Fig.~\ref{fig:qua-nerf-syn}. }

\begin{figure}[htbp]
  \centering
  \includegraphics[width=0.85\textwidth]{./Figures/nerf-syn.pdf} % Adjust the size and filename as needed
  \caption{\textbf{Qualitative comparisons of Multi-view results on the Drums class of the NeRF synthetic dataset. } }  % Caption for the figure
  \label{fig:qua-nerf-syn}
  % \vspace{-3mm}
\end{figure}







\section{Extending G-NPF to NeRFs}
\label{sec:appendix-neural-radiance-fields}


% \js{Double check connection sentences between subsections}

\noindent \textbf{Notations.} We denote 3D world coordinates by \(\mathbf{p} = (x, y, z)\) and a camera viewing direction by \(\mathbf{d} = (\theta, \phi)\). Each point in 3D space have its color \(\mathbf{c}(\mathbf{p}, \mathbf{d})\), which depends on the location \(\mathbf{p}\) and viewing direction \(\mathbf{d}\). Points also have a density value \(\sigma(\mathbf{p})\) that encodes opacity. We represent coordinates and view direction together as $\mathbf{x} = \{\mathbf{p},\mathbf{d} \}$, color and density together as \(\mathbf{y}(\mathbf{p}, \mathbf{d}) = \{\mathbf{c}(\mathbf{p}, \mathbf{d}), \sigma(\mathbf{p})\}\).
When observing a 3D object from multiple locations, we denote all 3D points as \(\mathbf{X} = \{\mathbf{x}_n \}_{n=1}^N\) and their colors and densities as \(\mathbf{Y} = \{\mathbf{y}_n\}_{n=1}^N\).
Assuming a ray \(\mathbf{r} = (\mathbf{o}, \mathbf{d})\) starting from the camera origin \(\mathbf{o}\) and along direction \(\mathbf{d}\), we sample $P$ points along the ray, with \(\mathbf{x}^{\mathbf{r}} = \{{x}_i^\mathbf{r}\}_{i=1}^P\) and corresponding colors and densities \(\mathbf{y}^{\mathbf{r}} = \{{y}_i^{\mathbf{r}}\}_{i=1}^P\). Further, we denote the observations \(\widetilde{\mathbf{X}}\) and \(\widetilde{\mathbf{Y}}\) as: the set of camera rays \(\widetilde{\mathbf{X}} = \{\widetilde{\mathbf{x}}_n = \mathbf{r}_n\}_{n=1}^N\) and the projected 2D pixels from the rays \(\widetilde{\mathbf{Y}} = \{\widetilde{\mathbf{y}}_n\}_{n=1}^N\).



% \begin{figure}[t]
%   \centering
%   \includegraphics[width=0.49\textwidth]{Figures/problemstate.pdf} % Adjust the size and filename as needed
%   \caption{Framework.} % Caption for the figure
%   \label{fig:problem}
% \end{figure}



% \str{Can you clarify what exactly each notation style corresponds to? What is the $\sim$ for instance? The sampled new pixel views? And no tilde are the actual observations?}

\begin{figure}[htbp]
%\vspace{-5mm}
\centering
\centerline{
\includegraphics[width=0.4\columnwidth]{Figures/problemstate.pdf} 
} 
%\vspace{-2mm}
\caption{\textbf{Complete rendering from 3D points to a 2D pixel.}
}
%\vspace{-4mm}
\label{fig:problem}
\end{figure}

\textbf{Background on Neural Radiance Fields.}
We formally describe Neural Radiance Field (NeRF)~\citep{mildenhall2021nerf, arandjelovic2021nerf} as a continuous function \( f_{\text{NeRF}}: \mathbf{x} \mapsto \mathbf{y} \), which maps 3D world coordinates \(\mathbf{p}\) and viewing directions \(\mathbf{d}\) to color and density values \(\mathbf{y}\). 
That is, a NeRF function, \( f_{\text{NeRF}} \), is a neural network-based function that represents the whole 3D object (e.g., a car in Fig.~\ref{fig:problem}) as coordinates to color and density mappings. Learning a NeRF function of a 3D object is an inverse problem where we only have indirect observations of arbitrary 2D views of the 3D object, and we want to infer the entire 3D object's geometry and appearance.
With the NeRF function, given any camera pose, we can render a view on the corresponding 2D image plane by marching rays and using the corresponding colors and densities at the 3D points along the rays. Specifically, given a set of rays \(\mathbf{r}\) with view directions \(\mathbf{d}\), we obtain a corresponding 2D image. The integration along each ray corresponds to a specific pixel on the 2D image using the volume rendering technique described in~\cite{kajiya1984ray}, which is also illustrated in Fig.~\ref{fig:problem}. Details about the integration are given in Appendix~\ref{supp:nerf-render}. 



%\str{Write down the integral.}
%


% Neural Fields are normally considered as an optimization routine in a deterministic setting, whereby95
% the function fNeRF is fit perfectly to the available observations (akin to “overfitting” training data).


\subsection{Probabilistic NeRF Generalization}
% \js{probabilistic NeRF is not new}

\paragraph{Deterministic Neural Radiance Fields} Neural Radiance Fields are normally considered as an optimization routine in a deterministic setting~\citep{mildenhall2021nerf,barron2021mip}, whereby the function $f_{\text{NeRF}}$ fits specifically to the available observations (akin to ``overfitting'' training data).

\paragraph{Probabilistic Neural Radiance Fields} As we are not just interested in fitting a single and specific 3D object but want to learn how to infer the Neural Radiance Field of any 3D object,  we focus on probabilistic Neural Radiance Fields with the following factorization:
\begin{equation}
    p({\bf{\widetilde{Y}}} | {\bf{\widetilde{X}}}) \varpropto
    \underbrace{p({\bf{\widetilde{Y}}} | {\bf{{Y}}}, {\bf{{X}}})}_{\text{Integration}}
    \underbrace{p({\bf{{Y}}} | {\bf{{X}}})}_{\text{NeRF Model}}
    \underbrace{p({\bf{{X}}} | {\bf{\widetilde{X}}})}_{\text{Sampling}}.
\label{eq: probabilitic_NeRF}
\end{equation}
%
% \str{Is $f_\text{NeRF}$ now a random variable? Normally it is not.}
%\str{This can also be writtena  more fluently}
The generation process of this probabilistic formulation is as follows.
We first start from (or sample) a set of rays $\widetilde{\mathbf{X}}$.
Conditioning on these rays, we sample 3D points in space $\mathbf{X} \big|\widetilde{\mathbf{X}}$.
Then, we map these 3D points into their colors and density values with the NeRF function, ${\bf{Y}} = f_{\text{NeRF}}({\bf{{X}}})$.
Last, we sample the 2D pixels of the viewing image that corresponds to the 3D ray ${\widetilde{\bf{Y}}}| {\bf{{Y}}}, {\bf{X}}$ with a probabilistic process. This corresponds to integrating colors and densities ${\bf{{Y}}}$ along the ray on locations ${\bf{X}}$.

% In the following sections, we will define the various probabilistic terms.
% \str{Here it would be good to be more explicit and say how are the various probabistic terms are defined. Or we can say that we will specify later, also in the context of Geometric NP. Either way, the current text below looks like deterministic relations, so I think we can not write them down here.}

%\str{I suggest we go directly on conditional neural fields. The way we have it now, we only create extra confusion, unless we are the first to propose this decomposition (but there have been other probabilistic NeRFs before, no?). Or perhaps have better structure in the writing, otherwise it is confusing.. What is context, what target?}
The probabilistic model in \cref{eq: probabilitic_NeRF} is for a single 3D object, thus requiring optimizing a function $f_{\text{NeRF}}$ afresh for every new object, which is time-consuming. For NeRF generalization, we accelerate learning and improve generalization by amortizing the probabilistic model over multiple objects, obtaining per-object reconstructions by conditioning on context sets ${{\widetilde {\bf{X}}}_C, {\widetilde {\bf{Y}}}_C}$.
% \str{What about $\widetilde {\bf{X}}_T$? What is that? Also, why (1) has small letters, and here we have capitals?}
% These context variables are few observations from any new object, that is, the rays and the corresponding observed colors.
For clarity, we use ${(\cdot)}_{C}$ to indicate context sets with {a few new observations for a new object}, while ${(\cdot)}_{T}$ indicates target sets containing 3D points or camera rays from novel views of the same object.
Thus, we formulate a probabilistic NeRF for generalization as:
% \str{update this according to the above equation}
%
\begin{equation}
\begin{aligned}
    &p({\bf{\widetilde{Y}}}_{T} | {\bf{\widetilde{X}}}_{T}, {\bf{\widetilde{X}}}_{C}, {\bf{\widetilde{Y}}}_{C}) \varpropto \\
&    \underbrace{p({\bf{\widetilde{Y}}}_{T} | {\bf{{Y}}}_{T}, {\bf{{X}}}_{T})}_{\text{Integration}}
    \underbrace{p({\bf{{Y}}}_{T} | {\bf{{X}}}_{T}, {\bf{\widetilde{X}}}_{C}, {\bf{\widetilde{Y}}}_{C})}_{\text{NeRF Generalization}}
    \underbrace{p({\bf{{X}}}_{T} | {\bf{\widetilde{X}}}_{T})}_{\text{Sampling}}.
\end{aligned}
\label{eq: probabilitic_NeRF_generalization}
\end{equation}
%
%\str{Not sure if this sentence is good enough, please check later again.}
As this paper focuses on generalization with new 3D objects, we keep the same sampling and integrating processes as in ~\cref{eq: probabilitic_NeRF}. We turn our attention to the modeling of the predictive distribution $p({\bf{{Y}}}_{T}| {\bf{{X}}}_{T}, {\bf{\widetilde{X}}}_{C}, {\bf{\widetilde{Y}}}_{C})$ in the generalization step, which implies inferring the NeRF function.

\paragraph{Misalignment between 2D context and 3D structures} It is worth mentioning that the predictive distribution in 3D space is conditioned on 2D context pixels with their ray $\{{\bf{\widetilde{X}}}_{C}, {\bf{\widetilde{Y}}}_{C}\}$ and 3D target points ${\bf{X}}_{T}$, which is challenging due to potential information misalignment. Thus, we need strong inductive biases with 3D structure information to ensure that 2D and 3D conditional information is fused reliably.




% \subsection{Geometric Neural Processes for NeRF} 
\subsection{Geometric Bases} 
\label{sec: geometrybases}
% In NeRF generalization, given that the context set is expected to correspond to too few views with few 3D information, 
% % In NeRF generalization, given the context set corresponding to too few 2D views providing few 3D information, 
% fitting a model for $p({\bf{Y}}_{T}| {\bf{X}}_{T}, {\bf{\widetilde{X}}}_{C}, {\bf{\widetilde{Y}}}_{C})$ that generalizes well is challenging. 
To mitigate the information misalignment between 2D context views and 3D target points, we introduce geometric bases ${\bf{{B}}}_{C}=\{{\bf{b}}_i\}_{i=1}^{M}$, which {induces prior structure to the context set} $\{{\bf{\widetilde{X}}}_{C}, {\bf{\widetilde{Y}}}_{C}\}$ geometrically. $M$ is the number of geometric bases. 

\begin{figure*}[t]
  \centering  \includegraphics[width=0.99\textwidth]{./Figures/architecture-0.pdf} % Adjust the size and filename as needed
  % \vspace{-2mm}
\caption{\textbf{Illustration of our Geometric Neural Processes.} 
% We solve the problem of radiance field generalization by Neural Processes. 
% captures uncertainty induced by few available observations.
We cast radiance field generalization as a probabilistic modeling problem. Specifically, we first construct geometric bases ${\bf{B}}_C$ in 3D space from the 2D context sets ${\bf{\widetilde{X}}}_{C}, {\bf{\widetilde{Y}}}_{C}$ to model the 3D NeRF function (Section~\ref{sec: geometrybases}). We then infer the NeRF function by modulating a shared MLP through hierarchical latent variables ${\bf{z}}_{o}, {\bf{z}}_{r}$ and make predictions by the modulated MLP (Section~\ref{sec: hierar}). 
  The posterior distributions of the latent variables are inferred from the target sets ${\bf{\widetilde{X}}}_{T}, {\bf{\widetilde{Y}}}_{T}$, which supervises the priors during training (Section~\ref{sec: object}). 
  } % Caption for the figure
  \label{fig: framework}
  %\vspace{-2mm}
\end{figure*}

%\str{where is the semantic representation coming from? Self-supervised models? Or is it learned?}
Each geometric basis consists of a Gaussian distribution in the 3D point space and a semantic representation, \textit{i.e.,} ${\bf{b}}_i = \{ \mathcal{N}(\mu_i, \Sigma_i); \omega_i\}$, 
%\str{What is $\omega_i$ in the equation? The weight of the Gaussian?We have mixtures of Gaussians? Please clarify.} 
where $\mu_i$ and $\Sigma_i$ are the mean and covariance matrix of $i$-th Gaussian in 3D space, and $\omega_i$ is its corresponding latent representation. 
Intuitively, the mixture of all 3D Gaussian distributions implies the structure of the object, while $\omega_i$ stores the corresponding semantic information.
% from a 2D context set, e.g., color and texture. 
In practice, we use a transformer-based encoder to learn the Gaussian distributions and representations from the context sets, \textit{i.e.,} $\{(\mu_i, \Sigma_i, \omega_i)\} = \texttt{Encoder} [{\bf{\widetilde{X}}}_{C}, {\bf{\widetilde{Y}}}_{C}]$. Detailed architecture of the encoder is provided in Appendix~\ref{supp:gaussian}. 



With the geometric bases $\mathbf{B}_C$, we review the predictive distribution from  $p({\bf{Y}}_{T}| {\bf{X}}_{T}, {\bf{\widetilde{X}}}_{C}, {\bf{\widetilde{Y}}}_{C})$ to $p({\bf{Y}}_{T}| {\bf{X}}_{T},{\bf{{B}}}_{C})$.  By inferring the function distribution $p(f_{\text{NeRF}})$, we reformulate the predictive distribution as: 
\begin{equation}
    % p({\bf{{Y}}}_{T} | {\bf{{X}}}_{T}, {\bf{\widetilde{X}}}_{C}, {\bf{\widetilde{Y}}}_{C}) = 
    p({\bf{{Y}}}_{T} | {\bf{{X}}}_{T}, {\bf{{B}}}_{C}) = \int p({\bf{Y}}_{T}|f_{\text{NeRF}}, {\bf{X}}_{T}) p(f_{\text{NeRF}}| {\bf{X}}_{T}, {\bf{B}}_{C}) df_{\text{NeRF}},
\label{eq: predictive_w_B}
\end{equation}
where $p(f_{\text{NeRF}}| {\bf{X}}_{T}, {\bf{B}}_{C})$ is the prior distribution of the NeRF function, and $p({\bf{Y}}_{T}|f_{\text{NeRF}}, {\bf{X}}_{T})$ is the likelihood term. 
% We integrate the likelihood term with all possible NeRF functions. 
%\str{How do we do this? The space to integrate over must be huge, no?}
%\str{The following sentence is a bit weird, can you check it again.}
% \wy{We integrate the likelihood term over the latent space of all possible variables for modulating NeRF functions by monte carlo sampling, which can be seen as integrating over a function distribution.}
Note that the prior distribution of the NeRF function is conditioned on the target points ${\bf{X}}_{T}$ and the geometric bases ${\bf{B}}_{C}$. 
Thus, the prior distribution is data-dependent on the target inputs, yielding a better generalization on novel target views of new objects. 
Moreover, since ${\bf{B}}_{C}$ is constructed with continuous Gaussian distributions in the 3D space, the geometric bases can enrich the locality and semantic information of each discrete target point, enhancing the capture of high-frequency details~\citep{chen2023neurbf,chen2022tensorf,muller2022instant}.



\subsection{Geometric Neural Processes with Hierarchical Latent Variables}
\label{sec: hierar}

% To achieve the 
With the geometric bases, we propose Geometric Neural Processes (\textbf{\method{}}) by inferring the NeRF function distribution $p(f_{\text{NeRF}}|{\bf{X}}_{T}, {\bf{{B}}}_{C})$ in a probabilistic way.  
% We can generalize NeRF learning and efficiently adapt the functional distribution to new 3D objects.
Based on the probabilistic NeRF generalization in~\cref{eq: probabilitic_NeRF_generalization}, we introduce hierarchical latent variables to encode various spatial-specific information into $p(f_{\text{NeRF}}|{\bf{X}}_{T}, {\bf{{B}}}_{C})$, improving the generalization ability in different spatial levels.
%\str{Make sure that notation is consistent and not overloaded. Eg, $x^r$ rather than $x^\mathbf{r}$ since it is not that we use the $1:P$ somewhere specific, besides it is not clear that this corresponds to a ray, since the $P$ points could be anywhere.}
Since all rays are independent of each other, we decompose the predictive distribution in \cref{eq: predictive_w_B} as:
\begin{equation}
    p({\bf{Y}}_{T}| {\bf{X}}_{T},  {\bf{B}}_{C})  = \prod_{n=1}^{N} p({\bf{y}}_{T}^{\mathbf{r}, n}| {\bf{x}}_{T}^{{\mathbf{r}}, n},  {\bf{B}}_{C}),
\label{eq: predictive_distribution_ray_specific}
\end{equation}
where the target input ${\bf{X}}_{T}$ consists of $N \times P$ location points $\{{\bf{x}}_{T}^{{\mathbf{r}}, n}\}_{n=1}^{N}$ for $N$ rays.


\begin{figure}[htbp]
\centering
\includegraphics[width=0.45\columnwidth]{./Figures/graphical_model2.pdf} 
\caption{\textbf{Graphical model for the proposed geometric neural processes.}}
\label{fig-supp: graphical_model}
\end{figure}

Further, we develop a hierarchical Bayes framework for \method{} to accommodate the data structure of the target input ${\bf{X}}_{T}$ in \cref{eq: predictive_distribution_ray_specific}.
We introduce an object-specific latent variable $\mathbf{z}_o$ and $N$ individual ray-specific latent variables $\{\mathbf{z}_r^{n}\}_{n=1}^{N}$ to represent the randomness of $f_\text{NeRF}$.
% the probabilistic NeRF function. 


%\str{The formatting here looks weird. Is this the right template?}
Within the hierarchical Bayes framework, $\mathbf{z}_o$ encodes the entire object information from all target inputs and the geometric bases $\{\mathbf{X}_T, \mathbf{B}_C\}$ in the global level; while every $\mathbf{z}_r^{n}$ encodes ray-specific information from $\{ \mathbf{x}_T^{\mathbf{r}, n}, \mathbf{B}_C\}$ in the local level, which is also conditioned on the global latent variable $\mathbf{z}_o$. 
The hierarchical architecture allows the model to exploit the structure information from the geometric bases $\mathbf{B}_C$ in different levels, improving the model's expressiveness ability.
By introducing the hierarchical latent variables in \cref{eq: predictive_distribution_ray_specific}, we model \method{} as:
{\small
\begin{equation}
\begin{aligned}
        p({\bf{Y}}_{T}| {\bf{X}}_{T}, {\bf{B}}_{C}) &= \int \prod_{n=1}^{N} \Big\{ \int p({\bf{y}}_{T}^{\mathbf{r}, n}| {\bf{x}}_{T}^{\mathbf{r}, n}, {\bf{B}}_{C}, {\bf{z}}_r^n,{\bf{z}}_o ) \\
        &p({\bf{z}}_{r}^n| {\bf{z}}_o,  {\bf{x}}_{T}^{\mathbf{r}, n}, {\bf{B}}_C) d {\bf{z}}_r^n \Big\} p({\bf{z}}_o |{\bf{X}}_T, {\bf{B}}_C) d {\bf{z}}_o,
\end{aligned}
\label{eq:ganp-model}
\end{equation}
}where $p({\bf{y}}_{T}^{\mathbf{r}, n}| {\bf{x}}_{T}^{\mathbf{r}, n}, {\bf{B}}_{C}, {\bf{z}}_o, {\bf{z}}_r^i)$ denotes the ray-specific likelihood term. In this term, we use the hierarchical latent variables $\{{\bf{z}}_o, {\bf{z}}_r^i\}$ to modulate a ray-specific NeRF function $f_{\text{NeRF}}$ for prediction, as shown in Fig.~\ref{fig: framework}.
% In general, we first use the object-specific latent variable $\mathbf{z}_o$ to make $f_{NeRF}$ object-specific. Then, the ray-specific latent variable $\mathbf{z}_r$ to enable $f_{\text{NeRF}}$ to capture the local texture information.
Hence, $f_{\text{NeRF}}$ can explore global information of the entire object and local information of each specific ray, leading to better generalization ability on new scenes and new views.
A graphical model of our method is provided in Fig.~\ref{fig-supp: graphical_model}. 


In the modeling of {\method{}}, the prior distribution of each hierarchical latent variable is conditioned on the geometric bases and target input. 
%\textcolor{blue}{To infer each latent variable, we first integrate the geometric bases and each target input, yielding a location representation specific to the target input. The location representation has access to relevant locality information from the geometry bases, \textit{i.e.}, $<{\bf{x}}_{T}^{\mathbf{r}, n}, {\bf{B}}_C >$. }
% For generalization, we need to infer latent variables that are specific to the target input. 
% To this end, 
We first represent each target location by integrating the geometric bases, \textit{i.e.}, $<{\bf{x}}_{T}^{n}, {\bf{B}}_C >$, which aggregates the relevant locality and semantic information for the given input. 
Since ${\bf{B}}_{C}$ contains $M$ Gaussians, we employ a Gaussian radial basis function in \cref{suppeq:rbf_agg} between each target input ${\bf{x}}_{T}^{ n}$ and each geometric basis ${\bf{b}}_i$ to aggregate the structural and semantic information to the 3D location representation. Thus, we obtain the 3D location representation as follows:
\begin{equation}
\label{suppeq:rbf_agg}
    <{\bf{x}}_{T}^{n}, {\bf{B}}_C > = \texttt{MLP}\Big[\sum_i^{M} \exp (-\frac{1}{2}({\bf{x}}_{T}^{n}-\mu_i)^T\Sigma_i^{-1}({\bf{x}}_{T}^{n}-\mu_i) ) \cdot \omega_i\Big],
\end{equation} 
where $\texttt{MLP}[\cdot]$ is a learnable neural network.
With the location representation $<{\bf{x}}_{T}^{n}, {\bf{B}}_C >$, we next infer each latent variable hierarchically, in object and ray levels. 

\noindent {\textbf{Object-specific Latent Variable.}} The distribution of the object-specific latent variable ${\bf{z}}_o$ is obtained by aggregating all location representations:
\begin{equation}
    [\mu_{{o}}, \sigma_{{o}}] 
    = \texttt{MLP}\Big[\frac{1}{N \times P}\sum_{n = 1}^{N}\sum_{\mathbf{r}}
    <{\bf{x}}_{T}^{n}, {\bf{B}}_C >\Big],
\end{equation} 
where we assume $p({\bf{z}}_o | {\bf{B}}_C,  {\bf{X}}_T)$ is a standard Gaussian distribution and generate its mean $\mu_{o}$ and variance $\sigma_{o}$ by a ~\texttt{MLP}. 
Thus, our model captures objective-specific uncertainty in the NeRF function.


\noindent {\textbf{Ray-specific Latent Variable.}} 
% By ray-specific latent variable, the object-specific is expected to capture the local details.
To generate the distribution of the ray-specific latent variable, we first average the location representations ray-wisely. 
We then obtain the ray-specific latent variable by aggregating the averaged location representation and the object latent variable through a lightweight transformer. We formulate the inference of the ray-specific latent variable as:
\begin{equation}
    [\mu_{{r}}, \sigma_{{r}}] = \texttt{Transformer} \Big[\texttt{MLP}[\frac{1}{P}\sum_{\mathbf{r}}
    <{\bf{x}}_{T}^{n}, {\bf{B}}_C >]; \hat{{\bf{z}}}_o \Big],
\end{equation}
where $\hat{{\bf{z}}}_o$ is a sample from the prior distribution $p({\bf{z}}_o | {\bf{X}}_T, {\bf{B}}_C)$. 
Similar to the object-specific latent variable, we also assume the distribution $p({\bf{z}}_r^n| {\bf{z}}_o,  {\bf{x}}_{T}^{\mathbf{r}, n}, {\bf{B}}_C)$ is a mean-field Gaussian distribution with the mean $\mu_{{r}}$ and variance $\sigma_{{r}}$. We provide more details of the latent variables in Appendix~\ref{supp:latent-variables}.



\noindent  \textbf{NeRF Function Modulation.}
With the hierarchical latent variables $\{{\bf{z}}_o, {\bf{z}}_r^n\}$, we modulate a neural network for a 3D object in both object-specific and ray-specific levels.  Specifically, the modulation of each layer is achieved by scaling its weight matrix with a style vector~\citep{guo2023versatile}. 
The object-specific latent variable ${\bf{z}}_o$ and ray-specific latent variable ${\bf{z}}_r^n$ are taken as style vectors of the low-level layers and high-level layers, respectively. The prediction distribution $p({\bf{Y}}_{T}| {\bf{X}}_{T}, {\bf{B}}_{C})$ are finally obtained by passing each location representation through the modulated neural network for the NeRF function. 
More details are provided in Appendix~\ref{supp:modulate}. 
% The modulated MLP layer used in our paper is similar to the style \textit{modulation} in ~\cite{guo2023versatile}. Essentially, we predict a style vector $s\in \mathbb{R}^{d_{in}}$ to multiply or scale the weight matrix of an MLP, $W \in \mathbb{R}^{d_{in} \times d_{out}}$.


% \subsection{Inference of Geometry-aware Neural Processes}
% \label{sec: elbo}
% ELBO

% \noindent{\textbf{Variational Posteriors with the Geometry Bases.}} Solving \textbf{GANPs} with Eq.~\ref{eq:ganp-model} involves estimating the true posterior, $p({\bf{g}},  \{{\bf{r}}_i\}_{r=1}^{N_{ray}} | {\widetilde{\bf{X}}}_T, {\widetilde{\bf{Y}}}_T)$ which is intractable. Hence, we introduce a variational posterior distribution, which can be factorized as follows:
% \begin{equation}
% p({\bf{g}},  \{{\bf{r}}_i\}_{r=1}^{N_{ray}} | {\widetilde{\bf{X}}}_T, {\widetilde{\bf{Y}}}_T) \approx q_{\theta, \phi}({\bf{g}},  \{{\bf{r}}_i\}_{r=1}^{N_{ray}} | {\bf{X}}_T, {\bf{B}}_T),
% \end{equation}

\subsection{Empirical Objective}
\label{sec: object}

\noindent{\textbf{Evidence Lower Bound.}} 
To optimize the proposed \method{},
we apply variational inference~\citep{garnelo2018neural} and derive the evidence lower bound (ELBO) as:
\begin{equation}
\begin{aligned}
% \mathcal{L}_{\text{ELBO}}
& \log   p({\bf{Y}}_{T}| {\bf{X}}_{T}, {\bf{B}}_{C})
\geq \\
&\mathbb{E}_{q({\bf{z}}_o | {\bf{B}}_T,  {\bf{X}}_T)}  \Big\{  \sum_{n=1}^{N}  \mathbb{E}_{q({\bf{z}}_r^n| {\bf{z}}_o,  {\bf{x}}_{T}^{\mathbf{r}, n}, {\bf{B}_T})} \log p({\bf{y}}_{T}^{{\mathbf{r}}, n}| {\bf{x}}_{T}^{{\mathbf{r}}, n}, {\bf{z}}_o, {\bf{z}}_r^n) \\
&- D_{\text{KL}}[q({\bf{z}}_r^n| {\bf{z}}_o,  {\bf{x}}_{T}^{{\mathbf{r}}, n}, {{\bf{B}}_T}) || p({\bf{z}}_r^n| {\bf{z}}_o,  {\bf{x}}_{T}^{{\mathbf{r}}, n}, {{\bf{B}}_C}) ] \Big\} \\
& - D_{\text{KL}}[q({\bf{z}}_o | {\bf{B}}_T,  {\bf{X}}_T) || p({\bf{z}}_o | {\bf{B}}_C,  {\bf{X}}_T)], \\
\end{aligned}
\end{equation}
where $q_{\theta, \phi}({\bf{z}}_o,  \{{\bf{z}}_r^i\}_{i=1}^{N} | {\bf{X}}_T, {\bf{B}}_T) = \Pi_{i=1}^{N}q({\bf{z}}_r^n| {\bf{z}}_o,  {\bf{x}}_{T}^{{\mathbf{r}}, n}, {{\bf{B}}_T}) q({\bf{z}}_o | {\bf{B}}_T,  {\bf{X}}_T)$ is the involved variational posterior for the hierarchical latent variables.  ${\bf{B}}_T$ is the geometric bases constructed from the target sets $\{{\bf{\widetilde{X}}}_{T}, {\bf{\widetilde{Y}}}_{T}\}$, which are only accessible during training. 
The variational posteriors are inferred from the target sets during training, which introduces more information on the object. 
The prior distributions are supervised by the variational posterior using Kullback–Leibler (KL) divergence, learning to model more object information with limited context data and generalize to new scenes. Detailed derivations are provided in Appendix~\ref{supp:elbo}.

% \noindent{\textbf{Empirical Objective.}} 
For the geometric bases $\mathbf{B}_C$, we regularize the spatial shape of the context geometric bases to be closer to that of the target one $\mathbf{B}_T$ by introducing a KL divergence. 
Therefore, given the above ELBO, our objective function consists of three parts: a reconstruction loss (MSE loss), KL divergences for hierarchical latent variables, and a KL divergence for the geometric bases. 
%constraint for matching the two sets of Gaussian basis to ensure the basis obtained from context is as close to the one from the target. 
The empirical objective for the proposed \method{} is formulated as:
\begin{equation}
\begin{aligned}
& \mathcal{L}_{\text{\method{}}}  =  ||y - y'||^2_2 + \alpha \cdot \big(  D_{\text{KL}} [p(\mathbf{z}_o|{\bf{B}}_C)|q(\mathbf{z}_o|{\bf{B}}_T)] \\
    & + D_{\text{KL}}[p(\mathbf{z}_r|\mathbf{z}_o,{\bf{B}}_C)|q(\mathbf{z}_r|\mathbf{z}_o,{\bf{B}}_T)] \big) + \beta \cdot D_{\text{KL}}[{\bf{B}}_C, {\bf{B}}_T],
\end{aligned}
\end{equation}
where $y'$ is the prediction. $\alpha$ and $\beta$ are hyperparameters to balance the three parts of the objective. The KL divergence on ${\bf{B}}_C, {\bf{B}}_T$ is to align the spatial location and the shape of two sets of bases. 

% +++++++++++++

% \noindent 
% \textbf{Neural Fields.}
% % \zx{First introduce NeRF (with problem definition and notations) and its disadvantages of inefficient inference, then say what we will do to avoid this problem by NP?} 


% In general, training a NeRF requires overfitting each scene, which is time-consuming. Hence, how to leverage the observation for generalization on the new scene remains a problem. As each INR is a function of a data sample, the problem can be viewed as estimating the distribution of functions. This motivates us to use the Neural Process (NP) in this problem. 
 

% \noindent {\textbf{Neural Process.}} Neural Process~\cite{garnelo2018neural} parameterizes the distribution of functions. Given the context set comprising of the observation $X$ and the corresponding labels $Y$, $D_C = (X_C, Y_C) := (\mathbf{x}_i, \mathbf{y}_{i})_{i\in C} $. The aim of NP is to learn a mapping function from the target points $X_T$ to the target labels $Y_T$,  $D_T = (X_T, Y_T) := (\mathbf{x}_i, \mathbf{y}_{i})_{i\in T} $. The conditional distribution of target points is:
% \begin{equation}
%     p_{\phi}(Y_T|X_T,D_C) = \prod_{\mathbf{x}, \mathbf{y}\in D_{T}} \mathcal{N}(\mathbf{y};\mu_{\mathbf{y}}(\mathbf{x},D_c), \sigma^2_{\mathbf{y}}(\mathbf{x},D_c)).
% \end{equation}

% \begin{equation}
%     p_{\phi}(Y_T|X_T, \mathbf{z}) = \prod_{\mathbf{x}, \mathbf{y}\in D_{T}} \mathcal{N}(\mathbf{y};\mu_{\mathbf{y}}(\mathbf{z}, X_T,D_c), \sigma^2_{\mathbf{y}}(\mathbf{z}, X_T,D_c)),
% \end{equation}
% where $\mathbf{z} \sim p_{\theta}(\mathbf{z|X_T,D_C})$. Using NP can efficiently leverage the limited context/observation to infer the function of INR for the target from a probabilistic perspective. It also can incorporate uncertainty estimation for the unseen view. This is reasonable as the value in the unseen target location should not be deterministic. 

%\zx{advantages of NP? incorporating uncertainty for limited context information, which is suitable for reconstruction tasks?}







% \begin{align}
%     p(\mathbf{y}|\mathbf{x}, I) & = \int_{g} \int_{r}  p(\mathbf{y}, g, r |\mathbf{x}, I) \mathrm{d}g  \mathrm{d}r \\
%     &=  \int_{g} \int_{r}  p(\mathbf{y}| g, r) p(g, r | \mathbf{x}, I) \mathrm{d}g  \mathrm{d}r \\
%     % &= \int_{g} \int_{r} \int_{B} p_{\theta_1}(\mathbf{y}| g, r) p_{\theta_2}(g, r | B) p_{\theta_3}(B|\mathbf{x}, I) \mathrm{d}g  \mathrm{d}r \mathrm{d}B 
%     &= \int_{g} \int_{r}  p_{\theta_1}(\mathbf{y}| g, r) p_{\theta_2}(g, r | B) p_{\theta_3}(B|\mathbf{x}, I) \mathrm{d}g  \mathrm{d}r 
% \end{align}


%\subsection{Gaussian Basis for INR}
%As shown in Fig.~\ref{fig:framework}, 


%\subsection{Hierarchical Neural Process for INR}









\subsection{Derivation of Evidence Lower Bound}
\label{supp:elbo}
\noindent{\textbf{Evidence Lower Bound.}}
We optimize the model via variational inference~\citep{garnelo2018neural}, deriving the evidence lower bound (ELBO):
\begin{equation}
\begin{aligned}
& \log p(\mathbf{Y}_T \mid \mathbf{X}_T, \mathbf{B}_C) \geq \\
&\mathbb{E}_{q(\mathbf{z}_g | \mathbf{X}_T, \mathbf{B}_T)} \Bigg[ \sum_{m=1}^M \mathbb{E}_{q(\mathbf{z}_l^m | \mathbf{z}_g, \mathbf{x}_T^m, \mathbf{B}_T)} \log p(\mathbf{y}_T^m \mid \mathbf{z}_g, \mathbf{z}_l^m, \mathbf{x}_T^m) \\
& \quad - D_{\text{KL}}\Big[q(\mathbf{z}_l^m | \mathbf{z}_g, \mathbf{x}_T^m, \mathbf{B}_T) \,\big|\big|\, p(\mathbf{z}_l^m | \mathbf{z}_g, \mathbf{x}_T^m, \mathbf{B}_C)\Big] \Bigg] \\
& - D_{\text{KL}}\Big[q(\mathbf{z}_g | \mathbf{X}_T, \mathbf{B}_T) \,\big|\big|\, p(\mathbf{z}_g | \mathbf{X}_T, \mathbf{B}_C)\Big],
\end{aligned}
\end{equation}
where the variational posterior factorizes as $q(\mathbf{z}_g, \{\mathbf{z}_l^m\}_{m=1}^M | \mathbf{X}_T, \mathbf{B}_T) = q(\mathbf{z}_g | \mathbf{X}_T, \mathbf{B}_T) \prod_{m=1}^M q(\mathbf{z}_l^m | \mathbf{z}_g, \mathbf{x}_T^m, \mathbf{B}_T)$. Here, $\mathbf{B}_T$ denotes geometric bases constructed from target data $\{\widetilde{\mathbf{X}}_T, \widetilde{\mathbf{Y}}_T\}$ (available only during training). The KL terms regularize the hierarchical priors $p(\mathbf{z}_g | \mathbf{B}_C)$ and $p(\mathbf{z}_l^m | \mathbf{z}_g, \mathbf{B}_C)$ to align with variational posteriors inferred from $\mathbf{B}_T$, enhancing generalization to context-only settings. 


The propose \textbf{GeomNP} is formulated as:
{\small
\begin{equation}
        p({\bf{Y}}_{T}| {\bf{X}}_{T}, {\bf{B}}_{C}) = \int \prod_{n=1}^{N} \Big\{ \int p({\bf{y}}_{T}^{\mathbf{r}, n}| {\bf{x}}_{T}^{\mathbf{r}, n}, {\bf{B}}_{C}, {\bf{z}}_r^n,{\bf{z}}_o, ) p({\bf{r}}^n| {\bf{z}}_o,  {\bf{x}}_{T}^{\mathbf{r}, n}, {\bf{B}}_C) d {\bf{z}}_r^n \Big\} p({\bf{z}}_o |{\bf{X}}_T, {\bf{B}}_C) d {\bf{z}}_o, 
\label{eq:ganp-model-supp}
\end{equation}}where $p({\bf{z}}_o | {\bf{B}}_C,  {\bf{X}}_T)$ and $p({\bf{z}}_r^n| {\bf{z}}_o,  {\bf{x}}_{T}^{r, n}, {\bf{B}}_C)$ denote prior distributions of a object-specific and each ray-specific latent variables, respectively. Then, the evidence lower bound is derived as follows.

\begin{equation}
\begin{aligned}
        &\log p({\bf{Y}}_{T}| {\bf{X}}_{T}, {\bf{B}}_{C}) \\
        &= \log \int \prod_{n=1}^{N} \Big\{ \int p({\bf{y}}_{T}^{\mathbf{r}, n}| {\bf{x}}_{T}^{\mathbf{r}, n}, {\bf{z}}_o, {\bf{z}}_r^n) p({\bf{z}}_r^n| {\bf{z}}_o,  {\bf{x}}_{T}^{\mathbf{r}, n}, {\bf{B}_C}) d {\bf{z}}_r^n \Big\} p({\bf{z}}_o | {\bf{B}}_C,  {\bf{X}}_T) d {\bf{z}}_o  \\
    &= \log \int  \prod_{i=1}^{N} \Big\{ \int p({\bf{y}}_{T}^{\mathbf{r}, n}| {\bf{x}}_{T}^{\mathbf{r}, n}, {\bf{z}}_o, {\bf{z}}_r^n) p({\bf{z}}_r^n| {\bf{z}}_o,  {\bf{x}}_{T}^{\mathbf{r}, n}, {\bf{B}_C}) \frac{q({\bf{z}}_r^n| {\bf{z}}_o,  {\bf{x}}_{T}^{\mathbf{r}, n}, {\bf{B}_T})}{q({\bf{z}}_r^n| {\bf{z}}_o,  {\bf{x}}_{T}^{\mathbf{r}, n}, {\bf{B}_T})} d {\bf{z}}_r^n \Big\} \\
    & p({\bf{z}}_o | {\bf{B}}_C,  {\bf{X}}_T) \frac{q({\bf{z}}_o | {\bf{B}}_T,  {\bf{X}}_T)}{q({\bf{z}}_o | {\bf{B}}_T,  {\bf{X}}_T,)} d {\bf{z}}_o  \\
    &\geq  \mathbb{E}_{q({\bf{z}}_o | {\bf{B}}_T,  {\bf{X}}_T)}  \Big\{  \sum_{i=1}^{N} \log  \int p({\bf{y}}_{T}^{\mathbf{r}, n}| {\bf{x}}_{T}^{\mathbf{r}, n}, {\bf{z}}_o, {\bf{z}}_r^n) p({\bf{z}}_r^n| {\bf{z}}_o,  {\bf{x}}_{T}^{\mathbf{r}, n}, {\bf{B}_C}) \frac{q({\bf{z}}_r^n| {\bf{z}}_o,  {\bf{x}}_{T}^{\mathbf{r}, n}, {\bf{B}_T})}{q({\bf{z}}_r^n| {\bf{z}}_o,  {\bf{x}}_{T}^{\mathbf{r}, n}, {\bf{B}_T})} d {\bf{z}}_r^n \Big\} \\
    &- D_{\text{KL}}(q({\bf{z}}_o | {\bf{B}}_T,  {\bf{X}}_T,) || p({\bf{z}}_o | {\bf{B}}_C,  {\bf{X}}_T)) \\
    &\geq  \mathbb{E}_{q({\bf{z}}_o | {\bf{B}}_T,  {\bf{X}}_T)}  \Big\{  \sum_{n=1}^{N}  \mathbb{E}_{q({\bf{z}}_r^n| {\bf{z}}_o,  {\bf{x}}_{T}^{\mathbf{r}, n}, {\bf{B}_T})} \log p({\bf{y}}_{T}^{{\mathbf{r}}, n}| {\bf{x}}_{T}^{{\mathbf{r}}, n}, {\bf{z}}_o, {\bf{z}}_r^n) \\
&- D_{\text{KL}}[q({\bf{z}}_r^n| {\bf{z}}_o,  {\bf{x}}_{T}^{{\mathbf{r}}, n}, {\bf{B}_T}) || p({\bf{z}}_r^n| {\bf{z}}_o,  {\bf{x}}_{T}^{{\mathbf{r}}, n}, {\bf{B}_C}) ] \Big\} 
- D_{\text{KL}}[q({\bf{z}}_o | {\bf{B}}_T,  {\bf{X}}_T) || p({\bf{z}}_o | {\bf{B}}_C,  {\bf{X}}_T)], \\
\end{aligned}      
\end{equation}
where $q_{\theta, \phi}({\bf{z}}_o,  \{{\bf{z}}_r^i\}_{i=1}^{N} | {\bf{X}}_T, {\bf{B}}_T) = q({\bf{z}}_r^n| {\bf{z}}_o,  {\bf{x}}_{T}^{{\mathbf{r}}, n}, {\bf{B}_T}) q({\bf{z}}_o | {\bf{B}}_T,  {\bf{X}}_T)$ is the variational posterior of the hierarchical latent variables. 




\section{More Related Work}
% \textcolor{blue}{
% \cite{szymanowicz2024splatter}, \cite{charatan2024pixelsplat}, \cite{chen2025mvsplat}, \cite{hong2023lrm}, \cite{muller2023diffrf}, \cite{tewari2023diffusion}, \cite{xu2022point}, \cite{wang2024learning}, \cite{liu2024geometry}}


{\paragraph{Generalizable Neural Radiance Fields (NeRF)}
Advancements in neural radiance fields have focused on improving generalization across diverse scenes and objects. \cite{wang2022attention} propose an attention-based NeRF architecture, demonstrating enhanced capabilities in capturing complex scene geometries by focusing on informative regions. \cite{suhail2022generalizable} introduce a generalizable patch-based neural rendering approach, enabling models to adapt to new scenes without retraining. \cite{xu2022point} present \textit{Point-NeRF}, leveraging point-based representations for efficient scene modeling and scalability. \cite{wang2024learning} further enhance point-based methods by incorporating visibility and feature augmentation to improve robustness and generalization. \cite{liu2024geometry} propose a geometry-aware reconstruction with fusion-refined rendering for generalizable NeRFs, improving geometric consistency and visual fidelity. Recently, the \textit{Large Reconstruction Model (LRM)}~\citep{hong2023lrm} has drawn attention. It aims for single-image to 3D reconstruction, emphasizing scalability and handling of large datasets.}

{\paragraph{Gaussian Splatting-based Methods}
Gaussian splatting~\citep{kerbl20233d} has emerged as an effective technique for efficient 3D reconstruction from sparse views. \cite{szymanowicz2024splatter} propose \textit{Splatter Image} for ultra-fast single-view 3D reconstruction. \cite{charatan2024pixelsplat} introduce \textit{pixelsplat}, utilizing 3D Gaussian splats from image pairs for scalable generalizable reconstruction. \cite{chen2025mvsplat} present \textit{MVSplat}, focusing on efficient Gaussian splatting from sparse multi-view images. Our approach can be a complementary module for these methods by introducing a probabilistic neural processing scheme to fully leverage the observation. }

{\paragraph{Diffusion-based 3D Reconstruction}
Integrating diffusion models into 3D reconstruction has shown promise in handling uncertainty and generating high-quality results. \cite{muller2023diffrf} introduce \textit{DiffRF}, a rendering-guided diffusion model for 3D radiance fields. \cite{tewari2023diffusion} explore solving stochastic inverse problems without direct supervision using diffusion with forward models. \cite{liu2023zero} propose \textit{Zero-1-to-3}, a zero-shot method for generating 3D objects from a single image without training on 3D data, utilizing diffusion models. \cite{shi2023zero123++} introduce \textit{Zero123++}, generating consistent multi-view images from a single input image using diffusion-based techniques. \cite{shi2023mvdream} present \textit{MVDream}, which uses multi-view diffusion for 3D generation, enhancing the consistency and quality of reconstructed models.}

% \section{NP with Gaussian Splatting}
% \begin{table}[htbp]
%     \centering
%     \caption{Comparison of methods}
%     \begin{tabular}{lccc}
%         \toprule
%         \textbf{Method} & \textbf{PSNR $\uparrow$} & \textbf{SSIM $\uparrow$} & \textbf{LPIPS $\downarrow$} \\
%         \midrule
%         PixelNeRF & 21.76 & 0.78 & 0.203 \\
%         {Splatter Image} & 21.80 & {0.80} & {0.150} \\
%         \bottomrule
%     \end{tabular}
% \end{table}


% Recently, 3D Gaussian Splatting~\citep{kerbl20233d} has gained significant attention for its efficiency and strong performance in reconstructing 3D scenes. Like NeRF, Gaussian Splatting requires overfitting on a specific scene to optimize the 3D Gaussian parameters. To improve generalization, given a single-view context image, Splatter Image~\citep{szymanowicz2024splatter} employs a UNet to predict Gaussian parameters for a new scene. However, Splatter Image remains a deterministic method and does not account for scene uncertainty. Therefore, in this section, we demonstrate that integrating neural processes can enhance Splatter Image's performance.

% Specifically, we employ the UNet encoder to generate a latent variable, and then sample a scene-specific latent vector to estimate Gaussian parameters through the decoder. For multiple-view ($N$) images, we first aggregate multi-view latent features and then infer the latent variables (mean and variance). We sample $N$ times to probabilistically estimate the Gaussian parameters. The prior and posterior distributions are derived from context and target images, respectively. In addition to the original reconstruction loss, we introduce a KL divergence constraint between the prior and posterior distributions, guiding the model to achieve richer representation with limited observations.

% Experiments are conducted on the CO3D dataset. 


%%%%%%%%%%%%%%%%%%%%%%%%%%%%%%%%%%%%%%%%%%%%%%%%%%%%%%%%%%%%%%%%%%%%%%%%%%%%%%%%%%%%%%%%%%%%%%%%%%

% \subsection{Image Regression}
% \label{sec:image-regression}



% \begin{figure}[t]
%     \centering
%     \begin{subtable}[b]{0.42\textwidth}
%     % \vspace{-6mm}
%     \begin{tabular}{lcc}
%     \toprule
%                  & CelebA & Imagenette \\ \midrule
%     Learned Init \citep{tancik2021learned} & 30.37  & 27.07       \\
%     TransINR~\citep{chen2022transformers}         & 31.96  & 29.01       \\ 
%     % \hline
%     \rowcolor{lightblue}
%     \method{} (Ours)         & \textbf{33.41}  & \textbf{29.82}      \\ 
%     \bottomrule
%     \end{tabular}
%     \caption{Quantitative results. \method{} outperforms baseline methods consistently on both datasets.}
%     \label{tab:image-regression}
%     \end{subtable} \hfill
%     \begin{subtable}[b]{0.54\textwidth}
%     \includegraphics[width=\textwidth]{Figures/image-regression0.pdf} % Adjust the size and filename as needed
%     \caption{Visualizations on CelebA (left) and Imagenette (right), respectively.} % Caption for the figure
%     \label{fig:visualization-image-regression}
% \end{subtable}
% % \vspace{-2mm}
% \caption{\textbf{Quantitative results and visualizations} of image regression on CelebA and Imagenette.}
% \vspace{-3mm}
% \end{figure}

% \begin{figure}[t!]
%   \centering
%   \includegraphics[width=0.99\textwidth]{Figures/image-completion.pdf} % Adjust the size and filename as needed
%   \vspace{-3mm}
%   \caption{\textbf{Image completion visualization} on CelebA using $10\%$ (left) and $20\%$ (right) context.}
%   \label{fig:completion}
%   \vspace{-3mm}
% \end{figure}


% \noindent{\textbf{Setup.}} Image regression is a common task used for evaluating INRs' capacity of representing a signal~\citep{tancik2021learned,sitzmann2020implicit}. 
% We employ two real-world image datasets as used in previous works~\citep{chen2022transformers,tancik2021learned,gu2023generalizable}. The CelebA dataset~\citep{liu2015deep} encompasses approximately 202,000 images of celebrities, partitioned into training (162,000 images), validation (20,000 images), and test (20,000 images) sets. The Imagenette dataset~\citep{imagenette}, a curated subset comprising 10 classes from the 1,000 classes in ImageNet~\citep{deng2009imagenet}, consists of roughly 9,000 training images and 4,000 testing images. In order to compare with previous methods, we conduct image regression experiments. The context set is an image and the task is to learn an implicit function that regresses the image pixels well in terms of PSNR.
% %\str{What is the point of image regresssion when the image itself is used as context? Why not just copy the context??}
% %\str{The following is a bit strante, is this still image regression? Why not have a separate section?}
% %\str{Is Image Regression the standard name for this task?}


% \noindent{\textbf{Implementation Details.}} 
% Following TransINR~\citep{chen2022transformers}, we resize each image into $178\times 178$, and use patch size 9 for the tokenizer. The self-attention module remains the same as the one in the NeRF experiments (Sec. \ref{sec:nerf-results}). For the Gaussian bases, we predict the 2D Gaussians instead of the 3D. 
% The hierarchical latent variables are inferred in image-level and pixel-level. 





% %The self-attention and global variable remain the same as the one in the NeRF experiments. %We do not use the pixel variable modulation for the computation concern. However, our method still has competitive performance.   

% % \begin{table}[t]
% % \centering
% % \vspace{-6mm}
% % \caption{Quantitative results of image regression.}
% % \label{tab:image-regression}
% % \begin{tabular}{lcc}
% % \toprule
% %              & CelebA & Imagenette \\ \midrule
% % Learned Init \citep{tancik2021learned} & 30.37  & 27.07       \\
% % TransINR~\citep{chen2022transformers}         & 31.96  & 29.01       \\ 
% % \hline
% % \method{} (Ours)         & \textbf{33.41}  & \textbf{29.82}      \\ 
% % \bottomrule
% % \end{tabular}
% % \end{table}



% \noindent{\textbf{Results.}} The quantitative comparison of \method{} for representing the 2D image signals is presented in Table~\ref{tab:image-regression}. \method{} outperforms the baseline methods on both CelebA and Imagenette datasets significantly, showing better generalization ability and representation capacity than baselines. 
% %Note that the Imagenette is a more diverse dataset than the CelebA. The better performance shows that. 
% Fig.~\ref{fig:visualization-image-regression} shows the ability of \method{} to recover the high-frequency details for image regression.
% %regress the image closely to the ground truth, indicating the capability of capturing the detailed texture information. 


% \noindent {\textbf{Image Completion Visualization.}} We also conduct experiments of \method{} on image completion (also called image inpainting), which is a more challenging variant of image regression. Essentially, only part of the pixels are given as context, while the INR functions are required to complete the full image. Visualizations in Fig.~\ref{fig:completion} demonstrate the generalization ability of our method to recover realistic images with fine details based on very limited context ($10 \% - 20\%$ pixels). %, 




% \begin{figure}[t!]
%   \centering
%   \includegraphics[width=0.99\textwidth]{Figures/image-completion.pdf} % Adjust the size and filename as needed
%   \vspace{-3mm}
%   \caption{\textbf{Image completion visualization} on CelebA using $10\%$ (left) and $20\%$ (right) context.}
%   \label{fig:completion}
%   \vspace{-3mm}
% \end{figure}



% \begin{figure}[t!]
%   \centering
%   \includegraphics[width=1\textwidth]{Figures/image-regression-basis.pdf} % Adjust the size and filename as needed
%   \caption{\textbf{Visualization of geometric bases (Gaussian)} on the context image, which reveals the structure of the object.} % Caption for the figure
%   \label{fig:visualization}
%   \vspace{-5mm}
% \end{figure}


% \noindent{\textbf{Visualization of Geometric Bases.}}
% Moreover, we also visualize the learned Gaussian bases on the image regression task. As shown in Fig. \ref{fig:visualization}, the bases are more concentrated on the objects and complex backgrounds in the image, while sparse on the simple complex. The visualizations indicate that the geometric bases do encode structure information from the context data.




