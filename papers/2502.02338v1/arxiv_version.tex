%%%%%%%% ICML 2025 EXAMPLE LATEX SUBMISSION FILE %%%%%%%%%%%%%%%%%

\documentclass{article}

% Recommended, but optional, packages for figures and better typesetting:
\usepackage{microtype}
\usepackage{graphicx}
\usepackage{subfigure}
\usepackage{booktabs} % for professional tables

% hyperref makes hyperlinks in the resulting PDF.
% If your build breaks (sometimes temporarily if a hyperlink spans a page)
% please comment out the following usepackage line and replace
% \usepackage{icml2025} with \usepackage[nohyperref]{icml2025} above.
\usepackage{hyperref}


% Attempt to make hyperref and algorithmic work together better:
\newcommand{\theHalgorithm}{\arabic{algorithm}}

% Use the following line for the initial blind version submitted for review:
%\usepackage{icml2025}

% If accepted, instead use the following line for the camera-ready submission:
\usepackage[accepted]{icml2025}

% For theorems and such
\usepackage{amsmath}
\usepackage{amssymb}
\usepackage{mathtools}
\usepackage{amsthm}

% if you use cleveref..
\usepackage[capitalize,noabbrev]{cleveref}
\usepackage{microtype}      % microtypography
\usepackage{color}
\usepackage{algorithm} % For algorithm environment
\usepackage{algorithmic} % For algorithmic environment
\usepackage{minitoc}     % For separate table of contents in appendix


\usepackage{xcolor}      % For color customization
%\usepackage{tocloft}
\usepackage{minitoc}
\usepackage{appendix}

\usepackage{graphicx} 
\usepackage{amsmath} 
\usepackage{float}
\usepackage{wrapfig}
\usepackage{lipsum} % for dummy text
\usepackage{subcaption}
\usepackage{multirow}
\usepackage{derivative}
\usepackage{latexsym}
\usepackage{pifont}
\usepackage{booktabs}
\usepackage{subcaption}
%\usepackage{algpseudocode} % For pseudocode

%%%%%%%%%%%%%%%%%%%%%%%%%%%%%%%%
% THEOREMS
%%%%%%%%%%%%%%%%%%%%%%%%%%%%%%%%
\theoremstyle{plain}
\newtheorem{theorem}{Theorem}[section]
\newtheorem{proposition}[theorem]{Proposition}
\newtheorem{lemma}[theorem]{Lemma}
\newtheorem{corollary}[theorem]{Corollary}
\theoremstyle{definition}
\newtheorem{definition}[theorem]{Definition}
\newtheorem{assumption}[theorem]{Assumption}
\theoremstyle{remark}
\newtheorem{remark}[theorem]{Remark}

\newcommand{\method}{{\textit{GeomNP}}}  % Define your new command here
\usepackage{colortbl} % colored cells

\newcommand{\name}{{\emph{G}-NPF}}

\crefname{equation}{Eq.}{Eqs.} % For singular and plural forms

\definecolor{lightblue}{rgb}{0.81, 0.94, 1.0}
\definecolor{cvprblue}{rgb}{0.21,0.49,0.74}
\usepackage{hyperref}
\hypersetup{
    colorlinks=true,
    linkcolor=red,
    citecolor=cvprblue,      
    urlcolor=cyan
    }

%\usepackage{tocloft}
\usepackage{xcolor}

% \newcommand{\INPUT}{\textbf{Input: }}
% \newcommand{\OUTPUT}{\textbf{Output: }}

% Todonotes is useful during development; simply uncomment the next line
%    and comment out the line below the next line to turn off comments
%\usepackage[disable,textsize=tiny]{todonotes}
\usepackage[textsize=tiny]{todonotes}
\usepackage{booktabs}
\usepackage{pifont}


\newcommand{\zx}[1]{\textcolor{blue}{[\textbf{Zehao}: #1]}}
\newcommand{\js}[1]{\textcolor{orange}{[\textbf{Jiayi}: #1]}}
\newcommand{\wy}[1]{\textcolor{red}{[\textbf{WY}: #1]}}
\newcommand{\str}[1]{\textcolor{red}{[\textbf{STR}: #1]}}
\newcommand{\yc}[1]{\textcolor{magenta}{[\textbf{yunlu}: #1]}}
\newcommand{\addref}{\textcolor{red}{\textbf{[REF]}}}
\newcommand{\ymark}{\ding{51}}%
\newcommand{\cmark}{\ding{51}} % Checkmark
\newcommand{\xmark}{\ding{55}} % Crossmark

\newcommand{\RETURN}[1]{\textbf{return} #1}


% The \icmltitle you define below is probably too long as a header.
% Therefore, a short form for the running title is supplied here:
\icmltitlerunning{Geometric Neural Process Fields}

\begin{document}



\twocolumn[
\icmltitle{Geometric Neural Process Fields}

% It is OKAY to include author information, even for blind
% submissions: the style file will automatically remove it for you
% unless you've provided the [accepted] option to the icml2025
% package.

% List of affiliations: The first argument should be a (short)
% identifier you will use later to specify author affiliations
% Academic affiliations should list Department, University, City, Region, Country
% Industry affiliations should list Company, City, Region, Country

% You can specify symbols, otherwise they are numbered in order.
% Ideally, you should not use this facility. Affiliations will be numbered
% in order of appearance and this is the preferred way.
%\icmlsetsymbol{equal}{*}

\begin{icmlauthorlist}
\icmlauthor{Wenzhe Yin}{uva}
\icmlauthor{Zehao Xiao}{uva}
\icmlauthor{Jiayi Shen}{uva}
\icmlauthor{Yunlu Chen}{cmu}
\icmlauthor{Cees G. M. Snoek}{uva}
\icmlauthor{Jan-Jakob Sonke}{nki}
\icmlauthor{Efstratios Gavves}{uva}
%\icmlauthor{}{sch}
% \icmlauthor{Firstname8 Lastname8}{sch}
% \icmlauthor{Firstname8 Lastname8}{yyy,comp}
%\icmlauthor{}{sch}
%\icmlauthor{}{sch}
\end{icmlauthorlist}

\icmlaffiliation{uva}{University of Amsterdam}
\icmlaffiliation{nki}{The Netherlands Cancer Institute}
\icmlaffiliation{cmu}{Carnegie Mellon University}

\icmlcorrespondingauthor{Jiayi Shen}{j.shen@uva.nl}
% \icmlcorrespondingauthor{Firstname2 Lastname2}{first2.last2@www.uk}

% You may provide any keywords that you
% find helpful for describing your paper; these are used to populate
% the "keywords" metadata in the PDF but will not be shown in the document

%\icmlkeywords{Machine Learning, ICML}

\vskip 0.3in
]

% this must go after the closing bracket ] following \twocolumn[ ...

% This command actually creates the footnote in the first column
% listing the affiliations and the copyright notice.
% The command takes one argument, which is text to display at the start of the footnote.
% The \icmlEqualContribution command is standard text for equal contribution.
% Remove it (just {}) if you do not need this facility.

\printAffiliationsAndNotice{}  % leave blank if no need to mention equal contribution

%\printAffiliationsAndNotice{\icmlEqualContribution} % otherwise use the standard text.


\begin{abstract}
%Implicit Neural Representations (INRs) have been widely used to represent 3D scenes due to their continuous function properties. However, one of the major drawbacks is that a separate neural network must be trained from scratch for each signal. 
This paper addresses the challenge of Neural Field (NeF) generalization, where models must efficiently adapt to new signals given only a few observations. To tackle this, we propose Geometric Neural Process Fields (\name{}), a probabilistic framework for neural radiance fields that explicitly captures uncertainty. We formulate NeF generalization as a probabilistic problem, enabling direct inference of NeF function distributions from limited context observations.
To incorporate structural inductive biases, we introduce a set of geometric bases that encode spatial structure and facilitate the inference of NeF function distributions. Building on these bases, we design a hierarchical latent variable model, allowing \name{} to integrate structural information across multiple spatial levels and effectively parameterize INR functions. This hierarchical approach improves generalization to novel scenes and unseen signals.
Experiments on novel-view synthesis for 3D scenes, as well as 2D image and 1D signal regression, demonstrate the effectiveness of our method in capturing uncertainty and leveraging structural information for improved generalization.

% Specifically, in Neural Radiance Field (NeRF) generalization, there exists an information gap between the context set (continuous 2D RGB and rays) and the target set (3D discrete points), which is commonly ignored by the previous NeRF generalization methods. 
% To this end, we introduce a set of posterior Gaussian bases based on the context set to provide the structure geometry information of the scene for each spatial location. Furthermore, we propose a hierarchical neural process modeling to modulate the INR function globally and locally. The experimental results on both the ShapNet novel views synthesis task and the 2D image regression task show state-of-the-art performance. 
\end{abstract}


\section{Introduction}
\section{Introduction}

In today’s rapidly evolving digital landscape, the transformative power of web technologies has redefined not only how services are delivered but also how complex tasks are approached. Web-based systems have become increasingly prevalent in risk control across various domains. This widespread adoption is due their accessibility, scalability, and ability to remotely connect various types of users. For example, these systems are used for process safety management in industry~\cite{kannan2016web}, safety risk early warning in urban construction~\cite{ding2013development}, and safe monitoring of infrastructural systems~\cite{repetto2018web}. Within these web-based risk management systems, the source search problem presents a huge challenge. Source search refers to the task of identifying the origin of a risky event, such as a gas leak and the emission point of toxic substances. This source search capability is crucial for effective risk management and decision-making.

Traditional approaches to implementing source search capabilities into the web systems often rely on solely algorithmic solutions~\cite{ristic2016study}. These methods, while relatively straightforward to implement, often struggle to achieve acceptable performances due to algorithmic local optima and complex unknown environments~\cite{zhao2020searching}. More recently, web crowdsourcing has emerged as a promising alternative for tackling the source search problem by incorporating human efforts in these web systems on-the-fly~\cite{zhao2024user}. This approach outsources the task of addressing issues encountered during the source search process to human workers, leveraging their capabilities to enhance system performance.

These solutions often employ a human-AI collaborative way~\cite{zhao2023leveraging} where algorithms handle exploration-exploitation and report the encountered problems while human workers resolve complex decision-making bottlenecks to help the algorithms getting rid of local deadlocks~\cite{zhao2022crowd}. Although effective, this paradigm suffers from two inherent limitations: increased operational costs from continuous human intervention, and slow response times of human workers due to sequential decision-making. These challenges motivate our investigation into developing autonomous systems that preserve human-like reasoning capabilities while reducing dependency on massive crowdsourced labor.

Furthermore, recent advancements in large language models (LLMs)~\cite{chang2024survey} and multi-modal LLMs (MLLMs)~\cite{huang2023chatgpt} have unveiled promising avenues for addressing these challenges. One clear opportunity involves the seamless integration of visual understanding and linguistic reasoning for robust decision-making in search tasks. However, whether large models-assisted source search is really effective and efficient for improving the current source search algorithms~\cite{ji2022source} remains unknown. \textit{To address the research gap, we are particularly interested in answering the following two research questions in this work:}

\textbf{\textit{RQ1: }}How can source search capabilities be integrated into web-based systems to support decision-making in time-sensitive risk management scenarios? 
% \sq{I mention ``time-sensitive'' here because I feel like we shall say something about the response time -- LLM has to be faster than humans}

\textbf{\textit{RQ2: }}How can MLLMs and LLMs enhance the effectiveness and efficiency of existing source search algorithms? 

% \textit{\textbf{RQ2:}} To what extent does the performance of large models-assisted search align with or approach the effectiveness of human-AI collaborative search? 

To answer the research questions, we propose a novel framework called Auto-\
S$^2$earch (\textbf{Auto}nomous \textbf{S}ource \textbf{Search}) and implement a prototype system that leverages advanced web technologies to simulate real-world conditions for zero-shot source search. Unlike traditional methods that rely on pre-defined heuristics or extensive human intervention, AutoS$^2$earch employs a carefully designed prompt that encapsulates human rationales, thereby guiding the MLLM to generate coherent and accurate scene descriptions from visual inputs about four directional choices. Based on these language-based descriptions, the LLM is enabled to determine the optimal directional choice through chain-of-thought (CoT) reasoning. Comprehensive empirical validation demonstrates that AutoS$^2$-\ 
earch achieves a success rate of 95–98\%, closely approaching the performance of human-AI collaborative search across 20 benchmark scenarios~\cite{zhao2023leveraging}. 

Our work indicates that the role of humans in future web crowdsourcing tasks may evolve from executors to validators or supervisors. Furthermore, incorporating explanations of LLM decisions into web-based system interfaces has the potential to help humans enhance task performance in risk control.







\section{Background}
\section{Background}\label{sec:backgrnd}

\subsection{Cold Start Latency and Mitigation Techniques}

Traditional FaaS platforms mitigate cold starts through snapshotting, lightweight virtualization, and warm-state management. Snapshot-based methods like \textbf{REAP} and \textbf{Catalyzer} reduce initialization time by preloading or restoring container states but require significant memory and I/O resources, limiting scalability~\cite{dong_catalyzer_2020, ustiugov_benchmarking_2021}. Lightweight virtualization solutions, such as \textbf{Firecracker} microVMs, achieve fast startup times with strong isolation but depend on robust infrastructure, making them less adaptable to fluctuating workloads~\cite{agache_firecracker_2020}. Warm-state management techniques like \textbf{Faa\$T}~\cite{romero_faa_2021} and \textbf{Kraken}~\cite{vivek_kraken_2021} keep frequently invoked containers ready, balancing readiness and cost efficiency under predictable workloads but incurring overhead when demand is erratic~\cite{romero_faa_2021, vivek_kraken_2021}. While these methods perform well in resource-rich cloud environments, their resource intensity challenges applicability in edge settings.

\subsubsection{Edge FaaS Perspective}

In edge environments, cold start mitigation emphasizes lightweight designs, resource sharing, and hybrid task distribution. Lightweight execution environments like unikernels~\cite{edward_sock_2018} and \textbf{Firecracker}~\cite{agache_firecracker_2020}, as used by \textbf{TinyFaaS}~\cite{pfandzelter_tinyfaas_2020}, minimize resource usage and initialization delays but require careful orchestration to avoid resource contention. Function co-location, demonstrated by \textbf{Photons}~\cite{v_dukic_photons_2020}, reduces redundant initializations by sharing runtime resources among related functions, though this complicates isolation in multi-tenant setups~\cite{v_dukic_photons_2020}. Hybrid offloading frameworks like \textbf{GeoFaaS}~\cite{malekabbasi_geofaas_2024} balance edge-cloud workloads by offloading latency-tolerant tasks to the cloud and reserving edge resources for real-time operations, requiring reliable connectivity and efficient task management. These edge-specific strategies address cold starts effectively but introduce challenges in scalability and orchestration.

\subsection{Predictive Scaling and Caching Techniques}

Efficient resource allocation is vital for maintaining low latency and high availability in serverless platforms. Predictive scaling and caching techniques dynamically provision resources and reduce cold start latency by leveraging workload prediction and state retention.
Traditional FaaS platforms use predictive scaling and caching to optimize resources, employing techniques (OFC, FaasCache) to reduce cold starts. However, these methods rely on centralized orchestration and workload predictability, limiting their effectiveness in dynamic, resource-constrained edge environments.



\subsubsection{Edge FaaS Perspective}

Edge FaaS platforms adapt predictive scaling and caching techniques to constrain resources and heterogeneous environments. \textbf{EDGE-Cache}~\cite{kim_delay-aware_2022} uses traffic profiling to selectively retain high-priority functions, reducing memory overhead while maintaining readiness for frequent requests. Hybrid frameworks like \textbf{GeoFaaS}~\cite{malekabbasi_geofaas_2024} implement distributed caching to balance resources between edge and cloud nodes, enabling low-latency processing for critical tasks while offloading less critical workloads. Machine learning methods, such as clustering-based workload predictors~\cite{gao_machine_2020} and GRU-based models~\cite{guo_applying_2018}, enhance resource provisioning in edge systems by efficiently forecasting workload spikes. These innovations effectively address cold start challenges in edge environments, though their dependency on accurate predictions and robust orchestration poses scalability challenges.

\subsection{Decentralized Orchestration, Function Placement, and Scheduling}

Efficient orchestration in serverless platforms involves workload distribution, resource optimization, and performance assurance. While traditional FaaS platforms rely on centralized control, edge environments require decentralized and adaptive strategies to address unique challenges such as resource constraints and heterogeneous hardware.



\subsubsection{Edge FaaS Perspective}

Edge FaaS platforms adopt decentralized and adaptive orchestration frameworks to meet the demands of resource-constrained environments. Systems like \textbf{Wukong} distribute scheduling across edge nodes, enhancing data locality and scalability while reducing network latency. Lightweight frameworks such as \textbf{OpenWhisk Lite}~\cite{kravchenko_kpavelopenwhisk-light_2024} optimize resource allocation by decentralizing scheduling policies, minimizing cold starts and latency in edge setups~\cite{benjamin_wukong_2020}. Hybrid solutions like \textbf{OpenFaaS}~\cite{noauthor_openfaasfaas_2024} and \textbf{EdgeMatrix}~\cite{shen_edgematrix_2023} combine edge-cloud orchestration to balance resource utilization, retaining latency-sensitive functions at the edge while offloading non-critical workloads to the cloud. While these approaches improve flexibility, they face challenges in maintaining coordination and ensuring consistent performance across distributed nodes.



\section{Geometric Neural Process Fields}
\section{Method}\label{sec:method}
\begin{figure}
    \centering
    \includegraphics[width=0.85\textwidth]{imgs/heatmap_acc.pdf}
    \caption{\textbf{Visualization of the proposed periodic Bayesian flow with mean parameter $\mu$ and accumulated accuracy parameter $c$ which corresponds to the entropy/uncertainty}. For $x = 0.3, \beta(1) = 1000$ and $\alpha_i$ defined in \cref{appd:bfn_cir}, this figure plots three colored stochastic parameter trajectories for receiver mean parameter $m$ and accumulated accuracy parameter $c$, superimposed on a log-scale heatmap of the Bayesian flow distribution $p_F(m|x,\senderacc)$ and $p_F(c|x,\senderacc)$. Note the \emph{non-monotonicity} and \emph{non-additive} property of $c$ which could inform the network the entropy of the mean parameter $m$ as a condition and the \emph{periodicity} of $m$. %\jj{Shrink the figures to save space}\hanlin{Do we need to make this figure one-column?}
    }
    \label{fig:vmbf_vis}
    \vskip -0.1in
\end{figure}
% \begin{wrapfigure}{r}{0.5\textwidth}
%     \centering
%     \includegraphics[width=0.49\textwidth]{imgs/heatmap_acc.pdf}
%     \caption{\textbf{Visualization of hyper-torus Bayesian flow based on von Mises Distribution}. For $x = 0.3, \beta(1) = 1000$ and $\alpha_i$ defined in \cref{appd:bfn_cir}, this figure plots three colored stochastic parameter trajectories for receiver mean parameter $m$ and accumulated accuracy parameter $c$, superimposed on a log-scale heatmap of the Bayesian flow distribution $p_F(m|x,\senderacc)$ and $p_F(c|x,\senderacc)$. Note the \emph{non-monotonicity} and \emph{non-additive} property of $c$. \jj{Shrink the figures to save space}}
%     \label{fig:vmbf_vis}
%     \vspace{-30pt}
% \end{wrapfigure}


In this section, we explain the detailed design of CrysBFN tackling theoretical and practical challenges. First, we describe how to derive our new formulation of Bayesian Flow Networks over hyper-torus $\mathbb{T}^{D}$ from scratch. Next, we illustrate the two key differences between \modelname and the original form of BFN: $1)$ a meticulously designed novel base distribution with different Bayesian update rules; and $2)$ different properties over the accuracy scheduling resulted from the periodicity and the new Bayesian update rules. Then, we present in detail the overall framework of \modelname over each manifold of the crystal space (\textit{i.e.} fractional coordinates, lattice vectors, atom types) respecting \textit{periodic E(3) invariance}. 

% In this section, we first demonstrate how to build Bayesian flow on hyper-torus $\mathbb{T}^{D}$ by overcoming theoretical and practical problems to provide a low-noise parameter-space approach to fractional atom coordinate generation. Next, we present how \modelname models each manifold of crystal space respecting \textit{periodic E(3) invariance}. 

\subsection{Periodic Bayesian Flow on Hyper-torus \texorpdfstring{$\mathbb{T}^{D}$}{}} 
For generative modeling of fractional coordinates in crystal, we first construct a periodic Bayesian flow on \texorpdfstring{$\mathbb{T}^{D}$}{} by designing every component of the totally new Bayesian update process which we demonstrate to be distinct from the original Bayesian flow (please see \cref{fig:non_add}). 
 %:) 
 
 The fractional atom coordinate system \citep{jiao2023crystal} inherently distributes over a hyper-torus support $\mathbb{T}^{3\times N}$. Hence, the normal distribution support on $\R$ used in the original \citep{bfn} is not suitable for this scenario. 
% The key problem of generative modeling for crystal is the periodicity of Cartesian atom coordinates $\vX$ requiring:
% \begin{equation}\label{eq:periodcity}
% p(\vA,\vL,\vX)=p(\vA,\vL,\vX+\vec{LK}),\text{where}~\vec{K}=\vec{k}\vec{1}_{1\times N},\forall\vec{k}\in\mathbb{Z}^{3\times1}
% \end{equation}
% However, there does not exist such a distribution supporting on $\R$ to model such property because the integration of such distribution over $\R$ will not be finite and equal to 1. Therefore, the normal distribution used in \citet{bfn} can not meet this condition.

To tackle this problem, the circular distribution~\citep{mardia2009directional} over the finite interval $[-\pi,\pi)$ is a natural choice as the base distribution for deriving the BFN on $\mathbb{T}^D$. 
% one natural choice is to 
% we would like to consider the circular distribution over the finite interval as the base 
% we find that circular distributions \citep{mardia2009directional} defined on a finite interval with lengths of $2\pi$ can be used as the instantiation of input distribution for the BFN on $\mathbb{T}^D$.
Specifically, circular distributions enjoy desirable periodic properties: $1)$ the integration over any interval length of $2\pi$ equals 1; $2)$ the probability distribution function is periodic with period $2\pi$.  Sharing the same intrinsic with fractional coordinates, such periodic property of circular distribution makes it suitable for the instantiation of BFN's input distribution, in parameterizing the belief towards ground truth $\x$ on $\mathbb{T}^D$. 
% \yuxuan{this is very complicated from my perspective.} \hanlin{But this property is exactly beautiful and perfectly fit into the BFN.}

\textbf{von Mises Distribution and its Bayesian Update} We choose von Mises distribution \citep{mardia2009directional} from various circular distributions as the form of input distribution, based on the appealing conjugacy property required in the derivation of the BFN framework.
% to leverage the Bayesian conjugacy property of von Mises distribution which is required by the BFN framework. 
That is, the posterior of a von Mises distribution parameterized likelihood is still in the family of von Mises distributions. The probability density function of von Mises distribution with mean direction parameter $m$ and concentration parameter $c$ (describing the entropy/uncertainty of $m$) is defined as: 
\begin{equation}
f(x|m,c)=vM(x|m,c)=\frac{\exp(c\cos(x-m))}{2\pi I_0(c)}
\end{equation}
where $I_0(c)$ is zeroth order modified Bessel function of the first kind as the normalizing constant. Given the last univariate belief parameterized by von Mises distribution with parameter $\theta_{i-1}=\{m_{i-1},\ c_{i-1}\}$ and the sample $y$ from sender distribution with unknown data sample $x$ and known accuracy $\alpha$ describing the entropy/uncertainty of $y$,  Bayesian update for the receiver is deducted as:
\begin{equation}
 h(\{m_{i-1},c_{i-1}\},y,\alpha)=\{m_i,c_i \}, \text{where}
\end{equation}
\begin{equation}\label{eq:h_m}
m_i=\text{atan2}(\alpha\sin y+c_{i-1}\sin m_{i-1}, {\alpha\cos y+c_{i-1}\cos m_{i-1}})
\end{equation}
\begin{equation}\label{eq:h_c}
c_i =\sqrt{\alpha^2+c_{i-1}^2+2\alpha c_{i-1}\cos(y-m_{i-1})}
\end{equation}
The proof of the above equations can be found in \cref{apdx:bayesian_update_function}. The atan2 function refers to  2-argument arctangent. Independently conducting  Bayesian update for each dimension, we can obtain the Bayesian update distribution by marginalizing $\y$:
\begin{equation}
p_U(\vtheta'|\vtheta,\bold{x};\alpha)=\mathbb{E}_{p_S(\bold{y}|\bold{x};\alpha)}\delta(\vtheta'-h(\vtheta,\bold{y},\alpha))=\mathbb{E}_{vM(\bold{y}|\bold{x},\alpha)}\delta(\vtheta'-h(\vtheta,\bold{y},\alpha))
\end{equation} 
\begin{figure}
    \centering
    \vskip -0.15in
    \includegraphics[width=0.95\linewidth]{imgs/non_add.pdf}
    \caption{An intuitive illustration of non-additive accuracy Bayesian update on the torus. The lengths of arrows represent the uncertainty/entropy of the belief (\emph{e.g.}~$1/\sigma^2$ for Gaussian and $c$ for von Mises). The directions of the arrows represent the believed location (\emph{e.g.}~ $\mu$ for Gaussian and $m$ for von Mises).}
    \label{fig:non_add}
    \vskip -0.15in
\end{figure}
\textbf{Non-additive Accuracy} 
The additive accuracy is a nice property held with the Gaussian-formed sender distribution of the original BFN expressed as:
\begin{align}
\label{eq:standard_id}
    \update(\parsn{}'' \mid \parsn{}, \x; \alpha_a+\alpha_b) = \E_{\update(\parsn{}' \mid \parsn{}, \x; \alpha_a)} \update(\parsn{}'' \mid \parsn{}', \x; \alpha_b)
\end{align}
Such property is mainly derived based on the standard identity of Gaussian variable:
\begin{equation}
X \sim \mathcal{N}\left(\mu_X, \sigma_X^2\right), Y \sim \mathcal{N}\left(\mu_Y, \sigma_Y^2\right) \Longrightarrow X+Y \sim \mathcal{N}\left(\mu_X+\mu_Y, \sigma_X^2+\sigma_Y^2\right)
\end{equation}
The additive accuracy property makes it feasible to derive the Bayesian flow distribution $
p_F(\boldsymbol{\theta} \mid \mathbf{x} ; i)=p_U\left(\boldsymbol{\theta} \mid \boldsymbol{\theta}_0, \mathbf{x}, \sum_{k=1}^{i} \alpha_i \right)
$ for the simulation-free training of \cref{eq:loss_n}.
It should be noted that the standard identity in \cref{eq:standard_id} does not hold in the von Mises distribution. Hence there exists an important difference between the original Bayesian flow defined on Euclidean space and the Bayesian flow of circular data on $\mathbb{T}^D$ based on von Mises distribution. With prior $\btheta = \{\bold{0},\bold{0}\}$, we could formally represent the non-additive accuracy issue as:
% The additive accuracy property implies the fact that the "confidence" for the data sample after observing a series of the noisy samples with accuracy ${\alpha_1, \cdots, \alpha_i}$ could be  as the accuracy sum  which could be  
% Here we 
% Here we emphasize the specific property of BFN based on von Mises distribution.
% Note that 
% \begin{equation}
% \update(\parsn'' \mid \parsn, \x; \alpha_a+\alpha_b) \ne \E_{\update(\parsn' \mid \parsn, \x; \alpha_a)} \update(\parsn'' \mid \parsn', \x; \alpha_b)
% \end{equation}
% \oyyw{please check whether the below equation is better}
% \yuxuan{I fill somehow confusing on what is the update distribution with $\alpha$. }
% \begin{equation}
% \update(\parsn{}'' \mid \parsn{}, \x; \alpha_a+\alpha_b) \ne \E_{\update(\parsn{}' \mid \parsn{}, \x; \alpha_a)} \update(\parsn{}'' \mid \parsn{}', \x; \alpha_b)
% \end{equation}
% We give an intuitive visualization of such difference in \cref{fig:non_add}. The untenability of this property can materialize by considering the following case: with prior $\btheta = \{\bold{0},\bold{0}\}$, check the two-step Bayesian update distribution with $\alpha_a,\alpha_b$ and one-step Bayesian update with $\alpha=\alpha_a+\alpha_b$:
\begin{align}
\label{eq:nonadd}
     &\update(c'' \mid \parsn, \x; \alpha_a+\alpha_b)  = \delta(c-\alpha_a-\alpha_b)
     \ne  \mathbb{E}_{p_U(\parsn' \mid \parsn, \x; \alpha_a)}\update(c'' \mid \parsn', \x; \alpha_b) \nonumber \\&= \mathbb{E}_{vM(\bold{y}_b|\bold{x},\alpha_a)}\mathbb{E}_{vM(\bold{y}_a|\bold{x},\alpha_b)}\delta(c-||[\alpha_a \cos\y_a+\alpha_b\cos \y_b,\alpha_a \sin\y_a+\alpha_b\sin \y_b]^T||_2)
\end{align}
A more intuitive visualization could be found in \cref{fig:non_add}. This fundamental difference between periodic Bayesian flow and that of \citet{bfn} presents both theoretical and practical challenges, which we will explain and address in the following contents.

% This makes constructing Bayesian flow based on von Mises distribution intrinsically different from previous Bayesian flows (\citet{bfn}).

% Thus, we must reformulate the framework of Bayesian flow networks  accordingly. % and do necessary reformulations of BFN. 

% \yuxuan{overall I feel this part is complicated by using the language of update distribution. I would like to suggest simply use bayesian update, to provide intuitive explantion.}\hanlin{See the illustration in \cref{fig:non_add}}

% That introduces a cascade of problems, and we investigate the following issues: $(1)$ Accuracies between sender and receiver are not synchronized and need to be differentiated. $(2)$ There is no tractable Bayesian flow distribution for a one-step sample conditioned on a given time step $i$, and naively simulating the Bayesian flow results in computational overhead. $(3)$ It is difficult to control the entropy of the Bayesian flow. $(4)$ Accuracy is no longer a function of $t$ and becomes a distribution conditioned on $t$, which can be different across dimensions.
%\jj{Edited till here}

\textbf{Entropy Conditioning} As a common practice in generative models~\citep{ddpm,flowmatching,bfn}, timestep $t$ is widely used to distinguish among generation states by feeding the timestep information into the networks. However, this paper shows that for periodic Bayesian flow, the accumulated accuracy $\vc_i$ is more effective than time-based conditioning by informing the network about the entropy and certainty of the states $\parsnt{i}$. This stems from the intrinsic non-additive accuracy which makes the receiver's accumulated accuracy $c$ not bijective function of $t$, but a distribution conditioned on accumulated accuracies $\vc_i$ instead. Therefore, the entropy parameter $\vc$ is taken logarithm and fed into the network to describe the entropy of the input corrupted structure. We verify this consideration in \cref{sec:exp_ablation}. 
% \yuxuan{implement variant. traditionally, the timestep is widely used to distinguish the different states by putting the timestep embedding into the networks. citation of FM, diffusion, BFN. However, we find that conditioned on time in periodic flow could not provide extra benefits. To further boost the performance, we introduce a simple yet effective modification term entropy conditional. This is based on that the accumulated accuracy which represents the current uncertainty or entropy could be a better indicator to distinguish different states. + Describe how you do this. }



\textbf{Reformulations of BFN}. Recall the original update function with Gaussian sender distribution, after receiving noisy samples $\y_1,\y_2,\dots,\y_i$ with accuracies $\senderacc$, the accumulated accuracies of the receiver side could be analytically obtained by the additive property and it is consistent with the sender side.
% Since observing sample $\y$ with $\alpha_i$ can not result in exact accuracy increment $\alpha_i$ for receiver, the accuracies between sender and receiver are not synchronized which need to be differentiated. 
However, as previously mentioned, this does not apply to periodic Bayesian flow, and some of the notations in original BFN~\citep{bfn} need to be adjusted accordingly. We maintain the notations of sender side's one-step accuracy $\alpha$ and added accuracy $\beta$, and alter the notation of receiver's accuracy parameter as $c$, which is needed to be simulated by cascade of Bayesian updates. We emphasize that the receiver's accumulated accuracy $c$ is no longer a function of $t$ (differently from the Gaussian case), and it becomes a distribution conditioned on received accuracies $\senderacc$ from the sender. Therefore, we represent the Bayesian flow distribution of von Mises distribution as $p_F(\btheta|\x;\alpha_1,\alpha_2,\dots,\alpha_i)$. And the original simulation-free training with Bayesian flow distribution is no longer applicable in this scenario.
% Different from previous BFNs where the accumulated accuracy $\rho$ is not explicitly modeled, the accumulated accuracy parameter $c$ (visualized in \cref{fig:vmbf_vis}) needs to be explicitly modeled by feeding it to the network to avoid information loss.
% the randomaccuracy parameter $c$ (visualized in \cref{fig:vmbf_vis}) implies that there exists information in $c$ from the sender just like $m$, meaning that $c$ also should be fed into the network to avoid information loss. 
% We ablate this consideration in  \cref{sec:exp_ablation}. 

\textbf{Fast Sampling from Equivalent Bayesian Flow Distribution} Based on the above reformulations, the Bayesian flow distribution of von Mises distribution is reframed as: 
\begin{equation}\label{eq:flow_frac}
p_F(\btheta_i|\x;\alpha_1,\alpha_2,\dots,\alpha_i)=\E_{\update(\parsnt{1} \mid \parsnt{0}, \x ; \alphat{1})}\dots\E_{\update(\parsn_{i-1} \mid \parsnt{i-2}, \x; \alphat{i-1})} \update(\parsnt{i} | \parsnt{i-1},\x;\alphat{i} )
\end{equation}
Naively sampling from \cref{eq:flow_frac} requires slow auto-regressive iterated simulation, making training unaffordable. Noticing the mathematical properties of \cref{eq:h_m,eq:h_c}, we  transform \cref{eq:flow_frac} to the equivalent form:
\begin{equation}\label{eq:cirflow_equiv}
p_F(\vec{m}_i|\x;\alpha_1,\alpha_2,\dots,\alpha_i)=\E_{vM(\y_1|\x,\alpha_1)\dots vM(\y_i|\x,\alpha_i)} \delta(\vec{m}_i-\text{atan2}(\sum_{j=1}^i \alpha_j \cos \y_j,\sum_{j=1}^i \alpha_j \sin \y_j))
\end{equation}
\begin{equation}\label{eq:cirflow_equiv2}
p_F(\vec{c}_i|\x;\alpha_1,\alpha_2,\dots,\alpha_i)=\E_{vM(\y_1|\x,\alpha_1)\dots vM(\y_i|\x,\alpha_i)}  \delta(\vec{c}_i-||[\sum_{j=1}^i \alpha_j \cos \y_j,\sum_{j=1}^i \alpha_j \sin \y_j]^T||_2)
\end{equation}
which bypasses the computation of intermediate variables and allows pure tensor operations, with negligible computational overhead.
\begin{restatable}{proposition}{cirflowequiv}
The probability density function of Bayesian flow distribution defined by \cref{eq:cirflow_equiv,eq:cirflow_equiv2} is equivalent to the original definition in \cref{eq:flow_frac}. 
\end{restatable}
\textbf{Numerical Determination of Linear Entropy Sender Accuracy Schedule} ~Original BFN designs the accuracy schedule $\beta(t)$ to make the entropy of input distribution linearly decrease. As for crystal generation task, to ensure information coherence between modalities, we choose a sender accuracy schedule $\senderacc$ that makes the receiver's belief entropy $H(t_i)=H(p_I(\cdot|\vtheta_i))=H(p_I(\cdot|\vc_i))$ linearly decrease \emph{w.r.t.} time $t_i$, given the initial and final accuracy parameter $c(0)$ and $c(1)$. Due to the intractability of \cref{eq:vm_entropy}, we first use numerical binary search in $[0,c(1)]$ to determine the receiver's $c(t_i)$ for $i=1,\dots, n$ by solving the equation $H(c(t_i))=(1-t_i)H(c(0))+tH(c(1))$. Next, with $c(t_i)$, we conduct numerical binary search for each $\alpha_i$ in $[0,c(1)]$ by solving the equations $\E_{y\sim vM(x,\alpha_i)}[\sqrt{\alpha_i^2+c_{i-1}^2+2\alpha_i c_{i-1}\cos(y-m_{i-1})}]=c(t_i)$ from $i=1$ to $i=n$ for arbitrarily selected $x\in[-\pi,\pi)$.

After tackling all those issues, we have now arrived at a new BFN architecture for effectively modeling crystals. Such BFN can also be adapted to other type of data located in hyper-torus $\mathbb{T}^{D}$.

\subsection{Equivariant Bayesian Flow for Crystal}
With the above Bayesian flow designed for generative modeling of fractional coordinate $\vF$, we are able to build equivariant Bayesian flow for each modality of crystal. In this section, we first give an overview of the general training and sampling algorithm of \modelname (visualized in \cref{fig:framework}). Then, we describe the details of the Bayesian flow of every modality. The training and sampling algorithm can be found in \cref{alg:train} and \cref{alg:sampling}.

\textbf{Overview} Operating in the parameter space $\bthetaM=\{\bthetaA,\bthetaL,\bthetaF\}$, \modelname generates high-fidelity crystals through a joint BFN sampling process on the parameter of  atom type $\bthetaA$, lattice parameter $\vec{\theta}^L=\{\bmuL,\brhoL\}$, and the parameter of fractional coordinate matrix $\bthetaF=\{\bmF,\bcF\}$. We index the $n$-steps of the generation process in a discrete manner $i$, and denote the corresponding continuous notation $t_i=i/n$ from prior parameter $\thetaM_0$ to a considerably low variance parameter $\thetaM_n$ (\emph{i.e.} large $\vrho^L,\bmF$, and centered $\bthetaA$).

At training time, \modelname samples time $i\sim U\{1,n\}$ and $\bthetaM_{i-1}$ from the Bayesian flow distribution of each modality, serving as the input to the network. The network $\net$ outputs $\net(\parsnt{i-1}^\mathcal{M},t_{i-1})=\net(\parsnt{i-1}^A,\parsnt{i-1}^F,\parsnt{i-1}^L,t_{i-1})$ and conducts gradient descents on loss function \cref{eq:loss_n} for each modality. After proper training, the sender distribution $p_S$ can be approximated by the receiver distribution $p_R$. 

At inference time, from predefined $\thetaM_0$, we conduct transitions from $\thetaM_{i-1}$ to $\thetaM_{i}$ by: $(1)$ sampling $\y_i\sim p_R(\bold{y}|\thetaM_{i-1};t_i,\alpha_i)$ according to network prediction $\predM{i-1}$; and $(2)$ performing Bayesian update $h(\thetaM_{i-1},\y^\calM_{i-1},\alpha_i)$ for each dimension. 

% Alternatively, we complete this transition using the flow-back technique by sampling 
% $\thetaM_{i}$ from Bayesian flow distribution $\flow(\btheta^M_{i}|\predM{i-1};t_{i-1})$. 

% The training objective of $\net$ is to minimize the KL divergence between sender distribution and receiver distribution for every modality as defined in \cref{eq:loss_n} which is equivalent to optimizing the negative variational lower bound $\calL^{VLB}$ as discussed in \cref{sec:preliminaries}. 

%In the following part, we will present the Bayesian flow of each modality in detail.

\textbf{Bayesian Flow of Fractional Coordinate $\vF$}~The distribution of the prior parameter $\bthetaF_0$ is defined as:
\begin{equation}\label{eq:prior_frac}
    p(\bthetaF_0) \defeq \{vM(\vm_0^F|\vec{0}_{3\times N},\vec{0}_{3\times N}),\delta(\vc_0^F-\vec{0}_{3\times N})\} = \{U(\vec{0},\vec{1}),\delta(\vc_0^F-\vec{0}_{3\times N})\}
\end{equation}
Note that this prior distribution of $\vm_0^F$ is uniform over $[\vec{0},\vec{1})$, ensuring the periodic translation invariance property in \cref{De:pi}. The training objective is minimizing the KL divergence between sender and receiver distribution (deduction can be found in \cref{appd:cir_loss}): 
%\oyyw{replace $\vF$ with $\x$?} \hanlin{notations follow Preliminary?}
\begin{align}\label{loss_frac}
\calL_F = n \E_{i \sim \ui{n}, \flow(\parsn{}^F \mid \vF ; \senderacc)} \alpha_i\frac{I_1(\alpha_i)}{I_0(\alpha_i)}(1-\cos(\vF-\predF{i-1}))
\end{align}
where $I_0(x)$ and $I_1(x)$ are the zeroth and the first order of modified Bessel functions. The transition from $\bthetaF_{i-1}$ to $\bthetaF_{i}$ is the Bayesian update distribution based on network prediction:
\begin{equation}\label{eq:transi_frac}
    p(\btheta^F_{i}|\parsnt{i-1}^\calM)=\mathbb{E}_{vM(\bold{y}|\predF{i-1},\alpha_i)}\delta(\btheta^F_{i}-h(\btheta^F_{i-1},\bold{y},\alpha_i))
\end{equation}
\begin{restatable}{proposition}{fracinv}
With $\net_{F}$ as a periodic translation equivariant function namely $\net_F(\parsnt{}^A,w(\parsnt{}^F+\vt),\parsnt{}^L,t)=w(\net_F(\parsnt{}^A,\parsnt{}^F,\parsnt{}^L,t)+\vt), \forall\vt\in\R^3$, the marginal distribution of $p(\vF_n)$ defined by \cref{eq:prior_frac,eq:transi_frac} is periodic translation invariant. 
\end{restatable}
\textbf{Bayesian Flow of Lattice Parameter \texorpdfstring{$\boldsymbol{L}$}{}}   
Noting the lattice parameter $\bm{L}$ located in Euclidean space, we set prior as the parameter of a isotropic multivariate normal distribution $\btheta^L_0\defeq\{\vmu_0^L,\vrho_0^L\}=\{\bm{0}_{3\times3},\bm{1}_{3\times3}\}$
% \begin{equation}\label{eq:lattice_prior}
% \btheta^L_0\defeq\{\vmu_0^L,\vrho_0^L\}=\{\bm{0}_{3\times3},\bm{1}_{3\times3}\}
% \end{equation}
such that the prior distribution of the Markov process on $\vmu^L$ is the Dirac distribution $\delta(\vec{\mu_0}-\vec{0})$ and $\delta(\vec{\rho_0}-\vec{1})$, 
% \begin{equation}
%     p_I^L(\boldsymbol{L}|\btheta_0^L)=\mathcal{N}(\bm{L}|\bm{0},\bm{I})
% \end{equation}
which ensures O(3)-invariance of prior distribution of $\vL$. By Eq. 77 from \citet{bfn}, the Bayesian flow distribution of the lattice parameter $\bm{L}$ is: 
\begin{align}% =p_U(\bmuL|\btheta_0^L,\bm{L},\beta(t))
p_F^L(\bmuL|\bm{L};t) &=\mathcal{N}(\bmuL|\gamma(t)\bm{L},\gamma(t)(1-\gamma(t))\bm{I}) 
\end{align}
where $\gamma(t) = 1 - \sigma_1^{2t}$ and $\sigma_1$ is the predefined hyper-parameter controlling the variance of input distribution at $t=1$ under linear entropy accuracy schedule. The variance parameter $\vrho$ does not need to be modeled and fed to the network, since it is deterministic given the accuracy schedule. After sampling $\bmuL_i$ from $p_F^L$, the training objective is defined as minimizing KL divergence between sender and receiver distribution (based on Eq. 96 in \citet{bfn}):
\begin{align}
\mathcal{L}_{L} = \frac{n}{2}\left(1-\sigma_1^{2/n}\right)\E_{i \sim \ui{n}}\E_{\flow(\bmuL_{i-1} |\vL ; t_{i-1})}  \frac{\left\|\vL -\predL{i-1}\right\|^2}{\sigma_1^{2i/n}},\label{eq:lattice_loss}
\end{align}
where the prediction term $\predL{i-1}$ is the lattice parameter part of network output. After training, the generation process is defined as the Bayesian update distribution given network prediction:
\begin{equation}\label{eq:lattice_sampling}
    p(\bmuL_{i}|\parsnt{i-1}^\calM)=\update^L(\bmuL_{i}|\predL{i-1},\bmuL_{i-1};t_{i-1})
\end{equation}
    

% The final prediction of the lattice parameter is given by $\bmuL_n = \predL{n-1}$.
% \begin{equation}\label{eq:final_lattice}
%     \bmuL_n = \predL{n-1}
% \end{equation}

\begin{restatable}{proposition}{latticeinv}\label{prop:latticeinv}
With $\net_{L}$ as  O(3)-equivariant function namely $\net_L(\parsnt{}^A,\parsnt{}^F,\vQ\parsnt{}^L,t)=\vQ\net_L(\parsnt{}^A,\parsnt{}^F,\parsnt{}^L,t),\forall\vQ^T\vQ=\vI$, the marginal distribution of $p(\bmuL_n)$ defined by \cref{eq:lattice_sampling} is O(3)-invariant. 
\end{restatable}


\textbf{Bayesian Flow of Atom Types \texorpdfstring{$\boldsymbol{A}$}{}} 
Given that atom types are discrete random variables located in a simplex $\calS^K$, the prior parameter of $\boldsymbol{A}$ is the discrete uniform distribution over the vocabulary $\parsnt{0}^A \defeq \frac{1}{K}\vec{1}_{1\times N}$. 
% \begin{align}\label{eq:disc_input_prior}
% \parsnt{0}^A \defeq \frac{1}{K}\vec{1}_{1\times N}
% \end{align}
% \begin{align}
%     (\oh{j}{K})_k \defeq \delta_{j k}, \text{where }\oh{j}{K}\in \R^{K},\oh{\vA}{KD} \defeq \left(\oh{a_1}{K},\dots,\oh{a_N}{K}\right) \in \R^{K\times N}
% \end{align}
With the notation of the projection from the class index $j$ to the length $K$ one-hot vector $ (\oh{j}{K})_k \defeq \delta_{j k}, \text{where }\oh{j}{K}\in \R^{K},\oh{\vA}{KD} \defeq \left(\oh{a_1}{K},\dots,\oh{a_N}{K}\right) \in \R^{K\times N}$, the Bayesian flow distribution of atom types $\vA$ is derived in \citet{bfn}:
\begin{align}
\flow^{A}(\parsn^A \mid \vA; t) &= \E_{\N{\y \mid \beta^A(t)\left(K \oh{\vA}{K\times N} - \vec{1}_{K\times N}\right)}{\beta^A(t) K \vec{I}_{K\times N \times N}}} \delta\left(\parsn^A - \frac{e^{\y}\parsnt{0}^A}{\sum_{k=1}^K e^{\y_k}(\parsnt{0})_{k}^A}\right).
\end{align}
where $\beta^A(t)$ is the predefined accuracy schedule for atom types. Sampling $\btheta_i^A$ from $p_F^A$ as the training signal, the training objective is the $n$-step discrete-time loss for discrete variable \citep{bfn}: 
% \oyyw{can we simplify the next equation? Such as remove $K \times N, K \times N \times N$}
% \begin{align}
% &\calL_A = n\E_{i \sim U\{1,n\},\flow^A(\parsn^A \mid \vA ; t_{i-1}),\N{\y \mid \alphat{i}\left(K \oh{\vA}{KD} - \vec{1}_{K\times N}\right)}{\alphat{i} K \vec{I}_{K\times N \times N}}} \ln \N{\y \mid \alphat{i}\left(K \oh{\vA}{K\times N} - \vec{1}_{K\times N}\right)}{\alphat{i} K \vec{I}_{K\times N \times N}}\nonumber\\
% &\qquad\qquad\qquad-\sum_{d=1}^N \ln \left(\sum_{k=1}^K \out^{(d)}(k \mid \parsn^A; t_{i-1}) \N{\ydd{d} \mid \alphat{i}\left(K\oh{k}{K}- \vec{1}_{K\times N}\right)}{\alphat{i} K \vec{I}_{K\times N \times N}}\right)\label{discdisc_t_loss_exp}
% \end{align}
\begin{align}
&\calL_A = n\E_{i \sim U\{1,n\},\flow^A(\parsn^A \mid \vA ; t_{i-1}),\N{\y \mid \alphat{i}\left(K \oh{\vA}{KD} - \vec{1}\right)}{\alphat{i} K \vec{I}}} \ln \N{\y \mid \alphat{i}\left(K \oh{\vA}{K\times N} - \vec{1}\right)}{\alphat{i} K \vec{I}}\nonumber\\
&\qquad\qquad\qquad-\sum_{d=1}^N \ln \left(\sum_{k=1}^K \out^{(d)}(k \mid \parsn^A; t_{i-1}) \N{\ydd{d} \mid \alphat{i}\left(K\oh{k}{K}- \vec{1}\right)}{\alphat{i} K \vec{I}}\right)\label{discdisc_t_loss_exp}
\end{align}
where $\vec{I}\in \R^{K\times N \times N}$ and $\vec{1}\in\R^{K\times D}$. When sampling, the transition from $\bthetaA_{i-1}$ to $\bthetaA_{i}$ is derived as:
\begin{equation}
    p(\btheta^A_{i}|\parsnt{i-1}^\calM)=\update^A(\btheta^A_{i}|\btheta^A_{i-1},\predA{i-1};t_{i-1})
\end{equation}

The detailed training and sampling algorithm could be found in \cref{alg:train} and \cref{alg:sampling}.






% \section{Methodology}
% 
% \js{Double check connection sentences between subsections}

\noindent \textbf{Notations.}
%We work on the 3D space.
% We denote 3D world coordinates with ${\bf{p}}=(x,y,z)$, and the camera viewing direction with $v=(\theta,\phi)$. 
% %from any 3D point in space with the angles $v=(\theta,\phi)$. %denoting pitch and yaw.
% The points in the 3D space have color $c^{{\bf{p}}, v}$ that depends on the 3D location $p$ and the viewpoint of the camera $\bf{v}$.
% The points also have a density value $\sigma^p$ that encodes how opaque that point is.
% We couple them together in $\bf{y}^p=\{c^{{\mathbf{p}}, v}, \sigma^p\}$.
% When examining a whole 3D object from multiple 3D locations in space, $i=1,...,N$, we denote the set of all 3D locations with $\mathbf{X}=\{\mathbf{x}^n\}_{n=1}^N$, and $\mathbf{Y}=\{\mathbf{y}^n\}_{n=1}^N$.
% Assuming a ray $r=(p, v)$ starting from location $p$ and from viewpoint $v$, with $x^r=\{x^r_i\}_{i=1}^P$ and $y^r=\{y^r_i\}_{i=1}^P$ we denote all the 3D locations, and colors on $P$ points sampled from the ray.
% Further, with ${\tilde{\bf{X}}}$ and ${\tilde{\bf{Y}}}$ we denote the observations we have, the set of camera rays ${\tilde{\bf{X}}} = \{  
% \tilde{\bf{x}}_n = r_n \}_n^{N}$, and the projected 2D pixels from the rays ${\tilde{\bf{Y}}} = \{  
% \tilde{\bf{y}}_n^p \}_n^{N}$. 
%
We denote 3D world coordinates by \(\mathbf{p} = (x, y, z)\) and a camera viewing direction by \(\mathbf{d} = (\theta, \phi)\). Each point in 3D space have its color \(\mathbf{c}(\mathbf{p}, \mathbf{d})\), which depends on the location \(\mathbf{p}\) and viewing direction \(\mathbf{d}\). Points also have a density value \(\sigma(\mathbf{p})\) that encodes opacity. We represent coordinates and view direction together as $\mathbf{x} = \{\mathbf{p},\mathbf{d} \}$, color and density together as \(\mathbf{y}(\mathbf{p}, \mathbf{d}) = \{\mathbf{c}(\mathbf{p}, \mathbf{d}), \sigma(\mathbf{p})\}\).
\str{This sentence sounds a bit strange, we can denote all 3D points like that anyways, no need to observe them from 'multiple locations', no? In general, I think the paragraph can be written more cleanly.}
When observing a 3D object from multiple locations, we denote all 3D points as \(\mathbf{X} = \{\mathbf{x}_n \}_{n=1}^N\) and their colors and densities as \(\mathbf{Y} = \{\mathbf{y}_n\}_{n=1}^N\).
Assuming a ray \(\mathbf{r} = (\mathbf{o}, \mathbf{d})\) starting from the camera origin \(\mathbf{o}\) and along direction \(\mathbf{d}\), we sample $P$ points along the ray, with \(\mathbf{x}^{\mathbf{r}} = \{{x}_i^\mathbf{r}\}_{i=1}^P\) and corresponding colors and densities \(\mathbf{y}^{\mathbf{r}} = \{{y}_i^{\mathbf{r}}\}_{i=1}^P\). Further, we denote the observations \(\widetilde{\mathbf{X}}\) and \(\widetilde{\mathbf{Y}}\) as: the set of camera rays \(\widetilde{\mathbf{X}} = \{\widetilde{\mathbf{x}}_n = \mathbf{r}_n\}_{n=1}^N\) and the projected 2D pixels from the rays \(\widetilde{\mathbf{Y}} = \{\widetilde{\mathbf{y}}_n\}_{n=1}^N\).



% \begin{figure}[t]
%   \centering
%   \includegraphics[width=0.49\textwidth]{Figures/problemstate.pdf} % Adjust the size and filename as needed
%   \caption{Framework.} % Caption for the figure
%   \label{fig:problem}
% \end{figure}



% \str{Can you clarify what exactly each notation style corresponds to? What is the $\sim$ for instance? The sampled new pixel views? And no tilde are the actual observations?}

\begin{figure}[htbp]
%\vspace{-5mm}
\centering
\centerline{
\includegraphics[width=0.95\columnwidth]{Figures/problemstate.pdf} 
} 
%\vspace{-2mm}
\caption{\textbf{Complete rendering from 3D points to a 2D pixel.}
}
%\vspace{-4mm}
\label{fig:problem}
\end{figure}

\textbf{Background on Neural Radiance Fields.}
We formally describe Neural Radiance Field (NeRF)~\citep{mildenhall2021nerf, arandjelovic2021nerf} as a continuous function \( f_{\text{NeRF}}: \mathbf{x} \mapsto \mathbf{y} \), which maps 3D world coordinates \(\mathbf{p}\) and viewing directions \(\mathbf{d}\) to color and density values \(\mathbf{y}\). 
That is, a NeRF function, \( f_{\text{NeRF}} \), is a neural network-based function that represents the whole 3D object (e.g., a car in Fig.~\ref{fig:problem}) as coordinates to color and density mappings. Learning a NeRF function of a 3D object is an inverse problem where we only have indirect observations of arbitrary 2D views of the 3D object, and we want to infer the entire 3D object's geometry and appearance.
With the NeRF function, given any camera pose, we can render a view on the corresponding 2D image plane by marching rays and using the corresponding colors and densities at the 3D points along the rays. Specifically, given a set of rays \(\mathbf{r}\) with view directions \(\mathbf{d}\), we obtain a corresponding 2D image. The integration along each ray corresponds to a specific pixel on the 2D image using the volume rendering technique described in~\cite{kajiya1984ray}, which is also illustrated in Fig.~\ref{fig:problem}. Details about the integration are given in Appendix~\ref{supp:nerf-render}. 



%\str{Write down the integral.}
%


% Neural Fields are normally considered as an optimization routine in a deterministic setting, whereby95
% the function fNeRF is fit perfectly to the available observations (akin to “overfitting” training data).


\subsection{Probabilistic NeRF Generalization}
% \js{probabilistic NeRF is not new}

\paragraph{Deterministic Neural Radiance Fields} Neural Radiance Fields are normally considered as an optimization routine in a deterministic setting~\citep{mildenhall2021nerf,barron2021mip}, whereby the function $f_{\text{NeRF}}$ fits specifically to the available observations (akin to ``overfitting'' training data).

\paragraph{Probabilistic Neural Radiance Fields} As we are not just interested in fitting a single and specific 3D object but want to learn how to infer the Neural Radiance Field of any 3D object,  we focus on probabilistic Neural Radiance Fields with the following factorization:
%
% \begin{equation}
%     p({\bf{\widetilde{y}}}|{\bf{\widetilde{x}}}, f_{NeRF}) \propto
%     \underbrace{p({\bf{\widetilde{y}}}| {\bf{{y}}}^{1:P}, {\bf{{x}}}^{1:P})}_{\text{Integration}}
%     \underbrace{p({\bf{{y}}}^{1:P}|{\bf{{x}}}^{1:P}, f_{\text{NeRF}})}_{\text{NeRF model}}
%     \underbrace{p({\bf{{x}}}^{1:P}|{\bf{\widetilde{x}}})}_{\text{Sampling}}.
% \label{eq: rendering}
% \end{equation}
% \begin{equation}
%     p({\bf{\widetilde{y}}}|{\bf{\widetilde{x}}}, f_{NeRF}) \propto
%     \underbrace{p({\bf{\widetilde{y}}}| {\bf{{y}}}^{\mathbf{r}}, {\bf{{x}}}^{\mathbf{r}})}_{\text{Integration}}
%     \underbrace{p({\bf{{y}}}^{\mathbf{r}}|{\bf{{x}}}^{\mathbf{r}}, f_{\text{NeRF}})}_{\text{NeRF model}}
%     \underbrace{p({\bf{{x}}}^{\mathbf{r}}|{\bf{\widetilde{x}}})}_{\text{Sampling}}.
% \label{eq: rendering}
% \end{equation}
% \begin{equation}
%     p({\bf{\widetilde{Y}}}_{T} | {\bf{\widetilde{X}}}_{T}) \varpropto
%     \underbrace{p({\bf{\widetilde{Y}}}_{T} | {\bf{{Y}}}_{T}, {\bf{{X}}}_{T})}_{\text{Integration}}
%     \underbrace{p({\bf{{Y}}}_{T} | {\bf{{X}}}_{T})}_{\text{NeRF Model}}
%     \underbrace{p({\bf{{X}}}_{T} | {\bf{\widetilde{X}}}_{T})}_{\text{Sampling}},
% \label{eq: definiation}
% \end{equation}
\begin{equation}
    p({\bf{\widetilde{Y}}} | {\bf{\widetilde{X}}}) \varpropto
    \underbrace{p({\bf{\widetilde{Y}}} | {\bf{{Y}}}, {\bf{{X}}})}_{\text{Integration}}
    \underbrace{p({\bf{{Y}}} | {\bf{{X}}})}_{\text{NeRF Model}}
    \underbrace{p({\bf{{X}}} | {\bf{\widetilde{X}}})}_{\text{Sampling}}.
\label{eq: probabilitic_NeRF}
\end{equation}
%
% \str{Is $f_\text{NeRF}$ now a random variable? Normally it is not.}
\str{This can also be writtena  more fluently}
The generation process of this probabilistic formulation is as follows.
We first start from (or sample) a set of rays $\widetilde{\mathbf{X}}$.
Conditioning on these rays, we sample 3D points in space $\mathbf{X} \big|\widetilde{\mathbf{X}}$.
Then, we map these 3D points into their colors and density values with the NeRF function, ${\bf{Y}} = f_{\text{NeRF}}({\bf{{X}}})$.
Last, we sample the 2D pixels of the viewing image that corresponds to the 3D ray ${\widetilde{\bf{Y}}}| {\bf{{Y}}}, {\bf{X}}$ with a probabilistic process. This corresponds to integrating colors and densities ${\bf{{Y}}}$ along the ray on locations ${\bf{X}}$.

% In the following sections, we will define the various probabilistic terms.
% \str{Here it would be good to be more explicit and say how are the various probabistic terms are defined. Or we can say that we will specify later, also in the context of Geometric NP. Either way, the current text below looks like deterministic relations, so I think we can not write them down here.}

\str{I suggest we go directly on conditional neural fields. The way we have it now, we only create extra confusion, unless we are the first to propose this decomposition (but there have been other probabilistic NeRFs before, no?). Or perhaps have better structure in the writing, otherwise it is confusing.. What is context, what target?}
The probabilistic model in \cref{eq: probabilitic_NeRF} is for a single 3D object, thus requiring optimizing a function $f_{\text{NeRF}}$ afresh for every new object, which is time-consuming. For NeRF generalization, we accelerate learning and improve generalization by amortizing the probabilistic model over multiple objects, obtaining per-object reconstructions by conditioning on context sets ${{\widetilde {\bf{X}}}_C, {\widetilde {\bf{Y}}}_C}$.
% \str{What about $\widetilde {\bf{X}}_T$? What is that? Also, why (1) has small letters, and here we have capitals?}
% These context variables are few observations from any new object, that is, the rays and the corresponding observed colors.
For clarity, we use ${(\cdot)}_{C}$ to indicate context sets with {a few new observations for a new object}, while ${(\cdot)}_{T}$ indicates target sets containing 3D points or camera rays from novel views of the same object.
Thus, we formulate a probabilistic NeRF for generalization as:
% \str{update this according to the above equation}
%
\begin{equation}
\begin{aligned}
    &p({\bf{\widetilde{Y}}}_{T} | {\bf{\widetilde{X}}}_{T}, {\bf{\widetilde{X}}}_{C}, {\bf{\widetilde{Y}}}_{C}) \varpropto \\
&    \underbrace{p({\bf{\widetilde{Y}}}_{T} | {\bf{{Y}}}_{T}, {\bf{{X}}}_{T})}_{\text{Integration}}
    \underbrace{p({\bf{{Y}}}_{T} | {\bf{{X}}}_{T}, {\bf{\widetilde{X}}}_{C}, {\bf{\widetilde{Y}}}_{C})}_{\text{NeRF Generalization}}
    \underbrace{p({\bf{{X}}}_{T} | {\bf{\widetilde{X}}}_{T})}_{\text{Sampling}}.
\end{aligned}
\label{eq: probabilitic_NeRF_generalization}
\end{equation}
%
%\str{Not sure if this sentence is good enough, please check later again.}
As this paper focuses on generalization with new 3D objects, we keep the same sampling and integrating processes as in ~\cref{eq: probabilitic_NeRF}. We turn our attention to the modeling of the predictive distribution $p({\bf{{Y}}}_{T}| {\bf{{X}}}_{T}, {\bf{\widetilde{X}}}_{C}, {\bf{\widetilde{Y}}}_{C})$ in the generalization step, which implies inferring the NeRF function.

\paragraph{Misalignment between 2D context and 3D structures} It is worth mentioning that the predictive distribution in 3D space is conditioned on 2D context pixels with their ray $\{{\bf{\widetilde{X}}}_{C}, {\bf{\widetilde{Y}}}_{C}\}$ and 3D target points ${\bf{X}}_{T}$, which is challenging due to potential information misalignment. Thus, we need strong inductive biases with 3D structure information to ensure that 2D and 3D conditional information is fused reliably.


% \str{Why do you call this the query step?}

% #############################
% By inferring the function distribution $p(f_{\text{NeRF}})$ from the context sets, we can obtain the predictive distribution as: 
% %
% \begin{equation}
%       p({\bf{Y}}_{T}| {\bf{X}}_{T}, {\bf{\widetilde{X}}}_{C}, {\bf{\widetilde{Y}}}_{C})  
%       = \int p({\bf{Y}}_{T}|f_{\text{NeRF}}, {\bf{X}}_{T}) p(f_{\text{NeRF}}| {{\bf{{X}}}_{T}, \bf{\widetilde{X}}}_{C}, {\bf{\widetilde{Y}}}_{C}) df_{\text{NeRF}} 
% \label{eq: gp_w/o_B}
% \end{equation}
% %
% where $p(f_{\text{NeRF}}| {\bf{X}}_{T} {\bf{\widetilde{X}}}_{C}, {\bf{\widetilde{Y}}}_{C})$ is the prior distribution of the NeRF function, and $p({\bf{Y}}_{T}|f_{\text{NeRF}}, {\bf{X}}_{T})$ is the likelihood term. We integrate the likelihood term over the latent space of all possible NeRF functions.
% %\str{The following two sentences may beed to be rewritten.}
% %It is important to note that the context variables by nature contain fewer views and thus less information per a new object.
% It is worth mentioning that inferring the NeRF function needs to incorporate 2D context pixels with their ray $\{{\bf{\widetilde{X}}}_{C}, {\bf{\widetilde{Y}}}_{C}\}$ and 3D target points ${\bf{X}}_{T}$, which is challenging due to potential information misalignment.  Thus, we need strong inductive biases with 3D structure information to ensure that 2D and 3D conditional information is fused reliably.
% #############################

% \subsection{Geometric Neural Processes for NeRF} 
\subsection{Geometric Bases} 
\label{sec: geometrybases}
% In NeRF generalization, given that the context set is expected to correspond to too few views with few 3D information, 
% % In NeRF generalization, given the context set corresponding to too few 2D views providing few 3D information, 
% fitting a model for $p({\bf{Y}}_{T}| {\bf{X}}_{T}, {\bf{\widetilde{X}}}_{C}, {\bf{\widetilde{Y}}}_{C})$ that generalizes well is challenging. 
To mitigate the information misalignment between 2D context views and 3D target points, we introduce geometric bases ${\bf{{B}}}_{C}=\{{\bf{b}}_i\}_{i=1}^{M}$, which {induces prior structure to the context set} $\{{\bf{\widetilde{X}}}_{C}, {\bf{\widetilde{Y}}}_{C}\}$ geometrically. $M$ is the number of geometric bases. 

\begin{figure*}[t]
  \centering  \includegraphics[width=0.99\textwidth]{ICLR2025/Figures/architecture-0.pdf} % Adjust the size and filename as needed
  % \vspace{-2mm}
\caption{\textbf{Illustration of our Geometric Neural Processes.} 
% We solve the problem of radiance field generalization by Neural Processes. 
% captures uncertainty induced by few available observations.
We cast radiance field generalization as a probabilistic modeling problem. Specifically, we first construct geometric bases ${\bf{B}}_C$ in 3D space from the 2D context sets ${\bf{\widetilde{X}}}_{C}, {\bf{\widetilde{Y}}}_{C}$ to model the 3D NeRF function (Section~\ref{sec: geometrybases}). We then infer the NeRF function by modulating a shared MLP through hierarchical latent variables ${\bf{z}}_{o}, {\bf{z}}_{r}$ and make predictions by the modulated MLP (Section~\ref{sec: hierar}). 
  The posterior distributions of the latent variables are inferred from the target sets ${\bf{\widetilde{X}}}_{T}, {\bf{\widetilde{Y}}}_{T}$, which supervises the priors during training (Section~\ref{sec: object}). 
  } % Caption for the figure
  \label{fig: framework}
  %\vspace{-2mm}
\end{figure*}

\str{where is the semantic representation coming from? Self-supervised models? Or is it learned?}
Each geometric basis consists of a Gaussian distribution in the 3D point space and a semantic representation, \textit{i.e.,} ${\bf{b}}_i = \{ \mathcal{N}(\mu_i, \Sigma_i); \omega_i\}$, 
%\str{What is $\omega_i$ in the equation? The weight of the Gaussian?We have mixtures of Gaussians? Please clarify.} 
where $\mu_i$ and $\Sigma_i$ are the mean and covariance matrix of $i$-th Gaussian in 3D space, and $\omega_i$ is its corresponding latent representation. 
Intuitively, the mixture of all 3D Gaussian distributions implies the structure of the object, while $\omega_i$ stores the corresponding semantic information.
% from a 2D context set, e.g., color and texture. 
In practice, we use a transformer-based encoder to learn the Gaussian distributions and representations from the context sets, \textit{i.e.,} $\{(\mu_i, \Sigma_i, \omega_i)\} = \texttt{Encoder} [{\bf{\widetilde{X}}}_{C}, {\bf{\widetilde{Y}}}_{C}]$. Detailed architecture of the encoder is provided in Appendix~\ref{supp:gaussian}. 

% \begin{equation}
%     {\bf{{B}}}_{C} = \{{\bf{b}}_i\}_{i=1}^{M}, {\bf{b}}_i=\{\mathcal{N}(\mu_i, \Sigma_i); \omega_i\},
%     \\
%      \mu_i, \Sigma_i, \omega_i = \texttt{Encoder} [{\bf{\widetilde{X}}}_{C}, {\bf{\widetilde{Y}}}_{C}],
% \end{equation}
% where $M$ is the number of Gaussian bases. 

% where ${\bf{{B}}}_{C}$ are Gaussian bases inferred from the context views $\{{\bf{\widetilde{X}}}_{C}, {\bf{\widetilde{Y}}}_{C}\}$ with 3D structure information, \textit{i.e.,} 
% ${\bf{{B}}}_{C}=\texttt{Encoder}\Big({\bf{\widetilde{X}}}_{C}, {\bf{\widetilde{Y}}}_{C}\Big)$. 


% To address the information loss in the context views, we develop a geometry-aware prior distribution for the NeRF function. The geometry-aware prior integrates a set of geometry bases ${\bf{{B}}}_{C}$ and the target location points ${\bf{X}}_{T}$, which enrich the context sets with the {structure locality information}. 
% By doing so, we reformulate the prior distribution of the NeRF function as:
% \begin{equation}
%     p(f_{\text{NeRF}}| {\bf{X}}_{T}, {\bf{\widetilde{X}}}_{C}, {\bf{\widetilde{Y}}}_{C}) = p(f_{\text{NeRF}}| {\bf{X}}_{T}, {\bf{{B}}}_{C}), 
% \label{eq: prior_f}
% \end{equation}
% \str{Can a deterministic variable be part of a probabilistic expression?d}
% where ${\bf{{B}}}_{C}$ is a set of Gaussian bases inferred from the context views $\{{\bf{\widetilde{X}}}_{C}, {\bf{\widetilde{Y}}}_{C}\}$ with 3D structure information, \textit{i.e.,} ${\bf{{B}}}_{C}=\texttt{Encoder}[{\bf{\widetilde{X}}}_{C}, {\bf{\widetilde{Y}}}_{C}]$. 
% Specifically, we construct ${\bf{{B}}}_{C}$ as:
% Geometry basis-agnostic ideas, like 4D scene tensor~\cite{chen2022tensorf}, and RBF kernels~\cite{chen2023neurbf} has been used in deterministic NeRF to store the 3D scene geometry and semantic information. This motivates us to use a set of geometry bases to represent both the geometry structure and semantic information of the scene. 
% We assume the space is spanned by a set of basis, with geometric shapes and high-dimensional representation. 
% The geometry basis ${\bf{B}}_C$ is given by a posterior distribution, $p_{\pi}( {\bf{{B}}}_{C}| {\bf{\widetilde{X}}}_{C}, {\bf{\widetilde{Y}}}_{C})$. This can be explained as the posterior knowledge (color and spatial location) of the scene when a human sees a view of a scene (context image). Then, we model the function distribution as:
% \begin{align}
%     &{\bf{{B}}}_{C} = \{{\bf{b}}_i\}_{i=1}^{M}, {\bf{b}}_i=\{\mathcal{N}(\mu_i, \Sigma_i); \omega_i\},
%     \label{eq: generation_B_1}
%     \\
%     & \mu_i, \Sigma_i = \texttt{Att}({\bf{\widetilde{X}}}_{C}, {\bf{\widetilde{Y}}}_{C}), \texttt{Att}({\bf{\widetilde{X}}}_{C}, {\bf{\widetilde{Y}}}_{C}),
%     \label{eq: generation_B_2}
%     \\
%     & \omega_i = \texttt{Att}({\bf{\widetilde{X}}}_{C}, {\bf{\widetilde{Y}}}_{C}),
%     \label{eq: generation_B_3}
% \end{align}
% where $M$ is the number of the Gaussian bases. $\mu \mathbb \in {R}^3$ is the Gaussian center, $\Sigma \in  \mathbb{R}^{3\times 3}$ is the covariance matrix, and $\omega \in \mathbb{R}^{d_B}$ is the corresponding ${d_B}$-dimension semantic representation. 
%Each Gaussian basis represents a 3D Gaussian kernel and its corresponding semantic information. The shape of a Gaussian kernel can reflect a local object structure.
%For each kernel, \js{details here: we use a visual self-attention to estimate the mean $\mu \mathbb \in {R}^3$ and covariance matrix $\Sigma \in  \mathbb{R}^{3\times 3}$, and a corresponding ${d_B}$-dimension semantic representation $\omega \in \mathbb{R}^{d_B}$. }

With the geometric bases $\mathbf{B}_C$, we review the predictive distribution from  $p({\bf{Y}}_{T}| {\bf{X}}_{T}, {\bf{\widetilde{X}}}_{C}, {\bf{\widetilde{Y}}}_{C})$ to $p({\bf{Y}}_{T}| {\bf{X}}_{T},{\bf{{B}}}_{C})$.  By inferring the function distribution $p(f_{\text{NeRF}})$, we reformulate the predictive distribution as: 
\begin{equation}
    % p({\bf{{Y}}}_{T} | {\bf{{X}}}_{T}, {\bf{\widetilde{X}}}_{C}, {\bf{\widetilde{Y}}}_{C}) = 
    p({\bf{{Y}}}_{T} | {\bf{{X}}}_{T}, {\bf{{B}}}_{C}) = \int p({\bf{Y}}_{T}|f_{\text{NeRF}}, {\bf{X}}_{T}) p(f_{\text{NeRF}}| {\bf{X}}_{T}, {\bf{B}}_{C}) df_{\text{NeRF}},
\label{eq: predictive_w_B}
\end{equation}
where $p(f_{\text{NeRF}}| {\bf{X}}_{T}, {\bf{B}}_{C})$ is the prior distribution of the NeRF function, and $p({\bf{Y}}_{T}|f_{\text{NeRF}}, {\bf{X}}_{T})$ is the likelihood term. 
% We integrate the likelihood term with all possible NeRF functions. 
%\str{How do we do this? The space to integrate over must be huge, no?}
%\str{The following sentence is a bit weird, can you check it again.}
% \wy{We integrate the likelihood term over the latent space of all possible variables for modulating NeRF functions by monte carlo sampling, which can be seen as integrating over a function distribution.}
Note that the prior distribution of the NeRF function is conditioned on the target points ${\bf{X}}_{T}$ and the geometric bases ${\bf{B}}_{C}$. 
Thus, the prior distribution is data-dependent on the target inputs, yielding a better generalization on novel target views of new objects. 
Moreover, since ${\bf{B}}_{C}$ is constructed with continuous Gaussian distributions in the 3D space, the geometric bases can enrich the locality and semantic information of each discrete target point, enhancing the capture of high-frequency details~\citep{chen2023neurbf,chen2022tensorf,muller2022instant}.

% Since ${\bf{B}}_{C}$ is constructed in the 3D space, 3D target points ${\bf{X}}_{T})$ is able to effectively interact with ${\bf{B}}_{C}$, alleviating the information misalignment.

%\zx{We need some advantages of B in this paragraph}
%\zx{maybe also refer some splatting/RBF kernel methods to introduce what B is, how does B contain 3D structure information}
% \wy{The inferred posterior about the Gaussian bases shapes is able to reflect the object shape, while the semantic posterior (e.g. color and texture) is embedded in the representation of the bases. Moreover, the Gaussian basis spanned in the space is able to help aggregate the locality information for each queried point, which facilitates learning representing high-frequency details~\cite{chen2023neurbf,chen2022tensorf,muller2022instant}.}

% By integrating the prior distribution in equation~\cref{eq: prior_f} into the predictive distribution in equation~\cref{eq: gp_w/o_B}, we replace  $p({\bf{Y}}_{T}| {\bf{X}}_{T}, {\bf{\widetilde{X}}}_{C}, {\bf{\widetilde{Y}}}_{C})$ with $p({\bf{{Y}}}_{T} | {\bf{{X}}}_{T}, {\bf{{B}}}_{C})$.
% The stochastic processes for NeRF with geometry bases are then derived as:
% % With the encoded geometry base ${\bf{B}}_{C}$, we reformulate the querying step as $p({\bf{{Y}}}_{T} | {\bf{{X}}}_{T}, {\bf{{B}}}_{C})$. By integrating the approximated prior with the bases Eq.~\ref{eq: prior_f} into the generative process in Eq.~\ref{eq: gp_w/o_B}, we derive the stochastic processes for NeRF with Geometry Bases as:
% \str{In 3, $p(f_{Nerf})$ is not conditioned on $X_T$. Is there a difference?}
% \begin{equation}
%     % p({\bf{{Y}}}_{T} | {\bf{{X}}}_{T}, {\bf{\widetilde{X}}}_{C}, {\bf{\widetilde{Y}}}_{C}) = 
%     p({\bf{{Y}}}_{T} | {\bf{{X}}}_{T}, {\bf{{B}}}_{C}) = \int p({\bf{Y}}_{T}|f_{\text{NeRF}}, {\bf{X}}_{T}) p(f_{\text{NeRF}}| {\bf{X}}_{T}, {\bf{B}}_{C}) df_{\text{NeRF}}.
% \label{eq: gp_w_B}
% \end{equation}
%
% Inferred from the context-generated 3D information in ${\bf{{B}}}_{C}$, the NeRF function prior $p(f_{\text{NeRF}}|{\bf{X}}_{T}, {\bf{{B}}}_{C})$ reduces the impact of information loss in context views (\wy{due to sampling and integration}) and becomes more suitable for the \zx{query step} $p({\bf{Y}}_{T}|f_{\text{NeRF}}, {\bf{X}}_{T})$ in 3D space.

\subsection{Geometric Neural Processes with Hierarchical Latent Variables}
\label{sec: hierar}

% To achieve the 
With the geometric bases, we propose Geometric Neural Processes (\textbf{\method{}}) by inferring the NeRF function distribution $p(f_{\text{NeRF}}|{\bf{X}}_{T}, {\bf{{B}}}_{C})$ in a probabilistic way.  
% We can generalize NeRF learning and efficiently adapt the functional distribution to new 3D objects.
Based on the probabilistic NeRF generalization in~\cref{eq: probabilitic_NeRF_generalization}, we introduce hierarchical latent variables to encode various spatial-specific information into $p(f_{\text{NeRF}}|{\bf{X}}_{T}, {\bf{{B}}}_{C})$, improving the generalization ability in different spatial levels.
%\str{Make sure that notation is consistent and not overloaded. Eg, $x^r$ rather than $x^\mathbf{r}$ since it is not that we use the $1:P$ somewhere specific, besides it is not clear that this corresponds to a ray, since the $P$ points could be anywhere.}
Since all rays are independent of each other, we decompose the predictive distribution in \cref{eq: predictive_w_B} as:
\begin{equation}
    p({\bf{Y}}_{T}| {\bf{X}}_{T},  {\bf{B}}_{C})  = \prod_{n=1}^{N} p({\bf{y}}_{T}^{\mathbf{r}, n}| {\bf{x}}_{T}^{{\mathbf{r}}, n},  {\bf{B}}_{C}),
\label{eq: predictive_distribution_ray_specific}
\end{equation}
where the target input ${\bf{X}}_{T}$ consists of $N \times P$ location points $\{{\bf{x}}_{T}^{{\mathbf{r}}, n}\}_{n=1}^{N}$ for $N$ rays.


\begin{figure}[htbp]
\centering
\includegraphics[width=0.9\columnwidth]{ICLR2025/Figures/graphical_model2.pdf} 
\caption{\textbf{Graphical model for the proposed geometric neural processes.}}
\label{fig: graphical_model}
\end{figure}

Further, we develop a hierarchical Bayes framework for \method{} to accommodate the data structure of the target input ${\bf{X}}_{T}$ in \cref{eq: predictive_distribution_ray_specific}.
We introduce an object-specific latent variable $\mathbf{z}_o$ and $N$ individual ray-specific latent variables $\{\mathbf{z}_r^{n}\}_{n=1}^{N}$ to represent the randomness of $f_\text{NeRF}$.
% the probabilistic NeRF function. 


\str{The formatting here looks weird. Is this the right template?}
Within the hierarchical Bayes framework, $\mathbf{z}_o$ encodes the entire object information from all target inputs and the geometric bases $\{\mathbf{X}_T, \mathbf{B}_C\}$ in the global level; while every $\mathbf{z}_r^{n}$ encodes ray-specific information from $\{ \mathbf{x}_T^{\mathbf{r}, n}, \mathbf{B}_C\}$ in the local level, which is also conditioned on the global latent variable $\mathbf{z}_o$. 
The hierarchical architecture allows the model to exploit the structure information from the geometric bases $\mathbf{B}_C$ in different levels, improving the model's expressiveness ability.
By introducing the hierarchical latent variables in \cref{eq: predictive_distribution_ray_specific}, we model \method{} as:
% \begin{equation}
% \small
%         p({\bf{Y}}_{T}| {\bf{X}}_{T}, {\bf{B}}_{C}) = \int \prod_{n=1}^{N} \Big\{ \int p({\bf{y}}_{T}^{\mathbf{r}, n}| {\bf{x}}_{T}^{\mathbf{r}, n}, {\bf{B}}_{C}, {\bf{z}}_r^n,{\bf{z}}_o, ) p({\bf{z}}_{r}^n| {\bf{z}}_o,  {\bf{x}}_{T}^{\mathbf{r}, n}, {\bf{B}}_C) d {\bf{z}}_r^n \Big\} p({\bf{z}}_o |{\bf{X}}_T, {\bf{B}}_C) d {\bf{z}}_o, 
% \label{eq:ganp-model}
% \end{equation}
{\small
\begin{equation}
\begin{aligned}
        p({\bf{Y}}_{T}| {\bf{X}}_{T}, {\bf{B}}_{C}) &= \int \prod_{n=1}^{N} \Big\{ \int p({\bf{y}}_{T}^{\mathbf{r}, n}| {\bf{x}}_{T}^{\mathbf{r}, n}, {\bf{B}}_{C}, {\bf{z}}_r^n,{\bf{z}}_o ) \\
        &p({\bf{z}}_{r}^n| {\bf{z}}_o,  {\bf{x}}_{T}^{\mathbf{r}, n}, {\bf{B}}_C) d {\bf{z}}_r^n \Big\} p({\bf{z}}_o |{\bf{X}}_T, {\bf{B}}_C) d {\bf{z}}_o,
\end{aligned}
\label{eq:ganp-model}
\end{equation}
}where $p({\bf{y}}_{T}^{\mathbf{r}, n}| {\bf{x}}_{T}^{\mathbf{r}, n}, {\bf{B}}_{C}, {\bf{z}}_o, {\bf{z}}_r^i)$ denotes the ray-specific likelihood term. In this term, we use the hierarchical latent variables $\{{\bf{z}}_o, {\bf{z}}_r^i\}$ to modulate a ray-specific NeRF function $f_{\text{NeRF}}$ for prediction, as shown in Fig.~\ref{fig: framework}.
% In general, we first use the object-specific latent variable $\mathbf{z}_o$ to make $f_{NeRF}$ object-specific. Then, the ray-specific latent variable $\mathbf{z}_r$ to enable $f_{\text{NeRF}}$ to capture the local texture information.
Hence, $f_{\text{NeRF}}$ can explore global information of the entire object and local information of each specific ray, leading to better generalization ability on new scenes and new views.
A graphical model of our method is provided in Fig.~\ref{fig: graphical_model}. 

% As mentioned in Eq.~\ref{eq: definiation}, the target inputs can be obtained by randomly sampling ${\bf{x}}_{T}^{1:P}$ from each ray ${\widetilde{\bf{x}}}_{T}$. Thus, target input ${\bf{X}}_{T}$ consists of $N \times P$ location points $\{{\bf{x}}_{T}^{1:P, n}\}_{n=1}^{N}$ given $N$ rays. 
% Since the rendering of each ray is independently \cite{martin2021nerf}, we reformulate the predictive distribution for NeRF in a ray-specific manner:
% \begin{equation}
%     p({\bf{Y}}_{T}| {\bf{X}}_{T},  {\bf{B}}_{C})  = \prod_{n=1}^{N} p({\bf{y}}_{T}^{1:P, n}| {\bf{x}}_{T}^{1:P, n},  {\bf{B}}_{C}).
% \label{eq: predictive_distribution_ray_specific}
% \end{equation}
% Based on the hierarchical data structure of the target inputs, we design a hierarchical latent variable model. In the proposed model, object-specific latent variables ${\bf{g}}$ encode the entire 3D object information; each ${\bf{g}}$ corresponds $N$ individual ray-specific latent variables $\{{\bf{r}}^{n}\}_{n=1}^{N}$.

%\str{Until here. I will check below tomorrow, but maybe you can use the same style like above, more concise and to the point.}
%\str{In general, we have to explain the hierarchical framework better. What is $g$ for instance. }



% that corresponds to the ray's viewpoint $v$. 

% Each viewing image is rendered from a camera pose by marching rays into the 2D image plane and computing colors.

% To optimize a NeRF function for a 3D object, we only have access to several viewing images and their ray information, \textit{e.g., $p({\bf{\widetilde{y}}}|{\bf{\widetilde{x}}}, f_{NeRF})$}.
% \zx{When learning the NeRF function of a 3D object, we only have access to arbitrary 2D views of the 3D object.
% Each viewing image is rendered from a camera pose by marching rays into the 2D image plane and computing colors.
% can render a viewing image of the object from any camera pose by marching rays into the 2D image plane and computing colors. 
% \str{The notation below looks inconsistent with that of the previou svariable, where we have $c_r$, ...}
% In general, each ray corresponds to a pixel in the 2D image from this view. 
% We use ${\bf{\widetilde{x}}} = [\mathbf{r}_o;\mathbf{r}_d] \in \mathbb{R}^{6}$ and ${\bf{\widetilde{y}}} = [r, g, b] \in \mathbb{R}^{3}$ to denote one specific ray and its corresponding pixel color, respectively. 
% Here $\mathbf{r}_o \in \mathbb{R}^3$ and $\mathbf{r}_d \in \mathbb{R}^3$ denote the origin and the direction of the ray, respectively. Each ray color ${\bf{\widetilde{y}}}$ is the RGB value for the corresponding pixel in the 2D image~\cite{arandjelovic2021nerf}.

% To render the ray color of each pixel, as illustrated in Fig.~\ref{fig:problem}, the volume rendering technique~\cite{kajiya1984ray} is used.
% $P$ discrete location points ${\bf{x}}^{1:P} =\{{\bf{x}}^p\}_{p=1}^P$ are first sampled along the ray, and their densities and colors ${\bf{y}}^{1:P} = \{{\bf{y}}^p\}_{p=1}^P$ are queried by the function $f_{\text{NeRF}}$.
% The information from all discrete locations is then integrated via a rendering equation. 
% % An illustration of the rendering process is shown in Fig.~\ref{fig:problem}.
% We factorize the complete rendering process $p({\bf{\widetilde{y}}}|{\bf{\widetilde{x}}})$ as follows:  
% \begin{equation}
%     p({\bf{\widetilde{y}}}|{\bf{\widetilde{x}}}, f_{NeRF}) \propto
%     \underbrace{p({\bf{\widetilde{y}}}| {\bf{{y}}}^{1:P}, {\bf{{x}}}^{1:P})}_{\text{Integration}}
%     \underbrace{p({\bf{{y}}}^{1:P}|{\bf{{x}}}^{1:P}, f_{NeRF})}_{\text{NeRF model}}
%     \underbrace{p({\bf{{x}}}^{1:P}|{\bf{\widetilde{x}}})}_{\text{Sampling}},
% \label{eq: rendering}
% \end{equation}
% \begin{equation}
%     p({\bf{\widetilde{y}}}|{\bf{\widetilde{x}}}) \varpropto p({\bf{\widetilde{y}}}| {\bf{{y}}}_{1:P}, {\bf{{x}}}_{1:P}) p({\bf{{y}}}_{1:P}|{\bf{{x}}}_{1:P})p({\bf{{x}}}_{1:P}|{\bf{\widetilde{x}}}),
% \label{eq: rendering}
% \end{equation}
% where $p({\bf{\widetilde{y}}}| {\bf{{y}}}^{1:P}, {\bf{{x}}}^{1:P})$ represents the integrating step
% % , where the pre-defined render equation computes the ray color 
% by weighted mixing all location colors using the pre-defined render equation ${\bf{\widetilde{y}}} = f_{\text{render}}({\bf{y}}^{1:P}, {\bf{x}}^{1:P})$  \cite{arandjelovic2021nerf,mildenhall2021nerf}.
% % etails of the equation can be found in~\cite{arandjelovic2021nerf,mildenhall2021nerf}.
% $p({\bf{y}}^{1:P}|{\bf{x}}^{1:P})$ denotes the NeRF modeling step for each queried points, \textit{e.g.}, ${\bf{y}}^{1:P}:= f_{\text{NeRF}}({\bf{x}}^{1}), f_{\text{NeRF}}({\bf{x}}^{2}), ..., f_{\text{NeRF}}({\bf{x}}^{P})$.
% $p({\bf{{x}}}^{1:P}|{\bf{\widetilde{x}}})$ denotes the randomly sampling step along the ray $\bf{\widetilde{x}}$. 
% \zx{Given a set of rays ${\bf{\widetilde{X}}}=\{ {\bf{\widetilde{x}}} \}^{N}$ from the same camera pose, we can obtain its corresponding 2D image ${\bf{\widetilde{Y}}}= \{ {\bf{\widetilde{y}}} \}^{N}$ consisting of all pixel colors by ray-specific rendering. Each specific ray corresponds to a pixel on the image.}
% Moreover, we use ${\bf{\widetilde{X}}}$ and ${\bf{\widetilde{Y}}}$ to represent sets of rays and their corresponding colors in 2D image plane.


% \subsection{Probabilistic Radiance Field Generalization} 

% Conventional methods~\cite{martin2021nerf,chen2022tensorf} usually train a deterministic neural network to optimize the NeRF function $f_{\text{NeRF}}$ in Eq. \ref{eq: rendering}. However, this requires the model to overfit large amounts of viewing images for each 3D object, which is data intensive and time-consuming, with poor generalization ability.

%However, it is difficult to collect location points and their location densities and colors in 3D space (\textcolor{red}{citation}). 
% By contrast, this paper focuses on radiance field generalization on arbitrary new 3D objects with few viewing images. 
% However, the method is applicable to arbitrary implicit neural representation (e.g. 2D images). 
% For a new 3D object
% Therefore, $f_{\text{NeRF}}$ needs to be quickly generalized to a new scene based on \textit{few-shot} context views $\{ {\bf{\widetilde{X}}}_C, {\bf{\widetilde{Y}}}_C\}$. 
% %images and their ray information, which are represented as the context set including ray and pixel color $\{ {\bf{\widetilde{X}}}_C, {\bf{\widetilde{Y}}}_C\}$. 
% The generalized NeRF function is then utilized to render any target set of ray ${\bf{\widetilde{X}}}_{T}$ to get the corresponding pixel color set ${\bf{\widetilde{Y}}}_{T}$ in 2D image plane.

% To achieve fast generalization with limited views, we formulate the problem of radiance field generalization in a probabilistic manner.
% The probabilistic formulation enables us to infer the NeRF function while considering uncertainty, improving the generalization ability on limited context data.
% Formally, we formulate the goal of radiance field generalization as $p({\bf{\widetilde{Y}}}_{T}| {\bf{\widetilde{X}}}_{T}, {\bf{\widetilde{X}}}_{C}, {\bf{\widetilde{Y}}}_{C} )$. 
% % The target distribution of radiance field generalization is $p({\bf{\widetilde{Y}}}_{T}| {\bf{\widetilde{X}}}_{T}, {\bf{\widetilde{X}}}_{C}, {\bf{\widetilde{Y}}}_{C} )$. 
% Considering the rendering with a NeRF function in Eq.~\ref{eq: rendering}, we review the target distribution as:
% % \begin{equation}
% %     p({\bf{\widetilde{Y}}}_{T} | {\bf{\widetilde{X}}}_{T}, {\bf{\widetilde{X}}}_{C}, {\bf{\widetilde{Y}}}_{C}) \varpropto
% %     \underbrace{p({\bf{\widetilde{Y}}}_{T} | {\bf{{Y}}}_{T}, {\bf{{X}}}_{T})}_{\text{Integration}}
% %     \underbrace{p({\bf{{Y}}}_{T} | {\bf{{X}}}_{T}, {\bf{\widetilde{X}}}_{C}, {\bf{\widetilde{Y}}}_{C})}_{\text{Generalization}}
% %     \underbrace{p({\bf{{X}}}_{T} | {\bf{\widetilde{X}}}_{T})}_{\text{Sampling}},
% % \label{eq: definiation}
% % \end{equation}
% \begin{equation}
%     p({\bf{\widetilde{Y}}}_{T} | {\bf{\widetilde{X}}}_{T}, {\bf{\widetilde{X}}}_{C}, {\bf{\widetilde{Y}}}_{C}) \varpropto
%     \underbrace{p({\bf{\widetilde{Y}}}_{T} | {\bf{{Y}}}_{T}, {\bf{{X}}}_{T})}_{\text{Integration}}
%     \underbrace{p({\bf{{Y}}}_{T} | {\bf{{X}}}_{T}, {\bf{\widetilde{X}}}_{C}, {\bf{\widetilde{Y}}}_{C})}_{\text{Generalization}}
%     \underbrace{p({\bf{{X}}}_{T} | {\bf{\widetilde{X}}}_{T})}_{\text{Sampling}},
% \label{eq: definiation}
% \end{equation}
% Without loss of generality, we share the fixed sampling and integrating processes as in Eq.~\ref{eq: rendering}. Thus, we focus on the modeling of query step $p({\bf{{Y}}}_{T}| {\bf{{X}}}_{T}, {\bf{\widetilde{X}}}_{C}, {\bf{\widetilde{Y}}}_{C})$.  

%\wy{The ray casting used in volume rendering (sampling process described above) enables sampling in the 3D space continuously, which }
% The probabilistic formulation enables us to obtain the NeRF function by neural processes, which capture uncertainty and directly infer the function in a single feed-forward pass. 
% This leads to both effective and efficient generalization of the NeRF function on new scenes.


% Conventional methods \zx{references\cite{}} usually learn a deterministic neural network $\theta_{NeRF}$ to model a NeRF function. This requires the neural network $\theta_{NeRF}$ to overfit on amounts of viewing images for a 3D object, which is time-consuming. 
% \begin{equation}
% p({\bf{{Y}}}_{T} | {\bf{{X}}}_{T}, {\bf{\widetilde{X}}}_{C}, {\bf{\widetilde{Y}}}_{C}) \approx  p({\bf{{Y}}}_{T} | {\bf{{X}}}_{T}, \theta^*_{NeRF}), 
% \end{equation}
% where the best neural network is estimated by maximum a posterior (MAP) on the support sets of ray and ray colors, \textit{e.g.}, $\theta^*_{NeRF} =\mathop{\arg\min}_{\theta} p(\theta_{NeRF}|{\bf{\widetilde{X}}}_{C}, {\bf{\widetilde{Y}}}_{C})$. For each new 3D object, $\theta_{NeRF}$ will be re-trained for overfitting, yielding worse generalizations. Hence, leveraging limited viewing images for radiance field generalization remains a problem. 

% \textbf{Neural Processes for Radiance Field Generalization.}
% \js{Motivation of generative processes in a probabilistic way}
% In radiance field generalization, each new 3D object only accesses a few viewing images. In this case, a deterministic NeRF function could not capture uncertainty caused by the limited viewing images, leading to worse generalization ability on new 3D objects. 


% \subsection{Stochastic Processes for Radiance Field Generalization with Geometry Basis} 
% \label{sec: geometrybases}

% \noindent \textbf{Stochastic Processes for NeRF.}
% To capture uncertainty and improve generalization ability on limited context views, we cast radiance field generalization as stochastic processes over the NeRF function in 3D space. 
% {The processes directly infer the NeRF function in a feedforward pass, leading to both effective and efficient generalization on new scenes.}
% Specifically, we formulate the predictive distribution of radiance field generalization as follows:
% \begin{equation}
%       p({\bf{Y}}_{T}| {\bf{X}}_{T}, {\bf{\widetilde{X}}}_{C}, {\bf{\widetilde{Y}}}_{C})  
%       = \int p({\bf{Y}}_{T}|f_{NeRF}, {\bf{X}}_{T}) p(f_{NeRF}| {\bf{\widetilde{X}}}_{C}, {\bf{\widetilde{Y}}}_{C}) df_{NeRF} 
% \label{eq: gp_w/o_B}
% \end{equation}
% where $p(f_{\text{NeRF}}| {\bf{\widetilde{X}}}_{C}, {\bf{\widetilde{Y}}}_{C})$ is a prior distribution of the NeRF function.
% Since $f_{\text{NeRF}}$ is constructed on 3D space, it should ideally be inferred from the 3D location context sets $\{{\bf{{X}}}_{C}, {\bf{{Y}}}_{C}\}$. 
% \str{I am not sure the following sentences really make a (important) point.}
% However, the 3D location information is not available during practice training.
% The only accessible information of the 3D object is the limited viewing images $\{{\bf{\widetilde{X}}}_{C}, {\bf{\widetilde{Y}}}_{C}\}$.
% Moreover, the discrete sampling and integration for rendering results in information loss of the context sets from 3D location space to 2D views (Fig. \ref{fig:problem}).
% Therefore, inferring $f_{\text{NeRF}}$ by the context views is less optimal.
% However, NeRF mapping location points to their location colors and densities should be ideally inferred from the location sets $\{{\bf{{X}}}_{C}, {\bf{{Y}}}_{C}\}$, rather than the ray sets $\{{\bf{\widetilde{X}}}_{C}, {\bf{\widetilde{Y}}}_{C}\}$.
% In practice, the former is not available during training. 
% We can not directly transfer the context sets in the ray or camera space $\{{\bf{{X}}}_{C}, {\bf{{Y}}}_{C}\}$ to the 3D location space $\{{\bf{\widetilde{X}}}_{C}, {\bf{\widetilde{Y}}}_{C}\}$. This is intractable since the information loss of the context sets from the location space (3D) to the ray space (2D) due to the discrete sampling and integration. 

% To better capture the uncertainty with the context views, we cast radiance field generation as a stochastic process over a NeRF function in the 3D space. 
% The prior distribution of the function should be formulated as $p(f_{\text{NeRF}}| {\bf{X}}_{T}, {\bf{{X}}}_{C}, {\bf{{Y}}}_{C})$. 
% However, in practice, ${\bf{{X}}}_{C}, {\bf{{Y}}}_{C}$ (the context radiance field) is not available during training. We can not directly transfer the context sets in the ray or camera space $\{{\bf{{X}}}_{C}, {\bf{{Y}}}_{C}\}$ to the 3D location space $\{{\bf{\widetilde{X}}}_{C}, {\bf{\widetilde{Y}}}_{C}\}$. This is intractable since the information loss of the context sets from the location space (3D) to the ray space (2D) due to the discrete sampling and integration. We introduce a set of geometry basis to address this issue, which can also enrich the query points with the structure locality information.

% Thus, we can factorize the query process in Eq.~\ref{eq: definiation} as:
% \begin{equation}
%     p({\bf{Y}}_{T}| {\bf{X}}_{T}, {\bf{\widetilde{X}}}_{C}, {\bf{\widetilde{Y}}}_{C}) \varpropto p({\bf{Y}}_{T}, f_{NeRF}| {\bf{X}}_{T}, {\bf{{X}}}_{C}, {\bf{{Y}}}_{C}) p( {\bf{{X}}}_{C}, {\bf{{Y}}}_{C} | {\bf{\widetilde{X}}}_{C}, {\bf{\widetilde{Y}}}_{C}),
% \label{eq: query process1}
% \end{equation}
% where $p({\bf{{X}}}_{C}, {\bf{{Y}}}_{C} | {\bf{\widetilde{X}}}_{C}, {\bf{\widetilde{Y}}}_{C})$ denotes generating process from ray to location spaces, which is reverse from the complete rendering from location to ray spaces in Eq.~\ref{eq: rendering}. Therefore, this generating process is intractable since the information loss of the support sets from the location space (3D) to the ray space (2D). 

% \noindent \textbf{Geometry Basis.}
% To address the information loss in the context views, we develop a geometry-aware prior distribution for the NeRF function. The geometry-aware prior integrates a set of geometry bases ${\bf{{B}}}_{C}$ and the target location points ${\bf{X}}_{T}$, which enrich the context sets with the {structure locality information}. 
% By doing so, we reformulate the prior distribution of the NeRF function as:
% \begin{equation}
%     p(f_{\text{NeRF}}| {\bf{X}}_{T}, {\bf{\widetilde{X}}}_{C}, {\bf{\widetilde{Y}}}_{C}) = p(f_{\text{NeRF}}| {\bf{X}}_{T}, {\bf{{B}}}_{C}), 
% \label{eq: prior_f}
% \end{equation}
% \str{Can a deterministic variable be part of a probabilistic expression?d}
% where ${\bf{{B}}}_{C}$ is a set of Gaussian bases inferred from the context views $\{{\bf{\widetilde{X}}}_{C}, {\bf{\widetilde{Y}}}_{C}\}$ with 3D structure information, \textit{i.e.,} ${\bf{{B}}}_{C}=\texttt{Encoder}[{\bf{\widetilde{X}}}_{C}, {\bf{\widetilde{Y}}}_{C}]$. Specifically, we construct ${\bf{{B}}}_{C}$ as:
% % Geometry basis-agnostic ideas, like 4D scene tensor~\cite{chen2022tensorf}, and RBF kernels~\cite{chen2023neurbf} has been used in deterministic NeRF to store the 3D scene geometry and semantic information. This motivates us to use a set of geometry bases to represent both the geometry structure and semantic information of the scene. 
% % We assume the space is spanned by a set of basis, with geometric shapes and high-dimensional representation. 
% % The geometry basis ${\bf{B}}_C$ is given by a posterior distribution, $p_{\pi}( {\bf{{B}}}_{C}| {\bf{\widetilde{X}}}_{C}, {\bf{\widetilde{Y}}}_{C})$. This can be explained as the posterior knowledge (color and spatial location) of the scene when a human sees a view of a scene (context image). Then, we model the function distribution as:
% \begin{align}
%     &{\bf{{B}}}_{C} = \{{\bf{b}}_i\}_{i=1}^{M}, {\bf{b}}_i=\{\mathcal{N}(\mu_i, \Sigma_i); \omega_i\},
%     \label{eq: generation_B_1}
%     \\
%     & \mu_i, \Sigma_i = \texttt{Att}({\bf{\widetilde{X}}}_{C}, {\bf{\widetilde{Y}}}_{C}), \texttt{Att}({\bf{\widetilde{X}}}_{C}, {\bf{\widetilde{Y}}}_{C}),
%     \label{eq: generation_B_2}
%     \\
%     & \omega_i = \texttt{Att}({\bf{\widetilde{X}}}_{C}, {\bf{\widetilde{Y}}}_{C}),
%     \label{eq: generation_B_3}
% \end{align}
% where $M$ is the number of the Gaussian bases. $\mu \mathbb \in {R}^3$ is the Gaussian center, $\Sigma \in  \mathbb{R}^{3\times 3}$ is the covariance matrix, and $\omega \in \mathbb{R}^{d_B}$ is the corresponding ${d_B}$-dimension semantic representation. 
% %Each Gaussian basis represents a 3D Gaussian kernel and its corresponding semantic information. The shape of a Gaussian kernel can reflect a local object structure.
% %For each kernel, \js{details here: we use a visual self-attention to estimate the mean $\mu \mathbb \in {R}^3$ and covariance matrix $\Sigma \in  \mathbb{R}^{3\times 3}$, and a corresponding ${d_B}$-dimension semantic representation $\omega \in \mathbb{R}^{d_B}$. }
% Since the Gaussian bases are continuous in the location space, we can refine arbitrary location points with the Gaussian bases.
% \zx{We need some advantages of B in this paragraph}
% \zx{maybe also refer some splatting/RBF kernel methods to introduce what B is, how does B contain 3D structure information}

% \js{too details, use some \texttt{Atten} to obtain each parameters.}
% Given the context set $[\widetilde{X};\widetilde{Y}] \in \mathbb{R}^{H\times W \times (3+3+3)}$, a visual self-attention module first produces a $M\times D$ tokens with $M$ is the number of visual tokens and $D$ is the hidden dimension. The number of Gaussians we use equals to the number of tokens $M$. Then, we use one MLP to predict centers $\mu$, as well as the rotation $R$ and scaling $S$ matrices parameters for producing covariance matrix $\Sigma$, and one MLP to produce the latent representations $\omega$. The covariance matrix is obtained by $\Sigma^C = RSS^TR^T$.
% \begin{equation}
%     \Sigma = RSS^TR^T.
%     \label{eq:cov-matrix}
% \end{equation}
%where $R\in \mathbb{R}^{3\times3}$ is the rotation matrix, and $S \in \mathbb{R}^3$ is the scaling matrix. 



% The generative process of the NeRF function can be formulated in a probabilistic way:

% Different from previous works, we cast radiance field generalization as a 

% we the distribution over a
% single function $p( )$




% \begin{equation}
%     p({\bf{Y}}_{T}| {\bf{X}}_{T}, {\bf{\widetilde{X}}}_{C}, {\bf{\widetilde{Y}}}_{C}) =  \int p({\bf{Y}}_{T}| f_{NeRF}, {\bf{X}}_{T}) p(f_{NeRF} | {\bf{\widetilde{X}}}_{C}, {\bf{\widetilde{Y}}}_{C}) d f_{NeRF}
% \end{equation}


%To improve the generalization ability and achieve efficient inference on new scenes of Neural Field methods, we propose to learn to infer specific model parameters for each scene by Neural Processes.
%\zx{Advantages}
%In the following, we use NeRF generalization to illustrate the proposed method.  In NeRF generalization, a few views (one or two) of images $I_c$ and its corresponding camera pose $\mathbf{p}$ serve as the context information. 

%We represent each location in this scene by aggregating this Gaussian basis based on the radial basis function (RBF). 
%Then, based on the context information, the task is to estimate an implicit function that can be used to infer the target for unseen views.

% The proposed GP-INR framework assumes a space composed of a set of Gaussian basis that encode the spatial location and semantic information of a scene. Initially, GP-INR predicts a Gaussian basis $B_C$ from the observed context. Similar to the recent advances in implicit neural fields (e.g., Neurbf~\cite{chen2023neurbf}), at arbitrary new locations $X_T \in \mathbb{R}^{N\times P\times 3}$, we aggregate the Gaussians to form a new representation of this location $X'_C$ via the radial basis function (RBF). By averaging the representations both globally and locally along rays, we obtain a global latent $g_C$ and a ray-specific latent $r_C$ to modulate an MLP for predicting the implicit neural field. During training, with the target view and image $I_T$ also provided, we perform the same procedures for $I_T$ and align latent variables from both the context and target. For simplicity, in the following, we will use $I_C$ as an example to illustrate. 

% Introducing B
% ; 2). enrich the query points representation by aggregating the locality scene information

% \noindent \textbf{Stochastic Processes with Geometry Bases.} 
% By integrating the prior distribution with geometry bases ${\bf{B}}_{C}$ (Eq. \ref{eq: prior_f}) into the predictive distribution in Eq. \ref{eq: gp_w/o_B}, we reformulate the predictive distribution $p({\bf{Y}}_{T}| {\bf{X}}_{T}, {\bf{\widetilde{X}}}_{C}, {\bf{\widetilde{Y}}}_{C})$ as $p({\bf{{Y}}}_{T} | {\bf{{X}}}_{T}, {\bf{{B}}}_{C})$.
% The stochastic processes for NeRF with geometry bases are then derived as:
% % With the encoded geometry base ${\bf{B}}_{C}$, we reformulate the querying step as $p({\bf{{Y}}}_{T} | {\bf{{X}}}_{T}, {\bf{{B}}}_{C})$. By integrating the approximated prior with the bases Eq.~\ref{eq: prior_f} into the generative process in Eq.~\ref{eq: gp_w/o_B}, we derive the stochastic processes for NeRF with Geometry Bases as:
% \begin{equation}
%     % p({\bf{{Y}}}_{T} | {\bf{{X}}}_{T}, {\bf{\widetilde{X}}}_{C}, {\bf{\widetilde{Y}}}_{C}) = 
%     p({\bf{{Y}}}_{T} | {\bf{{X}}}_{T}, {\bf{{B}}}_{C}) = \int p({\bf{Y}}_{T}|f_{NeRF}, {\bf{X}}_{T}) p(f_{NeRF}| {\bf{X}}_{T}, {\bf{B}}_{C}) df_{NeRF}.
% \label{eq: gp_w_B}
% \end{equation}
% Inferred from the context-generated 3D information in ${\bf{{B}}}_{C}$, the NeRF function prior $p(f_{\text{NeRF}}|{\bf{X}}_{T}, {\bf{{B}}}_{C})$ reduces the impact of information loss in context views and becomes more suitable for the \zx{query step} $p({\bf{Y}}_{T}|f_{\text{NeRF}}, {\bf{X}}_{T})$ in 3D space.
% Moreover, since we consider the target location information ${\bf{X}}_{T}$ as well in $p(f_{\text{NeRF}}|{\bf{X}}_{T}, {\bf{{B}}}_{C})$, the NeRF function $f_{\text{NeRF}}$ is more generalizable to new views of new objects.

% This formulation provides a proxy between 2D context information and the 3D space by representing the 3D space with the continuous Gaussian basis. $p(f_{\text{NeRF}}|{\bf{X}}_{T}, {\bf{{B}}}_{C})$ is the conditional function distribution where we sample the modulation variable to modulate a shared MLP as the NeRF function, and $p({\bf{Y}}_{T}|f_{\text{NeRF}}, {\bf{X}}_{T})$ is NeRF \zx{rendering process}. We provide a new perspective to model the function distribution fully in 3D space. By doing so, we eliminate the information loss from 2D context to 3D location space.

% \begin{equation}
%     X'_T = \text{MLP}(\sum_i^{M} \varphi(x, \mu_i, \Sigma_i)\omega_i)
% \end{equation}s

% \begin{equation}
%     \varphi(x, \mu_i, \Sigma_i) = \exp (-(x-\mu_i)^T\Sigma_i^{-1}(x-\mu_i) /2)
% \end{equation}

%Hence, instead of directly using the 2D context set, we update the Gaussian basis (both shape and representation) in 3D space. Then, it is natural to use a continuous aggregation function (e.g. Gaussian radial basis function) to enrich the information at each location. By doing so, we eliminate the information loss from 2D context to 3D location space. 




% \subsection{Modeling of Geometry-aware Neural Processes}
% \subsection{Geometry-aware Neural Processes with Hierarchical Modulation}
% \label{sec: hierar}

% % To achieve the 
% To implement the inference of the NeRF function, we propose the Geometric Neural Process (\textbf{GeomNP}) for radiance field generalization, which efficiently adapt the functional distribution to new 3D objects. Specifically, \textbf{GeomNP} are constructed on a hierarchical Bayes framework. For 3D objects, the framework encodes object-specific and ray-specific latent variables in different levels, which helps the developed model querying the structure information from the Geometry bases $\mathbf{B}_C$. 

% \zx{why and how neural processes?}
% In this section, we present how to perform the \zx{rendering process} $p({\bf{Y}}_{T}| {\bf{X}}_{T},  {\bf{B}}_{C})$, including sampling a function from $p(f_{\text{NeRF}}|{\bf{X}}_{T}, {\bf{{B}}}_{C})$. 

% \noindent{{\textbf{Hierarchical Latent Variables.}}}
% The motivation of hierarchical latent variables in our model is the data structure of target inputs ${\bf{X}}_{T}$.
% As mentioned in Eq.~\ref{eq: definiation}, the target inputs can be obtained by randomly sampling ${\bf{x}}_{T}^{1:P}$ from each ray ${\widetilde{\bf{x}}}_{T}$. Thus, target input ${\bf{X}}_{T}$ consists of $N \times P$ location points $\{{\bf{x}}_{T}^{1:P, n}\}_{n=1}^{N}$ given $N$ rays. 
% Since the rendering of each ray is independently \cite{martin2021nerf}, we reformulate the predictive distribution for NeRF in a ray-specific manner:
% \begin{equation}
%     p({\bf{Y}}_{T}| {\bf{X}}_{T},  {\bf{B}}_{C})  = \prod_{n=1}^{N} p({\bf{y}}_{T}^{1:P, n}| {\bf{x}}_{T}^{1:P, n},  {\bf{B}}_{C}).
% \label{eq: predictive_distribution_ray_specific}
% \end{equation}
% Based on the hierarchical data structure of the target inputs, we design a hierarchical latent variable model. In the proposed model, object-specific latent variables ${\bf{g}}$ encode the entire 3D object information; each ${\bf{g}}$ corresponds $N$ individual ray-specific latent variables $\{{\bf{r}}^{n}\}_{n=1}^{N}$.

% \begin{wrapfigure}{r}{0.5\textwidth}
% \vspace{-5mm}
% \centering
% \centerline{
% \includegraphics[width=0.45\columnwidth]{Figures/graphical_model.png} 
% } 
% \vspace{-2mm}
% \caption{\textbf{Graphical model for the proposed geometric neural processes.}
% }
% \vspace{-2mm}
% \label{fig: graphical_model}
% \end{wrapfigure}
%${\mathbf{z}_o}$

% \subsection{Inferring with Geometric Bases and Modulation}

% \str{The following paragraph is not very clear.}

In the modeling of {\method{}}, the prior distribution of each hierarchical latent variable is conditioned on the geometric bases and target input. 
%\textcolor{blue}{To infer each latent variable, we first integrate the geometric bases and each target input, yielding a location representation specific to the target input. The location representation has access to relevant locality information from the geometry bases, \textit{i.e.}, $<{\bf{x}}_{T}^{\mathbf{r}, n}, {\bf{B}}_C >$. }
% For generalization, we need to infer latent variables that are specific to the target input. 
% To this end, 
We first represent each target location by integrating the geometric bases, \textit{i.e.}, $<{\bf{x}}_{T}^{n}, {\bf{B}}_C >$, which aggregates the relevant locality and semantic information for the given input. 
Since ${\bf{B}}_{C}$ contains $M$ Gaussians, we employ a Gaussian radial basis function in \cref{eq:rbf_agg} between each target input ${\bf{x}}_{T}^{ n}$ and each geometric basis ${\bf{b}}_i$ to aggregate the structural and semantic information to the 3D location representation. Thus, we obtain the 3D location representation as follows:
\begin{equation}
\label{eq:rbf_agg}
    <{\bf{x}}_{T}^{n}, {\bf{B}}_C > = \texttt{MLP}\Big[\sum_i^{M} \exp (-\frac{1}{2}({\bf{x}}_{T}^{n}-\mu_i)^T\Sigma_i^{-1}({\bf{x}}_{T}^{n}-\mu_i) ) \cdot \omega_i\Big],
\end{equation} 
where $\texttt{MLP}[\cdot]$ is a learnable neural network.
With the location representation $<{\bf{x}}_{T}^{n}, {\bf{B}}_C >$, we next infer each latent variable hierarchically, in object and ray levels. 

\noindent {\textbf{Object-specific Latent Variable.}} The distribution of the object-specific latent variable ${\bf{z}}_o$ is obtained by aggregating all location representations:
\begin{equation}
    [\mu_{{o}}, \sigma_{{o}}] 
    = \texttt{MLP}\Big[\frac{1}{N \times P}\sum_{n = 1}^{N}\sum_{\mathbf{r}}
    <{\bf{x}}_{T}^{n}, {\bf{B}}_C >\Big],
\end{equation} 
where we assume $p({\bf{z}}_o | {\bf{B}}_C,  {\bf{X}}_T)$ is a standard Gaussian distribution and generate its mean $\mu_{o}$ and variance $\sigma_{o}$ by a ~\texttt{MLP}. 
Thus, our model captures objective-specific uncertainty in the NeRF function.


\noindent {\textbf{Ray-specific Latent Variable.}} 
% By ray-specific latent variable, the object-specific is expected to capture the local details.
To generate the distribution of the ray-specific latent variable, we first average the location representations ray-wisely. 
We then obtain the ray-specific latent variable by aggregating the averaged location representation and the object latent variable through a lightweight transformer. We formulate the inference of the ray-specific latent variable as:
\begin{equation}
    [\mu_{{r}}, \sigma_{{r}}] = \texttt{Transformer} \Big[\texttt{MLP}[\frac{1}{P}\sum_{\mathbf{r}}
    <{\bf{x}}_{T}^{n}, {\bf{B}}_C >]; \hat{{\bf{z}}}_o \Big],
\end{equation}
where $\hat{{\bf{z}}}_o$ is a sample from the prior distribution $p({\bf{z}}_o | {\bf{X}}_T, {\bf{B}}_C)$. 
Similar to the object-specific latent variable, we also assume the distribution $p({\bf{z}}_r^n| {\bf{z}}_o,  {\bf{x}}_{T}^{\mathbf{r}, n}, {\bf{B}}_C)$ is a mean-field Gaussian distribution with the mean $\mu_{{r}}$ and variance $\sigma_{{r}}$. We provide more details of the latent variables in Appendix~\ref{supp:latent-variables}.

% The detail of the used Transformer is given in the Appendix. In summary, our latent variable is in a hierarchical structure. 
% The location-specific latent variable $r_C$ is to modulate the NeRF function specific to the queried ray (or pixel location in 2D image). We infer a ray-specific variable $r_c$ by first aggregating the feature ray-wisely:
% \begin{equation}
%     r'_c = \text{MLP}(\frac{1}{ P}\sum_j^{P}(\bf{X}'_T)[j])
% \end{equation}


%Then, we perform a hierarchical formulation by incorporating the global variable to obtain the ray-specific variable. This enables us to design the latent variable in a hierarchical structure, which is rational as it learns the scene coarse-to-fine. 

% Then, $r_c$ is obtained by feeding $r'_c$ together with the sampled $\hat{g}_c$ into a Transformer: 
% \begin{equation}
%     r_c = [\mu^c_r, \sigma^c_r] = \text{Transformer}([\hat{g}_c; \hat{r}_C]),
% \end{equation}
% where $\mu^r_g$ and $\sigma^r_g$ are the mean and variance of global variables, respectively. The detail of the used Transformer is given in the Appendix. In summary, our latent variable is in a hierarchical structure. 

\noindent  \textbf{NeRF Function Modulation.}
With the hierarchical latent variables $\{{\bf{z}}_o, {\bf{z}}_r^n\}$, we modulate a neural network for a 3D object in both object-specific and ray-specific levels.  Specifically, the modulation of each layer is achieved by scaling its weight matrix with a style vector~\citep{guo2023versatile}. 
The object-specific latent variable ${\bf{z}}_o$ and ray-specific latent variable ${\bf{z}}_r^n$ are taken as style vectors of the low-level layers and high-level layers, respectively. The prediction distribution $p({\bf{Y}}_{T}| {\bf{X}}_{T}, {\bf{B}}_{C})$ are finally obtained by passing each location representation through the modulated neural network for the NeRF function. 
More details are provided in Appendix~\ref{supp:modulate}. 
% The modulated MLP layer used in our paper is similar to the style \textit{modulation} in ~\cite{guo2023versatile}. Essentially, we predict a style vector $s\in \mathbb{R}^{d_{in}}$ to multiply or scale the weight matrix of an MLP, $W \in \mathbb{R}^{d_{in} \times d_{out}}$.


% \subsection{Inference of Geometry-aware Neural Processes}
% \label{sec: elbo}
% ELBO

% \noindent{\textbf{Variational Posteriors with the Geometry Bases.}} Solving \textbf{GANPs} with Eq.~\ref{eq:ganp-model} involves estimating the true posterior, $p({\bf{g}},  \{{\bf{r}}_i\}_{r=1}^{N_{ray}} | {\widetilde{\bf{X}}}_T, {\widetilde{\bf{Y}}}_T)$ which is intractable. Hence, we introduce a variational posterior distribution, which can be factorized as follows:
% \begin{equation}
% p({\bf{g}},  \{{\bf{r}}_i\}_{r=1}^{N_{ray}} | {\widetilde{\bf{X}}}_T, {\widetilde{\bf{Y}}}_T) \approx q_{\theta, \phi}({\bf{g}},  \{{\bf{r}}_i\}_{r=1}^{N_{ray}} | {\bf{X}}_T, {\bf{B}}_T),
% \end{equation}

\subsection{Empirical Objective}
\label{sec: object}

\noindent{\textbf{Evidence Lower Bound.}} 
To optimize the proposed \method{},
we apply variational inference~\citep{garnelo2018neural} and derive the evidence lower bound (ELBO) as:
\begin{equation}
\begin{aligned}
% \mathcal{L}_{\text{ELBO}}
& \log   p({\bf{Y}}_{T}| {\bf{X}}_{T}, {\bf{B}}_{C})
\geq \\
&\mathbb{E}_{q({\bf{z}}_o | {\bf{B}}_T,  {\bf{X}}_T)}  \Big\{  \sum_{n=1}^{N}  \mathbb{E}_{q({\bf{z}}_r^n| {\bf{z}}_o,  {\bf{x}}_{T}^{\mathbf{r}, n}, {\bf{B}_T})} \log p({\bf{y}}_{T}^{{\mathbf{r}}, n}| {\bf{x}}_{T}^{{\mathbf{r}}, n}, {\bf{z}}_o, {\bf{z}}_r^n) \\
&- D_{\text{KL}}[q({\bf{z}}_r^n| {\bf{z}}_o,  {\bf{x}}_{T}^{{\mathbf{r}}, n}, {{\bf{B}}_T}) || p({\bf{z}}_r^n| {\bf{z}}_o,  {\bf{x}}_{T}^{{\mathbf{r}}, n}, {{\bf{B}}_C}) ] \Big\} \\
& - D_{\text{KL}}[q({\bf{z}}_o | {\bf{B}}_T,  {\bf{X}}_T) || p({\bf{z}}_o | {\bf{B}}_C,  {\bf{X}}_T)], \\
\end{aligned}
\end{equation}
where $q_{\theta, \phi}({\bf{z}}_o,  \{{\bf{z}}_r^i\}_{i=1}^{N} | {\bf{X}}_T, {\bf{B}}_T) = \Pi_{i=1}^{N}q({\bf{z}}_r^n| {\bf{z}}_o,  {\bf{x}}_{T}^{{\mathbf{r}}, n}, {{\bf{B}}_T}) q({\bf{z}}_o | {\bf{B}}_T,  {\bf{X}}_T)$ is the involved variational posterior for the hierarchical latent variables.  ${\bf{B}}_T$ is the geometric bases constructed from the target sets $\{{\bf{\widetilde{X}}}_{T}, {\bf{\widetilde{Y}}}_{T}\}$, which are only accessible during training. 
The variational posteriors are inferred from the target sets during training, which introduces more information on the object. 
The prior distributions are supervised by the variational posterior using Kullback–Leibler (KL) divergence, learning to model more object information with limited context data and generalize to new scenes. Detailed derivations are provided in Appendix~\ref{supp:elbo}.

% \noindent{\textbf{Empirical Objective.}} 
For the geometric bases $\mathbf{B}_C$, we regularize the spatial shape of the context geometric bases to be closer to that of the target one $\mathbf{B}_T$ by introducing a KL divergence. 
Therefore, given the above ELBO, our objective function consists of three parts: a reconstruction loss (MSE loss), KL divergences for hierarchical latent variables, and a KL divergence for the geometric bases. 
%constraint for matching the two sets of Gaussian basis to ensure the basis obtained from context is as close to the one from the target. 
The empirical objective for the proposed \method{} is formulated as:
\begin{equation}
\begin{aligned}
& \mathcal{L}_{\text{\method{}}}  =  ||y - y'||^2_2 + \alpha \cdot \big(  D_{\text{KL}} [p(\mathbf{z}_o|{\bf{B}}_C)|q(\mathbf{z}_o|{\bf{B}}_T)] \\
    & + D_{\text{KL}}[p(\mathbf{z}_r|\mathbf{z}_o,{\bf{B}}_C)|q(\mathbf{z}_r|\mathbf{z}_o,{\bf{B}}_T)] \big) + \beta \cdot D_{\text{KL}}[{\bf{B}}_C, {\bf{B}}_T],
\end{aligned}
\end{equation}
where $y'$ is the prediction. $\alpha$ and $\beta$ are hyperparameters to balance the three parts of the objective. The KL divergence on ${\bf{B}}_C, {\bf{B}}_T$ is to align the spatial location and the shape of two sets of bases. 

% +++++++++++++

% \noindent 
% \textbf{Neural Fields.}
% % \zx{First introduce NeRF (with problem definition and notations) and its disadvantages of inefficient inference, then say what we will do to avoid this problem by NP?} 


% In general, training a NeRF requires overfitting each scene, which is time-consuming. Hence, how to leverage the observation for generalization on the new scene remains a problem. As each INR is a function of a data sample, the problem can be viewed as estimating the distribution of functions. This motivates us to use the Neural Process (NP) in this problem. 
 

% \noindent {\textbf{Neural Process.}} Neural Process~\cite{garnelo2018neural} parameterizes the distribution of functions. Given the context set comprising of the observation $X$ and the corresponding labels $Y$, $D_C = (X_C, Y_C) := (\mathbf{x}_i, \mathbf{y}_{i})_{i\in C} $. The aim of NP is to learn a mapping function from the target points $X_T$ to the target labels $Y_T$,  $D_T = (X_T, Y_T) := (\mathbf{x}_i, \mathbf{y}_{i})_{i\in T} $. The conditional distribution of target points is:
% \begin{equation}
%     p_{\phi}(Y_T|X_T,D_C) = \prod_{\mathbf{x}, \mathbf{y}\in D_{T}} \mathcal{N}(\mathbf{y};\mu_{\mathbf{y}}(\mathbf{x},D_c), \sigma^2_{\mathbf{y}}(\mathbf{x},D_c)).
% \end{equation}

% \begin{equation}
%     p_{\phi}(Y_T|X_T, \mathbf{z}) = \prod_{\mathbf{x}, \mathbf{y}\in D_{T}} \mathcal{N}(\mathbf{y};\mu_{\mathbf{y}}(\mathbf{z}, X_T,D_c), \sigma^2_{\mathbf{y}}(\mathbf{z}, X_T,D_c)),
% \end{equation}
% where $\mathbf{z} \sim p_{\theta}(\mathbf{z|X_T,D_C})$. Using NP can efficiently leverage the limited context/observation to infer the function of INR for the target from a probabilistic perspective. It also can incorporate uncertainty estimation for the unseen view. This is reasonable as the value in the unseen target location should not be deterministic. 

%\zx{advantages of NP? incorporating uncertainty for limited context information, which is suitable for reconstruction tasks?}







% \begin{align}
%     p(\mathbf{y}|\mathbf{x}, I) & = \int_{g} \int_{r}  p(\mathbf{y}, g, r |\mathbf{x}, I) \mathrm{d}g  \mathrm{d}r \\
%     &=  \int_{g} \int_{r}  p(\mathbf{y}| g, r) p(g, r | \mathbf{x}, I) \mathrm{d}g  \mathrm{d}r \\
%     % &= \int_{g} \int_{r} \int_{B} p_{\theta_1}(\mathbf{y}| g, r) p_{\theta_2}(g, r | B) p_{\theta_3}(B|\mathbf{x}, I) \mathrm{d}g  \mathrm{d}r \mathrm{d}B 
%     &= \int_{g} \int_{r}  p_{\theta_1}(\mathbf{y}| g, r) p_{\theta_2}(g, r | B) p_{\theta_3}(B|\mathbf{x}, I) \mathrm{d}g  \mathrm{d}r 
% \end{align}


%\subsection{Gaussian Basis for INR}
%As shown in Fig.~\ref{fig:framework}, 


%\subsection{Hierarchical Neural Process for INR}









\section{Experiments}
\section{Experiments}
\label{sec:experiments}
The experiments are designed to address two key research questions.
First, \textbf{RQ1} evaluates whether the average $L_2$-norm of the counterfactual perturbation vectors ($\overline{||\perturb||}$) decreases as the model overfits the data, thereby providing further empirical validation for our hypothesis.
Second, \textbf{RQ2} evaluates the ability of the proposed counterfactual regularized loss, as defined in (\ref{eq:regularized_loss2}), to mitigate overfitting when compared to existing regularization techniques.

% The experiments are designed to address three key research questions. First, \textbf{RQ1} investigates whether the mean perturbation vector norm decreases as the model overfits the data, aiming to further validate our intuition. Second, \textbf{RQ2} explores whether the mean perturbation vector norm can be effectively leveraged as a regularization term during training, offering insights into its potential role in mitigating overfitting. Finally, \textbf{RQ3} examines whether our counterfactual regularizer enables the model to achieve superior performance compared to existing regularization methods, thus highlighting its practical advantage.

\subsection{Experimental Setup}
\textbf{\textit{Datasets, Models, and Tasks.}}
The experiments are conducted on three datasets: \textit{Water Potability}~\cite{kadiwal2020waterpotability}, \textit{Phomene}~\cite{phomene}, and \textit{CIFAR-10}~\cite{krizhevsky2009learning}. For \textit{Water Potability} and \textit{Phomene}, we randomly select $80\%$ of the samples for the training set, and the remaining $20\%$ for the test set, \textit{CIFAR-10} comes already split. Furthermore, we consider the following models: Logistic Regression, Multi-Layer Perceptron (MLP) with 100 and 30 neurons on each hidden layer, and PreactResNet-18~\cite{he2016cvecvv} as a Convolutional Neural Network (CNN) architecture.
We focus on binary classification tasks and leave the extension to multiclass scenarios for future work. However, for datasets that are inherently multiclass, we transform the problem into a binary classification task by selecting two classes, aligning with our assumption.

\smallskip
\noindent\textbf{\textit{Evaluation Measures.}} To characterize the degree of overfitting, we use the test loss, as it serves as a reliable indicator of the model's generalization capability to unseen data. Additionally, we evaluate the predictive performance of each model using the test accuracy.

\smallskip
\noindent\textbf{\textit{Baselines.}} We compare CF-Reg with the following regularization techniques: L1 (``Lasso''), L2 (``Ridge''), and Dropout.

\smallskip
\noindent\textbf{\textit{Configurations.}}
For each model, we adopt specific configurations as follows.
\begin{itemize}
\item \textit{Logistic Regression:} To induce overfitting in the model, we artificially increase the dimensionality of the data beyond the number of training samples by applying a polynomial feature expansion. This approach ensures that the model has enough capacity to overfit the training data, allowing us to analyze the impact of our counterfactual regularizer. The degree of the polynomial is chosen as the smallest degree that makes the number of features greater than the number of data.
\item \textit{Neural Networks (MLP and CNN):} To take advantage of the closed-form solution for computing the optimal perturbation vector as defined in (\ref{eq:opt-delta}), we use a local linear approximation of the neural network models. Hence, given an instance $\inst_i$, we consider the (optimal) counterfactual not with respect to $\model$ but with respect to:
\begin{equation}
\label{eq:taylor}
    \model^{lin}(\inst) = \model(\inst_i) + \nabla_{\inst}\model(\inst_i)(\inst - \inst_i),
\end{equation}
where $\model^{lin}$ represents the first-order Taylor approximation of $\model$ at $\inst_i$.
Note that this step is unnecessary for Logistic Regression, as it is inherently a linear model.
\end{itemize}

\smallskip
\noindent \textbf{\textit{Implementation Details.}} We run all experiments on a machine equipped with an AMD Ryzen 9 7900 12-Core Processor and an NVIDIA GeForce RTX 4090 GPU. Our implementation is based on the PyTorch Lightning framework. We use stochastic gradient descent as the optimizer with a learning rate of $\eta = 0.001$ and no weight decay. We use a batch size of $128$. The training and test steps are conducted for $6000$ epochs on the \textit{Water Potability} and \textit{Phoneme} datasets, while for the \textit{CIFAR-10} dataset, they are performed for $200$ epochs.
Finally, the contribution $w_i^{\varepsilon}$ of each training point $\inst_i$ is uniformly set as $w_i^{\varepsilon} = 1~\forall i\in \{1,\ldots,m\}$.

The source code implementation for our experiments is available at the following GitHub repository: \url{https://anonymous.4open.science/r/COCE-80B4/README.md} 

\subsection{RQ1: Counterfactual Perturbation vs. Overfitting}
To address \textbf{RQ1}, we analyze the relationship between the test loss and the average $L_2$-norm of the counterfactual perturbation vectors ($\overline{||\perturb||}$) over training epochs.

In particular, Figure~\ref{fig:delta_loss_epochs} depicts the evolution of $\overline{||\perturb||}$ alongside the test loss for an MLP trained \textit{without} regularization on the \textit{Water Potability} dataset. 
\begin{figure}[ht]
    \centering
    \includegraphics[width=0.85\linewidth]{img/delta_loss_epochs.png}
    \caption{The average counterfactual perturbation vector $\overline{||\perturb||}$ (left $y$-axis) and the cross-entropy test loss (right $y$-axis) over training epochs ($x$-axis) for an MLP trained on the \textit{Water Potability} dataset \textit{without} regularization.}
    \label{fig:delta_loss_epochs}
\end{figure}

The plot shows a clear trend as the model starts to overfit the data (evidenced by an increase in test loss). 
Notably, $\overline{||\perturb||}$ begins to decrease, which aligns with the hypothesis that the average distance to the optimal counterfactual example gets smaller as the model's decision boundary becomes increasingly adherent to the training data.

It is worth noting that this trend is heavily influenced by the choice of the counterfactual generator model. In particular, the relationship between $\overline{||\perturb||}$ and the degree of overfitting may become even more pronounced when leveraging more accurate counterfactual generators. However, these models often come at the cost of higher computational complexity, and their exploration is left to future work.

Nonetheless, we expect that $\overline{||\perturb||}$ will eventually stabilize at a plateau, as the average $L_2$-norm of the optimal counterfactual perturbations cannot vanish to zero.

% Additionally, the choice of employing the score-based counterfactual explanation framework to generate counterfactuals was driven to promote computational efficiency.

% Future enhancements to the framework may involve adopting models capable of generating more precise counterfactuals. While such approaches may yield to performance improvements, they are likely to come at the cost of increased computational complexity.


\subsection{RQ2: Counterfactual Regularization Performance}
To answer \textbf{RQ2}, we evaluate the effectiveness of the proposed counterfactual regularization (CF-Reg) by comparing its performance against existing baselines: unregularized training loss (No-Reg), L1 regularization (L1-Reg), L2 regularization (L2-Reg), and Dropout.
Specifically, for each model and dataset combination, Table~\ref{tab:regularization_comparison} presents the mean value and standard deviation of test accuracy achieved by each method across 5 random initialization. 

The table illustrates that our regularization technique consistently delivers better results than existing methods across all evaluated scenarios, except for one case -- i.e., Logistic Regression on the \textit{Phomene} dataset. 
However, this setting exhibits an unusual pattern, as the highest model accuracy is achieved without any regularization. Even in this case, CF-Reg still surpasses other regularization baselines.

From the results above, we derive the following key insights. First, CF-Reg proves to be effective across various model types, ranging from simple linear models (Logistic Regression) to deep architectures like MLPs and CNNs, and across diverse datasets, including both tabular and image data. 
Second, CF-Reg's strong performance on the \textit{Water} dataset with Logistic Regression suggests that its benefits may be more pronounced when applied to simpler models. However, the unexpected outcome on the \textit{Phoneme} dataset calls for further investigation into this phenomenon.


\begin{table*}[h!]
    \centering
    \caption{Mean value and standard deviation of test accuracy across 5 random initializations for different model, dataset, and regularization method. The best results are highlighted in \textbf{bold}.}
    \label{tab:regularization_comparison}
    \begin{tabular}{|c|c|c|c|c|c|c|}
        \hline
        \textbf{Model} & \textbf{Dataset} & \textbf{No-Reg} & \textbf{L1-Reg} & \textbf{L2-Reg} & \textbf{Dropout} & \textbf{CF-Reg (ours)} \\ \hline
        Logistic Regression   & \textit{Water}   & $0.6595 \pm 0.0038$   & $0.6729 \pm 0.0056$   & $0.6756 \pm 0.0046$  & N/A    & $\mathbf{0.6918 \pm 0.0036}$                     \\ \hline
        MLP   & \textit{Water}   & $0.6756 \pm 0.0042$   & $0.6790 \pm 0.0058$   & $0.6790 \pm 0.0023$  & $0.6750 \pm 0.0036$    & $\mathbf{0.6802 \pm 0.0046}$                    \\ \hline
%        MLP   & \textit{Adult}   & $0.8404 \pm 0.0010$   & $\mathbf{0.8495 \pm 0.0007}$   & $0.8489 \pm 0.0014$  & $\mathbf{0.8495 \pm 0.0016}$     & $0.8449 \pm 0.0019$                    \\ \hline
        Logistic Regression   & \textit{Phomene}   & $\mathbf{0.8148 \pm 0.0020}$   & $0.8041 \pm 0.0028$   & $0.7835 \pm 0.0176$  & N/A    & $0.8098 \pm 0.0055$                     \\ \hline
        MLP   & \textit{Phomene}   & $0.8677 \pm 0.0033$   & $0.8374 \pm 0.0080$   & $0.8673 \pm 0.0045$  & $0.8672 \pm 0.0042$     & $\mathbf{0.8718 \pm 0.0040}$                    \\ \hline
        CNN   & \textit{CIFAR-10} & $0.6670 \pm 0.0233$   & $0.6229 \pm 0.0850$   & $0.7348 \pm 0.0365$   & N/A    & $\mathbf{0.7427 \pm 0.0571}$                     \\ \hline
    \end{tabular}
\end{table*}

\begin{table*}[htb!]
    \centering
    \caption{Hyperparameter configurations utilized for the generation of Table \ref{tab:regularization_comparison}. For our regularization the hyperparameters are reported as $\mathbf{\alpha/\beta}$.}
    \label{tab:performance_parameters}
    \begin{tabular}{|c|c|c|c|c|c|c|}
        \hline
        \textbf{Model} & \textbf{Dataset} & \textbf{No-Reg} & \textbf{L1-Reg} & \textbf{L2-Reg} & \textbf{Dropout} & \textbf{CF-Reg (ours)} \\ \hline
        Logistic Regression   & \textit{Water}   & N/A   & $0.0093$   & $0.6927$  & N/A    & $0.3791/1.0355$                     \\ \hline
        MLP   & \textit{Water}   & N/A   & $0.0007$   & $0.0022$  & $0.0002$    & $0.2567/1.9775$                    \\ \hline
        Logistic Regression   &
        \textit{Phomene}   & N/A   & $0.0097$   & $0.7979$  & N/A    & $0.0571/1.8516$                     \\ \hline
        MLP   & \textit{Phomene}   & N/A   & $0.0007$   & $4.24\cdot10^{-5}$  & $0.0015$    & $0.0516/2.2700$                    \\ \hline
       % MLP   & \textit{Adult}   & N/A   & $0.0018$   & $0.0018$  & $0.0601$     & $0.0764/2.2068$                    \\ \hline
        CNN   & \textit{CIFAR-10} & N/A   & $0.0050$   & $0.0864$ & N/A    & $0.3018/
        2.1502$                     \\ \hline
    \end{tabular}
\end{table*}

\begin{table*}[htb!]
    \centering
    \caption{Mean value and standard deviation of training time across 5 different runs. The reported time (in seconds) corresponds to the generation of each entry in Table \ref{tab:regularization_comparison}. Times are }
    \label{tab:times}
    \begin{tabular}{|c|c|c|c|c|c|c|}
        \hline
        \textbf{Model} & \textbf{Dataset} & \textbf{No-Reg} & \textbf{L1-Reg} & \textbf{L2-Reg} & \textbf{Dropout} & \textbf{CF-Reg (ours)} \\ \hline
        Logistic Regression   & \textit{Water}   & $222.98 \pm 1.07$   & $239.94 \pm 2.59$   & $241.60 \pm 1.88$  & N/A    & $251.50 \pm 1.93$                     \\ \hline
        MLP   & \textit{Water}   & $225.71 \pm 3.85$   & $250.13 \pm 4.44$   & $255.78 \pm 2.38$  & $237.83 \pm 3.45$    & $266.48 \pm 3.46$                    \\ \hline
        Logistic Regression   & \textit{Phomene}   & $266.39 \pm 0.82$ & $367.52 \pm 6.85$   & $361.69 \pm 4.04$  & N/A   & $310.48 \pm 0.76$                    \\ \hline
        MLP   &
        \textit{Phomene} & $335.62 \pm 1.77$   & $390.86 \pm 2.11$   & $393.96 \pm 1.95$ & $363.51 \pm 5.07$    & $403.14 \pm 1.92$                     \\ \hline
       % MLP   & \textit{Adult}   & N/A   & $0.0018$   & $0.0018$  & $0.0601$     & $0.0764/2.2068$                    \\ \hline
        CNN   & \textit{CIFAR-10} & $370.09 \pm 0.18$   & $395.71 \pm 0.55$   & $401.38 \pm 0.16$ & N/A    & $1287.8 \pm 0.26$                     \\ \hline
    \end{tabular}
\end{table*}

\subsection{Feasibility of our Method}
A crucial requirement for any regularization technique is that it should impose minimal impact on the overall training process.
In this respect, CF-Reg introduces an overhead that depends on the time required to find the optimal counterfactual example for each training instance. 
As such, the more sophisticated the counterfactual generator model probed during training the higher would be the time required. However, a more advanced counterfactual generator might provide a more effective regularization. We discuss this trade-off in more details in Section~\ref{sec:discussion}.

Table~\ref{tab:times} presents the average training time ($\pm$ standard deviation) for each model and dataset combination listed in Table~\ref{tab:regularization_comparison}.
We can observe that the higher accuracy achieved by CF-Reg using the score-based counterfactual generator comes with only minimal overhead. However, when applied to deep neural networks with many hidden layers, such as \textit{PreactResNet-18}, the forward derivative computation required for the linearization of the network introduces a more noticeable computational cost, explaining the longer training times in the table.

\subsection{Hyperparameter Sensitivity Analysis}
The proposed counterfactual regularization technique relies on two key hyperparameters: $\alpha$ and $\beta$. The former is intrinsic to the loss formulation defined in (\ref{eq:cf-train}), while the latter is closely tied to the choice of the score-based counterfactual explanation method used.

Figure~\ref{fig:test_alpha_beta} illustrates how the test accuracy of an MLP trained on the \textit{Water Potability} dataset changes for different combinations of $\alpha$ and $\beta$.

\begin{figure}[ht]
    \centering
    \includegraphics[width=0.85\linewidth]{img/test_acc_alpha_beta.png}
    \caption{The test accuracy of an MLP trained on the \textit{Water Potability} dataset, evaluated while varying the weight of our counterfactual regularizer ($\alpha$) for different values of $\beta$.}
    \label{fig:test_alpha_beta}
\end{figure}

We observe that, for a fixed $\beta$, increasing the weight of our counterfactual regularizer ($\alpha$) can slightly improve test accuracy until a sudden drop is noticed for $\alpha > 0.1$.
This behavior was expected, as the impact of our penalty, like any regularization term, can be disruptive if not properly controlled.

Moreover, this finding further demonstrates that our regularization method, CF-Reg, is inherently data-driven. Therefore, it requires specific fine-tuning based on the combination of the model and dataset at hand.

% \section{Experiments}
% \section{Experiments}
\label{sec:experiments}


We measure IBURD's efficacy with quantitative evaluations using $10$ classes from the TrashCan~\cite{hong2020trashcan} dataset. We also perform robotic experiments with the LoCO AUV~\cite{loco_paper_2020} to evaluate our approach on visually similar but unseen environments (\eg pool and sea).
In both cases, we start by collecting some source object images and background images that are representative of the environment we selected to evaluate a detector. 
We then use IBURD to blend the object images into the selected backgrounds and train a detector using this synthetic data generated from the pipeline.

\begin{table}[t]

\caption{Class distribution of source object image used for quantitative evaluation.}
\label{data_dist_quant}
\begin{center}
\begin{tabular}{ccc}
\hline
\toprule
    Class name & No. of source images using Dall-E & No. of source images from TrashCan\\
    \midrule
     animal\_starfish & $3$ & $4$ \\
     trash\_bag & $3$ & $22$\\
     animal\_shell & $3$ & $2$\\
     animal\_crab & $2$ & $5$ \\
     trash\_pipe & $3$ & $4$\\
     trash\_bottle & $3$ &$5$\\
     trash\_snack\_wrapper & $3$ & $3$\\
     trash\_can & $3$ & $3$ \\
     trash\_cup & $3$ & $2$\\
     trash\_container & $3$ & $11$\\
     
\bottomrule
\end{tabular}
\end{center}
\end{table}


\subsection{Data Collection}
\subsubsection{Quantitative Experiments}
To evaluate a detector's performance on the TrashCan dataset, we select $10$ background images within the dataset. 
We choose images that are representative of the TrashCan dataset and contain as few objects in each image as possible. 
We collect source images both semi-automatically ($29$ images from Dall-E) and manually ($68$ images from TrashCan). 
Both Dall-E and TrashCan images have the same $10$ object classes  (Table~\ref{data_dist_quant}).
Unlike the source images from Dall-E, the distribution of the ones collected from TrashCan is non-uniform across the classes since TrashCan has more data from common debris items (\eg bottle) compared to other classes (\eg starfish).


\subsubsection{Robotic Experiments}
We collect $47$ source images containing $7$ classes of objects commonly found in marine debris (Table~\ref{data_dist}). We have a smaller number of classes compared to the quantitative experiments since not all $10$ classes of objects can be easily placed underwater (\eg pipe and crab).
We then manually annotate the images, providing class labels, bounding boxes, and segmentation information.
We use $2$ types of background images for blending: sea and pool. 
We collect $7$ images captured at different locations from the pool used for experiments and add $3$ pool images from online sources to add more variety. 
For the sea background, we use $10$ background images sourced from the Internet. 


\subsection{Image Blending and Data Generation} 
\label{subsec:data_gen}
\subsubsection{Quantitative experiments}
\label{subsec:blend_quant}
To generate a diverse dataset with $512\times512$ pixel-size images, we use $4$ different rotations ($0^{\circ}$, $90^{\circ}$, $180^{\circ}$, $270^{\circ}$) for the source object images. 
For Dall-E generated images, we use $4$ different source image sizes,
 ($96\times96$, $128\times128$, $192\times192$, $256\times256$ pixels). 
With source objects from TrashCan, we only use $2$ source image sizes ($192\times192$, $256\times256$ pixels).
This is because TrashCan has a mixture of close-up and long-range views of objects, and using the smaller scales might produce blended objects that are not visible to the human eye, making them impossible to annotate. Close-up version of objects generated by Dall-E enables the use of smaller-scale images.


For both cases, the image is split into a $2\times2$ grid to avoid overlap with previously blended objects. We determine the object location by randomly selecting one section within the grid on the background image plane. Using the grid we blend up to $4$ objects in the same background. This way we create $3$ different datasets: 1) TrashCan training data with $2$k images generated using Dall-E source objects (\textit{T+D$2$k}), 2) TrashCan training data with $10$k images generated using Dall-E source objects (\textit{T+D$10$k}), and 3) TrashCan training data with $10$k images generated using source objects from TrashCan (\textit{T+T$10$k}). 
For these $3$ cases, we also generate images using just Poisson image editing (\ie first pass only) to train a detector on Poisson-blended data alone. This makes it possible to assess the effect of adding the second pass in the IBURD pipeline on detector performance compared to the same detector trained on basic Poisson-blended data. For both blending techniques (Poisson Image Editing and IBURD), the images used are the same in terms of objects, size, orientation, and location (\eg Fig.~\ref{fig:compare}). We use IBURD generated data for training and evaluate the performance using real-world validation data from TrashCan for all cases.

\begin{table}[t]
\footnotesize
\centering
\setlength{\tabcolsep}{2.8pt}
\caption{\small{Description of eligible data from selected datasets.}}
\label{tab:data_source}
\begin{tabular}{c|ccccccccc|c|c}
\toprule
 & \multicolumn{10}{c|}{KITTI} & \multicolumn{1}{c}{nuScenes}\\
 & $00$ & $01$ & $02$ & $03$ & $04$ & $05$ & $06$ & $18$ & $20$ & sum & $61$ \\
\midrule
\midrule
 eligible objects & $2$ & - & - & $2$ & - & $1$ & - & $2$ & $8$ & $15$ & $1$ \\
 \midrule
 testing instances & $144$ & - & - & $67$ & - & $81$ & - & $47$ & $183$ & $522$ & $149$ \\
\bottomrule
\end{tabular}
\vspace{-4mm}
\end{table}

\subsubsection{Robotic Experiments}
Similar to the case of quantitative experiments, we generate datasets with $4$ rotations and $4$ sizes for the $47$ source object images and create images with up to $4$ objects blended in the same background.
We determine the object location by randomly selecting one section within the grid on the background image plane. 
For multi-object blending, we divide the background ($512\times512$ pixels) into $2\times2$ grid considering the maximum size of source images (\ie $256\times256$ pixels). 
For the single object case, the image plane is divided based on the current source image size (\ie $4\times4$ grid for $96\times96$ and $128\times128$ pixel-size images, $2\times2$ grid for $192\times192$ and $256\times256$ pixel-size images). 
For the sea backgrounds, we create $1,880$ images with $1$ object, $2,209$ images with $2$ objects, $3,008$ images with $3$ objects, and $4,096$ images with $4$ objects making a total of $11,193$ images for training. 
Similarly, in the case of pool background, we create $1,880$ images with $1$ object, $2,209$ images with $2$ objects, $2,396$ images with $3$ objects, and $1,731$ images with $4$ objects, giving a total of $8,216$ images for training. 
We generate a smaller dataset for pool images since it is a controlled environment and has limited variety in the type of backgrounds that can occur in real-world images.





In both experiments, the background image and the final blended image are of size $512\times512$. We use $100$ iterations for style transfer during the second pass. We empirically fix the number of iterations. Similar to ~\cite{zhang_deep_2020}, we use a $L-BFG$ solver to optimize the total loss, set the content loss weight $\mu$ to $1$, and choose the total variation loss weight $\nu$ to be $10^{-6}$. In our method, the style loss weight $\lambda$ is based on image blurriness (Table~\ref{survey_weight}). Some examples of generated images are shown in Fig.~\ref{blend_example}.
\begin{figure}[t]

  \centering
  \includegraphics[width=.3\linewidth]{imgs/blend_pool_2.png}
  \hspace{2mm}
  \includegraphics[width=.3\linewidth]{imgs/blend_sea_2.png}\\
  \vspace{2mm}
  \includegraphics[width=.3\linewidth]{imgs/blend_pool_4.png}
  \hspace{2mm}
  \includegraphics[width=.3\linewidth]{imgs/blend_sea_4.png}
  \caption{Sample images generated from IBURD. The first column shows images blended on pool backgrounds. The second column contains objects blended on ocean backgrounds.}
  \label{blend_example}
\end{figure}


 

\subsection{Object Detection and Instance Segmentation}
To evaluate the efficacy of our framework, we first select YOLACT~\cite{bolya2019yolact} as an instance segmentation model based on its inference speed and performance with pretrained weights. 
We choose ResNet50-FPN as a backbone of YOLACT to obtain a reasonable inference time on the low-power mobile GPU (see Sec.~\ref{sec:robot_setup}) on the LoCO AUV.
We train the model with $4$ different datasets for quantitative experiments (Sec.~\ref{subsec:blend_quant}): \textit{T+D$2$k}, \textit{T+D$10$k}, \textit{T+T$10$k}, and original TrashCan data. For robotic experiments, we train the model with $2$ synthetic datasets: pool and sea data. The model is trained on an NVIDIA GeForce RTX 2080 Ti. 

\subsection{Robot Setup}\label{sec:robot_setup}
We use the LoCO AUV to run the detector model with trained weights on its mobile GPU (\ie NVIDIA Jetson TX2). 
We utilize image frames from the right camera of the LoCO AUV to make inferences. 
We adopt an LED lighting indicator system~\cite{fulton2023hreyes} 
installed on the LoCO AUV's left camera to visually examine the performance of different network weights during deployments. 
The $40$ LEDs in the system are split into $7$ groups, where each group denotes a specific class with a unique color (as shown in Fig.~\ref{fig:images}). 



\section{Related Work}
\section{Related Work}\label{sec:relatedwork}
\subsection{Parameter Efficient Fine-Tuning}
The increasing size of LLM parameters motivates the development of parameter efficient fine-tuning \cite{han2024parameter}. Unlike full-parameter fine-tuning, PEFT methods selectively adjust or introduce a small number of trainable parameters without modifying the entire parameter set. Adapter tuning \cite{houlsby2019parameter}, prompt tuning \cite{lester2021power}, prefix tuning \cite{li2021prefix}, and LoRA \cite{hulora} are some representative PEFT schemes, and LoRA is one of the most widely employed strategies. It integrates the low-rank decomposition matrices into the weight update of the model and greatly reduces the number of trainable parameters. Specifically, the modified parameter matrix $w'\in \mathbb{R}^{n\times m}$ is computed as $w'=w+AB$, where $w \in \mathbb{R}^{n\times m}$ is the original parameter matrix, $A\in \mathbb{R}^{n\times r}$ and $B \in \mathbb{R}^{r\times m}$ are the low-rank decomposition matrices, and $r$ is much smaller than $n$, i.e., $r\ll min(n,m)$. 

Based on the basic LoRA, several optimizations have been proposed to align with the specific characteristics of LLMs and implementation requirements. The authors in \cite{dettmers2024qlora} further minimize parameter overhead by quantizing adapter parameters. LoRA+ \cite{hayoulora} applies differentiated learning rates based on the initial values of the adapter matrices $A$ and $B$, thereby improving fine-tuning performance in certain scenarios. Unlike conventional approaches, which typically initialize $A$ with random values sampled from a normal distribution and $B$ with zeros, VeRA \cite{kopiczkovera} initializes all of them with a normal distributed random values but freezes them, and adds a trainable vector for each of $A$ and $B$. It can further reduce the trainable parameters involved in fine-tuning with slight accuracy sacrifice.

\subsection{Mixture of Experts}
The mixture of experts architecture was first proposed by \cite{jacobs1991adaptive}, introducing an assignment mechanism based on input data to distribute tasks across multiple expert modules, thereby achieving efficient task specialization and model capacity utilization. 
With the growing popularity of the Transformer \cite{vaswani2017attention}, many studies revisited MoE by applying it to the corresponding Feed-Forward Network (FFN) layers, extending these layers into multiple expert networks. However, a prominent feature of the widely adopted MoE in transformer-based LLMs is the use of a sparse gating mechanism, which selects only a subset of experts for token processing, enabling LLMs to scale to an extreme scale. 

GShard \cite{lepikhin2020gshard} and Switch Transformer \cite{fedus2022switch} are pioneers that employ learnable top-2 or top-1 expert selection strategies, and DeepSeek \cite{dai2024deepseekmoe} implements a shared expert isolation scheme. HashLayer \cite{roller2021hash} uses a hashing-based way to select experts for tokens, improving the stability of model training. \cite{huang2024harder,zhou2022mixture,yang2024xmoe} allow different numbers of experts to be assigned for different tokens, enhancing model flexibility. Besides studies on architectures and training strategies of MoE, recent years have also witnessed the emergence of many MoE-based multimodal models \cite{riquelme2021scaling,mustafa2022multimodal,du2022glam}.

\subsection{LoRA Meets MoE}
Based on basic LoRA technology, some studies have further introduced the MoE architecture, where each weight matrix's LoRA adapter is no longer a single module but are a set of adapters controlled by a gating network \cite{yang2024moral,wu2024parameter}. Such the approach enhances the capability of the fine-tuned model while maintaining scalability \cite{li2024mixlora,wumixture}. 

MoELoRA \cite{liu2024moe} combines the strengths of these two techniques to achieve an efficient multi-task fine-tuning framework. 
AdaMoE \cite{zeng2024adamoe} ingeniously introduces additional null expert networks, enabling a learnable and dynamic selection of the experts. 
LoRAMoE \cite{dou2024loramoe} classifies all experts of each layer into two categories, assigning them to process either world knowledge or fine-tuning knowledge, thereby mitigating the issue of world knowledge forgetting of the fine-tuning. 
MoLA \cite{gao2024higher} implements a layer-wise expert allocation strategy and demonstrates through experimental results that deeper networks require more experts than shallower ones.




\section{Conclusion}
\section{Conclusion}
% In conclusion, this study highlights the efficacy and potential of integrating Parameter-Efficient Fine-Tuning (PEFT) methods with game theory principles through the innovative approach of LoRA with Mixture of Gamers (\ourmethod). By melding Low-Rank Adaptation (LoRA) with Mixture of Experts (MoE) and utilizing game theory-based dynamics, \ourmethod{} significantly advances the field by addressing critical gaps in flexibility and dynamic expert selection inherent in previous methods. The employment of submatrix decomposition alongside Shapley values in \ourmethod{} enables a more granular understanding of the interactions and contributions of different components within PEFT setups. The promising experimental outcomes across a variety of tasks not only underscore \ourmethod's superior performance but also illuminate its versatile applicability and the potential for future adaptations in complex, domain-specific applications. Moving forward, it will be crucial to refine these approaches, ensuring robustness and scalability, to fully harness the transformative power of PEFT in enhancing machine learning models.

In this paper, we introduce \ourmethod{}, a Med-LVLM that unifies medical vision-language comprehension and generation through a novel heterogeneous knowledge adaptation approach.
% integrates H-LoRA and a three-stage fine-tuning approach, aim at unifying medical understanding and generation tasks. 
% To enhance the multi-task performance of \ourmethod{}, we introduce the \texttt{VL-Health} dataset for training. 
Experimental results demonstrate that \ourmethod{} achieves significant performance improvements across multiple medical comprehension and generation tasks, showcasing its potential for healthcare applications. 
\input{}

\section*{Impact Statement}
This paper contributes to the advancement of Machine Learning. While our work may have various societal implications, none require specific emphasis at this stage.


\bibliography{icml_main}
\bibliographystyle{icml2025}


%%%%%%%%%%%%%%%%%%%%%%%%%%%%%%%%%%%%%%%%%%%%%%%%%%%%%%%%%%%%%%%%%%%%%%%%%%%%%%%
%%%%%%%%%%%%%%%%%%%%%%%%%%%%%%%%%%%%%%%%%%%%%%%%%%%%%%%%%%%%%%%%%%%%%%%%%%%%%%%
% APPENDIX
%%%%%%%%%%%%%%%%%%%%%%%%%%%%%%%%%%%%%%%%%%%%%%%%%%%%%%%%%%%%%%%%%%%%%%%%%%%%%%%
%%%%%%%%%%%%%%%%%%%%%%%%%%%%%%%%%%%%%%%%%%%%%%%%%%%%%%%%%%%%%%%%%%%%%%%%%%%%%%%
\newpage
\appendix
\onecolumn
% \section*{Appendix}
%{\color{lightblue}
%\tableofcontents
%}
%
% \section*{Table of Contents}
% \begin{itemize}
%     \item[\textbf{\textcolor{blue}{A}}] \textbf{\textcolor{blue}{Implementation Details}}
%     \begin{itemize}
%         \item[A.1] Gaussian Construction
%         \item[A.2] Neural Radiance Field Rendering
%         \item[A.3] Hierarchical Latent Variables
%         \item[A.4] Modulation
%     \end{itemize}
    
%     \item[\textbf{\textcolor{blue}{B}}] \textbf{\textcolor{blue}{Derivation of Evidence Lower Bound}}
    
%     \item[\textbf{\textcolor{blue}{C}}] \textbf{\textcolor{blue}{Training Details}}
    
%     \item[\textbf{\textcolor{blue}{D}}] \textbf{\textcolor{blue}{More Experimental Results}}
% \end{itemize}

%%%%%%%%%%%%%%%%%%%%%%%%%%%%%%%%%%%%%%%%%%%%
% NerF: $F: (x,y,z,\theta,\phi) \mapsto (c,\sigma)$. 

% Context set (camera pose and image rbg level): 
% \begin{align}
% \tilde{X}_c:&= \{ \tilde{x}_i \}_{i=1}^{N_c}, \tilde{x}_i \in \mathbb{R}^{6} (\text{ray-o}, \text{ray-d})\\
% \tilde{Y}_c:&= \{ \tilde{y}_i \}_{i=1}^{N_c}, \tilde{y}_i \in \mathbb{R}^{3} (\text{RGB})
% \end{align}

% Target set (3d coordinates and 3d rbg with density level):

% $x_i$ is a set of 3D points corresponding to a specific ray direction. 
% \begin{align}
% X_t:&= \{ x_i \}_{i=1}^{N_t}, x_i \in \mathbb{R}^{P\times3} (\text{sampled 3D points})\\
% Y_t:&= \{ y_i \}_{i=1}^{N_t}, y_i \in \mathbb{R}^{P\times4} (c,\sigma)
% \end{align}

% \begin{align}
%     p(Y_t|X_t, \tilde{X}_c, \tilde{Y}_c) & = p(Y_t|X_t,B_c)p(B_c|\tilde{X}_c, \tilde{Y}_c) \\
%     &= \Pi_{i=1}^{N_t} p(y_i|x_i, g, r_i, B_c) p(r_i|g, x_i) p(g|B_c, X_t) p(B_c|\tilde{X}_c, \tilde{Y}_c)
% \end{align}

% \begin{align}
%     p(Y_t|X_t, \tilde{X}_c, \tilde{Y}_c) & \approx p(Y_t|X_t,B_c) \\
%     &= \Pi_{i=1}^{N_t} p(y_i|x_i, g, r_i, B_c) p(r_i|g, x_i) p(g|B_c, X_t)
% \end{align}
%%%%%%%%%%%%%%%%%%%%%%%%%%%%%%%%%%%%%%%%%%%%
\newpage
\section{Neural Radiance Field Rendering}
\label{supp:nerf-render}
In this section, we outline the rendering function of NeRF~\citep{mildenhall2021nerf}. A 5D neural radiance field represents a scene by specifying the volume density and the directional radiance emitted at every point in space. NeRF calculates the color of any ray traversing the scene based on principles from classical volume rendering~\citep{kajiya1984ray}. The volume density $\sigma(\mathbf{x})$ quantifies the differential likelihood of a ray terminating at an infinitesimal particle located at $\mathbf{x}$. The anticipated color $C(\mathbf{r})$ of a camera ray $\mathbf{r}(t) = \mathbf{o} + t\mathbf{d}$, within the bounds $t_n$ and $t_f$, is determined as follows:
\begin{equation}
C(\mathbf{r}) = \int_{t_n}^{t_f} T(t) \sigma(\mathbf{r}(t)) c(\mathbf{r}(t), \mathbf{d}) dt, \quad \text{where} \quad T(t) = \exp \left( - \int_{t_n}^{t} \sigma(\mathbf{r}(s)) ds \right).
\end{equation}

Here, the function $T(t)$ represents the accumulated transmittance along the ray from $t_n$ to $t$, which is the probability that the ray travels from $t_n$ to $t$ without encountering any other particles. To render a view from our continuous neural radiance field, we need to compute this integral $C(\mathbf{r})$ for a camera ray traced through each pixel of the desired virtual camera.

\section{Implementation Details}
\label{sec:implementation-details}

\subsection{Gaussian Construction}
\label{supp:gaussian}

As introduced in Sec.~\ref{sec: geometrybases}, we introduce geometric bases ${\bf{{B}}}_{C}$ to structure the context variables geometrically.
${\bf{{B}}}_{C}$ are geometric  bases (Gaussians) inferred from the context views $\{{\bf{\widetilde{X}}}_{C}, {\bf{\widetilde{Y}}}_{C}\}$ with 3D structure information, 
\textit{i.e.,} ${\bf{b}}_i = \{ \mathcal{N}(\mu_i, \Sigma_i); \omega_i\}$,
%\textit{i.e., object shape, color and texture.}. $B_C$ is obtained by: 
% ${\bf{{B}}}_{C}=\texttt{Encoder}\Big({\bf{\widetilde{X}}}_{C}, {\bf{\widetilde{Y}}}_{C}\Big)$. 
\begin{align}
    &{\bf{{B}}}_{C} = \{{\bf{b}}_i\}_{i=1}^{M}, {\bf{b}}_i=\{\mathcal{N}(\mu_i, \Sigma_i); \omega_i\},
    \label{eq: generation_B_1}
    \\
    & \mu_i, \Sigma_i = \texttt{Att}({\bf{\widetilde{X}}}_{C}, {\bf{\widetilde{Y}}}_{C}), \texttt{Att}({\bf{\widetilde{X}}}_{C}, {\bf{\widetilde{Y}}}_{C}),
    \label{eq: generation_B_2}
    \\
    & \omega_i = \texttt{Att}({\bf{\widetilde{X}}}_{C}, {\bf{\widetilde{Y}}}_{C}),
    \label{eq: generation_B_3}
\end{align}
where $M$ is the number of the Gaussian bases. $\mu \mathbb \in {R}^3$ is the Gaussian center, $\Sigma \in  \mathbb{R}^{3\times 3}$ is the covariance matrix, and $\omega \in \mathbb{R}^{d_B}$ is the corresponding ${d_B}$-dimension semantic representation. In our implementation, we choose $d_{B}$ as $32$. $\texttt{Att}$ is a self-attention module. Specifically, given the context set $[\widetilde{\mathbf{X}};\widetilde{\mathbf{Y}}] \in \mathbb{R}^{H\times W \times (3+3+3)}$, the visual self-attention module, $\texttt{Att}$, first produces a $M\times D$ tokens with $M$ is the number of visual tokens and $D$ is the hidden dimension. The number of Gaussians we use equals the number of tokens $M$. %Then, we use one MLP to predict centers $\mu$, as well as the rotation $R$ and scaling $S$ matrices parameters for producing covariance matrix $\Sigma$, and one MLP to produce the latent representations $\omega$. 
Then, we use one MLP with 2 linear layers to map the tokens into a 10-dimensional vector, which includes 3-dimensional Gaussian centers, a 3-dimensional vector for constructing the scaling matrix, and a 4-dimensional vector for quaternion parameters of the rotation matrix. Both the scaling matrix and rotation matrix are used to build the \(3 \times 3\) covariance matrix. This procedure is similar to Gaussian construction in the 3D Gaussian Splatting~\citep{kerbl20233d}.
Another MLP estimates the latent representation of each Gaussian basis, using a 32-dimensional vector for each Gaussian basis. 

The covariance matrix is obtained by:
\begin{equation}
    \Sigma = RSS^TR^T,
    \label{eq:cov-matrix}
\end{equation}
where $R\in \mathbb{R}^{3\times3}$ is the rotation matrix, and $S \in \mathbb{R}^3$ is the scaling matrix. 











\subsection{Hierarchical Latent Variables}
\label{supp:latent-variables}

\begin{figure}[t]
  \centering
  \includegraphics[width=0.7\textwidth]{Figures/Transformer.pdf} % Adjust the size and filename as needed
  \caption{\textbf{Using transformer encoder to generate ray-specific latent variable $\mathbf{z}_r$.}} % Caption for the figure
  \label{fig:latent-transformer}
  % \vspace{-3mm}
\end{figure}

At the object level, the distribution of an object-specific latent variable \(\mathbf{z}_o\) is obtained by aggregating all location representations from \((\mathbf{B}_C, \mathbf{X}_T)\). We assume \(p(\mathbf{z}_o | \mathbf{B}_C, \mathbf{X}_T)\) follows a standard Gaussian distribution and generate its mean \(\mu_{o}\) and variance \(\sigma_{o}\) using MLPs. We sample an object-specific modulation vector, \(\hat{\mathbf{z}}_o\), from its prior distribution \(p(\mathbf{z}_o | \mathbf{X}_T, \mathbf{B}_C)\).

Similarly, as shown in Fig.~\ref{fig:latent-transformer}, we aggregate the information per ray using \(\mathbf{B}_C\), which is then fed into a Transformer along with \(\hat{\mathbf{z}}_o\) to predict the latent variable \(\mathbf{z}_r\) with mean \(\mu_r\) and \(\sigma_r\) for each ray.


 

\subsection{Modulation}
\label{supp:modulate}
The latent variables for modulating the MLP are represented as \([\mathbf{z}_o; \mathbf{z}_r]\). Our approach to the modulated MLP layer follows the style modulation techniques described in \citep{karras2020analyzing, guo2023versatile}. Specifically, we consider the weights of an MLP layer (or 1x1 convolution) as \( W \in \mathbb{R}^{d_{\text{in}} \times d_{\text{out}}} \), where \( d_{\text{in}} \) and \( d_{\text{out}} \) are the input and output dimensions respectively, and \( w_{ij} \) is the element at the \(i\)-th row and \(j\)-th column of \( W \).

To generate the style vector \( s \in \mathbb{R}^{d_{\text{in}}} \), we pass the latent variable \( z \) through two MLP layers. Each element \( s_i \) of the style vector \( s \) is then used to modulate the corresponding parameter in \( W \).
\begin{equation}
    w'_{ij} = s_i \cdot w_{ij}, \quad j = 1, \ldots, d_{\text{out}},
\end{equation}
where $w_{ij}$ and $w'_{ij}$ denote the original and modulated weights, respectively.

The modulated weights are normalized to preserve training stability,
\begin{equation}
    w''_{ij} = \frac{w'_{ij}}{\sqrt{\sum_i w'^2_{ij} + \epsilon}}, \quad j = 1, \ldots, d_{\text{out}}.
\end{equation}




\begin{algorithm}[H]
\caption{Modulation Layer}
\begin{algorithmic}[1]
\REQUIRE Latent variable $z$, weight matrix $W \in \mathbb{R}^{d_{\mathrm{in}} \times d_{\mathrm{out}}}$
\ENSURE Modulated and normalized weight matrix $W''$
\STATE \textbf{Compute style vector:}
\STATE $s \leftarrow \mathrm{MLP}_2\big(\mathrm{MLP}_1(z)\big)$
\STATE \textbf{Modulate weights:}
\STATE $W' \leftarrow \operatorname{diag}(s) \times W$
\STATE \textbf{Normalize modulated weights:}
\STATE For each column $j$ in $W'$:
\STATE \hskip1em $\sigma_j \leftarrow \sqrt{\sum_{i=1}^{d_{\mathrm{in}}} (W'_{ij})^2 + \epsilon}$
\STATE Normalize column $j$ of $W'$: $W''_{:,j} \leftarrow W'_{:,j} / \sigma_j$
\RETURN $W''$
\end{algorithmic}
\end{algorithm}










\begin{algorithm}[H]
\caption{Training Procedure}
\begin{algorithmic}[1]
\REQUIRE Context set $({\bf{X}}_{C}, {\bf{Y}}_C)$, target set $({\bf{X}}_{T}, {\bf{Y}}_T)$
\ENSURE Prediction ${\bf{Y}}'_T$
\vspace{0.5em}
\STATE Estimate the context bases ${\bf{B}}_C$ and the target bases ${\bf{B}}_T$ (Eq. 12).
\vspace{0.5em}
\STATE Estimate the object-specific latent variables:
\begin{itemize}
    \item For the context set ${\bf{z}}_o^C$:
    \[
    {\bf{z}}_o^C \sim p({\bf{z}}_o \mid {\bf{X}}_C, {\bf{B}}_C)
    \]
    \item For the target set ${\bf{z}}_o^T$:
    \[
    {\bf{z}}_o^T \sim q({\bf{z}}_o \mid {\bf{X}}_T, {\bf{B}}_T) \quad \text{(Eq. 7)}
    \]
\end{itemize}
\vspace{0.5em}
\STATE Estimate the ray-specific latent variables:
\begin{itemize}
    \item For the context set ${\bf{z}}_r^{C}$:
    \[
    {\bf{z}}_r^{C} \sim p({\bf{z}}_r^n \mid {\bf{z}}_o^C, {\bf{x}}_C^{n}, {\bf{B}}_C) \quad \text{(Eq. 8)}
    \]
    \item For the target set ${\bf{z}}_r^{T}$:
    \[
    {\bf{z}}_r^{T} \sim q({\bf{z}}_r^n \mid {\bf{z}}_o^T, {\bf{x}}_T^{n}, {\bf{B}}_T) \quad \text{(Eq. 8)}
    \]
\end{itemize}
\vspace{0.5em}
\STATE Modulate MLP $f$ using the target latent variables $\{{\bf{z}}_o^T, {\bf{z}}_r^{T}\}$ (Eqs. 16 \& 17).
\vspace{0.5em}
\STATE Render novel views $\hat{\bf{Y}}_T$ using the modulated MLP $f$.
\vspace{0.5em}
\STATE \textbf{Compute losses:}
\begin{itemize}
    \item Reconstruction loss between predictions and ground truth:
    \[
    \mathcal{L}_{\text{recon}} = \text{Loss}(\hat{\bf{Y}}_T, {\bf{Y}}_T)
    \]
    \item Latent variable alignment losses (KL divergence) using context and target latent variables (Eq. 10).
\end{itemize}
\end{algorithmic}
\end{algorithm}


\begin{algorithm}[H]
\caption{Inference Procedure}
\begin{algorithmic}[1]
\REQUIRE Context set $({\bf{X}}_C, {\bf{Y}}_C)$, target input ${\bf{X}}_T$
\ENSURE Prediction ${\bf{Y}}'_T$
\vspace{0.5em}
\STATE Estimate the context bases ${\bf{B}}_C$ (Eq. 12).
\vspace{0.5em}
\STATE Estimate the object-specific latent variable ${\bf{z}}_o$ based on the context set:
\[
{\bf{z}}_o \sim p({\bf{z}}_o \mid {\bf{X}}_C, {\bf{B}}_C) \quad \text{(Eq. 7)}
\]
\vspace{0.5em}
\STATE Estimate the ray-specific latent variables ${\bf{z}}_r^{T}$:
\[
{\bf{z}}_r^{T} \sim p({\bf{z}}_r^n \mid {\bf{z}}_o, {\bf{x}}_T^{n}, {\bf{B}}_C) \quad \text{(Eq. 8)}
\]
\vspace{0.5em}
\STATE Modulate the MLP $f$ using the latent variables $\{{\bf{z}}_o, {\bf{z}}_r^{T}\}$ (Eqs. 16 \& 17).
\vspace{0.5em}
\STATE Render novel views $\hat{\bf{Y}}_T$ using the modulated MLP $f$.
\end{algorithmic}
\end{algorithm}


% $f_C$ by $\{{\bf{z}}_o, {\bf{z}}_r^n\}_C$, 

\section{Derivation of Evidence Lower Bound}
\label{supp:elbo}



\noindent{\textbf{Evidence Lower Bound.}} 
We optimize the model via variational inference~\citep{garnelo2018neural}, deriving the evidence lower bound (ELBO):
\begin{equation}
\begin{aligned}
& \log p(\mathbf{Y}_T \mid \mathbf{X}_T, \mathbf{B}_C) \geq \\
&\mathbb{E}_{q(\mathbf{z}_g | \mathbf{X}_T, \mathbf{B}_T)} \Bigg[ \sum_{m=1}^M \mathbb{E}_{q(\mathbf{z}_l^m | \mathbf{z}_g, \mathbf{x}_T^m, \mathbf{B}_T)} \log p(\mathbf{y}_T^m \mid \mathbf{z}_g, \mathbf{z}_l^m, \mathbf{x}_T^m) \\
& \quad - D_{\text{KL}}\Big[q(\mathbf{z}_l^m | \mathbf{z}_g, \mathbf{x}_T^m, \mathbf{B}_T) \,\big|\big|\, p(\mathbf{z}_l^m | \mathbf{z}_g, \mathbf{x}_T^m, \mathbf{B}_C)\Big] \Bigg] \\
& - D_{\text{KL}}\Big[q(\mathbf{z}_g | \mathbf{X}_T, \mathbf{B}_T) \,\big|\big|\, p(\mathbf{z}_g | \mathbf{X}_T, \mathbf{B}_C)\Big],
\end{aligned}
\end{equation}
where the variational posterior factorizes as $q(\mathbf{z}_g, \{\mathbf{z}_l^m\}_{m=1}^M | \mathbf{X}_T, \mathbf{B}_T) = q(\mathbf{z}_g | \mathbf{X}_T, \mathbf{B}_T) \prod_{m=1}^M q(\mathbf{z}_l^m | \mathbf{z}_g, \mathbf{x}_T^m, \mathbf{B}_T)$. Here, $\mathbf{B}_T$ denotes geometric bases constructed from target data $\{\widetilde{\mathbf{X}}_T, \widetilde{\mathbf{Y}}_T\}$ (available only during training). The KL terms regularize the hierarchical priors $p(\mathbf{z}_g | \mathbf{B}_C)$ and $p(\mathbf{z}_l^m | \mathbf{z}_g, \mathbf{B}_C)$ to align with variational posteriors inferred from $\mathbf{B}_T$, enhancing generalization to context-only settings. Derivations are in Appendix~\ref{supp:elbo}.


The propose \textbf{GeomNP} is formulated as:
{\small
\begin{equation}
        p({\bf{Y}}_{T}| {\bf{X}}_{T}, {\bf{B}}_{C}) = \int \prod_{n=1}^{N} \Big\{ \int p({\bf{y}}_{T}^{\mathbf{r}, n}| {\bf{x}}_{T}^{\mathbf{r}, n}, {\bf{B}}_{C}, {\bf{z}}_r^n,{\bf{z}}_o, ) p({\bf{r}}^n| {\bf{z}}_o,  {\bf{x}}_{T}^{\mathbf{r}, n}, {\bf{B}}_C) d {\bf{z}}_r^n \Big\} p({\bf{z}}_o |{\bf{X}}_T, {\bf{B}}_C) d {\bf{z}}_o, 
\label{eq:ganp-model-supp}
\end{equation}}where $p({\bf{z}}_o | {\bf{B}}_C,  {\bf{X}}_T)$ and $p({\bf{z}}_r^n| {\bf{z}}_o,  {\bf{x}}_{T}^{r, n}, {\bf{B}}_C)$ denote prior distributions of a object-specific and each ray-specific latent variables, respectively. Then, the evidence lower bound is derived as follows.

\begin{equation}
\begin{aligned}
        &\log p({\bf{Y}}_{T}| {\bf{X}}_{T}, {\bf{B}}_{C}) \\
        &= \log \int \prod_{n=1}^{N} \Big\{ \int p({\bf{y}}_{T}^{\mathbf{r}, n}| {\bf{x}}_{T}^{\mathbf{r}, n}, {\bf{z}}_o, {\bf{z}}_r^n) p({\bf{z}}_r^n| {\bf{z}}_o,  {\bf{x}}_{T}^{\mathbf{r}, n}, {\bf{B}_C}) d {\bf{z}}_r^n \Big\} p({\bf{z}}_o | {\bf{B}}_C,  {\bf{X}}_T) d {\bf{z}}_o  \\
    &= \log \int  \prod_{i=1}^{N} \Big\{ \int p({\bf{y}}_{T}^{\mathbf{r}, n}| {\bf{x}}_{T}^{\mathbf{r}, n}, {\bf{z}}_o, {\bf{z}}_r^n) p({\bf{z}}_r^n| {\bf{z}}_o,  {\bf{x}}_{T}^{\mathbf{r}, n}, {\bf{B}_C}) \frac{q({\bf{z}}_r^n| {\bf{z}}_o,  {\bf{x}}_{T}^{\mathbf{r}, n}, {\bf{B}_T})}{q({\bf{z}}_r^n| {\bf{z}}_o,  {\bf{x}}_{T}^{\mathbf{r}, n}, {\bf{B}_T})} d {\bf{z}}_r^n \Big\} \\
    & p({\bf{z}}_o | {\bf{B}}_C,  {\bf{X}}_T) \frac{q({\bf{z}}_o | {\bf{B}}_T,  {\bf{X}}_T)}{q({\bf{z}}_o | {\bf{B}}_T,  {\bf{X}}_T,)} d {\bf{z}}_o  \\
    &\geq  \mathbb{E}_{q({\bf{z}}_o | {\bf{B}}_T,  {\bf{X}}_T)}  \Big\{  \sum_{i=1}^{N} \log  \int p({\bf{y}}_{T}^{\mathbf{r}, n}| {\bf{x}}_{T}^{\mathbf{r}, n}, {\bf{z}}_o, {\bf{z}}_r^n) p({\bf{z}}_r^n| {\bf{z}}_o,  {\bf{x}}_{T}^{\mathbf{r}, n}, {\bf{B}_C}) \frac{q({\bf{z}}_r^n| {\bf{z}}_o,  {\bf{x}}_{T}^{\mathbf{r}, n}, {\bf{B}_T})}{q({\bf{z}}_r^n| {\bf{z}}_o,  {\bf{x}}_{T}^{\mathbf{r}, n}, {\bf{B}_T})} d {\bf{z}}_r^n \Big\} \\
    &- D_{\text{KL}}(q({\bf{z}}_o | {\bf{B}}_T,  {\bf{X}}_T,) || p({\bf{z}}_o | {\bf{B}}_C,  {\bf{X}}_T)) \\
    &\geq  \mathbb{E}_{q({\bf{z}}_o | {\bf{B}}_T,  {\bf{X}}_T)}  \Big\{  \sum_{n=1}^{N}  \mathbb{E}_{q({\bf{z}}_r^n| {\bf{z}}_o,  {\bf{x}}_{T}^{\mathbf{r}, n}, {\bf{B}_T})} \log p({\bf{y}}_{T}^{{\mathbf{r}}, n}| {\bf{x}}_{T}^{{\mathbf{r}}, n}, {\bf{z}}_o, {\bf{z}}_r^n) \\
&- D_{\text{KL}}[q({\bf{z}}_r^n| {\bf{z}}_o,  {\bf{x}}_{T}^{{\mathbf{r}}, n}, {\bf{B}_T}) || p({\bf{z}}_r^n| {\bf{z}}_o,  {\bf{x}}_{T}^{{\mathbf{r}}, n}, {\bf{B}_C}) ] \Big\} 
- D_{\text{KL}}[q({\bf{z}}_o | {\bf{B}}_T,  {\bf{X}}_T) || p({\bf{z}}_o | {\bf{B}}_C,  {\bf{X}}_T)], \\
        % &=  \int \log \prod_{i=1}^{N_{ray}} \Big\{ \int p({\bf{y}}^{T}_{1:P, i}| {\bf{x}}^{T}_{1:P, i}, {\bf{g}}, {\bf{r}}_i) p({\bf{r}}_i| {\bf{g}},  {\bf{x}}^{T}_{1:P, i}) d {\bf{r}}_i \Big\} +  \log p({\bf{g}} | {\bf{B}}_C,  {\bf{X}}_T,) d {\bf{g}} \\
        % &= \int \sum_{i=1}^{N_{ray}} \log \Big\{ \int p({\bf{y}}^{T}_{1:P, i}| {\bf{x}}^{T}_{1:P, i}, {\bf{g}}, {\bf{r}}_i) p({\bf{r}}_i| {\bf{g}},  {\bf{x}}^{T}_{1:P, i}) d {\bf{r}}_i \Big\} +  \log p({\bf{g}} | {\bf{B}}_C,  {\bf{X}}_T,) d {\bf{g}} \\
        % &= \int \sum_{i=1}^{N_{ray}}  \Big\{ \int \log p({\bf{y}}^{T}_{1:P, i}| {\bf{x}}^{T}_{1:P, i}, {\bf{g}}, {\bf{r}}_i) + \log p({\bf{r}}_i| {\bf{g}},  {\bf{x}}^{T}_{1:P, i}) d {\bf{r}}_i \Big\} +  \log p({\bf{g}} | {\bf{B}}_C,  {\bf{X}}_T,) d {\bf{g}} \\
\end{aligned}      
\end{equation}
where $q_{\theta, \phi}({\bf{z}}_o,  \{{\bf{z}}_r^i\}_{i=1}^{N} | {\bf{X}}_T, {\bf{B}}_T) = q({\bf{z}}_r^n| {\bf{z}}_o,  {\bf{x}}_{T}^{{\mathbf{r}}, n}, {\bf{B}_T}) q({\bf{z}}_o | {\bf{B}}_T,  {\bf{X}}_T)$ is the variational posterior of the hierarchical latent variables. 


\section{More Related Work}
% \textcolor{blue}{
% \cite{szymanowicz2024splatter}, \cite{charatan2024pixelsplat}, \cite{chen2025mvsplat}, \cite{hong2023lrm}, \cite{muller2023diffrf}, \cite{tewari2023diffusion}, \cite{xu2022point}, \cite{wang2024learning}, \cite{liu2024geometry}}


\textcolor{blue}{\paragraph{Generalizable Neural Radiance Fields (NeRF)}
Advancements in neural radiance fields have focused on improving generalization across diverse scenes and objects. \cite{wang2022attention} propose an attention-based NeRF architecture, demonstrating enhanced capabilities in capturing complex scene geometries by focusing on informative regions. \cite{suhail2022generalizable} introduce a generalizable patch-based neural rendering approach, enabling models to adapt to new scenes without retraining. \cite{xu2022point} present \textit{Point-NeRF}, leveraging point-based representations for efficient scene modeling and scalability. \cite{wang2024learning} further enhance point-based methods by incorporating visibility and feature augmentation to improve robustness and generalization. \cite{liu2024geometry} propose a geometry-aware reconstruction with fusion-refined rendering for generalizable NeRFs, improving geometric consistency and visual fidelity. Recently, the \textit{Large Reconstruction Model (LRM)}~\citep{hong2023lrm} has drawn attention. It aims for single-image to 3D reconstruction, emphasizing scalability and handling of large datasets.}

\textcolor{blue}{\paragraph{Gaussian Splatting-based Methods}
Gaussian splatting~\citep{kerbl20233d} has emerged as an effective technique for efficient 3D reconstruction from sparse views. \cite{szymanowicz2024splatter} propose \textit{Splatter Image} for ultra-fast single-view 3D reconstruction. \cite{charatan2024pixelsplat} introduce \textit{pixelsplat}, utilizing 3D Gaussian splats from image pairs for scalable generalizable reconstruction. \cite{chen2025mvsplat} present \textit{MVSplat}, focusing on efficient Gaussian splatting from sparse multi-view images. Our approach can be a complementary module for these methods by introducing a probabilistic neural processing scheme to fully leverage the observation. }

\textcolor{blue}{\paragraph{Diffusion-based 3D Reconstruction}
Integrating diffusion models into 3D reconstruction has shown promise in handling uncertainty and generating high-quality results. \cite{muller2023diffrf} introduce \textit{DiffRF}, a rendering-guided diffusion model for 3D radiance fields. \cite{tewari2023diffusion} explore solving stochastic inverse problems without direct supervision using diffusion with forward models. \cite{liu2023zero} propose \textit{Zero-1-to-3}, a zero-shot method for generating 3D objects from a single image without training on 3D data, utilizing diffusion models. \cite{shi2023zero123++} introduce \textit{Zero123++}, generating consistent multi-view images from a single input image using diffusion-based techniques. \cite{shi2023mvdream} present \textit{MVDream}, which uses multi-view diffusion for 3D generation, enhancing the consistency and quality of reconstructed models.}


\section{Implementation Details}
We train all our models with PyTorch. Adam optimizer is used with a learning rate of $1e-4$. For NeRF-related experiments, we follow the baselines~\citep{chen2022transformers,guo2023versatile} to train the model for 1000 epochs. All experiments are conducted on four NVIDIA A5000 GPUs. For the hyper-parameters $\alpha$ and $\beta$, we simply set them as $0.001$.  


\textcolor{blue}{\paragraph{Model Complexity} The comparison of the number of parameters is presented in Table.~\ref{tab:params_psnr}. Our method, GeomNP, utilizes fewer parameters than the baseline, VNP, while achieving better performance on the ShapeNet Car dataset in terms of PSNR.}

\begin{table}[h!]
\centering
\caption{Comparison of the number of parameters and PSNR on the ShapeNet Car dataset.}
\begin{tabular}{lcc}
\toprule
Method & {\# Parameters} & {PSNR} \\ 
\midrule
VNP     & 34.3M   & 24.21 \\ 
GeomNP  & \textbf{24.0M}   & \textbf{25.13} \\ 
\bottomrule
\end{tabular}
\label{tab:params_psnr}
\end{table}

\paragraph{Integration with PixelNeRF} 
\textcolor{blue}{To integrate our method into PixelNeRF, we utilize the same feature extractor and NeRF architecture. Specifically, we employ a pre-trained ResNet to extract features from the observed images. From the latent space of the feature encoder, we predict geometric bases, which are used to re-represent each 3D point in a higher-dimensional space. These re-represented point features are aggregated into latent variables, which are then used to modulate the first two input MLP layers of PixelNeRF's NeRF network. During training, we align the latent variables derived from the context images with those from the target views to ensure consistency.}

% \newpage

\section{More Experimental Results}
\label{supp:more-results}
%%%%%%%%%%%%%%%%%%%%%%%%%%%%%%%%%%%%%%%%%%%%%%%%%%%%%%%%%%%%
\subsection{Image Regression}

% \begin{figure*}[htbp]
%     \centering
%     \begin{minipage}[b]{0.45\textwidth} 
%         \includegraphics[width=\textwidth]{Figures/image-regression0.pdf} % Adjust filename as needed
%         \caption{CelebA}
%         \label{fig:celeba}
%     \end{minipage}
%     \hfill
%     \begin{minipage}[b]{0.45\textwidth} 
%         \includegraphics[width=\textwidth]{Figures/image-regression1.pdf} % Adjust filename as needed
%         \caption{Imagenette}
%         \label{fig:imagenette}
%     \end{minipage}
%     \caption{\textbf{Visualizations} of image regression results on CelebA (left) and Imagenette (right).}
%     \label{fig:visualization-image-regression}
% \end{figure*}

\subsection{Image Completion}
 We also conduct experiments of \method{} on image completion (also called image inpainting), which is a more challenging variant of image regression. Essentially, only part of the pixels are given as context, while the INR functions are required to complete the full image. Visualizations in Fig.~\ref{fig:completion} demonstrate the generalization ability of our method to recover realistic images with fine details based on very limited context ($10 \% - 20\%$ pixels).


\begin{figure}[t!]
  \centering
  \includegraphics[width=0.99\textwidth]{Figures/image-completion.pdf} % Adjust the size and filename as needed
  \vspace{-3mm}
  \caption{\textbf{Image completion visualization} on CelebA using $10\%$ (left) and $20\%$ (right) context.}
  \label{fig:completion}
  \vspace{-5mm}
\end{figure}

\subsection{Comparison with GNT.}
\label{sec:compare_gnt}
Specfiically, we use GNT's image encoder and predict the geometric bases, and GNT's NeRF' network for prediction. 



\begin{figure}[htbp]
  \centering
  \includegraphics[width=0.99\textwidth]{ICML25/Figures/nerf-syn-1view.pdf} % Adjust the size and filename as needed
  \vspace{-3mm}
  \caption{\textbf{Qualitative comparison with GNT on 1-view setting.}}
  \label{fig:1-view-compare}
  \vspace{-5mm}
\end{figure}

\subsubsection{Cross-Category Example.}
\label{sec:cross-category}

\begin{figure}[htbp]
  \centering
  \includegraphics[width=0.99\textwidth]{ICML25/Figures/cross-category.pdf} % Adjust the size and filename as needed
  \vspace{-3mm}
  \caption{}
  \label{fig:cross-category}
  \vspace{-5mm}
\end{figure}



% \subsection{Comparison with PixelNeRF}

% \begin{wraptable}{r}{0.42\textwidth}
% \vspace{-4mm}
% \caption{\textbf{Comparison on the DTU MVS dataset.} Training with 1-view context and testing with both 1-view and 3-view context images. Integrating \method{} into the pixelNeRF framework leads to improvement in terms of both PSNR and SSIM.}
% \centering
% \resizebox{0.48\textwidth}{!}{
% \begin{tabular}{llcc}
% \toprule
%  & {Method} & {PSNR} & {SSIM} \\
% \midrule
% \multirow{2}{*}{1-view} 
% & pixelNeRF & 15.51 & 0.51 \\
% \multirow{-1}{*}{} & \cellcolor{lightblue}\textbf{\method{}} (Ours) & \cellcolor{lightblue}\textbf{15.89} & \cellcolor{lightblue}\textbf{0.58} \\
% \midrule
% \multirow{2}{*}{3-view} 
% & pixelNeRF & 15.80 & 0.56 \\
% \multirow{-1}{*}{} & \cellcolor{lightblue}\textbf{\method{}} (Ours) & \cellcolor{lightblue}\textbf{16.99} & \cellcolor{lightblue}\textbf{0.61} \\
% \bottomrule
% \vspace{-3mm}
% \end{tabular}
% }
% \label{tab:dtu-compare}
% \end{wraptable}
% \noindent {\textbf{Comparison on DTU.}}
% %Our method can be flexibly integrated with other approaches. 
% To ensure a fair comparison with pixelNeRF~\citep{yu2021pixelnerf} using the same encoder and NeRF network architecture, we incorporate our probabilistic framework into pixelNeRF. We conducted experiments on real-world scenes from the DTU MVS dataset~\citep{aanaes2016large}. To explore the capability of dealing with extremely limited context information, we 
% train both models with 1-view context and test the 1-view and 3-view results in terms of PSNR and SSIM~\citep{wang2004image} metrics. Both qualitative results in Table~\ref{tab:dtu-compare} and qualitative results in Fig.~\ref{fig:dtu-visualization} demonstrate our probabilistic modeling can improve the existing methods. Notably, even when trained with a 1-view context image and tested with 3-view context images, our method significantly outperforms pixelNeRF, demonstrating that our probabilistic framework effectively utilizes limited observations.


% \begin{figure}[t]
%   \centering
%   \includegraphics[width=1\textwidth]{ICLR2025/Figures/dtu-results.pdf} % Adjust the size and filename as needed
%   \vspace{-6mm}  
%   \caption{\textbf{Novel view synthesis results with 1-view context on the DTU dataset.} \method{} has a more realistic rendering quality than pixelNeRF~\citep{yu2021pixelnerf} for novel views with extremely limited context views (1-view).} % Caption for the figure
%   \label{fig:dtu-visualization}
%   \vspace{-5mm}
% \end{figure}




\subsection{More results on ShapeNet}
In this section, we demonstrate more experimental results on the novel view synthesis task on ShapeNet in Fig~\ref{fig:nerf-supp-shapenet}, comparison with VNP~\cite{guo2023versatile} in Fig.~\ref{fig:nerf-supp-compare}, and image regression on the Imagenette dataset in Fig.~\ref{fig:image-supp-image}. The proposed method is able to generate realistic novel view synthesis and 2D images.


\begin{figure}[t]
  \centering
\includegraphics[width=1\textwidth]{Figures/nerf-results-more-supp.pdf} % Adjust the size and filename as needed
  \caption{\textbf{More NeRF results on novel view synthesis task on ShapeNet objects.}} % Caption for the figure
  \label{fig:nerf-supp-shapenet}
\end{figure}


\begin{figure}[t]
  \centering
\includegraphics[width=1\textwidth]{Figures/comparsion_vnp.pdf} % Adjust the size and filename as needed
  \caption{\textbf{Comparison between the proposed method and VNP} on novel view synthesis task for ShapeNet objects. Our method has a better rendering quality than VNP for novel views.} % Caption for the figure
  \label{fig:nerf-supp-compare}
\end{figure}

\begin{figure}[htbp]
  \centering
\includegraphics[width=0.8\textwidth]{Figures/imagenette-more.png} % Adjust the size and filename as needed
  \caption{\textbf{More image regression results on the Imagenette dataset.} Left: ground truth; Right: prediction.} % Caption for the figure
  \label{fig:image-supp-image}
  % \vspace{-3mm}
\end{figure}

\subsection{Training Time Comparison}

\textcolor{blue}{As illustrated in Fig.\ref{fig:train-time}, with the same training time, our method (GeomNP) demonstrates faster convergence and higher final PSNR compared to the baseline (VNP). }

\begin{figure}[t]
  \centering
  \includegraphics[width=0.7\textwidth]{ICLR2025/Figures/train_time_psnr.png} % Adjust the size and filename as needed
  \caption{\textbf{Training time vs. PSNR on the ShapeNet Car dataset.} Our method (GeomNP) demonstrates faster convergence and higher final PSNR compared to the baseline (VNP).} % Caption for the figure
  \label{fig:train-time}
  % \vspace{-3mm}
\end{figure}






\subsection{Qualitative ablation of the hierarchical latent variables}
\label{sec:abl-bases}
\textcolor{blue}{In this section, we perform a qualitative ablation study on the hierarchical latent variables. As illustrated in Fig.~\ref{fig:hier-abl}, the absence of the global variable prevents the model from accurately predicting the object's outline, whereas the local variable captures fine-grained details. When both global and local variables are incorporated, GeomNP successfully estimates the novel view with high accuracy.}



\begin{figure}[t]
  \centering
  \includegraphics[width=0.7\textwidth]{./Figures/hierarchical-ablation-new.pdf} % Adjust the size and filename as needed
  \caption{\textbf{Qualitative ablation of the hierarchical latent variables (global and local variables)}. }  % Caption for the figure
  \label{fig:hier-abl}
  % \vspace{-3mm}
\end{figure}



\subsection{More multi-view reconstruction results}
\textcolor{blue}{We integrate our method into GNT~\citep{wang2022attention} framework and perform experiments on the Drums class of the NeRF synthetic dataset. Qualitative comparisons of multi-view results are presented in Fig.~\ref{fig:qua-nerf-syn}. }

\begin{figure}[t]
  \centering
  \includegraphics[width=1.0\textwidth]{ICLR2025/Figures/nerf-syn.pdf} % Adjust the size and filename as needed
  \caption{\textbf{Qualitative comparisons of Multi-view results on the Drums class of the NeRF synthetic dataset. } }  % Caption for the figure
  \label{fig:qua-nerf-syn}
  % \vspace{-3mm}
\end{figure}

% \section{NP with Gaussian Splatting}
% \begin{table}[htbp]
%     \centering
%     \caption{Comparison of methods}
%     \begin{tabular}{lccc}
%         \toprule
%         \textbf{Method} & \textbf{PSNR $\uparrow$} & \textbf{SSIM $\uparrow$} & \textbf{LPIPS $\downarrow$} \\
%         \midrule
%         PixelNeRF & 21.76 & 0.78 & 0.203 \\
%         {Splatter Image} & 21.80 & {0.80} & {0.150} \\
%         \bottomrule
%     \end{tabular}
% \end{table}


% Recently, 3D Gaussian Splatting~\citep{kerbl20233d} has gained significant attention for its efficiency and strong performance in reconstructing 3D scenes. Like NeRF, Gaussian Splatting requires overfitting on a specific scene to optimize the 3D Gaussian parameters. To improve generalization, given a single-view context image, Splatter Image~\citep{szymanowicz2024splatter} employs a UNet to predict Gaussian parameters for a new scene. However, Splatter Image remains a deterministic method and does not account for scene uncertainty. Therefore, in this section, we demonstrate that integrating neural processes can enhance Splatter Image's performance.

% Specifically, we employ the UNet encoder to generate a latent variable, and then sample a scene-specific latent vector to estimate Gaussian parameters through the decoder. For multiple-view ($N$) images, we first aggregate multi-view latent features and then infer the latent variables (mean and variance). We sample $N$ times to probabilistically estimate the Gaussian parameters. The prior and posterior distributions are derived from context and target images, respectively. In addition to the original reconstruction loss, we introduce a KL divergence constraint between the prior and posterior distributions, guiding the model to achieve richer representation with limited observations.

% Experiments are conducted on the CO3D dataset. 


%%%%%%%%%%%%%%%%%%%%%%%%%%%%%%%%%%%%%%%%%%%%%%%%%%%%%%%%%%%%%%%%%%%%%%%%%%%%%%%%%%%%%%%%%%%%%%%%%%

% \subsection{Image Regression}
% \label{sec:image-regression}



% \begin{figure}[t]
%     \centering
%     \begin{subtable}[b]{0.42\textwidth}
%     % \vspace{-6mm}
%     \begin{tabular}{lcc}
%     \toprule
%                  & CelebA & Imagenette \\ \midrule
%     Learned Init \citep{tancik2021learned} & 30.37  & 27.07       \\
%     TransINR~\citep{chen2022transformers}         & 31.96  & 29.01       \\ 
%     % \hline
%     \rowcolor{lightblue}
%     \method{} (Ours)         & \textbf{33.41}  & \textbf{29.82}      \\ 
%     \bottomrule
%     \end{tabular}
%     \caption{Quantitative results. \method{} outperforms baseline methods consistently on both datasets.}
%     \label{tab:image-regression}
%     \end{subtable} \hfill
%     \begin{subtable}[b]{0.54\textwidth}
%     \includegraphics[width=\textwidth]{Figures/image-regression0.pdf} % Adjust the size and filename as needed
%     \caption{Visualizations on CelebA (left) and Imagenette (right), respectively.} % Caption for the figure
%     \label{fig:visualization-image-regression}
% \end{subtable}
% % \vspace{-2mm}
% \caption{\textbf{Quantitative results and visualizations} of image regression on CelebA and Imagenette.}
% \vspace{-3mm}
% \end{figure}

% \begin{figure}[t!]
%   \centering
%   \includegraphics[width=0.99\textwidth]{Figures/image-completion.pdf} % Adjust the size and filename as needed
%   \vspace{-3mm}
%   \caption{\textbf{Image completion visualization} on CelebA using $10\%$ (left) and $20\%$ (right) context.}
%   \label{fig:completion}
%   \vspace{-3mm}
% \end{figure}


% \noindent{\textbf{Setup.}} Image regression is a common task used for evaluating INRs' capacity of representing a signal~\citep{tancik2021learned,sitzmann2020implicit}. 
% We employ two real-world image datasets as used in previous works~\citep{chen2022transformers,tancik2021learned,gu2023generalizable}. The CelebA dataset~\citep{liu2015deep} encompasses approximately 202,000 images of celebrities, partitioned into training (162,000 images), validation (20,000 images), and test (20,000 images) sets. The Imagenette dataset~\citep{imagenette}, a curated subset comprising 10 classes from the 1,000 classes in ImageNet~\citep{deng2009imagenet}, consists of roughly 9,000 training images and 4,000 testing images. In order to compare with previous methods, we conduct image regression experiments. The context set is an image and the task is to learn an implicit function that regresses the image pixels well in terms of PSNR.
% %\str{What is the point of image regresssion when the image itself is used as context? Why not just copy the context??}
% %\str{The following is a bit strante, is this still image regression? Why not have a separate section?}
% %\str{Is Image Regression the standard name for this task?}


% \noindent{\textbf{Implementation Details.}} 
% Following TransINR~\citep{chen2022transformers}, we resize each image into $178\times 178$, and use patch size 9 for the tokenizer. The self-attention module remains the same as the one in the NeRF experiments (Sec. \ref{sec:nerf-results}). For the Gaussian bases, we predict the 2D Gaussians instead of the 3D. 
% The hierarchical latent variables are inferred in image-level and pixel-level. 





% %The self-attention and global variable remain the same as the one in the NeRF experiments. %We do not use the pixel variable modulation for the computation concern. However, our method still has competitive performance.   

% % \begin{table}[t]
% % \centering
% % \vspace{-6mm}
% % \caption{Quantitative results of image regression.}
% % \label{tab:image-regression}
% % \begin{tabular}{lcc}
% % \toprule
% %              & CelebA & Imagenette \\ \midrule
% % Learned Init \citep{tancik2021learned} & 30.37  & 27.07       \\
% % TransINR~\citep{chen2022transformers}         & 31.96  & 29.01       \\ 
% % \hline
% % \method{} (Ours)         & \textbf{33.41}  & \textbf{29.82}      \\ 
% % \bottomrule
% % \end{tabular}
% % \end{table}



% \noindent{\textbf{Results.}} The quantitative comparison of \method{} for representing the 2D image signals is presented in Table~\ref{tab:image-regression}. \method{} outperforms the baseline methods on both CelebA and Imagenette datasets significantly, showing better generalization ability and representation capacity than baselines. 
% %Note that the Imagenette is a more diverse dataset than the CelebA. The better performance shows that. 
% Fig.~\ref{fig:visualization-image-regression} shows the ability of \method{} to recover the high-frequency details for image regression.
% %regress the image closely to the ground truth, indicating the capability of capturing the detailed texture information. 


% \noindent {\textbf{Image Completion Visualization.}} We also conduct experiments of \method{} on image completion (also called image inpainting), which is a more challenging variant of image regression. Essentially, only part of the pixels are given as context, while the INR functions are required to complete the full image. Visualizations in Fig.~\ref{fig:completion} demonstrate the generalization ability of our method to recover realistic images with fine details based on very limited context ($10 \% - 20\%$ pixels). %, 




% \begin{figure}[t!]
%   \centering
%   \includegraphics[width=0.99\textwidth]{Figures/image-completion.pdf} % Adjust the size and filename as needed
%   \vspace{-3mm}
%   \caption{\textbf{Image completion visualization} on CelebA using $10\%$ (left) and $20\%$ (right) context.}
%   \label{fig:completion}
%   \vspace{-3mm}
% \end{figure}



% \begin{figure}[t!]
%   \centering
%   \includegraphics[width=1\textwidth]{Figures/image-regression-basis.pdf} % Adjust the size and filename as needed
%   \caption{\textbf{Visualization of geometric bases (Gaussian)} on the context image, which reveals the structure of the object.} % Caption for the figure
%   \label{fig:visualization}
%   \vspace{-5mm}
% \end{figure}


% \noindent{\textbf{Visualization of Geometric Bases.}}
% Moreover, we also visualize the learned Gaussian bases on the image regression task. As shown in Fig. \ref{fig:visualization}, the bases are more concentrated on the objects and complex backgrounds in the image, while sparse on the simple complex. The visualizations indicate that the geometric bases do encode structure information from the context data.





\appendix
\section*{Appendix}
\section{Discussion: Scope and Ethics}
\label{appendix:scope}
In this work, we evaluate our method on six core scene-aware tasks: existence, count, position, color, scene, and HOI reasoning. We select these tasks as they represent core aspects of multimodal understanding which are essential for many applications. Meanwhile, we do not extend our evaluation to more complex reasoning tasks, such as numerical calculations or code generation, because SOTA diffusion models like SDXL are not yet capable of handling these tasks effectively. Fine-tuning alone cannot overcome the fundamental limitations of these models in generating images that require symbolic logic or complex reasoning. Additionally, we avoid tasks with ethical concerns, such as generating images of specific individuals (e.g., for celebrity recognition task), to mitigate risks related to privacy and misuse. Our goal was to ensure that our approach focuses on technically feasible and responsible AI applications. Expanding to other tasks will require significant advancements in diffusion model capabilities and careful consideration of ethical implications.

\section{Limitations and Future Work}
While our Multimodal Context Evaluator proves effective in enhancing the fidelity of generated images and maintaining diversity, \method is built using pre-trained diffusion models such as SDXL and MLLMs like LLaVA, it inherently shares the limitations of these foundation models. \method still faces challenges with complex reasoning tasks such as numerical calculations or code generation due to the symbolic logic limitations inherent to SDXL. Additionally, during inference, the MLLM context descriptor occasionally generates incorrect information or ambiguous descriptions initially, which can lead to lower fidelity in the generated images. Figure~\ref{fig:failure} further illustrates these observations.

\method currently focuses on single attributes like count, position, and color as part of the multimodal context. This is because this task alone poses significant challenges to existing methods, which \method effectively addresses. A potential direction for future work is to broaden the applicability of \method to synthesize images with multiple scene attributes in the multimodal context as part of compositional reasoning tasks.


\begin{figure}[!h]
    \centering
    \includegraphics[width=\linewidth]{figures/failures.pdf}
    \caption{Failure cases of \method. (a) Our method fails due to the symbolic logic limitation of existing pre-trained SDXL. (b) Initially incorrect descriptions generated by MLLMs lead to low fidelity of generated images. (c) Context description generated by MLLMs is ambiguous and does not directly relate to the text guidance, the spoon can be both inside or outside the bowl.}
    \label{fig:failure}
\end{figure}

\section{Prompt Templates}
\label{appendix:prompts}
Figure~\ref{fig:prompt_templates}~(a-c) showcases the prompt templates used by \method to fine-tune diffusion models specifically on each task including VQA, HOI Reasoning, and Object-Centric benchmarks. It's worth noting that we designed the prompt such that it provides detailed instruction to MLLMs on which scene attributes to focus. We also evaluate the effectiveness of our designed prompt templates by fine-tuning \method with a generic prompt as illustrated in Figure~\ref{fig:prompt_templates}~(d). Table~\ref{table:prommpt} indicates that without using our designed prompt template, the MLLM is not properly instructed to generate specific context description thus leading to reduced performance after fine-tuning on MME tasks. We believe that when using a generic prompt, MLLM is not able to receive sufficient grounding about the multimodal context leading to information loss on key scene attributes.


\begin{table}[!h]
\centering
\footnotesize
\caption{Effectiveness of the prompt template on fine-tuning \method on MME Perception.}
\resizebox{1\linewidth}{!}{
\begin{tabular}{clcccccccccc}
\toprule
 \textbf{MLLM} & \multirow{2}{*}{\textbf{\method}} & \multicolumn{2}{c}{\textbf{Existence}} & \multicolumn{2}{c}{\textbf{Count}} & \multicolumn{2}{c}{\textbf{Position}} & \multicolumn{2}{c}{\textbf{Color}} & \multicolumn{2}{c}{\textbf{Scene}} \\
 \textbf{Name} & & ACC & ACC+ & ACC & ACC+ & ACC & ACC+ & ACC & ACC+ & ACC & ACC+ \\
 \midrule
 \multirow{3}{*}{\makecell{\textbf{LLaVA }  \\ \textbf{v1.6 7B} \\ \citep{liu2024improved}}}
 &w/ prompt template & \textbf{96.67}  & \textbf{93.33}  & \textbf{83.33}  & \textbf{70.00}  & \textbf{81.67}  & \textbf{66.67} & \textbf{95.00}  & \textbf{93.33}  & \textbf{87.75} & \textbf{74.00} \\
 \cmidrule{2-12}
 & \multirow{2}{*}{w/ generic prompt} & 91.67 & 83.33 & 75.00 & 56.67 & \textbf{81.67} & 63.33 & 91.67 & 83.33 & 87.25 & 73.00 \\
 & & {\scriptsize \color{red}\textbf{$\downarrow$ 5.00}} & {\scriptsize \color{red}\textbf{$\downarrow$ 10.00}} & {\scriptsize \color{red}\textbf{$\downarrow$ 8.33}} & {\scriptsize \color{red}\textbf{$\downarrow$ 13.33}} & - &  {\scriptsize \color{red}\textbf{$\downarrow$ 3.34}} & {\scriptsize \color{red}\textbf{$\downarrow$ 3.33}} & {\scriptsize \color{red}\textbf{$\downarrow$ 10.00}} & {\scriptsize \color{red}\textbf{$\downarrow$ 0.50}} & {\scriptsize \color{red}\textbf{$\downarrow$ 1.00}}\\
 \midrule
 \multirow{3}{*}{\makecell{\textbf{InternVL }  \\ \textbf{2.0 8B}\\ \citep{chen2024internvl}}} 
 &w/ prompt template & \textbf{98.33}  & \textbf{96.67} & \textbf{86.67} & \textbf{73.33}  & \textbf{78.33}  & \textbf{63.33}  & \textbf{98.33}  & \textbf{96.67}  & \textbf{86.25} & \textbf{71.00} \\
 \cmidrule{2-12}
 & \multirow{2}{*}{w/ generic prompt} & 91.67 & 83.33 & 80.00 & 60.00 & 71.67 & 50.00 & 91.67 & 83.33 & 84.50 & 69.00 \\
 & & {\scriptsize \color{red}\textbf{$\downarrow$ 6.66}} &  {\scriptsize \color{red}\textbf{$\downarrow$ 13.34}} & {\scriptsize \color{red}\textbf{$\downarrow$ 6.67}} & {\scriptsize \color{red}\textbf{$\downarrow$ 13.33}} & {\scriptsize \color{red}\textbf{$\downarrow$ 6.66}} & {\scriptsize \color{red}\textbf{$\downarrow$ 13.33}} & {\scriptsize \color{red}\textbf{$\downarrow$ 6.66}} & {\scriptsize \color{red}\textbf{$\downarrow$ 13.34}} & {\scriptsize \color{red}\textbf{$\downarrow$ 1.75}} & {\scriptsize \color{red}\textbf{$\downarrow$ 2.00}}\\
\bottomrule
\end{tabular}
}
\label{table:prommpt}
\end{table}

\begin{figure}[!h]
    \centering
    \includegraphics[width=\linewidth]{figures/prompt_template.pdf}
    \caption{Prompt templates (a-c) used by \method to fine-tune the diffusion model on each task including VQA, HOI Reasoning, and Object Centric benchmarks. The generic prompt (d) is also included to evaluate the effectiveness of prompt template.}
    \label{fig:prompt_templates}
\end{figure}
\section{Inference Pipeline}
\label{appendix:inference}
In the inference pipeline of \method (Figure~\ref{fig:inference}), the text guidance $\mathbf{g}$ includes only the question corresponding to the reference image $\mathbf{x}$. The answer is excluded for fair evaluation. Moreover, we remove Multimodal Context Evaluator, and the generated image $\hat{\mathbf{x}}$ is the final output.
\begin{figure}[!h]
    \centering
    \includegraphics[width=\linewidth]{figures/inference.pdf}
    \caption{Inference pipeline of \method}
    \label{fig:inference}
\end{figure}

\begin{figure}[!h]
    \centering
    \includegraphics[width=\linewidth]{figures/diversity_compact_caption.pdf}
    \vspace{-5mm}
    \caption{Examples of context description from MLLM in the inference pipeline where answers are not included in text guidance.}
    \label{fig:diversity_compact_caption}
\end{figure}



\section{Ablation Study on BLIP-2 QFormer}
Our design choice to leverage BLIP-2 QFormer in \method as the multimodal context evaluator facilitates the formulation of our novel Global Semantic and Fine-grained Consistency Rewards. These rewards enable \method to be effective across all tasks as seen in Table~\ref{table:clip}. On replace with a less powerful multimodal context encoder such as CLIP ViT-G/14, we can only implement the global semantic reward as the cosine similarity between the text features and generated image features. As a result, while the setting can maintain performance on coarse-level tasks such as Scene and Existence, there is a noticeable decline on fine-grained tasks like Count and Position. This demonstrates the effectiveness of our design choices in \method and shows that using less powerful alternatives, without the ability to provide both global and fine-grained alignment, affects the fidelity of generated images.

\begin{figure*}[t]
\centering
\includegraphics[width=15.5cm]{figures/clip_zeroshot.png}\\
\caption{CLIP a) training and b) zero-shot inference framework}
\label{fig:clip} 
\end{figure*}


\section{Additional Evaluation on MME Artwork}

To explore the method's ability to work on tasks involving more nuanced or abstract text guidance beyond factual scene attributes, we evaluate \method on an additional task of MME Artwork. This task focuses on image style attributes that are more nuanced/abstract such as the following question-answer pair -- Question: ``Does this artwork exist in the form of mosaic?'', Answer: ``No''.

Table~\ref{table:artwork_reasoning} summarizes the evaluation. We can observe that \method outperforms all existing methods on both ACC and ACC+, implying its higher effectiveness in generating images with high fidelity (in this case, image style preservation) compared to existing methods. This provides evidence that \method can generalize to tasks involving abstract/nuanced attributes such as image style. Figure~\ref{fig:artwork} further shows qualitative comparison between image generation methods on the MME Artwork task.

\begin{table}[h]
\centering
\caption{Comparison on Artwork benchmark and Visual Reasoning task. \method outperforms SOTA image generation and augmentation techniques.}
\resizebox{\linewidth}{!}{
\begin{tabular}{@{}l@{ }ccccccc@{}}
\toprule
\textbf{Method} & \textbf{Real only} & \textbf{RandAugment} &  \textbf{Image Variation} & \textbf{Image Translation} & \textbf{Textual Inversion} & \textbf{I2T2I SDXL} & \textbf{\method} \\
\midrule
\textbf{Artwork ACC} & 69.50 & 69.25 & 69.00 & 67.00 & 66.75 & 68.00 & \textbf{70.25} \\
\textbf{Artwork ACC+} & 41.00 & 41.00 & 40.00 & 38.00 & 37.50 & 38.00 & \textbf{41.50} \\
\midrule
\textbf{Reasoning ACC} & 69.29 & 67.86 & 69.29 & 69.29 & 67.14 & 72.14 & \textbf{72.86} \\
\textbf{Reasoning ACC+} & 42.86 & 40.00 & 41.40 & 40.00 & 37.14 & 47.14 & \textbf{48.57} \\

\bottomrule
\end{tabular}
}
\label{table:artwork_reasoning}
\end{table}


\begin{figure}[!h]
    \centering
    \includegraphics[width=\linewidth]{figures/artwork.pdf}
    \caption{Qualitative comparison on the Artwork task between image generation method. \method can preserve both diversity and fidelity of the reference image in a more abstract domain.}
    \label{fig:artwork}
\end{figure}


\section{Additional Evaluation on MME Commonsense Reasoning}
We have additionally performed our evaluation to more complex tasks such as Visual Reasoning using the MME Commonsense Reasoning benchmark. Results in Table~\ref{table:artwork_reasoning} highlight \method's ability to generalize effectively across diverse domains and complex reasoning tasks, demonstrating its broader applicability. Figure~\ref{fig:reasoning} further shows qualitative comparison between image generation methods on the MME Commonsense Reasoning task.

\begin{figure}[!h]
    \centering
    \includegraphics[width=\linewidth]{figures/reasoning.pdf}
    \caption{Qualitative comparison on the Commonsense Reasoning task between image generation method. \method can preserve both diversity and fidelity of the reference image in a more abstract domain.}
    \label{fig:reasoning}
\end{figure}
\section{FID Scores}
% \textcolor{blue}{We compute FID scores of traditional augmentation and image generation methods. Table~\ref{table:fid} shows that the data distribution of generated images by RandAugment and Image Translation are closer to the real distribution as these methods only change images minimally. We also want to emphasize that even though the FID metric evaluates the quality of generated images, it can not measure the diversity of generated images. \method with rewards fine-tuning achieves a competitive score. As we showed in the diversity analysis in Table~\ref{table:diversity} in the main paper, \method performs significantly better than these ``minimal change" methods while still achieving a competitive FID score. We believe this is a worldwide trade-off.}

We compute FID scores for \method and the different baselines (traditional augmentation and image generation methods) and tabulate the numbers in Table~\ref{table:fid}. FID is a valuable metric for assessing the quality of generated images and how closely the distribution of generated images matches the real distribution. However, \textit{FID does not account for the diversity among the generated images}, which is a critical aspect of the task our work targets~(i.e., how can we generate high fidelity images, preserving certain scene attributes, while still maintaining high diversity?). We also illustrate the shortcomings of FID for the task in Figure~\ref{fig:fid_diversity} where we compare generated images across methods. We observe that RandAugment and Image Translation achieve lower FID scores than \method~(w/ finetuning) because they compromise on diversity by only minimally changing the input image, allowing their generated image distribution to be much closer to the real distribution. While \method has a higher FID score than RandAugment and Image Translation, Figure~\ref{fig:fid_diversity} shows that it is able to preserve the scene attribute w.r.t.~multimodal context while generating an image that is significantly different from than original input image. Therefore, it accomplishes the targeted task more effectively, with both high fidelity and high diversity.

\begin{table}[h]
\centering
\caption{FID scores of traditional augmentation and image generation methods. Lower is better.}
\resizebox{\linewidth}{!}{
\begin{tabular}{@{}l@{ }ccccccc@{}}
\toprule
\multirow{2}{*}{\textbf{Method}} & \multirow{2}{*}{\textbf{RandAugment}} & \multirow{2}{*}{\textbf{I2T2I SDXL}} & \multirow{2}{*}{\textbf{Image Variation}} & \multirow{2}{*}{\textbf{Image Translation}} & \multirow{2}{*}{\textbf{Textual Inversion}} & \multicolumn{2}{c}{\textbf{\method}} \\
& & & & & & \ding{55} fine-tuning & \ding{51} fine-tuning\\
\midrule
\textbf{FID score $\downarrow$} & \textbf{15.93} & 18.35 & 17.66 & 16.29 & 20.84 & 17.78 & 16.55 \\
\bottomrule
\end{tabular}
}
\label{table:fid}
\end{table}

\begin{figure}[!h]
    \centering
    \includegraphics[width=\linewidth]{figures/fid_diversity.pdf}
    \caption{While RandAugment and Image Translation achieve lower FID scores, \method balances fidelity and diversity effectively.}
    \label{fig:fid_diversity}
\end{figure}

\section{User Study}
% \textcolor{blue}{We created a survey form with 50 questions (10 questions per MME task). In each survey question, users were shown: a reference image, a related question, and two generated images from different methods (I2T2I SDXL vs. \method). Users are asked to select the generated image(s) that preserve the attribute referred to by the question in relation to reference image. We collected form responses from 70 people. Table~\ref{table:user_study} shows that \method significantly outperforms I2T2I SDXL in terms of fidelity across all tasks on MME benchmark. We have some examples of survey questions in Figure~\ref{fig:user_study_examples}.}

We conduct a user study where we create a survey form with 50 questions (10 questions per MME Perception task). In each survey question, we show users a reference image, a related question, and a generated image each from two different methods (baseline I2T2I SDXL vs \method). We ask users to select the generated images(s) (either one or both or neither of them) that preserve the attribute referred to by the question in relation to the reference image. If an image is selected, it denotes high fidelity in generation. We collect form responses from 70 people for this study. We compute the percentage of total generated images for each method that were selected by the users as a measure of fidelity. Table~\ref{table:user_study} summarizes the results and shows that \method significantly outperforms I2T2I SDXL in terms of fidelity across all tasks on the MME Perception benchmark. We have some examples of survey questions in Figure~\ref{fig:user_study_examples}.

\begin{figure}[htp]
  \centering
   \includegraphics[width=\columnwidth]{Assets/userstudy_grid.pdf}
   
   \caption{\textbf{User study results.} Users preference percentage of our method compared to other methods in terms of text alignment, visual quality, and overall preference.
   }
   \label{fig:user_study}
\end{figure}
\begin{figure}[!h]
    \centering
    \includegraphics[width=\linewidth]{figures/user_study_examples.pdf}
    \caption{Some examples of our survey questions to evaluate the fidelity of generated images from I2T2I SDXL and \method.}
    \label{fig:user_study_examples}
\end{figure}
\section{Training Performance on Bongard HOI Dataset}
% \textcolor{blue}{We conducted an additional experiment by training a CNN baseline ResNet50 \citep{he2016deep} model on the Bongard-HOI training set with traditional augmentation and other image generation methods, using the same number of training iterations. As shown in Table~\ref{table:hoi_training}, \method consistently outperforms other methods across all test splits. However, as discussed in Subsection~\ref{sec:benchmark_formulation}, our primary focus on test-time evaluation ensures fair comparisons by avoiding variability in training behavior caused by differences in model architectures, data distributions, and training configurations.}

Following the existing method \citep{shu2022testtime}, we conduct an additional experiment by training a ResNet50 \citep{he2016deep} model on the Bongard-HOI \citep{jiang2022bongard} training set with traditional augmentation and Hummingbird generated images. We compare the performance with other image generation methods, using the same
number of training iterations. As shown in Table~\ref{table:hoi_training}, \method consistently outperforms all the baselines across all test splits. In the paper, as discussed in Section~\ref{sec:benchmark_formulation}, we focus primarily on test-time evaluation because it eliminates the variability introduced by model training due to multiple external variables such as model architecture, data distribution, and training configurations, and allows for a fairer comparison where the evaluation setup remains fixed.

\begin{table}[!h]
\centering
\footnotesize
\caption{Comparison on Human-Object Interaction~(HOI) Reasoning by training a CNN-baseline ResNet50 with image augmentation and generation methods. \method outperforms SOTA methods on all $4$ test splits of Bongard-HOI dataset.}
\resizebox{0.8\linewidth}{!}{
\begin{tabular}{lccccc}
\toprule
\multirow{3}{*}{Method} & \multicolumn{4}{c}{Test Splits} & \multirow{3}{*}{Average} \\
\cmidrule{2-5}
 & seen act., & unseen act., & seen act., & unseen act., &  \\
 & seen obj. & seen obj. & unseen obj. & unseen obj. & \\
  % & seen act., seen obj. & unseen act., seen obj. & seen act., unseen obj. & unseen act., unseen obj. &  \\
 % &  &  &  & & \\
\midrule
CNN-baseline (ResNet50) & 50.03\xspace\xspace\xspace\xspace\xspace\xspace\xspace\xspace\xspace\xspace & 49.89\xspace\xspace\xspace\xspace\xspace\xspace\xspace\xspace\xspace\xspace & 49.77\xspace\xspace\xspace\xspace\xspace\xspace\xspace\xspace\xspace\xspace & 50.01\xspace\xspace\xspace\xspace\xspace\xspace\xspace\xspace\xspace\xspace & 49.92\xspace\xspace\xspace\xspace\xspace\xspace\xspace\xspace\xspace\xspace \\
RandAugment \citep{cubuk2020randaugment} & 51.07 {\scriptsize \color{ForestGreen}$\uparrow$ 1.04} & 51.14 {\scriptsize \color{ForestGreen}$\uparrow$ 1.25} & 51.79 {\scriptsize \color{ForestGreen}$\uparrow$ 2.02} & 51.73 {\scriptsize \color{ForestGreen}$\uparrow$ 1.72} & 51.43 {\scriptsize \color{ForestGreen}$\uparrow$ 1.51} \\
Image Variation \citep{xu2023versatile} & 41.78 {\scriptsize \color{red}$\downarrow$ 8.25} & 41.29 {\scriptsize \color{red}$\downarrow$ 8.60} & 41.15 {\scriptsize \color{red}$\downarrow$ 8.62} & 41.25 {\scriptsize \color{red}$\downarrow$ 8.76} & 41.37 {\scriptsize \color{red}$\downarrow$ 8.55} \\
Image Translation \citep{pan2023boomerang} & 46.60 {\scriptsize \color{red}$\downarrow$ 3.43} & 46.94 {\scriptsize \color{red}$\downarrow$ 2.95} & 46.38 {\scriptsize \color{red}$\downarrow$ 3.39} & 46.50 {\scriptsize \color{red}$\downarrow$ 3.51} & 46.61 {\scriptsize \color{red}$\downarrow$ 3.31} \\
Textual Inversion \citep{gal2022image} & \xspace37.67 {\scriptsize \color{red}$\downarrow$ 12.36} & \xspace37.52 {\scriptsize \color{red}$\downarrow$ 12.37} & \xspace38.12 {\scriptsize \color{red}$\downarrow$ 11.65} & \xspace38.06 {\scriptsize \color{red}$\downarrow$ 11.95} & \xspace37.84 {\scriptsize \color{red}$\downarrow$ 12.08} \\
I2T2I SDXL \citep{podell2023sdxl} & 51.92 {\scriptsize \color{ForestGreen}$\uparrow$ 1.89} & 52.18 {\scriptsize \color{ForestGreen}$\uparrow$ 2.29} & 52.25 {\scriptsize \color{ForestGreen}$\uparrow$ 2.48} & 52.15 {\scriptsize \color{ForestGreen}$\uparrow$ 2.14} & 52.13 {\scriptsize \color{ForestGreen}$\uparrow$ 2.21}\\
\textbf{\method} & \textbf{53.71 {\scriptsize \color{ForestGreen}$\uparrow$ 3.68}} & \textbf{53.55 {\scriptsize \color{ForestGreen}$\uparrow$ 3.66}} & \textbf{53.69 {\scriptsize \color{ForestGreen}$\uparrow$ 3.92}} & \textbf{53.41 {\scriptsize \color{ForestGreen}$\uparrow$ 3.40}} & \textbf{53.59 {\scriptsize \color{ForestGreen}$\uparrow$ 3.67}} \\
\bottomrule
\end{tabular}
}
\label{table:hoi_training}
\end{table}



\section{Random Seeds Selection Analysis}
We conduct an additional experiment, varying the number of random seeds from $10$ to $100$. The results are presented in the boxplot in Figure~\ref{fig:boxplot}, which shows the distribution of the mean L2 distances of generated image features from Hummingbird across different numbers of seeds.


The figure demonstrates that the difference in the distribution of the diversity (L2) scores across the different numbers of random seeds is statistically insignificant. So while it is helpful to increase the number of seeds for improved confidence, we observe that it stabilizes at 20 random seeds. This analysis suggests that using $20$ random seeds also suffices to capture the diversity of generated images without significantly affecting the robustness of the analysis.

% We conduct an additional experiment where we vary the number of seeds from 10 to 100. We present the results as a boxplot in Appendix K, Figure 15 which shows the distribution of the mean L2 distances of generated image features from Hummingbird across different numbers of seeds.

% The figure demonstrates that the difference in the distribution of the diversity (L2) scores across the different numbers of random seeds is statistically insignificant. So while it is helpful to increase the number of seeds for improved confidence, we observe that it stabilizes at 20 random seeds. This analysis suggests that using 20 random seeds also suffices to capture the diversity of generated images without significantly affecting the robustness of the analysis.

\begin{figure}[!h]
    \centering
    \includegraphics[width=0.8\linewidth]{figures/diversity_boxplot_rectangular.pdf}
    \caption{Diversity analysis across varying numbers of random seeds (10 to 100) using mean L2 distances of generated image features from \method. The box plot demonstrates consistent diversity scores as the number of seeds increases, indicating that performance stabilizes around 20 random seeds.}
    \label{fig:boxplot}
\end{figure}

\section{Further Explanation of Multimodal Context Evaluator}
The Global Semantic Reward, \(\mathcal{R}_\textrm{global}\), ensures alignment between the global semantic features of the generated image \(\mathbf{\hat{x}}\) and the textual context description \(\mathcal{C}\). This reward leverages cosine similarity to measure the directional alignment between two feature vectors, which can be interpreted as maximizing the mutual information \(I(\mathbf{\hat{x}}, \mathcal{C})\) between the generated image \(\mathbf{\hat{x}}\) and the context description \(\mathcal{C}\). Mutual information quantifies the dependency between the joint distribution \(p_{\theta}(\mathbf{\hat{x}}, \mathcal{C})\) and the marginal distributions. In conditional diffusion models, the likelihood \(p_{\theta}(\mathbf{\hat{x}} \vert \mathcal{C})\) of generating \(\mathbf{\hat{x}}\) given \(\mathcal{C}\) is proportional to the joint distribution:
\[
p_{\theta}(\mathbf{\hat{x}} \vert \mathcal{C}) = \frac{p_{\theta}(\mathbf{\hat{x}}, \mathcal{C})}{p(\mathcal{C})} \propto p_{\theta}(\mathbf{\hat{x}}, \mathcal{C}),
\]
where \(p(\mathcal{C})\) is the marginal probability of the context description, treated as a constant during optimization. By maximizing \(\mathcal{R}_\textrm{global}\), which aligns global semantic features, the model indirectly maximizes the mutual information \(I(\mathbf{\hat{x}}, \mathcal{C})\), thereby enhancing the likelihood \(p_{\theta}(\mathbf{\hat{x}} \vert \mathcal{C})\) in the conditional diffusion model.


The Fine-Grained Consistency Reward, $\mathcal{R}_{\textrm{fine-grained}}$, captures detailed multimodal alignment between the generated image $\mathbf{\hat{x}}$ and the textual context description $\mathcal{C}$. It operates at a token level, leveraging bidirectional self-attention and cross-attention mechanisms in the BLIP-2 QFormer, followed by the Image-Text Matching (ITM) classifier to maximize the alignment score.

\textbf{Self-Attention on Text Tokens:}
    Text tokens $\mathcal{T}_{\mathrm{tokens}}$ undergo self-attention, allowing interactions among words to capture intra-text dependencies:
    \begin{equation}
        \mathcal{T}_{\mathrm{attn}} = \tt{SelfAttention}(\mathcal{T}_{\mathrm{tokens}})
    \end{equation}

\textbf{Self-Attention on Image Tokens:}
    Image tokens $\mathcal{Z}$ are derived from visual features of the generated image $\mathbf{\hat{x}}$ using a cross-attention mechanism:
    \begin{equation}
        \mathcal{Z} = \tt{CrossAttention}(\mathcal{Q}_{\mathrm{learned}}, \mathcal{I}_{\mathrm{tokens}}(\mathbf{\hat{x}}))
    \end{equation}
    These tokens then pass through self-attention to extract intra-image relationships:
    \begin{equation}
        \mathcal{Z}_{\mathrm{attn}} = \tt{SelfAttention}(\mathcal{Z})
    \end{equation}

\textbf{Cross-Attention between Text and Image Tokens:}
    The text tokens $\mathcal{T}_{\mathrm{attn}}$ and image tokens $\mathcal{Z}_{\mathrm{attn}}$ interact through cross-attention to focus on multimodal correlations:
    \begin{equation}
        \mathcal{F} = \tt{CrossAttention}(\mathcal{T}_{\mathrm{attn}}, \mathcal{Z}_{\mathrm{attn}})
    \end{equation}

\textbf{ITM Classifier for Alignment:}
    The resulting multimodal features $\mathcal{F}$ are fed into the ITM classifier, which outputs two logits: one for positive match ($j=1$) and one for negative match ($j=0$). The positive class ($j=1$) indicates strong alignment between the image-text pair, while the negative class ($j=0$) indicates misalignment:
    \begin{equation}
        \mathcal{R}_{\textrm{fine-grained}} = \tt{ITM\_Classifier}(\mathcal{F})_{j=1}
    \end{equation}

The ITM classifier predicts whether the generated image and the textual context description match. Maximizing the logit for the positive class $j=1$ corresponds to maximizing the log probability $\log p(\mathbf{\hat{x}}, \mathcal{C})$ of the joint distribution of image and text. This process aligns the fine-grained details in $\mathbf{\hat{x}}$ with $\mathcal{C}$, increasing the mutual information between the generated image and the text features.

\textbf{Improving fine-grained relationships of CLIP.} While the CLIP Text Encoder, at times, struggles to accurately capture spatial features when processing longer sentences in the Multimodal Context Description, \method addresses this limitation by distilling the global semantic and fine-grained semantic rewards from BLIP-2 QFormer into a specific set of UNet denoiser layers, as mentioned in the implementation details under Appendix~\ref{appendix:impl}~(i.e., Q, V transformation layers including $\tt{to\_q, to\_v, query, value}$). This strengthens the alignment between the generated image tokens~(Q) and input text tokens from the Multimodal Context Description~(K, V) in the cross-attention mechanism of the UNet denoiser. As a result, we obtain generated images with improved fidelity, particularly w.r.t.~spatial relationships, thereby helping to mitigate the shortcomings of vanilla CLIP Text Encoder in processing the long sentences of the Multimodal Context Description.

To illustrate further, a Context Description like “the dog under the pool” is processed in three steps: (1) self-attention is applied to the text tokens (K, V), enabling spatial terms like “dog,” “under,” and “pool” to interact; (2) self-attention is applied to visual features represented by the generated image tokens (Q) to extract intra-image relationships (3) cross-attention aligns this text features with visual features. The resulting alignment scores are used to compute the mean and select the positive class for the reward. Our approach to distill this reward into the cross-attention layers therefore ensures that spatial relationships and other fine-grained attributes are effectively captured, improving the fidelity of generated images.


\section{The Choice of Text Encoder in SDXL and BLIP-2 QFormer}

The choice of text encoder in our pipeline is to leverage pre-trained models for their respective strengths. SDXL inherently uses the CLIP Text Encoder for its generative pipeline, as it is designed to process text embeddings aligned with the CLIP Image Encoder. In the Multimodal Context Evaluator, we use the BLIP-2 QFormer, which is pre-trained with a BERT-based text encoder.

\section{Textual Inversion for Data Augmentation}
In our experiments, we applied Textual Inversion for data augmentation as follows: given a reference image, Textual Inversion learns a new text embedding that captures the context of the reference image (denoted as $<$context$>$). This embedding is then used to generate multiple augmented images by employing the prompt: ``a photo of $<$context$>$". This approach allows Textual Inversion to create context-relevant augmentations for comparison in our experiments.

\section{Convergence Curve}
To evaluate convergence, we monitor the training process using the Global Semantic Reward and Fine-Grained Consistency Reward as criteria. Specifically, we observe the stabilization of these rewards over training iterations. Figure~\ref{fig:convergence} presents the convergence curves for both rewards, illustrating their gradual increase followed by stabilization around 50k iterations. This steady state indicates that the model has learned to effectively align the generated images with the multimodal context.

\begin{figure}[!h]
    \centering
    \includegraphics[width=\linewidth]{figures/convergence.pdf}
    \caption{Convergence curves of Global Semantic and Fine-Grained Consistency Rewards}
    \label{fig:convergence}
\end{figure}


\section{Fidelity Evaluation using GPT-4o}
In addition to the results above, we compute additional metrics for fidelity, which measure how well the model preserves scene attributes when generating new images from a reference image. For this, we use GPT-4o (model version: 2024-05-13) as the MLLM oracle for a VQA task on the MME Perception benchmark \citep{fu2024mme}. 
% We use a MLLM as an oracle for a visual question-answering (VQA) task on the MME Perception benchmark \citep{fu2024mme}. In this experiment, we use GPT-4o (model version: 2024-05-13) as the oracle. 
We evaluate \method with and without fine-tuning process.

The MME dataset consists of Yes/No questions, with a positive and a negative question for every reference image. To measure fidelity, we measure the rate at which the oracle's answer remains consistent across the reference and the generated image for every image in the dataset. We run the experiment multiple times and report the average numbers in Table~\ref{table:fidelity_comparison}. We see that fine-tuning the base SDXL with our novel rewards results in an average increase of $2.99\%$ in fidelity.

\begin{table}[!h]
\centering
\footnotesize
\caption{Fidelity between reference and generated images from \method with and without fine-tuning.}
\resizebox{0.9\linewidth}{!}{
\begin{tabular}{clccc}
\toprule
 \textbf{MLLM Oracle} & \textbf{\method} & \textbf{Fidelity on ``Yes"} & \textbf{Fidelity on ``No"} & \textbf{Overall Fidelity} \\
 \midrule
 \multirow{2}{*}{\makecell{\textbf{GPT-4o}\\\textbf{Ver: 2024-05-13}}}
 & w/o fine-tuning & 68.33\xspace\xspace\xspace\xspace\xspace\xspace\xspace\xspace\xspace\xspace & 70.55\xspace\xspace\xspace\xspace\xspace\xspace\xspace\xspace\xspace\xspace & 71.18\xspace\xspace\xspace\xspace\xspace\xspace\xspace\xspace\xspace\xspace \\
 \cmidrule{2-5}
 % \cmidrule{2-12}
 & w/ fine-tuning & \textbf{69.72} {\scriptsize \color{ForestGreen}\textbf{$\uparrow$ 1.39}}  & \textbf{73.61} {\scriptsize \color{ForestGreen}\textbf{$\uparrow$ 3.06}}  & \textbf{74.17} {\scriptsize \color{ForestGreen}\textbf{$\uparrow$ 2.99}} \\
\bottomrule
\end{tabular}
}
\label{table:fidelity_comparison}
\end{table}


\section{Implementation Details}
\label{appendix:impl}
We implement \method using PyTorch \citep{paszke2019pytorch} and HuggingFace diffusers \citep{huggingface2023diffusers} libraries. For the generative model, we utilize the SDXL Base $1.0$ which is a standard and commonly used pre-trained diffusion model in natural images domain. In the pipeline, we employ CLIP ViT-G/14 as image encoder and both CLIP-L/14 \& CLIP-G/14 as text encoders \citep{radford2021learning}. We perform LoRA fine-tuning on the following modules of SDXL UNet denoiser including $Q$, $V$ transformation layers, fully-connected layers ($\tt{to\_q, to\_v, query, value, ff.net.0.proj}$) with rank parameter $r = 8$, which results in $11$M trainable parameters $\approx 0.46\%$ of total $2.6$B parameters. The fine-tuning is done on $8$ NVIDIA A100 80GB GPUs using AdamW \citep{loshchilov2017decoupled} optimizer, a learning rate of \texttt{5e-6}, and gradient accumulation steps of $8$.

\section{Additional Qualitative Results}
\label{appendix:visuals}
Figure~\ref{fig:diversity_compact_caption} showcases two examples of context description from MLLM in the inference pipeline where answers are not included in text guidance. Figure~\ref{fig:diversity_full} illustrates additional qualitative results highlighting the diversity and multimodal context fidelity between reference and synthetic images, as well as across images generated by \method with different random seeds. Figure~\ref{fig:qualitative_full} shows additional qualitative comparisons between \method and SOTA image generation methods on VQA and HOI Reasoning tasks.
\begin{figure}[!h]
    \centering
    \includegraphics[width=\linewidth]{figures/diversity_full.pdf}
    \vspace{-5mm}
    \caption{Diversity and multimodal context fidelity between reference and synthetic image and across generated ones from \method with different random seeds.}
    \label{fig:diversity_full}
\end{figure}
\begin{figure}[!h]
    \centering
    \includegraphics[width=\linewidth]{figures/qualitative_full_v1.pdf}
    \vspace{-5mm}
    \caption{Qualitative comparison between \method and other image generation methods on MME Perception and HOI Reasoning benchmarks.}
    \label{fig:qualitative_full}
\end{figure}
%%%%%%%%%%%%%%%%%%%%%%%%%%%%%%%%%%%%%%%%%%%%%%%%%%%%%%%%%%%%%%%%%%%%%%%%%%%%%%%
%%%%%%%%%%%%%%%%%%%%%%%%%%%%%%%%%%%%%%%%%%%%%%%%%%%%%%%%%%%%%%%%%%%%%%%%%%%%%%%


\end{document}


% This document was modified from the file originally made available by
% Pat Langley and Andrea Danyluk for ICML-2K. This version was created
% by Iain Murray in 2018, and modified by Alexandre Bouchard in
% 2019 and 2021 and by Csaba Szepesvari, Gang Niu and Sivan Sabato in 2022.
% Modified again in 2023 and 2024 by Sivan Sabato and Jonathan Scarlett.
% Previous contributors include Dan Roy, Lise Getoor and Tobias
% Scheffer, which was slightly modified from the 2010 version by
% Thorsten Joachims & Johannes Fuernkranz, slightly modified from the
% 2009 version by Kiri Wagstaff and Sam Roweis's 2008 version, which is
% slightly modified from Prasad Tadepalli's 2007 version which is a
% lightly changed version of the previous year's version by Andrew
% Moore, which was in turn edited from those of Kristian Kersting and
% Codrina Lauth. Alex Smola contributed to the algorithmic style files.
