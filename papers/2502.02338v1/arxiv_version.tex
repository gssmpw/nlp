%%%%%%%% ICML 2025 EXAMPLE LATEX SUBMISSION FILE %%%%%%%%%%%%%%%%%

\documentclass{article}

% Recommended, but optional, packages for figures and better typesetting:
\usepackage{microtype}
\usepackage{graphicx}
\usepackage{subfigure}
\usepackage{booktabs} % for professional tables

% hyperref makes hyperlinks in the resulting PDF.
% If your build breaks (sometimes temporarily if a hyperlink spans a page)
% please comment out the following usepackage line and replace
% \usepackage{icml2025} with \usepackage[nohyperref]{icml2025} above.
\usepackage{hyperref}


% Attempt to make hyperref and algorithmic work together better:
\newcommand{\theHalgorithm}{\arabic{algorithm}}

% Use the following line for the initial blind version submitted for review:
%\usepackage{icml2025}

% If accepted, instead use the following line for the camera-ready submission:
\usepackage[accepted]{icml2025}

% For theorems and such
\usepackage{amsmath}
\usepackage{amssymb}
\usepackage{mathtools}
\usepackage{amsthm}

% if you use cleveref..
\usepackage[capitalize,noabbrev]{cleveref}
\usepackage{microtype}      % microtypography
\usepackage{color}
\usepackage{algorithm} % For algorithm environment
\usepackage{algorithmic} % For algorithmic environment
\usepackage{minitoc}     % For separate table of contents in appendix


\usepackage{xcolor}      % For color customization
%\usepackage{tocloft}
\usepackage{minitoc}
\usepackage{appendix}

\usepackage{graphicx} 
\usepackage{amsmath} 
\usepackage{float}
\usepackage{wrapfig}
\usepackage{lipsum} % for dummy text
\usepackage{subcaption}
\usepackage{multirow}
\usepackage{derivative}
\usepackage{latexsym}
\usepackage{pifont}
\usepackage{booktabs}
\usepackage{subcaption}
%\usepackage{algpseudocode} % For pseudocode

%%%%%%%%%%%%%%%%%%%%%%%%%%%%%%%%
% THEOREMS
%%%%%%%%%%%%%%%%%%%%%%%%%%%%%%%%
\theoremstyle{plain}
\newtheorem{theorem}{Theorem}[section]
\newtheorem{proposition}[theorem]{Proposition}
\newtheorem{lemma}[theorem]{Lemma}
\newtheorem{corollary}[theorem]{Corollary}
\theoremstyle{definition}
\newtheorem{definition}[theorem]{Definition}
\newtheorem{assumption}[theorem]{Assumption}
\theoremstyle{remark}
\newtheorem{remark}[theorem]{Remark}

\newcommand{\method}{{\textit{GeomNP}}}  % Define your new command here
\usepackage{colortbl} % colored cells

\newcommand{\name}{{\emph{G}-NPF}}

\crefname{equation}{Eq.}{Eqs.} % For singular and plural forms

\definecolor{lightblue}{rgb}{0.81, 0.94, 1.0}
\definecolor{cvprblue}{rgb}{0.21,0.49,0.74}
\usepackage{hyperref}
\hypersetup{
    colorlinks=true,
    linkcolor=red,
    citecolor=cvprblue,      
    urlcolor=cyan
    }

%\usepackage{tocloft}
\usepackage{xcolor}

% \newcommand{\INPUT}{\textbf{Input: }}
% \newcommand{\OUTPUT}{\textbf{Output: }}

% Todonotes is useful during development; simply uncomment the next line
%    and comment out the line below the next line to turn off comments
%\usepackage[disable,textsize=tiny]{todonotes}
\usepackage[textsize=tiny]{todonotes}
\usepackage{booktabs}
\usepackage{pifont}


\newcommand{\zx}[1]{\textcolor{blue}{[\textbf{Zehao}: #1]}}
\newcommand{\js}[1]{\textcolor{orange}{[\textbf{Jiayi}: #1]}}
\newcommand{\wy}[1]{\textcolor{red}{[\textbf{WY}: #1]}}
\newcommand{\str}[1]{\textcolor{red}{[\textbf{STR}: #1]}}
\newcommand{\yc}[1]{\textcolor{magenta}{[\textbf{yunlu}: #1]}}
\newcommand{\addref}{\textcolor{red}{\textbf{[REF]}}}
\newcommand{\ymark}{\ding{51}}%
\newcommand{\cmark}{\ding{51}} % Checkmark
\newcommand{\xmark}{\ding{55}} % Crossmark

\newcommand{\RETURN}[1]{\textbf{return} #1}


% The \icmltitle you define below is probably too long as a header.
% Therefore, a short form for the running title is supplied here:
\icmltitlerunning{Geometric Neural Process Fields}

\begin{document}



\twocolumn[
\icmltitle{Geometric Neural Process Fields}

% It is OKAY to include author information, even for blind
% submissions: the style file will automatically remove it for you
% unless you've provided the [accepted] option to the icml2025
% package.

% List of affiliations: The first argument should be a (short)
% identifier you will use later to specify author affiliations
% Academic affiliations should list Department, University, City, Region, Country
% Industry affiliations should list Company, City, Region, Country

% You can specify symbols, otherwise they are numbered in order.
% Ideally, you should not use this facility. Affiliations will be numbered
% in order of appearance and this is the preferred way.
%\icmlsetsymbol{equal}{*}

\begin{icmlauthorlist}
\icmlauthor{Wenzhe Yin}{uva}
\icmlauthor{Zehao Xiao}{uva}
\icmlauthor{Jiayi Shen}{uva}
\icmlauthor{Yunlu Chen}{cmu}
\icmlauthor{Cees G. M. Snoek}{uva}
\icmlauthor{Jan-Jakob Sonke}{nki}
\icmlauthor{Efstratios Gavves}{uva}
%\icmlauthor{}{sch}
% \icmlauthor{Firstname8 Lastname8}{sch}
% \icmlauthor{Firstname8 Lastname8}{yyy,comp}
%\icmlauthor{}{sch}
%\icmlauthor{}{sch}
\end{icmlauthorlist}

\icmlaffiliation{uva}{University of Amsterdam}
\icmlaffiliation{nki}{The Netherlands Cancer Institute}
\icmlaffiliation{cmu}{Carnegie Mellon University}

\icmlcorrespondingauthor{Jiayi Shen}{j.shen@uva.nl}
% \icmlcorrespondingauthor{Firstname2 Lastname2}{first2.last2@www.uk}

% You may provide any keywords that you
% find helpful for describing your paper; these are used to populate
% the "keywords" metadata in the PDF but will not be shown in the document

%\icmlkeywords{Machine Learning, ICML}

\vskip 0.3in
]

% this must go after the closing bracket ] following \twocolumn[ ...

% This command actually creates the footnote in the first column
% listing the affiliations and the copyright notice.
% The command takes one argument, which is text to display at the start of the footnote.
% The \icmlEqualContribution command is standard text for equal contribution.
% Remove it (just {}) if you do not need this facility.

\printAffiliationsAndNotice{}  % leave blank if no need to mention equal contribution

%\printAffiliationsAndNotice{\icmlEqualContribution} % otherwise use the standard text.


\begin{abstract}
%Implicit Neural Representations (INRs) have been widely used to represent 3D scenes due to their continuous function properties. However, one of the major drawbacks is that a separate neural network must be trained from scratch for each signal. 
This paper addresses the challenge of Neural Field (NeF) generalization, where models must efficiently adapt to new signals given only a few observations. To tackle this, we propose Geometric Neural Process Fields (\name{}), a probabilistic framework for neural radiance fields that explicitly captures uncertainty. We formulate NeF generalization as a probabilistic problem, enabling direct inference of NeF function distributions from limited context observations.
To incorporate structural inductive biases, we introduce a set of geometric bases that encode spatial structure and facilitate the inference of NeF function distributions. Building on these bases, we design a hierarchical latent variable model, allowing \name{} to integrate structural information across multiple spatial levels and effectively parameterize INR functions. This hierarchical approach improves generalization to novel scenes and unseen signals.
Experiments on novel-view synthesis for 3D scenes, as well as 2D image and 1D signal regression, demonstrate the effectiveness of our method in capturing uncertainty and leveraging structural information for improved generalization.

% Specifically, in Neural Radiance Field (NeRF) generalization, there exists an information gap between the context set (continuous 2D RGB and rays) and the target set (3D discrete points), which is commonly ignored by the previous NeRF generalization methods. 
% To this end, we introduce a set of posterior Gaussian bases based on the context set to provide the structure geometry information of the scene for each spatial location. Furthermore, we propose a hierarchical neural process modeling to modulate the INR function globally and locally. The experimental results on both the ShapNet novel views synthesis task and the 2D image regression task show state-of-the-art performance. 
\end{abstract}


\section{Introduction}
\section{Introduction}\label{sec:introduction}
% -- Outline
% ---- LLMs are popular
% ---- There're many stakeholders in the training and inference loop
% ---- Adversaries in the training loop are a problem -- malpractice, poisoning
% ---- Also, showing compliance
% ---- Need a framework to prove the integrity of the pipeline
% ---- Enter Atlas

% ---- LLMs are popular
In recent years, machine learning (ML) models, have become increasingly popular.
The pervasive use of large language models (LLMs), in particular, and multi-stakeholder
involvement in model creation and deployment exacerbate security and privacy risks.
These considerations are emphasized by the global nature and the complexity of
large-scale ML deployments with different lifecycle stages:
%gathering and sanitizing the data from different sources,
%training and inferencing across many data centers,
%compliance with local laws or corporate policies.

% ---- There're many stakeholders in the training and inference loop
%Additionally, different stages of the ML development pipeline come with their own stakeholders:
\begin{enumerate}[label=\arabic*)]
    \item Collection and sanitation of a \emph{training} dataset from several public and proprietary sources.
    %\item Solicitation and facilitation of training.
    \item Provisioning of the training environment (hardware and software).
    \item Execution of training across many data centers.
    \item Construction of a \emph{testing} dataset from several sources, and the evaluation.
    \item Deployment and use of the model for inference that is compliant with local laws or corporate policies.
    %\item Use of the model in compliance with local laws or corporate policies.
\end{enumerate}

% ---- Adversaries in the training loop are a problem -- malpractice, poisoning
Each of these stages is vulnerable to malicious or dishonest parties.
For example, data can be poisoned~\cite{biggio2012poisoning,carlini2024poisoning} during collection or training.
Service providers executing outsourced training can shorten or omit critical steps to reduce their cost.
Model providers can serve smaller models in SaaS, or even distribute malicious ones.

% ---- Also, showing compliance
On the other hand, responsible model builders and other stakeholders may be incentivised or required to provide security and trust guarantees.
They may want to prove low bias in their training data, offer easily verifiable performance claims, or guarantee end-to-end integrity of the model creation in high risk domains.

% ---- Need a framework to prove the integrity of the pipeline
To address these challenges, it is necessary to guarantee the integrity of the entire ML lifecycle --
beginning with the data, through the training, and finally, the evaluation and deployment.
Was the data modified?
Did the hardware and software environment adhere to the specification?
Did the contractor follow the specified training procedure?
Can I trust the evaluation?
How can I guarantee that I am interacting with the intended model?
These are example questions that showcase the breadth of the involved challenges that must be tackled to provide end-to-end security.

% --- Enter Atlas
In this work, we introduce \atlas, a framework for enhancing the security and transparency of the lifecycle of ML models.
\atlas establishes the baseline of fundamental components and capabilities needed for comprehensive provenance tracking
at each stage of the ML lifecycle.
Subsequently, \atlas defines the core integrity requirements for verifiable ML lifecycle transparency.
We provide a reference implementation that instantiates \atlas using hardware-based security mechanisms -- with trusted execution environment (TEE),
including attestations.% , and comprehensive metadata-based provenance tracking.
%Our implementation satisfies all \atlas requirements.

We claim the following contributions:
\begin{enumerate}[label=\arabic*.]\label{sec:introduction:contributions}
    \item We introduce \atlas, a framework designed for end-to-end ML lifecycle transparency.
    \item We instantiate \atlas using TEEs and metadata-based provenance tracking.
    \item We evaluate our \atlas prototype through two case studies:
        \begin{enumerate*}[label=\arabic*)]
            \item fine-tuning of a BERT model~\cite{lin2023metabert, lin2023metabertimpl};
            \item fine-tuning of a bge-reranker model~\cite{chen2023bge}
        \end{enumerate*}.
\end{enumerate}

%\msm{revise: Integrate this motivation into intro}
%Organizations frequently leverage pre-trained models, outsource training processes, and integrate components from multiple sources,
%making it difficult to verify the authenticity and trustworthiness of their ML systems. This complexity is further compounded
%by the potential for malicious modifications at various stages of the model lifecycle, from data preparation through deployment.
%The involvement of various third parties in ML model development and deployment
%creates critical challenges in ensuring supply chain integrity.
%
%While Software Bills of Materials (SBOMs) and AI Bills of Materials (AI BOMs) provide basic inventory tracking for model components,
%they fall short in addressing the dynamic nature of ML pipelines. These approaches typically offer point-in-time snapshots but
%fail to capture the complex transformations, fine-tuning operations, and runtime modifications that characterize modern ML workflows.
%Additionally, they lack cryptographic guarantees about the integrity of recorded information and cannot effectively track the provenance
% of model weights and training data.
%
% These approaches demonstrate the growing importance of ML supply chain security.
% However, they are typically applied in an ad-hoc fashion, highlighting the need
% for a more integrated approach that combines comprehensive lineage tracking,
% strong cryptographic properties, and practical integration capabilities with existing ML development and deployment pipelines.
%
%A comprehensive solution requires not just documentation of components, but verifiable evidence of their origins,
%transformations, and integrity throughout the entire model lifecycle. This need has driven interest in more robust
%provenance tracking mechanisms that can:
%
%\begin{itemize}
%\item Provide cryptographic proof of model lineage
%\item Track and verify all pipeline transformations
%\item Maintain tamper-evident records of training processes
%\item Ensure integrity of model artifacts across organizational boundaries
%\end{itemize}
%
%Several existing tools and frameworks
%commonly focusing on different components of the model lifecycle and provenance tracking.
%While these solutions offer valuable capabilities, they often address only specific parts of the end-to-end ML
%supply chain rather than providing comprehensive coverage.
%\msm{end-revise}
%
%\todo{add discussion of EU-CRA AI Act requirements for model documentation and FDA guidelines for AI/ML in healthcare}

%The remainder of this paper is organized as follows:
%in Section~\ref{sec:background-related} we provide an overview of the necessary background, and the related work;
%Section~\ref{sec:problem} presents the challenge of providing integrity in the ML pipeline, the threat model, and the system assumptions;
%in Section~\ref{sec:framework} we present \atlas -- our framework for providing ML integrity;
%Section~\ref{sec:implementation} covers implementation details;
%in Section~\ref{sec:eval}, we show that \atlas is effective across three dimensions: training overhead $<8\%$, the verification time increases linearly with the size of the model, and it is compatible with PyTorch and Tensorflow;
%in Section~\ref{sec:casestudies} we present the case studies;
%in Section~\ref{sec:discussion} we discuss additional considerations for \atlas,
%and Section~\ref{sec:conclusion} concludes the paper and provides directions for future work.


\section{Background}
\section{Basic Background: Supervised Learning and the PAC Model}
\label{sec:background}

At this point almost everyone has heard of machine learning (ML). Anyone likely to stumble upon this article will have also heard of its most influential special case, supervised learning, and those theoretically inclined will also be familiar with the PAC model. Nonetheless, I will set the stage by  recapping the basics.

\subsection{Basics of Supervised Learning}%Let's set the stage in any case

\emph{Supervised Learning} is the task of ``coming up'' with a function $f: \X \to \Y$ to ``explain'' or ``fit'' a sequence of input/output examples   $(x_1,y_1), \ldots, (x_n,y_n)$, with $x_i \in \X$ and $y_i \in \Y$.  Here $\X$ is a \emph{data domain} consisting of \emph{datapoints} $x \in \X$, $\Y$ is a \emph{label set} consisting of \emph{labels} $y \in \Y$, and the sequence $(x_1,y_1),\ldots,(x_n,y_n)$ is the \emph{training data} consisting of \emph{labeled examples (a.k.a. samples)}~$(x_i,y_i)$.  I~will refer to the chosen function $f$ as a \emph{predictor}, and to $n$ as the \emph{sample size}. A \emph{learning algorithm} takes as input training data, and outputs (some representation of) a predictor $f \in \Y^\X$.\footnote{Note that this describes the usual \emph{batch}, a.k.a.~\emph{offline}, setting of supervised learning. I do not discuss other paradigms such as online or active learning in this article.} 



Success in supervised learning is defined as \emph{generalization} to  future examples: For a typical \emph{test example}  $(x_{\tst},y_{\tst})$, the predicted label $y'_{\tst}=f(x_{\tst})$ should ``equal'' $y_{\tst}$, perhaps approximately. We usually assume the test example is drawn from the same  ``source'' as the training data  --- commonly, i.i.d.~from the same distribution. The quality of the prediction is quantified by $\ell(y'_{\tst},y_{\tst})$, where $\ell:~\Y~\times~\Y \to \RR_{\geq 0}$ is a \emph{loss function} chosen as part of the problem definition. Common loss functions include the 0-1 loss $\ell_{0-1}(y',y) = [y' \neq y]$ for \emph{classification} problems,\footnote{The notation $[P]$ denotes $1$ when predicate $P$ is true, and denotes $0$ when $P$ is false.} as well as the absolute loss $|y'-y|$ or squared loss $(y'-y)^2$ for \emph{regression problems} featuring $\Y  \sse \RR$.

Nontrivial generalization properties are typically only possible if one assumes something about the data.\footnote{The need for such an assumption is formalized by the  \emph{no free lunch theorems} of supervised learning \cite{wolpert_connection_1992,wolpert_lack_1996,schaffer_conservation_1994}.} The Bayesian approach to  machine learning, common in many applications, assumes some parametric form for the distribution generating the data, and postulates a prior on the parameters. This is not the approach I will take in this article. Instead, I will focus on the frequentist --- and some would say ``worst-case'' or ``adversarial'' ---  approach that is common in the computational learning theory community, embodied by the PAC model. Here we assume that the (training and test) data can be explained, perhaps approximately, by a function in some ``simple enough to learn'' class of functions $\H \sse \Y^\X$, often called the \emph{hypotheses}. Equivalently, we  seek a predictor which explains the unseen data roughly  as well as the best hypothesis $h^* \in \H$, whether or not we assume that $h^*$ itself provides a perfect explanation.



 \paragraph{Common Algorithmic Templates.} Perhaps the best known general-purpose supervised learning algorithm is \emph{empirical risk minimization (ERM)}, which chooses as its predictor a hypothesis $f \in \H$ minimizing $\frac{1}{n} \sum_{i=1}^n \ell(f(x_i),y_i)$ --- a quantity called the \emph{training error}, \emph{empirical error}, or \emph{empirical risk} of $f$. %\footnote{When multiple hypotheses minimize the empirical risk, we assume ERM breaks ties arbitrarily.}
A common template for generalizing ERM involves adding a \emph{regularization term} $\psi(f)$ to the  objective function, typically chosen to measure some notion of ``hypothesis complexity.'' An algorithm instantiating this template is known as a \emph{structural risk minimizer (SRM)}, and chooses as its predictor the hypothesis $f \in \H$ minimizing the \emph{structural risk} $\frac{1}{n} \sum_{i=1}^n \ell(f(x_i),y_i) + \psi(f)$. Other well-known algorithms, such as gradient descent and its variations,  can frequently be interpreted as approximate implementations of ERM or SRM.


\paragraph{Proper vs Improper Learning.} A learning algorithm is said to be \emph{proper} if its predictor $f$ is always chosen from the hypothesis class, i.e., $f \in \H$, otherwise it is said to be \emph{improper}. ERM  is an example of a proper learning algorithm, as are SRM algorithms of the form described above.  In the \emph{proper regime} of learning, algorithms are required to be proper. This article will be concerned with the more flexible \emph{improper regime} (a.k.a \emph{representation-independent learning}), where no such constraint is placed on the learner. In other words, all we care about is predictive power at test time, rather than any insights derived from the functional form or representation of the predictor~itself.


\subsection{The PAC Model}
A standard mathematical setup for evaluation of supervised learning algorithms, at least in the theoretical computer science community, is Valiant's \emph{Probably Approximately Correct (PAC) model} of learning (see e.g.~\cite{kearns_introduction_1994,mohri_foundations_2018}). Here, we assume there is an unknown distribution $\D$ on $\X \times \Y$ from which training and test data are  drawn.  Specifically, the labeled datapoints of the training set  $(x_1,y_1), \ldots, (x_n,y_n)$, as well as the test data  $(x_\tst,y_\tst)$, are i.i.d.~from $\D$. Often it is assumed that $\D$ lies in some class of distributions of interest. The \emph{true expected loss}, or simply \emph{loss}, of a predictor $f: \X \to \Y$ is the expected loss it incurs on draws from $\D$, written $L_\D(f) = \Ex_{(x,y) \sim \D} \ell(f(x),y)$.


There are two main ``settings'' in PAC learning. The  \emph{realizable setting} only requires that the data be perfectly explained by some hypothesis in $\H$. More generally, the \emph{agnostic setting} makes no assumption relating the data to the hypotheses, but shifts the goalposts as necessary to allow nontrivial guarantees: the expected loss at test time is evaluated only ``relative'' to that of the best hypothesis $h^* \in \H$. There are other settings which make more nuanced assumptions, such as $\D$ being of a particular parametric form or its support living in some (unknown) lower-dimensional space, etc. I will mostly discuss the realizable and agnostic settings in this article, those being the simplest and most studied from a theoretical perspective. %TODO:We will briefly discuss other settings in Section ??

The PAC model demands high probability guarantees of learners, in the worst case over distributions of interest. Consider first the realizable setting, where $\D$ is such that $\min_{h \in \H} L_{\D}(h) = 0$. A PAC learner has \emph{error} $\epsilon=\epsilon(n)$ and \emph{confidence} $\delta=\delta(n)$ if, when training data consists of $n$ i.i.d~samples from a realizable distribution $\D$, it produces a predictor $f$  satisfying $L_\D(f) \leq \epsilon$ with probability at least $1-\delta$. In the agnostic setting, where $\D$ can be arbitrary, we require $L_\D(f) - \min_{h \in \H} L_\D(h) \leq \epsilon$ with probability $1-\delta$.

In both the realizable and agnostic settings, we look for PAC learners with small $\epsilon$ and $\delta$ as a function of the sample size $n$. An equivalent perspective looks at the sample complexity $m(\epsilon,\delta)$, which is the minimum sample size which guarantees error  at most $\epsilon$ with probability at least $1-\delta$. We say a problem is \emph{PAC learnable} if its PAC sample complexity is finite whenever $\epsilon,\delta > 0$.

For most PAC learning problems, learnability and sample complexity are characterized in terms of a  ``dimension'' of the hypothesis class. Most prominently this is the \emph{VC dimension} for binary classification, the \emph{fat shattering dimension} for agnostic regression, and the \emph{DS dimension} for multiclass classification (see \cite{anthony_neural_1999,daniely_optimal_2014,brukhim_characterization_2022}). Treatment of these is beyond the scope of this article. The unfamiliar reader need not worry, however,  as dimensions will feature only tangentially in our~discussion.




%\paragraph{Learning settings: Realizable, Agnostic, etc.} In learning theory, evaluating a supervised learning algorithm requires specifying a data model and an objective. We will leave the details of the data model flexible for now, to allow for both the PAC model and the adversarial transductive model. Nonetheless we will describe two variations, which we call ``settings'', which cut across different models. The  \emph{realizable setting}  requires only that the data be perfectly explained by some hypothesis $h \in \H$ --- i.e., there exists a hypothesis which is guaranteed to suffer a loss of $0$ on training and test data. The performance of the learning algorithm is its expected loss at test time for some ``worst case'' realizable instance. More generally, the \emph{agnostic setting} makes no assumption relating the data to the hypotheses, but shifts the goalposts as necessary to allow nontrivial guarantees: the expected loss at test time is evaluated only ``relative'' to that of the best hypothesis $h^* \in \H$, again for some ``worst case'' instance. There are other settings which make more nuanced assumptions about the data, such as it is drawn from a distribution of a particular parametric form, or that it lives in some (unknown) lower-dimensional space, etc. We will mostly discuss the realizable and agnostic settings, those being the simplest and most studied from a theoretical perspective.




%%% Local Variables:
%%% mode: latex
%%% TeX-master: "learning_matching"
%%% End:


\section{Geometric Neural Process Fields}
% \begin{figure}
%     \centering
%     \includegraphics[width=0.5\linewidth]{Move_teaser.pdf}
%     \caption{Comparison of different dynamic compute approaches. length of arrow indicates residual transformation per token while width indicates velocity of transformation.}
%     \label{fig:enter-label}
% \end{figure}

\section{Method}
\label{sec:method}
Residual connections play a crucial role in shaping token representations, yet their dynamics remain underexplored in the context of efficient decoding. In this work, we delve deeper into transformer residual dynamics and investigate how modulating residual transformation velocity can improve inference efficiency in token-level processing, optimizing both dense and sparse MoE transformers.


\subsection{Residual Dynamics and Motivation for Multi-rate Residuals} \label{sec:motivation}

To analyze how hidden representations evolve across different layers of a transformer architecture, it's crucial to consider the effect of residual connections. Each transformer decoder layer typically has residual connections across attention and MLP submodules. As the residual stream $h_i$ traverses from interval $E_j$ to $E_{j+1}$, it undergoes a residual transformation given by:  
% \begin{equation}
% \label{eq:slow_residual_transformation}
% H_{E_{j+1}} = H_{E_j} \prod_{i=E_j}^{E_{j+1}} \left( I + \mathcal{A}_i \right) \left( I + \mathcal{M}_i \right) \quad \text{where} \quad \mathcal{A}_i = f(c_i, h_{i}), \mathcal{M}_i = g(h_i)
% \end{equation}

\begin{equation} \label{eq:slow_residual_transformation}
h_{E_{j+1}} = h_{E_j} + \sum_{i=E_j}^{E_{j+1}-1} \left( \mathcal{A}_i(h_i) + \mathcal{M}_i(h_i + \mathcal{A}_i(h_i)) \right) \quad \text{where} \quad \mathcal{A}_i = f(c_i, h_{i}), \mathcal{M}_i = g(h_i). 
\end{equation}

Here, \( \mathcal{A}_i \) denotes the non-linear transformation introduced by the multi-head attention mechanism at layer \( i \), while \( \mathcal{M}_i \) corresponds to the non-linear transformation of the MLP block at the same layer. These transformations depend on the input residual stream \( h_i \) and, in the case of \( \mathcal{A}_i \), the previous contextual representation \( c_i \).\footnote{Normalization layers are typically applied in practice but are omitted here for simplicity of the argument.}


% For easy tokens, the magnitude and direction of this delta transformation become progressively smaller with each successive layer as shown in \cref{fig:delta_transformation}. Consequently, it is feasible to predict these tokens after only a few residual connections, whereas harder tokens necessitate more extensive processing through additional layers.

\begin{figure}[ht]
    \centering
    \begin{subfigure}{0.48\textwidth}
        \centering
        \includegraphics[width=\textwidth]{sections/figures/residual_change.pdf}
        \caption{}
        \label{fig:residual_change}
    \end{subfigure}%
    \hfill
    \begin{subfigure}{0.48\textwidth}
        \centering
        \includegraphics[width=\textwidth]{sections/figures/alignment_wrt_dedicated_model.pdf}
        \caption{}
    \label{fig:alignment_wrt_dedicated_model}
    \end{subfigure}
    \caption{(a) As residual streams propagate through the model, the directional shifts in the residuals become progressively smaller. (b) A dedicated model with $k$ layers achieves a faster rate of change in residual streams and higher alignment than base model leveraging early exit mechanisms at layer $k$.}
    \label{fig}
\end{figure}


To examine whether residual transformations can be accelerated across layers, we conducted experiments using a diverse set of prompts on a pre-trained Phi3 model~\cite{phi3_report}. As illustrated in \cref{fig:residual_change}, we measured the directional shift in residual states as \( 1 - \mathcal{C}(h_{i-1}, h_i) \), where \(\mathcal{C}\) denotes normalized cosine similarity. This shift is notably higher in the initial layers, gradually decreasing in subsequent layers. This behavior allows traditional early exit approaches to effectively accelerate decoding by enabling earlier exits for simpler tokens. However, these approaches typically rely on a distance-based approximation, where the full residual transformation of the model is approximated by the residual transformations of the initial layers. To gain deeper insights into the distance versus velocity aspects of residual transformation, we conducted a comparative study. Specifically, we trained an early exit head at layer $k$ of the Phi3 model, which consists of 32 layers, restricting the distance traveled by each token. To accelerate the residual transformation relative to number of layers, we trained a smaller model consisting of only $k$ layers, while keeping all other hyperparameters consistent. We then compared the next-token prediction accuracy of the early exit head of the base model with that of the smaller model. To ensure an equal number of trainable parameters, we inserted low-rank adapters into the smaller model and trained only these adapters, whereas, in the distance-based approach, we trained solely the early exit head. In addition, to accelerate the residual transformation in smaller model, we distilled the residual streams from the larger model by incorporating a distillation loss ~\cite{sanh2019distilbert} between the residual state at layer \(i\) of the smaller model and the residual state at layer \(4 \times i\) of the larger model. As shown in ~\cref{fig:alignment_wrt_dedicated_model} the smaller model demonstrates a significantly faster rate of change in residual streams, leading to higher next token prediction accuracy after $k$ layers compared to the base model that employs traditional early exit mechanisms after $k$ layers \cite{schuster2022confident, chen2023eellm, varshney-etal-2024-investigating}. This experimental setup, which modifies only the rate of change in residual streams while keeping other factors constant, suggests that dense transformers, trained with a fixed number of layers, may inherently possess a slow residual transformation bias.

This observation raises an intriguing question: if the rate of change in residual streams could be accelerated relative to the number of layers, is it possible to facilitate earlier alignment for a greater proportion of tokens? Earlier alignment would be beneficial to not only facilitate dynamic computation but also for generating speculative tokens efficiently with high acceptance rates in speculative decoding setups ~\cite{leviathan2023fast, chen2023accelerating}. 

%thereby enhancing the efficiency of early exiting? 
 % This bias likely constrains the effectiveness of early exiting, particularly for easier tokens. By addressing this limitation through accelerated residual transformations, we hypothesize that it is possible to substantially improve the efficiency and accuracy of early exit strategies in transformer models.

\subsection{Multi-Rate Residual Transformation} \label{m2r2_method}

To address the slow residual transformation bias described in ~\cref{sec:motivation}, we introduce \textit{accelerated residual streams} that operate at rate $R$ relative to original slow residual stream. We pair slow residual stream, $h$ with an accelerated residual stream, $p$, which has an intrinsic bias towards earlier alignment. Relative to ~\cref{eq:slow_residual_transformation}, accelerated residual transformation from interval $E_j$ to $E_{j+1}$ can be represented as: 

% \begin{equation}
% \label{eq:fast_residual_transformation}
% P_{E_{j+1}} = P_{E_j} \prod_{i=E_j}^{E_{j+1}} \left( I + \hat{\mathcal{A}_i} \right) \left( I + \hat{\mathcal{M}_i} \right) \quad \text{where} \quad \hat{\mathcal{A}_i} = \hat{f}(c_i, P_{i}), \hat{\mathcal{M}_i} = \hat{g}(P_{i})
% \end{equation}


\begin{equation} \label{eq:fast_residual_transformation}
p_{E_{j+1}} = p_{E_j} + \sum_{i=E_j}^{E_{j+1}-1} \left( \hat{\mathcal{A}_i}(p_i) + \hat{\mathcal{M}_i}(p_i + \hat{\mathcal{A}_i}(p_i)) \right) \quad \text{where} \quad \hat{\mathcal{A}_i} = \hat{f}(c_i, p_{i}), \hat{\mathcal{M}_i} = \hat{g}(h_i), 
\end{equation}



where $\hat{\mathcal{A}_i}$ and $\hat{\mathcal{M}_i}$ denote non-linear transformation added by layer $i$ to previous accelerated residual $p_{i}$. Similar to $\mathcal{A}_i$, non-linear transformation $\hat{\mathcal{A}_i}$ attends to same context $c_i$ but uses a different transformation $\hat{f}$ for accelerating $p_{E_j}$ relative to $h_{E_j}$. 

We integrate accelerated residual transformation directly into the base network using parallel accelerator adapters such that rank of accelerator adapters $R_p << d$ where $d$ denotes base model hidden dimension. This setup allows the slow residual stream $h_{E_j}$ to pass through the base model layers while the accelerated residual stream $p_{E_j}$ utilizes these parallel adapters as shown in ~\cref{fig:m2r2_main}. Both slow and accelerated residuals are processed in same forward pass via attention masking and incur negligible additional inference latency in memory bound decoding setups, while in compute bound decoding setups where FLOPs optimization is essential, accelerated residual stream utilizes a fraction of attention heads that of slow residual (see ~\cref{sec:flops_optimization}). Additionally, to maximize the utility of accelerated residual transformations without introducing dedicated KV caches, we propose a shared caching mechanism between the slow and accelerated streams which minimally impact alignment benefits of our approach while offering substantial memory savings (see ~\cref{fig:koala_alignment}). Specifically, the attention operation on the slow residuals \( \text{MHA}(h_t, h_{\leq t}, h_{\leq t}) \) is redefined for accelerated residuals as 
\[
\hat{\mathcal{A}} = MHA(p_t, h_{<t} \oplus p_t, h_{<t} \oplus p_t),
\]
where the accelerated residual at time-step $t$, \( p_t \) attends to the slow residual’s KV cache, facilitating the reuse of contextual information across both residual streams without incurring additional caching costs. Here, \(MHA(q, k, v) \) represents multi-head attention between query \( q \), key \( k \), and value \( v \).

\begin{figure}
    \centering
    \includegraphics[width=0.8\linewidth]{sections//figures/m2r2_main2.pdf}
    \caption{Multi-rate Residuals Framework: Slow residual stream of base model is accompanied by a faster stream that operates at a $2-(J+1)\times$ rate relative to the slow stream, undergoing transformations via accelerator adapters as detailed in \cref{m2r2_method}, where J denotes number of early exit intervals. Colors within the slow and fast residual streams indicate similarity, with matching colors representing the most closely aligned residual states. At the beginning of the forward pass and at each exit point, the accelerated residual state is initialized from the corresponding slow residual state to avoid gradient conflict during training (see ~\cref{sec:grad_conflict}). Early exiting decisions are informed by the Accelerated Residual Latent Attention (ARLA) mechanism, described in \cref{method_arla}, which evaluates residual dynamics across consecutive exit gates.}
    \label{fig:m2r2_main}
\end{figure}

% Furthermore. to maximize the benefits of fast residual transformations without using dedicated KV caches, we propose sharing the fast network’s cache with the slow network. Formally speaking, We modify attention operation on slow residuals $MHA(H_t, H_{<=t}, H_{<=t})$ as $MHA(P_{t}, H_{<t} \oplus P_t, H_{<t}  \oplus P_t)$ such that accelerated residuals attend to previous slow context KV cache, where $MHA(q,k,v)$ denotes multi head attention between query, $q$, key $k$ and value $v$.


\subsection{Enhanced Early Residual Alignment}
Early residual alignment is instrumental in optimizing early exiting, speculative decoding, and Mixture-of-Experts (MoE) inference mechanisms. In this section, we provide a detailed analysis of how accelerated residuals enhance these inference setups.

% By aligning the residual states of intermediate layers with the final output representations, the model can maintain high prediction accuracy even when computations are truncated at earlier layers. This enables more reliable early exiting, reducing the overall computational cost while preserving performance. Additionally, in speculative decoding, early residual alignment allows the model to make confident predictions using faster, partial computations, thereby accelerating inference without sacrificing output quality.


\subsubsection{Early Exiting} \label{method_early_exiting}

A prevalent strategy for enabling early exiting at an intermediate layer $E_{j}$ involves approximating the residual transformation between $E_{j}$ and the final layer $N-1$ using a linear, context independent mapping, $\mathcal{T}$, such that $H_{N-1} \approx \mathcal{T}(H_{E_{j}})$. This approximation has been extensively employed in conventional approaches ~\cite{schuster2022confident, chen2023eellm, varshney-etal-2024-investigating}, providing a computationally efficient means to project the output of deeper layers from intermediate states. Specifically, residual state of layer $N-1$ with this approximation can be expressed as:


% \begin{equation}
% \label{eq: vanila_ea_assumption}
% \Phi(H_{E_{j}}) \sim H_{E_{j}} \prod_{i=E_{j}}^{N}\left( I + \mathcal{A}_i \right) \left( I + \mathcal{M}_i \right) \quad \text{where} \quad \Phi \perp C
% \end{equation}

\begin{equation} \label{eq:early_exiting}
h_{E_j} + \sum_{i=E_j}^{N-1} \left( \mathcal{A}_i(h_i) + \mathcal{M}_i(h_i + \mathcal{A}_i(h_i)) \right) \sim \mathcal{T}(h_{E_{j}})  \quad \text{where} \quad \mathcal{T} \perp c. 
\end{equation}


Here, $\mathcal{A}_i$ and $\mathcal{M}_i$ represent the residual contributions of the multi-head attention and MLP layers, respectively, while $\mathcal{T}$ remains independent of $c$, the preceding context.

This approach is inherently limited by two major factors: first, the assumption of linearity between $h_{E_{j}}$ and $h_{N-1}$ may not hold uniformly for all tokens, particularly when $E_j \ll N$. Second, the linear transformation $\mathcal{T}$ disregards the influence of the context $c$ and fails to account for the latent representations of previous contextual states. In contrast, M2R2 accelerated residual states mitigate both of these challenges by approximating the slow residual transformation of all layers via a faster residual transformation of fewer layers as:
% \begin{equation}
% H_{E_j} \prod_{i=E_j}^{N}\left( I + \mathcal{A}_i \right) \left( I + \mathcal{M}_i \right) \sim P_{E_j} \prod_{i=E_j}^{E_j+1}\left( I + \hat{\mathcal{A}_i} \right) \left( I + \hat{\mathcal{M}_i} \right)
% \end{equation}


\begin{equation} \label{eq:m2r2_approximating_ea}
h_{E_j} + \sum_{i=E_j}^{N-1} \left( \mathcal{A}_i(h_i) + \mathcal{M}_i(h_i + \mathcal{A}_i(h_i)) \right) \sim p_{E_j} + \sum_{i=E_j}^{E_{j+1}-1} \left( \hat{\mathcal{A}_i}(p_i) + \hat{\mathcal{M}_i}(p_i + \hat{\mathcal{A}_i}(p_i)) \right), 
\end{equation}

% \begin{equation} \label{eq:fast_residual_transformation}
% p_{E_{j+1}} = p_{E_j} + \sum_{i=E_j}^{E_{j+1}-1} \left( \hat{\mathcal{A}_i}(p_i) + \hat{\mathcal{M}_i}(p_i + \hat{\mathcal{A}_i}(p_i)) \right) \quad \text{where} \quad \hat{\mathcal{A}_i} = \hat{f}(c_i, p_{i}), \hat{\mathcal{M}_i} = \hat{g}(h_i) 
% \end{equation}






where $p_{E_j}$ is initialized from the slow residual state $h_{E_j}$ at each early exit interval $E_j$ using an identity transformation (see ~\cref{fig:m2r2_main}). As shown in ~\cref{fig:m2r2_residual_sim}, accelerated residuals offer a smoother, more consistent shift in residual direction across layers, in contrast to the abrupt changes typically seen at early exit points in standard early exit methods. Moreover, the normalized cosine similarity between accelerated states at early exit intervals and final residual states is substantially higher compared to traditional early exit techniques, highlighting improved alignment with final layer representations. Traditional adaptive compute methods are constrained by two principal factors: the number of tokens eligible for early exit at intermediate layers and the precision of early exit decision. If residual streams fail to saturate early, the majority of tokens remain ineligible for exit, thereby diminishing potential speedups. Additionally, imprecise delineations between tokens suitable for early exit can lead to underthinking (premature exits that adversely affect accuracy) or overthinking (unnecessary processing that compromises efficiency) ~\cite{zhou2020self, dai2020dynamic}. Enhanced early alignment using ~\cref{eq:m2r2_approximating_ea} helps to address  first issue. To address the second issue we introduce Accelerated Residual Latent Attention, which dynamically assesses the saturation of the residual stream, allowing for a more precise differentiation between tokens that can exit early and those requiring further processing.

% This results in uniform change in residual direction    
% % We keep $\mathcal{A} = \hat{\mathcal{A}}$, while $\hat{\mathcal{M}}$ is accelerated by a factor of $2 - (N_{E}+1)X$ relative to the slower residual transformation $\mathcal{M}$, where $N_E$ represents number of early exiting intervals.
% Figure~\cref{fig:rate_change_comparison} illustrates the comparative rate of change between these transformation streams.



% fig:rate_change_comparison
% - grid plot x axis -> layer id (0, 8) , y axis -> layer id -> dark color cell for max similarity , lighter for lower 
% 
-------------------------------------------------------
Let's consider residual stream $h_i$ traverses through interval $E_j$ to $E_{j+1}$ and undergoes residual transformation given by 
\begin{equation}
h_{E_{j+1}} = h_{E_j} \prod_{i=E_j}^{E_{j+1}} \left( 1 + \delta_i \right)    
\end{equation}

where $\delta_i$ denotes non-linear transformation added by layer $i$. Each non-linear transformation of layer $i$ is a function of previous contextual representation, $c_i$ and input residual stream $h_i-1$ as
$\delta_i = f(c_i, h_{i-1})$ 

One way to exit early at exit $E_j+1$ is to assume that residual transformation from $E_j+1$ to final layer $N-1$ can be approximated by a linear function $\phi$ as $h_{N-1} \sim \Phi(h_{E_j+1})$ and most conventional approaches such as \todo{cite EA papers} use this approach. In other words, 

\begin{equation}
\Phi(h_{E_j+1} \sim h_{E_j+1} \prod_{i=E_j+1}^{N} \left( 1 + \delta_i \right)   
\end{equation}

This approach suffers from two primary issues, linearity assumption from $h_E_j+1$ to $H_N-1$ if often incorrect, particularly when $E_j << N$. More importantly, linear transformation $\Phi$ doesn't consider effect of context $C_i$. M2R2  effectively addresses these issues as accelerated residual stream at interval $E_j+1$ can be represented as 

\begin{equation}
r_{E_{j+1}} = r_{E_j} \prod_{i=E_j}^{E_{j+1}} \left( 1 + \gamma_i \right)    
\end{equation}

where $\gamma_i$ denotes non-linear transformation added by layer $i$ to previous accelerated residual $r_i-1$. Similar to $\delta_i$, non-linear transformation $\gamma_i$ considers context $C_i$ as 
$\gamma_i = g(c_i, r_{i-1})$. So in summary, slow residual transformation is approximated by accelerated residual as: 

\begin{equation}
h_{E_j} \prod_{i=E_j}^{N} \left( 1 + \delta_i \right) \sim h_{E_j} \prod_{i=E_j}^{E_j+1} \left( 1 + \gamma_i \right)
\end{equation}

It's worth noting that accelerated residual $r_i$ and slow residual $h_i$ are processed concurrently at layer $i$ by constructing proper attention mask such as attention of slow residual is represented as 

$MHA(H_it, H_{i<=t}, H_{i<=t}$ while attention of fast residual is computed as 

$MHA(r_it, H_{i<=t}, H_{i<=t}$ where $MHA(q,k,v$ denotes multi head attention between query, $q$, key $k$ and value $v$.


------------------------------------------------------------------

Vertical latent attention on accelerated residual is computed as 
$MHA(S_mt, S(Ej<=i<=m)t, S(Ej<=i<=m)t)$ where $Smt$ denotes query/key/value projection in latent domain at layer $m$ at time $t$. 
------------------------------------------------------------------

Gradient conflict Avoidance: 

Let's consider $w_j$ is a trainable parameter that belongs to a layer between $E_j$ and $E_j+1$. Consider early exit loss at gate $E_j+1$, $L_j+1$, gradient propagation of $w_j$ at another trainable parameter $w_j-n$ can be gives as 

$\sum_{k=E_j-n}^{E_j} \beta_k \frac{\partial L_{E_k}}{\partial w_k}$

where $\beta_j$ denotes backward transformation coefficient for weight $w_j$ to reach gate $E_j$. 
 
On the other hand, gradient propagation in proposed approach can be represented as 

\[
\frac{\partial L_{E_j}}{\partial w_j} = 
\begin{cases} 
\beta_j \frac{\partial L_{E_j}}{\partial w_j} & \text{if } E_j \leq w_j \leq E_{j+1} \\
0 & \text{otherwise}
\end{cases}
\]







% \begin{figure}[ht]
%     \centering
%     \includegraphics[width=0.8\textwidth, height=5cm]{rate_change_comparison.png}
%     \caption{Rate of change comparison between fast and slow residual streams.}
%     \label{fig:rate_change_comparison}
% \end{figure}

%vary k and and plot EA accuracy for larger and smaller models. 

% \begin{figure}[ht]
%     \centering
%     \includegraphics[width=0.5\textwidth,height=5cm]{sections/figures/alignment_comparison_dialogsum.pdf}
%     \caption{Alignment of exited tokens for different early exit layers using traditional early exiting heads, dedicated faster networks, and faster residuals.}
%     \label{fig:small_model_early_exiting}
% \end{figure}


\textbf{Accelerated Residual Latent Attention} \label{method_arla}

In the context of residual streams, we observe that the decision to exit at a given layer can be more effectively informed by analyzing the dynamics of residual stream transformations, instead of solely relying on a classification head applied at the early exit interval $E_j$. To capture the subtle dynamics of residual acceleration, we propose a \textit{Accelerated Residual Latent Attention} (ARLA) mechanism. This approach involves making the exit decision at gate $E_j$ by attending to the residuals spanning from gate $E_{j-1}$ to $E_j$, rather than considering only the residual at gate $E_j$. To minimize the computational overhead associated with exit decision-making, the attention mechanism operates within the latent domain as depicted in ~\cref{fig:arla_arch}. Formally, for each interval $[E_j, E_{j+1}]$, the accelerated residuals are projected into Query ($Q^s_{E_j}, \ldots, Q^s_{E_{j+1}}$), Key ($K^s_{E_j}, \ldots, K^s_{E_{j+1}}$), and Value ($V^s_{E_j}, \ldots, V^s_{E_{j+1}}$) vectors, with latent dimension $d^s$ for $Q^s$, $K^s$, and $V^s$ being significantly smaller than hidden dimension of $p$.\footnote{We use $d^s = 64$ for experiments described in ~\cref{sec:experiments}.} Notably, when the router is allowed to make exit decisions at gate $E_j$ based on residual change dynamics, we observe that the attention is not confined to the residual state at $E_j$ but is distributed across residual states from $E_{j-1}$ to $E_j$, %as illustrated in Figure~\ref{fig:vertical_latent_attention_dynamics}. 
This broader focus on residual dynamics significantly reduces decision ambiguity in early exits, as demonstrated in Figure~\ref{fig:roc_arla}, which contrasts routers based on the last hidden state, and the proposed ARLA router.

%show R -> S transformation. 
%show parameter and flop overhead as compared to adapter on last hidden state.

% \begin{figure}[ht]
%     \centering
%     \includegraphics[width=0.5\textwidth,height=5cm]{sections/figures/roc_arla.pdf}
%     \caption{ROC curves of early exit decision strategies: confidence-based methods (CALM/LITE), routers based on the accelerated hidden state, and latent attention routers.}
%     \label{fig:decision_making_comparison}
% \end{figure}

% \begin{figure}[ht]
%     \centering
%     \includegraphics[width=0.5\textwidth,height=5cm]{vertical_latent_attention.png}
%     \caption{Vertical latent attention mechanism for optimizing early exit decisions by considering residuals from gate \(M\) through \(M-1\).}
%     \label{fig:vertical_latent_attention}
% \end{figure}

\begin{figure}[ht]
    \centering
    \begin{subfigure}{0.52\textwidth}
        \centering
        \includegraphics[width=\textwidth, height = 4cm]{sections/figures/arla_arch.pdf}
        \caption{Accelerated Residual Latent Attention (ARLA): Accelerated residuals between early exit gates are projected into latent domain and attention over residual states within the interval is computed to capture residual dynamics and exit decision is made based on residual saturation.}
        \label{fig:arla_arch}
    \end{subfigure}%
    \hfill
    \begin{subfigure}{0.45\textwidth}
        \centering
        \includegraphics[width=\textwidth, height = 4.5cm]{sections/figures/vla_roc.pdf}
        \caption{ROC classification curves of early exit decision strategies using a linear router used on last residual state ~\cite{schuster2022confident, varshney-etal-2024-investigating, chen2023eellm}  and using ARLA approach that considers residual dynamics. }
        \label{fig:roc_arla}
    \end{subfigure}
    \caption{Effectiveness of ARLA in capturing residual dynamics for early exiting decisions.}


\end{figure}



% \begin{figure}[ht]
%     \centering
%     \includegraphics[width=1\textwidth,height=5cm]{sections/figures/arla.pdf}
%     \caption{fig that plots 32 rows 2 cols heatmap showing attention at each gate}
%     \label{fig:vertical_latent_attention_dynamics}
% \end{figure}

\subsubsection{Self Speculative Decoding} \label{method_self_speculative_decoding}

An alternative means to exploit the early alignment properties of our approach is through the use of accelerated residual states for speculative token sampling to accelerate autoregressive decoding. Speculative decoding aims to speed up memory-bound transformer inference by employing a lightweight draft model to predict candidate tokens, while verifying speculated tokens in parallel and advancing token generation by more than one token per full model invocation \cite{leviathan2023fast, chen2023accelerating, xia2023speculative, miao2023specinfer}. Despite its effectiveness in accelerating large language models (LLMs), speculative decoding introduces substantial complexity in both deployment and training. A separate draft model must be specifically trained and aligned with the target model for each application, which increases the training load and operational complexity ~\cite{chen2023accelerating}. Additionally, this approach is resource-inefficient, as it requires both the draft and target models to be simultaneously maintained in memory during inference \cite{leviathan2023fast, chen2023accelerating}. 

One strategy to address this inefficiency is to leverage the initial layers of the target model itself to generate speculative candidates, as depicted in ~\cite{Tang2024}. While this method reduces the autoregressive overhead associated with speculation, it suffers from suboptimal acceptance rates. This occurs because the linear transformation employed for translating hidden states from layer $k$ to the final layer $N$ is typically a poor approximation, as discussed in ~\cref{sec:motivation} and ~\cref{method_early_exiting}. Our approach resolves this limitation by utilizing accelerated residuals, which demonstrate higher fidelity to their slower counterparts. By utilizing accelerated residuals operating at a rate of $N/k$, where $k$ denotes the number of layers used for candidate speculation, we are able to efficiently generate speculative tokens for decoding.\footnote{We typically set $k = 4$ to balance the trade-off between autoregressive drafting overhead and acceptance rate, as discussed in~\cref{sec:experiments}.}
 This technique not only obviates the need for multiple models during inference but also improves the overall efficiency and effectiveness of speculative decoding.

\begin{figure}
    \centering    \includegraphics[width=1\linewidth]{sections/figures/m2r2_aot_loading.pdf}
    \caption{Ahead-of-Time Expert Loading: M2R2 accelerated residual stream predicts experts required for future layers, reducing reliance on on-demand lazy loading. Speculative pre-loading is efficiently overlapped with computation of multi-head attention (MHA) and MLP transformations. Only incorrectly speculated experts are loaded lazily, resulting in faster inference steps and improved computational efficiency. Here, H indicates LBM Host while D indicates HBM Device.}
    \label{fig:moe_expert_aot_loading}
\end{figure}


\subsubsection{Ahead of Time Expert Loading:} \label{method_aot_expert_loading}

Recent advancements in sparse Mixture-of-Experts (MoE) architectures ~\cite{shazeer2017outrageously, fedus2022switch, artetxe2019massively, lepikhin2020gshard, zoph2022designing} have introduced a paradigm shift in token generation by dynamically activating only a subset of experts per input, achieving superior efficiency in comparison to dense models, particularly under memory-bound constraints of autoregressive decoding \cite{fedus2022switch, zoph2022designing}. This sparse activation approach enables MoE-based language models to generate tokens more swiftly, leveraging the efficiency of selective expert usage and avoiding the overhead of full dense layer invocation. In dense transformer models, pre-loading layers is a common strategy to enhance throughput, as computations of current layer can be overlapped with pre-loading of next layer parameters ~\cite{narayanan2021efficient, shoeybi2020megatron}. However, MoE models face a unique challenge: expert selection occurs dynamically based on previous layer’s output, making it infeasible to preload next layer’s experts in parallel. This limitation results in inherent latency, as expert loading becomes a sequential, on-demand process ~\cite{lepikhin2020gshard, fedus2022switch}.

To address this inefficiency, our method introduces a mechanism with \textit{accelerated residuals}, which not only captures key characteristics of base slower residual states but also exhibit high cosine similarity with their final counterparts (as illustrated in \cref{fig:m2r2_residual_sim}). By employing accelerated residual streams, we can effectively predict the necessary experts for future layers well in advance of their actual invocation. Specifically, using a $2\times$ accelerated residual, the experts needed for layers $2i+2$ and $2i+3$ can be identified while still computing in layer $i$, thus overcoming the bottleneck of sequential, on-demand expert selection and mitigating latency in the decoding pipeline, as shown in \cref{fig:moe_expert_aot_loading}. Note that, we use fixed set of accelerator adapters for transforming accelerated residuals (as discussed in ~\cref{m2r2_method}) while slow residual is transformed via expert routing mechanism. 

Furthermore, our approach integrates a Least Recently Used (LRU) caching strategy, which enhances memory efficiency by replacing the least recently used experts with speculated experts that are anticipated to be needed in upcoming layers. This hybrid approach of preemptive expert loading with LRU caching yields substantial improvements over traditional on-demand loading or standalone caching strategies. By minimizing cache misses and efficiently managing memory, this approach addresses both compute and memory bottlenecks, leading to faster, more resource-efficient token generation in MoE architectures. A comprehensive evaluation of this strategy, in relation to state-of-the-art methods, is provided in \cref{experiments_aot}, and the compute and memory traces on an A100 GPU are detailed in \cref{fig:moe_aot_cuda_trace}.



% Recent advancements in sparse Mixture-of-Experts (MoE) architectures have introduced the concept of utilizing distinct computational paths for different tokens \cite{shazeer2017outrageously}. This approach, wherein only a subset of experts are activated per input, enables MoE-based language models to generate tokens more swiftly compared to their dense counterparts due to memory-bound nature of auto-regressive decoding. In dense models, pre-loading layers in advance is a common strategy to enhance computational efficiency. However, this technique is not applicable to MoE models, where expert selection occurs dynamically based on the outputs of previous layers, preventing parallel pre-fetching of experts.

% Our proposed method addresses this inefficiency. Accelerated residuals, which are highly similar to their slower counterparts (see \cref{fig:similarity}), can reliably predict the necessary experts ahead of time. For instance, by utilizing $2X$ accelerated residual stream, we can predict the experts needed for the layer $2i+1$ and $2i+3$ while carrying out computation in layer $i$. This enables us to commence expert loading significantly earlier, as illustrated in \cref{expert_loading}, effectively mitigating the delays observed with the naive on-demand expert loading. Additionally, our method benefits from incorporating a Least Recently Used (LRU) strategy, where speculated experts replace those that are least recently utilized, resulting in improved performance compared to using either strategy alone. For a comprehensive evaluation, refer to \cref{moe_trace}, which provides a CUDA compute and memory trace of our approach executed on <>.



% A naive solution involves using the residual state of the previous layer along with the gating function of the next layer to predict which experts need to be loaded, and initiating the expert loading process in parallel with the attention computation of the next layer. Yet, as shown in \cref{fig:MOE_attn_vs_loading_time}, the attention computation for medium to long contexts is considerably faster than the expert loading time, making this approach inefficient.




\subsection{Training} \label{method_training}
% This approach is feasible due to the absence of gradient conflicts, as discussed in \cref{sec:grad_conflict}.

To accelerate residual streams, we employ parallel accelerator adapters as described in \cref{m2r2_method}.  For the early exiting use-case outlined in \cref{method_early_exiting}, we define the training objective for these adapters using the following loss function, which combines cross-entropy loss at each exit $E_j$ with distillation loss at each layer $i$. Loss weights coefficients $\alpha_0$ and $\alpha_1$ are employed to balance contribution of corresponding losses.

\begin{align} \label{eq:mr_loss}
L_{\text{m2r2}} = \underbrace{-\alpha_0 \sum_{j=1}^{J} \sum_{t=1}^{T} \log p_{\theta} \left( \hat{y}_t^{E_j} \mid y_{<t}, x \right)}_{\text{cross-entropy loss}} 
+ \underbrace{\alpha_1\sum_{i=1}^{E_{J-1}} \sum_{t=1}^{T} \| \mathbf{p}_{t}^{i} - \mathbf{h}_{t}^{((i - E_{j(i)}) \cdot R_i) + E_{j(i)})} \|^2}_{\text{distillation loss}}.
\end{align}

where $\hat{y}_t^{E_j}$ denotes the predictions from the accelerated residual stream at layer $E_j$ and time step $t$, $y_t$ represents the corresponding ground truth tokens, and $x$ indicates previous context tokens. The distillation loss at each layer $i$ is computed by comparing accelerated residuals at layer $i$ with slow residuals at layer $(i - E_{j(i)}) \cdot R_i + E_{j(i)}$, where $R_i$ denotes the rate of accelerated residuals at layer $i$ while $E_{j(i)}$ represents the most recent gate layer index such that $E_{j(i)} <= i$. \( J \) represents the total number of early exit gates, N denotes number of hidden layers and $E_j$ denotes layer index corresponding to gate index $j$ and \( T \) denotes the sequence length. 

In dynamic compute settings, after training of accelerator adapters, we optimize the query, key, and value parameters governing the ARLA routers (see ~\cref{method_arla}) across all exits in parallel on binary cross entropy loss between predicted decision and ground truth exiting decision. The ground truth labels for the router are determined based on whether the application of the final logit head on $\hat{y}_t^{E_j}$ yields the correct next-token prediction. 


% The objective for this optimization is defined by the following loss function:


%TODO are equations required ? 
% \begin{equation} \label{eq:arla_loss_combined}\small
%     L_{\text{arla}} = -\frac{1}{N} \sum_{t=1}^{T} \left( \sum_{j=1}^{E_n} \left[ O_t^{E_j} \log(\hat{O}_t^{E_j}) + (1 - O_t^{E_j}) \log(1 - \hat{O}_t^{E_j}) \right] \right), \quad \text{where} \quad 
%     O_t^{E_j} = \begin{cases} 
%     1, & \text{if } L(\hat{y}_t^{E_j}) = y_t^{E_j} \\
%     0, & \text{otherwise}
%     \end{cases}
% \end{equation}

% where $\hat{O}_t^{E_j}$ represents the binary predicted logits produced by the vertical latent attention router, as described in \cref{sec:arla}, at gate $E_j$ and time step $t$, and $O_t^{E_j}$ denotes the corresponding ground truth labels. The ground truth labels for the router are determined based on whether the application of the logit head on $\hat{y}_t^{E_j}$ yields the correct next-token prediction. The parameters controlling vertical latent attention are trained concurrently to ensure consistency and efficient use of computational resources.

For self-speculative decoding, as described in \cref{method_self_speculative_decoding}, the training objective remains the same as \cref{eq:mr_loss}, but with the number of intervals set to $J = 1$ and the rate of residual transformation set to $R_n = N/k$, where the first $k$ layers generate speculative candidate tokens. In the context of Ahead-of-Time Expert Loading for Mixture-of-Experts (MoE) models (see \cref{method_aot_expert_loading}), setting the rate of residual transformation to $R_n = 2$ typically offers a good trade-off between the accuracy of expert speculation and AoT pre-loading of experts. 

% Thus, we set $J = 1$ and $E_1 = 16$.


~\subsection{FLOPs Optimization} \label{sec:flops_optimization}

Naively implemented, M2R2 incurs higher FLOP overhead compared to traditional speculative decoding and early exiting approaches such as ~\cite{medusa, schuster2022confident, Tang2024}. However, modern accelerators demonstrate compute bandwidth that exceeds memory access bandwidth by an order of magnitude or more~\cite{databricksLLMInference2023, jouppi2021ten}, meaning increased FLOPs do not necessarily translate to increased decoding latency. Nevertheless, to ensure fair comparison and efficiency in compute bound scenarios, we introduce targeted optimizations.

~\textbf{Attention FLOPs Optimization} For medium-to-long context lengths, attention computation dominates FLOPs in the self-attention layer, surpassing the contribution from MLP layers. Specifically, matrix multiplications involving queries, cached keys, and cached values scale with $l_{kv} * l_{q}$ where $l_{kv}$ denotes previous context length and $l_q$ denotes current query length. Since M2R2 pairs accelerated residuals with slow residuals, a naive implementation results in twice the FLOPs consumption compared to a standard attention layer. To address this, we limit the attention of accelerated residual stream to selectively attend to the top-k most relevant tokens, identified by the slow residual stream based on top attention coefficients\footnote{We set to k = 64 and attend to top 64 tokens as identified by the slow residual stream.}. This is possible since slow and accelerated residual streams are processed in same forward pass and accelerated streams have access to attention coefficients of slow stream. Note that, the faster residual stream still retains the flexibility to assign distinct attention coefficients to these tokens. Furthermore, we design the faster residual stream to employ only 8 attention heads, compared to the 32 heads used in the slow residual stream of the Phi-3 model, reducing query, key, value, and output projection FLOPs by a factor of 1/4. ~\cref{fig:m2r2_num_heads_ablation} indicates effect of using a slicker stream on alignment. As depicted, using $\hat{n}_h = 8$ offers a good trade-off between alignment and FLOPs overhead. 

~\textbf{MLP FLOPs Optimization} The accelerator adapters operating on the accelerated residual stream are intentionally designed with lower rank than their counterparts in the base model. This reduces FLOP overhead by a factor proportional to $hiddenSize / rank$. Additionally, since the faster residual stream uses only 8 attention heads (compared to 32 in the slow residual stream of Phi-3), the subsequent MLP layers process a smaller set of activations, further reducing FLOPs by another factor of 1/4.

These optimizations significantly reduce the FLOP overhead per speculative draft generation, as illustrated in ~\cref{fig:flops_optmization}. Notably, while traditional early-exiting speculative approaches such as DEED require propagating the full slow residual state through the initial layers, incurring substantial computational costs, M2R2 achieves efficient token generation via slimmer, low-rank faster residual streams. In contrast, Medusa introduces considerable FLOP overhead due to per-head computations scaling with $d^2+dv$\footnote{Here $d$ denotes hidden state dimension while $v$ denotes vocab size.}, whereas M2R2 employs low-rank layers for both MLP and language modeling heads, maintaining computational efficiency. All experiments involving the M2R2 approach, as detailed in ~\cref{sec:experiments}, are conducted using these FLOPs optimizations.









% \[
% O_t^{E_j} = 
% \begin{cases} 
% 1, & \text{if } L(\hat{y}_t^{E_j}) = y_t^{E_j} \\
% 0, & \text{otherwise}
% \end{cases}
% \]




%add distillation
% We train accelerator adapters described in \cref{m2r2_method} to accelerate residual streams on next token prediction all in parallel since there are no gradient conflict issues as described in \cref{sec:grad_conflict}.

% \begin{align} \label{eq:mr_loss}
% L_{mr} =  & -\sum_{j = 1}^{E_n} (\sum_{t=1}^{T}\log p_{\theta} (\hat{y}_t^{E_j} | \hat{y}_{<t}, x)) \nonumber
% \end{align}

% where $\hat{y_t^{E_j}}$ denotes predicted logits obtained from accelerated residual stream at gate $E_j$ and time-step $t$ while $y_t^{E_j}$ denotes corresponding truth tokens. 

% Upon training of adapters responsible for accelerating residual streams, we train query, key, value parameters responsible for vertical latent attention of all gates in parallel as

% \begin{equation} \label{eq:arla_loss}
%     L_{arla} = -\frac{1}{N} (\sum_{t=1}^{T}(1\sum_{j=1}^{E_n} \left[ O_t^{E_j} \log(\hat{O}_t^{E_j}) + (1 - o_t^{E_j}) \log(1 - \hat{o_t}_{E_j}) \right]))
% \end{equation}

% where $\hat{O_t^{E_j}}$ denotes binary predicted logits obtained from vertical latent attention router described in \cref{sec:arla} at gate $E_j$ and timestep $t$ while $O_t^{E_j}$ denotes corresponding truth label. Truth labels for router are obtained by computing whether logit head application on $\hat{y}_t^j$ results in true next token prediction. Formally speaking, 

% $O_t^{E_j} = 1 if L(\hat{y_t^{E_j}}) == y_t^{E_j} , 0 otherwise$. 

% Parameters responsible for vertical latent attention are also trained in parallel as well. 

%todo: training slow and fast residuals together and distillation can be two training mdoes. 
%Distillation can be an ablation. 




% Although transformer decoding is memory bound on most mainstream accelerators, there could be scenarios where flop savings are crucial. For instance, on on-device settings power consumption is directly correlated with flops per decoding step and reducing flops does help with overall energy consumption. Vanilla early exiting methods help with flop reduction but suffer from mismatch between training and inference due to early exited tokens. If token at decoding step $t$, $T_t$ exited at layer $E_i$, while token $T_{t+k}$ exits at layer $E_j$ such that $E_i < E_j$, hidden state $H_{t+k}l$ does not have corresponding hidden state $H_tl$ to attend to where $E_i < l <= E_j$. One solution that's often used in literature is to rely on last hidden state available, $H_t{E_j}$, however it tends to be sub-optimal and does affect generation quality \cite{ref}.  To alleviate this mismatch while reducing flops, we train router such that attention mask between token $T_{t+k}$ and token $T_{<t+k}$ is given by: 

% \begin{equation}
%     a_{T_{{t+k}{T_{<t+k}}} = 1 if  E_{T_{<t+k}} >= E{T_{t+k}}
%     else 0
% \end{equation}

% This attention mask enables router to account for exited tokens and get trained accordingly. Since attention mechanism during decoding remains exactly same as that during training, impact on generation quality tends to be minimal as noted in \cref{fig:gen_auality_with_and_without_recompute_attention_show_flops}.  Although MoD does not suffer from training and inference mismatch, we observe that it suffers from discountinuity between pre-training and super-vised fine-tuning resulting in sub-optimal perplexity. On the other hand, our method doesn't not require pre-training , doesn't suffer from discountinuity, and achieves much better perplexity in super-vised fine-tuning and instruction tuning setups as shown in \cref{fig:Mod_vs_m2r2_loss_curves}.






% Our techniques are directly applicable in such scenarios.    




%expert loading with cuda streams in experiments


% \section{Methodology}
% 
% \js{Double check connection sentences between subsections}

\noindent \textbf{Notations.}
%We work on the 3D space.
% We denote 3D world coordinates with ${\bf{p}}=(x,y,z)$, and the camera viewing direction with $v=(\theta,\phi)$. 
% %from any 3D point in space with the angles $v=(\theta,\phi)$. %denoting pitch and yaw.
% The points in the 3D space have color $c^{{\bf{p}}, v}$ that depends on the 3D location $p$ and the viewpoint of the camera $\bf{v}$.
% The points also have a density value $\sigma^p$ that encodes how opaque that point is.
% We couple them together in $\bf{y}^p=\{c^{{\mathbf{p}}, v}, \sigma^p\}$.
% When examining a whole 3D object from multiple 3D locations in space, $i=1,...,N$, we denote the set of all 3D locations with $\mathbf{X}=\{\mathbf{x}^n\}_{n=1}^N$, and $\mathbf{Y}=\{\mathbf{y}^n\}_{n=1}^N$.
% Assuming a ray $r=(p, v)$ starting from location $p$ and from viewpoint $v$, with $x^r=\{x^r_i\}_{i=1}^P$ and $y^r=\{y^r_i\}_{i=1}^P$ we denote all the 3D locations, and colors on $P$ points sampled from the ray.
% Further, with ${\tilde{\bf{X}}}$ and ${\tilde{\bf{Y}}}$ we denote the observations we have, the set of camera rays ${\tilde{\bf{X}}} = \{  
% \tilde{\bf{x}}_n = r_n \}_n^{N}$, and the projected 2D pixels from the rays ${\tilde{\bf{Y}}} = \{  
% \tilde{\bf{y}}_n^p \}_n^{N}$. 
%
We denote 3D world coordinates by \(\mathbf{p} = (x, y, z)\) and a camera viewing direction by \(\mathbf{d} = (\theta, \phi)\). Each point in 3D space have its color \(\mathbf{c}(\mathbf{p}, \mathbf{d})\), which depends on the location \(\mathbf{p}\) and viewing direction \(\mathbf{d}\). Points also have a density value \(\sigma(\mathbf{p})\) that encodes opacity. We represent coordinates and view direction together as $\mathbf{x} = \{\mathbf{p},\mathbf{d} \}$, color and density together as \(\mathbf{y}(\mathbf{p}, \mathbf{d}) = \{\mathbf{c}(\mathbf{p}, \mathbf{d}), \sigma(\mathbf{p})\}\).
\str{This sentence sounds a bit strange, we can denote all 3D points like that anyways, no need to observe them from 'multiple locations', no? In general, I think the paragraph can be written more cleanly.}
When observing a 3D object from multiple locations, we denote all 3D points as \(\mathbf{X} = \{\mathbf{x}_n \}_{n=1}^N\) and their colors and densities as \(\mathbf{Y} = \{\mathbf{y}_n\}_{n=1}^N\).
Assuming a ray \(\mathbf{r} = (\mathbf{o}, \mathbf{d})\) starting from the camera origin \(\mathbf{o}\) and along direction \(\mathbf{d}\), we sample $P$ points along the ray, with \(\mathbf{x}^{\mathbf{r}} = \{{x}_i^\mathbf{r}\}_{i=1}^P\) and corresponding colors and densities \(\mathbf{y}^{\mathbf{r}} = \{{y}_i^{\mathbf{r}}\}_{i=1}^P\). Further, we denote the observations \(\widetilde{\mathbf{X}}\) and \(\widetilde{\mathbf{Y}}\) as: the set of camera rays \(\widetilde{\mathbf{X}} = \{\widetilde{\mathbf{x}}_n = \mathbf{r}_n\}_{n=1}^N\) and the projected 2D pixels from the rays \(\widetilde{\mathbf{Y}} = \{\widetilde{\mathbf{y}}_n\}_{n=1}^N\).



% \begin{figure}[t]
%   \centering
%   \includegraphics[width=0.49\textwidth]{Figures/problemstate.pdf} % Adjust the size and filename as needed
%   \caption{Framework.} % Caption for the figure
%   \label{fig:problem}
% \end{figure}



% \str{Can you clarify what exactly each notation style corresponds to? What is the $\sim$ for instance? The sampled new pixel views? And no tilde are the actual observations?}

\begin{figure}[htbp]
%\vspace{-5mm}
\centering
\centerline{
\includegraphics[width=0.95\columnwidth]{Figures/problemstate.pdf} 
} 
%\vspace{-2mm}
\caption{\textbf{Complete rendering from 3D points to a 2D pixel.}
}
%\vspace{-4mm}
\label{fig:problem}
\end{figure}

\textbf{Background on Neural Radiance Fields.}
We formally describe Neural Radiance Field (NeRF)~\citep{mildenhall2021nerf, arandjelovic2021nerf} as a continuous function \( f_{\text{NeRF}}: \mathbf{x} \mapsto \mathbf{y} \), which maps 3D world coordinates \(\mathbf{p}\) and viewing directions \(\mathbf{d}\) to color and density values \(\mathbf{y}\). 
That is, a NeRF function, \( f_{\text{NeRF}} \), is a neural network-based function that represents the whole 3D object (e.g., a car in Fig.~\ref{fig:problem}) as coordinates to color and density mappings. Learning a NeRF function of a 3D object is an inverse problem where we only have indirect observations of arbitrary 2D views of the 3D object, and we want to infer the entire 3D object's geometry and appearance.
With the NeRF function, given any camera pose, we can render a view on the corresponding 2D image plane by marching rays and using the corresponding colors and densities at the 3D points along the rays. Specifically, given a set of rays \(\mathbf{r}\) with view directions \(\mathbf{d}\), we obtain a corresponding 2D image. The integration along each ray corresponds to a specific pixel on the 2D image using the volume rendering technique described in~\cite{kajiya1984ray}, which is also illustrated in Fig.~\ref{fig:problem}. Details about the integration are given in Appendix~\ref{supp:nerf-render}. 



%\str{Write down the integral.}
%


% Neural Fields are normally considered as an optimization routine in a deterministic setting, whereby95
% the function fNeRF is fit perfectly to the available observations (akin to “overfitting” training data).


\subsection{Probabilistic NeRF Generalization}
% \js{probabilistic NeRF is not new}

\paragraph{Deterministic Neural Radiance Fields} Neural Radiance Fields are normally considered as an optimization routine in a deterministic setting~\citep{mildenhall2021nerf,barron2021mip}, whereby the function $f_{\text{NeRF}}$ fits specifically to the available observations (akin to ``overfitting'' training data).

\paragraph{Probabilistic Neural Radiance Fields} As we are not just interested in fitting a single and specific 3D object but want to learn how to infer the Neural Radiance Field of any 3D object,  we focus on probabilistic Neural Radiance Fields with the following factorization:
%
% \begin{equation}
%     p({\bf{\widetilde{y}}}|{\bf{\widetilde{x}}}, f_{NeRF}) \propto
%     \underbrace{p({\bf{\widetilde{y}}}| {\bf{{y}}}^{1:P}, {\bf{{x}}}^{1:P})}_{\text{Integration}}
%     \underbrace{p({\bf{{y}}}^{1:P}|{\bf{{x}}}^{1:P}, f_{\text{NeRF}})}_{\text{NeRF model}}
%     \underbrace{p({\bf{{x}}}^{1:P}|{\bf{\widetilde{x}}})}_{\text{Sampling}}.
% \label{eq: rendering}
% \end{equation}
% \begin{equation}
%     p({\bf{\widetilde{y}}}|{\bf{\widetilde{x}}}, f_{NeRF}) \propto
%     \underbrace{p({\bf{\widetilde{y}}}| {\bf{{y}}}^{\mathbf{r}}, {\bf{{x}}}^{\mathbf{r}})}_{\text{Integration}}
%     \underbrace{p({\bf{{y}}}^{\mathbf{r}}|{\bf{{x}}}^{\mathbf{r}}, f_{\text{NeRF}})}_{\text{NeRF model}}
%     \underbrace{p({\bf{{x}}}^{\mathbf{r}}|{\bf{\widetilde{x}}})}_{\text{Sampling}}.
% \label{eq: rendering}
% \end{equation}
% \begin{equation}
%     p({\bf{\widetilde{Y}}}_{T} | {\bf{\widetilde{X}}}_{T}) \varpropto
%     \underbrace{p({\bf{\widetilde{Y}}}_{T} | {\bf{{Y}}}_{T}, {\bf{{X}}}_{T})}_{\text{Integration}}
%     \underbrace{p({\bf{{Y}}}_{T} | {\bf{{X}}}_{T})}_{\text{NeRF Model}}
%     \underbrace{p({\bf{{X}}}_{T} | {\bf{\widetilde{X}}}_{T})}_{\text{Sampling}},
% \label{eq: definiation}
% \end{equation}
\begin{equation}
    p({\bf{\widetilde{Y}}} | {\bf{\widetilde{X}}}) \varpropto
    \underbrace{p({\bf{\widetilde{Y}}} | {\bf{{Y}}}, {\bf{{X}}})}_{\text{Integration}}
    \underbrace{p({\bf{{Y}}} | {\bf{{X}}})}_{\text{NeRF Model}}
    \underbrace{p({\bf{{X}}} | {\bf{\widetilde{X}}})}_{\text{Sampling}}.
\label{eq: probabilitic_NeRF}
\end{equation}
%
% \str{Is $f_\text{NeRF}$ now a random variable? Normally it is not.}
\str{This can also be writtena  more fluently}
The generation process of this probabilistic formulation is as follows.
We first start from (or sample) a set of rays $\widetilde{\mathbf{X}}$.
Conditioning on these rays, we sample 3D points in space $\mathbf{X} \big|\widetilde{\mathbf{X}}$.
Then, we map these 3D points into their colors and density values with the NeRF function, ${\bf{Y}} = f_{\text{NeRF}}({\bf{{X}}})$.
Last, we sample the 2D pixels of the viewing image that corresponds to the 3D ray ${\widetilde{\bf{Y}}}| {\bf{{Y}}}, {\bf{X}}$ with a probabilistic process. This corresponds to integrating colors and densities ${\bf{{Y}}}$ along the ray on locations ${\bf{X}}$.

% In the following sections, we will define the various probabilistic terms.
% \str{Here it would be good to be more explicit and say how are the various probabistic terms are defined. Or we can say that we will specify later, also in the context of Geometric NP. Either way, the current text below looks like deterministic relations, so I think we can not write them down here.}

\str{I suggest we go directly on conditional neural fields. The way we have it now, we only create extra confusion, unless we are the first to propose this decomposition (but there have been other probabilistic NeRFs before, no?). Or perhaps have better structure in the writing, otherwise it is confusing.. What is context, what target?}
The probabilistic model in \cref{eq: probabilitic_NeRF} is for a single 3D object, thus requiring optimizing a function $f_{\text{NeRF}}$ afresh for every new object, which is time-consuming. For NeRF generalization, we accelerate learning and improve generalization by amortizing the probabilistic model over multiple objects, obtaining per-object reconstructions by conditioning on context sets ${{\widetilde {\bf{X}}}_C, {\widetilde {\bf{Y}}}_C}$.
% \str{What about $\widetilde {\bf{X}}_T$? What is that? Also, why (1) has small letters, and here we have capitals?}
% These context variables are few observations from any new object, that is, the rays and the corresponding observed colors.
For clarity, we use ${(\cdot)}_{C}$ to indicate context sets with {a few new observations for a new object}, while ${(\cdot)}_{T}$ indicates target sets containing 3D points or camera rays from novel views of the same object.
Thus, we formulate a probabilistic NeRF for generalization as:
% \str{update this according to the above equation}
%
\begin{equation}
\begin{aligned}
    &p({\bf{\widetilde{Y}}}_{T} | {\bf{\widetilde{X}}}_{T}, {\bf{\widetilde{X}}}_{C}, {\bf{\widetilde{Y}}}_{C}) \varpropto \\
&    \underbrace{p({\bf{\widetilde{Y}}}_{T} | {\bf{{Y}}}_{T}, {\bf{{X}}}_{T})}_{\text{Integration}}
    \underbrace{p({\bf{{Y}}}_{T} | {\bf{{X}}}_{T}, {\bf{\widetilde{X}}}_{C}, {\bf{\widetilde{Y}}}_{C})}_{\text{NeRF Generalization}}
    \underbrace{p({\bf{{X}}}_{T} | {\bf{\widetilde{X}}}_{T})}_{\text{Sampling}}.
\end{aligned}
\label{eq: probabilitic_NeRF_generalization}
\end{equation}
%
%\str{Not sure if this sentence is good enough, please check later again.}
As this paper focuses on generalization with new 3D objects, we keep the same sampling and integrating processes as in ~\cref{eq: probabilitic_NeRF}. We turn our attention to the modeling of the predictive distribution $p({\bf{{Y}}}_{T}| {\bf{{X}}}_{T}, {\bf{\widetilde{X}}}_{C}, {\bf{\widetilde{Y}}}_{C})$ in the generalization step, which implies inferring the NeRF function.

\paragraph{Misalignment between 2D context and 3D structures} It is worth mentioning that the predictive distribution in 3D space is conditioned on 2D context pixels with their ray $\{{\bf{\widetilde{X}}}_{C}, {\bf{\widetilde{Y}}}_{C}\}$ and 3D target points ${\bf{X}}_{T}$, which is challenging due to potential information misalignment. Thus, we need strong inductive biases with 3D structure information to ensure that 2D and 3D conditional information is fused reliably.


% \str{Why do you call this the query step?}

% #############################
% By inferring the function distribution $p(f_{\text{NeRF}})$ from the context sets, we can obtain the predictive distribution as: 
% %
% \begin{equation}
%       p({\bf{Y}}_{T}| {\bf{X}}_{T}, {\bf{\widetilde{X}}}_{C}, {\bf{\widetilde{Y}}}_{C})  
%       = \int p({\bf{Y}}_{T}|f_{\text{NeRF}}, {\bf{X}}_{T}) p(f_{\text{NeRF}}| {{\bf{{X}}}_{T}, \bf{\widetilde{X}}}_{C}, {\bf{\widetilde{Y}}}_{C}) df_{\text{NeRF}} 
% \label{eq: gp_w/o_B}
% \end{equation}
% %
% where $p(f_{\text{NeRF}}| {\bf{X}}_{T} {\bf{\widetilde{X}}}_{C}, {\bf{\widetilde{Y}}}_{C})$ is the prior distribution of the NeRF function, and $p({\bf{Y}}_{T}|f_{\text{NeRF}}, {\bf{X}}_{T})$ is the likelihood term. We integrate the likelihood term over the latent space of all possible NeRF functions.
% %\str{The following two sentences may beed to be rewritten.}
% %It is important to note that the context variables by nature contain fewer views and thus less information per a new object.
% It is worth mentioning that inferring the NeRF function needs to incorporate 2D context pixels with their ray $\{{\bf{\widetilde{X}}}_{C}, {\bf{\widetilde{Y}}}_{C}\}$ and 3D target points ${\bf{X}}_{T}$, which is challenging due to potential information misalignment.  Thus, we need strong inductive biases with 3D structure information to ensure that 2D and 3D conditional information is fused reliably.
% #############################

% \subsection{Geometric Neural Processes for NeRF} 
\subsection{Geometric Bases} 
\label{sec: geometrybases}
% In NeRF generalization, given that the context set is expected to correspond to too few views with few 3D information, 
% % In NeRF generalization, given the context set corresponding to too few 2D views providing few 3D information, 
% fitting a model for $p({\bf{Y}}_{T}| {\bf{X}}_{T}, {\bf{\widetilde{X}}}_{C}, {\bf{\widetilde{Y}}}_{C})$ that generalizes well is challenging. 
To mitigate the information misalignment between 2D context views and 3D target points, we introduce geometric bases ${\bf{{B}}}_{C}=\{{\bf{b}}_i\}_{i=1}^{M}$, which {induces prior structure to the context set} $\{{\bf{\widetilde{X}}}_{C}, {\bf{\widetilde{Y}}}_{C}\}$ geometrically. $M$ is the number of geometric bases. 

\begin{figure*}[t]
  \centering  \includegraphics[width=0.99\textwidth]{ICLR2025/Figures/architecture-0.pdf} % Adjust the size and filename as needed
  % \vspace{-2mm}
\caption{\textbf{Illustration of our Geometric Neural Processes.} 
% We solve the problem of radiance field generalization by Neural Processes. 
% captures uncertainty induced by few available observations.
We cast radiance field generalization as a probabilistic modeling problem. Specifically, we first construct geometric bases ${\bf{B}}_C$ in 3D space from the 2D context sets ${\bf{\widetilde{X}}}_{C}, {\bf{\widetilde{Y}}}_{C}$ to model the 3D NeRF function (Section~\ref{sec: geometrybases}). We then infer the NeRF function by modulating a shared MLP through hierarchical latent variables ${\bf{z}}_{o}, {\bf{z}}_{r}$ and make predictions by the modulated MLP (Section~\ref{sec: hierar}). 
  The posterior distributions of the latent variables are inferred from the target sets ${\bf{\widetilde{X}}}_{T}, {\bf{\widetilde{Y}}}_{T}$, which supervises the priors during training (Section~\ref{sec: object}). 
  } % Caption for the figure
  \label{fig: framework}
  %\vspace{-2mm}
\end{figure*}

\str{where is the semantic representation coming from? Self-supervised models? Or is it learned?}
Each geometric basis consists of a Gaussian distribution in the 3D point space and a semantic representation, \textit{i.e.,} ${\bf{b}}_i = \{ \mathcal{N}(\mu_i, \Sigma_i); \omega_i\}$, 
%\str{What is $\omega_i$ in the equation? The weight of the Gaussian?We have mixtures of Gaussians? Please clarify.} 
where $\mu_i$ and $\Sigma_i$ are the mean and covariance matrix of $i$-th Gaussian in 3D space, and $\omega_i$ is its corresponding latent representation. 
Intuitively, the mixture of all 3D Gaussian distributions implies the structure of the object, while $\omega_i$ stores the corresponding semantic information.
% from a 2D context set, e.g., color and texture. 
In practice, we use a transformer-based encoder to learn the Gaussian distributions and representations from the context sets, \textit{i.e.,} $\{(\mu_i, \Sigma_i, \omega_i)\} = \texttt{Encoder} [{\bf{\widetilde{X}}}_{C}, {\bf{\widetilde{Y}}}_{C}]$. Detailed architecture of the encoder is provided in Appendix~\ref{supp:gaussian}. 

% \begin{equation}
%     {\bf{{B}}}_{C} = \{{\bf{b}}_i\}_{i=1}^{M}, {\bf{b}}_i=\{\mathcal{N}(\mu_i, \Sigma_i); \omega_i\},
%     \\
%      \mu_i, \Sigma_i, \omega_i = \texttt{Encoder} [{\bf{\widetilde{X}}}_{C}, {\bf{\widetilde{Y}}}_{C}],
% \end{equation}
% where $M$ is the number of Gaussian bases. 

% where ${\bf{{B}}}_{C}$ are Gaussian bases inferred from the context views $\{{\bf{\widetilde{X}}}_{C}, {\bf{\widetilde{Y}}}_{C}\}$ with 3D structure information, \textit{i.e.,} 
% ${\bf{{B}}}_{C}=\texttt{Encoder}\Big({\bf{\widetilde{X}}}_{C}, {\bf{\widetilde{Y}}}_{C}\Big)$. 


% To address the information loss in the context views, we develop a geometry-aware prior distribution for the NeRF function. The geometry-aware prior integrates a set of geometry bases ${\bf{{B}}}_{C}$ and the target location points ${\bf{X}}_{T}$, which enrich the context sets with the {structure locality information}. 
% By doing so, we reformulate the prior distribution of the NeRF function as:
% \begin{equation}
%     p(f_{\text{NeRF}}| {\bf{X}}_{T}, {\bf{\widetilde{X}}}_{C}, {\bf{\widetilde{Y}}}_{C}) = p(f_{\text{NeRF}}| {\bf{X}}_{T}, {\bf{{B}}}_{C}), 
% \label{eq: prior_f}
% \end{equation}
% \str{Can a deterministic variable be part of a probabilistic expression?d}
% where ${\bf{{B}}}_{C}$ is a set of Gaussian bases inferred from the context views $\{{\bf{\widetilde{X}}}_{C}, {\bf{\widetilde{Y}}}_{C}\}$ with 3D structure information, \textit{i.e.,} ${\bf{{B}}}_{C}=\texttt{Encoder}[{\bf{\widetilde{X}}}_{C}, {\bf{\widetilde{Y}}}_{C}]$. 
% Specifically, we construct ${\bf{{B}}}_{C}$ as:
% Geometry basis-agnostic ideas, like 4D scene tensor~\cite{chen2022tensorf}, and RBF kernels~\cite{chen2023neurbf} has been used in deterministic NeRF to store the 3D scene geometry and semantic information. This motivates us to use a set of geometry bases to represent both the geometry structure and semantic information of the scene. 
% We assume the space is spanned by a set of basis, with geometric shapes and high-dimensional representation. 
% The geometry basis ${\bf{B}}_C$ is given by a posterior distribution, $p_{\pi}( {\bf{{B}}}_{C}| {\bf{\widetilde{X}}}_{C}, {\bf{\widetilde{Y}}}_{C})$. This can be explained as the posterior knowledge (color and spatial location) of the scene when a human sees a view of a scene (context image). Then, we model the function distribution as:
% \begin{align}
%     &{\bf{{B}}}_{C} = \{{\bf{b}}_i\}_{i=1}^{M}, {\bf{b}}_i=\{\mathcal{N}(\mu_i, \Sigma_i); \omega_i\},
%     \label{eq: generation_B_1}
%     \\
%     & \mu_i, \Sigma_i = \texttt{Att}({\bf{\widetilde{X}}}_{C}, {\bf{\widetilde{Y}}}_{C}), \texttt{Att}({\bf{\widetilde{X}}}_{C}, {\bf{\widetilde{Y}}}_{C}),
%     \label{eq: generation_B_2}
%     \\
%     & \omega_i = \texttt{Att}({\bf{\widetilde{X}}}_{C}, {\bf{\widetilde{Y}}}_{C}),
%     \label{eq: generation_B_3}
% \end{align}
% where $M$ is the number of the Gaussian bases. $\mu \mathbb \in {R}^3$ is the Gaussian center, $\Sigma \in  \mathbb{R}^{3\times 3}$ is the covariance matrix, and $\omega \in \mathbb{R}^{d_B}$ is the corresponding ${d_B}$-dimension semantic representation. 
%Each Gaussian basis represents a 3D Gaussian kernel and its corresponding semantic information. The shape of a Gaussian kernel can reflect a local object structure.
%For each kernel, \js{details here: we use a visual self-attention to estimate the mean $\mu \mathbb \in {R}^3$ and covariance matrix $\Sigma \in  \mathbb{R}^{3\times 3}$, and a corresponding ${d_B}$-dimension semantic representation $\omega \in \mathbb{R}^{d_B}$. }

With the geometric bases $\mathbf{B}_C$, we review the predictive distribution from  $p({\bf{Y}}_{T}| {\bf{X}}_{T}, {\bf{\widetilde{X}}}_{C}, {\bf{\widetilde{Y}}}_{C})$ to $p({\bf{Y}}_{T}| {\bf{X}}_{T},{\bf{{B}}}_{C})$.  By inferring the function distribution $p(f_{\text{NeRF}})$, we reformulate the predictive distribution as: 
\begin{equation}
    % p({\bf{{Y}}}_{T} | {\bf{{X}}}_{T}, {\bf{\widetilde{X}}}_{C}, {\bf{\widetilde{Y}}}_{C}) = 
    p({\bf{{Y}}}_{T} | {\bf{{X}}}_{T}, {\bf{{B}}}_{C}) = \int p({\bf{Y}}_{T}|f_{\text{NeRF}}, {\bf{X}}_{T}) p(f_{\text{NeRF}}| {\bf{X}}_{T}, {\bf{B}}_{C}) df_{\text{NeRF}},
\label{eq: predictive_w_B}
\end{equation}
where $p(f_{\text{NeRF}}| {\bf{X}}_{T}, {\bf{B}}_{C})$ is the prior distribution of the NeRF function, and $p({\bf{Y}}_{T}|f_{\text{NeRF}}, {\bf{X}}_{T})$ is the likelihood term. 
% We integrate the likelihood term with all possible NeRF functions. 
%\str{How do we do this? The space to integrate over must be huge, no?}
%\str{The following sentence is a bit weird, can you check it again.}
% \wy{We integrate the likelihood term over the latent space of all possible variables for modulating NeRF functions by monte carlo sampling, which can be seen as integrating over a function distribution.}
Note that the prior distribution of the NeRF function is conditioned on the target points ${\bf{X}}_{T}$ and the geometric bases ${\bf{B}}_{C}$. 
Thus, the prior distribution is data-dependent on the target inputs, yielding a better generalization on novel target views of new objects. 
Moreover, since ${\bf{B}}_{C}$ is constructed with continuous Gaussian distributions in the 3D space, the geometric bases can enrich the locality and semantic information of each discrete target point, enhancing the capture of high-frequency details~\citep{chen2023neurbf,chen2022tensorf,muller2022instant}.

% Since ${\bf{B}}_{C}$ is constructed in the 3D space, 3D target points ${\bf{X}}_{T})$ is able to effectively interact with ${\bf{B}}_{C}$, alleviating the information misalignment.

%\zx{We need some advantages of B in this paragraph}
%\zx{maybe also refer some splatting/RBF kernel methods to introduce what B is, how does B contain 3D structure information}
% \wy{The inferred posterior about the Gaussian bases shapes is able to reflect the object shape, while the semantic posterior (e.g. color and texture) is embedded in the representation of the bases. Moreover, the Gaussian basis spanned in the space is able to help aggregate the locality information for each queried point, which facilitates learning representing high-frequency details~\cite{chen2023neurbf,chen2022tensorf,muller2022instant}.}

% By integrating the prior distribution in equation~\cref{eq: prior_f} into the predictive distribution in equation~\cref{eq: gp_w/o_B}, we replace  $p({\bf{Y}}_{T}| {\bf{X}}_{T}, {\bf{\widetilde{X}}}_{C}, {\bf{\widetilde{Y}}}_{C})$ with $p({\bf{{Y}}}_{T} | {\bf{{X}}}_{T}, {\bf{{B}}}_{C})$.
% The stochastic processes for NeRF with geometry bases are then derived as:
% % With the encoded geometry base ${\bf{B}}_{C}$, we reformulate the querying step as $p({\bf{{Y}}}_{T} | {\bf{{X}}}_{T}, {\bf{{B}}}_{C})$. By integrating the approximated prior with the bases Eq.~\ref{eq: prior_f} into the generative process in Eq.~\ref{eq: gp_w/o_B}, we derive the stochastic processes for NeRF with Geometry Bases as:
% \str{In 3, $p(f_{Nerf})$ is not conditioned on $X_T$. Is there a difference?}
% \begin{equation}
%     % p({\bf{{Y}}}_{T} | {\bf{{X}}}_{T}, {\bf{\widetilde{X}}}_{C}, {\bf{\widetilde{Y}}}_{C}) = 
%     p({\bf{{Y}}}_{T} | {\bf{{X}}}_{T}, {\bf{{B}}}_{C}) = \int p({\bf{Y}}_{T}|f_{\text{NeRF}}, {\bf{X}}_{T}) p(f_{\text{NeRF}}| {\bf{X}}_{T}, {\bf{B}}_{C}) df_{\text{NeRF}}.
% \label{eq: gp_w_B}
% \end{equation}
%
% Inferred from the context-generated 3D information in ${\bf{{B}}}_{C}$, the NeRF function prior $p(f_{\text{NeRF}}|{\bf{X}}_{T}, {\bf{{B}}}_{C})$ reduces the impact of information loss in context views (\wy{due to sampling and integration}) and becomes more suitable for the \zx{query step} $p({\bf{Y}}_{T}|f_{\text{NeRF}}, {\bf{X}}_{T})$ in 3D space.

\subsection{Geometric Neural Processes with Hierarchical Latent Variables}
\label{sec: hierar}

% To achieve the 
With the geometric bases, we propose Geometric Neural Processes (\textbf{\method{}}) by inferring the NeRF function distribution $p(f_{\text{NeRF}}|{\bf{X}}_{T}, {\bf{{B}}}_{C})$ in a probabilistic way.  
% We can generalize NeRF learning and efficiently adapt the functional distribution to new 3D objects.
Based on the probabilistic NeRF generalization in~\cref{eq: probabilitic_NeRF_generalization}, we introduce hierarchical latent variables to encode various spatial-specific information into $p(f_{\text{NeRF}}|{\bf{X}}_{T}, {\bf{{B}}}_{C})$, improving the generalization ability in different spatial levels.
%\str{Make sure that notation is consistent and not overloaded. Eg, $x^r$ rather than $x^\mathbf{r}$ since it is not that we use the $1:P$ somewhere specific, besides it is not clear that this corresponds to a ray, since the $P$ points could be anywhere.}
Since all rays are independent of each other, we decompose the predictive distribution in \cref{eq: predictive_w_B} as:
\begin{equation}
    p({\bf{Y}}_{T}| {\bf{X}}_{T},  {\bf{B}}_{C})  = \prod_{n=1}^{N} p({\bf{y}}_{T}^{\mathbf{r}, n}| {\bf{x}}_{T}^{{\mathbf{r}}, n},  {\bf{B}}_{C}),
\label{eq: predictive_distribution_ray_specific}
\end{equation}
where the target input ${\bf{X}}_{T}$ consists of $N \times P$ location points $\{{\bf{x}}_{T}^{{\mathbf{r}}, n}\}_{n=1}^{N}$ for $N$ rays.


\begin{figure}[htbp]
\centering
\includegraphics[width=0.9\columnwidth]{ICLR2025/Figures/graphical_model2.pdf} 
\caption{\textbf{Graphical model for the proposed geometric neural processes.}}
\label{fig: graphical_model}
\end{figure}

Further, we develop a hierarchical Bayes framework for \method{} to accommodate the data structure of the target input ${\bf{X}}_{T}$ in \cref{eq: predictive_distribution_ray_specific}.
We introduce an object-specific latent variable $\mathbf{z}_o$ and $N$ individual ray-specific latent variables $\{\mathbf{z}_r^{n}\}_{n=1}^{N}$ to represent the randomness of $f_\text{NeRF}$.
% the probabilistic NeRF function. 


\str{The formatting here looks weird. Is this the right template?}
Within the hierarchical Bayes framework, $\mathbf{z}_o$ encodes the entire object information from all target inputs and the geometric bases $\{\mathbf{X}_T, \mathbf{B}_C\}$ in the global level; while every $\mathbf{z}_r^{n}$ encodes ray-specific information from $\{ \mathbf{x}_T^{\mathbf{r}, n}, \mathbf{B}_C\}$ in the local level, which is also conditioned on the global latent variable $\mathbf{z}_o$. 
The hierarchical architecture allows the model to exploit the structure information from the geometric bases $\mathbf{B}_C$ in different levels, improving the model's expressiveness ability.
By introducing the hierarchical latent variables in \cref{eq: predictive_distribution_ray_specific}, we model \method{} as:
% \begin{equation}
% \small
%         p({\bf{Y}}_{T}| {\bf{X}}_{T}, {\bf{B}}_{C}) = \int \prod_{n=1}^{N} \Big\{ \int p({\bf{y}}_{T}^{\mathbf{r}, n}| {\bf{x}}_{T}^{\mathbf{r}, n}, {\bf{B}}_{C}, {\bf{z}}_r^n,{\bf{z}}_o, ) p({\bf{z}}_{r}^n| {\bf{z}}_o,  {\bf{x}}_{T}^{\mathbf{r}, n}, {\bf{B}}_C) d {\bf{z}}_r^n \Big\} p({\bf{z}}_o |{\bf{X}}_T, {\bf{B}}_C) d {\bf{z}}_o, 
% \label{eq:ganp-model}
% \end{equation}
{\small
\begin{equation}
\begin{aligned}
        p({\bf{Y}}_{T}| {\bf{X}}_{T}, {\bf{B}}_{C}) &= \int \prod_{n=1}^{N} \Big\{ \int p({\bf{y}}_{T}^{\mathbf{r}, n}| {\bf{x}}_{T}^{\mathbf{r}, n}, {\bf{B}}_{C}, {\bf{z}}_r^n,{\bf{z}}_o ) \\
        &p({\bf{z}}_{r}^n| {\bf{z}}_o,  {\bf{x}}_{T}^{\mathbf{r}, n}, {\bf{B}}_C) d {\bf{z}}_r^n \Big\} p({\bf{z}}_o |{\bf{X}}_T, {\bf{B}}_C) d {\bf{z}}_o,
\end{aligned}
\label{eq:ganp-model}
\end{equation}
}where $p({\bf{y}}_{T}^{\mathbf{r}, n}| {\bf{x}}_{T}^{\mathbf{r}, n}, {\bf{B}}_{C}, {\bf{z}}_o, {\bf{z}}_r^i)$ denotes the ray-specific likelihood term. In this term, we use the hierarchical latent variables $\{{\bf{z}}_o, {\bf{z}}_r^i\}$ to modulate a ray-specific NeRF function $f_{\text{NeRF}}$ for prediction, as shown in Fig.~\ref{fig: framework}.
% In general, we first use the object-specific latent variable $\mathbf{z}_o$ to make $f_{NeRF}$ object-specific. Then, the ray-specific latent variable $\mathbf{z}_r$ to enable $f_{\text{NeRF}}$ to capture the local texture information.
Hence, $f_{\text{NeRF}}$ can explore global information of the entire object and local information of each specific ray, leading to better generalization ability on new scenes and new views.
A graphical model of our method is provided in Fig.~\ref{fig: graphical_model}. 

% As mentioned in Eq.~\ref{eq: definiation}, the target inputs can be obtained by randomly sampling ${\bf{x}}_{T}^{1:P}$ from each ray ${\widetilde{\bf{x}}}_{T}$. Thus, target input ${\bf{X}}_{T}$ consists of $N \times P$ location points $\{{\bf{x}}_{T}^{1:P, n}\}_{n=1}^{N}$ given $N$ rays. 
% Since the rendering of each ray is independently \cite{martin2021nerf}, we reformulate the predictive distribution for NeRF in a ray-specific manner:
% \begin{equation}
%     p({\bf{Y}}_{T}| {\bf{X}}_{T},  {\bf{B}}_{C})  = \prod_{n=1}^{N} p({\bf{y}}_{T}^{1:P, n}| {\bf{x}}_{T}^{1:P, n},  {\bf{B}}_{C}).
% \label{eq: predictive_distribution_ray_specific}
% \end{equation}
% Based on the hierarchical data structure of the target inputs, we design a hierarchical latent variable model. In the proposed model, object-specific latent variables ${\bf{g}}$ encode the entire 3D object information; each ${\bf{g}}$ corresponds $N$ individual ray-specific latent variables $\{{\bf{r}}^{n}\}_{n=1}^{N}$.

%\str{Until here. I will check below tomorrow, but maybe you can use the same style like above, more concise and to the point.}
%\str{In general, we have to explain the hierarchical framework better. What is $g$ for instance. }



% that corresponds to the ray's viewpoint $v$. 

% Each viewing image is rendered from a camera pose by marching rays into the 2D image plane and computing colors.

% To optimize a NeRF function for a 3D object, we only have access to several viewing images and their ray information, \textit{e.g., $p({\bf{\widetilde{y}}}|{\bf{\widetilde{x}}}, f_{NeRF})$}.
% \zx{When learning the NeRF function of a 3D object, we only have access to arbitrary 2D views of the 3D object.
% Each viewing image is rendered from a camera pose by marching rays into the 2D image plane and computing colors.
% can render a viewing image of the object from any camera pose by marching rays into the 2D image plane and computing colors. 
% \str{The notation below looks inconsistent with that of the previou svariable, where we have $c_r$, ...}
% In general, each ray corresponds to a pixel in the 2D image from this view. 
% We use ${\bf{\widetilde{x}}} = [\mathbf{r}_o;\mathbf{r}_d] \in \mathbb{R}^{6}$ and ${\bf{\widetilde{y}}} = [r, g, b] \in \mathbb{R}^{3}$ to denote one specific ray and its corresponding pixel color, respectively. 
% Here $\mathbf{r}_o \in \mathbb{R}^3$ and $\mathbf{r}_d \in \mathbb{R}^3$ denote the origin and the direction of the ray, respectively. Each ray color ${\bf{\widetilde{y}}}$ is the RGB value for the corresponding pixel in the 2D image~\cite{arandjelovic2021nerf}.

% To render the ray color of each pixel, as illustrated in Fig.~\ref{fig:problem}, the volume rendering technique~\cite{kajiya1984ray} is used.
% $P$ discrete location points ${\bf{x}}^{1:P} =\{{\bf{x}}^p\}_{p=1}^P$ are first sampled along the ray, and their densities and colors ${\bf{y}}^{1:P} = \{{\bf{y}}^p\}_{p=1}^P$ are queried by the function $f_{\text{NeRF}}$.
% The information from all discrete locations is then integrated via a rendering equation. 
% % An illustration of the rendering process is shown in Fig.~\ref{fig:problem}.
% We factorize the complete rendering process $p({\bf{\widetilde{y}}}|{\bf{\widetilde{x}}})$ as follows:  
% \begin{equation}
%     p({\bf{\widetilde{y}}}|{\bf{\widetilde{x}}}, f_{NeRF}) \propto
%     \underbrace{p({\bf{\widetilde{y}}}| {\bf{{y}}}^{1:P}, {\bf{{x}}}^{1:P})}_{\text{Integration}}
%     \underbrace{p({\bf{{y}}}^{1:P}|{\bf{{x}}}^{1:P}, f_{NeRF})}_{\text{NeRF model}}
%     \underbrace{p({\bf{{x}}}^{1:P}|{\bf{\widetilde{x}}})}_{\text{Sampling}},
% \label{eq: rendering}
% \end{equation}
% \begin{equation}
%     p({\bf{\widetilde{y}}}|{\bf{\widetilde{x}}}) \varpropto p({\bf{\widetilde{y}}}| {\bf{{y}}}_{1:P}, {\bf{{x}}}_{1:P}) p({\bf{{y}}}_{1:P}|{\bf{{x}}}_{1:P})p({\bf{{x}}}_{1:P}|{\bf{\widetilde{x}}}),
% \label{eq: rendering}
% \end{equation}
% where $p({\bf{\widetilde{y}}}| {\bf{{y}}}^{1:P}, {\bf{{x}}}^{1:P})$ represents the integrating step
% % , where the pre-defined render equation computes the ray color 
% by weighted mixing all location colors using the pre-defined render equation ${\bf{\widetilde{y}}} = f_{\text{render}}({\bf{y}}^{1:P}, {\bf{x}}^{1:P})$  \cite{arandjelovic2021nerf,mildenhall2021nerf}.
% % etails of the equation can be found in~\cite{arandjelovic2021nerf,mildenhall2021nerf}.
% $p({\bf{y}}^{1:P}|{\bf{x}}^{1:P})$ denotes the NeRF modeling step for each queried points, \textit{e.g.}, ${\bf{y}}^{1:P}:= f_{\text{NeRF}}({\bf{x}}^{1}), f_{\text{NeRF}}({\bf{x}}^{2}), ..., f_{\text{NeRF}}({\bf{x}}^{P})$.
% $p({\bf{{x}}}^{1:P}|{\bf{\widetilde{x}}})$ denotes the randomly sampling step along the ray $\bf{\widetilde{x}}$. 
% \zx{Given a set of rays ${\bf{\widetilde{X}}}=\{ {\bf{\widetilde{x}}} \}^{N}$ from the same camera pose, we can obtain its corresponding 2D image ${\bf{\widetilde{Y}}}= \{ {\bf{\widetilde{y}}} \}^{N}$ consisting of all pixel colors by ray-specific rendering. Each specific ray corresponds to a pixel on the image.}
% Moreover, we use ${\bf{\widetilde{X}}}$ and ${\bf{\widetilde{Y}}}$ to represent sets of rays and their corresponding colors in 2D image plane.


% \subsection{Probabilistic Radiance Field Generalization} 

% Conventional methods~\cite{martin2021nerf,chen2022tensorf} usually train a deterministic neural network to optimize the NeRF function $f_{\text{NeRF}}$ in Eq. \ref{eq: rendering}. However, this requires the model to overfit large amounts of viewing images for each 3D object, which is data intensive and time-consuming, with poor generalization ability.

%However, it is difficult to collect location points and their location densities and colors in 3D space (\textcolor{red}{citation}). 
% By contrast, this paper focuses on radiance field generalization on arbitrary new 3D objects with few viewing images. 
% However, the method is applicable to arbitrary implicit neural representation (e.g. 2D images). 
% For a new 3D object
% Therefore, $f_{\text{NeRF}}$ needs to be quickly generalized to a new scene based on \textit{few-shot} context views $\{ {\bf{\widetilde{X}}}_C, {\bf{\widetilde{Y}}}_C\}$. 
% %images and their ray information, which are represented as the context set including ray and pixel color $\{ {\bf{\widetilde{X}}}_C, {\bf{\widetilde{Y}}}_C\}$. 
% The generalized NeRF function is then utilized to render any target set of ray ${\bf{\widetilde{X}}}_{T}$ to get the corresponding pixel color set ${\bf{\widetilde{Y}}}_{T}$ in 2D image plane.

% To achieve fast generalization with limited views, we formulate the problem of radiance field generalization in a probabilistic manner.
% The probabilistic formulation enables us to infer the NeRF function while considering uncertainty, improving the generalization ability on limited context data.
% Formally, we formulate the goal of radiance field generalization as $p({\bf{\widetilde{Y}}}_{T}| {\bf{\widetilde{X}}}_{T}, {\bf{\widetilde{X}}}_{C}, {\bf{\widetilde{Y}}}_{C} )$. 
% % The target distribution of radiance field generalization is $p({\bf{\widetilde{Y}}}_{T}| {\bf{\widetilde{X}}}_{T}, {\bf{\widetilde{X}}}_{C}, {\bf{\widetilde{Y}}}_{C} )$. 
% Considering the rendering with a NeRF function in Eq.~\ref{eq: rendering}, we review the target distribution as:
% % \begin{equation}
% %     p({\bf{\widetilde{Y}}}_{T} | {\bf{\widetilde{X}}}_{T}, {\bf{\widetilde{X}}}_{C}, {\bf{\widetilde{Y}}}_{C}) \varpropto
% %     \underbrace{p({\bf{\widetilde{Y}}}_{T} | {\bf{{Y}}}_{T}, {\bf{{X}}}_{T})}_{\text{Integration}}
% %     \underbrace{p({\bf{{Y}}}_{T} | {\bf{{X}}}_{T}, {\bf{\widetilde{X}}}_{C}, {\bf{\widetilde{Y}}}_{C})}_{\text{Generalization}}
% %     \underbrace{p({\bf{{X}}}_{T} | {\bf{\widetilde{X}}}_{T})}_{\text{Sampling}},
% % \label{eq: definiation}
% % \end{equation}
% \begin{equation}
%     p({\bf{\widetilde{Y}}}_{T} | {\bf{\widetilde{X}}}_{T}, {\bf{\widetilde{X}}}_{C}, {\bf{\widetilde{Y}}}_{C}) \varpropto
%     \underbrace{p({\bf{\widetilde{Y}}}_{T} | {\bf{{Y}}}_{T}, {\bf{{X}}}_{T})}_{\text{Integration}}
%     \underbrace{p({\bf{{Y}}}_{T} | {\bf{{X}}}_{T}, {\bf{\widetilde{X}}}_{C}, {\bf{\widetilde{Y}}}_{C})}_{\text{Generalization}}
%     \underbrace{p({\bf{{X}}}_{T} | {\bf{\widetilde{X}}}_{T})}_{\text{Sampling}},
% \label{eq: definiation}
% \end{equation}
% Without loss of generality, we share the fixed sampling and integrating processes as in Eq.~\ref{eq: rendering}. Thus, we focus on the modeling of query step $p({\bf{{Y}}}_{T}| {\bf{{X}}}_{T}, {\bf{\widetilde{X}}}_{C}, {\bf{\widetilde{Y}}}_{C})$.  

%\wy{The ray casting used in volume rendering (sampling process described above) enables sampling in the 3D space continuously, which }
% The probabilistic formulation enables us to obtain the NeRF function by neural processes, which capture uncertainty and directly infer the function in a single feed-forward pass. 
% This leads to both effective and efficient generalization of the NeRF function on new scenes.


% Conventional methods \zx{references\cite{}} usually learn a deterministic neural network $\theta_{NeRF}$ to model a NeRF function. This requires the neural network $\theta_{NeRF}$ to overfit on amounts of viewing images for a 3D object, which is time-consuming. 
% \begin{equation}
% p({\bf{{Y}}}_{T} | {\bf{{X}}}_{T}, {\bf{\widetilde{X}}}_{C}, {\bf{\widetilde{Y}}}_{C}) \approx  p({\bf{{Y}}}_{T} | {\bf{{X}}}_{T}, \theta^*_{NeRF}), 
% \end{equation}
% where the best neural network is estimated by maximum a posterior (MAP) on the support sets of ray and ray colors, \textit{e.g.}, $\theta^*_{NeRF} =\mathop{\arg\min}_{\theta} p(\theta_{NeRF}|{\bf{\widetilde{X}}}_{C}, {\bf{\widetilde{Y}}}_{C})$. For each new 3D object, $\theta_{NeRF}$ will be re-trained for overfitting, yielding worse generalizations. Hence, leveraging limited viewing images for radiance field generalization remains a problem. 

% \textbf{Neural Processes for Radiance Field Generalization.}
% \js{Motivation of generative processes in a probabilistic way}
% In radiance field generalization, each new 3D object only accesses a few viewing images. In this case, a deterministic NeRF function could not capture uncertainty caused by the limited viewing images, leading to worse generalization ability on new 3D objects. 


% \subsection{Stochastic Processes for Radiance Field Generalization with Geometry Basis} 
% \label{sec: geometrybases}

% \noindent \textbf{Stochastic Processes for NeRF.}
% To capture uncertainty and improve generalization ability on limited context views, we cast radiance field generalization as stochastic processes over the NeRF function in 3D space. 
% {The processes directly infer the NeRF function in a feedforward pass, leading to both effective and efficient generalization on new scenes.}
% Specifically, we formulate the predictive distribution of radiance field generalization as follows:
% \begin{equation}
%       p({\bf{Y}}_{T}| {\bf{X}}_{T}, {\bf{\widetilde{X}}}_{C}, {\bf{\widetilde{Y}}}_{C})  
%       = \int p({\bf{Y}}_{T}|f_{NeRF}, {\bf{X}}_{T}) p(f_{NeRF}| {\bf{\widetilde{X}}}_{C}, {\bf{\widetilde{Y}}}_{C}) df_{NeRF} 
% \label{eq: gp_w/o_B}
% \end{equation}
% where $p(f_{\text{NeRF}}| {\bf{\widetilde{X}}}_{C}, {\bf{\widetilde{Y}}}_{C})$ is a prior distribution of the NeRF function.
% Since $f_{\text{NeRF}}$ is constructed on 3D space, it should ideally be inferred from the 3D location context sets $\{{\bf{{X}}}_{C}, {\bf{{Y}}}_{C}\}$. 
% \str{I am not sure the following sentences really make a (important) point.}
% However, the 3D location information is not available during practice training.
% The only accessible information of the 3D object is the limited viewing images $\{{\bf{\widetilde{X}}}_{C}, {\bf{\widetilde{Y}}}_{C}\}$.
% Moreover, the discrete sampling and integration for rendering results in information loss of the context sets from 3D location space to 2D views (Fig. \ref{fig:problem}).
% Therefore, inferring $f_{\text{NeRF}}$ by the context views is less optimal.
% However, NeRF mapping location points to their location colors and densities should be ideally inferred from the location sets $\{{\bf{{X}}}_{C}, {\bf{{Y}}}_{C}\}$, rather than the ray sets $\{{\bf{\widetilde{X}}}_{C}, {\bf{\widetilde{Y}}}_{C}\}$.
% In practice, the former is not available during training. 
% We can not directly transfer the context sets in the ray or camera space $\{{\bf{{X}}}_{C}, {\bf{{Y}}}_{C}\}$ to the 3D location space $\{{\bf{\widetilde{X}}}_{C}, {\bf{\widetilde{Y}}}_{C}\}$. This is intractable since the information loss of the context sets from the location space (3D) to the ray space (2D) due to the discrete sampling and integration. 

% To better capture the uncertainty with the context views, we cast radiance field generation as a stochastic process over a NeRF function in the 3D space. 
% The prior distribution of the function should be formulated as $p(f_{\text{NeRF}}| {\bf{X}}_{T}, {\bf{{X}}}_{C}, {\bf{{Y}}}_{C})$. 
% However, in practice, ${\bf{{X}}}_{C}, {\bf{{Y}}}_{C}$ (the context radiance field) is not available during training. We can not directly transfer the context sets in the ray or camera space $\{{\bf{{X}}}_{C}, {\bf{{Y}}}_{C}\}$ to the 3D location space $\{{\bf{\widetilde{X}}}_{C}, {\bf{\widetilde{Y}}}_{C}\}$. This is intractable since the information loss of the context sets from the location space (3D) to the ray space (2D) due to the discrete sampling and integration. We introduce a set of geometry basis to address this issue, which can also enrich the query points with the structure locality information.

% Thus, we can factorize the query process in Eq.~\ref{eq: definiation} as:
% \begin{equation}
%     p({\bf{Y}}_{T}| {\bf{X}}_{T}, {\bf{\widetilde{X}}}_{C}, {\bf{\widetilde{Y}}}_{C}) \varpropto p({\bf{Y}}_{T}, f_{NeRF}| {\bf{X}}_{T}, {\bf{{X}}}_{C}, {\bf{{Y}}}_{C}) p( {\bf{{X}}}_{C}, {\bf{{Y}}}_{C} | {\bf{\widetilde{X}}}_{C}, {\bf{\widetilde{Y}}}_{C}),
% \label{eq: query process1}
% \end{equation}
% where $p({\bf{{X}}}_{C}, {\bf{{Y}}}_{C} | {\bf{\widetilde{X}}}_{C}, {\bf{\widetilde{Y}}}_{C})$ denotes generating process from ray to location spaces, which is reverse from the complete rendering from location to ray spaces in Eq.~\ref{eq: rendering}. Therefore, this generating process is intractable since the information loss of the support sets from the location space (3D) to the ray space (2D). 

% \noindent \textbf{Geometry Basis.}
% To address the information loss in the context views, we develop a geometry-aware prior distribution for the NeRF function. The geometry-aware prior integrates a set of geometry bases ${\bf{{B}}}_{C}$ and the target location points ${\bf{X}}_{T}$, which enrich the context sets with the {structure locality information}. 
% By doing so, we reformulate the prior distribution of the NeRF function as:
% \begin{equation}
%     p(f_{\text{NeRF}}| {\bf{X}}_{T}, {\bf{\widetilde{X}}}_{C}, {\bf{\widetilde{Y}}}_{C}) = p(f_{\text{NeRF}}| {\bf{X}}_{T}, {\bf{{B}}}_{C}), 
% \label{eq: prior_f}
% \end{equation}
% \str{Can a deterministic variable be part of a probabilistic expression?d}
% where ${\bf{{B}}}_{C}$ is a set of Gaussian bases inferred from the context views $\{{\bf{\widetilde{X}}}_{C}, {\bf{\widetilde{Y}}}_{C}\}$ with 3D structure information, \textit{i.e.,} ${\bf{{B}}}_{C}=\texttt{Encoder}[{\bf{\widetilde{X}}}_{C}, {\bf{\widetilde{Y}}}_{C}]$. Specifically, we construct ${\bf{{B}}}_{C}$ as:
% % Geometry basis-agnostic ideas, like 4D scene tensor~\cite{chen2022tensorf}, and RBF kernels~\cite{chen2023neurbf} has been used in deterministic NeRF to store the 3D scene geometry and semantic information. This motivates us to use a set of geometry bases to represent both the geometry structure and semantic information of the scene. 
% % We assume the space is spanned by a set of basis, with geometric shapes and high-dimensional representation. 
% % The geometry basis ${\bf{B}}_C$ is given by a posterior distribution, $p_{\pi}( {\bf{{B}}}_{C}| {\bf{\widetilde{X}}}_{C}, {\bf{\widetilde{Y}}}_{C})$. This can be explained as the posterior knowledge (color and spatial location) of the scene when a human sees a view of a scene (context image). Then, we model the function distribution as:
% \begin{align}
%     &{\bf{{B}}}_{C} = \{{\bf{b}}_i\}_{i=1}^{M}, {\bf{b}}_i=\{\mathcal{N}(\mu_i, \Sigma_i); \omega_i\},
%     \label{eq: generation_B_1}
%     \\
%     & \mu_i, \Sigma_i = \texttt{Att}({\bf{\widetilde{X}}}_{C}, {\bf{\widetilde{Y}}}_{C}), \texttt{Att}({\bf{\widetilde{X}}}_{C}, {\bf{\widetilde{Y}}}_{C}),
%     \label{eq: generation_B_2}
%     \\
%     & \omega_i = \texttt{Att}({\bf{\widetilde{X}}}_{C}, {\bf{\widetilde{Y}}}_{C}),
%     \label{eq: generation_B_3}
% \end{align}
% where $M$ is the number of the Gaussian bases. $\mu \mathbb \in {R}^3$ is the Gaussian center, $\Sigma \in  \mathbb{R}^{3\times 3}$ is the covariance matrix, and $\omega \in \mathbb{R}^{d_B}$ is the corresponding ${d_B}$-dimension semantic representation. 
% %Each Gaussian basis represents a 3D Gaussian kernel and its corresponding semantic information. The shape of a Gaussian kernel can reflect a local object structure.
% %For each kernel, \js{details here: we use a visual self-attention to estimate the mean $\mu \mathbb \in {R}^3$ and covariance matrix $\Sigma \in  \mathbb{R}^{3\times 3}$, and a corresponding ${d_B}$-dimension semantic representation $\omega \in \mathbb{R}^{d_B}$. }
% Since the Gaussian bases are continuous in the location space, we can refine arbitrary location points with the Gaussian bases.
% \zx{We need some advantages of B in this paragraph}
% \zx{maybe also refer some splatting/RBF kernel methods to introduce what B is, how does B contain 3D structure information}

% \js{too details, use some \texttt{Atten} to obtain each parameters.}
% Given the context set $[\widetilde{X};\widetilde{Y}] \in \mathbb{R}^{H\times W \times (3+3+3)}$, a visual self-attention module first produces a $M\times D$ tokens with $M$ is the number of visual tokens and $D$ is the hidden dimension. The number of Gaussians we use equals to the number of tokens $M$. Then, we use one MLP to predict centers $\mu$, as well as the rotation $R$ and scaling $S$ matrices parameters for producing covariance matrix $\Sigma$, and one MLP to produce the latent representations $\omega$. The covariance matrix is obtained by $\Sigma^C = RSS^TR^T$.
% \begin{equation}
%     \Sigma = RSS^TR^T.
%     \label{eq:cov-matrix}
% \end{equation}
%where $R\in \mathbb{R}^{3\times3}$ is the rotation matrix, and $S \in \mathbb{R}^3$ is the scaling matrix. 



% The generative process of the NeRF function can be formulated in a probabilistic way:

% Different from previous works, we cast radiance field generalization as a 

% we the distribution over a
% single function $p( )$




% \begin{equation}
%     p({\bf{Y}}_{T}| {\bf{X}}_{T}, {\bf{\widetilde{X}}}_{C}, {\bf{\widetilde{Y}}}_{C}) =  \int p({\bf{Y}}_{T}| f_{NeRF}, {\bf{X}}_{T}) p(f_{NeRF} | {\bf{\widetilde{X}}}_{C}, {\bf{\widetilde{Y}}}_{C}) d f_{NeRF}
% \end{equation}


%To improve the generalization ability and achieve efficient inference on new scenes of Neural Field methods, we propose to learn to infer specific model parameters for each scene by Neural Processes.
%\zx{Advantages}
%In the following, we use NeRF generalization to illustrate the proposed method.  In NeRF generalization, a few views (one or two) of images $I_c$ and its corresponding camera pose $\mathbf{p}$ serve as the context information. 

%We represent each location in this scene by aggregating this Gaussian basis based on the radial basis function (RBF). 
%Then, based on the context information, the task is to estimate an implicit function that can be used to infer the target for unseen views.

% The proposed GP-INR framework assumes a space composed of a set of Gaussian basis that encode the spatial location and semantic information of a scene. Initially, GP-INR predicts a Gaussian basis $B_C$ from the observed context. Similar to the recent advances in implicit neural fields (e.g., Neurbf~\cite{chen2023neurbf}), at arbitrary new locations $X_T \in \mathbb{R}^{N\times P\times 3}$, we aggregate the Gaussians to form a new representation of this location $X'_C$ via the radial basis function (RBF). By averaging the representations both globally and locally along rays, we obtain a global latent $g_C$ and a ray-specific latent $r_C$ to modulate an MLP for predicting the implicit neural field. During training, with the target view and image $I_T$ also provided, we perform the same procedures for $I_T$ and align latent variables from both the context and target. For simplicity, in the following, we will use $I_C$ as an example to illustrate. 

% Introducing B
% ; 2). enrich the query points representation by aggregating the locality scene information

% \noindent \textbf{Stochastic Processes with Geometry Bases.} 
% By integrating the prior distribution with geometry bases ${\bf{B}}_{C}$ (Eq. \ref{eq: prior_f}) into the predictive distribution in Eq. \ref{eq: gp_w/o_B}, we reformulate the predictive distribution $p({\bf{Y}}_{T}| {\bf{X}}_{T}, {\bf{\widetilde{X}}}_{C}, {\bf{\widetilde{Y}}}_{C})$ as $p({\bf{{Y}}}_{T} | {\bf{{X}}}_{T}, {\bf{{B}}}_{C})$.
% The stochastic processes for NeRF with geometry bases are then derived as:
% % With the encoded geometry base ${\bf{B}}_{C}$, we reformulate the querying step as $p({\bf{{Y}}}_{T} | {\bf{{X}}}_{T}, {\bf{{B}}}_{C})$. By integrating the approximated prior with the bases Eq.~\ref{eq: prior_f} into the generative process in Eq.~\ref{eq: gp_w/o_B}, we derive the stochastic processes for NeRF with Geometry Bases as:
% \begin{equation}
%     % p({\bf{{Y}}}_{T} | {\bf{{X}}}_{T}, {\bf{\widetilde{X}}}_{C}, {\bf{\widetilde{Y}}}_{C}) = 
%     p({\bf{{Y}}}_{T} | {\bf{{X}}}_{T}, {\bf{{B}}}_{C}) = \int p({\bf{Y}}_{T}|f_{NeRF}, {\bf{X}}_{T}) p(f_{NeRF}| {\bf{X}}_{T}, {\bf{B}}_{C}) df_{NeRF}.
% \label{eq: gp_w_B}
% \end{equation}
% Inferred from the context-generated 3D information in ${\bf{{B}}}_{C}$, the NeRF function prior $p(f_{\text{NeRF}}|{\bf{X}}_{T}, {\bf{{B}}}_{C})$ reduces the impact of information loss in context views and becomes more suitable for the \zx{query step} $p({\bf{Y}}_{T}|f_{\text{NeRF}}, {\bf{X}}_{T})$ in 3D space.
% Moreover, since we consider the target location information ${\bf{X}}_{T}$ as well in $p(f_{\text{NeRF}}|{\bf{X}}_{T}, {\bf{{B}}}_{C})$, the NeRF function $f_{\text{NeRF}}$ is more generalizable to new views of new objects.

% This formulation provides a proxy between 2D context information and the 3D space by representing the 3D space with the continuous Gaussian basis. $p(f_{\text{NeRF}}|{\bf{X}}_{T}, {\bf{{B}}}_{C})$ is the conditional function distribution where we sample the modulation variable to modulate a shared MLP as the NeRF function, and $p({\bf{Y}}_{T}|f_{\text{NeRF}}, {\bf{X}}_{T})$ is NeRF \zx{rendering process}. We provide a new perspective to model the function distribution fully in 3D space. By doing so, we eliminate the information loss from 2D context to 3D location space.

% \begin{equation}
%     X'_T = \text{MLP}(\sum_i^{M} \varphi(x, \mu_i, \Sigma_i)\omega_i)
% \end{equation}s

% \begin{equation}
%     \varphi(x, \mu_i, \Sigma_i) = \exp (-(x-\mu_i)^T\Sigma_i^{-1}(x-\mu_i) /2)
% \end{equation}

%Hence, instead of directly using the 2D context set, we update the Gaussian basis (both shape and representation) in 3D space. Then, it is natural to use a continuous aggregation function (e.g. Gaussian radial basis function) to enrich the information at each location. By doing so, we eliminate the information loss from 2D context to 3D location space. 




% \subsection{Modeling of Geometry-aware Neural Processes}
% \subsection{Geometry-aware Neural Processes with Hierarchical Modulation}
% \label{sec: hierar}

% % To achieve the 
% To implement the inference of the NeRF function, we propose the Geometric Neural Process (\textbf{GeomNP}) for radiance field generalization, which efficiently adapt the functional distribution to new 3D objects. Specifically, \textbf{GeomNP} are constructed on a hierarchical Bayes framework. For 3D objects, the framework encodes object-specific and ray-specific latent variables in different levels, which helps the developed model querying the structure information from the Geometry bases $\mathbf{B}_C$. 

% \zx{why and how neural processes?}
% In this section, we present how to perform the \zx{rendering process} $p({\bf{Y}}_{T}| {\bf{X}}_{T},  {\bf{B}}_{C})$, including sampling a function from $p(f_{\text{NeRF}}|{\bf{X}}_{T}, {\bf{{B}}}_{C})$. 

% \noindent{{\textbf{Hierarchical Latent Variables.}}}
% The motivation of hierarchical latent variables in our model is the data structure of target inputs ${\bf{X}}_{T}$.
% As mentioned in Eq.~\ref{eq: definiation}, the target inputs can be obtained by randomly sampling ${\bf{x}}_{T}^{1:P}$ from each ray ${\widetilde{\bf{x}}}_{T}$. Thus, target input ${\bf{X}}_{T}$ consists of $N \times P$ location points $\{{\bf{x}}_{T}^{1:P, n}\}_{n=1}^{N}$ given $N$ rays. 
% Since the rendering of each ray is independently \cite{martin2021nerf}, we reformulate the predictive distribution for NeRF in a ray-specific manner:
% \begin{equation}
%     p({\bf{Y}}_{T}| {\bf{X}}_{T},  {\bf{B}}_{C})  = \prod_{n=1}^{N} p({\bf{y}}_{T}^{1:P, n}| {\bf{x}}_{T}^{1:P, n},  {\bf{B}}_{C}).
% \label{eq: predictive_distribution_ray_specific}
% \end{equation}
% Based on the hierarchical data structure of the target inputs, we design a hierarchical latent variable model. In the proposed model, object-specific latent variables ${\bf{g}}$ encode the entire 3D object information; each ${\bf{g}}$ corresponds $N$ individual ray-specific latent variables $\{{\bf{r}}^{n}\}_{n=1}^{N}$.

% \begin{wrapfigure}{r}{0.5\textwidth}
% \vspace{-5mm}
% \centering
% \centerline{
% \includegraphics[width=0.45\columnwidth]{Figures/graphical_model.png} 
% } 
% \vspace{-2mm}
% \caption{\textbf{Graphical model for the proposed geometric neural processes.}
% }
% \vspace{-2mm}
% \label{fig: graphical_model}
% \end{wrapfigure}
%${\mathbf{z}_o}$

% \subsection{Inferring with Geometric Bases and Modulation}

% \str{The following paragraph is not very clear.}

In the modeling of {\method{}}, the prior distribution of each hierarchical latent variable is conditioned on the geometric bases and target input. 
%\textcolor{blue}{To infer each latent variable, we first integrate the geometric bases and each target input, yielding a location representation specific to the target input. The location representation has access to relevant locality information from the geometry bases, \textit{i.e.}, $<{\bf{x}}_{T}^{\mathbf{r}, n}, {\bf{B}}_C >$. }
% For generalization, we need to infer latent variables that are specific to the target input. 
% To this end, 
We first represent each target location by integrating the geometric bases, \textit{i.e.}, $<{\bf{x}}_{T}^{n}, {\bf{B}}_C >$, which aggregates the relevant locality and semantic information for the given input. 
Since ${\bf{B}}_{C}$ contains $M$ Gaussians, we employ a Gaussian radial basis function in \cref{eq:rbf_agg} between each target input ${\bf{x}}_{T}^{ n}$ and each geometric basis ${\bf{b}}_i$ to aggregate the structural and semantic information to the 3D location representation. Thus, we obtain the 3D location representation as follows:
\begin{equation}
\label{eq:rbf_agg}
    <{\bf{x}}_{T}^{n}, {\bf{B}}_C > = \texttt{MLP}\Big[\sum_i^{M} \exp (-\frac{1}{2}({\bf{x}}_{T}^{n}-\mu_i)^T\Sigma_i^{-1}({\bf{x}}_{T}^{n}-\mu_i) ) \cdot \omega_i\Big],
\end{equation} 
where $\texttt{MLP}[\cdot]$ is a learnable neural network.
With the location representation $<{\bf{x}}_{T}^{n}, {\bf{B}}_C >$, we next infer each latent variable hierarchically, in object and ray levels. 

\noindent {\textbf{Object-specific Latent Variable.}} The distribution of the object-specific latent variable ${\bf{z}}_o$ is obtained by aggregating all location representations:
\begin{equation}
    [\mu_{{o}}, \sigma_{{o}}] 
    = \texttt{MLP}\Big[\frac{1}{N \times P}\sum_{n = 1}^{N}\sum_{\mathbf{r}}
    <{\bf{x}}_{T}^{n}, {\bf{B}}_C >\Big],
\end{equation} 
where we assume $p({\bf{z}}_o | {\bf{B}}_C,  {\bf{X}}_T)$ is a standard Gaussian distribution and generate its mean $\mu_{o}$ and variance $\sigma_{o}$ by a ~\texttt{MLP}. 
Thus, our model captures objective-specific uncertainty in the NeRF function.


\noindent {\textbf{Ray-specific Latent Variable.}} 
% By ray-specific latent variable, the object-specific is expected to capture the local details.
To generate the distribution of the ray-specific latent variable, we first average the location representations ray-wisely. 
We then obtain the ray-specific latent variable by aggregating the averaged location representation and the object latent variable through a lightweight transformer. We formulate the inference of the ray-specific latent variable as:
\begin{equation}
    [\mu_{{r}}, \sigma_{{r}}] = \texttt{Transformer} \Big[\texttt{MLP}[\frac{1}{P}\sum_{\mathbf{r}}
    <{\bf{x}}_{T}^{n}, {\bf{B}}_C >]; \hat{{\bf{z}}}_o \Big],
\end{equation}
where $\hat{{\bf{z}}}_o$ is a sample from the prior distribution $p({\bf{z}}_o | {\bf{X}}_T, {\bf{B}}_C)$. 
Similar to the object-specific latent variable, we also assume the distribution $p({\bf{z}}_r^n| {\bf{z}}_o,  {\bf{x}}_{T}^{\mathbf{r}, n}, {\bf{B}}_C)$ is a mean-field Gaussian distribution with the mean $\mu_{{r}}$ and variance $\sigma_{{r}}$. We provide more details of the latent variables in Appendix~\ref{supp:latent-variables}.

% The detail of the used Transformer is given in the Appendix. In summary, our latent variable is in a hierarchical structure. 
% The location-specific latent variable $r_C$ is to modulate the NeRF function specific to the queried ray (or pixel location in 2D image). We infer a ray-specific variable $r_c$ by first aggregating the feature ray-wisely:
% \begin{equation}
%     r'_c = \text{MLP}(\frac{1}{ P}\sum_j^{P}(\bf{X}'_T)[j])
% \end{equation}


%Then, we perform a hierarchical formulation by incorporating the global variable to obtain the ray-specific variable. This enables us to design the latent variable in a hierarchical structure, which is rational as it learns the scene coarse-to-fine. 

% Then, $r_c$ is obtained by feeding $r'_c$ together with the sampled $\hat{g}_c$ into a Transformer: 
% \begin{equation}
%     r_c = [\mu^c_r, \sigma^c_r] = \text{Transformer}([\hat{g}_c; \hat{r}_C]),
% \end{equation}
% where $\mu^r_g$ and $\sigma^r_g$ are the mean and variance of global variables, respectively. The detail of the used Transformer is given in the Appendix. In summary, our latent variable is in a hierarchical structure. 

\noindent  \textbf{NeRF Function Modulation.}
With the hierarchical latent variables $\{{\bf{z}}_o, {\bf{z}}_r^n\}$, we modulate a neural network for a 3D object in both object-specific and ray-specific levels.  Specifically, the modulation of each layer is achieved by scaling its weight matrix with a style vector~\citep{guo2023versatile}. 
The object-specific latent variable ${\bf{z}}_o$ and ray-specific latent variable ${\bf{z}}_r^n$ are taken as style vectors of the low-level layers and high-level layers, respectively. The prediction distribution $p({\bf{Y}}_{T}| {\bf{X}}_{T}, {\bf{B}}_{C})$ are finally obtained by passing each location representation through the modulated neural network for the NeRF function. 
More details are provided in Appendix~\ref{supp:modulate}. 
% The modulated MLP layer used in our paper is similar to the style \textit{modulation} in ~\cite{guo2023versatile}. Essentially, we predict a style vector $s\in \mathbb{R}^{d_{in}}$ to multiply or scale the weight matrix of an MLP, $W \in \mathbb{R}^{d_{in} \times d_{out}}$.


% \subsection{Inference of Geometry-aware Neural Processes}
% \label{sec: elbo}
% ELBO

% \noindent{\textbf{Variational Posteriors with the Geometry Bases.}} Solving \textbf{GANPs} with Eq.~\ref{eq:ganp-model} involves estimating the true posterior, $p({\bf{g}},  \{{\bf{r}}_i\}_{r=1}^{N_{ray}} | {\widetilde{\bf{X}}}_T, {\widetilde{\bf{Y}}}_T)$ which is intractable. Hence, we introduce a variational posterior distribution, which can be factorized as follows:
% \begin{equation}
% p({\bf{g}},  \{{\bf{r}}_i\}_{r=1}^{N_{ray}} | {\widetilde{\bf{X}}}_T, {\widetilde{\bf{Y}}}_T) \approx q_{\theta, \phi}({\bf{g}},  \{{\bf{r}}_i\}_{r=1}^{N_{ray}} | {\bf{X}}_T, {\bf{B}}_T),
% \end{equation}

\subsection{Empirical Objective}
\label{sec: object}

\noindent{\textbf{Evidence Lower Bound.}} 
To optimize the proposed \method{},
we apply variational inference~\citep{garnelo2018neural} and derive the evidence lower bound (ELBO) as:
\begin{equation}
\begin{aligned}
% \mathcal{L}_{\text{ELBO}}
& \log   p({\bf{Y}}_{T}| {\bf{X}}_{T}, {\bf{B}}_{C})
\geq \\
&\mathbb{E}_{q({\bf{z}}_o | {\bf{B}}_T,  {\bf{X}}_T)}  \Big\{  \sum_{n=1}^{N}  \mathbb{E}_{q({\bf{z}}_r^n| {\bf{z}}_o,  {\bf{x}}_{T}^{\mathbf{r}, n}, {\bf{B}_T})} \log p({\bf{y}}_{T}^{{\mathbf{r}}, n}| {\bf{x}}_{T}^{{\mathbf{r}}, n}, {\bf{z}}_o, {\bf{z}}_r^n) \\
&- D_{\text{KL}}[q({\bf{z}}_r^n| {\bf{z}}_o,  {\bf{x}}_{T}^{{\mathbf{r}}, n}, {{\bf{B}}_T}) || p({\bf{z}}_r^n| {\bf{z}}_o,  {\bf{x}}_{T}^{{\mathbf{r}}, n}, {{\bf{B}}_C}) ] \Big\} \\
& - D_{\text{KL}}[q({\bf{z}}_o | {\bf{B}}_T,  {\bf{X}}_T) || p({\bf{z}}_o | {\bf{B}}_C,  {\bf{X}}_T)], \\
\end{aligned}
\end{equation}
where $q_{\theta, \phi}({\bf{z}}_o,  \{{\bf{z}}_r^i\}_{i=1}^{N} | {\bf{X}}_T, {\bf{B}}_T) = \Pi_{i=1}^{N}q({\bf{z}}_r^n| {\bf{z}}_o,  {\bf{x}}_{T}^{{\mathbf{r}}, n}, {{\bf{B}}_T}) q({\bf{z}}_o | {\bf{B}}_T,  {\bf{X}}_T)$ is the involved variational posterior for the hierarchical latent variables.  ${\bf{B}}_T$ is the geometric bases constructed from the target sets $\{{\bf{\widetilde{X}}}_{T}, {\bf{\widetilde{Y}}}_{T}\}$, which are only accessible during training. 
The variational posteriors are inferred from the target sets during training, which introduces more information on the object. 
The prior distributions are supervised by the variational posterior using Kullback–Leibler (KL) divergence, learning to model more object information with limited context data and generalize to new scenes. Detailed derivations are provided in Appendix~\ref{supp:elbo}.

% \noindent{\textbf{Empirical Objective.}} 
For the geometric bases $\mathbf{B}_C$, we regularize the spatial shape of the context geometric bases to be closer to that of the target one $\mathbf{B}_T$ by introducing a KL divergence. 
Therefore, given the above ELBO, our objective function consists of three parts: a reconstruction loss (MSE loss), KL divergences for hierarchical latent variables, and a KL divergence for the geometric bases. 
%constraint for matching the two sets of Gaussian basis to ensure the basis obtained from context is as close to the one from the target. 
The empirical objective for the proposed \method{} is formulated as:
\begin{equation}
\begin{aligned}
& \mathcal{L}_{\text{\method{}}}  =  ||y - y'||^2_2 + \alpha \cdot \big(  D_{\text{KL}} [p(\mathbf{z}_o|{\bf{B}}_C)|q(\mathbf{z}_o|{\bf{B}}_T)] \\
    & + D_{\text{KL}}[p(\mathbf{z}_r|\mathbf{z}_o,{\bf{B}}_C)|q(\mathbf{z}_r|\mathbf{z}_o,{\bf{B}}_T)] \big) + \beta \cdot D_{\text{KL}}[{\bf{B}}_C, {\bf{B}}_T],
\end{aligned}
\end{equation}
where $y'$ is the prediction. $\alpha$ and $\beta$ are hyperparameters to balance the three parts of the objective. The KL divergence on ${\bf{B}}_C, {\bf{B}}_T$ is to align the spatial location and the shape of two sets of bases. 

% +++++++++++++

% \noindent 
% \textbf{Neural Fields.}
% % \zx{First introduce NeRF (with problem definition and notations) and its disadvantages of inefficient inference, then say what we will do to avoid this problem by NP?} 


% In general, training a NeRF requires overfitting each scene, which is time-consuming. Hence, how to leverage the observation for generalization on the new scene remains a problem. As each INR is a function of a data sample, the problem can be viewed as estimating the distribution of functions. This motivates us to use the Neural Process (NP) in this problem. 
 

% \noindent {\textbf{Neural Process.}} Neural Process~\cite{garnelo2018neural} parameterizes the distribution of functions. Given the context set comprising of the observation $X$ and the corresponding labels $Y$, $D_C = (X_C, Y_C) := (\mathbf{x}_i, \mathbf{y}_{i})_{i\in C} $. The aim of NP is to learn a mapping function from the target points $X_T$ to the target labels $Y_T$,  $D_T = (X_T, Y_T) := (\mathbf{x}_i, \mathbf{y}_{i})_{i\in T} $. The conditional distribution of target points is:
% \begin{equation}
%     p_{\phi}(Y_T|X_T,D_C) = \prod_{\mathbf{x}, \mathbf{y}\in D_{T}} \mathcal{N}(\mathbf{y};\mu_{\mathbf{y}}(\mathbf{x},D_c), \sigma^2_{\mathbf{y}}(\mathbf{x},D_c)).
% \end{equation}

% \begin{equation}
%     p_{\phi}(Y_T|X_T, \mathbf{z}) = \prod_{\mathbf{x}, \mathbf{y}\in D_{T}} \mathcal{N}(\mathbf{y};\mu_{\mathbf{y}}(\mathbf{z}, X_T,D_c), \sigma^2_{\mathbf{y}}(\mathbf{z}, X_T,D_c)),
% \end{equation}
% where $\mathbf{z} \sim p_{\theta}(\mathbf{z|X_T,D_C})$. Using NP can efficiently leverage the limited context/observation to infer the function of INR for the target from a probabilistic perspective. It also can incorporate uncertainty estimation for the unseen view. This is reasonable as the value in the unseen target location should not be deterministic. 

%\zx{advantages of NP? incorporating uncertainty for limited context information, which is suitable for reconstruction tasks?}







% \begin{align}
%     p(\mathbf{y}|\mathbf{x}, I) & = \int_{g} \int_{r}  p(\mathbf{y}, g, r |\mathbf{x}, I) \mathrm{d}g  \mathrm{d}r \\
%     &=  \int_{g} \int_{r}  p(\mathbf{y}| g, r) p(g, r | \mathbf{x}, I) \mathrm{d}g  \mathrm{d}r \\
%     % &= \int_{g} \int_{r} \int_{B} p_{\theta_1}(\mathbf{y}| g, r) p_{\theta_2}(g, r | B) p_{\theta_3}(B|\mathbf{x}, I) \mathrm{d}g  \mathrm{d}r \mathrm{d}B 
%     &= \int_{g} \int_{r}  p_{\theta_1}(\mathbf{y}| g, r) p_{\theta_2}(g, r | B) p_{\theta_3}(B|\mathbf{x}, I) \mathrm{d}g  \mathrm{d}r 
% \end{align}


%\subsection{Gaussian Basis for INR}
%As shown in Fig.~\ref{fig:framework}, 


%\subsection{Hierarchical Neural Process for INR}









\section{Experiments}
\section{Experiments: Planning outperforms Heuristics}
\label{sec:experiment}

We begin our empirical demonstrations by showcasing the effectiveness of our planning framework on both synthetic and real datasets. We focus on the simplest planning algorithm, 1-step lookaheads (Algorithm~\ref{alg:complete}), and show that even basic planning can hold great promise. 
We illustrate our framework using two uncertainty quantification modules---GPs and 
\ensembles/ \ensembleplus. 

Throughout this section, we focus on evaluating the mean squared error of 
a regression model $\model$,  and develop adaptive policies that minimize uncertainty on $g(f)$ defined in~\eqref{eqn:l2-g-f}.
When GPs provide a valid model of uncertainty, 
our experiments show that our planning framework significantly outperforms other baselines. 
We further demonstrate that our conceptual framework extends to deep learning-based uncertainty quantification methods such as  \ensembleplus while highlighting computational challenges that need to be resolved in order to scale our ideas. 
For simplicity, we assume a naive predictor, i.e., $\psi(\cdot) \equiv 0$. However, we emphasize that this problem is just as complex as if we were using a sophisticated model $\psi(.)$. The performance gap between the algorithms 
primarily depends
on the level  of uncertainty in our prior beliefs.

To evaluate the performance of our algorithm, we benchmark it against several baselines. 
%Active learning baselines use an acquisition function $\ac$ to select points that have the highest   function value: $X\opt_t \in \argmax_{X \in \xpoolj{t}} \ac({X})$ at every step $t$. These methods may also need an UQ module, which we simply use the same UQ module as in our algorithm, and it  outputs $V(X)$ that measures the the uncertainty of each point $X \in \xpoolj{t}$.
Our first set of baselines are from active learning~\citep{AggarwalKoGuHaPh14}:
\\ % \noindent\textbf{Active Learning Heuristics:} 
\textbf{(1)} 
\textsf{Uncertainty Sampling (Static):}  In this approach, we query the samples for which the model is least certain about. Specifically, we estimate the variance of the latent output $f(X)$ for each $X \in \xpool$ using the UQ module and select the top-$K$ points with the highest uncertainty. \\
\textbf{(2)} \textsf{Uncertainty Sampling (Sequential):} This is a greedy heuristic that sequentially selects the points with the highest uncertainty within a batch, while updating the posterior beliefs using pseudo labels from the current posterior state. Unlike \textsf{Uncertainty Sampling (Static)}, this method takes into account the information gained from each point within batch, and hence tries to diversify the selected points within a batch. 

 
We also compare our approach to the  \textbf{(3)} \textsf{Random Sampling}, which selects each batch uniformly at random from the pool. Additionally, we compare solving the planning problem using  \textsf{REINFORCE}-based policy gradients with   $\mathsf{Smoothed\text{-}Autodiff}$ policy gradients.\footnote{Our code repository is available at
  \url{https://github.com/namkoong-lab/adaptive-labeling}.}
%Detailed experimental setups are provided in Section \ref{sec:details-experiments}.

%We repeat all experiments with 10 random seeds.




\begin{figure}[t]
\centering
\begin{minipage}[b]{0.49\textwidth}
\centering
\includegraphics[width=\textwidth, height=5cm]{figures/original_scale/Var_of_l_2_loss.pdf}
\caption{(Synthetic data) Variance of mean squared loss evaluated through the posterior belief $\mu_t$ at each horizon $t$. This is the objective that policy gradient methods like \textsf{REINFORCE} and $\ouralgo$ optimizes. 1-step lookaheads are surprisingly effective even in long horizons.}
\label{fig:var-l2-sim}
\end{minipage}
\hfill
\begin{minipage}[b]{0.49\textwidth}
\centering \includegraphics[width=\textwidth, height=5cm]{figures/original_scale/Error_of_estimated_model_l_2_loss.pdf}
\caption{(Synthetic data) Error between MSE calculated based on collected data $\mc{D}^{0:T}$ vs. population oracle MSE over $\mc{D}_{\rm eval} \sim P_X$. Reducing uncertainty over posteriors directly leads to better OOD evaluations. 1-step lookaheads significantly outperform active learning heuristics in small horizons.}
\label{fig:mean-l2-sim}
\end{minipage}
%\caption{Simulated data for GPs}
%\label{fig:both_plots}
\end{figure}

\subsection{Planning with Gaussian processes}
\label{sec:experiment-plan-GP}
We now briefly describe the data generation process for the GP experiments,  deferring a more detailed discussion of the dataset generation to Section~\ref{sec:details-experiments}. 
We use both the synthetic data and the real data to test our methodology.
For the \emph{simulated data},  we construct a setting where the general population is distributed across \emph{51 non-overlapping clusters} while the initial labeled data $\dtrain$ just comes from one cluster. In contrast, both $\dpool \defeq (\xpool,\ypool),\deval \defeq (\xeval,\yeval)$ are generated   from all the clusters. 
We begin with a low-dimensional scenario, generating a one-dimensional regression setting using a GP. %Gaussian Process (GP).
Although the data-generating process is not known to the algorithms,  we assume that the GP hyperparameters are known to all the algorithms
to ensure fair comparisons. This can be viewed as a setting where our prior is well-specified, allowing us to isolate the effects
of different policy optimization approaches
 without any concerns about the misspecified priors. We select $10$ batches, each of size $K=5$ across $T = 10$ time horizons.

To examine the robustness of our method against the distributional assumptions made  in the simulated case, we then move to a real dataset where the correct prior is not known. We simulate selection bias from the eICU dataset~\citep{PollardJoRaCeMaBa18}, which contains real-world patient data with in-hospital mortality outcomes. 
We conduct a $k$-means clustering to generate 51 clusters and then select data from those clusters. We view this to be a credible replication of practice, as severe distribution shifts are common due to selection bias in clinical labels.  To convert the binary mortality labels into a regression setting, we train a  random forest classifier and fit a GP on predicted scores, which serves as the UQ module for all the algorithms. As before, the task is to select 10 batches, each consisting of 5 samples, across 10 time horizons.

 In Figures~\ref{fig:var-l2-sim} and~\ref{fig:mean-l2-sim}, we present results for the simulated data. 
Figure~\ref{fig:var-l2-sim} shows the variance of $\ell_2$ loss, and Figure~\ref{fig:mean-l2-sim} presents the error in the estimated $\ell_2$ loss using $\mu_t$ (relative to true $\ell_2$ loss, that is unknown to the algorithm). 
As we can see from these plots, our method one-step lookahead  gives substantial improvements  over active learning baselines and random sampling. In addition,
compared to the one-step lookahead planning approach using \textsf{REINFORCE}-based policy gradients, 
we observe that $\mathsf{Smoothed\text{-}Autodiff}$-based policy gradients provide significantly more robust performance over all horizons.

In Figures~\ref{fig:var-l2-real}~and~\ref{fig:mean-l2-real}, we observe similar findings on the eICU data. We see that planning policies (\textsf{REINFORCE} and $\mathsf{Smoothed\text{-}Autodiff}$) consistently outperform other heuristics by a large margin.  Active learning baselines perform poorly in these small-horizon batched problems and can sometimes be even worse than the random search baselines.  Overall, our results show the importance of careful planning in adaptive labeling for reliable model evaluation. 

We offer some intuition as to why one-step lookahead planning may outperform other heuristic algorithms. 
 First,  \textsf{Uncertainty sampling (Static)} while myopically selects the
 top-$K$ inputs with the highest uncertainty, it fails to consider 
the overlap in information content among the ``best” instances; see \citep{AggarwalKoGuHaPh14} for more details. 
In other words,  it might acquire points from the same region with high uncertainty while failing to induce diversity among the batch.
Although \textsf{Uncertainty Sampling (Sequential)} somewhat addresses the issue of information overlap, a significant drawback of 
this algorithm
is the disconnect between the objective we aim to optimize and the algorithm. For example, it might sample from a region with high uncertainty but very low density. 

\begin{figure}[t]
\centering
\begin{minipage}[b]{0.48\textwidth}
\centering
\includegraphics[width=\textwidth, height=5cm]{figures/original_scale/Var_of_l_2_loss_real.pdf}
\caption{(Real-world eICU data) Variance of mean squared loss evaluated through the posterior belief $\mu_t$ at each horizon $t$. Even 1-step lookaheads are extremely effective planners, and auto-differentiation-based pathwise policy gradients provide a reliable optimization algorithm based on low-variance gradient estimates.}
\label{fig:var-l2-real}
\end{minipage}
\hfill
\begin{minipage}[b]{0.48\textwidth}
\centering \includegraphics[width=\textwidth, height=5cm]{figures/original_scale/Error_of_estimated_model_l_2_loss_real.pdf}
\caption{(Real-world eICU data) Error between MSE calculated based on collected data $\mc{D}^{0:T}$ vs. population oracle MSE over $\mc{D}_{\rm eval} \sim P_X$. Reducing uncertainty over posteriors directly leads to better OOD evaluations. Our method significantly outperforms active learning-based heuristics, and random sampling.}
\label{fig:mean-l2-real}
\end{minipage}
%\caption{Real data for GPs}
\end{figure}
 
%\vspace{-1.5cm}
% \begin{wrapfigure}{r}{.32\columnwidth}
%   \vspace{-.5cm} 
%   \centering
% \includegraphics[scale=.29]{figures/Var of l2l_2 loss.pdf}
%   \vspace{-0.2cm}
%   \caption{Results of GP}
% \label{fig:var-l2-gp}
%   \vspace{-0.1cm}
% \end{wrapfigure}


% Attempts have been made  in the past to address these  drawbacks heuristically  (see \citep{AggarwalKoGuHaPh14}). We give a unified computational framework while approaching the problem in a more principled manner and solving it more optimally.




\subsection{Planning with  neural network-based uncertainty quantification methods ($\ensembleplus$)}


We now provide a proof-of-concept that shows the generalizability of our conceptual framework  to the deep learning-based UQ modules, specifically focusing on $\ensembleplus$ due to their previously observed superior performance~\citep{OsbandWenAsDwIbLuRo23}. Recall that implementing our framework with deep learning-based UQ modules  requires us to retrain the model across multiple possible random actions $\bm{a}(\theta)$ sampled from the current policy $\pi_\theta$.
This requires significant computational resources, in sharp contrast to the GPs where the posteriors are in closed form and can be readily updated and differentiated. 

Due to the computational constraints, we test $\ensembleplus$ on a toy setting to demonstrate the generalizability of our framework. We consider a setting where the general population consists of four clusters, while the initial labeled data only comes from one cluster. Again we generate data using GPs.  The task is to select a batch of 2 points in one horizon. We detail the $\ensembleplus$ architecture in Section \ref{sec:details-experiments}, and we assume prior uncertainty to be large (depends on the scaling of the prior generating functions). 
The results are summarized in the Table~\ref{tab:UQ_ensemble}.

% \begin{table}[H]
% \vspace{-10pt}
% \caption{Performance under \ensembleplus as UQ module}
%     \centering
%     \begin{tabular}{|m{3cm}|m{2.5cm}|m{2cm}|} 
%     \hline
%       Algorithm   & Variance of $\loss_2$ loss estimate & Error of $\loss_2$ loss estimate  \\ \hline Random Sampling 
%          & $1710.9 \pm 1352.1$ & $8.67\pm6.62$ 
%       \\ \hline \ouralgo & $1.30 \pm 0.68$ & $0.91\pm0.25$ \\ \hline
%     \end{tabular}
%     \label{tab:UQ_ensemble}
%     %\vspace{-10pt}
% \end{table}




\begin{table}[h]
\vspace{-10pt}
\caption{Performance under \ensembleplus as the UQ module}
\centering
\begin{tabular}{|l|l|l|}
\hline
Algorithm   & Variance of $\loss_2$ loss estimate & Error of $\loss_2$ loss estimate  \\
\hline
\textsf{Random sampling} & 7129.8 $\pm$ 1027.0 & 136.2 $\pm$ 8.28 \\ \hline
\textsf{Uncertainty sampling (Static)} & 10852 $\pm$ 0.0 & 162.156 $\pm$ 0.0 \\ \hline
\textsf{Uncertainty sampling (Sequential)} & 8585.5 $\pm$ 898.9 & 144 $\pm$ 6.93 \\ \hline
\textsf{REINFORCE} & 1697.1 $\pm$ 0.0 & 45.27 $\pm$ 0.0 \\ \hline
\ouralgo & 1697.1 $\pm$ 0.0 & 45.27 $\pm$ 0.0 \\ \hline
\end{tabular}
%\caption{Comparison of different algorithms based on variance   and   error in $\ell_2$ loss estimation with Ensemble $+$ as the UQ module. Our results demonstrate that {\ouralgo} and REINFORCE outperformthe other active learning based heuristics, confirming the benefits of our MDP formulation for the adaptive labeling problem, as also demonstrated in Section 4.\\
%\footnotesize{Experimental details: We use Gaussian Processes as our data generating process, GP parameters are the same as in Section D.3.  The task is to select a batch of 2 points along one horizon.The marginal distribution $p_X$ has 4 \textit{non-overlapping} clusters. Initial data comes from one cluster, while pool and evaluation points comes from all the clusters. We have $20$ initial labeled data points, $10$ pool points, and $252$ evaluation points.  Training procedures are similar to the one in Section D.3.} }
\label{tab:UQ_ensemble}
\end{table}



% We faced  issues in scaling up these experiments which will be our focus in the future. 





% \begin{itemize}
%     \item Posteriors should be consistent. Two dimensions: even with less training,  
%     \item the inference should be  fast enough
% \end{itemize}


% Potential research directions for uncertainty quantification

% In this section we consider a simple setting We consider a simpler setting and 


% For synthetic dataset generation, we use ...... For real datasets, we use ...... We compare our methodolgy to several baselines ()    This Section is structured as follows:
% \begin{itemize}
%     \item \textbf{GPs, square loss objective} (Section \ref{}): 
%     %the broad aim of the experiments  in this section is to isolate the performance of our methodology without any concerns for the inefficiencies induced due to a mis-specified prior or imperfect posterior inference. To accomplish this we generate synthetic datasets using GPs (detailed later). We use the well specified prior (GPs - with same hyperparameter setting) as our UQ module.   
%      As GPs provide differentaible posterior inference - any errors induced due to imperfect posterior updates are also isolated. We note that under this setting
%      \item In Section\ref{} we demonstrate why our methodology performs better than other baselines - by devising various synthetic experiments ()
%     \item  \textbf{UQ Benchmarking }(Section \ref{}): Before diving into the experiments using $\ensembleplus$ and ENNs,  we showcase our benchmarking experiments in Section \ref{}. We use real datasets We observe that ENNs perform better
%      \item \textbf{Ensemble $+$}, objective: recall, accuracy
%     \item \textbf{ENN}, objective: recall, accuracy
% \end{itemize}




% In Section {}, we test 
% \subsection{Experimental details}

% \begin{itemize}
%     \item UQ methodologies - GPs, ENNs
%     \item Objectives - Recall,  ATE
%     \item Datasets - ATE-synthetic datasets, Recall-synthetic, real datasets
%     \item Baselines - 
%     \begin{itemize}
%         \item Random sampling
%         \item Active learning - Uncertainty based sampling - In regression setting almost all of the 
%         \item Myopic greedy - Greedy Batch based sampling
%         \item Policy Gradient
%     \end{itemize}
    
% \end{itemize}

% \subsection{Experiments}
%     \begin{itemize}
%     \item GPs with square loss
%     \item Benchmarking ENN
%         \item ENNs with ATE
%         \item ENNs with Recall
%     \end{itemize}

% \subsection{Benefits over other algorithms - intuition and experiments}

%Active learning - Myopic greedy / Don't rely on the objective rather some entropy version.


%%% Local Variables:
%%% mode: latex
%%% TeX-master: "main"
%%% End:


% \section{Experiments}
% 
% We evaluate the proposed \method{} on different tasks.
% In Sec.~\ref{sec:nerf-results}, we demonstrate the performance on novel view synthesis for 3D scenes. 
% Further, we show the method can also perform competitively on the 2D image regression task, as well as image completion in Sec.~\ref{sec:image-regression}, although it is originally designed for 3D view synthesis.
% Last, we perform ablation studies in Sec.~\ref{sec:abl-study}. 

\noindent{\textbf{Baselines.}}
%\str{Explain why meta-learning is the type of baseline we want here.}
We compare \method{} with three recent probabilistic INR generalization methods: NeRF-VAE~\citep{kosiorek2021nerf}, PONP~\citep{gu2023generalizable} and VNP~\citep{guo2023versatile} on ShapeNet novel view synthesis and image regression tasks.
PONP~\citep{gu2023generalizable} and VNP~\citep{guo2023versatile} also rely on Neural Processes, however, they neglect structure information and the probabilistic interaction between 3D functions and 2D partial observations. Additionally, we choose two previous well-known deterministic INR generalization approaches, LearnInit~\citep{tancik2021learned} and  TransINR~\citep{chen2022transformers} as our baselines. Moreover, to demonstrate the flexibility of our method and its ability to handle complex scenes, we integrate \method{} with GNT~\citep{wang2022attention} and conduct experiments on the NeRF Synthetic dataset~\citep{mildenhall2021nerf}. \textcolor{blue}{In addition, we demonstrate our method is also effective for 1-D signals in the Appendix.}


% We choose two previous well-known deterministic INR generalization approaches, LearnInit~\citep{tancik2021learned} and  TransINR~\citep{chen2022transformers} as our baselines. We use the self-attention module similar to TransINR to extract features from the context set.
% Additionally, we compare with three recent probabilistic INR generalization methods: NeRF-VAE~\citep{kosiorek2021nerf}, PONP~\citep{gu2023generalizable} and VNP~\citep{guo2023versatile}.
% PONP~\citep{gu2023generalizable} and VNP~\citep{guo2023versatile} also rely on Neural Processes, however, they neglect structure information and the probabilstic interaction between 3D functions and 2D partial observations.


\subsection{Novel View Synthesis}
\label{sec:nerf-results}

% \noindent{\textbf{Setup.}} We perform the 3D novel view synthesis task on ShapeNet~\citep{chang2015shapenet} objects. Following previous works' setup, the dataset consists of objects from three ShapeNet categories: chairs, cars, and lamps. For each 3D object, 25 views of size $128 \times 128$ images are generated from viewpoints randomly selected on a sphere. The objects in each category are divided into training and testing sets, with each training object consisting of 25 views with known camera poses. During testing, a random input view is sampled to evaluate the performance of the novel view synthesis. Following the setting from previous methods, we focus on the single-view (1 shot) and 2-views (2 shot) versions of the task, where single-view or two-view images with their corresponding camera rays are provided as the context set.
\noindent{\textbf{ShapeNet.}} We perform the 3D novel view synthesis task on ShapeNet~\citep{chang2015shapenet} objects. Following previous works' setup~\citep{tancik2021learned}, the dataset consists of objects from three ShapeNet categories: chairs, cars, and lamps. For each 3D object, 25 views of size \(128 \times 128\) images are generated from viewpoints randomly selected on a sphere. The objects in each category are divided into training and testing sets, with each training object consisting of 25 views with known camera poses. At test time, a random input view is sampled to evaluate the performance of the novel view synthesis. Following the setting of previous methods~\citep{chen2022transformers}, we focus on the single-view (1-shot) and 2-view (2-shot) versions of the task, where one or two images with their corresponding camera rays are provided as the context.


% \noindent{\textbf{Implementation Details.}} 
% %We use the visual tokenizer and self-attention modules as TransINR~\citep{chen2022transformers} and VNP~\citep{guo2023versatile}. 
% Our context input is the concatenation of a set of camera rays and the corresponding image pixels from a few views (one or two),
% %the context image and the corresponding ray maps %\str{What is a ray map?}, 
% which are then split into different visual tokens. We use the same patch size $8\times8$ as TransINR~\citep{chen2022transformers} and VNP~\citep{guo2023versatile}, which gives us 256 tokens. 
% Then, a linear layer and a self-attention module are used to project each token into a 512-dimension vector. Based on the 256 tokens, we predict 256 Gaussian bases by two MLP modules. Specifically, one MLP with 2 linear layers is used to map the tokens into a 10-dimension vector, which includes 3-dimension Gaussian centers a 3-dimension vector for constructing the scaling matrix, and a 4-dimension vector for quaternions parameters of the rotation matrix. Both the scaling matrix and rotation matrix are used to build the $3\times 3$ covariance matrix. This procedure is similar to Gaussian construction in the 3D Gassian Splatting~\citep{kerbl20233d}. Details are provided in the Appendix. Another MLP is used to estimate the latent representation of each Gaussian basis. Here, we simply use 32-dimension for each Gaussian basis. 
% Note that using more views as the context set will increase the number of Gaussian basis accordingly, which enriches the ability of describing the geometry of the space. 


\begin{table*}[t]
    \centering
    \caption{\textbf{Qualitative comparison (PSNR) on novel view synthesis of ShapeNet objects.} \method{} consistently outperforms baselines across all categories with both 1-view and 2-view context.}
    \begin{tabular}{lccccc}
        \toprule
        Method & Views & Car & Lamps & Chairs & Average \\
        \midrule
        Learn Init~\citep{tancik2021learned}  & 25 & 22.80 & 22.35 & 18.85 & 21.33 \\
        \midrule
        Tran-INR~\citep{chen2022transformers}  & 1 & 23.78 & 22.76 & 19.66 & 22.07 \\
        NeRF-VAE~\citep{kosiorek2021nerf}  & 1 & 21.79 & 21.58 & 17.15 & 20.17 \\
        PONP~\citep{gu2023generalizable}  & 1 & 24.17 & 22.78 & 19.48 & 22.14 \\
        VNP~\citep{guo2023versatile}  & 1 & 24.21 & 24.10 & 19.54 & 22.62 \\
        \rowcolor{lightblue}
        \textbf{\method{}} (Ours) & 1 & \textbf{25.13} & \textbf{24.59} & \textbf{20.74} & \textbf{23.49} \\
        \midrule
        Tran-INR~\citep{chen2022transformers}  & 2 & 25.45 & 23.11 & 21.13 & 23.27 \\
        PONP~\citep{gu2023generalizable}  & 2 & 25.98 & 23.28 & 19.48 & 22.91 \\
        \rowcolor{lightblue}
        \textbf{\method{}} (Ours) & 2 & \textbf{26.39} & \textbf{25.32} & \textbf{22.68} & \textbf{24.80} \\
        \bottomrule
    \end{tabular}
    \label{tab:nerf-psnr}
    \vspace{-2mm}
\end{table*}

\begin{figure*}[t]
  \centering
  \includegraphics[width=1\textwidth]{Figures/nerf-results-new.pdf} % Adjust the size and filename as needed
  \vspace{-6mm}  \caption{\textbf{Qualitative results of the proposed \method{} on novel view synthesis of ShapeNet objects.} Both 1-view (top) and 2-view (bottom) context results are presented.} % Caption for the figure
  \label{fig:nerf-visualization}
  \vspace{-3mm}
\end{figure*}

\noindent{\textbf{Implementation Details.}} Our context input is the concatenation of a set of camera rays and the corresponding image pixels from one or two views, which are then split into different visual tokens. We use the same patch size \(8 \times 8\) as TransINR~\citep{chen2022transformers} and VNP~\citep{guo2023versatile}, resulting in 256 tokens. A linear layer and a self-attention module project each token into a 512-dimensional vector. Based on the 256 tokens, we predict 256 geometric bases using two MLP modules: one for 3D Gaussian distribution parameters and the other for the latent representation (32 dimensions).
% \wy{Can be removed to appendix: Specifically, one MLP with 2 linear layers maps the tokens into a 10-dimensional vector, which includes 3-dimensional Gaussian centers, a 3-dimensional vector for constructing the scaling matrix, and a 4-dimensional vector for quaternion parameters of the rotation matrix. Both the scaling matrix and rotation matrix are used to build the \(3 \times 3\) covariance matrix. This procedure is similar to Gaussian construction in the 3D Gaussian Splatting~\citep{kerbl20233d}.} 
%We use one MLP to predict Gaussian basis parameters, which is detailed in the Appendix. Another MLP estimates the latent representation of each Gaussian basis, using a 32-dimensional vector for each Gaussian basis. 
More details are given in Appendix~\ref{supp:gaussian}. 
%Using more views as the context set increases the number of geometric base accordingly, enriching the ability to describe the geometry of the space.
% To have a fair comparison with baseline methods, we report the results of using $8\times 8$ patch size. \str{What do you mean here? Other baselines also use 8x8 patches? If yes, say that.}
We obtain the object-specific and ray-specific modulating vectors (both are 512 dimensions) based on the geometric base. Our NeRF function consists of four layers, including two modulated layers and two shared layers. %\str{This can also be better written. The terminology is not consisten, what is the global and local-raywise modulate vectors? We not really use that words before.}

\noindent{\textbf{Quantitative Results.}} The quantitative comparison in terms of Peak Signal-to-Noise Ratio (PSNR) is presented in Table~\ref{tab:nerf-psnr}. The proposed \method{} consistently outperforms all other baselines across all three categories by a significant margin. On average, \method{} exceeds the previous NP-based method, VNP~\citep{guo2023versatile}, by 0.87 PSNR, %(23.49 vs. 22.62), 
indicating that the proposed geometric bases and probabilistic hierarchical modulation result in better generalization ability. Moreover, with two views of context information, \method{}'s performance improves significantly by around $1$ PSNR. This improvement is expected, as the richer geometric bases information allows for a better representation of the 3D space, leading to improved object-specific and ray-specific latent variables. % to modulate the MLP.




{\noindent {\textbf{Qualitative Results.}}
In Fig.~\ref{fig:nerf-visualization}, we visualize the results of \method{} on novel view synthesis of ShapeNet objects.
\method{} can infer object-specific radiance fields and render high-quality 2D images of the objects from novel camera views, even with only 1 or 2 views as context. More results and comparisons with other VNP are provided in Appendix~\ref{supp:more-results}.}


\textcolor{blue}{\noindent \textbf{Comparison with More SOTAs on NeRF Synthetic.} To illustrate the flexibility and advantages of our proposed method, we evaluate it on the NeRF Synthetic dataset~\citep{mildenhall2021nerf} against recent state-of-the-art methods, including GNT~\citep{wang2022attention}, MatchNeRF~\citep{chen2023explicit}, and GeFu~\citep{liu2024geometry}. For a fair comparison, we use the same encoder and NeRF network architecture while integrating our probabilistic framework into GNT. Following GeFu, we assess performance in 2-view and 3-view settings. As shown in Table~\ref{tab:comparison-gnt}, our method surpasses GeFu by approximately 1 PSNR in the 3-view setting, validating the effectiveness of our probabilistic framework and geometric bases.}

\textcolor{blue}{We also consider a challenging 1-view setting to examine the model’s robustness under extremely limited context. Both Table~\ref{tab:comparison-gnt} and Figure~\ref{fig:1-view-compare} indicate that our method can still reconstruct novel views effectively, whereas GNT fails to do so. Furthermore, we test cross-category generalization for our model and GNT without retraining, training on the \texttt{drums} category and evaluating on \texttt{lego}. As shown in Figure~\ref{fig:cross-category}, our method leverages the available context information more effectively, producing higher-quality generations with superior color fidelity compared to GNT. Additional details are provided in Appendix~\ref{sec:compare_gnt}.}


\begin{table}[htbp]
\centering
\resizebox{\columnwidth}{!}{ % Resize the table to fit within the column width
\begin{tabular}{lcccc}
\toprule
\textbf{Models} & \textbf{\# Views} & \textbf{PSNR ($\uparrow$)} & \textbf{SSIM ($\uparrow$)} &  \textbf{LPIPS ($\downarrow$)}\\ 
\midrule
GNT  & 1 & 10.25 & 0.583 & 0.496\\ 
GeomNP & 1 & \textbf{20.07} & \textbf{0.815} & \textbf{0.208} \\ 
\midrule
GNT & 2 & 23.47 &0.877 &0.151 \\ 
MatchNeRF & 2 &20.57 & 0.864 &0.200  \\ 
GeFu & 2 & 25.30 &0.939 &0.082  \\ 
GeomNP & 2 &  &  &  \\ 
\midrule
GNT & 3 &  25.80 & 0.905 & 0.104 \\ 
MatchNeRF & 3 & 23.20 & 0.897 & 0.164  \\ 
GeFu & 3 &  26.99 & \textbf{0.952} & 0.070  \\ 
GeomNP & 3 & \textbf{27.85} & 0.948 & \textbf{0.068} \\ 
\bottomrule
\end{tabular}
}
\caption{Performance comparison of different models with varying numbers of context views.}
\label{tab:comparison-gnt}
\end{table}





\begin{table}[htbp]
    \centering
    \begin{tabular}{lcc}
        \toprule
                 & CelebA & Imagenette \\ \midrule
        Learned Init \citep{tancik2021learned} & 30.37  & 27.07       \\
        TransINR~\citep{chen2022transformers}  & 31.96  & 29.01       \\ 
        \rowcolor{lightblue}
        \textbf{\method{} (Ours)}         & \textbf{33.41}  & \textbf{29.82}      \\ 
        \bottomrule
    \end{tabular}
    \caption{Quantitative results. \method{} outperforms baseline methods consistently on both datasets.}
    \label{tab:image-regression}
\end{table}


\begin{figure}[htbp]
    \centering
    \begin{minipage}[b]{0.49\textwidth} 
        \includegraphics[width=\textwidth]{Figures/image-regression0.pdf} % Adjust filename as needed
        \caption{\textbf{Visualizations of image regression results} on CelebA (left) and Imagenette (right).}
        \label{fig:visualization-image-regression}
    \end{minipage}
\end{figure}


% \begin{figure*}[htbp]
%     \centering
%     \begin{subtable}[b]{151.2pt} 
%     % \vspace{-6mm}
%     \begin{tabular}{lcc}
%     \toprule
%                  & CelebA & Imagenette \\ \midrule
%     Learned Init \citep{tancik2021learned} & 30.37  & 27.07       \\
%     TransINR~\citep{chen2022transformers}         & 31.96  & 29.01       \\ 
%     % \hline
%     \rowcolor{lightblue}
%     \textbf{\method{}} (Ours)         & \textbf{33.41}  & \textbf{29.82}      \\ 
%     \bottomrule
%     \end{tabular}
%     \caption{Quantitative results. \method{} outperforms baseline methods consistently on both datasets.}
%     \label{tab:image-regression}
%     \end{subtable} \hfill
%     \begin{subtable}[b]{151.2pt} 
%     \includegraphics[width=\textwidth]{Figures/image-regression0.pdf} % Adjust the size and filename as needed
%     \caption{Visualizations on CelebA (left) and Imagenette (right), respectively.} % Caption for the figure
%     \label{fig:visualization-image-regression}
% \end{subtable}
% % \vspace{-2mm}
% \caption{\textbf{Quantitative results and visualizations} of image regression on CelebA and Imagenette.}
% \vspace{-3mm}
% \end{figure*}





% \begin{figure}[t]
%     \centering
%     \begin{subtable}[b]{0.42\textwidth}
%     % \vspace{-6mm}
%     \begin{tabular}{lcc}
%     \toprule
%                  & CelebA & Imagenette \\ \midrule
%     Learned Init \citep{tancik2021learned} & 30.37  & 27.07       \\
%     TransINR~\citep{chen2022transformers}         & 31.96  & 29.01       \\ 
%     % \hline
%     \rowcolor{lightblue}
%     \method{} (Ours)         & \textbf{33.41}  & \textbf{29.82}      \\ 
%     \bottomrule
%     \end{tabular}
%     \caption{Quantitative results. \method{} outperforms baseline methods consistently on both datasets.}
%     \label{tab:image-regression}
%     \end{subtable} \hfill
%     \begin{subtable}[b]{0.54\textwidth}
%     \includegraphics[width=\textwidth]{Figures/image-regression0.pdf} % Adjust the size and filename as needed
%     \caption{Visualizations on CelebA (left) and Imagenette (right), respectively.} % Caption for the figure
%     \label{fig:visualization-image-regression}
% \end{subtable}
% % \vspace{-2mm}
% \caption{\textbf{Quantitative results and visualizations} of image regression on CelebA and Imagenette.}
% \vspace{-3mm}
% \end{figure}



\subsection{Image Regression}
\label{sec:image-regression}


\noindent{\textbf{Setup.}} Our method is flexible to different signals and can also be seamlessly applied to 2D signals. Here, we evaluate our method on the image regression task, a common task for evaluating INRs' capacity of representing a signal~\citep{tancik2021learned,sitzmann2020implicit}. 
We employ two real-world image datasets as used in previous works~\citep{chen2022transformers,tancik2021learned,gu2023generalizable}. The CelebA dataset~\citep{liu2015deep} encompasses approximately 202,000 images of celebrities, partitioned into training (162,000 images), validation (20,000 images), and test (20,000 images) sets. The Imagenette dataset~\citep{imagenette}, a curated subset comprising 10 classes from the 1,000 classes in ImageNet~\citep{deng2009imagenet}, consists of roughly 9,000 training images and 4,000 testing images. In order to compare with previous methods, we conduct image regression experiments. The context set is an image and the task is to learn an implicit function that regresses the image pixels well. % in terms of PSNR.
%\str{What is the point of image regresssion when the image itself is used as context? Why not just copy the context??}
%\str{The following is a bit strante, is this still image regression? Why not have a separate section?}
%\str{Is Image Regression the standard name for this task?}


\noindent{\textbf{Implementation Details.}} 
Following TransINR~\citep{chen2022transformers}, we resize each image into $178\times 178$, and use patch size 9 for the tokenizer. The self-attention module remains the same as the one in the NeRF experiments (Sec. \ref{sec:nerf-results}). For the Gaussian bases, we predict the 2D Gaussians instead of the 3D. 
The hierarchical latent variables are inferred in image-level and pixel-level. 
%The self-attention and global variable remain the same as the one in the NeRF experiments. %We do not use the pixel variable modulation for the computation concern. However, our method still has competitive performance.   

% \begin{table}[t]
% \centering
% \vspace{-6mm}
% \caption{Quantitative results of image regression.}
% \label{tab:image-regression}
% \begin{tabular}{lcc}
% \toprule
%              & CelebA & Imagenette \\ \midrule
% Learned Init \citep{tancik2021learned} & 30.37  & 27.07       \\
% TransINR~\citep{chen2022transformers}         & 31.96  & 29.01       \\ 
% \hline
% \method{} (Ours)         & \textbf{33.41}  & \textbf{29.82}      \\ 
% \bottomrule
% \end{tabular}
% \end{table}



\noindent{\textbf{Results.}} The quantitative comparison of \method{} for representing the 2D image signals is presented in Table~\ref{tab:image-regression}. \method{} outperforms the baseline methods on both CelebA and Imagenette datasets significantly, showing better generalization ability and representation capacity than baselines. 
%Note that the Imagenette is a more diverse dataset than the CelebA. The better performance shows that. 
Fig.~\ref{fig:visualization-image-regression} shows the ability of \method{} to recover the high-frequency details for image regression.
%regress the image closely to the ground truth, indicating the capability of capturing the detailed texture information. 





%, implying good capability of .



% \begin{figure}[t]
%   \centering
%   \includegraphics[width=0.9\textwidth]{Figures/image-regression-vis.pdf} % Adjust the size and filename as needed
%   \caption{Image regression visualization on CelebA and Imagenette, respectively. Left: Ground truth; Right: Prediction.} % Caption for the figure
%   \label{fig:visualization-image-regression}
% \end{figure}



%\noindent {\textbf{Ablation study}}
\subsection{Ablations}
\label{sec:abl-study}

%\noindent

\begin{table}[htbp]
% \centering
\vspace{-4mm}
\caption{\textbf{Sensitivity to the number of geometric bases} on NeRF and image regression.}
\label{tab: ablate_basis}
\resizebox{0.5\textwidth}{!}{
\begin{tabular}{lccccc}
\toprule
 & \multicolumn{3}{c}{\textbf{Image Regression}} & \multicolumn{2}{c}{\textbf{NeRF}} \\
 \cmidrule(lr){2-4} \cmidrule(lr){5-6}
\# Bases & 49 & 169 & 484 & 100 & 250 \\ \midrule
PSNR~($\uparrow$)    & 28.59  & 33.74   &  44.24  &  24.31 &  24.59  \\ \bottomrule
\label{table:num-gaussian}
\end{tabular}
}
\vspace{-6mm}
\end{table}
% \vspace{-2mm}
\textbf{Sensitivity to Number of Geometric Bases}.
We further analyze the sensitivity to the number of geometric bases in the CelebA image regression and Lamps NeRF tasks. 
We further analyze the sensitivity to the number of geometric bases in the CelebA image regression and Lamps NeRF tasks. 
In image regression, we resize the images to \(64 \times 64\) and use different patch sizes to construct 49, 169, and 484 bases. In the NeRF task, we keep the same setup as in Sec. \ref{sec:nerf-results} and construct 100, 250 bases. The results are provided in Table~\ref{table:num-gaussian}. With more bases, \method{} achieves better performance consistently, indicating that large numbers of geometric Gaussian bases further enrich the structure information and lead to stronger predictive functions. 
%However, beyond a certain point, specifically 169 in this experiment, the performance gains diminish. 
We choose the number of bases by balancing the performance and computational costs.





\noindent{\textbf{Importance of Hierarchical Latent Variables.}}
To demonstrate the effectiveness of the hierarchical nature of \method{} with object-specific and ray-specific latent variables for modulation, we performed an ablation study on a subset of the Lamps dataset for fast evaluation. As shown in the last four rows in Table~\ref{table:abl-np}, either object-specific or ray-specific latent variable improves the performance of neural processes, indicating the effectiveness of the specific function modulation. With both ${\bf{z}}_o$ and ${\bf{z}}_r$, the method performs best, demonstrating the importance of the hierarchical modulation by latent variables. 
In addition, the hierarchical modulation also performs well without the geometric bases.
% proposed NPs effectively provides both object-specific and ray-specific functions to capture detailed textures.
%Additionally, without all hierarchical latent variables, \method{} still remains a relatively good performance, which implies the outstanding capacity of the Gaussian basis. 

%drops by $0.5$ PSNR.
%The reason is that the same MLP is used for different scenes and novel views, while with the hierarchical Neural Process \str{TODO}.


% \begin{table}[htbp]
% \centering
% \caption{Ablation Study on a subset of the Lamps scene synthesis (PSNR). }
% \label{tab:my_label}
% \begin{tabular}{lc}
% \toprule
% %Basis & $\mathbf{z}_o$ (Object) & $\mathbf{z}_r$ (Ray) & PSNR \\
% Basis & PSNR \\ 
% %\midrule
% \midrule
% %\cmark & \xmark & \xmark &  25.98\\ 
% \xmark  & 23.06 \\ 
% %\cmark & \xmark & \cmark & 26.29\\ 
% %\cmark & \cmark & \xmark & 26.24\\ 
% \cmark  & 26.48\\ 
% \bottomrule
% \label{table:abl-basis}
% \end{tabular}
% \end{table}



\begin{table}[htbp]
\centering
\vspace{-4mm}
\caption{\textbf{Importance of geometric bases and hierarchical latent variables} on a subset of the Lamps scene synthesis (PSNR). $\mathbf{z}_o$ and $\mathbf{z}_r$ are object-specific variable and ray-specific variable, respectively.  \ymark~ and \xmark~denote whether the component joins the pipeline or not.}
%\vspace{-2mm}
%\label{tab:my_label}
\resizebox{0.25\textwidth}{!}{
\begin{tabular}{lccc}
\toprule
 $\mathbf{B}_C$ & $\mathbf{z}_o$& $\mathbf{z}_r$ & PSNR ($\uparrow$) \\ 
\midrule
\xmark & \cmark & \cmark & 23.06 \\
\cmark & \xmark & \xmark & 25.98 \\
\cmark & \cmark & \xmark & 26.24\\ 
\cmark & \xmark & \cmark & 26.29\\ 
\cmark & \cmark & \cmark & \textbf{26.48}\\ 
% \cmark & \cmark & \xmark & \\ 
% \cmark & \cmark & \cmark & 24.59\\ 
\bottomrule
\label{table:abl-np}
\end{tabular}
}
\vspace{-8mm}
\end{table}

% \noindent 
\noindent{\textbf{Importance of Geometric Bases.}}
%\str{The table seems a bit redundant, similar numbers are in the previous table. Maybe keep the new numbers only? Or merge somehow the tables?}
We also explore the effectiveness of the proposed geometric bases. As shown in Table~\ref{table:abl-np} (rows 1 and 5), with the geometric bases, \method{} performs clearly better. This indicates the importance of the 3D structure information modeled in the geometric bases, which provide specific inferences of the INR function in different spatial levels. Moreover, the bases perform well without hierarchical latent variables, demonstrating their ability to construct 3D information and reduce misalignment between 2D and 3D spaces.


% provide  effectively encodes modulates inputs, making them context-specific. Consequently, even with a fully shared MLP, the model can produce reliable estimations.

% \wy{Need to figure out how to structure two tabless}
%The reason is that the input is re-represented by the Gaussian basis, which indicates the effectiveness of our Gaussian basis. Ho
%\str{What do you mean here by re-represented?}
%For the experiment without the Gaussian basis, we simply use the coordinate embedding with Fourier features and an MLP to obtain the representation of each location. \str{What are coordinate embeddings? I think you must again make sure the wording, terminology and styling is consistent.}
%Then, the global and location variables are obtained by aggregating this representation. The experiment shows that the performance significantly drops without the help of the Gaussian basis. 

%\noindent {\textbf{Ablation study on NP hierarchical maybe? }}

% \noindent 


% 
% the performance is close to the best one. 

% \begin{table}[h]
% \centering
% \caption{Ablation study of the number of Gaussians basis on image regression.}
% \label{your-label}
% \begin{tabular}{lccc}
% \toprule
% \# Basis & 49 & 169 & 484 \\ \midrule
% -     & 28.59  & 33.74   &  44.24  \\ \bottomrule
% \label{table:num-gaussian}
% \end{tabular}
% \end{table}



% \begin{table}[h]
% \centering
% \caption{Ablation study of the number of Gaussians basis on NeRF.}
% \label{your-label}
% \begin{tabular}{lccc}
% \toprule
% \# Basis & 100 & 250 & 441 \\ \midrule
% -     &   & 24.59   &    \\ \bottomrule
% \label{table:num-gaussian-3d}
% \end{tabular}
% \end{table}

% \noindent{\textbf{Uncertainty Visualization}.} 
% As a probabilistic framework, our method can provide uncertainty estimation. To obtain the uncertainty map, we sample ten times from the predicted prior distribution to generate corresponding images and then use the variance map to represent the uncertainty. As shown in Fig.~\ref{fig:uncertainty-visual}, high uncertainty is concentrated around the edges, which is expected, as capturing detailed, sharp changes at the edges is more challenging for the model.

% \begin{figure}[t!]
%   \centering
%   \includegraphics[width=0.99\textwidth]{ICLR2025/Figures/uncertainty.pdf} % Adjust the size and filename as needed
%   \vspace{-3mm}
%   \caption{\textbf{Uncertainty Map of the predictions.} Edges of objects have higher uncertainty since it is more challenging for the model to capture the detailed, sharp changes at the edges.}
%   \label{fig:uncertainty-visual}
%   \vspace{-3mm}
% \end{figure}



% \begin{figure}[t]
%     \centering
    % \begin{minipage}[b]{0.45\textwidth}
    %     \centering
    %     \includegraphics[width=\textwidth]{Figures/image_with_heatmap_2.png}
    %     %\caption{Figure 1}
    %     \label{fig:figure1}
    % \end{minipage}
    % \hfill
    % \begin{minipage}[b]{0.45\textwidth}
    %     \centering
    %     \includegraphics[width=\textwidth]{Figures/image_with_heatmap_9.png}
    %     %\caption{Figure 2}
    %     \label{fig:figure2}
    % \end{minipage}
    % \vskip\baselineskip
%     \begin{minipage}[b]{0.45\textwidth}
%         \centering
%         \includegraphics[width=\textwidth]{Figures/image_with_heatmap_14.png}
%         %\caption{Figure 3}
%         \label{fig:figure3}
%     \end{minipage}
%     \hfill
%     \begin{minipage}[b]{0.45\textwidth}
%         \centering
%         \includegraphics[width=\textwidth]{Figures/image_with_heatmap_16}
%         %\caption{Figure 4}
%         \label{fig:figure4}
%     \end{minipage}
%     \caption{\textbf{Uncertainty Map of the predictions.} Edges of objects have higher uncertainty since it is more challenging for the model to capture the detailed, sharp changes at the edges.}
%     \label{fig:uncertainty-visual}
%     \vspace{-4mm}
% \end{figure}



\section{Related Work}



% \noindent {\textbf{Neural Fields.}} 
% 3d shape (\cite{chen2019learning}, \cite{park2019deepsdf}, \cite{mescheder2019occupancy})
% 3D scene reconstruction (NeRF\cite{mildenhall2021nerf}, ; \cite{niemeyer2021giraffe}),

% Neural Fields (NeFs) map coordinates to signals, providing a compact and flexible continuous data representation~\citep{sitzmann2020implicit, tancik2020fourier}. They are widely used for 3D object and scene modeling~\citep{chen2019learning, park2019deepsdf, mescheder2019occupancy, genova2020local, niemeyer2021giraffe}. NeRF~\citep{mildenhall2021nerf} learns neural radiance fields for view synthesis, mapping spatial coordinates to colors and densities via differentiable volumetric rendering. Extensions include Mip-NeRF~\citep{barron2021mip} for multiscale representations, TensoRF~\citep{chen2022tensorf} for low-rank tensor factorization, and NeuRBF~\citep{chen2023neurbf} for radial basis function aggregation. Unlike these methods, which rely on pre-defined structured information, we infer geometric bases to encode spatial structure.



%RBF-based works (\cite{ramasinghe2021learning}, \cite{ramasinghe2022beyond}, NeuRBF~\cite{chen2023neurbf}, 


\noindent {\textbf{Neural Fields (NeFs) and Generalization.}} Neural Fields (NeFs) map coordinates to signals, providing a compact and flexible continuous data representation~\citep{sitzmann2020implicit, tancik2020fourier}. They are widely used for 3D object and scene modeling~\citep{chen2019learning, park2019deepsdf, mescheder2019occupancy, genova2020local, niemeyer2021giraffe}. However, how to generalize to new scenes without retraining remains a problem. 
Many previous methods attempt to use meta-learning to achieve NeF generalization. Specifically, gradient-based meta-learning algorithms such as Model-Agnostic Meta Learning (MAML)~\citep{finn2017model} and Reptile~\citep{nichol2018first} have been used to adapt NeFs to unseen data samples in a few gradient steps~\citep{lee2021meta, sitzmann2020metasdf, tancik2021learned}. Another line of work uses HyperNet~\citep{Ha2016HyperNetworks} to predict modulation vectors for each data instance, scaling and shifting the activations in all layers of the shared MLP~\citep{mehta2021modulated, dupont2022data, dupont2022coin++}. Some methods use HyperNet to predict the weight matrix of NeF functions~\citep{dupont2021generative, zhang20233dshape2vecset}. Transformers~\citep{vaswani2017attention} have also been used as hypernetworks to predict column vectors in the weight matrix of MLP layers~\citep{chen2022transformers, dupont2022coin++}. In addition, \cite{reizenstein2021common,wang2022attention} use transformers specifically for NeRF. Such methods are deterministic and do not consider the uncertainty of a scene when only partially observed. Other approaches model NeRF from a probabilistic perspective~\citep{kosiorek2021nerf, hoffman2023probnerf, dupont2021generative, moreno2023laser,erkocc2023hyperdiffusion}. For instance, NeRF-VAE~\citep{kosiorek2021nerf} learns a distribution over radiance fields using latent scene representations based on VAE~\citep{kingma2013auto} with amortized inference. Normalizing flow~\citep{winkler2019learning} has also been used with variational inference to quantify uncertainty in NeRF representations~\citep{shen2022conditional, wei2023fg}. However, these methods do not consider potential structural information, such as the geometric characteristics of signals, which our approach explicitly models.



\noindent {\textbf{Neural Processes.}} Neural Processes (NPs)~\citep{garnelo2018neural} is a meta-learning framework that characterizes distributions over functions, enabling probabilistic inference, rapid adaptation to novel observations, and the capability to estimate uncertainties. This framework is divided into two classes of research. The first one concentrates on the marginal distribution of latent variables~\citep{garnelo2018neural}, whereas the second targets the conditional distributions of functions given a set of observations~\citep{garnelo2018conditional, gordon2019convolutional}. Typically, MLP is employed in Neural Processes methods. To improve this, Attentive Neural Processes (ANP)~\citep{kim2019attentive} integrate the attention mechanism to improve the representation of individual context points. Similarly, Transformer Neural Processes (TNP)~\citep{nguyen2022transformer} view each context point as a token and utilize transformer architecture to effectively approximate functions.
Additionally, the Versatile Neural Process (VNP)~\citep{guo2023versatile} employs attentive neural processes for neural field generalization but does not consider the information misalignment between the 2D context set and the 3D target points. The hierarchical structure in VNP is more sequential than global-to-local. Conversely, PONP~\citep{gu2023generalizable} is agnostic to neural-field specifics and concentrates on the neural process perspective. In this work, we consider a hierarchical neural process to model the structure information of the scene. 





\section{Conclusion}
\section{Conclusion}
% In conclusion, this study highlights the efficacy and potential of integrating Parameter-Efficient Fine-Tuning (PEFT) methods with game theory principles through the innovative approach of LoRA with Mixture of Gamers (\ourmethod). By melding Low-Rank Adaptation (LoRA) with Mixture of Experts (MoE) and utilizing game theory-based dynamics, \ourmethod{} significantly advances the field by addressing critical gaps in flexibility and dynamic expert selection inherent in previous methods. The employment of submatrix decomposition alongside Shapley values in \ourmethod{} enables a more granular understanding of the interactions and contributions of different components within PEFT setups. The promising experimental outcomes across a variety of tasks not only underscore \ourmethod's superior performance but also illuminate its versatile applicability and the potential for future adaptations in complex, domain-specific applications. Moving forward, it will be crucial to refine these approaches, ensuring robustness and scalability, to fully harness the transformative power of PEFT in enhancing machine learning models.

In this paper, we introduce \ourmethod{}, a Med-LVLM that unifies medical vision-language comprehension and generation through a novel heterogeneous knowledge adaptation approach.
% integrates H-LoRA and a three-stage fine-tuning approach, aim at unifying medical understanding and generation tasks. 
% To enhance the multi-task performance of \ourmethod{}, we introduce the \texttt{VL-Health} dataset for training. 
Experimental results demonstrate that \ourmethod{} achieves significant performance improvements across multiple medical comprehension and generation tasks, showcasing its potential for healthcare applications. 
\input{}

\section*{Impact Statement}
This paper contributes to the advancement of Machine Learning. While our work may have various societal implications, none require specific emphasis at this stage.


\bibliography{icml_main}
\bibliographystyle{icml2025}


%%%%%%%%%%%%%%%%%%%%%%%%%%%%%%%%%%%%%%%%%%%%%%%%%%%%%%%%%%%%%%%%%%%%%%%%%%%%%%%
%%%%%%%%%%%%%%%%%%%%%%%%%%%%%%%%%%%%%%%%%%%%%%%%%%%%%%%%%%%%%%%%%%%%%%%%%%%%%%%
% APPENDIX
%%%%%%%%%%%%%%%%%%%%%%%%%%%%%%%%%%%%%%%%%%%%%%%%%%%%%%%%%%%%%%%%%%%%%%%%%%%%%%%
%%%%%%%%%%%%%%%%%%%%%%%%%%%%%%%%%%%%%%%%%%%%%%%%%%%%%%%%%%%%%%%%%%%%%%%%%%%%%%%
\newpage
\appendix
\onecolumn
% \section*{Appendix}
%{\color{lightblue}
%\tableofcontents
%}
%
% \section*{Table of Contents}
% \begin{itemize}
%     \item[\textbf{\textcolor{blue}{A}}] \textbf{\textcolor{blue}{Implementation Details}}
%     \begin{itemize}
%         \item[A.1] Gaussian Construction
%         \item[A.2] Neural Radiance Field Rendering
%         \item[A.3] Hierarchical Latent Variables
%         \item[A.4] Modulation
%     \end{itemize}
    
%     \item[\textbf{\textcolor{blue}{B}}] \textbf{\textcolor{blue}{Derivation of Evidence Lower Bound}}
    
%     \item[\textbf{\textcolor{blue}{C}}] \textbf{\textcolor{blue}{Training Details}}
    
%     \item[\textbf{\textcolor{blue}{D}}] \textbf{\textcolor{blue}{More Experimental Results}}
% \end{itemize}

%%%%%%%%%%%%%%%%%%%%%%%%%%%%%%%%%%%%%%%%%%%%
% NerF: $F: (x,y,z,\theta,\phi) \mapsto (c,\sigma)$. 

% Context set (camera pose and image rbg level): 
% \begin{align}
% \tilde{X}_c:&= \{ \tilde{x}_i \}_{i=1}^{N_c}, \tilde{x}_i \in \mathbb{R}^{6} (\text{ray-o}, \text{ray-d})\\
% \tilde{Y}_c:&= \{ \tilde{y}_i \}_{i=1}^{N_c}, \tilde{y}_i \in \mathbb{R}^{3} (\text{RGB})
% \end{align}

% Target set (3d coordinates and 3d rbg with density level):

% $x_i$ is a set of 3D points corresponding to a specific ray direction. 
% \begin{align}
% X_t:&= \{ x_i \}_{i=1}^{N_t}, x_i \in \mathbb{R}^{P\times3} (\text{sampled 3D points})\\
% Y_t:&= \{ y_i \}_{i=1}^{N_t}, y_i \in \mathbb{R}^{P\times4} (c,\sigma)
% \end{align}

% \begin{align}
%     p(Y_t|X_t, \tilde{X}_c, \tilde{Y}_c) & = p(Y_t|X_t,B_c)p(B_c|\tilde{X}_c, \tilde{Y}_c) \\
%     &= \Pi_{i=1}^{N_t} p(y_i|x_i, g, r_i, B_c) p(r_i|g, x_i) p(g|B_c, X_t) p(B_c|\tilde{X}_c, \tilde{Y}_c)
% \end{align}

% \begin{align}
%     p(Y_t|X_t, \tilde{X}_c, \tilde{Y}_c) & \approx p(Y_t|X_t,B_c) \\
%     &= \Pi_{i=1}^{N_t} p(y_i|x_i, g, r_i, B_c) p(r_i|g, x_i) p(g|B_c, X_t)
% \end{align}
%%%%%%%%%%%%%%%%%%%%%%%%%%%%%%%%%%%%%%%%%%%%
\newpage
\section{Neural Radiance Field Rendering}
\label{supp:nerf-render}
In this section, we outline the rendering function of NeRF~\citep{mildenhall2021nerf}. A 5D neural radiance field represents a scene by specifying the volume density and the directional radiance emitted at every point in space. NeRF calculates the color of any ray traversing the scene based on principles from classical volume rendering~\citep{kajiya1984ray}. The volume density $\sigma(\mathbf{x})$ quantifies the differential likelihood of a ray terminating at an infinitesimal particle located at $\mathbf{x}$. The anticipated color $C(\mathbf{r})$ of a camera ray $\mathbf{r}(t) = \mathbf{o} + t\mathbf{d}$, within the bounds $t_n$ and $t_f$, is determined as follows:
\begin{equation}
C(\mathbf{r}) = \int_{t_n}^{t_f} T(t) \sigma(\mathbf{r}(t)) c(\mathbf{r}(t), \mathbf{d}) dt, \quad \text{where} \quad T(t) = \exp \left( - \int_{t_n}^{t} \sigma(\mathbf{r}(s)) ds \right).
\end{equation}

Here, the function $T(t)$ represents the accumulated transmittance along the ray from $t_n$ to $t$, which is the probability that the ray travels from $t_n$ to $t$ without encountering any other particles. To render a view from our continuous neural radiance field, we need to compute this integral $C(\mathbf{r})$ for a camera ray traced through each pixel of the desired virtual camera.

\section{Implementation Details}
\label{sec:implementation-details}

\subsection{Gaussian Construction}
\label{supp:gaussian}

As introduced in Sec.~\ref{sec: geometrybases}, we introduce geometric bases ${\bf{{B}}}_{C}$ to structure the context variables geometrically.
${\bf{{B}}}_{C}$ are geometric  bases (Gaussians) inferred from the context views $\{{\bf{\widetilde{X}}}_{C}, {\bf{\widetilde{Y}}}_{C}\}$ with 3D structure information, 
\textit{i.e.,} ${\bf{b}}_i = \{ \mathcal{N}(\mu_i, \Sigma_i); \omega_i\}$,
%\textit{i.e., object shape, color and texture.}. $B_C$ is obtained by: 
% ${\bf{{B}}}_{C}=\texttt{Encoder}\Big({\bf{\widetilde{X}}}_{C}, {\bf{\widetilde{Y}}}_{C}\Big)$. 
\begin{align}
    &{\bf{{B}}}_{C} = \{{\bf{b}}_i\}_{i=1}^{M}, {\bf{b}}_i=\{\mathcal{N}(\mu_i, \Sigma_i); \omega_i\},
    \label{eq: generation_B_1}
    \\
    & \mu_i, \Sigma_i = \texttt{Att}({\bf{\widetilde{X}}}_{C}, {\bf{\widetilde{Y}}}_{C}), \texttt{Att}({\bf{\widetilde{X}}}_{C}, {\bf{\widetilde{Y}}}_{C}),
    \label{eq: generation_B_2}
    \\
    & \omega_i = \texttt{Att}({\bf{\widetilde{X}}}_{C}, {\bf{\widetilde{Y}}}_{C}),
    \label{eq: generation_B_3}
\end{align}
where $M$ is the number of the Gaussian bases. $\mu \mathbb \in {R}^3$ is the Gaussian center, $\Sigma \in  \mathbb{R}^{3\times 3}$ is the covariance matrix, and $\omega \in \mathbb{R}^{d_B}$ is the corresponding ${d_B}$-dimension semantic representation. In our implementation, we choose $d_{B}$ as $32$. $\texttt{Att}$ is a self-attention module. Specifically, given the context set $[\widetilde{\mathbf{X}};\widetilde{\mathbf{Y}}] \in \mathbb{R}^{H\times W \times (3+3+3)}$, the visual self-attention module, $\texttt{Att}$, first produces a $M\times D$ tokens with $M$ is the number of visual tokens and $D$ is the hidden dimension. The number of Gaussians we use equals the number of tokens $M$. %Then, we use one MLP to predict centers $\mu$, as well as the rotation $R$ and scaling $S$ matrices parameters for producing covariance matrix $\Sigma$, and one MLP to produce the latent representations $\omega$. 
Then, we use one MLP with 2 linear layers to map the tokens into a 10-dimensional vector, which includes 3-dimensional Gaussian centers, a 3-dimensional vector for constructing the scaling matrix, and a 4-dimensional vector for quaternion parameters of the rotation matrix. Both the scaling matrix and rotation matrix are used to build the \(3 \times 3\) covariance matrix. This procedure is similar to Gaussian construction in the 3D Gaussian Splatting~\citep{kerbl20233d}.
Another MLP estimates the latent representation of each Gaussian basis, using a 32-dimensional vector for each Gaussian basis. 

The covariance matrix is obtained by:
\begin{equation}
    \Sigma = RSS^TR^T,
    \label{eq:cov-matrix}
\end{equation}
where $R\in \mathbb{R}^{3\times3}$ is the rotation matrix, and $S \in \mathbb{R}^3$ is the scaling matrix. 











\subsection{Hierarchical Latent Variables}
\label{supp:latent-variables}

\begin{figure}[t]
  \centering
  \includegraphics[width=0.7\textwidth]{Figures/Transformer.pdf} % Adjust the size and filename as needed
  \caption{\textbf{Using transformer encoder to generate ray-specific latent variable $\mathbf{z}_r$.}} % Caption for the figure
  \label{fig:latent-transformer}
  % \vspace{-3mm}
\end{figure}

At the object level, the distribution of an object-specific latent variable \(\mathbf{z}_o\) is obtained by aggregating all location representations from \((\mathbf{B}_C, \mathbf{X}_T)\). We assume \(p(\mathbf{z}_o | \mathbf{B}_C, \mathbf{X}_T)\) follows a standard Gaussian distribution and generate its mean \(\mu_{o}\) and variance \(\sigma_{o}\) using MLPs. We sample an object-specific modulation vector, \(\hat{\mathbf{z}}_o\), from its prior distribution \(p(\mathbf{z}_o | \mathbf{X}_T, \mathbf{B}_C)\).

Similarly, as shown in Fig.~\ref{fig:latent-transformer}, we aggregate the information per ray using \(\mathbf{B}_C\), which is then fed into a Transformer along with \(\hat{\mathbf{z}}_o\) to predict the latent variable \(\mathbf{z}_r\) with mean \(\mu_r\) and \(\sigma_r\) for each ray.


 

\subsection{Modulation}
\label{supp:modulate}
The latent variables for modulating the MLP are represented as \([\mathbf{z}_o; \mathbf{z}_r]\). Our approach to the modulated MLP layer follows the style modulation techniques described in \citep{karras2020analyzing, guo2023versatile}. Specifically, we consider the weights of an MLP layer (or 1x1 convolution) as \( W \in \mathbb{R}^{d_{\text{in}} \times d_{\text{out}}} \), where \( d_{\text{in}} \) and \( d_{\text{out}} \) are the input and output dimensions respectively, and \( w_{ij} \) is the element at the \(i\)-th row and \(j\)-th column of \( W \).

To generate the style vector \( s \in \mathbb{R}^{d_{\text{in}}} \), we pass the latent variable \( z \) through two MLP layers. Each element \( s_i \) of the style vector \( s \) is then used to modulate the corresponding parameter in \( W \).
\begin{equation}
    w'_{ij} = s_i \cdot w_{ij}, \quad j = 1, \ldots, d_{\text{out}},
\end{equation}
where $w_{ij}$ and $w'_{ij}$ denote the original and modulated weights, respectively.

The modulated weights are normalized to preserve training stability,
\begin{equation}
    w''_{ij} = \frac{w'_{ij}}{\sqrt{\sum_i w'^2_{ij} + \epsilon}}, \quad j = 1, \ldots, d_{\text{out}}.
\end{equation}




\begin{algorithm}[H]
\caption{Modulation Layer}
\begin{algorithmic}[1]
\REQUIRE Latent variable $z$, weight matrix $W \in \mathbb{R}^{d_{\mathrm{in}} \times d_{\mathrm{out}}}$
\ENSURE Modulated and normalized weight matrix $W''$
\STATE \textbf{Compute style vector:}
\STATE $s \leftarrow \mathrm{MLP}_2\big(\mathrm{MLP}_1(z)\big)$
\STATE \textbf{Modulate weights:}
\STATE $W' \leftarrow \operatorname{diag}(s) \times W$
\STATE \textbf{Normalize modulated weights:}
\STATE For each column $j$ in $W'$:
\STATE \hskip1em $\sigma_j \leftarrow \sqrt{\sum_{i=1}^{d_{\mathrm{in}}} (W'_{ij})^2 + \epsilon}$
\STATE Normalize column $j$ of $W'$: $W''_{:,j} \leftarrow W'_{:,j} / \sigma_j$
\RETURN $W''$
\end{algorithmic}
\end{algorithm}










\begin{algorithm}[H]
\caption{Training Procedure}
\begin{algorithmic}[1]
\REQUIRE Context set $({\bf{X}}_{C}, {\bf{Y}}_C)$, target set $({\bf{X}}_{T}, {\bf{Y}}_T)$
\ENSURE Prediction ${\bf{Y}}'_T$
\vspace{0.5em}
\STATE Estimate the context bases ${\bf{B}}_C$ and the target bases ${\bf{B}}_T$ (Eq. 12).
\vspace{0.5em}
\STATE Estimate the object-specific latent variables:
\begin{itemize}
    \item For the context set ${\bf{z}}_o^C$:
    \[
    {\bf{z}}_o^C \sim p({\bf{z}}_o \mid {\bf{X}}_C, {\bf{B}}_C)
    \]
    \item For the target set ${\bf{z}}_o^T$:
    \[
    {\bf{z}}_o^T \sim q({\bf{z}}_o \mid {\bf{X}}_T, {\bf{B}}_T) \quad \text{(Eq. 7)}
    \]
\end{itemize}
\vspace{0.5em}
\STATE Estimate the ray-specific latent variables:
\begin{itemize}
    \item For the context set ${\bf{z}}_r^{C}$:
    \[
    {\bf{z}}_r^{C} \sim p({\bf{z}}_r^n \mid {\bf{z}}_o^C, {\bf{x}}_C^{n}, {\bf{B}}_C) \quad \text{(Eq. 8)}
    \]
    \item For the target set ${\bf{z}}_r^{T}$:
    \[
    {\bf{z}}_r^{T} \sim q({\bf{z}}_r^n \mid {\bf{z}}_o^T, {\bf{x}}_T^{n}, {\bf{B}}_T) \quad \text{(Eq. 8)}
    \]
\end{itemize}
\vspace{0.5em}
\STATE Modulate MLP $f$ using the target latent variables $\{{\bf{z}}_o^T, {\bf{z}}_r^{T}\}$ (Eqs. 16 \& 17).
\vspace{0.5em}
\STATE Render novel views $\hat{\bf{Y}}_T$ using the modulated MLP $f$.
\vspace{0.5em}
\STATE \textbf{Compute losses:}
\begin{itemize}
    \item Reconstruction loss between predictions and ground truth:
    \[
    \mathcal{L}_{\text{recon}} = \text{Loss}(\hat{\bf{Y}}_T, {\bf{Y}}_T)
    \]
    \item Latent variable alignment losses (KL divergence) using context and target latent variables (Eq. 10).
\end{itemize}
\end{algorithmic}
\end{algorithm}


\begin{algorithm}[H]
\caption{Inference Procedure}
\begin{algorithmic}[1]
\REQUIRE Context set $({\bf{X}}_C, {\bf{Y}}_C)$, target input ${\bf{X}}_T$
\ENSURE Prediction ${\bf{Y}}'_T$
\vspace{0.5em}
\STATE Estimate the context bases ${\bf{B}}_C$ (Eq. 12).
\vspace{0.5em}
\STATE Estimate the object-specific latent variable ${\bf{z}}_o$ based on the context set:
\[
{\bf{z}}_o \sim p({\bf{z}}_o \mid {\bf{X}}_C, {\bf{B}}_C) \quad \text{(Eq. 7)}
\]
\vspace{0.5em}
\STATE Estimate the ray-specific latent variables ${\bf{z}}_r^{T}$:
\[
{\bf{z}}_r^{T} \sim p({\bf{z}}_r^n \mid {\bf{z}}_o, {\bf{x}}_T^{n}, {\bf{B}}_C) \quad \text{(Eq. 8)}
\]
\vspace{0.5em}
\STATE Modulate the MLP $f$ using the latent variables $\{{\bf{z}}_o, {\bf{z}}_r^{T}\}$ (Eqs. 16 \& 17).
\vspace{0.5em}
\STATE Render novel views $\hat{\bf{Y}}_T$ using the modulated MLP $f$.
\end{algorithmic}
\end{algorithm}


% $f_C$ by $\{{\bf{z}}_o, {\bf{z}}_r^n\}_C$, 

\section{Derivation of Evidence Lower Bound}
\label{supp:elbo}



\noindent{\textbf{Evidence Lower Bound.}} 
We optimize the model via variational inference~\citep{garnelo2018neural}, deriving the evidence lower bound (ELBO):
\begin{equation}
\begin{aligned}
& \log p(\mathbf{Y}_T \mid \mathbf{X}_T, \mathbf{B}_C) \geq \\
&\mathbb{E}_{q(\mathbf{z}_g | \mathbf{X}_T, \mathbf{B}_T)} \Bigg[ \sum_{m=1}^M \mathbb{E}_{q(\mathbf{z}_l^m | \mathbf{z}_g, \mathbf{x}_T^m, \mathbf{B}_T)} \log p(\mathbf{y}_T^m \mid \mathbf{z}_g, \mathbf{z}_l^m, \mathbf{x}_T^m) \\
& \quad - D_{\text{KL}}\Big[q(\mathbf{z}_l^m | \mathbf{z}_g, \mathbf{x}_T^m, \mathbf{B}_T) \,\big|\big|\, p(\mathbf{z}_l^m | \mathbf{z}_g, \mathbf{x}_T^m, \mathbf{B}_C)\Big] \Bigg] \\
& - D_{\text{KL}}\Big[q(\mathbf{z}_g | \mathbf{X}_T, \mathbf{B}_T) \,\big|\big|\, p(\mathbf{z}_g | \mathbf{X}_T, \mathbf{B}_C)\Big],
\end{aligned}
\end{equation}
where the variational posterior factorizes as $q(\mathbf{z}_g, \{\mathbf{z}_l^m\}_{m=1}^M | \mathbf{X}_T, \mathbf{B}_T) = q(\mathbf{z}_g | \mathbf{X}_T, \mathbf{B}_T) \prod_{m=1}^M q(\mathbf{z}_l^m | \mathbf{z}_g, \mathbf{x}_T^m, \mathbf{B}_T)$. Here, $\mathbf{B}_T$ denotes geometric bases constructed from target data $\{\widetilde{\mathbf{X}}_T, \widetilde{\mathbf{Y}}_T\}$ (available only during training). The KL terms regularize the hierarchical priors $p(\mathbf{z}_g | \mathbf{B}_C)$ and $p(\mathbf{z}_l^m | \mathbf{z}_g, \mathbf{B}_C)$ to align with variational posteriors inferred from $\mathbf{B}_T$, enhancing generalization to context-only settings. Derivations are in Appendix~\ref{supp:elbo}.


The propose \textbf{GeomNP} is formulated as:
{\small
\begin{equation}
        p({\bf{Y}}_{T}| {\bf{X}}_{T}, {\bf{B}}_{C}) = \int \prod_{n=1}^{N} \Big\{ \int p({\bf{y}}_{T}^{\mathbf{r}, n}| {\bf{x}}_{T}^{\mathbf{r}, n}, {\bf{B}}_{C}, {\bf{z}}_r^n,{\bf{z}}_o, ) p({\bf{r}}^n| {\bf{z}}_o,  {\bf{x}}_{T}^{\mathbf{r}, n}, {\bf{B}}_C) d {\bf{z}}_r^n \Big\} p({\bf{z}}_o |{\bf{X}}_T, {\bf{B}}_C) d {\bf{z}}_o, 
\label{eq:ganp-model-supp}
\end{equation}}where $p({\bf{z}}_o | {\bf{B}}_C,  {\bf{X}}_T)$ and $p({\bf{z}}_r^n| {\bf{z}}_o,  {\bf{x}}_{T}^{r, n}, {\bf{B}}_C)$ denote prior distributions of a object-specific and each ray-specific latent variables, respectively. Then, the evidence lower bound is derived as follows.

\begin{equation}
\begin{aligned}
        &\log p({\bf{Y}}_{T}| {\bf{X}}_{T}, {\bf{B}}_{C}) \\
        &= \log \int \prod_{n=1}^{N} \Big\{ \int p({\bf{y}}_{T}^{\mathbf{r}, n}| {\bf{x}}_{T}^{\mathbf{r}, n}, {\bf{z}}_o, {\bf{z}}_r^n) p({\bf{z}}_r^n| {\bf{z}}_o,  {\bf{x}}_{T}^{\mathbf{r}, n}, {\bf{B}_C}) d {\bf{z}}_r^n \Big\} p({\bf{z}}_o | {\bf{B}}_C,  {\bf{X}}_T) d {\bf{z}}_o  \\
    &= \log \int  \prod_{i=1}^{N} \Big\{ \int p({\bf{y}}_{T}^{\mathbf{r}, n}| {\bf{x}}_{T}^{\mathbf{r}, n}, {\bf{z}}_o, {\bf{z}}_r^n) p({\bf{z}}_r^n| {\bf{z}}_o,  {\bf{x}}_{T}^{\mathbf{r}, n}, {\bf{B}_C}) \frac{q({\bf{z}}_r^n| {\bf{z}}_o,  {\bf{x}}_{T}^{\mathbf{r}, n}, {\bf{B}_T})}{q({\bf{z}}_r^n| {\bf{z}}_o,  {\bf{x}}_{T}^{\mathbf{r}, n}, {\bf{B}_T})} d {\bf{z}}_r^n \Big\} \\
    & p({\bf{z}}_o | {\bf{B}}_C,  {\bf{X}}_T) \frac{q({\bf{z}}_o | {\bf{B}}_T,  {\bf{X}}_T)}{q({\bf{z}}_o | {\bf{B}}_T,  {\bf{X}}_T,)} d {\bf{z}}_o  \\
    &\geq  \mathbb{E}_{q({\bf{z}}_o | {\bf{B}}_T,  {\bf{X}}_T)}  \Big\{  \sum_{i=1}^{N} \log  \int p({\bf{y}}_{T}^{\mathbf{r}, n}| {\bf{x}}_{T}^{\mathbf{r}, n}, {\bf{z}}_o, {\bf{z}}_r^n) p({\bf{z}}_r^n| {\bf{z}}_o,  {\bf{x}}_{T}^{\mathbf{r}, n}, {\bf{B}_C}) \frac{q({\bf{z}}_r^n| {\bf{z}}_o,  {\bf{x}}_{T}^{\mathbf{r}, n}, {\bf{B}_T})}{q({\bf{z}}_r^n| {\bf{z}}_o,  {\bf{x}}_{T}^{\mathbf{r}, n}, {\bf{B}_T})} d {\bf{z}}_r^n \Big\} \\
    &- D_{\text{KL}}(q({\bf{z}}_o | {\bf{B}}_T,  {\bf{X}}_T,) || p({\bf{z}}_o | {\bf{B}}_C,  {\bf{X}}_T)) \\
    &\geq  \mathbb{E}_{q({\bf{z}}_o | {\bf{B}}_T,  {\bf{X}}_T)}  \Big\{  \sum_{n=1}^{N}  \mathbb{E}_{q({\bf{z}}_r^n| {\bf{z}}_o,  {\bf{x}}_{T}^{\mathbf{r}, n}, {\bf{B}_T})} \log p({\bf{y}}_{T}^{{\mathbf{r}}, n}| {\bf{x}}_{T}^{{\mathbf{r}}, n}, {\bf{z}}_o, {\bf{z}}_r^n) \\
&- D_{\text{KL}}[q({\bf{z}}_r^n| {\bf{z}}_o,  {\bf{x}}_{T}^{{\mathbf{r}}, n}, {\bf{B}_T}) || p({\bf{z}}_r^n| {\bf{z}}_o,  {\bf{x}}_{T}^{{\mathbf{r}}, n}, {\bf{B}_C}) ] \Big\} 
- D_{\text{KL}}[q({\bf{z}}_o | {\bf{B}}_T,  {\bf{X}}_T) || p({\bf{z}}_o | {\bf{B}}_C,  {\bf{X}}_T)], \\
        % &=  \int \log \prod_{i=1}^{N_{ray}} \Big\{ \int p({\bf{y}}^{T}_{1:P, i}| {\bf{x}}^{T}_{1:P, i}, {\bf{g}}, {\bf{r}}_i) p({\bf{r}}_i| {\bf{g}},  {\bf{x}}^{T}_{1:P, i}) d {\bf{r}}_i \Big\} +  \log p({\bf{g}} | {\bf{B}}_C,  {\bf{X}}_T,) d {\bf{g}} \\
        % &= \int \sum_{i=1}^{N_{ray}} \log \Big\{ \int p({\bf{y}}^{T}_{1:P, i}| {\bf{x}}^{T}_{1:P, i}, {\bf{g}}, {\bf{r}}_i) p({\bf{r}}_i| {\bf{g}},  {\bf{x}}^{T}_{1:P, i}) d {\bf{r}}_i \Big\} +  \log p({\bf{g}} | {\bf{B}}_C,  {\bf{X}}_T,) d {\bf{g}} \\
        % &= \int \sum_{i=1}^{N_{ray}}  \Big\{ \int \log p({\bf{y}}^{T}_{1:P, i}| {\bf{x}}^{T}_{1:P, i}, {\bf{g}}, {\bf{r}}_i) + \log p({\bf{r}}_i| {\bf{g}},  {\bf{x}}^{T}_{1:P, i}) d {\bf{r}}_i \Big\} +  \log p({\bf{g}} | {\bf{B}}_C,  {\bf{X}}_T,) d {\bf{g}} \\
\end{aligned}      
\end{equation}
where $q_{\theta, \phi}({\bf{z}}_o,  \{{\bf{z}}_r^i\}_{i=1}^{N} | {\bf{X}}_T, {\bf{B}}_T) = q({\bf{z}}_r^n| {\bf{z}}_o,  {\bf{x}}_{T}^{{\mathbf{r}}, n}, {\bf{B}_T}) q({\bf{z}}_o | {\bf{B}}_T,  {\bf{X}}_T)$ is the variational posterior of the hierarchical latent variables. 


\section{More Related Work}
% \textcolor{blue}{
% \cite{szymanowicz2024splatter}, \cite{charatan2024pixelsplat}, \cite{chen2025mvsplat}, \cite{hong2023lrm}, \cite{muller2023diffrf}, \cite{tewari2023diffusion}, \cite{xu2022point}, \cite{wang2024learning}, \cite{liu2024geometry}}


\textcolor{blue}{\paragraph{Generalizable Neural Radiance Fields (NeRF)}
Advancements in neural radiance fields have focused on improving generalization across diverse scenes and objects. \cite{wang2022attention} propose an attention-based NeRF architecture, demonstrating enhanced capabilities in capturing complex scene geometries by focusing on informative regions. \cite{suhail2022generalizable} introduce a generalizable patch-based neural rendering approach, enabling models to adapt to new scenes without retraining. \cite{xu2022point} present \textit{Point-NeRF}, leveraging point-based representations for efficient scene modeling and scalability. \cite{wang2024learning} further enhance point-based methods by incorporating visibility and feature augmentation to improve robustness and generalization. \cite{liu2024geometry} propose a geometry-aware reconstruction with fusion-refined rendering for generalizable NeRFs, improving geometric consistency and visual fidelity. Recently, the \textit{Large Reconstruction Model (LRM)}~\citep{hong2023lrm} has drawn attention. It aims for single-image to 3D reconstruction, emphasizing scalability and handling of large datasets.}

\textcolor{blue}{\paragraph{Gaussian Splatting-based Methods}
Gaussian splatting~\citep{kerbl20233d} has emerged as an effective technique for efficient 3D reconstruction from sparse views. \cite{szymanowicz2024splatter} propose \textit{Splatter Image} for ultra-fast single-view 3D reconstruction. \cite{charatan2024pixelsplat} introduce \textit{pixelsplat}, utilizing 3D Gaussian splats from image pairs for scalable generalizable reconstruction. \cite{chen2025mvsplat} present \textit{MVSplat}, focusing on efficient Gaussian splatting from sparse multi-view images. Our approach can be a complementary module for these methods by introducing a probabilistic neural processing scheme to fully leverage the observation. }

\textcolor{blue}{\paragraph{Diffusion-based 3D Reconstruction}
Integrating diffusion models into 3D reconstruction has shown promise in handling uncertainty and generating high-quality results. \cite{muller2023diffrf} introduce \textit{DiffRF}, a rendering-guided diffusion model for 3D radiance fields. \cite{tewari2023diffusion} explore solving stochastic inverse problems without direct supervision using diffusion with forward models. \cite{liu2023zero} propose \textit{Zero-1-to-3}, a zero-shot method for generating 3D objects from a single image without training on 3D data, utilizing diffusion models. \cite{shi2023zero123++} introduce \textit{Zero123++}, generating consistent multi-view images from a single input image using diffusion-based techniques. \cite{shi2023mvdream} present \textit{MVDream}, which uses multi-view diffusion for 3D generation, enhancing the consistency and quality of reconstructed models.}


\section{Implementation Details}
We train all our models with PyTorch. Adam optimizer is used with a learning rate of $1e-4$. For NeRF-related experiments, we follow the baselines~\citep{chen2022transformers,guo2023versatile} to train the model for 1000 epochs. All experiments are conducted on four NVIDIA A5000 GPUs. For the hyper-parameters $\alpha$ and $\beta$, we simply set them as $0.001$.  


\textcolor{blue}{\paragraph{Model Complexity} The comparison of the number of parameters is presented in Table.~\ref{tab:params_psnr}. Our method, GeomNP, utilizes fewer parameters than the baseline, VNP, while achieving better performance on the ShapeNet Car dataset in terms of PSNR.}

\begin{table}[h!]
\centering
\caption{Comparison of the number of parameters and PSNR on the ShapeNet Car dataset.}
\begin{tabular}{lcc}
\toprule
Method & {\# Parameters} & {PSNR} \\ 
\midrule
VNP     & 34.3M   & 24.21 \\ 
GeomNP  & \textbf{24.0M}   & \textbf{25.13} \\ 
\bottomrule
\end{tabular}
\label{tab:params_psnr}
\end{table}

\paragraph{Integration with PixelNeRF} 
\textcolor{blue}{To integrate our method into PixelNeRF, we utilize the same feature extractor and NeRF architecture. Specifically, we employ a pre-trained ResNet to extract features from the observed images. From the latent space of the feature encoder, we predict geometric bases, which are used to re-represent each 3D point in a higher-dimensional space. These re-represented point features are aggregated into latent variables, which are then used to modulate the first two input MLP layers of PixelNeRF's NeRF network. During training, we align the latent variables derived from the context images with those from the target views to ensure consistency.}

% \newpage

\section{More Experimental Results}
\label{supp:more-results}
%%%%%%%%%%%%%%%%%%%%%%%%%%%%%%%%%%%%%%%%%%%%%%%%%%%%%%%%%%%%
\subsection{Image Regression}

% \begin{figure*}[htbp]
%     \centering
%     \begin{minipage}[b]{0.45\textwidth} 
%         \includegraphics[width=\textwidth]{Figures/image-regression0.pdf} % Adjust filename as needed
%         \caption{CelebA}
%         \label{fig:celeba}
%     \end{minipage}
%     \hfill
%     \begin{minipage}[b]{0.45\textwidth} 
%         \includegraphics[width=\textwidth]{Figures/image-regression1.pdf} % Adjust filename as needed
%         \caption{Imagenette}
%         \label{fig:imagenette}
%     \end{minipage}
%     \caption{\textbf{Visualizations} of image regression results on CelebA (left) and Imagenette (right).}
%     \label{fig:visualization-image-regression}
% \end{figure*}

\subsection{Image Completion}
 We also conduct experiments of \method{} on image completion (also called image inpainting), which is a more challenging variant of image regression. Essentially, only part of the pixels are given as context, while the INR functions are required to complete the full image. Visualizations in Fig.~\ref{fig:completion} demonstrate the generalization ability of our method to recover realistic images with fine details based on very limited context ($10 \% - 20\%$ pixels).


\begin{figure}[t!]
  \centering
  \includegraphics[width=0.99\textwidth]{Figures/image-completion.pdf} % Adjust the size and filename as needed
  \vspace{-3mm}
  \caption{\textbf{Image completion visualization} on CelebA using $10\%$ (left) and $20\%$ (right) context.}
  \label{fig:completion}
  \vspace{-5mm}
\end{figure}

\subsection{Comparison with GNT.}
\label{sec:compare_gnt}
Specfiically, we use GNT's image encoder and predict the geometric bases, and GNT's NeRF' network for prediction. 



\begin{figure}[htbp]
  \centering
  \includegraphics[width=0.99\textwidth]{ICML25/Figures/nerf-syn-1view.pdf} % Adjust the size and filename as needed
  \vspace{-3mm}
  \caption{\textbf{Qualitative comparison with GNT on 1-view setting.}}
  \label{fig:1-view-compare}
  \vspace{-5mm}
\end{figure}

\subsubsection{Cross-Category Example.}
\label{sec:cross-category}

\begin{figure}[htbp]
  \centering
  \includegraphics[width=0.99\textwidth]{ICML25/Figures/cross-category.pdf} % Adjust the size and filename as needed
  \vspace{-3mm}
  \caption{}
  \label{fig:cross-category}
  \vspace{-5mm}
\end{figure}



% \subsection{Comparison with PixelNeRF}

% \begin{wraptable}{r}{0.42\textwidth}
% \vspace{-4mm}
% \caption{\textbf{Comparison on the DTU MVS dataset.} Training with 1-view context and testing with both 1-view and 3-view context images. Integrating \method{} into the pixelNeRF framework leads to improvement in terms of both PSNR and SSIM.}
% \centering
% \resizebox{0.48\textwidth}{!}{
% \begin{tabular}{llcc}
% \toprule
%  & {Method} & {PSNR} & {SSIM} \\
% \midrule
% \multirow{2}{*}{1-view} 
% & pixelNeRF & 15.51 & 0.51 \\
% \multirow{-1}{*}{} & \cellcolor{lightblue}\textbf{\method{}} (Ours) & \cellcolor{lightblue}\textbf{15.89} & \cellcolor{lightblue}\textbf{0.58} \\
% \midrule
% \multirow{2}{*}{3-view} 
% & pixelNeRF & 15.80 & 0.56 \\
% \multirow{-1}{*}{} & \cellcolor{lightblue}\textbf{\method{}} (Ours) & \cellcolor{lightblue}\textbf{16.99} & \cellcolor{lightblue}\textbf{0.61} \\
% \bottomrule
% \vspace{-3mm}
% \end{tabular}
% }
% \label{tab:dtu-compare}
% \end{wraptable}
% \noindent {\textbf{Comparison on DTU.}}
% %Our method can be flexibly integrated with other approaches. 
% To ensure a fair comparison with pixelNeRF~\citep{yu2021pixelnerf} using the same encoder and NeRF network architecture, we incorporate our probabilistic framework into pixelNeRF. We conducted experiments on real-world scenes from the DTU MVS dataset~\citep{aanaes2016large}. To explore the capability of dealing with extremely limited context information, we 
% train both models with 1-view context and test the 1-view and 3-view results in terms of PSNR and SSIM~\citep{wang2004image} metrics. Both qualitative results in Table~\ref{tab:dtu-compare} and qualitative results in Fig.~\ref{fig:dtu-visualization} demonstrate our probabilistic modeling can improve the existing methods. Notably, even when trained with a 1-view context image and tested with 3-view context images, our method significantly outperforms pixelNeRF, demonstrating that our probabilistic framework effectively utilizes limited observations.


% \begin{figure}[t]
%   \centering
%   \includegraphics[width=1\textwidth]{ICLR2025/Figures/dtu-results.pdf} % Adjust the size and filename as needed
%   \vspace{-6mm}  
%   \caption{\textbf{Novel view synthesis results with 1-view context on the DTU dataset.} \method{} has a more realistic rendering quality than pixelNeRF~\citep{yu2021pixelnerf} for novel views with extremely limited context views (1-view).} % Caption for the figure
%   \label{fig:dtu-visualization}
%   \vspace{-5mm}
% \end{figure}




\subsection{More results on ShapeNet}
In this section, we demonstrate more experimental results on the novel view synthesis task on ShapeNet in Fig~\ref{fig:nerf-supp-shapenet}, comparison with VNP~\cite{guo2023versatile} in Fig.~\ref{fig:nerf-supp-compare}, and image regression on the Imagenette dataset in Fig.~\ref{fig:image-supp-image}. The proposed method is able to generate realistic novel view synthesis and 2D images.


\begin{figure}[t]
  \centering
\includegraphics[width=1\textwidth]{Figures/nerf-results-more-supp.pdf} % Adjust the size and filename as needed
  \caption{\textbf{More NeRF results on novel view synthesis task on ShapeNet objects.}} % Caption for the figure
  \label{fig:nerf-supp-shapenet}
\end{figure}


\begin{figure}[t]
  \centering
\includegraphics[width=1\textwidth]{Figures/comparsion_vnp.pdf} % Adjust the size and filename as needed
  \caption{\textbf{Comparison between the proposed method and VNP} on novel view synthesis task for ShapeNet objects. Our method has a better rendering quality than VNP for novel views.} % Caption for the figure
  \label{fig:nerf-supp-compare}
\end{figure}

\begin{figure}[htbp]
  \centering
\includegraphics[width=0.8\textwidth]{Figures/imagenette-more.png} % Adjust the size and filename as needed
  \caption{\textbf{More image regression results on the Imagenette dataset.} Left: ground truth; Right: prediction.} % Caption for the figure
  \label{fig:image-supp-image}
  % \vspace{-3mm}
\end{figure}

\subsection{Training Time Comparison}

\textcolor{blue}{As illustrated in Fig.\ref{fig:train-time}, with the same training time, our method (GeomNP) demonstrates faster convergence and higher final PSNR compared to the baseline (VNP). }

\begin{figure}[t]
  \centering
  \includegraphics[width=0.7\textwidth]{ICLR2025/Figures/train_time_psnr.png} % Adjust the size and filename as needed
  \caption{\textbf{Training time vs. PSNR on the ShapeNet Car dataset.} Our method (GeomNP) demonstrates faster convergence and higher final PSNR compared to the baseline (VNP).} % Caption for the figure
  \label{fig:train-time}
  % \vspace{-3mm}
\end{figure}






\subsection{Qualitative ablation of the hierarchical latent variables}
\label{sec:abl-bases}
\textcolor{blue}{In this section, we perform a qualitative ablation study on the hierarchical latent variables. As illustrated in Fig.~\ref{fig:hier-abl}, the absence of the global variable prevents the model from accurately predicting the object's outline, whereas the local variable captures fine-grained details. When both global and local variables are incorporated, GeomNP successfully estimates the novel view with high accuracy.}



\begin{figure}[t]
  \centering
  \includegraphics[width=0.7\textwidth]{./Figures/hierarchical-ablation-new.pdf} % Adjust the size and filename as needed
  \caption{\textbf{Qualitative ablation of the hierarchical latent variables (global and local variables)}. }  % Caption for the figure
  \label{fig:hier-abl}
  % \vspace{-3mm}
\end{figure}



\subsection{More multi-view reconstruction results}
\textcolor{blue}{We integrate our method into GNT~\citep{wang2022attention} framework and perform experiments on the Drums class of the NeRF synthetic dataset. Qualitative comparisons of multi-view results are presented in Fig.~\ref{fig:qua-nerf-syn}. }

\begin{figure}[t]
  \centering
  \includegraphics[width=1.0\textwidth]{ICLR2025/Figures/nerf-syn.pdf} % Adjust the size and filename as needed
  \caption{\textbf{Qualitative comparisons of Multi-view results on the Drums class of the NeRF synthetic dataset. } }  % Caption for the figure
  \label{fig:qua-nerf-syn}
  % \vspace{-3mm}
\end{figure}

% \section{NP with Gaussian Splatting}
% \begin{table}[htbp]
%     \centering
%     \caption{Comparison of methods}
%     \begin{tabular}{lccc}
%         \toprule
%         \textbf{Method} & \textbf{PSNR $\uparrow$} & \textbf{SSIM $\uparrow$} & \textbf{LPIPS $\downarrow$} \\
%         \midrule
%         PixelNeRF & 21.76 & 0.78 & 0.203 \\
%         {Splatter Image} & 21.80 & {0.80} & {0.150} \\
%         \bottomrule
%     \end{tabular}
% \end{table}


% Recently, 3D Gaussian Splatting~\citep{kerbl20233d} has gained significant attention for its efficiency and strong performance in reconstructing 3D scenes. Like NeRF, Gaussian Splatting requires overfitting on a specific scene to optimize the 3D Gaussian parameters. To improve generalization, given a single-view context image, Splatter Image~\citep{szymanowicz2024splatter} employs a UNet to predict Gaussian parameters for a new scene. However, Splatter Image remains a deterministic method and does not account for scene uncertainty. Therefore, in this section, we demonstrate that integrating neural processes can enhance Splatter Image's performance.

% Specifically, we employ the UNet encoder to generate a latent variable, and then sample a scene-specific latent vector to estimate Gaussian parameters through the decoder. For multiple-view ($N$) images, we first aggregate multi-view latent features and then infer the latent variables (mean and variance). We sample $N$ times to probabilistically estimate the Gaussian parameters. The prior and posterior distributions are derived from context and target images, respectively. In addition to the original reconstruction loss, we introduce a KL divergence constraint between the prior and posterior distributions, guiding the model to achieve richer representation with limited observations.

% Experiments are conducted on the CO3D dataset. 


%%%%%%%%%%%%%%%%%%%%%%%%%%%%%%%%%%%%%%%%%%%%%%%%%%%%%%%%%%%%%%%%%%%%%%%%%%%%%%%%%%%%%%%%%%%%%%%%%%

% \subsection{Image Regression}
% \label{sec:image-regression}



% \begin{figure}[t]
%     \centering
%     \begin{subtable}[b]{0.42\textwidth}
%     % \vspace{-6mm}
%     \begin{tabular}{lcc}
%     \toprule
%                  & CelebA & Imagenette \\ \midrule
%     Learned Init \citep{tancik2021learned} & 30.37  & 27.07       \\
%     TransINR~\citep{chen2022transformers}         & 31.96  & 29.01       \\ 
%     % \hline
%     \rowcolor{lightblue}
%     \method{} (Ours)         & \textbf{33.41}  & \textbf{29.82}      \\ 
%     \bottomrule
%     \end{tabular}
%     \caption{Quantitative results. \method{} outperforms baseline methods consistently on both datasets.}
%     \label{tab:image-regression}
%     \end{subtable} \hfill
%     \begin{subtable}[b]{0.54\textwidth}
%     \includegraphics[width=\textwidth]{Figures/image-regression0.pdf} % Adjust the size and filename as needed
%     \caption{Visualizations on CelebA (left) and Imagenette (right), respectively.} % Caption for the figure
%     \label{fig:visualization-image-regression}
% \end{subtable}
% % \vspace{-2mm}
% \caption{\textbf{Quantitative results and visualizations} of image regression on CelebA and Imagenette.}
% \vspace{-3mm}
% \end{figure}

% \begin{figure}[t!]
%   \centering
%   \includegraphics[width=0.99\textwidth]{Figures/image-completion.pdf} % Adjust the size and filename as needed
%   \vspace{-3mm}
%   \caption{\textbf{Image completion visualization} on CelebA using $10\%$ (left) and $20\%$ (right) context.}
%   \label{fig:completion}
%   \vspace{-3mm}
% \end{figure}


% \noindent{\textbf{Setup.}} Image regression is a common task used for evaluating INRs' capacity of representing a signal~\citep{tancik2021learned,sitzmann2020implicit}. 
% We employ two real-world image datasets as used in previous works~\citep{chen2022transformers,tancik2021learned,gu2023generalizable}. The CelebA dataset~\citep{liu2015deep} encompasses approximately 202,000 images of celebrities, partitioned into training (162,000 images), validation (20,000 images), and test (20,000 images) sets. The Imagenette dataset~\citep{imagenette}, a curated subset comprising 10 classes from the 1,000 classes in ImageNet~\citep{deng2009imagenet}, consists of roughly 9,000 training images and 4,000 testing images. In order to compare with previous methods, we conduct image regression experiments. The context set is an image and the task is to learn an implicit function that regresses the image pixels well in terms of PSNR.
% %\str{What is the point of image regresssion when the image itself is used as context? Why not just copy the context??}
% %\str{The following is a bit strante, is this still image regression? Why not have a separate section?}
% %\str{Is Image Regression the standard name for this task?}


% \noindent{\textbf{Implementation Details.}} 
% Following TransINR~\citep{chen2022transformers}, we resize each image into $178\times 178$, and use patch size 9 for the tokenizer. The self-attention module remains the same as the one in the NeRF experiments (Sec. \ref{sec:nerf-results}). For the Gaussian bases, we predict the 2D Gaussians instead of the 3D. 
% The hierarchical latent variables are inferred in image-level and pixel-level. 





% %The self-attention and global variable remain the same as the one in the NeRF experiments. %We do not use the pixel variable modulation for the computation concern. However, our method still has competitive performance.   

% % \begin{table}[t]
% % \centering
% % \vspace{-6mm}
% % \caption{Quantitative results of image regression.}
% % \label{tab:image-regression}
% % \begin{tabular}{lcc}
% % \toprule
% %              & CelebA & Imagenette \\ \midrule
% % Learned Init \citep{tancik2021learned} & 30.37  & 27.07       \\
% % TransINR~\citep{chen2022transformers}         & 31.96  & 29.01       \\ 
% % \hline
% % \method{} (Ours)         & \textbf{33.41}  & \textbf{29.82}      \\ 
% % \bottomrule
% % \end{tabular}
% % \end{table}



% \noindent{\textbf{Results.}} The quantitative comparison of \method{} for representing the 2D image signals is presented in Table~\ref{tab:image-regression}. \method{} outperforms the baseline methods on both CelebA and Imagenette datasets significantly, showing better generalization ability and representation capacity than baselines. 
% %Note that the Imagenette is a more diverse dataset than the CelebA. The better performance shows that. 
% Fig.~\ref{fig:visualization-image-regression} shows the ability of \method{} to recover the high-frequency details for image regression.
% %regress the image closely to the ground truth, indicating the capability of capturing the detailed texture information. 


% \noindent {\textbf{Image Completion Visualization.}} We also conduct experiments of \method{} on image completion (also called image inpainting), which is a more challenging variant of image regression. Essentially, only part of the pixels are given as context, while the INR functions are required to complete the full image. Visualizations in Fig.~\ref{fig:completion} demonstrate the generalization ability of our method to recover realistic images with fine details based on very limited context ($10 \% - 20\%$ pixels). %, 




% \begin{figure}[t!]
%   \centering
%   \includegraphics[width=0.99\textwidth]{Figures/image-completion.pdf} % Adjust the size and filename as needed
%   \vspace{-3mm}
%   \caption{\textbf{Image completion visualization} on CelebA using $10\%$ (left) and $20\%$ (right) context.}
%   \label{fig:completion}
%   \vspace{-3mm}
% \end{figure}



% \begin{figure}[t!]
%   \centering
%   \includegraphics[width=1\textwidth]{Figures/image-regression-basis.pdf} % Adjust the size and filename as needed
%   \caption{\textbf{Visualization of geometric bases (Gaussian)} on the context image, which reveals the structure of the object.} % Caption for the figure
%   \label{fig:visualization}
%   \vspace{-5mm}
% \end{figure}


% \noindent{\textbf{Visualization of Geometric Bases.}}
% Moreover, we also visualize the learned Gaussian bases on the image regression task. As shown in Fig. \ref{fig:visualization}, the bases are more concentrated on the objects and complex backgrounds in the image, while sparse on the simple complex. The visualizations indicate that the geometric bases do encode structure information from the context data.





% \documentclass[twoside]{article}

% \usepackage{aistats2025}
% If your paper is accepted, change the options for the package
% aistats2025 as follows:
%
%\usepackage[accepted]{aistats2025}
%
% This option will print headings for the title of your paper and
% headings for the authors names, plus a copyright note at the end of
% the first column of the first page.

% If you set papersize explicitly, activate the following three lines:
%\special{papersize = 8.5in, 11in}
%\setlength{\pdfpageheight}{11in}
%\setlength{\pdfpagewidth}{8.5in}

% If you use natbib package, activate the following three lines:
%\usepackage[round]{natbib}
%\renewcommand{\bibname}{References}
%\renewcommand{\bibsection}{\subsubsection*{\bibname}}

% If you use BibTeX in apalike style, activate the following line:
%\bibliographystyle{apalike}

% \begin{document}

% If your paper is accepted and the title of your paper is very long,
% the style will print as headings an error message. Use the following
% command to supply a shorter title of your paper so that it can be
% used as headings.
%
%\runningtitle{I use this title instead because the last one was very long}

% If your paper is accepted and the number of authors is large, the
% style will print as headings an error message. Use the following
% command to supply a shorter version of the authors names so that
% they can be used as headings (for example, use only the surnames)
%
%\runningauthor{Surname 1, Surname 2, Surname 3, ...., Surname n}

% Supplementary material: To improve readability, you must use a single-column format for the supplementary material.
\onecolumn
\appendix
\aistatstitle{From Deep Additive Kernel Learning to Last-Layer \\ Bayesian Neural Networks via Induced Prior Approximation: \\
Supplementary Materials}

\section{SPARSE CHOLESKY DECOMPOSITION}
\label{sec:sparse chol decompose}
In this section, we present the algorithm for constructing the induced grids $\mathbf{U}$ as defined in \cref{eq:GPlayer} by using sorted dyadic points, and obtaining the sparse Choleksy decomposition of the Laplace kernel in one dimension, as proposed in \citep{ding2024sparse}.

A set of one-dimensional level-$L$ dyadic points $\Xv_L$ in increasing order over the interval $[0,1]$ is defined as:
\begin{align}
    \Xv_{L}:= \left\{ \frac{1}{2^{L}}, \frac{2}{2^{L}}, \frac{3}{2^{L}}, \ldots, \frac{2^{L}-1}{2^{L}} \right\}.
\end{align}
However, this increasing order does not yield a sparse representation of the Markov kernel $k(\cdot,\cdot)$ on the points $\Xv_L$, i.e., Cholesky decomposition of the covariance matrix $k(\Xv_L, \Xv_L)$ is not sparse. To achieve a sparse hierarchical expansion, we first sort the dyadic points $\Xv_L$ according to their levels.

\paragraph{Sorted Dyadic Points}
For level-$\ell$ dyadic points $\Xv_{\ell}$ where $ \ell=1,\ldots,L$, we first define the set $\rho(\ell)$ consisting of odd numbers as follows:
\begin{align}
    \rho(\ell) = \left\{ 1,3,5,\ldots,2^{\ell}-1 \right\}.
\end{align}
Next, we define the sorted incremental set $\Dv_{\ell}$ (with $\Xv_{0}:= \varnothing$) as:
\begin{align}
    \Dv_{\ell} = 
    \left\{ \frac{i}{2^{\ell}}: i\in \rho(\ell) \right\} = \Xv_{\ell} - \Xv_{\ell-1}, \quad  \ell=1,\ldots L.
\end{align}
Thus, the level-$L$ dyadic points $\Xv_L$ can be decomposed into disjoint incremental sets $\{ \Dv_{\ell} \}_{\ell=1}^{L}$:
\begin{align}
    \Xv_{L} = \cup_{\ell=1}^{L} \Dv_{\ell}, \quad \Dv_{i} \cap \Dv_{j} = \varnothing \text{ for $i\neq j$}.
\end{align}
Therefore, we can define the sorted level-$L$ dyadic points using these incremental sets as:
\begin{align}\label{eq:sorted dyadic}
    \Xv_{L}^{\text{sort}}:= \left\{ \Dv_1,\Dv_2, \ldots, \Dv_{L} \right\} 
    = \left\{ \frac{i \in \rho(\ell) }{2^{\ell}}, \ell=1,\ldots,L \right\}.
\end{align}
For example, the sorted level-3 dyadic points are given by:
\begin{align}
    \Xv_{3}^{\text{sort}} 
    = \bigg\{ 
    \begingroup
        \color{blue}
        \underbracket{
            \color{black}
            \frac{1}{2^1}
        }_{\color{blue}
            \Dv_1
        }
    \endgroup
    , 
    \begingroup
        \color{blue}
        \underbracket{
            \color{black}
            \frac{1}{2^2}, \frac{3}{2^2}
        }_{\color{blue}
            \Dv_2
        }
    \endgroup
    ,
    \begingroup
        \color{blue}
        \underbracket{
            \color{black}
            \frac{1}{2^3}, \frac{3}{2^3}, \frac{5}{2^3}, \frac{7}{2^3}
        }_{\color{blue}
            \Dv_3
        }
    \endgroup
     \bigg\}.
\end{align}

\paragraph{Algorithm}
We now present the algorithm for computing the inverse of the upper triangular Cholesky factor $[ \Lv_{\Xv_{L}^{\text{sort}}}^{\top} ]^{-1}$ of the covariance matrix $k(\Xv_{L}^{\text{sort}}, \Xv_{L}^{\text{sort}})$ in \Cref{alg:cholesky}, where $\Lv_{\Xv_{L}^{\text{sort}}} \Lv_{\Xv_{L}^{\text{sort}}}^{\top} = k(\Xv_{L}^{\text{sort}}, \Xv_{L}^{\text{sort}})$.. The corresponding proof can be found in \citep{ding2024sparse}. The output of \Cref{alg:cholesky} is a sparse matrix with $\Oc(3 \cdot (2^{L}-1))$ nonzero entries. Since each iteration of the for-loop only requires solving a $3 \times 3$ linear system, which costs $\Oc(3^3)$ time, the total computational complexity of \Cref{alg:cholesky} is $\Oc(2^L-1)$. This implies that the complexity of computing $\left[ \Lv_{\Uv}^{\top} \right]^{-1}$ in \cref{eq:GPlayer} is $\Oc(M)$ when $\Uv$, the induced grid of size $M$, consists of sorted dyadic points as defined in \cref{eq:sorted dyadic}.

\begin{algorithm}[hbt!]
\caption{Computation of the inverse Cholesky factor for the Markov kernel $k(\cdot, \cdot)$ on sorted one-dimensional level-$L$ dyadic points $\Xv_L^{\text{sort}}$.}
\label{alg:cholesky}
\setstretch{0.99} % set the line spacing to 0.99
\begin{algorithmic}[1]
    \STATE {\bfseries Input:} Markov kernel $k(\cdot,\cdot)$, sorted level-$L$ dyadic points $\Xv_{L}^{\text{sort}}$
    \STATE {\bfseries Output:} inverse of the upper triangular Cholesky factor $\Rv:= [ \Lv_{\Xv_{L}^{\text{sort}}}^{\top} ]^{-1}$, s.t. $\Lv_{\Xv_{L}^{\text{sort}}} \Lv_{\Xv_{L}^{\text{sort}}}^{\top} = k(\Xv_{L}^{\text{sort}}, \Xv_{L}^{\text{sort}})$
    \STATE Initialize $\Rv \leftarrow \text{zeros($2^L-1$,$2^L-1$)}$;
    \STATE Define $k(\pm \infty, \cdot) = k(\cdot, \pm \infty) = 0$;
    \FOR{$\ell=1$ {\bfseries to} $L$}
        \FOR{$i \in \rho(\ell)=\{1,3,\ldots,2^{\ell}-1\}$}
            \STATE $x_{\text{mid}} := \frac{i}{2^{\ell}}$;\quad
            $x_{\text{left}}:=\frac{i-1}{2^{\ell}}$ {\bfseries if} $i>1$ {\bfseries else} $-\infty$;\quad
            $x_{\text{right}}:=\frac{i+1}{2^{\ell}}$ {\bfseries if} $i<2^{\ell}-1$ {\bfseries else} $+\infty$;
            \STATE Get $i_{\text{mid}}$, $i_{\text{left}}$, $i_{\text{right}}$, the indices of the points $x_{\text{mid}}$, $x_{\text{left}}$, $x_{\text{right}}$ in the sorted set $\Xv_{L}^{\text{sort}}$ respectively;
            \STATE Get the coefficients $c_1$, $c_2$, $c_3$ by solving the following linear system:
            \begin{align}
                \begin{bmatrix}
                     & k(x_{\text{left}}, x_{\text{left}})
                     & k(x_{\text{left}}, x_{\text{mid}})
                     & k(x_{\text{left}}, x_{\text{right}}) \\
                     & k(x_{\text{mid}}, x_{\text{left}})
                     & k(x_{\text{mid}}, x_{\text{mid}})
                     & k(x_{\text{mid}}, x_{\text{right}}) \\
                     & k(x_{\text{right}}, x_{\text{left}})
                     & k(x_{\text{right}}, x_{\text{mid}})
                    &k(x_{\text{right}}, x_{\text{right}})
                \end{bmatrix}
                \begin{bmatrix}
                    c1\\
                    c2\\
                    c3
                \end{bmatrix}=
                \begin{bmatrix}
                    0\\
                    1\\
                    0
                \end{bmatrix}.
            \end{align}
            \STATE $[c_1,c_2,c_3] := [c_1,c_2,c_3] / \sqrt{c_2}$;
            \STATE {\bfseries if} $x_{\text{left}} \neq - \infty$, 
            {\bfseries then} $\Rv[i_{\text{left}} ,i_{\text{mid}}] = c_1$; \quad
            {\bfseries if} $x_{\text{right}} \neq + \infty$, 
            {\bfseries then} $\Rv[i_{\text{right}} ,i_{\text{mid}}] = c_3$;
            \STATE $\Rv[i_{\text{mid}} ,i_{\text{mid}}] = c_2$;
        \ENDFOR
    \ENDFOR
\end{algorithmic}
\end{algorithm}


\section{REPARAMETERIZATION OF KERNEL LENGTHSCALES}
\label{sec:theo}
Considering the additive Laplace kernel with fixed lengthscale $\tilde{\theta}$ for all base kernels, applying linear projections $\left\{ \wv_{p}^{\top}\xv \right\}_{p=1}^{P}$ on inputs $\xv\in \Rb^D$ will give:
\begin{align}
    &\sum_{p=1}^{P}\sigma^2_p k_p\left( \wv^{\top}_{p}\xv,\wv^{\top}_{p}\xv^{\prime} \right)\nonumber \\
    = & \sum_{p=1}^{P} \sigma^2_p\exp \left( -  \frac{\sum_{d=1}^{D} \left| w_{p,d}\left( x_{d}-x_{d}^{\prime} \right) \right|}{\tilde{\theta}} \right)\nonumber \\
    = & \sum_{p=1}^{P} \prod_{d=1}^{D} \sigma^2_p\exp \left( - \frac{\left| x_{d}-x_{d}^{\prime} \right|}{\tilde{\theta} / \left| w_{p,d}\right| } \right)\nonumber \\
    = & \sum_{p=1}^{P} \prod_{d=1}^{D} \sigma^2_p\exp \left( - \frac{\left| x_{d}-x_{d}^{\prime} \right|}{\theta_{p,d}} \right),
\end{align}
This still leads to an additive Laplace kernel but with adaptive lengthscale $\theta_{p,d}$ for base kernels. The resulting kernel also retains \emph{sparse} Cholesky decomposition by the properties of Markov kernels so that the complexity of inference is $\Oc(M)$.

\section{INFERENCE OF PREDICTIVE DISTRIBUTION}
\label{sec:uq of inference}
Given an input $\xv \in \Rb^D$, the prediction of the DAK model can be written in the following equation according to \cref{eq:DAK prediction}: 
\begin{align}
    \tilde{f}_{\xv}
    &= \sum_{p=1}^{P}
    \sigma_p \Big(
        \phi(h_{\psi}^{[p]}(\xv)) \zv_p
    \Big) + \mu \nonumber\\
    &= \sum_{p=1}^{P}
    \sigma_p \Big(
        \bm{\phi}_{p}^{\top} \zv_p
    \Big) + \mu,
\end{align}
where $\bm{\phi}_{p}^{\top}:=\phi(h_{\psi}^{[p]}(\xv)) \in \Rb^{1 \times M}$
% , $\mu_p:=\mu_p(h_{\psi}^{[p]}(\xv)) \in \Rb$
. We assume the variational distribution over the independent Gaussian weights $\zv_p \sim \Nc(\bm{m}_{\zv_p}, \Sv_{\zv_p})$ and the bias $\mu \sim \Nc(m_{\mu}, \sigma_{\mu}^2)$. Then it's straighforward to deduce that
\begin{align}
    \bm{\phi}_{p}^{\top} \zv_p + \mu 
    &\sim
    \Nc\left(
    \bm{\phi}_{p}^{\top} \bm{m}_{\zv_p} + m_{\mu},\hspace{0.2em}
    \bm{\phi}_{p}^{\top} \Sv_{\zv_p} \bm{\phi}_{p} + \sigma_{\mu}^2
    \right), \\
    \sigma_p \left(
    \bm{\phi}_{p}^{\top} \zv_p 
    \right) + \mu
    & \sim
    \Nc\left(
    \sigma_p ( \bm{\phi}_{p}^{\top} \bm{m}_{\zv_p} )+ m_{\mu} ,\hspace{0.2em}
    \sigma_p^2( \bm{\phi}_{p}^{\top} \Sv_{\zv_p} \bm{\phi}_{p}) + \sigma_{\mu}^2
    \right), \\
    \tilde{f}_{\xv} = 
    \sum_{p=1}^{P}
    \sigma_p \left(
    \bm{\phi}_{p}^{\top} \zv_p
    \right) + \mu
    & \sim
    \Nc\left(
    \sum_{p=1}^{P}
    \sigma_p ( \bm{\phi}_{p}^{\top} \bm{m}_{\zv_p}) + m_{\mu} ,\hspace{0.2em}
    \sum_{p=1}^{P}
    \sigma_p^2( \bm{\phi}_{p}^{\top} \Sv_{\zv_p} \bm{\phi}_{p} ) + \sigma_{\mu}^2
    \right).
\end{align}
Therefore, we obtain the predictive distribution of the $\tilde{f}(\xv)$ at the point $\xv \in \Rb^D$ and its mean and variance are given by:
\begin{subequations}
\label{eq:dak inference closed form}
\begin{align}
    \Eb\left[ \tilde{f}_{\xv} \right]
        = \sum_{p=1}^{P}
        \sigma_p ( \bm{\phi}_{p}^{\top} \bm{m}_{\zv_p}) + m_{\mu},
\end{align}
\begin{align}
    \text{Var}\left[ \tilde{f}_{\xv} \right]
        =\sum_{p=1}^{P}
        \sigma_p^2( \bm{\phi}_{p}^{\top} \Sv_{\zv_p} \bm{\phi}_{p}) + \sigma_{\mu}^2.
\end{align}
\end{subequations}
% \begin{subequations}
% \label{eq:dak inference closed form}
%     \begin{align}
%         \Eb\left[ \tilde{f}(\xv) \right]
%         = \sum_{p=1}^{P}
%         \sigma_p ( \bm{\phi}_{p}^{\top} \bm{m}_{\zv_p} + m_{\mu_p} ),
%     \end{align}
%     \begin{align}
%         \text{Var}\left[ \tilde{f}(\xv) \right]
%         =\sum_{p=1}^{P}
%         \sigma_p^2( \bm{\phi}_{p}^{\top} \Sigma_{\zv_p} \bm{\phi}_{p} + \sigma_{\mu_p}^2).
%     \end{align}
% \end{subequations}


\section{TRAINING OF VARIATIONAL INFERENCE}
\label{sec:training}
Given the dataset $\mathcal{D}=\{ \Xv, \yv \}$ where $\Xv:=\{ \xv_i \}_{i=1}^N$, $\yv=(y_1,\ldots,y_N)^{\top}$, $\xv_i \in \Rb^D$, $y_i\in\Rb$, the prediction $\tilde{f}_{\Xv}\in \Rb^N$ of DAK is given by all the parameters $\bm{\theta}=\left\{ \psi, \bm{\sigma} \right\}$, $\bm{\eta}=\left\{ \{ \mv_{\zv_{p}},\Sv_{\zv_{p}}\}_{p=1}^{P} , \{m_{\mu},\sigma_{\mu} \} \right\}$ according to \cref{eq:DAK prediction}:
\begin{align}
    \tilde{f}_{\Xv}:= \tilde{f}(\Xv; \bm{\theta}, \bm{\eta})
    = \sum_{p=1}^{P}
    \sigma_p \Big(
        \phi(h_{\psi}^{[p]}(\Xv)) \zv_p
    \Big) + \mu,
\end{align}
where $\zv_{p} \sim \mathcal{N} (\bm{m}_{\zv_p} ,\Sv_{\zv_p})$, $p=1,\ldots,P$, and $\mu \sim \mathcal{N} ( m_{\mu},\sigma^2_{\mu} )$ are variational variables $\Theta_{\text{var}}$ parameterized by $\bm{\eta}$. The variational distribution is denoted by $q_{\bm{\eta}}(\Theta_{\text{var}})= q(\mu)\prod_{p=1}^{P} q(\zv_{p}) = \Nc ( m_{\mu} ,\sigma_{\mu}^2 )\prod_{p=1}^{P} 
\Nc ( \bm{m}_{\zv_p} ,\Sv_{\zv_p} )$, and the variational prior is denoted by $p(\Theta_{\text{var}})$.

We consider the KL divergence between $q_{\bm{\eta}}(\Theta_{\text{var}})$ and the true posterior $p(\Theta_{\text{var}}\vert \yv, \Xv, \bm{\theta})$:
\begin{align}
& \qquad \text{KL} \left[ q_{\bm{\eta}}(\Theta_{\text{var}}) \| p(\Theta_{\text{var}} \vert \yv,\Xv, \bm{\theta} ) \right] \nonumber \\
= & \int q_{\bm{\eta}}(\Theta_{\text{var}} )\log \frac{q_{\bm{\eta}}(\Theta_{\text{var}} )}{p(\Theta_{\text{var}} \vert \yv,\Xv,\bm{\theta} )} d\Theta_{\text{var}} \nonumber \\
= & \int q_{\bm{\eta}}(\Theta_{\text{var}} )\log \frac{q_{\bm{\eta}}(\Theta_{\text{var}} )p(\yv \vert \Xv,\bm{\theta})}{p(\yv \vert \Xv,\bm{\theta} ,\Theta_{\text{var}} )p(\Theta_{\text{var}} )} d\Theta_{\text{var}} \nonumber \\
= & \int q_{\bm{\eta}}(\Theta_{\text{var}} )\log \frac{q_{\bm{\eta}}(\Theta_{\text{var}} )}{p(\Theta_{\text{var}} )} d\Theta_{\text{var}} -\int q_{\bm{\eta}}(\Theta_{\text{var}} )\log p(\yv \vert \tilde{f}_{\Xv} )d\Theta_{\text{var}} +\log p(\yv\vert \Xv,\bm{\theta}).
\end{align}
Using the fact that $\text{KL}[\cdot \| \cdot] \geq 0$, we have
\begin{align}
\label{eq:variational lower bound}
    \log p(\yv\vert \Xv,\bm{\theta}) & \geq \int q_{\bm{\eta}}(\Theta_{\text{var}} )\log p(\yv \vert \tilde{f}_{\Xv} )d\Theta_{\text{var}} - \text{KL} \left[ q_{\bm{\eta}}(\Theta_{\text{var}} ) \| p(\Theta_{\text{var}}) \right] \nonumber \\
    & = \Eb_{q_{\bm{\eta}}(\Theta_{\text{var}} )} \left[ \log p(\yv \vert \tilde{f}_{\Xv} ) \right] - \text{KL} \left[ q_{\bm{\eta}}(\Theta_{\text{var}} ) \| p(\Theta_{\text{var}}) \right].
\end{align}

\paragraph{Full-training.}
Firstly, we present the joint training of $\bm{\theta}$ and $\bm{\eta}$. The most common approach optimizes the marginal log-likelihood (the left-hand side of \cref{eq:variational lower bound}):
\begin{align}
    \bm{\theta}^{\ast} &=\argmax_{\bm{\theta}} \log p(\yv\vert \Xv,\bm{\theta} ) \\
    &= \argmax_{\bm{\theta}} \log \int p\left( y\vert X,\bm{\theta},\Theta_{\text{var}} \right) p(\Theta_{\text{var}})d\Theta_{\text{var}},
\end{align}
which involves intractable integral in some tasks such as classification. Instead, we optimize the variational lower bound (the right-hand side of \cref{eq:variational lower bound}):
\begin{align}
    \Theta^{\ast} := \argmax_{\bm{\theta},\bm{\eta}} \mathcal{L}(\bm{\theta},\bm{\eta}) =\argmax_{\bm{\theta},\bm{\eta}}\left\{ E_{q_{\bm{\eta}}(\Theta_{\text{var}} )}\left[ \log p(\yv|\tilde{f}_{\Xv} ) \right] -\text{KL} \left[ q_{\bm{\eta}}(\Theta_{\text{var}} )\| p(\Theta_{\text{var}} ) \right] \right\}.
\end{align}

\paragraph{Fine-tuning.}
An alternative training approach is to firstly pre-train the deterministic parameters of feature extractor by standard neural network training, with mean squared error for regression or cross-entropy for classification as the loss function, and then fine-tune the last layer additive GP with fixed features. The objective function is identical to \cref{eq:elbo}, but $\bm{\theta}$ is learned during the pre-training step and is no longer optimized during fine-tuning.


\section{ELBO}%{DERIVATION OF ELBO}
\label{sec:elbo}
\subsection{Assumptions}
Consider the model $y_i = \tilde{f}(\xv_i) + \epsilon_i$ with the i.i.d. noise $\epsilon_i \overset{\text{i.i.d.}}{\sim} \Nc(0, \sigma_{f}^2)$ and $\tilde{f} : \Rb^D \rightarrow \Rb$ is defined in \cref{eq:DAK prediction}. The training dataset is $\mathcal{D} = \{ \Xv, \yv \}$ where $\Xv:=\{ \xv_i \}_{i=1}^N$, $\yv=(y_1,\ldots,y_N)^{\top}$, $\xv_i \in \Rb^D$, $y_i\in\Rb$. $\Theta_{\text{var}}:= \{ \mu ,\{ \zv_{p}\}_{p=1}^{P} \}$ are the variational random variables consisting of Gaussian weights and bias of $P$ units, $\psi$ are the parameters of the NN, $\bm{\sigma}:=(\sigma_1, \ldots, \sigma_p)^{\top}$ are the scale parameters of base GP layers. The variational distributions are $q(\mu)=\Nc(m_{\mu}, \sigma_{\mu}^2)$, $q(\zv_p)=\Nc(\bm{m}_{\zv_p}, \Sv_{\zv_p})$ and the variational priors are $p(\mu)=\Nc(\check{m}_{\mu} ,\check{\sigma}^2_{\mu})$, $p(\zv_p)=\Nc(\check{\bm{m}}_{\zv_p} ,\check{\Sv}_{\zv_p})$. Note that $\Sv_{\zv_p}\in\Rb^{M \times M}$ is a diagonal covariance matrix due to the independence of $\zv_p$, $M$ is the number of inducing points $\Uv$ defined in \cref{eq:GPlayer}, and $\bm{m}_{\zv_p} \in \Rb^M$, $m_{\mu} \in \Rb$, $\sigma_{\mu}^2 \in \Rb$. We derive the ELBO in VI to learn the preditive posterior over the variational variables $\Theta_{\text{var}}:= \{ \mu ,\{ \zv_{p}\}_{p=1}^{P} \}$ parameterized by $\bm{\eta}:=\left\{ \{ \mv_{\zv_{p}},\Sv_{\zv_{p}}\}_{p=1}^{P} , \{m_{\mu},\sigma_{\mu} \} \right\}$, and optimize the deterministic parameters $\bm{\theta}:=\{\psi, \bm{\sigma}\}$.

\subsection{Expected Log Likelihood}
\paragraph{Closed Form}
The \emph{expected log likelihood}, which is the first term in ELBO defined in \cref{eq:elbo}, is given by 
\begin{align}
    {\Eb}_{q_{\bm{\eta}}(\Theta_{\text{var}})} \left[ \log \text{Pr} (\yv \vert \tilde{f}_{\Xv} ) \right]
    &= {\Eb}_{q_{\bm{\eta}}(\Theta_{\text{var}})} \left[ 
    \log \prod_{i=1}^{N} 
    p (y_i \vert \tilde{f}_{\xv_i} )
    \right] \nonumber\\
    &= \sum_{i=1}^{N} 
    {\Eb}_{q_{\bm{\eta}}(\Theta_{\text{var}})} \left[ 
    \log
    p (y_i \vert \tilde{f}_{\xv_i} )
    \right] \nonumber\\
    &= \sum_{i=1}^{N} 
    {\Eb}_{q_{\bm{\eta}}(\Theta_{\text{var}})} \left[ 
    \log
    \Nc( \tilde{f}_i,\hspace{0.2em} \sigma_{f}^2 )
    \right] \nonumber\\
    &= \sum_{i=1}^{N} 
    {\Eb}_{q_{\bm{\eta}}(\Theta_{\text{var}})} \left[ 
    \log \left(
    (2\pi \sigma_{f}^2)^{-\frac{1}{2}}
    \exp\left\{  
        -\frac{ (y_i - \tilde{f}_i)^2 }{2 \sigma_{f}^2}
    \right\}
    \right)
    \right] \nonumber\\
    &= \sum_{i=1}^{N} 
    {\Eb}_{q_{\bm{\eta}}(\Theta_{\text{var}})} \left[
    -\frac{1}{2} \log(2\pi) 
    - \frac{1}{2}\log(\sigma_{f}^2)
    - \frac{1}{2 \sigma_{f}^2}
    (y_i - \tilde{f}_i)^2
    \right] \nonumber\\
    &= - \frac{N}{2} \log(2\pi)
    - \frac{N}{2} \log(\sigma_{f}^2)
    - \frac{1}{2 \sigma_{f}^2}
    \sum_{i=1}^{N}
    {\Eb}_{q_{\bm{\eta}}(\Theta_{\text{var}})} \left[
    (y_i - \tilde{f}_i)^2
    \right] \nonumber\\
    &= - \frac{N}{2} \log(2\pi)
    - \frac{N}{2} \log(\sigma_{f}^2)
    - \frac{1}{2 \sigma_{f}^2}
    \sum_{i=1}^{N} \left(
    \left({\Eb}_{q(\Theta_{\text{var}})} \left[
    (y_i - \tilde{f}_i)
    \right] \right)^2
    + \text{Var}_{q(\Theta_{\text{var}})} \left[
    (y_i - \tilde{f}_i)
    \right]
    \right) \label{eq:evidence halfway},
\end{align}
where
\begin{align}
    \tilde{f}_i
    % \mu_{\tilde{f}_i} &:= \tilde{f}(\xv_i;\Theta_{\text{var}}, \Theta_{\text{det}} ) \nonumber\\
    &= \sum_{p=1}^{P} \sigma_p \Big(
    \begingroup
        \color{blue}
        \underbracket{
            \color{black}
            \phi(h_{\psi}^{[p]}(\xv_i))
        }_{\color{blue}
            :=\bm{\phi}_{i,p}^{\top} \in \Rb^{1 \times M}
        }
    \endgroup
    \zv_p
    \Big)
    + \mu
    % \begingroup
    %     \color{blue}
    %     \underbracket{
    %         \color{black}
    %         \mu_{p}(h_{\psi}^{[p]}(\xv_i))
    %     }_{\color{blue}
    %         :=\mu_{i,p} \in \Rb
    %     }
    % \endgroup 
    \nonumber\\
    &= \sum_{p=1}^{P} \sigma_p \left(
    \bm{\phi}_{i,p}^{\top} \zv_p 
    \right) + \mu.
\end{align}
Recall that the variational assumptions $q(\zv_p)=\Nc(\bm{m}_{\zv_p}, \Sv_{\zv_p})$ and $q(\mu)=\Nc(m_{\mu}, \sigma_{\mu}^2)$, we can infer that
\begin{align}
    \bm{\phi}_{i,p}^{\top} \zv_p + \mu 
    &\sim
    \Nc\left(
    \bm{\phi}_{i,p}^{\top} \bm{m}_{\zv_p} + m_{\mu},\hspace{0.2em}
    \bm{\phi}_{i,p}^{\top} \Sv_{\zv_p} \bm{\phi}_{i,p} + \sigma_{\mu}^2
    \right), \\
    \sigma_p \left(
    \bm{\phi}_{i,p}^{\top} \zv_p 
    \right) + \mu
    & \sim
    \Nc\left(
    \sigma_p ( \bm{\phi}_{i,p}^{\top} \bm{m}_{\zv_p} ) + m_{\mu},\hspace{0.2em}
    \sigma_p^2( \bm{\phi}_{i,p}^{\top} \Sv_{\zv_p} \bm{\phi}_{i,p} ) + \sigma_{\mu}^2
    \right), \\
    \tilde{f}_i = 
    \sum_{p=1}^{P}
    \sigma_p \left(
    \bm{\phi}_{i,p}^{\top} \zv_p 
    \right)+ \mu
    & \sim
    \Nc\left(
    \sum_{p=1}^{P}
    \sigma_p ( \bm{\phi}_{i,p}^{\top} \bm{m}_{\zv_p} )+ m_{\mu},\hspace{0.2em}
    \sum_{p=1}^{P}
    \sigma_p^2( \bm{\phi}_{i,p}^{\top} \Sv_{\zv_p} \bm{\phi}_{i,p} ) + \sigma_{\mu}^2
    \right), \\
    y_i - \tilde{f}_i
    & \sim 
    \Nc\left(
    y_i - 
    \sum_{p=1}^{P}
    \sigma_p ( \bm{\phi}_{i,p}^{\top} \bm{m}_{\zv_p} ) -m_{\mu},\hspace{0.2em}
    \sum_{p=1}^{P}
    \sigma_p^2( \bm{\phi}_{i,p}^{\top} \Sv_{\zv_p} \bm{\phi}_{i,p} ) + \sigma_{\mu}^2
    \right).
\end{align}
Therefore, 
\begin{subequations}\label{eq:exp and var in evidence}
    \begin{align}
        \left({\Eb}_{q(\Theta_{\text{var}})} \left[
        (y_i - \tilde{f}_i)
        \right] \right)^2
        = \left(
         y_i - 
        \sum_{p=1}^{P}
        \sigma_p ( \bm{\phi}_{i,p}^{\top} \bm{m}_{\zv_p} ) -m_{\mu}
        \right)^2,
    \end{align}
    \begin{align}
        \text{Var}_{q(\Theta_{\text{var}})}
        \left[
        (y_i - \tilde{f}_i)
        \right]
        = \sum_{p=1}^{P}
        \sigma_p^2( \bm{\phi}_{i,p}^{\top} \Sv_{\zv_p} \bm{\phi}_{i,p} ) + \sigma_{\mu}^2.
    \end{align}
\end{subequations}
By applying \cref{eq:exp and var in evidence} to \cref{eq:evidence halfway}, we derive the analytical formula for the expected evidence, expressed as
\begin{align}
    {\Eb}_{q_{\bm{\eta}}(\Theta_{\text{var}})} \left[ \log \text{Pr} (\yv \vert \tilde{f}_{\Xv} ) \right]
    &= - \frac{N}{2} \log(2\pi)
    - \frac{N}{2} \log(\sigma_{f}^2) \nonumber\\
    &- \frac{1}{2 \sigma_{f}^2}
    \sum_{i=1}^{N} \left(
        \Big(
         y_i - 
        \sum_{p=1}^{P}
        \sigma_p ( \bm{\phi}_{i,p}^{\top} \bm{m}_{\zv_p} ) -m_{\mu}
        \Big)^2
        + \sum_{p=1}^{P}
        \sigma_p^2( \bm{\phi}_{i,p}^{\top} \Sv_{\zv_p} \bm{\phi}_{i,p} )+ \sigma_{\mu}^2
    \right). \label{eq:evidence final}
\end{align}

\paragraph{Monte Carlo Approximation}
For comparison, we provide the equation for computing the Monte Carlo estimate of the ELBO in the paragraph that follows.
\begin{align}
    {\Eb}_{q_{\bm{\eta}}(\Theta_{\text{var}})} \left[ \log \text{Pr} (\yv \vert \tilde{f}_{\Xv} ) \right]
    % &= {\Eb}_{q(\Theta)} \left[ 
    % \log \prod_{i=1}^{N} 
    % p (y_i \vert \xv_i,\Theta, \psi, \bm{\sigma})
    % \right] \nonumber\\
    &= \sum_{i=1}^{N} 
    {\Eb}_{q_{\bm{\eta}}(\Theta_{\text{var}} )} \left[ 
    \log
    p (y_i \vert \tilde{f}_{\xv_i} )
    \right] \nonumber\\
    & \approx \sum_{i=1}^{N}
    \frac{1}{S}
     \sum_{s=1}^{S}
    \log
    p (y_i \vert \xv_i,\tilde{\Theta}^{(s)}_{\text{var}}, \bm{\theta} ) \nonumber\\
    &= \frac{1}{S} \sum_{i=1}^{N} 
    \sum_{s=1}^{S} 
    \log
    \Nc(y_i \left\vert\right. \tilde{f}_{i}^{(s)},\hspace{0.2em} \sigma_{f}^2 )
    \nonumber\\
    &= \frac{1}{S} \sum_{i=1}^{N} 
    \sum_{s=1}^{S} 
    \log \left(
    (2\pi \sigma_{f}^2)^{-\frac{1}{2}}
    \exp\left\{  
        -\frac{ (y_i - \tilde{f}_{i}^{(s)})^2 }{2 \sigma_{f}^2}
    \right\}
    \right)
    \nonumber\\
    &= \frac{1}{S} \sum_{i=1}^{N} 
    \sum_{s=1}^{S} \left(
    -\frac{1}{2} \log(2\pi) 
    - \frac{1}{2}\log(\sigma_{f}^2)
    - \frac{1}{2 \sigma_{f}^2}
    (y_i - \tilde{f}_{i}^{(s)})^2
    \right) \nonumber\\
    &= - \frac{N}{2} \log(2\pi)
    - \frac{N}{2} \log(\sigma_{f}^2)
    - \frac{1}{2 \sigma_{f}^2}
    \sum_{i=1}^{N}
    \frac{1}{S} \sum_{s=1}^{S}
    (y_i - \tilde{f}_{i}^{(s)})^2, \label{eq:evidence halfway mc approx}
\end{align}
where $S$ is the number of Monte Carlo samples, $\{  \tilde{\mu}^{(s)} ,\{ \tilde{\zv}_{p}^{(s)} \}_{p=1}^{P} \} := \tilde{\Theta}^{(s)}_{\text{var}}$ are the $s$-th Monte Carlo samplings over the variational parameters $\Theta_{\text{var}}$ and $\tilde{\Theta}^{(s)}_{\text{var}} \sim q_{\bm{\eta}}(\Theta_{\text{var}})$, $\tilde{f}_{i}^{(s)}$ is given as follows:
\begin{align}
    \tilde{f}_{i}^{(s)} &:= \tilde{f}(\xv_i;\tilde{\Theta}^{(s)}_{\text{var}},\bm{\theta} ) \nonumber\\
    &= \sum_{p=1}^{P} \sigma_p \Big(
    \begingroup
        \color{blue}
        \underbracket{
            \color{black}
            \phi(h_{\psi}^{[p]}(\xv_i))
        }_{\color{blue}
            :=\bm{\phi}_{i,p}^{\top} \in \Rb^{1 \times M}
        }
    \endgroup
    \tilde{\zv}_p^{(s)} 
    \Big) + \tilde{\mu}^{(s)} \nonumber\\
    &= \sum_{p=1}^{P} \sigma_p \left(
    \bm{\phi}_{i,p}^{\top} \tilde{\zv}_p^{(s)} 
    \right)+ \tilde{\mu}^{(s)}. \label{eq:mc approx mean}
\end{align}
Therefore, we plug \cref{eq:mc approx mean} into \cref{eq:evidence halfway mc approx} and get the the Monte Carlo estimate of the ELBO written in the following formula:
\begin{align}
    {\Eb}_{q_{\bm{\eta}}(\Theta_{\text{var}})} \left[ \log \text{Pr} (\yv \vert \tilde{f}_{\Xv} ) \right]
    &\approx
    - \frac{N}{2} \log(2\pi)
    - \frac{N}{2} \log(\sigma_{f}^2)
    - \frac{1}{2 \sigma_{f}^2}
    \sum_{i=1}^{N}
    \frac{1}{S} \sum_{s=1}^{S}
    \Big(y_i - 
    \sum_{p=1}^{P} \sigma_p \left(
    \bm{\phi}_{i,p}^{\top} \tilde{\zv}_p^{(s)} 
    \Big)- \tilde{\mu}^{(s)}
    \right)^2, \label{eq:evidence final mc approx} \\
    \tilde{\zv}_p^{(s)} &\sim \Nc(\bm{m}_{\zv_p}, \Sv_{\zv_p}),\qquad
    \tilde{\mu}^{(s)} \sim \Nc(m_{\mu}, \sigma_{\mu}^2).
\end{align}


\subsection{KL Divergence}
Since we place Gaussian assumptions over the variational parameters $\Theta_{\text{var}}$,  the \emph{KL divergence}, which is the second term in ELBO defined in \cref{eq:elbo}, is then given by
\begin{align}
    \text{KL} \left[ q(\Theta_{\text{var}} ) \| p(\Theta_{\text{var}}) \right]
    &= \text{KL} \left[ q( \mu ,\{ \zv_{p}\}_{p=1}^{P} ) \Vert p( \mu ,\{ \zv_{p}\}_{p=1}^{P}) \right] \nonumber\\
    & =  
    \text{KL} \left[ q(\mu) \Vert p(\mu) \right] 
    + \sum_{p=1}^{P} 
    \text{KL} \left[ q(\zv_{p}) \Vert p(\zv_{p}) \right],
\end{align}

\begin{align}
     \text{KL} \left[ q(\mu) \Vert p(\mu) \right]
     = \frac{1}{2} \left(
     \frac{\sigma_{\mu}^2}{\check{\sigma}_{\mu}^2} 
     + \frac{(m_{\mu} - \check{m}_{\mu})^2}{\check{\sigma}_{\mu}^2} 
     -\log\left( \frac{\sigma_{\mu}^2}{\check{\sigma}_{\mu}^2} \right)
     -1
     \right),
\end{align}

\begin{align}
    \text{KL} \left[ q(\zv_{p}) \Vert p(\zv_{p}) \right]
    = \frac{1}{2} \sum_{i=1}^{M} \left(
     \frac{[\Sv_{\zv_p}]_{ii}}{[\check{\Sv}_{\zv_p}]_{ii}} 
     + \frac{([\bm{m}_{\zv_p}]_{i} - [\check{\bm{m}}_{\zv_p}]_i)^2}{[\check{\Sv}_{\zv_p}]_{ii}}
     -\log\left( 
     \frac{[\Sv_{\zv_p}]_{ii}}{[\check{\Sv}_{\zv_p}]_{ii}}  
     \right)
     -1
     \right),
\end{align}
where $[\Sv_{\zv_p}]_{ii}$ is the $(i,i)$-th element of the diagonal covariance matrix $\Sv_{\zv_p} \in \Rb^{M \times M}$, $[\bm{m}_{\zv_p}]_{i}$ is the $i$-th element of the mean vector $\bm{m}_{\zv_p} \in \Rb^M$, the approximated posteriors are $q(\mu)=\Nc(m_{\mu}, \sigma_{\mu}^2)$, $q(\zv_p)=\Nc(\bm{m}_{\zv_p}, \Sv_{\zv_p})$ and the priors are $p(\mu)=\Nc(\check{m}_{\mu} ,\check{\sigma}^2_{\mu})$, $p(\zv_p)=\Nc(\check{\bm{m}}_{\zv_p} ,\check{\Sv}_{\zv_p})$.

% \subsection{Performance Comparison}
% \label{sec:toy exp compare}
% We compare the perforamce of computing the ELBO in \cref{eq:elbo} by using closed form in \cref{eq:evidence final} and using Monte Carlo approximation in \cref{eq:evidence final mc approx} in a toy example.
% \textcolor{red}{Table or Figure to add if time available}


\subsection{Limitations of the Closed-Form ELBO}

The closed-form ELBO is only applicable to regression problems. In classification, applying the softmax function to $\tilde{f}(\xv;\bm{\theta}, \bm{\eta})$ results in a non-analytic predictive distribution, meaning the ELBO must still be computed via Monte Carlo sampling during training. Similarly, the closed-form expressions for the predictive mean and variance, as provided in \cref{eq:dak inference closed form} in \Cref{sec:uq of inference}, are not applicable to classification but only apply to regression problems.


\section{COMPUTATIONAL COMPLEXITY}
\label{sec:complexity}
In this section, we discuss the computational complexity of various DKL models compared to the proposed DAK method, focusing on the GP layer as the most computationally demanding component. \Cref{tab:complexity supp} shows the computational complexity of our model compared to other state-of-the-art GP and DKL methods.

\begin{table}[ht]
    \caption{Computational complexity of the DKL models for $N$ training points. The reported training complexity is for one iteration. $\hat{M}$ is the number of inducing points in SVGP and KISS-GP, while $M$ is the size of induced grids in DAK, $M < \hat{M}$. $S$ is the number of Monte Carlo samples, $B$ is the size of mini-batch, $D_w$ is the dimension of the NN outputs in DKL, $P$ is the dimension of the outputs after applying linear transformations to the NN outputs in the proposed DAK model. DAK-MC refers to the DAK model using Monte Carlo approximation, while DAK-CF refers to the DAK model using closed-form inference and ELBO.}
    \centering
    \begin{tabular}{lcc}
    \toprule[1pt]
                  & \textbf{Inference}       & \textbf{Training} (per iteration) \\
    \midrule[0.5pt]
    NN + SVGP     & $\Oc(\hat{M}^2 N)$    & $\Oc( S D_w MB + \hat{M}^3)$ \\
    NN + KISS-GP  & $\Oc(D_w \hat{M}^{1+\frac{1}{D_w}})$  & $\Oc(S D_w MB + D_w \hat{M}^{\frac{3}{D_w}})$ \\
    DAK-MC (ours) & $\Oc(SM)$       & $\Oc(SPMB + PM)$   \\
    DAK-CF (ours) & $\Oc(M)$        & $\Oc(PMB + PM)$    \\
    \bottomrule[1pt]
    \end{tabular}
    \label{tab:complexity supp}
\end{table}

\paragraph{Inference Complexity.}
In inference based on induced approximation, computing the multiplication of the inverse of the covariance matrix $k(\Uv, \Uv)$ and a vector takes $\Oc(\hat{M}^2N)$ time for $\hat{M}$ inducing points $\Uv$ and $N$ training points when using SVGP. This cost is reduced by KISS-GP to $\Oc(D \hat{M}^{1+\frac{1}{D}})$ by decomposing the covariance matrix into a Kronecker product of $D$ one-dimensional covariance matrices of the inducing points: $k(\Uv, \Uv) = \bigotimes_{d=1}^{D} k(\Uv^{[d]}, \Uv^{[d]})$. Despite the significant reduction on complexity, it requires inducing points $\Uv$ arranged on a Cartesian grid of size $\hat{M} = \prod_{d=1}^{D} \hat{M}_d$, where $\hat{M}_d$ is the number of inducing points in the $d$-th dimension. In high-dimensional spaces, fixing $\hat{M}$ leads to very small $\hat{M}_d$ per dimension, which can degrade model performance. To address this, we propose the DAK model via sparse finite-rank approximation, which employs an additive Laplace kernel for GPs. The inverse Cholesky factor $\Lv_{\Uv}^{\top}$ for one-dimensional induced grids $\Uv$ of size $M$, where $M < \hat{M}$, as defined in \cref{eq:GPlayer}, is sparse and can be computed in $\Oc(M)$ time.

\paragraph{Training Complexity.}
In training, VI requires computing the ELBO as described in \cref{eq:elbo}, which consists of two terms: the \emph{expected log likelihood} and the \emph{KL divergence} between the variational distributions and priors. 

1) The \emph{expected log likelihood} is usually approximated via Monte Carlo sampling at a cost of $\Oc(S N_{\Theta} N)$, where $S$ is the number of Monte Carlo samples, $N_{\Theta}$ is the total number of variational parameters $\Theta_{\text{var}}$, and $N$ is the number of training points. This complexity can be reduced to $\Oc(S N_{\Theta} B)$ by applying stochastic variational inference with a mini-batch of size $B \ll N$. For DKL models using SVGP and KISS-GP, $\Theta_{\text{var}}$ are inducing variables, and the expectation does not have a closed form, requiring Monte Carlo sampling. In contrast, in the proposed DAK model, $\Theta_{\text{var}}= \{ \{ \zv_{p}\}_{p=1}^{P}, \mu \}$ consists of independent Gaussian weights $\zv_p\in \Rb^M$ and bias $\mu$. This allows us to derive an analytical form for this term, as shown in \cref{eq:evidence final} in \Cref{sec:elbo}, reducing the computational cost to $\Oc(N_{\Theta} B) = \Oc(PM B)$ when using a mini-batch of size $B$.

2) The \emph{KL divergence} between two Gaussian distributions can be computed in closed form. This leads to a linear time complexity of $\Oc(N_{\Theta})$ if the parameters $\Theta_{\text{var}}$ are independent, or cubic time $\Oc(N_{\Theta}^3)$ if they are fully correlated. In SVGP and KISS-GP, $\Theta_{\text{var}}$ represents fully correlated Gaussian distributed inducing variables, so computing the KL divergence takes $\Oc(\hat{M}^3)$ for SVGP. In KISS-GP, this can be reduced to $\Oc(D \hat{M}^{\frac{3}{D}})$ using fast eigendecomposition of Kronecker matrices. In the DAK model, the weights $\{\zv_p\}_{p=1}^{P}$ as defined in \cref{eq:GPlayer} are independent Gaussian random variables, allowing the KL divergence to be computed in $\Oc(N_{\Theta}) = \Oc(PM)$ time, where $P$ is the number of base GP layers.


\section{ADDITIONAL DISCUSSIONS}

Although interpretability is one advantage of additive models, the main motivation for replacing a GP layer with an additive GP layer in our work is to handle high-dimensional data. When the input dimension is low, it is reasonable that GPs are superior to additive GPs since the additive kernel is an approximated and restrictive kernel. However, when the input dimension increases, the computational complexity grows considerably even in GPs with sparse approximation. For example, in DKL, the output dimension of NN encoder is usually chosen as small as 2, while in pixel data experiments, DKL cannot handle the computation associated with the dimensionality when the output dimension of ResNet is 512 or more. Although DKL is superior in low-dimensional and simple cases, we view additive structure as a necessary component to achieve scalability and good performance with high-dimensional data.

\subsection{Why choosing the induced grids instead of learning the inducing points?}

From an approximation accuracy point of view, there are two separate strategies to increase the accuracy. The first one is to learn the inducing point locations. The second one, however, is to simply increase the number of inducing points on a pre-specified finer grid. The second method is much easier to implement and has a theoretical guarantee by the GP regression theory: as the inducing points become dense in the input region, the approximation will become exact. In contrast, the first approach does not have such a favorable theoretical guarantee. 

The second approach would become difficult to use for many existing methodologies as in general the computational cost would scale as $\mathcal{O}(M^3)$ with $M$ inducing points, which is particularly problematic in high dimensions. 
% The first approach can be viewed as a compromise in those situations, and that is why many existing methods chose to learn the locations of the inducing points instead.
This difficulty is resolved by additive GPs, since approximating an additive GP boils down to approximating one dimensional GPs, which can be accomplished by using a set of pre-specified inducing points on a fine grid in 1-D. One major benefit of the proposed methodology is that the computation now scales at $\mathcal{O}(M)$, enabled by the Markov kernel and the additive kernel. Therefore, a large number of inducing points can be used in an efficient way. 

The proposed method also has several additional benefits: 1) It can decouple to some extent the neural network component and GP component by avoiding learning the inducing points, which may help reduce overfitting/overconfidence; 2) The equivalence to BNN holds exactly with the fixed inducing points, whereas for learned inducing points, this BNN equivalence breaks down, and the proposed computation/training framework would not be possible to carry through; 3) It can simplify the overall optimization since there is no need to learn the inducing points.

\subsection{Limitations and future directions}

Generally, a finer grid will lead to better approximations, but the number of parameters to be trained will also increase. Therefore, there is a trade-off between the accuracy and the computational cost that we can afford. This current work is using a specific Laplace kernel, which can utilize sparse Cholesky decomposition. More general kernels may result in more computational complexity but better representation power of the model. In addition, the current variational family is restricted under mean-field assumptions. A more general variational family, e.g. full/low-rank covariance, may lead to superior performance in some applications. 


\section{EXPERIMENTAL DETAILS}
\label{sec:expdetail}
In this section, we provide additional details regarding the experiments.

\subsection{Benchmarks for Regression}
\label{subsec:regression supp}
\paragraph{Experiment Setup}
For all models, the NN architecture is a fully connected NN with rectified linear unit (ReLU) activation function \citep{nair2010rectified} and two hidden layers containing 64 and 32 neurons, respectively, structured as $D \rightarrow 64 \rightarrow 32 \rightarrow D_w$, where $D$ is the input feature size (also the size of input $\Xv$) and $D_w$ is the output feature size. The models are evaluated with $D_w=16$, 64, and 256, respectively. The number of Monte Carlo samples is set to 8 during training and 20 during inference.

The NN is a deterministic model, and we use the negative Gaussian log-likelihood as the loss function to quantify the uncertainty of the NN outputs and compute the NLPD.

For NN+SVGP, the inducing points are set to the size of 64 in $D_w$ dimension. We implement the \texttt{ApproximateGP} model in GPyTorch \citep{gardner2018gpytorch}, defining the inducing variables as variational parameters, and use \texttt{VariationalELBO} in GPyTorch to perform variational inference and compute the loss.

SV-DKL is originally designed for classification, so for a fair comparison in regression tasks, we modify it by first applying a linear embedding layer $\Wv: \Rb^{D_w} \rightarrow \Rb^P$ with $P=16$ and normalizing the outputs to the interval $[0,1]$ for each base GP, similar to the DAK model. To adapt the additive GP layer for regression, we remove the softmax function from the model in eq. (1) of \citep{wilson2016stochastic}. Given training data $\{ \xv_i, \yv_i \}_{i=1}^{N}$, the model is modified as follows:
\begin{align}
    p(\yv_i \vert \fv_i, A) = \mathcal{A}(\fv_i)^{\top} \yv_i
\end{align}
where $\fv_i \in \Rb^P$ is a vector of independent GPs followed by a linear mixing layer $\mathcal{A}(\fv_i) = A \fv_i$, with $A \in \Rb^{C \times P}$ as the transformation matrix. Here, $C=1$ for single-task regression. For each $p$-th GP ($1 \leq p \leq P$) in the additive GP layer, the corresponding inducing variables $\uv_p$ are set to the size of 64 and treated as variational parameters for training. We use the \texttt{GridInterpolationVariationalStrategy} model with \texttt{LMCVariationalStrategy} in GPyTorch to perform KISS-GP with variational inducing variables, augmented by a linear mixing layer.

For AV-DKL, the inducing points are set to size of 64 in $D_{w}$ dimension. We implement the AV-DKL model based on the source code~\cite{matias2024amortized}.

Both DAK-MC and DAK-CF use the same additive GP layer size as SV-DKL, with $P=16$, and employ fixed induced grids $\Uv = \{1/8, 2/8, \ldots, 7/8\}$ of size 7 for each base GP, which is much smaller than that of SV-DKL.

\paragraph{Metrics}
Let $\{\xv_t, y_t\}_{t=1}^{T}$ represent a test dataset of size $T$, where $\mu_t$ and $\sigma_t^2$ are the predictive mean and variance. We evaluate model performance using two common metrics: Root Mean Squared Error (RMSE) and Negative Log Predictive Density (NLPD).

RMSE is widely used to assess the accuracy of predictions, measuring how far predictions deviate from the true target values. It is calculated as:
\begin{align}
    \text{RMSE} = \sqrt{ \frac{1}{T} \sum_{t=1}^{T}(y_t - \mu_t)^2 }.
\end{align}

NLPD is a standard probabilistic metric for evaluating the quality of a model's uncertainty quantification. It represents the negative log likelihood of the test data given the predictive distribution. For GPs, NLPD is calculated as:
\begin{align}
    \text{NLPD}
    &= - \sum_{t=1}^{T} \log p(y_t = \mu_t \vert \xv_t) \\
    &= \frac{1}{T}
    \sum_{t=1}^{T} \Big[
    \frac{(y_t - \mu_t)^2}{2\sigma_t^2} + \frac{1}{2} \log(2\pi \sigma_t^2)
    \Big].
\end{align}
Both RMSE and NLPD are widely used in the GP regression literature, where smaller values indicate better model performance.

\paragraph{Computing Infrastructure}
The experiments for regression were run on Macbook Pro M1 with 8 cores and 16GB RAM.

\subsection{Benchmarks for Classification}
\label{subsec:classification supp}
We use PyTorch \citep{paszke2019pytorch} baseline of NN models, GPyTorch \citep{gardner2018gpytorch} baseline of SVGP and SV-DKL models. In classification tasks, we apply a softmax likelihood to normalize the output digits to probability distributions. The NN is a deterministic model trained via negative log-likelihood loss, while DKL and DAK models are trained via ELBO loss. The setting of all training tasks are described in \Cref{tab:model classification} and \Cref{tab:optimizer classification}.

SVGP is originally designed for single-output regression. To make it fit for multi-output classification, we used \texttt{IndependentMultitaskVariationalStrategy} in GPyTorch to implement the multi-task \texttt{ApproximateGP} model, and use \texttt{VariationalELBO} with \texttt{SoftmaxLikelihood} in GPyTorch to perform variational inference and compute the loss. 

For SV-DKL, we employed the same \texttt{VariationalELBO} with \texttt{SoftmaxLikelihood} as the variational loss objective. \texttt{GridInterpolationVariationalStrategy} is applied within \texttt{IndependentMultitaskVariationalStrategy} to perform additive KISS-GP approximation. For each KISS-GP unit, we used $64$ variational inducing points initialized on a grid of size $[-1,1]$. 

For DAK, we implemented DAK-MC using Monte Carlo estimation given the intractable softmax likelihood. We employed fixed induced grids $\Uv=\{ -31/32, -30/32, \ldots, 30/32, 31/32 \}$ of size 63 for each base GP component.

\begin{table}[ht]
\caption{Model architectures for image classification on MNIST, CIFAR-10 and CIFAR-100.}
\centering
\resizebox{0.7\linewidth}{!}{
\begin{tabular}{l|l|ccc}
\toprule[1pt]
Model                   & Hyper-parameter          & MNIST       & CIFAR-10    & CIFAR-100   \\
\midrule[0.5pt]
\multirow{4}{*}{NN+SVGP}   & Feature extractor        & CNN         & ResNet-18   & ResNet-34   \\
                        & NN out features $D_w$         & 128         & 512         & 512         \\
                        & Embedding features $P$               & 16          & 64          & 128         \\
                        & \# inducing points $\hat{M}$      & 512         & 512         & 512         \\
                        & \# epochs       & 20         & 200         & 200         \\
                        & Training strategy      & Full-training         & Full-training         & Fine-tuning         \\
\midrule[0.5pt]
\multirow{5}{*}{SV-DKL} & Feature extractor        & CNN         & ResNet-18   & ResNet-34   \\
                        & NN out features $D_w$         & 128         & 512         & 512         \\
                        & Embedding features $P$               & 16          & 64          & 128         \\
                        & \# inducing points $\hat{M}$      & 64          & 64          & 64          \\
                        & Grid bounds              & {[}-1,1{]} & {[}-1,1{]} & {[}-1,1{]} \\
                        & \# epochs       & 20         & 200         & 200         \\
                        & Training strategy       & Full-training         & Full-training         & Fine-tuning         \\
\midrule[0.5pt]
\multirow{4}{*}{DAK}    & Feature extractor        & CNN         & ResNet-18   & ResNet-34   \\
                        & NN out features $D_w$         & 128         & 512         & 512         \\
                        & Embedding features $P$               & 16          & 64          & 128         \\
                        & \# induced interpolation $M$ & 63          & 63          & 63         \\
                        & \# epochs       & 20         & 200         & 200         \\
                        & Training strategy      & Full-training         & Full-training         & Full-training         \\
\bottomrule[1pt]
\end{tabular}

}
\label{tab:model classification}
\end{table}

\paragraph{MNIST} We used a CNN implemented in PyTorch as the feature extractor: \texttt{Conv2d}(1,32,3) $\rightarrow$ \texttt{Conv2d}(32,64,3) $\rightarrow$ \texttt{MaxPool2d}(2) $\rightarrow$ \texttt{Dropout}(0.25) $\rightarrow$ \texttt{Linear}(9216,128) $\rightarrow$ \texttt{Dropout}(0.5). To make a fair comparison, for both SV-DKL and DAK, we applied an embedding module through a linear layer that transform $128$ output features into $P=16$ base GP channels. 

\paragraph{CIFAR-10} We used a ResNet-18 as the feature extractor followed by a linear embedding layer that compressed the $512$ output features into $P=64$ base GP channels. 

\paragraph{CIFAR-100} We used a pretrained ResNet-34 as the feature extractor for SV-DKL and fine-tuned GP output layers since SV-DKL struggled to fit using full-training. For proposed DAK, we used full-training. The number of base GP channels is selected as $P=128$. 

\begin{table}[ht]
\caption{Details of training optimizer for image classification on MNIST, CIFAR-10 and CIFAR-100.}
\centering
\resizebox{0.7\linewidth}{!}{

\begin{tabular}{l|ccc}
\toprule[1pt]
Optimization      & MNIST                                                             & CIFAR-10                                                                                                  & CIFAR-100                                                                                                 \\
\midrule[0.5pt]
Optimizer         & Adadelta                                                          & SGD                                                                                                       & SGD                                                                                                       \\
Initial lr.       & 1.0                                                               & 0.1                                                                                                       & 0.1                                                                                                       \\
Weight decay      & 0.0001                                                            & 0.0001                                                                                                    & 0.0001                                                                                                    \\
Scheduler         & StepLR                                                            & CosineAnnealingLR                                                                                         & CosineAnnealingLR                                                                                         \\
\midrule[0.5pt]
Data Augmentation & MNIST                                                             & CIFAR-10                                                                                                  & CIFAR-100                                                                                                 \\
\midrule[0.5pt]
RandomCrop        & -                                                                 & size=32, padding=4                                                                                        & size=32, padding=4                                                                                        \\
HorizontalFlip    & -                                                                 & p=0.5                                                                                                     & p=0.5                                                                                                     \\
% Normalization     & \begin{tabular}[c]{@{}l@{}}mean=0.1307,\\ std=0.3081\end{tabular} & \begin{tabular}[c]{@{}l@{}}mean={[}0.4914,0.4822,0.4465{]},\\ std={[}0.2023,0.1994,0.2010{]}\end{tabular} & \begin{tabular}[c]{@{}l@{}}mean={[}0.5071,0.4867,0.4408{]},\\ std={[}0.2675,0.2565,0.2761{]}\end{tabular} \\
\bottomrule[1pt]
\end{tabular}
}
\label{tab:optimizer classification}
\end{table}

\paragraph{Additional Benchmark.}  \citet{matias2024amortized} proposed Amortized Variational DKL (AV-DKL), which is a variant SV-DKL using amortization network to compute the inducing locations and variational parameters, thus attenuating the overcorrelation of NN extracted features. AV-DKL is included as the additional benchmark for classification tasks in \Cref{tab:img avdkl}. The training recipe is the same with SV-DKL. 


\begin{table*}[ht]
\caption{\small{Accuracy, NLL, ECE for AV-DKL, SV-DKL, DAK-MC on CIFAR-10/100 averaged over 3 runs. CIFAR-10 uses ResNet-18 with 64 features extracted; CIFAR-100 uses ResNet-34 with 512 features. The best results are highlighted in \textbf{bold}; the second best results are highlighted by \underline{underline}.}}
\centering
\vspace{-0.1cm}
\resizebox{\linewidth}{!}{%
\begin{tabular}{rccclccc}
\toprule[1pt]
\multicolumn{1}{l}{} & \multicolumn{3}{c}{Batch size: 128}  &  & \multicolumn{3}{c}{Batch size: 1024} \\ \cline{2-4} \cline{6-8} \vspace{-8pt} \\
\multicolumn{1}{l}{} & AV-DKL & SV-DKL & \cellcolor{Gray} DAK-MC &   & AV-DKL  & SV-DKL & \cellcolor{Gray} DAK-MC \\ 
\midrule[1pt]
CIFAR-10 - Acc. (\%) $\uparrow$    & \underline{94.23 $\pm$ 0.65}  & 93.44 $\pm$ 0.28    &  \cellcolor{Gray} \textbf{94.81 $\pm$ 0.13}   &     &  \textbf{93.32} $\pm$ \textbf{0.13}        & 90.22 $\pm$ 1.42       & \cellcolor{Gray} \underline{93.02 $\pm$ 0.18}        \\
NLL $\downarrow$     & 0.352 $\pm$ 0.084    & \underline{0.312 $\pm$ 0.033}       &  \cellcolor{Gray} \textbf{0.256} $\pm$ \textbf{0.014}     &      & \underline{0.439 $\pm$ 0.022}         & 0.485 $\pm$ 0.061       & \cellcolor{Gray} \textbf{0.345 $\pm$ 0.001}    \\
ECE $\downarrow$      & 0.048 $\pm$ 0.006    & \underline{0.046 $\pm$ 0.003}       &  \cellcolor{Gray} \textbf{0.039 $\pm$ 0.002}          &     & \underline{0.054 $\pm$ 0.001}       & 0.060 $\pm$ 0.004       & \cellcolor{Gray} \textbf{0.052 $\pm$ 0.001}           \\
\midrule[1pt]
CIFAR-100 -  Acc. (\%) $\uparrow$    & \textbf{77.47 $\pm$ 0.19}  & 74.52 $\pm$ 0.13       & \cellcolor{Gray}  \underline{76.75 $\pm$ 0.18}     &     &  \textbf{77.07 $\pm$ 0.10}        & 66.54 $\pm$ 0.74       & \cellcolor{Gray} \underline{70.38 $\pm$ 1.25}        \\
NLL $\downarrow$     & 1.787 $\pm$ 0.011    & \underline{1.041 $\pm$ 0.007}       & \cellcolor{Gray}  \textbf{1.001 $\pm$ 0.027}     &      & 2.326 $\pm$ 0.030    & \underline{1.738 $\pm$  0.058}      & \cellcolor{Gray} \textbf{1.203 $\pm$ 0.040}        \\
ECE $\downarrow$      & 0.166 $\pm$ 0.002    & \underline{0.049 $\pm$ 0.002}       & \cellcolor{Gray}  \textbf{0.041 $\pm$ 0.004}        &     & 0.175 $\pm$ 0.001         & \underline{0.148 $\pm$ 0.007}       &\cellcolor{Gray}  \textbf{0.056 $\pm$ 0.006}           \\
\bottomrule[1pt]
\end{tabular}
}
\vspace{-0.2cm}
\label{tab:img avdkl}
\end{table*}

\paragraph{Metrics} 
We evaluate model performance using four common metrics: Top-1 accuracy, ELBO, Negative Log Likelihood (NLL), and Expected Calibration Error (ECE). 

ECE is a metric used to quantify the degree of ``calibration'' of a probabilistic model in UQ, specifically for classification problems. It is defined as the weighted average of the absolute difference between the model's predicted probability (confidence) and the actual outcome (accuracy) over several bins of predicted probability. Mathematically, ECE is given by:
\begin{align}
    \text{ECE} =\sum_{m=1}^{M} \frac{\left| B_{m} \right|}{n} \left| \text{acc} (B_{m})-\text{conf} (B_{m}) \right|,
\end{align}
where $M$ is the number of bins into which the confidence values are partitioned, $B_m$ is the set of indices of samples whose predicted confidence falls into the $m$-th bin, $n$ is the total number of samples.

\paragraph{Computing Infrastructure}
The experiments for classification were run on a Linux machine with NVIDIA RTX4080 GPU, and 32GB of RAM.




\subsection{Additional Tables and Figures}
\label{sec:additional exp results}

\paragraph{Choices of learning rates.}
We evaluate the choices of learning rates on 1D regression examples. DKL requires a separate tuning of the learning rate of the GP covariance parameters, which differs from the learning rate of the NN feature extractor. In \Cref{fig:dkl lr}, we choose the learning rate of the NN feature extractor as $0.01$, while the learning rate of the GP covariance is set to different values. (a)-(c) show that different learning rates of covariance in DKL result in different predictive posterior. In particular, although the training losses for DKL in both (a) and (b) are minimal, the regressions do not fit well. On the other hand, DAK does not need a distinct recipe for tuning GP covariances because of the BNN interpretation. Furthermore, the poor posterior is indicated by the higher training loss, as illustrated in (d)-(f).

\begin{figure}[ht]
\centering
\subfloat[$\begin{gathered}\text{DKL: last-layer lr} =0.01.\\ \text{Training loss:} -0.21.\end{gathered}$]{\includegraphics[width=.3\textwidth]{toy_dkl_lr_01.pdf}}
\subfloat[$\begin{gathered}\text{DKL: last-layer lr} =0.001.\\ \text{Training loss: } -0.07.\end{gathered}$]{\includegraphics[width=.3\textwidth]{toy_dkl_lr_001.pdf}}
\subfloat[$\begin{gathered}\text{DKL: last-layer lr} =0.0001.\\ \text{Training loss: } 0.22.\end{gathered}$]{\includegraphics[width=.3\textwidth]{toy_dkl_lr_0001.pdf}}

\subfloat[$\begin{gathered}\text{DAK: last-layer lr} =0.1.\\ \text{Training loss: } 0.10.\end{gathered}$]{\includegraphics[width=.3\textwidth]{toy_dak_lr_1.pdf}}
\subfloat[$\begin{gathered}\text{DAK: last-layer lr} =0.01.\\ \text{Training loss: } 0.10.\end{gathered}$]{\includegraphics[width=.3\textwidth]{toy_dak_lr_01.pdf}}
\subfloat[$\begin{gathered}\text{DAK: last-layer lr} =0.001.\\ \text{Training loss: } 0.22.\end{gathered}$]{\includegraphics[width=.3\textwidth]{toy_dak_lr_001.pdf}}

\caption{Results on 1D regression with different last-layer learning rates. The learning rate of NN feature extractor is set as $0.01$. (a)--(f) shows the regression fits and corresponding training losses. DAK fits for the same learning rate strategy with NN feature extractor (lr=0.01), while DKL requires a separate tuning for last-layer learning rate of GPs. Additionally, a better training loss does not necessarily prevent overfitting for DKL.}
\label{fig:dkl lr}
\end{figure}


\paragraph{Learning curves.} We plot the learning curves of CIFAR-10/100 in \Cref{fig:cifar10 curves} and \ref{fig:cifar100 curves}. The learning curves of SVDKL in \Cref{fig:cifar10 curves} is more unstable, with many significant spikes, and the convergence is slower than DAK. Futhermore, SVDKL struggles to fit with full-training in CIFAR-100, and a pretrained feature extractor is used in CIFAR-100. Therefore, the learning curves of SVDKL look smoothing, but DAK fits well with full-training in CIFAR-100.


\begin{figure}[ht]
\centering
\subfloat[Test Error (\%).]{\includegraphics[width=.3\textwidth]{CIFAR_10_test_error.pdf}}
\subfloat[Test NLL.]{\includegraphics[width=.3\textwidth]{CIFAR_10_nll.pdf}}
\subfloat[ELBO.]{\includegraphics[width=.3\textwidth]{CIFAR_10_elbo.pdf}}
\caption{Test errors, test NLLs, ELBOs of NN, SVDKL, and DAK curves with batch size of 128/1024 for CIFAR-10 averaged on 3 runs. DAK outperforms SVDKL on both test error and NLL along the training epochs. Additionally, SVDKL degrades more and struggles to fit when the batch size becomes larger.}
\label{fig:cifar10 curves}
\end{figure}

\begin{figure}[ht]
\centering
\subfloat[Test Error (\%).]{\includegraphics[width=.3\textwidth]{CIFAR_100_test_error.pdf}}
\subfloat[Test NLL.]{\includegraphics[width=.3\textwidth]{CIFAR_100_nll.pdf}}
\subfloat[ELBO.]{\includegraphics[width=.3\textwidth]{CIFAR_100_elbo.pdf}}
\caption{Test errors, test NLLs, ELBOs of NN, SVDKL, and DAK curves with batch size of 128/1024 for CIFAR-100 averaged on 3 runs. DAK trained NN and last-layer additive GPs jointly, while SVDKL used the pre-trained NN and fine-tuned the last-layer GP since SVDKL struggles to fit using full-training. DAK outperforms SVDKL on both test error and NLL along the training epochs. SVDKL struggled to fit in high-dimensional multitask cases, indicating the necessity of pre-training in SVDKL. However, DAK fitted well with high dimensionality and large batch sizes.}
\label{fig:cifar100 curves}
\end{figure}







% \end{document}

%%%%%%%%%%%%%%%%%%%%%%%%%%%%%%%%%%%%%%%%%%%%%%%%%%%%%%%%%%%%%%%%%%%%%%%%%%%%%%%
%%%%%%%%%%%%%%%%%%%%%%%%%%%%%%%%%%%%%%%%%%%%%%%%%%%%%%%%%%%%%%%%%%%%%%%%%%%%%%%


\end{document}


% This document was modified from the file originally made available by
% Pat Langley and Andrea Danyluk for ICML-2K. This version was created
% by Iain Murray in 2018, and modified by Alexandre Bouchard in
% 2019 and 2021 and by Csaba Szepesvari, Gang Niu and Sivan Sabato in 2022.
% Modified again in 2023 and 2024 by Sivan Sabato and Jonathan Scarlett.
% Previous contributors include Dan Roy, Lise Getoor and Tobias
% Scheffer, which was slightly modified from the 2010 version by
% Thorsten Joachims & Johannes Fuernkranz, slightly modified from the
% 2009 version by Kiri Wagstaff and Sam Roweis's 2008 version, which is
% slightly modified from Prasad Tadepalli's 2007 version which is a
% lightly changed version of the previous year's version by Andrew
% Moore, which was in turn edited from those of Kristian Kersting and
% Codrina Lauth. Alex Smola contributed to the algorithmic style files.
