\documentclass[10pt,conference]{IEEEtran}
\IEEEoverridecommandlockouts
% The preceding line is only needed to identify funding in the first footnote. If that is unneeded, please comment it out.
%Template version as of 6/27/2024

\usepackage{cite}
\usepackage{amsmath,amssymb,amsfonts}
\usepackage{algorithmic}
\usepackage{graphicx}
\usepackage{textcomp}
\usepackage{tcolorbox}
\usepackage{todonotes}
\usepackage{url}
\usepackage{listings}
\usepackage{booktabs}
\usepackage{xspace}
\usepackage{hyperref}


\newcommand{\change}[1]{\textcolor{red}{#1}\xspace}

\lstdefinestyle{mystyle}{
%    backgroundcolor=\color{backcolour},   
%    commentstyle=\color{codegreen},
 %   keywordstyle=\color{magenta},
 %   numberstyle=\tiny\color{codegray},
 %   stringstyle=\color{codepurple},
    basicstyle=\ttfamily\footnotesize,
    language=Java,
    %basicstyle=\ttfamily\tiny,
 %   breakatwhitespace=false,         
 %   breaklines=true,                 
    captionpos=b,                    
 %   keepspaces=true,                 
    %numbers=left,                    
    %numbersep=5pt,                  
    showspaces=false,                
    showstringspaces=false,
 %   showtabs=false,                  
    tabsize=2
}
\lstset{style=mystyle}

\newcommand{\urlRepo}[0]{\url{https://osf.io/yh6v2/?view_only=a069f5a56f3640abaf235eed1320b46f}\xspace} 

%Erano presenti
\usepackage[inline]{enumitem}
\setlist[enumerate,1]{label=\textit{\alph*)}}
\definecolor{review}{RGB}{0,0,0}

\usepackage{xcolor}
\def\BibTeX{{\rm B\kern-.05em{\sc i\kern-.025em b}\kern-.08em
    T\kern-.1667em\lower.7ex\hbox{E}\kern-.125emX}}

\usepackage{booktabs}






\begin{document}

\title{On the Possibility of Breaking Copyleft Licenses When Reusing Code Generated by ChatGPT
%\title{Investigating Possible IP Rights Infringement in AI-Generated Code: A Case Study on ChatGPT\\
%{\footnotesize \textsuperscript{*}Note: Sub-titles are not captured for https://ieeexplore.ieee.org  and
%should not be used}
%\thanks{Identify applicable funding agency here. If none, delete this.}
}

\author{\IEEEauthorblockN{Gaia Colombo}
\IEEEauthorblockA{
\textit{University of Milano-Bicocca}\\
Milano, Italy \\
g.colombo147@campus.unimib.it}
\and
\IEEEauthorblockN{Leonardo Mariani}
\IEEEauthorblockA{
\textit{University of Milano-Bicocca}\\
Milano, Italy \\
leonardo.mariani@unimib.it}
 \and
%\linebreakand
%\hspace{-1.0cm} % Adjust spacing between authors if needed
\IEEEauthorblockN{Daniela Micucci}
\IEEEauthorblockA{
\textit{University of Milano-Bicocca}\\
Milano, Italy \\
daniela.micucci@unimib.it}
\and
\IEEEauthorblockN{Oliviero Riganelli}
\IEEEauthorblockA{
\textit{University of Milano-Bicocca}\\
Milano, Italy \\
oliviero.riganelli@unimib.it}

}

%\author{\IEEEauthorblockN{Anonymous Authors}}
%\author{\IEEEauthorblockN{1\textsuperscript{st} Given Name Surname}
%\IEEEauthorblockA{\textit{dept. name of organization (of Aff.)} \\
%\textit{name of organization (of Aff.)}\\
%City, Country \\
%email address or ORCID}
%\and
%\IEEEauthorblockN{2\textsuperscript{nd} Given Name Surname}
%\IEEEauthorblockA{\textit{dept. name of organization (of Aff.)} \\
%\textit{name of organization (of Aff.)}\\
%City, Country \\
%email address or ORCID}
%\and
%\IEEEauthorblockN{3\textsuperscript{rd} Given Name Surname}
%\IEEEauthorblockA{\textit{dept. name of organization (of Aff.)} \\
%\textit{name of organization (of Aff.)}\\
%City, Country \\
%email address or ORCID}
%\and
%\IEEEauthorblockN{4\textsuperscript{th} Given Name Surname}
%\IEEEauthorblockA{\textit{dept. name of organization (of Aff.)} \\
%\textit{name of organization (of Aff.)}\\
%City, Country \\
%email address or ORCID}
%\and
%\IEEEauthorblockN{5\textsuperscript{th} Given Name Surname}
%\IEEEauthorblockA{\textit{dept. name of organization (of Aff.)} \\
%\textit{name of organization (of Aff.)}\\
%City, Country \\
%email address or ORCID}
%\and
%\IEEEauthorblockN{6\textsuperscript{th} Given Name Surname}
%\IEEEauthorblockA{\textit{dept. name of organization (of Aff.)} \\
%\textit{name of organization (of Aff.)}\\
%City, Country \\
%email address or ORCID}
%}

\maketitle


\begin{abstract}
AI assistants can help developers by recommending code to be included in their implementations (e.g., suggesting the implementation of a method from its signature). Although useful, these recommendations may mirror copyleft code available in public repositories, exposing developers to the risk of reusing code that they are allowed to reuse only under certain constraints (e.g., a specific license for the derivative software). 

This paper presents a large-scale study about the frequency and magnitude of this phenomenon in ChatGPT. In particular, we generate more than 70,000 method implementations using a range of configurations and prompts, revealing that a larger context increases the likelihood of reproducing copyleft code, but higher temperature settings can mitigate this issue.
 %Our findings reveal that a larger matching context increases the likelihood of GPT-4 reproducing copyleft code, though higher temperature settings can help mitigate this issue.

\end{abstract}

\begin{IEEEkeywords}
AI-assisted coding, code generation, copyleft licenses, intellectual property\end{IEEEkeywords}

\section{Introduction}

Several AI assistants, such as ChatGPT\footnote{https://openai.com/chatgpt/}, GitHub Copilot\footnote{https://github.com/features/copilot}, and Google Gemini\footnote{https://gemini.google.com/}, are available to help developers complete their development tasks. Multiple studies reveal that, although far from perfect, these tools may produce useful results with the potential of accelerating code development~\cite{Hou:LLMforSE:SLR:TOSEM:2024,Fan:LLMChallenges:ICSEFOSE:2023,Mastropaolo:RobustnessGenCode:ICSE:2023,Corso:EmpiricalAssessment:ICPC:2024,Fagadau:EmpiricalStudy:ICPC:2024}. 

AI assistants exploit LLMs trained on a huge corpus of data, including code, to provide recommendations. This raises concerns in terms of the \emph{ownership and rights to reuse} the code generated by these tools when the recommended code matches the code in the training set. This issue is particularly severe if we consider that training code could be protected by restrictive licenses that forbid or limit reuse\footnote{For instance, open source code protected by copyleft licenses can be reused only under specific constraints on how the derivative code is licensed.}. That is, a developer may inadvertently break the license terms associated with the reused code by simply accepting a recommendation. 


A key question about the recommendations produced by AI assistants is thus: \emph{``Is the recommended code original, or is it a copy of existing code protected by a restrictive license?"} Answering this question can be extremely important to prevent potentially unethical behavior and possible legal issues. 

Previous studies based on the now-obsolete GPT-2 models suggest that LLMs can memorize and reproduce code present in their training data. In particular, Al-Kaswan et al.~\cite{AlKaswan:TracesMemorisation:ICSE:2024}  reported that LLMs may generate completions for code prefixes encountered during training by appending the corresponding code suffix from the training data.
%OLD
%Previous studies based on, now obsolete, GPT-2 models show that LLMs may memorize and reproduce the code that is part of the training data. In particular, Al-Kaswan et al.~\cite{AlKaswan:TracesMemorisation:ICSE:2024} reported that LLMs may suggest to complete a code-prefix that occurred in the training data, with the suffix-code present in the training data. 
Similarly, Yang et al.~\cite{Yang:UnveilingMemorization:ICSE:2024} found that training code can be observed in the output for a range of diverse prompts. 

A more recent study~\cite{Yu:codeipprompt:ICML:2023} investigated the concern of code memorization in GPT-4 models, considering code protected by restrictive licenses. The investigation reported a high probability (more than 50\%) for models to return code protected by copyleft licenses if queried for the generation of a function's implementation whose signature is present in the training set. 

%Such a high percentage of potential violations is, also, due to the configuration of the plagiarism detection tools used in the study. In fact, the code returned by AI-assistants is considered as a potential evidence of plagiarism if its similarity is above 0.5 for at least one of the two plagiarism detection tools used: JPlag and Devos. 

%This setup is in contrast with previous studies~\cite{AlKaswan:TracesMemorisation:ICSE:2024,Yang:UnveilingMemorization:ICSE:2024} that considered nearly identical code as an evidence of plagiarism (e.g., Type I clones), and reported a definitely lower probability (\textbf{LEO quanto?}) for AI-assistants to generate code that plagiarizes code in the training data. 

This study aims to \emph{broaden existing evidence and derive further insights} about the risks related to the reuse of the code originated by GPT models. In particular, we investigate how GPT-4 performs in the generation of Java methods, studying the \emph{similarity} between the recommended code and the corresponding open source code available under restrictive licenses using the JPlag plagiarism detection tool\footnote{https://helmholtz.software/software/jplag}. Further, our investigation considers multiple dimensions not studied in the context of GPT-4 models, so far. 

First, we study if and to what extent the \emph{context} may influence the generation of code protected by restrictive licenses. In particular, %we consider the case of developers who have already accepted some code recommendations that include licensed code, and we study if this may increase the likelihood of receiving further recommendations of code still protected by restrictive licenses. 
we investigate cases in which developers, having already accepted recommendations containing licensed code, may face an increased likelihood of receiving additional recommendations of code protected by restrictive licenses. We specifically consider the case of classes that already contain code that mirrors open-source code protected by copyleft licenses, and study its impact on recommendations.
%
We investigate two paradigmatic cases: one considering the presence of \emph{all the methods in the class but the one that has to be generated} matching an open-source implementation protected by a copyleft license, and another one only considering the \emph{presence of access methods} matching an open-source implementation protected by a copyleft license. The first case captures the situation where the context is the largest possible, encompassing every other method within the class. Instead, the second case investigates to what extent the presence of simple and commonly implemented methods with minimal semantic content (i.e., the access methods) may influence the generation of licensed code.

Second, we consider the impact of the \emph{temperature} on the likelihood of generating code protected by restrictive licenses. The temperature is a parameter that influences the creativity of the model, and thus it may have a role in the likelihood that the model replicates licensed code observed during training.

%Last, we investigate if GPT-4 shows any level of \emph{awareness} in the generation of code distributed under restrictive licenses, by studying if explicitly asking for original code may decrease the change code that cannot be freely reused is recommended. 

Finally, we examine if GPT-4 demonstrates any level of \emph{awareness} regarding the generation of code subject to restrictive licenses. Specifically, we investigate whether explicitly requesting code that is not a copy of any known implementation may influence the likelihood of receiving recommendations for code that cannot be freely reused.


We performed this investigation on a body of 7,347 methods, assessing more than 70,000 method implementations. Results show a low probability of receiving recommendations of copyleft code for individual requests, but an increasingly large matching context may increase the likelihood that GPT-4 returns copies of code protected by copyleft licenses. 
%That is, the 
This suggests that the early acceptance of copyleft code may lead developers to incidentally accumulate more copyleft code in their implementations. High-temperature values may help mitigate this problem. Finally, GPT-4 has not demonstrated any awareness of this phenomenon.

In a nutshell, this paper contributes as follows:
\begin{enumerate*}[leftmargin=*,label=\textit{\alph*)}]
\item Reporting a large-scale study on the generation of licensed code with GPT-4,

\item Presenting an analysis of how the context and the temperature may influence the results,

\item Investigating if any trace of awareness is present in GPT-4 models concerning the generation of code under restrictive licenses,

\item Producing recommendations for developers who use GPT-4 to generate code,

\item Publicly releasing the experimental material\footnote{\urlRepo} for replication and extension of the study.

\end{enumerate*}


The paper is organized as follows. Section~\ref{sec:RW} discusses related work. Section~\ref{sec:Met} introduces the five research questions investigated in this paper and presents the methodology used to answer them. Section~\ref{sec:Res} reports the empirical results collected to answer the five research questions. Finally, Section~\ref{sec:con} provides final remarks and suggests directions for future research.


\section{Related Work}

\section{Related Work} \label{sec:related}

% \textbf{Adversarial Attack}
\textbf{Attacks on SLAM.} 
%With the rise of machine learning, 
The robustness of computer vision systems is being actively investigated. With the emergence of adversarial images in the digital domain by adding optimized noise directly to images~\cite{szegedy2013intriguing,carlini2017towards}, researchers find that such attacks also exist physically in the real world \cite{eykholt2018robust,song2018physical,zhao2019seeing}. To fill the gap between attacks in the digital and physical worlds, recent studies have demonstrated that attacks on real-world computer vision systems are practical \cite{eykholt2018robust,li2019adversarial,man2020ghostimage,sharif2016accessorize,zhao2019seeing,zhou2018invisible}. However, attacks on traditional computer vision methods such as SLAM are relatively less explored. \cite{yoshida2022adversarial} proposes an attack against the scan matching algorithm in LiDAR-based SLAM, while most SLAMs in AR/VR devices rely on different sensors like RGB/depth cameras and IMUs. \cite{ikram2022perceptual} and \cite{chen2024adversary} mislead visual SLAM by poisoning the images with special patterns, and \cite{wang2021can} causes the camera to fail using infrared light. In our work, we demonstrate attacks on Visual-Inertial SLAM (VI-SLAM) by perturbing the IMU readings, rather than cameras, and showing its impact on XR user experience. 

\textbf{Acoustic Injection Attacks.} Among various physical attacks, acoustic injection attacks are attractive due to their low cost. Son~\etal~\cite{son2015rocking} were the first to introduce acoustic attacks on MEMS gyroscopes, demonstrating how these attacks could lead to sensor denial-of-service and result in drone crashes. WALNUT~\cite{trippel2017walnut} expanded on this by developing output biasing and control attacks that enable precise manipulation of MEMS accelerometer outputs using modulated sound waves. Wang et al.~\cite{wang2017sonic} demonstrated a sonic gun, showcasing the vulnerability of various smart devices (\eg drones and self-balancing vehicles) to acoustic attacks. Tu et al. \cite{tu2018injected} designed side-swing and switching attacks to alter the outputs of MEMS gyroscopes and accelerometers. Furthermore, Ji et al. \cite{ji2021poltergeist} fool the object detectors by applying acoustic attack to the image stabilizers commonly used in modern cameras. However, none of the existing works study the relationship between the acoustic injections and SLAM outputs on recent XR devices. 

% \zijian{Do we need one session about security in AR/VR?}
% \yicheng{TODO}
%\jiasi{cite the AIVR paper (UMass Amherst?) paper is we have not already. They add IMU perturbation but w/o SLAM, iirc} \yicheng{Cited}

\textbf{XR Security and Privacy.} 
%Security and privacy concerns in XR systems have gained significant attention. 
For single-user XR systems, researchers have demonstrated various side-channel attacks to extract sensitive information (\eg keystrokes) through video feeds~\cite{ling2019know}, head movements~\cite{nair2023unique, slocum2023going}, architectural hints~\cite{zhang2023its,shang2020arspy}, power usage~\cite{li2024dangers}, and EM side-channel leakages~\cite{al2021vr}. In multi-user XR systems, Su et al.~\cite{su2024remote} use avatar motion data to infer keystrokes in shared VR environments. Slocum et al.~\cite{slocum2024doesn} reveal vulnerabilities in the shared state frameworks of multi-user AR. Similarly, Lebeck et al.~\cite{lebeck2017securing} highlight risks like deceptive virtual objects and emphasize access control for managing shared physical and virtual spaces. Ruth et al.~\cite{ruth2019secure} further propose a secure multi-user AR framework focusing on content sharing and permissions.
Chandio et al.~\cite{chandio2024stealthy} %introduced a multi-modal spatiotemporal attack that 
simultaneously manipulated visual and inertial sensors to disrupt XR pose estimation. However, their study evaluated the attack using offline datasets and assumed the attacker's capability to manipulate IMU data streams through acoustic means, without real experiments. Ours is the first to demonstrate acoustic injection attacks on recent XR devices, like the Hololens 2, in the real world.
 



\section{Methodology}

\section{\label{sec:method}Methodology}

Each SIEM system uses its own RDL to define threat detection rules, and each RDL has its own schema.
For example, the Splunk SIEM uses the SPL to define its threat detection rules.
The task of understanding threat detection rules and recommending relevant MITRE ATT\&CK techniques (or sub-techniques) requires complex reasoning skills.
In the case of LLMs, this can be achieved with a technique called prompt chaining in which each task is divided into multiple sub-tasks in order to understand the complex reasoning behind the task.
Therefore, we employ a multi-phase architecture based on prompt chaining that leverages the power of LLMs to take a SIEM rule defined in any RDL and map it to relevant MITRE ATT\&CK techniques using the power of LLMs.
Our approach is based on the following intuitions:
\begin{itemize}[nosep,leftmargin=*]
    \item \textit{LLMs' implicit knowledge}: LLMs possess deep understanding of diverse RDLs. This enables them to interpret any rule, regardless of the RDL it is defined in, and convert it into comprehensible natural language text.
    \item \textit{LLMs' similarity comparison capability}: LLMs are adept at analyzing and comparing textual descriptions. 
    They can intelligently assess the similarity between two textual inputs to establish a meaningful connection.
\end{itemize}

\methodName has two main phases: (1) the rule to text translation phase, and (2) the MITRE ATT\&CK techniques recommendation phase.
These two phases in the pipeline include six key steps to determine relevant TTPs, as illustrated in Figure~\ref{fig:r2t}.

Although LLMs excel at translating SIEM rules into natural language, they often lack critical domain-specific contextual information related to IoCs in the rules.
To overcome this limitation, the \textit{rule to text translation} phase includes three steps: IoC extraction, contextual information retrieval, and natural language translation.

The workflow begins with the extraction of IoCs from the rules (for example, processes, log source, event codes, and file names) that the rule searches for in the logs (step (1)).In the next sstep a web search agent performs the task of obtaining additional contextual information about the IoCs discovered ((step 2)).
By incorporating this additional domain-specific information, the pipeline enhances the language translation, resulting in a more accurate and meaningful interpretation of SIEM rules.
The rule itself and the IoCs' contextual additional information from the previous stage are then used to translate the rule from RDL to natural language (step (3)).

The \textit{MITRE ATT\&CK techniques} recommendation phase of the pipeline includes the following three steps.
The rule in processed in data source identification step in which the probable origin of the data is identified (step (4)).
The description of the rule is then used to determine probable MITRE ATT\&CK techniques based on the implicit knowledge of the LLM (step (5)).
Finally, using chain-of-thought~\cite{wei2022chain} prompting, the most relevant techniques are extracted from the list of probable techniques (step (6)).
Each step of our method is further described in detail below.


% [bb=0 0 1440 900,width=1.43\linewidth,height=0.9\textwidth]
\begin{figure*}[htbp]
   \includegraphics[width=\textwidth]{Images/stages.jpg}
    
   \caption{An illustration of the different steps in \methodName.}
   \label{fig:stages}
\end{figure*} 

\subsection{IoC Extraction}
The context associated with a SIEM detection rule is crucial for its accurate interpretation and effective application. 
Obtaining this contextual understanding requires comprehensive analysis of the embedded IoCs in the SIEM rule.
In the first step, \methodName systematically identifies and extracts all IoCs, identifying the types of IoCs and their corresponding values that form the foundational elements of the detection rules. 
Leveraging the LLM's inherent understanding of rule structures and IoCs, we employ a zero-shot prompting approach for this task. 
Zero-shot prompting enables the direct extraction of IoCs from the rules without requiring extensive pre-training on specific datasets.

\noindent The result of this stage is a dictionary structure, where:
\begin{itemize}[nosep,leftmargin=*]
    \item Keys represent types of IoC, such as processes, files, IP addresses, and log sources.
    \item Values are lists containing specific IoC details, such as process names, file names, IP addresses, and log source identifiers.
\end{itemize}

In the example depicted in Figure~\ref{fig:stages}(a), the pipeline processes a rule for which relevant MITRE ATT\&CK techniques need to be recommended. 
The IoC extractor LLM produces a dictionary structure as output, organizing the IoCs in a structured format to support subsequent stages in the analysis pipeline. 



\subsection{Contextual Information Retrieval}
In this step, an LLM agent is employed to retrieve relevant information pertaining to the IoCs extracted from the rule.
A REACT agent~\cite{react} was used in this case to generate both reasoning traces and task-specific actions in an interleaved manner.
REACT agents interact with external tools to retrieve additional information that leads to more factual and reliable responses.
The LLM agent conducts a systematic search across web resources to gather additional contextual information for each IoC value present in the rule. 
This step addresses LLMS' lack of up-to-date knowledge or specialized domain expertise (which is critical to understanding the role and significance of the IoCs in the rule), without the need for retraining or fine-tuning.
Figure~\ref{fig:stages}(b) presents an example in which the rule includes the process name \texttt{soaphound.exe} as an IoC.
As can be seen, the web search results indicate that \texttt{soaphound.exe} is being used for active directory (AD) enumeration, which is important for the understanding of the attack. 

\subsection{Natural Language Translation}

The translation of detection rules into natural language textual descriptions fulfills three key objectives:
\begin{enumerate}[nosep,leftmargin=*]
    \item \textbf{Ensures that \methodName is format-agnostic}: It converts rules defined in various RDL formats into a generic, unstructured text format, ensuring compatibility with different SIEM systems, regardless of the specific rule format.
    \item \textbf{Provides contextual explanation}: It includes all relevant contextual information to produce a concise and comprehensible explanation of the rule.
    \item \textbf{Enhances the comprehension for LLMs}: It enables LLMs to more effectively compare the translated rule with descriptions in the MITRE ATT\&CK framework by providing a unified textual representation.
\end{enumerate}
To achieve these objectives, a zero-shot prompting technique is employed. 
The input to the LLM comprises two components:
\begin{itemize}
    \item \textbf{Syntactical information}: The rule itself, providing the structural and operational details.
    \item \textbf{Contextual information}: Details of the IoCs extracted from the rule, providing semantic insights into the rule's intent and function.
\end{itemize}
The LLM utilizes these inputs to generate a natural language textual description of the rule. 
This transformation not only ensures a more interpretable representation but also facilitates further steps of analysis and comparison, particularly in aligning the rule with MITRE ATT\&CK techniques and sub-techniques.



\subsection{Data Source or Mitigation Identification}
Identifying the most relevant data component or mitigation associated with the rule description in this step is critical for filtering out irrelevant MITRE ATT\&CK techniques (or sub-techniques) in subsequent steps of the pipeline.
In the MITRE ATT\&CK framework, data sources represent various categories of information that can be gathered from sensors or logs. 
These data sources include \textit{data components}, which are specific attributes or properties within a data source that are directly relevant to detecting a particular technique or sub-technique~. 
For example, in the context of the rule described in Figure~\ref{fig:stages}(a), the term \texttt{Endpoint.Processes} indicates that the activity is happening on an endpoint. 
Presence of the terms such as, \texttt{soaphound.exe}, \texttt{--buildcache}, \texttt{--certdump} and etc. indicate that the rule searches for command line execution of an executable named \texttt{soaphound.exe} with specific parameters. 
Therefore, the appropriate data source in this example is \textit{Command}, with the corresponding data component being \textit{Command Execution}.
Additionally, \textit{mitigations} are defined as categories of technologies or strategies that can prevent or reduce the impact of specific techniques or sub-techniques. 
The MITRE ATT\&CK framework explicitly establishes relationships between data components, mitigations, and techniques (or sub-techniques), enabling a systematic approach for identifying relevant elements.

To identify the most relevant data component or mitigation associated with a given rule description, we utilize agentic retrieval augmented generation (RAG), which incorporates an AI Agent-based implementation of the RAG framework.
Data from the MITRE ATT\&CK framework, specifically related to data components and mitigations, is stored in a vector database (e.g., ChromaDB). 
The process begins with the rule description from the previous stage, which serves as the input to the AI Agent. 
The LLM-powered agent automatically generates a search query tailored to retrieve relevant information from the RAG database.

For each query, the system retrieves the five most similar documents from the database, each containing contextual information about data components or mitigations. 
These documents are then utilized by the LLM agent to contextualize the rule description. 
By comparing the content of these retrieved documents with the rule description, the LLM agent determines and outputs the most relevant data component or mitigation along with a chain-of-thought as to why the data component or mitigation is related to the rule.


\subsection{Probable Technique Recommendation}

In this step, an LM agent is utilized to propose probable MITRE ATT\&CK techniques (and sub-techniques) that may be relevant to the description of the provided rule.
We used a REACT agent in this step as well to utilize both implicit and explicit knowledge during reasoning.
For explicit knowledge, the agent searches the MITRE ATT\&CK framework to obtain the list of probable techniques (and sub-techniques).
The natural language description of the rule from the previous step serves as input to the LLM agent.
The output of this stage consists of a list of JSON objects, each containing the MITRE technique ID, technique name, and technique description as seen in Figure~\ref{fig:stages}(c).

Throughout our experiments, we observed that as the number of recommendations ($k$) increases, both the framework's average recall and precision initially improve, however beyond a certain threshold of $k$, the %average 
precision begins to decline.
Based on these observations(please refer Table~\ref{tab:results3}), we selected a $k$-value of 11 to ensure a high recall.



\subsection{Relevant Technique Extraction}
In this step, \methodName refines the set of probable MITRE ATT\&CK techniques identified in the previous stage by eliminating irrelevant entries.
This step in the pipeline serves two primary purposes: (1) to enhance precision while maintaining recall achieved in previous step, and (2) to provide a clear rationale for the selection of the labels, ensuring transparency and interpretability of the mapping process.
This refinement process is grounded in the assumption that LLMs are effective for text similarity matching tasks.

The process comprises two key steps:
\begin{itemize}
    \item \textit{Rule-technique comparison}: The description of each technique in the set of probable techniques is compared with the rule description. 
    A chain-of-thought technique is then applied to elucidate the reasoning behind the association of each technique with the rule.
    \item \textit{Confidence calculation}: The generated chain-of-thought rationale for each technique (or sub-technique) is compared with the rule description to compute a relevance (or confidence) score, as done in prior work~\cite{freitas2024ai}.
    % \item \textbf{Reasoning}: \new{Add here the reasoning that it provides - explaining in NLP why it was selected...}
\end{itemize}

Techniques with higher confidence scores are deemed more relevant to the rule. 
Conversely, techniques with scores falling below a predefined threshold are excluded.
The techniques retained after this filtering step represent the most relevant techniques corresponding to the given rule's description. 


The chain-of-thought (CoT) rationale generated during the comparison of each rule to its probable technique is also provided as an output in this step.
This rationale offers a detailed natural language explanation, articulating why a particular technique is relevant to the given rule. 
Such explanations are highly valuable for security analysts, as they provide clear and transparent reasoning behind the mapping, enabling analysts to better understand and validate the association between the rule and the technique.
Other classification models proposed in previous works within this domain also suffer from the limitation of being black-box models, which lack the ability to provide clear reasoning or explanations. 
Unlike \methodName, these models fail to generate transparent, CoT rationales that explain why a particular rule is mapped to a specific technique, making them less interpretable and less useful for security analysts.

\section{Results}
\begin{table}[ht!]
\centering
\caption{\textbf{Super Resolution Performance Results.} Our proposed WGAN EEG Spatial Upsampling method significantly outperforms a baseline of Bicubic Interpolation commonly used in EEG upsampling pipelines.}
\label{tab:results}
\resizebox{0.8\linewidth}{!}{%
\begin{tabular}{@{}cccccc@{}}
\toprule
\multirow{2}{*}{\textbf{Dataset}} & \multirow{2}{*}{\textbf{Scale}} & \multicolumn{2}{c}{\textbf{Bicubic}} & \multicolumn{2}{c}{\textbf{WGAN}} \\ \cmidrule(l){3-6} 
                      &   & \textbf{MSE} & \textbf{MAE} & \textbf{MSE}    & \textbf{MAE}   \\
\toprule
\multirow{2}{*}{Val}  & 2 & 3.71E7       & 3.89E3       & \textbf{2.01E3} & \textbf{24.38} \\
                      & 4 & 7.23E7       & 6.42E3       & \textbf{8.53E3} & \textbf{63.83} \\
\midrule
\multirow{2}{*}{Test} & 2 & 3.75E7       & 3.91E3       & \textbf{2.06E3} & \textbf{24.66} \\
                      & 4 & 7.30E7       & 6.45E3       & \textbf{8.68E3} & \textbf{64.39} \\
\bottomrule
\end{tabular}%
}
\end{table}

\section{Conclusions}
\section{Conclusion Remarks}
This work proposes a RBG graph model for disease spreading via hubs. We study the joint effect of the agent density, hub density, and connection function. The existence of a critical hub density depends only on the boundedness of the support of the connection function, which relates to curbing the traveling distance of individuals. When it comes to dispersion, both the degree distribution and the percolation threshold suggest that increasing dispersion helps spread the disease. The percolation properties of RBG graphs relate to unipartite graphs with modified connection functions. 
An interesting question in this direction is if and when the properties of the RBG graphs can be well represented by unipartite graphs with some modified connection functions. Our conjecture is that for independent connections between different pairs of agents, such representation is unlikely due to the oblivion of the local dependence (present in the RBG models). 
 Another direction is to consider hybrid models where agents may get infected either through common hubs or direct interactions between agents. The former infection mechanism is more centralized than the latter. 

\section*{Acknowledgment}
This work was partially supported by the MUR under the grant “Dipartimenti di Eccellenza 2023-2027" of the DISCo Department of the University of Milano - Bicocca; and by the Engineered MachinE Learning-intensive IoT systems (EMELIOT) national research project, funded by the MUR under the PRIN 2020 program (Contract 2020W3A5FY).

\bibliographystyle{IEEEtran}
  \bibliography{main}

%\vspace{12pt}


\end{document}
