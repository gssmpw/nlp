\section{Problem Description} 
\label{sec:problem_desc}
We assume the availability of a source dataset \scalebox{0.8}{{$\mathcal{D_S}= \{(x_{i}^{\mathcal{S}}, u^{\mathcal{S}}_{i}, {x^{\prime}_{i}}^{\mathcal{S}},$}} \scalebox{0.8}{{$ {r}_{i}^{\mathcal{S}})\}_{i=1}^{N} \sim \mathcal{P}_{\mathcal{S}}$}} and a target dataset \scalebox{0.8}{$\mathcal{D_T}=$ $\{(x_{i}^{\mathcal{T}}, u^{\mathcal{T}}_{i}, {x^{\prime}_{i}}^{\mathcal{T}}, {r}_{i}^{\mathcal{T}})\}_{i=1}^{M} \sim \mathcal{P}_{\mathcal{T}}$}, where $x \in \mathbb{R}^{m}$ is a state; $u \in \mathbb{R}^{n}$ is the action that the agent performs at the state $x$; $r \in \mathbb{R}$ is the reward that the agent gets by performing the action $u$ in the state $x$; and $x^{\prime} \in \mathbb{R}^{m}$ is the state that the agent transitions to (i.e., next state). We also assume that the target dataset $\mathcal{D_T}$ is much smaller than the the source dataset $\mathcal{D_S}$, therefore $N \gg M$. Furthermore, the source distribution $\mathcal{P}_{\mathcal{S}}$ can be different from the target distribution $\mathcal{P}_{\mathcal{T}}$ (i.e. $\mathcal{P}_{\mathcal{S}} \neq \mathcal{P}_{\mathcal{T}}$), confronting the agent with the source-target domain mismatch problem. 


Moreover, while intrinsic reward functions may be the same across source and target domains (i.e., $r^{\mathcal{S}} = r^{\mathcal{T}}$), modifications to the reward functions in either domain can lead to effective differences (i.e., $r^{\mathcal{S}} \neq r^{\mathcal{T}}$). We assume that the reward functions can potentially be different across the domains. These potential differences in reward functions could exacerbate the domain mismatch and introduce additional challenges for effective knowledge transfer. Each trajectory $\tau_k^{\mathcal{S}}$ in $\mathcal{D_S}$ is defined as a sequence of transitions \scalebox{0.8}{$\tau_k^{\mathcal{S}} = \{(x_{i}^{k,\mathcal{S}}, u_{i}^{k,\mathcal{S}}, {x^{\prime}_{i}}^{k,\mathcal{S}}, r_{i}^{k,\mathcal{S}})\}_{i=1}^{L_k^{\mathcal{S}}}$}, where $L_k^{\mathcal{S}}$ denotes the length of $k$-th trajectory. The degree of similarity between these transitions and those in the target dataset $\mathcal{D_T}$ varies.

The goal is to identify high-quality trajectories by quantifying the similarity of each trajectory $\tau_k^{\mathcal{S}}$ from the source dataset to the target dataset $\mathcal{D_T}$, aggregating the similarities of the individual transitions within $\tau_k^{\mathcal{S}}$ to those in $\mathcal{D_T}$, and selecting trajectories most relevant for effective knowledge transfer.



