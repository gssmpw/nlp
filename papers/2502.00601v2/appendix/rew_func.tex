\subsection{Choice of Reward Function in CLTV}
\label{rew-choice}
To demonstrate the effectiveness of our reward function, we conducted two comparative experiments: one using the temporal difference (TD)~\citep{tdl} and the other using the reward shaping (RS)~\citep{pbrs} to modify the rewards of transitions. 

The results of the comparison are presented in \autoref{tab:reward-functions}, where CLTV-TD corresponds to the CLTV model using TD as the reward function (\( r_{TD} \)), while CLTV-RS uses RS as the reward function (\( r_{RS} \)). 


The TD reward function is presented in \autoref{td-rf}:

\begin{equation}
\label{td-rf}
r_{\mathrm{TD}}=V(x)+\alpha\left(r+\gamma V\left(x^{\prime}\right)-V(x)\right)
\end{equation}

Similarly, the RS reward function is presented in \autoref{rs-rf}:

\begin{equation}
\label{rs-rf}
r_{\mathrm{RS}}=\gamma V\left(x^{\prime}\right)-V(x)
\end{equation}







\autoref{tab:reward-functions} compares the performance of CLTV with its variants, CLTV-TD and CLTV-RS, across all considered environments and datasets, using CQL and IQL as base algorithms.

In the Ant domain, CLTV achieves higher normalized scores with both CQL and IQL algorithms. For example, on the random-medium dataset, CLTV reaches a score of \(97.48\), compared to CLTV-TD's \(73.36\) and CLTV-RS's \(73.50\). Similarly, in the random-expert dataset, CLTV attains \(115.86\), outperforming the other methods. 

In the HalfCheetah domain, CLTV also exhibits superior performance, particularly on the medium-expert dataset, where it achieves \(59.88\) using the CQL algorithm. This result is higher than the scores of \(0.50\) and \(28.31\) recorded for CLTV-TD and CLTV-RS, respectively. For the IQL algorithm, CLTV performs well, reaching a score of \(77.10\) and clearly surpassing the other methods. 

The Walker2d and Hopper domains further highlight the advantages of CLTV. In the random-expert dataset of Walker2d, CLTV achieves \(97.43\) using CQL, outperforming both variants by a substantial margin. Similarly, in the Hopper domain, CLTV outperforms its counterparts on the medium-expert dataset, achieving \(83.44\) compared to the lower scores of the TD and RS variants. 

The results show that our reward function outperforms the other two in almost all cases. This suggests that our reward function is particularly effective for domain transfer learning.


\begin{table}[!ht]
\caption{{Performance of our CLTV method compared to two variants: CLTV-TD, which uses temporal difference (TD), and CLTV-RS, which uses reward shaping (RS). Normalized scores with standard deviations over 100 episodes and 5 seeds on mixed D4RL datasets are reported using the base algorithms CQL and IQL. The highest scores are highlighted in blue.}}
\label{tab:reward-functions}
\centering
\resizebox{\linewidth}{!}{
\renewcommand{\arraystretch}{1.4}
\begin{tabular}{@{}ccllll@{}}
\toprule
\textbf{Domain} & \textbf{\shortstack{RL \\ Algorithm}} & \textbf{Method} & \multicolumn{3}{c}{\textbf{Dataset}}\\
\multicolumn{1}{l}{} & \multicolumn{1}{l}{} & \multicolumn{1}{l}{} & \multicolumn{1}{c}{\textbf{random-medium}} & \textbf{random-expert} & \textbf{medium-expert}\\
\hline
\multirow{6}{*}{\rotatebox{90}{\Large Ant}} & \multirow{3}{*}{CQL} & CLTV-TD &  73.36 ± 6.72 & 95.08 ± 23.14 & 1.17 ± 46.66 \\
 &  & CLTV-RS & 73.50 ± 4.48 & 62.98 ± 12.38 & 6.48 ± 26.03 \\
 &  & CLTV & \paddedcolorbox{LightCyan}{97.48} ± 3.63 & \paddedcolorbox{LightCyan}{115.86} ± 6.82 & \paddedcolorbox{LightCyan}{24.84} ± 13.70 \\ \cmidrule{3-6}
 & \multirow{3}{*}{IQL} & CLTV-TD & 18.78 ± 8.78 & 7.87 ± 1.94 & 113.07 ± 3.89 \\
 &  & CLTV-RS & 75.11 ± 7.19 & 50.80 ± 11.16 & 111.82 ± 3.99 \\
 &  & CLTV & \paddedcolorbox{LightCyan}{78.67} ± 8.26 & \paddedcolorbox{LightCyan}{88.26} ± 4.67 & \paddedcolorbox{LightCyan}{117.00} ± 7.28 \\ 
  \hline
\multirow{6}{*}{\rotatebox{90}{\Large HalfCheetah}} & \multirow{3}{*}{CQL} & CLTV-TD & 36.87 ± 1.73 & 2.69 ± 2.20 & 0.50 ± 3.82 \\
 &  & CLTV-RS & 36.44 ± 2.07 & 0.11 ± 1.33 & 28.31 ± 7.40 \\
 &  & CLTV & \paddedcolorbox{LightCyan}{44.13} ± 3.47 & \paddedcolorbox{LightCyan}{10.37} ± 2.51 & \paddedcolorbox{LightCyan}{59.88} ± 7.91 \\ \cmidrule{3-6}
 & \multirow{3}{*}{IQL} & CLTV-TD & 3.94 ± 2.28 & 6.65 ± 2.60  & 46.52 ± 0.00 \\
 &  & CLTV-RS & 36.78 ± 1.33 & 8.91 ± 3.32 & 51.45 ± 3.03 \\
 &  & CLTV & \paddedcolorbox{LightCyan}{41.83} ± 0.63 & \paddedcolorbox{LightCyan}{16.28} ± 7.22 & \paddedcolorbox{LightCyan}{77.10} ± 5.16 \\ 
 \hline
\multirow{6}{*}{\rotatebox{90}{\Large Hopper}} & \multirow{3}{*}{CQL} & CLTV-TD & 7.30 ± 10.74 & 7.42 ± 6.90 & 30.42 ± 35.61 \\
 &  & CLTV-RS & 12.28 ± 11.99 & 6.79 ± 0.00 & 64.48 ± 35.66 \\
 &  & CLTV & \paddedcolorbox{LightCyan}{51.04} ± 3.21 & \paddedcolorbox{LightCyan}{56.23} ± 15.14 & \paddedcolorbox{LightCyan}{83.44} ± 16.95 \\ \cmidrule{3-6}
 & \multirow{3}{*}{IQL} & CLTV-TD & 0.19 ± 0.00 &  0.17 ± 0.00 & 42.31 ± 16.43 \\
 &  & CLTV-RS & 61.95 ± 6.24 & 30.79 ± 2.23 & 32.53 ± 12.08 \\
 &  & CLTV & \paddedcolorbox{LightCyan}{55.79} ± 4.44 & \paddedcolorbox{LightCyan}{39.56} ± 2.78 & \paddedcolorbox{LightCyan}{70.73} ± 4.11 \\ 
 \hline
\multirow{6}{*}{\rotatebox{90}{\Large Walker2d}} & \multirow{3}{*}{CQL} & CLTV-TD & 26.68 ± 20.97 & 74.62 ± 14.59 & 7.06 ± 4.56 \\
 &  & CLTV-RS & 41.53 ± 10.37 & 66.38 ± 40.72 & 2.59 ± 1.55 \\
 &  & CLTV & \paddedcolorbox{LightCyan}{70.45} ± 8.72 & \paddedcolorbox{LightCyan}{97.43} ± 7.74 & \paddedcolorbox{LightCyan}{30.23} ± 41.25 \\ \cmidrule{3-6}
 & \multirow{3}{*}{IQL} & CLTV-TD & 39.59 ± 13.09 & 18.84 ± 19.42 & 73.06 ± 5.16 \\
 &  & CLTV-RS & 64.97 ± 5.54 & 35.34 ± 17.71 & 89.05 ± 9.94 \\
 &  & CLTV & \paddedcolorbox{LightCyan}{68.37} ± 4.12 & \paddedcolorbox{LightCyan}{89.51} ± 9.03 & \paddedcolorbox{LightCyan}{110.74} ± 0.66 \\ 
\bottomrule
\end{tabular}
}
\end{table}