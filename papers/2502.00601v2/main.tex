\documentclass[sigconf,nonacm]{aamas}
\usepackage{balance}
\usepackage{natbib}
\usepackage[utf8]{inputenc}
\usepackage[T1]{fontenc}
\usepackage{url}
\usepackage{amsfonts}
\usepackage{nicefrac}
\usepackage{microtype}
\usepackage{xcolor}
\usepackage{afterpage}
\usepackage{cuted}
\usepackage{color, colortbl}
\usepackage{graphicx}
\usepackage{wrapfig,lipsum,booktabs}
\usepackage{bm}
\usepackage{tikz}
\usepackage{algorithm}
\usepackage[noend]{algpseudocode}
\usepackage{mdframed}
\usepackage{enumitem}
\usepackage{makecell}
\usepackage{float}
\usepackage{adjustbox}
\usepackage{multirow}
\usepackage{tabularx}
\usepackage{mathtools}
\usepackage[figuresright]{rotating}
\usepackage{pdflscape}
\usepackage{float}
\usepackage{cleveref}
\usepackage[english]{babel}
\usepackage{amsthm}


\let\labelindent\relax
\newcommand{\commentsymbol}{//}
\algrenewcommand\algorithmiccomment[1]{\hfill \commentsymbol{} #1}
\makeatletter
\newcommand{\LineComment}[2][\algorithmicindent]{\Statex \hspace{#1}\commentsymbol{} #2}
\makeatother
\newcommand{\varfont}{\texttt}

\newcommand\Tau{\mathrm{T}}

\newcommand{\eg}{e.\,g.,\ }
\newcommand{\ie}{i.\,e.,\ }
\newcommand{\wrt}{w.\,r.\,t.\ }
\newcommand{\et}{{et al.\ }}
\newcommand{\cf}{{cf.\,}}

\usepackage{breqn}
\DeclareMathOperator{\E}{\mathbb{E}}

\usepackage[english]{babel}


\def\sectionautorefname{Section}
\def\subsectionautorefname{Section}
\renewcommand{\arraystretch}{1.4}


\addto\extrasenglish{
  \def\algorithmautorefname{Algorithm}
  \def\theoremautorefname{Theorem}
  \def\lemmaautorefname{Lemma}
  \def\corollaryautorefname{Corollary}
}
\newtheorem*{remark}{Remark}

\usepackage{etoolbox}
\newtoggle{inappendix}
\togglefalse{inappendix}

\apptocmd{\appendix}{\toggletrue{inappendix}}{}{\errmessage{failed to patch \appendix}}

\addto\extrasenglish{%
  \def\appendixautorefname{Appendix}
}

\makeatletter
\patchcmd{\hyper@makecurrent}{%
    \ifx\Hy@param\Hy@chapterstring
        \let\Hy@param\Hy@chapapp
    \fi
}{%
    \iftoggle{inappendix}{
        \@checkappendixparam{chapter}%
        \@checkappendixparam{section}%
        \@checkappendixparam{subsection}%
        \@checkappendixparam{subsubsection}%
    }{}%
}{}{\errmessage{failed to patch}}

\newcommand*{\@checkappendixparam}[1]{%
    \def\@checkappendixparamtmp{#1}%
    \ifx\Hy@param\@checkappendixparamtmp
        \let\Hy@param\Hy@appendixstring
    \fi
}
\makeatother





\newcommand{\triangleqq}{\coloneqq}
\DeclarePairedDelimiterX{\infdivx}[2]{(}{)}{%
  #1\;\delimsize\|\;#2%
}

\newcommand{\infdiv}{D_{\mathrm{KL}}\infdivx}

\DeclarePairedDelimiter{\norm}{\lVert}{\rVert}



\newtheorem{example}{Example}
\newtheorem{theorem}{Theorem}
\newtheorem{corollary}{Corollary}
\newtheorem{lemma}{Lemma}

\definecolor{LightCyan}{rgb}{0.75,0.9,1}
\usepackage{caption} 
\captionsetup[table]{skip=6pt}


\newcommand{\paddedcolorbox}[2]{
  \begingroup
  \setlength{\fboxsep}{5pt}
  \colorbox{#1}{#2}
  \endgroup
}

\usepackage{paracol}
\usepackage{afterpage}
\usepackage{blindtext}

\newcommand{\fillrightcolumn}{
  \noindent
  \begin{minipage}[t][\textheight][t]{\columnwidth}
    \null
  \end{minipage}
}




\makeatletter
\gdef\@copyrightpermission{
  \begin{minipage}{0.2\columnwidth}
   \href{https://creativecommons.org/licenses/by/4.0/}{\includegraphics[width=0.90\textwidth]{by}}
  \end{minipage}\hfill
  \begin{minipage}{0.8\columnwidth}
   \href{https://creativecommons.org/licenses/by/4.0/}{This work is licensed under a Creative Commons Attribution International 4.0 License.}
  \end{minipage}
  \vspace{5pt}
}
\makeatother

\setcopyright{ifaamas}
\acmConference[AAMAS '25]{Proc.\@ of the 24th International Conference
on Autonomous Agents and Multiagent Systems (AAMAS 2025)}{May 19 -- 23, 2025}
{Detroit, Michigan, USA}{Y.~Vorobeychik, S.~Das, A.~Nowé  (eds.)}
\copyrightyear{2025}
\acmYear{2025}
\acmDOI{}
\acmPrice{}
\acmISBN{}



\acmSubmissionID{6}

\title{Enhancing Offline Reinforcement Learning with Curriculum Learning-Based Trajectory Valuation}


\author{Amir Abolfazli}
\affiliation{
  \institution{L3S Research Center}
  \city{Hannover}
  \country{Germany}}
\email{abolfazli@l3s.de}

\author{Zekun Song}
\affiliation{
  \institution{Technical University of Berlin}
  \city{Berlin}
  \country{Germany}}
\email{zekun.song@tu-berlin.de}


\author{Avishek Anand}
\affiliation{
  \institution{Delft University of Technology}
  \city{Delft}
  \country{Netherlands}}
\email{avishek.anand@tudelft.nl}


\author{Wolfgang Nejdl}
\affiliation{
  \institution{L3S Research Center}
  \city{Hannover}
  \country{Germany}}
\email{nejdl@l3s.de}

\begin{abstract}
The success of deep reinforcement learning (DRL) relies on the availability and quality of training data, often requiring extensive interactions with specific environments. In many real-world scenarios, where data collection is costly and risky, offline reinforcement learning (RL) offers a solution by utilizing data collected by domain experts and searching for a batch-constrained optimal policy. This approach is further augmented by incorporating external data sources, expanding the range and diversity of data collection possibilities. However, existing offline RL methods often struggle with challenges posed by non-matching data from these external sources. In this work, we specifically address the problem of source-target domain mismatch in scenarios involving mixed datasets, characterized by a predominance of source data generated from random or suboptimal policies and a limited amount of target data generated from higher-quality policies. To tackle this problem, we introduce Transition Scoring (TS), a novel method that assigns scores to transitions based on their similarity to the target domain, and propose Curriculum Learning-Based Trajectory Valuation (CLTV), which effectively leverages these transition scores to identify and prioritize high-quality trajectories through a curriculum learning approach. Our extensive experiments across various offline RL methods and MuJoCo environments, complemented by rigorous theoretical analysis, demonstrate that CLTV enhances the overall performance and transferability of policies learned by offline RL algorithms.
\end{abstract}


\keywords{Offline Reinforcement Learning, Trajectory Valuation}

  
\newcommand{\BibTeX}{\rm B\kern-.05em{\sc i\kern-.025em b}\kern-.08em\TeX}


\begin{document}


\pagestyle{fancy}
\fancyhead{}


\maketitle 

%%%%%%%%%%%%%%%%%%%%%%%%%%%%%%%%%%%%%%%%%%%%%%%%%%%%%%%%%%%%%%%%%%%%%%%%


% 
% 
The widespread integration of communication networks and smart devices in modern control systems has increased the vulnerability of industrial systems to online cyber-attacks, e.g., Industroyer, Blackenergy, etc \citep{osti_1505628}.
% Modern control systems have seen a large push to include communication networks and smart devices to increase performance, made possible by improvements in communication device cost and energy consumption. This trend has been coupled with the usage of open-standard communication protocols among industrial control systems, making them vulnerable to online cyber-attacks such as Industroyer, Blackenergy, etc \citep{osti_1505628}. 
To counter this, methods have been developed to improve security by achieving attack detection, mitigation, and monitoring, among others \citep{sandberg2022secure}. This paper focuses on active attack diagnosis to mitigate stealthy attacks. 
%
%\subsection{Literature review}

Active diagnosis techniques rely on the inclusion of additional moduli to control systems
% inclusion within the control system of additional moduli 
to alter the behavior of the system compared to information known by the attacker. 
For instance, the concept of additive watermarking was introduced in \cite{mo2015physical}, where noise signals of known mean and variance are added at the plant and compensated for it at the controller. 
This compensation, however, is not exact, causing some performance degradation. Thus, trade-offs between performance and detectability  are necessary \citep{zhu2023detection}.
% A later work \citep{zhu2023detection} designs the watermark signal by trading performance for detection. Thus, although additive watermarking serves as a good detection scheme, they endure performance losses even in the nominal case. 

In encrypted control \citep{darup2021encrypted}, the sensor data is encrypted, sent to the controller, and then operated on directly. Encrypted input signals are sent back to the plant for decryption. Although encryption is widespread in IT security, in control systems it presents some concerns, such as the introduction of time delays \citep{stabile2024verifiable}, while it may present inherent weaknesses \citep{alisic2023model}.
% they are not preferred as they introduce time delays \citep{stabile2024verifiable} which can cause instability, and some encryption schemes can be very weak  \citep{alisic2023model}. 

In moving target defense \citep{griffioen2020moving}, the plant is augmented with fictitious dynamics, known to the controller. The plant output is transmitted to the controller along with the fictitious states over a network under attack. 
The additional measurements then aide in the detection of attacks. 
This comes at the cost of higher communication bandwidth needs, which increases rapidly with the dimension of the augmented systems.
% Since the dynamics of the fictitious dynamics are exactly known to the controller, the attack is detected easily. However, when the scale of the system increases, the communication bandwidth used by moving the target defense approach increases rapidly. 

Other recently proposed works include two-way coding \citep{fang2019two}, a weak encryuption technique, and dynamic masking \citep{abdalmoaty2023privacy}, which enhances privacy as well as security, have been shown to be effective against zero-dynamics attacks.
% Two-way coding \citep{fang2019two} and dynamic masking \citep{abdalmoaty2023privacy} are other recently proposed approaches. Two-way coding is another form of weak encryption technique whilst dynamic masking proposes an architecture that enhances both privacy and security. These schemes are shown to be effective against zero dynamics attacks but remain to be studied for other classes of attacks. 
% Recent extensions include \citep{mukherjee2021secure,ramos2024privacy}.
% Some other works which are related are \citep{mukherjee2021secure}, an extension of \cite{fang2019two}. The work \citep{ramos2024privacy} is an extension of moving target defense for multi-agent systems. 
Furthermore, filtering techniques for attack detection are proposed by \cite{murguia2020security,hashemi2022codesign,escudero2023safety}, while not focusing on stealthy attacks.
% The works \citep{murguia2020security,hashemi2022codesign,escudero2023safety} develop filtering techniques to guarantee safety, without being focused on stealthy covert attacks.

Multiplicative watermarking (mWM) has been proposed by the authors as a diagnosis technique \citep{ferrari2020switching}. mWM consists of a pair of filters on each communication channel between the plant and its controller; the scheme is affine to weak encryption, whereby ``encoding'' and ``decoding'' are done by changing signals' dynamic characteristics through inverse pairs of filters. This enables original signals to be recovered exactly, and thus does not lead to performance degradation.
% A multiplicative watermark is an affine to a weak encryption technique, through which the signal is ``encoded'' by a filter, changing its dynamic behavior. The use of inverse pairs means that the original signal can be recovered, through ``decoding'' via an inverse filter. As such, differently to techniques based on additive watermarking, no performance is lost due to the injection of noise, and there are no bandwidth limitations.

%\subsection{Contributions}
One of the critical features of multiplicative watermarking is that to detect stealthy attacks, the mWM filter parameters must be switched over time. In this paper, an algorithm to optimally design the mWM parameters after a switching event is presented, enhancing detection performance, without changing the switching time.
% This is done without changing the switching time, which is taken as given.

\textcolor{black}{
To formalize the filter design problem, we suppose the defender is interested in optimal performance against adversaries injecting covert attacks with matched system parameters \citep{smith2015covert}, including the mWM parameters prior to the switch. This scenario represents a worst case where malicious agents can take full control of the system while remaining undetected.
Thus, the attack strategy is explicitly included within the formulation of the closed-loop system, and the mWM filters are chosen by solving an optimization problem minimizing the attack-energy-constrained output-to-output gain (AEC-OOG) \citep{anand2023risk}, a variation of the output-to-output gain proposed in  \cite{teixeira2015strategic}.
}
The main contributions of this paper are:
% We consider an adversary injecting a covert attack with matched system parameters \citep{smith2015covert}, i.e., an attacker with full knowledge of the control system parameters, including those of the mWM filters before the switch. This scenario is taken as a worst case, as it has been shown that this class of attacks can be made stealthy. To quantitatively define a cost, the output-to-output gain (OOG) \citep{teixeira2015strategic} is leveraged,
% a metric introduced to evaluate the impact of an additive attack in a control system. %Specifically, OOG evaluates the worst-case performance loss that an attacker injecting an undetectable attack can obtain. 
% Here, the maximum performance loss caused by a stealthy adversary with limited energy is taken, the attack-energy-constrained OOG (AEC-OOG) \citep{anand2023risk}. The main contributions of this paper are:
\begin{enumerate}
%[label=\alph*.]
\item The problem of optimally designing the switching mWM filters is formulated as an optimization problem, with the AEC-OOG is taken as the objective;%where the AEC-OOG is taken as the impact metric; 
\item The worst-case scenario of a covert attack with exact knowledge of plant and mWM filter parameters is embedded within the design problem;
% The optimization problem is defined to incorporate the worst-case scenario of a covert attack with exact knowledge of plant and mWM filter parameters;
\item The feasibility of the optimization problem is shown to be dependent only on stability conditions; 
\item A solution scheme is proposed to promote randomization of the mWM filter parameters such that an eavesdropping adversary cannot remain stealthy.
\end{enumerate} 

This builds on the results of \cite{ferrari2020switching}, where the focus was on the design of the switching protocols, rather than the parameters themselves.
Compared to previous work \citep{gallo2021design}, this paper introduces an optimization problem which is always feasible (thanks to the use of AEC-OOG in the objective), while also considering a more sophisticated class of covert attacks, where the presence of watermark is known to the adversary. 
Moreover, this paper poses a different objective than \citep{zhang2023hybrid}; indeed, while \citep{zhang2023hybrid} provided a design strategy to ensure certain privacy properties, in this paper we address the problem of optimal parameter design following a switching event.


%\subsection{Organization}
The rest of the paper is organized as follows. 
After formulating the problem in Section~\ref{sec:PF}, we propose our design algorithm in Section~\ref{sec:main}, and analyze its properties. It is then evaluated through a numerical example in Section~\ref{sec:NE}, and concluding remarks are given Section~\ref{sec:Con}.
% We provide the problem background in Section~\ref{sec:PF}. We formulate the design problem in Section~\ref{sec:main}, together with an analysis of its properties. The proposed algorithm is evaluated through a numerical example in Section \ref{sec:NE}. Concluding remarks are offered in Section \ref{sec:Con}.
\section{Mobile Networks Powered by \glspl{LLM}}
\label{sec:LLM_enabled_MNs}
\begin{figure*}[t!]
\centering
\includegraphics[width=.99\textwidth]{Fig1.eps}
    \caption{Possible architectural designs for integrated \gls{LLM} and \gls{MNO} ecosystem.}
    \label{fig:LLM_possible_architectures}
\end{figure*}
The historical data of the \gls{MNO}, archived over years of expertise, constitutes a solid foundation for training the \gls{LLM} using structured and unstructured multi-modal inputs (as illustrated in Fig.~\ref{fig:LLM_possible_architectures}a) such as user intents, network logs, alarm descriptions, trouble tickets, \gls{PCAP} files (e.g. from Wireshark or tcpdump), dashboard screenshots, audio recordings (e.g. from \gls{IVR} systems), video feeds (e.g. from infrastructure surveillance), and \gls{NWDAF} analytics. To this end, a separate collection framework aggregates data from various sources into a centralized repository, and extracts most informative features such as warnings, error codes, timestamps, and user/gNB/session/bearer/\gls{QoS} flow/slice IDs. The extracted features are then converted into unified embeddings that are combined into a common vector space with suitable metadata (e.g. to differentiate data formats). The resulting vector store is used to fine-tune the \gls{LLM} to deeply internalize \gls{MNO}-specific knowledge \cite{Bariah2023understanding}. This allows the \gls{LLM} to learn patterns, sequences, and deviations that correlate with normal or faulty network operations. This is made possible using a timestamp-based cross-referencing to link different entries from several data sources, allowing detailed description and context for each flagged event as well as the resolution workflow for the spotted anomalies.

In live mobile networks, fresh multi-modal data is continuously fed into the \gls{LLM}, either uploaded in batches or streamed in real-time. The \gls{LLM} analyzes this data and identifies potential anomalous behaviors in light of its accumulated learning. In case of new anomalies not covered during the fine-tuning stage, the \gls{LLM} can rely on clustering techniques to group similar patterns and flag outliers as suspected behaviors. The \gls{LLM} is also capable of using \gls{RAG}-enabled external knowledge databases such as \gls{3GPP} documents \cite{Said2024instruct}, \gls{IEEE} standards, \gls{IETF} RFCs and vendors documentation \cite{soman2023observations} to compare the actual network behavior with the expected one to identify misconfigurations and spot unusual trends in protocols and communication flows. Well-crafted prompts, on the other hand, can guide the \gls{LLM} responses to provide focused solutions. Paradigms such as the \gls{CoT} reasoning can be used to break down the \gls{LLM} insights into a series of simplified and actionable sub-tasks. It can be extended by the \gls{ToT} technique to explore different reasoning paths and identify the most optimal solution. The \gls{LLM} can naturally produce stepwise reasoning if datasets used for fine-tuning contain \gls{CoT} and \gls{ToT} examples, or through creative prompting \cite{Zhou2024survey}. In parallel, \gls{NOC} engineers can intervene to confirm, guide or reject the \gls{LLM} findings, if needed, e.g. using its intuitive conversational interface. Through continuous self-learning, the \gls{LLM} will dynamically adapt to evolving network conditions, optimizing its performance over time \cite{Chaparadza2023optimization}.

%For instance, when a network experiences latency issues, the \gls{LLM} seamlessly analyze multi-modal information from diverse origins to identify the root cause, e.g. overloaded \gls{UPF} due to insufficient capacity, and then suggest a solution, e.g. step-by-step instructions including suitable code scripts for the involved \glspl{NF} to autonomously reroute traffic or modify policies. Conventional 5G networks can only alert about anomalies using suitable \gls{NWDAF} analytics that track the violated thresholds and notify the \gls{OAM} center to display the details on complex dashboards.

By incorporating \glspl{LLM} (e.g. as \glspl{NF}) into upcoming 6G networks, expected to be designed with \gls{SbD} principles \cite{Khaloopour2024Resilience}, \glspl{LLM} will naturally inherit the same built-in security safeguards rather than adding them as an afterthought. This design-driven approach focuses on proactive threat management, end-to-end encryption, authentication, network slicing isolation, \gls{AI}-driven threat detection with automated reactions, and stateless designs, fostering a resilient \gls{LLM}.
%The design-driven security in 5G and upcoming 6G networks ensures that security is natively integrated at every layer of the architecture rather than added as an afterthought. This approach focuses on proactive threat management, end-to-end encryption, authentication, network slicing, and \gls{AI}-driven threat detection and automated reactions to counter evolving cyber threats.



\section{Problem Description} 
\label{sec:problem_desc}
We assume the availability of a source dataset \scalebox{0.8}{{$\mathcal{D_S}= \{(x_{i}^{\mathcal{S}}, u^{\mathcal{S}}_{i}, {x^{\prime}_{i}}^{\mathcal{S}},$}} \scalebox{0.8}{{$ {r}_{i}^{\mathcal{S}})\}_{i=1}^{N} \sim \mathcal{P}_{\mathcal{S}}$}} and a target dataset \scalebox{0.8}{$\mathcal{D_T}=$ $\{(x_{i}^{\mathcal{T}}, u^{\mathcal{T}}_{i}, {x^{\prime}_{i}}^{\mathcal{T}}, {r}_{i}^{\mathcal{T}})\}_{i=1}^{M} \sim \mathcal{P}_{\mathcal{T}}$}, where $x \in \mathbb{R}^{m}$ is a state; $u \in \mathbb{R}^{n}$ is the action that the agent performs at the state $x$; $r \in \mathbb{R}$ is the reward that the agent gets by performing the action $u$ in the state $x$; and $x^{\prime} \in \mathbb{R}^{m}$ is the state that the agent transitions to (i.e., next state). We also assume that the target dataset $\mathcal{D_T}$ is much smaller than the the source dataset $\mathcal{D_S}$, therefore $N \gg M$. Furthermore, the source distribution $\mathcal{P}_{\mathcal{S}}$ can be different from the target distribution $\mathcal{P}_{\mathcal{T}}$ (i.e. $\mathcal{P}_{\mathcal{S}} \neq \mathcal{P}_{\mathcal{T}}$), confronting the agent with the source-target domain mismatch problem. 


Moreover, while intrinsic reward functions may be the same across source and target domains (i.e., $r^{\mathcal{S}} = r^{\mathcal{T}}$), modifications to the reward functions in either domain can lead to effective differences (i.e., $r^{\mathcal{S}} \neq r^{\mathcal{T}}$). We assume that the reward functions can potentially be different across the domains. These potential differences in reward functions could exacerbate the domain mismatch and introduce additional challenges for effective knowledge transfer. Each trajectory $\tau_k^{\mathcal{S}}$ in $\mathcal{D_S}$ is defined as a sequence of transitions \scalebox{0.8}{$\tau_k^{\mathcal{S}} = \{(x_{i}^{k,\mathcal{S}}, u_{i}^{k,\mathcal{S}}, {x^{\prime}_{i}}^{k,\mathcal{S}}, r_{i}^{k,\mathcal{S}})\}_{i=1}^{L_k^{\mathcal{S}}}$}, where $L_k^{\mathcal{S}}$ denotes the length of $k$-th trajectory. The degree of similarity between these transitions and those in the target dataset $\mathcal{D_T}$ varies.

The goal is to identify high-quality trajectories by quantifying the similarity of each trajectory $\tau_k^{\mathcal{S}}$ from the source dataset to the target dataset $\mathcal{D_T}$, aggregating the similarities of the individual transitions within $\tau_k^{\mathcal{S}}$ to those in $\mathcal{D_T}$, and selecting trajectories most relevant for effective knowledge transfer.




\section{Related Work}
\subsection{Multimodal Large Language Models}
% Building on the success of large language models (LLMs) \citep{yao2024tree, glm2024chatglm, achiam2023gpt, touvron2023llama, brown2020language}, multimodal large language models (MLLMs) \citep{liu2024improved, li2023blip, zhu2023minigpt, wang2023cogvlm, liu2024visual} extend these capabilities by integrating vision and text processing, achieving remarkable performance in tasks involving images, videos, and multimodal reasoning. However, handling visual data poses computational challenges due to the redundancy and low information density of high-resolution tokens \citep{liang2022evit} and the quadratic scaling of attention mechanisms \citep{vaswani2017attention}.
% For instance, models like LLaVA \citep{liu2023improvedllava} and mini-Gemini-HD \citep{li2024mini} encode high-resolution images into thousands of tokens, while video-based models such as VideoLLaVA \citep{lin2023video} and VideoPoet \citep{kondratyuk2023videopoet} allocate even more tokens to process multiple frames. These challenges highlight the need for more efficient token representations and longer context lengths to enable scalability. Recent advancements, such as Gemini \citep{geminiteam2023gemini} and LWM \citep{liu2024world}, have focused on addressing these issues by optimizing token efficiency and extending the context length, paving the way for more scalable and effective MLLMs.

The remarkable success of large language models (LLMs) \citep{radford2019language, brown2020language} has spurred a growing trend of extending their advanced reasoning capabilities to multi-modal tasks, leading to the development of vision-language models (VLMs) \citep{huang2023languageneedaligningperception, driess2023palmeembodiedmultimodallanguage, liu2024visual, Qwen-VL}. These VLMs typically consist of a visual encoder \citep{radford2021learning} that serializes input image representations and an LLM responsible for text generation. To enable the LLM to process visual inputs, an alignment module is employed to bridge the gap between visual and textual modalities. This module can take various forms, such as a simple linear layer, an MLP projector, or a more complex query-based network. While this integration allows the LLM to gain visual perception, it also introduces significant computational challenges due to the long sequences of visual tokens.

Moreover, existing VLMs often exhibit limitations, such as visual shortcomings or hallucinations, which hinder their performance. Efforts to enhance VLM capabilities by increasing input image resolution have further exacerbated computational demands. For instance, encoding higher-resolution images results in a substantial increase in the number of visual tokens. A model like LLaVA-1.5 \citep{liu2024improved} generates 576 visual tokens for a single image, while its successor, LLaVA-NeXT \citep{liu2024llavanext}, produces up to 2880 tokens at double the resolution, far exceeding the length of typical textual prompts.
Optimizing the inference efficiency of VLMs is thus a critical task to facilitate their deployment in real-world scenarios with limited computational resources.

\subsection{Visual Token Compression}
% Visual tokens often exceed text tokens by tens to hundreds of times, with visual signals being more spatially redundant compared to information dense text \citep{marr2010vision}.
% Various methods have been proposed to address this issue. For instance, LLaMA-VID \citep{li2023llama} uses a Q-Former with context tokens, and DeCo \citep{yao2024deco} applies adaptive pooling to downsample visual tokens at the patch level.
% However, these approaches require modifying model components and additional training, increasing computational and training costs.
% ToMe~\citep{bolya2022tome} reduces tokens without training by adding a token merge module to ViTs, but this disrupts early cross-modal interactions in language models~\citep{xing2024PyramidDrop}. FastV~\citep{chen2024image} selects important visual tokens using attention scores, while SparseVLM~\citep{zhang2024sparsevlm} incorporates text guidance via cross-modal attention.
% However, these methods forgo flash-attention~\citep{dao2022flashattention, dao2023flashattention2} and primarily focus on token importance, overlooking the impact of token duplication.
% In our work, we preserve hardware acceleration compatibility, including flash attention, while considering both token importance and duplication for token reduction.

Visual tokens are often significantly more numerous than text tokens, with higher spatial redundancy and lower information density. To address this issue, various methods have been proposed for reducing visual token counts in vision language models. For instance, some approaches modify model components, such as using context tokens in Q-Former \citep{li2023llama} or applying adaptive pooling at the patch level, but these typically require additional training and increase computational costs. Other techniques, like Token Merging (ToMe) \citep{bolya2022tome} and FastV \citep{chen2024image}, focus on reducing tokens without retraining by merging tokens or selecting important ones based on attention scores. SparseVLM \cite{zhang2024sparsevlm} incorporates text guidance through cross-modal attention to refine token selection. However, these methods often overlook hardware acceleration compatibility and fail to account for token duplication alongside token importance. Furthermore, while token pruning has been extensively explored in natural language processing and computer vision to improve inference efficiency, its application to VLMs remains under-explored. Existing pruning strategies, such as those in FastV and SparseVLM, rely on text-visual attention within large language models (LLMs) to evaluate token importance, which may not align well with actual visual token relevance.



\begin{figure*}
	\centering
	\includegraphics[width = \linewidth]{figure/AgentArena.pdf}
	\caption{\textbf{Stock Trading Workflow in \textit{Agent Trading Arena}.} 
	\textbf{Top:} Workflow of a trading day, including preparation, trading, and post-trading reflection. Agents discuss insights in the chat pool, analyze market trends, execute trades, and refine strategies based on performance.  
	\textbf{Bottom:} Example of agents' interactions in the chat pool and dynamic strategy updates.}
	\label{fig:AgentArena}
	\vspace{-3pt}
\end{figure*}

\section{Proposed Method}

% 核心部分visual representation,

To mitigate the influence of human prior knowledge and memory, we designed a closed-loop economic system~\citep{guo2024economics} called the \textit{Agent Trading Arena}, a zero-sum game simulating complex, quantitative real-world scenarios. The simulation workflow is illustrated in \autoref{fig:AgentArena} and further detailed in \autoref{appendix_arena}. In the \textit{Agent Trading Arena}, agents can invest in assets, earn dividends from holding assets, and pay daily expenses using virtual currency. The agent with the highest total return wins the game.

\subsection{Agent Trading Arena}

\paragraph{Structure of Agent Trading Arena.} 

To eliminate external knowledge biases, asset prices are determined by a bid-ask system, reflecting the prices at which buyers and sellers are willing to transact. The system evolves solely based on agents' actions and interactions, without external influences. This design ensures that the outcomes of agents' actions are not immediately apparent but unfold gradually, influenced by other agents' decisions.

To encourage active participation, a dividend mechanism is introduced. There are two primary sources of income in this system: capital gains from asset price differentials and dividends from holding assets. Dividends for each asset are distributed according to a predefined ratio, serving as an implicit anchor for asset prices. Agents holding more low-cost assets receive higher dividends. To prevent passive asset holding until the end of the game, agents must pay a daily capital cost proportional to their total wealth. These expenses are offset by asset dividends, and only agents with sufficient low-cost assets can cover costs. Under the pressure of significant daily expenses, agents must act swiftly and strategically, triggering frequent trades and price fluctuations to stimulate market activity. This dynamic mechanism ensures fairness in the zero-sum game while preventing agents from relying on fixed strategies to find optimal solutions.

\vspace{-3pt}

\paragraph{Agents Learn and Compete in Arena.}

The zero-sum game structure is crucial to eliminating the possibility of a universally optimal strategy. In fixed scenarios with a static optimal solution, agents could rely on predefined rules or memory-based approaches, bypassing adaptive decision-making. The zero-sum game ensures that there is no universally correct solution, with outcomes evolving dynamically based on agent interactions and competition. This design forces agents to continually adapt, learn from feedback, and develop context-dependent strategies, promoting deeper environmental exploration and preventing reliance on static or memory-driven solutions.

In the \textit{Agent Trading Arena}, agents are unaware of implicit rules, except for the objective to maximize their virtual wealth throughout the simulation. To win this zero-sum game, agents must effectively learn from experience, decipher hidden game rules, and develop strategies to counter competitors. This requires the ability to comprehend numerical feedback, formulate enduring strategies, and make informed decisions. Unlike other mathematical reasoning problems, the results of their actions unfold gradually and dynamically. Moreover, agents are easily misled by erroneous information from competitors, hindering their ability to discern strategic cues from competitors' textual data. Importantly, agents remain unaware of these implicit rules, so applying real-world knowledge does not benefit their performance. Therefore, agents must rely on experiential learning to decipher the hidden game rules and ultimately achieve victory.

\subsection{Types of Numerical Data Input}

\paragraph{Limitations of Textual Numerical Data.}

In the \textit{Agent Trading Arena}, the generated stock data is stored in numerical format. When used directly as input to an LLM, the models often struggle to interpret numerical data accurately or make sound decisions. To mitigate this, we convert the data into textual formats~\citep{numerical_text, long_text}, enhancing semantic features and clarifying output requirements to improve the models' understanding. During interactions, the LLMs process stock prices, trading volumes, and market indices presented as textual numerical data.

\begin{figure*}
	\centering
	\includegraphics[width = \linewidth]{figure/v_t.pdf}
	\caption{\textbf{Textual and Visual Representations of Corresponding Inputs and Outputs.} The left images display the agent’s Buy and Sell trading records, daily trade prices, and K-line charts for three stocks. The output from visual inputs (bottom right) captures overall stock trends and long-term behavior, while the output from textual inputs (top right) focuses on specific current prices.}
	\label{textual_visualized}
	\vspace{-3pt}
\end{figure*}

However, this textual approach reveals significant limitations. While the data is presented clearly, LLMs tend to focus excessively on specific values rather than identifying long-term trends or global patterns. They also struggle with understanding correlative relations and percentage changes, limiting their ability to assess differences and identify connections between data points. When analyzing time-series data with complex patterns, LLMs often fixate on individual data points, overlooking overarching relations. This issue is evident in the analysis output in the top-right corner of \autoref{textual_visualized}, where LLMs' focus on individual values impedes their ability to generalize, reducing their capacity to extract meaningful global insights.

Additionally, LLMs often overemphasize recent data while undervaluing historical information, even when prompted to consider its importance. This prevents them from effectively integrating past data and recognizing long-term patterns, complicating their understanding of numerical relations and trends. These challenges highlight the need for improved mechanisms to process numerical relations, identify global trends, and derive deeper insights from textual numerical data.

\vspace{-3pt}

\paragraph{Potential of Visual Numerical Data.}

Since textual numerical data often leads LLMs to focus on local details while neglecting broader relations, we investigated whether visual representations, such as scatter plots, line charts, and bar charts, could help LLMs better understand overall trends, similar to human reasoning. Thus, we transition from textual numerical data inputs to visualized formats ~\citep{storyllava}. As demonstrated in the bottom-right corner of \autoref{textual_visualized}, visual representations enable LLMs to more effectively grasp global trends, patterns, and relations that are often difficult to discern from textual numerical data alone.

These findings highlight the advantages of structured, visual numerical data, indicating that this format allows LLMs to more intuitively and comprehensively understand complex data, better capturing overall fluctuations, whereas text tends to focus on local details. By combining visualization and textual representations, LLMs not only overcome the challenges of relations in time-series data but also demonstrate better performance in identifying long-term trends and global patterns, while still attending to local details.

\subsection{Reflection Module}

We propose a strategy distillation method, illustrated in \autoref{fig:reflection}, that delivers real-time feedback to LLMs by analyzing both descriptive textual and visual numerical data. This enables the generation of new strategies and optimization of action plans. The approach allows agents to evaluate their results, refine strategies, and adapt continuously based on feedback. The process begins with assessing the day’s trajectory memory and associated strategies using an evaluation function. The strategic generation process leverages contrastive analysis of peak and nadir performers from the evaluation phase, creating bidirectional learning signals that inform subsequent iterations. This iterative cycle ensures continuous strategy evolution, fostering sustained improvement in decision-making.

\begin{figure}[t]
	\centering
	\includegraphics[width = \linewidth]{figure/reflection.pdf}
	\caption{\textbf{Design of the Reflection Module.} The process evaluates daily trajectory memory and strategies (top right), then generates new strategies (center) based on evaluation, environmental feedback (bottom right), and feedback from the 5 top- and bottom-performing strategies. Stock visualization (bottom left) enhances reflection, driving continuous improvement.}
	%The process evaluates daily trajectory memory and strategies, generating new strategies based on positive and negative feedback from the top- and bottom-performing strategies. Stock visualizations (bottom left) further enhance the reflection process, reinforcing continuous strategy refinement.}
	\label{fig:reflection}
	\vspace{-3pt}
\end{figure}

% We propose a strategy distillation method, illustrated in \autoref{fig:reflection}, that provides real-time feedback to LLMs by analyzing both descriptive textual and visualized numerical data. This enables the generation of new strategies and the optimization of action plans. The approach allows agents to assess their results, refine strategies, and continuously adapt based on feedback. The process begins by evaluating the day's trajectory memory and associated strategies using an evaluation function. From this assessment, new strategies are generated by selecting the top-performing and lowest-performing strategies, offering both positive and negative feedback. This iterative cycle ensures continuous strategy evolution, driving sustained improvement in decision-making.

The reflection module plays a crucial role in refining strategies by offering real-time feedback. It analyzes both descriptive textual and visual numerical data to generate new strategies and optimize action plans. Within the \textit{Agent Trading Arena}, the reflection module is triggered regularly to consolidate daily trading records and evaluate the effectiveness of strategies, refining both successful and unsuccessful experiences to guide future decisions. Ineffective strategies are stored in a strategy library for future reference, allowing agents to review and learn from past experiences. Further details can be found in \autoref{appendix_arena}.

In this section, we empirically compare the proposed algorithm on both sequence windows and time windows with existing methods.
\paragraph{Datasets} For the sequence-based model, we used two synthetic datasets and two cross-language datasets. The statistics of the datasets are provided in Table \ref{table:statistics}:

\begin{table}[t]
    \centering
    \caption{The statistics of the datasets. The datasets satisfy $1 \leq \|\vx\|\|\vy\| \leq R $.}
    \label{table:statistics}
    \begin{tabular}{|c|c|c|c|c|c|}
    \hline
        Dataset & $n$ & $m_x$ & $m_y$ & $N$ & $R$ \\ \hline
        SYNTHETIC(1) & 100,000 & 1,000 & 2,000 & 50,000 & 65 \\ \hline
        SYNTHETIC(2) & 100,000 & 1,000 & 2,000 & 50,000 & 724 \\ \hline
        APR & 23,235 & 28,017 & 42,833 & 10,000 & 773 \\ \hline
        PAN11 & 88,977 & 5,121 & 9,959 & 10,000 & 5,548 \\ \hline
        EURO & 475,834 & 7,247 & 8,768 & 100,000 & 107,840 \\ \hline
    \end{tabular}
\end{table}

\begin{itemize}
    \item Synthetic: The elements of the two synthetic datasets are initially uniformly sampled from the range (0,1), then multiplied by a coefficient to adjust the maximum column squared norm $R$. The X matrix has 1,000 rows, and the Y matrix has 2,000 rows, each with 100,000 columns. The window size is set to 50,000.
    \item APR: The Amazon Product Reviews (APR) dataset is a publicly available collection containing product reviews and related information from the Amazon website. This dataset consists of millions of sentences in both English and French. We structured it into a review matrix where the X matrix has 28,017 rows, and the Y matrix has 42,833 rows, with both matrices sharing 23,235 columns. The window size is 10,000.
    \item PAN11: PANPC-11 (PAN11) is a dataset designed for text analysis, particularly for tasks such as plagiarism detection, author identification, and near-duplicate detection. The dataset includes texts in English and French. The X and Y matrices contain 5,121 and 9,959 rows, respectively, with both matrices having 88,977 columns. The window size is 10,000.
\end{itemize}
We evaluate the time-based model on another real-world dataset:
\begin{itemize}
    \item EURO: The Europarl (EURO) dataset is a widely used multilingual parallel corpus, comprising the proceedings of the European Parliament. We selected a subset of its English and French text portions. The X and Y matrices contain 7,247 and 8,768 rows, respectively, and both matrices share 475,834 columns. Timestamps are generated using the $Poisson$ $Arrival$ $Process$ with a rate parameter of $\lambda=2$. The window size is set to 100,000, with approximately 30,000 columns of data on average in each window.
\end{itemize}

\paragraph{Setup} For the sequence-based model, we compare the proposed hDS-COD and  aDS-COD with EH-COD~\cite{yao2024approximate} and DI-COD~\cite{yao2024approximate}. We do not consider the Sampling algorithm as a baseline, as its performance is inferior to that of EH-COD and DI-CID, as demonstrated in \cite{yao2024approximate}. %The hDS-COD is adjusted by the parameter $\ell$ and the maximum number of levels $L = \log{R}$, where $R$ is the prior estimate of the maximum squared column norm of the dataset. DI-COD similarly requires a prior estimate of $R$ to limit the maximum number of levels $L = \log{(R/\varepsilon})$. In contrast, aDS-COD and EH-COD do not require an estimate of $R$; their error-space balance is controlled by the parameter $\ell = \frac{1}{\varepsilon}$. 
For the time-based model, we compare the proposed hDS-COD and  aDS-COD with EH-COD and the Sampling algorithm since DI-COD cannot be applied to time-based sliding window model. To achieve the same error bound, the maximum number of levels for hDS-COD is set to $L = \log{(\varepsilon NR)}$, and the initial threshold for aDS-COD is set to $1$.

Our experiments aim to illustrate the trade-offs between space and approximation errors. The x-axis represents two metrics for space: final sketch size and total space cost. The final sketch size refers to the number of columns in the result sketches $\mA$ and $\mB$ generated by the algorithm, representing a compression ratio. The total space cost refers to the maximum space required during the algorithm's execution, measured by the number of columns.We evaluate the approximation performance of all algorithms based on correlation errors $\operatorname{corr-err}(\mathbf{X}_W \mathbf{Y}_W^\top, \mathbf{A} \mathbf{B}^\top)$, which is reflected on the y-axis. Every 1,000 iterations, all algorithms query the window and record the average and maximum errors across all sampled windows.

The experiments for all algorithms were conducted using MATLAB (R2023a), with all algorithms running on a Windows server equipped with 32GB of memory and a single processor of Intel i9-13900K.

\paragraph{Performance} Figure \ref{fig:error vs l} and Figure \ref{fig:error vs space} illustrate the space efficiency comparison of the algorithms on sequence-based datasets. Panels (a-d) show the average errors across all sampled windows, while panels (e-h) display the maximum errors.

Figure \ref{fig:error vs l} evaluates the compression effect of the final sketch. The hDS-COD, aDS-COD, and EH-COD show similar compression performances. But the DS series is more stable, particularly on the synthetic datasets, where they significantly outperform EH-COD and DI-COD. The performance of hDS-COD and aDS-COD is nearly the same, indicating that the adaptive threshold trick in aDS-COD does not have a noticeable negative impact on it, maintaining the same error as hDS-COD.

Figure \ref{fig:error vs space} measures the total space cost of the algorithms. hDS-COD and aDS-COD show a significant advantage over existing methods, as they can achieve the  $\varepsilon$-approximation error with much less space. For the same space cost, the correlation errors of hDS-COD and aDS-COD are much smaller than those of EH-COD and DI-COD. Also, aDS-COD has better space efficiency than hDS-COD because aDS only uses a single-level structure while hDS requires $\log R+1$ levels. We find that hDS-COD requires more space on  SYNTHETIC(2) dataset compared to SYNTHETIC(1) dataset. This phenomenon occurs because SYNTHETIC(2) dataset has a larger $R$, which confirms the dependence on $R$ as stated in Theorem~\ref{thm:hds}. 

Figure \ref{fig:time-based} compares the performance of algorithms on time-based windows. Panels (a) and (b) present the error against the final sketch size, which show that our aDS-COD and hDS-COD algorithms enjoy similar performance as EH-COD and significantly outperform the sampling algorithm. On the other hand, as shown in panels (c) and (d), our methods outperform baselines in terms of total space cost.

Software development is increasingly conceived as a collaboration activity between developers and AIs. Indeed, IDEs already implement features to enable interactive development, with AI suggesting implementations that are reused by developers.

Although multiple studies show this interaction can be successful, there is still limited understanding of how the models must be configured and used in the context of code generation tasks. This study addresses this gap, systematically investigating the impact of several key parameters, including the repeated submission of a prompt to accommodate for the non-deterministic nature of the models.

Our study reveals several key findings about the usage of ChatGPT. In particular, we discovered how creativity, although up to a limited extent, is useful to increase the range of methods whose code can be generated correctly. A major role is played by parameter top-p, which is commonly underrated, and instead has a major impact on the correctness of the results, with lower values producing better results. Finally, prompts should be submitted multiple times, with $5$ repetitions combined with a temperature of $1.2$ resulting in an effective configuration in our experiments.  

Future work concerns two main research directions. One is about replicating this experiment with other AI assistants, to validate our findings in multiple contexts. The second research direction concerns finding strategies to deal with the need to submit the same prompt multiple times to obtain a useful result, and thus developing approaches able to select or merge multiple responses automatically. 


%%%%%%%%%%%%%%%%%%%%%%%%%%%%%%%%%%%%%%%%%%%%%%%%%%%%%%%%%%%%%%%%%%%%%%%%

\begin{acks}
This work was supported by the research projects ``QuBRA'' and ``BIFOLD'', funded by the Federal Ministry of Education and Research (BMBF) under grant IDs 13N16052 and BIFOLD24B, respectively.
\end{acks}



%%%%%%%%%%%%%%%%%%%%%%%%%%%%%%%%%%%%%%%%%%%%%%%%%%%%%%%%%%%%%%%%%%%%%%%%


\bibliographystyle{ACM-Reference-Format} 
\bibliography{references}

\clearpage
\newpage

\appendix
\section*{Appendix}
\section{Theoretical analysis}
We prove two theorems that give bounds for the quality of the solutions, which are expected to be reached with the cost functions. We have the same initial assumptions for both proofs, which we state next. We consider that dynamic programming (DP) and greedy algorithms perform as many steps as we perform joins. We assume the reader is familiar with these two algorithms but have included their definitions in the appendix. 

By Def.~\eqref{def:binary_variables} of the binary variables, the number of steps coincides with the rank index in the cost functions. Thus, we can prove the theorems by induction on the rank parameter $r$ in the proposed algorithms, as well as the steps in the DP and greedy algorithms, by showing that for each $r$, there exists a solution from HUBO that corresponds to the same plan computed by the classical algorithms.

\begin{theorem}\label{thm:dp_bound}
Let  $H_{\text{cost}}$ be the cost HUBO defined in Subsection \ref{subsection:cost_function} and let $H_{\text{val}}$ be the binary formulation for the join order validity constraints. Let $x$ be the point that minimizes the full problem $H_{\text{cost}} + CH_{\text{val}}$. Then, the join order cost $H_{\text{cost}}(x)$ is equal to the cost computed by the dynamic programming algorithm without cross-products.
\end{theorem}
\begin{proof}
First, assume that the rank is $0$, and we are at the first iteration in the dynamic programming (DP) algorithm. By Alg.~\ref{alg:hubo_term_construction} (line 5), we include all the joins between the leaf tables to the cost function $H_{\text{cost}}$. Considering the DP algorithm, we compute the same costs for joining the leaf tables and include the combinations and costs in the DP table. Thus, the HUBO encodes the same plans as the DP algorithm.

Let us assume that we are at rank $r > 0$, and we have applied Algorithm \ref{alg:hubo_term_construction} and the DP algorithm up to $r - 1$ steps. The DP algorithm considers all the intermediate joins from the previous step (i.e., at rank $r-1$). For each of these joins, it performs the possible join for each table that is not yet included in the intermediate result with respect to the query graph. These results are kept in the DP table for the next iteration. While the DP algorithm keeps the total cost of each intermediate result in the DP table, our cost objective $H_{\text{cost}}$ encodes only the ''local'' intermediate costs, which are computed with Eq.~\eqref{eq:term_coefficient}. Based on Algorithm \ref{alg:hubo_term_construction}, we take the terms of rank $r-1$ and compute the corresponding new terms (i.e., intermediate join plans) and coefficients (i.e., intermediate join costs) for each table that is not yet included in the intermediate result with respect to the query graph. Instead of computing the total cost, we add the terms and coefficients to the cost objective $H_{\text{cost}}$. This process encodes the same cost that is stored in the DP table because we can choose to activate those $H_{\text{cost}}$ terms, which produce the total cost for each value stored in the DP table. This leads to the claim that there is a point where $H_{\text{cost}}$ achieves the same cost as the DP algorithm.
\end{proof}

\begin{theorem}\label{thm:greedy_bound}
Let $H_{\text{cost}}$ be the heuristic method's cost HUBO defined in Subsection \ref{subsection:heuristic_cost_function}, and let $H_{\text{val}}$ be the binary formulation for the join order validity constraints. Let $x$ be the point that minimizes the full problem $H_{\text{cost}} + CH_{\text{val}}$. Let $C_{\text{greedy}}$ be the cost computed by the greedy algorithm without cross-products. Then, $H(x) \leq C_{\text{greedy}}$, i.e., the cost from the greedy algorithm gives an upper bound for the cost from the heuristic algorithm.
\end{theorem}
\begin{proof}
First, assume that the rank is $0$, and we are at the first iteration in the greedy algorithm. By the definition of the heuristic method in Subsection \ref{subsection:heuristic_cost_function}, we include all terms corresponding to the joins between leaf tables to the cost function $H_{\text{cost}}$. The greedy algorithm includes only one join with the minimum cost to the join tree at the same step. Thus, our heuristic method encodes the first step of the greedy algorithm.

Let us consider a general rank $r > 0$ and assume that we have applied the heuristic HUBO construction and the greedy algorithm up to $r - 1$ steps. By the definition of our heuristic method, we select a subset of $n$-many table combinations from the previous rank $r - 1$ with the smallest coefficients. Simultaneously, in the greedy algorithm, we compute the costs of joining the previous step's intermediate result with the possible joins that are left with respect to the query graph and keep the join tree with the minimum total cost. This minimum total cost is always achieved by including the join tree corresponding to the terms with the smallest coefficient in the $H_{\text{cost}}$ at rank $r$. Since the terms with the smallest coefficients correspond to the cheapest plans, the heuristic algorithm will encode the same plan that the greedy algorithm finds. For larger numbers of $n$, $H_{\text{cost}}$ will also include other intermediate results. Since our heuristic method and the greedy algorithm keep the join plans with the minimum cost at each step, we can deduce that our algorithm encodes the same plan (and others depending on value $n$) as the greedy algorithm. This leads to the claim that $H(x) \leq C_{\text{greedy}}$.
\end{proof}

Besides the theorems, our method archives advantageous variable scalability compared to the most scalable methods \cite{Schonberger_Scherzinger_Mauerer, 10.14778/3632093.3632112}. The model in \cite{DBLP:conf/q-data/SaxenaSS24} uses the same variable definitions as \cite{10.14778/3632093.3632112}, so the scalability comparison also applies to this paper as well. The scalability is visualized in Fig.~\ref{fig:variable_scalability_cycle} for cycle query graphs. The chain, star, and tree graphs have identical relative scalability in all three cases. We want to point out that these are the only mandatory variables for the two compared methods. The previous techniques require more variables to estimate the cost thresholds, which depend on the problem. We excluded the variable scalability of \cite{Nayak_Winker_Groppe_Groppe_2024} since the growth of variable count is exponential in their work.

\begin{figure}
    \centering
    \includegraphics[width=0.4\linewidth]{figures/variable_scalability_cycle.png}
    \caption{We compare the number of mandatory variables in \cite{Schonberger_Scherzinger_Mauerer,10.14778/3632093.3632112} to all variables in our optimization model.}
    \label{fig:variable_scalability_cycle}
    \Description[Comparison of variable scalability between this and previous methods]{The plot shows that our method has the best variable scalability compared to the previous methods.}
\end{figure}
\section{Additional Experimental Results}
\subsection{Runtime Analysis of Offline RL Methods}
\label{runtime-analysis}
\autoref{fig-runtime} shows the runtimes of CLTV compared to Vanilla, CUORL, and Harness across different datasets and offline RL algorithms.

In the Ant domain, CLTV has a higher runtime compared to CQL and IQL. A similar pattern is observed in the Hopper and Walker2d domains, where CLTV’s runtime is generally higher than the other methods across most cases. However, the gains achieved in key learning tasks, particularly in the expert-level datasets, justify the additional computational cost. In the HalfCheetah domain, the runtime differences between CQL and IQL methods are smaller, but CLTV still incurs higher computational overhead compared to other methods, particularly in the random-medium setting. 

In conclusion, while CLTV may not always achieve the shortest runtimes, the performance improvements it provides (as reported in \autoref{tab:normalized-score}) justify the additional computational time. Across different environments and datasets, CLTV offers a reasonable trade-off between runtime and learning performance, making it a practical and effective option for offline RL tasks.

\begin{figure}[!ht]
\center
\includegraphics[width=0.32\textwidth]{figures/runtime.pdf}
\caption{Runtime analysis of offline RL algorithms.}
\label{fig-runtime}
\end{figure}
\subsection{Choice of Reward Function in CLTV}
\label{rew-choice}
To demonstrate the effectiveness of our reward function, we conducted two comparative experiments: one using the temporal difference (TD)~\citep{tdl} and the other using the reward shaping (RS)~\citep{pbrs} to modify the rewards of transitions. 

The results of the comparison are presented in \autoref{tab:reward-functions}, where CLTV-TD corresponds to the CLTV model using TD as the reward function (\( r_{TD} \)), while CLTV-RS uses RS as the reward function (\( r_{RS} \)). 


The TD reward function is presented in \autoref{td-rf}:

\begin{equation}
\label{td-rf}
r_{\mathrm{TD}}=V(x)+\alpha\left(r+\gamma V\left(x^{\prime}\right)-V(x)\right)
\end{equation}

Similarly, the RS reward function is presented in \autoref{rs-rf}:

\begin{equation}
\label{rs-rf}
r_{\mathrm{RS}}=\gamma V\left(x^{\prime}\right)-V(x)
\end{equation}







\autoref{tab:reward-functions} compares the performance of CLTV with its variants, CLTV-TD and CLTV-RS, across all considered environments and datasets, using CQL and IQL as base algorithms.

In the Ant domain, CLTV achieves higher normalized scores with both CQL and IQL algorithms. For example, on the random-medium dataset, CLTV reaches a score of \(97.48\), compared to CLTV-TD's \(73.36\) and CLTV-RS's \(73.50\). Similarly, in the random-expert dataset, CLTV attains \(115.86\), outperforming the other methods. 

In the HalfCheetah domain, CLTV also exhibits superior performance, particularly on the medium-expert dataset, where it achieves \(59.88\) using the CQL algorithm. This result is higher than the scores of \(0.50\) and \(28.31\) recorded for CLTV-TD and CLTV-RS, respectively. For the IQL algorithm, CLTV performs well, reaching a score of \(77.10\) and clearly surpassing the other methods. 

The Walker2d and Hopper domains further highlight the advantages of CLTV. In the random-expert dataset of Walker2d, CLTV achieves \(97.43\) using CQL, outperforming both variants by a substantial margin. Similarly, in the Hopper domain, CLTV outperforms its counterparts on the medium-expert dataset, achieving \(83.44\) compared to the lower scores of the TD and RS variants. 

The results show that our reward function outperforms the other two in almost all cases. This suggests that our reward function is particularly effective for domain transfer learning.


\begin{table}[!ht]
\caption{{Performance of our CLTV method compared to two variants: CLTV-TD, which uses temporal difference (TD), and CLTV-RS, which uses reward shaping (RS). Normalized scores with standard deviations over 100 episodes and 5 seeds on mixed D4RL datasets are reported using the base algorithms CQL and IQL. The highest scores are highlighted in blue.}}
\label{tab:reward-functions}
\centering
\resizebox{\linewidth}{!}{
\renewcommand{\arraystretch}{1.4}
\begin{tabular}{@{}ccllll@{}}
\toprule
\textbf{Domain} & \textbf{\shortstack{RL \\ Algorithm}} & \textbf{Method} & \multicolumn{3}{c}{\textbf{Dataset}}\\
\multicolumn{1}{l}{} & \multicolumn{1}{l}{} & \multicolumn{1}{l}{} & \multicolumn{1}{c}{\textbf{random-medium}} & \textbf{random-expert} & \textbf{medium-expert}\\
\hline
\multirow{6}{*}{\rotatebox{90}{\Large Ant}} & \multirow{3}{*}{CQL} & CLTV-TD &  73.36 ± 6.72 & 95.08 ± 23.14 & 1.17 ± 46.66 \\
 &  & CLTV-RS & 73.50 ± 4.48 & 62.98 ± 12.38 & 6.48 ± 26.03 \\
 &  & CLTV & \paddedcolorbox{LightCyan}{97.48} ± 3.63 & \paddedcolorbox{LightCyan}{115.86} ± 6.82 & \paddedcolorbox{LightCyan}{24.84} ± 13.70 \\ \cmidrule{3-6}
 & \multirow{3}{*}{IQL} & CLTV-TD & 18.78 ± 8.78 & 7.87 ± 1.94 & 113.07 ± 3.89 \\
 &  & CLTV-RS & 75.11 ± 7.19 & 50.80 ± 11.16 & 111.82 ± 3.99 \\
 &  & CLTV & \paddedcolorbox{LightCyan}{78.67} ± 8.26 & \paddedcolorbox{LightCyan}{88.26} ± 4.67 & \paddedcolorbox{LightCyan}{117.00} ± 7.28 \\ 
  \hline
\multirow{6}{*}{\rotatebox{90}{\Large HalfCheetah}} & \multirow{3}{*}{CQL} & CLTV-TD & 36.87 ± 1.73 & 2.69 ± 2.20 & 0.50 ± 3.82 \\
 &  & CLTV-RS & 36.44 ± 2.07 & 0.11 ± 1.33 & 28.31 ± 7.40 \\
 &  & CLTV & \paddedcolorbox{LightCyan}{44.13} ± 3.47 & \paddedcolorbox{LightCyan}{10.37} ± 2.51 & \paddedcolorbox{LightCyan}{59.88} ± 7.91 \\ \cmidrule{3-6}
 & \multirow{3}{*}{IQL} & CLTV-TD & 3.94 ± 2.28 & 6.65 ± 2.60  & 46.52 ± 0.00 \\
 &  & CLTV-RS & 36.78 ± 1.33 & 8.91 ± 3.32 & 51.45 ± 3.03 \\
 &  & CLTV & \paddedcolorbox{LightCyan}{41.83} ± 0.63 & \paddedcolorbox{LightCyan}{16.28} ± 7.22 & \paddedcolorbox{LightCyan}{77.10} ± 5.16 \\ 
 \hline
\multirow{6}{*}{\rotatebox{90}{\Large Hopper}} & \multirow{3}{*}{CQL} & CLTV-TD & 7.30 ± 10.74 & 7.42 ± 6.90 & 30.42 ± 35.61 \\
 &  & CLTV-RS & 12.28 ± 11.99 & 6.79 ± 0.00 & 64.48 ± 35.66 \\
 &  & CLTV & \paddedcolorbox{LightCyan}{51.04} ± 3.21 & \paddedcolorbox{LightCyan}{56.23} ± 15.14 & \paddedcolorbox{LightCyan}{83.44} ± 16.95 \\ \cmidrule{3-6}
 & \multirow{3}{*}{IQL} & CLTV-TD & 0.19 ± 0.00 &  0.17 ± 0.00 & 42.31 ± 16.43 \\
 &  & CLTV-RS & 61.95 ± 6.24 & 30.79 ± 2.23 & 32.53 ± 12.08 \\
 &  & CLTV & \paddedcolorbox{LightCyan}{55.79} ± 4.44 & \paddedcolorbox{LightCyan}{39.56} ± 2.78 & \paddedcolorbox{LightCyan}{70.73} ± 4.11 \\ 
 \hline
\multirow{6}{*}{\rotatebox{90}{\Large Walker2d}} & \multirow{3}{*}{CQL} & CLTV-TD & 26.68 ± 20.97 & 74.62 ± 14.59 & 7.06 ± 4.56 \\
 &  & CLTV-RS & 41.53 ± 10.37 & 66.38 ± 40.72 & 2.59 ± 1.55 \\
 &  & CLTV & \paddedcolorbox{LightCyan}{70.45} ± 8.72 & \paddedcolorbox{LightCyan}{97.43} ± 7.74 & \paddedcolorbox{LightCyan}{30.23} ± 41.25 \\ \cmidrule{3-6}
 & \multirow{3}{*}{IQL} & CLTV-TD & 39.59 ± 13.09 & 18.84 ± 19.42 & 73.06 ± 5.16 \\
 &  & CLTV-RS & 64.97 ± 5.54 & 35.34 ± 17.71 & 89.05 ± 9.94 \\
 &  & CLTV & \paddedcolorbox{LightCyan}{68.37} ± 4.12 & \paddedcolorbox{LightCyan}{89.51} ± 9.03 & \paddedcolorbox{LightCyan}{110.74} ± 0.66 \\ 
\bottomrule
\end{tabular}
}
\end{table}
\subsection{Similarity-Reward Trade-off Parameters}
\label{par-impact}
We examine how the parameters \(\delta\) and \(\lambda\) affect performance across different domains, datasets, and algorithms in offline RL.

The parameter \(\delta\) balances the importance of transition similarity, which refers to the transition score or its relevance to the target domain, against the actual reward received, allowing the model to focus appropriately on both aspects during learning. Tuning \(\delta\) is important when applying the model to datasets with different dynamics, ensuring the model generalizes well without overfitting to the source data. 

The parameter \(\lambda\), on the other hand, plays a key role in balancing exploration and exploitation. A higher \(\lambda\) value encourages more exploration by allowing the model to take actions that may not immediately seem optimal, but could lead to better long-term outcomes. In contrast, a lower \(\lambda\) value favors exploitation, where the model sticks to actions that have previously yielded high rewards.

The heatmaps in \autoref{fig-deltalambda-cql} show that the choice of \(\delta\) and \(\lambda\) has a noticeable effect on CLTV (CQL) performance across different environments. Moderate values for both parameters generally lead to better results. For instance, in environments like Ant and Hopper, higher \(\lambda\) values enhance the balance between exploration and exploitation, while moderate \(\delta\) values allow for greater flexibility in learning without overfitting.

Similarly, the heatmaps in \autoref{fig-deltalambda-iql} show that performance in CLTV (IQL) is influenced by \(\delta\) and \(\lambda\), though the algorithm tends to be more stable across different settings. The heatmaps indicate that higher \(\lambda\) values help the model explore effectively, preventing it from getting stuck in suboptimal solutions. At the same time, moderate \(\delta\) values strike a balance between leveraging current estimates and improving both policy and value functions. This suggests that fine-tuning \(\lambda\) helps the model explore better, while \(\delta\) adjusts how much it sticks to what it has already learned.


When we analyze the impact of \(\lambda\) and \(\delta\) on CLTV (CQL) and CLTV (IQL) across different datasets, we see clear differences in how each algorithm reacts. In CLTV (CQL), especially in the Ant environment with the medium-expert dataset, performance is sensitive to changes in these parameters. For instance, as \(\lambda\) increases from 0.0 to 1.0, we see improvements in rewards. This highlights how \(\lambda\) helps manage the balance between policy conservativeness and exploration. In contrast, CLTV (IQL) appears to be less affected by changes in \(\lambda\) and \(\delta\). For example, in the Ant environment, CLTV (IQL) achieves consistently high rewards across various parameter settings. This shows that CLTV (IQL) handles exploration and exploitation more efficiently without needing significant tuning of these parameters. 

In summary, \(\lambda\) and \(\delta\) have different impacts on CLTV (CQL) and CLTV (IQL). For CLTV (CQL), higher \(\lambda\) values and moderate \(\delta\) values tend to result in better performance, especially in environments like Ant. On the other hand, CLTV (IQL) performs well across a wide range of \(\lambda\) and \(\delta\) values, reducing the need for fine-tuning. Understanding the effect of these parameters is essential for configuring algorithms in offline RL, where they can have a major influence on the performance of the learned policies.


\begin{figure*}[!ht]
\center
\includegraphics[width=\textwidth]{figures/heatmaps_cql.pdf}
\caption{{Heatmaps illustrating the performance of CLTV (CQL) on mixed datasets with respect to different $\delta$ (Delta) and $\lambda$ (Lambda) values, ranging from 0.2 to 1.0 in increments of 0.2, evaluated over 100 episodes with one seed.}}
\label{fig-deltalambda-cql}
\end{figure*}



\begin{figure*}[!ht]
\center
\includegraphics[width=\textwidth]{figures/heatmaps_iql.pdf}
\caption{{Heatmaps illustrating the performance of CLTV (IQL) on mixed datasets with respect to different $\delta$ (Delta) and $\lambda$ (Lambda) values, ranging from 0.2 to 1.0 in increments of 0.2, evaluated over 100 episodes with one seed.}}

\label{fig-deltalambda-iql}
\end{figure*}







\clearpage  
\makeatletter
\@twocolumnfalse
\makeatother


\begin{paracol}{2}  
    \switchcolumn[0]
    

\section{Experimental Details}
\label{app:experimental_details}

\subsection{Loss of plasticity with Dropout}
\label{app:loss_pl_dropout}

\subsubsection{Trainability for Dropout}
\label{app:loss_pl_dropout_trainability}
We employed an 8-layer MLP featuring 512 hidden units and trained it on 1400 samples from the MNIST dataset.
The model is trained for 50 different tasks, with each task running for 100 epochs.
To evaluate subnetwork plasticity, we extracted 10 subnetworks at the conclusion of each epoch, training these on new tasks for an additional 100 epochs and then calculated the mean final accuracy.
Adam optimizer was utilized for model optimization.


\subsubsection{Warm-start}
We used the ResNet-18 architecture described in Appendix \ref{app:generalizability}.
In the warm-start scenario, the model was pre-trained on 10\% of the CIFAR100 dataset for 1,000 epochs and continued training on the full dataset for 100 epochs, with the optimizer reset at the start of full dataset training.
In the cold-start scenario, the model was trained on the full dataset for 100 epochs. Adam optimizer with learning rate $0.001$ was utilized.


\subsection{Trainability}
\label{app:trainability}
\textbf{Permuted MNIST. }
We followed the setup of \cite{dohare2024loss}.
It consists of a total of 800 tasks that 60,000 images are fed into models only once with 512 batch size.
In the beginning of each task, the pixel of images are permuted arbitrarily.
We trained fully connected neural networks with three hidden layers.
Each hidden layer has 2,000 units and followed by ReLU activaiton function.

\textbf{Random Label MNIST. }
We conducted a variant of \cite{kumar2023maintaining}.
It consists of a total of 200 tasks that 1,600 images are fed into models with 64 batch size.
In the beginning of each task, the label class of images are changed to other class arbitrarily.
We trained fully connected neural networks with three hidden layers 100 epochs per each task.
Each hidden layer has 2,000 units and followed by ReLU activaiton function.


\subsection{Generalizability}
\label{app:generalizability}
We conducted experiments on CIFAR-10, CIFAR-100 \cite{krizhevsky2009learning}, and TinyImageNet \cite{le2015tiny} datasets using a 4-layer CNN, ResNet-18, and VGG-16, respectively, to evaluate the effectiveness of AID across different model architectures and datasets.
We provide detailed model architecture below:
\begin{itemize}
    \item \textbf{CNN}: We employed a convolutional neural network (CNN), which is used in relatively small image classification.
    The model includes two convolutional layers with a $5\times5$ kernel and 16 channels and max-pooling layer is followed after activation function.
    The two fully connected layers follow with 100 hidden units. 
    % residual block에도 잘 적용되는지 확인하기 위해 modern architecture인 resnet-18을 실험에 사용 
    \item \textbf{Resnet-18} \citep{he2016deep}: We utilized ResNet-18 to examine how well AID integrates with modern deep architectures featuring residual connections.
    Following \cite{lee2024slow}, the stem layers were removed to accommodate the smaller image size of the dataset. Additionally, a gradient clipping threshold of 0.5 was applied to ensure stable training.
    \item \textbf{VGG-16} \citep{simonyan2014very}: We adopted VGG-16 with batch normalization to investigate whether AID adapts properly in large-size models. The number of hidden units of the classifiers was set to 4096 without dropout.
\end{itemize}

In addition, we replaced the max-pooling layer with an average-pooling layer for methods such as Fourier activation, DropReLU, and AID, where large values may not necessarily represent important features.
Next, we provide the detailed experimental settings below.

\textbf{Continual Full. } Similar to the setup provided by \cite{shen2024step}, the entire data is randomly split into 10 chunks.
The training process consists of 10 stages and the model gains access up to the $k$-th chunk in each stage $k$.
In each stage, the dataset is fed into models with a batch size of 256.
We trained the model for 100 epochs per each stage, and we reset the optimizer when each stage starts training.

\textbf{Continual Limited. } This setting follows the same configuration as continual full, with the key distinction that the model does not retain access to previously seen data chunks.
At each stage, the model is trained only on the current chunk, without revisiting earlier data, simulating real-world constraints such as memory limitations and privacy concerns.

\textbf{Class-Incremental. } For CIFAR100 and TinyImageNet, the dataset was divided into 20 class-based chunks, with new classes introduced incrementally at each stage.
Unless otherwise specified, test accuracy is evaluated based on the corresponding classes encountered up to each stage.
The rest of the setup, including the batch size, and training epochs per stage, follows the Continual Full setting.










\subsection{Reinforcement Learning}
\label{app:reinforcement_learning}
To evaluate whether AID enhances sample efficiency on reinforcement learning, we conducted experiments on 17 Atari games from the Arcade Learning Environment \cite{bellemare2013arcade}, selected based on prior studies \cite{kumar2020implicit,sokar2023dormant}. We trained a DQN model following \citet{mnih2015human} using the CleanRL framework \cite{huang2022cleanrl}. The replay ratio was set to 1, as adopted in \citet{sokar2023dormant, elsayed2024weight}. We followed the hyperparameter settings for environment from \citet{sokar2023dormant}, with details provided in Table \ref{tab:hyperparameter_rl_env}.



\begin{table}[h]
    \centering
    \caption{Hyperparameters used in reinforcement learning environment}
    \begin{tabular}{l r}
        \toprule
        \textbf{Parameter} & \textbf{Value} \\
        \midrule
        Optimizer & Adam \cite{kingma2014adam} \\
        Optimizer: $\epsilon$ & $1.5\mathrm{e}-4$ \\
        Optimizer: Learning rate & $6.25\mathrm{e}-4$ \\
        \midrule
        Minimum $\epsilon$ for training & $0.01$ \\
        Evaluation $\epsilon$ & $0.001$\\
        Discount Factor $\gamma$ & $0.99$ \\
        Replay buffer size & $10^6$ \\
        Minibatch size & $32$ \\
        Initial collect steps & $20000$ \\
        Training iterations & $10$ \\
        Training environment steps per iteration & $250K$ \\
        Updates per environment step (Replay Ratio) & $1$ \\
        Target network update period & $2000$\\
        Loss function & Huber Loss \cite{huber1992robust} \\
        
        \bottomrule
    \end{tabular}
    \label{tab:hyperparameter_rl_env}
\end{table}







\subsection{Standard Supervised Learning}
For the same model architecture and dataset used in the generalizability experiments, we trained with Adam optimizer for 200 epochs, applying learning rate decay at the 100th and 150th epochs. The initial learning rate was set to $0.001$ and was reduced by a factor of $10$ at each decay step.



\subsection{Hyperparameter Search Space}
\label{app:hyperparameter_seach_space}
We present the hyperparameter search space considered for each experiment in Table \ref{tab:hyperparameter_search_space}.
Without mentioned, we performed a sweep over 5 different seeds for all experiments, except for VGG-16 model on the TinyImageNet dataset, where we used only 3 seeds due to computational cost.
\cref{tab:hyperparameter_permuted_mnist,tab:hyperparameter_random_label_mnist,tab:hyperparameter_continual_full,tab:hyperparameter_continual_limited,tab:hyperparameter_class_incremental,tab:hyperparameter_rl,tab:hyperparameter_sl} shows the best hyperparameter set that we found in various experiments. 

\begin{table}[H]
    \centering
    \caption{Hyperparameter search space for every experiment}
    \begin{tabular}{l l l l}
        \toprule
        \textbf{Experiment} & \textbf{Method} & \textbf{Hyperparameters} & \textbf{Search Space}\\
        \midrule
        Warm-Start & Dropout & $p$ & $0.1, 0.3, 0.5$ \\
        & DIA & $p$ & $0.7, 0.8, 0.9$ \\
        \midrule
        Trainability & ADAM & learning rate & $1\mathrm{e}-3, 1\mathrm{e}-4$ \\
                     & SGD & learning rate & $3\mathrm{e}-2, 3\mathrm{e}-3$ \\
                     & L2 & $\lambda$ & $1\mathrm{e}-2, 1\mathrm{e}-3, 1\mathrm{e}-4, 1\mathrm{e}-5, 1\mathrm{e}-6$ \\
                     & L2 Init & $\lambda$ & $1\mathrm{e}-2, 1\mathrm{e}-3, 1\mathrm{e}-4, 1\mathrm{e}-5, 1\mathrm{e}-6$ \\
                     & Dropout & $p$ & $0.01, 0.05, 0.1, 0.15, 0.2, 0.25, 0.3$ \\
                     & S\&P & $\lambda$ & $0.1, 0.2, 0.3, 0.4, 0.5, 0.6, 0.7, 0.8, 0.9$ \\
                     & ReDo & \text{threshold} & $0.0, 1.0, 5.0, 10.0, 50.0$ \\
                     &      & \text{period} & $2500$ \\
                     & CBP & \text{replacement rate}($\rho$) & $1\mathrm{e}-1, 1\mathrm{e}-2, 1\mathrm{e}-3, 1\mathrm{e}-4, 1\mathrm{e}-5$ \\
                     &     & \text{maturity threshold} & $100$ \\
                     & RReLU & \text{lower} & $0.0625, 0.125, 0.25$ \\
                     &       & \text{upper} & $0.125, 0.25, 0.333, 0.5$ \\
                     & DropReLU & $p$ & $0.1, 0.2, 0.3, 0.4, 0.5, 0.6, 0.7, 0.8, 0.9, 0.99$ \\
                     & DIA & $p$ & $0.1, 0.2, 0.3, 0.4, 0.5, 0.6, 0.7, 0.8, 0.9, 0.99$ \\
        \midrule
        Generalizability & ADAM & learning rate & $1\mathrm{e}-3, 1\mathrm{e}-4$ \\
                         & S\&P & $\lambda$ & $0.2, 0.4, 0.6, 0.8$ \\
                         & CBP & \text{replacement rate}($\rho$) & $1\mathrm{e}-4, 1\mathrm{e}-5$\\
                         &     & \text{maturity threshold} & $100, 1000$\\
                         & DropReLU & $p$ & $0.7, 0.8, 0.9$ \\
                         & L2 & $\lambda$ & $1\mathrm{e}-2, 1\mathrm{e}-3, 1\mathrm{e}-4, 1\mathrm{e}-5$ \\
                         & Dropout & $p$ & $0.1, 0.3, 0.5$\\
                         & RReLU & lower & $0.125$\\
                         &       & upper & $0.333$\\
                         & DIA & $p$ & $0.7, 0.8, 0.9$ \\
        \hdashline
        Continual Full   & DASH & $\alpha$ & $0.1, 0.3$\\
                         &      & $\lambda$ & $0.05, 0.1, 0.3$\\
        \hdashline
        Class-Incremental & DASH & $\alpha$ & $0.1, 0.3$\\
                         &      & $\lambda$ & $0.05, 0.1, 0.3$\\
        \hdashline
        Continual Limited & DASH & $\alpha$ & $0.1, 0.3, 0.7, 1.0$\\
                         &      & $\lambda$ & $0.3$\\
                         & S-EWC & $\lambda$ & $100, 10, 1, 0.1, 0.01$\\
        \midrule
        Standard SL & L2   & $\lambda$ & $1\mathrm{e}-2, 1\mathrm{e}-3, 1\mathrm{e}-4, 1\mathrm{e}-5$ \\
                    & Dropout & $p$ & $0.1, 0.3, 0.5$ \\
                    & DIA & $p$ & $0.8, 0.9, 0.95$ \\
        \midrule
        Reinforcement Learning & DIA & $p$ & $0.99, 0.999$ \\
        \bottomrule
    \end{tabular}
    \label{tab:hyperparameter_search_space}
\end{table}



\begin{table}[p]
    \centering
    \caption{Hyperparameter set of each method on permuted MNIST}
    \begin{tabular}{l|l|l|l}
        \toprule
        \textbf{Method} & \textbf{Optimizer} & \textbf{Learning Rate} & \textbf{Optimal Hyperparameters} \\
        \midrule
        Baseline                & Adam & $1\mathrm{e}{-3}$ & -- \\
        Dropout                 & Adam & $1\mathrm{e}{-3}$ & $p = 0.05$ \\
        L2                      & Adam & $1\mathrm{e}{-3}$ & $\lambda = 1\mathrm{e}{-5}$ \\
        L2 Init                 & Adam & $1\mathrm{e}{-3}$ & $\lambda = 1\mathrm{e}{-4}$ \\
        Shrink \& Perturb       & Adam & $1\mathrm{e}{-3}$ & $\lambda = 0.2$ \\
        ReDo                    & Adam & $1\mathrm{e}{-3}$ & $\text{recycle period} = 118, \text{recycle threshold} = 50$ \\
        Continual Backprop      & Adam & $1\mathrm{e}{-3}$ & $\rho = 1\mathrm{e}{-4}, \text{maturity threshold} = 100$ \\
        Concat ReLU             & Adam & $1\mathrm{e}{-3}$ & -- \\
        RReLU                   & Adam & $1\mathrm{e}{-3}$ & $\text{lower} = 0.0625, \text{upper} = 0.333$ \\
        DropReLU                & Adam & $1\mathrm{e}{-3}$ & $p = 0.99$ \\
        DIA                     & Adam & $1\mathrm{e}{-3}$ & $p = 0.99$ \\
        \midrule
        Baseline                & Adam & $1\mathrm{e}{-4}$ & -- \\
        Dropout                 & Adam & $1\mathrm{e}{-4}$ & $p = 0.05$ \\
        L2                      & Adam & $1\mathrm{e}{-4}$ & $\lambda = 1\mathrm{e}{-5}$ \\
        L2 Init                 & Adam & $1\mathrm{e}{-4}$ & $\lambda = 1\mathrm{e}{-4}$ \\
        Shrink \& Perturb       & Adam & $1\mathrm{e}{-3}$ & $\lambda = 0.1$ \\
        ReDo                    & Adam & $1\mathrm{e}{-4}$ & $\text{recycle period} = 118, \text{recycle threshold} = 50$ \\
        Continual Backprop      & Adam & $1\mathrm{e}{-4}$ & $\rho = 1\mathrm{e}{-4}, \text{maturity threshold} = 100$ \\
        Concat ReLU             & Adam & $1\mathrm{e}{-4}$ & -- \\
        RReLU                   & Adam & $1\mathrm{e}{-4}$ & $\text{lower} = 0.0625, \text{upper} = 0.333$ \\
        DropReLU                & Adam & $1\mathrm{e}{-4}$ & $p = 0.99$ \\
        DIA                     & Adam & $1\mathrm{e}{-4}$ & $p = 0.99$ \\
        \midrule
        Baseline                & SGD  & $3\mathrm{e}{-2}$ & -- \\
        Dropout                 & SGD  & $3\mathrm{e}{-2}$ & $p = 0.25$ \\
        L2                      & SGD  & $3\mathrm{e}{-2}$ & $\lambda = 1\mathrm{e}{-5}$ \\
        L2 Init                 & SGD  & $3\mathrm{e}{-2}$ & $\lambda = 1\mathrm{e}{-3}$ \\
        Shrink \& Perturb       & SGD  & $3\mathrm{e}{-3}$ & $\lambda = 0.1$ \\
        ReDo                    & SGD  & $3\mathrm{e}{-2}$ & $\text{recycle period} = 118, \text{recycle threshold} = 5$ \\
        Continual Backprop      & SGD  & $3\mathrm{e}{-2}$ & $\rho = 1\mathrm{e}{-3}, \text{maturity threshold} = 100$ \\
        Concat ReLU             & SGD  & $3\mathrm{e}{-2}$ & -- \\
        RReLU                   & SGD  & $3\mathrm{e}{-2}$ & $\text{lower} = 0.0625, \text{upper} = 0.333$ \\
        DropReLU                & SGD  & $3\mathrm{e}{-2}$ & $p = 0.99$ \\
        DIA                     & SGD  & $3\mathrm{e}{-2}$ & $p = 0.99$ \\
        \midrule
        Baseline                & SGD  & $3\mathrm{e}{-3}$ & -- \\
        Dropout                 & SGD  & $3\mathrm{e}{-3}$ & $p = 0.01$ \\
        L2                      & SGD  & $3\mathrm{e}{-3}$ & $\lambda = 1\mathrm{e}{-5}$ \\
        L2 Init                 & SGD  & $3\mathrm{e}{-3}$ & $\lambda = 1\mathrm{e}{-3}$ \\
        Shrink \& Perturb       & SGD  & $3\mathrm{e}{-3}$ & $\lambda = 0.1$ \\
        ReDo                    & SGD  & $3\mathrm{e}{-3}$ & $\text{recycle period} = 118, \text{recycle threshold} = 1$ \\
        Continual Backprop      & SGD  & $3\mathrm{e}{-3}$ & $\rho = 1\mathrm{e}{-5}, \text{maturity threshold} = 100$ \\
        Concat ReLU             & SGD  & $3\mathrm{e}{-3}$ & -- \\
        RReLU                   & SGD  & $3\mathrm{e}{-3}$ & $\text{lower} = 0.0625, \text{upper} = 0.333$ \\
        DropReLU                & SGD  & $3\mathrm{e}{-3}$ & $p = 0.99$ \\
        DIA                     & SGD  & $3\mathrm{e}{-3}$ & $p = 0.99$ \\
        \bottomrule
    \end{tabular}
    \label{tab:hyperparameter_permuted_mnist}
\end{table}




\begin{table}[p]
    \centering
    \caption{Hyperparameter set of each method on random label MNIST}
    \begin{tabular}{l|l|l|l}
        \toprule
        \textbf{Method} & \textbf{Optimizer} & \textbf{Learning Rate} & \textbf{Optimal Hyperparameters} \\
        \midrule
        Baseline                & Adam & $1\mathrm{e}{-3}$ & -- \\
        Dropout                 & Adam & $1\mathrm{e}{-3}$ & $p = 0.15$ \\
        L2                      & Adam & $1\mathrm{e}{-3}$ & $\lambda = 1\mathrm{e}{-4}$ \\
        L2 Init                 & Adam & $1\mathrm{e}{-3}$ & $\lambda = 1\mathrm{e}{-3}$ \\
        Shrink \& Perturb       & Adam & $1\mathrm{e}{-3}$ & $\lambda = 0.8$ \\
        ReDo                    & Adam & $1\mathrm{e}{-3}$ & $\text{recycle period} = 2500, \text{recycle threshold} = 0$ \\
        Continual Backprop      & Adam & $1\mathrm{e}{-3}$ & $\rho = 1\mathrm{e}{-3}, \text{maturity threshold} = 100$ \\
        Concat ReLU             & Adam & $1\mathrm{e}{-3}$ & -- \\
        RReLU                   & Adam & $1\mathrm{e}{-3}$ & $\text{lower} = 0.0625, \text{upper} = 0.125$ \\
        DropReLU                & Adam & $1\mathrm{e}{-3}$ & $p = 0.9$ \\
        DIA                     & Adam & $1\mathrm{e}{-3}$ & $p = 0.9$ \\
        \midrule
        Baseline                & Adam & $1\mathrm{e}{-4}$ & -- \\
        Dropout                 & Adam & $1\mathrm{e}{-4}$ & $p = 0.25$ \\
        L2                      & Adam & $1\mathrm{e}{-4}$ & $\lambda = 1\mathrm{e}{-3}$ \\
        L2 Init                 & Adam & $1\mathrm{e}{-4}$ & $\lambda = 1\mathrm{e}{-4}$ \\
        Shrink \& Perturb       & Adam & $1\mathrm{e}{-3}$ & $\lambda = 0.7$ \\
        ReDo                    & Adam & $1\mathrm{e}{-4}$ & $\text{recycle period} = 2500, \text{recycle threshold} = 0$ \\
        Continual Backprop      & Adam & $1\mathrm{e}{-4}$ & $\rho = 1\mathrm{e}{-4}, \text{maturity threshold} = 100$ \\
        Concat ReLU             & Adam & $1\mathrm{e}{-4}$ & -- \\
        RReLU                   & Adam & $1\mathrm{e}{-4}$ & $\text{lower} = 0.25, \text{upper} = 0.333$ \\
        DropReLU                & Adam & $1\mathrm{e}{-4}$ & $p = 0.99$ \\
        DIA                     & Adam & $1\mathrm{e}{-4}$ & $p = 0.99$ \\
        \midrule
        Baseline                & SGD  & $3\mathrm{e}{-2}$ & -- \\
        Dropout                 & SGD  & $3\mathrm{e}{-2}$ & $p = 0.15$ \\
        L2                      & SGD  & $3\mathrm{e}{-2}$ & $\lambda = 1\mathrm{e}{-3}$ \\
        L2 Init                 & SGD  & $3\mathrm{e}{-2}$ & $\lambda = 1\mathrm{e}{-3}$ \\
        Shrink \& Perturb       & SGD  & $3\mathrm{e}{-3}$ & $\lambda = 0.2$ \\
        ReDo                    & SGD  & $3\mathrm{e}{-2}$ & $\text{recycle period} = 2500, \text{recycle threshold} = 1$ \\
        Continual Backprop      & SGD  & $3\mathrm{e}{-2}$ & $\rho = 1\mathrm{e}{-5}, \text{maturity threshold} = 100$ \\
        Concat ReLU             & SGD  & $3\mathrm{e}{-2}$ & -- \\
        RReLU                   & SGD  & $3\mathrm{e}{-2}$ & $\text{lower} = 0.0625, \text{upper} = 0.125$ \\
        DropReLU                & SGD  & $3\mathrm{e}{-2}$ & $p = 0.99$ \\
        DIA                     & SGD  & $3\mathrm{e}{-2}$ & $p = 0.99$ \\
        \midrule
        Baseline                & SGD  & $3\mathrm{e}{-3}$ & -- \\
        Dropout                 & SGD  & $3\mathrm{e}{-3}$ & $p = 0.01$ \\
        L2                      & SGD  & $3\mathrm{e}{-3}$ & $\lambda = 1\mathrm{e}{-2}$ \\
        L2 Init                 & SGD  & $3\mathrm{e}{-3}$ & $\lambda = 1\mathrm{e}{-3}$ \\
        Shrink \& Perturb       & SGD  & $3\mathrm{e}{-3}$ & $\lambda = 0.1$ \\
        ReDo                    & SGD  & $3\mathrm{e}{-3}$ & $\text{recycle period} = 2500, \text{recycle threshold} = 1$ \\
        Continual Backprop      & SGD  & $3\mathrm{e}{-3}$ & $\rho = 1\mathrm{e}{-3}, \text{maturity threshold} = 100$ \\
        Concat ReLU             & SGD  & $3\mathrm{e}{-3}$ & -- \\
        RReLU                   & SGD  & $3\mathrm{e}{-3}$ & $\text{lower} = 0.0625, \text{upper} = 0.25$ \\
        DropReLU                & SGD  & $3\mathrm{e}{-3}$ & $p = 0.99$ \\
        DIA                     & SGD  & $3\mathrm{e}{-3}$ & $p = 0.99$ \\
        \bottomrule
    \end{tabular}
    \label{tab:hyperparameter_random_label_mnist}
\end{table}




\begin{table}[p]
    \centering
    \caption{Hyperparameter set of each method on continual full setting.}
    \begin{tabular}{l|l|c|c}
        \toprule
        \textbf{Dataset (Model)} & \textbf{Method} & \textbf{Optimal Learning Rate} & \textbf{Optimal Hyperparameters} \\
        \midrule
        CIFAR10 & Baseline                & $1\mathrm{e}{-4}$ & -- \\
        (CNN)&Full Reset              & $1\mathrm{e}{-4}$ & -- \\
        &L2                      & $1\mathrm{e}{-3}$ & $\lambda = 1\mathrm{e}{-2}$ \\
        &Dropout                 & $1\mathrm{e}{-4}$ & $p = 0.3$ \\
        &RReLU                   & $1\mathrm{e}{-4}$ & -- \\
        &CReLU                   & $1\mathrm{e}{-4}$ & -- \\
        &Fourier                 & $1\mathrm{e}{-4}$ & -- \\
        &S\&P                    & $1\mathrm{e}{-4}$ & $\lambda = 0.8$ \\
        &CBP                     & $1\mathrm{e}{-4}$ & $\rho = 1\mathrm{e}{-5}, \text{maturity threshold} = 100$ \\
        &DASH                    & $1\mathrm{e}{-4}$ & $\alpha=0.1$, $\lambda=0.3$ \\
        &DropReLU                & $1\mathrm{e}{-4}$ & $p = 0.9$ \\
        &AID                     & $1\mathrm{e}{-3}$ & $p=0.9$ \\
        \midrule
        CIFAR100 & Baseline                & $1\mathrm{e}{-3}$ & -- \\
        (Resnet-18)&Full Reset              & $1\mathrm{e}{-3}$ & -- \\
        &L2                      & $1\mathrm{e}{-3}$ & $\lambda = 1\mathrm{e}{-2}$ \\
        &Dropout                 & $1\mathrm{e}{-4}$ & $p = 0.3$ \\
        &RReLU                   & $1\mathrm{e}{-3}$ & -- \\
        &CReLU                   & $1\mathrm{e}{-3}$ & -- \\
        &Fourier                 & $1\mathrm{e}{-3}$ & -- \\
        &S\&P                    & $1\mathrm{e}{-3}$ & $\lambda = 0.8$ \\
        &CBP                     & $1\mathrm{e}{-3}$ & $\rho = 1\mathrm{e}{-5}, \text{maturity threshold} = 1000$ \\
        &DASH                    & $1\mathrm{e}{-4}$ & $\alpha=0.1$, $\lambda=0.05$ \\
        &DropReLU                & $1\mathrm{e}{-3}$ & $p = 0.8$ \\
        &AID                     & $1\mathrm{e}{-3}$ & $p=0.7$ \\
        \midrule
        TinyImageNet & Baseline                & $1\mathrm{e}{-3}$ & -- \\
        (VGG-16)&Full Reset              & $1\mathrm{e}{-3}$ & -- \\
        &L2                      & $1\mathrm{e}{-3}$ & $\lambda = 1\mathrm{e}{-4}$ \\
        &Dropout                 & $1\mathrm{e}{-4}$ & $p = 0.1$ \\
        &RReLU                   & $1\mathrm{e}{-4}$ & -- \\
        &CReLU                   & $1\mathrm{e}{-3}$ & -- \\
        &Fourier                 & $1\mathrm{e}{-4}$ & -- \\
        &S\&P                    & $1\mathrm{e}{-3}$ & $\lambda = 0.8$ \\
        &CBP                     & $1\mathrm{e}{-3}$ & $\rho = 1\mathrm{e}{-4}, \text{maturity threshold} = 100$ \\
        &DASH                    & $1\mathrm{e}{-4}$ & $\alpha=0.3$, $\lambda=0.1$ \\
        &DropReLU                & $1\mathrm{e}{-4}$ & $p = 0.7$ \\
        &AID                     & $1\mathrm{e}{-4}$ & $p=0.7$ \\
        \bottomrule
    \end{tabular}
    \label{tab:hyperparameter_continual_full}
\end{table}






\begin{table}[p]
    \centering
    \caption{Hyperparameter set of each method on continual limited setting.}
    \begin{tabular}{l|l|c|c}
        \toprule
        \textbf{Dataset (Model)} & \textbf{Method} & \textbf{Optimal Learning Rate} & \textbf{Optimal Hyperparameters} \\
        \midrule
        CIFAR10 & Baseline                & $1\mathrm{e}{-4}$ & -- \\
        (CNN)&L2                      & $1\mathrm{e}{-4}$ & $\lambda = 1\mathrm{e}{-5}$ \\
        &Dropout                 & $1\mathrm{e}{-3}$ & $p = 0.5$ \\
        &S-EWC                 & $1\mathrm{e}{-4}$ & $\lambda = 0.01$ \\
        &RReLU                   & $1\mathrm{e}{-4}$ & -- \\
        &CReLU                   & $1\mathrm{e}{-4}$ & -- \\
        &Fourier                 & $1\mathrm{e}{-4}$ & -- \\
        &S\&P                    & $1\mathrm{e}{-4}$ & $\lambda = 0.2$ \\
        &CBP                     & $1\mathrm{e}{-4}$ & $\rho = 1\mathrm{e}{-4}, \text{maturity threshold} = 100$ \\
        &DASH                    & $1\mathrm{e}{-3}$ & $\alpha=0.1$, $\lambda=0.3$ \\
        &DropReLU                & $1\mathrm{e}{-4}$ & $p = 0.9$ \\
        &AID                     & $1\mathrm{e}{-3}$ & $p=0.8$ \\
        \midrule
        CIFAR100 & Baseline                & $1\mathrm{e}{-3}$ & -- \\
        (Resnet-18)&L2                      & $1\mathrm{e}{-3}$ & $\lambda = 1\mathrm{e}{-5}$ \\
        &Dropout                 & $1\mathrm{e}{-4}$ & $p = 0.1$ \\
        &S-EWC                 & $1\mathrm{e}{-3}$ & $\lambda = 1$ \\
        &RReLU                   & $1\mathrm{e}{-3}$ & -- \\
        &CReLU                   & $1\mathrm{e}{-3}$ & -- \\
        &Fourier                 & $1\mathrm{e}{-3}$ & -- \\
        &S\&P                    & $1\mathrm{e}{-3}$ & $\lambda = 0.2$ \\
        &CBP                     & $1\mathrm{e}{-3}$ & $\rho = 1\mathrm{e}{-5}, \text{maturity threshold} = 1000$ \\
        &DASH                    & $1\mathrm{e}{-4}$ & $\alpha=0.1$, $\lambda=0.3$ \\
        &DropReLU                & $1\mathrm{e}{-4}$ & $p = 0.7$ \\
        &AID                     & $1\mathrm{e}{-3}$ & $p=0.8$ \\
        \midrule
        TinyImageNet & Baseline                & $1\mathrm{e}{-4}$ & -- \\
        (VGG-16)&L2                      & $1\mathrm{e}{-3}$ & $\lambda = 1\mathrm{e}{-4}$ \\
        &Dropout                 & $1\mathrm{e}{-4}$ & $p = 0.1$ \\
        &S-EWC                 & $1\mathrm{e}{-3}$ & $\lambda = 100$ \\
        &RReLU                   & $1\mathrm{e}{-4}$ & -- \\
        &CReLU                   & $1\mathrm{e}{-3}$ & -- \\
        &Fourier                 & $1\mathrm{e}{-4}$ & -- \\
        &S\&P                    & $1\mathrm{e}{-3}$ & $\lambda = 0.4$ \\
        &CBP                     & $1\mathrm{e}{-4}$ & $\rho = 1\mathrm{e}{-4}, \text{maturity threshold} = 1000$ \\
        &DASH                    & $1\mathrm{e}{-3}$ & $\alpha=0.1$, $\lambda=0.3$ \\
        &DropReLU                & $1\mathrm{e}{-4}$ & $p = 0.8$ \\
        &AID                     & $1\mathrm{e}{-4}$ & $p=0.8$ \\
        \bottomrule
    \end{tabular}
    \label{tab:hyperparameter_continual_limited}
\end{table}




\input{tables/hyperparameter_class_incremental}
\newpage


\begin{table}[H]
    \vspace{2.5cm}
    \centering
    \caption{\centering Hyperparameter set of AID on reinforcement learning setting.}
    \begin{tabular}{l|c}
        \toprule
        \textbf{Game} &\textbf{Optimal Hyperparameters} \\
        \midrule
        Seaquest & $p=0.999$\\
        DemonAttack & $p=0.99$\\
        SpaceInvaders & $p=0.99$\\
        Qbert & $p=0.999$\\
        DoubleDunk & $p=0.99$\\
        MsPacman & $p=0.999$\\
        Enduro & $p=0.99$\\
        BeamRider & $p=0.99$\\
        WizardOfWor & $p=0.999$\\
        Jamesbond & $p=0.99$\\
        RoadRunner & $p=0.999$\\
        Asterix & $p=0.99$\\
        Pong & $p=0.999$\\
        Zaxxon & $p=0.999$\\
        YarsRevenge & $p=0.99$\\
        Breakout & $p=0.99$\\
        IceHockey & $p=0.99$\\
        
        \bottomrule
    \end{tabular}
    \label{tab:hyperparameter_rl}
\end{table}
\vfill




\begin{table}[H]
    \centering
    \caption{Hyperparameter set of each method on standard supervised learning setting.}
    \begin{tabular}{l|l|c}
        \toprule
        \textbf{Dataset(Model)} & \textbf{Method} & \textbf{Optimal Hyperparameters} \\
        \midrule
        CIFAR10 &L2     & $\lambda = 1\mathrm{e}{-2}$ \\
        (CNN)
        &Dropout               & $p = 0.3$ \\
        &AID                    & $p=0.95$ \\
        \midrule
        CIFAR100 &L2     & $\lambda = 1\mathrm{e}{-5}$ \\
        (Resnet-18)
        
        &Dropout              & $p = 0.1$ \\
        &AID                  & $p=0.8$ \\
        \midrule
        TinyImageNet &L2 & $\lambda = 1\mathrm{e}{-5}$ \\
        (VGG-16)
        &Dropout            & $p = 0.1$ \\
        &AID               & $p=0.9$ \\
        \bottomrule
    \end{tabular}
    \label{tab:hyperparameter_sl}
\end{table}
\vfill




    \switchcolumn
    \vfill
\end{paracol}

\end{document}