% \documentclass{uai2025} % for initial submission
\documentclass[accepted]{uai2025} % after acceptance, for a revised version; 
% also before submission to see how the non-anonymous paper would look like 
                        
%% There is a class option to choose the math font
% \documentclass[mathfont=ptmx]{uai2025} % ptmx math instead of Computer
                                         % Modern (has noticeable issues)
% \documentclass[mathfont=newtx]{uai2025} % newtx fonts (improves upon
                                          % ptmx; less tested, no support)
% NOTE: Only keep *one* line above as appropriate, as it will be replaced
%       automatically for papers to be published. Do not make any other
%       change above this note for an accepted version.

%% Choose your variant of English; be consistent
\usepackage[american]{babel}
% \usepackage[british]{babel}

%% Some suggested packages, as needed:
\usepackage{natbib} % has a nice set of citation styles and commands
    \bibliographystyle{plainnat}
    \renewcommand{\bibsection}{\subsubsection*{References}}
\usepackage{mathtools} % amsmath with fixes and additions
% \usepackage{siunitx} % for proper typesetting of numbers and units
\usepackage{booktabs} % commands to create good-looking tables
\usepackage{tikz} % nice language for creating drawings and diagrams



\usepackage{tikz-cd}
\usetikzlibrary{arrows}
\usepackage{comment}
\usepackage{amssymb}
\usepackage{tikz}
\usepackage{tikz-qtree,tikz-qtree-compat}
\usetikzlibrary{positioning}
\usepackage{float}


\usepackage{amsmath}
\usepackage{amsthm}
\usepackage{tikz}
\usepackage{tikz-qtree,tikz-qtree-compat}

\newcommand\independent{\protect\mathpalette{\protect\independenT}{\perp}}
\def\independenT#1#2{\mathrel{\rlap{$#1#2$}\mkern2mu{#1#2}}}
\newcommand{\notindep}{\not\!\independent}

\usepackage{stackengine}
\def\delequal{\mathrel{\ensurestackMath{\stackon[1pt]{=}{\scriptstyle\Delta}}}}
%%%%%%%%%%%%%%%%%%%%%%%%%%%%%%%%
% THEOREMS
%%%%%%%%%%%%%%%%%%%%%%%%%%%%%%%%
% \newtheorem{theorem}{Theorem}
% \newtheorem{definition}[theorem]{Definition}
% \newtheorem{lemma}[theorem]{Lemma}
% \newtheorem{conjecture}[theorem]{Conjecture}
% \newtheorem{corollary}[theorem]{Corollary}
\theoremstyle{plain}
\newtheorem{theorem}{Theorem}
\newtheorem{proposition}[theorem]{Proposition}
\newtheorem{lemma}[theorem]{Lemma}
\newtheorem{corollary}[theorem]{Corollary}
\theoremstyle{definition}
\newtheorem{definition}[theorem]{Definition}
\newtheorem{assumption}[theorem]{Assumption}
\theoremstyle{remark}
\newtheorem{remark}[theorem]{Remark}



\usepackage{subcaption}

%% Provided macros
% \smaller: Because the class footnote size is essentially LaTeX's \small,
%           redefining \footnotesize, we provide the original \footnotesize
%           using this macro.
%           (Use only sparingly, e.g., in drawings, as it is quite small.)

%% Self-defined macros
\newcommand{\swap}[3][-]{#3#1#2} % just an example

\title{Estimating Probabilities of Causation with Machine Learning Models}

% The standard author block has changed for UAI 2025 to provide
% more space for long author lists and allow for complex affiliations
%
% All author information is authomatically removed by the class for the
% anonymous submission version of your paper, so you can already add your
% information below.
%
% Add authors
\author[1]{\href{mailto:<sw23v@fsu.edu>?Subject=Your UAI 2025 paper}{Shuai Wang}}
\author[1]{\href{mailto:<angli@cs.fsu.edu>?Subject=Your UAI 2025 paper}Ang Li}
% Add affiliations after the authors
\affil[1]{%
    Dept. of Computer Science\par
    Florida State University\par
    Tallahassee, FL, USA
}

  
  \begin{document}
\maketitle
\thispagestyle{empty} % 避免过多的空白
\begin{abstract}
% \sloppy
Probabilities of causation play a crucial role in modern decision-making. This paper addresses the challenge of predicting probabilities of causation for subpopulations with insufficient data using machine learning models. Tian and Pearl first defined and derived tight bounds for three fundamental probabilities of causation: the probability of necessity and sufficiency (PNS), the probability of sufficiency (PS), and the probability of necessity (PN). However, estimating these probabilities requires both experimental and observational distributions specific to each subpopulation, which are often unavailable or impractical to obtain with limited population-level data. We assume that the probabilities of causation for each subpopulation are determined by its characteristics. To estimate these probabilities for subpopulations with insufficient data, we propose using machine learning models that draw insights from subpopulations with sufficient data. Our evaluation of multiple machine learning models indicates that, given sufficient population-level data and an appropriate choice of machine learning model and activation function, PNS can be effectively predicted. Through simulation studies, we show that our multilayer perceptron (MLP) model with the Mish activation function achieves a mean absolute error (MAE) of approximately 0.02 in predicting PNS for 32,768 subpopulations using data from around 2,000 subpopulations.
\end{abstract}


\section{Introduction}\label{sec:intro}
Understanding causal relationships and estimating probabilities of causation are crucial in fields such as healthcare, policy evaluation, and economics \citep{pearl2009causality, imbens2015causal, heckman2015causal}. Unlike correlation-based methods, causal inference enables decision-makers to determine whether an action or intervention directly leads to a desired outcome. This is particularly essential in personalized medicine, where accurately assessing treatment effects ensures both efficacy and safety \citep{mueller:pea23-r530}. Moreover, causal reasoning enhances machine learning applications by improving accuracy \citep{li2020training}, interpretability, and fairness \citep{plecko2022causal} in automated decision-making. Despite its broad significance, estimating probabilities of causation remains challenging due to data limitations. In this paper, we address this challenge by leveraging machine learning techniques to predict probabilities of causation for subpopulations with insufficient data.

The study of probabilities of causation began around 2000 when \cite{pearl1999probabilities} first defined three fundamental probabilities—PNS, PS, and PN—within Structural Causal Models \citep{galles1998axiomatic,halpern2000axiomatizing,pearl2009causality}. Subsequently, \cite{tian2000probabilities} derived tight bounds for these probabilities using Balke's linear programming \citep{balke1995probabilistic}, incorporating both observational and experimental data. Nearly two decades later, \cite{li2019unit} formally proved these bounds and introduced the unit selection model, a decision-making framework based on their linear combination. More recently, \cite{li2024unit} extended the definitions and bounds to a more general form. Additionally, \cite{pearl:etal21-r505}, as well as \cite{dawid2017}, demonstrated that these bounds could be further refined given specific causal structures.

However, any above estimation of the probabilities of causation requires both observational and experimental data. Additionally, estimating (sub)populations, based on \cite{li2022probabilities}'s suggestions, requires approximately $1,300$ entries of both data types for each (sub)population, making the process impractical. \cite{li2022learning,li2022unitlearning} demonstrated the potential of machine learning models to achieve accurate estimations for (sub)populations. In this research, we select five diverse machine learning models based on the characteristics of the probabilities of causation. We then evaluate their performance in accomplishing the task.

\subsection{Contributions}
Despite the extensive theoretical research on probabilities of causation, practical estimation methods have remained unexplored. Our work provides the first systematic approach to predicting probabilities of causation using machine learning. Specifically, we make the following contributions:
\begin{itemize}[nosep]
    \item \textbf{First Machine Learning Pipeline for Predicting Probabilities of Causation:} We propose a novel machine learning framework to estimate the bounds of PNS, PS, and PN, filling a critical gap between theoretical causal inference and practical applications.
    \item \textbf{First Accurate Machine Learning Model for PNS Prediction:} We demonstrate that a MLP can accurately predict PNS, proving that machine learning is a feasible and effective tool for estimating probabilities of causation.
    \item \textbf{First Dataset for PNS Bound Prediction:} We construct and release the first synthetic dataset specifically designed to evaluate machine learning models for estimating PNS, providing a foundation for future research in this area.
\end{itemize}
To the best of our knowledge, no prior work has applied machine learning to the problem of predicting probabilities of causation. Our study establishes a new research direction by bridging causal inference and machine learning for practical estimation tasks.

The remainder of the paper is structured as follows: first, we review key causal inference concepts to provide necessary context. Next, we introduce the model and dataset used in our study. Finally, we present our five machine learning models developed for the task. All code for data generation and machine learning models is included in the appendix.

\section{Preliminaries}
\label{related work}
In this section, we review the fundamental concepts of causal inference necessary for understanding the rest of the paper. We begin by discussing the definitions of PNS, PS, and PN as introduced by \cite{pearl1999probabilities}, followed by the definitions of identifiability and the conditions required to identify PNS, PS, and PN \citep{tian2000probabilities}. Additionally, we examine the tight bounds of PNS, PS, and PN in cases where they are unidentifiable \citep{tian2000probabilities}. Readers already familiar with these concepts may skip this section.

Similar to the works mentioned above, we adopt the causal language of Structural Causal Models (SCMs) \citep{galles1998axiomatic,halpern2000axiomatizing}. In this framework, the counterfactual statement ``Variable \( Y \) would have the value \( y \) had \( X \) been \( x \)'' is denoted as \( Y_x = y \), abbreviated as \( y_x \). We consider two types of data: experimental data, expressed as causal effects \( P(y_x) \), and observational data, represented by the joint probability function \( P(x, y) \). Unless otherwise specified, we assume \( X \) and \( Y \) are binary variables in a causal model \( M \), with \( x \) and \( y \) denoting the propositions \( X = \text{true} \) and \( Y = \text{true} \), respectively, and \( x' \) and \( y' \) representing their complements. For simplicity, we focus on binary variables; extensions to multi-valued cases are discussed by \cite{pearl2009causality} (p. 286, footnote 5) and \cite{li2024probabilities}.

First, the definitions of three basic probabilities of causation defined using SCM are as follow \citep{pearl1999probabilities}:

\begin{definition}[Probability of necessity (PN)]
Let $X$ and $Y$ be two binary variables in a causal model $M$, let $x$ and $y$ stand for the propositions $X=true$ and $Y=true$, respectively, and $x'$ and $y'$ for their complements. The probability of necessity is defined as the expression 
\begin{eqnarray}
\text{PN} &\delequal& P(Y_{X=false}=false|X=true,Y=true)\nonumber\\
 &\delequal&  P(y'_{x'}|x,y) \nonumber
\end{eqnarray}
\end{definition}

\begin{definition}[Probability of sufficiency (PS)]
Let $X$ and $Y$ be two binary variables in a causal model $M$, let $x$ and $y$ stand for the propositions $X=true$ and $Y=true$, respectively, and $x'$ and $y'$ for their complements. The probability of sufficiency is defined as the expression
\begin{eqnarray}
\text{PS} &\delequal& P(Y_{X=true}=true|X=false,Y=false)\nonumber\\
&\delequal& P(y_x|x',y') \nonumber
\end{eqnarray}
\end{definition}

\begin{definition}[Probability of necessity and sufficiency (PNS)] Let $X$ and $Y$ be two binary variables in a causal model $M$, let $x$ and $y$ stand for the propositions $X=true$ and $Y=true$, respectively, and $x'$ and $y'$ for their complements. The probability of necessity and sufficiency is defined as the expression
\begin{eqnarray}
\text{PNS} &\delequal& P(Y_{X=true}=true,Y_{X=false}=false)\nonumber\\
&\delequal& P(y_x,y'_{x'}) \nonumber
\end{eqnarray}
\end{definition}
Then, we review the identification conditions for PNS, PS, and PN \citep{tian2000probabilities}.

\begin{definition} (Monotonicity)
A Variable $Y$ is said to be monotonic relative to variable $X$ in a causal model $M$ iff
\begin{eqnarray*}
y'_x\land y_{x'}=\text{false}.
\end{eqnarray*}
\end{definition}

\begin{theorem}
If $Y$ is monotonic relative to $X$, then PNS, PN, and PS are all identifiable, and
\begin{eqnarray*}
PNS = P(y_x) - P(y_{x'}),\\
PN = \frac{P(y) - P(y_{x'})}{P(x,y)},\\
PS = \frac{P(y_x) - P(y)}{P(x', y')}.
\end{eqnarray*}
\end{theorem}

If PNS, PN, and PS are not identifiable, informative bounds are given by \cite{tian2000probabilities}.

% \begin{eqnarray}
% \max \left \{
% \begin{array}{cc}
% 0, \\
% P(y_x) - P(y_{x'}), \\
% P(y) - P(y_{x'}), \\
% P(y_x) - P(y)
% \end{array}
% \right \} \leqslant
% \text{PNS}\leqslant
% \min \left \{
% \begin{array}{cc}
%  P(y_x), \\
%  P(y'_{x'}), \\
% P(x,y) + P(x',y'), \\
% P(y_x) - P(y_{x'}) +\\
% P(x, y') + P(x', y)
% \end{array} 
% \right \}
% \label{pns}
% \end{eqnarray}

% \begin{eqnarray}
% \max \left \{
% \begin{array}{cc}
% 0, \\
% \frac{P(y)-P(y_{x'})}{P(x,y)}
% \end{array} 
% \right \} \leqslant
% \text{PN} \leqslant
% \min \left \{
% \begin{array}{cc}
% 1, \\
% \frac{P(y'_{x'})-P(x',y')}{P(x,y)}
% \end{array}
% \right \}
% \label{pn}
% \end{eqnarray}

% \begin{eqnarray}
% \max \left \{
% \begin{array}{cc}
% 0, \\
% \frac{P(y')-P(y'_{x})}{P(x',y')}
% \end{array} 
% \right \} \leqslant
% \text{PS}\leqslant
% \min \left \{
% \begin{array}{cc}
% 1, \\
% \frac{P(y_{x})-P(x,y)}{P(x',y')}
% \end{array}
% \right \}
% \label{ps}
% \end{eqnarray}

\begin{eqnarray}
\max \left \{
\begin{array}{cc}
0, \\
P(y_x) - P(y_{x'}), \\
P(y) - P(y_{x'}), \\
P(y_x) - P(y)
\end{array}
\right \} \le
\text{PNS}\label{pnslb}\\
\min \left \{
\begin{array}{cc}
 P(y_x), \\
 P(y'_{x'}), \\
P(x,y) + P(x',y'), \\
P(y_x) - P(y_{x'}) +\\
P(x, y') + P(x', y)
\end{array} 
\right \}\ge
\text{PNS}
\label{pnsub}\\
\max \left \{
\begin{array}{cc}
0, \\
\frac{P(y)-P(y_{x'})}{P(x,y)}
\end{array} 
\right \} \le
\text{PN} \label{pnlb}\\
\min \left \{
\begin{array}{cc}
1, \\
\frac{P(y'_{x'})-P(x',y')}{P(x,y)}
\end{array}
\right \}\ge \text{PN}
\label{pnub}\\
\max \left \{
\begin{array}{cc}
0, \\
\frac{P(y')-P(y'_{x})}{P(x',y')}
\end{array} 
\right \} \le
\text{PS} \label{pslb}\\
\min \left \{
\begin{array}{cc}
1, \\
\frac{P(y_{x})-P(x,y)}{P(x',y')}
\end{array}
\right \} \ge \text{PS}
\label{psub}
\end{eqnarray}

Therefore, the primary objective of this paper is then to predict Equations \ref{pnslb} to \ref{psub} (i.e., the lower and upper bounds of the PNS, PS, and PN) for any (sub)populations using those with sufficient data (i.e., sufficient data to estimate the distributions $P(X,Y)$ and $P(Y_X)$.) Due to space constraints, the focus will be on the bounds of PNS (i.e., Equations \ref{pnslb} and \ref{pnsub}). Unless otherwise specified, the discussion will be limited to binary treatment and effect, meaning both $X$ and $Y$ are binary.

\section{Structural Causal Model}
\label{scm}
In general, the equations in SCMs are in implicitly form (e.g., $Z=f_Z(X,Y,U_Z)$). However, in order to verify the accuracy of the learned bounds of PNS, we need to explicitly define the SCM and the data-generating process to determine the true PNS value and its bounds. Followed the setup in \cite{li2022learning}, we will use the following SCM. 
\begin{eqnarray*}
    \begin{cases}
        Z_i &= U_{Z_i} \text{ for } i \in \{1,...,20\},\\
        X&=f_X(M_X,U_X)\\
        &=\begin{cases}
            1& \text{ if } M_X+U_X > 0.5\\
            0& \text{ otherwise, }\\
        \end{cases}\\
        Y&=f_Y(X,M_Y,U_Y)\\
        &=\begin{cases}
            1& \text{ if } 0<C_Y \cdot X+M_Y+U_Y <1 \\
            1& \text{ if } 1<C_Y \cdot X+M_Y+U_Y <2 \\
            0& \text{ otherwise. }\\
        \end{cases}
    \end{cases}
\end{eqnarray*}
where $X,Y,Z_i$ are all binary, $U_{Z_i}, U_X, U_Y$ are binary exogenous variables with Bernoulli distributions, $C_Y$ is a constant, and $M_X, M_Y$ are linear combinations of $Z_i$. The randomly generated value of $C_Y,M_X,M_Y$ and the distributions of $U_X,U_Y,U_{Z_i}$ for the model are provided in the appendix.

\section{Data Generating Process}
Based on the defined model, $20$ binary features are considered (i.e., $Z_1,...,Z_{20}$). We made $15$ observable ($Z_1,...,Z_{15}$) and $5$ unobservable, and the exogenous variables are also unobservable, leading to $2^{15}$ observed subpopulations (i.e., the combination of $Z_1,...,Z_{15}$ defined a subpopulation).

\subsection{Informer Data}  
To evaluate the learned bounds, the informer data must have access to the actual PNS bounds for each subpopulation. Given the explicit form of the SCM and the distributions of all exogenous variables, the PNS bounds, as well as the experimental and observational distributions, can be computed for each combination of the features \( Z_1, \dots, Z_{15} \) (i.e., a subpopulation) using the SCM. For detailed mathematical formulations, refer to the appendix.

\subsection{Sample Collection}
A total of $50,000,000$ experimental and $50,000,000$ observational samples were generated as follows for each sample: In both settings, the exogenous variables \( U_X \), \( U_Y \), and \( U_{Z_i} \) were randomly generated according to their distributions specified in Section \ref{scm}. In the experimental setting, \( X \) was then assigned according to a \( \text{Bernoulli}(0.5) \) distribution, while \( Y \) and \( Z_i \) were computed using the structural functions described in Section \ref{scm}. In the observational setting, \( X \), \( Y \), and \( Z_i \) were all determined by the structural functions. The final datasets include only the observable features \( Z_1, \dots, Z_{15} \), along with \( X \) and \( Y \), while \( Z_{16}, \dots, Z_{20} \) were masked.

\subsection{Data for Machine Learning Models}  
We selected subpopulations from the $2^{15}$ possible groups that contained at least $1,300$ experimental and observational samples ($1,300$ based on \cite{li2022probabilities}'s suggestions). For these selected subpopulations, we computed the experimental and observational distributions and determined the bounds of PNS using Equations \ref{pnslb} and \ref{pnsub}. These results served as the data for our machine learning models (i.e., each data entry consists of 15 features and the PNS bounds as the label.) The obtained data includes $2,054$ entries for the lower bound (LB) and $2,065$ entries for the upper bound (UB) of the PNS.
\section{Machine Learning Prediction}\label{sec:ml}
To evaluate the feasibility of machine learning in predicting the bounds of the PNS, we employed five distinct machine learning models to assess their effectiveness in this task: Support Vector Machine (SVM) \citep{svm}, Random Forest (RF) \citep{rf}, Gradient Boosting Decision Trees (GBDT) \citep{gbdt}, Transformer \citep{vaswani2017attention}, and Multilayer Perceptron (MLP) \citep{mlp}. These models were chosen to represent a diverse range of machine learning paradigms, including kernel-based methods (SVM), ensemble learning techniques (RF and GBDT), and deep learning approaches (MLP and Transformer). This selection ensures a comprehensive evaluation of their ability to approximate causal quantities across different settings. A detailed pipeline is illustrated in Figure \ref{fig:flow chart}.

\begin{figure*}[!htbp] 
    \centering
    \includegraphics[width=\textwidth]{imgs/flow_chart.pdf}
    \caption{Framework for Causal Data Generation and Machine Learning Prediction.}
    \label{fig:flow chart}
\end{figure*}

\begin{figure*}[!htb]
    \centering
    % 第一行:SVM, RF, GBDT, Transformer Lower Bound
    \begin{subfigure}[b]{0.24\linewidth}
        \centering
        \includegraphics[width=\linewidth]{imgs/svm_true_vs_pred_lb.pdf}
        \caption{SVM (Lower bound)}
        \label{fig:svm2}
    \end{subfigure}
    \hfill
    \begin{subfigure}[b]{0.24\linewidth}
        \centering
        \includegraphics[width=\linewidth]{imgs/rf_true_vs_pred_lb.pdf}
        \caption{RF (Lower bound)}
        \label{fig:rf2}
    \end{subfigure}
    \hfill
    \begin{subfigure}[b]{0.24\linewidth}
        \centering
        \includegraphics[width=\linewidth]{imgs/gbdt_true_vs_pred_lb.pdf}
        \caption{GBDT (Lower bound)}
        \label{fig:gbdt2}
    \end{subfigure}
    \hfill
    \begin{subfigure}[b]{0.24\linewidth}
        \centering
        \includegraphics[width=\linewidth]{imgs/transformer_true_vs_pred_lb.pdf}
        \caption{Transformer (Lower bound)}
        \label{fig:transformer2}
    \end{subfigure}

    \vspace{0.3cm} % 调整上下间距

    % 第二行:SVM, RF, GBDT, Transformer Upper Bound
    \begin{subfigure}[b]{0.24\linewidth}
        \centering
        \includegraphics[width=\linewidth]{imgs/svm_true_vs_pred_ub.pdf}
        \caption{SVM (Upper bound)}
        \label{fig:svm3}
    \end{subfigure}
    \hfill
    \begin{subfigure}[b]{0.24\linewidth}
        \centering
        \includegraphics[width=\linewidth]{imgs/rf_true_vs_pred_ub.pdf}
        \caption{RF (Upper bound)}
        \label{fig:rf3}
    \end{subfigure}
    \hfill
    \begin{subfigure}[b]{0.24\linewidth}
        \centering
        \includegraphics[width=\linewidth]{imgs/gbdt_true_vs_pred_ub.pdf}
        \caption{GBDT (Upper bound)}
        \label{fig:gbdt3}
    \end{subfigure}
    \hfill
    \begin{subfigure}[b]{0.24\linewidth}
        \centering
        \includegraphics[width=\linewidth]{imgs/transformer_true_vs_pred_ub.pdf}
        \caption{Transformer (Upper bound)}
        \label{fig:transformer3}
    \end{subfigure}

    \caption{Comparison of true and predicted values across different models for both lower and upper bounds.}
    \label{fig:combined_model_comparison}
\end{figure*}
\subsection{Support Vector Machine}
Support Vector Machines (SVM) \citep{svm} are widely used and well-established supervised learning models. Given their strengths, we selected Support Vector Regression (SVR), a variant of SVM, as the first model for our experiments. To effectively capture complex patterns, we employed the Radial Basis Function (RBF) kernel to map the data into a high-dimensional feature space.

Key hyperparameters include the penalty parameter (\( C \)), the insensitive loss threshold (\( \epsilon \)), and the kernel coefficient (\( \gamma \)). The parameter \( C \) controls the trade-off between model complexity and error tolerance, where larger values may lead to overfitting. The threshold \( \epsilon \) defines the margin of tolerance for errors, while \( \gamma \) determines the influence range of individual data points.

A two-stage hyperparameter tuning strategy was adopted. First, Randomized Search \citep{randomsearch} was employed to efficiently explore the parameter space and identify promising ranges. Then, Grid Search \citep{randomsearch} was used to fine-tune parameters within these ranges. Cross-validation ensured robust generalization throughout the tuning process.

\begin{figure*}[!htb]
    \centering
    % 第一行:SVM, RF, GBDT Lower Bounds
    \begin{subfigure}[b]{0.32\linewidth}
        \centering
        \includegraphics[width=\linewidth]{imgs/confusion_matrix_svm_lb.pdf}
        \caption{SVM Lower bound.}
        \label{fig:svm_lb} % 放在 caption 后
    \end{subfigure}
    \hfill
    \begin{subfigure}[b]{0.32\linewidth}
        \centering
        \includegraphics[width=\linewidth]{imgs/confusion_matrix_rf_lb.pdf}
        \caption{RF Lower bound.}
        \label{fig:rf_lb}
    \end{subfigure}
    \hfill
    \begin{subfigure}[b]{0.32\linewidth}
        \centering
        \includegraphics[width=\linewidth]{imgs/confusion_matrix_gbdt_lb.pdf}
        \caption{GBDT Lower bound.}
        \label{fig:gbdt_lb}
    \end{subfigure}

    % 第二行:SVM, RF, GBDT Upper Bounds
    \begin{subfigure}[b]{0.32\linewidth}
        \centering
        \includegraphics[width=\linewidth]{imgs/confusion_matrix_svm_ub.pdf}
        \caption{SVM Upper bound.}
        \label{fig:svm_ub}
    \end{subfigure}
    \hfill
    \begin{subfigure}[b]{0.32\linewidth}
        \centering
        \includegraphics[width=\linewidth]{imgs/confusion_matrix_rf_ub.pdf}
        \caption{RF Upper bound.}
        \label{fig:rf_ub}
    \end{subfigure}
    \hfill
    \begin{subfigure}[b]{0.32\linewidth}
        \centering
        \includegraphics[width=\linewidth]{imgs/confusion_matrix_gbdt_ub.pdf}
        \caption{GBDT Upper bound.}
        \label{fig:gbdt_ub}
    \end{subfigure}

    % 第三行:Transformer
    \begin{subfigure}[b]{0.32\linewidth}
        \centering
        \includegraphics[width=\linewidth]{imgs/confusion_matrix_transformer_lb.pdf}
        \caption{Transformer Lower bound.}
        \label{fig:transformer_lb}
    \end{subfigure}
    \begin{subfigure}[b]{0.32\linewidth}
        \centering
        \includegraphics[width=\linewidth]{imgs/confusion_matrix_transformer_ub.pdf}
        \caption{Transformer Upper bound.}
        \label{fig:transformer_ub}
    \end{subfigure}

    \caption{Confusion matrices of SVM, RF, GBDT, and Transformer models.}
    \label{fig:confusion_matrices_combined}
\end{figure*}

% \begin{table}[!h]
%     \centering
%     \caption{SVM Result} \label{tab:svm}
%     \begin{tabular}{rll}
%         \toprule % from booktabs package
%         \bfseries Dataset & \bfseries  MSE & \bfseries MAE\\
%         \midrule % from booktabs package
%         Lower bound & 0.0112 & 0.0868\\
%         Upper bound & 0.0304 & 0.1527\\
%         \bottomrule % from booktabs package
%     \end{tabular}
% \end{table}
Finally, the mean squared error (MSE) and mean absolute error (MAE) values of the SVR model can be found in Table \ref{tab:comparison}. Confusion matrices are presented in Figures \ref{fig:svm_lb} and \ref{fig:svm_ub}, while Figures \ref{fig:svm2} and \ref{fig:svm3} provide a clearer comparison with the true PNS bounds. For the prediction of the lower bound, SVR demonstrates reasonable effectiveness; however, for the more complex upper bound, it exhibits a significant decline in accuracy.

\subsection{Random Forest}
Random Forests (RF) \citep{rf} are a widely used ensemble learning method for classification, regression, and other predictive tasks. The core idea behind RF is to construct multiple decision trees during training and aggregate their outputs to enhance overall performance. As an ensemble model, RF exhibits strong robustness, motivating us to assess its effectiveness in predicting PNS bounds.

Key hyperparameters of RF include the number of trees (\( n_{\text{estimators}} \)), maximum tree depth (\( \text{max\_depth} \)), minimum samples required to split a node (\( \text{min\_samples\_split} \)), and the number of features considered for splitting (\( \text{max\_features} \)). Increasing \( n_{\text{estimators}} \) generally improves performance but at the expense of higher computational costs. The parameters \( \text{max\_depth} \), \( \text{min\_samples\_split} \), and \( \text{max\_features} \) regulate tree complexity, balancing bias-variance trade-offs.

For hyperparameter optimization, we employed a two-stage tuning strategy similar to that used for SVM. Table \ref{tab:comparison} also presents RF's MAE and MSE results, while Figures \ref{fig:rf_lb} and \ref{fig:rf_ub} show its confusion matrices. A more direct comparison with true PNS bounds is provided in Figures \ref{fig:rf2} and \ref{fig:rf3}. RF performs comparably to SVM on the lower bound but exhibits significantly higher accuracy on the upper bound.
% \begin{table}[!h]
%     \centering
%     \caption{RF Result} \label{tab:rf}
%     \begin{tabular}{rll}
%         \toprule % from booktabs package
%         \bfseries Dataset & \bfseries  MSE & \bfseries MAE\\
%         \midrule % from booktabs package
%         Lower bound & 0.0116 & 0.0919\\
%         Upper bound & 0.0205 & 0.1242\\
%         \bottomrule % from booktabs package
%     \end{tabular}
% \end{table}
\subsection{Gradient Boosting Decision Trees}
Gradient Boosting Decision Trees (GBDT) \citep{gbdt} is an ensemble learning method that builds models sequentially, with each new tree correcting the errors of its predecessors. Unlike traditional boosting, GBDT optimizes pseudo-residuals, enabling flexible loss function optimization. Simple decision trees serve as weak learners, allowing GBDT to effectively capture complex data patterns.

Key hyperparameters include the number of trees (\( n_{\text{estimators}} \)), learning rate (\( \text{learning\_rate} \)), maximum tree depth (\( \text{max\_depth} \)), and subsample ratio (\( \text{subsample} \)). The learning rate determines each tree’s contribution, while \( n_{\text{estimators}} \) and \( \text{max\_depth} \) regulate model complexity and performance.

Following the approach used for SVM and RF, we applied a two-stage tuning strategy. Again, table \ref{tab:comparison} presents the MSE and MAE results, while Figures \ref{fig:gbdt_lb} and \ref{fig:gbdt_ub} show the confusion matrices. A more direct comparison with true PNS bounds is provided in Figures \ref{fig:gbdt2} and \ref{fig:gbdt3}. GBDT demonstrates moderate performance on both the lower and upper bounds.
% \begin{table}[!h]
%     \centering
%     \caption{GBDT Result} \label{tab:gbdt}
%     \begin{tabular}{rll}
%         \toprule
%         \bfseries Dataset & \bfseries MSE & \bfseries MAE \\
%         \midrule
%         Lower bound & 0.0159 & 0.1049 \\
%         Upper bound & 0.0261 & 0.1399 \\
%         \bottomrule
%     \end{tabular}
% \end{table}
\subsection{Transformer}
The Transformer \citep{vaswani2017attention}, originally developed for Natural Language Processing, has expanded into Computer Vision and become a cornerstone of deep learning, particularly with the rise of large language models. Given its significant impact, this study also evaluates the Transformer for testing.

The model architecture begins with an input layer processing 15-dimensional feature vectors, followed by a linear embedding layer that projects inputs into a 64-dimensional space. Positional encoding is applied to retain feature order information, and two Transformer encoder layers with four attention heads each capture complex feature interactions. The final output is generated through a fully connected layer with a Sigmoid activation function, ensuring predictions remain within the range \([0, 1]\). Key hyperparameters include an embedding dimension of 64, four attention heads, two encoder layers, and a dropout rate of 0.1.

Similarly, table \ref{tab:comparison} presents the MSE and MAE results, while Figures \ref{fig:transformer_lb} and \ref{fig:transformer_ub} show the confusion matrices. A direct comparison with true PNS bounds is provided in Figures \ref{fig:transformer2} and \ref{fig:transformer3}. The Transformer demonstrates strong performance on the lower bound and moderate performance on the upper bound.
% \begin{table}[!h]
%     \centering
%     \caption{Transformer Result} \label{tab:transformer}
%     \begin{tabular}{rll}
%         \toprule
%         \bfseries Dataset & \bfseries MSE & \bfseries MAE \\
%         \midrule
%         Lower bound & 0.0030 & 0.0348 \\
%         Upper bound & 0.0156 & 0.1060 \\
%         \bottomrule
%     \end{tabular}
% \end{table}
\subsection{Multilayer Perceptron}
MLP \citep{mlp} consists of an input layer, one or more hidden layers, and an output layer. With appropriate activation functions, it can effectively model both linear and nonlinear relationships. As a fundamental structure in deep learning, MLP holds significant representativeness, motivating its inclusion in our experiments.

A key consideration for MLP is the choice of activation function, particularly for predicting the lower bound. Since the lower bound of PNS cannot be negative, we initially selected the ReLU \citep{relu} activation function (\ref{equ:relu}). However, ReLU can lead to the loss of negatively correlated features, prompting us to adopt LeakyReLU \citep{lkrelu} (\ref{equ:lkrelu}) as a complementary solution. Furthermore, given the considerable number of zero values in the data, the non-differentiability of ReLU and LeakyReLU at \(s = 0\) imposes limitations on backpropagation. To address this, we proposed using Mish \citep{mish} (\ref{equ:mish}) as an alternative activation function. The corresponding equations are:

\begin{equation}\label{equ:relu}
    \text{ReLU}(s) = \max(0, s)
\end{equation}

\begin{equation}\label{equ:lkrelu}
    \text{LeakyReLU}(s) = 
    \begin{cases}
        s, & \text{if } s \ge 0 \\
        \alpha s, & \text{if } s < 0
    \end{cases}
\end{equation}

\begin{equation}\label{equ:mish}
    \text{Mish}(s) = s \cdot \tanh(\ln(1 + e^s))
\end{equation}

Additionally, we implemented an MLP with the architecture \( 15 \rightarrow 64 \rightarrow 32 \rightarrow 16 \rightarrow 1 \), utilizing ReLU-like functions and Sigmoid as activation functions. The model was optimized using the Adam optimizer with a learning rate of $0.01$ and trained for $1000$ epochs. 

Again, the final results are presented in Table \ref{tab:comparison}. With the Mish activation function, the MLP achieved an MSE of \textbf{0.0011} on the lower bound and \textbf{0.0010} on the upper bound. For MAE, it attained \textbf{0.0225} on the lower bound and \textbf{0.0247} on the upper bound. The confusion matrix is shown in Figure \ref{fig:mlp_comparison}, and a clearer comparison with the true PNS bounds is provided in Figure \ref{fig:mlp2} (Only the best performance comparisons with Mish are shown). 

Overall, MLP significantly outperformed other machine learning models, with Mish yielding the best results among the activation functions. The comparison with the true PNS bounds further confirms that MLP (Mish) provides an accurate and practical model for predicting PNS.
% \begin{table}[!h]
%     \centering
%     \caption{MLP Result} \label{tab:mlp}
%     \begin{tabular}{rlll}
%         \toprule
%         \bfseries Dataset & \bfseries Activation Function & \bfseries MSE & \bfseries MAE \\
%         \midrule
%         Lower bound &ReLU & 0.0045 & 0.0434 \\
%         Upper bound &ReLU & 0.0023 & 0.0357 \\
%         Lower bound &LeakyReLU & 0.0038 & 0.0379 \\
%         Upper bound &LeakyReLU & 0.0024 & 0.0380 \\
%         Lower bound &Mish & \textbf{0.0011} & \textbf{0.0225} \\
%         Upper bound &Mish & \textbf{0.0010} & \textbf{0.0247} \\
%         \bottomrule
%     \end{tabular}
% \end{table}
%%%%%%%%%%%%%%%%%%%%%%%%%%%%
\begin{figure*}[!htb]  % 使用 figure* 跨越两列
    \centering
    % 第一排:ReLU, Leaky ReLU, Mish 的 Lower Bound
    \begin{subfigure}[b]{0.32\linewidth}
        \centering
        \includegraphics[width=\linewidth]{imgs/confusion_matrix_nn_relu_lb.pdf}
        \caption{ReLU (Lower bound)}
    \end{subfigure}
    \hfill
    \begin{subfigure}[b]{0.32\linewidth}
        \centering
        \includegraphics[width=\linewidth]{imgs/confusion_matrix_nn_lkrelu_lb.pdf}
        \caption{Leaky ReLU (Lower bound)}
    \end{subfigure}
    \hfill
    \begin{subfigure}[b]{0.32\linewidth}
        \centering
        \includegraphics[width=\linewidth]{imgs/confusion_matrix_nn_mish_lb.pdf}
        \caption{Mish (Lower bound)}
    \end{subfigure}

    \vspace{0.3cm} % 调整上下间距

    % 第二排:ReLU, Leaky ReLU, Mish 的 Upper Bound
    \begin{subfigure}[b]{0.32\linewidth}
        \centering
        \includegraphics[width=\linewidth]{imgs/confusion_matrix_nn_relu_ub.pdf}
        \caption{ReLU (Upper bound)}
    \end{subfigure}
    \hfill
    \begin{subfigure}[b]{0.32\linewidth}
        \centering
        \includegraphics[width=\linewidth]{imgs/confusion_matrix_nn_lkrelu_ub.pdf}
        \caption{Leaky ReLU (Upper bound)}
    \end{subfigure}
    \hfill
    \begin{subfigure}[b]{0.32\linewidth}
        \centering
        \includegraphics[width=\linewidth]{imgs/confusion_matrix_nn_mish_ub.pdf}
        \caption{Mish (Upper bound)}
    \end{subfigure}

    \caption{Confusion matrices of MLP with different activation functions: ReLU, Leaky ReLU, and Mish for both lower and upper bounds.}
    \label{fig:mlp_comparison}
\end{figure*}

%%%%%%%%%%%%%%%%%%%%%%%%%%%%


\begin{figure}[!htb]
    \centering
    % 左侧:Lower bound
    \begin{subfigure}[b]{0.48\linewidth}
        \centering
        \includegraphics[width=\linewidth]{imgs/mlp_true_vs_pred_lb.pdf}
        \caption{Lower bound.}
        \label{fig:mlp_lb}
    \end{subfigure}
    \hfill
    % 右侧:Upper bound
    \begin{subfigure}[b]{0.48\linewidth}
        \centering
        \includegraphics[width=\linewidth]{imgs/mlp_true_vs_pred_ub.pdf}
        \caption{Upper bound.}
        \label{fig:mlp_ub}
    \end{subfigure}
    
    \caption{Comparison of MLP (Mish) for lower and upper bounds.}
    \label{fig:mlp2}
\end{figure}





\subsection{Experimental Comparison}
% As shown in Table \ref{tab:comparison}, the overall performance of MLP is undoubtedly the best. Among the other four machine learning models, SVM achieves good results on the lower bound but performs the worst on the upper bound, with performance close to complete failure. RF performs significantly better, achieving acceptable results on both the lower and upper bounds. In contrast, GBDT performs worse than RF, despite also being a tree-based model, showing inferior results across the board with only a slight improvement over SVM on the upper bound. The Transformer, as a model based on MLP, performs better than the other machine learning models but still falls short of MLP's performance.

% For MLP models, due to the special characteristics of the dataset around zero, different activation functions exhibit significant performance differences. The basic ReLU exhibits suboptimal performance on the lower bound. Considering negative values, LeakyReLU shows slightly better performance than ReLU on the lower bound. Mish, which not only accounts for negative values but also ensures differentiability around zero, achieves the best results.
As shown in Table \ref{tab:comparison}, MLP delivers the best overall performance. Among the other four machine learning models, SVM performs well on the lower bound but fails almost entirely on the upper bound. RF shows significantly better results, achieving acceptable performance on both bounds. Despite also being a tree-based model, GBDT underperforms compared to RF, with only a slight improvement over SVM on the upper bound. The Transformer, as an MLP-based model, outperforms the other machine learning models but still falls short of MLP’s performance.  

For MLP models, the dataset’s characteristics around zero (we will discuss these characteristics in the discussion section) lead to notable differences in activation function performance. Basic ReLU shows suboptimal performance on the lower bound, while LeakyReLU, which accounts for negative values, performs slightly better. Mish, which not only handles negative values but also ensures differentiability around zero, achieves the best results.

\begin{table}[!ht]
    \centering
-    \caption{Comparison of Model Performance} 
    \label{tab:comparison}
    \begin{tabular}{rlll}
        \toprule
        \bfseries Model & \bfseries Dataset & \bfseries MSE & \bfseries MAE \\
        \midrule
        SVM  & Lower bound & 0.0112 & 0.0868 \\
             & Upper bound & 0.0304 & 0.1527 \\
        RF   & Lower bound & 0.0116 & 0.0919 \\
             & Upper bound & 0.0205 & 0.1242 \\
        GBDT & Lower bound & 0.0159 & 0.1049 \\
             & Upper bound & 0.0261 & 0.1399 \\
        Transformer & Lower bound & 0.0030 & 0.0348 \\
             & Upper bound & 0.0156 & 0.1060 \\
        MLP(ReLU)  & Lower bound & 0.0045 & 0.0434 \\
             & Upper bound & 0.0023 & 0.0357 \\
        MLP(LeakyReLU)  & Lower bound & 0.0038 & 0.0379 \\
             & Upper bound & 0.0024 & 0.0380 \\
        MLP(Mish)  & Lower bound & \textbf{0.0011} & \textbf{0.0225} \\
             & Upper bound & \textbf{0.0010} & \textbf{0.0247} \\
             
        \bottomrule
    \end{tabular}
\end{table}
\section{ Task Generalization Beyond i.i.d. Sampling and Parity Functions
}\label{sec:Discussion}
% Discussion: From Theory to Beyond
% \misha{what is beyond?}
% \amir{we mean two things: in the first subsection beyond i.i.d subsampling of parity tasks and in the second subsection beyond parity task}
% \misha{it has to be beyond something, otherwise it is not clear what it is about} \hz{this is suggested by GPT..., maybe can be interpreted as from theory to beyond theory. We can do explicit like Discussion: Beyond i.i.d. task sampling and the Parity Task}
% \misha{ why is "discussion" in the title?}\amir{Because it is a discussion, it is not like separate concrete explnation about why these thing happens or when they happen, they just discuss some interesting scenraios how it relates to our theory.   } \misha{it is not really a discussion -- there is a bunch of experiments}

In this section, we extend our experiments beyond i.i.d. task sampling and parity functions. We show an adversarial example where biased task selection substantially hinders task generalization for sparse parity problem. In addition, we demonstrate that exponential task scaling extends to a non-parity tasks including arithmetic and multi-step language translation.

% In this section, we extend our experiments beyond i.i.d. task sampling and parity functions. On the one hand, we find that biased task selection can significantly degrade task generalization; on the other hand, we show that exponential task scaling generalizes to broader scenarios.
% \misha{we should add a sentence or two giving more detail}


% 1. beyond i.i.d tasks sampling
% 2. beyond parity -> language, arithmetic -> task dependency + implicit bias of transformer (cannot implement this algorithm for arithmatic)



% In this section, we emphasize the challenge of quantifying the level of out-of-distribution (OOD) differences between training tasks and testing tasks, even for a simple parity task. To illustrate this, we present two scenarios where tasks differ between training and testing. For each scenario, we invite the reader to assess, before examining the experimental results, which cases might appear “more” OOD. All scenarios consider \( d = 10 \). \kaiyue{this sentence should be put into 5.1}






% for parity problem




% \begin{table*}[th!]
%     \centering
%     \caption{Generalization Results for Scenarios 1 and 2 for $d=10$.}
%     \begin{tabular}{|c|c|c|c|}
%         \hline
%         \textbf{Scenario} & \textbf{Type/Variation} & \textbf{Coordinates} & \textbf{Generalization accuracy} \\
%         \hline
%         \multirow{3}{*}{Generalization with Missing Pair} & Type 1 & \( c_1 = 4, c_2 = 6 \) & 47.8\%\\ 
%         & Type 2 & \( c_1 = 4, c_2 = 6 \) & 96.1\%\\ 
%         & Type 3 & \( c_1 = 4, c_2 = 6 \) & 99.5\%\\ 
%         \hline
%         \multirow{3}{*}{Generalization with Missing Pair} & Type 1 &  \( c_1 = 8, c_2 = 9 \) & 40.4\%\\ 
%         & Type 2 & \( c_1 = 8, c_2 = 9 \) & 84.6\% \\ 
%         & Type 3 & \( c_1 = 8, c_2 = 9 \) & 99.1\%\\ 
%         \hline
%         \multirow{1}{*}{Generalization with Missing Coordinate} & --- & \( c_1 = 5 \) & 45.6\% \\ 
%         \hline
%     \end{tabular}
%     \label{tab:generalization_results}
% \end{table*}

\subsection{Task Generalization Beyond i.i.d. Task Sampling }\label{sec: Experiment beyond iid sampling}

% \begin{table*}[ht!]
%     \centering
%     \caption{Generalization Results for Scenarios 1 and 2 for $d=10, k=3$.}
%     \begin{tabular}{|c|c|c|}
%         \hline
%         \textbf{Scenario}  & \textbf{Tasks excluded from training} & \textbf{Generalization accuracy} \\
%         \hline
%         \multirow{1}{*}{Generalization with Missing Pair}
%         & $\{4,6\} \subseteq \{s_1, s_2, s_3\}$ & 96.2\%\\ 
%         \hline
%         \multirow{1}{*}{Generalization with Missing Coordinate}
%         & \( s_2 = 5 \) & 45.6\% \\ 
%         \hline
%     \end{tabular}
%     \label{tab:generalization_results}
% \end{table*}




In previous sections, we focused on \textit{i.i.d. settings}, where the set of training tasks $\mathcal{F}_{train}$ were sampled uniformly at random from the entire class $\mathcal{F}$. Here, we explore scenarios that deliberately break this uniformity to examine the effect of task selection on out-of-distribution (OOD) generalization.\\

\textit{How does the selection of training tasks influence a model’s ability to generalize to unseen tasks? Can we predict which setups are more prone to failure?}\\

\noindent To investigate this, we consider two cases parity problems with \( d = 10 \) and \( k = 3 \), where each task is represented by its tuple of secret indices \( (s_1, s_2, s_3) \):

\begin{enumerate}[leftmargin=0.4 cm]
    \item \textbf{Generalization with a Missing Coordinate.} In this setup, we exclude all training tasks where the second coordinate takes the value \( s_2 = 5 \), such as \( (1,5,7) \). At test time, we evaluate whether the model can generalize to unseen tasks where \( s_2 = 5 \) appears.
    \item \textbf{Generalization with Missing Pair.} Here, we remove all training tasks that contain both \( 4 \) \textit{and} \( 6 \) in the tuple \( (s_1, s_2, s_3) \), such as \( (2,4,6) \) and \( (4,5,6) \). At test time, we assess whether the model can generalize to tasks where both \( 4 \) and \( 6 \) appear together.
\end{enumerate}

% \textbf{Before proceeding, consider the following question:} 
\noindent \textbf{If you had to guess.} Which scenario is more challenging for generalization to unseen tasks? We provide the experimental result in Table~\ref{tab:generalization_results}.

 % while the model struggles for one of them while as it generalizes almost perfectly in the other one. 

% in the first scenario, it generalizes almost perfectly in the second. This highlights how exposure to partial task structures can enhance generalization, even when certain combinations are entirely absent from the training set. 

In the first scenario, despite being trained on all tasks except those where \( s_2 = 5 \), which is of size $O(\d^T)$, the model struggles to generalize to these excluded cases, with prediction at chance level. This is intriguing as one may expect model to generalize across position. The failure  suggests that positional diversity plays a crucial role in the task generalization of Transformers. 

In contrast, in the second scenario, though the model has never seen tasks with both \( 4 \) \textit{and} \( 6 \) together, it has encountered individual instances where \( 4 \) appears in the second position (e.g., \( (1,4,5) \)) or where \( 6 \) appears in the third position (e.g., \( (2,3,6) \)). This exposure appears to facilitate generalization to test cases where both \( 4 \) \textit{and} \( 6 \) are present. 



\begin{table*}[t!]
    \centering
    \caption{Generalization Results for Scenarios 1 and 2 for $d=10, k=3$.}
    \resizebox{\textwidth}{!}{  % Scale to full width
        \begin{tabular}{|c|c|c|}
            \hline
            \textbf{Scenario}  & \textbf{Tasks excluded from training} & \textbf{Generalization accuracy} \\
            \hline
            Generalization with Missing Pair & $\{4,6\} \subseteq \{s_1, s_2, s_3\}$ & 96.2\%\\ 
            \hline
            Generalization with Missing Coordinate & \( s_2 = 5 \) & 45.6\% \\ 
            \hline
        \end{tabular}
    }
    \label{tab:generalization_results}
\end{table*}

As a result, when the training tasks are not i.i.d, an adversarial selection such as exclusion of specific positional configurations may lead to failure to unseen task generalization even though the size of $\mathcal{F}_{train}$ is exponentially large. 


% \paragraph{\textbf{Key Takeaways}}
% \begin{itemize}
%     \item Out-of-distribution generalization in the parity problem is highly sensitive to the diversity and positional coverage of training tasks.
%     \item Adversarial exclusion of specific pairs or positional configurations can lead to systematic failures, even when most tasks are observed during training.
% \end{itemize}




%################ previous veriosn down
% \textit{How does the choice of training tasks affect the ability of a model to generalize to unseen tasks? Can we predict which setups are likely to lead to failure?}

% To explore these questions, we crafted specific training and test task splits to investigate what makes one setup appear “more” OOD than another.

% \paragraph{Generalization with Missing Pair.}

% Imagine we have tasks constructed from subsets of \(k=3\) elements out of a larger set of \(d\) coordinates. What happens if certain pairs of coordinates are adversarially excluded during training? For example, suppose \(d=5\) and two specific coordinates, \(c_1 = 1\) and \(c_2 = 2\), are excluded. The remaining tasks are formed from subsets of the other coordinates. How would a model perform when tested on tasks involving the excluded pair \( (c_1, c_2) \)? 

% To probe this, we devised three variations in how training tasks are constructed:
%     \begin{enumerate}
%         \item \textbf{Type 1:} The training set includes all tasks except those containing both \( c_1 = 1 \) and \( c_2 = 2 \). 
%         For this example, the training set includes only $\{(3,4,5)\}$. The test set consists of all tasks containing the rest of tuples.

%         \item \textbf{Type 2:} Similar to Type 1, but the training set additionally includes half of the tasks containing either \( c_1 = 1 \) \textit{or} \( c_2 = 2 \) (but not both). 
%         For the example, the training set includes all tasks from Type 1 and adds tasks like \(\{(1, 3, 4), (2, 3, 5)\}\) (half of those containing \( c_1 = 1 \) or \( c_2 = 2 \)).

%         \item \textbf{Type 3:} Similar to Type 2, but the training set also includes half of the tasks containing both \( c_1 = 1 \) \textit{and} \( c_2 = 2 \). 
%         For the example, the training set includes all tasks from Type 2 and adds, for instance, \(\{(1, 2, 5)\}\) (half of the tasks containing both \( c_1 \) and \( c_2 \)).
%     \end{enumerate}

% By systematically increasing the diversity of training tasks in a controlled way, while ensuring no overlap between training and test configurations, we observe an improvement in OOD generalization. 

% % \textit{However, the question is this improvement similar across all coordinate pairs, or does it depend on the specific choices of \(c_1\) and \(c_2\) in the tasks?} 

% \textbf{Before proceeding, consider the following question:} Is the observed improvement consistent across all coordinate pairs, or does it depend on the specific choices of \(c_1\) and \(c_2\) in the tasks? 

% For instance, consider two cases for \(d = 10, k = 3\): (i) \(c_1 = 4, c_2 = 6\) and (ii) \(c_1 = 8, c_2 = 9\). Would you expect similar OOD generalization behavior for these two cases across the three training setups we discussed?



% \paragraph{Answer to the Question.} for both cases of \( c_1, c_2 \), we observe that generalization fails in Type 1, suggesting that the position of the tasks the model has been trained on significantly impacts its generalization capability. For Type 2, we find that \( c_1 = 4, c_2 = 6 \) performs significantly better than \( c_1 = 8, c_2 = 9 \). 

% Upon examining the tasks where the transformer fails for \( c_1 = 8, c_2 = 9 \), we see that the model only fails at tasks of the form \((*, 8, 9)\) while perfectly generalizing to the rest. This indicates that the model has never encountered the value \( 8 \) in the second position during training, which likely explains its failure to generalize. In contrast, for \( c_1 = 4, c_2 = 6 \), while the model has not seen tasks of the form \((*, 4, 6)\), it has encountered tasks where \( 4 \) appears in the second position, such as \((1, 4, 5)\), and tasks where \( 6 \) appears in the third position, such as \((2, 3, 6)\). This difference may explain why the model generalizes almost perfectly in Type 2 for \( c_1 = 4, c_2 = 6 \), but not for \( c_1 = 8, c_2 = 9 \).



% \paragraph{Generalization with Missing Coordinates.}
% Next, we investigate whether a model can generalize to tasks where a specific coordinate appears in an unseen position during training. For instance, consider \( c_1 = 5 \), and exclude all tasks where \( c_1 \) appears in the second position. Despite being trained on all other tasks, the model fails to generalize to these excluded cases, highlighting the importance of positional diversity in training tasks.



% \paragraph{Key Takeaways.}
% \begin{itemize}
%     \item OOD generalization depends heavily on the diversity and positional coverage of training tasks for the parity problem.
%     \item adversarial exclusion of specific pairs or positional configurations in the parity problem can lead to failure, even when the majority of tasks are observed during training.
% \end{itemize}


%################ previous veriosn up

% \paragraph{Key Takeaways} These findings highlight the complexity of OOD generalization, even in seemingly simple tasks like parity. They also underscore the importance of task design: the diversity of training tasks can significantly influence a model’s ability to generalize to unseen tasks. By better understanding these dynamics, we can design more robust training regimes that foster generalization across a wider range of scenarios.


% #############


% Upon examining the tasks where the transformer fails for \( c_1 = 8, c_2 = 9 \), we see that the model only fails at tasks of the form \((*, 8, 9)\) while perfectly generalizing to the rest. This indicates that the model has never encountered the value \( 8 \) in the second position during training, which likely explains its failure to generalize. In contrast, for \( c_1 = 4, c_2 = 6 \), while the model has not seen tasks of the form \((*, 4, 6)\), it has encountered tasks where \( 4 \) appears in the second position, such as \((1, 4, 5)\), and tasks where \( 6 \) appears in the third position, such as \((2, 3, 6)\). This difference may explain why the model generalizes almost perfectly in Type 2 for \( c_1 = 4, c_2 = 6 \), but not for \( c_1 = 8, c_2 = 9 \).

% we observe a striking pattern: generalization fails entirely in Type 1, regardless of the coordinate pair (\(c_1, c_2\)). However, in Type 2, generalization varies: \(c_1 = 4, c_2 = 6\) achieves 96\% accuracy, while \(c_1 = 8, c_2 = 9\) lags behind at 70\%. Why? Upon closer inspection, the model struggles specifically with tasks like \((*, 8, 9)\), where the combination \(c_1 = 8\) and \(c_2 = 9\) is entirely novel. In contrast, for \(c_1 = 4, c_2 = 6\), the model benefits from having seen tasks where \(4\) appears in the second position or \(6\) in the third. This suggests that positional exposure during training plays a key role in generalization.

% To test whether task structure influences generalization, we consider two variations:
% \begin{enumerate}
%     \item \textbf{Sorted Tuples:} Tasks are always sorted in ascending order.
%     \item \textbf{Unsorted Tuples:} Tasks can appear in any order.
% \end{enumerate}

% If the model struggles with generalizing to the excluded position, does introducing variability through unsorted tuples help mitigate this limitation?

% \paragraph{Discussion of Results}

% In \textbf{Generalization with Missing Pairs}, we observe a striking pattern: generalization fails entirely in Type 1, regardless of the coordinate pair (\(c_1, c_2\)). However, in Type 2, generalization varies: \(c_1 = 4, c_2 = 6\) achieves 96\% accuracy, while \(c_1 = 8, c_2 = 9\) lags behind at 70\%. Why? Upon closer inspection, the model struggles specifically with tasks like \((*, 8, 9)\), where the combination \(c_1 = 8\) and \(c_2 = 9\) is entirely novel. In contrast, for \(c_1 = 4, c_2 = 6\), the model benefits from having seen tasks where \(4\) appears in the second position or \(6\) in the third. This suggests that positional exposure during training plays a key role in generalization.

% In \textbf{Generalization with Missing Coordinates}, the results confirm this hypothesis. When \(c_1 = 5\) is excluded from the second position during training, the model fails to generalize to such tasks in the sorted case. However, allowing unsorted tuples introduces positional diversity, leading to near-perfect generalization. This raises an intriguing question: does the model inherently overfit to positional patterns, and can task variability help break this tendency?




% In this subsection, we show that the selection of training tasks can affect the quality of the unseen task generalization significantly in practice. To illustrate this, we present two scenarios where tasks differ between training and testing. For each scenario, we invite the reader to assess, before examining the experimental results, which cases might appear “more” OOD. 

% % \amir{add examples, }

% \kaiyue{I think the name of scenarios here are not very clear}
% \begin{itemize}
%     \item \textbf{Scenario 1:  Generalization Across Excluded Coordinate Pairs (\( k = 3 \))} \\
%     In this scenario, we select two coordinates \( c_1 \) and \( c_2 \) out of \( d \) and construct three types of training sets. 

%     Suppose \( d = 5 \), \( c_1 = 1 \), and \( c_2 = 2 \). The tuples are all possible subsets of \( \{1, 2, 3, 4, 5\} \) with \( k = 3 \):
%     \[
%     \begin{aligned}
%     \big\{ & (1, 2, 3), (1, 2, 4), (1, 2, 5), (1, 3, 4), (1, 3, 5), \\
%            & (1, 4, 5), (2, 3, 4), (2, 3, 5), (2, 4, 5), (3, 4, 5) \big\}.
%     \end{aligned}
%     \]

%     \begin{enumerate}
%         \item \textbf{Type 1:} The training set includes all tuples except those containing both \( c_1 = 1 \) and \( c_2 = 2 \). 
%         For this example, the training set includes only $\{(3,4,5)\}$ tuple. The test set consists of tuples containing the rest of tuples.

%         \item \textbf{Type 2:} Similar to Type 1, but the training set additionally includes half of the tuples containing either \( c_1 = 1 \) \textit{or} \( c_2 = 2 \) (but not both). 
%         For the example, the training set includes all tuples from Type 1 and adds tuples like \(\{(1, 3, 4), (2, 3, 5)\}\) (half of those containing \( c_1 = 1 \) or \( c_2 = 2 \)).

%         \item \textbf{Type 3:} Similar to Type 2, but the training set also includes half of the tuples containing both \( c_1 = 1 \) \textit{and} \( c_2 = 2 \). 
%         For the example, the training set includes all tuples from Type 2 and adds, for instance, \(\{(1, 2, 5)\}\) (half of the tuples containing both \( c_1 \) and \( c_2 \)).
%     \end{enumerate}

% % \begin{itemize}
% %     \item \textbf{Type 1:} The training set includes tuples \(\{1, 3, 4\}, \{2, 3, 4\}\) (excluding tuples with both \( c_1 \) and \( c_2 \): \(\{1, 2, 3\}, \{1, 2, 4\}\)). The test set contains the excluded tuples.
% %     \item \textbf{Type 2:} The training set includes all tuples in Type 1 plus half of the tuples containing either \( c_1 = 1 \) or \( c_2 = 2 \) (e.g., \(\{1, 2, 3\}\)).
% %     \item \textbf{Type 3:} The training set includes all tuples in Type 2 plus half of the tuples containing both \( c_1 = 1 \) and \( c_2 = 2 \) (e.g., \(\{1, 2, 4\}\)).
% % \end{itemize}
    
%     \item \textbf{Scenario 2: Scenario 2: Generalization Across a Fixed Coordinate (\( k = 3 \))} \\
%     In this scenario, we select one coordinate \( c_1 \) out of \( d \) (\( c_1 = 5 \)). The training set includes all task tuples except those where \( c_1 \) is the second coordinate of the tuple. For this scenario, we examine two variations:
%     \begin{enumerate}
%         \item \textbf{Sorted Tuples:} Task tuples are always sorted (e.g., \( (x_1, x_2, x_3) \) with \( x_1 \leq x_2 \leq x_3 \)).
%         \item \textbf{Unsorted Tuples:} Task tuples can appear in any order.
%     \end{enumerate}
% \end{itemize}




% \paragraph{Discussion of Results.} In the first scenario, for both cases of \( c_1, c_2 \), we observe that generalization fails in Type 1, suggesting that the position of the tasks the model has been trained on significantly impacts its generalization capability. For Type 2, we find that \( c_1 = 4, c_2 = 6 \) performs significantly better than \( c_1 = 8, c_2 = 9 \). 

% Upon examining the tasks where the transformer fails for \( c_1 = 8, c_2 = 9 \), we see that the model only fails at tasks of the form \((*, 8, 9)\) while perfectly generalizing to the rest. This indicates that the model has never encountered the value \( 8 \) in the second position during training, which likely explains its failure to generalize. In contrast, for \( c_1 = 4, c_2 = 6 \), while the model has not seen tasks of the form \((*, 4, 6)\), it has encountered tasks where \( 4 \) appears in the second position, such as \((1, 4, 5)\), and tasks where \( 6 \) appears in the third position, such as \((2, 3, 6)\). This difference may explain why the model generalizes almost perfectly in Type 2 for \( c_1 = 4, c_2 = 6 \), but not for \( c_1 = 8, c_2 = 9 \).

% This position-based explanation appears compelling, so in the second scenario, we focus on a single position to investigate further. Here, we find that the transformer fails to generalize to tasks where \( 5 \) appears in the second position, provided it has never seen any such tasks during training. However, when we allow for more task diversity in the unsorted case, the model achieves near-perfect generalization. 

% This raises an important question: does the transformer have a tendency to overfit to positional patterns, and does introducing more task variability, as in the unsorted case, prevent this overfitting and enable generalization to unseen positional configurations?

% These findings highlight that even in a simple task like parity, it is remarkably challenging to understand and quantify the sources and levels of OOD behavior. This motivates further investigation into the nuances of task design and its impact on model generalization.


\subsection{Task Generalization Beyond Parity Problems}

% \begin{figure}[t!]
%     \centering
%     \includegraphics[width=0.45\textwidth]{Figures/arithmetic_v1.png}
%     \vspace{-0.3cm}
%     \caption{Task generalization for arithmetic task with CoT, it has $\d =2$ and $T = d-1$ as the ambient dimension, hence $D\ln(DT) = 2\ln(2T)$. We show that the empirical scaling closely follows the theoretical scaling.}
%     \label{fig:arithmetic}
% \end{figure}



% \begin{wrapfigure}{r}{0.4\textwidth}  % 'r' for right, 'l' for left
%     \centering
%     \includegraphics[width=0.4\textwidth]{Figures/arithmetic_v1.png}
%     \vspace{-0.3cm}
%     \caption{Task generalization for the arithmetic task with CoT. It has $d =2$ and $T = d-1$ as the ambient dimension, hence $D\ln(DT) = 2\ln(2T)$. We show that the empirical scaling closely follows the theoretical scaling.}
%     \label{fig:arithmetic}
% \end{wrapfigure}

\subsubsection{Arithmetic Task}\label{subsec:arithmetic}











We introduce the family of \textit{Arithmetic} task that, like the sparse parity problem, operates on 
\( d \) binary inputs \( b_1, b_2, \dots, b_d \). The task involves computing a structured arithmetic expression over these inputs using a sequence of addition and multiplication operations.
\newcommand{\op}{\textrm{op}}

Formally, we define the function:
\[
\text{Arithmetic}_{S} \colon \{0,1\}^d \to \{0,1,\dots,d\},
\]
where \( S = (\op_1, \op_2, \dots, \op_{d-1}) \) is a sequence of \( d-1 \) operations, each \( \op_k \) chosen from \( \{+, \times\} \). The function evaluates the expression by applying the operations sequentially from left-to-right order: for example, if \( S = (+, \times, +) \), then the arithmetic function would compute:
\[
\text{Arithmetic}_{S}(b_1, b_2, b_3, b_4) = ((b_1 + b_2) \times b_3) + b_4.
\]
% Thus, the sequence of operations \( S \) defines how the binary inputs are combined to produce an integer output between \( 0 \) and \( d \).
% \[
% \text{Arithmetic}_{S} 
% (b_1,\,b_2,\,\dots,b_d)
% =
% \Bigl(\dots\bigl(\,(b_1 \;\op_1\; b_2)\;\op_2\; b_3\bigr)\,\dots\Bigr) 
% \;\op_{d-1}\; b_d.
% \]
% We now introduce an \emph{Arithmetic} task that, like the sparse parity problem, operates on $d$ binary inputs $b_1, b_2, \dots, b_d$. Specifically, we define an arithmetic function
% \[
% \text{Arithmetic}_{S}\colon \{0,1\}^d \;\to\; \{0,1,\dots,d\},
% \]
% where $S = (i_1, i_2, \dots, i_{d-1})$ is a sequence of $d-1$ operations, each $i_k \in \{+,\,\times\}$. The value of $\text{Arithmetic}_{S}$ is obtained by applying the prescribed addition and multiplication operations in order, namely:
% \[
% \text{Arithmetic}_{S}(b_1,\,b_2,\,\dots,b_d)
% \;=\;
% \Bigl(\dots\bigl(\,(b_1 \;i_1\; b_2)\;i_2\; b_3\bigr)\,\dots\Bigr) 
% \;i_{d-1}\; b_d.
% \]

% This is an example of our framework where $T = d-1$ and $|\Theta_t| = 2$ with total $2^d$ possible tasks. 




By introducing a step-by-step CoT, arithmetic class belongs to $ARC(2, d-1)$: this is because at every step, there is only $\d = |\Theta_t| = 2$ choices (either $+$ or $\times$) while the length is  $T = d-1$, resulting a total number of $2^{d-1}$ tasks. 


\begin{minipage}{0.5\textwidth}  % Left: Text
    Task generalization for the arithmetic task with CoT. It has $d =2$ and $T = d-1$ as the ambient dimension, hence $D\ln(DT) = 2\ln(2T)$. We show that the empirical scaling closely follows the theoretical scaling.
\end{minipage}
\hfill
\begin{minipage}{0.4\textwidth}  % Right: Image
    \centering
    \includegraphics[width=\textwidth]{Figures/arithmetic_v1.png}
    \refstepcounter{figure}  % Manually advances the figure counter
    \label{fig:arithmetic}  % Now this label correctly refers to the figure
\end{minipage}

Notably, when scaling with \( T \), we observe in the figure above that the task scaling closely follow the theoretical $O(D\log(DT))$ dependency. Given that the function class grows exponentially as \( 2^T \), it is truly remarkable that training on only a few hundred tasks enables generalization to an exponentially larger space—on the order of \( 2^{25} > 33 \) Million tasks. This exponential scaling highlights the efficiency of structured learning, where a modest number of training examples can yield vast generalization capability.





% Our theory suggests that only $\Tilde{O}(\ln(T))$ i.i.d training tasks is enough to generalize to the rest of unseen tasks. However, we show in Figure \ref{fig:arithmetic} that transformer is not able to match  that. The transformer out-of distribution generalization behavior is not consistent across different dimensions when we scale the number of training tasks with $\ln(T)$. \hongzhou{implicit bias, optimization, etc}
 






% \subsection{Task generalization Beyond parity problem}

% \subsection{Arithmetic} In this setting, we still use the set-up we introduced in \ref{subsec:parity_exmaple}, the input is still a set of $d$ binary variable, $b_1, b_2,\dots,b_d$ and ${Arithmatic_{S}}:\{0,1\}\rightarrow \{0, 1, \dots, d\}$, where $S = (i_1,i_2,\dots,i_{d-1})$ is a tuple of size $d-1$ where each coordinate is either add($+
% $) or multiplication ($\times$). The function is as following,

% \begin{align*}
%     Arithmatic_{S}(b_1, b_2,\dots,b_d) = (\dots(b1(i1)b2)(i3)b3\dots)(i{d-1})
% \end{align*}
    


\subsubsection{Multi-Step Language Translation Task}

 \begin{figure*}[h!]
    \centering
    \includegraphics[width=0.9\textwidth]{Figures/combined_plot_horiz.png}
    \vspace{-0.2cm}
    \caption{Task generalization for language translation task: $\d$ is the number of languages and $T$ is the length of steps.}
    \vspace{-2mm}
    \label{fig:language}
\end{figure*}
% \vspace{-2mm}

In this task, we study a sequential translation process across multiple languages~\cite{garg2022can}. Given a set of \( D \) languages, we construct a translation chain by randomly sampling a sequence of \( T \) languages \textbf{with replacement}:  \(L_1, L_2, \dots, L_T,\)
where each \( L_t \) is a sampled language. Starting with a word, we iteratively translate it through the sequence:
\vspace{-2mm}
\[
L_1 \to L_2 \to L_3 \to \dots \to L_T.
\]
For example, if the sampled sequence is EN → FR → DE → FR, translating the word "butterfly" follows:
\vspace{-1mm}
\[
\text{butterfly} \to \text{papillon} \to \text{schmetterling} \to \text{papillon}.
\]
This task follows an \textit{AutoRegressive Compositional} structure by itself, specifically \( ARC(D, T-1) \), where at each step, the conditional generation only depends on the target language, making \( D \) as the number of languages and the total number of possible tasks is \( D^{T-1} \). This example illustrates that autoregressive compositional structures naturally arise in real-world languages, even without explicit CoT. 

We examine task scaling along \( D \) (number of languages) and \( T \) (sequence length). As shown in Figure~\ref{fig:language}, empirical  \( D \)-scaling closely follows the theoretical \( O(D \ln D T) \). However, in the \( T \)-scaling case, we observe a linear dependency on \( T \) rather than the logarithmic dependency \(O(\ln T) \). A possible explanation is error accumulation across sequential steps—longer sequences require higher precision in intermediate steps to maintain accuracy. This contrasts with our theoretical analysis, which focuses on asymptotic scaling and does not explicitly account for compounding errors in finite-sample settings.

% We examine task scaling along \( D \) (number of languages) and \( T \) (sequence length). As shown in Figure~\ref{fig:language}, empirical scaling closely follows the theoretical \( O(D \ln D T) \) trend, with slight exceptions at $ T=10 \text{ and } 3$ in Panel B. One possible explanation for this deviation could be error accumulation across sequential steps—longer sequences require each intermediate translation to be approximated with higher precision to maintain test accuracy. This contrasts with our theoretical analysis, which primarily focuses on asymptotic scaling and does not explicitly account for compounding errors in finite-sample settings.

Despite this, the task scaling is still remarkable — training on a few hundred tasks enables generalization to   $4^{10} \approx 10^6$ tasks!






% , this case, we are in a regime where \( D \ll T \), we observe  that the task complexity empirically scales as \( T \log T \) rather than \( D \log T \). 


% the model generalizes to an exponentially larger space of \( 2^T \) unseen tasks. In case $T=25$, this is $2^{25} > 33$ Million tasks. This remarkable exponential generalization demonstrates the power of structured task composition in enabling efficient generalization.


% In the case of parity tasks, introducing CoT effectively decomposes the problem from \( ARC(D^T, 1) \) to \( ARC(D, T) \), significantly improving task generalization.

% Again, in the regime scaling $T$, we again observe a $T\log T$ dependency. Knowing that the function class is scaling as $D^T$, it is remarkable that training on a few hundreds tasks can generalize to $4^{10} \approx 1M$ tasks. 





% We further performed a preliminary investigation on a semi-synthetic word-level translation task to show that (1) task generalization via composition structure is feasible beyond parity and (2) understanding the fine-grained mechanism leading to this generalization is still challenging. 
% \noindent
% \noindent
% \begin{minipage}[t]{\columnwidth}
%     \centering
%     \textbf{\scriptsize In-context examples:}
%     \[
%     \begin{array}{rl}
%         \textbf{Input} & \hspace{1.5em} \textbf{Output} \\
%         \hline
%         \textcolor{blue}{car}   & \hspace{1.5em} \textcolor{red}{voiture \;,\; coche} \\
%         \textcolor{blue}{house} & \hspace{1.5em} \textcolor{red}{maison \;,\; casa} \\
%         \textcolor{blue}{dog}   & \hspace{1.5em} \textcolor{red}{chien \;,\; perro} 
%     \end{array}
%     \]
%     \textbf{\scriptsize Query:}
%     \[
%     \begin{array}{rl}
%         \textbf{Input} & \textbf{Output} \\
%         \hline
%         \textcolor{blue}{cat} & \hspace{1.5em} \textcolor{red}{?} \\
%     \end{array}
%     \]
% \end{minipage}



% \begin{figure}[h!]
%     \centering
%     \includegraphics[width=0.45\textwidth]{Figures/translation_scale_d.png}
%     \vspace{-0.2cm}
%     \caption{Task generalization behavior for word translation task.}
%     \label{fig:arithmetic}
% \end{figure}


\vspace{-1mm}
\section{Conclusions}
% \misha{is it conclusion of the section or of the whole paper?}    
% \amir{The whole paper. It is very short, do we need a separate section?}
% \misha{it should not be a subsection if it is the conclusion the whole thing. We can just remove it , it does not look informative} \hz{let's do it in a whole section, just to conclude and end the paper, even though it is not informative}
%     \kaiyue{Proposal: Talk about the implication of this result on theory development. For example, it calls for more fine-grained theoretical study in this space.  }

% \huaqing{Please feel free to edit it if you have better wording or suggestions.}

% In this work, we propose a theoretical framework to quantitatively investigate task generalization with compositional autoregressive tasks. We show that task to $D^T$ task is theoretically achievable by training on only $O (D\log DT)$ tasks, and empirically observe that transformers trained on parity problem indeed achieves such task generalization. However, for other tasks beyond parity, transformers seem to fail to achieve this bond. This calls for more fine-grained theoretical study the phenomenon of task generalization specific to transformer model. It may also be interesting to study task generalization beyond the setting of in-context learning. 
% \misha{what does this add?} \amir{It does not, i dont have any particular opinion to keep it. @Hongzhou if you want to add here?}\hz{While it may not introduce anything new, we are following a good practice to have a short conclusion. It provides a clear closing statement, reinforces key takeaways, and helps the reader leave with a well-framed understanding of our contributions. }
% In this work, we quantitatively investigate task generalization under autoregressive compositional structure. We demonstrate that task generalization to $D^T$ tasks is theoretically achievable by training on only $\tilde O(D)$ tasks. Empirically, we observe that transformers trained indeed achieve such exponential task generalization on problems such as parity, arithmetic and multi-step language translation. We believe our analysis opens up a new angle to understand the remarkable generalization ability of Transformer in practice. 

% However, for tasks beyond the parity problem, transformers appear to fail to reach this bound. This highlights the need for a more fine-grained theoretical exploration of task generalization, especially for transformer models. Additionally, it may be valuable to investigate task generalization beyond the scope of in-context learning.



In this work, we quantitatively investigated task generalization under the autoregressive compositional structure, demonstrating both theoretically and empirically that exponential task generalization to $D^T$ tasks can be achieved with training on only $\tilde{O}(D)$ tasks. %Our theoretical results establish a fundamental scaling law for task generalization, while our experiments validate these insights across problems such as parity, arithmetic, and multi-step language translation. The remarkable ability of transformers to generalize exponentially highlights the power of structured learning and provides a new perspective on how large language models extend their capabilities beyond seen tasks. 
We recap our key contributions  as follows:
\begin{itemize}
    \item \textbf{Theoretical Framework for Task Generalization.} We introduced the \emph{AutoRegressive Compositional} (ARC) framework to model structured task learning, demonstrating that a model trained on only $\tilde{O}(D)$ tasks can generalize to an exponentially large space of $D^T$ tasks.
    
    \item \textbf{Formal Sample Complexity Bound.} We established a fundamental scaling law that quantifies the number of tasks required for generalization, proving that exponential generalization is theoretically achievable with only a logarithmic increase in training samples.
    
    \item \textbf{Empirical Validation on Parity Functions.} We showed that Transformers struggle with standard in-context learning (ICL) on parity tasks but achieve exponential generalization when Chain-of-Thought (CoT) reasoning is introduced. Our results provide the first empirical demonstration of structured learning enabling efficient generalization in this setting.
    
    \item \textbf{Scaling Laws in Arithmetic and Language Translation.} Extending beyond parity functions, we demonstrated that the same compositional principles hold for arithmetic operations and multi-step language translation, confirming that structured learning significantly reduces the task complexity required for generalization.
    
    \item \textbf{Impact of Training Task Selection.} We analyzed how different task selection strategies affect generalization, showing that adversarially chosen training tasks can hinder generalization, while diverse training distributions promote robust learning across unseen tasks.
\end{itemize}



We introduce a framework for studying the role of compositionality in learning tasks and how this structure can significantly enhance generalization to unseen tasks. Additionally, we provide empirical evidence on learning tasks, such as the parity problem, demonstrating that transformers follow the scaling behavior predicted by our compositionality-based theory. Future research will  explore how these principles extend to real-world applications such as program synthesis, mathematical reasoning, and decision-making tasks. 


By establishing a principled framework for task generalization, our work advances the understanding of how models can learn efficiently beyond supervised training and adapt to new task distributions. We hope these insights will inspire further research into the mechanisms underlying task generalization and compositional generalization.

\section*{Acknowledgements}
We acknowledge support from the National Science Foundation (NSF) and the Simons Foundation for the Collaboration on the Theoretical Foundations of Deep Learning through awards DMS-2031883 and \#814639 as well as the  TILOS institute (NSF CCF-2112665) and the Office of Naval Research (ONR N000142412631). 
This work used the programs (1) XSEDE (Extreme science and engineering discovery environment)  which is supported by NSF grant numbers ACI-1548562, and (2) ACCESS (Advanced cyberinfrastructure coordination ecosystem: services \& support) which is supported by NSF grants numbers \#2138259, \#2138286, \#2138307, \#2137603, and \#2138296. Specifically, we used the resources from SDSC Expanse GPU compute nodes, and NCSA Delta system, via allocations TG-CIS220009. 

% \section{Machine Learning Prediction}\label{sec:ml}
To evaluate the feasibility of machine learning in predicting the bounds of the PNS, we employed five distinct machine learning models to assess their effectiveness in this task: Support Vector Machine (SVM) \citep{svm}, Random Forest (RF) \citep{rf}, Gradient Boosting Decision Trees (GBDT) \citep{gbdt}, Transformer \citep{vaswani2017attention}, and Multilayer Perceptron (MLP) \citep{mlp}. These models were chosen to represent a diverse range of machine learning paradigms, including kernel-based methods (SVM), ensemble learning techniques (RF and GBDT), and deep learning approaches (MLP and Transformer). This selection ensures a comprehensive evaluation of their ability to approximate causal quantities across different settings. A detailed pipeline is illustrated in Figure \ref{fig:flow chart}.

\begin{figure*}[!htbp] 
    \centering
    \includegraphics[width=\textwidth]{imgs/flow_chart.pdf}
    \caption{Framework for Causal Data Generation and Machine Learning Prediction.}
    \label{fig:flow chart}
\end{figure*}

\begin{figure*}[!htb]
    \centering
    % 第一行:SVM, RF, GBDT, Transformer Lower Bound
    \begin{subfigure}[b]{0.24\linewidth}
        \centering
        \includegraphics[width=\linewidth]{imgs/svm_true_vs_pred_lb.pdf}
        \caption{SVM (Lower bound)}
        \label{fig:svm2}
    \end{subfigure}
    \hfill
    \begin{subfigure}[b]{0.24\linewidth}
        \centering
        \includegraphics[width=\linewidth]{imgs/rf_true_vs_pred_lb.pdf}
        \caption{RF (Lower bound)}
        \label{fig:rf2}
    \end{subfigure}
    \hfill
    \begin{subfigure}[b]{0.24\linewidth}
        \centering
        \includegraphics[width=\linewidth]{imgs/gbdt_true_vs_pred_lb.pdf}
        \caption{GBDT (Lower bound)}
        \label{fig:gbdt2}
    \end{subfigure}
    \hfill
    \begin{subfigure}[b]{0.24\linewidth}
        \centering
        \includegraphics[width=\linewidth]{imgs/transformer_true_vs_pred_lb.pdf}
        \caption{Transformer (Lower bound)}
        \label{fig:transformer2}
    \end{subfigure}

    \vspace{0.3cm} % 调整上下间距

    % 第二行:SVM, RF, GBDT, Transformer Upper Bound
    \begin{subfigure}[b]{0.24\linewidth}
        \centering
        \includegraphics[width=\linewidth]{imgs/svm_true_vs_pred_ub.pdf}
        \caption{SVM (Upper bound)}
        \label{fig:svm3}
    \end{subfigure}
    \hfill
    \begin{subfigure}[b]{0.24\linewidth}
        \centering
        \includegraphics[width=\linewidth]{imgs/rf_true_vs_pred_ub.pdf}
        \caption{RF (Upper bound)}
        \label{fig:rf3}
    \end{subfigure}
    \hfill
    \begin{subfigure}[b]{0.24\linewidth}
        \centering
        \includegraphics[width=\linewidth]{imgs/gbdt_true_vs_pred_ub.pdf}
        \caption{GBDT (Upper bound)}
        \label{fig:gbdt3}
    \end{subfigure}
    \hfill
    \begin{subfigure}[b]{0.24\linewidth}
        \centering
        \includegraphics[width=\linewidth]{imgs/transformer_true_vs_pred_ub.pdf}
        \caption{Transformer (Upper bound)}
        \label{fig:transformer3}
    \end{subfigure}

    \caption{Comparison of true and predicted values across different models for both lower and upper bounds.}
    \label{fig:combined_model_comparison}
\end{figure*}
\subsection{Support Vector Machine}
Support Vector Machines (SVM) \citep{svm} are widely used and well-established supervised learning models. Given their strengths, we selected Support Vector Regression (SVR), a variant of SVM, as the first model for our experiments. To effectively capture complex patterns, we employed the Radial Basis Function (RBF) kernel to map the data into a high-dimensional feature space.

Key hyperparameters include the penalty parameter (\( C \)), the insensitive loss threshold (\( \epsilon \)), and the kernel coefficient (\( \gamma \)). The parameter \( C \) controls the trade-off between model complexity and error tolerance, where larger values may lead to overfitting. The threshold \( \epsilon \) defines the margin of tolerance for errors, while \( \gamma \) determines the influence range of individual data points.

A two-stage hyperparameter tuning strategy was adopted. First, Randomized Search \citep{randomsearch} was employed to efficiently explore the parameter space and identify promising ranges. Then, Grid Search \citep{randomsearch} was used to fine-tune parameters within these ranges. Cross-validation ensured robust generalization throughout the tuning process.

\begin{figure*}[!htb]
    \centering
    % 第一行:SVM, RF, GBDT Lower Bounds
    \begin{subfigure}[b]{0.32\linewidth}
        \centering
        \includegraphics[width=\linewidth]{imgs/confusion_matrix_svm_lb.pdf}
        \caption{SVM Lower bound.}
        \label{fig:svm_lb} % 放在 caption 后
    \end{subfigure}
    \hfill
    \begin{subfigure}[b]{0.32\linewidth}
        \centering
        \includegraphics[width=\linewidth]{imgs/confusion_matrix_rf_lb.pdf}
        \caption{RF Lower bound.}
        \label{fig:rf_lb}
    \end{subfigure}
    \hfill
    \begin{subfigure}[b]{0.32\linewidth}
        \centering
        \includegraphics[width=\linewidth]{imgs/confusion_matrix_gbdt_lb.pdf}
        \caption{GBDT Lower bound.}
        \label{fig:gbdt_lb}
    \end{subfigure}

    % 第二行:SVM, RF, GBDT Upper Bounds
    \begin{subfigure}[b]{0.32\linewidth}
        \centering
        \includegraphics[width=\linewidth]{imgs/confusion_matrix_svm_ub.pdf}
        \caption{SVM Upper bound.}
        \label{fig:svm_ub}
    \end{subfigure}
    \hfill
    \begin{subfigure}[b]{0.32\linewidth}
        \centering
        \includegraphics[width=\linewidth]{imgs/confusion_matrix_rf_ub.pdf}
        \caption{RF Upper bound.}
        \label{fig:rf_ub}
    \end{subfigure}
    \hfill
    \begin{subfigure}[b]{0.32\linewidth}
        \centering
        \includegraphics[width=\linewidth]{imgs/confusion_matrix_gbdt_ub.pdf}
        \caption{GBDT Upper bound.}
        \label{fig:gbdt_ub}
    \end{subfigure}

    % 第三行:Transformer
    \begin{subfigure}[b]{0.32\linewidth}
        \centering
        \includegraphics[width=\linewidth]{imgs/confusion_matrix_transformer_lb.pdf}
        \caption{Transformer Lower bound.}
        \label{fig:transformer_lb}
    \end{subfigure}
    \begin{subfigure}[b]{0.32\linewidth}
        \centering
        \includegraphics[width=\linewidth]{imgs/confusion_matrix_transformer_ub.pdf}
        \caption{Transformer Upper bound.}
        \label{fig:transformer_ub}
    \end{subfigure}

    \caption{Confusion matrices of SVM, RF, GBDT, and Transformer models.}
    \label{fig:confusion_matrices_combined}
\end{figure*}

% \begin{table}[!h]
%     \centering
%     \caption{SVM Result} \label{tab:svm}
%     \begin{tabular}{rll}
%         \toprule % from booktabs package
%         \bfseries Dataset & \bfseries  MSE & \bfseries MAE\\
%         \midrule % from booktabs package
%         Lower bound & 0.0112 & 0.0868\\
%         Upper bound & 0.0304 & 0.1527\\
%         \bottomrule % from booktabs package
%     \end{tabular}
% \end{table}
Finally, the mean squared error (MSE) and mean absolute error (MAE) values of the SVR model can be found in Table \ref{tab:comparison}. Confusion matrices are presented in Figures \ref{fig:svm_lb} and \ref{fig:svm_ub}, while Figures \ref{fig:svm2} and \ref{fig:svm3} provide a clearer comparison with the true PNS bounds. For the prediction of the lower bound, SVR demonstrates reasonable effectiveness; however, for the more complex upper bound, it exhibits a significant decline in accuracy.

\subsection{Random Forest}
Random Forests (RF) \citep{rf} are a widely used ensemble learning method for classification, regression, and other predictive tasks. The core idea behind RF is to construct multiple decision trees during training and aggregate their outputs to enhance overall performance. As an ensemble model, RF exhibits strong robustness, motivating us to assess its effectiveness in predicting PNS bounds.

Key hyperparameters of RF include the number of trees (\( n_{\text{estimators}} \)), maximum tree depth (\( \text{max\_depth} \)), minimum samples required to split a node (\( \text{min\_samples\_split} \)), and the number of features considered for splitting (\( \text{max\_features} \)). Increasing \( n_{\text{estimators}} \) generally improves performance but at the expense of higher computational costs. The parameters \( \text{max\_depth} \), \( \text{min\_samples\_split} \), and \( \text{max\_features} \) regulate tree complexity, balancing bias-variance trade-offs.

For hyperparameter optimization, we employed a two-stage tuning strategy similar to that used for SVM. Table \ref{tab:comparison} also presents RF's MAE and MSE results, while Figures \ref{fig:rf_lb} and \ref{fig:rf_ub} show its confusion matrices. A more direct comparison with true PNS bounds is provided in Figures \ref{fig:rf2} and \ref{fig:rf3}. RF performs comparably to SVM on the lower bound but exhibits significantly higher accuracy on the upper bound.
% \begin{table}[!h]
%     \centering
%     \caption{RF Result} \label{tab:rf}
%     \begin{tabular}{rll}
%         \toprule % from booktabs package
%         \bfseries Dataset & \bfseries  MSE & \bfseries MAE\\
%         \midrule % from booktabs package
%         Lower bound & 0.0116 & 0.0919\\
%         Upper bound & 0.0205 & 0.1242\\
%         \bottomrule % from booktabs package
%     \end{tabular}
% \end{table}
\subsection{Gradient Boosting Decision Trees}
Gradient Boosting Decision Trees (GBDT) \citep{gbdt} is an ensemble learning method that builds models sequentially, with each new tree correcting the errors of its predecessors. Unlike traditional boosting, GBDT optimizes pseudo-residuals, enabling flexible loss function optimization. Simple decision trees serve as weak learners, allowing GBDT to effectively capture complex data patterns.

Key hyperparameters include the number of trees (\( n_{\text{estimators}} \)), learning rate (\( \text{learning\_rate} \)), maximum tree depth (\( \text{max\_depth} \)), and subsample ratio (\( \text{subsample} \)). The learning rate determines each tree’s contribution, while \( n_{\text{estimators}} \) and \( \text{max\_depth} \) regulate model complexity and performance.

Following the approach used for SVM and RF, we applied a two-stage tuning strategy. Again, table \ref{tab:comparison} presents the MSE and MAE results, while Figures \ref{fig:gbdt_lb} and \ref{fig:gbdt_ub} show the confusion matrices. A more direct comparison with true PNS bounds is provided in Figures \ref{fig:gbdt2} and \ref{fig:gbdt3}. GBDT demonstrates moderate performance on both the lower and upper bounds.
% \begin{table}[!h]
%     \centering
%     \caption{GBDT Result} \label{tab:gbdt}
%     \begin{tabular}{rll}
%         \toprule
%         \bfseries Dataset & \bfseries MSE & \bfseries MAE \\
%         \midrule
%         Lower bound & 0.0159 & 0.1049 \\
%         Upper bound & 0.0261 & 0.1399 \\
%         \bottomrule
%     \end{tabular}
% \end{table}
\subsection{Transformer}
The Transformer \citep{vaswani2017attention}, originally developed for Natural Language Processing, has expanded into Computer Vision and become a cornerstone of deep learning, particularly with the rise of large language models. Given its significant impact, this study also evaluates the Transformer for testing.

The model architecture begins with an input layer processing 15-dimensional feature vectors, followed by a linear embedding layer that projects inputs into a 64-dimensional space. Positional encoding is applied to retain feature order information, and two Transformer encoder layers with four attention heads each capture complex feature interactions. The final output is generated through a fully connected layer with a Sigmoid activation function, ensuring predictions remain within the range \([0, 1]\). Key hyperparameters include an embedding dimension of 64, four attention heads, two encoder layers, and a dropout rate of 0.1.

Similarly, table \ref{tab:comparison} presents the MSE and MAE results, while Figures \ref{fig:transformer_lb} and \ref{fig:transformer_ub} show the confusion matrices. A direct comparison with true PNS bounds is provided in Figures \ref{fig:transformer2} and \ref{fig:transformer3}. The Transformer demonstrates strong performance on the lower bound and moderate performance on the upper bound.
% \begin{table}[!h]
%     \centering
%     \caption{Transformer Result} \label{tab:transformer}
%     \begin{tabular}{rll}
%         \toprule
%         \bfseries Dataset & \bfseries MSE & \bfseries MAE \\
%         \midrule
%         Lower bound & 0.0030 & 0.0348 \\
%         Upper bound & 0.0156 & 0.1060 \\
%         \bottomrule
%     \end{tabular}
% \end{table}
\subsection{Multilayer Perceptron}
MLP \citep{mlp} consists of an input layer, one or more hidden layers, and an output layer. With appropriate activation functions, it can effectively model both linear and nonlinear relationships. As a fundamental structure in deep learning, MLP holds significant representativeness, motivating its inclusion in our experiments.

A key consideration for MLP is the choice of activation function, particularly for predicting the lower bound. Since the lower bound of PNS cannot be negative, we initially selected the ReLU \citep{relu} activation function (\ref{equ:relu}). However, ReLU can lead to the loss of negatively correlated features, prompting us to adopt LeakyReLU \citep{lkrelu} (\ref{equ:lkrelu}) as a complementary solution. Furthermore, given the considerable number of zero values in the data, the non-differentiability of ReLU and LeakyReLU at \(s = 0\) imposes limitations on backpropagation. To address this, we proposed using Mish \citep{mish} (\ref{equ:mish}) as an alternative activation function. The corresponding equations are:

\begin{equation}\label{equ:relu}
    \text{ReLU}(s) = \max(0, s)
\end{equation}

\begin{equation}\label{equ:lkrelu}
    \text{LeakyReLU}(s) = 
    \begin{cases}
        s, & \text{if } s \ge 0 \\
        \alpha s, & \text{if } s < 0
    \end{cases}
\end{equation}

\begin{equation}\label{equ:mish}
    \text{Mish}(s) = s \cdot \tanh(\ln(1 + e^s))
\end{equation}

Additionally, we implemented an MLP with the architecture \( 15 \rightarrow 64 \rightarrow 32 \rightarrow 16 \rightarrow 1 \), utilizing ReLU-like functions and Sigmoid as activation functions. The model was optimized using the Adam optimizer with a learning rate of $0.01$ and trained for $1000$ epochs. 

Again, the final results are presented in Table \ref{tab:comparison}. With the Mish activation function, the MLP achieved an MSE of \textbf{0.0011} on the lower bound and \textbf{0.0010} on the upper bound. For MAE, it attained \textbf{0.0225} on the lower bound and \textbf{0.0247} on the upper bound. The confusion matrix is shown in Figure \ref{fig:mlp_comparison}, and a clearer comparison with the true PNS bounds is provided in Figure \ref{fig:mlp2} (Only the best performance comparisons with Mish are shown). 

Overall, MLP significantly outperformed other machine learning models, with Mish yielding the best results among the activation functions. The comparison with the true PNS bounds further confirms that MLP (Mish) provides an accurate and practical model for predicting PNS.
% \begin{table}[!h]
%     \centering
%     \caption{MLP Result} \label{tab:mlp}
%     \begin{tabular}{rlll}
%         \toprule
%         \bfseries Dataset & \bfseries Activation Function & \bfseries MSE & \bfseries MAE \\
%         \midrule
%         Lower bound &ReLU & 0.0045 & 0.0434 \\
%         Upper bound &ReLU & 0.0023 & 0.0357 \\
%         Lower bound &LeakyReLU & 0.0038 & 0.0379 \\
%         Upper bound &LeakyReLU & 0.0024 & 0.0380 \\
%         Lower bound &Mish & \textbf{0.0011} & \textbf{0.0225} \\
%         Upper bound &Mish & \textbf{0.0010} & \textbf{0.0247} \\
%         \bottomrule
%     \end{tabular}
% \end{table}
%%%%%%%%%%%%%%%%%%%%%%%%%%%%
\begin{figure*}[!htb]  % 使用 figure* 跨越两列
    \centering
    % 第一排:ReLU, Leaky ReLU, Mish 的 Lower Bound
    \begin{subfigure}[b]{0.32\linewidth}
        \centering
        \includegraphics[width=\linewidth]{imgs/confusion_matrix_nn_relu_lb.pdf}
        \caption{ReLU (Lower bound)}
    \end{subfigure}
    \hfill
    \begin{subfigure}[b]{0.32\linewidth}
        \centering
        \includegraphics[width=\linewidth]{imgs/confusion_matrix_nn_lkrelu_lb.pdf}
        \caption{Leaky ReLU (Lower bound)}
    \end{subfigure}
    \hfill
    \begin{subfigure}[b]{0.32\linewidth}
        \centering
        \includegraphics[width=\linewidth]{imgs/confusion_matrix_nn_mish_lb.pdf}
        \caption{Mish (Lower bound)}
    \end{subfigure}

    \vspace{0.3cm} % 调整上下间距

    % 第二排:ReLU, Leaky ReLU, Mish 的 Upper Bound
    \begin{subfigure}[b]{0.32\linewidth}
        \centering
        \includegraphics[width=\linewidth]{imgs/confusion_matrix_nn_relu_ub.pdf}
        \caption{ReLU (Upper bound)}
    \end{subfigure}
    \hfill
    \begin{subfigure}[b]{0.32\linewidth}
        \centering
        \includegraphics[width=\linewidth]{imgs/confusion_matrix_nn_lkrelu_ub.pdf}
        \caption{Leaky ReLU (Upper bound)}
    \end{subfigure}
    \hfill
    \begin{subfigure}[b]{0.32\linewidth}
        \centering
        \includegraphics[width=\linewidth]{imgs/confusion_matrix_nn_mish_ub.pdf}
        \caption{Mish (Upper bound)}
    \end{subfigure}

    \caption{Confusion matrices of MLP with different activation functions: ReLU, Leaky ReLU, and Mish for both lower and upper bounds.}
    \label{fig:mlp_comparison}
\end{figure*}

%%%%%%%%%%%%%%%%%%%%%%%%%%%%


\begin{figure}[!htb]
    \centering
    % 左侧:Lower bound
    \begin{subfigure}[b]{0.48\linewidth}
        \centering
        \includegraphics[width=\linewidth]{imgs/mlp_true_vs_pred_lb.pdf}
        \caption{Lower bound.}
        \label{fig:mlp_lb}
    \end{subfigure}
    \hfill
    % 右侧:Upper bound
    \begin{subfigure}[b]{0.48\linewidth}
        \centering
        \includegraphics[width=\linewidth]{imgs/mlp_true_vs_pred_ub.pdf}
        \caption{Upper bound.}
        \label{fig:mlp_ub}
    \end{subfigure}
    
    \caption{Comparison of MLP (Mish) for lower and upper bounds.}
    \label{fig:mlp2}
\end{figure}





\subsection{Experimental Comparison}
% As shown in Table \ref{tab:comparison}, the overall performance of MLP is undoubtedly the best. Among the other four machine learning models, SVM achieves good results on the lower bound but performs the worst on the upper bound, with performance close to complete failure. RF performs significantly better, achieving acceptable results on both the lower and upper bounds. In contrast, GBDT performs worse than RF, despite also being a tree-based model, showing inferior results across the board with only a slight improvement over SVM on the upper bound. The Transformer, as a model based on MLP, performs better than the other machine learning models but still falls short of MLP's performance.

% For MLP models, due to the special characteristics of the dataset around zero, different activation functions exhibit significant performance differences. The basic ReLU exhibits suboptimal performance on the lower bound. Considering negative values, LeakyReLU shows slightly better performance than ReLU on the lower bound. Mish, which not only accounts for negative values but also ensures differentiability around zero, achieves the best results.
As shown in Table \ref{tab:comparison}, MLP delivers the best overall performance. Among the other four machine learning models, SVM performs well on the lower bound but fails almost entirely on the upper bound. RF shows significantly better results, achieving acceptable performance on both bounds. Despite also being a tree-based model, GBDT underperforms compared to RF, with only a slight improvement over SVM on the upper bound. The Transformer, as an MLP-based model, outperforms the other machine learning models but still falls short of MLP’s performance.  

For MLP models, the dataset’s characteristics around zero (we will discuss these characteristics in the discussion section) lead to notable differences in activation function performance. Basic ReLU shows suboptimal performance on the lower bound, while LeakyReLU, which accounts for negative values, performs slightly better. Mish, which not only handles negative values but also ensures differentiability around zero, achieves the best results.

\begin{table}[!ht]
    \centering
-    \caption{Comparison of Model Performance} 
    \label{tab:comparison}
    \begin{tabular}{rlll}
        \toprule
        \bfseries Model & \bfseries Dataset & \bfseries MSE & \bfseries MAE \\
        \midrule
        SVM  & Lower bound & 0.0112 & 0.0868 \\
             & Upper bound & 0.0304 & 0.1527 \\
        RF   & Lower bound & 0.0116 & 0.0919 \\
             & Upper bound & 0.0205 & 0.1242 \\
        GBDT & Lower bound & 0.0159 & 0.1049 \\
             & Upper bound & 0.0261 & 0.1399 \\
        Transformer & Lower bound & 0.0030 & 0.0348 \\
             & Upper bound & 0.0156 & 0.1060 \\
        MLP(ReLU)  & Lower bound & 0.0045 & 0.0434 \\
             & Upper bound & 0.0023 & 0.0357 \\
        MLP(LeakyReLU)  & Lower bound & 0.0038 & 0.0379 \\
             & Upper bound & 0.0024 & 0.0380 \\
        MLP(Mish)  & Lower bound & \textbf{0.0011} & \textbf{0.0225} \\
             & Upper bound & \textbf{0.0010} & \textbf{0.0247} \\
             
        \bottomrule
    \end{tabular}
\end{table}
% \section{ Task Generalization Beyond i.i.d. Sampling and Parity Functions
}\label{sec:Discussion}
% Discussion: From Theory to Beyond
% \misha{what is beyond?}
% \amir{we mean two things: in the first subsection beyond i.i.d subsampling of parity tasks and in the second subsection beyond parity task}
% \misha{it has to be beyond something, otherwise it is not clear what it is about} \hz{this is suggested by GPT..., maybe can be interpreted as from theory to beyond theory. We can do explicit like Discussion: Beyond i.i.d. task sampling and the Parity Task}
% \misha{ why is "discussion" in the title?}\amir{Because it is a discussion, it is not like separate concrete explnation about why these thing happens or when they happen, they just discuss some interesting scenraios how it relates to our theory.   } \misha{it is not really a discussion -- there is a bunch of experiments}

In this section, we extend our experiments beyond i.i.d. task sampling and parity functions. We show an adversarial example where biased task selection substantially hinders task generalization for sparse parity problem. In addition, we demonstrate that exponential task scaling extends to a non-parity tasks including arithmetic and multi-step language translation.

% In this section, we extend our experiments beyond i.i.d. task sampling and parity functions. On the one hand, we find that biased task selection can significantly degrade task generalization; on the other hand, we show that exponential task scaling generalizes to broader scenarios.
% \misha{we should add a sentence or two giving more detail}


% 1. beyond i.i.d tasks sampling
% 2. beyond parity -> language, arithmetic -> task dependency + implicit bias of transformer (cannot implement this algorithm for arithmatic)



% In this section, we emphasize the challenge of quantifying the level of out-of-distribution (OOD) differences between training tasks and testing tasks, even for a simple parity task. To illustrate this, we present two scenarios where tasks differ between training and testing. For each scenario, we invite the reader to assess, before examining the experimental results, which cases might appear “more” OOD. All scenarios consider \( d = 10 \). \kaiyue{this sentence should be put into 5.1}






% for parity problem




% \begin{table*}[th!]
%     \centering
%     \caption{Generalization Results for Scenarios 1 and 2 for $d=10$.}
%     \begin{tabular}{|c|c|c|c|}
%         \hline
%         \textbf{Scenario} & \textbf{Type/Variation} & \textbf{Coordinates} & \textbf{Generalization accuracy} \\
%         \hline
%         \multirow{3}{*}{Generalization with Missing Pair} & Type 1 & \( c_1 = 4, c_2 = 6 \) & 47.8\%\\ 
%         & Type 2 & \( c_1 = 4, c_2 = 6 \) & 96.1\%\\ 
%         & Type 3 & \( c_1 = 4, c_2 = 6 \) & 99.5\%\\ 
%         \hline
%         \multirow{3}{*}{Generalization with Missing Pair} & Type 1 &  \( c_1 = 8, c_2 = 9 \) & 40.4\%\\ 
%         & Type 2 & \( c_1 = 8, c_2 = 9 \) & 84.6\% \\ 
%         & Type 3 & \( c_1 = 8, c_2 = 9 \) & 99.1\%\\ 
%         \hline
%         \multirow{1}{*}{Generalization with Missing Coordinate} & --- & \( c_1 = 5 \) & 45.6\% \\ 
%         \hline
%     \end{tabular}
%     \label{tab:generalization_results}
% \end{table*}

\subsection{Task Generalization Beyond i.i.d. Task Sampling }\label{sec: Experiment beyond iid sampling}

% \begin{table*}[ht!]
%     \centering
%     \caption{Generalization Results for Scenarios 1 and 2 for $d=10, k=3$.}
%     \begin{tabular}{|c|c|c|}
%         \hline
%         \textbf{Scenario}  & \textbf{Tasks excluded from training} & \textbf{Generalization accuracy} \\
%         \hline
%         \multirow{1}{*}{Generalization with Missing Pair}
%         & $\{4,6\} \subseteq \{s_1, s_2, s_3\}$ & 96.2\%\\ 
%         \hline
%         \multirow{1}{*}{Generalization with Missing Coordinate}
%         & \( s_2 = 5 \) & 45.6\% \\ 
%         \hline
%     \end{tabular}
%     \label{tab:generalization_results}
% \end{table*}




In previous sections, we focused on \textit{i.i.d. settings}, where the set of training tasks $\mathcal{F}_{train}$ were sampled uniformly at random from the entire class $\mathcal{F}$. Here, we explore scenarios that deliberately break this uniformity to examine the effect of task selection on out-of-distribution (OOD) generalization.\\

\textit{How does the selection of training tasks influence a model’s ability to generalize to unseen tasks? Can we predict which setups are more prone to failure?}\\

\noindent To investigate this, we consider two cases parity problems with \( d = 10 \) and \( k = 3 \), where each task is represented by its tuple of secret indices \( (s_1, s_2, s_3) \):

\begin{enumerate}[leftmargin=0.4 cm]
    \item \textbf{Generalization with a Missing Coordinate.} In this setup, we exclude all training tasks where the second coordinate takes the value \( s_2 = 5 \), such as \( (1,5,7) \). At test time, we evaluate whether the model can generalize to unseen tasks where \( s_2 = 5 \) appears.
    \item \textbf{Generalization with Missing Pair.} Here, we remove all training tasks that contain both \( 4 \) \textit{and} \( 6 \) in the tuple \( (s_1, s_2, s_3) \), such as \( (2,4,6) \) and \( (4,5,6) \). At test time, we assess whether the model can generalize to tasks where both \( 4 \) and \( 6 \) appear together.
\end{enumerate}

% \textbf{Before proceeding, consider the following question:} 
\noindent \textbf{If you had to guess.} Which scenario is more challenging for generalization to unseen tasks? We provide the experimental result in Table~\ref{tab:generalization_results}.

 % while the model struggles for one of them while as it generalizes almost perfectly in the other one. 

% in the first scenario, it generalizes almost perfectly in the second. This highlights how exposure to partial task structures can enhance generalization, even when certain combinations are entirely absent from the training set. 

In the first scenario, despite being trained on all tasks except those where \( s_2 = 5 \), which is of size $O(\d^T)$, the model struggles to generalize to these excluded cases, with prediction at chance level. This is intriguing as one may expect model to generalize across position. The failure  suggests that positional diversity plays a crucial role in the task generalization of Transformers. 

In contrast, in the second scenario, though the model has never seen tasks with both \( 4 \) \textit{and} \( 6 \) together, it has encountered individual instances where \( 4 \) appears in the second position (e.g., \( (1,4,5) \)) or where \( 6 \) appears in the third position (e.g., \( (2,3,6) \)). This exposure appears to facilitate generalization to test cases where both \( 4 \) \textit{and} \( 6 \) are present. 



\begin{table*}[t!]
    \centering
    \caption{Generalization Results for Scenarios 1 and 2 for $d=10, k=3$.}
    \resizebox{\textwidth}{!}{  % Scale to full width
        \begin{tabular}{|c|c|c|}
            \hline
            \textbf{Scenario}  & \textbf{Tasks excluded from training} & \textbf{Generalization accuracy} \\
            \hline
            Generalization with Missing Pair & $\{4,6\} \subseteq \{s_1, s_2, s_3\}$ & 96.2\%\\ 
            \hline
            Generalization with Missing Coordinate & \( s_2 = 5 \) & 45.6\% \\ 
            \hline
        \end{tabular}
    }
    \label{tab:generalization_results}
\end{table*}

As a result, when the training tasks are not i.i.d, an adversarial selection such as exclusion of specific positional configurations may lead to failure to unseen task generalization even though the size of $\mathcal{F}_{train}$ is exponentially large. 


% \paragraph{\textbf{Key Takeaways}}
% \begin{itemize}
%     \item Out-of-distribution generalization in the parity problem is highly sensitive to the diversity and positional coverage of training tasks.
%     \item Adversarial exclusion of specific pairs or positional configurations can lead to systematic failures, even when most tasks are observed during training.
% \end{itemize}




%################ previous veriosn down
% \textit{How does the choice of training tasks affect the ability of a model to generalize to unseen tasks? Can we predict which setups are likely to lead to failure?}

% To explore these questions, we crafted specific training and test task splits to investigate what makes one setup appear “more” OOD than another.

% \paragraph{Generalization with Missing Pair.}

% Imagine we have tasks constructed from subsets of \(k=3\) elements out of a larger set of \(d\) coordinates. What happens if certain pairs of coordinates are adversarially excluded during training? For example, suppose \(d=5\) and two specific coordinates, \(c_1 = 1\) and \(c_2 = 2\), are excluded. The remaining tasks are formed from subsets of the other coordinates. How would a model perform when tested on tasks involving the excluded pair \( (c_1, c_2) \)? 

% To probe this, we devised three variations in how training tasks are constructed:
%     \begin{enumerate}
%         \item \textbf{Type 1:} The training set includes all tasks except those containing both \( c_1 = 1 \) and \( c_2 = 2 \). 
%         For this example, the training set includes only $\{(3,4,5)\}$. The test set consists of all tasks containing the rest of tuples.

%         \item \textbf{Type 2:} Similar to Type 1, but the training set additionally includes half of the tasks containing either \( c_1 = 1 \) \textit{or} \( c_2 = 2 \) (but not both). 
%         For the example, the training set includes all tasks from Type 1 and adds tasks like \(\{(1, 3, 4), (2, 3, 5)\}\) (half of those containing \( c_1 = 1 \) or \( c_2 = 2 \)).

%         \item \textbf{Type 3:} Similar to Type 2, but the training set also includes half of the tasks containing both \( c_1 = 1 \) \textit{and} \( c_2 = 2 \). 
%         For the example, the training set includes all tasks from Type 2 and adds, for instance, \(\{(1, 2, 5)\}\) (half of the tasks containing both \( c_1 \) and \( c_2 \)).
%     \end{enumerate}

% By systematically increasing the diversity of training tasks in a controlled way, while ensuring no overlap between training and test configurations, we observe an improvement in OOD generalization. 

% % \textit{However, the question is this improvement similar across all coordinate pairs, or does it depend on the specific choices of \(c_1\) and \(c_2\) in the tasks?} 

% \textbf{Before proceeding, consider the following question:} Is the observed improvement consistent across all coordinate pairs, or does it depend on the specific choices of \(c_1\) and \(c_2\) in the tasks? 

% For instance, consider two cases for \(d = 10, k = 3\): (i) \(c_1 = 4, c_2 = 6\) and (ii) \(c_1 = 8, c_2 = 9\). Would you expect similar OOD generalization behavior for these two cases across the three training setups we discussed?



% \paragraph{Answer to the Question.} for both cases of \( c_1, c_2 \), we observe that generalization fails in Type 1, suggesting that the position of the tasks the model has been trained on significantly impacts its generalization capability. For Type 2, we find that \( c_1 = 4, c_2 = 6 \) performs significantly better than \( c_1 = 8, c_2 = 9 \). 

% Upon examining the tasks where the transformer fails for \( c_1 = 8, c_2 = 9 \), we see that the model only fails at tasks of the form \((*, 8, 9)\) while perfectly generalizing to the rest. This indicates that the model has never encountered the value \( 8 \) in the second position during training, which likely explains its failure to generalize. In contrast, for \( c_1 = 4, c_2 = 6 \), while the model has not seen tasks of the form \((*, 4, 6)\), it has encountered tasks where \( 4 \) appears in the second position, such as \((1, 4, 5)\), and tasks where \( 6 \) appears in the third position, such as \((2, 3, 6)\). This difference may explain why the model generalizes almost perfectly in Type 2 for \( c_1 = 4, c_2 = 6 \), but not for \( c_1 = 8, c_2 = 9 \).



% \paragraph{Generalization with Missing Coordinates.}
% Next, we investigate whether a model can generalize to tasks where a specific coordinate appears in an unseen position during training. For instance, consider \( c_1 = 5 \), and exclude all tasks where \( c_1 \) appears in the second position. Despite being trained on all other tasks, the model fails to generalize to these excluded cases, highlighting the importance of positional diversity in training tasks.



% \paragraph{Key Takeaways.}
% \begin{itemize}
%     \item OOD generalization depends heavily on the diversity and positional coverage of training tasks for the parity problem.
%     \item adversarial exclusion of specific pairs or positional configurations in the parity problem can lead to failure, even when the majority of tasks are observed during training.
% \end{itemize}


%################ previous veriosn up

% \paragraph{Key Takeaways} These findings highlight the complexity of OOD generalization, even in seemingly simple tasks like parity. They also underscore the importance of task design: the diversity of training tasks can significantly influence a model’s ability to generalize to unseen tasks. By better understanding these dynamics, we can design more robust training regimes that foster generalization across a wider range of scenarios.


% #############


% Upon examining the tasks where the transformer fails for \( c_1 = 8, c_2 = 9 \), we see that the model only fails at tasks of the form \((*, 8, 9)\) while perfectly generalizing to the rest. This indicates that the model has never encountered the value \( 8 \) in the second position during training, which likely explains its failure to generalize. In contrast, for \( c_1 = 4, c_2 = 6 \), while the model has not seen tasks of the form \((*, 4, 6)\), it has encountered tasks where \( 4 \) appears in the second position, such as \((1, 4, 5)\), and tasks where \( 6 \) appears in the third position, such as \((2, 3, 6)\). This difference may explain why the model generalizes almost perfectly in Type 2 for \( c_1 = 4, c_2 = 6 \), but not for \( c_1 = 8, c_2 = 9 \).

% we observe a striking pattern: generalization fails entirely in Type 1, regardless of the coordinate pair (\(c_1, c_2\)). However, in Type 2, generalization varies: \(c_1 = 4, c_2 = 6\) achieves 96\% accuracy, while \(c_1 = 8, c_2 = 9\) lags behind at 70\%. Why? Upon closer inspection, the model struggles specifically with tasks like \((*, 8, 9)\), where the combination \(c_1 = 8\) and \(c_2 = 9\) is entirely novel. In contrast, for \(c_1 = 4, c_2 = 6\), the model benefits from having seen tasks where \(4\) appears in the second position or \(6\) in the third. This suggests that positional exposure during training plays a key role in generalization.

% To test whether task structure influences generalization, we consider two variations:
% \begin{enumerate}
%     \item \textbf{Sorted Tuples:} Tasks are always sorted in ascending order.
%     \item \textbf{Unsorted Tuples:} Tasks can appear in any order.
% \end{enumerate}

% If the model struggles with generalizing to the excluded position, does introducing variability through unsorted tuples help mitigate this limitation?

% \paragraph{Discussion of Results}

% In \textbf{Generalization with Missing Pairs}, we observe a striking pattern: generalization fails entirely in Type 1, regardless of the coordinate pair (\(c_1, c_2\)). However, in Type 2, generalization varies: \(c_1 = 4, c_2 = 6\) achieves 96\% accuracy, while \(c_1 = 8, c_2 = 9\) lags behind at 70\%. Why? Upon closer inspection, the model struggles specifically with tasks like \((*, 8, 9)\), where the combination \(c_1 = 8\) and \(c_2 = 9\) is entirely novel. In contrast, for \(c_1 = 4, c_2 = 6\), the model benefits from having seen tasks where \(4\) appears in the second position or \(6\) in the third. This suggests that positional exposure during training plays a key role in generalization.

% In \textbf{Generalization with Missing Coordinates}, the results confirm this hypothesis. When \(c_1 = 5\) is excluded from the second position during training, the model fails to generalize to such tasks in the sorted case. However, allowing unsorted tuples introduces positional diversity, leading to near-perfect generalization. This raises an intriguing question: does the model inherently overfit to positional patterns, and can task variability help break this tendency?




% In this subsection, we show that the selection of training tasks can affect the quality of the unseen task generalization significantly in practice. To illustrate this, we present two scenarios where tasks differ between training and testing. For each scenario, we invite the reader to assess, before examining the experimental results, which cases might appear “more” OOD. 

% % \amir{add examples, }

% \kaiyue{I think the name of scenarios here are not very clear}
% \begin{itemize}
%     \item \textbf{Scenario 1:  Generalization Across Excluded Coordinate Pairs (\( k = 3 \))} \\
%     In this scenario, we select two coordinates \( c_1 \) and \( c_2 \) out of \( d \) and construct three types of training sets. 

%     Suppose \( d = 5 \), \( c_1 = 1 \), and \( c_2 = 2 \). The tuples are all possible subsets of \( \{1, 2, 3, 4, 5\} \) with \( k = 3 \):
%     \[
%     \begin{aligned}
%     \big\{ & (1, 2, 3), (1, 2, 4), (1, 2, 5), (1, 3, 4), (1, 3, 5), \\
%            & (1, 4, 5), (2, 3, 4), (2, 3, 5), (2, 4, 5), (3, 4, 5) \big\}.
%     \end{aligned}
%     \]

%     \begin{enumerate}
%         \item \textbf{Type 1:} The training set includes all tuples except those containing both \( c_1 = 1 \) and \( c_2 = 2 \). 
%         For this example, the training set includes only $\{(3,4,5)\}$ tuple. The test set consists of tuples containing the rest of tuples.

%         \item \textbf{Type 2:} Similar to Type 1, but the training set additionally includes half of the tuples containing either \( c_1 = 1 \) \textit{or} \( c_2 = 2 \) (but not both). 
%         For the example, the training set includes all tuples from Type 1 and adds tuples like \(\{(1, 3, 4), (2, 3, 5)\}\) (half of those containing \( c_1 = 1 \) or \( c_2 = 2 \)).

%         \item \textbf{Type 3:} Similar to Type 2, but the training set also includes half of the tuples containing both \( c_1 = 1 \) \textit{and} \( c_2 = 2 \). 
%         For the example, the training set includes all tuples from Type 2 and adds, for instance, \(\{(1, 2, 5)\}\) (half of the tuples containing both \( c_1 \) and \( c_2 \)).
%     \end{enumerate}

% % \begin{itemize}
% %     \item \textbf{Type 1:} The training set includes tuples \(\{1, 3, 4\}, \{2, 3, 4\}\) (excluding tuples with both \( c_1 \) and \( c_2 \): \(\{1, 2, 3\}, \{1, 2, 4\}\)). The test set contains the excluded tuples.
% %     \item \textbf{Type 2:} The training set includes all tuples in Type 1 plus half of the tuples containing either \( c_1 = 1 \) or \( c_2 = 2 \) (e.g., \(\{1, 2, 3\}\)).
% %     \item \textbf{Type 3:} The training set includes all tuples in Type 2 plus half of the tuples containing both \( c_1 = 1 \) and \( c_2 = 2 \) (e.g., \(\{1, 2, 4\}\)).
% % \end{itemize}
    
%     \item \textbf{Scenario 2: Scenario 2: Generalization Across a Fixed Coordinate (\( k = 3 \))} \\
%     In this scenario, we select one coordinate \( c_1 \) out of \( d \) (\( c_1 = 5 \)). The training set includes all task tuples except those where \( c_1 \) is the second coordinate of the tuple. For this scenario, we examine two variations:
%     \begin{enumerate}
%         \item \textbf{Sorted Tuples:} Task tuples are always sorted (e.g., \( (x_1, x_2, x_3) \) with \( x_1 \leq x_2 \leq x_3 \)).
%         \item \textbf{Unsorted Tuples:} Task tuples can appear in any order.
%     \end{enumerate}
% \end{itemize}




% \paragraph{Discussion of Results.} In the first scenario, for both cases of \( c_1, c_2 \), we observe that generalization fails in Type 1, suggesting that the position of the tasks the model has been trained on significantly impacts its generalization capability. For Type 2, we find that \( c_1 = 4, c_2 = 6 \) performs significantly better than \( c_1 = 8, c_2 = 9 \). 

% Upon examining the tasks where the transformer fails for \( c_1 = 8, c_2 = 9 \), we see that the model only fails at tasks of the form \((*, 8, 9)\) while perfectly generalizing to the rest. This indicates that the model has never encountered the value \( 8 \) in the second position during training, which likely explains its failure to generalize. In contrast, for \( c_1 = 4, c_2 = 6 \), while the model has not seen tasks of the form \((*, 4, 6)\), it has encountered tasks where \( 4 \) appears in the second position, such as \((1, 4, 5)\), and tasks where \( 6 \) appears in the third position, such as \((2, 3, 6)\). This difference may explain why the model generalizes almost perfectly in Type 2 for \( c_1 = 4, c_2 = 6 \), but not for \( c_1 = 8, c_2 = 9 \).

% This position-based explanation appears compelling, so in the second scenario, we focus on a single position to investigate further. Here, we find that the transformer fails to generalize to tasks where \( 5 \) appears in the second position, provided it has never seen any such tasks during training. However, when we allow for more task diversity in the unsorted case, the model achieves near-perfect generalization. 

% This raises an important question: does the transformer have a tendency to overfit to positional patterns, and does introducing more task variability, as in the unsorted case, prevent this overfitting and enable generalization to unseen positional configurations?

% These findings highlight that even in a simple task like parity, it is remarkably challenging to understand and quantify the sources and levels of OOD behavior. This motivates further investigation into the nuances of task design and its impact on model generalization.


\subsection{Task Generalization Beyond Parity Problems}

% \begin{figure}[t!]
%     \centering
%     \includegraphics[width=0.45\textwidth]{Figures/arithmetic_v1.png}
%     \vspace{-0.3cm}
%     \caption{Task generalization for arithmetic task with CoT, it has $\d =2$ and $T = d-1$ as the ambient dimension, hence $D\ln(DT) = 2\ln(2T)$. We show that the empirical scaling closely follows the theoretical scaling.}
%     \label{fig:arithmetic}
% \end{figure}



% \begin{wrapfigure}{r}{0.4\textwidth}  % 'r' for right, 'l' for left
%     \centering
%     \includegraphics[width=0.4\textwidth]{Figures/arithmetic_v1.png}
%     \vspace{-0.3cm}
%     \caption{Task generalization for the arithmetic task with CoT. It has $d =2$ and $T = d-1$ as the ambient dimension, hence $D\ln(DT) = 2\ln(2T)$. We show that the empirical scaling closely follows the theoretical scaling.}
%     \label{fig:arithmetic}
% \end{wrapfigure}

\subsubsection{Arithmetic Task}\label{subsec:arithmetic}











We introduce the family of \textit{Arithmetic} task that, like the sparse parity problem, operates on 
\( d \) binary inputs \( b_1, b_2, \dots, b_d \). The task involves computing a structured arithmetic expression over these inputs using a sequence of addition and multiplication operations.
\newcommand{\op}{\textrm{op}}

Formally, we define the function:
\[
\text{Arithmetic}_{S} \colon \{0,1\}^d \to \{0,1,\dots,d\},
\]
where \( S = (\op_1, \op_2, \dots, \op_{d-1}) \) is a sequence of \( d-1 \) operations, each \( \op_k \) chosen from \( \{+, \times\} \). The function evaluates the expression by applying the operations sequentially from left-to-right order: for example, if \( S = (+, \times, +) \), then the arithmetic function would compute:
\[
\text{Arithmetic}_{S}(b_1, b_2, b_3, b_4) = ((b_1 + b_2) \times b_3) + b_4.
\]
% Thus, the sequence of operations \( S \) defines how the binary inputs are combined to produce an integer output between \( 0 \) and \( d \).
% \[
% \text{Arithmetic}_{S} 
% (b_1,\,b_2,\,\dots,b_d)
% =
% \Bigl(\dots\bigl(\,(b_1 \;\op_1\; b_2)\;\op_2\; b_3\bigr)\,\dots\Bigr) 
% \;\op_{d-1}\; b_d.
% \]
% We now introduce an \emph{Arithmetic} task that, like the sparse parity problem, operates on $d$ binary inputs $b_1, b_2, \dots, b_d$. Specifically, we define an arithmetic function
% \[
% \text{Arithmetic}_{S}\colon \{0,1\}^d \;\to\; \{0,1,\dots,d\},
% \]
% where $S = (i_1, i_2, \dots, i_{d-1})$ is a sequence of $d-1$ operations, each $i_k \in \{+,\,\times\}$. The value of $\text{Arithmetic}_{S}$ is obtained by applying the prescribed addition and multiplication operations in order, namely:
% \[
% \text{Arithmetic}_{S}(b_1,\,b_2,\,\dots,b_d)
% \;=\;
% \Bigl(\dots\bigl(\,(b_1 \;i_1\; b_2)\;i_2\; b_3\bigr)\,\dots\Bigr) 
% \;i_{d-1}\; b_d.
% \]

% This is an example of our framework where $T = d-1$ and $|\Theta_t| = 2$ with total $2^d$ possible tasks. 




By introducing a step-by-step CoT, arithmetic class belongs to $ARC(2, d-1)$: this is because at every step, there is only $\d = |\Theta_t| = 2$ choices (either $+$ or $\times$) while the length is  $T = d-1$, resulting a total number of $2^{d-1}$ tasks. 


\begin{minipage}{0.5\textwidth}  % Left: Text
    Task generalization for the arithmetic task with CoT. It has $d =2$ and $T = d-1$ as the ambient dimension, hence $D\ln(DT) = 2\ln(2T)$. We show that the empirical scaling closely follows the theoretical scaling.
\end{minipage}
\hfill
\begin{minipage}{0.4\textwidth}  % Right: Image
    \centering
    \includegraphics[width=\textwidth]{Figures/arithmetic_v1.png}
    \refstepcounter{figure}  % Manually advances the figure counter
    \label{fig:arithmetic}  % Now this label correctly refers to the figure
\end{minipage}

Notably, when scaling with \( T \), we observe in the figure above that the task scaling closely follow the theoretical $O(D\log(DT))$ dependency. Given that the function class grows exponentially as \( 2^T \), it is truly remarkable that training on only a few hundred tasks enables generalization to an exponentially larger space—on the order of \( 2^{25} > 33 \) Million tasks. This exponential scaling highlights the efficiency of structured learning, where a modest number of training examples can yield vast generalization capability.





% Our theory suggests that only $\Tilde{O}(\ln(T))$ i.i.d training tasks is enough to generalize to the rest of unseen tasks. However, we show in Figure \ref{fig:arithmetic} that transformer is not able to match  that. The transformer out-of distribution generalization behavior is not consistent across different dimensions when we scale the number of training tasks with $\ln(T)$. \hongzhou{implicit bias, optimization, etc}
 






% \subsection{Task generalization Beyond parity problem}

% \subsection{Arithmetic} In this setting, we still use the set-up we introduced in \ref{subsec:parity_exmaple}, the input is still a set of $d$ binary variable, $b_1, b_2,\dots,b_d$ and ${Arithmatic_{S}}:\{0,1\}\rightarrow \{0, 1, \dots, d\}$, where $S = (i_1,i_2,\dots,i_{d-1})$ is a tuple of size $d-1$ where each coordinate is either add($+
% $) or multiplication ($\times$). The function is as following,

% \begin{align*}
%     Arithmatic_{S}(b_1, b_2,\dots,b_d) = (\dots(b1(i1)b2)(i3)b3\dots)(i{d-1})
% \end{align*}
    


\subsubsection{Multi-Step Language Translation Task}

 \begin{figure*}[h!]
    \centering
    \includegraphics[width=0.9\textwidth]{Figures/combined_plot_horiz.png}
    \vspace{-0.2cm}
    \caption{Task generalization for language translation task: $\d$ is the number of languages and $T$ is the length of steps.}
    \vspace{-2mm}
    \label{fig:language}
\end{figure*}
% \vspace{-2mm}

In this task, we study a sequential translation process across multiple languages~\cite{garg2022can}. Given a set of \( D \) languages, we construct a translation chain by randomly sampling a sequence of \( T \) languages \textbf{with replacement}:  \(L_1, L_2, \dots, L_T,\)
where each \( L_t \) is a sampled language. Starting with a word, we iteratively translate it through the sequence:
\vspace{-2mm}
\[
L_1 \to L_2 \to L_3 \to \dots \to L_T.
\]
For example, if the sampled sequence is EN → FR → DE → FR, translating the word "butterfly" follows:
\vspace{-1mm}
\[
\text{butterfly} \to \text{papillon} \to \text{schmetterling} \to \text{papillon}.
\]
This task follows an \textit{AutoRegressive Compositional} structure by itself, specifically \( ARC(D, T-1) \), where at each step, the conditional generation only depends on the target language, making \( D \) as the number of languages and the total number of possible tasks is \( D^{T-1} \). This example illustrates that autoregressive compositional structures naturally arise in real-world languages, even without explicit CoT. 

We examine task scaling along \( D \) (number of languages) and \( T \) (sequence length). As shown in Figure~\ref{fig:language}, empirical  \( D \)-scaling closely follows the theoretical \( O(D \ln D T) \). However, in the \( T \)-scaling case, we observe a linear dependency on \( T \) rather than the logarithmic dependency \(O(\ln T) \). A possible explanation is error accumulation across sequential steps—longer sequences require higher precision in intermediate steps to maintain accuracy. This contrasts with our theoretical analysis, which focuses on asymptotic scaling and does not explicitly account for compounding errors in finite-sample settings.

% We examine task scaling along \( D \) (number of languages) and \( T \) (sequence length). As shown in Figure~\ref{fig:language}, empirical scaling closely follows the theoretical \( O(D \ln D T) \) trend, with slight exceptions at $ T=10 \text{ and } 3$ in Panel B. One possible explanation for this deviation could be error accumulation across sequential steps—longer sequences require each intermediate translation to be approximated with higher precision to maintain test accuracy. This contrasts with our theoretical analysis, which primarily focuses on asymptotic scaling and does not explicitly account for compounding errors in finite-sample settings.

Despite this, the task scaling is still remarkable — training on a few hundred tasks enables generalization to   $4^{10} \approx 10^6$ tasks!






% , this case, we are in a regime where \( D \ll T \), we observe  that the task complexity empirically scales as \( T \log T \) rather than \( D \log T \). 


% the model generalizes to an exponentially larger space of \( 2^T \) unseen tasks. In case $T=25$, this is $2^{25} > 33$ Million tasks. This remarkable exponential generalization demonstrates the power of structured task composition in enabling efficient generalization.


% In the case of parity tasks, introducing CoT effectively decomposes the problem from \( ARC(D^T, 1) \) to \( ARC(D, T) \), significantly improving task generalization.

% Again, in the regime scaling $T$, we again observe a $T\log T$ dependency. Knowing that the function class is scaling as $D^T$, it is remarkable that training on a few hundreds tasks can generalize to $4^{10} \approx 1M$ tasks. 





% We further performed a preliminary investigation on a semi-synthetic word-level translation task to show that (1) task generalization via composition structure is feasible beyond parity and (2) understanding the fine-grained mechanism leading to this generalization is still challenging. 
% \noindent
% \noindent
% \begin{minipage}[t]{\columnwidth}
%     \centering
%     \textbf{\scriptsize In-context examples:}
%     \[
%     \begin{array}{rl}
%         \textbf{Input} & \hspace{1.5em} \textbf{Output} \\
%         \hline
%         \textcolor{blue}{car}   & \hspace{1.5em} \textcolor{red}{voiture \;,\; coche} \\
%         \textcolor{blue}{house} & \hspace{1.5em} \textcolor{red}{maison \;,\; casa} \\
%         \textcolor{blue}{dog}   & \hspace{1.5em} \textcolor{red}{chien \;,\; perro} 
%     \end{array}
%     \]
%     \textbf{\scriptsize Query:}
%     \[
%     \begin{array}{rl}
%         \textbf{Input} & \textbf{Output} \\
%         \hline
%         \textcolor{blue}{cat} & \hspace{1.5em} \textcolor{red}{?} \\
%     \end{array}
%     \]
% \end{minipage}



% \begin{figure}[h!]
%     \centering
%     \includegraphics[width=0.45\textwidth]{Figures/translation_scale_d.png}
%     \vspace{-0.2cm}
%     \caption{Task generalization behavior for word translation task.}
%     \label{fig:arithmetic}
% \end{figure}


\vspace{-1mm}
\section{Conclusions}
% \misha{is it conclusion of the section or of the whole paper?}    
% \amir{The whole paper. It is very short, do we need a separate section?}
% \misha{it should not be a subsection if it is the conclusion the whole thing. We can just remove it , it does not look informative} \hz{let's do it in a whole section, just to conclude and end the paper, even though it is not informative}
%     \kaiyue{Proposal: Talk about the implication of this result on theory development. For example, it calls for more fine-grained theoretical study in this space.  }

% \huaqing{Please feel free to edit it if you have better wording or suggestions.}

% In this work, we propose a theoretical framework to quantitatively investigate task generalization with compositional autoregressive tasks. We show that task to $D^T$ task is theoretically achievable by training on only $O (D\log DT)$ tasks, and empirically observe that transformers trained on parity problem indeed achieves such task generalization. However, for other tasks beyond parity, transformers seem to fail to achieve this bond. This calls for more fine-grained theoretical study the phenomenon of task generalization specific to transformer model. It may also be interesting to study task generalization beyond the setting of in-context learning. 
% \misha{what does this add?} \amir{It does not, i dont have any particular opinion to keep it. @Hongzhou if you want to add here?}\hz{While it may not introduce anything new, we are following a good practice to have a short conclusion. It provides a clear closing statement, reinforces key takeaways, and helps the reader leave with a well-framed understanding of our contributions. }
% In this work, we quantitatively investigate task generalization under autoregressive compositional structure. We demonstrate that task generalization to $D^T$ tasks is theoretically achievable by training on only $\tilde O(D)$ tasks. Empirically, we observe that transformers trained indeed achieve such exponential task generalization on problems such as parity, arithmetic and multi-step language translation. We believe our analysis opens up a new angle to understand the remarkable generalization ability of Transformer in practice. 

% However, for tasks beyond the parity problem, transformers appear to fail to reach this bound. This highlights the need for a more fine-grained theoretical exploration of task generalization, especially for transformer models. Additionally, it may be valuable to investigate task generalization beyond the scope of in-context learning.



In this work, we quantitatively investigated task generalization under the autoregressive compositional structure, demonstrating both theoretically and empirically that exponential task generalization to $D^T$ tasks can be achieved with training on only $\tilde{O}(D)$ tasks. %Our theoretical results establish a fundamental scaling law for task generalization, while our experiments validate these insights across problems such as parity, arithmetic, and multi-step language translation. The remarkable ability of transformers to generalize exponentially highlights the power of structured learning and provides a new perspective on how large language models extend their capabilities beyond seen tasks. 
We recap our key contributions  as follows:
\begin{itemize}
    \item \textbf{Theoretical Framework for Task Generalization.} We introduced the \emph{AutoRegressive Compositional} (ARC) framework to model structured task learning, demonstrating that a model trained on only $\tilde{O}(D)$ tasks can generalize to an exponentially large space of $D^T$ tasks.
    
    \item \textbf{Formal Sample Complexity Bound.} We established a fundamental scaling law that quantifies the number of tasks required for generalization, proving that exponential generalization is theoretically achievable with only a logarithmic increase in training samples.
    
    \item \textbf{Empirical Validation on Parity Functions.} We showed that Transformers struggle with standard in-context learning (ICL) on parity tasks but achieve exponential generalization when Chain-of-Thought (CoT) reasoning is introduced. Our results provide the first empirical demonstration of structured learning enabling efficient generalization in this setting.
    
    \item \textbf{Scaling Laws in Arithmetic and Language Translation.} Extending beyond parity functions, we demonstrated that the same compositional principles hold for arithmetic operations and multi-step language translation, confirming that structured learning significantly reduces the task complexity required for generalization.
    
    \item \textbf{Impact of Training Task Selection.} We analyzed how different task selection strategies affect generalization, showing that adversarially chosen training tasks can hinder generalization, while diverse training distributions promote robust learning across unseen tasks.
\end{itemize}



We introduce a framework for studying the role of compositionality in learning tasks and how this structure can significantly enhance generalization to unseen tasks. Additionally, we provide empirical evidence on learning tasks, such as the parity problem, demonstrating that transformers follow the scaling behavior predicted by our compositionality-based theory. Future research will  explore how these principles extend to real-world applications such as program synthesis, mathematical reasoning, and decision-making tasks. 


By establishing a principled framework for task generalization, our work advances the understanding of how models can learn efficiently beyond supervised training and adapt to new task distributions. We hope these insights will inspire further research into the mechanisms underlying task generalization and compositional generalization.

\section*{Acknowledgements}
We acknowledge support from the National Science Foundation (NSF) and the Simons Foundation for the Collaboration on the Theoretical Foundations of Deep Learning through awards DMS-2031883 and \#814639 as well as the  TILOS institute (NSF CCF-2112665) and the Office of Naval Research (ONR N000142412631). 
This work used the programs (1) XSEDE (Extreme science and engineering discovery environment)  which is supported by NSF grant numbers ACI-1548562, and (2) ACCESS (Advanced cyberinfrastructure coordination ecosystem: services \& support) which is supported by NSF grants numbers \#2138259, \#2138286, \#2138307, \#2137603, and \#2138296. Specifically, we used the resources from SDSC Expanse GPU compute nodes, and NCSA Delta system, via allocations TG-CIS220009. 


\section{Conclusion}
In this paper, we demonstrated that the bounds of probabilities of causation can be effectively learned and predicted using machine learning models. Specifically, we proposed five different models to predict the bounds of PNS. Experiments showed that an MLP with the Mish activation function achieved a mean absolute error of approximately 0.02 for an SCM with 15 observed and 5 unobserved confounders. Our results suggest that machine learning is a powerful tool for causal inference, particularly in real-world scenarios where direct estimation using SCM formulas is infeasible due to data limitations. Future research will explore larger datasets with more complex SCMs.

Although our study demonstrates the feasibility of machine learning for estimating probabilities of causation, we acknowledge that our experiments are based on synthetic data generated from a structured SCM. Most existing research on probabilities of causation remains theoretical, often without practical validation, despite claims of real-world applicability. Due to page limitations, we could not extend our study to real-world applications, but this remains a critical direction for future research. We believe that bridging this gap will require developing datasets from real-world causal systems where experimental and observational data can be systematically collected. Our work serves as a first step in this direction, providing a foundation for future studies to explore the practical deployment of machine learning models for causality estimation.


% \begin{acknowledgements} % will be removed in pdf for initial submission,
% 						 % (without ‘accepted’ option in \documentclass)
%                          % so you can already fill it to test with the
%                          % ‘accepted’ class option
%     Briefly acknowledge people and organizations here.

%     \emph{All} acknowledgements go in this section.
% \end{acknowledgements}


% \newpage
\clearpage
\begin{thebibliography}{00}

\bibitem{1} Bhide, Nirmal, and Christopher M. Bishop. "Pathophysiology of Non-Dopaminergic Monoamine Systems in Parkinson's Disease: Implications for Mood Dysfunction." InTech EBooks, 2011,  https://doi.org/10.5772/21140.


\bibitem{2} What Is Parkinson’s? — Parkinson’s Foundation. Parkinson’s Foundation. https://www.parkinson.org/understanding-parkinsons/what-is-parkinsons . Accessed 20 Aug 2022

\bibitem{3} Hamzehei, Sahand. "Gateways and Wearable Tools for Monitoring Patient Movements in a Hospital Environment-Webthesis." (2022).

\bibitem{4} Karami, Mostafa. "Machine Learning Algorithms for Radiogenomics: Application to Prediction of the MGMT promoter methylation status in mpMRI scans." PhD diss., Politecnico di Torino, 2022.

\bibitem{5} Andreas Maier, Christopher Syben, Tobias Lasser, Christian Riess, A gentle introduction to deep learning in medical image processing, Zeitschrift für Medizinische Physik, Volume 29, Issue 2,2019,Pages 86-101,ISSN 0939-3889

\bibitem{6} Tsanas A, Little MA, McSharry PE, Ramig LO (2010) Accurate telemonitoring of Parkinson’s disease progression by noninvasive speech tests. IEEE Trans Biomed Eng 57(4):884–93. https://doi.org/10.1109/TBME.2009.2036000. Epub 2009 Nov 20 PMID: 19932995

\bibitem{7} Tin Kam Ho, "Random decision forests," Proceedings of 3rd International Conference on Document Analysis and Recognition, Montreal, QC, Canada, 1995, pp. 278-282 vol.1, doi: 10.1109/ICDAR.1995.598994.

\bibitem{8} Guyon I.; Weston J.; Barnhill S.; Vapnik V. (2002). "Gene selection for cancer classification using support vector machines". Machine Learning. 46 (1–3): 389–422. doi:10.1023/A:1012487302797

\bibitem{9} Wolaver, Dan H. (1991). Phase-Locked Loop Circuit Design. Prentice Hall. p. 211. ISBN 978-0-13-662743-2.

\bibitem{10} Hamzehei, S., Akbarzadeh, O., Attar, H. et al. Predicting the total Unified Parkinson’s Disease Rating Scale (UPDRS) based on ML techniques and cloud-based update. J Cloud Comp 12, 12 (2023). https://doi.org/10.1186/s13677-022-00388-1

\bibitem{11} "Mean Squared Error (MSE)" www.probabilitycourse.com. Retrieved 2020-09-12.

\bibitem{12} Steel, R. G. D.; Torrie, J. H. (1960). Principles and Procedures of Statistics with Special Reference to the Biological Sciences. McGraw Hill.

%%%%%%%%%%% LR
\bibitem{13} A. U. Haq et al., "Feature Selection Based on L1-Norm Support Vector Machine and Effective Recognition System for Parkinson’s Disease Using Voice Recordings," in IEEE Access, vol. 7, pp. 37718-37734, 2019, doi: 10.1109/ACCESS.2019.2906350.

\bibitem{14} A. A. Spadoto, R. C. Guido, F. L. Carnevali, A. F. Pagnin, A. X. Falcão and J. P. Papa, "Improving Parkinson's disease identification through evolutionary-based feature selection," 2011 Annual International Conference of the IEEE Engineering in Medicine and Biology Society, Boston, MA, USA, 2011, pp. 7857-7860, doi: 10.1109/IEMBS.2011.6091936.

\bibitem{15} S. Aich, K. Younga, K. L. Hui, A. A. Al-Absi and M. Sain, "A nonlinear decision tree based classification approach to predict the Parkinson's disease using different feature sets of voice data," 2018 20th International Conference on Advanced Communication Technology (ICACT), Chuncheon, Korea (South), 2018, pp. 638-642, doi: 10.23919/ICACT.2018.8323864.

\bibitem{16} Amato F, Borzì L, Olmo G, Orozco-Arroyave JR. An algorithm for Parkinson's disease speech classification based on isolated words analysis. Health Inf Sci Syst. 2021 Jul 30;9(1):32. doi: 10.1007/s13755-021-00162-8. PMID: 34422258; PMCID: PMC8324609.





\end{thebibliography}


% References
% \bibliography{uai2025-template,ang}
\clearpage
% \newpage

\onecolumn

\title{Supplementary Material}
\maketitle

\appendix
\section{The Causal Model}
The coefficients for \( M_X, M_Y \), and \( C_Y \) were uniformly generated from the range \([-1,1]\), while the parameters of the Bernoulli distribution were uniformly generated from \([0,1]\). The detailed model is as follows:
\begin{eqnarray*}
    &&\begin{cases}
        Z_i &= U_{Z_i} \text{ for } i \in \{1,...,20\},\\
        X&=f_X(M_X,U_X)\\
        &=\begin{cases}
            1& \text{ if } M_X+U_X > 0.5\\
            0& \text{ otherwise, }\\
        \end{cases}\\
        Y&=f_Y(X,M_Y,U_Y)\\
        &=\begin{cases}
            1& \text{ if } 0<C_Y \cdot X+M_Y+U_Y <1 \\
            1& \text{ if } 1<C_Y \cdot X+M_Y+U_Y <2 \\
            0& \text{ otherwise. }\\
        \end{cases}
    \end{cases}\\
&&\text{where, } U_{Z_i}, U_X, U_Y \text{ are binary exogenous variables with Bernoulli distributions.}\\
&&s.t., \\
&M_X& =
\begin{bmatrix}
Z_1~Z_2~...~Z_{20}
\end{bmatrix}\times
\begin{bmatrix}
0.259223510143\\ -0.658140989167\\ -0.75025831768\\ 0.162906462426\\ 0.652023463285\\ -0.0892939586541\\ 0.421469107769\\ -0.443129684766\\ 0.802624388789\\ -0.225740978499\\ 0.716621631717\\ 0.0650682260309\\ -0.220690334026\\ 0.156355773665\\ -0.50693672491\\ -0.707060278115\\ 0.418812816935\\ -0.0822118703986\\ 0.769299853833\\ -0.511585391002
\end{bmatrix},
M_Y =
\begin{bmatrix}
Z_1~Z_2~...~Z_{20}
\end{bmatrix}\times
\begin{bmatrix}
-0.792867111918\\ 0.759967136147\\ 0.55437722369\\ 0.503970540409\\ -0.527187144651\\ 0.378619988091\\ 0.269255196301\\ 0.671597043594\\ 0.396010142274\\ 0.325228576643\\ 0.657808327574\\ 0.801655023993\\ 0.0907679484097\\ -0.0713852594543\\ -0.0691046005285\\ -0.222582013343\\ -0.848408031595\\ -0.584285069026\\ -0.324874831799\\ 0.625621583197
\end{bmatrix}
\end{eqnarray*}
\begin{eqnarray*}
&&U_{Z_1} \sim \text{Bernoulli}(0.352913861526), U_{Z_2} \sim \text{Bernoulli}(0.460995855543),\\
&&U_{Z_3} \sim \text{Bernoulli}(0.331702473392), U_{Z_4} \sim \text{Bernoulli}(0.885505026779),\\
&&U_{Z_5} \sim \text{Bernoulli}(0.017026872706), U_{Z_6} \sim \text{Bernoulli}(0.380772701708),\\
&&U_{Z_7} \sim \text{Bernoulli}(0.028092602705), U_{Z_8} \sim \text{Bernoulli}(0.220819399962),\\
&&U_{Z_9} \sim \text{Bernoulli}(0.617742227477), U_{Z_{10}} \sim \text{Bernoulli}(0.981975046713),\\
&&U_{Z_{11}} \sim \text{Bernoulli}(0.142042291381), U_{Z_{12}} \sim \text{Bernoulli}(0.833602592350),\\
&&U_{Z_{13}} \sim \text{Bernoulli}(0.882938907115), U_{Z_{14}} \sim \text{Bernoulli}(0.542143191999),\\
&&U_{Z_{15}} \sim \text{Bernoulli}(0.085023436884), U_{Z_{16}} \sim \text{Bernoulli}(0.645357252864),\\
&&U_{Z_{17}} \sim \text{Bernoulli}(0.863787135134), U_{Z_{18}} \sim \text{Bernoulli}(0.460539711624),\\
&&U_{Z_{19}} \sim \text{Bernoulli}(0.314014079207), U_{Z_{20}} \sim \text{Bernoulli}(0.685879388218),\\
&&U_{X} \sim \text{Bernoulli}(0.601680857267), U_{Y} \sim \text{Bernoulli}(0.497668975278),\\
&&C_Y=-0.77953605542.
\end{eqnarray*}

\section{Detailed Data Generating Process}
If all 20 binary features are observable, then for a given feature set \( z = (z_1, \dots, z_{20}) \), the values of \( M_X \) and \( M_Y \) are fixed (denoted as \( M_X(z) \) and \( M_Y(z) \)). Under these conditions, the PNS, experimental distribution, and observational distribution corresponding to this set of features are:
\begin{eqnarray*}
PNS(z) &=& P(Y=0_{X=0}, Y=1_{X=1}|z)\\
&=& P(U_Y=0)\cdot T_0 + P(U_Y=1)\cdot T_1, \\
\text{where}, &T_0& =
\left \{
\begin{array}{cc}
1& \text{ if } f_Y(0,M_Y(z),0)=0 \text{ and } f_Y(1,M_Y(z),0)=1, \\
0& \text{otherwize},
\end{array}
\right.\\
&T_1& =
\left \{
\begin{array}{cc}
1& \text{ if } f_Y(0,M_Y(z),1)=0 \text{ and } f_Y(1,M_Y(z),1)=1, \\
0& \text{otherwize.}
\end{array}
\right.
\end{eqnarray*}
\begin{eqnarray*}
&&P(Y=1|do(X),z)\\
&=& P(U_Y=0)\cdot f_Y(X,M_Y(z),0) + P(U_Y=1)\cdot f_Y(X,M_Y(z),1).
\end{eqnarray*}
\begin{eqnarray*}
&&P(Y=1|X,z)\\
&=& P(U_X=0)\cdot P(U_Y=0)\cdot f_Y(f_X(M_X(z),0),M_Y(z),0)+\\
&&P(U_X=0)\cdot P(U_Y=1)\cdot f_Y(f_X(M_X(z),0),M_Y(z),1)) +\\ &&P(U_X=1)\cdot P(U_Y=0)\cdot f_Y(f_X(M_X(z),1),M_Y(z),0)) +\\ &&P(U_X=1)\cdot P(U_Y=1)\cdot f_Y(f_X(M_X(z),1),M_Y(z),1)).
\end{eqnarray*}

% We assumed $15$ of the features are observable (i.e., $Z_1,...,Z_{15}$), which means each subpopulation $c=(z_1,...,z_{15})$ consists $32$ sets of $20$ binary features (i.e., $s_{0}=(z_1,...,z_{15},0,0,0,0,0), s_{1}=(z_1,...,z_{15},0,0,0,0,1), s_{2}=(z_1,...,z_{15},0,0,0,1,0), ...,s_{31}=(z_1,...,z_{15},1,1,1,1,1)$), then we have the PNS, experimental distribution, and observational distribution of all observed subpopulations as follow:
We assume that $15$ of the features are observable (i.e., \( Z_1, \dots, Z_{15} \)). This implies that each subpopulation, denoted as \( c = (z_1, \dots, z_{15}) \), consists of 32 possible sets of $20$ binary features. Specifically, these sets are: $s_{0}=(z_1,...,z_{15},0,0,0,0,0), s_{1}=(z_1,...,z_{15},0,0,0,0,1), s_{2}=(z_1,...,z_{15},0,0,0,1,0), ...,s_{31}=(z_1,...,z_{15},1,1,1,1,1)$.

Under this setup, we obtain the $PNS_{\text{subpopulation}}$, experimental distribution, and observational distribution for any observed subpopulation $c$ as follows:
\begin{eqnarray*}
PNS_{\text{subpopulation}}(c) &=& P(Y=0_{X=0}, Y=1_{X=1}|c)\\
&=& P(s_{0})/P(c)PNS(s_{0})+P(s_{1})/P(c)PNS(s_{1})+\\
&&P(s_{2})/P(c)PNS(s_{2})+...+P(s_{31})/P(c)PNS(s_{31})\\
&=& P(Z_{16}=0)P(Z_{17}=0)P(Z_{18}=0)P(Z_{19}=0)P(Z_{20}=0)PNS(s_{0})+\\
&&P(Z_{16}=0)P(Z_{17}=0)P(Z_{18}=0)P(Z_{19}=0)P(Z_{20}=1)PNS(s_{1})+...+\\
&&P(Z_{16}=1)P(Z_{17}=1)P(Z_{18}=1)P(Z_{19}=1)P(Z_{20}=1)PNS(s_{31}).
\end{eqnarray*}
\begin{eqnarray*}
&&P(Y=1|do(X),c)\\
&=& P(Z_{16}=0)P(Z_{17}=0)P(Z_{18}=0)P(Z_{19}=0)P(Z_{20}=0)P(Y=1|do(X),s_{0})+\\
&& P(Z_{16}=0)P(Z_{17}=0)P(Z_{18}=0)P(Z_{19}=0)P(Z_{20}=1)P(Y=1|do(X),s_{1})+\\
&& P(Z_{16}=0)P(Z_{17}=0)P(Z_{18}=0)P(Z_{19}=1)P(Z_{20}=0)P(Y=1|do(X),s_{2})+...+\\
&& P(Z_{16}=1)P(Z_{17}=1)P(Z_{18}=1)P(Z_{19}=1)P(Z_{20}=1)P(Y=1|do(X),s_{31}).
\end{eqnarray*}
\begin{eqnarray*}
&&P(Y=1|X,c)\\
&=& P(Z_{16}=0)P(Z_{17}=0)P(Z_{18}=0)P(Z_{19}=0)P(Z_{20}=0)P(Y=1|X,s_{0})+\\
&& P(Z_{16}=0)P(Z_{17}=0)P(Z_{18}=0)P(Z_{19}=0)P(Z_{20}=1)P(Y=1|X,s_{1})+\\
&& P(Z_{16}=0)P(Z_{17}=0)P(Z_{18}=0)P(Z_{19}=1)P(Z_{20}=0)P(Y=1|X,s_{2})+...+\\
&& P(Z_{16}=1)P(Z_{17}=1)P(Z_{18}=1)P(Z_{19}=1)P(Z_{20}=1)P(Y=1|X,s_{31}).
\end{eqnarray*}
The true bounds of the $PNS_{\text{subpopulation}}(c)$ can be obtained using Equations \ref{pnslb} and \ref{pnsub}, along with the above observational and experimental distributions.

\section{Code}
All code for data generation and machine learning models is available at the following anonymous link: \url{https://anonymous.4open.science/r/2025uai-ED50/}.
\end{document}
