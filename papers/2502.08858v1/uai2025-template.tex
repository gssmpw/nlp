% \documentclass{uai2025} % for initial submission
\documentclass[accepted]{uai2025} % after acceptance, for a revised version; 
% also before submission to see how the non-anonymous paper would look like 
                        
%% There is a class option to choose the math font
% \documentclass[mathfont=ptmx]{uai2025} % ptmx math instead of Computer
                                         % Modern (has noticeable issues)
% \documentclass[mathfont=newtx]{uai2025} % newtx fonts (improves upon
                                          % ptmx; less tested, no support)
% NOTE: Only keep *one* line above as appropriate, as it will be replaced
%       automatically for papers to be published. Do not make any other
%       change above this note for an accepted version.

%% Choose your variant of English; be consistent
\usepackage[american]{babel}
% \usepackage[british]{babel}

%% Some suggested packages, as needed:
\usepackage{natbib} % has a nice set of citation styles and commands
    \bibliographystyle{plainnat}
    \renewcommand{\bibsection}{\subsubsection*{References}}
\usepackage{mathtools} % amsmath with fixes and additions
% \usepackage{siunitx} % for proper typesetting of numbers and units
\usepackage{booktabs} % commands to create good-looking tables
\usepackage{tikz} % nice language for creating drawings and diagrams



\usepackage{tikz-cd}
\usetikzlibrary{arrows}
\usepackage{comment}
\usepackage{amssymb}
\usepackage{tikz}
\usepackage{tikz-qtree,tikz-qtree-compat}
\usetikzlibrary{positioning}
\usepackage{float}


\usepackage{amsmath}
\usepackage{amsthm}
\usepackage{tikz}
\usepackage{tikz-qtree,tikz-qtree-compat}

\newcommand\independent{\protect\mathpalette{\protect\independenT}{\perp}}
\def\independenT#1#2{\mathrel{\rlap{$#1#2$}\mkern2mu{#1#2}}}
\newcommand{\notindep}{\not\!\independent}

\usepackage{stackengine}
\def\delequal{\mathrel{\ensurestackMath{\stackon[1pt]{=}{\scriptstyle\Delta}}}}
%%%%%%%%%%%%%%%%%%%%%%%%%%%%%%%%
% THEOREMS
%%%%%%%%%%%%%%%%%%%%%%%%%%%%%%%%
% \newtheorem{theorem}{Theorem}
% \newtheorem{definition}[theorem]{Definition}
% \newtheorem{lemma}[theorem]{Lemma}
% \newtheorem{conjecture}[theorem]{Conjecture}
% \newtheorem{corollary}[theorem]{Corollary}
\theoremstyle{plain}
\newtheorem{theorem}{Theorem}
\newtheorem{proposition}[theorem]{Proposition}
\newtheorem{lemma}[theorem]{Lemma}
\newtheorem{corollary}[theorem]{Corollary}
\theoremstyle{definition}
\newtheorem{definition}[theorem]{Definition}
\newtheorem{assumption}[theorem]{Assumption}
\theoremstyle{remark}
\newtheorem{remark}[theorem]{Remark}



\usepackage{subcaption}

%% Provided macros
% \smaller: Because the class footnote size is essentially LaTeX's \small,
%           redefining \footnotesize, we provide the original \footnotesize
%           using this macro.
%           (Use only sparingly, e.g., in drawings, as it is quite small.)

%% Self-defined macros
\newcommand{\swap}[3][-]{#3#1#2} % just an example

\title{Estimating Probabilities of Causation with Machine Learning Models}

% The standard author block has changed for UAI 2025 to provide
% more space for long author lists and allow for complex affiliations
%
% All author information is authomatically removed by the class for the
% anonymous submission version of your paper, so you can already add your
% information below.
%
% Add authors
\author[1]{\href{mailto:<sw23v@fsu.edu>?Subject=Your UAI 2025 paper}{Shuai Wang}}
\author[1]{\href{mailto:<angli@cs.fsu.edu>?Subject=Your UAI 2025 paper}Ang Li}
% Add affiliations after the authors
\affil[1]{%
    Dept. of Computer Science\par
    Florida State University\par
    Tallahassee, FL, USA
}

  
  \begin{document}
\maketitle
\thispagestyle{empty} % 避免过多的空白
\begin{abstract}
% \sloppy
Probabilities of causation play a crucial role in modern decision-making. This paper addresses the challenge of predicting probabilities of causation for subpopulations with insufficient data using machine learning models. Tian and Pearl first defined and derived tight bounds for three fundamental probabilities of causation: the probability of necessity and sufficiency (PNS), the probability of sufficiency (PS), and the probability of necessity (PN). However, estimating these probabilities requires both experimental and observational distributions specific to each subpopulation, which are often unavailable or impractical to obtain with limited population-level data. We assume that the probabilities of causation for each subpopulation are determined by its characteristics. To estimate these probabilities for subpopulations with insufficient data, we propose using machine learning models that draw insights from subpopulations with sufficient data. Our evaluation of multiple machine learning models indicates that, given sufficient population-level data and an appropriate choice of machine learning model and activation function, PNS can be effectively predicted. Through simulation studies, we show that our multilayer perceptron (MLP) model with the Mish activation function achieves a mean absolute error (MAE) of approximately 0.02 in predicting PNS for 32,768 subpopulations using data from around 2,000 subpopulations.
\end{abstract}


\section{Introduction}\label{sec:intro}
Understanding causal relationships and estimating probabilities of causation are crucial in fields such as healthcare, policy evaluation, and economics \citep{pearl2009causality, imbens2015causal, heckman2015causal}. Unlike correlation-based methods, causal inference enables decision-makers to determine whether an action or intervention directly leads to a desired outcome. This is particularly essential in personalized medicine, where accurately assessing treatment effects ensures both efficacy and safety \citep{mueller:pea23-r530}. Moreover, causal reasoning enhances machine learning applications by improving accuracy \citep{li2020training}, interpretability, and fairness \citep{plecko2022causal} in automated decision-making. Despite its broad significance, estimating probabilities of causation remains challenging due to data limitations. In this paper, we address this challenge by leveraging machine learning techniques to predict probabilities of causation for subpopulations with insufficient data.

The study of probabilities of causation began around 2000 when \cite{pearl1999probabilities} first defined three fundamental probabilities—PNS, PS, and PN—within Structural Causal Models \citep{galles1998axiomatic,halpern2000axiomatizing,pearl2009causality}. Subsequently, \cite{tian2000probabilities} derived tight bounds for these probabilities using Balke's linear programming \citep{balke1995probabilistic}, incorporating both observational and experimental data. Nearly two decades later, \cite{li2019unit} formally proved these bounds and introduced the unit selection model, a decision-making framework based on their linear combination. More recently, \cite{li2024unit} extended the definitions and bounds to a more general form. Additionally, \cite{pearl:etal21-r505}, as well as \cite{dawid2017}, demonstrated that these bounds could be further refined given specific causal structures.

However, any above estimation of the probabilities of causation requires both observational and experimental data. Additionally, estimating (sub)populations, based on \cite{li2022probabilities}'s suggestions, requires approximately $1,300$ entries of both data types for each (sub)population, making the process impractical. \cite{li2022learning,li2022unitlearning} demonstrated the potential of machine learning models to achieve accurate estimations for (sub)populations. In this research, we select five diverse machine learning models based on the characteristics of the probabilities of causation. We then evaluate their performance in accomplishing the task.

\subsection{Contributions}
Despite the extensive theoretical research on probabilities of causation, practical estimation methods have remained unexplored. Our work provides the first systematic approach to predicting probabilities of causation using machine learning. Specifically, we make the following contributions:
\begin{itemize}[nosep]
    \item \textbf{First Machine Learning Pipeline for Predicting Probabilities of Causation:} We propose a novel machine learning framework to estimate the bounds of PNS, PS, and PN, filling a critical gap between theoretical causal inference and practical applications.
    \item \textbf{First Accurate Machine Learning Model for PNS Prediction:} We demonstrate that a MLP can accurately predict PNS, proving that machine learning is a feasible and effective tool for estimating probabilities of causation.
    \item \textbf{First Dataset for PNS Bound Prediction:} We construct and release the first synthetic dataset specifically designed to evaluate machine learning models for estimating PNS, providing a foundation for future research in this area.
\end{itemize}
To the best of our knowledge, no prior work has applied machine learning to the problem of predicting probabilities of causation. Our study establishes a new research direction by bridging causal inference and machine learning for practical estimation tasks.

The remainder of the paper is structured as follows: first, we review key causal inference concepts to provide necessary context. Next, we introduce the model and dataset used in our study. Finally, we present our five machine learning models developed for the task. All code for data generation and machine learning models is included in the appendix.

\section{Preliminaries}
\label{related work}
In this section, we review the fundamental concepts of causal inference necessary for understanding the rest of the paper. We begin by discussing the definitions of PNS, PS, and PN as introduced by \cite{pearl1999probabilities}, followed by the definitions of identifiability and the conditions required to identify PNS, PS, and PN \citep{tian2000probabilities}. Additionally, we examine the tight bounds of PNS, PS, and PN in cases where they are unidentifiable \citep{tian2000probabilities}. Readers already familiar with these concepts may skip this section.

Similar to the works mentioned above, we adopt the causal language of Structural Causal Models (SCMs) \citep{galles1998axiomatic,halpern2000axiomatizing}. In this framework, the counterfactual statement ``Variable \( Y \) would have the value \( y \) had \( X \) been \( x \)'' is denoted as \( Y_x = y \), abbreviated as \( y_x \). We consider two types of data: experimental data, expressed as causal effects \( P(y_x) \), and observational data, represented by the joint probability function \( P(x, y) \). Unless otherwise specified, we assume \( X \) and \( Y \) are binary variables in a causal model \( M \), with \( x \) and \( y \) denoting the propositions \( X = \text{true} \) and \( Y = \text{true} \), respectively, and \( x' \) and \( y' \) representing their complements. For simplicity, we focus on binary variables; extensions to multi-valued cases are discussed by \cite{pearl2009causality} (p. 286, footnote 5) and \cite{li2024probabilities}.

First, the definitions of three basic probabilities of causation defined using SCM are as follow \citep{pearl1999probabilities}:

\begin{definition}[Probability of necessity (PN)]
Let $X$ and $Y$ be two binary variables in a causal model $M$, let $x$ and $y$ stand for the propositions $X=true$ and $Y=true$, respectively, and $x'$ and $y'$ for their complements. The probability of necessity is defined as the expression 
\begin{eqnarray}
\text{PN} &\delequal& P(Y_{X=false}=false|X=true,Y=true)\nonumber\\
 &\delequal&  P(y'_{x'}|x,y) \nonumber
\end{eqnarray}
\end{definition}

\begin{definition}[Probability of sufficiency (PS)]
Let $X$ and $Y$ be two binary variables in a causal model $M$, let $x$ and $y$ stand for the propositions $X=true$ and $Y=true$, respectively, and $x'$ and $y'$ for their complements. The probability of sufficiency is defined as the expression
\begin{eqnarray}
\text{PS} &\delequal& P(Y_{X=true}=true|X=false,Y=false)\nonumber\\
&\delequal& P(y_x|x',y') \nonumber
\end{eqnarray}
\end{definition}

\begin{definition}[Probability of necessity and sufficiency (PNS)] Let $X$ and $Y$ be two binary variables in a causal model $M$, let $x$ and $y$ stand for the propositions $X=true$ and $Y=true$, respectively, and $x'$ and $y'$ for their complements. The probability of necessity and sufficiency is defined as the expression
\begin{eqnarray}
\text{PNS} &\delequal& P(Y_{X=true}=true,Y_{X=false}=false)\nonumber\\
&\delequal& P(y_x,y'_{x'}) \nonumber
\end{eqnarray}
\end{definition}
Then, we review the identification conditions for PNS, PS, and PN \citep{tian2000probabilities}.

\begin{definition} (Monotonicity)
A Variable $Y$ is said to be monotonic relative to variable $X$ in a causal model $M$ iff
\begin{eqnarray*}
y'_x\land y_{x'}=\text{false}.
\end{eqnarray*}
\end{definition}

\begin{theorem}
If $Y$ is monotonic relative to $X$, then PNS, PN, and PS are all identifiable, and
\begin{eqnarray*}
PNS = P(y_x) - P(y_{x'}),\\
PN = \frac{P(y) - P(y_{x'})}{P(x,y)},\\
PS = \frac{P(y_x) - P(y)}{P(x', y')}.
\end{eqnarray*}
\end{theorem}

If PNS, PN, and PS are not identifiable, informative bounds are given by \cite{tian2000probabilities}.

% \begin{eqnarray}
% \max \left \{
% \begin{array}{cc}
% 0, \\
% P(y_x) - P(y_{x'}), \\
% P(y) - P(y_{x'}), \\
% P(y_x) - P(y)
% \end{array}
% \right \} \leqslant
% \text{PNS}\leqslant
% \min \left \{
% \begin{array}{cc}
%  P(y_x), \\
%  P(y'_{x'}), \\
% P(x,y) + P(x',y'), \\
% P(y_x) - P(y_{x'}) +\\
% P(x, y') + P(x', y)
% \end{array} 
% \right \}
% \label{pns}
% \end{eqnarray}

% \begin{eqnarray}
% \max \left \{
% \begin{array}{cc}
% 0, \\
% \frac{P(y)-P(y_{x'})}{P(x,y)}
% \end{array} 
% \right \} \leqslant
% \text{PN} \leqslant
% \min \left \{
% \begin{array}{cc}
% 1, \\
% \frac{P(y'_{x'})-P(x',y')}{P(x,y)}
% \end{array}
% \right \}
% \label{pn}
% \end{eqnarray}

% \begin{eqnarray}
% \max \left \{
% \begin{array}{cc}
% 0, \\
% \frac{P(y')-P(y'_{x})}{P(x',y')}
% \end{array} 
% \right \} \leqslant
% \text{PS}\leqslant
% \min \left \{
% \begin{array}{cc}
% 1, \\
% \frac{P(y_{x})-P(x,y)}{P(x',y')}
% \end{array}
% \right \}
% \label{ps}
% \end{eqnarray}

\begin{eqnarray}
\max \left \{
\begin{array}{cc}
0, \\
P(y_x) - P(y_{x'}), \\
P(y) - P(y_{x'}), \\
P(y_x) - P(y)
\end{array}
\right \} \le
\text{PNS}\label{pnslb}\\
\min \left \{
\begin{array}{cc}
 P(y_x), \\
 P(y'_{x'}), \\
P(x,y) + P(x',y'), \\
P(y_x) - P(y_{x'}) +\\
P(x, y') + P(x', y)
\end{array} 
\right \}\ge
\text{PNS}
\label{pnsub}\\
\max \left \{
\begin{array}{cc}
0, \\
\frac{P(y)-P(y_{x'})}{P(x,y)}
\end{array} 
\right \} \le
\text{PN} \label{pnlb}\\
\min \left \{
\begin{array}{cc}
1, \\
\frac{P(y'_{x'})-P(x',y')}{P(x,y)}
\end{array}
\right \}\ge \text{PN}
\label{pnub}\\
\max \left \{
\begin{array}{cc}
0, \\
\frac{P(y')-P(y'_{x})}{P(x',y')}
\end{array} 
\right \} \le
\text{PS} \label{pslb}\\
\min \left \{
\begin{array}{cc}
1, \\
\frac{P(y_{x})-P(x,y)}{P(x',y')}
\end{array}
\right \} \ge \text{PS}
\label{psub}
\end{eqnarray}

Therefore, the primary objective of this paper is then to predict Equations \ref{pnslb} to \ref{psub} (i.e., the lower and upper bounds of the PNS, PS, and PN) for any (sub)populations using those with sufficient data (i.e., sufficient data to estimate the distributions $P(X,Y)$ and $P(Y_X)$.) Due to space constraints, the focus will be on the bounds of PNS (i.e., Equations \ref{pnslb} and \ref{pnsub}). Unless otherwise specified, the discussion will be limited to binary treatment and effect, meaning both $X$ and $Y$ are binary.

\section{Structural Causal Model}
\label{scm}
In general, the equations in SCMs are in implicitly form (e.g., $Z=f_Z(X,Y,U_Z)$). However, in order to verify the accuracy of the learned bounds of PNS, we need to explicitly define the SCM and the data-generating process to determine the true PNS value and its bounds. Followed the setup in \cite{li2022learning}, we will use the following SCM. 
\begin{eqnarray*}
    \begin{cases}
        Z_i &= U_{Z_i} \text{ for } i \in \{1,...,20\},\\
        X&=f_X(M_X,U_X)\\
        &=\begin{cases}
            1& \text{ if } M_X+U_X > 0.5\\
            0& \text{ otherwise, }\\
        \end{cases}\\
        Y&=f_Y(X,M_Y,U_Y)\\
        &=\begin{cases}
            1& \text{ if } 0<C_Y \cdot X+M_Y+U_Y <1 \\
            1& \text{ if } 1<C_Y \cdot X+M_Y+U_Y <2 \\
            0& \text{ otherwise. }\\
        \end{cases}
    \end{cases}
\end{eqnarray*}
where $X,Y,Z_i$ are all binary, $U_{Z_i}, U_X, U_Y$ are binary exogenous variables with Bernoulli distributions, $C_Y$ is a constant, and $M_X, M_Y$ are linear combinations of $Z_i$. The randomly generated value of $C_Y,M_X,M_Y$ and the distributions of $U_X,U_Y,U_{Z_i}$ for the model are provided in the appendix.

\section{Data Generating Process}
Based on the defined model, $20$ binary features are considered (i.e., $Z_1,...,Z_{20}$). We made $15$ observable ($Z_1,...,Z_{15}$) and $5$ unobservable, and the exogenous variables are also unobservable, leading to $2^{15}$ observed subpopulations (i.e., the combination of $Z_1,...,Z_{15}$ defined a subpopulation).

\subsection{Informer Data}  
To evaluate the learned bounds, the informer data must have access to the actual PNS bounds for each subpopulation. Given the explicit form of the SCM and the distributions of all exogenous variables, the PNS bounds, as well as the experimental and observational distributions, can be computed for each combination of the features \( Z_1, \dots, Z_{15} \) (i.e., a subpopulation) using the SCM. For detailed mathematical formulations, refer to the appendix.

\subsection{Sample Collection}
A total of $50,000,000$ experimental and $50,000,000$ observational samples were generated as follows for each sample: In both settings, the exogenous variables \( U_X \), \( U_Y \), and \( U_{Z_i} \) were randomly generated according to their distributions specified in Section \ref{scm}. In the experimental setting, \( X \) was then assigned according to a \( \text{Bernoulli}(0.5) \) distribution, while \( Y \) and \( Z_i \) were computed using the structural functions described in Section \ref{scm}. In the observational setting, \( X \), \( Y \), and \( Z_i \) were all determined by the structural functions. The final datasets include only the observable features \( Z_1, \dots, Z_{15} \), along with \( X \) and \( Y \), while \( Z_{16}, \dots, Z_{20} \) were masked.

\subsection{Data for Machine Learning Models}  
We selected subpopulations from the $2^{15}$ possible groups that contained at least $1,300$ experimental and observational samples ($1,300$ based on \cite{li2022probabilities}'s suggestions). For these selected subpopulations, we computed the experimental and observational distributions and determined the bounds of PNS using Equations \ref{pnslb} and \ref{pnsub}. These results served as the data for our machine learning models (i.e., each data entry consists of 15 features and the PNS bounds as the label.) The obtained data includes $2,054$ entries for the lower bound (LB) and $2,065$ entries for the upper bound (UB) of the PNS.
\section{Machine Learning Prediction}\label{sec:ml}
To evaluate the feasibility of machine learning in predicting the bounds of the PNS, we employed five distinct machine learning models to assess their effectiveness in this task: Support Vector Machine (SVM) \citep{svm}, Random Forest (RF) \citep{rf}, Gradient Boosting Decision Trees (GBDT) \citep{gbdt}, Transformer \citep{vaswani2017attention}, and Multilayer Perceptron (MLP) \citep{mlp}. These models were chosen to represent a diverse range of machine learning paradigms, including kernel-based methods (SVM), ensemble learning techniques (RF and GBDT), and deep learning approaches (MLP and Transformer). This selection ensures a comprehensive evaluation of their ability to approximate causal quantities across different settings. A detailed pipeline is illustrated in Figure \ref{fig:flow chart}.

\begin{figure*}[!htbp] 
    \centering
    \includegraphics[width=\textwidth]{imgs/flow_chart.pdf}
    \caption{Framework for Causal Data Generation and Machine Learning Prediction.}
    \label{fig:flow chart}
\end{figure*}

\begin{figure*}[!htb]
    \centering
    % 第一行:SVM, RF, GBDT, Transformer Lower Bound
    \begin{subfigure}[b]{0.24\linewidth}
        \centering
        \includegraphics[width=\linewidth]{imgs/svm_true_vs_pred_lb.pdf}
        \caption{SVM (Lower bound)}
        \label{fig:svm2}
    \end{subfigure}
    \hfill
    \begin{subfigure}[b]{0.24\linewidth}
        \centering
        \includegraphics[width=\linewidth]{imgs/rf_true_vs_pred_lb.pdf}
        \caption{RF (Lower bound)}
        \label{fig:rf2}
    \end{subfigure}
    \hfill
    \begin{subfigure}[b]{0.24\linewidth}
        \centering
        \includegraphics[width=\linewidth]{imgs/gbdt_true_vs_pred_lb.pdf}
        \caption{GBDT (Lower bound)}
        \label{fig:gbdt2}
    \end{subfigure}
    \hfill
    \begin{subfigure}[b]{0.24\linewidth}
        \centering
        \includegraphics[width=\linewidth]{imgs/transformer_true_vs_pred_lb.pdf}
        \caption{Transformer (Lower bound)}
        \label{fig:transformer2}
    \end{subfigure}

    \vspace{0.3cm} % 调整上下间距

    % 第二行:SVM, RF, GBDT, Transformer Upper Bound
    \begin{subfigure}[b]{0.24\linewidth}
        \centering
        \includegraphics[width=\linewidth]{imgs/svm_true_vs_pred_ub.pdf}
        \caption{SVM (Upper bound)}
        \label{fig:svm3}
    \end{subfigure}
    \hfill
    \begin{subfigure}[b]{0.24\linewidth}
        \centering
        \includegraphics[width=\linewidth]{imgs/rf_true_vs_pred_ub.pdf}
        \caption{RF (Upper bound)}
        \label{fig:rf3}
    \end{subfigure}
    \hfill
    \begin{subfigure}[b]{0.24\linewidth}
        \centering
        \includegraphics[width=\linewidth]{imgs/gbdt_true_vs_pred_ub.pdf}
        \caption{GBDT (Upper bound)}
        \label{fig:gbdt3}
    \end{subfigure}
    \hfill
    \begin{subfigure}[b]{0.24\linewidth}
        \centering
        \includegraphics[width=\linewidth]{imgs/transformer_true_vs_pred_ub.pdf}
        \caption{Transformer (Upper bound)}
        \label{fig:transformer3}
    \end{subfigure}

    \caption{Comparison of true and predicted values across different models for both lower and upper bounds.}
    \label{fig:combined_model_comparison}
\end{figure*}
\subsection{Support Vector Machine}
Support Vector Machines (SVM) \citep{svm} are widely used and well-established supervised learning models. Given their strengths, we selected Support Vector Regression (SVR), a variant of SVM, as the first model for our experiments. To effectively capture complex patterns, we employed the Radial Basis Function (RBF) kernel to map the data into a high-dimensional feature space.

Key hyperparameters include the penalty parameter (\( C \)), the insensitive loss threshold (\( \epsilon \)), and the kernel coefficient (\( \gamma \)). The parameter \( C \) controls the trade-off between model complexity and error tolerance, where larger values may lead to overfitting. The threshold \( \epsilon \) defines the margin of tolerance for errors, while \( \gamma \) determines the influence range of individual data points.

A two-stage hyperparameter tuning strategy was adopted. First, Randomized Search \citep{randomsearch} was employed to efficiently explore the parameter space and identify promising ranges. Then, Grid Search \citep{randomsearch} was used to fine-tune parameters within these ranges. Cross-validation ensured robust generalization throughout the tuning process.

\begin{figure*}[!htb]
    \centering
    % 第一行:SVM, RF, GBDT Lower Bounds
    \begin{subfigure}[b]{0.32\linewidth}
        \centering
        \includegraphics[width=\linewidth]{imgs/confusion_matrix_svm_lb.pdf}
        \caption{SVM Lower bound.}
        \label{fig:svm_lb} % 放在 caption 后
    \end{subfigure}
    \hfill
    \begin{subfigure}[b]{0.32\linewidth}
        \centering
        \includegraphics[width=\linewidth]{imgs/confusion_matrix_rf_lb.pdf}
        \caption{RF Lower bound.}
        \label{fig:rf_lb}
    \end{subfigure}
    \hfill
    \begin{subfigure}[b]{0.32\linewidth}
        \centering
        \includegraphics[width=\linewidth]{imgs/confusion_matrix_gbdt_lb.pdf}
        \caption{GBDT Lower bound.}
        \label{fig:gbdt_lb}
    \end{subfigure}

    % 第二行:SVM, RF, GBDT Upper Bounds
    \begin{subfigure}[b]{0.32\linewidth}
        \centering
        \includegraphics[width=\linewidth]{imgs/confusion_matrix_svm_ub.pdf}
        \caption{SVM Upper bound.}
        \label{fig:svm_ub}
    \end{subfigure}
    \hfill
    \begin{subfigure}[b]{0.32\linewidth}
        \centering
        \includegraphics[width=\linewidth]{imgs/confusion_matrix_rf_ub.pdf}
        \caption{RF Upper bound.}
        \label{fig:rf_ub}
    \end{subfigure}
    \hfill
    \begin{subfigure}[b]{0.32\linewidth}
        \centering
        \includegraphics[width=\linewidth]{imgs/confusion_matrix_gbdt_ub.pdf}
        \caption{GBDT Upper bound.}
        \label{fig:gbdt_ub}
    \end{subfigure}

    % 第三行:Transformer
    \begin{subfigure}[b]{0.32\linewidth}
        \centering
        \includegraphics[width=\linewidth]{imgs/confusion_matrix_transformer_lb.pdf}
        \caption{Transformer Lower bound.}
        \label{fig:transformer_lb}
    \end{subfigure}
    \begin{subfigure}[b]{0.32\linewidth}
        \centering
        \includegraphics[width=\linewidth]{imgs/confusion_matrix_transformer_ub.pdf}
        \caption{Transformer Upper bound.}
        \label{fig:transformer_ub}
    \end{subfigure}

    \caption{Confusion matrices of SVM, RF, GBDT, and Transformer models.}
    \label{fig:confusion_matrices_combined}
\end{figure*}

% \begin{table}[!h]
%     \centering
%     \caption{SVM Result} \label{tab:svm}
%     \begin{tabular}{rll}
%         \toprule % from booktabs package
%         \bfseries Dataset & \bfseries  MSE & \bfseries MAE\\
%         \midrule % from booktabs package
%         Lower bound & 0.0112 & 0.0868\\
%         Upper bound & 0.0304 & 0.1527\\
%         \bottomrule % from booktabs package
%     \end{tabular}
% \end{table}
Finally, the mean squared error (MSE) and mean absolute error (MAE) values of the SVR model can be found in Table \ref{tab:comparison}. Confusion matrices are presented in Figures \ref{fig:svm_lb} and \ref{fig:svm_ub}, while Figures \ref{fig:svm2} and \ref{fig:svm3} provide a clearer comparison with the true PNS bounds. For the prediction of the lower bound, SVR demonstrates reasonable effectiveness; however, for the more complex upper bound, it exhibits a significant decline in accuracy.

\subsection{Random Forest}
Random Forests (RF) \citep{rf} are a widely used ensemble learning method for classification, regression, and other predictive tasks. The core idea behind RF is to construct multiple decision trees during training and aggregate their outputs to enhance overall performance. As an ensemble model, RF exhibits strong robustness, motivating us to assess its effectiveness in predicting PNS bounds.

Key hyperparameters of RF include the number of trees (\( n_{\text{estimators}} \)), maximum tree depth (\( \text{max\_depth} \)), minimum samples required to split a node (\( \text{min\_samples\_split} \)), and the number of features considered for splitting (\( \text{max\_features} \)). Increasing \( n_{\text{estimators}} \) generally improves performance but at the expense of higher computational costs. The parameters \( \text{max\_depth} \), \( \text{min\_samples\_split} \), and \( \text{max\_features} \) regulate tree complexity, balancing bias-variance trade-offs.

For hyperparameter optimization, we employed a two-stage tuning strategy similar to that used for SVM. Table \ref{tab:comparison} also presents RF's MAE and MSE results, while Figures \ref{fig:rf_lb} and \ref{fig:rf_ub} show its confusion matrices. A more direct comparison with true PNS bounds is provided in Figures \ref{fig:rf2} and \ref{fig:rf3}. RF performs comparably to SVM on the lower bound but exhibits significantly higher accuracy on the upper bound.
% \begin{table}[!h]
%     \centering
%     \caption{RF Result} \label{tab:rf}
%     \begin{tabular}{rll}
%         \toprule % from booktabs package
%         \bfseries Dataset & \bfseries  MSE & \bfseries MAE\\
%         \midrule % from booktabs package
%         Lower bound & 0.0116 & 0.0919\\
%         Upper bound & 0.0205 & 0.1242\\
%         \bottomrule % from booktabs package
%     \end{tabular}
% \end{table}
\subsection{Gradient Boosting Decision Trees}
Gradient Boosting Decision Trees (GBDT) \citep{gbdt} is an ensemble learning method that builds models sequentially, with each new tree correcting the errors of its predecessors. Unlike traditional boosting, GBDT optimizes pseudo-residuals, enabling flexible loss function optimization. Simple decision trees serve as weak learners, allowing GBDT to effectively capture complex data patterns.

Key hyperparameters include the number of trees (\( n_{\text{estimators}} \)), learning rate (\( \text{learning\_rate} \)), maximum tree depth (\( \text{max\_depth} \)), and subsample ratio (\( \text{subsample} \)). The learning rate determines each tree’s contribution, while \( n_{\text{estimators}} \) and \( \text{max\_depth} \) regulate model complexity and performance.

Following the approach used for SVM and RF, we applied a two-stage tuning strategy. Again, table \ref{tab:comparison} presents the MSE and MAE results, while Figures \ref{fig:gbdt_lb} and \ref{fig:gbdt_ub} show the confusion matrices. A more direct comparison with true PNS bounds is provided in Figures \ref{fig:gbdt2} and \ref{fig:gbdt3}. GBDT demonstrates moderate performance on both the lower and upper bounds.
% \begin{table}[!h]
%     \centering
%     \caption{GBDT Result} \label{tab:gbdt}
%     \begin{tabular}{rll}
%         \toprule
%         \bfseries Dataset & \bfseries MSE & \bfseries MAE \\
%         \midrule
%         Lower bound & 0.0159 & 0.1049 \\
%         Upper bound & 0.0261 & 0.1399 \\
%         \bottomrule
%     \end{tabular}
% \end{table}
\subsection{Transformer}
The Transformer \citep{vaswani2017attention}, originally developed for Natural Language Processing, has expanded into Computer Vision and become a cornerstone of deep learning, particularly with the rise of large language models. Given its significant impact, this study also evaluates the Transformer for testing.

The model architecture begins with an input layer processing 15-dimensional feature vectors, followed by a linear embedding layer that projects inputs into a 64-dimensional space. Positional encoding is applied to retain feature order information, and two Transformer encoder layers with four attention heads each capture complex feature interactions. The final output is generated through a fully connected layer with a Sigmoid activation function, ensuring predictions remain within the range \([0, 1]\). Key hyperparameters include an embedding dimension of 64, four attention heads, two encoder layers, and a dropout rate of 0.1.

Similarly, table \ref{tab:comparison} presents the MSE and MAE results, while Figures \ref{fig:transformer_lb} and \ref{fig:transformer_ub} show the confusion matrices. A direct comparison with true PNS bounds is provided in Figures \ref{fig:transformer2} and \ref{fig:transformer3}. The Transformer demonstrates strong performance on the lower bound and moderate performance on the upper bound.
% \begin{table}[!h]
%     \centering
%     \caption{Transformer Result} \label{tab:transformer}
%     \begin{tabular}{rll}
%         \toprule
%         \bfseries Dataset & \bfseries MSE & \bfseries MAE \\
%         \midrule
%         Lower bound & 0.0030 & 0.0348 \\
%         Upper bound & 0.0156 & 0.1060 \\
%         \bottomrule
%     \end{tabular}
% \end{table}
\subsection{Multilayer Perceptron}
MLP \citep{mlp} consists of an input layer, one or more hidden layers, and an output layer. With appropriate activation functions, it can effectively model both linear and nonlinear relationships. As a fundamental structure in deep learning, MLP holds significant representativeness, motivating its inclusion in our experiments.

A key consideration for MLP is the choice of activation function, particularly for predicting the lower bound. Since the lower bound of PNS cannot be negative, we initially selected the ReLU \citep{relu} activation function (\ref{equ:relu}). However, ReLU can lead to the loss of negatively correlated features, prompting us to adopt LeakyReLU \citep{lkrelu} (\ref{equ:lkrelu}) as a complementary solution. Furthermore, given the considerable number of zero values in the data, the non-differentiability of ReLU and LeakyReLU at \(s = 0\) imposes limitations on backpropagation. To address this, we proposed using Mish \citep{mish} (\ref{equ:mish}) as an alternative activation function. The corresponding equations are:

\begin{equation}\label{equ:relu}
    \text{ReLU}(s) = \max(0, s)
\end{equation}

\begin{equation}\label{equ:lkrelu}
    \text{LeakyReLU}(s) = 
    \begin{cases}
        s, & \text{if } s \ge 0 \\
        \alpha s, & \text{if } s < 0
    \end{cases}
\end{equation}

\begin{equation}\label{equ:mish}
    \text{Mish}(s) = s \cdot \tanh(\ln(1 + e^s))
\end{equation}

Additionally, we implemented an MLP with the architecture \( 15 \rightarrow 64 \rightarrow 32 \rightarrow 16 \rightarrow 1 \), utilizing ReLU-like functions and Sigmoid as activation functions. The model was optimized using the Adam optimizer with a learning rate of $0.01$ and trained for $1000$ epochs. 

Again, the final results are presented in Table \ref{tab:comparison}. With the Mish activation function, the MLP achieved an MSE of \textbf{0.0011} on the lower bound and \textbf{0.0010} on the upper bound. For MAE, it attained \textbf{0.0225} on the lower bound and \textbf{0.0247} on the upper bound. The confusion matrix is shown in Figure \ref{fig:mlp_comparison}, and a clearer comparison with the true PNS bounds is provided in Figure \ref{fig:mlp2} (Only the best performance comparisons with Mish are shown). 

Overall, MLP significantly outperformed other machine learning models, with Mish yielding the best results among the activation functions. The comparison with the true PNS bounds further confirms that MLP (Mish) provides an accurate and practical model for predicting PNS.
% \begin{table}[!h]
%     \centering
%     \caption{MLP Result} \label{tab:mlp}
%     \begin{tabular}{rlll}
%         \toprule
%         \bfseries Dataset & \bfseries Activation Function & \bfseries MSE & \bfseries MAE \\
%         \midrule
%         Lower bound &ReLU & 0.0045 & 0.0434 \\
%         Upper bound &ReLU & 0.0023 & 0.0357 \\
%         Lower bound &LeakyReLU & 0.0038 & 0.0379 \\
%         Upper bound &LeakyReLU & 0.0024 & 0.0380 \\
%         Lower bound &Mish & \textbf{0.0011} & \textbf{0.0225} \\
%         Upper bound &Mish & \textbf{0.0010} & \textbf{0.0247} \\
%         \bottomrule
%     \end{tabular}
% \end{table}
%%%%%%%%%%%%%%%%%%%%%%%%%%%%
\begin{figure*}[!htb]  % 使用 figure* 跨越两列
    \centering
    % 第一排:ReLU, Leaky ReLU, Mish 的 Lower Bound
    \begin{subfigure}[b]{0.32\linewidth}
        \centering
        \includegraphics[width=\linewidth]{imgs/confusion_matrix_nn_relu_lb.pdf}
        \caption{ReLU (Lower bound)}
    \end{subfigure}
    \hfill
    \begin{subfigure}[b]{0.32\linewidth}
        \centering
        \includegraphics[width=\linewidth]{imgs/confusion_matrix_nn_lkrelu_lb.pdf}
        \caption{Leaky ReLU (Lower bound)}
    \end{subfigure}
    \hfill
    \begin{subfigure}[b]{0.32\linewidth}
        \centering
        \includegraphics[width=\linewidth]{imgs/confusion_matrix_nn_mish_lb.pdf}
        \caption{Mish (Lower bound)}
    \end{subfigure}

    \vspace{0.3cm} % 调整上下间距

    % 第二排:ReLU, Leaky ReLU, Mish 的 Upper Bound
    \begin{subfigure}[b]{0.32\linewidth}
        \centering
        \includegraphics[width=\linewidth]{imgs/confusion_matrix_nn_relu_ub.pdf}
        \caption{ReLU (Upper bound)}
    \end{subfigure}
    \hfill
    \begin{subfigure}[b]{0.32\linewidth}
        \centering
        \includegraphics[width=\linewidth]{imgs/confusion_matrix_nn_lkrelu_ub.pdf}
        \caption{Leaky ReLU (Upper bound)}
    \end{subfigure}
    \hfill
    \begin{subfigure}[b]{0.32\linewidth}
        \centering
        \includegraphics[width=\linewidth]{imgs/confusion_matrix_nn_mish_ub.pdf}
        \caption{Mish (Upper bound)}
    \end{subfigure}

    \caption{Confusion matrices of MLP with different activation functions: ReLU, Leaky ReLU, and Mish for both lower and upper bounds.}
    \label{fig:mlp_comparison}
\end{figure*}

%%%%%%%%%%%%%%%%%%%%%%%%%%%%


\begin{figure}[!htb]
    \centering
    % 左侧:Lower bound
    \begin{subfigure}[b]{0.48\linewidth}
        \centering
        \includegraphics[width=\linewidth]{imgs/mlp_true_vs_pred_lb.pdf}
        \caption{Lower bound.}
        \label{fig:mlp_lb}
    \end{subfigure}
    \hfill
    % 右侧:Upper bound
    \begin{subfigure}[b]{0.48\linewidth}
        \centering
        \includegraphics[width=\linewidth]{imgs/mlp_true_vs_pred_ub.pdf}
        \caption{Upper bound.}
        \label{fig:mlp_ub}
    \end{subfigure}
    
    \caption{Comparison of MLP (Mish) for lower and upper bounds.}
    \label{fig:mlp2}
\end{figure}





\subsection{Experimental Comparison}
% As shown in Table \ref{tab:comparison}, the overall performance of MLP is undoubtedly the best. Among the other four machine learning models, SVM achieves good results on the lower bound but performs the worst on the upper bound, with performance close to complete failure. RF performs significantly better, achieving acceptable results on both the lower and upper bounds. In contrast, GBDT performs worse than RF, despite also being a tree-based model, showing inferior results across the board with only a slight improvement over SVM on the upper bound. The Transformer, as a model based on MLP, performs better than the other machine learning models but still falls short of MLP's performance.

% For MLP models, due to the special characteristics of the dataset around zero, different activation functions exhibit significant performance differences. The basic ReLU exhibits suboptimal performance on the lower bound. Considering negative values, LeakyReLU shows slightly better performance than ReLU on the lower bound. Mish, which not only accounts for negative values but also ensures differentiability around zero, achieves the best results.
As shown in Table \ref{tab:comparison}, MLP delivers the best overall performance. Among the other four machine learning models, SVM performs well on the lower bound but fails almost entirely on the upper bound. RF shows significantly better results, achieving acceptable performance on both bounds. Despite also being a tree-based model, GBDT underperforms compared to RF, with only a slight improvement over SVM on the upper bound. The Transformer, as an MLP-based model, outperforms the other machine learning models but still falls short of MLP’s performance.  

For MLP models, the dataset’s characteristics around zero (we will discuss these characteristics in the discussion section) lead to notable differences in activation function performance. Basic ReLU shows suboptimal performance on the lower bound, while LeakyReLU, which accounts for negative values, performs slightly better. Mish, which not only handles negative values but also ensures differentiability around zero, achieves the best results.

\begin{table}[!ht]
    \centering
-    \caption{Comparison of Model Performance} 
    \label{tab:comparison}
    \begin{tabular}{rlll}
        \toprule
        \bfseries Model & \bfseries Dataset & \bfseries MSE & \bfseries MAE \\
        \midrule
        SVM  & Lower bound & 0.0112 & 0.0868 \\
             & Upper bound & 0.0304 & 0.1527 \\
        RF   & Lower bound & 0.0116 & 0.0919 \\
             & Upper bound & 0.0205 & 0.1242 \\
        GBDT & Lower bound & 0.0159 & 0.1049 \\
             & Upper bound & 0.0261 & 0.1399 \\
        Transformer & Lower bound & 0.0030 & 0.0348 \\
             & Upper bound & 0.0156 & 0.1060 \\
        MLP(ReLU)  & Lower bound & 0.0045 & 0.0434 \\
             & Upper bound & 0.0023 & 0.0357 \\
        MLP(LeakyReLU)  & Lower bound & 0.0038 & 0.0379 \\
             & Upper bound & 0.0024 & 0.0380 \\
        MLP(Mish)  & Lower bound & \textbf{0.0011} & \textbf{0.0225} \\
             & Upper bound & \textbf{0.0010} & \textbf{0.0247} \\
             
        \bottomrule
    \end{tabular}
\end{table}
\section{Discussion and Future Work}\label{sec:discussion}
This paper pioneers the novel approach of selective response, showing that withholding responses can be a powerful tool for GenAI systems. By opting not to answer every query as accurately as it can---particularly when new or complex topics emerge---GenAI can encourage user participation on community-driven platforms and thereby generate more high-quality data for future training. This mechanism ultimately enhances GenAI's long-term performance and revenue. From a welfare perspective, our results indicate that such selective engagement can also benefit users, leading to better solutions and increased overall satisfaction. Since this work is the first to address selective response strategies for GenAI, numerous promising directions remain for future research; we highlight some of them below. 

First, from a technical standpoint, all of the results in this paper rely on Assumption~\ref{assumption: data lip}, involving the lipshitz condition of the accuracy function and the sensitivity parameter $\beta$. Future work could seek to relax this assumption. Furthermore, our constrained optimization approach in Subsection~\ref{sec: welfare constrained revenue maximization} could be extended to approximate the optimal (continuous) strategy instead of the optimal discrete strategy.

Second, our stylized model adopts the simplifying---though unrealistic---assumption that only a single GenAI platform exists. Admittedly, this makes it easier to focus on the idea of selective responses, and indeed, this assumption is pivotal in keeping our analysis tractable. Future research could explore scenarios with multiple GenAI platforms and human-centered forums. In such settings, one platform's selective response might redirect users not only to forums but also to competing GenAI platforms, leading to the tragedy of the commons \cite{hardin1968tragedy}: Although all GenAI platforms benefit from fresh data generation, none may choose to respond selectively if it means losing users to competitors. 

Third, we assumed Forum behaves non-strategically. In reality, human-centered platforms often monetize their data by selling it to GenAI platforms, adding a further layer of strategic interaction for GenAI. Moreover, data transfer between the platforms can form the basis for collaboration: GenAI could employ selective response to bolster Forum content creation, and Forum could, in turn, attribute that content to GenAI for subsequent use in retraining.


%Third, we make the (again) simplifying assumption that Forum is non-strategic. However, in practice, human-centered platforms can sell their data to GenAI platforms. This adds additional considerations for GenAI. Furthermore, data transmission between the platforms can also become the basis for collaboration: GenAI can use selective response to ensure enough content is generated in Forum, and Forum could provide the data attributed to this mechanism back to GenAI. 


%Second, this paper makes the simplifying yet unrealistic assumption of the existence of one GenAI platform. Indeed, this simplifies many aspects and allows us to analyze selective responses. Future work could address the data generation process with more than one GenAI platform and possibly several human-centered forums. In such a case, selective response of one GenAI platform can either drive users to forums or to other GenAI platforms; thus, we might face a tragedy of the commons situation~\ref{hardin1968tragedy}, where all GenAI platforms are interested in fresh data generation but none volunteer to selectively respond and lose users. 

%This paper examines the competition between a generative AI platform and human-based platforms, challenging the assumption that always providing answers is optimal. We analyzed the impact of withholding answers on GenAI's revenue and developed an efficient approximately optimal algorithm for this purpose. We further explored how withholding affects users, showing that it can lead to better outcomes compared to always answering. Specifically, we demonstrated that withholding can Pareto-dominate this strategy and derived the necessary and sufficient conditions for that. Finally, we proposed a second approximately optimal algorithm that maximizes GenAI's revenue while ensuring users are better off than when GenAI answers all queries.

%On a more conceptual level, our model assumes that GenAI’s data comes solely from the competing platform (Forum). Future research could explore a scenario where GenAI can purchase additional data from a third party. This extension could provide valuable insights into the interplay between withholding answers and data purchasing, and whether these two strategies can complement each other or must be traded off.
% \section{Machine Learning Prediction}\label{sec:ml}
To evaluate the feasibility of machine learning in predicting the bounds of the PNS, we employed five distinct machine learning models to assess their effectiveness in this task: Support Vector Machine (SVM) \citep{svm}, Random Forest (RF) \citep{rf}, Gradient Boosting Decision Trees (GBDT) \citep{gbdt}, Transformer \citep{vaswani2017attention}, and Multilayer Perceptron (MLP) \citep{mlp}. These models were chosen to represent a diverse range of machine learning paradigms, including kernel-based methods (SVM), ensemble learning techniques (RF and GBDT), and deep learning approaches (MLP and Transformer). This selection ensures a comprehensive evaluation of their ability to approximate causal quantities across different settings. A detailed pipeline is illustrated in Figure \ref{fig:flow chart}.

\begin{figure*}[!htbp] 
    \centering
    \includegraphics[width=\textwidth]{imgs/flow_chart.pdf}
    \caption{Framework for Causal Data Generation and Machine Learning Prediction.}
    \label{fig:flow chart}
\end{figure*}

\begin{figure*}[!htb]
    \centering
    % 第一行:SVM, RF, GBDT, Transformer Lower Bound
    \begin{subfigure}[b]{0.24\linewidth}
        \centering
        \includegraphics[width=\linewidth]{imgs/svm_true_vs_pred_lb.pdf}
        \caption{SVM (Lower bound)}
        \label{fig:svm2}
    \end{subfigure}
    \hfill
    \begin{subfigure}[b]{0.24\linewidth}
        \centering
        \includegraphics[width=\linewidth]{imgs/rf_true_vs_pred_lb.pdf}
        \caption{RF (Lower bound)}
        \label{fig:rf2}
    \end{subfigure}
    \hfill
    \begin{subfigure}[b]{0.24\linewidth}
        \centering
        \includegraphics[width=\linewidth]{imgs/gbdt_true_vs_pred_lb.pdf}
        \caption{GBDT (Lower bound)}
        \label{fig:gbdt2}
    \end{subfigure}
    \hfill
    \begin{subfigure}[b]{0.24\linewidth}
        \centering
        \includegraphics[width=\linewidth]{imgs/transformer_true_vs_pred_lb.pdf}
        \caption{Transformer (Lower bound)}
        \label{fig:transformer2}
    \end{subfigure}

    \vspace{0.3cm} % 调整上下间距

    % 第二行:SVM, RF, GBDT, Transformer Upper Bound
    \begin{subfigure}[b]{0.24\linewidth}
        \centering
        \includegraphics[width=\linewidth]{imgs/svm_true_vs_pred_ub.pdf}
        \caption{SVM (Upper bound)}
        \label{fig:svm3}
    \end{subfigure}
    \hfill
    \begin{subfigure}[b]{0.24\linewidth}
        \centering
        \includegraphics[width=\linewidth]{imgs/rf_true_vs_pred_ub.pdf}
        \caption{RF (Upper bound)}
        \label{fig:rf3}
    \end{subfigure}
    \hfill
    \begin{subfigure}[b]{0.24\linewidth}
        \centering
        \includegraphics[width=\linewidth]{imgs/gbdt_true_vs_pred_ub.pdf}
        \caption{GBDT (Upper bound)}
        \label{fig:gbdt3}
    \end{subfigure}
    \hfill
    \begin{subfigure}[b]{0.24\linewidth}
        \centering
        \includegraphics[width=\linewidth]{imgs/transformer_true_vs_pred_ub.pdf}
        \caption{Transformer (Upper bound)}
        \label{fig:transformer3}
    \end{subfigure}

    \caption{Comparison of true and predicted values across different models for both lower and upper bounds.}
    \label{fig:combined_model_comparison}
\end{figure*}
\subsection{Support Vector Machine}
Support Vector Machines (SVM) \citep{svm} are widely used and well-established supervised learning models. Given their strengths, we selected Support Vector Regression (SVR), a variant of SVM, as the first model for our experiments. To effectively capture complex patterns, we employed the Radial Basis Function (RBF) kernel to map the data into a high-dimensional feature space.

Key hyperparameters include the penalty parameter (\( C \)), the insensitive loss threshold (\( \epsilon \)), and the kernel coefficient (\( \gamma \)). The parameter \( C \) controls the trade-off between model complexity and error tolerance, where larger values may lead to overfitting. The threshold \( \epsilon \) defines the margin of tolerance for errors, while \( \gamma \) determines the influence range of individual data points.

A two-stage hyperparameter tuning strategy was adopted. First, Randomized Search \citep{randomsearch} was employed to efficiently explore the parameter space and identify promising ranges. Then, Grid Search \citep{randomsearch} was used to fine-tune parameters within these ranges. Cross-validation ensured robust generalization throughout the tuning process.

\begin{figure*}[!htb]
    \centering
    % 第一行:SVM, RF, GBDT Lower Bounds
    \begin{subfigure}[b]{0.32\linewidth}
        \centering
        \includegraphics[width=\linewidth]{imgs/confusion_matrix_svm_lb.pdf}
        \caption{SVM Lower bound.}
        \label{fig:svm_lb} % 放在 caption 后
    \end{subfigure}
    \hfill
    \begin{subfigure}[b]{0.32\linewidth}
        \centering
        \includegraphics[width=\linewidth]{imgs/confusion_matrix_rf_lb.pdf}
        \caption{RF Lower bound.}
        \label{fig:rf_lb}
    \end{subfigure}
    \hfill
    \begin{subfigure}[b]{0.32\linewidth}
        \centering
        \includegraphics[width=\linewidth]{imgs/confusion_matrix_gbdt_lb.pdf}
        \caption{GBDT Lower bound.}
        \label{fig:gbdt_lb}
    \end{subfigure}

    % 第二行:SVM, RF, GBDT Upper Bounds
    \begin{subfigure}[b]{0.32\linewidth}
        \centering
        \includegraphics[width=\linewidth]{imgs/confusion_matrix_svm_ub.pdf}
        \caption{SVM Upper bound.}
        \label{fig:svm_ub}
    \end{subfigure}
    \hfill
    \begin{subfigure}[b]{0.32\linewidth}
        \centering
        \includegraphics[width=\linewidth]{imgs/confusion_matrix_rf_ub.pdf}
        \caption{RF Upper bound.}
        \label{fig:rf_ub}
    \end{subfigure}
    \hfill
    \begin{subfigure}[b]{0.32\linewidth}
        \centering
        \includegraphics[width=\linewidth]{imgs/confusion_matrix_gbdt_ub.pdf}
        \caption{GBDT Upper bound.}
        \label{fig:gbdt_ub}
    \end{subfigure}

    % 第三行:Transformer
    \begin{subfigure}[b]{0.32\linewidth}
        \centering
        \includegraphics[width=\linewidth]{imgs/confusion_matrix_transformer_lb.pdf}
        \caption{Transformer Lower bound.}
        \label{fig:transformer_lb}
    \end{subfigure}
    \begin{subfigure}[b]{0.32\linewidth}
        \centering
        \includegraphics[width=\linewidth]{imgs/confusion_matrix_transformer_ub.pdf}
        \caption{Transformer Upper bound.}
        \label{fig:transformer_ub}
    \end{subfigure}

    \caption{Confusion matrices of SVM, RF, GBDT, and Transformer models.}
    \label{fig:confusion_matrices_combined}
\end{figure*}

% \begin{table}[!h]
%     \centering
%     \caption{SVM Result} \label{tab:svm}
%     \begin{tabular}{rll}
%         \toprule % from booktabs package
%         \bfseries Dataset & \bfseries  MSE & \bfseries MAE\\
%         \midrule % from booktabs package
%         Lower bound & 0.0112 & 0.0868\\
%         Upper bound & 0.0304 & 0.1527\\
%         \bottomrule % from booktabs package
%     \end{tabular}
% \end{table}
Finally, the mean squared error (MSE) and mean absolute error (MAE) values of the SVR model can be found in Table \ref{tab:comparison}. Confusion matrices are presented in Figures \ref{fig:svm_lb} and \ref{fig:svm_ub}, while Figures \ref{fig:svm2} and \ref{fig:svm3} provide a clearer comparison with the true PNS bounds. For the prediction of the lower bound, SVR demonstrates reasonable effectiveness; however, for the more complex upper bound, it exhibits a significant decline in accuracy.

\subsection{Random Forest}
Random Forests (RF) \citep{rf} are a widely used ensemble learning method for classification, regression, and other predictive tasks. The core idea behind RF is to construct multiple decision trees during training and aggregate their outputs to enhance overall performance. As an ensemble model, RF exhibits strong robustness, motivating us to assess its effectiveness in predicting PNS bounds.

Key hyperparameters of RF include the number of trees (\( n_{\text{estimators}} \)), maximum tree depth (\( \text{max\_depth} \)), minimum samples required to split a node (\( \text{min\_samples\_split} \)), and the number of features considered for splitting (\( \text{max\_features} \)). Increasing \( n_{\text{estimators}} \) generally improves performance but at the expense of higher computational costs. The parameters \( \text{max\_depth} \), \( \text{min\_samples\_split} \), and \( \text{max\_features} \) regulate tree complexity, balancing bias-variance trade-offs.

For hyperparameter optimization, we employed a two-stage tuning strategy similar to that used for SVM. Table \ref{tab:comparison} also presents RF's MAE and MSE results, while Figures \ref{fig:rf_lb} and \ref{fig:rf_ub} show its confusion matrices. A more direct comparison with true PNS bounds is provided in Figures \ref{fig:rf2} and \ref{fig:rf3}. RF performs comparably to SVM on the lower bound but exhibits significantly higher accuracy on the upper bound.
% \begin{table}[!h]
%     \centering
%     \caption{RF Result} \label{tab:rf}
%     \begin{tabular}{rll}
%         \toprule % from booktabs package
%         \bfseries Dataset & \bfseries  MSE & \bfseries MAE\\
%         \midrule % from booktabs package
%         Lower bound & 0.0116 & 0.0919\\
%         Upper bound & 0.0205 & 0.1242\\
%         \bottomrule % from booktabs package
%     \end{tabular}
% \end{table}
\subsection{Gradient Boosting Decision Trees}
Gradient Boosting Decision Trees (GBDT) \citep{gbdt} is an ensemble learning method that builds models sequentially, with each new tree correcting the errors of its predecessors. Unlike traditional boosting, GBDT optimizes pseudo-residuals, enabling flexible loss function optimization. Simple decision trees serve as weak learners, allowing GBDT to effectively capture complex data patterns.

Key hyperparameters include the number of trees (\( n_{\text{estimators}} \)), learning rate (\( \text{learning\_rate} \)), maximum tree depth (\( \text{max\_depth} \)), and subsample ratio (\( \text{subsample} \)). The learning rate determines each tree’s contribution, while \( n_{\text{estimators}} \) and \( \text{max\_depth} \) regulate model complexity and performance.

Following the approach used for SVM and RF, we applied a two-stage tuning strategy. Again, table \ref{tab:comparison} presents the MSE and MAE results, while Figures \ref{fig:gbdt_lb} and \ref{fig:gbdt_ub} show the confusion matrices. A more direct comparison with true PNS bounds is provided in Figures \ref{fig:gbdt2} and \ref{fig:gbdt3}. GBDT demonstrates moderate performance on both the lower and upper bounds.
% \begin{table}[!h]
%     \centering
%     \caption{GBDT Result} \label{tab:gbdt}
%     \begin{tabular}{rll}
%         \toprule
%         \bfseries Dataset & \bfseries MSE & \bfseries MAE \\
%         \midrule
%         Lower bound & 0.0159 & 0.1049 \\
%         Upper bound & 0.0261 & 0.1399 \\
%         \bottomrule
%     \end{tabular}
% \end{table}
\subsection{Transformer}
The Transformer \citep{vaswani2017attention}, originally developed for Natural Language Processing, has expanded into Computer Vision and become a cornerstone of deep learning, particularly with the rise of large language models. Given its significant impact, this study also evaluates the Transformer for testing.

The model architecture begins with an input layer processing 15-dimensional feature vectors, followed by a linear embedding layer that projects inputs into a 64-dimensional space. Positional encoding is applied to retain feature order information, and two Transformer encoder layers with four attention heads each capture complex feature interactions. The final output is generated through a fully connected layer with a Sigmoid activation function, ensuring predictions remain within the range \([0, 1]\). Key hyperparameters include an embedding dimension of 64, four attention heads, two encoder layers, and a dropout rate of 0.1.

Similarly, table \ref{tab:comparison} presents the MSE and MAE results, while Figures \ref{fig:transformer_lb} and \ref{fig:transformer_ub} show the confusion matrices. A direct comparison with true PNS bounds is provided in Figures \ref{fig:transformer2} and \ref{fig:transformer3}. The Transformer demonstrates strong performance on the lower bound and moderate performance on the upper bound.
% \begin{table}[!h]
%     \centering
%     \caption{Transformer Result} \label{tab:transformer}
%     \begin{tabular}{rll}
%         \toprule
%         \bfseries Dataset & \bfseries MSE & \bfseries MAE \\
%         \midrule
%         Lower bound & 0.0030 & 0.0348 \\
%         Upper bound & 0.0156 & 0.1060 \\
%         \bottomrule
%     \end{tabular}
% \end{table}
\subsection{Multilayer Perceptron}
MLP \citep{mlp} consists of an input layer, one or more hidden layers, and an output layer. With appropriate activation functions, it can effectively model both linear and nonlinear relationships. As a fundamental structure in deep learning, MLP holds significant representativeness, motivating its inclusion in our experiments.

A key consideration for MLP is the choice of activation function, particularly for predicting the lower bound. Since the lower bound of PNS cannot be negative, we initially selected the ReLU \citep{relu} activation function (\ref{equ:relu}). However, ReLU can lead to the loss of negatively correlated features, prompting us to adopt LeakyReLU \citep{lkrelu} (\ref{equ:lkrelu}) as a complementary solution. Furthermore, given the considerable number of zero values in the data, the non-differentiability of ReLU and LeakyReLU at \(s = 0\) imposes limitations on backpropagation. To address this, we proposed using Mish \citep{mish} (\ref{equ:mish}) as an alternative activation function. The corresponding equations are:

\begin{equation}\label{equ:relu}
    \text{ReLU}(s) = \max(0, s)
\end{equation}

\begin{equation}\label{equ:lkrelu}
    \text{LeakyReLU}(s) = 
    \begin{cases}
        s, & \text{if } s \ge 0 \\
        \alpha s, & \text{if } s < 0
    \end{cases}
\end{equation}

\begin{equation}\label{equ:mish}
    \text{Mish}(s) = s \cdot \tanh(\ln(1 + e^s))
\end{equation}

Additionally, we implemented an MLP with the architecture \( 15 \rightarrow 64 \rightarrow 32 \rightarrow 16 \rightarrow 1 \), utilizing ReLU-like functions and Sigmoid as activation functions. The model was optimized using the Adam optimizer with a learning rate of $0.01$ and trained for $1000$ epochs. 

Again, the final results are presented in Table \ref{tab:comparison}. With the Mish activation function, the MLP achieved an MSE of \textbf{0.0011} on the lower bound and \textbf{0.0010} on the upper bound. For MAE, it attained \textbf{0.0225} on the lower bound and \textbf{0.0247} on the upper bound. The confusion matrix is shown in Figure \ref{fig:mlp_comparison}, and a clearer comparison with the true PNS bounds is provided in Figure \ref{fig:mlp2} (Only the best performance comparisons with Mish are shown). 

Overall, MLP significantly outperformed other machine learning models, with Mish yielding the best results among the activation functions. The comparison with the true PNS bounds further confirms that MLP (Mish) provides an accurate and practical model for predicting PNS.
% \begin{table}[!h]
%     \centering
%     \caption{MLP Result} \label{tab:mlp}
%     \begin{tabular}{rlll}
%         \toprule
%         \bfseries Dataset & \bfseries Activation Function & \bfseries MSE & \bfseries MAE \\
%         \midrule
%         Lower bound &ReLU & 0.0045 & 0.0434 \\
%         Upper bound &ReLU & 0.0023 & 0.0357 \\
%         Lower bound &LeakyReLU & 0.0038 & 0.0379 \\
%         Upper bound &LeakyReLU & 0.0024 & 0.0380 \\
%         Lower bound &Mish & \textbf{0.0011} & \textbf{0.0225} \\
%         Upper bound &Mish & \textbf{0.0010} & \textbf{0.0247} \\
%         \bottomrule
%     \end{tabular}
% \end{table}
%%%%%%%%%%%%%%%%%%%%%%%%%%%%
\begin{figure*}[!htb]  % 使用 figure* 跨越两列
    \centering
    % 第一排:ReLU, Leaky ReLU, Mish 的 Lower Bound
    \begin{subfigure}[b]{0.32\linewidth}
        \centering
        \includegraphics[width=\linewidth]{imgs/confusion_matrix_nn_relu_lb.pdf}
        \caption{ReLU (Lower bound)}
    \end{subfigure}
    \hfill
    \begin{subfigure}[b]{0.32\linewidth}
        \centering
        \includegraphics[width=\linewidth]{imgs/confusion_matrix_nn_lkrelu_lb.pdf}
        \caption{Leaky ReLU (Lower bound)}
    \end{subfigure}
    \hfill
    \begin{subfigure}[b]{0.32\linewidth}
        \centering
        \includegraphics[width=\linewidth]{imgs/confusion_matrix_nn_mish_lb.pdf}
        \caption{Mish (Lower bound)}
    \end{subfigure}

    \vspace{0.3cm} % 调整上下间距

    % 第二排:ReLU, Leaky ReLU, Mish 的 Upper Bound
    \begin{subfigure}[b]{0.32\linewidth}
        \centering
        \includegraphics[width=\linewidth]{imgs/confusion_matrix_nn_relu_ub.pdf}
        \caption{ReLU (Upper bound)}
    \end{subfigure}
    \hfill
    \begin{subfigure}[b]{0.32\linewidth}
        \centering
        \includegraphics[width=\linewidth]{imgs/confusion_matrix_nn_lkrelu_ub.pdf}
        \caption{Leaky ReLU (Upper bound)}
    \end{subfigure}
    \hfill
    \begin{subfigure}[b]{0.32\linewidth}
        \centering
        \includegraphics[width=\linewidth]{imgs/confusion_matrix_nn_mish_ub.pdf}
        \caption{Mish (Upper bound)}
    \end{subfigure}

    \caption{Confusion matrices of MLP with different activation functions: ReLU, Leaky ReLU, and Mish for both lower and upper bounds.}
    \label{fig:mlp_comparison}
\end{figure*}

%%%%%%%%%%%%%%%%%%%%%%%%%%%%


\begin{figure}[!htb]
    \centering
    % 左侧:Lower bound
    \begin{subfigure}[b]{0.48\linewidth}
        \centering
        \includegraphics[width=\linewidth]{imgs/mlp_true_vs_pred_lb.pdf}
        \caption{Lower bound.}
        \label{fig:mlp_lb}
    \end{subfigure}
    \hfill
    % 右侧:Upper bound
    \begin{subfigure}[b]{0.48\linewidth}
        \centering
        \includegraphics[width=\linewidth]{imgs/mlp_true_vs_pred_ub.pdf}
        \caption{Upper bound.}
        \label{fig:mlp_ub}
    \end{subfigure}
    
    \caption{Comparison of MLP (Mish) for lower and upper bounds.}
    \label{fig:mlp2}
\end{figure}





\subsection{Experimental Comparison}
% As shown in Table \ref{tab:comparison}, the overall performance of MLP is undoubtedly the best. Among the other four machine learning models, SVM achieves good results on the lower bound but performs the worst on the upper bound, with performance close to complete failure. RF performs significantly better, achieving acceptable results on both the lower and upper bounds. In contrast, GBDT performs worse than RF, despite also being a tree-based model, showing inferior results across the board with only a slight improvement over SVM on the upper bound. The Transformer, as a model based on MLP, performs better than the other machine learning models but still falls short of MLP's performance.

% For MLP models, due to the special characteristics of the dataset around zero, different activation functions exhibit significant performance differences. The basic ReLU exhibits suboptimal performance on the lower bound. Considering negative values, LeakyReLU shows slightly better performance than ReLU on the lower bound. Mish, which not only accounts for negative values but also ensures differentiability around zero, achieves the best results.
As shown in Table \ref{tab:comparison}, MLP delivers the best overall performance. Among the other four machine learning models, SVM performs well on the lower bound but fails almost entirely on the upper bound. RF shows significantly better results, achieving acceptable performance on both bounds. Despite also being a tree-based model, GBDT underperforms compared to RF, with only a slight improvement over SVM on the upper bound. The Transformer, as an MLP-based model, outperforms the other machine learning models but still falls short of MLP’s performance.  

For MLP models, the dataset’s characteristics around zero (we will discuss these characteristics in the discussion section) lead to notable differences in activation function performance. Basic ReLU shows suboptimal performance on the lower bound, while LeakyReLU, which accounts for negative values, performs slightly better. Mish, which not only handles negative values but also ensures differentiability around zero, achieves the best results.

\begin{table}[!ht]
    \centering
-    \caption{Comparison of Model Performance} 
    \label{tab:comparison}
    \begin{tabular}{rlll}
        \toprule
        \bfseries Model & \bfseries Dataset & \bfseries MSE & \bfseries MAE \\
        \midrule
        SVM  & Lower bound & 0.0112 & 0.0868 \\
             & Upper bound & 0.0304 & 0.1527 \\
        RF   & Lower bound & 0.0116 & 0.0919 \\
             & Upper bound & 0.0205 & 0.1242 \\
        GBDT & Lower bound & 0.0159 & 0.1049 \\
             & Upper bound & 0.0261 & 0.1399 \\
        Transformer & Lower bound & 0.0030 & 0.0348 \\
             & Upper bound & 0.0156 & 0.1060 \\
        MLP(ReLU)  & Lower bound & 0.0045 & 0.0434 \\
             & Upper bound & 0.0023 & 0.0357 \\
        MLP(LeakyReLU)  & Lower bound & 0.0038 & 0.0379 \\
             & Upper bound & 0.0024 & 0.0380 \\
        MLP(Mish)  & Lower bound & \textbf{0.0011} & \textbf{0.0225} \\
             & Upper bound & \textbf{0.0010} & \textbf{0.0247} \\
             
        \bottomrule
    \end{tabular}
\end{table}
% \section{Discussion and Future Work}\label{sec:discussion}
This paper pioneers the novel approach of selective response, showing that withholding responses can be a powerful tool for GenAI systems. By opting not to answer every query as accurately as it can---particularly when new or complex topics emerge---GenAI can encourage user participation on community-driven platforms and thereby generate more high-quality data for future training. This mechanism ultimately enhances GenAI's long-term performance and revenue. From a welfare perspective, our results indicate that such selective engagement can also benefit users, leading to better solutions and increased overall satisfaction. Since this work is the first to address selective response strategies for GenAI, numerous promising directions remain for future research; we highlight some of them below. 

First, from a technical standpoint, all of the results in this paper rely on Assumption~\ref{assumption: data lip}, involving the lipshitz condition of the accuracy function and the sensitivity parameter $\beta$. Future work could seek to relax this assumption. Furthermore, our constrained optimization approach in Subsection~\ref{sec: welfare constrained revenue maximization} could be extended to approximate the optimal (continuous) strategy instead of the optimal discrete strategy.

Second, our stylized model adopts the simplifying---though unrealistic---assumption that only a single GenAI platform exists. Admittedly, this makes it easier to focus on the idea of selective responses, and indeed, this assumption is pivotal in keeping our analysis tractable. Future research could explore scenarios with multiple GenAI platforms and human-centered forums. In such settings, one platform's selective response might redirect users not only to forums but also to competing GenAI platforms, leading to the tragedy of the commons \cite{hardin1968tragedy}: Although all GenAI platforms benefit from fresh data generation, none may choose to respond selectively if it means losing users to competitors. 

Third, we assumed Forum behaves non-strategically. In reality, human-centered platforms often monetize their data by selling it to GenAI platforms, adding a further layer of strategic interaction for GenAI. Moreover, data transfer between the platforms can form the basis for collaboration: GenAI could employ selective response to bolster Forum content creation, and Forum could, in turn, attribute that content to GenAI for subsequent use in retraining.


%Third, we make the (again) simplifying assumption that Forum is non-strategic. However, in practice, human-centered platforms can sell their data to GenAI platforms. This adds additional considerations for GenAI. Furthermore, data transmission between the platforms can also become the basis for collaboration: GenAI can use selective response to ensure enough content is generated in Forum, and Forum could provide the data attributed to this mechanism back to GenAI. 


%Second, this paper makes the simplifying yet unrealistic assumption of the existence of one GenAI platform. Indeed, this simplifies many aspects and allows us to analyze selective responses. Future work could address the data generation process with more than one GenAI platform and possibly several human-centered forums. In such a case, selective response of one GenAI platform can either drive users to forums or to other GenAI platforms; thus, we might face a tragedy of the commons situation~\ref{hardin1968tragedy}, where all GenAI platforms are interested in fresh data generation but none volunteer to selectively respond and lose users. 

%This paper examines the competition between a generative AI platform and human-based platforms, challenging the assumption that always providing answers is optimal. We analyzed the impact of withholding answers on GenAI's revenue and developed an efficient approximately optimal algorithm for this purpose. We further explored how withholding affects users, showing that it can lead to better outcomes compared to always answering. Specifically, we demonstrated that withholding can Pareto-dominate this strategy and derived the necessary and sufficient conditions for that. Finally, we proposed a second approximately optimal algorithm that maximizes GenAI's revenue while ensuring users are better off than when GenAI answers all queries.

%On a more conceptual level, our model assumes that GenAI’s data comes solely from the competing platform (Forum). Future research could explore a scenario where GenAI can purchase additional data from a third party. This extension could provide valuable insights into the interplay between withholding answers and data purchasing, and whether these two strategies can complement each other or must be traded off.

\section{Conclusion}
In this paper, we demonstrated that the bounds of probabilities of causation can be effectively learned and predicted using machine learning models. Specifically, we proposed five different models to predict the bounds of PNS. Experiments showed that an MLP with the Mish activation function achieved a mean absolute error of approximately 0.02 for an SCM with 15 observed and 5 unobserved confounders. Our results suggest that machine learning is a powerful tool for causal inference, particularly in real-world scenarios where direct estimation using SCM formulas is infeasible due to data limitations. Future research will explore larger datasets with more complex SCMs.

Although our study demonstrates the feasibility of machine learning for estimating probabilities of causation, we acknowledge that our experiments are based on synthetic data generated from a structured SCM. Most existing research on probabilities of causation remains theoretical, often without practical validation, despite claims of real-world applicability. Due to page limitations, we could not extend our study to real-world applications, but this remains a critical direction for future research. We believe that bridging this gap will require developing datasets from real-world causal systems where experimental and observational data can be systematically collected. Our work serves as a first step in this direction, providing a foundation for future studies to explore the practical deployment of machine learning models for causality estimation.


% \begin{acknowledgements} % will be removed in pdf for initial submission,
% 						 % (without ‘accepted’ option in \documentclass)
%                          % so you can already fill it to test with the
%                          % ‘accepted’ class option
%     Briefly acknowledge people and organizations here.

%     \emph{All} acknowledgements go in this section.
% \end{acknowledgements}


% \newpage
\clearpage
\begin{thebibliography}{00}

\bibitem{1} Bhide, Nirmal, and Christopher M. Bishop. "Pathophysiology of Non-Dopaminergic Monoamine Systems in Parkinson's Disease: Implications for Mood Dysfunction." InTech EBooks, 2011,  https://doi.org/10.5772/21140.


\bibitem{2} What Is Parkinson’s? — Parkinson’s Foundation. Parkinson’s Foundation. https://www.parkinson.org/understanding-parkinsons/what-is-parkinsons . Accessed 20 Aug 2022

\bibitem{3} Hamzehei, Sahand. "Gateways and Wearable Tools for Monitoring Patient Movements in a Hospital Environment-Webthesis." (2022).

\bibitem{4} Karami, Mostafa. "Machine Learning Algorithms for Radiogenomics: Application to Prediction of the MGMT promoter methylation status in mpMRI scans." PhD diss., Politecnico di Torino, 2022.

\bibitem{5} Andreas Maier, Christopher Syben, Tobias Lasser, Christian Riess, A gentle introduction to deep learning in medical image processing, Zeitschrift für Medizinische Physik, Volume 29, Issue 2,2019,Pages 86-101,ISSN 0939-3889

\bibitem{6} Tsanas A, Little MA, McSharry PE, Ramig LO (2010) Accurate telemonitoring of Parkinson’s disease progression by noninvasive speech tests. IEEE Trans Biomed Eng 57(4):884–93. https://doi.org/10.1109/TBME.2009.2036000. Epub 2009 Nov 20 PMID: 19932995

\bibitem{7} Tin Kam Ho, "Random decision forests," Proceedings of 3rd International Conference on Document Analysis and Recognition, Montreal, QC, Canada, 1995, pp. 278-282 vol.1, doi: 10.1109/ICDAR.1995.598994.

\bibitem{8} Guyon I.; Weston J.; Barnhill S.; Vapnik V. (2002). "Gene selection for cancer classification using support vector machines". Machine Learning. 46 (1–3): 389–422. doi:10.1023/A:1012487302797

\bibitem{9} Wolaver, Dan H. (1991). Phase-Locked Loop Circuit Design. Prentice Hall. p. 211. ISBN 978-0-13-662743-2.

\bibitem{10} Hamzehei, S., Akbarzadeh, O., Attar, H. et al. Predicting the total Unified Parkinson’s Disease Rating Scale (UPDRS) based on ML techniques and cloud-based update. J Cloud Comp 12, 12 (2023). https://doi.org/10.1186/s13677-022-00388-1

\bibitem{11} "Mean Squared Error (MSE)" www.probabilitycourse.com. Retrieved 2020-09-12.

\bibitem{12} Steel, R. G. D.; Torrie, J. H. (1960). Principles and Procedures of Statistics with Special Reference to the Biological Sciences. McGraw Hill.

%%%%%%%%%%% LR
\bibitem{13} A. U. Haq et al., "Feature Selection Based on L1-Norm Support Vector Machine and Effective Recognition System for Parkinson’s Disease Using Voice Recordings," in IEEE Access, vol. 7, pp. 37718-37734, 2019, doi: 10.1109/ACCESS.2019.2906350.

\bibitem{14} A. A. Spadoto, R. C. Guido, F. L. Carnevali, A. F. Pagnin, A. X. Falcão and J. P. Papa, "Improving Parkinson's disease identification through evolutionary-based feature selection," 2011 Annual International Conference of the IEEE Engineering in Medicine and Biology Society, Boston, MA, USA, 2011, pp. 7857-7860, doi: 10.1109/IEMBS.2011.6091936.

\bibitem{15} S. Aich, K. Younga, K. L. Hui, A. A. Al-Absi and M. Sain, "A nonlinear decision tree based classification approach to predict the Parkinson's disease using different feature sets of voice data," 2018 20th International Conference on Advanced Communication Technology (ICACT), Chuncheon, Korea (South), 2018, pp. 638-642, doi: 10.23919/ICACT.2018.8323864.

\bibitem{16} Amato F, Borzì L, Olmo G, Orozco-Arroyave JR. An algorithm for Parkinson's disease speech classification based on isolated words analysis. Health Inf Sci Syst. 2021 Jul 30;9(1):32. doi: 10.1007/s13755-021-00162-8. PMID: 34422258; PMCID: PMC8324609.





\end{thebibliography}


% References
% \bibliography{uai2025-template,ang}
\clearpage
% \newpage

\onecolumn

\title{Supplementary Material}
\maketitle

\appendix
\section{The Causal Model}
The coefficients for \( M_X, M_Y \), and \( C_Y \) were uniformly generated from the range \([-1,1]\), while the parameters of the Bernoulli distribution were uniformly generated from \([0,1]\). The detailed model is as follows:
\begin{eqnarray*}
    &&\begin{cases}
        Z_i &= U_{Z_i} \text{ for } i \in \{1,...,20\},\\
        X&=f_X(M_X,U_X)\\
        &=\begin{cases}
            1& \text{ if } M_X+U_X > 0.5\\
            0& \text{ otherwise, }\\
        \end{cases}\\
        Y&=f_Y(X,M_Y,U_Y)\\
        &=\begin{cases}
            1& \text{ if } 0<C_Y \cdot X+M_Y+U_Y <1 \\
            1& \text{ if } 1<C_Y \cdot X+M_Y+U_Y <2 \\
            0& \text{ otherwise. }\\
        \end{cases}
    \end{cases}\\
&&\text{where, } U_{Z_i}, U_X, U_Y \text{ are binary exogenous variables with Bernoulli distributions.}\\
&&s.t., \\
&M_X& =
\begin{bmatrix}
Z_1~Z_2~...~Z_{20}
\end{bmatrix}\times
\begin{bmatrix}
0.259223510143\\ -0.658140989167\\ -0.75025831768\\ 0.162906462426\\ 0.652023463285\\ -0.0892939586541\\ 0.421469107769\\ -0.443129684766\\ 0.802624388789\\ -0.225740978499\\ 0.716621631717\\ 0.0650682260309\\ -0.220690334026\\ 0.156355773665\\ -0.50693672491\\ -0.707060278115\\ 0.418812816935\\ -0.0822118703986\\ 0.769299853833\\ -0.511585391002
\end{bmatrix},
M_Y =
\begin{bmatrix}
Z_1~Z_2~...~Z_{20}
\end{bmatrix}\times
\begin{bmatrix}
-0.792867111918\\ 0.759967136147\\ 0.55437722369\\ 0.503970540409\\ -0.527187144651\\ 0.378619988091\\ 0.269255196301\\ 0.671597043594\\ 0.396010142274\\ 0.325228576643\\ 0.657808327574\\ 0.801655023993\\ 0.0907679484097\\ -0.0713852594543\\ -0.0691046005285\\ -0.222582013343\\ -0.848408031595\\ -0.584285069026\\ -0.324874831799\\ 0.625621583197
\end{bmatrix}
\end{eqnarray*}
\begin{eqnarray*}
&&U_{Z_1} \sim \text{Bernoulli}(0.352913861526), U_{Z_2} \sim \text{Bernoulli}(0.460995855543),\\
&&U_{Z_3} \sim \text{Bernoulli}(0.331702473392), U_{Z_4} \sim \text{Bernoulli}(0.885505026779),\\
&&U_{Z_5} \sim \text{Bernoulli}(0.017026872706), U_{Z_6} \sim \text{Bernoulli}(0.380772701708),\\
&&U_{Z_7} \sim \text{Bernoulli}(0.028092602705), U_{Z_8} \sim \text{Bernoulli}(0.220819399962),\\
&&U_{Z_9} \sim \text{Bernoulli}(0.617742227477), U_{Z_{10}} \sim \text{Bernoulli}(0.981975046713),\\
&&U_{Z_{11}} \sim \text{Bernoulli}(0.142042291381), U_{Z_{12}} \sim \text{Bernoulli}(0.833602592350),\\
&&U_{Z_{13}} \sim \text{Bernoulli}(0.882938907115), U_{Z_{14}} \sim \text{Bernoulli}(0.542143191999),\\
&&U_{Z_{15}} \sim \text{Bernoulli}(0.085023436884), U_{Z_{16}} \sim \text{Bernoulli}(0.645357252864),\\
&&U_{Z_{17}} \sim \text{Bernoulli}(0.863787135134), U_{Z_{18}} \sim \text{Bernoulli}(0.460539711624),\\
&&U_{Z_{19}} \sim \text{Bernoulli}(0.314014079207), U_{Z_{20}} \sim \text{Bernoulli}(0.685879388218),\\
&&U_{X} \sim \text{Bernoulli}(0.601680857267), U_{Y} \sim \text{Bernoulli}(0.497668975278),\\
&&C_Y=-0.77953605542.
\end{eqnarray*}

\section{Detailed Data Generating Process}
If all 20 binary features are observable, then for a given feature set \( z = (z_1, \dots, z_{20}) \), the values of \( M_X \) and \( M_Y \) are fixed (denoted as \( M_X(z) \) and \( M_Y(z) \)). Under these conditions, the PNS, experimental distribution, and observational distribution corresponding to this set of features are:
\begin{eqnarray*}
PNS(z) &=& P(Y=0_{X=0}, Y=1_{X=1}|z)\\
&=& P(U_Y=0)\cdot T_0 + P(U_Y=1)\cdot T_1, \\
\text{where}, &T_0& =
\left \{
\begin{array}{cc}
1& \text{ if } f_Y(0,M_Y(z),0)=0 \text{ and } f_Y(1,M_Y(z),0)=1, \\
0& \text{otherwize},
\end{array}
\right.\\
&T_1& =
\left \{
\begin{array}{cc}
1& \text{ if } f_Y(0,M_Y(z),1)=0 \text{ and } f_Y(1,M_Y(z),1)=1, \\
0& \text{otherwize.}
\end{array}
\right.
\end{eqnarray*}
\begin{eqnarray*}
&&P(Y=1|do(X),z)\\
&=& P(U_Y=0)\cdot f_Y(X,M_Y(z),0) + P(U_Y=1)\cdot f_Y(X,M_Y(z),1).
\end{eqnarray*}
\begin{eqnarray*}
&&P(Y=1|X,z)\\
&=& P(U_X=0)\cdot P(U_Y=0)\cdot f_Y(f_X(M_X(z),0),M_Y(z),0)+\\
&&P(U_X=0)\cdot P(U_Y=1)\cdot f_Y(f_X(M_X(z),0),M_Y(z),1)) +\\ &&P(U_X=1)\cdot P(U_Y=0)\cdot f_Y(f_X(M_X(z),1),M_Y(z),0)) +\\ &&P(U_X=1)\cdot P(U_Y=1)\cdot f_Y(f_X(M_X(z),1),M_Y(z),1)).
\end{eqnarray*}

% We assumed $15$ of the features are observable (i.e., $Z_1,...,Z_{15}$), which means each subpopulation $c=(z_1,...,z_{15})$ consists $32$ sets of $20$ binary features (i.e., $s_{0}=(z_1,...,z_{15},0,0,0,0,0), s_{1}=(z_1,...,z_{15},0,0,0,0,1), s_{2}=(z_1,...,z_{15},0,0,0,1,0), ...,s_{31}=(z_1,...,z_{15},1,1,1,1,1)$), then we have the PNS, experimental distribution, and observational distribution of all observed subpopulations as follow:
We assume that $15$ of the features are observable (i.e., \( Z_1, \dots, Z_{15} \)). This implies that each subpopulation, denoted as \( c = (z_1, \dots, z_{15}) \), consists of 32 possible sets of $20$ binary features. Specifically, these sets are: $s_{0}=(z_1,...,z_{15},0,0,0,0,0), s_{1}=(z_1,...,z_{15},0,0,0,0,1), s_{2}=(z_1,...,z_{15},0,0,0,1,0), ...,s_{31}=(z_1,...,z_{15},1,1,1,1,1)$.

Under this setup, we obtain the $PNS_{\text{subpopulation}}$, experimental distribution, and observational distribution for any observed subpopulation $c$ as follows:
\begin{eqnarray*}
PNS_{\text{subpopulation}}(c) &=& P(Y=0_{X=0}, Y=1_{X=1}|c)\\
&=& P(s_{0})/P(c)PNS(s_{0})+P(s_{1})/P(c)PNS(s_{1})+\\
&&P(s_{2})/P(c)PNS(s_{2})+...+P(s_{31})/P(c)PNS(s_{31})\\
&=& P(Z_{16}=0)P(Z_{17}=0)P(Z_{18}=0)P(Z_{19}=0)P(Z_{20}=0)PNS(s_{0})+\\
&&P(Z_{16}=0)P(Z_{17}=0)P(Z_{18}=0)P(Z_{19}=0)P(Z_{20}=1)PNS(s_{1})+...+\\
&&P(Z_{16}=1)P(Z_{17}=1)P(Z_{18}=1)P(Z_{19}=1)P(Z_{20}=1)PNS(s_{31}).
\end{eqnarray*}
\begin{eqnarray*}
&&P(Y=1|do(X),c)\\
&=& P(Z_{16}=0)P(Z_{17}=0)P(Z_{18}=0)P(Z_{19}=0)P(Z_{20}=0)P(Y=1|do(X),s_{0})+\\
&& P(Z_{16}=0)P(Z_{17}=0)P(Z_{18}=0)P(Z_{19}=0)P(Z_{20}=1)P(Y=1|do(X),s_{1})+\\
&& P(Z_{16}=0)P(Z_{17}=0)P(Z_{18}=0)P(Z_{19}=1)P(Z_{20}=0)P(Y=1|do(X),s_{2})+...+\\
&& P(Z_{16}=1)P(Z_{17}=1)P(Z_{18}=1)P(Z_{19}=1)P(Z_{20}=1)P(Y=1|do(X),s_{31}).
\end{eqnarray*}
\begin{eqnarray*}
&&P(Y=1|X,c)\\
&=& P(Z_{16}=0)P(Z_{17}=0)P(Z_{18}=0)P(Z_{19}=0)P(Z_{20}=0)P(Y=1|X,s_{0})+\\
&& P(Z_{16}=0)P(Z_{17}=0)P(Z_{18}=0)P(Z_{19}=0)P(Z_{20}=1)P(Y=1|X,s_{1})+\\
&& P(Z_{16}=0)P(Z_{17}=0)P(Z_{18}=0)P(Z_{19}=1)P(Z_{20}=0)P(Y=1|X,s_{2})+...+\\
&& P(Z_{16}=1)P(Z_{17}=1)P(Z_{18}=1)P(Z_{19}=1)P(Z_{20}=1)P(Y=1|X,s_{31}).
\end{eqnarray*}
The true bounds of the $PNS_{\text{subpopulation}}(c)$ can be obtained using Equations \ref{pnslb} and \ref{pnsub}, along with the above observational and experimental distributions.

\section{Code}
All code for data generation and machine learning models is available at the following anonymous link: \url{https://anonymous.4open.science/r/2025uai-ED50/}.
\end{document}
