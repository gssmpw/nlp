\section{Related Work}
\subsection{fMRI-Based Brain Decoding}
% 任务定义
% Brain Visual decoding 目的是从脑活动中还原出受试者得到的对应视觉刺激. Naselaris2011
Brain decoding seeks to reconstruct the visual stimuli perceived by subjects based on their brain activity, offering a deeper understanding of the brain's mechanisms for processing external information~\cite{Naselaris2011}.
% Early
Earlier work~\cite{Horikawa2017} reveals a correlation between Deep Neural Networks (DNNs) image representations and neural activity in the visual cortex using sparse linear regression.
% 生成模型入场: GAN 入场 -> Diffusion + CLIP
With the advent of generative models~\cite{gan, diffusion} and extensive fMRI datasets~\cite{nsd}, visual decoding has shifted towards mapping brain signals to the latent spaces of large models, facilitating the reconstruction of diverse visual stimuli~\cite{gu2022decoding, ozcelik2022reconstruction, Shen2019, gao2023mind, maisurvey, gao2024fmri}.
% Diffusion + CLIP: % 现有数据集缺乏配对数据集,大多依赖于拉到潜空间的策略
This approach has proven effective in utilizing latent diffusion models for image reconstruction~\cite{Lin2022, Takagi2023, maiunibrain, mindeyev1, chen2023seeing}, addressing inter-subject differences by either training separate models for individual subjects or employing partially unified models with subject-specific parameters.
% 我们的优势
% However, such models focus solely on individual latent patterns and, limit their potential to uncover inter-subject neural variability.
However, influenced by neural variability, cross-subject brain signals in the latent space are prone to semantic conflicts, which can lead to convergence at suboptimal points.
% MindAligner introduces a region-level explicit alignment method designed to enhance the interpretation of inter-subject neural variability. 
MindAligner addresses this by leveraging an explicit functional alignment framework across brains, this approach more effectively utilizes shared functionalities among subjects, thereby mitigating semantic conflicts.

% 框架图
\begin{figure*}[!t]
\begin{center}
\centerline{\includegraphics[width=0.97\textwidth]{images/Framework.pdf}}
\vspace{-2mm}
\caption{
Overview of MindAligner. 
To achieve explicit brain functional alignment, given a pre-trained brain decoding model, we design a \textbf{Brain Functional Alignment Module (BFA)} that learns a Brain Transfer Matrix (BTM) $\boldsymbol{\mathcal{M}}$ for fMRI mapping between the known and novel subjects. BTM is decomposed into two low-rank matrices $\boldsymbol{\mathcal{A}}$ and $\boldsymbol{\mathcal{B}}$ to create latent space for further alignment. 
The Cross-Stimulus Neural Mapper is proposed to create fMRI pairs under shared stimuli. In addition to the alignment losses $\boldsymbol{\mathcal{L}}_{rec}$ and $\boldsymbol{\mathcal{L}}_{KL}$ between generated and real fMRI, a latent alignment loss $\boldsymbol{\mathcal{L}}_{latent}$ guides functional alignment based on stimulus similarities.
In the inference stage, only the BTM is utilized for functional mapping, enabling cross-subject brain decoding.
}
\label{fig:framework}
\end{center}
\vskip -0.3in
\end{figure*}




\subsection{Cross-Subject Functional Alignment}
% 文件夹清空, 
% 问题: 为什么要研究跨受试功能对齐
As brains differ both in size and processing mechanisms \cite{nsd, finn2017brain}, the resulting variability in fMRI signals has spurred research into brain alignment methods.
% 促使了功能对齐的产生, 传统功能对齐往往依赖于共享刺激,即需要提供不同受试者在相同视觉刺激下的数据对,通过优化重建损失完成对齐.
The ideal condition for functional alignment methods often depends on \textbf{shared stimuli}, 
requiring paired data from multiple subjects exposed to identical visual inputs, with alignment achieved through reconstruction loss optimization \cite{difumo, rastegarnia2023brain}.
A new paradigm~\cite{Bazeille2019, fgwu, Thual2023, througtheireye} has emerged, enabling explicit alignment by transforming one subject's signal to that of another, thereby preserving functionality and facilitating knowledge transfer across subjects.
To enable cross-subject visual decoding on the Natural Scenes Dataset (NSD)~\cite{nsd}, the largest open-source dataset available, which lacks shared stimuli, 
current methods focus on either anatomical alignment~\cite{Bao2024, mindshot, shenneuro} or functional alignment in latent space~\cite{mindeyev1, mindeyev2, mindbridge, mindtuner}.
Benefiting from a focus on region-level functional differences, functional alignment outperforms anatomical alignment in effectiveness.
MindEye2~\cite{mindeyev2} employs ridge regression to align different subjects into a shared latent space, followed by a shared decoding module.
%, subsequently using the resulting latent vectors in a common module for visual decoding.
MindBridge~\cite{mindbridge} generates pseudo stimuli to create shared stimuli for brain alignment.
% 现有方法存在的问题, %  我们方法的优越性
However, these alignment methods either rely on shared stimuli, restricting their applicability to datasets without such conditions, or utilize latent space alignment, which impedes their ability to uncover inter-subject neural variability.
We introduce an explicit brain functional alignment model to conquer the restriction of the shared stimuli and enhance the interpretation of inter-subject neural variability.
%and design a region-level explicit alignment method to enhance the interpretation of inter-subject neural variability.