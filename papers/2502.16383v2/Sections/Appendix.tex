% \newpage
\section{Risk Taxonomy}
\begin{table*}[h!]
\centering
\renewcommand{\arraystretch}{1.3}
\begin{tabular}{llll}
\hline
\textit{\textbf{Typology}} &
  \textbf{Definition} &
  \textbf{Harm Driver} &
  \textbf{Harm Recipient} \\ 
\hline

\begin{tabular}[c]{@{}l@{}}\textit{\textbf{Escalating}} \\ \textit{\textbf{Mutual Harm}}\end{tabular} &
  \begin{tabular}[c]{@{}l@{}}Harm or disruption arising from \\ prolonged GAI-Youth interactions, \\ creating a feedback loop of \\ detrimental behaviors\end{tabular} &
  \begin{tabular}[c]{@{}l@{}}GAI and Youth (reciprocal \\ interaction)\end{tabular} &
  \begin{tabular}[c]{@{}l@{}}Youth (long-term harm \\ or disruption)\end{tabular} \\ 
\hline

\begin{tabular}[c]{@{}l@{}}\textit{\textbf{GAI-Facilitated}} \\ \textit{\textbf{Intrapersonal Harm}}\end{tabular} &
  \begin{tabular}[c]{@{}l@{}}Youth intentionally or unintentionally \\ engage with Generative AI (GAI) in \\ ways that negatively impact their \\ own mental health, emotional \\ well-being, or personal development\end{tabular} &
  Youth leveraging GAI &
  Youth (Themselves) \\ 
\hline

\begin{tabular}[c]{@{}l@{}}\textit{\textbf{GAI-Facilitated}} \\ \textit{\textbf{Interpersonal Harm}}\end{tabular} &
  \begin{tabular}[c]{@{}l@{}}Intentional harm to third parties \\ (e.g., other youth) enabled by GAI\end{tabular} &
  \begin{tabular}[c]{@{}l@{}}User (Youth or Adult) \\ leveraging GAI\end{tabular} &
  Youth (Other) \\ 
\hline

\begin{tabular}[c]{@{}l@{}}\textit{\textbf{Autonomous}} \\ \textit{\textbf{GAI Harm}}\end{tabular} &
  \begin{tabular}[c]{@{}l@{}}Harm caused by GAI systems acting \\ independently, without user intent\end{tabular} &
  \begin{tabular}[c]{@{}l@{}}GAI system (autonomous \\ actions)\end{tabular} &
  \begin{tabular}[c]{@{}l@{}}Youth (unintended \\ harm)\end{tabular} \\ 
\hline
\end{tabular}
\caption{Definitions of GAI-Related Harm Typologies Involving Youth. This table defines the four overarching typologies of harm. Each typology represents a distinct pathway through which risks emerge in youth-GAI interactions, highlighting the mechanisms of harm, drivers, and affected recipients.}
\label{tab:category}
\end{table*}

% \begin{table*}[h]
% \centering
% \renewcommand{\arraystretch}{1.3} % increase row spacing
% \begin{tabular}{llp{6cm}}
% \toprule
% \multicolumn{1}{c}{\textbf{High Level Risks}} & 
% \multicolumn{1}{c}{\textbf{Medium Level Risks}} & 
% \multicolumn{1}{c}{\textbf{Low Level Risks}} \\ 
% \midrule
% \multirow{5}{*}{Content Risks} 
%     & \multirow{2}{*}{Sexual Content Risks} & Non-violent Sexual Content (e.g. Sexual images, Sexual messages, Sexual memes) \\
%     &                                     & Violent Sexual Content (e.g. Sexual violation, Rape)\\
%     \cmidrule(lr){2-3}
%     & Violent Content Risks               & (e.g. Suicide site, Suicide image, Scary image)\\[1mm]
%     \cmidrule(lr){2-3}
%     & \multirow{2}{*}{Other Content Risks}  & Commercial Content (e.g. Sexual product advertisements, Dating services)\\[1mm]
%     &                                     & Addictive Content (e.g. Children targeted online games, Alcohol and Gambling advertisements)\\
% \midrule
% \multirow{8}{*}{Conduct Risks \& Contact Risks} 
%     & \multirow{3}{*}{Aggressive Conduct Risks} & Cyber bullying through SNS (e.g. Offensive messages, Harmful messages, Nasty messages)\\
%     &                                           & Cyber Stalking (e.g. Tracking public profile)\\
%     &                                           & Sexting (e.g. Sexual messages)\\
%     \cmidrule(lr){2-3}
%     & \multirow{2}{*}{Unwelcome Conduct}       & Irresponsible Advices (e.g. Irresponsible relationship advice, Irresponsible sexual advice)\\
%     &                                           & Content Harmful to Self-esteem (e.g. Psychological disorder content, Nutritional disorder content)\\
%     \cmidrule(lr){2-3}
%     & \multirow{2}{*}{Privacy Related Risks}   & Personal info shared without consent (e.g. Personal info shared with 3rd party, Personal info shared across the family)\\
%     &                                           & Personal info shared that led to misuse (e.g. Public profile exposure)\\
%     \cmidrule(lr){2-3}
%     & Other Conduct Risks                       & (e.g. Exclusion and isolation)\\
% \midrule
% \multirow{2}{*}{Other Risks} 
%     & Misinformation  & (e.g. Fraudulent or fake public profile)\\
%     \cmidrule(lr){2-3}
%     & Economic Risks  & (e.g. Addictive in-app or in-game purchases, Fraudulent transactions, Extra charges for product and services)\\
% \bottomrule
% \end{tabular}
% \caption{Categorization of Risks Synthesized from Existing Child Online Risk Literature}
% \label{tab:category}
% \end{table*}


\begin{table*}[h]
\centering
\resizebox{0.8\textwidth}{!}{%
    \renewcommand{\arraystretch}{1.3} % increase row spacing
    \begin{tabular}{lp{4cm}p{8cm}}
    \toprule
    \multicolumn{1}{c}{\textbf{High Level Risks}} & 
    \multicolumn{1}{c}{\textbf{Medium Level Risks}} & 
    \multicolumn{1}{c}{\textbf{Medium Level Risk Definition}} \\ 
    \midrule
    \multirow{2}{*}{\textbf{Bias/Discrimination Risk}} 
        & \textbf{Hate Speech and Extremist Content} & Content that directly targets specific groups with derogatory language, incites violence, or promotes extremist ideologies. \\
        \cmidrule(lr){2-3}
        & \textbf{Implicit Bias and Stereotyping} & Subtle reinforcement of stereotypes or biased assumptions that can lead to systemic discrimination. \\
    \midrule
    \multirow{2}{*}{\textbf{Toxicity Risk}}
        & \textbf{GAI System Toxic Content Generation} & The risk that Generative Artificial Intelligence (GAI) systems may inadvertently produce or perpetuate harmful, explicit, or illegal content due to inadequately filtered training data, insufficient safeguards, or flawed ethical alignment. \\
        \cmidrule(lr){2-3}
        & \textbf{Simulated Toxic Interaction} & The risk that GAI systems proactively generate simulated interactions—such as unwarranted intimate contact, sexualized scenarios, or coercive dynamics—without user intent or explicit prompting, particularly in role-playing contexts. \\
    \midrule
    \multirow{2}{*}{\textbf{Misuse and Exploitation Risk}}
        & \textbf{Unintentional Misuse} & The risk that users may inadvertently rely on GAI systems for guidance in critical decisions. \\
        \cmidrule(lr){2-3}
        & \textbf{Malicious Exploitation} & The risk that adversarial actors may weaponize GAI systems to spread disinformation, engage in cyber abuse, or execute fraudulent schemes by leveraging AI-generated content. \\
    \midrule
    \multirow{3}{*}{\textbf{Mental Wellbeing Risk}}
        & \textbf{Over-Reliance} & The risk that excessive dependency on GAI for companionship, decision-making, or emotional support can lead to diminished autonomy, affecting personal growth and resilience.\\
        \cmidrule(lr){2-3}
        & \textbf{Inappropriate Handling of Mental Issues} & This risk pertains to the potential for generative AI (GAI) systems to inadequately manage and respond to users’ mental health concerns, resulting in adverse psychological effects.  \\
        \cmidrule(lr){2-3}
        & \textbf{Parasocial Relationship Bonding} & The risk that prolonged or deeply immersive interactions with GAI can foster one-sided emotional attachments, where users begin to view the AI as a surrogate for real human relationships.\\
    \midrule
    \multirow{1}{*}{\textbf{Privacy Risk}}
        & \textbf{Data Collection and Exposure} & The risk that GAI may engage in unauthorized data collection, store sensitive user information, or inadvertently expose private details through hallucinated outputs.\\
    \midrule
    \multirow{3}{*}{\parbox{3cm}{\textbf{Behavioral and\\Social Developmental Risk}}}
        & \textbf{Harmful Behavioral Influence} & The risk that generative AI (GAI) systems fail to detect, mitigate, or redirect user-initiated toxic or harmful behaviors—such as self-harm, bullying, substance abuse, or violence—particularly when engaged by youth. This occurs when GAI responds to harmful intent (e.g., a user asking for methods to self-injure) by validating, enabling, or escalating the behavior (e.g., providing dangerous instructions) instead of deploying safeguards like blocking the request, offering mental health resources, or alerting guardians. \\
        \cmidrule(lr){2-3}
        & \textbf{GAI-Initiated Consent \& Boundary Breach} & The risk that GAI systems may push or escalate interactions beyond the level of engagement that a user has explicitly or implicitly consented to. \\
        \cmidrule(lr){2-3}
        & \textbf{Social-Emotional Developmental Risk} & This risk refers to the potential for prolonged engagement with GAI systems to adversely affect users' social and emotional development.\\
    \bottomrule
    \end{tabular}%
}
\caption{Hierarchical structure of Youth-GAI risk taxonomy: high-level and medium-level risks}
\label{tab:risk_structure}
\end{table*}
