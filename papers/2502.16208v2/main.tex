\documentclass[runningheads]{llncs}
%\usepackage[paperheight=235mm, paperwidth=155mm,textwidth=12.2cm,textheight=19.3cm]{geometry}

%\usepackage{defsFT}

\usepackage[dvipsnames]{xcolor}

\usepackage{xspace}

\usepackage{microtype}%if unwanted, comment out or use option "draft"
\usepackage{todonotes}
\usepackage{amsmath}
\usepackage{amsmath,amsfonts,amssymb} % Math packages
\usepackage{url}
\usepackage{stmaryrd}
\usepackage{mathrsfs}
\usepackage{wasysym}
\usepackage{bm}
\usepackage{alltt}

%\usepackage{enumerate}
\usepackage[inline,shortlabels]{enumitem}
\usepackage{todonotes}
\usepackage{wrapfig}
\usepackage{subcaption}
\usepackage{pgfplots}
\usepackage{multirow}
\usepackage{verbatim}
\usepackage{mathtools}
\usepackage{booktabs}
%\usepackage{marvosym}


%\usepackage{referencing}
%\usepackage{paralist}
\usepackage{extarrows}
\usepackage{mathtools}

\usepackage{algorithm}
\usepackage{algorithmic}
\usepackage{listings}
\usepackage[title]{appendix}
\usepackage{marginnote}
\usepackage{listings}
\usepackage{url}
\usepackage{wrapfig}
\usepackage{graphicx}
\usepackage{colortbl}
\usepackage{pifont}
\usepackage{hhline}
%\usepackage{bbding}
\usepackage[colorlinks=true,citecolor=blue,linkcolor=blue,urlcolor=blue]{hyperref}
\usepackage{graphicx}
\usepackage{adjustbox}
\usepackage[autostyle]{csquotes}


\definecolor{Blueberry}{RGB}{4,51,255}

%% \usepackage{tikz}
%% \usetikzlibrary{positioning,shadows.blur}
%% \usetikzlibrary{arrows,automata}

\usepackage{filecontents}

\usepackage{tikz, tikzscale, pgfplots}
\usetikzlibrary{arrows,arrows.meta,calc,automata,positioning,decorations.pathreplacing,shapes.geometric,shapes.misc,graphs,backgrounds,shadows.blur,snakes}
\tikzset{align at top/.style={baseline=(current bounding box.north)}}
\tikzstyle{every node}=[font=\scriptsize]
%\tikzstyle{state} = [draw,fill=white,circle,thick,align=center,inner sep=0pt,minimum size=4.5mm]
\tikzstyle{state} = [draw,fill=white,ellipse,thick,align=center,inner sep=0pt,minimum size=4.5mm]
\tikzstyle{vvert} = [draw,fill=white,ellipse,thick,align=center,inner sep=-2pt,minimum size=8mm]
\tikzstyle{rvert} = [draw,fill=white,rectangle,thick,align=center,inner sep=3pt,minimum size=7mm]
\tikzstyle{dot} = [fill,circle,inner sep=0mm,minimum size=1.25mm,line width=0mm]

\makeatletter
%\RequirePackage[bookmarks,unicode,colorlinks=true]{hyperref}%
%   \def\@citecolor{blue}%
%   \def\@urlcolor{blue}%
%   \def\@linkcolor{blue}%
\def\UrlFont{\rmfamily}
\def\orcidID#1{\smash{\href{http://orcid.org/#1}{\protect\raisebox{-1.25pt}{\protect\includegraphics{orcid_color.eps}}}}}
\makeatother

\allowdisplaybreaks


%
\setlength\unitlength{1mm}
\newcommand{\twodots}{\mathinner {\ldotp \ldotp}}
% bb font symbols
\newcommand{\Rho}{\mathrm{P}}
\newcommand{\Tau}{\mathrm{T}}

\newfont{\bbb}{msbm10 scaled 700}
\newcommand{\CCC}{\mbox{\bbb C}}

\newfont{\bb}{msbm10 scaled 1100}
\newcommand{\CC}{\mbox{\bb C}}
\newcommand{\PP}{\mbox{\bb P}}
\newcommand{\RR}{\mbox{\bb R}}
\newcommand{\QQ}{\mbox{\bb Q}}
\newcommand{\ZZ}{\mbox{\bb Z}}
\newcommand{\FF}{\mbox{\bb F}}
\newcommand{\GG}{\mbox{\bb G}}
\newcommand{\EE}{\mbox{\bb E}}
\newcommand{\NN}{\mbox{\bb N}}
\newcommand{\KK}{\mbox{\bb K}}
\newcommand{\HH}{\mbox{\bb H}}
\newcommand{\SSS}{\mbox{\bb S}}
\newcommand{\UU}{\mbox{\bb U}}
\newcommand{\VV}{\mbox{\bb V}}


\newcommand{\yy}{\mathbbm{y}}
\newcommand{\xx}{\mathbbm{x}}
\newcommand{\zz}{\mathbbm{z}}
\newcommand{\sss}{\mathbbm{s}}
\newcommand{\rr}{\mathbbm{r}}
\newcommand{\pp}{\mathbbm{p}}
\newcommand{\qq}{\mathbbm{q}}
\newcommand{\ww}{\mathbbm{w}}
\newcommand{\hh}{\mathbbm{h}}
\newcommand{\vvv}{\mathbbm{v}}

% Vectors

\newcommand{\av}{{\bf a}}
\newcommand{\bv}{{\bf b}}
\newcommand{\cv}{{\bf c}}
\newcommand{\dv}{{\bf d}}
\newcommand{\ev}{{\bf e}}
\newcommand{\fv}{{\bf f}}
\newcommand{\gv}{{\bf g}}
\newcommand{\hv}{{\bf h}}
\newcommand{\iv}{{\bf i}}
\newcommand{\jv}{{\bf j}}
\newcommand{\kv}{{\bf k}}
\newcommand{\lv}{{\bf l}}
\newcommand{\mv}{{\bf m}}
\newcommand{\nv}{{\bf n}}
\newcommand{\ov}{{\bf o}}
\newcommand{\pv}{{\bf p}}
\newcommand{\qv}{{\bf q}}
\newcommand{\rv}{{\bf r}}
\newcommand{\sv}{{\bf s}}
\newcommand{\tv}{{\bf t}}
\newcommand{\uv}{{\bf u}}
\newcommand{\wv}{{\bf w}}
\newcommand{\vv}{{\bf v}}
\newcommand{\xv}{{\bf x}}
\newcommand{\yv}{{\bf y}}
\newcommand{\zv}{{\bf z}}
\newcommand{\zerov}{{\bf 0}}
\newcommand{\onev}{{\bf 1}}

% Matrices

\newcommand{\Am}{{\bf A}}
\newcommand{\Bm}{{\bf B}}
\newcommand{\Cm}{{\bf C}}
\newcommand{\Dm}{{\bf D}}
\newcommand{\Em}{{\bf E}}
\newcommand{\Fm}{{\bf F}}
\newcommand{\Gm}{{\bf G}}
\newcommand{\Hm}{{\bf H}}
\newcommand{\Id}{{\bf I}}
\newcommand{\Jm}{{\bf J}}
\newcommand{\Km}{{\bf K}}
\newcommand{\Lm}{{\bf L}}
\newcommand{\Mm}{{\bf M}}
\newcommand{\Nm}{{\bf N}}
\newcommand{\Om}{{\bf O}}
\newcommand{\Pm}{{\bf P}}
\newcommand{\Qm}{{\bf Q}}
\newcommand{\Rm}{{\bf R}}
\newcommand{\Sm}{{\bf S}}
\newcommand{\Tm}{{\bf T}}
\newcommand{\Um}{{\bf U}}
\newcommand{\Wm}{{\bf W}}
\newcommand{\Vm}{{\bf V}}
\newcommand{\Xm}{{\bf X}}
\newcommand{\Ym}{{\bf Y}}
\newcommand{\Zm}{{\bf Z}}

% Calligraphic

\newcommand{\Ac}{{\cal A}}
\newcommand{\Bc}{{\cal B}}
\newcommand{\Cc}{{\cal C}}
\newcommand{\Dc}{{\cal D}}
\newcommand{\Ec}{{\cal E}}
\newcommand{\Fc}{{\cal F}}
\newcommand{\Gc}{{\cal G}}
\newcommand{\Hc}{{\cal H}}
\newcommand{\Ic}{{\cal I}}
\newcommand{\Jc}{{\cal J}}
\newcommand{\Kc}{{\cal K}}
\newcommand{\Lc}{{\cal L}}
\newcommand{\Mc}{{\cal M}}
\newcommand{\Nc}{{\cal N}}
\newcommand{\nc}{{\cal n}}
\newcommand{\Oc}{{\cal O}}
\newcommand{\Pc}{{\cal P}}
\newcommand{\Qc}{{\cal Q}}
\newcommand{\Rc}{{\cal R}}
\newcommand{\Sc}{{\cal S}}
\newcommand{\Tc}{{\cal T}}
\newcommand{\Uc}{{\cal U}}
\newcommand{\Wc}{{\cal W}}
\newcommand{\Vc}{{\cal V}}
\newcommand{\Xc}{{\cal X}}
\newcommand{\Yc}{{\cal Y}}
\newcommand{\Zc}{{\cal Z}}

% Bold greek letters

\newcommand{\alphav}{\hbox{\boldmath$\alpha$}}
\newcommand{\betav}{\hbox{\boldmath$\beta$}}
\newcommand{\gammav}{\hbox{\boldmath$\gamma$}}
\newcommand{\deltav}{\hbox{\boldmath$\delta$}}
\newcommand{\etav}{\hbox{\boldmath$\eta$}}
\newcommand{\lambdav}{\hbox{\boldmath$\lambda$}}
\newcommand{\epsilonv}{\hbox{\boldmath$\epsilon$}}
\newcommand{\nuv}{\hbox{\boldmath$\nu$}}
\newcommand{\muv}{\hbox{\boldmath$\mu$}}
\newcommand{\zetav}{\hbox{\boldmath$\zeta$}}
\newcommand{\phiv}{\hbox{\boldmath$\phi$}}
\newcommand{\psiv}{\hbox{\boldmath$\psi$}}
\newcommand{\thetav}{\hbox{\boldmath$\theta$}}
\newcommand{\tauv}{\hbox{\boldmath$\tau$}}
\newcommand{\omegav}{\hbox{\boldmath$\omega$}}
\newcommand{\xiv}{\hbox{\boldmath$\xi$}}
\newcommand{\sigmav}{\hbox{\boldmath$\sigma$}}
\newcommand{\piv}{\hbox{\boldmath$\pi$}}
\newcommand{\rhov}{\hbox{\boldmath$\rho$}}
\newcommand{\upsilonv}{\hbox{\boldmath$\upsilon$}}

\newcommand{\Gammam}{\hbox{\boldmath$\Gamma$}}
\newcommand{\Lambdam}{\hbox{\boldmath$\Lambda$}}
\newcommand{\Deltam}{\hbox{\boldmath$\Delta$}}
\newcommand{\Sigmam}{\hbox{\boldmath$\Sigma$}}
\newcommand{\Phim}{\hbox{\boldmath$\Phi$}}
\newcommand{\Pim}{\hbox{\boldmath$\Pi$}}
\newcommand{\Psim}{\hbox{\boldmath$\Psi$}}
\newcommand{\Thetam}{\hbox{\boldmath$\Theta$}}
\newcommand{\Omegam}{\hbox{\boldmath$\Omega$}}
\newcommand{\Xim}{\hbox{\boldmath$\Xi$}}


% Sans Serif small case

\newcommand{\Gsf}{{\sf G}}

\newcommand{\asf}{{\sf a}}
\newcommand{\bsf}{{\sf b}}
\newcommand{\csf}{{\sf c}}
\newcommand{\dsf}{{\sf d}}
\newcommand{\esf}{{\sf e}}
\newcommand{\fsf}{{\sf f}}
\newcommand{\gsf}{{\sf g}}
\newcommand{\hsf}{{\sf h}}
\newcommand{\isf}{{\sf i}}
\newcommand{\jsf}{{\sf j}}
\newcommand{\ksf}{{\sf k}}
\newcommand{\lsf}{{\sf l}}
\newcommand{\msf}{{\sf m}}
\newcommand{\nsf}{{\sf n}}
\newcommand{\osf}{{\sf o}}
\newcommand{\psf}{{\sf p}}
\newcommand{\qsf}{{\sf q}}
\newcommand{\rsf}{{\sf r}}
\newcommand{\ssf}{{\sf s}}
\newcommand{\tsf}{{\sf t}}
\newcommand{\usf}{{\sf u}}
\newcommand{\wsf}{{\sf w}}
\newcommand{\vsf}{{\sf v}}
\newcommand{\xsf}{{\sf x}}
\newcommand{\ysf}{{\sf y}}
\newcommand{\zsf}{{\sf z}}


% mixed symbols

\newcommand{\sinc}{{\hbox{sinc}}}
\newcommand{\diag}{{\hbox{diag}}}
\renewcommand{\det}{{\hbox{det}}}
\newcommand{\trace}{{\hbox{tr}}}
\newcommand{\sign}{{\hbox{sign}}}
\renewcommand{\arg}{{\hbox{arg}}}
\newcommand{\var}{{\hbox{var}}}
\newcommand{\cov}{{\hbox{cov}}}
\newcommand{\Ei}{{\rm E}_{\rm i}}
\renewcommand{\Re}{{\rm Re}}
\renewcommand{\Im}{{\rm Im}}
\newcommand{\eqdef}{\stackrel{\Delta}{=}}
\newcommand{\defines}{{\,\,\stackrel{\scriptscriptstyle \bigtriangleup}{=}\,\,}}
\newcommand{\<}{\left\langle}
\renewcommand{\>}{\right\rangle}
\newcommand{\herm}{{\sf H}}
\newcommand{\trasp}{{\sf T}}
\newcommand{\transp}{{\sf T}}
\renewcommand{\vec}{{\rm vec}}
\newcommand{\Psf}{{\sf P}}
\newcommand{\SINR}{{\sf SINR}}
\newcommand{\SNR}{{\sf SNR}}
\newcommand{\MMSE}{{\sf MMSE}}
\newcommand{\REF}{{\RED [REF]}}

% Markov chain
\usepackage{stmaryrd} % for \mkv 
\newcommand{\mkv}{-\!\!\!\!\minuso\!\!\!\!-}

% Colors

\newcommand{\RED}{\color[rgb]{1.00,0.10,0.10}}
\newcommand{\BLUE}{\color[rgb]{0,0,0.90}}
\newcommand{\GREEN}{\color[rgb]{0,0.80,0.20}}

%%%%%%%%%%%%%%%%%%%%%%%%%%%%%%%%%%%%%%%%%%
\usepackage{hyperref}
\hypersetup{
    bookmarks=true,         % show bookmarks bar?
    unicode=false,          % non-Latin characters in AcrobatÕs bookmarks
    pdftoolbar=true,        % show AcrobatÕs toolbar?
    pdfmenubar=true,        % show AcrobatÕs menu?
    pdffitwindow=false,     % window fit to page when opened
    pdfstartview={FitH},    % fits the width of the page to the window
%    pdftitle={My title},    % title
%    pdfauthor={Author},     % author
%    pdfsubject={Subject},   % subject of the document
%    pdfcreator={Creator},   % creator of the document
%    pdfproducer={Producer}, % producer of the document
%    pdfkeywords={keyword1} {key2} {key3}, % list of keywords
    pdfnewwindow=true,      % links in new window
    colorlinks=true,       % false: boxed links; true: colored links
    linkcolor=red,          % color of internal links (change box color with linkbordercolor)
    citecolor=green,        % color of links to bibliography
    filecolor=blue,      % color of file links
    urlcolor=blue           % color of external links
}
%%%%%%%%%%%%%%%%%%%%%%%%%%%%%%%%%%%%%%%%%%%



\urldef{\mailsa}\path|pedro.dargenio@unc.edu.ar|
\urldef{\mailsb}\path|castro@dc.exa.unrc.edu.ar|

\graphicspath{{./Figs/}}%helpful if your graphic files are in another directory.

\title{Polytopal Stochastic Games
  \thanks{This work was supported by
    Agencia {I+D+i} PICT 2019-03134,
    SeCyT-UNC 33620230100384CB (MECANO), and
    EU Grant agreement ID: 101008233 (MISSION).}}

\titlerunning{\vspace{-3cm}Polytopal Stochastic Games}

\author{
Pablo F. Castro \inst{1,3} \orcidID{0000-0002-5835-4333}\and
Pedro R. D'Argenio \inst{2,3} \orcidID{0000-0002-8528-9215}%\and
%Ramiro Demasi \inst{2,3} \orcidID{0000-0003-1651-624X}
%%Luciano Putruele \inst{1,3}{(\Envelope)}  \orcidID{0000-0002-3063-4704} \and 
}
%
\authorrunning{P.F. Castro et al.}
\institute{Universidad Nacional de 
  R\'{\i}o Cuarto, FCEFQyN, Departamento de Computaci\'on,
  R\'{\i}o Cuarto, C\'ordoba,
  Argentina,
  \mailsb
  \and 
  Universidad Nacional de C\'ordoba, FAMAF,
  C\'ordoba,
  Argentina,
  \mailsa
  \and
  Consejo Nacional de Investigaciones Cient\'ificas y T\'ecnicas (CONICET), Argentina}

%\setcounter{tocdepth}{3}
%\tableofcontents
%\listoftodos

\begin{document}
\maketitle

\begin{abstract}
  In this paper we introduce \emph{polytopal stochastic games}, an
  extension of two-player, zero-sum, turn-based stochastic games, in
  which we may have uncertainty over the transition probabilities.  In
  these games the uncertainty over the probability distributions is
  captured via linear (in)equalities whose space of solutions forms a
  polytope.  We give a formal definition of these games and prove
  their basic properties: determinacy and existence of optimal
  memoryless and deterministic strategies.  We do this for
  reachability and different types of reward objectives and show that
  the solution exists in a finite representation of the game.  We also
  state that the corresponding decision problems are in
  $\NP\cap\coNP$.  We motivate the use of polytopal stochastic games
  via a simple example.
%%   Finally, we report some experiments we
%%   performed with a prototype tool.
\end{abstract}

\section{Introduction}

    In the last decades,  stochastic systems have become ubiquitous in computer science: communication and security protocols, fault analysis in critical systems,  autonomous devices,  to name a few examples, typically use techniques coming from probability theory.  Furthermore,  well-known techniques in artificial intelligence, such as reinforcement learning \cite{ReinfLearning}, are based on stochastic models.
    In view of this, the verification and formal analysis of stochastic systems is one of the most active areas of research in software verification.
    Christel Baier and Joost-Pieter Katoen's book \cite{BaierK08} is considered a standard reference in the area,  it introduces common concepts and techniques for model checking probabilistic systems, this includes algorithms for verifying temporal assertions over Markov chains (MCs) and Markov Decision processes (MDPs). The latter can be considered as one player stochastic games, in which the system has to select strategies to solve non-determinism in stochastic settings.  
%\remarkPC{quizas acá también podemos destacar otros trabajos de Baier}
In general,  game theory offers a powerful mathematical framework for specifying and verifying computing systems. The idea is appealing,  a computing system can be thought of as a player playing against an environment, or another system, while trying to achieve certain goals.
%
For instance, a security system can be seen as a player that selects different countermeasures to possibly different types of maneuvers executed by an attacker (a second player) each of which may succeed with certain probabilities.  The objective of the defense system is to minimize the probability that the attack succeeds while the attacker wants to maximize it.
%\remarkPRD{Pablo: fijate si te gusta este otro ejemplo}
%% For instance, consider a communicating protocol,  we may think of it  as a player that wants  to send a message, while the possible faults that may occur during the communication could be thought of as another player.  The objective of the system is to guarantee certain probability of successfully sending the message to the receiver.
%
This scenario can be modeled as a stochastic game, and then analysed using techniques coming from game theory.   Examples of applications of game theory  to the analysis of systems can be found almost everywhere in the last years: self-driving cars~\cite{WangWTGS21},  robotics~\cite{DBLP:conf/qest/Junges0KTZH18},  UAVs~\cite{DBLP:journals/tase/FengWHT16},  security~\cite{DBLP:conf/csfw/AslanyanNP16},  etc.  Furthermore,  in recent years,  some model checkers have been extended  to provide support for stochastic games,  e.g., this is the case of \PRISMGAMES~\cite{DBLP:conf/tacas/ChenFKPS13}, which offers support for several versions of stochastic games. % any other supporting games?

In this paper we focus on two-player, zero-sum,  turn-based perfect-information stochastic games.  Intuitively,  they are non-deterministic probabilistic transition systems in which the vertices are partitioned into two sets:  vertices belonging to player $\maxplay$ and vertices belonging to player $\minplay$.  When the current state belongs to a given player, say $\maxplay$, she performs an action by selecting one of the non-deterministic outgoing transitions which would lead to different states with some given probabilities.
%% In this paper we focus on two-player, zero-sum,  turn-based perfect-information stochastic games.  Intuitively,  they are graph games in which the vertices are partitioned into two sets:  vertices belonging to player 1 and vertices belonging to player 2.  Players' actions are graph transitions, thus the players take turns to move a token from one vertex to another.
Typically,  the players want to fulfill or maximize/minimize some objectives. Standard quantitative objectives are discounted sum (the players collect an amount of rewards during the play which are multiplied by a discount factor in each step), total sum (the players want to maximize/minimize the cumulative sum of the rewards collected during a play), mean-payoff (the objective is to maximize or minimize the long-run average reward), or simply a reachability objective, that is, they aim to  maximize/minimize the probability of reaching certain subset of states. These kinds of objectives can be used and combined to model different kinds of systems,  e.g., the case of a self-driving car intending to maximize the probability of reaching some zone in a city can be seen as a multiobjective game \cite{DBLP:conf/qest/ChenKSW13}.   
     
Most of the time, when modeling stochastic systems, one assumes that the probability distributions are exactly  known,  which may not always be the case due to measurement inaccuracies, lack of data, or other issues.  In this paper we propose an extension of stochastic games that adds the possibility of having uncertainty over the probabilities.  Games with some kinds of uncertainty have been considered for $1\frac{1}{2}$-player games, i.e.,  Markov Decision Processes (MDPs). For instance, Interval-valued Discrete-Time Markov Chains (IDTMCs)~\cite{JonssonL91,KozineU02,SenVA06}, Interval Markov Decision Process (IMDP)~\cite{SenVA06}, and Convex MDPs~\cite{DBLP:conf/cav/PuggelliLSS13}.  To the best of our knowledge,  these approaches  have not been extended to stochastic games (i.e., $2\frac{1}{2}$-player games).  A key challenge for doing so is that in multiplayer games one needs to prove determinacy results,  this ensures that the games possess  a well-defined value,  which does not depend on the players' knowledge.
%  
In the aforementioned approaches the notion of uncertainty is usually adversarially resolved,  that is,  each time a state is visited,  the adversary 
picks a transition distribution that respects the constraints, and takes a probabilistic step according to the chosen distribution. However, it is interesting to note that, in two-player games we may adopt two ways of resolving uncertainty:   a controllable one, in which the actual player resolves the uncertainty following her goals; and an adversarial one in which the adversary resolves the uncertainty in her favor. The former approach  is useful in those scenarios in which the uncertainty affects the adversary as she does not precisely know the possible movements of  our player; while the latter is helpful to reason in worst-case scenarios.

We therefore introduce \emph{polytopal stochastic games (PSG)}. PSGs, as defined in Section \ref{sec:polytopal-games}, allow one to model uncertainties over probability distributions using linear (closed) inequalities.  Geometrically, these linear inequalities correspond to polytopes,  i.e., bounded polyhedra.  As PSGs are two-player games,  both ways of resolving uncertainty are possible: the adversarial approach and the controllable one.  Furthermore, we show that in all the cases  these kinds of games preserve some good properties of standard stochastic games for several objectives: reachability, total rewards, average sum,  and mean payoff.  In particular we show that these games are determined and admit optimal memoryless and deterministic strategies.
%
We also show that these inherently infinite games can be reduced to equivalent finite stochastic games that traverse exclusively through the vertices of the original polytopes.  As such, they are amenable to standard algorithmic solutions.
%
Finally, we prove that the complexity of these games for the aforementioned objectives remain  in $\NP \cap \coNP$, that is,  they stay in the standard complexity class of simple stochastic games, even when polytopal games support for  an uncountable number of actions for the players and the discretization may grow exponentially.

\paragraph{Related work.}
%
Definitions of infinite stochastic games do exist (see, for instance, \cite{Kucera11}) though they are of discrete nature, contrary to the type of games presented here.  In fact,  PSGs are related to IDTMCs~\cite{JonssonL91,KozineU02,SenVA06}, IMDPs~\cite{SenVA06}, and Convex MDPs~\cite{DBLP:conf/cav/PuggelliLSS13},  but they are variants of MDPs and hence they are $1\frac{1}{2}$-games.  In particular, PSGs adopt a semantics similar to IMDPs and Convex MDPs~\cite{DBLP:conf/cav/PuggelliLSS13} in which the uncertainty introduced by the polytope is interpreted as an uncountable non-deterministic branching.
While  in \cite{DBLP:conf/cav/PuggelliLSS13}  interior-point algorithms are used to solve Convex MDPs, we use a discretization through the vertices of polytopes to solve PSGs.  Though this has an exponential impact, this is very mild in practice as we will show later.
%
A much simpler variant of PSG was used in~\cite{CastroDPD23} to provide an algorithmic solution for a fault tolerant measure.  This incipient idea served as the starting point for the generalization presented here.

Somewhat related are the stochastic timed games (STGs)~\cite{BouyerF09,AkshayBKMT16}.  However, the continuous non-determinism introduced by the time in STGs is resolved by uniform and exponential distributions and the remnant non-determinism (resolved by the strategies) is still discrete.  This does not make these models simpler since undecidability has been shown for games with at least 3 clocks~\cite{BouyerF09}.

\paragraph{Outline of the paper.}
%
Section~\ref{sec:roborta} presents a motivating example.  Section~\ref{sec:preliminaries} introduces the background needed for tackling the rest of the paper.  The definition of PSGs, their semantics and basic properties are given in Section~\ref{sec:polytopal-games}.  The main results are presented in Section~\ref{sec:discretazition}.  
%Finally,  we describe some experiments performed with a prototype tool  that extends {\PRISMGAMES} to support polytopal stochastic games, and draw some conclusions. 
Full proofs are gathered in the Appendix.


\section{Roborta vs. Rigoborto in the land of uncertainties} \label{sec:roborta}
%
% This is a Tikz draw for an example of grid
\begin{filecontents}[overwrite]{rob-vs-rig.tikz}
\begin{tikzpicture}
\matrix [nodes=draw, column sep=-0mm]
{
                
    \node (a00) [fill=red! 42.2210925762524 ,minimum size=2cm] { \begin{tabular}{r}    \  \; \ \end{tabular}}; 
     & 
                            
    \node (a01) [fill=red! 37.89772014701512 ,minimum size=2cm]  {\begin{tabular}{c}   \   \;  \  \end{tabular}}; 
     & 
                            
    \node (a02) [fill=red! 21.02857904154225 ,minimum size=2cm]  {\begin{tabular}{c}   \   \;  \  \end{tabular}}; 
     & 
                            
    \node (a03) [fill=red! 12.945837514648167 ,minimum size=2cm]  {\begin{tabular}{c}   \   \;  \  \end{tabular}}; 
     \\ 
    \node (a10) [fill=red! 25.563736068430426 ,minimum size=2cm]  {\begin{tabular}{c}   \   \;  \  \end{tabular}}; 
     & 
                            
    \node (a11) [fill=red! 20.246706872520715 ,minimum size=2cm]  {\begin{tabular}{c}   \   \;  \  \end{tabular}}; 
     & 
                            
    \node (a12) [fill=red! 39.18992945173863 ,minimum size=2cm]  {\begin{tabular}{c}   \   \;  \  \end{tabular}}; 
     & 
                            
    \node (a13) [fill=red! 15.165636303946373 ,minimum size=2cm]  {\begin{tabular}{c}   \   \;  \  \end{tabular}}; 
     \\ 
    \node (a20) [fill=red! 23.82984770761779 ,minimum size=2cm]  {\begin{tabular}{c}   \   \;  \  \end{tabular}}; 
     & 
                            
    \node (a21) [fill=red! 29.16910197275156 ,minimum size=2cm]  {\begin{tabular}{c}   \   \;  \  \end{tabular}}; 
     & 
                            
    \node (a22) [fill=red! 45.40564425976676 ,minimum size=2cm]  {\begin{tabular}{c}   \   \;  \  \end{tabular}}; 
     & 
                            
    \node (a23) [fill=red! 25.234342790869512 ,minimum size=2cm]  {\begin{tabular}{c}   \   \;  \  \end{tabular}}; 
     \\ 
    \node (a30) [fill=red! 14.09189221998519 ,minimum size=2cm]  {\begin{tabular}{c}   \   \;  \  \end{tabular}}; 
     & 
                            
    \node (a31) [fill=red! 37.79021020786119 ,minimum size=2cm]  {\begin{tabular}{c}   \   \;  \  \end{tabular}}; 
     & 
                            
    \node (a32) [fill=red! 30.91844983376658 ,minimum size=2cm]  {\begin{tabular}{c}   \   \;  \  \end{tabular}}; 
     & 
                            
    \node (a33) [fill=red! 12.525317068122026 ,minimum size=2cm]  {\begin{tabular}{c}   \   \  \end{tabular}}; 
     \\ 
    };
    %% \node (rigoborto) [below of =  a33, node distance=0.7in] {\begin{tabular}{c}   \   \includegraphics[scale=0.3]{pics/rigoborto.png}  \  \end{tabular}};
    %% \node (roborta) [above of =  a00, node distance=0.7in] {\begin{tabular}{c}   \   \includegraphics[scale=0.3]{pics/roborta.png}  \  \end{tabular}};
    \draw[line width=1ex,black!47.90153792160812,->] ([xshift=-5pt]a00.center) -- (a00.west); 
    \draw[line width=1ex,black!61.00556540260447,->] ([xshift=5pt]a01.center) -- (a01.east); 
    \draw[line width=1ex,black!9.739860767117857,->] ([xshift=5pt]a02.center) -- (a02.east); 
    \draw[line width=1ex,black!97.19165996719622,->] ([xshift=-5pt]a03.center) -- (a03.west); 
    \draw[line width=1ex,black!43.94093728079083,->] ([xshift=5pt]a10.center) -- (a10.east); 
    \draw[line width=1ex,black!20.23529155514625,->] ([xshift=-5pt]a11.center) -- (a11.west); 
    \draw[line width=1ex,black!64.9689954296466,->] ([xshift=5pt]a12.center) -- (a12.east); 
    \draw[line width=1ex,black!33.63064024637017,->] ([xshift=5pt]a13.center) -- (a13.east); 
    \draw[line width=1ex,black!99.77143613711435,->] ([xshift=-5pt]a20.center) -- (a20.west); 
    \draw[line width=1ex,black!1.2844267069350712,->] ([xshift=-5pt]a21.center) -- (a21.west); 
    \draw[line width=1ex,black!73.52055509855617,->] ([xshift=5pt]a22.center) -- (a22.east); 
    \draw[line width=1ex,black!51.2178246225736,->] ([xshift=-5pt]a23.center) -- (a23.west); 
    \draw[line width=1ex,black!34.95912745052199,->] ([xshift=-5pt]a30.center) -- (a30.west); 
    \draw[line width=1ex,black!74.09424642173093,->] ([xshift=5pt]a31.center) -- (a31.east); 
    \draw[line width=1ex,black!61.78658169952189,->] ([xshift=-5pt]a32.center) -- (a32.west); 
    \draw[line width=1ex,black!13.502148124134372,->] ([xshift=5pt]a33.center) -- (a33.east); 
    \draw[line width=1ex,black!81.94925119364802,->] ([yshift=5pt]a00.center) -- (a00.north);  
    
    \draw[line width=1ex,black!96.55709520753062,->] ([yshift=5pt]a01.center) -- (a01.north);  
    
    \draw[line width=1ex,black!62.04344719931791,->] ([yshift=5pt]a02.center) -- (a02.north);  
    
    \draw[line width=1ex,black!80.43319008791654,->] ([yshift=5pt]a03.center) -- (a03.north);  
    
    \draw[line width=1ex,black!37.970486136133474,->] ([yshift=-5pt]a10.center) -- (a10.south); 
                    
    \draw[line width=1ex,black!45.96634965202573,->] ([yshift=5pt]a11.center) -- (a11.north);  
    
    \draw[line width=1ex,black!79.7676575935987,->] ([yshift=5pt]a12.center) -- (a12.north);  
    
    \draw[line width=1ex,black!36.796786383088254,->] ([yshift=5pt]a13.center) -- (a13.north);  
    
    \draw[line width=1ex,black!5.5714569094573285,->] ([yshift=-5pt]a20.center) -- (a20.south); 
                    
    \draw[line width=1ex,black!79.85975838632685,->] ([yshift=-5pt]a21.center) -- (a21.south); 
                    
    \draw[line width=1ex,black!13.165632909243264,->] ([yshift=-5pt]a22.center) -- (a22.south); 
                    
    \draw[line width=1ex,black!22.177394688760323,->] ([yshift=5pt]a23.center) -- (a23.north);  
    
    \draw[line width=1ex,black!82.60221064757964,->] ([yshift=5pt]a30.center) -- (a30.north);  
    
    \draw[line width=1ex,black!93.32127355415176,->] ([yshift=5pt]a31.center) -- (a31.north);  
    
    \draw[line width=1ex,black!4.598044689456593,->] ([yshift=-5pt]a32.center) -- (a32.south); 
                    
    \draw[line width=1ex,black!73.06198555432802,->] ([yshift=5pt]a33.center) -- (a33.north);  
    
    \node (rigoborto) [below of =  a23, node distance=.88in, xshift=-.18in] {\begin{tabular}{c}   \   \includegraphics[scale=0.3]{pics/rigoborto.png}  \  \end{tabular}};
    \node (roborta) [above of =  a10, node distance=0.7in, xshift=.18in] {\begin{tabular}{c}   \   \includegraphics[scale=0.3]{pics/roborta.png}  \  \end{tabular}};

\end{tikzpicture}
\end{filecontents}
%
%
\begin{wrapfigure}[15]{r}{.40\textwidth}
\vspace{-.8cm}
\resizebox{.4\textwidth}{!}{\input{rob-vs-rig.tikz}\unskip}
\vspace*{-.5cm}
\caption{An example of a grid for the Roborta vs Rigoborto game.}\label{fig:rob-vs-rig}
\end{wrapfigure}     
%
We illustrate our approach by means of a simple example.  Consider a field represented as a bidimensional grid and two robots --which we call Roborta and Rigoborto-- that navigate it.   Roborta can move sideways and forward,  Rigoborto can move sideways and backward. The robots start at a certain initial  position.  Roborta intends to reach the end of the grid, i.e., she wants to reach position $(i,n+1)$ for any $i$, whereas Rigoborto wants to stop Roborta.  He can achieve this by reaching Roborta's location.
The robots play in turns.
%% The robots play in turns (i.e., it is a turn-based game).
%
The objective of Roborta is to maximize the probability of reaching the exit, while the objective of Rigoborto is to minimize this value.
We spice up this example by considering the terrain quality (which depends on factors like,  e.g., stones, mud or grass) and slope,  which may cause imprecisions and uncertainties in the robots mobility, probably making them slide towards some undesired direction.  The terrain quality and slope may vary in each grid position.  In Fig.~\ref{fig:rob-vs-rig}, we show an example of such a scenario.  Therein, the robots start at the corners,  the arrows indicate the slopes in the terrain, and the colors in the cells indicate the terrain quality.  Darker arrows correspond to sharper slopes.  Similarly, cells with lower quality are colored with stronger red colors.

More precisely, for each $(x,y)$-cell, the terrain quality $q_{xy}\in[0,0.5]$ gives the uncertainty factor, where $q_{xy}=0$ means that probabilities are completely determined, and, as $q_{xy}$ grows, the probability values become increasingly fuzzier.   In addition, we consider two factors associated with the terrain slopes: $l_{xy},f_{xy}\in[-1,1]$, representing the inclination of the lateral and frontal slopes respectively.  Thus, as $l_{xy}$ get closer to $1$ ($-1$), the likelihood of shifting to the right (left) increases, with $l_{xy}=0$ not favouring any particular side.  Similarly $f_{xy}>1$ ($f_{xy}<-1$) biases the robot towards the front (back).
%
Let $p_c$ be the probability that the robot command is successful (that is, that it moves in the intended direction), and let $p_l$, $p_r$, $p_f$, and $p_b$ be the probabilities that the command is unsuccessful and the robot uncontrollably slides respectively to the left, right, front, and back.  Then, the space of all probability values can be defined by the following set of inequalities:
%
\begin{align*}
  1   & \ = \    p_c + p_l + p_r + p_f + p_b \\
  p_c & \ \geq \ 0,\ \ \ p_l \ \geq \ 0,\ \ \ p_r \ \geq \ 0,\ \ \ p_f \ \geq \ 0,\ \ \ p_b \ \geq \ 0 \\
  p_c & \ \leq \ 1-(q_{xy}+{\textstyle\frac{1}{2}}\cdot(1-(1-|l_{xy}|) \cdot (1-|f_{xy}|))) \\
  0   & \ \leq \ (1-\max(0,-l_{xy})) \cdot p_l-(1-q_{xy}) \cdot (1-\max(0,l_{xy})) \cdot p_r \\
  0   & \ \leq \ (1-\max(0,l_{xy})) \cdot p_r-(1-q_{xy}) \cdot (1-\max(0,-l_{xy})) \cdot p_l \\
  0   & \ \leq \ (1-\max(0,f_{xy})) \cdot p_f-(1-q_{xy}) \cdot (1-\max(0,-f_{xy})) \cdot p_b \\
  0   & \ \leq \ (1-\max(0,-f_{xy})) \cdot p_b-(1-q_{xy}) \cdot (1-\max(0,f_{xy})) \cdot p_f
\end{align*}
%
Note that if $q_{xy}=0$, the system has a unique solution. If, in addition, $l_{xy}>0$, $1/(1-l_{xy})=p_r/p_l$ giving the likelihood ratio of sliding towards the right.

%
\begin{figure}[t]
%\vspace*{-0.5cm}
\begin{subfigure}{\textwidth}
\lstset{style=polyprism}
\begin{lstlisting}{language=Java}
// action specification for Roborta moving to the left
[robl] (turn = 0) & (roby<L) & !Collision -> (rob_mov'=1) & (turn'=1)$\medskip$
[robl-cont] (turn = 1) & (rob_mov = 1) ->
  //The first four probabilistic options  correspond to environments setbacks
  pl : (robx'=max(0,robx-1)) & (rob_mov'=0) + pr : (robx'=min(W-1,robx+1)) & (rob_mov'=0) 
  + pf : (roby'=roby+1) & (rob_mov'=0) + pb : (roby'=max(0,roby+1)) & (rob_mov'=0)
  + pc : (robx'=max(0,robx-1)) & (rob_mov'=0) 
  {// inequations for uncertainty
    1-(Q[robx,roby]+(1-(1-abs(L[robx,roby]))*(1-abs(F[robx,roby])))/2) >= pc,
    (1-max(0,-L[robx,roby]))*pl - (1-Q[robx,roby])*(1-max(0,L[robx,roby]))*pr >= 0,
    (1-max(0,L[robx,roby]))*pr - (1-Q[robx,roby])*(1-max(0,-L[robx,roby]))*pl >= 0,
    (1-max(0,F[robx,roby]))*pf - (1-Q[robx,roby])*(1-max(0,-F[robx,roby]))*pb >= 0,
    (1-max(0,-F[robx,roby]))*pb - (1-Q[robx,roby])*(1-max(0,F[robx,roby]))*pf >= 0  
  };
\end{lstlisting}
\vspace*{-0.3cm}
\caption{Roborta moves left}\label{fig:roborta-code}
\end{subfigure}

\smallskip

\begin{subfigure}{\textwidth}
\lstset{style=polyprism}
\begin{lstlisting}{language=Java}
// action specification for Rigoborto moving to the left
[rigl] (turn = 1) & (rob_mov = 0) & (rigy<L) & !Collision ->
  //The first four probabilistic options  correspond to environments setbacks
  pl : (rigx'=max(0,rigx-1)) & (turn'=0) & (Collision'=(robx=rigx && roby=rigy))
  + pr : (rigx'=min(W-1,rigx+1)) & (turn'=0) & (Collision'=(robx=rigx && roby=rigy))
  + pf : (rigy'=rigy+1) & (turn'=0) & (Collision'=(robx=rigx && roby=rigy))
  + pb : (rigy'=max(0,rigy+1)) & (turn'=0) & (Collision'=(robx=rigx && roby=rigy))
  + pc : (rigx'=max(0,rigx-1)) & (turn'=0) & (Collision'=(robx=rigx && roby=rigy)) 
  {// inequations for uncertainty
    1-(Q[rigx,rigy]+(1-(1-abs(L[rigx,rigy]))*(1-abs(F[rigx,rigy])))/2) >= pc,
    (1-max(0,-L[rigx,rigy]))*pl - (1-Q[rigx,rigy])*(1-max(0,L[rigx,rigy]))*pr >= 0,
    (1-max(0,L[rigx,rigy]))*pr - (1-Q[rigx,rigy])*(1-max(0,-L[rigx,rigy]))*pl >= 0,
    (1-max(0,F[rigx,rigy]))*pf - (1-Q[rigx,rigy])*(1-max(0,-F[rigx,rigy]))*pb >= 0,
    (1-max(0,-F[rigx,rigy]))*pb - (1-Q[rigx,rigy])*(1-max(0,F[rigx,rigy]))*pf >= 0  
  };
\end{lstlisting}
\vspace*{-0.3cm}
\caption{Rigoborto moves left}\label{fig:rigoborto-code}
\end{subfigure}
%\vspace*{-0.3cm}
\caption{Fragment of code for Roborta vs Rigoborto}\label{fig:robvsrig-code}
%\vspace*{-0.2cm}
\end{figure}


Our aim is to find the best strategy for Roborta to win against all
odds.  This implies that the terrain uncertainty behaves adversarially
to Roborta but favourably to Rigoborto.  Thus,  in our model,  
Rigoborto controls the non-determinism introduced by the terrain
uncertainty.
%
Assuming an extension of the \PRISMGAMES language, the code could
look like in Fig.~\ref{fig:robvsrig-code}, where
subfigures~\ref{fig:roborta-code} and~\ref{fig:rigoborto-code} show
the decisions to move left by Roborta and Rigoborto respectively.

Variable \verb"turn" indicates who is the next player to move (with
\verb"0" for Roborta and \verb"1" for Rigoborto).  If it is Roborta's
turn (see first line in Fig.~\ref{fig:roborta-code}) and she decides
to move left, she indicates it by setting \verb"rob_mov'=1" (\verb"1"
indicates a left move while \verb"2", \verb"3", and \verb"4" are used
for the other directions, and \verb"0" to indicate that Roborta is not
moving).  At the same time, she yields her turn by setting
\verb"turn'=1".
%
Notice that the action is not yet complete: the reaction of the
terrain to the move is encoded in the next line (action
\verb"robl-cont" in Fig.~\ref{fig:roborta-code}).  Notice that this
action happens in a state in which \verb"turn=1", making the terrain
uncertainty --defined by the polytope-- adversarial to Roborta.
%
Here, variables \verb"robx" and \verb"roby" correspond to Roborta's
coordinates $x$ and $y$ and constant matrices \verb"Q", \verb"L", and
\verb"F" contain the respective values for $q_{xy}$, $l_{xy}$, and
$f_{xy}$. The rest of the variables are as expected.  Once this step is
taken, variable \verb"rob_mov" is set to \verb"0", thus enabling
Rigoborto's move.
%
Rigoborto's decision to move left is given in
Fig~\ref{fig:rigoborto-code}.  Notice that this is performed in a
single action since we assume that the terrain uncertainty plays in
favour of him.  Something particular to this transition is the setting
of variable \verb"Collision" to indicate whether Rigoborto has
caught Roborta.









%% We propose an extension of {\PRISMGAMES} language to represent this type of games which includes notation \verb"-U->" to indicate a transition with uncertain behaviour.
%% %
%% \begin{figure}[t]
%% %\vspace*{-0.5cm}
%% \lstset{style=polyprism}
%% \begin{lstlisting}{language=Java}
%% // action specification for Roborta moving to the left
%% [robl] (sched = 0) & (roby<L) & !Collision -U->
%%   //The first four probabilistic options  correspond to environments setbacks
%%   pl:(robx'=(max(0, robx-1)) & (sched' = 1) + pr:(robx'=(min(W-1, robx+1)) & (sched' = 1) 
%%   + pf:(roby'=(roby+1) & (sched' = 1) + pb:(roby'=(max(0, roby+1)) & (sched' = 1) 
%%   + pc:(robx'=max(0, robx-1)) & (sched' = 1) 
%%   {// inequations for uncertainty
%%     1-(Q[robx,roby]+(1-(1-abs(L[robx,roby]))*(1-abs(F[robx,roby])))/2) >= pc, 
%%     (1-max(0,-L[robx,roby]))*pl-(1-Q[robx,roby])*(1-max(0,L[robx,roby]))*pr >= 0, 
%%     (1-max(0,L[robx,roby]))*pr-(1-Q[robx,roby])*(1-max(0,-L[robx,roby]))*pl >= 0, 
%%     (1-max(0,F[robx,roby]))*pf-(1-Q[robx,roby])*(1-max(0,-F[robx,roby]))*pb >= 0, 
%%     (1-max(0,-F[robx,roby]))*pb-(1-Q[robx,roby])*(1-max(0,F[robx,roby]))*pf >= 0  
%%   };
%% \end{lstlisting}
%% \vspace*{-0.3cm}
%% \caption{Fragment of code for Roborta vs Rigoborto.}\label{fig:roborta-code}
%% \vspace*{-0.2cm}
%% \end{figure}
%% %
%% Fig.~\ref{fig:roborta-code} illustrates the command in which Roborta is instructed to move left.  Variables \verb"robx" and \verb"roby" correspond to Roborta's coordinates $x$ and $y$ and constant matrices \verb"Q", \verb"L", and \verb"F" contain the respective values for $q_{xy}$, $l_{xy}$, and $f_{xy}$.  The rest of the values are as expected except for \verb"sched" that is used for modeling a scheduler that signals who is the next player to play, and the Boolean variable \verb"Collision" to indicate that Rigoborto has caught Roborta.
%% %
%% In Sec.~\ref{sec:experiments} we report some results on this game as
%% well as some variants that include reward objectives.

%%%%%%%%%%%%%%%%%%%%%%%%%%%%%%%%%%%%%%%%%%%%%%%%%%%%%%%%%%%%%%%%%%%%%%%%%%%%%%%%
%%%%%%%%%%%%%%%%%%%%%%%%%%%%%%%%%%%%%%%%%%%%%%%%%%%%%%%%%%%%%%%%%%%%%%%%%%%%%%%%
%%%%%%%%%%%%%%%%%%%%%%%%%%%%%%%%%%%%%%%%%%%%%%%%%%%%%%%%%%%%%%%%%%%%%%%%%%%%%%%%
\iffalse

In these cases,  we have uncertainty about the next position of the robots; this uncertainty is modeled via linear equations.   For doing so,  we propose an extension of {\PRISMGAMES} language.  We illustrate this with a small fragment of the code for this example, see Fig.~\ref{fig:roborta-code}.  The code mostly  follows  {\PRISM} notation,  but we add the notation  \verb"-U->" for pointing out that an action has uncertain behavior.  Variables \verb"robc" and \verb"robr" indicate the row and column of Roborta location, respectively.  Variable \verb"sched" is used for 
modeling a scheduler that signals who is the next player to play.   For modeling the irregularities in the terrain,  we consider an array \verb"LS[i,j]" (the lateral inclination, given by a number  \verb"-1<=k<=1") where  \verb"0" indicates no inclination at all,   \verb"1" is the maximum inclination towards right, and \verb"-1" maximum inclination towards left.  Similarly,  matrix \verb"RS[i,j]" is used for modeling the inclination towards right.  Furthermore, we use an array \verb"QT[i,j]" to indicate the level of muddy in the cells,  the value in this array are picked from \verb"[0,0.5]",  being \verb"0.5" the maximum amount of mud. 
\begin{figure}[h]
\vspace*{-0.5cm}
\lstset{style=polyprism}
\begin{lstlisting}{language=Java}
// action specification for Roborta moving to the left
[robl] (sched = 0) & (robr<L) & !Collision -U->
  //The first four probabilistic options  corresponds to environments setbacks
  pl:(robc'=(max(0, robc-1)) & (sched' = 1) + pr:(robc'=(min(W-1, robc+1)) & (sched' = 1) 
  + pf:(robr'=(robr+1) & (sched' = 1) + pb:(robr'=(max(0, robr+1)) & (sched' = 1) 
  + pm:(robc'=max(0, robc-1)) & (sched' = 1) 
  {// inequations for uncertainty
    1-(QT[robc,robr]+(1-(1-abs(LS[robc,robr]))*(1-abs(FS[robc,robr])))/2) >= pm, // IEQ0
    (1-max(0,-LS[robc,robr]))*pl-(1-QT[robc,robr])*(1-max(0,LS[robc,robr]))*pr >= 0, // IEQ1
    (1-max(0,LS[robc,robr]))*pr-(1-QT[robc,robr])*(1-max(0,-LS[robc,robr]))*pl >= 0, // IEQ2
    (1-max(0,FS[robc,robr]))*pf-(1-QT[robc,robr])*(1-max(0,-FS[robc,robr]))*pb >= 0, // IEQ3
    (1-max(0,-FS[robc,robr]))*pb (1-QT[robc,robr])*(1-max(0,FS[robc,robr]))*pf >= 0  // IEQ4
  };
\end{lstlisting}
\vspace*{-0.2cm}
\caption{Fragment of code for Roborta vs Rigoborto.}\label{fig:roborta-code}
\vspace*{-0.2cm}
\end{figure}
\sloppy Consider the Roborta action of moving to the left (named \verb"robl"), the result of the action depends on the current position of Roborta; if there is no irregularities in the current position, the next position is \verb"(i+1,j)", where   \verb"(i,j)" is the current position; otherwise, the result depends on the terrain.  Taking into account these observations, we can define this Roborta's action as in Fig.~\ref{fig:roborta-code}. Therein we have a collection of linear inequations stating the uncertain behavior of the robot, where \verb"pl,pr,pl,pb" represent the probabilities that the robot moves to the left, right, forward, or backward, whereas \verb"pm" is the probability that the command is successfully executed (i.e,  she goes to the left).  For instance,  the line annotated with \verb"IEQ0" bounds the probability of a successful command.  Intuitively, the expression at the left of the inequality  expresses the probability of not sliding.  More precisely,  \verb"(1-(1-abs(LS[robc,robr]))*(1-abs(FS[robc,robr])))/2)" states that
the slopes contribute  half of the probability of sliding,  \verb"QT[robc,robr]" contributes the other half,  and the complement of that gives us an upper bound for the probability of not sliding. The other inequations have a similar reading.
%

\fi
%%%%%%%%%%%%%%%%%%%%%%%%%%%%%%%%%%%%%%%%%%%%%%%%%%%%%%%%%%%%%%%%%%%%%%%%%%%%%%%%
%%%%%%%%%%%%%%%%%%%%%%%%%%%%%%%%%%%%%%%%%%%%%%%%%%%%%%%%%%%%%%%%%%%%%%%%%%%%%%%%
%%%%%%%%%%%%%%%%%%%%%%%%%%%%%%%%%%%%%%%%%%%%%%%%%%%%%%%%%%%%%%%%%%%%%%%%%%%%%%%%

\section{Preliminaries}\label{sec:preliminaries}
In this section we introduce notation and basic concepts of polytopes and games. Interested readers are referred to \cite{Ziegler95,Kucera11}.

In the following $\powerset(S)$ denotes the powerset of set $S$, and $\powerset_f(S)$ denotes the set of finite subsets of set $S$.
A \emph{convex polytope} in $\Reals^n$ is a bounded set
$K=\{\bvec{x}\in\Reals^n \mid A\bvec{x}\leq b\}$, with
$A\in\Reals^{m\times n}$ and $b\in\Reals^m$, for some $m\in\Nat$.  By
\emph{bounded} we mean (in our case) that there exists $M\in\Realsnn$ such that
$\sum_{i=1}^n\abs{x_i}\leq M$ for all $\bvec{x}\in K$ ($x_i$ denotes
the $i$th element of $\bvec{x}$).
%% \remarkPRD{In fact, more generally it is requested that
%%   $\norm{\bvec{x}}\leq M$ for all $\bvec{x}\in K$ and some given norm
%%   $\norm{\_}$.}
%
Let $S$ be a finite set. As functions in $\Reals^S$ can be
equivalently seen as vectors in $\Reals^{|S|}$, we will in general
refer to polytopes in $\Reals^S$.  Let $\Poly(S)$ be the set of all
convex polytopes in $\Reals^S$.
%
Notice that the set of all probability functions on $S$ form the
convex polytope
%
$\Dist(S)=\{\mu\in\Reals^S\mid \sum_{s\in S}\mu(s)=1 \text{ and } \forall {s\in S} \colon {\mu(s)\geq 0} \}$.
%
Let $\DPoly(S)=\{K\cap\Dist(S)\mid K\in\Poly(S)\}$.  Thus,
$K\in\DPoly(S)$ is a convex polytope whose elements are also
probability functions on $S$ and therefore its defining set of
inequality $A\bvec{x}\leq b$ already encodes the inequalities
$\sum_{s\in S}x_s=1$ and $x_s\geq 0$ for $s\in S$.

Any convex polytope $K\in\Poly(S)$ can alternatively be characterized
as the convex hull of its finite set of vertices.   Let $\vertices(K)$ denote the set of all
vertices of polytope $K$.
%
If $\vertices(K)=\{\bvec{v}^1,\ldots,\bvec{v}^k\}$, then every
$\bvec{x}\in K$ is a convex combination of
$\{\bvec{v}^1,\ldots,\bvec{v}^k\}$, that is,
%
$\bvec{x} = \sum_{i=1}^k\lambda_i\bvec{v}^k$ with $\lambda_i\geq 0$,
for $i\in[1..k]$, and $\sum_{i=1}^k\lambda_i=1$.
%
A \emph{simplex} is any convex polytope $K\in\Poly(S)$ whose set of
vertices $\vertices(K)$ is affinely independent, that is, for any
family $\{\lambda_{\bvec{v}}\in\Reals\}_{\bvec{v}\in\vertices(K)}$
such that $\sum_{\bvec{v}\in\vertices(K)}\lambda_{\bvec{v}}=0$,
$\sum_{\bvec{v}\in\vertices(K)}\lambda_{\bvec{v}}\bvec{v}=0$ implies
that $\lambda_{\bvec{v}}=0$ for all $\bvec{v}\in\vertices(K)$.
%
This implies that for every $\bvec{x}\in K$, with $K$ being a simplex,
the convex combination $\bvec{x} = \sum_{i=1}^k\lambda_i\bvec{v}^k$ is
unique.
%
We also remark that any convex polytope $K$ can be expressed as the
union of a (finite) set of simplices $\{K_i\}_{i\in I}$ so that
$\vertices(K)=\bigcup_{i\in I}\vertices(K_i)$ (this is a consequence
of Charath\'eodory's Theorem~\cite{Ziegler95,McMullen2020}).  We will
call such decomposition a \emph{vertex-preserving triangulation}.
% The last claim is spelled out as Theorem 1.11 in "disgeoIII_notes.pdf"
% and corresponds to Theorem 2.15(4) in Ziegler95.
% I like Theorem 1B4 of Caratheodory's theorem in McMullen2020.
%
Let $\Simp(S)$ denote the set of all simplices in $\Reals^S$ and
$\DSimp(S)=\Simp(S)\cap\DPoly(S)$.

A \emph{stochastic game} \cite{Shapley53,Condon92,FilarV96}
is a tuple
%
$\StochG = (\sgnodes, (\sgnmax, \sgnmin), \sgactions, \sgtrans)$,
%
where $\sgnodes$ is a finite set of \emph{states} with
$\sgnmax,\sgnmin\subseteq\sgnodes$ being a partition of $\sgnodes$,
$\sgactions$ is a (finite) set of \emph{actions},
and
%
$\sgtrans : \sgnodes \times \sgactions \times \sgnodes \rightarrow [0,1]$
%
is a \emph{probabilistic transition function} such that for every
$s\in\sgnodes$ and $a\in\sgactions$,
$\sgtrans(s,a,\cdot)\in\Dist(\sgnodes)$ or $\sgtrans(s,a,\sgnodes)=0$.
Let
$\enabled(s)=\{{a\in\sgactions}\mid{\sgtrans(s,a,\sgnodes)=1}\}$ be
the set of actions enabled at state $s$.
%
If $\sgnmax=\emptyset$ or $\sgnmin=\emptyset$,
then $\StochG$ is a \emph{Markov decision process} (or MDP).
If, in addition, $|\enabled(s)|=1$ for all $s\in\sgnodes$,
$\StochG$ is a \emph{Markov chain} (or MC).
%
%
%
A \emph{path} in the game $\StochG$ is an infinite sequence of states
$\rho=s_0 s_1 \dots$ such that,  for every $k\in\Nat$,  there is an $a\in\sgactions$ with $\sgtrans(s_k, a, s_{k+1})>0$.  For $i\geq0$, $\rho_i$
indicates the $i$th state in the path $\rho$ (notice that $\rho_0$ is
the first state in $\rho$). $\GamePaths_{\StochG}$ denotes the set of
all paths, and $\FinGamePaths_{\StochG}$ denotes the set of finite
prefixes of paths.  Similarly, $\GamePaths_{\StochG,s}$ and
$\FinGamePaths_{\StochG,s}$ denote the set of paths and the set of
finite paths starting at state $s$.
%
%
%
A \emph{strategy} for the $i$-player (for $i\in\{\maxplay,\minplay\}$)
in a game $\StochG$ is a function
$\strat_{i}:{\sgnodes^*\sgnodes_i}\to{\Dist(\sgactions)}$ that assigns
a probabilistic distribution to each finite sequence of states such
that $\strat_{i}(\hat{\rho}s)(a) > 0$ only if $a\in \enabled(s)$.  The
set of all strategies for the $i$-player is named
$\Strategies{i}$. Whenever convenient, we indicate that the set of
strategies $\Strategies{i}$ belongs to the game $\StochG$ by writing
by $\Strategies{\StochG,i}$
%
A strategy $\strat_{i}$ is said to be \emph{pure} or
\emph{deterministic} if, for every
$\hat{\rho}s\in\sgnodes^*\sgnodes_i$, $\strat_{i}(\hat{\rho} s)$ is a
Dirac distribution (that is a distribution $\Dirac_a$ s.t.,
$\Dirac_a(a)=1$ and $\Dirac_a(b)=0$ for all $b\neq a$), and it is
called \emph{memoryless} if $\strat_{i}(\hat{\rho} s) =
\strat_{i}(s)$, for every $\hat{\rho}\in\sgnodes^*$.
%% Let
%% $\MemorylessStrats{i}$ and $\DetStrats{i}$ be respectively the set of
%% all memoryless strategies and the set of all deterministic strategies
%% for the $i$-player.  $\DetMemorylessStrats{i} = \MemorylessStrats{i}
%% \cap \DetStrats{i}$ is the set of all its deterministic and memoryless
%% strategies.
Let $\MemorylessStrats{i}$ be the set of all memoryless strategies for
the $i$-player and $\DetMemorylessStrats{i}$ be the set of all its
deterministic and memoryless strategies.  Note that the definition of strategy given above works for  set of actions that are finite,  in Section \ref{sec:polytopal-games} we define strategies for uncountable sets of actions.


Given strategies $\strat_{\maxplay} \in \Strategies{\maxplay}$ and
$\strat_{\minplay} \in \Strategies{\minplay}$, and an initial state
$s$, the \emph{result} of the game is a Markov chain
\cite{ChatterjeeH12}, denoted
$\StochG^{\strat_{\maxplay},\strat_{\minplay}}_s$.
%
The Markov chain $\StochG^{\strat_{\maxplay},\strat_{\minplay}}_s$
defines a probability measure
$\Prob^{\strat_{\maxplay},\strat_{\minplay}}_{\StochG,s}$ on the Borel
$\sigma$-algebra generated by the cylinders of $\GamePaths_{\StochG,s}$.
%
If $\xi$ is a measurable set in such a Borel $\sigma$-algebra,
$\Prob^{\strat_{\maxplay},\strat_{\minplay}}_{\StochG,s}(\xi)$ is the
probability that strategies $\strat_{\maxplay}$ and
$\strat_{\minplay}$ follow a path in $\xi$ starting from state $s$.
%
We use {\LTL} notation to represent specific set of paths, in particular,
%
$D \Until^n C =
\{{\rho \in \sgnodes^\omega} \mid {{{\rho_n\in C} \wedge {\forall{j<n}\colon{\rho_j\in D}}}}\} =
D^n\times C\times \sgnodes^\omega$
%
is the set of paths that reach $C\subseteq\sgnodes$ in exactly $n\geq0$ steps
traversing before only states in $D\subseteq\sgnodes$;
%
$\Finally^n C = \sgnodes\Until^n C$ is the set of all paths reaching
states in $C$ in exactly $n$ steps; and
%
$\Finally C = \bigcup_{n\geq0}(\sgnodes\setminus C)\Until^nC$
is the set of all paths that reach a state in $C$.

%% A stochastic game is said to be \emph{almost surely
%% stopping}~\cite{Condon92} if for all pair of memoryless and deterministic strategies\remarkPRD{por qu\'e solo memoryless y deterministic?}
%% $\strat_{\maxplay}$, $\strat_{\minplay}$ the probability of reaching a
%% terminal state is~$1$.
%% %
%% A state $s$ is \emph{terminal} if $\sgtrans(s,a,s)=1$, for all
%% $a\in\enabled(s)$.
%% %\DetMemorylessStrats{i}
%% In other words, a game is stopping if
%% $\inf_{\strat_\minplay\in\DetMemorylessStrats{\minplay}}\inf_{\strat_\maxplay\in\DetMemorylessStrats{\maxplay}}\Prob^{\strat_{\maxplay},\strat_{\minplay}}_{s}(\Finally T)=1$,
%% where $T\subseteq\sgnodes$ is the set of terminal states.
%% %
%% A stochastic game is \emph{irreducible}~\cite{FilarV96} if for all
%% pair of memoryless and deterministic strategies, the probability of reaching a state from any other
%% state is positive, that is, if
%% $\inf_{\strat_\minplay\in\DetMemorylessStrats{\minplay}}\inf_{\strat_\maxplay\in\DetMemorylessStrats{\maxplay}}\Prob^{\strat_{\maxplay},\strat_{\minplay}}_{s}(\Finally s')>0$
%% for all pair of states $s,s'\in\sgnodes$.
%
%
A stochastic game is said to be \emph{almost surely
stopping}~\cite{Condon92,FilarV96} if for all pair of
%memoryless
strategies
$\strat_{\maxplay}$, $\strat_{\minplay}$ the probability of reaching a
terminal state is~$1$.
%
A state $s$ is \emph{terminal} if $\sgtrans(s,a,s)=1$, for all
$a\in\enabled(s)$.
%\MemorylessStrats{i}
In other words, a game is stopping if
$\inf_{\strat_\minplay\in\Strategies{\minplay}}\inf_{\strat_\maxplay\in\Strategies{\maxplay}}\Prob^{\strat_{\maxplay},\strat_{\minplay}}_{s}(\Finally T)=1$,
%$\inf_{\strat_\minplay\in\MemorylessStrats{\minplay}}\inf_{\strat_\maxplay\in\MemorylessStrats{\maxplay}}\Prob^{\strat_{\maxplay},\strat_{\minplay}}_{s}(\Finally T)=1$,
where $T\subseteq\sgnodes$ is the set of terminal states.
%
A stochastic game is \emph{irreducible}~\cite{FilarV96} if for all
pair of
%memoryless
strategies,
%% strategies%
%% \footnote{Normally the definition of irreducible stochastic games is
%% limited to memoryless and deterministic strategies (which coincide
%% with the general definition for finite models).  Since we are
%% introducing a variant with uncountably many transitions, we opt for
%% the general case.},
%
the probability of reaching a state from any other state is positive,
that is, if
$\inf_{\strat_\minplay\in\Strategies{\minplay}}\inf_{\strat_\maxplay\in\Strategies{\maxplay}}\Prob^{\strat_{\maxplay},\strat_{\minplay}}_{s}(\Finally
s')>0$
%% $\inf_{\strat_\minplay\in\MemorylessStrats{\minplay}}\inf_{\strat_\maxplay\in\MemorylessStrats{\maxplay}}\Prob^{\strat_{\maxplay},\strat_{\minplay}}_{s}(\Finally s')>0$
for all pair of states $s,s'\in\sgnodes$.

A \emph{quantitative objective} or \emph{payoff function} is a
measurable function $f: \sgnodes^{\omega} \to \Reals$.  Let
$\Expect^{\strat_{\maxplay},\strat_{\minplay}}_{\StochG,s}[f]$ be the
expectation of measurable function $f$ under probability
$\Prob^{\strat_{\maxplay},\strat_{\minplay}}_{\StochG,s}$.
%
The goal of the $\maxplay$-player is to maximize this value whereas the
goal of the $\minplay$-player is to minimize it.  Sometimes quantitative
objective functions can be defined via \emph{rewards}. These are
assigned by a \emph{reward function} $\reward:S \to \Reals^+$.  We
usually consider stochastic games augmented with a reward
function.  Moreover, we assume that for every terminal state $s$,
$\reward(s) = 0$.
%
The value of the game for the $\maxplay$-player at state $s$ under
strategy $\strat_{\maxplay}$ is defined as the infimum over all the
values resulting from the $\minplay$-player strategies in that state,
i.e.,
$\inf_{\strat_{\minplay} \in \Strategies{\minplay}} \Expect^{\strat_{\maxplay},\strat_{\minplay}}_{\StochG,s}[f]$.
%
The \emph{value of the game} for the $\maxplay$-player is defined as the
supremum of the values of all the $\maxplay$-player strategies, i.e.,
$\sup_{\strat_{\maxplay} \in \Strategies{\maxplay}} \inf_{\strat_{\minplay} \in \Strategies{\minplay}} \Expect^{\strat_{\maxplay},\strat_{\minplay}}_{\StochG,s}[f]$.
%
Similarly, the value of the game for the $\minplay$-player under
strategy $\strat_{\minplay}$ and the value of the game for the
$\minplay$-player are defined as
$\sup_{\strat_{\maxplay} \in \Strategies{\maxplay}}  \Expect^{\strat_{\maxplay},\strat_{\minplay}}_{\StochG,s}[f]$
and
$\inf_{\strat_{\minplay} \in \Strategies{\minplay}} \sup_{\strat_{\maxplay} \in \Strategies{\maxplay}}  \Expect^{\strat_{\maxplay},\strat_{\minplay}}_{\StochG,s}[f]$,
respectively.
%
We say that a game is \emph{determined} if both values are the same,
that is,
$\sup_{\strat_{\maxplay} \in \Strategies{\maxplay}} \inf_{\strat_{\minplay} \in \Strategies{\minplay}} \Expect^{\strat_{\maxplay},\strat_{\minplay}}_{\StochG,s}[f]
=
\inf_{\strat_{\minplay} \in \Strategies{\minplay}} \sup_{\strat_{\maxplay} \in \Strategies{\maxplay}} \Expect^{\strat_{\maxplay},\strat_{\minplay}}_{\StochG,s}[f]$.
%
%% Martin \cite{Martin98} proved the determinacy of stochastic games for
%% Borel and bounded objective functions.


% The following definitions are taken from Puterman 1994, Sec. 5.1,
% p. 120-121
%
In this paper we focus on \emph{total accumulated reward}, where the
payoff function is defined by
$\TRewards(\rho)=\lim_{n\to\infty}\sum^n_{i=0} \reward(\rho_i)$,
%
\emph{total discounted reward}, defined by
$\DRewards{\discfactor}(\rho)=\lim_{n\to\infty}\sum^n_{i=0}\discfactor^i\reward(\rho_i)$,
where $\discfactor\in(0,1)$ is the discount factor,
%
and \emph{average reward}, defined by
$\ARewards(\rho)=\lim_{n\to\infty}\frac{1}{n+1}\sum^n_{i=0}\reward(\rho_i)$.
%
By taking, respectively, $\fgen(i,n)=1$, $\fgen(i,n)=\discfactor^i$, or
$\fgen(i,n)=\frac{1}{n+1}$, we refer simultaneously to the above
payoff functions with the single function
$\GRewards(\rho)=\lim_{n\to\infty}\sum^n_{i=0}\fgen(i,n)\reward(\rho_i)$.

We also focus on \emph{reachability objective}.  In this case, the
goal of the $\maxplay$-player is to maximize the probability of reaching a
state on a goal set $\goal\subseteq\sgnodes$ whereas the goal of the
$\minplay$-player is to minimize it.  Therefore, similar to quantitative
objectives, the \emph{value of the reachability game for the
$\maxplay$-player} is defined by
$\sup_{\strat_{\maxplay} \in \Strategies{\maxplay}} \inf_{\strat_{\minplay} \in \Strategies{\minplay}} \Prob^{\strat_{\maxplay},\strat_{\minplay}}_{\StochG,s}(\Finally\goal)$
%
and the \emph{value of the reachability game for the $\minplay$-player}
is defined by
$\inf_{\strat_{\minplay} \in \Strategies{\minplay}} \sup_{\strat_{\maxplay} \in \Strategies{\maxplay}}  \Prob^{\strat_{\maxplay},\strat_{\minplay}}_{\StochG,s}(\Finally\goal)$,
%
and the game is \emph{determined} if both values are the same.





\section{Polytopal Stochastic Games} \label{sec:polytopal-games}

A polytopal stochastic game is characterized through a structure that
contains a finite set of states divided into two sets, each owned by
a different player.  In addition, each state has assigned a finite
set of convex polytopes of probability distributions over states.
The formal definition is as follows.
\begin{definition}\label{def:psg}
  A \emph{polytopal stochastic game}
  %% (PSG, for short\footnote{The
  %% choice of the acronym is just coincidental.  No association should
  %% be made to any football club that may have not apropriately welcomed
  %% the greatest football player of all time.})
  (PSG, for short)
  %
  is a structure
  $\StochK=(\psgnodes,(\psgnmax,\psgnmin),\psgtrans)$ such that
  $\psgnodes$ is a finite set of states partitioned into 
  $\psgnodes=\psgnmax\uplus\psgnmin$ and
  $\psgtrans:\psgnodes\to\powersetf(\DPoly(\psgnodes))$.
%
  If, in particular,
  $\psgtrans:\psgnodes\to\powersetf(\DSimp(\psgnodes))$, we call
  $\StochK$ a \emph{simplicial stochastic game} (SSG for short).
\end{definition}

The idea of a PSG is as expected: in a state $s\in\psgnodes_i$
($i\in\{\maxplay,\minplay\}$), player $i$ chooses to play a polytope
$K\in\psgtrans(s)$ and a distribution $\mu\in K$.  The next state $s'$
is sampled according to distribution $\mu$ and the game continues from
$s'$ repeating the same process.

%% \begin{example}
%%   One can imagine a stochastic game variant of interval Markov
%%   decision processes (IMDP)~\cite{JonssonL91,KozineU02}.  In this
%%   case, every polytope $K\in\psgtrans(s)$, for all $s\in\psgnodes$, is
%%   defined by $\mu\in K$ iff $\sum_{s'\in\psgnodes}\mu(s')=1$ and, for
%%   all $s'\in\psgnodes$ and some fixed $0\leq l_{s'}\leq u_{s'}\leq 1$,
%%   $l_{s'}\leq\mu(s')\leq u_{s'}$ (note that the intervals need to be
%%   closed).
%% \end{example}


As a particular example, one can devise a stochastic game variant of
Interval Markov Decision Processes (IMDPs)~\cite{JonssonL91,KozineU02}.
This type of games can be interpreted as a PSG where every polytope
$K\in\psgtrans(s)$, for all $s\in\psgnodes$, is defined by $\mu\in K$
iff $\sum_{s'\in\psgnodes}\mu(s')=1$ and, for all $s'\in\psgnodes$ and
some fixed $0\leq l_{s'}\leq u_{s'}\leq 1$,
$l_{s'}\leq\mu(s')\leq u_{s'}$ (note that the intervals need to be closed).


The behaviour of a polytopal stochastic game is formally interpreted
in terms of a stochastic game where the number of transitions outgoing
the players' states may be uncountably large.
We choose a controllable view on the uncertainty introduced by the polytope since the adversarial alternative can be encoded as was shown in Sec.~\ref{sec:roborta}.
Formally, the
interpretation of a PSG is as follows.
\begin{definition}\label{def:interpretation}
  The \emph{interpretation of the polytopal stochastic game $\StochK$}
  is defined by the stochastic game
  %
  $\StochGK = (\sgnodes, (\sgnmax,\sgnmin), \sgactions, \sgtrans)$,
  %
  where
  %
  $\sgactions=\left(\bigcup_{s\in\sgnodes}\psgtrans(s)\right)\times\Dist(\sgnodes)$
  and
  %
  \[\sgtrans(s,(K,\mu),s')=
    \begin{cases}
      \mu(s') & \text{if } K\in\psgtrans(s) \text{ and } \mu \in K \\
      0 & \text{otherwise}
    \end{cases}
  \]
\end{definition}

Notice that the set of actions $\sgactions$ can be uncountably large,
as well as each set $\enabled(s)=\bigcup_{K\in\psgtrans(s)}\{K\}{\times}K$.
%as well as each set $\enabled(s)$, $s\in\sgnodes$.
Therefore we need to extend the strategies to this uncountable domain
which should be properly endowed with a $\sigma$-algebra.
%
For this we make use of a standard construction to provide a
$\sigma$-algebra to $\Dist(\sgnodes)$~\cite{Giry82}:
$\Salg_{\Dist(\sgnodes)}$ is defined as the smallest $\sigma$-algebra
containing the sets $\{\mu\in\Dist(\sgnodes)\mid\mu(S)\geq p\}$ for
all $S\subseteq\sgnodes$ and $p\in[0,1]$.
%
Now, we endow $\sgactions$ with the product $\sigma$-algebra
$\Salg_\sgactions=\powerset\left(\bigcup_{s\in\sgnodes}\psgtrans(s)\right)\otimes\Salg_{\Dist(\sgnodes)}$
(i.e., the smallest $\sigma$-algebra containing all rectangles
$\boldsymbol{K}\times \boldsymbol{M}$ with
$\boldsymbol{K}\subseteq\bigcup_{s\in\sgnodes}\psgtrans(s)$ and
$\boldsymbol{M}\in\Salg_{\Dist(\sgnodes)}$) and let $\PMeas(\sgactions)$ be the set
of all probability measures on $\Salg_\sgactions$.  It is not
difficult to check that each set of enabled actions $\enabled(s)$ is
measurable (i.e., $\enabled(s)\in\Salg_\sgactions$) and that function
$\sgtrans(s,\cdot,s')$ is measurable (i.e.,
$\{{a\in\sgactions}\mid{\sgtrans(s,a,s')\leq p}\}\in\Salg_\sgactions$
for all $p\in[0,1]$).

We extend the definition of \emph{strategy} for the $i$-player
($i\in\{\maxplay,\minplay\}$) in $\StochGK$ to be a function
$\strat_{i}:{\sgnodes^*\sgnodes_i}\to{\PMeas(\sgactions)}$ that
assigns a probability measure to each finite sequence of states such
that $\strat_{i}(\hat{\rho}s)(\enabled(s)) = 1$.  All other concepts
on strategies defined in~Sec.~\ref{sec:preliminaries} apply to this
new definition as well.


In the following we present the formal definition of
$\Prob^{\strat_{\maxplay},\strat_{\minplay}}_{\StochGK,s}$.
%
First, for each $n\geq 0$ and $s\in \sgnodes$, define 
$\Prob^{\strat_{\maxplay},\strat_{\minplay},n}_{\StochGK,s}:\sgnodes^{n+1}\to[0,1]$
for all $s'\in \sgnodes$ and $\hat{\rho}\in \sgnodes^{n+1}$ inductively as follows:
%
\begin{align*}
  &\Prob^{\strat_{\maxplay},\strat_{\minplay},0}_{\StochGK,s}(s') =  \Dirac_{s}(s') \\
  &\Prob^{\strat_{\maxplay},\strat_{\minplay},n+1}_{\StochGK,s}(\hat{\rho} s') =% {} \\
  %&\qquad
  \begin{dcases}
    \Prob^{\strat_{\maxplay},\strat_{\minplay},n}_{\StochGK,s}(\hat{\rho})\int_{\sgactions}\sgtrans(\last(\hat{\rho}),\cdot,s')\ \diff(\strat_{\maxplay}(\hat{\rho})(\cdot)) & \text{if } \last(\hat{\rho})\in \sgnmax \\%[1ex]
    \Prob^{\strat_{\maxplay},\strat_{\minplay},n}_{\StochGK,s}(\hat{\rho})\int_{\sgactions}\sgtrans(\last(\hat{\rho}),\cdot,s')\ \diff(\strat_{\minplay}(\hat{\rho})(\cdot)) & \text{if } \last(\hat{\rho})\in \sgnmin
  \end{dcases}
\end{align*}
%
%% \begin{align*}
%%   &\Prob^{\strat_{\maxplay},\strat_{\minplay},0}_{\StochGK,s}(s') =  \Dirac_{s}(s') \\
%%   &\Prob^{\strat_{\maxplay},\strat_{\minplay},n+1}_{\StochGK,s}(\hat{\rho} s') = 
%%    \Prob^{\strat_{\maxplay},\strat_{\minplay},n}_{\StochGK,s}(\hat{\rho})\int_{\sgactions}\sgtrans(\last(\hat{\rho}),\cdot,s')\ \diff(\strat_{i}(\hat{\rho})(\cdot))\\
%%   & & \hspace{-5em}\text{if } \last(\hat{\rho})\in \sgnodes_i \text{ with } i\in\{\maxplay,\minplay\}
%% \end{align*}
%
and extend $\Prob^{\strat_{\maxplay},\strat_{\minplay},n}_{\StochGK,s}:\powerset(\sgnodes^{n+1})\to[0,1]$ to sets as the sum of the points.

Let $\Salg_{\sgnodes}$ denote the discrete $\sigma$-algebra on $\sgnodes$
and $\Salg_{\sgnodes^\omega}$ the usual product $\sigma$-algebra on
$\sgnodes^\omega$.
%
By Carath\'eodory extension theorem~\cite{AshDoleans99},
$\Prob^{\strat_{\maxplay},\strat_{\minplay}}_{\StochGK,s}:\Salg_{\sgnodes^\omega}\to[0,1]$
is defined as the unique probability measure such that for all
$n\geq 0$, and $S_i\in \Salg_{\sgnodes}$, $0\leq i\leq n$,
%
\[\Prob^{\strat_{\maxplay},\strat_{\minplay}}_{\StochGK,s}(S_0\times\cdots\times S_n\times\sgnodes^\omega) = \Prob^{\strat_{\maxplay},\strat_{\minplay},n}_{\StochGK,s}(S_0\times\cdots\times S_n)\]

%\medskip

The notions of \emph{deterministic} and \emph{memoryless} extends
directly to this type of strategies.
%
In addition, a strategy $\strat_{i}$, $i\in\{\maxplay,\minplay\}$, is
\emph{semi-Markov} if for every $\hat{\rho},\hat{\rho}'\in\sgnodes^*$
and $s\in \sgnodes_i$, $|\hat{\rho}|=|\hat{\rho}'|$ implies
$\strat_{i}(\hat{\rho}s)=\strat_{i}(\hat{\rho}'s)$, that is, the
decisions of $\strat_{i}$ depend only on the length of the run and its
last state. Thus, we write $\strat_{i}(n,s)$ instead of
$\strat_{i}(\hat{\rho}s)$ whenever $|\hat{\rho}|=n$.  Let
$\SemiMarkovStrats{i}$ denote the set of all semi-Markov strategies
for the $i$-player.
%
Also, we say that a strategy $\strat_{i}\in\Strategies{i}$ is
\emph{extreme} if for all $\hat{\rho}\in\sgnodes^*$,
$\strat_{i}(\hat{\rho}s)(\{(K,\mu)\in\sgactions(s)\mid\mu\in\vertices(K)\})=1$.
Notice that extreme strategies only selects transitions on vertices
of polytopes.
%
Let $\XSemiMarkovStrats{i}$ and $\XDetMemorylessStrats{i}$ be,
respectively, the set of all extreme semi-Markov strategies and the
set of all extreme deterministic and memoryless srategies for the
$i$-player.


Polytopal stochastic games can be translated into
simplicial stochastic games preserving all the stochastic behaviour.
More precisely, for every PSG $\StochK$ there is a SSG $\StochK'$ such
that for every pair of strategies for $\StochK$ in a particular class
(i.e., memoryless, semi-Markov, etc.), there is a pair of strategies
for $\StochK'$ in the same class that yields the same probability
measure and vice versa.
%
Let $\Triang\colon\DPoly\to\powerset(\DSimp)$ be a function that assigns a
vertex-preserving triangulation $\Triang(K)$ to each polytope $K$. 
%
Then:

\begin{proposition}\label{prop:PSG:SSG}
  Let $\StochK=(\psgnodes,(\psgnmax,\psgnmin),\psgtrans)$ be a PSG and
  define the SSG $\StochK'=(\psgnodes,(\psgnmax,\psgnmin),\psgtrans')$
  such that $\psgtrans'(s) = \bigcup_{K\in\psgtrans(s)}\Triang(K)$.
  Let $\StochGK$ and $\StochGKp$ be their respective interpretations.
%
  Then,
  \begin{enumerate}
  \item\label{item:PSG:SSG:i}%
    for all pair of strategies $\strat_{\maxplay}$ and
    $\strat_{\minplay}$ for $\StochGK$ there is a pair of strategies
    $\strat'_{\maxplay}$ and $\strat'_{\minplay}$ for $\StochGKp$ such
    that
    \begin{enumerate*}
    \item%
      $\Prob^{\strat_{\maxplay},\strat_{\minplay}}_{\StochGK,s}=\Prob^{\strat'_{\maxplay},\strat'_{\minplay}}_{\StochGKp,s}$
      for all $s\in\sgnodes$, and
    \item%
      if $\strat_{i}$, $i\in\{\maxplay,\minplay\}$, is memoryless (resp.
      deterministic, semi-Markov or extreme) then so is $\strat'_{i}$;
    \end{enumerate*}
    and
  \item\label{item:PSG:SSG:ii}%
    the same holds with the roles of $\StochGK$ and $\StochGKp$
    exchanged.
  \end{enumerate}  
\end{proposition}
%
\begin{proof}[Sketch]
  Let $\StochGK=(\sgnodes,(\sgnmax,\sgnmin),\sgactions,\sgtrans)$
%%   with
%%   $\sgactions=\left(\bigcup_{s\in\sgnodes}\psgtrans(s)\right)\times\Dist(\sgnodes)$
  and
  $\StochGKp=(\sgnodes,(\sgnmax,\sgnmin),\sgactions',\sgtrans')$.
%%   with
%%   $\sgactions'=\left(\bigcup_{s\in\sgnodes}\psgtrans'(s)\right)\times\Dist(\sgnodes)=\left(\bigcup_{\substack{s\in\sgnodes,~\\K\in\psgtrans(s)}}\Triang(K)\right)\times\Dist(\sgnodes)$.
%
  To prove item~\ref{item:PSG:SSG:i}, the new strategies are defined
  so that they preserve the same measure on the probability part of
  the labels in $\sgactions'$ as the one the old strategies measure on
  the probability part of $\sgactions$ while properly distributing the
  probabilities on the simplices of the triangulation of the original
  polytopes.
  %
  For this, first fix a function $f_K:\Triang(K)\to\powerset(K)$ for
  each polytope $K\in\DPoly(\sgnodes)$ satisfying
  \begin{enumerate*}[(i)]
  \item%
    $\forall {K'\in\Triang(K)}\colon{f_K(K')\subseteq K'}$,
  \item%
    $\bigcup_{K'\in\Triang(K)}f_K(K') = K$, and
  \item%
    $\forall {K'_1,K'_2\in\Triang(K)}\colon {{f_K(K'_1)\cap f_K(K'_2)\neq\emptyset} \limp {K'_1=K'_2}}$.
  \end{enumerate*}
  %
  Thus, $f_K(K')$ is almost the simplex $K'$ but ensuring that
  distributions on the faces of $K'$ are exactly in one of the
  $f_K(K'')$, $K''\in\Triang(K)$.
  
  Given strategies $\strat_{i}$, $i\in\{\maxplay,\minplay\}$,
  for $\StochGK$ define $\strat'_{i}$ for $\StochGKp$, for all
  $\hat{\rho}\in\sgnodes^*$, $s\in\sgnodes_{i}$, and
  $A'\in\Salg_{\sgactions'}$ by
%
  \begin{equation}\label{eq:def:stratp:ssg:main}\textstyle
    \strat'_{i}(\hat{\rho}s)(A') = \sum_{K\in\psgtrans(s)} \sum_{K'\in\Triang(K)} \strat_{i}(\hat{\rho}s)(\{K\}\times({A'\sect{K'}}\cap f_K(K')))
  \end{equation}
%
  where ${A'\sect{K'}}=\{\mu\mid (K',\mu)\in A'\}$ is the $K'$ section
  of the measurable set $A'$.
%
  Notice that $f_K$ ensures that the faces of each $K'\in\Triang(K)$
  are considered in exactly one summand of the inner summation
  of~(\ref{eq:def:stratp:ssg:main}).
  %
%%   Thus, $\strat'_{i}(\hat{\rho})$ is a well defined probability
%%   measure and hence $\strat'_{\maxplay}$ and $\strat'_{\minplay}$ is a
%%   pair of strategies for $\StochGKp$.


  For item~\ref{item:PSG:SSG:ii},
  %% as for~\ref{item:PSG:SSG:i},
  the new strategies preserve the same measure on the probability part
  of $\sgactions$ as the old strategies while gathering the
  probability of the simplices in the original polytope.
  %
  So, for each state $s\in\sgnodes$, fix
  %% $f_s:\psgtrans(s)\to\bigcup_{K\in\psgtrans(s)}\Triang(K)$
  $f_s:{\psgtrans(s)\to\powerset(\DSimp(\sgnodes)})$
  %
  such that
  \begin{enumerate*}[(i)]
  \item%
    $\forall {K\in\psgtrans(s)}\colon {f_s(K)\subseteq\Triang(K)}$,
  \item%
    $\bigcup_{K\in\psgtrans(s)}f_s(K)=\bigcup_{K\in\psgtrans(s)}\Triang(K)$, and
  \item%
    $\forall {K_1,K_2\in\psgtrans(s)}\colon {{f_s(K_1)\cap f_s(K_2)\neq\emptyset} \limp {K_1=K_2}}$.
  \end{enumerate*}
  %
  Given strategies $\strat'_{i}$, $i\in\{\maxplay,\minplay\}$, for
  $\StochGKp$ define $\strat_{i}$ for $\StochGK$, for all
  $\hat{\rho}\in\sgnodes^*$, $s\in\sgnodes_{i}$, and
  $A\in\Salg_{\sgactions}$ by
%
  \begin{equation}\label{eq:def:stratp:psg:main}\textstyle
    \strat_{i}(\hat{\rho}s)(A) = \sum_{K\in\psgtrans(s)} \sum_{K'\in f_{s}(K)} \strat'_{i}(\hat{\rho}s)(\{K'\}\times{A\sect{K}})
  \end{equation}
%
  Notice that, by definition, $K'\in\psgtrans'(s)$.
  Moreover, notice that $f_s$ ensures that a simplex in a
  triangulation of a polytope outgoing $s$ is considered in exactly one
  summand of~(\ref{eq:def:stratp:psg:main}).
  %
%%   Thus $\strat_{i}(\hat{\rho})$ is a well defined probability measure
%%   on $\sgnodes$ and hence $\strat_{\maxplay}$ and $\strat_{\minplay}$
%%   is a pair of strategies for $\StochGK$.

  In both cases, it requires some straightforward calculations to
  check that the properties of memoryless, semi-Markov, deterministic,
  and extreme are preserved by the new strategies. Also in both cases,
  to prove that
  $\Prob^{\strat_{\maxplay},\strat_{\minplay}}_{\StochGK,s}=\Prob^{\strat'_{\maxplay},\strat'_{\minplay}}_{\StochGKp,s}$
  it sufficies to state that
  $\Prob^{\strat_{\maxplay},\strat_{\minplay},n}_{\StochGK,s}=\Prob^{\strat'_{\maxplay},\strat'_{\minplay},n}_{\StochGKp,s}$
  for all $n\geq0$ which is done by induction using results from
  measure theory.
%
  \qed
\end{proof}


\section{Discretizing Polytopal Stochastic Games} \label{sec:discretazition}

In this section we show that a PSG can be solved by translating it
into a finite stochastic game that is just like the original PSG but
it only has the transitions corresponding to the vertices of the
polytopes.  We focus on reachability games, and the reward games
introduced above: total accumulated reward, total discounted reward,
and average reward.

The first lemma we introduce states that the calculation of the
expected values of the different reward games only depend on the
probability of reaching each state and the reward collected in each
state regardless the path that lead to such states.  In particular,
Lemma~\ref{lm:encoding:reach:and:expectation}.\ref{lm:encoding:reach:and:expectation:i}
refers to the reward collected in a finite number of steps while
Lemma~\ref{lm:encoding:reach:and:expectation}.\ref{lm:encoding:reach:and:expectation:ii}
refers to the general case stated before.

For $k\geq 0$ define
$\Finally^{k} s = \sgnodes^k\times\{s\}\times\sgnodes^\omega$
to be the set of all runs in which $s\in\sgnodes$ is reached in
exactly $k$ steps.
%
Let
$\FGRewards^n(\hat{\rho})=\sum^{n}_{i=0}\fgen(i,n)\reward(\hat{\rho}_i)$
for all $\hat{\rho}\in\sgnodes^{n+1}$.
%
Then $\GRewards(\rho)=\lim_{n\to\infty}\FGRewards^n(\rho[..n+1])$ where
$\rho[..n+1]$ is the $(n+1)$th prefix of $\rho$, i.e.,
$\rho[..n+1]=\rho_0\rho_1\rho_2...\rho_n$.


\begin{lemma}\label{lm:encoding:reach:and:expectation}
  Let $\StochGK$ be a stochastic game resulting from interpreting a
  PSG $\StochK$.  For all strategies
  $\strat_{\maxplay}\in\Strategies{\maxplay}$ and
  $\strat_{\minplay}\in\Strategies{\minplay}$,
  \begin{enumerate}
  \item\label{lm:encoding:reach:and:expectation:i}%
    $%\displaystyle
    \sum_{\hat{\rho} \in \sgnodes^{n+1}} \Prob^{\strat_{\maxplay},\strat_{\minplay},n}_{\StochGK,s}(\hat{\rho})\mult\FGRewards^n(\hat{\rho}) =
    \sum^{n}_{i=0} \sum_{s' \in \sgnodes} \Prob^{\strat_{\maxplay},\strat_{\minplay}}_{\StochGK,s}(\Finally^i s')\mult\fgen(i,n)\mult\reward(s')$, for all $n\geq 0$,
    and\smallskip
  \item\label{lm:encoding:reach:and:expectation:ii}%
    $%\displaystyle
    \Expect^{\strat_{\maxplay},\strat_{\minplay}}_{\StochGK,s}[\GRewards] =
    \lim_{n\to\infty} \sum^{n}_{i=0} \sum_{s' \in \sgnodes} \Prob^{\strat_{\maxplay},\strat_{\minplay}}_{\StochGK,s}(\Finally^i s')\mult\fgen(i,n)\mult\reward(s')$.    
  \end{enumerate}
\end{lemma}

The proof of
Lemma~\ref{lm:encoding:reach:and:expectation}.\ref{lm:encoding:reach:and:expectation:i}
follows by induction on $n$ while
Lemma~\ref{lm:encoding:reach:and:expectation}.\ref{lm:encoding:reach:and:expectation:ii}
can be calculated using the first item.



The next lemma states that if the $\minplay$-player plays a
semi-Markov strategy, the $\maxplay$-player can achieve equal results
whether she plays an arbitrary strategy or limits to playing only
semi-Markov strategies.
%% The first two items of the lemma focus on
%% reachability while the second one on the expected value of the
%% different reward games.
%
\begin{lemma}\label{lm:semimarkov}
  Let $\StochGK$ be a stochastic game resulting from interpreting a
  PSG $\StochK$.
  %
  If $\strat_{\minplay} \in \SemiMarkovStrats{\minplay}$
  is a semi-Markov strategy,
  then, for any $\strat_{\maxplay} \in \Strategies{\maxplay}$,
  there is a semi-Markov strategy
  $\starredstrat_{\maxplay} \in \SemiMarkovStrats{\maxplay}$
  such that:
  \begin{enumerate}
  \item\label{lm:semimarkov:i}%
    $\Prob^{\strat_{\maxplay},\strat_{\minplay}}_{\StochGK,s}(D \Until^n s') =
    \Prob^{\starredstrat_{\maxplay},\strat_{\minplay}}_{\StochGK,s}(D \Until^n s')$,
    for all $n\geq 0$, $D\subseteq\sgnodes$ and $s'\in \sgnodes$;
  \item\label{lm:semimarkov:ii}%
    $\Prob^{\strat_{\maxplay},\strat_{\minplay}}_{\StochGK,s}(\Finally C) =
    \Prob^{\starredstrat_{\maxplay},\strat_{\minplay}}_{\StochGK,s}(\Finally C)$,
    for all $C\subseteq\sgnodes$; and
  \item\label{lm:semimarkov:iii}%
    $\Expect^{\strat_{\maxplay},\strat_{\minplay}}_{\StochGK,s}[\GRewards] =
    \Expect^{\starredstrat_{\maxplay},\strat_{\minplay}}_{\StochGK,s}[\GRewards]$.
  \end{enumerate}
  %
  Similarly, if $\strat_{\maxplay} \in \SemiMarkovStrats{\maxplay}$
  % is a semi-Markov strategy,
  then, for any $\strat_{\minplay} \in \Strategies{\minplay}$,
  there exists % a semi-Markov strategy
  $\starredstrat_{\minplay} \in \SemiMarkovStrats{\minplay}$
  satisfying, mutatis mutandis, the same equalities.
\end{lemma}
%
\begin{proof}[Sketch]
  To prove item~\ref{lm:semimarkov:i}, we define the new strategy
  $\starredstrat_{\maxplay}$ so that the probability of choosing from
  $A\in\Salg_\sgactions$ after a path of length $n$ ending on a state
  $s$ with the original strategy is uniformly distributed among the
  paths of this type in the new strategy.
  %% To prove item~\ref{lm:semimarkov:i}, we define the new strategy
  %% $\starredstrat_{\maxplay}$ so that the probability of choosing
  %% from $A\in\Salg_\sgactions$ after finite paths of equal lengths, say
  %% $n$, ending on the same state $s$ uniformly distributes the total
  %% probability of chosing from $A$ induced by the original strategy
  %% after $n$ steps.
  %
  Thus, $\starredstrat_{\maxplay}$ is formally defined as follows.
  For $\hat{\rho}\in\sgnodes^*$, $s'\in\sgnodes$, and $A\in\Salg_\sgactions$,
  such that
  $\Prob^{\strat_{\maxplay},\strat_{\minplay}}_{\StochGK,s}(D \Until^n s') > 0$
  and $|\hat{\rho}| = n\geq 0$, let
  %
  \begin{align*}
  \starredstrat_{\maxplay}(\hat{\rho}s')(A)
  & = 
  \frac{\sum_{\hat{\rho}'\in D^n}\Prob^{\strat_{\maxplay},\strat_{\minplay},n}_{\StochGK,s}(\hat{\rho}'s')\mult\strat_{\maxplay}(\hat{\rho}' s')(A)}{\Prob^{\strat_{\maxplay},\strat_{\minplay}}_{\StochGK,s}(D \Until^n s')}
  \end{align*}
  %
  For $s'\in\sgnodes$ with
  $\Prob^{\strat_{\maxplay},\strat_{\minplay}}_{\StochGK,s}(D \Until^n s') = 0$
  and $|\hat{\rho}s'| = n$, define
  $\starredstrat_{\maxplay}(\hat{\rho}s')$ to be
  $\Dirac_{\textsf{f}(s')}$ for a globally fixed function $\textsf{f}$
  such that $\textsf{f}(s')\in\enabled(s')$.
  %
  Notice that
  $\starredstrat_{\maxplay}\in\SemiMarkovStrats{\maxplay}$.
%%   Therefore, we write $\starredstrat_{\maxplay}(n,s')$ for
%%   $\starredstrat_{\maxplay}(\hat{\rho}s')$ whenever
%%   $|\hat{\rho}|=n$.

  Then, the proof of item~\ref{lm:semimarkov:i} follows by induction
  with particular care in the case of
  $\Prob^{\strat_{\maxplay},\strat_{\minplay}}_{\StochGK,s}(D \Until^n s') = 0$.
  %
  Item~\ref{lm:semimarkov:ii} follows straightforwardly from
  item~\ref{lm:semimarkov:i} and item~\ref{lm:semimarkov:iii} follows
  directly from item~\ref{lm:semimarkov:ii} using
  Lemma~\ref{lm:encoding:reach:and:expectation}.\ref{lm:encoding:reach:and:expectation:ii}.
  %
  The proof can be replicated mutatis mutandi with
  $\maxplay$ and $\minplay$ exchanged yielding the last part of the
  lemma.
  \qed
\end{proof}


Since $\psgtrans(s)$ is finite,  there can be finitely many
polytopes $K$ such that $(K,\mu)\in\enabled(s)$.  Besides, the set of
vertices $\vertices(K)$ of $K$ is finite.  Therefore the set
$\{(K,\mu)\in\sgactions(s)\mid\mu\in\vertices(K)\}$ is also finite
and, as a consequence, extreme strategies only resolve with discrete
(finite) probability distributions.  That is, if $\strat_{i}$ is
extreme, $\strat_{i}(\hat{\rho}s)$ has finite support for all
$\hat{\rho}\in\sgnodes^*$ and $s\in\sgnodes$.

It turns out that Lemma~\ref{lm:semimarkov} can be strengthened to
obtain \emph{extreme} semi-Markov strategies.
%% Thus if one of the players plays a semi-Markov strategy, the other
%% can achieve equal results to the general case if she limits herself
%% to playing only extreme semi-Markov strategies, i.e., she only
%% chooses actions $(K,\mu)$ such that the distribution $\mu$ is a
%% vertex of the simplex $K$.
%
We first prove this new lemma for simplicial stochastic games since
simplices have the particular property that any vector in a simplex can
be uniquely defined as a convex combination of the simplex vertices
which is crucial for the proof of the lemma.

\begin{lemma}\label{lm:xsemimarkov}
  Let $\StochGK$ be a stochastic game resulting from interpreting a
  SSG $\StochK$.
  %
  If $\strat_{\minplay} \in \SemiMarkovStrats{\minplay}$
  is a semi-Markov strategy,
  then, for any $\strat_{\maxplay} \in \SemiMarkovStrats{\maxplay}$,
  there is an extreme semi-Markov strategy
  $\starredstrat_{\maxplay} \in \XSemiMarkovStrats{\maxplay}$
  such that:
  \begin{enumerate}
  \item\label{lm:xsemimarkov:i}%
    $\Prob^{\strat_{\maxplay},\strat_{\minplay}}_{\StochGK,s}(D \Until^n s') =
    \Prob^{\starredstrat_{\maxplay},\strat_{\minplay}}_{\StochGK,s}(D \Until^n s')$,
    for all $n\geq 0$, $D\subseteq\sgnodes$ and $s'\in \sgnodes$;
  \item\label{lm:xsemimarkov:ii}%
    $\Prob^{\strat_{\maxplay},\strat_{\minplay}}_{\StochGK,s}(\Finally C) =
    \Prob^{\starredstrat_{\maxplay},\strat_{\minplay}}_{\StochGK,s}(\Finally C)$,
    for all $C\subseteq\sgnodes$; and
  \item\label{lm:xsemimarkov:iii}%
    $\Expect^{\strat_{\maxplay},\strat_{\minplay}}_{\StochGK,s}[\GRewards] =
    \Expect^{\starredstrat_{\maxplay},\strat_{\minplay}}_{\StochGK,s}[\GRewards]$.
  \end{enumerate}
  %
  Similarly, if $\strat_{\maxplay} \in \SemiMarkovStrats{\maxplay}$
  then, for any $\strat_{\minplay} \in \SemiMarkovStrats{\minplay}$,
  there exists
  $\starredstrat_{\minplay} \in \XSemiMarkovStrats{\minplay}$
  satisfying, mutatis mutandis, the same equalities.
\end{lemma}
%
\begin{proof}[Sketch]
  \newcommand{\convp}{\mathsf{p}}%
  %
  For any $K\in\DSimp(\sgnodes)$, $\mu\in K$ and
  $\hat{\mu}\in\vertices(K)$ define $\convp^K(\mu,\hat{\mu})\in[0,1]$
  such that
  $\sum_{\hat{\mu}\in\vertices(K)}\convp^K(\mu,\hat{\mu})\mult\hat{\mu}=\mu$.
  That is, all $\convp^K(\mu,\hat{\mu})$, $\hat{\mu}\in\vertices(K)$,
  are the unique factors that define the convex combination for $\mu$
  in the simplex $K$.
%%   Therefore, $\convp^K(\mu,\hat{\mu})$ is well
%%   defined for all $K\in\DSimp(\sgnodes)$, $\mu\in K$ and
%%   $\hat{\mu}\in\vertices(K)$.
  In any other case, let $\convp^K(\mu,\hat{\mu})=0$.

  Let $\convp((K,\mu),(K,\hat{\mu}))=\convp^K(\mu,\hat{\mu})$ for all
  $K\in\DSimp(\sgnodes)$, $\mu\in K$ and $\hat{\mu}\in\vertices(K)$,
  and let $\convp(a,b)=0$ for any other $a,b\in\sgactions$.
  %
  For every $(K,\mu)\in\sgactions$ such that $\mu\in K$, let
  $\vertices(K,\mu)=\{(K,\hat{\mu})\mid\hat{\mu}\in\vertices(K)\}$ and
  let $\vertices(K,\mu)=\emptyset$ otherwise.
  %
  Thus, for every $s\in\sgnodes$ and $a\in\sgactions$,
%%   \begin{equation}\label{eq:xsemimarkov:sgtrans:convp:main}
%%     \sgtrans(s,a,\cdot) =
%%     \sum_{b\in\vertices(a)}\convp(a,b)\mult\sgtrans(s,b,\cdot).
%%   \end{equation}
  $\sgtrans(s,a,\cdot) =
  \sum_{b\in\vertices(a)}\convp(a,b)\mult\sgtrans(s,b,\cdot)$.

  We also extend $\convp$ to measurable sets $B\in\Salg_\sgactions$
  and $a\in\sgactions$ by
  $\convp(a,B)=\sum_{b\in B\cap\vertices(a)}\convp(a,b)$.

  For every $\hat{\rho}\in\sgnodes^*$, $s'\in\sgnodes$ and
  $B\in\Salg_\sgactions$, define $\starredstrat_\maxplay$ by
  %
  \[\starredstrat_\maxplay(\hat{\rho}s')(B) =
  \int_{\sgactions} \convp(\cdot,B)\ \diff(\strat_\maxplay(\hat{\rho}s')(\cdot)).\]
  %
  $\starredstrat_\maxplay(\hat{\rho}s')$ is defined so that it assigns
  to each vertex of a simplex the weighted contribution (according to
  $\strat_\maxplay(\hat{\rho}s')$) of each distribution (in the said
  simplex) to such vertex.

  Because $\strat_\maxplay$ is semi-Markov, so is
  $\starredstrat_\maxplay$.  Moreover, notice that if $b$ is not a
  vertex label, then $\convp(a,b)=0$ (and hence $\convp(a,B)>0$ only
  if $B$ contains vertices).  This should hint that
  $\starredstrat_\maxplay$ is also extreme.

  Item~\ref{lm:xsemimarkov:i} proceeds by induction on $n$.
  Item~\ref{lm:xsemimarkov:ii} follows straightforwardly
  using~\ref{lm:xsemimarkov:i}, and item~\ref{lm:xsemimarkov:iii}
  follows from~item~\ref{lm:xsemimarkov:ii} using
  Lemma~\ref{lm:encoding:reach:and:expectation}.\ref{lm:encoding:reach:and:expectation:ii}.
  The proof can be replicated mutatis mutandi with
  $\maxplay$ and $\minplay$ exchanged which yields the last part of the
  lemma.
  \qed
\end{proof}

Because of Proposition~\ref{prop:PSG:SSG}, Lemma~\ref{lm:xsemimarkov}
extends immediately to PSG. Moreover, by applying
Lemma~\ref{lm:xsemimarkov} twice and Proposition~\ref{prop:PSG:SSG},
we have the next corollary.

\begin{corollary}\label{cor:xsemimarkov}
  Let $\StochGK$ be a stochastic game resulting from interpreting a
  PSG $\StochK$.
  %
  For all semi-Markov strategies
  $\strat_{\minplay} \in \SemiMarkovStrats{\minplay}$ and
  $\strat_{\maxplay} \in \SemiMarkovStrats{\maxplay}$,
  there are extreme semi-Markov strategies
  $\starredstrat_{\minplay} \in \XSemiMarkovStrats{\minplay}$ and
  $\starredstrat_{\maxplay} \in \XSemiMarkovStrats{\maxplay}$
  such that
  \begin{enumerate}
  \item\label{cor:xsemimarkov:i}%
    $\Prob^{\strat_{\maxplay},\strat_{\minplay}}_{\StochGK,s}(\Finally C) =
    \Prob^{\starredstrat_{\maxplay},\starredstrat_{\minplay}}_{\StochGK,s}(\Finally C)$,
    for all $C\subseteq\sgnodes$; and
  \item\label{cor:xsemimarkov:ii}%
    $\Expect^{\strat_{\maxplay},\strat_{\minplay}}_{\StochGK,s}[\GRewards] =
    \Expect^{\starredstrat_{\maxplay},\starredstrat_{\minplay}}_{\StochGK,s}[\GRewards]$.
  \end{enumerate}
  %
\end{corollary}

Given $\StochGK$, define the \emph{extreme interpretation of
$\StochK$} as the stochastic game
$\StochHK = (\sgnodes, (\sgnmax,\sgnmin), \vertices(\sgactions), \sgtrans_\StochHK)$
where $\sgtrans_\StochHK$ is the restriction of
$\sgtrans$ to actions in
$\vertices(\sgactions) = \{{(K,\mu)\in\sgactions}\mid{\mu\in\vertices(K)}\}$,
that is,
$\sgtrans_\StochHK(s,a,s)=\sgtrans(s,a,s')$ for all $s,s'\in\sgnodes$
and $a\in\vertices(\sgactions)$.
%
Since $\vertices(\sgactions)$ is finite, $\StochHK$ is a finite
stochastic game.

Given an extreme semi-Markov strategy
$\strat_i\in\XSemiMarkovStrats{\StochGK,i}$ for the $i$-player in the 
stochastic game $\StochGK$, $i\in\{\maxplay,\minplay\}$, define
$\stratv_i(\hat{\rho}s)(A)=\strat_i(\hat{\rho}s)(A)$ for all
$\hat{\rho}\in\sgnodes^*$, $s\in\sgnodes$, and
$A\subseteq\vertices(\sgactions)$ ($A\in\Salg_\sgactions$ since it is
finite).
%
Notice that
$\stratv_i(\hat{\rho}s)(\enabled_\StochHK(s))=\strat_i(\hat{\rho}s)(\vertices(\enabled(s)))=1$.
Therefore $\stratv_i\in\SemiMarkovStrats{\StochHK,i}$ is a semi-Markov
strategy in $\StochHK$.
%
Conversely, for a semi-Markov strategy
$\strat_i\in\SemiMarkovStrats{\StochHK,i}$ for the $i$-player in the
stochastic game $\StochHK$, define
$\stratx_i(\hat{\rho}s)(A)=\strat_i(\hat{\rho}s)(A\cap\vertices(\sgactions))$
for all $\hat{\rho}\in\sgnodes^*$, $s\in\sgnodes$, and
$A\in\Salg_\sgactions$.
%
$\stratx_i\in\XSemiMarkovStrats{\StochGK,i}$ is a well defined extreme
semi-Markov strategy in $\StochGK$ since
$\stratx_i(\hat{\rho}s)(\vertices(\enabled(s)))=\strat_i(\hat{\rho}s)(\enabled_\StochHK(s))=1$
and
$\stratx_i(\hat{\rho}s)(\sgactions\setminus\vertices(\sgactions))=\strat_i(\hat{\rho}s)(\emptyset)=0$.
%
Then, it can be calculated by induction on $n$ that
$\Prob^{\strat_{\maxplay},\strat_{\minplay},n}_{\StochGK,s} = \Prob^{\stratv_{\maxplay},\stratv_{\minplay},n}_{\StochHK,s}$
and
$\Prob^{\stratx_{\maxplay},\stratx_{\minplay},n}_{\StochGK,s} = \Prob^{\strat_{\maxplay},\strat_{\minplay},n}_{\StochHK,s}$
which yield
\begin{equation}\label{eq:StochGK:StochHK}
  \Prob^{\strat_{\maxplay},\strat_{\minplay}}_{\StochGK,s} = \Prob^{\stratv_{\maxplay},\stratv_{\minplay}}_{\StochHK,s}
  \text{ and }
  \Prob^{\stratx_{\maxplay},\stratx_{\minplay}}_{\StochGK,s} = \Prob^{\strat_{\maxplay},\strat_{\minplay}}_{\StochHK,s}.
\end{equation}
%
This suggests that the solution of a PSG under extreme semi-Markov
strategies is equivalent to the solution the game on its extreme
interpretation limited to semi-Markov strategies, which is stated in
the following:

%% Then, it can be calculated by induction on $n$ that
%% $\Prob^{\strat_{\maxplay},\strat_{\minplay},n}_{\StochGK,s} = \Prob^{\stratv_{\maxplay},\stratv_{\minplay},n}_{\StochHK,s}$
%% and
%% $\Prob^{\stratx_{\maxplay},\stratx_{\minplay},n}_{\StochGK,s} = \Prob^{\strat_{\maxplay},\strat_{\minplay},n}_{\StochHK,s}$
%% which yield
%% $\Prob^{\strat_{\maxplay},\strat_{\minplay}}_{\StochGK,s} = \Prob^{\stratv_{\maxplay},\stratv_{\minplay}}_{\StochHK,s}$
%% and
%% $\Prob^{\stratx_{\maxplay},\stratx_{\minplay}}_{\StochGK,s} = \Prob^{\strat_{\maxplay},\strat_{\minplay}}_{\StochHK,s}$ and hence the following proposition.

%% \begin{proposition}\label{prop:StochGK:StochHK}
%%   Let $\StochGK$ and $\StochHK$ be respectively the interpretation and
%%   the extreme interpretation of $\StochK$. Then
%%   \begin{enumerate}
%%   \item\label{prop:StochGK:StochHK:i}%
%%     For every
%%     $\strat_\maxplay\in\XSemiMarkovStrats{\StochGK,\maxplay}$ and
%%     $\strat_\minplay\in\XSemiMarkovStrats{\StochGK,\minplay}$,
%%     \begin{enumerate*}
%%     \item\label{prop:StochGK:StochHK:i:a}%
%%       $\Prob^{\strat_{\maxplay},\strat_{\minplay}}_{\StochGK,s}(\Finally C) =
%%       \Prob^{\stratv_{\maxplay},\stratv_{\minplay}}_{\StochHK,s}(\Finally C)$,
%%       for all $C\subseteq\sgnodes$, and
%%     \item\label{prop:StochGK:StochHK:i:b}%
%%       $\Expect^{\strat_{\maxplay},\strat_{\minplay}}_{\StochGK,s}[\GRewards] =
%%       \Expect^{\stratv_{\maxplay},\stratv_{\minplay}}_{\StochHK,s}[\GRewards]$; and
%%     \end{enumerate*}
%%   \item\label{prop:StochGK:StochHK:ii}%
%%     For every
%%     $\strat_\maxplay\in\SemiMarkovStrats{\StochHK,\maxplay}$ and
%%     $\strat_\minplay\in\SemiMarkovStrats{\StochHK,\minplay}$,
%%     \begin{enumerate*}
%%     \item\label{prop:StochGK:StochHK:ii:a}%
%%       $\Prob^{\stratx_{\maxplay},\stratx_{\minplay}}_{\StochGK,s}(\Finally C) =
%%       \Prob^{\strat_{\maxplay},\strat_{\minplay}}_{\StochHK,s}(\Finally C)$,
%%       for all $C\subseteq\sgnodes$, and
%%     \item\label{prop:StochGK:StochHK:ii:b}%
%%       $\Expect^{\stratx_{\maxplay},\stratx_{\minplay}}_{\StochGK,s}[\GRewards] =
%%       \Expect^{\strat_{\maxplay},\strat_{\minplay}}_{\StochHK,s}[\GRewards]$.
%%     \end{enumerate*}
%%   \end{enumerate}
%% \end{proposition}

%% %% From Proposition~\ref{prop:StochGK:StochHK} we can eventually get to
%% %% the fact that solving polytopal stochatstic games under extreme
%% %% semi-Markov strategies is equivalent to solve the games on their
%% %% extreme interpretation limited to semi-Markov strategies.  This is
%% %% stated in the following:

%% Proposition~\ref{prop:StochGK:StochHK} immediately suggests that the
%% solution of a PSG under extreme semi-Markov strategies is equivalent
%% to the solution the game on its extreme interpretation limited to
%% semi-Markov strategies.  This is stated in the following:


\begin{proposition}\label{prop:infsup:supinf:StochGK:StochHK}
  Let $\StochGK$ and $\StochHK$ be respectively the interpretation and
  the extreme interpretation of $\StochK$. Then, the following
  equalities hold
  \begin{enumerate}
  \item\label{prop:infsup:supinf:StochGK:StochHK:i}%
    $
    \inf_{\strat_\minplay\in\XSemiMarkovStrats{\StochGK,\minplay}}\!\sup_{\strat_\maxplay\in\XSemiMarkovStrats{\StochGK,\maxplay}}\!\Prob^{\strat_{\maxplay},\strat_{\minplay}}_{\StochGK,s}(\Finally C) =
    \inf_{\strat_\minplay\in\SemiMarkovStrats{\StochHK,\minplay}}\!\sup_{\strat_\maxplay\in\SemiMarkovStrats{\StochHK,\maxplay}}\!\Prob^{\strat_{\maxplay},\strat_{\minplay}}_{\StochHK,s}(\Finally C)$
  \item\label{prop:infsup:supinf:StochGK:StochHK:ii}%
    $
    \sup_{\strat_\maxplay\in\XSemiMarkovStrats{\StochGK,\maxplay}}\!\inf_{\strat_\minplay\in\XSemiMarkovStrats{\StochGK,\minplay}}\!\Prob^{\strat_{\maxplay},\strat_{\minplay}}_{\StochGK,s}(\Finally C) =
    \sup_{\strat_\maxplay\in\SemiMarkovStrats{\StochHK,\maxplay}}\!\inf_{\strat_\minplay\in\SemiMarkovStrats{\StochHK,\minplay}}\!\Prob^{\strat_{\maxplay},\strat_{\minplay}}_{\StochHK,s}(\Finally C)$
  \item\label{prop:infsup:supinf:StochGK:StochHK:iii}%
    $
    \inf_{\strat_\minplay\in\XSemiMarkovStrats{\StochGK,\minplay}}\!\sup_{\strat_\maxplay\in\XSemiMarkovStrats{\StochGK,\maxplay}}\!\Expect^{\strat_{\maxplay},\strat_{\minplay}}_{\StochGK,s}[\GRewards] =
    \inf_{\strat_\minplay\in\SemiMarkovStrats{\StochHK,\minplay}}\!\sup_{\strat_\maxplay\in\SemiMarkovStrats{\StochHK,\maxplay}}\!\Expect^{\stratv_{\maxplay},\stratv_{\minplay}}_{\StochHK,s}[\GRewards]$
  \item\label{prop:infsup:supinf:StochGK:StochHK:iv}%
    $
    \sup_{\strat_\maxplay\in\XSemiMarkovStrats{\StochGK,\maxplay}}\!\inf_{\strat_\minplay\in\XSemiMarkovStrats{\StochGK,\minplay}}\!\Expect^{\strat_{\maxplay},\strat_{\minplay}}_{\StochGK,s}[\GRewards] =
    \sup_{\strat_\maxplay\in\SemiMarkovStrats{\StochHK,\maxplay}}\!\inf_{\strat_\minplay\in\SemiMarkovStrats{\StochHK,\minplay}}\!\Expect^{\stratv_{\maxplay},\stratv_{\minplay}}_{\StochHK,s}[\GRewards]$
  \end{enumerate}
  %% \begin{enumerate}
  %% \item\label{prop:infsup:supinf:StochGK:StochHK:i}%
  %%   $\displaystyle
  %%   \inf_{\strat_\minplay\in\XSemiMarkovStrats{\StochGK,\minplay}}\sup_{\strat_\maxplay\in\XSemiMarkovStrats{\StochGK,\maxplay}}\Prob^{\strat_{\maxplay},\strat_{\minplay}}_{\StochGK,s}(\Finally C) =
  %%   \inf_{\strat_\minplay\in\SemiMarkovStrats{\StochHK,\minplay}}\sup_{\strat_\maxplay\in\SemiMarkovStrats{\StochHK,\maxplay}}\Prob^{\strat_{\maxplay},\strat_{\minplay}}_{\StochHK,s}(\Finally C)$,
  %% \item\label{prop:infsup:supinf:StochGK:StochHK:ii}%
  %%   $\displaystyle
  %%   \sup_{\strat_\maxplay\in\XSemiMarkovStrats{\StochGK,\maxplay}}\inf_{\strat_\minplay\in\XSemiMarkovStrats{\StochGK,\minplay}}\Prob^{\strat_{\maxplay},\strat_{\minplay}}_{\StochGK,s}(\Finally C) =
  %%   \sup_{\strat_\maxplay\in\SemiMarkovStrats{\StochHK,\maxplay}}\inf_{\strat_\minplay\in\SemiMarkovStrats{\StochHK,\minplay}}\Prob^{\strat_{\maxplay},\strat_{\minplay}}_{\StochHK,s}(\Finally C)$,
  %% \item\label{prop:infsup:supinf:StochGK:StochHK:iii}%
  %%   $\displaystyle
  %%   \inf_{\strat_\minplay\in\XSemiMarkovStrats{\StochGK,\minplay}}\sup_{\strat_\maxplay\in\XSemiMarkovStrats{\StochGK,\maxplay}}\Expect^{\strat_{\maxplay},\strat_{\minplay}}_{\StochGK,s}[\GRewards] =
  %%   \inf_{\strat_\minplay\in\SemiMarkovStrats{\StochHK,\minplay}}\sup_{\strat_\maxplay\in\SemiMarkovStrats{\StochHK,\maxplay}}\Expect^{\stratv_{\maxplay},\stratv_{\minplay}}_{\StochHK,s}[\GRewards]$, and
  %% \item\label{prop:infsup:supinf:StochGK:StochHK:iv}%
  %%   $\displaystyle
  %%   \sup_{\strat_\maxplay\in\XSemiMarkovStrats{\StochGK,\maxplay}}\inf_{\strat_\minplay\in\XSemiMarkovStrats{\StochGK,\minplay}}\Expect^{\strat_{\maxplay},\strat_{\minplay}}_{\StochGK,s}[\GRewards] =
  %%   \sup_{\strat_\maxplay\in\SemiMarkovStrats{\StochHK,\maxplay}}\inf_{\strat_\minplay\in\SemiMarkovStrats{\StochHK,\minplay}}\Expect^{\stratv_{\maxplay},\stratv_{\minplay}}_{\StochHK,s}[\GRewards]$.
  %% \end{enumerate}
\end{proposition}

The next proposition, whose proof also uses (\ref{eq:StochGK:StochHK}),
%% Proposition~\ref{prop:StochGK:StochHK},
provides necessary conditions
for the polytopal stochastic game to be almost surely stopping or
irreducible in terms of the extreme interpretation.

\begin{proposition}\label{prop:stopping:irreducible:StochGK:StochHK}
  Let $\StochGK$ and $\StochHK$ be respectively the interpretation and
  the extreme interpretation of $\StochK$.
  Then,
  \begin{enumerate*}[(1)]
  \item\label{prop:stopping:irreducible:StochGK:StochHK:i}%
    if $\StochGK$ is almost surely stopping, so is $\StochHK$, and
  \item\label{prop:stopping:irreducible:StochGK:StochHK:ii}%
    if $\StochGK$ is irreducible, so is $\StochHK$.
  \end{enumerate*}
  %% Then, the following two
  %% statements hold:
  %% \begin{enumerate}
  %% \item\label{prop:stopping:irreducible:StochGK:StochHK:i}%
  %%   if $\StochGK$ is almost surely stopping, so is $\StochHK$;
  %% \item\label{prop:stopping:irreducible:StochGK:StochHK:ii}%
  %%   if $\StochGK$ is irreducible, so is $\StochHK$.
  %% \end{enumerate}
\end{proposition}


Notice that by fixing one strategy in $\StochHK$ to be the memoryless,
the remaining structure is a Markov decision process.  Then the
statements in the following proposition are consequences of standard
results in MDP~\cite{Puterman94}.

\begin{proposition}\label{prop:mdp:results}
  For all
  $\starredstrat_\maxplay\in\DetMemorylessStrats{\StochHK,\maxplay}$ and
  $\starredstrat_\minplay\in\DetMemorylessStrats{\StochHK,\minplay}$,
  \begin{enumerate}
  \item\label{prop:mdp:results:i}%
    $%\displaystyle
%%     \sup_{\strat_\maxplay\in\Strategies{\StochHK,\maxplay}}\Prob^{\strat_{\maxplay},\starredstrat_{\minplay}}_{\StochHK,s}(\Finally C)
%%     =
    \sup_{\strat_\maxplay\in\SemiMarkovStrats{\StochHK,\maxplay}}\Prob^{\strat_{\maxplay},\starredstrat_{\minplay}}_{\StochHK,s}(\Finally C)
    =
    \sup_{\strat_\maxplay\in\DetMemorylessStrats{\StochHK,\maxplay}}\Prob^{\strat_{\maxplay},\starredstrat_{\minplay}}_{\StochHK,s}(\Finally C)$;
  \item\label{prop:mdp:results:ii}%
    $%\displaystyle
%%     \inf_{\strat_\minplay\in\Strategies{\StochHK,\minplay}}\Prob^{\starredstrat_{\maxplay},\strat_{\minplay}}_{\StochHK,s}(\Finally C)
%%     =
    \inf_{\strat_\minplay\in\SemiMarkovStrats{\StochHK,\minplay}}\Prob^{\starredstrat_{\maxplay},\strat_{\minplay}}_{\StochHK,s}(\Finally C)
    =
    \inf_{\strat_\minplay\in\DetMemorylessStrats{\StochHK,\minplay}}\Prob^{\starredstrat_{\maxplay},\strat_{\minplay}}_{\StochHK,s}(\Finally C)$;
  \item\label{prop:mdp:results:iii}%
    $%\displaystyle
%%     \sup_{\strat_\maxplay\in\Strategies{\StochHK,\maxplay}}\Expect^{\strat_{\maxplay},\starredstrat_{\minplay}}_{\StochHK,s}(\GRewards)
%%     =
    \sup_{\strat_\maxplay\in\SemiMarkovStrats{\StochHK,\maxplay}}\Expect^{\strat_{\maxplay},\starredstrat_{\minplay}}_{\StochHK,s}(\GRewards)
    =
    \sup_{\strat_\maxplay\in\DetMemorylessStrats{\StochHK,\maxplay}}\Expect^{\strat_{\maxplay},\starredstrat_{\minplay}}_{\StochHK,s}(\GRewards)$,
    provided
    $\Expect^{\strat_{\maxplay},\starredstrat_{\minplay}}_{\StochHK,s}(\GRewards)$
    is defined for all
    $\strat_\maxplay\in\SemiMarkovStrats{\StochHK,\maxplay}$; and
  \item\label{prop:mdp:results:iv}%
    $%\displaystyle
%%     \inf_{\strat_\minplay\in\Strategies{\StochHK,\minplay}}\Expect^{\starredstrat_{\maxplay},\strat_{\minplay}}_{\StochHK,s}(\GRewards)
%%     =
    \inf_{\strat_\minplay\in\SemiMarkovStrats{\StochHK,\minplay}}\Expect^{\starredstrat_{\maxplay},\strat_{\minplay}}_{\StochHK,s}(\GRewards)
    =
    \inf_{\strat_\minplay\in\DetMemorylessStrats{\StochHK,\minplay}}\Expect^{\starredstrat_{\maxplay},\strat_{\minplay}}_{\StochHK,s}(\GRewards)$,
    provided
    $\Expect^{\starredstrat_{\maxplay},\strat_{\minplay}}_{\StochHK,s}(\GRewards)$
    is defined for all
    $\strat_\minplay\in\SemiMarkovStrats{\StochHK,\minplay}$.
  \end{enumerate}
\end{proposition}

We are now in conditions to present our main result.  The following
theorem is two folded.  On the one hand, it states that the polytopal
stochastic games of all quantitative objectives of interest in this
paper --namely, quantitative reachability, expected total accumulated
reward, expected discounted accumulated rewards, and expected average
rewards-- are determined.
%
%% On the other hand, it states that any of these quantiative objectives
%% for PSG can be equivalently solved in its extreme interpretation.
%
On the other hand, it states that these objectives for PSG can be
equivalently solved in its extreme interpretation.



\begin{theorem}\label{th:determinacy:and:discretazation}%
  Let $\StochGK$ and $\StochHK$ be respectively the interpretation and
  the extreme interpretation of $\StochK$.  Then,
  \begin{enumerate}
  \item\label{th:determinacy:and:discretazation:i}%
    $\displaystyle
    \inf_{\strat_\minplay\in\Strategies{\StochGK,\minplay}}\sup_{\strat_\maxplay\in\Strategies{\StochGK,\maxplay}}\Prob^{\strat_{\maxplay},\strat_{\minplay}}_{\StochGK,s}(\Finally C)
    =
    \inf_{\strat_\minplay\in\DetMemorylessStrats{\StochHK,\minplay}}\sup_{\strat_\maxplay\in\DetMemorylessStrats{\StochHK,\maxplay}}\Prob^{\strat_{\maxplay},\strat_{\minplay}}_{\StochHK,s}(\Finally C) = {}$\newline
    \mbox{}\hfill
    $\displaystyle
    {} =
    \sup_{\strat_\maxplay\in\DetMemorylessStrats{\StochHK,\maxplay}}\inf_{\strat_\minplay\in\DetMemorylessStrats{\StochHK,\minplay}}\Prob^{\strat_{\maxplay},\strat_{\minplay}}_{\StochHK,s}(\Finally C)
    =
    \sup_{\strat_\maxplay\in\Strategies{\StochGK,\maxplay}}\inf_{\strat_\minplay\in\Strategies{\StochGK,\minplay}}\Prob^{\strat_{\maxplay},\strat_{\minplay}}_{\StochGK,s}(\Finally C)$\newline
    for all $C\subseteq\sgnodes$; and
  \item\label{th:determinacy:and:discretazation:ii}%
    $\displaystyle
    \inf_{\strat_\minplay\in\Strategies{\StochGK,\minplay}}\sup_{\strat_\maxplay\in\Strategies{\StochGK,\maxplay}}\Expect^{\strat_{\maxplay},\strat_{\minplay}}_{\StochGK,s}(\GRewards)
    =
    \inf_{\strat_\minplay\in\DetMemorylessStrats{\StochHK,\minplay}}\sup_{\strat_\maxplay\in\DetMemorylessStrats{\StochHK,\maxplay}}\Expect^{\strat_{\maxplay},\strat_{\minplay}}_{\StochHK,s}(\GRewards)
    = {}$\newline
    \mbox{}\hfill
    $\displaystyle
    {} =
    \sup_{\strat_\maxplay\in\DetMemorylessStrats{\StochHK,\maxplay}}\inf_{\strat_\minplay\in\DetMemorylessStrats{\StochHK,\minplay}}\Expect^{\strat_{\maxplay},\strat_{\minplay}}_{\StochHK,s}(\GRewards)
    =
    \sup_{\strat_\maxplay\in\Strategies{\StochGK,\maxplay}}\inf_{\strat_\minplay\in\Strategies{\StochGK,\minplay}}\Expect^{\strat_{\maxplay},\strat_{\minplay}}_{\StochGK,s}(\GRewards)$,\newline
    provided:
    %
    \begin{enumerate*}
    \item\label{th:determinacy:and:discretazation:ii:cond:total}%
      $\StochGK$ is almost surely stopping whenever
      $\GRewards=\TRewards$, and
    \item\label{th:determinacy:and:discretazation:ii:cond:average}%
      $\StochGK$ is irreducble whenever $\GRewards=\ARewards$.
    \end{enumerate*}
%%     $\Expect^{\strat_{\maxplay},\strat_{\minplay}}_{\StochHK,s}(\GRewards)$
%%     is defined for all
%%     $\strat_\maxplay\in\Strategies{\StochHK,\maxplay}$, and
%%     $\strat_\minplay\in\Strategies{\StochHK,\minplay}$.
%%     \remarkPRD{cuidado, quiz\'as haya que poner condiciones
%%       particulares para cada modo}
  \end{enumerate}
\end{theorem}
%
\begin{proof}
  For item~\ref{th:determinacy:and:discretazation:ii} we calculate as follows:
  %
  \begin{align*}
    \textstyle
    \inf_{\strat_\minplay\in\Strategies{\StochGK,\minplay}}\sup_{\strat_\maxplay\in\Strategies{\StochGK,\maxplay}}\Expect^{\strat_{\maxplay},\strat_{\minplay}}_{\StochGK,s}(\GRewards)
    \hspace{-14em} &
    \\
    & \textstyle \leq
    \inf_{\strat_\minplay\in\SemiMarkovStrats{\StochGK,\minplay}}\sup_{\strat_\maxplay\in\Strategies{\StochGK,\maxplay}}\Expect^{\strat_{\maxplay},\strat_{\minplay}}_{\StochGK,s}(\GRewards)
    \tag{$\SemiMarkovStrats{\StochGK,\minplay}\subseteq\Strategies{\StochGK,\minplay}$}\\
    & \textstyle =
    \inf_{\strat_\minplay\in\SemiMarkovStrats{\StochGK,\minplay}}\sup_{\strat_\maxplay\in\SemiMarkovStrats{\StochGK,\maxplay}}\Expect^{\strat_{\maxplay},\strat_{\minplay}}_{\StochGK,s}(\GRewards)
    \tag{by Lemma~\ref{lm:semimarkov}.\ref{lm:semimarkov:iii}}\\
    & \textstyle =
    \inf_{\strat_\minplay\in\XSemiMarkovStrats{\StochGK,\minplay}}\sup_{\strat_\maxplay\in\XSemiMarkovStrats{\StochGK,\maxplay}}\Expect^{\strat_{\maxplay},\strat_{\minplay}}_{\StochGK,s}(\GRewards)
    \tag{by Corollary~\ref{cor:xsemimarkov}.\ref{cor:xsemimarkov:ii}}\\
    & \textstyle =
    \inf_{\strat_\minplay\in\SemiMarkovStrats{\StochHK,\minplay}}\sup_{\strat_\maxplay\in\SemiMarkovStrats{\StochHK,\maxplay}}\Expect^{\strat_{\maxplay},\strat_{\minplay}}_{\StochHK,s}(\GRewards)
    \tag{by Prop.~\ref{prop:infsup:supinf:StochGK:StochHK}.\ref{prop:infsup:supinf:StochGK:StochHK:iii}}\\
    & \textstyle \leq
    \inf_{\strat_\minplay\in\DetMemorylessStrats{\StochHK,\minplay}}\sup_{\strat_\maxplay\in\SemiMarkovStrats{\StochHK,\maxplay}}\Expect^{\strat_{\maxplay},\strat_{\minplay}}_{\StochHK,s}(\GRewards)
    \tag{$\DetMemorylessStrats{\StochHK,\minplay}\subseteq\SemiMarkovStrats{\StochHK,\minplay}$}\\
    & \textstyle =
    \inf_{\strat_\minplay\in\DetMemorylessStrats{\StochHK,\minplay}}\sup_{\strat_\maxplay\in\DetMemorylessStrats{\StochHK,\maxplay}}\Expect^{\strat_{\maxplay},\strat_{\minplay}}_{\StochHK,s}(\GRewards)
    \tag{by Prop.~\ref{prop:mdp:results}.\ref{prop:mdp:results:iii}}\\
    & \textstyle =
    \sup_{\strat_\maxplay\in\DetMemorylessStrats{\StochHK,\maxplay}}\inf_{\strat_\minplay\in\DetMemorylessStrats{\StochHK,\minplay}}\Expect^{\strat_{\maxplay},\strat_{\minplay}}_{\StochHK,s}(\GRewards)
    \tag{*}\\
    & \textstyle =
    \sup_{\strat_\maxplay\in\DetMemorylessStrats{\StochHK,\maxplay}}\inf_{\strat_\minplay\in\SemiMarkovStrats{\StochHK,\minplay}}\Expect^{\strat_{\maxplay},\strat_{\minplay}}_{\StochHK,s}(\GRewards)
    \tag{by Prop.~\ref{prop:mdp:results}.\ref{prop:mdp:results:iv}}\\
    & \textstyle \leq
    \sup_{\strat_\maxplay\in\SemiMarkovStrats{\StochHK,\maxplay}}\inf_{\strat_\minplay\in\SemiMarkovStrats{\StochHK,\minplay}}\Expect^{\strat_{\maxplay},\strat_{\minplay}}_{\StochHK,s}(\GRewards)
    \tag{$\DetMemorylessStrats{\StochHK,\maxplay}\subseteq\SemiMarkovStrats{\StochHK,\maxplay}$}\\
    & \textstyle =
    \sup_{\strat_\maxplay\in\XSemiMarkovStrats{\StochGK,\maxplay}}\inf_{\strat_\minplay\in\XSemiMarkovStrats{\StochGK,\minplay}}\Expect^{\strat_{\maxplay},\strat_{\minplay}}_{\StochGK,s}(\GRewards)
    \tag{by Prop.~\ref{prop:infsup:supinf:StochGK:StochHK}.\ref{prop:infsup:supinf:StochGK:StochHK:iv}}\\
    & \textstyle =
    \sup_{\strat_\maxplay\in\SemiMarkovStrats{\StochGK,\maxplay}}\inf_{\strat_\minplay\in\SemiMarkovStrats{\StochGK,\minplay}}\Expect^{\strat_{\maxplay},\strat_{\minplay}}_{\StochGK,s}(\GRewards)
    \tag{by Corollary~\ref{cor:xsemimarkov}.\ref{cor:xsemimarkov:ii}}\\
    & \textstyle =
    \sup_{\strat_\maxplay\in\SemiMarkovStrats{\StochGK,\maxplay}}\inf_{\strat_\minplay\in\Strategies{\StochGK,\minplay}}\Expect^{\strat_{\maxplay},\strat_{\minplay}}_{\StochGK,s}(\GRewards)
    \tag{by Lemma~\ref{lm:semimarkov}.\ref{lm:semimarkov:iii}}\\
    & \textstyle 
    \leq
    \sup_{\strat_\maxplay\in\Strategies{\StochGK,\maxplay}}\inf_{\strat_\minplay\in\Strategies{\StochGK,\minplay}}\Expect^{\strat_{\maxplay},\strat_{\minplay}}_{\StochGK,s}(\GRewards)
    \tag{$\SemiMarkovStrats{\StochGK,\maxplay}\subseteq\Strategies{\StochGK,\maxplay}$}\\
    & \textstyle \leq
    \inf_{\strat_\minplay\in\Strategies{\StochGK,\minplay}}\sup_{\strat_\maxplay\in\Strategies{\StochGK,\maxplay}}\Expect^{\strat_{\maxplay},\strat_{\minplay}}_{\StochGK,s}(\GRewards)
    \tag{by prop. of $\sup$ and $\inf$}
  \end{align*}
  %
  Since the last term is equal to the first term in the calculation,
  item~\ref{th:determinacy:and:discretazation:ii} is concluded.
  %
  In particular, step (*) is justified as follows, depending on
  $\GRewards$:
  %
%%   \begin{itemize}
%%   \item%
    For $\GRewards=\TRewards$, (*) follows by
    \cite[Theorem~4.2.6]{FilarV96} since, by
    Proposition~\ref{prop:stopping:irreducible:StochGK:StochHK}.\ref{prop:stopping:irreducible:StochGK:StochHK:i},
    the game $\StochHK$ is also almost surely stopping.
%%   \item%
    For $\GRewards=\DRewards{\gamma}$ (*) follows by
    \cite[Theorem~4.3.2]{FilarV96}.
%%   \item%
    For $\GRewards=\ARewards$ (*) follows by
    \cite[Theorem~5.1.5]{FilarV96} since, by
    Proposition~\ref{prop:stopping:irreducible:StochGK:StochHK}.\ref{prop:stopping:irreducible:StochGK:StochHK:ii},
    the game $\StochHK$ is also irreducible.
%%   \end{itemize}

  Item~\ref{th:determinacy:and:discretazation:i} of the theorem
  follows similarly. In each step, propositions, lemmas and
  corollaries are the same only differing on the item, while
  step~(*) follows from~\cite[Lemma 6]{Condon92}.
%
  \qed
\end{proof}


Since extreme interpretations are finite, the values of the different
games can be calculated following known
algorithms~\cite{Condon92,FilarV96}.  Thus,
Theorem~\ref{th:determinacy:and:discretazation} immediately provides
an algorithmic solution for PSGs.

The number of vertices of a polytope grows exponentially in the
dimension of the polytope~\cite{KaibelP03}. More precisely if $d$ is
the dimension of a polytope $K$ and $m$ is the number of inequalities
that defines it, $\vertices(K)\sim\Omega(m^{\lfloor{d/2}\rfloor})$.
This implies that the extreme interpretation $\StochHK$ grows
exponentially on the largest size of the support sets of the
distributions involved in the original PSG $\StochK$ which we expect
not to be too large.  (In our example of Sec.~\ref{sec:roborta},
$\lfloor{d/2}\rfloor=2$)

Condon~\cite{Condon92} showed that deciding reachability in stochastic
games is in $\NP \cap \coNP$.  Despite the exponential grow, this is
still our case as we show in the following.
%
Let $\val{s}{\StochK}$ denote the value of the game at state $s$, that
is, it is equal to
%
$\sup_{\strat_{\maxplay} \in \Strategies{\maxplay}} \inf_{\strat_{\minplay} \in \Strategies{\minplay}} \Prob^{\strat_{\maxplay},\strat_{\minplay}}_{\StochGK,s}(\Finally\goal)$,
%
or
%
$\sup_{\strat_{\maxplay} \in \Strategies{\maxplay}} \inf_{\strat_{\minplay} \in \Strategies{\minplay}} \Expect^{\strat_{\maxplay},\strat_{\minplay}}_{\StochGK,s}[\GRewards]$.
%
The problem is then to decide whether $\val{s}{\StochK} \geq q$, for a
given $q \in \mathbb{Q}$ and $s \in \psgnodes$
%
Since for all the cases (total reward, discounted reward, average
reward and reachability objectives under the respective conditions)
the value $\val{s}{\StochK}$ of the game can be achieved with an extreme
memoryless and deterministic strategies, we can reason as follows:
%
\begin{enumerate*}[(i)]
\item%
  guess a memoryless and deterministic strategy for each player,
\item%
  on the resulting Markov chain compute the corresponding measure
  (i.e. total reward, discounted reward, average reward or
  reachability) on the respective set of linear equations, which can
  be done in polynomial time (for $\ARewards$ an extra linear
  summation is needed)~\cite{Kulkarni17}, 
\item%
  verify if it is a fixpoint of Bellman equations (for reachability, discounted, or total reward), or a fixpoint of the Alg.~5.1.1 of \cite{FilarV96}, in the case of average reward, and
\item  check whether $\val{s}{\StochK} \geq q$.
\end{enumerate*}
%
This puts our problem in $\NP$.
%
With the same process we can check whether $\val{s}{\StochK} < q$
which puts the problem also in $\coNP$.  Hence we have the next
theorem.
\begin{theorem}
  For any PSG $\StochK$, $q \in \mathbb{Q}$, and $s \in \psgnodes$,
  the problem of deciding whether $\val{s}{\StochK} \geq q$ is in
  ${\NP} \cap {\coNP}$.
  %
%%   For $\GRewards=\TRewards$ and $\GRewards=\ARewards$
  For $\GRewards\in\{\TRewards, \ARewards\}$
  the decision problem is restricted to
  $\StochGK$ being almost surely stopping and irreducible, respectively.
\end{theorem}
%% \begin{theorem}
%%   For any PSG $\StochK$, $q \in \mathbb{Q}$, and $s \in \psgnodes$,
%%   the following decision problems are in ${\NP} \cap {\coNP}$:
%%   %
%%   \begin{enumerate*}
%%   \item%
%%     $\sup_{\strat_{\maxplay} \in \Strategies{\maxplay}} \inf_{\strat_{\minplay} \in \Strategies{\minplay}} \Prob^{\strat_{\maxplay},\strat_{\minplay}}_{\StochGK,s}(\Finally\goal)\geq q$, and
%%   \item%
%%     $\sup_{\strat_{\maxplay} \in \Strategies{\maxplay}} \inf_{\strat_{\minplay} \in \Strategies{\minplay}} \Expect^{\strat_{\maxplay},\strat_{\minplay}}_{\StochGK,s}[\GRewards]\geq q$
%%     %
%%     provided $\StochGK$ is almost surely stopping whenever
%%     $\GRewards=\TRewards$, and $\StochGK$ is irreducble whenever
%%     $\GRewards=\ARewards$.
%%   \end{enumerate*}
%% \end{theorem}

%% \begin{theorem}
%%   For any PSG $\StochK$, $q \in \mathbb{Q}$, $s \in \psgnodes$, and
%%   ${\bowtie}\in\{{<},{\leq},=,{\geq},{>}\}$, the following decision
%%   problems are in $\NP \cap \coNP$:
%%   %
%%   \begin{enumerate*}
%%   \item%
%%     $\sup_{\strat_{\maxplay} \in \Strategies{\maxplay}} \inf_{\strat_{\minplay} \in \Strategies{\minplay}} \Prob^{\strat_{\maxplay},\strat_{\minplay}}_{\StochGK,s}(\Finally\goal)\bowtie q$, and
%%   \item%
%%     $\sup_{\strat_{\maxplay} \in \Strategies{\maxplay}} \inf_{\strat_{\minplay} \in \Strategies{\minplay}} \Expect^{\strat_{\maxplay},\strat_{\minplay}}_{\StochGK,s}[\GRewards]\bowtie q$
%%     %
%%     provided $\StochGK$ is almost surely stopping whenever
%%     $\GRewards=\TRewards$, and $\StochGK$ is irreducble whenever
%%     $\GRewards=\ARewards$.
%%   \end{enumerate*}
%% \end{theorem}



%% We end this section we some remarks about the complexity of polytopal games.  Given a PSG $\StochK=(\psgnodes,(\psgnmax,\psgnmin),\psgtrans)$, its size  (denoted $| \StochK |$) is given by the 
%% expression: $|S| + \sum_{s \in S} |\Theta(s)|*(|S|+d_{\Theta(s)})$, where $d_{\Theta(s)}$ is the number of constrains in $\Theta(s)$. We assume that the equation system $\Theta(s)$ has $|S|$ variables, it is straightforward to see that this is enough for defining any polytope in $\DPoly(S)$.  Let $\val{s}{\StochK}$ denote the value of the game at state $s$,  that is,  it is equal to $\sup_{\strat_{\maxplay} \in \Strategies{\maxplay}} \inf_{\strat_{\minplay} \in \Strategies{\minplay}} \Prob^{\strat_{\maxplay},\strat_{\minplay}}_{\StochGK,s}(\Finally\goal)$, or $\sup_{\strat_{\maxplay} \in \Strategies{\maxplay}} \inf_{\strat_{\minplay} \in \Strategies{\minplay}} \Expect^{\strat_{\maxplay},\strat_{\minplay}}_{\StochGK,s}[f]$ (for $f \in \{\TRewards, \DRewards,  \ARewards\}$).

%% The decision problem corresponding to a given PSG is to decide if  $\val{s}{\StochK} \geq k$,  for a given $k \in \mathbb{Q}$ and $s \in \psgnodes$.    The following result follows from the fact that polytopal games for  total reward, average reward, discounted reward and reachability objectives admit optimal memoryless and deterministic strategies, and the corresponding Bellman equation has a unique fixpoint.  Thus,  one can guess memoryless and deterministic  strategies for the players, which just choose one polytope for each state,  compute the maximal or minimal solutions of the corresponding linear programs, which can be done in polynomial time,  and verify if this is a fixpoint of the corresponding Bellman equation. Finally,  we check if this value is greater than or equal to  $k$.  For the $\coNP$ part the procedure is exactly the same as the complement of the decision problem is given by the inequation $\val{s}{\StochK} < k$.
%% \begin{theorem} For any polytopal game $\StochK$ its decision problem is in $\NP \cap \coNP$.
%% \end{theorem}


%\section{Experiments}\label{sec:experiments}
%We have implemented the algorithms described in earlier sections as an extension of the {\PRISMGAMES} tool.  
%This tool takes as input a game description in {\PRISMGAMES} notation, which may include 
%linear equations for defining uncertain probabilistic actions, in the style of the one shown in Fig.~\ref{fig:roborta-code}.   We have run the tool with several configurations of the example of Section~\ref{sec:roborta}.  
%All the experiments were run with a MacBook Air M2 with 16GB of memory.
%%
%\begin{table}[t]
%  \begin{minipage}[t]{.46\textwidth}
%    \vspace{0pt}\centering%
%    \begin{tabular}{|c|c|c|c|c|c|c|c|c|c|c|}
%    \hline
%        Size  &  States & Trans. & Value & $\begin{array}{c}\text{Time} \\ \text{(secs)} \end{array}$ \\ \hline
%        $2 {\times} 2$ & 46 & 2050 & 0.115 & 0.029 \\ \hline
%        $3 {\times} 3$ &  213 & 11773 & 0.571 & 0.137 \\ \hline
%        $4 {\times} 4$  &  636 & 38350 & 0.759 & 0.366 \\ \hline
%        $5 {\times} 5$  & 1495 & 93583 & 0.815 & 0.842 \\ \hline
%        $6 {\times} 6$ & 3018 & 193260 & 0.925 & 3.09 \\ \hline
%        $7 {\times} 7$  &  5481 & 357985 & 0.968 & 7.61 \\ \hline
%        $8 {\times} 8$  &  9208 & 607840 & 0.945 & 14.2 \\ \hline
%        $9 {\times} 9$ &  14571 & 972035 & 0.964 & 40.98 \\ \hline
%        $10 {\times} 10$  &21990 & 1477732 & 0.971 & 63.3 \\ \hline
%        $11 {\times} 11$  & 31933 & 2151925 & 0.992 & 85.77 \\ \hline
%        $12 {\times} 12$ &  44916 & 3047446 & 0.993 & 182.3 \\ \hline
%        $13 {\times} 13$  &  61503 & 4180327 & 0.993 & 222.4 \\ \hline
%        $14 {\times} 14$ & 82306 & 5616190 & 0.994 & 387.8 \\ \hline
%        $15 {\times} 15$  &  107985 & 7404777 & 0.996 & 499.1 \\ \hline
%    \end{tabular}\bigskip
%    
%    \caption{Results for the Roborta vs Rigoborto game. Roborta always
%      starts at $(0,0)$, Rigoborto starts at the opposite
%      corner.}\label{table:rob-vs-rig}
%  \end{minipage}
%  \hfill
%  \begin{minipage}[t]{.49\textwidth}
%    \vspace{0pt}\centering%
%    \begin{tabular}{|c|c|c|c|c|c|c|c|}
%    \hline
%        Size &  States & Trans. &Value & $\begin{array}{c}\text{Time} \\ \text{(secs)} \end{array}$ \\ \hline
%        $2 {\times} 2$  & 46 & 2050 & 8.564 & 0.054 \\ \hline
%        $3 {\times} 3$  &  213 & 11773 & 26.652 & 0.192 \\ \hline
%        $4 {\times} 4$  & 636 & 38350 & 62.759 & 0.556 \\ \hline
%        $5 {\times} 5$  &  1495 & 93583 & 136.645 & 3.849 \\ \hline
%        $6 {\times} 6$  & 3018 & 193260 & 256.284 & 13.73 \\ \hline
%        $7 {\times} 7$  & 5481 & 357985 & 681.504 & 96.11 \\ \hline
%        $8 {\times} 8$  &  9208 & 607840 & 851.329 & 316.7 \\ \hline
%        $9 {\times} 9$  &  14571 & 972035 & 2638.003 & 628.4 \\ \hline
%        $10 {\times} 10$ &  21990 & 1477732 & 5120.172 & 1869.3 \\ \hline
%        $11 {\times} 11$ &  31933 & 2151925 & 9660.622 & 5656.9 \\ \hline
%        $12 {\times} 12$  &  44916 & 3047446 & - & T/O \\ \hline
%    \end{tabular}\bigskip
%    
%    \caption{Results for the Roborta vs Rigoborto game with a total reward objective.  Roborta
%      always starts at $(0,0)$, Rigoborto starts at the opposite
%      corner.}\label{table:rob-vs-rig-rews}
%  \end{minipage}
%  \vspace{-1.6em}
%\end{table}
%%
%Table~\ref{table:rob-vs-rig} shows the results obtained for a randomly generated set of configurations of grids of size up to $15 \times 15$ cells.  For each configuration we computed the probability that Roborta wins.  As it can be seen in the table,  the probability that Roborta wins is greater as the size of the grid increases.  
%This outcome is expected,  as the robots begin at opposite corners,
%so the distance between Roborta and Rigoborto increases more than the number of cells Roborta have to cross to win.
%
%We have also run the tool for this example with a total reward objective.  In this case we have randomly added rewards of values $0$ and $1$ in the grid cells.  
%Table~\ref{table:rob-vs-rig-rews} shows the result obtained.  In this case, for grids of size $12 \times 12$, the tool timed out after $2$ hours.  This is mainly due to the iterative methods used by {\PRISMGAMES}. 
%
%\sloppy
%As a second example we consider the analysis of the  Bandwidth Amplification Attack (BAA)  introduced in  \cite{DBLP:conf/edcc/DeshpandeKSS14}.  This example is modeled as a game between an attacker and a defender. The attacker aims to disable a DNS server by flooding it with bogus messages sent via zombie servers,  while the other player is a defender that implements countermeasures to mitigate the consequences of the attack.  
%%
%\begin{wraptable}[20]{r}{.48\textwidth}
%\vspace{-1.5em}
%    \centering\small%
%    \begin{tabular}{|c|c|c|c|c|}
%    \hline
%        $\begin{array}{c}\text{\it max}\\\text{\it Time}\end{array}$ & States & Trans. & Value & $\begin{array}{c}\text{Time} \\ \text{(secs)} \end{array}$ \\ \hline
%        40 & 103618 & 2040677 & 15.47335 & 43.853 \\ \hline
%        41 & 104590 & 2047349 & 16.47335 & 44.113 \\ \hline
%        42 & 108370 & 2099585 & 16.49307 & 51.263 \\ \hline
%        43 & 112960 & 2200007 & 16.51279 & 47.35 \\ \hline
%        44 & 116362 & 2293301 & 16.53252 & 50.498 \\ \hline
%        45 & 116956 & 2305187 & 17.53252 & 56.037 \\ \hline
%        46 & 117928 & 2311859 & 18.53252 & 55.561 \\ \hline
%        47 & 121708 & 2364095 & 18.55224 & 62.705 \\ \hline
%        48 & 126298 & 2464517 & 18.57196 & 64.876 \\ \hline
%        49 & 129700 & 2557811 & 18.59169 & 66.276 \\ \hline
%        50 & 130294 & 2569697 & 19.59169 & 72.954 \\ \hline
%        51 & 131266 & 2576369 & 20.59169 & 74.878 \\ \hline
%        52 & 135046 & 2628605 & 20.61141 & 73.553 \\ \hline
%        53 & 139636 & 2729027 & 20.63113 & 78.143 \\ \hline
%        54 & 143038 & 2822321 & 20.65085 & 80.746 \\ \hline
%    \end{tabular}
% %\vspace{-0.5cm}
%\caption{Results for DNS example. Expected number of legal packages received.
%}\label{table:dns}
%%\end{table}
%\end{wraptable}
%%
%We have modeled three countermeasures. In the first case (\verb"nofix") no countermeasures are taken.  In the second countermeasure (\verb"fdr") \emph{bogus} packages are detected  with probability $\mathit{df}$ and  a false positive rate of $\mathit{fpf}$, both values are obtained from  intervals. For instance, the defender can  select  the value of  $\mathit{df}$ from   interval $[0.85,0.95]$ and the value of $\mathit{fpf}$  from $[0.05,0.15]$.   The last countermeasure considered (\verb"agf") combines  \verb"fdr" with the saturation of the network with \emph{legal} packages, which prevents the attacker from sending bogus packages to the server.
%To ensure termination, we use a constant $\mathit{maxTime}$, which limits the duration of the game.
%In Table~\ref{table:dns}, we report the results obtained when verifying property \verb|<<def>>R{"received_legal"}max=?[F(time=maxTime)]|, i.e.,  the expected number of legal packages that will be received by the server when both players use their best strategies.   

\section{Concluding remarks}

We believe that polytopal games may have several applications in practice,  particularly, in scenarios where the probabilities are not exact but can be characterized with linear equations.  We observe that one may expect that  the number of vertices of the polytopes keep small in practical examples, hence the game discretization may have no impact on the runtime of a tool implementing the approach described in the paper.  However,  we leave as further work the implementation of such a tool and an in-depth evaluation of it.
%

In addition, it would be also be of interest to explore other types of objectives, including $\omega$-regular objectives as already study for standard stochastic games in~\cite{ChatterjeeAH05} or even solving stochastic games for conditional probabilities of temporal properties or conditional expectations of rewards models as widely studied by Christel Baier and her team in the context of Markov decision processes~\cite{BaierKKM14,Baier0KW17,MarckerB0K17,PiribauerB19}.

%% as well as the use of other kinds of objectives including multiobjectives or lexicographic ones.


\bibliographystyle{splncs04}% the mandatory bibstyle
\bibliography{references}


\subsection{Lloyd-Max Algorithm}
\label{subsec:Lloyd-Max}
For a given quantization bitwidth $B$ and an operand $\bm{X}$, the Lloyd-Max algorithm finds $2^B$ quantization levels $\{\hat{x}_i\}_{i=1}^{2^B}$ such that quantizing $\bm{X}$ by rounding each scalar in $\bm{X}$ to the nearest quantization level minimizes the quantization MSE. 

The algorithm starts with an initial guess of quantization levels and then iteratively computes quantization thresholds $\{\tau_i\}_{i=1}^{2^B-1}$ and updates quantization levels $\{\hat{x}_i\}_{i=1}^{2^B}$. Specifically, at iteration $n$, thresholds are set to the midpoints of the previous iteration's levels:
\begin{align*}
    \tau_i^{(n)}=\frac{\hat{x}_i^{(n-1)}+\hat{x}_{i+1}^{(n-1)}}2 \text{ for } i=1\ldots 2^B-1
\end{align*}
Subsequently, the quantization levels are re-computed as conditional means of the data regions defined by the new thresholds:
\begin{align*}
    \hat{x}_i^{(n)}=\mathbb{E}\left[ \bm{X} \big| \bm{X}\in [\tau_{i-1}^{(n)},\tau_i^{(n)}] \right] \text{ for } i=1\ldots 2^B
\end{align*}
where to satisfy boundary conditions we have $\tau_0=-\infty$ and $\tau_{2^B}=\infty$. The algorithm iterates the above steps until convergence.

Figure \ref{fig:lm_quant} compares the quantization levels of a $7$-bit floating point (E3M3) quantizer (left) to a $7$-bit Lloyd-Max quantizer (right) when quantizing a layer of weights from the GPT3-126M model at a per-tensor granularity. As shown, the Lloyd-Max quantizer achieves substantially lower quantization MSE. Further, Table \ref{tab:FP7_vs_LM7} shows the superior perplexity achieved by Lloyd-Max quantizers for bitwidths of $7$, $6$ and $5$. The difference between the quantizers is clear at 5 bits, where per-tensor FP quantization incurs a drastic and unacceptable increase in perplexity, while Lloyd-Max quantization incurs a much smaller increase. Nevertheless, we note that even the optimal Lloyd-Max quantizer incurs a notable ($\sim 1.5$) increase in perplexity due to the coarse granularity of quantization. 

\begin{figure}[h]
  \centering
  \includegraphics[width=0.7\linewidth]{sections/figures/LM7_FP7.pdf}
  \caption{\small Quantization levels and the corresponding quantization MSE of Floating Point (left) vs Lloyd-Max (right) Quantizers for a layer of weights in the GPT3-126M model.}
  \label{fig:lm_quant}
\end{figure}

\begin{table}[h]\scriptsize
\begin{center}
\caption{\label{tab:FP7_vs_LM7} \small Comparing perplexity (lower is better) achieved by floating point quantizers and Lloyd-Max quantizers on a GPT3-126M model for the Wikitext-103 dataset.}
\begin{tabular}{c|cc|c}
\hline
 \multirow{2}{*}{\textbf{Bitwidth}} & \multicolumn{2}{|c|}{\textbf{Floating-Point Quantizer}} & \textbf{Lloyd-Max Quantizer} \\
 & Best Format & Wikitext-103 Perplexity & Wikitext-103 Perplexity \\
\hline
7 & E3M3 & 18.32 & 18.27 \\
6 & E3M2 & 19.07 & 18.51 \\
5 & E4M0 & 43.89 & 19.71 \\
\hline
\end{tabular}
\end{center}
\end{table}

\subsection{Proof of Local Optimality of LO-BCQ}
\label{subsec:lobcq_opt_proof}
For a given block $\bm{b}_j$, the quantization MSE during LO-BCQ can be empirically evaluated as $\frac{1}{L_b}\lVert \bm{b}_j- \bm{\hat{b}}_j\rVert^2_2$ where $\bm{\hat{b}}_j$ is computed from equation (\ref{eq:clustered_quantization_definition}) as $C_{f(\bm{b}_j)}(\bm{b}_j)$. Further, for a given block cluster $\mathcal{B}_i$, we compute the quantization MSE as $\frac{1}{|\mathcal{B}_{i}|}\sum_{\bm{b} \in \mathcal{B}_{i}} \frac{1}{L_b}\lVert \bm{b}- C_i^{(n)}(\bm{b})\rVert^2_2$. Therefore, at the end of iteration $n$, we evaluate the overall quantization MSE $J^{(n)}$ for a given operand $\bm{X}$ composed of $N_c$ block clusters as:
\begin{align*}
    \label{eq:mse_iter_n}
    J^{(n)} = \frac{1}{N_c} \sum_{i=1}^{N_c} \frac{1}{|\mathcal{B}_{i}^{(n)}|}\sum_{\bm{v} \in \mathcal{B}_{i}^{(n)}} \frac{1}{L_b}\lVert \bm{b}- B_i^{(n)}(\bm{b})\rVert^2_2
\end{align*}

At the end of iteration $n$, the codebooks are updated from $\mathcal{C}^{(n-1)}$ to $\mathcal{C}^{(n)}$. However, the mapping of a given vector $\bm{b}_j$ to quantizers $\mathcal{C}^{(n)}$ remains as  $f^{(n)}(\bm{b}_j)$. At the next iteration, during the vector clustering step, $f^{(n+1)}(\bm{b}_j)$ finds new mapping of $\bm{b}_j$ to updated codebooks $\mathcal{C}^{(n)}$ such that the quantization MSE over the candidate codebooks is minimized. Therefore, we obtain the following result for $\bm{b}_j$:
\begin{align*}
\frac{1}{L_b}\lVert \bm{b}_j - C_{f^{(n+1)}(\bm{b}_j)}^{(n)}(\bm{b}_j)\rVert^2_2 \le \frac{1}{L_b}\lVert \bm{b}_j - C_{f^{(n)}(\bm{b}_j)}^{(n)}(\bm{b}_j)\rVert^2_2
\end{align*}

That is, quantizing $\bm{b}_j$ at the end of the block clustering step of iteration $n+1$ results in lower quantization MSE compared to quantizing at the end of iteration $n$. Since this is true for all $\bm{b} \in \bm{X}$, we assert the following:
\begin{equation}
\begin{split}
\label{eq:mse_ineq_1}
    \tilde{J}^{(n+1)} &= \frac{1}{N_c} \sum_{i=1}^{N_c} \frac{1}{|\mathcal{B}_{i}^{(n+1)}|}\sum_{\bm{b} \in \mathcal{B}_{i}^{(n+1)}} \frac{1}{L_b}\lVert \bm{b} - C_i^{(n)}(b)\rVert^2_2 \le J^{(n)}
\end{split}
\end{equation}
where $\tilde{J}^{(n+1)}$ is the the quantization MSE after the vector clustering step at iteration $n+1$.

Next, during the codebook update step (\ref{eq:quantizers_update}) at iteration $n+1$, the per-cluster codebooks $\mathcal{C}^{(n)}$ are updated to $\mathcal{C}^{(n+1)}$ by invoking the Lloyd-Max algorithm \citep{Lloyd}. We know that for any given value distribution, the Lloyd-Max algorithm minimizes the quantization MSE. Therefore, for a given vector cluster $\mathcal{B}_i$ we obtain the following result:

\begin{equation}
    \frac{1}{|\mathcal{B}_{i}^{(n+1)}|}\sum_{\bm{b} \in \mathcal{B}_{i}^{(n+1)}} \frac{1}{L_b}\lVert \bm{b}- C_i^{(n+1)}(\bm{b})\rVert^2_2 \le \frac{1}{|\mathcal{B}_{i}^{(n+1)}|}\sum_{\bm{b} \in \mathcal{B}_{i}^{(n+1)}} \frac{1}{L_b}\lVert \bm{b}- C_i^{(n)}(\bm{b})\rVert^2_2
\end{equation}

The above equation states that quantizing the given block cluster $\mathcal{B}_i$ after updating the associated codebook from $C_i^{(n)}$ to $C_i^{(n+1)}$ results in lower quantization MSE. Since this is true for all the block clusters, we derive the following result: 
\begin{equation}
\begin{split}
\label{eq:mse_ineq_2}
     J^{(n+1)} &= \frac{1}{N_c} \sum_{i=1}^{N_c} \frac{1}{|\mathcal{B}_{i}^{(n+1)}|}\sum_{\bm{b} \in \mathcal{B}_{i}^{(n+1)}} \frac{1}{L_b}\lVert \bm{b}- C_i^{(n+1)}(\bm{b})\rVert^2_2  \le \tilde{J}^{(n+1)}   
\end{split}
\end{equation}

Following (\ref{eq:mse_ineq_1}) and (\ref{eq:mse_ineq_2}), we find that the quantization MSE is non-increasing for each iteration, that is, $J^{(1)} \ge J^{(2)} \ge J^{(3)} \ge \ldots \ge J^{(M)}$ where $M$ is the maximum number of iterations. 
%Therefore, we can say that if the algorithm converges, then it must be that it has converged to a local minimum. 
\hfill $\blacksquare$


\begin{figure}
    \begin{center}
    \includegraphics[width=0.5\textwidth]{sections//figures/mse_vs_iter.pdf}
    \end{center}
    \caption{\small NMSE vs iterations during LO-BCQ compared to other block quantization proposals}
    \label{fig:nmse_vs_iter}
\end{figure}

Figure \ref{fig:nmse_vs_iter} shows the empirical convergence of LO-BCQ across several block lengths and number of codebooks. Also, the MSE achieved by LO-BCQ is compared to baselines such as MXFP and VSQ. As shown, LO-BCQ converges to a lower MSE than the baselines. Further, we achieve better convergence for larger number of codebooks ($N_c$) and for a smaller block length ($L_b$), both of which increase the bitwidth of BCQ (see Eq \ref{eq:bitwidth_bcq}).


\subsection{Additional Accuracy Results}
%Table \ref{tab:lobcq_config} lists the various LOBCQ configurations and their corresponding bitwidths.
\begin{table}
\setlength{\tabcolsep}{4.75pt}
\begin{center}
\caption{\label{tab:lobcq_config} Various LO-BCQ configurations and their bitwidths.}
\begin{tabular}{|c||c|c|c|c||c|c||c|} 
\hline
 & \multicolumn{4}{|c||}{$L_b=8$} & \multicolumn{2}{|c||}{$L_b=4$} & $L_b=2$ \\
 \hline
 \backslashbox{$L_A$\kern-1em}{\kern-1em$N_c$} & 2 & 4 & 8 & 16 & 2 & 4 & 2 \\
 \hline
 64 & 4.25 & 4.375 & 4.5 & 4.625 & 4.375 & 4.625 & 4.625\\
 \hline
 32 & 4.375 & 4.5 & 4.625& 4.75 & 4.5 & 4.75 & 4.75 \\
 \hline
 16 & 4.625 & 4.75& 4.875 & 5 & 4.75 & 5 & 5 \\
 \hline
\end{tabular}
\end{center}
\end{table}

%\subsection{Perplexity achieved by various LO-BCQ configurations on Wikitext-103 dataset}

\begin{table} \centering
\begin{tabular}{|c||c|c|c|c||c|c||c|} 
\hline
 $L_b \rightarrow$& \multicolumn{4}{c||}{8} & \multicolumn{2}{c||}{4} & 2\\
 \hline
 \backslashbox{$L_A$\kern-1em}{\kern-1em$N_c$} & 2 & 4 & 8 & 16 & 2 & 4 & 2  \\
 %$N_c \rightarrow$ & 2 & 4 & 8 & 16 & 2 & 4 & 2 \\
 \hline
 \hline
 \multicolumn{8}{c}{GPT3-1.3B (FP32 PPL = 9.98)} \\ 
 \hline
 \hline
 64 & 10.40 & 10.23 & 10.17 & 10.15 &  10.28 & 10.18 & 10.19 \\
 \hline
 32 & 10.25 & 10.20 & 10.15 & 10.12 &  10.23 & 10.17 & 10.17 \\
 \hline
 16 & 10.22 & 10.16 & 10.10 & 10.09 &  10.21 & 10.14 & 10.16 \\
 \hline
  \hline
 \multicolumn{8}{c}{GPT3-8B (FP32 PPL = 7.38)} \\ 
 \hline
 \hline
 64 & 7.61 & 7.52 & 7.48 &  7.47 &  7.55 &  7.49 & 7.50 \\
 \hline
 32 & 7.52 & 7.50 & 7.46 &  7.45 &  7.52 &  7.48 & 7.48  \\
 \hline
 16 & 7.51 & 7.48 & 7.44 &  7.44 &  7.51 &  7.49 & 7.47  \\
 \hline
\end{tabular}
\caption{\label{tab:ppl_gpt3_abalation} Wikitext-103 perplexity across GPT3-1.3B and 8B models.}
\end{table}

\begin{table} \centering
\begin{tabular}{|c||c|c|c|c||} 
\hline
 $L_b \rightarrow$& \multicolumn{4}{c||}{8}\\
 \hline
 \backslashbox{$L_A$\kern-1em}{\kern-1em$N_c$} & 2 & 4 & 8 & 16 \\
 %$N_c \rightarrow$ & 2 & 4 & 8 & 16 & 2 & 4 & 2 \\
 \hline
 \hline
 \multicolumn{5}{|c|}{Llama2-7B (FP32 PPL = 5.06)} \\ 
 \hline
 \hline
 64 & 5.31 & 5.26 & 5.19 & 5.18  \\
 \hline
 32 & 5.23 & 5.25 & 5.18 & 5.15  \\
 \hline
 16 & 5.23 & 5.19 & 5.16 & 5.14  \\
 \hline
 \multicolumn{5}{|c|}{Nemotron4-15B (FP32 PPL = 5.87)} \\ 
 \hline
 \hline
 64  & 6.3 & 6.20 & 6.13 & 6.08  \\
 \hline
 32  & 6.24 & 6.12 & 6.07 & 6.03  \\
 \hline
 16  & 6.12 & 6.14 & 6.04 & 6.02  \\
 \hline
 \multicolumn{5}{|c|}{Nemotron4-340B (FP32 PPL = 3.48)} \\ 
 \hline
 \hline
 64 & 3.67 & 3.62 & 3.60 & 3.59 \\
 \hline
 32 & 3.63 & 3.61 & 3.59 & 3.56 \\
 \hline
 16 & 3.61 & 3.58 & 3.57 & 3.55 \\
 \hline
\end{tabular}
\caption{\label{tab:ppl_llama7B_nemo15B} Wikitext-103 perplexity compared to FP32 baseline in Llama2-7B and Nemotron4-15B, 340B models}
\end{table}

%\subsection{Perplexity achieved by various LO-BCQ configurations on MMLU dataset}


\begin{table} \centering
\begin{tabular}{|c||c|c|c|c||c|c|c|c|} 
\hline
 $L_b \rightarrow$& \multicolumn{4}{c||}{8} & \multicolumn{4}{c||}{8}\\
 \hline
 \backslashbox{$L_A$\kern-1em}{\kern-1em$N_c$} & 2 & 4 & 8 & 16 & 2 & 4 & 8 & 16  \\
 %$N_c \rightarrow$ & 2 & 4 & 8 & 16 & 2 & 4 & 2 \\
 \hline
 \hline
 \multicolumn{5}{|c|}{Llama2-7B (FP32 Accuracy = 45.8\%)} & \multicolumn{4}{|c|}{Llama2-70B (FP32 Accuracy = 69.12\%)} \\ 
 \hline
 \hline
 64 & 43.9 & 43.4 & 43.9 & 44.9 & 68.07 & 68.27 & 68.17 & 68.75 \\
 \hline
 32 & 44.5 & 43.8 & 44.9 & 44.5 & 68.37 & 68.51 & 68.35 & 68.27  \\
 \hline
 16 & 43.9 & 42.7 & 44.9 & 45 & 68.12 & 68.77 & 68.31 & 68.59  \\
 \hline
 \hline
 \multicolumn{5}{|c|}{GPT3-22B (FP32 Accuracy = 38.75\%)} & \multicolumn{4}{|c|}{Nemotron4-15B (FP32 Accuracy = 64.3\%)} \\ 
 \hline
 \hline
 64 & 36.71 & 38.85 & 38.13 & 38.92 & 63.17 & 62.36 & 63.72 & 64.09 \\
 \hline
 32 & 37.95 & 38.69 & 39.45 & 38.34 & 64.05 & 62.30 & 63.8 & 64.33  \\
 \hline
 16 & 38.88 & 38.80 & 38.31 & 38.92 & 63.22 & 63.51 & 63.93 & 64.43  \\
 \hline
\end{tabular}
\caption{\label{tab:mmlu_abalation} Accuracy on MMLU dataset across GPT3-22B, Llama2-7B, 70B and Nemotron4-15B models.}
\end{table}


%\subsection{Perplexity achieved by various LO-BCQ configurations on LM evaluation harness}

\begin{table} \centering
\begin{tabular}{|c||c|c|c|c||c|c|c|c|} 
\hline
 $L_b \rightarrow$& \multicolumn{4}{c||}{8} & \multicolumn{4}{c||}{8}\\
 \hline
 \backslashbox{$L_A$\kern-1em}{\kern-1em$N_c$} & 2 & 4 & 8 & 16 & 2 & 4 & 8 & 16  \\
 %$N_c \rightarrow$ & 2 & 4 & 8 & 16 & 2 & 4 & 2 \\
 \hline
 \hline
 \multicolumn{5}{|c|}{Race (FP32 Accuracy = 37.51\%)} & \multicolumn{4}{|c|}{Boolq (FP32 Accuracy = 64.62\%)} \\ 
 \hline
 \hline
 64 & 36.94 & 37.13 & 36.27 & 37.13 & 63.73 & 62.26 & 63.49 & 63.36 \\
 \hline
 32 & 37.03 & 36.36 & 36.08 & 37.03 & 62.54 & 63.51 & 63.49 & 63.55  \\
 \hline
 16 & 37.03 & 37.03 & 36.46 & 37.03 & 61.1 & 63.79 & 63.58 & 63.33  \\
 \hline
 \hline
 \multicolumn{5}{|c|}{Winogrande (FP32 Accuracy = 58.01\%)} & \multicolumn{4}{|c|}{Piqa (FP32 Accuracy = 74.21\%)} \\ 
 \hline
 \hline
 64 & 58.17 & 57.22 & 57.85 & 58.33 & 73.01 & 73.07 & 73.07 & 72.80 \\
 \hline
 32 & 59.12 & 58.09 & 57.85 & 58.41 & 73.01 & 73.94 & 72.74 & 73.18  \\
 \hline
 16 & 57.93 & 58.88 & 57.93 & 58.56 & 73.94 & 72.80 & 73.01 & 73.94  \\
 \hline
\end{tabular}
\caption{\label{tab:mmlu_abalation} Accuracy on LM evaluation harness tasks on GPT3-1.3B model.}
\end{table}

\begin{table} \centering
\begin{tabular}{|c||c|c|c|c||c|c|c|c|} 
\hline
 $L_b \rightarrow$& \multicolumn{4}{c||}{8} & \multicolumn{4}{c||}{8}\\
 \hline
 \backslashbox{$L_A$\kern-1em}{\kern-1em$N_c$} & 2 & 4 & 8 & 16 & 2 & 4 & 8 & 16  \\
 %$N_c \rightarrow$ & 2 & 4 & 8 & 16 & 2 & 4 & 2 \\
 \hline
 \hline
 \multicolumn{5}{|c|}{Race (FP32 Accuracy = 41.34\%)} & \multicolumn{4}{|c|}{Boolq (FP32 Accuracy = 68.32\%)} \\ 
 \hline
 \hline
 64 & 40.48 & 40.10 & 39.43 & 39.90 & 69.20 & 68.41 & 69.45 & 68.56 \\
 \hline
 32 & 39.52 & 39.52 & 40.77 & 39.62 & 68.32 & 67.43 & 68.17 & 69.30  \\
 \hline
 16 & 39.81 & 39.71 & 39.90 & 40.38 & 68.10 & 66.33 & 69.51 & 69.42  \\
 \hline
 \hline
 \multicolumn{5}{|c|}{Winogrande (FP32 Accuracy = 67.88\%)} & \multicolumn{4}{|c|}{Piqa (FP32 Accuracy = 78.78\%)} \\ 
 \hline
 \hline
 64 & 66.85 & 66.61 & 67.72 & 67.88 & 77.31 & 77.42 & 77.75 & 77.64 \\
 \hline
 32 & 67.25 & 67.72 & 67.72 & 67.00 & 77.31 & 77.04 & 77.80 & 77.37  \\
 \hline
 16 & 68.11 & 68.90 & 67.88 & 67.48 & 77.37 & 78.13 & 78.13 & 77.69  \\
 \hline
\end{tabular}
\caption{\label{tab:mmlu_abalation} Accuracy on LM evaluation harness tasks on GPT3-8B model.}
\end{table}

\begin{table} \centering
\begin{tabular}{|c||c|c|c|c||c|c|c|c|} 
\hline
 $L_b \rightarrow$& \multicolumn{4}{c||}{8} & \multicolumn{4}{c||}{8}\\
 \hline
 \backslashbox{$L_A$\kern-1em}{\kern-1em$N_c$} & 2 & 4 & 8 & 16 & 2 & 4 & 8 & 16  \\
 %$N_c \rightarrow$ & 2 & 4 & 8 & 16 & 2 & 4 & 2 \\
 \hline
 \hline
 \multicolumn{5}{|c|}{Race (FP32 Accuracy = 40.67\%)} & \multicolumn{4}{|c|}{Boolq (FP32 Accuracy = 76.54\%)} \\ 
 \hline
 \hline
 64 & 40.48 & 40.10 & 39.43 & 39.90 & 75.41 & 75.11 & 77.09 & 75.66 \\
 \hline
 32 & 39.52 & 39.52 & 40.77 & 39.62 & 76.02 & 76.02 & 75.96 & 75.35  \\
 \hline
 16 & 39.81 & 39.71 & 39.90 & 40.38 & 75.05 & 73.82 & 75.72 & 76.09  \\
 \hline
 \hline
 \multicolumn{5}{|c|}{Winogrande (FP32 Accuracy = 70.64\%)} & \multicolumn{4}{|c|}{Piqa (FP32 Accuracy = 79.16\%)} \\ 
 \hline
 \hline
 64 & 69.14 & 70.17 & 70.17 & 70.56 & 78.24 & 79.00 & 78.62 & 78.73 \\
 \hline
 32 & 70.96 & 69.69 & 71.27 & 69.30 & 78.56 & 79.49 & 79.16 & 78.89  \\
 \hline
 16 & 71.03 & 69.53 & 69.69 & 70.40 & 78.13 & 79.16 & 79.00 & 79.00  \\
 \hline
\end{tabular}
\caption{\label{tab:mmlu_abalation} Accuracy on LM evaluation harness tasks on GPT3-22B model.}
\end{table}

\begin{table} \centering
\begin{tabular}{|c||c|c|c|c||c|c|c|c|} 
\hline
 $L_b \rightarrow$& \multicolumn{4}{c||}{8} & \multicolumn{4}{c||}{8}\\
 \hline
 \backslashbox{$L_A$\kern-1em}{\kern-1em$N_c$} & 2 & 4 & 8 & 16 & 2 & 4 & 8 & 16  \\
 %$N_c \rightarrow$ & 2 & 4 & 8 & 16 & 2 & 4 & 2 \\
 \hline
 \hline
 \multicolumn{5}{|c|}{Race (FP32 Accuracy = 44.4\%)} & \multicolumn{4}{|c|}{Boolq (FP32 Accuracy = 79.29\%)} \\ 
 \hline
 \hline
 64 & 42.49 & 42.51 & 42.58 & 43.45 & 77.58 & 77.37 & 77.43 & 78.1 \\
 \hline
 32 & 43.35 & 42.49 & 43.64 & 43.73 & 77.86 & 75.32 & 77.28 & 77.86  \\
 \hline
 16 & 44.21 & 44.21 & 43.64 & 42.97 & 78.65 & 77 & 76.94 & 77.98  \\
 \hline
 \hline
 \multicolumn{5}{|c|}{Winogrande (FP32 Accuracy = 69.38\%)} & \multicolumn{4}{|c|}{Piqa (FP32 Accuracy = 78.07\%)} \\ 
 \hline
 \hline
 64 & 68.9 & 68.43 & 69.77 & 68.19 & 77.09 & 76.82 & 77.09 & 77.86 \\
 \hline
 32 & 69.38 & 68.51 & 68.82 & 68.90 & 78.07 & 76.71 & 78.07 & 77.86  \\
 \hline
 16 & 69.53 & 67.09 & 69.38 & 68.90 & 77.37 & 77.8 & 77.91 & 77.69  \\
 \hline
\end{tabular}
\caption{\label{tab:mmlu_abalation} Accuracy on LM evaluation harness tasks on Llama2-7B model.}
\end{table}

\begin{table} \centering
\begin{tabular}{|c||c|c|c|c||c|c|c|c|} 
\hline
 $L_b \rightarrow$& \multicolumn{4}{c||}{8} & \multicolumn{4}{c||}{8}\\
 \hline
 \backslashbox{$L_A$\kern-1em}{\kern-1em$N_c$} & 2 & 4 & 8 & 16 & 2 & 4 & 8 & 16  \\
 %$N_c \rightarrow$ & 2 & 4 & 8 & 16 & 2 & 4 & 2 \\
 \hline
 \hline
 \multicolumn{5}{|c|}{Race (FP32 Accuracy = 48.8\%)} & \multicolumn{4}{|c|}{Boolq (FP32 Accuracy = 85.23\%)} \\ 
 \hline
 \hline
 64 & 49.00 & 49.00 & 49.28 & 48.71 & 82.82 & 84.28 & 84.03 & 84.25 \\
 \hline
 32 & 49.57 & 48.52 & 48.33 & 49.28 & 83.85 & 84.46 & 84.31 & 84.93  \\
 \hline
 16 & 49.85 & 49.09 & 49.28 & 48.99 & 85.11 & 84.46 & 84.61 & 83.94  \\
 \hline
 \hline
 \multicolumn{5}{|c|}{Winogrande (FP32 Accuracy = 79.95\%)} & \multicolumn{4}{|c|}{Piqa (FP32 Accuracy = 81.56\%)} \\ 
 \hline
 \hline
 64 & 78.77 & 78.45 & 78.37 & 79.16 & 81.45 & 80.69 & 81.45 & 81.5 \\
 \hline
 32 & 78.45 & 79.01 & 78.69 & 80.66 & 81.56 & 80.58 & 81.18 & 81.34  \\
 \hline
 16 & 79.95 & 79.56 & 79.79 & 79.72 & 81.28 & 81.66 & 81.28 & 80.96  \\
 \hline
\end{tabular}
\caption{\label{tab:mmlu_abalation} Accuracy on LM evaluation harness tasks on Llama2-70B model.}
\end{table}

%\section{MSE Studies}
%\textcolor{red}{TODO}


\subsection{Number Formats and Quantization Method}
\label{subsec:numFormats_quantMethod}
\subsubsection{Integer Format}
An $n$-bit signed integer (INT) is typically represented with a 2s-complement format \citep{yao2022zeroquant,xiao2023smoothquant,dai2021vsq}, where the most significant bit denotes the sign.

\subsubsection{Floating Point Format}
An $n$-bit signed floating point (FP) number $x$ comprises of a 1-bit sign ($x_{\mathrm{sign}}$), $B_m$-bit mantissa ($x_{\mathrm{mant}}$) and $B_e$-bit exponent ($x_{\mathrm{exp}}$) such that $B_m+B_e=n-1$. The associated constant exponent bias ($E_{\mathrm{bias}}$) is computed as $(2^{{B_e}-1}-1)$. We denote this format as $E_{B_e}M_{B_m}$.  

\subsubsection{Quantization Scheme}
\label{subsec:quant_method}
A quantization scheme dictates how a given unquantized tensor is converted to its quantized representation. We consider FP formats for the purpose of illustration. Given an unquantized tensor $\bm{X}$ and an FP format $E_{B_e}M_{B_m}$, we first, we compute the quantization scale factor $s_X$ that maps the maximum absolute value of $\bm{X}$ to the maximum quantization level of the $E_{B_e}M_{B_m}$ format as follows:
\begin{align}
\label{eq:sf}
    s_X = \frac{\mathrm{max}(|\bm{X}|)}{\mathrm{max}(E_{B_e}M_{B_m})}
\end{align}
In the above equation, $|\cdot|$ denotes the absolute value function.

Next, we scale $\bm{X}$ by $s_X$ and quantize it to $\hat{\bm{X}}$ by rounding it to the nearest quantization level of $E_{B_e}M_{B_m}$ as:

\begin{align}
\label{eq:tensor_quant}
    \hat{\bm{X}} = \text{round-to-nearest}\left(\frac{\bm{X}}{s_X}, E_{B_e}M_{B_m}\right)
\end{align}

We perform dynamic max-scaled quantization \citep{wu2020integer}, where the scale factor $s$ for activations is dynamically computed during runtime.

\subsection{Vector Scaled Quantization}
\begin{wrapfigure}{r}{0.35\linewidth}
  \centering
  \includegraphics[width=\linewidth]{sections/figures/vsquant.jpg}
  \caption{\small Vectorwise decomposition for per-vector scaled quantization (VSQ \citep{dai2021vsq}).}
  \label{fig:vsquant}
\end{wrapfigure}
During VSQ \citep{dai2021vsq}, the operand tensors are decomposed into 1D vectors in a hardware friendly manner as shown in Figure \ref{fig:vsquant}. Since the decomposed tensors are used as operands in matrix multiplications during inference, it is beneficial to perform this decomposition along the reduction dimension of the multiplication. The vectorwise quantization is performed similar to tensorwise quantization described in Equations \ref{eq:sf} and \ref{eq:tensor_quant}, where a scale factor $s_v$ is required for each vector $\bm{v}$ that maps the maximum absolute value of that vector to the maximum quantization level. While smaller vector lengths can lead to larger accuracy gains, the associated memory and computational overheads due to the per-vector scale factors increases. To alleviate these overheads, VSQ \citep{dai2021vsq} proposed a second level quantization of the per-vector scale factors to unsigned integers, while MX \citep{rouhani2023shared} quantizes them to integer powers of 2 (denoted as $2^{INT}$).

\subsubsection{MX Format}
The MX format proposed in \citep{rouhani2023microscaling} introduces the concept of sub-block shifting. For every two scalar elements of $b$-bits each, there is a shared exponent bit. The value of this exponent bit is determined through an empirical analysis that targets minimizing quantization MSE. We note that the FP format $E_{1}M_{b}$ is strictly better than MX from an accuracy perspective since it allocates a dedicated exponent bit to each scalar as opposed to sharing it across two scalars. Therefore, we conservatively bound the accuracy of a $b+2$-bit signed MX format with that of a $E_{1}M_{b}$ format in our comparisons. For instance, we use E1M2 format as a proxy for MX4.

\begin{figure}
    \centering
    \includegraphics[width=1\linewidth]{sections//figures/BlockFormats.pdf}
    \caption{\small Comparing LO-BCQ to MX format.}
    \label{fig:block_formats}
\end{figure}

Figure \ref{fig:block_formats} compares our $4$-bit LO-BCQ block format to MX \citep{rouhani2023microscaling}. As shown, both LO-BCQ and MX decompose a given operand tensor into block arrays and each block array into blocks. Similar to MX, we find that per-block quantization ($L_b < L_A$) leads to better accuracy due to increased flexibility. While MX achieves this through per-block $1$-bit micro-scales, we associate a dedicated codebook to each block through a per-block codebook selector. Further, MX quantizes the per-block array scale-factor to E8M0 format without per-tensor scaling. In contrast during LO-BCQ, we find that per-tensor scaling combined with quantization of per-block array scale-factor to E4M3 format results in superior inference accuracy across models. 


\end{document}
