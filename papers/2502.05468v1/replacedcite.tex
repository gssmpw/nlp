\section{Related Works}
Decision-making under uncertainty has driven research in decision-focused learning, robust optimization, and risk-aware optimization. We review these approaches and their limitations for high-dimensional uncertainty and risk-sensitive decisions, motivating our proposed framework.

Decision-focused learning (DFL) enhances decision-making under uncertainty by integrating prediction and optimization into a single framework. ____ showed that optimizing predictive models for decision outcomes improves financial performance. 
Differentiable optimization layers have further expanded DFL applications ____. For example, ____ introduced differentiable quadratic programs, enabling backpropagation through constrained optimization, while ____ extended this to all convex programs. Parallel work has explored integrating integer programming into neural networks____.

However, existing DFL methods rely on single-point predictions, failing to capture uncertainty and leading to suboptimal decisions____. Additionally, they typically optimize for average-case performance, making them unsuitable for risk-sensitive applications____. Approaches like Conformal-Predict-Then-Optimize (CPO)____ attempt to address this by constructing fixed uncertainty sets but can be overly conservative, especially in high-dimensional settings.

Robust Optimization (RO) provides a principled approach to decision-making under uncertainty by ensuring solutions remain feasible under the worst-case scenario____. Instead of relying on probabilistic assumptions about uncertain parameters, RO constructs uncertainty sets that define the range of possible parameter values ____ and aims to find the decision that is robust against the worst-case in the uncertainty sets. This approach has found applications in domains such as supply chains ____, currency portfolio management ____, and power system optimization ____. 

% \woody{There are two limitations in RO: 1. difficult to specify the uncertainty set. most of the literature relies on the heuristic choice of uncertainty set; 2. being overly conservative due to relying on the single worst case scenarios.}
Despite its guarantees, the solutions suggested by RO suffer from two major limitations: 
($i$) Uncertainty set construction usually relies on heuristic choices, making it difficult to capture the real dynamics in the real-world applications____.
($ii$) Such pre-specified uncertainty sets tend to be overly conservative____ as it focuses solely on the worst-case outcome, whereas many high-stakes applications require accounting for multiple adverse scenarios.
% To alleviate this limitation, conditional value-at-risk (CVaR) has been widely adopted as a measure for quantifying high-loss scenarios beyond a given risk level____. CVaR optimizes over the worst $\alpha$-quantile of possible outcomes, providing a principled approach to minimizing expected costs in tail regions.

% \woody{Consider adding a discussion on generative model somewhere in the related work: The proposed framework also relates to generative modeling ...}

The proposed framework also relates to generative modeling. Generative modeling has shown promise for a number of fields such as image generation____, chemical species design____, and trajectory planning____. Recently, flow-based generative modeling approaches outperform others by establishing a mapping between complex distributions and a simple prior directly____. In this study, we also adopt the flow-based method, conditional normalizing flows (CNFs)____, to capture the target distribution in high-risk regions.