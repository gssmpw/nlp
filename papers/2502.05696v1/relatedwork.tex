\section{Related Work}
\label{sec:related}
To the best of our knowledge, no previous work explicitly studied one-shot dynamic soft tool use. So, we expand our literature to relevant areas, including manipulation of soft bodies and heterogeneous systems and identification of unobservable physical properties.
\subsection{Soft Bodies and Heterogeneous Systems Manipulation}
Since we use deformable linear objects (DLO) such as ropes as tools, we provide a review on DLO dynamic manipulation. Generally, tasks in this area aim to make a soft body reach a certain full or partial configuration. Such a problem needs to precisely model the rope dynamic patterns. Maybe the knocking task of the work from~\cite{zhang2021robots} has the closest relation to soft tool use. The method can perform rope vaulting and weaving and can knock down an object at a distance by whipping a rope. However, the interaction between the object and the rope is simple and short. In addition, the framework is unable to generalize to different environments without retraining. {Real2Sim2real}~\cite{9811651} has a similar casting moving primitive to our task. It proposes a planar robot casting task, where a robot gripper holds a cable at one endpoint and swings it along a planar surface so that the other end reaches a target position. However, this work does not deal with heterogeneous systems and also requires re-training to tackle any changes in the physical properties of the system. To enable adaptation to different physics properties,~\cite{chi2022irp} proposed to leverage the residual physics~\cite{zeng2020tossingbot} and neural network to iteratively refine the generated actions for goal-conditioned soft body manipulation tasks, such as rope whipping that require the rope end to reach certain positions. However, this approach also does not handle heterogeneous systems and cannot be applied to applications requiring task accomplishment in a single shot.  

Manipulation of heterogeneous systems involving both soft and rigid bodies remains relatively unexplored in the field of robotics. A seminal study by~\cite{donald2000distributed} presents a model-based, manually designed framework for packing and moving multiple objects using ropes, serving as a foundational exploration in this area. Subsequent research by~\cite{corke2000experiments,maneewarn2005mechanics} provides extensive experiment results and analyzes the mechanics of the approach presented in~\cite{donald2000distributed}. Recent methodologies~\cite{10225274, wang2023deriigp} in this domain leverage data-driven, model-free techniques. DeRi-Bot~\cite{10225274} focuses on moving a rigid body block to a target position by controlling robotic arms to pull connected ropes, addressing the intricate dynamics and stochastic nature of soft-rigid body systems with neural networks. The work by~\cite{wang2023deriigp} extends on DeRi-Bot and proposes a grasp-pull moving primitive with a subgoal planner for this task, significantly enhancing performance and generalization capabilities. To our knowledge, there is no existing work exclusively studying the dynamic, high-speed manipulation of heterogeneous systems. Compared to existing methods, our IPA policy performs more complex physics-sensitive tasks that require dynamic, one-shot manipulation of such systems. 

\subsection{Unobservable Physical Properties Identification}
\label{subsec:upps}
Certain unobservable physical properties, such as mass and friction, are crucial for various robotic tasks. However, these properties cannot be directly estimated through visual perception. Existing approaches estimate these properties through interaction with the environment and are divided into explicit and implicit methods.

The explicit methods use sensors to obtain physical readings. For instance, earlier studies by ~\cite{atkeson1986estimation} and~\cite{1570351} leverage force-torque sensors and mathematical models to estimate a grasped object's mass and moments of inertia through interactions. More recent approaches employ physical simulators and neural networks to efficiently estimate the physics properties~\cite{NIPS2015_d09bf415,NIPS2017_4c56ff4c,fragkiadaki2016learning}. However, these methods involve explicit estimations, leading to a significant sim-to-real gap. Additionally, obtaining real-world labels is costly due to the need for external sensors and is also error-prone.

The implicit estimation methods leverage the robot's interaction with the environment to indirectly capture the physics parameters governing the underlying system. The interaction is usually represented by a robot's action and the system's response to that action. The pioneering approaches for implicit physics estimation include PushNet~\cite{Li2018PushNetDP}, DensePhysNet~\cite{xu2019densephysnet}, and SwingBot~\cite{wang2020swingbot}. These methods are designed for rigid-body objects. The PushNet solves for a single-object planar pushing tasks. The DensePhysNet improves the PushNet method by learning pixel-wise representations from two non-quasi-static interactions: planar sliding and collision. Such a design allows it to dynamically manipulate multiple novel objects. Finally, the SwingBot approach leverages a tactile perception~\cite{yuan2017gelsight} to estimate physical properties through the shaking and tilting action. The tactile-based feedback aids the robot in executing the dynamic swing-up manipulation tasks. Inspired by these approaches, our IPA policy also senses physics properties indirectly by executing non-quasi-static actions and recording the system's response. These short-horizon interactions are utilized as a replacement for expansive physics labels. Note that the prior work considers rigid objects, whereas our approach uses the interactions to estimate the physics of a heterogenous system comprising DLO and rigid-body objects.


\begin{figure*}[t]
\vspace{2.5mm}
    \includegraphics[width=\textwidth]{imgs/pipeline.pdf}
    \caption{The workflow of the IPA policy for object transport task. IPA starts with implicitly identifying the related physical properties by performing a predefined action and recording the moving trajectory of the object. Next, the framework observes the environment to obtain the depth map and segmentation map encoding the environment and task configurations. Then, the IPA policy takes as input the aforementioned data to output a suitable action.}
    \label{fig:pipeline}
\vspace{-4.5mm}
\end{figure*}