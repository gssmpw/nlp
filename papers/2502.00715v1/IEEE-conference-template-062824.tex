\documentclass[conference]{IEEEtran}
%\usepackage{multirow}
\IEEEoverridecommandlockouts

\usepackage{cite}
\usepackage{amsmath,amssymb,amsfonts}
%\usepackage{algorithmic}
\usepackage{graphicx}
\usepackage{textcomp}
\usepackage{xcolor}
\usepackage{ifthen}
\usepackage{subcaption}
\usepackage{comment}
\usepackage{algorithm}
\usepackage{algpseudocode}
\usepackage{url}
\newboolean{longversion}
\setboolean{longversion}{false} % TOGGLE SWITCH
%\setlength{\textfloatsep}{100pt}  % Space between floats and text
\newcommand{\blue}[1]{\textcolor{blue}{#1}}
\newcommand{\red}[1]{{\textcolor{red}{#1}}}
\newcommand{\fa}[1]{\textcolor{magenta}{Fatemeh: #1}}
\newcommand{\ts}[1]{\textcolor{cyan}{Tolunay: #1}}
\newcommand{\rb}[1]{\textcolor{red}{Ryan: #1}}
\newcommand{\bd}[1]{\textcolor{green}{Basir: #1}}
\usepackage{enumitem}
\def\BibTeX{{\rm B\kern-.05em{\sc i\kern-.025em b}\kern-.08em
    T\kern-.1667em\lower.7ex\hbox{E}\kern-.125emX}}
\usepackage{multirow} 
\renewcommand{\baselinestretch}{0.85}
\begin{document}

\title{REAL: Reinforcement Learning-Enabled xApps for  Experimental Closed-Loop Optimization in O-RAN with OSC RIC and srsRAN
\thanks{This material is based upon work supported by the National Science Foundation under Grant Numbers  CNS-2318726, and CNS-2232048.}
}

\author{
	\IEEEauthorblockN{
	Ryan Barker, 
        Alireza Ebrahimi Dorcheh,
        Tolunay Seyfi,
        Fatemeh Afghah}

    \IEEEauthorblockA{Holcombe Department of Electrical and Computer Engineering, Clemson University, Clemson, SC, USA \\
        Emails: \{rcbarke, alireze, tseyfi, fafghah\}@clemson.edu}

}

\maketitle

%\rb{Long version: (1) ORAN Architecture overview, (2) Expanded srsRAN limitations, (3) Complementary Detail}

\begin{abstract}
Open Radio Access Network (O-RAN) offers an open, programmable architecture for next-generation wireless networks, enabling advanced control through AI-based applications on the near-Real-Time RAN Intelligent Controller (near-RT RIC). However, fully integrated, real-time demonstrations of closed-loop optimization in O-RAN remain scarce. In this paper, we present a complete framework that combines the O-RAN Software Community RIC (OSC RIC) with srsRAN for near-real-time network slicing using Reinforcement Learning (RL). Our system orchestrates resources across diverse slice types (eMBB, URLLC, mMTC) for up to 12 UEs. 
We incorporate GNU Radio blocks for channel modeling, including Free-Space Path Loss (FSPL), single-tap multipath, AWGN, and Doppler effects, to emulate an urban mobility scenario. Experimental results show that our RL-based xApps dynamically adapt resource allocation and maintain QoS under varying traffic demands, highlighting both the feasibility and challenges of end-to-end AI-driven optimization in a lightweight O-RAN testbed. Our findings establish a baseline for real-time RL-based slicing in a disaggregated 5G framework and underscore the need for further enhancements to support fully simulated PHY digital twins without reliance on commercial software.
\end{abstract}

\begin{IEEEkeywords}
O-RAN, %near-RT RIC, Open5GS, 
srsRAN, 
Reinforcement Learning, Network Slicing, Resource Allocation %, 5G SA, GNU Radio, E2, xApps
\end{IEEEkeywords}

\section{Introduction and Related Work}
\label{sec:intro}

Fifth-generation (5G) cellular networks introduce a range of capabilities, including enhanced Mobile Broadband (eMBB), Ultra-Reliable Low-Latency Communications (URLLC), and massive Machine-Type Communications (mMTC) \cite{AIin5G}. These services pose diverse Quality of Service (QoS) requirements, demanding a more flexible Radio Access Network (RAN) that can dynamically allocate resources in near real-time. The Open Radio Access Network (O-RAN) framework meets this need by disaggregating traditional, vendor-locked architectures into interoperable components. Central to this vision is the near-Real-Time RAN Intelligent Controller (near-RT RIC), an AI-driven platform that executes fine-grained optimization of RAN functions over the E2 interface \cite{ORAN, ORAN_Tommaso}.
A distinguishing feature of O-RAN is the ability to host third-party applications, known as xApps, on the near-RT RIC. These xApps can employ machine learning or control-theoretic strategies to orchestrate diverse slice types, ranging from high-throughput eMBB to latency-sensitive URLLC \cite{AIinORAN}. Among learning-based approaches, \emph{Reinforcement Learning} (RL) is well-suited for adaptive, real-time control: it can iteratively refine its policy based on a reward signal reflecting slice performance (e.g., throughput, latency, or reliability). 

In recent years, several platforms have explored the integration of RL and AI in O-RAN, leveraging testbeds to address resource allocation and network optimization challenges. X5G introduces a modular 5G O-RAN platform with NVIDIA ARC for GPU-accelerated PHY layer tasks and OpenAirInterface (OAI) for higher layers \cite{villa2023x5g}. It supports real-time control via the near-RT RIC, evaluating performance using up to eight concurrent Commercial Off-the-Shelf (COTS) UEs. ORANSlice, on the other hand, emphasizes multi-slice resource allocation with compliance to 3GPP and O-RAN standards, leveraging xApps for dynamic physical resource block (PRB) allocation and scheduling policies \cite{cheng2024oranslice}. ColO-RAN offers a large-scale framework for ML-based xApp development, integrating RL-based control strategies and large-scale data collection using the Colosseum wireless network emulator \cite{coloran}. 
The PandORA framework introduces an automated approach to designing and evaluating DRL-based xApps for Open RAN, leveraging the Colosseum wireless network emulator for large-scale testing under diverse traffic and channel conditions \cite{tsampazi2024pandora}. 
%This work benchmarks multiple xApps that integrate Deep Reinforcement Learning (DRL) agents with varying architectures, reward designs, action spaces, and decision-making timescales.
%demonstrating that fine-tuning control timers and optimizing reward structures can significantly improve network performance. \bd{Let's put way less emphasize on their work.}

%Despite these advancements, existing solutions have notable limitations. X5G lacks a real-time closed-loop system for dynamic optimization via RL-based xApps and demonstrates limited scalability. ORANSlice supports only two UEs in testing and does not incorporate RL for closed-loop adaptability under complex network conditions. ColO-RAN primarily relies on offline training, limiting its ability to respond to real-time changes such as shifting UE positions or traffic demands. These gaps highlight the need for novel frameworks that combine real-time RL control with scalability and comprehensive testing to address the evolving challenges of 5G and beyond.


\section{RELATED WORK}
\begin{comment}
\begin{table*}[ht]
\centering
\caption{Comparison of related works in terms of the simulation platform.}
\label{tab:relatedworkstable}
\resizebox{\textwidth}{!}{%
%\begin{tabular}{|l|l|l|l|l|}
\begin{tabular}{|l|p{4cm}|p{4cm}|p{4cm}|p{4cm}|}
\hline
\hline
\textbf{Ref.} & \textbf{Implementation Strategy} & \textbf{Platform/Simulation Tools} & \textbf{Evaluation Metrics} & \textbf{Dataset/Scenario} \\ \hline
\cite{tsampazi2023} %M. Tsampazi et al. 
& DRL-based xApps, actor-critic, hierarchical decision-making  & OpenRAN Gym, srsRAN & Latency, Throughput, Package transmission & 7 BS, 42 UEs (SDR-based) \\ \hline
\cite{marojevic2022actor} %V. Marojevic et al. 
& DRL-based xApps, actor-critic & OpenAI Gym, mobile-env Simulator \cite{mobile-env}  & Throughput, Normalized cumulative reward & Heterogeneous networks: 5 base stations and 10 moving UEs \\ \hline
\cite{Anand2023xApp} %D. Anand et al. 
& Multi-Classification and Offloading Scheme (MLMCOS) & NS-3 & VMAF, R-Factor, RUM Speed Index & Heterogeneous network environment (HetNet) \\ \hline
\cite{Colosseum2023} %L. Bonati et al. 
& OpenRAN Gym & Colosseum & Latency, Spectrum efficiency, Scalability & 128 pairs of generic compute servers and SDR \\ \hline
\end{tabular}%
}
\vspace{-5pt}
\end{table*}
\end{comment}

Various simulation platforms and frameworks have been utilized in recent works to evaluate the effectiveness of AI/ML-based optimization strategies within O-RAN architectures. These platforms, ranging from OpenRAN Gym to NS-3 and custom simulators, offer unique features tailored to specific research objectives such as resource allocation, interference management, and network slicing. Each study highlights distinct implementation strategies, performance metrics, and scenarios.
\cite{tsampazi2023} presents an O-RAN architecture using Deep RL (DRL)-based xApps for optimizing eMBB, URLLC, and mMTC slices. Tested on the Colosseum wireless network emulator via the OpenRAN Gym framework, these xApps apply Proximal Policy Optimization (PPO) for real-time resource management. The study evaluates multiple configurations, revealing trade-offs in action spaces and reward functions, showcasing scalability and adaptive performance across varied network conditions.
Building upon this, \cite{marojevic2022actor} investigates PPO for real-time resource management within O-RAN but adds a comparative analysis of Advantage Actor-Critic (A2C). Both models optimize resource allocation through the E2 interface in communication with the near-real-time RIC, but the study demonstrates PPO's faster convergence and superior rewards over A2C. Tested in a simulated RAN environment under dynamic conditions, the research highlights PPO's superior efficiency for resource management.
In a shift toward interference management, \cite{Anand2023xApp} introduces a machine learning-based xApp for mitigating co-tier interference in a Heterogeneous Network (HetNet) environment, with a focus on improving QoE for services like video and VoIP. The xApp employs a Multi-Classification and Offloading Scheme (MLMCOS), utilizing models such as Random Forest and CNN to classify users based on interference levels and offload them to femtocells. Tested using NS-3 \cite{ns3}, this study emphasizes interference management within HetNet environments, offering a more focused solution for dense networks compared to the previous studies on resource allocation.
Further expanding on AI-driven optimization, \cite{Colosseum2023} highlights Colosseum's role as an AI/ML-based digital twin platform for O-RAN development. Through its OpenRAN Gym framework, the platform supports real-time deployment and testing of algorithms such as network slicing, scheduling, and spectrum sharing. Integrated with SDRs and real-world RF conditions, Colosseum allows for scalable, high-fidelity testing of AI/ML xApps, enabling continuous interaction between the digital twin and physical network layers. This infrastructure bridges the gap between simulation and real-world deployment, offering detailed insights into the scalability and robustness of AI/ML solutions under complex conditions.

%These efforts reveal two primary gaps: (1) a lack of full-stack integration at scale---existing systems often handle only a small number of UEs or rely on constrained simulations---and (2) limited real-time RL-based slicing, as most frameworks depend on offline training or partial control loops.

These efforts reveal two primary gaps including 
%\begin{enumerate}[left=0pt, noitemsep]
i) \textbf{Full-stack integration at scale:}  Existing systems often handle only a small number of UEs or operate within constrained simulation setups (\cite{marojevic2022actor}, \cite{Colosseum2023}, \cite{Anand2023xApp}), and ii)
 \textbf{Real-time RL-based slicing:} Many frameworks rely on offline training or partial control loops, deferring real-time decision-making and adaptability (\cite{tsampazi2024pandora}, \cite{tsampazi2023},\cite{coloran}). Unlike many frameworks that rely on offline training, which cannot capture real-time environmental feedback, our work emphasizes online training. In this approach, the RL agent interacts directly with the srsRAN simulator during training, receiving immediate rewards based on its resource allocation decisions. This ensures dynamic adaptation to network traffic and performance, overcoming the limitations of offline methods that fail to reflect real-world conditions. Our framework thus enables responsive and optimized physical resource block allocation in 5G O-RAN architectures.
 %\fa{I feel like we did not emphasize enough on this gap when describing the related work. Either here, or in the last paragraph of the contribution subsection when we talk about PPO with online training, we should descrbe the advantages of an online trained PPO on the ability to adjusts the policy dynamically based on current observations and changes in network conditions. We should also discuss the its high demands for computational resources compared to an offline case and mention this as a bottleneck for our work. could be discussed in the conclusion. }
 %\fa{pls add references for each gap}
%\end{enumerate}
To address the aforementioned gaps, we introduce a full-stack O-RAN solution that unifies the OSC near-RT RIC with srsRAN \cite{srsran} for real-time slicing:

\begin{itemize}[left=0pt, noitemsep]
    \item \textbf{End-to-end Integration on OSC RIC and srsRAN:} 
    We develop a near-RT xApp that directly controls srsRAN’s gNB through the E2 interface, using E2AP messages to manage PRB allocations for up to 12 UEs.

    \item \textbf{Fully Online Closed-Loop Training:} 
    Our RL agent receives immediate feedback (downlink throughput, slice QoS) from srsRAN and updates its policy in real time. This ensures a genuine closed-loop approach, where each action on the RAN triggers immediate learning and fine-tuning in the xApp.

    \item \textbf{GNU Radio Channel Emulation:}
    We incorporate Free-Space Path Loss (FSPL), Additive White Gaussian Noise (AWGN), single-tap multipath fading, and Doppler shifts to approximate urban mobility.
    %within srsRAN’s ZeroMQ-based environment.

    \item \textbf{Practical Insights for Scalability:}
    We highlight operational constraints such as ZeroMQ saturation (limiting concurrency), partial uplink slicing, and attach-sequencing workarounds. These insights inform future O-RAN development for larger-scale deployments.
\end{itemize}

This paper presents an \emph{online}, RL-driven resource allocation strategy for managing multiple slices. 

The system model and E2-based control mechanisms are introduced in Section~\ref{sec:system}, followed by a detailed discussion of the RL methodology and its constraints in Section~\ref{sec:method}. Section~\ref{sec:scenarios} explores the simulation scenarios and channel conditions used to evaluate the framework. The performance of the system under varying mobility and traffic conditions is analyzed in Section~\ref{sec:results}. Finally, Section~\ref{sec:conclusion} summarizes the findings and outlines potential directions for future research in O-RAN.

\ifthenelse{\boolean{longversion}}{
% \section{O-RAN Architecture and E2 Interface}
% \label{sec:oran-arch}

% The Open Radio Access Network (O-RAN) initiative aims to redefine the traditional, monolithic RAN by disaggregating its hardware and software components, introducing open interfaces, and enabling AI-driven control loops. This section provides a deeper look into the O-RAN architectural framework, highlighting how the near-Real-Time RAN Intelligent Controller (near-RT RIC) communicates with the RAN via the E2 interface. We also discuss the broader 5G core ecosystem—particularly the role of the User Plane Function (UPF)—and emerging hardware platforms such as smart NICs and Data Processing Units (DPUs) that can further accelerate O-RAN deployments.

% \subsection{Disaggregated RAN Components}
% The O-RAN Alliance specifies a disaggregated architecture in which the Radio Unit (RU), Distributed Unit (DU), and Centralized Unit (CU) are split to allow multi-vendor interoperability. This approach fosters:
% \begin{itemize}
%     \item \textbf{Openness and Modularity:} Each component is accessible through open interfaces, avoiding vendor lock-in.
%     \item \textbf{Programmability:} Standardized APIs and service models enable the near-RT RIC to implement advanced control logic, including machine learning.
%     \item \textbf{Scalability:} Operators can scale individual RU, DU, or CU elements to handle dynamic traffic demands.
% \end{itemize}

% In our work, we use \emph{srsRAN} to emulate the CU/DU for a 5G Standalone (SA) deployment, while a simulated RU handles the physical layer. The near-RT RIC implements the intelligence—our \emph{xApp(s)}—that leverages real-time key performance indicators (KPIs) data and applies slice-specific actions over the E2 interface.

% \subsection{E2 Interface for Near-RT Control}
% \begin{figure}[!t]
%     \centering
%     \includegraphics[width=0.9\columnwidth]{E2.png}
%     \caption{E2 Interface, enabling near real time communication, resource management, and optimization between the core network and gNodeB. \rb{Rendered with mermaid, content is thorough but need to make this larger.}}
%     \label{fig:e2}
% \end{figure}

% A central novelty of O-RAN is the \emph{E2 interface}, which connects the near-RT RIC to the underlying RAN nodes (e.g., gNBs in 5G) as shown in Fig.~\ref{fig:e2}. Through E2 Application Protocol (E2AP) messages and function-specific service models (E2SM), the near-RT RIC can:
% \begin{itemize}
%     \item \textbf{Collect KPIs and Telemetry:} E2 allows near-real-time reporting of throughput, latency, PRB usage, and other physical- or MAC-layer metrics.
%     \item \textbf{Issue Control Actions:} The RIC can dynamically modify scheduling parameters, allocate or revoke PRBs, or even command handovers.
%     \item \textbf{Maintain Policy and Slicing:} By sending slice-level or UE-level policies, the RIC orchestrates traffic flows across URLLC, eMBB, and mMTC slices.
% \end{itemize}

% In our platform, the \emph{O-RAN Software Community RIC (OSC RIC)} terminates E2 messages via its \emph{E2Term} component. A specialized \emph{E2 Service Node (E2SN)} inside srsRAN translates E2 instructions into local RAN actions. Although srsRAN’s E2SN remains experimental—particularly for advanced uplink slicing or detailed PHY metrics—this interface still provides the essential hooks for our Reinforcement Learning (RL) xApp to adapt resource allocation in near real-time.

% \subsection{UPF Integration and the 5G Core Ecosystem}
% While the near-RT RIC focuses on optimizing RAN parameters, the broader 5G core network also plays a critical role in end-to-end connectivity. In 5G Non-Standalone (NSA) or Standalone (SA) modes, the \emph{User Plane Function (UPF)} is pivotal for forwarding user data between the RAN and external networks (e.g., the internet or private clouds). This decoupling of the user plane (UPF) from control-plane entities (AMF, SMF) allows:

% \begin{itemize}
%     \item \textbf{Scalable User Traffic Handling:} The UPF can be independently scaled or replicated to manage high data rates.
%     \item \textbf{Local Breakout and Edge Computing:} Operators can place UPFs closer to RAN sites to reduce latency or integrate them with Multi-access Edge Computing (MEC) platforms.
%     \item \textbf{Policy Enforcement:} The SMF (Session Management Function) configures the UPF to enforce QoS or traffic steering decisions, complementing near-RT RIC orchestration in the RAN.
% \end{itemize}

% Although our testbed primarily highlights the RIC--RAN control loop via E2, the UPF’s behavior can significantly impact end-to-end QoS. For instance, high-volume eMBB flows may induce congestion if the UPF is not provisioned or routed efficiently.

% \subsection{Smart NICs and DPUs for O-RAN Acceleration}
% One promising approach to further enhance O-RAN performance is offloading select data-plane tasks to \emph{Smart NICs} or \emph{Data Processing Units (DPUs)}:
% \begin{itemize}
%     \item \textbf{Smart NICs:} Specialized network interfaces that can parse traffic, handle encryption, or perform Quality of Service (QoS) classification without burdening the CPU. By accelerating packet processing, Smart NICs can reduce the latency between the RAN and UPF.
%     \item \textbf{DPUs:} More advanced than Smart NICs, DPUs often include embedded CPUs or programmable pipelines designed for high-throughput packet processing (e.g., programmable match-action tables, advanced telemetry). A DPU can offload parts of the UPF or even host the entire UPF function inline, alleviating bottlenecks in the user plane.
% \end{itemize}

% Such hardware acceleration aligns well with O-RAN’s open interfaces: the RIC can focus on high-level control (via E2), while Smart NICs or DPUs handle micro-level optimizations in the data plane. In future expansions of our testbed, we plan to investigate synergy between an RL-based near-RT xApp and a DPU-backed UPF, aiming for ultra-low-latency flows in URLLC scenarios and high-throughput offloading for eMBB.

% \subsection{Key Takeaways for Our System Model}
% The next section (\S\ref{sec:system}) describes how we combine OSC RIC, srsRAN, and GNU Radio to demonstrate a near-real-time RL loop. The O-RAN architectural concepts introduced here—disaggregated RAN components, E2-based near-RT control, and integration with a 5G core—form the foundation of our solution. While certain advanced features (e.g., mature UL slicing, DPU-based UPF) remain future work in our testbed, they represent promising directions for fully realizing the potential of open, AI-driven 5G and beyond.
}{}
\section{System Model}
\label{sec:system}

\subsection{Architecture Overview}
In this work, we integrate the O-RAN Software Community RIC (OSC RIC) with srsRAN’s 5G stack to form a complete end-to-end testbed capable of near-real-time radio resource control. Figure~\ref{fig:arch} illustrates the overall architecture, highlighting the following key components:

\begin{itemize}[left=0pt, noitemsep]
    \item \textbf{Open5GS Core:} Provides both \emph{4G (EPC)} and \emph{5G (5GC)} in a single software suite, enabling seamless UE attachment and data exchange with the RAN.
    \item \textbf{Near-RT RIC:} Hosts \emph{xApps} responsible for coordinating slicing decisions and sending resource-allocation commands via the E2 interface.
    \item \textbf{srsRAN gNB:} Emulates the \emph{Centralized Unit (CU)} and \emph{Distributed Unit (DU)} functionality; an \emph{E2 Service Node (E2SN)} bridges the OSC RIC’s E2Term and the gNB’s internal APIs.
    \item \textbf{E2 Interface:} Enables real-time orchestration of downlink PRB quotas, which can be assigned to multiple slices (URLLC, eMBB, mMTC).
    \item \textbf{GNU Radio Blocks (Channel Model):} Inserted between \emph{srsUE} and \emph{srsGNB} via ZeroMQ streams to emulate path loss, fading, noise, and Doppler effects.
\end{itemize}

In our testbed, \textbf{Open5GS Core} provides the following core network components as Dockerized services using a shared configuration:
\begin{itemize}[left=0pt, noitemsep]
    \item \textbf{EPC (Evolved Packet Core):} Allows 4G operation, with the \emph{MME (Mobility Management Entity)} handling control-plane signaling and the \emph{SGW (Serving Gateway)} routing user-plane data.
    \item \textbf{EPC 5GC (5G Core):} Enables 5G standalone functionality, where the \emph{AMF (Access and Mobility Management Function)} and \emph{SMF (Session Management Function)} manage UE registration and session establishment, and the \emph{UPF (User Plane Function)} forwards user traffic to external networks.
\end{itemize}

A dedicated subscriber database 
%in \emph{.CSV} format 
contains each UE’s \emph{IMSI}, \emph{encryption keys}, \emph{operator code}, \emph{QoS parameters}, and IP assignment details—referenced by the MME or AMF to authenticate users on network attach. Although our overall testbed supports both \emph{4G} and \emph{5G}, it is specifically deployed in \textbf{5G SA mode} to align with the \emph{3GPP 7.2 architecture} for O-RAN. In a 4G EPC configuration, the MME collaborates with the SGW and PGW (or UPF in a hybrid mode); meanwhile, for 5G SA, the AMF and SMF manage slicing parameters and session establishment, while the UPF routes user-plane traffic externally. \textbf{Network slicing} is enabled by assigning specific \emph{Slice/Service Types (SST)} and \emph{Slice Differentiators (SD)} in the AMF configuration.
%, ensuring that any UE registering with the correct SD and SST is placed into the desired slice (e.g., URLLC, eMBB, or mMTC). Any mismatch results in registration failure, reinforcing proper slice isolation.

\begin{figure}[!t]
    \centering
\includegraphics[width=1\columnwidth, ]{ORAN_Actor_Critic.jpg}
%\vspace{-10pt}
    \caption{%\fa{this figure does not serve the purpose of the proposed work. We want something to show the online training aspect, the online training aspect. We can either have another fig to show the learning steps/curve on integrate it with this fig. We could also add a pseudo code if space allows. } 
    Overall system architecture. The near-RT RIC controls the srsRAN gNB via the E2 interface, while GNU Radio injects channel impairments. %\fa{we need to define the terms UPF, others in the text. and discuss how implemented/integrated in the model. }
    }
    \label{fig:arch}
\end{figure}

%\begin{figure}[!t]
%    \centering
%\includegraphics[width=1\columnwidth, ]{actor_critic.jpg}
%\vspace{-10pt}
%    \caption{Reinforcement Learning Agent architecture.}
%    \label{fig:arch}
%\end{figure}


%\rb{Revise architecture to include gnuradio, use higher quality, easy to read figures (recommend draw.io).}

\subsection{Channel Assumptions and Modeling}
To ensure realistic network conditions, we incorporate a simplified channel model reflecting an \emph{urban mobility scenario}, accounting for various physical effects:

\begin{enumerate}[left=0pt, noitemsep]
    \item \textbf{Free-Space Path Loss (FSPL):} Signal attenuation is modeled based on the standard free-space path loss formula, which depends on the transmitter--receiver distance, the carrier frequency, and the speed of light. This captures the general propagation loss over distance, typical in open environments.
    
    \item \textbf{Single-Tap Fading:} We introduce a primary path with a complex gain of \([0.85 + 0.25j]\), representing moderate multipath conditions typical of low-rise urban environments. This fading factor is dimensionless, emphasizing the influence of reflections and scattering on signal integrity.
    
    \item \textbf{Additive White Gaussian Noise (AWGN):} Background noise is characterized by a power spectral density proportional to the system temperature, bandwidth, and Boltzmann constant. In practice, the noise voltage is capped below 0.01 in GNU Radio to preserve the stability of the srsRAN implementation.
    
    \item \textbf{Doppler Effects:} For mobile user equipment (UE), such as those supporting Ultra-Reliable Low-Latency Communications (URLLC) at moderate speeds (e.g., 40 km/h), the Doppler shift is computed based on the UE's velocity relative to the carrier frequency. This introduces time-varying phase offsets, presenting challenges for synchronization in dynamic urban scenarios.
\end{enumerate}
% \fa{we don't need to define the AWGN or FSPL formula. At least, have them in line with text not an equation  }

The channel modeling is implemented using GNU Radio, which communicates with srsUE and srsGNB via ZeroMQ streams for seamless data exchange. 
%This approach allows the RL agent’s actions—and srsRAN’s internal schedulers—to adapt in near real-time to the evolving channel state, simulating practical mobility and noise conditions without requiring external hardware. 
In Section~\ref{sec:method}, we elaborate on how these channel characteristics feed into our RL algorithm’s observation space.

%\begin{figure}[!t]
%    \centering
    %\includegraphics[width=0.9\columnwidth]{gnuradio.png}
    %\caption{Sample gnuradio flowgraph, illustrating multi-UE channel modeling and modulation.\ts{The reviewers will not understand this Figure since it is not readable as well lacks in explanation. We can add a conceptual diagram that visualizes the purpose of GNUradio.} \rb{Great idea, let's do it.}}
    %\label{fig:gnuradio}
%\end{figure}

\section{Methodology}
\label{sec:method}

\subsection{Reinforcement Learning Framework for RAN Slicing Optimization}
%We employ Proximal Policy Optimization (PPO) within an actor-critic framework to address the sequential decision-making challenges of RAN slicing. The actor network outputs action vectors \(\mathbf{a}_t\), representing PRB allocations for each slice–UE pair at timestep \(t\), balancing resource provisioning and over-allocation. The critic network estimates the value function \(V^\pi(\mathbf{s}_t)\) %\fa{define the notations for state and policy}, guiding policy updates by evaluating the impact of actions on slice QoS.

We employ Proximal Policy Optimization (PPO) which is an actor-critic framework to address the decision-making challenges of RAN slicing. Here, \(s_t\) represents the state at timestep \(t\), which encapsulates the current UE requirements and status. The actor network outputs action vectors \(\mathbf{a}_t\), representing PRB allocations for each slice at timestep \(t\), balancing resource provisioning and over-allocation. The critic network estimates the value function \(V^\pi(\mathbf{s}_t)\), where \(\pi\) denotes the policy, a mapping from states \(s_t\) to actions \(a_t\), which the agent learns to optimize the long-term reward.

\subsubsection{State and Action Spaces}
The state vector comprises
of UE type, Bit-rate or Buffer size (occupancy) and pathloss, relying primarily on E2 metrics provided by srsRAN. The action space defines PRB allocations at the slice level (eMBB, URLLC, mMTC), rather than per UE. PRBs are first distributed evenly among UEs within each slice, with any remaining PRBs allocated to UEs experiencing higher path loss. The total PRB usage remains constrained by the subframe limit, ensuring compliance with srsRAN’s channel bandwidth.


\subsubsection{Slice-Specific Reward Functions}
\label{subsec:rewards}

We design a reward function $R_t$ that combines the individual rewards $r_{\text{URLLC}, t}$, $r_{\text{eMBB}, t}$, and $r_{\text{mMTC}, t}$ at time $t$. Each slice $s \in \{\text{URLLC, eMBB, mMTC}\}$ has a distinct QoS priority and thus a specialized sub-reward. The final reward is a weighted sum:
\begin{align}
R_t \;=\; \alpha_{\text{URLLC}} \,r_{\text{URLLC},t}
\;+\;\alpha_{\text{eMBB}}\,r_{\text{eMBB},t}
\;+\;\alpha_{\text{mMTC}}\,r_{\text{mMTC},t},
\end{align}
where $\alpha_{\text{URLLC}}, \alpha_{\text{eMBB}}, \alpha_{\text{mMTC}}$ are slice importance weights. Below, we define each sub-reward:

\paragraph{URLLC Reward ($r_{\text{URLLC}}$).}
URLLC focuses on ultra-low latency and reliability. We approximate latency by monitoring buffer occupancy or per-packet delay. We define a negative penalty if the slice’s average delay exceeds $t_{\text{target}}$:
\begin{align}
r_{\text{URLLC}, t} \;=\;
\max\!\Bigl(-1,\,\min\!\Bigl(0,\,
\dfrac{t_{\text{target}} - t_{\text{avg}}}{t_{\text{target}}}\Bigr)\Bigr),
\end{align}
where $t_{\text{avg}}$ is the measured or estimated delay (e.g., from buffer timestamps), and $t_{\text{target}}$ is the URLLC delay threshold. Lower delay yields higher (less negative) reward.

\paragraph{eMBB Reward ($r_{\text{eMBB}}$).}
eMBB aims for high data throughput. We track the slice’s average bitrate $b_{\text{avg}}$ over the last measurement window and compare it against a target $b_{\text{target}}$:
\begin{align}
r_{\text{eMBB}, t} \;=\;
\max\!\Bigl(-1,\,\min\!\Bigl(0,\,
\dfrac{b_{\text{avg}} - b_{\text{target}}}{b_{\text{target}}}\Bigr)\Bigr).
\end{align}
If $b_{\text{avg}}$ meets $b_{\text{target}}$, the sub-reward approaches 0. Otherwise, it becomes increasingly negative, with a floor at $-1$.

\paragraph{mMTC Reward ($r_{\text{mMTC}}$).}
For mMTC, we focus on the number of successfully received packets—especially if the devices are mostly downlink or rely on sporadic transmissions. Let $b_{\text{received}}$ be the total bytes (or packets) received and $b_{\text{expected}}$ the desired or generated amount over the window:
\begin{align}
r_{\text{mMTC}, t} \;=\;
\max\!\Bigl(-1,\,\min\!\Bigl(0,\,
\dfrac{b_{\text{received}} - b_{\text{expected}}}{b_{\text{expected}}}\Bigr)\Bigr).
\end{align}
When $b_{\text{received}} \approx b_{\text{expected}}$, the slice gets a near-zero penalty. If $b_{\text{received}}$ is too low, it is penalized more harshly.

\paragraph{Clipping and Weights.}
We constrain each slice sub-reward to $[-1,\,0]$ to avoid exploding gradients and to unify the magnitude of slice-specific penalties. The weighting factors $\alpha_{\text{URLLC}}, \alpha_{\text{eMBB}}, \alpha_{\text{mMTC}}$ allow fine-tuning of slice priority and configurable fairness. For example, if URLLC must absolutely not be violated, one might set $\alpha_{\text{URLLC}} \gg \alpha_{\text{eMBB}}$.

This formulation ensures each slice’s QoS goals (latency, throughput, or packet reception) are explicitly represented within a single scalar reward $R_t$. During training, the RL agent learns how to adjust PRB allocations across slices to maximize $R_t$ over time, striking a balance among slice priorities and the channel constraints.

\subsubsection{Training Procedure}
The training procedure is as follows: the RL agent sets resource allocation while the xApp applies PRB assignments. During traffic simulation, KPI measurements are gathered every 500 ms. After simulation, the xApp averages these KPIs and saves them in a JSON file. The RL agent then computes the reward using the KPIs and stores the state, action, and reward in replay memory to update its policy. Table \ref{tab:task_spec} lists the parameters for each slice.

%The training procedure follows these steps: First, the RL agent determines the resource allocation, and the xApp applies the PRB assignments. Traffic simulation then begins, during which the xApp collects KPI measurements every 500 ms. Once the simulation concludes, the xApp computes the average KPIs and stores them in a JSON file. The RL agent then processes these KPIs to compute the reward, which, along with the corresponding state and action, is stored in the replay memory. This stored experience is subsequently used to update the RL agent, ensuring adaptive learning based on network dynamics. 

%Table \ref{tab:task_spec} shows the parameters used for each slice in our training.

\begin{table}[h]
    \centering
    \renewcommand{\arraystretch}{1.2}
    \setlength{\tabcolsep}{6pt}
    \begin{tabular}{c|c|c}
        \hline
        \textbf{Service} & \textbf{Parameter} & \textbf{Values} \\
        \hline
        \multirow{4}{*}{URLLC} & Gen. Freq. (Hz) & 2 \\
                               & Gen. Bytes (B) & Min: $10^5$, Max: $3\times10^5$ \\
                               & Latency (ms) & 500 \\
                               & $\alpha_{\text{URLLC}}$ & 1 \\
        \hline
        \multirow{2}{*}{eMBB} & Bit Rate & Min: $2\times10^5$, Max: $4\times10^5$ \\
                              & $\alpha_{\text{eMBB}}$ & 1 \\
        \hline
        \multirow{3}{*}{mMTC} & Gen. Freq. (Hz) & 4 \\
                              & Gen. Bytes (B) & Min: $25\times10^3$, Max: $60\times10^3$ \\
                              & $\alpha_{\text{mMTC}}$ & 1 \\
        \hline
    \end{tabular}
    \caption{Task Specification for PRB Allocation}
    \label{tab:task_spec}
\end{table}


\subsection{Resource Management Constraints: Scalability Limitations and Open-Source Commitment}
%\bd{32-core, 64-threads} \rb{24/32 is reflective of Ryan's machine, we should use Basir's specs here as that is where we trained and evaluated the final model. Does he have a different CPU model?} \rb{We confirmed that both machines are the same CPU, which is 24/32. Intel's architecture here: https://www.intel.com/content/www/us/en/products/sku/236773/intel-core-i9-processor-14900k-36m-cache-up-to-6-00-ghz/specifications.html}

While our RL approach is designed to be generic and adaptable to diverse network conditions, 
the limited computational resources on one PC would propose a limitation.
%the current srsRAN-based implementation exhibits inherent limitations that affect both scalability and performance. 
We run our environment via virtualization on a 24-core, 32-thread Intel(R) Core(TM) i9-14900K CPU. In this setup, ZeroMQ, being CPU-bounded and operating as a lightweight TCP/IP-based point-to-point transport layer, manages inter-process communication between srsGNB and each simulated srsUE. Although GNU Radio connects these individual message streams, the use of ZeroMQ in this design would propose a communicational limit. While it supports N-to-N socket instances, increasing the number of simultaneous endpoints 
%on a single message stream 
causes network strain, buffer overflow, and eventual saturation.
Extensive testing shows that an unsaturated channel in this environment supports a total of 28 Mbps across all UEs. However, when more than three UEs transmit simultaneously, the channel saturates, leading to substantial performance degradation and system instability. 
%This architecture cannot substitute for proper lower-PHY or O-RU protocols designed for dense network scenarios. 
To mitigate this constraint and enable simulations with up to 12 UEs, we employ a batching strategy that groups UEs in sets of three, rotating these groups over time intervals as described in Section~\ref{subsec:scenarios-limits}. It is worth noting that more CPU cores would likely improve performance.

In contrast, 
%commercial versions of srsRAN and 
proprietary solutions like \textit{Amarisoft}\cite{amarisoft} exhibit superior scalability, supporting significantly more UEs without similar bottlenecks, but their closed-source nature restricts reproducible research efforts. Our work underscores the importance of open-source frameworks to foster collaborative O-RAN development. Additionally, as of now, the current stable version of srsRAN supports only down-link slicing
%, constraining exploration of URLLC scenarios requiring robust uplink performance. 
We address these challenges by leveraging the O-RAN E2 interface in a closed-loop framework with our resource allocation xApp, enabling dynamic PRB allocation and near real-time KPI monitoring directly over E2. Although throughput measurements are sampled every 500 ms,
%—reducing the granularity of performance insights—
this setup provides timely feedback for the RL agent to update its policy, adapt resource allocations, and evaluate the impact of state changes. 
%Moreover, srsRAN offers limited KPI visibility. Critical metrics such as SINR and \textit{Hybrid Automatic Repeat Request} (HARQ) statistics are either absent or inaccessible. Consequently, 
the RL agent relies on higher-level indicators like throughput and slice occupancy, restricting its ability to optimize network performance effectively. 
From the KPI perspective, the RL agent relies on throughput readings.
Despite the constraints, our methodology demonstrates the feasibility of real-time RL-based slicing with srsRAN. 
%while highlighting areas for improvement, including enhanced channel data, finer scheduling granularity, and robust uplink support.

\section{Simulation Scenarios}
\label{sec:scenarios}

\subsection{ Traffic and Deployment Configuration}

We design our simulation scenarios around three primary slices—{URLLC}, {eMBB}, and {mMTC}—each reflecting distinct downlink-centric 5G services with corresponding QoS demands. The {URLLC} slice (\(4\) UEs) is modeled as autonomous driving systems requiring ultra-reliable, low-latency transmissions at \(40\,\mathrm{km/h}\), emulated by large, non-frequent downlink packets. The {eMBB} slice (\(4\) UEs) comprises two mobile UEs representing smartphone users streaming high-bandwidth content on the move, and two stationary UEs simulating fixed wireless access subscribers consuming large data flows (e.g., 4K video). Finally, the {mMTC} slice (\(4\) UEs) mimics low-throughput IoT endpoints such as digital billboards, where small, high frequency downlink updates reflect srsRAN’s limited uplink support.

In total, the deployment accommodates up to \(12\) UEs, though only three transmit actively at any given time to avoid ZeroMQ saturation. 
These active UEs rotate in groups, maintaining stable throughput while capturing dynamic traffic patterns. 
A realistic channel model—including single-tap path loss, AWGN, Doppler effects, and random distances (\(d \in [0.5\text{\,km}, 2\text{\,km}]\))—emulates a mobile urban environment. This setup provides an efficient testbed for evaluating RL-based resource allocation strategies under diverse, downlink-centric QoS requirements.

%Our simulation scenarios encompass three primary slices, each representing distinct 5G services and QoS requirements. \textbf{URLLC} focuses on ultra-reliable, low-latency applications with UEs moving at \(40\,\mathrm{km/h}\), generating small, frequent packets to emulate mission-critical traffic. \textbf{eMBB} targets high-throughput use cases like video streaming or file transfers, with \(50\%\) of UEs mobile at \(40\,\mathrm{km/h}\) and \(50\%\) stationary, capturing both fixed and mobile user patterns. \textbf{mMTC} represents IoT devices with low-throughput, sporadic downlink traffic, emphasizing passive data consumption due to uplink limitations in srsRAN.
%The deployment involves up to 12 UEs distributed across the three slices: 4 URLLC, 4 eMBB (split evenly between mobile and stationary), and 4 mMTC. Active transmissions are restricted to three UEs at a time to prevent ZMQ overload, with groups rotating every 10–30 seconds to ensure stable throughput. All UEs experience a realistic channel model incorporating single-tap path loss, AWGN, Doppler effects, and randomized distances (\(d \in [0.5\text{\,km}, 2\text{\,km}]\)), simulating a mobile urban environment. This efficient configuration supports the evaluation of RL strategies under dynamic traffic and channel conditions.

\subsection{Overcoming Scalability Constraints in Open Source Deployments}
\label{subsec:scenarios-limits}

 \ifthenelse{\boolean{longversion}}{
% Although \texttt{srsRAN} provides a robust open-source framework for 5G prototyping—especially for short-range, SDR-based experiments—our fully software-driven setup (using \texttt{srsGNB}, \texttt{srsUE}, and \texttt{ZeroMQ}, with no physical SDR) pushes these open-source components well beyond their typical lab-scale usage. This approach aligns with the \texttt{srsProject}’s official O-RAN tutorials, yet it deviates from the hardware-focused design that underpins much of \texttt{srsRAN}’s architecture. Consequently, we expose several limitations in \texttt{srsUE}’s \emph{PHY layer}, the \texttt{srsGNB} E2 interface, and the ZeroMQ transport, which together make a purely software-based digital twin challenging for realistic O-RAN research.

% Commercial variants of \texttt{srsRAN}—and third-party enterprise RU emulators—address some of these scalability problems, but they do not provide the open-source, fully simulated PHY layer needed for large-scale experimentation. Likewise, even \texttt{srsRAN}’s commercial offerings do not include a comprehensive PHY-level digital twin, which is why many researchers turn to \texttt{Amarisoft}, \texttt{Open Air Interface}, or other third party proprietary solutions. Our objective, however, is to keep this work as open and accessible as possible for the broader research community. Below, we document the key issues that arise when \texttt{srsUE} is used as a purely software-based UE emulator, so that others attempting a similar architecture understand the challenges and potential next steps.
% 
}
Since \texttt{srsRAN}, originally designed for hardware-SDR use, one may face few issues while scaling up the number of simulated UEs in a purely software-based digital twin.
%can face scalability and realism constraints when deployed in a purely software-based digital twin—particularly without an O-RU or enterprise-grade RU emulator. While commercial variants and third-party emulators mitigate some issues, none provide a fully open-source simulated PHY. 
We aim to keep our solution accessible for the research community and thus outline the following key challenges in replicating our efforts:

\begin{enumerate}[label=\textbf{(\arabic*)}, leftmargin=1.5em]
  \item \textbf{PRACH Attach Delays} \\
  Simultaneous attachments overwhelm \texttt{ZeroMQ} buffers, causing crashes. To mitigate this, each \texttt{srsUE} adds a random PRACH offset, staggering attach attempts.

  \item \textbf{Downlink-Only Slicing} \\
  The current stable version of \texttt{srsUE} as of writing this paper only supports downlink slicing. While adequate for downlink use cases (e.g., video streaming), this excludes the stringent uplink demands of URLLC and mMTC.

  \item \textbf{Batching Transmitting UEs} \\
  Activating more than three \texttt{srsUE} instances concurrently saturates ZeroMQ and results in a channel throughput degradation. 
  To address this, only three UEs transmit simultaneously.
\end{enumerate}

  % \ifthenelse{\boolean{longversion}}{%
  %   \item \textbf{Limited Bandwidth \& Duplexing} \\
  %   The \texttt{srsUE} code supports only certain FDD bandwidths (5/10/15/20\,MHz), excluding mid-band TDD deployments (e.g., 3.5\,GHz) that are dominant in modern 5G. Its fixed 15 kHz subcarrier spacing (SCS) further constrains throughput and per-UE PRB allocation. Wider channels or advanced numerologies (e.g., 40--100\,MHz, 30--60kHz SCS) remain out of scope without a more robust UE (e.g., \texttt{AmarisoftUE}) or hardware SDR solutions, limiting software-only scenarios to narrowband, short-range tests. 

  %   \item \textbf{Path Loss Constraints.} \\
  %   During our RL training experiments, \texttt{srsUE} became unstable when path loss values in \texttt{GNU~Radio} exceeded 10--20\,dB, with most UEs failing to attach or disconnecting quickly. Empirically, UEs operated most reliably around 10\,dB---corresponding to distances of just a few centimeters (e.g., $\approx4$\,cm at 10\,dB in free space at 1.805\,GHz). At 20\,dB ($\approx13$\,cm), \texttt{srsUE} connected erratically and soon dropped, while at or above 50\,dB ($\approx4.2$\,m), all devices failed to attach. These results highlight how the default PHY-layer logic in the \texttt{srsUE}–\texttt{srsGNB} ZeroMQ pipeline expects extremely short-range conditions. Larger path-loss values, even on the order of a few meters, require extensive customization or external amplification to maintain stable connectivity.

  %   \item \textbf{Power Scaling Issue} \\
  %   In further testing, we probed the I/Q sample power at the gNB Tx and UE Rx ports. Despite raising nominal transmit or receive gains in \texttt{srsGNB}/\texttt{srsUE} to extreme levels (e.g., 105\,dB, 1000\,dB), measured magnitudes remained around 0.0 and 0.0093 in linear scale ($\approx -20.3$\,dB). Periodic spikes near 127\,dB appear linked to buffer saturations rather than real power changes. Consequently, adjusting these software-configured gains has no meaningful effect on the amplitude of transmitted waveforms, undermining attempts to emulate realistic link budgets or path-loss margins in a software-only environment. Practical workarounds include scaling samples manually (e.g., via \texttt{Multiply Const} blocks in \texttt{GNU~Radio}) or leveraging hardware SDR pipelines where transmit/receive gains do shape the I/Q signal.

  %   \item \textbf{Fragile Multi-UE Scalability} \\
  %   While multiple \texttt{srsUE} instances can attach to one \texttt{srsGNB}, purely software-driven concurrency remains unstable with more than three concurrent flows. \texttt{ZeroMQ} can saturate and drop frames, leading to gNB crashes. Our batching approach avoids immediate crashes but diminishes realism for large-scale O-RAN tests.

  %   \item \textbf{Rudimentary E2 Interface \& ASN1C Gaps} \\
  %   The E2 Service Node (E2SN) in \texttt{srsGNB} implements only a partial subset of E2SM procedures. Its \texttt{ASN1C} code lacks support for power control, HARQ reporting, or channel estimation. Consequently, our RL agents must rely on coarse KPI metrics, limiting the granularity of resource-management strategies in a digital twin environment.

  %   \item \textbf{Inconsistent UE Identification \& Helper Scripts} \\
  %   By default, \texttt{srsUE} assigns local indices in the order of successful PRACH attach, independent from the core network ID. If multiple UEs attach nearly simultaneously, collisions reorder or drop them. We mitigate this via a custom script that sends brief traffic bursts in a known sequence, matching E2SM IDs to the core IDs. Although this unifies labeling across logs, it remains a manual workaround to keep software-only multi-UE identification consistent.
    
  %   \item \textbf{Sensitivity to Complex Channels \& ZeroMQ Overheads} \\
  %   Although \texttt{srsUE} and \texttt{srsGNB} handle modest AWGN or single-tap fading, multi-tap or high-Doppler channels often desynchronize the PHY. Even minor timing drift ($\epsilon \neq 1.0$) at higher sample rates (e.g., 11.52\,MSPS) can break connectivity altogether. Moreover, \texttt{ZeroMQ} inflates CPU load and buffering delays, undermining near-real-time operation under heavier or more complex traffic. To remain within practical limits:    
  %   \begin{itemize}
  %     \item \textbf{Taps}: [0.85+0.25j, 0.4-0.1j, 0.2+0.1j] (reduced to mitigate QoS degradation).
  %     \item \textbf{Noise Voltage}: 0.01
  %     \item \textbf{Frequency Offset}: 0.00005 (96\,m/s, constant velocity), often lowered to 0.00000584 ($\approx 40$\,km/h).
  %     \item \textbf{Epsilon}: 1.0
  %   \end{itemize}
    
  %   \item \textbf{Alternative Transport Protocols for Higher Bitrates} \\
  %   While \texttt{ZeroMQ} remains a convenient default in \texttt{srsUE}–\texttt{srsGNB} loops, it often saturates under multi-UE loads, causing latency spikes or outright crashes. For advanced scenarios demanding higher throughput and real-time concurrency, alternative protocols (e.g., \texttt{gRPC}, shared-memory pipelines, or other low-overhead messaging) should be considered. These could better sustain simultaneous flows and large data bursts, mitigating ZeroMQ bottlenecks.
    
  %   \item \textbf{Lack of Robust Uplink Slicing} \\
  %   Although \texttt{srsGNB} contains partial UL PRB-assignment code, it is incomplete and often introduces multi-second delays or system failures when paired with \texttt{srsUE}. URLLC or IoT traffic thus cannot be orchestrated in real time under the current implementation, undermining the viability of truly dynamic uplink slicing in software-only emulation.

  %   \end{enumerate}
  %   \noindent
  %   \textbf{Summary.} In essence, the \texttt{srsGNB} and \texttt{srsUE} components can provide a promising basis for lightweight 5G experimentation—particularly when \emph{hardware SDRs} or external UEs are used for short-range tests. However, because our project emphasizes a purely software-based workflow (i.e., the \texttt{srsUE}–\texttt{ZeroMQ}–\texttt{srsGNB} pipeline) without SDR acceleration, many of the internal constraints (e.g., narrow bandwidth, minimal path-loss tolerance, limited concurrency) become far more pronounced. As these limitations illustrate, the current \texttt{srsUE}/\texttt{srsGNB} loop is not yet adequate for fully realistic, large-scale, and time-critical O-RAN scenarios—especially those requiring robust UL slicing, multi-UE concurrency, or advanced slice coordination. Nevertheless, the openness and modifiability of the \texttt{srsRAN} environment remain valuable for prototyping AI-driven RAN features in controlled settings. Researchers pursuing mid- to large-scale 5G/6G testbeds with complex topologies and stronger uplink demands will likely need complementary solutions (e.g., \texttt{OpenAirInterface}, hardware-based testbeds) or significant custom extensions to achieve higher fidelity and near-real-time performance.
  % }

\subsection{Evaluation Metrics and Logging}
\label{subsec:scenarios-eval}
% \bd{Do we need to say we have pretraining phase?} \rb{Let's add the pretraining phase justification here}
Performance is evaluated using per-UE E2 throughput at a \(500\,\mathrm{ms}\) sampling rate.
%, URLLC latency via packet timestamps or buffer logs, and PPO loss values to track RL convergence. 
Training the agent consists of two phase.
The first phase is the pre-training phase which utilizes the channel throughput we accomplished during our experiments with different number if PRBs in the simulator, rather than simulating the traffic. We use this pre-training phase so that the RL agent has a good starting weights for online learning inside the simulator.
%For training, a pre-training phase leverages a slice experiment-based regression to predict expected throughput for each PRB assignment under ideal channel conditions without simulating the traffic using the . 
This model is then integrated into the simulator for the second training phase, enabling the agent to learn performance implications under realistic constraints such as FSPL, AWGN, fading, and frequency drift. 
%\bd{How about adding this to the conclusion.}Future work could explore multi-tap fading, robust uplink slicing, and scaling beyond 12 UEs for advanced RL applications in O-RAN.

Algorithm \ref{PPO_PRB_Allocation} outlines the procedural steps for PPO-based PRB allocation. The training process begins with the initialization of the PPO agent, where policy and value function parameters are defined. During each training iteration, new tasks are generated for all UEs, and the corresponding state vector is constructed and provided to the PPO agent. Based on the observed state, the PPO agent selects an action that determines the PRB allocation for each slice. The xApp subsequently applies the allocation through RAN Controller (RC) messages and UEs start generating traffic. The KPIs are continuously monitored and logged to evaluate the system's performance. The observed KPIs are used to compute a reward, which is stored along with the state-action pair in the experience buffer. When an update step is reached, the algorithm computes the advantage estimate using Generalized Advantage Estimation (GAE), optimizes the policy via clipped surrogate loss, and updates the value function through regression on discounted returns. The training process continues iteratively until the convergence criteria are met.

\begin{algorithm}
\caption{PPO-based PRB Allocation}
\label{PPO_PRB_Allocation}
\begin{algorithmic}[1]  % Adds line numbers for better readability

    \State \textbf{Initialize:} PPO agent with policy parameters $\theta$ and value function parameters $\phi$

    \While{Training is not complete}
        \State Generate new tasks for all UEs
        \State Construct the state vector $\mathbf{s}_t$
        \State Provide state $\mathbf{s}_t$ to the PPO agent
        \State PPO agent selects action $a_t \sim \pi_{\theta}(a_t | \mathbf{s}_t)$
        \State Allocate PRBs to UEs based on action $a_t$
        \State xApp applies PRB allocation via RC messages
        \State UEs generate traffic
        \State xApp monitors KPIs and logs data in a JSON file
        \State Compute reward $r_t$ based on observed KPIs
        \State Store transition $(\mathbf{s}_t, a_t, r_t)$ in experience buffer
        
        \If {Update step is reached}
            \State Compute advantage estimate $\hat{A}_t$ using Generalized Advantage Estimation (GAE)
            \State Optimize policy $\pi_{\theta}$ using the clipped surrogate loss function
            \State Update value function $V_{\phi}$ 
            \State Perform gradient ascent on $\theta$ and $\phi$
        \EndIf
    \EndWhile

\end{algorithmic}
\end{algorithm}

One of the key advantages of using online training for PPO in this context is its ability to dynamically adapt to changing network conditions. Unlike offline training, which relies on precollected data, online training enables the agent to continuously refine its policy based on the interaction it has with the environment. This adaptability is crucial in O-RAN environments, where traffic patterns and resource demands fluctuate dynamically. However, this approach comes with a significant computational cost. %Continuous training and inference steps require substantial processing power, making real-time execution challenging, especially on resource-constrained edge devices. This high computational demand represents a bottleneck in our work.


\section{Quality of Service Analysis}
\label{sec:results}

In this section, we evaluate the performance of our RL-based slicing framework under the simulation scenarios described in Section~\ref{sec:scenarios}. Our analysis focuses on the overall QoS compliance and the stability of the proposed framework.

We compare the performance of our PPO-based resource allocation approach, detailed in Section~\ref{sec:method}, against DQN and four baseline methods. To assess resource allocation efficiency, we utilize the cumulative distribution function (CDF) to analyze key performance metrics. Specifically, we examine the latency distribution of URLLC traffic to determine compliance with stringent delay requirements. Additionally, for eMBB services, we evaluate the difference between the achieved downlink bitrate and the requested bitrate. For mMTC, we assess the discrepancy between the number of received bytes and the actual transmitted bytes.

Furthermore, we analyze the impact of network congestion and resource contention on service quality, highlighting the ability of each strategy to meet dynamic service demands. Finally, we evaluate overall QoS compliance by measuring adherence to throughput and latency targets across network slices. This includes examining the trade-offs between latency-sensitive URLLC traffic and the bandwidth-intensive requirements of eMBB and mMTC services, providing insights into the effectiveness of resource allocation under varying network conditions.
%In other words it shows the URLLC buffer latency, bitrate distribution for eMBB and transmitted bytes for the mMTC users.
%on the bitrate distribution for mMTC and eMBB slices, the latency characteristics of URLLC traffic, and the overall QoS compliance and stability of the framework. 
%the cumulative distribution function (CDF) of achieved downlink bitrates for mMTC and eMBB slices \bd{minus the requested bitrate only for the eMBB. For the mMTC we measure the number of transmitted bytes}, highlighting how effectively each strategy meets service demands under dynamic conditions. Similarly, the latency distribution of URLLC traffic is analyzed to assess compliance with stringent delay requirements. This includes examining the effects of network congestion and resource contention. Finally, we assess overall QoS compliance by monitoring adherence to throughput and latency targets across slices. This includes evaluating trade-offs between latency-sensitive URLLC traffic and bandwidth-intensive eMBB and mMTC flows, providing insights into how well resources are balanced under varying conditions.

For benchmarking, we compare the RL-based methods with four baseline resource allocation strategies:

\begin{itemize}[left=0pt, noitemsep]
    \item \textbf{Equal Allocation Baseline:} This method distributes the total number of PRBs equally among all active users using integer division. This approach provides a straightforward and fair division of resources but does not adapt dynamically to traffic demands or channel conditions.
    
    \item \textbf{Proportional Allocation Baseline:} This method assigns PRBs in proportion to user demand, which is determined differently based on the traffic type. For URLLC and mMTC users, demand is computed as the product of generation frequency and packet size, assuming a fixed generation period. PRBs are allocated according to each user's share of the total demand, with any leftover PRBs assigned to users with the largest fractional remainder. This approach ensures a more adaptive distribution of resources based on instantaneous demand.
    
    \item \textbf{Pre-allocated Proportional Baseline:} This method ensures that each user receives at least one PRB before distributing the remaining resources based on proportional demand. This method prevents extreme resource starvation by guaranteeing a minimum allocation while still adapting to demand variations.
    
    \item \textbf{3GPP-Based Proportional Fair Frequency-Domain Packet Scheduling (3GPP-PF) \cite{5062197}:} This method follows the proportional fair scheduling principle standardized in 3GPP, balancing user fairness and spectral efficiency by prioritizing users based on their channel conditions and past resource allocations. The proportional fair scheduler dynamically adjusts allocations to mitigate resource starvation and enhance overall spectral efficiency.
\end{itemize}

%\begin{itemize}[left=0pt, noitemsep]
%    \item \textbf{Equal Allocation Baseline:} This method distributes the total number of PRBs equally among all active users. The allocation follows a basic integer division, and any remaining PRBs are assigned based on a predefined priority scheme, prioritizing URLLC users first, followed by eMBB and then mMTC. This approach provides a straightforward and fair division of resources but does not adapt dynamically to traffic demands or channel conditions.

%    \item \textbf{Proportional Allocation Baseline:} This method assigns PRBs in proportion to user demand, which is determined differently based on the traffic type. For URLLC and mMTC users, demand is computed as the product of generation frequency and packet size, assuming a fixed generation period. For eMBB users, demand is determined based on the required bitrate. PRBs are allocated according to each user's share of the total demand, with any leftover PRBs assigned to users with the largest fractional remainder. This approach ensures a more adaptive distribution of resources based on instantaneous demand.

%    \item \textbf{Pre-allocated Proportional Baseline:} This method ensures that each user receives at least one PRB before distributing the remaining resources based on proportional demand. After the initial allocation, the remaining PRBs are assigned using the same proportional logic as the previous baseline. This method prevents extreme resource starvation by guaranteeing a minimum allocation while still adapting to demand variations.

%    \item \textbf{3GPP-Based Proportional Fair Frequency-Domain Packet Scheduling (3GPP-PF) \cite{5062197}:} This method follows the proportional fair scheduling principle standardized in 3GPP, balancing user fairness and spectral efficiency by prioritizing users based on their channel conditions and past resource allocations. It assigns PRBs to maximize the long-term throughput while ensuring that users receive a fair share of the resources over time. The proportional fair scheduler dynamically adjusts allocations to mitigate resource starvation and enhance overall spectral efficiency.
%\end{itemize}

\begin{figure*}[!t]
    \centering
    %---- First subfigure ----
    \begin{subfigure}[b]{0.31\textwidth}
        \centering
        \includegraphics[width=\linewidth]{thumbnail_urlcc_cdf.png}
        \caption{CDF of URLLC Latency.}
        \label{fig:urlcc_cdf}
    \end{subfigure}
    \hfill 
    %---- Second subfigure ----
    \begin{subfigure}[b]{0.31\textwidth}
        \centering
        \includegraphics[width=\linewidth]{thumbnail_embb_cdf.png}
        \caption{CDF of eMBB $\Delta$ Bitrate.}
        \label{fig:embb_cdf}
    \end{subfigure}
    \hfill
    %---- Third subfigure ----
    \begin{subfigure}[b]{0.31\textwidth}
        \centering
        \includegraphics[width=\linewidth]{thumbnail_mmtc_cdf.png}
        \caption{CDF of mMTC $\Delta$ Payload (Log Scale).}
        \label{fig:mmtc_cdf}
    \end{subfigure}
    \caption{Comparison of network slice CDFs.}
    \label{fig:three_cdfs}
\end{figure*}

The evaluation of the CDF plots, shown in Figures \ref{fig:urlcc_cdf}, 
\ref{fig:embb_cdf} and ~\ref{fig:mmtc_cdf} highlights the comparative performance of the RL-based approaches (PPO and DQN) against the baseline resource allocation methods.
In Figure~\ref{fig:mmtc_cdf}, which depicts the bitrate distribution for mMTC slices, PPO demonstrates superior performance by achieving higher throughput levels for a larger proportion of users compared to DQN and the baselines. Similarly, in Figure~\ref{fig:embb_cdf}, which illustrates the throughput distribution for eMBB slices, PPO consistently outperforms other methods, indicating its ability to allocate resources adaptively to meet the higher bandwidth demands of eMBB traffic.
For URLLC traffic, as shown in Figure~\ref{fig:urlcc_cdf}, PPO achieves lower latency for a greater percentage of packets, ensuring compliance with stringent delay requirements. Specifically, the latency CDF indicates that PPO prioritizes low-latency traffic more effectively by dynamically allocating PRBs to meet the strict timing needs of URLLC packets, even under fluctuating network conditions and resource contention with other traffic types. This is particularly evident in the steepness of the latency curve for PPO, which reflects a concentrated distribution of packets with delays well below the critical thresholds. Compared to DQN and the baselines, PPO exhibits a more robust adaptation to channel variability, maintaining consistent performance across different scenarios. While the pre-allocated proportional baseline provides a reasonable minimum latency for many packets due to its fairness-driven approach, its inability to dynamically adjust to traffic demands results in a heavier tail, where a non-negligible percentage of packets experience higher delays. The 3GPP-PF approach, while effective in maintaining fairness, is limited by its reliance on past allocations, leading to occasional inefficiencies in handling dynamic traffic loads. Overall, PPO's latency optimization demonstrates its capability to fulfill the ultra-reliable and low-latency requirements critical for URLLC applications, ensuring that stringent QoS objectives are met.

% First Figure: mMTC CDF
%\begin{figure}[!t]
%    \centering
%    \includegraphics[width=0.85\columnwidth]{thumbnail_mmtc_cdf.png}
%    \caption{CDF of mMTC Bytes Difference (Log Scale).}
%    \label{fig:mmtc_cdf}
%\end{figure}

% Second Figure: eMBB CDF
%\begin{figure}[!t]
%    \centering
%    \includegraphics[width=0.85\columnwidth]{thumbnail_embb_cdf.png}
%    \caption{CDF of eMBB Bitrate Difference.}
%    \label{fig:embb_cdf}
%\end{figure}

% Third Figure: URLLC CDF
%\begin{figure}[!t]
%    \centering
%    \includegraphics[width=0.85\columnwidth]{thumbnail_urlcc_cdf.png}
%    \caption{CDF of URLLC Latency.}
%    \label{fig:urlcc_cdf}
%\end{figure}

\section{Conclusion}
\label{sec:conclusion}
%\fa{the conclusion should be more detailed and impressive.}
In this paper, we presented a framework integrating the OSC near-RT RIC with srsRAN for real-time slicing and resource management in O-RAN. Using PPO in an RL-based xApp, we demonstrated adaptive resource allocation for URLLC, eMBB, and mMTC slices under realistic channel conditions modeled with GNU Radio. Our results show improved QoS compliance, including enhanced throughput, reduced latency, and stable resource distribution.

Despite these achievements, several challenges remain. One key limitation is the high computational demand of PPO, particularly in online training scenarios where policies must dynamically adapt to real-time network variations. This contrasts with offline training methods that rely on pre-collected datasets, reducing computational overhead but limiting adaptability.
Additionally the constraints imposed by not using SDRs particularly in handling increased numbers of UEs due to ZeroMQ.
%increasing the number of simulated UEs
%Additionally, srsRAN imposes constraints on scalability, particularly in handling increased numbers of UEs due to its ZeroMQ-based message passing system. These bottlenecks restrict the full potential of our RL-based slicing framework, necessitating further improvements in both hardware optimization and efficient scheduling mechanisms. 
Moreover, our work underscores the importance of balancing URLLC's stringent latency requirements with eMBB's high throughput demands. The ability of PPO to dynamically reallocate resources based on evolving network conditions proves advantageous, but additional refinements in state-space representation and reward design could enhance performance further. Future research should explore hybrid approaches combining online and offline RL training to mitigate computational burdens while preserving adaptability. Additionally, integrating multi-agent RL techniques could improve decision-making scalability in dense O-RAN deployments.

Overall, this work establishes a foundation for real-time RL-based slicing in O-RAN while identifying areas for future enhancement. Addressing these limitations will be crucial in achieving robust, scalable, and intelligent resource allocation strategies for next-generation wireless networks.



\ifthenelse{\boolean{longversion}}{
% Looking ahead, there are several promising directions to expand this research:
% \begin{itemize}
%     \item \textbf{Mid-band TDD and Massive MIMO:} Extending to n77/n78 frequencies and testing advanced hardware with MIMO capabilities would more closely approximate modern 5G setups and reveal new scaling behaviors.
%     \item \textbf{Robust UL Slicing:} A stable uplink PRB control mechanism in srsRAN would enable fully bidirectional URLLC and mMTC studies, offering deeper insight into real-world IoT scenarios.
%     \item \textbf{Multi-Tap or Clustered Fading Models:} Incorporating more complex channel conditions (e.g., multi-path, Rayleigh/Rician) could stress the RL agent’s adaptability beyond single-tap assumptions.
%     \item \textbf{Scalability and Large-UE Testing:} Investigating performance with dozens or even hundreds of UEs would reveal how well the RL xApp manages resources in dense traffic environments and whether ZeroMQ can be replaced or optimized.
% \end{itemize}
}{}


%\nocite{*}
%\bibliographystyle{IEEEtran}
%\bibliography{References}
\documentclass[preprint,11pt]{elsarticle}

\usepackage{amssymb}
\usepackage{amsthm}
\theoremstyle{definition}
\newtheorem{remark}{Remark}
\newtheorem{proposition}{Proposition}

\usepackage{geometry}
\geometry{margin=1.8cm}
\usepackage{amsmath}
\usepackage{mathabx}
\usepackage[ruled,linesnumbered]{algorithm2e}
\usepackage{graphicx}
\usepackage{float}
\usepackage{upgreek}
\usepackage{hyperref}
\hypersetup{colorlinks=true,linkcolor=black}
\SetArgSty{textnormal}


\title{Multimaterial topology optimization for finite strain elastoplasticity: theory, methods, and applications}

\author[1]{Yingqi Jia}

\author[1,2,3]{Xiaojia Shelly Zhang\texorpdfstring{\corref{corr-author}}}
\cortext[corr-author]{Corresponding author.}
\ead{zhangxs@illinois.edu}

\address[1]{Department of Civil and Environmental Engineering, University of Illinois Urbana-Champaign, Urbana, IL 61801, USA}
\address[2]{Department of Mechanical Science and Engineering, University of Illinois Urbana-Champaign, Urbana, IL 61801, USA}
\address[3]{National Center for Supercomputing Applications, Urbana, IL 61801, USA}


\begin{document}

\begin{abstract}
Plasticity is inherent to many engineering materials such as metals. While it can degrade the load-carrying capacity of structures via material yielding, it can also protect structures through plastic energy dissipation. To fully harness plasticity, here we present the theory, method, and application of a topology optimization framework that simultaneously optimizes structural geometries and material phases to customize the stiffness, strength, and structural toughness of designs experiencing finite strain elastoplasticity. The framework accurately predicts structural responses by employing a rigorous, mechanics-based elastoplasticity theory that ensures isochoric plastic flow. It also effectively identifies optimal material phase distributions using a gradient-based optimizer, where gradient information is obtained via a reversed adjoint method to address history dependence, along with automatic differentiation to compute the complex partial derivatives. We demonstrate the framework's capabilities by optimizing a range of 2D and 3D multimaterial elastoplastic structures with real-world applications, including energy-dissipating dampers, load-carrying beams, impact-resisting bumpers, and cold working profiled sheets. These optimized multimaterial structures reveal important mechanisms for improving design performance under large deformation, such as the transition from kinematic to isotropic hardening with increasing displacement amplitudes and the formation of twisted regions that concentrate stress, enhancing plastic energy dissipation. Through the superior performance of these optimized designs, we demonstrate the framework’s effectiveness in tailoring elastoplastic responses across various spatial configurations, material types, hardening behaviors, and combinations of candidate materials. This work offers a systematic approach for optimizing next-generation multimaterial structures with elastoplastic behaviors under large deformations.

\keyword{Plasticity; Finite strain; Isochoric plastic flow; Topology optimization; Multimaterial; Energy dissipation}
\endkeyword
\end{abstract}

\maketitle


\section{Introduction}

Plasticity is a fundamental physical phenomenon describing the permanent deformation of solids \citep{chaboche_review_2008} --- a trait that transcends elasticity and is inherent to critical materials such as metals, woods, foams, and soils. These materials form the backbone of engineering structures. The irreversible nature of plasticity holds two faces: it can cause undesirable deformations and a loss of structural load-carrying capacity or enable the formation of desired geometries and the dissipation of energy. In addition, plasticity is an inherently complex behavior, involving history-dependent stress-strain responses, material hardening, and large deformations that interact across multiple scales. Thus, there is a need for innovative methods to design and control plasticity at will.

To harness material plasticity, topology optimization \citep{bendsoe_generating_1988, bendsoe_topology_2003, wang_comprehensive_2021} as a form of computational morphogenesis offers great promise. This approach strategically distributes limited materials to optimize structural responses under given constraints. Therefore, it is naturally suited for controlling mechanical responses such as displacement fields \citep{jia_topology_2024, jia_unstructured_2024}, stress distributions \citep{jia_modulate_2024}, and fracture nucleation and propagation \citep{jia_controlling_2023}.

In light of its potential for controlling mechanical responses, topology optimization has recently achieved notable progress in tailoring plastic behaviors under the assumption of infinitesimal strains. For instance, employing a single material type with restricted usage, researchers have shown success in limiting plastic yielding \citep{amir_stress-constrained_2017, zhang_framework_2023, li_three-dimensional_2023}, maximizing end compliance/force \citep{boissier_elastoplastic_2021, desai_topology_2021}, and maximizing energy \citep{maute_adaptive_1998, zhang_topology_2017, li_topology_2017, li_design_2017, li_design_2017-1, alberdi_topology_2017, alberdi_design_2019}. Furthermore, studies have explored maximizing both compliance and energy \citep{abueidda_topology_2021} and fitting target stress--strain responses \citep{kim_microstructure_2020}. Beyond single-material strategies, \citet{nakshatrala_topology_2015} minimized energy propagation by filling distinct materials into fixed geometries, while \citet{li_topology_2021} optimized end force in similar setups.

Despite these significant advances in controlling structural plasticity under small deformations, optimizing finite strain plastic responses remains in its infancy. Only a few pioneering studies have tackled this challenge in the past decade. For example, \citet{wallin_topology_2016} and \citet{ivarsson_topology_2018, ivarsson_topology_2020} focused on optimizing structural geometries to improve system energy by specifying a single candidate material. Similarly, \citet{ivarsson_plastic_2021} maximized compliance while constraining plastic work through single-material topology optimization. In addition, \citet{alberdi_bi-material_2019} employed two candidate materials in fixed geometries to maximize plastic work. More recently, \citet{han_topology_2024} separately optimized structural geometries and material phases in different design cases to enhance stiffness and plastic hardening.

A review of past efforts highlights a key limitation: most studies either optimize structural geometries while fixing material types or optimize material phases while fixing structural geometries. To fully leverage topology optimization’s potential and further improve structural performance, simultaneous optimization of structural geometries and material phases is desired.
Such an approach would enable the creation of free-form, multimaterial structures with optimized plastic responses. Recently, \citet{jia_multimaterial_2025} pursued this goal and demonstrated preliminary success for infinitesimal strain plasticity; however, simultaneous optimization for finite strain plasticity remains under-explored.

In this study, we propose a multimaterial topology optimization framework for finite strain elastoplasticity. As illustrated in Fig. \ref{Fig: Idea Figure}(a), our goal is to \textit{concurrently} optimize structural geometries and material phases to maximize stiffness, strength (end force), and total energy (effective structural toughness) of structures undergoing finite strain plasticity. To achieve this goal, we present a multimaterial optimization approach in Fig. \ref{Fig: Idea Figure}(b), which addresses two major challenges --- accurate prediction of finite strain elastoplastic responses and path-dependent sensitivity analysis. To precisely predict plastic behaviors, we adopt a rigorous mechanics-based elastoplasticity theory from \citet{simo_framework_1988-1, simo_framework_1988}, which accounts for large deformations. We further adapt it to enforce isochoric plastic flow, accommodating a wide range of plastic materials including metals. To handle the path-dependent sensitivity analysis necessary for updating design variables, we employ the reversed adjoint method \citep{alberdi_unified_2018, jia_multimaterial_2025} to address history dependency and leverage automatic differentiation to effortlessly compute the required partial derivatives. By resolving these challenges, the framework establishes a closed-loop process that enables the creation of freeform, multimaterial optimized elastoplastic structures.

\begin{figure}[!htbp]
    \centering
    \includegraphics[width=18cm]{Idea_Figure.pdf}
    \caption{Multimaterial and multiobjective topology optimization for finite strain elastoplasticity. (a) Optimization setups. (b) Multimaterial topology optimization framework. The variable $J^\texttt{p}$ is the determinant of the plastic part of the deformation gradient. (c) Optimized elastoplastic designs with real-world applications.}
    \label{Fig: Idea Figure}
\end{figure}

Based on the proposed framework, we optimize a spectrum of representative multimaterial elastoplastic structures with real-world applications (Fig. \ref{Fig: Idea Figure}(c)) and uncover the mechanisms behind their optimized behaviors. For example, we design two-dimensional (2D) metallic yielding dampers with maximized energy dissipation for vibration mitigation under cyclic loadings, revealing a shift in dominance from kinematic to isotropic hardening as displacement amplitudes increase. We also explore the synergistic use of hyperelastic and elastoplastic materials to achieve various stiffness--strength balances in load-carrying beams. Extending the design to three dimensions (3D), we maximize the crashworthiness of impact-resisting bumpers, uncovering the formation of X-shaped structures in the middle to shorten the load path and enhance elastic energy absorption, as well as twisted regions aside to concentrate stress and increase plastic energy dissipation. Finally, we present optimized cold working profiled sheets undergoing ultra-large plastic deformations, demonstrating the simultaneous optimization of metal-forming and load-carrying stages while incorporating practical constraints such as cost, lightweight, and sustainability \citep{kundu2025sustainability}.

The optimized designs showcase the effectiveness of the proposed framework in freely controlling elastoplastic responses under large deformations in 2D and 3D, across different material types (elastic/plastic/mixed), various hardening behaviors (perfect/isotropic/kinematic/combined and linear/nonlinear), and arbitrary numbers of candidate materials (single-material/bi-material/multimaterial). Ultimately, the framework can provide a systematic tool for designing the next generation of multimaterial elastoplastic structures.

The remaining parts of this study are organized as follows. Section \ref{Sec: Finite Strain Elastoplasticity} recaps the finite strain elastoplasticity theory presented in \citet{simo_framework_1988-1, simo_framework_1988}, with additional emphasis on enforcing isochoric plastic flow. Building on this theory, Section \ref{Sec: Topology Optimization Framework} introduces the proposed multimaterial topology optimization framework for tailoring structural elastoplastic responses. To demonstrate the framework’s effectiveness, Section \ref{Sec: Sample Examples} presents a broad spectrum of optimized elastoplastic designs with real-world applications. Finally, Section \ref{Sec: Conclusions} provides concluding remarks.

This study is further supplemented by six appendices. \ref{Sec: Derivation of Updating Formula of be_bar} derives the proposed updating formulae for the internal variable to ensure isochoric plastic flow. \ref{Sec: Second Elastoplastic Moduli} presents the derivation of the second elastoplastic moduli, a key component of the elastoplasticity theory for a quadratic convergence rate. \ref{Sec: FEA Verification} verifies the accuracy of the theory and implementation by comparing it to analytical solutions. \ref{Sec: FEA Convergence} investigates the convergence, precision, and computational time of the finite element analysis (FEA). \ref{Sec: Sensitivity Analysis and Verification} details the derivation and verification of the path-dependent sensitivity analysis. Finally, \ref{Sec: Comparison of Meshes} examines the consistency of elastoplastic responses across various mesh and finite element combinations for the optimized designs.


\section{Finite strain elastoplasticity: theory and implementation}
\label{Sec: Finite Strain Elastoplasticity}

In this section, we provide a recap of the finite strain elastoplasticity theory from \citet{simo_framework_1988-1, simo_framework_1988} and outline its numerical implementation. We begin with an overview of the fundamentals of finite strain deformation and strain tensors, followed by a presentation of the local governing equations. Lastly, we introduce the global governing equations for finite strain elastoplasticity. The detailed explanations are presented below.

\subsection{Prerequisites: finite strain deformation and strain tensors}

We consider an open bounded domain ($\Omega_0 \subset \mathbb{R}^3$) occupied by a piece of undeformed material. For any material point $\mathbf{X} \in \Omega_0$, we define its displacement field as $\mathbf{u}(\mathbf{X})$ and the total deformation gradient as $\mathbf{F}(\mathbf{X}) = \mathbf{I} + \nabla \mathbf{u}(\mathbf{X})$. We also define the left and right Cauchy--Green deformation tensors as
\begin{equation*}
    \mathbf{b}(\mathbf{X}) = \mathbf{F}(\mathbf{X}) \mathbf{F}^\top(\mathbf{X})
    \quad \text{and} \quad
    \mathbf{C}(\mathbf{X}) = \mathbf{F}^\top(\mathbf{X}) \mathbf{F}(\mathbf{X}),
\end{equation*}
respectively. Additionally, we define the Lagrangian strain tensor as
\begin{equation} \label{Finite Strain Tensors}
    \mathbf{E}(\mathbf{X}) = \dfrac{1}{2}(\mathbf{C} - \mathbf{I}).
\end{equation}
Based on the multiplicative decomposition, the total deformation gradient can be split as
\begin{equation*}
    \mathbf{F}(\mathbf{X}) = \mathbf{F}^\texttt{e}(\mathbf{X}) \mathbf{F}^\texttt{p}(\mathbf{X})
\end{equation*}
where $\mathbf{F}^\texttt{e}(\mathbf{X})$ and $\mathbf{F}^\texttt{p}(\mathbf{X})$ are the elastic and plastic parts of $\mathbf{F}(\mathbf{X})$, respectively, with $\mathbf{F}^\texttt{p}$ corresponding to stress-free plastic deformation. For later use, we further define the elastic part of $\mathbf{b}(\mathbf{X})$ as
\begin{equation} \label{Elastic Part of b}
    \mathbf{b}^\texttt{e}(\mathbf{X})
    = \mathbf{F}^\texttt{e}(\mathbf{X}) {\mathbf{F}^\texttt{e}}^\top(\mathbf{X})
\end{equation}
and the plastic part of $\mathbf{C}(\mathbf{X})$ as
\begin{equation} \label{Plastic Part of C}
    \mathbf{C}^\texttt{p}(\mathbf{X})
    = {\mathbf{F}^\texttt{p}}^\top(\mathbf{X}) \mathbf{F}^\texttt{p}(\mathbf{X}).
\end{equation}

The relationships among these deformation tensors are illustrated in Fig. \ref{Fig: Configurations}. At load step $n$, the undeformed configuration ($\Omega_0$) is mapped to the intermediate configuration ($\Xi_n$) through plastic deformation ($\mathbf{F}_n^\texttt{p}$ or $\mathbf{C}_n^\texttt{p}$), and further to the deformed configuration ($\Omega_n$) via elastic deformation ($\mathbf{F}_n^\texttt{e}$ or $\mathbf{b}_n^\texttt{e}$). Alternatively, $\Omega_0$ can be directly mapped to $\Omega_n$ through the total deformation ($\mathbf{F}_n$). Similarly, at the next load step $n+1$, $\Omega_0$ maps to the intermediate configuration ($\Xi_{n+1}$) through $\mathbf{F}_{n+1}^\texttt{p}$ or $\mathbf{C}_{n+1}^\texttt{p}$, and subsequently to the deformed configuration ($\Omega_{n+1}$) via $\mathbf{F}_{n+1}^\texttt{e}$ or $\mathbf{b}_{n+1}^\texttt{e}$. Note that $\Omega_0$ can also directly map to $\Omega_{n+1}$ through $\mathbf{F}_{n+1}$. Additionally, $\Omega_n$ can be mapped to $\Omega_{n+1}$ using the relative deformation gradient expressed as
\begin{equation*}
    \mathbf{f}_{n+1} = \mathbf{F}_{n+1} \mathbf{F}_n^{-1}.
\end{equation*}
Finally, for later use, we define the volume-preserving parts of these deformation and strain tensors as
\begin{equation*}
    \overline{\mathbf{F}} = J^{-1/3} \mathbf{F}, \quad
    \overline{\mathbf{b}}^\texttt{e} = {(J^\texttt{e})}^{-2/3} \mathbf{b}^\texttt{e}, \quad \text{and} \quad
    \overline{\mathbf{f}} = J_f^{-1/3} \mathbf{f},
\end{equation*}
and express the determinants as
\begin{equation*}
    J = \text{det}(\mathbf{F}), \quad
    J^\texttt{e} = \text{det}(\mathbf{F}^\texttt{e}), \quad
    J^\texttt{p} = \text{det}(\mathbf{F}^\texttt{p}), \quad \text{and} \quad
    J_f = \text{det}(\mathbf{f}).
\end{equation*}

\begin{figure}[!htbp]
    \centering
    \includegraphics[width=13cm]{Configurations.pdf}
    \caption{Relationships among the deformation tensors used in the finite strain elastoplasticity theory.}
    \label{Fig: Configurations}
\end{figure}

\subsection{Local governing equations for finite strain elastoplasticity}

In this subsection, we introduce the local governing equations for finite strain elastoplasticity, including the constitutive relationships, flow rules, hardening laws, and yield conditions. Combining these principles, we conclude with the radial return mapping scheme, which is employed to determine the elastoplastic response of a material point.

\subsubsection{Constitutive relationships}

We consider a hyperelastic material with a strain energy density function expressed as
\begin{equation} \label{Strain Energy Density Function}
    W(J^\texttt{e}, \overline{\mathbf{b}}^\texttt{e}) = U(J^\texttt{e}) + \overline{W}(\overline{\mathbf{b}}^\texttt{e}),
\end{equation}
which is characterized by separable volumetric ($U$) and deviatoric ($\overline{W}$) parts. One classical choice of such function reads as \citep{simo_computational_2006}
\begin{equation} \label{Volumetric and Deviatoric Engery}
    U(J^\texttt{e}) = \dfrac{\kappa}{2} \left\{ \frac{1}{2}\left[(J^\texttt{e})^2-1\right] - \ln J^\texttt{e} \right\}
    \quad \text{and} \quad
    \overline{W}(\overline{\mathbf{b}}^\texttt{e})
    = \dfrac{\mu}{2} \left[ \text{tr}(\overline{\mathbf{b}}^\texttt{e}) - 3 \right],
\end{equation}
where $\mu$ and $\kappa$ are the initial shear and bulk moduli, respectively. Based on the strain energy density function in \eqref{Strain Energy Density Function}, we derive the Kirchhoff stress tensor as
\begin{equation} \label{Kirchhoff Stress}
    \widehat{\boldsymbol{\tau}} = \dfrac{\partial W}{\partial \mathbf{F}^\texttt{e}} {\mathbf{F}^\texttt{e}}^\top
    = \boldsymbol{\tau}^\texttt{vol} + \widehat{\mathbf{s}}
    = J^\texttt{e} U'(J^\texttt{e}) \mathbf{I} + \mu\ \text{dev} (\overline{\mathbf{b}}^\texttt{e}),
\end{equation}
where we separate $\widehat{\boldsymbol{\tau}}$ into the volumetric ($\boldsymbol{\tau}^\texttt{vol} = J^\texttt{e} U'(J^\texttt{e}) \mathbf{I}$) and deviatoric ($\widehat{\mathbf{s}} = \mu\ \text{dev} (\overline{\mathbf{b}}^\texttt{e})$) parts. Finally, we define the net stress tensor as
\begin{equation*}
    \boldsymbol{\xi} = \widehat{\mathbf{s}} - \text{dev}(\overline{\boldsymbol{\beta}})
    \quad \text{with} \quad
    \overline{\boldsymbol{\beta}} = J^{-2/3} \boldsymbol{\beta}
\end{equation*}
where $\boldsymbol{\beta}$ is the back stress tensor triggered by the kinematic hardening of materials.

\subsubsection{Flow rules and kinematic hardening laws}

After defining the deformation and strain tensors as well as the constitutive relationships, we proceed with the flow rules and hardening laws. In the context of the finite strain elastoplasticity \citep{simo_framework_1988-1, simo_framework_1988}, we present the flow rules as 
\begin{equation} \label{Flow Rules}
    \mu J^{-2/3} \text{dev}( \mathcal{L}_v \mathbf{b}^\texttt{e} )
    = - 2 \overline{\overline{\mu}} \gamma \mathbf{n}
    \quad \text{and} \quad
    \text{tr}( \mathcal{L}_v \mathbf{b}^\texttt{e} ) = 0,
\end{equation}
and Prager--Ziegler-type kinematic hardening laws as
\begin{equation} \label{Kinematic Hardening Laws}
    \mu J^{-2/3} \text{dev}( \mathcal{L}_v \boldsymbol{\beta} )
    = \dfrac{2}{3} h \overline{\overline{\mu}} \gamma \mathbf{n}
    \quad \text{and} \quad
    \text{tr}( \mathcal{L}_v \boldsymbol{\beta} ) = 0.
\end{equation}
Here, $h$ is the kinematic hardening modulus of materials, and $\gamma \geq 0$ is a consistency parameter that will be determined later. The variable $\mathbf{n}$ is a unit tensor defined as $\mathbf{n} = \boldsymbol{\xi} / \lVert \boldsymbol{\xi} \rVert$ where $\lVert \boldsymbol{\xi} \rVert = \sqrt{\boldsymbol{\xi} : \boldsymbol{\xi}}$. The parameter $\overline{\overline{\mu}}$ is defined as
\begin{equation*}
    \overline{\overline{\mu}} = \overline{\mu} - \dfrac{1}{3} \text{tr}(\overline{\boldsymbol{\beta}})
    \quad \text{with} \quad
    \overline{\mu} = \dfrac{\mu}{3} J^{-2/3} \text{tr}(\mathbf{b}^\texttt{e}).
\end{equation*}
Additionally, the operator $\mathcal{L}_v$ in \eqref{Flow Rules}--\eqref{Kinematic Hardening Laws} is the Lie time derivative defined as
\begin{equation*}
    \mathcal{L}_v \mathbf{T} = \varphi_* \left[ \dfrac{\partial}{\partial t} \varphi^* (\mathbf{T}) \right]
\end{equation*}
for a dummy tensor $\mathbf{T}$ where $t$ represents the time. The symbols $\varphi^*$ and $\varphi_*$ are the pull-back and push-forward operators expressed as
\begin{equation*}
    \varphi^*(\mathbf{T}) = \mathbf{F}^{-1} \mathbf{T} \mathbf{F}^{-\top}
    \quad \text{and} \quad
    \varphi_*(\mathbf{T}) = \mathbf{F} \mathbf{T} \mathbf{F}^\top,
\end{equation*}
respectively.

\subsubsection{Yield criterion and isotropic hardening laws}

To account for the yielding of materials, we adopt the classical von Mises yield criterion defined as
\begin{equation} \label{Yield Criterion}
    f(\boldsymbol{\xi}, \alpha) = \lVert \boldsymbol{\xi} \rVert - \sqrt{\dfrac{2}{3}} k(\alpha) \leq 0
\end{equation}
where $f(\boldsymbol{\xi}, \alpha) < 0$ represents that the material deforms elastically, and $f(\boldsymbol{\xi}, \alpha) = 0$ denotes the onset of yielding. To enforce the yield criterion of $f(\boldsymbol{\xi}, \alpha) \leq 0$, we apply the Kuhn--Tucker conditions expressed as
\begin{equation} \label{Kuhn Tucker Conditions}
    \gamma \geq 0
    \quad \text{and} \quad
    \gamma f(\boldsymbol{\xi}, \alpha) = 0.
\end{equation}

In the above equations, the non-decreasing variable $\alpha$ is the equivalent plastic strain, whose evolution is governed by \citep{simo_framework_1988-1, simo_framework_1988}
\begin{equation} \label{Evolution of Equivalent Plastic Strain}
    \dfrac{\partial \alpha}{\partial t} = \sqrt{\dfrac{2}{3}} \gamma.
\end{equation}
The function $k(\alpha)$ in \eqref{Yield Criterion} determines the radius of the yield surface and characterizes the isotropic hardening of materials, which can follow either a linear or nonlinear form. For linear isotropic hardening, we set
\begin{equation} \label{Linear Isotropic Hardening Law}
    k(\alpha) = \sigma_y + K \alpha
\end{equation}
where $\sigma_y$ is the initial yield strength of the material, and $K$ is the isotropic hardening modulus. For nonlinear isotropic hardening, we adopt a classical expression as \citep{simo_framework_1988}
\begin{equation} \label{Nonlinear Isotropic Hardening Law}
    k(\alpha) = \sigma_y + K \alpha + (\sigma_\infty - \sigma_y)(1 - e^{- \delta \alpha})
\end{equation}
where $\sigma_\infty$ is the residual yield strength, and $\delta>0$ is a saturation exponent.

\subsubsection{Radial return mapping scheme}
\label{Return mapping scheme}

In previous discussions, we have established the constitutive relationship in \eqref{Kirchhoff Stress}, flow rules in \eqref{Flow Rules}, kinematic hardening laws in \eqref{Kinematic Hardening Laws}, yield criterion in \eqref{Yield Criterion}, Kuhn--Tucker conditions in \eqref{Kuhn Tucker Conditions}, the evolution of the equivalent plastic strain in \eqref{Evolution of Equivalent Plastic Strain}, and isotropic hardening laws in \eqref{Linear Isotropic Hardening Law}--\eqref{Nonlinear Isotropic Hardening Law}. A collection of these requirements fully governs the finite strain elastoplastic responses of any material point $\mathbf{X} \in \Omega_0$. To reconcile these governing equations and solve for the material elastoplastic responses, we adopt a canonical radial return mapping scheme \citep{wilkins1969calculation, krieg_accuracies_1977} as follows.

Applying a backward Euler difference scheme to the above local governing equations, we derive their time-discretized versions as
\begin{equation} \label{Local Governing Equations}
    \left\{ \begin{array}{l}
        \overline{\mathbf{b}}_{n+1}^\texttt{e}
        = \overline{\mathbf{f}}_{n+1} \overline{\mathbf{b}}_n^\texttt{e} \overline{\mathbf{f}}_{n+1}^\top
        - \dfrac{2 \overline{\overline{\mu}}_{n+1}}{\mu} \widehat{\gamma}_{n+1} \mathbf{n}_{n+1}, \quad
        \overline{\boldsymbol{\beta}}_{n+1} = \overline{\mathbf{f}}_{n+1} \overline{\boldsymbol{\beta}}_n \overline{\mathbf{f}}_{n+1}^\top
        + \dfrac{2 h \overline{\overline{\mu}}_{n+1}}{3 \mu} \widehat{\gamma}_{n+1} \mathbf{n}_{n+1}, \\[10pt]

        \alpha_{n+1} = \alpha_{n} + \sqrt{\dfrac{2}{3}} \widehat{\gamma}_{n+1}, \quad
        \widehat{\mathbf{s}}_{n+1} = \mu\ \text{dev}(\overline{\mathbf{b}}_{n+1}^\texttt{e}), \quad
        \boldsymbol{\xi}_{n+1} = \widehat{\mathbf{s}}_{n+1} - \text{dev} \left( \overline{\boldsymbol{\beta}}_{n+1} \right),
        \\[10pt]

        \mathbf{n}_{n+1} = \boldsymbol{\xi}_{n+1} / \lVert \boldsymbol{\xi}_{n+1} \rVert, \quad
        \overline{\mu}_{n+1} = \dfrac{\mu}{3} \text{tr}(\overline{\mathbf{b}}_{n+1}^\texttt{e}), \quad
        \overline{\overline{\mu}}_{n+1} = \overline{\mu}_{n+1} - \dfrac{1}{3} \text{tr} (\overline{\boldsymbol{\beta}}_{n+1}), \\[10pt]

        f_{n+1} = \lVert \boldsymbol{\xi}_{n+1} \rVert - \sqrt{\dfrac{2}{3}} k(\alpha_{n+1}) \leq 0, \quad
        \widehat{\gamma}_{n+1} \geq 0, \quad
        \widehat{\gamma}_{n+1} f_{n+1} = 0.
    \end{array} \right.
\end{equation}
In these expressions, we recall that $\mathbf{f}_{n+1} = \mathbf{F}_{n+1} \mathbf{F}_n^{-1}$ is the relative deformation gradient between load steps $n$ and $n+1$, and $\overline{\mathbf{f}}_{n+1} = (J_{n+1}^f)^{-1/3} \mathbf{f}_{n+1}$ is its volume-preserving part with $J_{n+1}^f = \det (\mathbf{f}_{n+1})$. The variable $\widehat{\gamma}_{n+1}$ is shorted for $\Delta t \gamma_{n+1}$, and $\Delta t$ is the time interval.

\begin{remark}
For the derivation of $\overline{\mathbf{b}}_{n+1}^\texttt{e}$ and $\overline{\mu}_{n+1}$ in \eqref{Local Governing Equations} and all subsequent analyses, we adopt the assumption of isochoric plastic flow ($J_n^\texttt{p} = J_{n+1}^\texttt{p} = 1$). This assumption is commonly accepted and even required \citep{weber_finite_1990, simo_algorithms_1992, simo_associative_1992, wang_how_2017} for many elastoplastic materials such as metals, and it implies that the volume of the material does not change during plastic deformation.
\end{remark}

To solve \eqref{Local Governing Equations} with given $\overline{\mathbf{b}}_n^\texttt{e}$, $\overline{\boldsymbol{\beta}}_n$, $\alpha_n$, $\mathbf{F}_n$, and $\mathbf{F}_{n+1}$, we \textit{temporarily} assume no new plastic flow ($\widehat{\gamma}_{n+1}=0$) and define the trial variables as
\begin{equation} \label{Trial Variables}
    \left\{ \begin{array}{llll}
        \overline{\mathbf{b}}_{n+1}^\texttt{e,tr}
        = \overline{\mathbf{f}}_{n+1} \overline{\mathbf{b}}_n^\texttt{e} \overline{\mathbf{f}}_{n+1}^\top,

        & \overline{\boldsymbol{\beta}}_{n+1}^\texttt{tr}
        = \overline{\mathbf{f}}_{n+1} \overline{\boldsymbol{\beta}}_n \overline{\mathbf{f}}_{n+1}^\top,

        & \alpha_{n+1}^\texttt{tr} = \alpha_n, \\[10pt]

        \mathbf{s}_{n+1}^\texttt{tr} = \mu\ \text{dev} \left( \overline{\mathbf{b}}_{n+1}^\texttt{e,tr} \right),

        &\boldsymbol{\xi}_{n+1}^\texttt{tr}
        = \mathbf{s}_{n+1}^\texttt{tr} - \text{dev} \left( \overline{\boldsymbol{\beta}}_{n+1}^\texttt{tr} \right),

        &\mathbf{n}_{n+1}^\texttt{tr} = \boldsymbol{\xi}_{n+1}^\texttt{tr} / \lVert \boldsymbol{\xi}_{n+1}^\texttt{tr} \rVert, \\[10pt]

        \overline{\mu}_{n+1}^\texttt{tr} = \dfrac{\mu}{3} \text{tr}\left( \overline{\mathbf{b}}_{n+1}^\texttt{e,tr} \right),

        & \overline{\overline{\mu}}_{n+1}^\texttt{tr} = \overline{\mu}_{n+1}^\texttt{tr} - \dfrac{1}{3} \text{tr} \left( \overline{\boldsymbol{\beta}}_{n+1}^\texttt{tr} \right),

        & f_{n+1}^\texttt{tr} = \lVert \boldsymbol{\xi}_{n+1}^\texttt{tr} \rVert
        - \sqrt{\dfrac{2}{3}} k \left( \alpha_{n+1}^\texttt{tr} \right).
    \end{array} \right.
\end{equation}
Next, we correct $\overline{\boldsymbol{\beta}}_{n+1}$ and $\alpha_{n+1}$ with
\begin{equation} \label{Updated beta and alpha}
    \overline{\boldsymbol{\beta}}_{n+1} = \overline{\boldsymbol{\beta}}_{n+1}^\texttt{tr}
    + \dfrac{2 h \overline{\overline{\mu}}_{n+1}^\texttt{tr}}{3 \mu} \widehat{\gamma}_{n+1} \mathbf{n}_{n+1}
    \quad \text{and} \quad
    \alpha_{n+1} = \alpha_{n+1}^\texttt{tr} + \sqrt{\dfrac{2}{3}} \widehat{\gamma}_{n+1}
\end{equation}
based on $\eqref{Local Governing Equations}_{2,3}$ and Proposition \ref{Compute mu_bar and mu_bar_bar} and correct $\overline{\mathbf{b}}_{n+1}^\texttt{e}$ with
\begin{equation} \label{Updated be_bar}
    \left\{ \begin{array}{l}
        \overline{\mathbf{b}}_{n+1}^\texttt{e}
        = \text{dev} \left( \overline{\mathbf{b}}_{n+1}^\texttt{e} \right) + \dfrac{1}{3} \mathcal{I}_1 \mathbf{I}, \\[12pt]

        \text{dev} \left( \overline{\mathbf{b}}_{n+1}^\texttt{e} \right) = \text{dev} \left( \overline{\mathbf{b}}_{n+1}^\texttt{e,tr} \right) - \dfrac{2 \overline{\overline{\mu}}_{n+1}^\texttt{tr}}{\mu} \widehat{\gamma}_{n+1} \mathbf{n}_{n+1}, \\[12pt]

        \mathcal{I}_1 = \left\{ \begin{array}{ll}
            \left( -\dfrac{\mathcal{Q}}{2} + \sqrt{-\Delta} \right)^{1/3} + \left( -\dfrac{\mathcal{Q}}{2} - \sqrt{-\Delta} \right)^{1/3}, & \text{if $\Delta < 0$;} \\[12pt]
    
            \max \left\{ \dfrac{3 \mathcal{Q}}{\mathcal{P}}, - \dfrac{3 \mathcal{Q}}{2 \mathcal{P}} \right\}, & \text{if $\Delta = 0$;} \\[12pt]  
    
            \mathcal{S}_1, & \text{if $\Delta > 0$, $\left( \overline{\mathbf{b}}_{n+1}^\texttt{e} \right)_{11} > 0$,
            and $\mathcal{B} > 0$}; \\[12pt]
    
            \mathcal{S}_2, & \text{otherwise.}
        \end{array} \right.
    \end{array} \right.
\end{equation}
Here, the related variables are defined as
\begin{equation*}
    \left\{ \begin{array}{l}
        \mathcal{P} = - \mathcal{J}_2 \leq 0, \quad
        \mathcal{Q} = \mathcal{J}_3 - 1, \quad
        \Delta = - \left( \dfrac{\mathcal{P}^3}{27} + \dfrac{\mathcal{Q}^2}{4} \right), \\[12pt]

        \mathcal{J}_2 = \dfrac{1}{2} \left\lVert \text{dev} \left( \overline{\mathbf{b}}_{n+1}^\texttt{e} \right) \right\rVert^2, \quad 
        \mathcal{J}_3 = \det \left[ \text{dev} \left( \overline{\mathbf{b}}_{n+1}^\texttt{e} \right) \right], \\[12pt]
   
        \mathcal{B} = \left( \overline{\mathbf{b}}_{n+1}^\texttt{e} \right)_{11} \left( \overline{\mathbf{b}}_{n+1}^\texttt{e} \right)_{22} - \left( \overline{\mathbf{b}}_{n+1}^\texttt{e} \right)_{12} \left( \overline{\mathbf{b}}_{n+1}^\texttt{e} \right)_{21}.
    \end{array} \right.
\end{equation*}
Additionally, we define $\mathcal{S}_1$ and $\mathcal{S}_2$ as the largest and second to largest values among $r_1$, $r_2$, and $r_3$ with $r_k$ defined as
\begin{equation} \label{Trigonometric Solutions}
    r_k = 2 \sqrt{-\dfrac{\mathcal{P}}{3}} \cos \left[ \dfrac{1}{3} \arccos \left( \dfrac{3 \mathcal{Q}}{2 \mathcal{P}} \sqrt{-\dfrac{3}{\mathcal{P}}} \right) - \dfrac{2 k \pi}{3} \right]
    \quad \text{for} \quad
    k = 1, 2, 3.
\end{equation}

\begin{proposition} \label{Compute mu_bar and mu_bar_bar}
The variables $\overline{\mu}$ and $\overline{\overline{\mu}}$ satisfy $\overline{\mu}_{n+1} = \overline{\mu}_{n+1}^\texttt{tr}$ and $\overline{\overline{\mu}}_{n+1} = \overline{\overline{\mu}}_{n+1}^\texttt{tr}$, respectively.
\end{proposition}

\begin{proof}
Applying the backward Euler difference on $\eqref{Flow Rules}_2$ derives
\begin{equation*}
    0 = \text{tr} \left[ \mathbf{F}_{n+1} \left( \mathbf{F}_{n+1}^{-1} \mathbf{b}_{n+1}^\texttt{e} \mathbf{F}_{n+1}^{-\top} - \mathbf{F}_n^{-1} \mathbf{b}_n^\texttt{e} \mathbf{F}_n^{-\top} \right) \mathbf{F}_{n+1}^\top \right]
    = \text{tr} \left( \mathbf{b}_{n+1}^\texttt{e} \right) - \text{tr} \left( \mathbf{f}_{n+1} \mathbf{b}_n^\texttt{e} \mathbf{f}_{n+1}^\top \right).
\end{equation*}
Note that
\begin{equation*}
    J_{n+1}^{-2/3} \mathbf{b}_{n+1}^\texttt{e} = (J_{n+1}^\texttt{e})^{-2/3} \mathbf{b}_{n+1}^\texttt{e} = \overline{\mathbf{b}}_{n+1}^\texttt{e}
\end{equation*}
and
\begin{equation*}
    J_{n+1}^{-2/3} \mathbf{f}_{n+1} \mathbf{b}_n^\texttt{e} \mathbf{f}_{n+1}^\top = \left( J_{n+1}^f J_n^\texttt{e} \right)^{-2/3} \mathbf{f}_{n+1} \mathbf{b}_n^\texttt{e} \mathbf{f}_{n+1}^\top = \overline{\mathbf{f}}_{n+1} \overline{\mathbf{b}}_n^\texttt{e} \overline{\mathbf{f}}_{n+1}^\top = \overline{\mathbf{b}}_{n+1}^\texttt{e,tr}.
\end{equation*}
We then prove
\begin{equation*}
    \text{tr} \left( \overline{\mathbf{b}}_{n+1}^\texttt{e} \right)
    = \text{tr} \left( \overline{\mathbf{b}}_{n+1}^\texttt{e,tr} \right)
    \quad \Rightarrow \quad
    \overline{\mu}_{n+1} = \dfrac{\mu}{3} \text{tr} \left( \overline{\mathbf{b}}_{n+1}^\texttt{e} \right) = \dfrac{\mu}{3} \text{tr} \left( \overline{\mathbf{b}}_{n+1}^\texttt{e,tr} \right) = \overline{\mu}_{n+1}^\texttt{tr}.
\end{equation*}
Similarly, applying the backward Euler difference on $\eqref{Kinematic Hardening Laws}_2$ derives
\begin{equation*}
    \text{tr} \left( \boldsymbol{\beta}_{n+1} \right) = \text{tr} \left( \mathbf{f}_{n+1} \boldsymbol{\beta}_n \mathbf{f}_{n+1}^\top \right)
    \quad \Rightarrow \quad
    \text{tr} \left( \overline{\boldsymbol{\beta}}_{n+1} \right) = \text{tr} \left( \overline{\boldsymbol{\beta}}_{n+1}^\texttt{tr} \right),
\end{equation*}
and we prove
\begin{equation*}
    \overline{\overline{\mu}}_{n+1} = \overline{\mu}_{n+1} - \dfrac{1}{3} \text{tr} \left( \overline{\boldsymbol{\beta}}_{n+1} \right) = \overline{\mu}_{n+1}^\texttt{tr} - \dfrac{1}{3} \text{tr} \left( \overline{\boldsymbol{\beta}}_{n+1}^\texttt{tr} \right) = \overline{\overline{\mu}}_{n+1}^\texttt{tr}.
\end{equation*}
\end{proof}

\begin{remark}
The updating formulae for $\overline{\mathbf{b}}_{n+1}^\texttt{e}$ in \eqref{Updated be_bar} is different from $\eqref{Local Governing Equations}_1$ as used in \citet{simo_framework_1988-1, simo_framework_1988}. It is because the expression in $\eqref{Local Governing Equations}_1$ is a necessary but \textit{insufficient} condition of the isochoric plastic flow ($J_{n+1}^\texttt{p} = 1$). As pointed out in \citet{wang_how_2017} and demonstrated in \ref{Sec: Comparison with Analytical Solution}, the violation of the isochoric plastic flow leads to incorrect stress prediction, especially during the unloading stage. Many remedies such as the exponential map algorithm \citep{simo_algorithms_1992, miehe_exponential_1996} can fix this issue. In the current work, we propose the updating formulae in \eqref{Updated be_bar} that require minimum modifications to the original theory \citep{simo_framework_1988-1, simo_framework_1988}, and this formula is essentially a generalized version compared to the one used in \citet{simo_associative_1992}. The derivation of \eqref{Updated be_bar} is presented in \ref{Sec: Derivation of Updating Formula of be_bar}. A comparison between the resultant FEA prediction and the analytical solution is shown in \ref{Sec: FEA Verification}, where we observe that the isochoric plastic flow is enforced precisely.
\end{remark}

We emphasize that the updating formulae in \eqref{Updated beta and alpha} and \eqref{Updated be_bar} are fully explicit up to the unit tensor ($\mathbf{n}_{n+1}$) and consistency parameter ($\widehat{\gamma}_{n+1}$), and we proceed to determine them. To derive the unit tensor ($\mathbf{n}_{n+1}$), we utilize $\eqref{Local Governing Equations}_4$, $\eqref{Trial Variables}_4$, and $\eqref{Updated be_bar}_2$ and yield
\begin{equation} \label{Deviatoric Kirchhoff Stress}
    \mathbf{s}_{n+1} = \mu\ \text{dev}(\overline{\mathbf{b}}_{n+1}^\texttt{e,tr})
    - 2 \overline{\overline{\mu}}_{n+1}^\texttt{tr} \widehat{\gamma}_{n+1} \mathbf{n}_{n+1}
    = \mathbf{s}_{n+1}^\texttt{tr}
    - 2 \overline{\overline{\mu}}_{n+1}^\texttt{tr} \widehat{\gamma}_{n+1} \mathbf{n}_{n+1}
\end{equation}
by noticing $\mathbf{n}_{n+1}$ is deviatoric. Based on $\eqref{Local Governing Equations}_2$, $\eqref{Trial Variables}_{2,5}$, and \eqref{Deviatoric Kirchhoff Stress}, we rewrite $\eqref{Local Governing Equations}_5$ as
\begin{equation*}
    \boldsymbol{\xi}_{n+1} = \mathbf{s}_{n+1} - \text{dev} \left( \overline{\boldsymbol{\beta}}_{n+1} \right)
    = \boldsymbol{\xi}_{n+1}^\texttt{tr}
    - 2 \overline{\overline{\mu}}_{n+1}^\texttt{tr}
    \left( 1 + \dfrac{h}{3\mu} \right) \widehat{\gamma}_{n+1} \mathbf{n}_{n+1},
\end{equation*}
which further renders
\begin{equation*}
    \lVert \boldsymbol{\xi}_{n+1} \rVert \mathbf{n}_{n+1}
    = \lVert \boldsymbol{\xi}_{n+1}^\texttt{tr} \rVert \mathbf{n}_{n+1}^\texttt{tr}
    - 2 \overline{\overline{\mu}}_{n+1}^\texttt{tr}
    \left( 1 + \dfrac{h}{3\mu} \right) \widehat{\gamma}_{n+1} \mathbf{n}_{n+1}
\end{equation*}
and implies
\begin{equation} \label{Relationship between xi and xi_trial}
    \lVert \boldsymbol{\xi}_{n+1} \rVert
    + 2 \overline{\overline{\mu}}_{n+1}^\texttt{tr}
    \left( 1 + \dfrac{h}{3\mu} \right) \widehat{\gamma}_{n+1}
    = \lVert \boldsymbol{\xi}_{n+1}^\texttt{tr} \rVert
    \quad \text{and} \quad
    \mathbf{n}_{n+1} = \mathbf{n}_{n+1}^\texttt{tr}
\end{equation}
under the assumption of $\overline{\overline{\mu}}_{n+1}^\texttt{tr} > 0$ and due to $\lVert \mathbf{n}_{n+1} \rVert = \lVert \mathbf{n}_{n+1}^\texttt{tr} \rVert = 1$.

Finally, to compute the consistency parameter ($\widehat{\gamma}_{n+1}$), we discuss the signs of the trial yield indicator, $f_{n+1}^\texttt{tr}$. In the case of $f_{n+1}^\texttt{tr} \leq 0$, the yield criterion is satisfied ($f_{n+1} < 0$ and $\widehat{\gamma}_{n+1} = 0$) with $\overline{\mathbf{b}}_{n+1}^\texttt{e,tr}$, $\overline{\boldsymbol{\beta}}_{n+1}^\texttt{tr}$, and $\alpha_{n+1}^\texttt{tr}$. We then use them to update the internal variables at load step $n+1$ as
\begin{equation*}
    \overline{\mathbf{b}}_{n+1}^\texttt{e} = \overline{\mathbf{b}}_{n+1}^\texttt{e,tr}, \quad
    \overline{\boldsymbol{\beta}}_{n+1} = \overline{\boldsymbol{\beta}}_{n+1}^\texttt{tr},
    \quad \text{and} \quad
    \alpha_{n+1} = \alpha_{n+1}^\texttt{tr}.
\end{equation*}
In the case of $f_{n+1}^\texttt{tr} > 0$, the yield criterion is violated ($f_{n+1} > 0$) with $\overline{\mathbf{b}}_{n+1}^\texttt{e,tr}$, $\overline{\boldsymbol{\beta}}_{n+1}^\texttt{tr}$, and $\alpha_{n+1}^\texttt{tr}$. We need to make them ``return" along the $\mathbf{n}_{n+1}$ direction with a distance determined by $\widehat{\gamma}_{n+1} > 0$. To figure out $\widehat{\gamma}_{n+1}$, we enforce the yield criterion
\begin{equation} \label{Function of gamma}
    0 = f_{n+1} = \lVert \boldsymbol{\xi}_{n+1} \rVert - \sqrt{\dfrac{2}{3}} k (\alpha_{n+1})
    = \lVert \boldsymbol{\xi}_{n+1}^\texttt{tr} \rVert
    - 2 \overline{\overline{\mu}}_{n+1}^\texttt{tr}
    \left( 1 + \dfrac{h}{3\mu} \right) \widehat{\gamma}_{n+1}
    - \sqrt{\dfrac{2}{3}} k \left( \alpha_{n+1}^\texttt{tr} + \sqrt{\dfrac{2}{3}} \widehat{\gamma}_{n+1} \right) := \mathcal{G}(\widehat{\gamma}_{n+1})
\end{equation}
based on $\eqref{Updated beta and alpha}_2$ and $\eqref{Relationship between xi and xi_trial}_1$. The expression in \eqref{Function of gamma} is an algebraic equation of $\widehat{\gamma}_{n+1}$. For the linear isotropic hardening law in \eqref{Linear Isotropic Hardening Law}, we analytically solve \eqref{Function of gamma} for $\widehat{\gamma}_{n+1}$ as
\begin{equation} \label{Linear Solution of gamma}
    \widehat{\gamma}_{n+1} =
    \dfrac{1}{2 \overline{\overline{\mu}}_{n+1}^\texttt{tr}}
    \left( 1 + \dfrac{h}{3\mu} + \dfrac{K}{3 \overline{\overline{\mu}}_{n+1}^\texttt{tr}} \right)^{-1}
    \left[ \lVert \boldsymbol{\xi}_{n+1}^\texttt{tr} \rVert - \sqrt{\dfrac{2}{3}} (\sigma_y + K \alpha_{n+1}^\texttt{tr}) \right].
\end{equation}
For the nonlinear hardening law in \eqref{Nonlinear Isotropic Hardening Law}, we adopt the Newton's method to iteratively solve $\widehat{\gamma}_{n+1}$ as
\begin{equation} \label{Nonlinear Solution of gamma}
    \widehat{\gamma}_{n+1}^{(k+1)} = \widehat{\gamma}_{n+1}^{(k)}
    - \mathcal{G}(\widehat{\gamma}_{n+1}^{(k)}) / \mathcal{G}'(\widehat{\gamma}_{n+1}^{(k)})
\end{equation}
with
\begin{equation*}
    \mathcal{G}'(\widehat{\gamma}_{n+1}^{(k)})
    = - 2 \overline{\overline{\mu}}_{n+1}^\texttt{tr} \left( 1 + \dfrac{h}{3\mu} \right)
    - \dfrac{2}{3} k' \left( \alpha_{n+1}^\texttt{tr} + \sqrt{\dfrac{2}{3}} \widehat{\gamma}_{n+1}^{(k)} \right).
\end{equation*}

After computing the unit tensor ($\mathbf{n}_{n+1}$) with $\eqref{Relationship between xi and xi_trial}_2$ and the consistency parameter ($\widehat{\gamma}_{n+1}$) with \eqref{Linear Solution of gamma} or \eqref{Nonlinear Solution of gamma}, we send them back to \eqref{Updated beta and alpha} and \eqref{Updated be_bar} for correcting $\overline{\mathbf{b}}_{n+1}^\texttt{e}$, $\overline{\boldsymbol{\beta}}_{n+1}$, and $\alpha_{n+1}$. This entire prediction--correction procedure forms a closed loop and is typically referred to as the return mapping scheme for elastoplasticity. For the reader's convenience, we present the implementation of this scheme in Algorithm \ref{Algorithm of Return Mapping Scheme} and the relationships between the primary state variables at load steps $n$ and $n+1$ in Fig. \ref{Fig: Independent Variables}.

\begin{algorithm}[!htbp]
\caption{Return mapping scheme for finite strain elastoplasticity}
\label{Algorithm of Return Mapping Scheme}

\textbf{Inputs}:

Volume-preserving part of the elastic left Cauchy--Green deformation tensor, $\overline{\mathbf{b}}_n^\texttt{e}$;

``Volume-preserving" part of the back stress tensor, $\overline{\boldsymbol{\beta}}_n$;

Equivalent plastic strain, $\alpha_n$;

Deformation gradients at load steps $n$ and $n+1$, $\mathbf{F}_n$ and $\mathbf{F}_{n+1}$, respectively;

\textbf{Prediction stage:}

Compute the relative deformation gradient as $\mathbf{f}_{n+1} = \mathbf{F}_{n+1} \mathbf{F}_n^{-1}$;

Compute the volume-preserving part of $\mathbf{f}_{n+1}$ as $\overline{\mathbf{f}}_{n+1} = {\left( J_{n+1}^f \right)}^{-1/3} \mathbf{f}_{n+1}$;

Compute the trial version of $\overline{\mathbf{b}}_{n+1}^\texttt{e}$ as 
$\overline{\mathbf{b}}_{n+1}^\texttt{e,tr} = \overline{\mathbf{f}}_{n+1} \overline{\mathbf{b}}_n^\texttt{e} \overline{\mathbf{f}}_{n+1}^\top$;

Compute the trial version of $\overline{\boldsymbol{\beta}}_{n+1}$ as $\overline{\boldsymbol{\beta}}_{n+1}^\texttt{tr} = \overline{\mathbf{f}}_{n+1} \overline{\boldsymbol{\beta}}_n \overline{\mathbf{f}}_{n+1}^\top$;

Compute the trial version of $\alpha_{n+1}$ as $\alpha_{n+1}^\texttt{tr} = \alpha_n$;

Compute the trial deviatoric Kirchhoff stress tensor as $\mathbf{s}_{n+1}^\texttt{tr} = \mu\ \text{dev} \left( \overline{\mathbf{b}}_{n+1}^\texttt{e,tr} \right)$;

Compute the trial net stress tensor as $\boldsymbol{\xi}_{n+1}^\texttt{tr} = \mathbf{s}_{n+1}^\texttt{tr} - \text{dev} \left( \overline{\boldsymbol{\beta}}_{n+1}^\texttt{tr} \right)$;

Compute the unit tensor as $\mathbf{n}_{n+1} = \mathbf{n}_{n+1}^\texttt{tr} = \boldsymbol{\xi}_{n+1}^\texttt{tr} / \lVert \boldsymbol{\xi}_{n+1}^\texttt{tr} \rVert$;

Compute the parameters $\overline{\mu}_{n+1}^\texttt{tr} = \mu\ \text{tr}\left( \overline{\mathbf{b}}_{n+1}^\texttt{e,tr} \right) / 3$ and $\overline{\overline{\mu}}_{n+1}^\texttt{tr} = \overline{\mu}_{n+1}^\texttt{tr} - \text{tr} \left( \overline{\boldsymbol{\beta}}_{n+1}^\texttt{tr} \right) / 3$;

Compute the trial yield indicator as $f_{n+1}^\texttt{tr} = \lVert \boldsymbol{\xi}_{n+1}^\texttt{tr} \rVert - \sqrt{2/3} k \left( \alpha_{n+1}^\texttt{tr} \right)$;

\textbf{Correction stage:}

\eIf{$f_{n+1}^\texttt{tr} \leq 0$}{
    Set the consistency parameter as $\widehat{\gamma}_{n+1} = 0$;
}{
    \eIf{the isotropic hardening law ($k(\alpha)$) is linear}{
        Compute $\widehat{\gamma}_{n+1} > 0$ with \eqref{Linear Solution of gamma};
    }{
        Compute $\widehat{\gamma}_{n+1} > 0$ with \eqref{Nonlinear Solution of gamma};
    }
}

Compute $\mathcal{I}_1$ based on $\eqref{Updated be_bar}_3$;

Update $\overline{\mathbf{b}}_{n+1}^\texttt{e}$ as $\overline{\mathbf{b}}_{n+1}^\texttt{e}= \text{dev} \left( \overline{\mathbf{b}}_{n+1}^\texttt{e,tr} \right) - 2 \left( \overline{\overline{\mu}}_{n+1}^\texttt{tr}/\mu \right) \widehat{\gamma}_{n+1} \mathbf{n}_{n+1} + \mathcal{I}_1 \mathbf{I}/3$;

Update $\overline{\boldsymbol{\beta}}_{n+1}$ as $\overline{\boldsymbol{\beta}}_{n+1} = \overline{\boldsymbol{\beta}}_{n+1}^\texttt{tr} + 2h/3 \left( \overline{\overline{\mu}}_{n+1}^\texttt{tr}/\mu \right) \widehat{\gamma}_{n+1} \mathbf{n}_{n+1}$;

Update $\alpha_{n+1}$ as $\alpha_{n+1} = \alpha_{n+1}^\texttt{tr} + \sqrt{2/3} \widehat{\gamma}_{n+1}$;
\end{algorithm}

\begin{figure}[!htbp]
    \centering
    \includegraphics[width=18cm]{Independent_Variables-V2.pdf}
    \caption{Relationships between the primary state variables at load steps $n$ and $n+1$. The dashed arrows signify that $\widehat{\gamma}_{n+1}$ generally does not admit to an explicit expression and needs to be numerically solved from \eqref{Function of gamma} with \eqref{Nonlinear Solution of gamma}.}
    \label{Fig: Independent Variables}
\end{figure}

\subsection{Global governing equations for finite strain elastoplasticity}

We remark again that the return mapping scheme in Section \ref{Return mapping scheme} pertains to the elastoplastic response of a single material point ($\mathbf{X} \in \Omega_0$). To predict the elastoplastic response of the entire structure defined on $\Omega_0$, we derive the global governing equations as follows.

\subsubsection{Stress measures and algorithmic tangent moduli}

Substituting \eqref{Deviatoric Kirchhoff Stress} into \eqref{Kirchhoff Stress} reformulates the Kirchhoff stress tensor ($\boldsymbol{\tau}_{n+1}$) at load step $n+1$ as
\begin{equation*}
    \boldsymbol{\tau}_{n+1} = \boldsymbol{\tau}_{n+1}^\texttt{vol}
    + \mathbf{s}_{n+1}^\texttt{tr}
    - 2 \overline{\overline{\mu}}_{n+1}^\texttt{tr} \widehat{\gamma}_{n+1} \mathbf{n}_{n+1}.
\end{equation*}
Accordingly, we compute the first Piola--Kirchhoff stress tensor ($\mathbf{P}_{n+1}$) as
\begin{equation} \label{First PK Stress}
    \mathbf{P}_{n+1} = \boldsymbol{\tau}_{n+1} \mathbf{F}_{n+1}^{-\top} = \left( \boldsymbol{\tau}_{n+1}^\texttt{vol}
    + \mathbf{s}_{n+1}^\texttt{tr}
    - 2 \overline{\overline{\mu}}_{n+1}^\texttt{tr} \widehat{\gamma}_{n+1} \mathbf{n}_{n+1} \right) \mathbf{F}_{n+1}^{-\top}
\end{equation}
and the second Piola--Kirchhoff stress tensor ($\mathbf{S}_{n+1}$) as
\begin{equation} \label{Second PK Stress}
    \mathbf{S}_{n+1} = \varphi^*(\boldsymbol{\tau}_{n+1})
    = \varphi^*(\boldsymbol{\tau}_{n+1}^\texttt{vol})
    + \varphi^*(\mathbf{s}_{n+1}^\texttt{tr})
    - 2 \overline{\overline{\mu}}_{n+1}^\texttt{tr} \widehat{\gamma}_{n+1} \varphi^*(\mathbf{n}_{n+1}).
\end{equation}

Based on the second Piola--Kirchhoff stress tensor ($\mathbf{S}_{n+1}$), we compute the second \textit{algorithmic} tangent moduli as
\begin{equation} \label{Second Algorithmic Tangent Moduli Initial}
    \mathbb{C}_{n+1} = \dfrac{\partial \mathbf{S}_{n+1}}{\partial \mathbf{E}_{n+1}} = \left\{ \begin{array}{cc}
        \mathbb{C}_{n+1}^\texttt{el}, & \text{if $f_{n+1}^\texttt{tr} \leq 0$ ($\widehat{\gamma}_{n+1} = 0$; hyperelastic)}; \\[12pt]
        \mathbb{C}_{n+1}^\texttt{ep}, & \text{if $f_{n+1}^\texttt{tr} > 0$ ($\widehat{\gamma}_{n+1} > 0$; elastoplastic)}.
    \end{array} \right.
\end{equation}
Here, the variable $\mathbb{C}_{n+1}^\texttt{el}$ is the second elastic moduli given as
\begin{equation} \label{Second Elastic Moduli}
    \begin{array}{ll}
        \mathbb{C}_{n+1}^\texttt{el} = \dfrac{\partial \varphi^*(\widehat{\boldsymbol{\tau}}_{n+1})}{\partial \mathbf{E}_{n+1}}
        = &\left( 2 \overline{\mu}_{n+1} - 2 J_{n+1} U'_{n+1} \right) \mathbb{I}_{\mathbf{C}_{n+1}^{-1}}
        + \left[ J_{n+1} (J_{n+1} U'_{n+1})'- \dfrac{2 \overline{\mu}_{n+1}}{3} \right] \mathbf{C}_{n+1}^{-1} \otimes \mathbf{C}_{n+1}^{-1} \\[12pt]
        &- \dfrac{2}{3} \left[ \varphi^* \left( \widehat{\mathbf{s}}_{n+1} \right) \otimes \mathbf{C}_{n+1}^{-1}
        + \mathbf{C}_{n+1}^{-1} \otimes \varphi^* \left( \widehat{\mathbf{s}}_{n+1} \right) \right]
    \end{array}
\end{equation}
where we define
\begin{equation*}
    \left( \mathbb{I}_{\mathbf{C}_{n+1}^{-1}} \right)_{ijkl} = \dfrac{1}{2} \left[ \left( C_{n+1}^{-1} \right)_{ik} \left( C_{n+1}^{-1} \right)_{jl} + \left( C_{n+1}^{-1} \right)_{il} \left( C_{n+1}^{-1} \right)_{jk} \right].
\end{equation*}
The variable $\mathbb{C}_{n+1}^\texttt{ep}$ is the second elastoplastic moduli (see \ref{Sec: Second Elastoplastic Moduli} for derivation) expressed as
\begin{equation} \label{Second Elastoplastic Moduli Final}
    \begin{array}{ll}
        \mathbb{C}_{n+1}^\texttt{ep} = &\left( 2 \overline{\mu}_{n+1}^\texttt{tr} - 2 c_1 \overline{\overline{\mu}}_{n+1}^\texttt{tr} - 2 J_{n+1} U'_{n+1} \right) \mathbb{I}_{\mathbf{C}_{n+1}^{-1}}
        
        + \left[ J_{n+1} (J_{n+1} U'_{n+1})' - \dfrac{2 \overline{\mu}_{n+1}^\texttt{tr}}{3} + \dfrac{2 c_1 \overline{\overline{\mu}}_{n+1}^\texttt{tr}}{3} \right] \mathbf{C}_{n+1}^{-1} \otimes \mathbf{C}_{n+1}^{-1} \\[12pt]
        
        &- \dfrac{2}{3} \left[ \varphi^*(\mathbf{s}_{n+1}^\texttt{tr}) \otimes \mathbf{C}_{n+1}^{-1} + \mathbf{C}_{n+1}^{-1} \otimes \varphi^*(\mathbf{s}_{n+1}^\texttt{tr}) \right]

        + \dfrac{2 c_1}{3} \left[ \varphi^*(\boldsymbol{\xi}_{n+1}^\texttt{tr}) \otimes \mathbf{C}_{n+1}^{-1} + \mathbf{C}_{n+1}^{-1} \otimes \varphi^*(\boldsymbol{\xi}_{n+1}^\texttt{tr}) \right] \\[12pt]

        &- c_3 \varphi^*(\mathbf{n}_{n+1}) \otimes \varphi^*(\mathbf{n}_{n+1})

        - \dfrac{c_4}{2} \left\{ \varphi^*(\mathbf{n}_{n+1}) \otimes \varphi^*\left[ \text{dev} (\mathbf{n}_{n+1}^2) \right] + \varphi^*\left[ \text{dev} (\mathbf{n}_{n+1}^2) \right] \otimes \varphi^*(\mathbf{n}_{n+1}) \right\}
    \end{array}
\end{equation}
where we define
\begin{equation} \label{Parameters in Elasticity}
    \left\{ \begin{array}{l}
        c_0 =  1 + \dfrac{h}{3\mu} + \dfrac{k'}{3 \overline{\overline{\mu}}_{n+1}^\texttt{tr}}, \quad
        c_1 = \dfrac{2 \overline{\overline{\mu}}_{n+1}^\texttt{tr} \widehat{\gamma}_{n+1}}{\lVert \overline{\boldsymbol{\xi}}_{n+1}^\texttt{tr} \rVert}, \quad
        c_2 = \dfrac{1}{c_0} - c_1, \quad \\[12pt]
        
        c_3 = 2 c_2 \overline{\overline{\mu}}_{n+1}^\texttt{tr}
        - \left[ \dfrac{1}{c_0} \left( 1 + \dfrac{h}{3 \mu} \right) - 1 \right] \dfrac{4}{3} \widehat{\gamma}_{n+1} \lVert \overline{\boldsymbol{\xi}}_{n+1}^\texttt{tr} \rVert, \quad
        c_4 = 2 c_2 \lVert \overline{\boldsymbol{\xi}}_{n+1}^\texttt{tr} \rVert.
    \end{array} \right.
\end{equation}

By examining the second elastic moduli ($\mathbb{C}_{n+1}^\texttt{el}$) in \eqref{Second Elastic Moduli} and elastoplastic moduli ($\mathbb{C}_{n+1}^\texttt{ep}$) in \eqref{Second Elastoplastic Moduli Final}, we reformulate the second algorithmic tangent moduli ($\mathbb{C}_{n+1}$) in \eqref{Second Algorithmic Tangent Moduli Initial} into a compact form as
\begin{equation} \label{Second Algorithmic Tangent Moduli Final}
    \begin{array}{ll}
        \mathbb{C}_{n+1} = &\left( 2 \overline{\mu}_{n+1}^\texttt{tr} - 2 \theta c_1 \overline{\overline{\mu}}_{n+1}^\texttt{tr} - 2 J_{n+1} U'_{n+1} \right) \mathbb{I}_{\mathbf{C}_{n+1}^{-1}}
        
        + \left[ J_{n+1} (J_{n+1} U'_{n+1})' - \dfrac{2 \overline{\mu}_{n+1}^\texttt{tr}}{3} + \dfrac{2 \theta c_1 \overline{\overline{\mu}}_{n+1}^\texttt{tr}}{3} \right] \mathbf{C}_{n+1}^{-1} \otimes \mathbf{C}_{n+1}^{-1} \\[12pt]
        
        &- \dfrac{2}{3} \left[ \varphi^*(\mathbf{s}_{n+1}^\texttt{tr}) \otimes \mathbf{C}_{n+1}^{-1} + \mathbf{C}_{n+1}^{-1} \otimes \varphi^*(\mathbf{s}_{n+1}^\texttt{tr}) \right]

        + \dfrac{2 \theta c_1}{3} \left[ \varphi^*(\boldsymbol{\xi}_{n+1}^\texttt{tr}) \otimes \mathbf{C}_{n+1}^{-1} + \mathbf{C}_{n+1}^{-1} \otimes \varphi^*(\boldsymbol{\xi}_{n+1}^\texttt{tr}) \right] \\[12pt]

        &- \theta c_3 \varphi^*(\mathbf{n}_{n+1}) \otimes \varphi^*(\mathbf{n}_{n+1})

        - \dfrac{\theta c_4}{2} \left\{ \varphi^*(\mathbf{n}_{n+1}) \otimes \varphi^*\left[ \text{dev} (\mathbf{n}_{n+1}^2) \right] + \varphi^*\left[ \text{dev} (\mathbf{n}_{n+1}^2) \right] \otimes \varphi^*(\mathbf{n}_{n+1}) \right\}
    \end{array}
\end{equation}
with
\begin{equation} \label{Theta Parameter}
    \theta = \dfrac{1}{2} \dfrac{f_{n+1}^\texttt{tr} + |f_{n+1}^\texttt{tr}|}{f_{n+1}^\texttt{tr}} = \left\{ \begin{array}{ll}
        0, & \text{if $f_{n+1}^\texttt{tr} \leq 0$ (hyperelastic)}; \\[8pt]
        1, & \text{if $f_{n+1}^\texttt{tr} > 0$ (elastoplastic)}.
    \end{array} \right.
\end{equation}

\subsubsection{Governing equations}

Based on the stress measures and tangent moduli defined above, we present the global governing equations in an incremental form as
\begin{equation} \label{Global Governing Equation}
    \left\{ \begin{array}{l}
        \displaystyle \int_{\Omega_0} \nabla \mathbf{v} : \widehat{\mathbb{C}}_{n+1}^{(k)} : \nabla \delta \mathbf{u}_{n+1}^{(k)}\ \text{d} \mathbf{X}
        = \int_{\Omega_0} \overline{\mathbf{q}}_{n+1} \cdot \mathbf{v}\ \text{d} \mathbf{X} 
        + \int_{\partial \Omega_0^\mathcal{N}} \overline{\mathbf{t}}_{n+1} \cdot \mathbf{v}\ \text{d} \mathbf{X}
        - \int_{\Omega_0} \mathbf{P}_{n+1}^{(k)} : \nabla \mathbf{v}\ \text{d} \mathbf{X}, \\[12pt]

        \delta \mathbf{u}_{n+1}^{(k)} = \overline{\mathbf{u}}_{n+1} - \mathbf{u}_{n+1}^{(k)} \quad \text{for} \quad \mathbf{X} \in \partial \Omega_0^\mathcal{D}, \\[12pt]
        
        \mathbf{u}_{n+1}^{(k+1)} = \mathbf{u}_{n+1}^{(k)} + l_{n+1}^{(k)} \delta \mathbf{u}_{n+1}^{(k)} \quad \text{for} \quad \mathbf{X} \in \overline{\Omega}_0.
    \end{array} \right.
\end{equation}
Here, the variable $\delta \mathbf{u}_{n+1}$ is the incremental displacement field, and $\mathbf{v}$ is the test displacement field. The variables $\overline{\mathbf{q}}_{n+1}$, $\overline{\mathbf{t}}_{n+1}$, and $\overline{\mathbf{u}}_{n+1}$ are the prescribed body force, traction, and displacement, respectively. The symbol $l_{n+1} \in (0, 1]$ represents a line search parameter such that the updated displacement field reduces the absolute residual (right-hand side of $\eqref{Global Governing Equation}_1$). The superscript $(k)$ of a variable represents the value evaluated at Newton iteration $k$. The symbol $\overline{\Omega}_0$ is the closed bounded domain in its undeformed configuration with Neumann ($\partial \Omega_0^\mathcal{N}$) and Dirichlet ($\partial \Omega_0^\mathcal{D}$) boundaries. The variable $\widehat{\mathbb{C}}_{n+1}$ is the first algorithmic tangent moduli, which can be computed from $\mathbf{S}_{n+1}$ in \eqref{Second PK Stress} and $\mathbb{C}_{n+1}$ in \eqref{Second Algorithmic Tangent Moduli Final} and expressed in indicial notation as
\begin{equation*}
    ( \widehat{\mathbb{C}}_{n+1} )_{rjsl} = \left( F_{n+1} \right)_{ri} \left( F_{n+1} \right)_{sk} \left( \mathbb{C}_{n+1} \right)_{ijkl} + \delta_{rs} \left( S_{n+1} \right)_{jl}.
\end{equation*}

After some tensor algebra, we write out the integrand of the left-hand side in $\eqref{Global Governing Equation}_1$ as
\begin{equation} \label{Integrand of Left-hand Side}
    \begin{array}{ll}
        \nabla \mathbf{v} : \widehat{\mathbb{C}} : \nabla \delta \mathbf{u} = &\left( 2 \overline{\mu}^{\texttt{tr}} - 2 \theta c_1 \overline{\overline{\mu}}^{\texttt{tr}} - 2 J U' \right) \left( \mathcal{U} \right)^\texttt{sym} : \mathcal{V}
        
        + \left[ J \left( J U' \right)' - \dfrac{2 \overline{\mu}^{\texttt{tr}}}{3} + \dfrac{2 \theta c_1 \overline{\overline{\mu}}^{\texttt{tr}}}{3} \right] \text{tr} \left( \mathcal{U} \right) \text{tr} \left( \mathcal{V} \right) \\[12pt]
        
        &- \dfrac{2}{3} \left[ \text{tr} \left( \mathcal{U} \right) \mathbf{s}^{\texttt{tr}} : \mathcal{V} + \text{tr} \left( \mathcal{V} \right) \mathbf{s}^{\texttt{tr}} : \mathcal{U} \right]

        + \dfrac{2 \theta c_1}{3} \left[ \text{tr} \left( \mathcal{U} \right) \boldsymbol{\xi}^{\texttt{tr}} : \mathcal{V} + \text{tr} \left( \mathcal{V} \right) \boldsymbol{\xi}^{\texttt{tr}} : \mathcal{U} \right] \\[12pt]

        &- \theta c_3 \left( \mathbf{n} : \mathcal{V} \right) \left( \mathbf{n} : \mathcal{U} \right)

        - \dfrac{\theta c_4}{2} \left\{ \left( \mathbf{n} : \mathcal{V} \right) \left[ \text{dev} \left( \mathbf{n}^2 \right) : \mathcal{U} \right] + \left[ \text{dev} \left( \mathbf{n}^2 \right) : \mathcal{V} \right] \left( \mathbf{n} : \mathcal{U} \right) \right\}

        + \left( \nabla \delta \mathbf{u}\ \mathbf{S} \right) : \nabla \mathbf{v}
    \end{array}
\end{equation}
by defining
\begin{equation*}
    \mathcal{U} = \nabla \delta \mathbf{u}\ \mathbf{F}^{-1}
    \quad \text{and} \quad
    \mathcal{V} = \nabla \mathbf{v}\ \mathbf{F}^{-1}
\end{equation*}
where we temporarily drop the subscript $n+1$ and superscript $(k)$ for conciseness. We remark that \eqref{Global Governing Equation} are explicit, linear equations of $\delta \mathbf{u}_{n+1}^{(k)}$, where all coefficients can be predetermined by $\overline{\mathbf{b}}_n^\texttt{e}$, $\overline{\boldsymbol{\beta}}_n$, $\alpha_n$, $\mathbf{u}_n$, and $\mathbf{u}_{n+1}^{(k)}$. Consequently, we can iteratively solve $\delta \mathbf{u}_{n+1}^{(k)}$ from \eqref{Global Governing Equation} and update $\mathbf{u}_{n+1}^{(k+1)}$ until convergence.

For the reader's convenience, we outline the solution scheme in Algorithm \ref{Algorithm of Overall Solution Scheme}. To enhance the convergence of the highly non-convex elastoplasticity problem, we incorporate an adaptive line search scheme (Algorithm \ref{Algorithm of Line Search}), which iteratively reduces the line search parameter ($l_{n+1}^{(k)}$) until the residual starts to decrease. Furthermore, we employ an adaptive loading scheme (Algorithm \ref{Algorithm of Adaptive Loading}) that dynamically adjusts the applied boundary values ($\overline{\mathbf{q}}$, $\overline{\mathbf{t}}$, and $\overline{\mathbf{u}}$) in cases where Newton iterations diverge occasionally. With the combination of Algorithms \ref{Algorithm of Return Mapping Scheme}–\ref{Algorithm of Adaptive Loading}, we successfully obtain converged FEA solutions for all examples introduced in Section \ref{Sec: Sample Examples}. The convergence, precision, and computational time of FEA for a sample design case are demonstrated in \ref{Sec: FEA Convergence}.

\begin{algorithm}[!htbp]
\caption{Overall solution scheme for finite strain elastoplasticity}
\label{Algorithm of Overall Solution Scheme}

\textbf{Inputs}: Initial state variables, $\overline{\mathbf{b}}_0^\texttt{e} \leftarrow \mathbf{I}$, $\overline{\boldsymbol{\beta}}_0 \leftarrow \mathbf{0}$, $\alpha_0 \leftarrow 0$, and $\mathbf{u}_0 \leftarrow \mathbf{0}$;
Prescribed body force ($\overline{\mathbf{q}}$), traction ($\overline{\mathbf{t}}$), and displacement ($\overline{\mathbf{u}}$);
Prescribed time steps, $\{t_1, t_2, \ldots, t_{M-1}, t_M\}$;
Maximum Newton iterations, $N^\texttt{iter}$;
Absolute and relative tolerances of Newton iterations, $\varepsilon^\texttt{abs}$ and $\varepsilon^\texttt{rel}$, respectively;  

Initialize the converged time step, $t^\texttt{conv} \leftarrow 0$;

Initialize the time step count, $m \leftarrow 1$, load step count, $n \leftarrow 0$, and try count, $r \leftarrow 0$;

\While{$m \leq M$}{
    Initialize the current time step, $t^\texttt{now} \leftarrow t_m$;

    \While{$t^\texttt{conv} < t_m$}{
        Update the boundary values, $\overline{\mathbf{q}}_{n+1} \leftarrow  t^\texttt{now}\ \overline{\mathbf{q}}$, $\overline{\mathbf{t}}_{n+1} \leftarrow  t^\texttt{now}\ \overline{\mathbf{t}}$, and $\overline{\mathbf{u}}_{n+1} \leftarrow t^\texttt{now}\ \overline{\mathbf{u}}$;

        Compute the previous deformation gradient, $\mathbf{F}_n \leftarrow \mathbf{I} + \nabla \mathbf{u}_n$;
    
        Initialize the displacement field at load step $n+1$ and Newton iteration 0, $\mathbf{u}_{n+1}^{(0)} \leftarrow \mathbf{u}_n$;
            
        Compute the absolute residual (right-hand side of $\eqref{Global Governing Equation}_1$), $a_{n+1}^{(0)}$;

        Initialize the convergence flag, $\texttt{convergence} \leftarrow \texttt{false}$;

        Initialize the Newton iteration count, $k \leftarrow 0$;
        
        \While{$k < N^\texttt{iter}$}{ 
            Compute the current deformation gradient, $\mathbf{F}_{n+1}^{(k)} \leftarrow \mathbf{I} + \nabla \mathbf{u}_{n+1}^{(k)}$;
    
            Compute $\alpha_{n+1}^{\texttt{tr},(k)}$, $\mathbf{s}_{n+1}^{\texttt{tr},(k)}$, $\boldsymbol{\xi}_{n+1}^{\texttt{tr},(k)}$, $\mathbf{n}_{n+1}^{\texttt{tr},(k)}$, $\overline{\mu}_{n+1}^{\texttt{tr},(k)}$, $\overline{\overline{\mu}}_{n+1}^{\texttt{tr},(k)}$, $f_{n+1}^{\texttt{tr},(k)}$, and $\widehat{\gamma}_{n+1}^{(k)}$ by providing Algorithm \ref{Algorithm of Return Mapping Scheme} with $\overline{\mathbf{b}}_n^\texttt{e}$, $\overline{\boldsymbol{\beta}}_n$,  $\alpha_n$, $\mathbf{F}_n$, and $\mathbf{F}_{n+1}^{(k)}$;
            
            Compute $\mathbf{P}_{n+1}^{(k)}$ in \eqref{First PK Stress} and $\mathbf{S}_{n+1}^{(k)}$ in \eqref{Second PK Stress} based on $\mathbf{F}_{n+1}^{(k)}$, $\mathbf{s}_{n+1}^{\texttt{tr},(k)}$, $\mathbf{n}_{n+1}^{\texttt{tr},(k)}$, $\overline{\overline{\mu}}_{n+1}^{\texttt{tr},(k)}$, and $\widehat{\gamma}_{n+1}^{(k)}$;
            
            Compute $c_1$, $c_3$, and $c_4$ in \eqref{Parameters in Elasticity} based on $\alpha_{n+1}^{\texttt{tr},(k)}$, $\boldsymbol{\xi}_{n+1}^{\texttt{tr},(k)}$, $\overline{\overline{\mu}}_{n+1}^{\texttt{tr},(k)}$, and $\widehat{\gamma}_{n+1}^{(k)}$;
    
            Compute $\theta$ in \eqref{Theta Parameter} based on $f_{n+1}^{(k)}$;
    
            Compute $\nabla \mathbf{v} : \widehat{\mathbb{C}}_{n+1}^{(k)} : \nabla \delta \mathbf{u}_{n+1}^{(k)}$ in \eqref{Integrand of Left-hand Side} (as a function of $\delta \mathbf{u}_{n+1}^{(k)}$ and $\mathbf{v}$) based on $\mathbf{F}_{n+1}^{(k)}$, $\mathbf{s}_{n+1}^{\texttt{tr},(k)}$,  
            $\boldsymbol{\xi}_{n+1}^{\texttt{tr},(k)}$,
            $\mathbf{n}_{n+1}^{\texttt{tr},(k)}$,
            $\overline{\mu}_{n+1}^{\texttt{tr},(k)}$, $\overline{\overline{\mu}}_{n+1}^{\texttt{tr},(k)}$, $c_1$, $c_3$, $c_4$, and $\theta$;
    
            Solve $\delta \mathbf{u}_{n+1}^{(k)}$ from $\eqref{Global Governing Equation}_{1,2}$ based on $\nabla \mathbf{v} : \widehat{\mathbb{C}}_{n+1}^{(k)} : \nabla \delta \mathbf{u}_{n+1}^{(k)}$, $\mathbf{P}_{n+1}^{(k)}$, $\overline{\mathbf{q}}_{n+1}$, $\overline{\mathbf{t}}_{n+1}$, and $\overline{\mathbf{u}}_{n+1}$;
            
            Compute $\mathbf{u}_{n+1}^{(k+1)}$ and $a_{n+1}^{(k+1)}$ by providing $\mathbf{u}_{n+1}^{(k)}$, $a_{n+1}^{(k)}$, and $\delta \mathbf{u}_{n+1}^{(k)}$ to the line search scheme in Algorithm \ref{Algorithm of Line Search};
    
            Compute the relative residual, $r_{n+1}^{(k+1)} \leftarrow a_{n+1}^{(k+1)} / a_{n+1}^{(1)}$;

            Update the Newton iteration count, $k \leftarrow k+1$;

            \If{$a_{n+1}^{(k)} \leq \varepsilon^\texttt{abs}$ or $r_{n+1}^{(k)} \leq \varepsilon^\texttt{rel}$}{
                Update the convergence flag, $\texttt{convergence} \leftarrow \texttt{true}$;

                Break the Newton iterations due to convergence;
            }
        }

        Update the current state variables, $\overline{\mathbf{b}}_{n+1}^\texttt{e}$, $\overline{\boldsymbol{\beta}}_{n+1}$, $\alpha_{n+1}$, $\widehat{\gamma}_{n+1}$, and $\mathbf{u}_{n+1}$, by providing $\overline{\mathbf{b}}_n^\texttt{e}$, $\overline{\boldsymbol{\beta}}_n$, $\alpha_n$, $\mathbf{F}_n$, $\mathbf{u}_{n+1}^{(k)}$, $t^\texttt{conv}$, $t^\texttt{now}$, $t_m$, $n$, and $r$ to the adaptive loading scheme in Algorithm \ref{Algorithm of Adaptive Loading};
    }

    Update the time step count, $m \leftarrow m+1$;
}

\end{algorithm}

\begin{algorithm}[!htbp]
\caption{Adaptive line search scheme}
\label{Algorithm of Line Search}

\textbf{Inputs}: Displacement and absolute residual from previous Newton iteration, $\mathbf{u}_{n+1}^{(k)}$ and $a_{n+1}^{(k)}$, respectively;
Incremental displacement, $\delta \mathbf{u}_{n+1}^{(k)}$;
Maximum iterations and tolerance for line search, $N^\texttt{search}$ and $\varepsilon^\texttt{search} \in (0, 1]$, respectively;

Initialize the line search parameter, $l_{n+1}^{(k)} \leftarrow 1$;

Initialize the iteration count for linear search, $s \leftarrow 1$;

\For{$s \leq N^\texttt{search}$}{
    Update $\mathbf{u}_{n+1}^{(k+1)}$ in $\eqref{Global Governing Equation}_3$ based on $\mathbf{u}_{n+1}^{(k)}$, $\delta \mathbf{u}_{n+1}^{(k)}$, and $l_{n+1}^{(k)}$;

    Compute the absolute residual, $a_{n+1}^{(k+1)}$;

    \eIf{$a_{n+1}^{(k+1)} < \varepsilon^\texttt{search}\ a_{n+1}^{(k)}$}{
        Break the line search due to convergence;
    }{
        Shrink the line search parameter, $l_{n+1}^{(k)} \leftarrow l_{n+1}^{(k)}/2$;

        Update the line search count, $s \leftarrow s+1$;
    }
}
\end{algorithm}

\begin{algorithm}[!htbp]
\caption{Adaptive loading scheme}
\label{Algorithm of Adaptive Loading}

\textbf{Inputs}: Convergence flag, \texttt{convergence};
Internal variables from the previous load step, $\overline{\mathbf{b}}_n^\texttt{e}$, $\overline{\boldsymbol{\beta}}_n$, and $\alpha_n$;
Deformation gradient from the previous load step, $\mathbf{F}_n$;
Displacement in the current Newton iteration, $\mathbf{u}_{n+1}^{(k)}$;
Maximum tries of the adaptive loading, $N^\texttt{try}$;
Converged, current, and target time steps, $t^\texttt{conv}$, $t^\texttt{now}$, and $t_m$, respectively;
Load step count, $n$;
Try count, $r$;

\eIf{\texttt{convergence} is \texttt{false} and $r \leq N^\texttt{try}$}{
    Shrink the current time step, $t^\texttt{now} \leftarrow (t^\texttt{conv} + t^\texttt{now}) / 2$;

    Update the try count, $r \leftarrow r + 1$; 
}{
    Reset the try count, $r \leftarrow 0$; 

    Update the converged time step, $t^\texttt{conv} \leftarrow t^\texttt{now}$;

    Update the current time step, $t^\texttt{now} \leftarrow t_m$;

    Update the current incremental consistency parameter, $\widehat{\gamma}_{n+1} \leftarrow \widehat{\gamma}_{n+1}^{(k)}$, and displacement field, $\mathbf{u}_{n+1} \leftarrow \mathbf{u}_{n+1}^{(k)}$;

    Update the current internal variables, $\overline{\mathbf{b}}_{n+1}^\texttt{e}$, $\overline{\boldsymbol{\beta}}_{n+1}$, and $\alpha_{n+1}$, by providing Algorithm \ref{Algorithm of Return Mapping Scheme} with $\overline{\mathbf{b}}_n^\texttt{e}$, $\overline{\boldsymbol{\beta}}_n$,  $\alpha_n$, $\mathbf{F}_n$, and $\mathbf{F}_{n+1} \leftarrow \mathbf{I} + \nabla \mathbf{u}_{n+1}$;

    Update the load step count, $n \leftarrow n+1$;
}
\end{algorithm}


\section{Topology optimization framework for finite strain elastoplasticity}
\label{Sec: Topology Optimization Framework}

Built upon the finite strain elastoplasticity theory in Section \ref{Sec: Finite Strain Elastoplasticity}, we now present the topology optimization framework for optimizing elastoplastic structures undergoing large deformations. This framework comprises design space parameterization, material property interpolation, topology optimization formulation, and sensitivity analysis and verification. This section details the first three components while the sensitivity analysis and verification are provided in \ref{Sec: Sensitivity Analysis and Verification}.

\subsection{Design space parameterization}

Before proceeding with topology optimization, one prerequisite is to identify suitable design variables that parameterize the large design space of structural geometries and material phases. Following the blueprint of topology optimization for infinitesimal strain elastoplasticity \citep{jia_multimaterial_2025}, we start by defining one density variable, $\rho(\mathbf{X}): \Omega_0 \mapsto [0, 1]$, and $N^\xi \geq 1$ material variables, $\xi_1(\mathbf{X}),\ \xi_2(\mathbf{X}),\ \ldots,\ \xi_{N^\xi}(\mathbf{X}): \Omega_0 \mapsto [0, 1]$. To reduce the mesh dependence and avoid the checkerboard patterns of designs, we apply a linear filter \citep{bourdin_filters_2001} on all design variables and derive their filtered versions as
\begin{equation*}
    \widetilde{\zeta}(\mathbf{X}) = \dfrac{\displaystyle \int_{\Omega_0} w_\zeta(\mathbf{X}, \mathbf{X}') \zeta(\mathbf{X}') \text{d} \mathbf{X}'}{\displaystyle \int_{\Omega_0} w_\zeta(\mathbf{X}, \mathbf{X}') \text{d} \mathbf{X}'}
    \quad \text{for} \quad
    \zeta \in \{\rho, \xi_1, \xi_2, \ldots, \xi_{N^\xi}\}.
\end{equation*}
Here, $w_\zeta(\mathbf{X}, \mathbf{X}') = \max \{ 0, R_\zeta-\lVert \mathbf{X}-\mathbf{X}' \rVert_2 \}$ is a weighting factor, and $R_\zeta \geq 0$ is the filter radius. 

To promote pure solid--void designs with discrete material interfaces, we further apply a Heaviside projection \citep{bendsoe_topology_2003} on the filtered design variables, $\widetilde{\rho}(\mathbf{X}),\ \widetilde{\xi}_1(\mathbf{X}),\ \widetilde{\xi}_2(\mathbf{X}),\ \ldots,\ \widetilde{\xi}_{N^\xi}(\mathbf{X}): \Omega_0 \mapsto [0, 1]$, and derive their projected versions as
\begin{equation} \label{Heaviside Projection}
    \widehat{\zeta}(\mathbf{X}) = \dfrac{\tanh(\beta_\zeta \theta_\zeta) + \tanh(\beta_\zeta (\widetilde{\zeta}(\mathbf{X}) - \theta_\zeta))}{\tanh(\beta_\zeta \theta_\zeta) + \tanh(\beta_\zeta (1 - \theta_\zeta))}
    \quad \text{for} \quad
     \zeta \in \{\rho, \xi_1, \xi_2, \ldots, \xi_{N^\xi}\},
\end{equation}
where $\beta_\zeta$ is a sharpness parameter, and $\theta_\zeta$ is the projection threshold. Based on the projected density variable, $\widehat{\rho}(\mathbf{X}): \Omega_0 \mapsto [0, 1]$, we define the physical density variable as $\overline{\rho}(\mathbf{X}) \equiv \widehat{\rho}(\mathbf{X}): \Omega_0 \mapsto [0, 1]$. Here, $\overline{\rho}(\mathbf{X}) = 1$ represents that material point $\mathbf{X}$ is occupied by the solid material, and $\overline{\rho}(\mathbf{X}) = 0$ signifies the void.

As for the projected material variables, $\widehat{\xi}_1(\mathbf{X}),\ \widehat{\xi}_2(\mathbf{X}),\ \ldots,\ \widehat{\xi}_{N^\xi}(\mathbf{X}): \Omega_0 \mapsto [0, 1]$, we further apply a modified version of the hypercube-to-simplex-projection (HSP) \citep{zhou_multi-component_2018} as
\begin{equation*}
    \left\{ \begin{array}{l}
        \overline{\xi}_n(\mathbf{X}) = \displaystyle{ \sum_{k=1}^{2^{N^\xi}} b_{nk}
        \left\{ (-1)^{N^\xi + \sum_{l=1}^{N^\xi}c_{kl}}
        \prod_{m=1}^{N^\xi} (\widehat{\xi}_m(\mathbf{X}) + c_{km} - 1) \right\} }
        \quad \text{for} \quad n=1,2,\ldots,N^\xi, \\[15pt]
        \overline{\xi}_{N^\texttt{mat}}(\mathbf{X}) = \displaystyle{ 1 - \sum_{n=1}^{N^\xi} } \overline{\xi}_n(\mathbf{X}),
    \end{array} \right.
\end{equation*}
to ensure that the summation of the physical material variables, $\overline{\xi}_1(\mathbf{X}),\ \overline{\xi}_2(\mathbf{X}),\ \ldots,\ \overline{\xi}_{N^\texttt{mat}}(\mathbf{X}): \Omega_0 \mapsto [0, 1]$, is equal to one ($\sum_{n=1}^{N^\texttt{mat}}\overline{\xi}_n(\mathbf{X}) = 1$). Here, the parameter $N^\texttt{mat} = N^\xi + 1$ represents the number of candidate materials. The parameter $c_{kl} \in \{0, 1\}$ is the $l$th coordinate component of the $k$th vertex of a unit hypercube in the $N^\texttt{mat}-1$ dimensional space. The parameter $b_{nk}$ is computed as
\begin{equation*}
    b_{nk} = \left\{ \begin{array}{cc}
        \dfrac{c_{nk}}{\sum_{l} c_{nl}}, & \text{if $\sum_{l} c_{nl} \geq 1$}; \\[8pt]
        0, & \text{otherwise}.
    \end{array} \right.
\end{equation*}

Note that for a material point with $\overline{\rho}(\mathbf{X}) = 1$, the expression $\overline{\xi}_n(\mathbf{X}) = 1$ for $n=1,2,\ldots,N^\texttt{mat}$ signifies that material $n$ is used. Consequently, we can use the physical density ($\overline{\rho}(\mathbf{X})$) and material ($\overline{\xi}_1(\mathbf{X}),\ \overline{\xi}_2(\mathbf{X}),\ \ldots,\ \overline{\xi}_{N^\texttt{mat}}(\mathbf{X})$) variables to describe the structural geometries and material phases in the design domain ($\Omega_0$), respectively.

\subsection{Material property interpolation}

After design space parameterization, one immediate next step is to interpolate the elasticity- and plasticity-related material constants/functions for various values of physical design variables, $\overline{\rho}(\mathbf{X}) \in [0, 1]$ and $\overline{\xi}_n(\mathbf{X}) \in [0, 1]$ for $n=1,2,\ldots,N^\texttt{mat}$. Following \citet{jia_multimaterial_2025}, we interpolate the initial bulk modulus ($\kappa$), initial shear modulus ($\mu$), isotropic hardening function ($k$), and kinematic hardening modulus ($h$) as
\begin{equation} \label{Material Interpolation Functions}
    \overline{\chi}(\mathbf{X}) = \left[\varepsilon_\rho + (1-\varepsilon_\rho) \overline{\rho}^{p_\chi}(\mathbf{X}) \right] 
    \sum_{n=1}^{N^\texttt{mat}} \overline{\xi}_n^{p_\xi}(\mathbf{X}) \chi_n
    \quad \text{for} \quad
    \chi \in \{ \kappa, \mu, k, h \}.
\end{equation}
Here, the variables $\overline{\kappa}(\mathbf{X})$, $\overline{\mu}(\mathbf{X})$, $\overline{k}(\mathbf{X})$, and $\overline{h}(\mathbf{X})$ are the interpolated material constants/functions. The material constant $\chi_n \in \{\kappa_n, \mu_n, k_n, h_n\}$ for $n=1,2,\ldots,N^\texttt{mat}$ represents the property/function of candidate material $n$. The parameters $p_\kappa = p_\mu$, $p_k$, and $p_h$ penalize the intermediate values of the physical density variable, $\overline{\rho}(\mathbf{X}) \in (0, 1)$, and $p_\xi$ penalizes the intermediate values of the physical material variables, $\overline{\xi}_n(\mathbf{X}) \in (0, 1)$ for $n=1,2,\ldots,N^\texttt{mat}$. The parameter $\varepsilon_\rho$ is a small positive number to prevent the singularity of the stiffness of void materials.

In addition to the interpolation rules in \eqref{Material Interpolation Functions} for material constants/functions, it is essential to interpolate the constitutive laws across the range of the physical density variable, $\overline{\rho} \in [0, 1]$. This consideration arises because many topology optimization problems involving finite strain analysis \citep{li_programming_2023} encounter divergence in FEA solutions. Such issues are typically caused by the excessive distortion of elements filled with void materials when subjected to large deformations. To address this challenge, a widely adopted and effective strategy for hyperelastic designs \citep{wang_interpolation_2014} assumes that void materials exhibit linear elastic behavior while solid materials retain their hyperelastic characteristics.

In the context of finite strain elastoplasticity, we propose to interpolate the first Piola--Kirchhoff stress tensor as
\begin{equation} \label{Interpolated First PK Stress}
    \widecheck{\mathbf{P}}(\mathbf{F}, \widecheck{\mathbf{F}}; \phi)
    = \phi \mathbf{P}(\widecheck{\mathbf{F}}) - \phi \boldsymbol{\sigma}^\texttt{l}(\boldsymbol{\epsilon}(\widecheck{\mathbf{F}})) + \boldsymbol{\sigma}^\texttt{l}(\boldsymbol{\epsilon}(\mathbf{F})).
\end{equation}
The variable $\widecheck{\mathbf{F}}$ is the interpolated total deformation gradient defined as
\begin{equation*}
    \widecheck{\mathbf{F}} = \mathbf{F} (\phi \mathbf{u}) = \mathbf{I} + \phi \nabla \mathbf{u} = (1-\phi) \mathbf{I} + \phi \mathbf{F}(\mathbf{u})
\end{equation*}
with
\begin{equation*}
    \phi(\mathbf{X}) = \dfrac{\tanh(\beta_\phi \theta_\phi) + \tanh(\beta_\phi (\overline{\rho}^{p_\kappa} (\mathbf{X}) - \theta_\phi))}{\tanh(\beta_\phi \theta_\phi) + \tanh(\beta_\phi (1 - \theta_\phi))}
\end{equation*}
where we assume $\phi(\mathbf{X})$ is element-wise constant and therefore $\nabla \phi(\mathbf{X}) = 0$ at the integration points. The parameters $\beta_\phi = 500$ and $\theta_\phi = 0.1$ are similar to $\beta_\zeta$ and $\theta_\zeta$ in \eqref{Heaviside Projection}, respectively. The variable $\boldsymbol{\sigma}^\texttt{l}$ is the infinitesimal stress tensor (associated with $W$ in \eqref{Strain Energy Density Function}) expressed as
\begin{equation*}
    \boldsymbol{\sigma}^\texttt{l}(\boldsymbol{\epsilon}) = \overline{\kappa}\ \text{tr}(\boldsymbol{\epsilon}) \mathbf{I} + 2 \overline{\mu}\ \text{dev}(\boldsymbol{\epsilon}),
\end{equation*}
and $\boldsymbol{\epsilon}$ is the infinitesimal strain tensor defined as
\begin{equation*}
    \boldsymbol{\epsilon}(\mathbf{F}) = \dfrac{1}{2} \left( \mathbf{F} + \mathbf{F}^\top \right) - \mathbf{I}.
\end{equation*}

\begin{remark}
The physical meaning of $\widecheck{\mathbf{P}}$ in \eqref{Interpolated First PK Stress} merits explicit clarification. For solid materials where $\overline{\rho}(\mathbf{X}) = 1$ and therefore $\phi(\mathbf{X}) = 1$, the variable $\widecheck{\mathbf{P}}$ reduces to $\mathbf{P}$ in \eqref{First PK Stress} for elastoplastic materials exactly. For void materials where $\overline{\rho}(\mathbf{X}) = 0$ and therefore $\phi(\mathbf{X}) = 0$, the variable $\widecheck{\mathbf{P}}$ reduces to $\boldsymbol{\sigma}^\texttt{l}$ for linear elastic materials exactly without any large deformations and yield behaviors (as in the air). This linear elasticity assumption for void materials brings negligible errors (as verified in \ref{Sec: Comparison of Meshes}) due to the tiny stiffness of void materials compared to solid ones.
\end{remark}

Corresponding to the interpolated first Piola--Kirchhoff stress ($\widecheck{\mathbf{P}}$) in \eqref{Interpolated First PK Stress}, we compute the interpolated first elastoplastic moduli as
\begin{equation*}
    \widecheck{\mathbb{C}} = \dfrac{\partial \widecheck{\mathbf{P}}}{\partial \mathbf{F}}
    = \phi^2 \widehat{\mathbb{C}}(\widecheck{\mathbf{F}}) + (1-\phi^2) \mathbb{C}^\texttt{l},
\end{equation*}
where $\mathbb{C}^\texttt{l}$ represents the tangent moduli of the associated linear elastic material expressed as
\begin{equation*}
    \mathbb{C}^\texttt{l} \equiv \overline{\kappa} \mathbf{I} \otimes \mathbf{I} + 2 \overline{\mu} \left( \mathbb{I} - \dfrac{1}{3} \mathbf{I} \otimes \mathbf{I} \right).
\end{equation*}
Consequently, the integrand of the left-hand side of the global incremental equilibrium equation in $\eqref{Global Governing Equation}_1$ becomes as
\begin{equation} \label{Interpolated Left-hand Side}
    \nabla \mathbf{v} : \widecheck{\mathbb{C}} : \nabla \delta \mathbf{u} = \phi^2 \nabla \mathbf{v} : \widehat{\mathbb{C}}(\widecheck{\mathbf{F}}) : \nabla \delta \mathbf{u} + (1-\phi^2) \nabla \mathbf{v} : \mathbb{C}^\texttt{l} : \nabla \delta \mathbf{u}.
\end{equation}
Note that one can effortlessly compute $\nabla \mathbf{v} : \widehat{\mathbb{C}}(\widecheck{\mathbf{F}}) : \nabla \delta \mathbf{u}$ by replacing $\mathbf{F}$ with $\widecheck{\mathbf{F}}$ in \eqref{Integrand of Left-hand Side}. As for the second term in \eqref{Interpolated Left-hand Side}, we write out
\begin{equation*}
    \nabla \mathbf{v} : \mathbb{C}^\texttt{l} : \nabla \delta \mathbf{u}
    = \left( \overline{\kappa} - \dfrac{2\overline{\mu}}{3} \right) \text{tr} (\nabla \delta \mathbf{u}) \text{tr} (\nabla \mathbf{v}) 
    + 2\overline{\mu} (\nabla \delta \mathbf{u})^\texttt{sym} : \nabla \mathbf{v}.
\end{equation*}
Finally, substituting the interpolation rules in \eqref{Material Interpolation Functions}, \eqref{Interpolated First PK Stress}, and \eqref{Interpolated Left-hand Side} into \eqref{Global Governing Equation} yields the parameterized global equilibrium equations, from which one can solve for the structural elastoplastic responses with various structural geometries and material phases.

\subsection{Topology optimization formulation}

Once deriving the parameterized global equilibrium equations, we can present the topology optimization formulation as
\begin{equation} \label{Topology Optimization Formulation}
    \left\{ \begin{array}{ll}
        \underset{\rho, \xi_1, \ldots, \xi_{N^\xi}}{\text{maximize:}} & J(\overline{\rho}, \overline{\xi}_1, \ldots, \overline{\xi}_{N^\texttt{mat}}; \mathbf{u}_1, \ldots, \mathbf{u}_N; \overline{\mathbf{b}}_1^\texttt{e}, \ldots, \overline{\mathbf{b}}_N^\texttt{e}; \overline{\boldsymbol{\beta}}_1, \ldots, \overline{\boldsymbol{\beta}}_N; \alpha_1, \ldots, \alpha_N; \widehat{\gamma}_1, \ldots, \widehat{\gamma}_N); \\[12pt]
        
        \text{subject to:} & \left\{ \begin{array}{l}
            g_{V0} = \displaystyle \dfrac{1}{|\Omega_0|} \int_{\Omega_0} \overline{\rho}(\mathbf{X}) \text{d} \mathbf{X} - \overline{V} \leq 0; \\[15pt]
            g_{Vn} = \displaystyle \dfrac{1}{|\Omega_0|} \int_{\Omega_0} \overline{\rho}(\mathbf{X}) \overline{\xi}_n(\mathbf{X}) \text{d} \mathbf{X} - \overline{V}_n \leq 0 \quad \text{for} \quad n = 1,2,\ldots,N^\texttt{mat}; \\[15pt]
            g_\Psi = \displaystyle \dfrac{1}{|\Omega_0|} \int_{\Omega_0} \overline{\rho}(\mathbf{X}) \sum_{n=1}^{N^\texttt{mat}} \left[ \overline{\xi}_n(\mathbf{X}) \Psi_n \right] \text{d} \mathbf{X} - \overline{\Psi} \leq 0 \quad \text{for} \quad \Psi \in \{P, M, C\}; \\[15pt]
            \rho(\mathbf{X}), \xi_1(\mathbf{X}), \ldots, \xi_{N^\xi}(\mathbf{X}) \in [0, 1];
        \end{array} \right. \\
        \\[2pt]

        \text{with:} & \left\{ \begin{array}{l}
            \text{Updating formulae for $\overline{\boldsymbol{\beta}}_1, \ldots, \overline{\boldsymbol{\beta}}_N$ and $\alpha_1, \ldots, \alpha_N$ in \eqref{Updated beta and alpha}}; \\[8pt]
            \text{Updating formulae for $\overline{\mathbf{b}}_1^\texttt{e}, \ldots, \overline{\mathbf{b}}_N^\texttt{e}$ in \eqref{Updated be_bar}}; \\[8pt]
            \text{Kuhn--Tucker condition for $\widehat{\gamma}_1, \ldots, \widehat{\gamma}_N$ in \eqref{Function of gamma}}; \\[8pt]
            \text{Global equilibrium equations for $\mathbf{u}_1, \ldots, \mathbf{u}_N$ in \eqref{Global Governing Equation}}.
        \end{array} \right.
    \end{array} \right.
\end{equation}
In these expressions, the variable $J$ represents a general objective function to be maximized, with its detailed formulation to be introduced later. The parameter $N$ represents the maximum load steps in FEA. The constraints $g_{V0}$ and $g_{Vn}$, for $n = 1, 2, \ldots, N^\texttt{mat}$, correspond to the total material volume fraction (with an upper bound $\overline{V} \in (0, 1]$) and the volume fraction of material $n$ (with an upper bound $\overline{V}_n \in [0, \overline{V}]$), respectively. Additionally, the constraints $g_P$, $g_M$, and $g_C$ impose limits on price, mass density, and CO$_2$ footprint, respectively, with $\overline{P}$, $\overline{M}$, and $\overline{C}$ representing their respective upper bounds. The parameters $P_n$, $M_n$, and $C_n$ denote the price, mass density, and CO$_2$ footprint of material $n$, respectively.

In this study, we aim to optimize the structural performance metrics of stiffness, strength, and effective structural toughness (through total energy). Following the footprint of \citet{jia_controlling_2023, jia_multimaterial_2025}, we design a comprehensive multi-objective function for finite strain elastoplasticity as
\begin{equation} \label{Multi-objective Function}
    J = w_\texttt{stiff} J_\texttt{stiff} + w_\texttt{force} J_\texttt{force} + w_\texttt{energy} J_\texttt{energy}.
\end{equation}
Here, $w_\texttt{stiff}$, $w_\texttt{force}$, and $w_\texttt{energy} \in [0, 1]$ are weighting factors subject to $w_\texttt{stiff} + w_\texttt{force} + w_\texttt{energy} = 1$. The energy-type variables $J_\texttt{stiff}$, $J_\texttt{force}$, and $J_\texttt{energy}$ represent stiffness, strength (end force), and structural toughness (total energy), respectively, and are defined as
\begin{equation} \label{Objective Function Terms}
    \left\{ \begin{array}{l}
        \displaystyle J_\texttt{stiff} = \dfrac{1}{2} \int_{\Omega_0} \widecheck{\mathbf{P}}_1 : \nabla \mathbf{u}_1 \text{d} \mathbf{X}, \quad 
        J_\texttt{force} = \int_{\Omega_0} \widecheck{\mathbf{P}}_N : \nabla \mathbf{u}_N \text{d} \mathbf{X}, \\[10pt]
        \displaystyle J_\texttt{energy} = \dfrac{1}{2} \sum_{n=1}^N \int_{\Omega_0} \left( \widecheck{\mathbf{P}}_n + \widecheck{\mathbf{P}}_{n-1} \right) : \left( \nabla \mathbf{u}_n - \nabla \mathbf{u}_{n-1} \right) \text{d} \mathbf{X}.
    \end{array} \right.
\end{equation}
Recall that $\mathbf{u}_n$ and $\widecheck{\mathbf{P}}_n$ for $n = 1,2,\ldots,N$ are the displacement and interpolated first Piola--Kirchhoff stress at load step $n$, respectively. The terms $J_\texttt{stiff}$, $J_\texttt{force}$, and $J_\texttt{energy}$ in \eqref{Objective Function Terms} are interpreted as follows. 
\begin{itemize}
    \item The variable $J_\texttt{stiff}$ represents the strain energy at the first load step, where plasticity has not yet evolved. Under displacement loading, $J_\texttt{stiff}$ is positively correlated with the initial stiffness of the structure. Therefore, maximizing $J_\texttt{stiff}$ enhances structural stiffness.
    \item The variable $J_\texttt{force}$ corresponds to the product of force and displacement at the final load step, also known as the ``end compliance" for hyperelastic structures. Under displacement loading, maximizing $J_\texttt{force}$ increases the final reaction force.
    \item Finally, the variable $J_\texttt{energy}$ captures the total energy absorption and dissipation of the structure, equivalent to the area enclosed by its force--displacement curve.
\end{itemize}

\begin{remark}
In the multi-objective function in \eqref{Multi-objective Function}, we opt to use the (interpolated) first Piola--Kirchhoff stress ($\widecheck{\mathbf{P}}$) as the the primary measure in contrast to the Kirchhoff stress ($\widehat{\boldsymbol{\tau}}$) in Section \ref{Sec: Finite Strain Elastoplasticity}. It is because $\widecheck{\mathbf{P}}$ is the easiest stress measure \citep{kumar_phase-field_2020} to gauge practically. However, we note that the multi-objective function in \eqref{Multi-objective Function} is general, and one can effortlessly replace $\widecheck{\mathbf{P}}$ with any other stress measures if desired.
\end{remark}

\begin{remark}
The proposed topology optimization formulation in \eqref{Topology Optimization Formulation} equipped with the general multi-objective function in \eqref{Multi-objective Function} can be simplified to recover several special cases that have demonstrated success. For example, setting $w_\texttt{stiff} = w_\texttt{energy} = 0$ and $w_\texttt{force}=1$ yields $J = J_\texttt{force}$, which mainly maximizes the ``end compliance" as in \citet{ivarsson_plastic_2021}. Similarly, setting $w_\texttt{stiff} = w_\texttt{force} = 0$ and $w_\texttt{energy} = 1$ derives $J = J_\texttt{energy}$, which primarily maximizes the energy of structures as in \citet{wallin_topology_2016}. Additionally, employing $w_\texttt{stiff} = 0$ and $w_\texttt{force} = w_\texttt{energy} = 0.5$ generates $J = 0.5 J_\texttt{force} + 0.5 J_\texttt{energy}$ as in \citet{abueidda_topology_2021}. As illustrated in Section \ref{Sec: Sample Examples}, utilizing the comprehensive multi-objective function in \eqref{Multi-objective Function} enables the generation of optimized structures exhibiting diverse and tailored elastoplastic responses under large deformations.
\end{remark}


\section{Optimized elastoplastic designs with real-world applications}
\label{Sec: Sample Examples}

In this section, we present optimized elastoplastic designs with real-world applications through four representative examples. These examples demonstrate the effectiveness of the proposed multimaterial topology optimization framework (Section \ref{Sec: Topology Optimization Framework}) in optimizing structural elastoplastic responses under large deformations. Additionally, we provide mechanical insights into achieving these optimized behaviors.

In the first example, we optimize the energy dissipation of multi-alloy dampers subjected to various cyclic loadings. This investigation reveals that employing multiple materials enhances energy dissipation compared to single-material ones under large deformations. Additionally, we demonstrate the transition from the kinematic hardening dominance to the isotropic hardening dominance, which improves energy dissipation as applied displacements increase. We also show that optimized dampers exhibit superior energy dissipation across multiple loading cycles compared to an intuitive design.

In the second example, we optimize the initial stiffness and end force of double-clamped beams. This example demonstrates the versatility of the proposed topology optimization framework in handling multiple design objectives for structures made of purely hyperelastic, purely elastoplastic, and mixed materials. We also explore the synergistic use of hyperelastic and elastoplastic materials to achieve various stiffness--strength interplays in composite structures.

In the third example, we maximize the crashworthiness of impact-resisting bumpers. Extending the optimization task to 3D geometries, we consider scenarios with more than two candidate materials. Through this example, we show that the proposed optimization framework applies to various spatial dimensions and an arbitrary number of candidate materials.

Finally, we optimize the end force of cold working profiled sheets accounting for both the processing (metal-forming) and service (load-carrying) stages, which involve ultra-large elastoplastic deformations. This example introduces multi-stage topology optimization, which incorporates both intra-stage and inter-stage history dependencies. Additionally, we integrate practical constraints --- cost, lightweight, and sustainability --- into the design and optimization of elastoplastic structures, bridging the gap between mechanical designs and non-mechanical considerations.

We remark that these examples span a spectrum of material properties, as summarized in Table \ref{Table: Material Properties}. For most examples, we consistently use titanium, bronze, nickel--chromium, and stainless steel, which exhibit perfect plasticity, linear isotropic hardening, nonlinear isotropic hardening, and kinematic hardening, respectively. Importantly, these four materials can be joined together using 3D printing techniques, as reported in \citet{wei_overview_2020, wei_cu10sn_2022}. In the double-clamped beam example, we utilize elastoplastic lithium, characterized by combined isotropic and kinematic hardening, alongside hyperelastic PCL, modeled by assigning a sufficiently large yield strength of $\sigma_y = 25$ MPa.

\begin{table}[!htbp]
    \caption{Material properties used in numerical examples}
    \label{Table: Material Properties}
    \centering
    \scriptsize
    \begin{tabular}{lccccccccc}
        \hline
        \textbf{Materials}
        & \begin{tabular}[x]{@{}c@{}}
            \textbf{Titanium} \\
            (Ti--6Al--4V)
        \end{tabular}
        & \begin{tabular}[x]{@{}c@{}}
            \textbf{Bronze} \\
            (CuSn10)
        \end{tabular}
        & \begin{tabular}[x]{@{}c@{}}
            \textbf{Nickel} \\
            \textbf{--chromium} \\
            (INCONEL 718)
        \end{tabular}
        & \begin{tabular}[x]{@{}c@{}}
            \textbf{Stainless steel} \\
            (AISI 316L)
        \end{tabular}
        & \begin{tabular}[x]{@{}c@{}}
            \textbf{Lithium} \\
            (commercial \\
            purity)
        \end{tabular}
        & \begin{tabular}[x]{@{}c@{}}
            \textbf{PCL} \\
            (polycapro \\
            -lactone)
        \end{tabular} \\
        \hline
        Features & \begin{tabular}[x]{@{}c@{}}
            Perfect \\
            plasticity
        \end{tabular} & \begin{tabular}[x]{@{}c@{}}
            Linear \\
            isotropic \\
            hardening
        \end{tabular} & \begin{tabular}[x]{@{}c@{}}
            Nonlinear \\
            isotropic \\
            hardening
        \end{tabular} & \begin{tabular}[x]{@{}c@{}}
            Kinematic \\
            hardening
        \end{tabular} & \begin{tabular}[x]{@{}c@{}}
            Combined \\
            hardening
        \end{tabular} & \begin{tabular}[x]{@{}c@{}}
            Near \\
            hyperelasticity \end{tabular} \\[10pt]

        \begin{tabular}[x]{@{}l@{}}
            Bulk moduli, \\
            $\kappa$ (GPa)
        \end{tabular} & 115.6 & 88.9 & 165.0 & 141.3 & 5.8 & 0.3880 \\[10pt]
        \begin{tabular}[x]{@{}l@{}}
            Shear moduli, \\
            $\mu$ (GPa)
        \end{tabular} & 41.4 & 29.6 & 76.2 & 76.8 & 1.8 & 0.0157 \\[10pt]
        \begin{tabular}[x]{@{}l@{}}
            Young's moduli, \\
            $E$ (GPa)
        \end{tabular} & 111.0 & 80.0 & 198.0 & 195.0 & 4.9 & 0.0466 \\[10pt]
        Poisson's ratios, $\nu$ & 0.34 & 0.35 & 0.30 & 0.27 & 0.36 & 0.48 \\[10pt]
        \begin{tabular}[x]{@{}l@{}}
            Kinematic hardening \\
            moduli, $h$ (MPa)
        \end{tabular} & 0.0 & 0.0 & 0.0 & 1339.1 & 2.5 & 0.0 \\[10pt]
        \begin{tabular}[x]{@{}l@{}}
            Isotropic hardening \\
            moduli, $K$ (MPa)
        \end{tabular} & 0.0 & 952.0 & 129.0 & 0.0 & 2.5 & 0.0 \\[10pt]
        \begin{tabular}[x]{@{}l@{}}
            Initial yield \\
            strengths, $\sigma_y$ (MPa)
        \end{tabular} & 853.0 & 145.0 & 450.0 & 226.0 & 1.0 & 25.0 \\[10pt]
        \begin{tabular}[x]{@{}l@{}}
            Residual yield \\
            strengths, $\sigma_\infty$ (MPa)
        \end{tabular} & 853.0 & 145.0 & 715.0 & 226.0 & 1.0 & 25.0 \\[10pt]
        \begin{tabular}[x]{@{}l@{}}
            Saturation \\
            exponents, $\delta$ 
        \end{tabular} & 0.0 & 0.0 & 16.9 & 0.0 & 0.0 & 0.0 \\[10pt]
        \begin{tabular}[x]{@{}l@{}}
            Isotropic hardening \\
            functions, $k(\alpha)$ (MPa)
        \end{tabular} & 853.0 & $145.0 + 952.0 \alpha$ & \begin{tabular}[x]{@{}c@{}}
            $450.0 + 129.0 \alpha$ \\
            $+ 265.0 (1-e^{-16.9 \alpha})$
        \end{tabular}  & 226.0 & $1.0 + 2.5 \alpha$ & 25.0 \\[10pt]
        Price (USD/kg) & 24.4 & 13.3 & 25.2 & 6.6 & 127.0 & 6.8 \\[10pt]
        \begin{tabular}[x]{@{}l@{}}
            Mass densities \\
             ($10^3$ kg/m$^3$)
        \end{tabular} & 4.4 & 8.8 & 8.2 & 8.0 & 0.5 & 1.1 \\[10pt]
        CO$_2$ footprint (kg/kg) & 40.4 & 6.0 & 16.6 & 7.4 & 79.6 & 2.3 \\
        \hline
    \end{tabular}
\end{table}

To implement the proposed topology optimization framework, we adapt the open-source software\footnote{Available in GitHub at \url{https://github.com/missionlab/fenitop}.} \citep{jia_fenitop_2024} to incorporate finite strain elastoplasticity. The software efficiently computes partial derivatives using automatic differentiation and accelerates computations through parallel processing. To ensure high solution precision, we utilize a direct solver with lower--upper factorization to solve the global equilibrium equations in \eqref{Global Governing Equation} and the adjoint equations in \eqref{Reduced Adjoint Equation at Final Step}--\eqref{Reduced Adjoint Equation at Remaining Steps}. Additionally, to determine the optimal design variables, we employ the well-established gradient-based optimizer, the method of moving asymptotes \citep{svanberg_method_1987}. This optimizer utilizes the sensitivity analysis presented in \ref{Sec: Sensitivity Analysis and Verification} and efficiently updates the design variables until convergence. The optimized elastoplastic structures and the mechanisms enabling these design objectives are detailed below.

\subsection{Metallic yielding dampers with maximized hysteretic energy}
\label{Sec: Dampers}

\subsubsection{Single-alloy versus multi-alloy dampers}

In this subsection, we optimize the energy dissipation of metallic dampers \citep{zhang_simplified_2018, jia_residual_2022}, which are typically used in structural engineering to suppress vibrations induced by earthquakes or winds. Through these optimized dampers, we aim to demonstrate the effectiveness of the proposed topology optimization framework in optimizing structural elastoplastic responses under large deformations. Additionally, we highlight the necessity of employing multiple alloys to achieve greater energy dissipation.

To achieve these goals, we revisit the damper optimization problem in \citet{jia_multimaterial_2025}. As shown in Fig. \ref{Fig: Damper-Part 1}(a), the design domain is a rectangle subjected to simple shear loadings under the plane strain condition. We aim to optimally distribute the bronze and steel materials, whose Kirchhoff stress--Lagrangian strain ($\tau_{11}$--$E_{11}$) curves are shown in Fig. \ref{Fig: Damper-Part 1}(b), to maximize the total hysteretic energy (Fig. \ref{Fig: Damper-Part 1}(c)). Notably, the current setup incorporates non-monotonic, cyclic loadings and accounts for large deformations, representing a more complex scenario compared to the original problem in \citet{jia_multimaterial_2025}.

To solve this problem, we employ the following FEA and optimization strategies. The design domain is discretized using a structured mesh consisting of 15,000 first-order quadrilateral elements. The filter radius is set to $R_\zeta = 10$ mm for $\zeta \in \{\rho, \xi_1, \ldots, \xi_{N^\xi} \}$. The Heaviside sharpness parameter, $\beta_\zeta$, is initially set to 1 and is doubled every 40 optimization iterations starting from iteration 41, until it reaches a maximum value of $\beta_\zeta = 512$. The penalty parameters for the density variable are fixed at $p_\kappa = p_\mu = p_h = 3$ and $p_k = 2.5$, while the penalty parameter for the material variables, $p_\xi$, starts at 1 and increases by 0.25 every 40 optimization iterations from iteration 41, up to a maximum of $p_\xi = 3$. The weighting factors in \eqref{Multi-objective Function} are set as $w_\texttt{stiff} = w_\texttt{force} = 0$ and $w_\texttt{energy} = 1$. The constraint is defined by the total material volume, $g_{V0} \leq 0$, with an upper bound of $\overline{V} = 0.5$. The maximum number of optimization iterations is set to 500. During the topology optimization process, we use 32 non-uniform load steps in the FEA. Following optimization, we refine the analysis by using 44 load steps to evaluate the elastoplastic responses of all designs.

\begin{figure}[!htbp]
    \centering
    \includegraphics[height=20cm]{Damper-Part_1.pdf}
    \caption{Design and optimization of metallic yielding dampers. (a) Design setups: the design domain, boundary conditions, and candidate materials (bronze, steel, and void). The variable $u$ is the applied displacement. (b) Uniaxial Kirchhoff stress--Lagrangian strain ($\tau_{11}$--$E_{11}$) curves of candidate materials. (c) Optimization objective: maximizing the total energy of dampers. The variable $F$ is the reaction force. (d) Intuitive and optimized dampers. The total energy ($\Pi$) is in kN$\cdot$m. The percentages are the total energy increments compared to the intuitive design. (e) Normalized equivalent plastic strain ($\alpha/\max (\alpha)$). (f) Force--displacement ($F$--$u$) curves. (g) Energy--displacement ($\Pi$--$u$) curves.}
    \label{Fig: Damper-Part 1}
\end{figure}

The damper designs under half-cycle loadings, along with the hysteretic energy ($\Pi$ in kN$\cdot$m), are illustrated in Fig. \ref{Fig: Damper-Part 1}(d). As a baseline, we include one intuitive composite design that uses the same material volume fraction ($\overline{V} = 0.5$) as the three optimized designs. These optimized designs are tailored for bronze, steel, and multiple materials, respectively. Compared to the intuitive composite design, the three optimized dampers achieve greater hysteretic energy, with increases of 4.58\%, 3.19\%, and 10.15\%, respectively. These improvements highlight the effectiveness of the proposed topology optimization framework in enhancing elastoplastic responses under large deformations. Moreover, the optimized composite design naturally favors the use of two materials and achieves higher energy dissipation than the two single-material optimized designs. This performance gain aligns with the infinitesimal strain scenario in \citet{jia_multimaterial_2025} and reinforces the benefit of using multiple materials to improve design performance under large deformations.

To understand the performance gains, we present the normalized equivalent plastic strain ($\alpha/\max (\alpha)$) for all dampers in Fig. \ref{Fig: Damper-Part 1}(e). In the intuitive composite design, plastic strain is confined to the central region, whereas the three optimized designs distribute plastic deformation more evenly, allowing a greater portion of the material to effectively contribute to total energy dissipation. Further, by examining the Kirchhoff stress--Lagrangian strain ($\tau_{11}$--$E_{11}$) curves in Fig. \ref{Fig: Damper-Part 1}(b), we observe that the steel material dissipates more energy during the first quadrant (loading stage), while the bronze material dissipates more energy during the fourth quadrant (unloading stage). This difference in energy dissipation between materials is manifested in the dissimilar hysteretic loops of the force--displacement ($F$--$u$) curves for the optimized bronze and steel designs (Fig. \ref{Fig: Damper-Part 1}(f)). In contrast, the optimized composite damper effectively utilizes both materials to balance energy dissipation during the loading and unloading stages, forming the largest hysteresis loop, which ultimately results in the highest total energy dissipation shown in Fig. \ref{Fig: Damper-Part 1}(g).

\subsubsection{Dampers under complete cyclic loadings with increasing displacement amplitudes}

Having demonstrated the effectiveness of the topology optimization framework and the necessity of employing multiple materials, we now evaluate the performance of optimized composite dampers under complete cyclic loadings with increasing displacement amplitudes. As shown in Fig. \ref{Fig: Damper-Part 2}(a), we present three optimized dampers with applied displacement amplitudes of $u_\texttt{max} = 10$, 20, and 30 mm, respectively. For each optimized damper, we report the total energy ($\Pi$) and its increment compared to the intuitive design in Fig. \ref{Fig: Damper-Part 1}(d). All three optimized dampers demonstrate performance gains --- 20.25\%, 17.28\%, and 18.60\%, respectively --- relative to the intuitive design. These gains are attributed to the optimized dampers' ability to engage more materials in energy dissipation, as evidenced by the normalized total energy density ($W/\max (W)$) in Fig. \ref{Fig: Damper-Part 2}(c). Notably, these performance improvements exceed the 10.15\% gain achieved by the optimized composite damper under half-cycle loadings (Fig. \ref{Fig: Damper-Part 1}(d)). This enhanced performance is due to the full-cycle optimized designs achieving significant energy increases in the second quadrant (reloading stage), despite slight energy decreases in the first quadrant (loading stage) for $u_\texttt{max} = 20$ and 30 mm, as shown in Fig. \ref{Fig: Damper-Part 2}(b) and (c).

A comparison among the three optimized designs in Fig. \ref{Fig: Damper-Part 2}(a) further reveals the influence of displacement amplitude ($u_\texttt{max}$) on material distribution. As displacement amplitudes increase, the optimized dampers increasingly favor the use of bronze over steel. This trend can be explained by the material behaviors illustrated in the Kirchhoff stress--Lagrangian strain ($\tau_{11}$--$E_{11}$) curves in Fig. \ref{Fig: Damper-Part 1}(b). The steel material, which exhibits kinematic hardening, produces a hysteretic loop approximating a parallelogram. Its total energy density is expressed as
\begin{equation} \label{Total Energy Density of Steel}
    W^\texttt{s} \approx 4 \sigma_y^\texttt{s} E_\texttt{max}
\end{equation}
where $\sigma_y^\texttt{s}$ is the yield strength of steel and $E_\texttt{max}$ is the Lagrangian strain amplitude. In contrast, the hysteretic loop of the bronze material expands with larger strains due to isotropic hardening, with its total energy density given by
\begin{equation} \label{Total Energy Density of Bronze}
    W^\texttt{b} \approx 2 E_\texttt{max} \left[ \sigma_y^\texttt{b} + (\sigma_y^\texttt{b} + 4 E_\texttt{t}^\texttt{b} E_\texttt{max}) \right]
    = 8 E_\texttt{t}^\texttt{b} E_\texttt{max}^2 + 4 \sigma_y^\texttt{b} E_\texttt{max}
\end{equation}
where $\sigma_y^\texttt{b}$ and $E_\texttt{t}^\texttt{b} > 0$ are the yield strength and average plastic tangent modulus of bronze, respectively. At small strains, $W^\texttt{b}$ is dominated by the linear term, with $W^\texttt{b} \rightarrow 4 \sigma_y^\texttt{b} E_\texttt{max} < W^\texttt{s}$ as $E_\texttt{max} \rightarrow 0$. At larger strains, $W^\texttt{b}$ is dominated by the quadratic term, with $W^\texttt{b} \rightarrow 8 E_\texttt{t}^\texttt{b} E_\texttt{max}^2 > W^\texttt{s}$ as $E_\texttt{max} \rightarrow \infty$. Consequently, as $E_\texttt{max}$ (and $u_\texttt{max}$) increases, the optimized dampers gradually favor bronze over steel.

We also note that these findings are specific to the context of large deformations, where both the materials (Fig. \ref{Fig: Damper-Part 1}(b)) and structures (Fig. \ref{Fig: Damper-Part 2}(b)) yield immediately upon loading. Therefore, the transition from kinematic to isotropic hardening --- rather than the stiffness-to-strength transition observed in the infinitesimal strain scenario \citep{jia_multimaterial_2025} --- dominates the material distribution in the optimized dampers.

\begin{figure}[!htbp]
    \centering
    \includegraphics[height=12.5cm]{Damper-Part_2.pdf}
    \caption{Optimized dampers under increasing applied displacements. (a) Optimized dampers. The total energy ($\Pi$) is in kN$\cdot$m, and the percentages are the total energy increments compared to the intuitive design in Fig. \ref{Fig: Damper-Part 1}(d). (b) Force--displacement ($F$--$u$) curves. (c) Energy--displacement ($\Pi$--$u$) curves. The insets are the normalized total energy density ($W/\max(W)$) at the final load step.}
    \label{Fig: Damper-Part 2}
\end{figure}

\subsubsection{Dampers under multiple-cycle loadings}

Practically, energy-dissipating dampers are subjected to multiple-cycle loadings with non-constant displacement amplitudes \citep{zhang_simplified_2018, jia_novel_2019, jia_double_2021, jia_residual_2022}. In this subsection, we apply such complex loading conditions and evaluate the total energy of the optimized damper. As shown in Fig. \ref{Fig: Damper-Part 3}(a), we compare the intuitive design from Fig. \ref{Fig: Damper-Part 1}(d) with a damper optimized under multiple-cycle loadings. The optimized design exhibits an organic material distribution and achieves a total energy increase of 32.84\%, rising from 4.45 kN$\cdot$m in the intuitive design to 5.91 kN$\cdot$m.

The force--displacement ($F$--$u$) and energy--displacement ($\Pi$--$u$) curves of the two designs are shown in Fig. \ref{Fig: Damper-Part 3}(b) and (c), respectively. In Fig. \ref{Fig: Damper-Part 3}(b), each force--displacement curve consists of 166 non-uniform load steps and spans three cycles with increasing displacement amplitudes of 10, 20, and 30 mm. Compared to the intuitive design, the optimized damper demonstrates a slightly smaller hysteresis loop during the first cycle but more expanded loops in the second and third cycles. Here is why.

Based on \eqref{Total Energy Density of Steel}, the total energy densities of steel during the three cycles are expressed as
\begin{equation*}
    W_1^\texttt{s} \approx 4 \sigma_y^\texttt{s} E_{1,\texttt{max}},
    \quad W_2^\texttt{s} \approx 4 \sigma_y^\texttt{s} E_{2,\texttt{max}},
    \quad \text{and} \quad
    W_3^\texttt{s} \approx 4 \sigma_y^\texttt{s} E_{3,\texttt{max}}
\end{equation*}
where $W_i^\texttt{s}$ for $i=1,2,3$ is the total energy density of steel during cycle $i$, and $E_{i,\texttt{max}}$ is the strain amplitude of that cycle. Similarly, based on \eqref{Total Energy Density of Bronze}, the total energy densities of bronze during the three cycles are
\begin{equation*}
    \left\{ \begin{array}{ll}
        W_1^\texttt{b} 
        & \approx 2 E_{1,\texttt{max}} \left[ \sigma_y^\texttt{b} + (\sigma_y^\texttt{b} + 4 E_\texttt{t}^\texttt{b} E_{1,\texttt{max}}) \right]
        = 8 E_\texttt{t}^\texttt{b} E_{1,\texttt{max}}^2 + 4 \sigma_y^\texttt{b} E_{1,\texttt{max}} \\[5pt]
        
        W_2^\texttt{b} 
        & \approx 2 E_{2,\texttt{max}} \left\{ (\sigma_y^\texttt{b} + 4 E_\texttt{t}^\texttt{b} E_{1,\texttt{max}}) + \left[ \sigma_y^\texttt{b} + 4 E_\texttt{t}^\texttt{b} (E_{1,\texttt{max}} + E_{2,\texttt{max}}) \right] \right\} \\[5pt]
        & = 8 E_\texttt{t}^\texttt{b} E_{2,\texttt{max}}^2 + 4 \sigma_y^\texttt{b} E_{2,\texttt{max}} + 16 E_\texttt{t}^\texttt{b} E_{1,\texttt{max}} E_{2,\texttt{max}} \\[5pt]

        W_3^\texttt{b} 
        & \approx 2 E_{3,\texttt{max}} \left\{ \left[ \sigma_y^\texttt{b} + 4 E_\texttt{t}^\texttt{b} (E_{1,\texttt{max}} + E_{2,\texttt{max}}) \right] + \left[ \sigma_y^\texttt{b} + 4 E_\texttt{t}^\texttt{b} (E_{1,\texttt{max}} + E_{2,\texttt{max}} + E_{3,\texttt{max}}) \right] \right\} \\[5pt]
        & = 8 E_\texttt{t}^\texttt{b} E_{3,\texttt{max}}^2 + 4 \sigma_y^\texttt{b} E_{3,\texttt{max}} + 16 E_\texttt{t}^\texttt{b} E_{1,\texttt{max}} E_{3,\texttt{max}} + 16 E_\texttt{t}^\texttt{b} E_{2,\texttt{max}} E_{3,\texttt{max}} \\[5pt]
    \end{array} \right.
\end{equation*}
where $W_i^\texttt{b}$ for $i=1,2,3$ is the total energy density of bronze during cycle $i$. Given that $E_{1,\texttt{max}} < E_{2,\texttt{max}} < E_{3,\texttt{max}}$, we observe $W_1^\texttt{s} < W_2^\texttt{s} < W_3^\texttt{s}$ and $W_1^\texttt{b} < W_2^\texttt{b} < W_3^\texttt{b}$. This progression indicates that the total energy contributions increase across successive cycles. Consequently, the optimized damper contracts the force--displacement curve and sacrifices the total energy during the first cycle. However, it compensates by expanding the force--displacement curves and increasing total energy during the second and third cycles. Eventually, the optimized damper shows greater end energy than the intuitive design (Fig. \ref{Fig: Damper-Part 3}(c)), demonstrating its superior performance under multiple-cycle loading.

\begin{figure}[!htbp]
    \centering
    \includegraphics[height=7.5cm]{Damper-Part_3.pdf}
    \caption{Dampers under multiple cycles of loadings. (a) Damper designs. The total energy ($\Pi$) is in kN$\cdot$m, and the percentage is the total energy increment compared to the intuitive design. (b) Force--displacement ($F$--$u$) curves. The insets are the deformed configurations of the optimized damper. (c) Energy--displacement ($\Pi$--$u$) curves. The insets are the equivalent plastic strains ($\alpha$), and the values above 1 are plotted as 1 for better visualization.}
    \label{Fig: Damper-Part 3}
\end{figure}

Through this example of optimizing metallic yielding dampers, we numerically prove the effectiveness of the proposed multimaterial topology optimization framework in enhancing elastoplastic responses of structures under large deformations. Exploiting this framework, we present a series of optimized dampers with superior energy dissipation compared to an intuitive design, irrespective of the applied loading conditions, including half-cycle, complete-cycle, and multiple-cycle loadings. Furthermore, we highlight the necessity of employing multiple materials to improve the energy dissipation of the optimized dampers. Our analysis also reveals key phenomena, such as the transition from kinematic to isotropic hardening under complete-cycle loadings with increasing displacement amplitudes and the dominance of later cycles over initial cycles in multiple-cycle loadings. Together with the framework for infinitesimal strain elastoplasticity in \citet{jia_multimaterial_2025}, this study completes a comprehensive narrative for optimizing energy-dissipating devices. 

\subsection{Hyperelastic--elastoplastic composite structures with tailored stiffness--strength balance}

In this example, we design and optimize composite structures composed of both hyperelastic and elastoplastic materials. These optimized designs showcase various stiffness--strength interplays by harnessing the complementary properties of elasticity and plasticity. Additionally, we demonstrate the versatility of the proposed topology optimization framework in optimizing material distributions across a range of configurations, including purely hyperelastic, purely elastoplastic, and mixed-material systems.

The design setups are shown in Fig. \ref{Fig: Beam}(a). We optimize a double-clamped beam, where the top middle edge is subjected to a downward displacement, $u$. The design objective is to distribute hyperelastic PCL and elastoplastic lithium, whose Kirchhoff stress--Lagrangian strain ($\tau_{11}$--$E_{11}$) curves are shown in Fig. \ref{Fig: Beam}(b), to simultaneously maximize the initial stiffness and end force of the structures (Fig. \ref{Fig: Beam}(c)).

\begin{figure}[!htbp]
    \centering
    \includegraphics[height=15cm]{Beam.pdf}
    \caption{Design and optimization of double-clamped beams. (a) Design setups: the design domain, boundary conditions, and candidate materials (lithium, PCL, and void). The variable $u$ is the applied displacement. (b) Uniaxial Kirchhoff stress--Lagrangian strain ($\tau_{11}$--$E_{11}$) curves of candidate materials. (c) Optimization objective: maximizing the initial stiffness and end force of beams. The variable $F$ is the reaction force. (d) Designs and deformations of optimized beams. The deformed configurations also show the elastic/plastic regions. (e) Force--displacement ($F$--$u$) curves. (f) Performance metrics of the initial stiffness and end force. The two metrics are normalized by the maxima of all designs, respectively.}
    \label{Fig: Beam}
\end{figure}

To achieve the design objective, we employ the following FEA and optimization treatments. The design domain is discretized using a structured mesh of 14,400 first-order quadrilateral elements. The filter radius is set to $R_\rho=1$ mm for the density variable ($\rho$) and $R_\zeta = 3$ mm for $\zeta \in \{ \xi_1, \ldots, \xi_{N^\xi} \}$. The Heaviside sharpness parameter ($\beta_\zeta$) is initialized at 1 and doubled every 20 optimization iterations starting from iteration 21 until it reaches a maximum value of $\beta_\zeta = 512$. The penalty parameters for the density variable are fixed at $p_\kappa = p_\mu = p_h = p_k = 3$, while the penalty parameter for the material variables ($p_\xi$) starts at 1 and increases by 0.25 every 20 optimization iterations from iteration 21, up to a maximum of $p_\xi = 5$. The constraint is defined by the total material volume ($g_{V0} \leq 0$) with an upper bound of $\overline{V} = 0.5$. The maximum number of optimization iterations is set to 500. For performance evaluation using FEA, we consistently apply 18 non-uniform load steps across all designs.

The optimized beams and their deformations are shown in Fig. \ref{Fig: Beam}(d). We present four optimized bi-material designs (Dsgs. 2--5) with decreasing weighting ratios of initial stiffness ($w_\texttt{stiff}$ = 0.30, 0.25, 0.20, and 0.05, respectively) and increasing weighting ratios of end force ($w_\texttt{force}$ = 0.70, 0.75, 0.80, and 0.95, respectively). Additionally, two single-material designs are included as references: a lithium design (Dsg. 1) optimized for initial stiffness and a PCL design (Dsg. 6) optimized for the end force.

A comparison of Dsgs. 2--5 reveals a gradual shift in material preference as $w_\texttt{stiff}$ decreases and $w_\texttt{force}$ increases. The optimized designs progressively favor the hyperelastic PCL over the elastoplastic lithium. This trend is driven by the inherent properties of the materials: while metals like lithium are stiffer than polymers like PCL, lithium yields immediately under large deformations, with its Kirchhoff stress confined by the yield surface (Figs. \ref{Fig: Beam}(b) and (d)). In contrast, PCL’s hyperelastic behavior allows its Kirchhoff stress to increase rapidly under deformation. Consequently, lithium contributes more to stiffness and is preferred when $w_\texttt{stiff}$ is larger, whereas PCL contributes more to the end force and is favored when $w_\texttt{force}$ is larger.

These material contributions are further verified by the force--displacement curves in Fig. \ref{Fig: Beam}(e). As $w_\texttt{stiff}$ decreases and $w_\texttt{force}$ increases from Dsg. 2 to Dsg. 5, the initial stiffness diminishes while the peak force increases due to the greater incorporation of PCL in the optimized designs. Notably, the stiffness and peak force of the bi-material designs (Dsgs. 2--5) are bounded by the stiffness of the optimized lithium design (Dsg. 1) and the peak force of the optimized PCL design (Dsg. 6), respectively. Through these optimized designs, we demonstrate the generality of the proposed framework in optimizing structures composed of hyperelastic and/or elastoplastic materials under finite deformations. This generality enables tailoring stiffness--strength (end force) interplay for composite structures (Fig. \ref{Fig: Beam}(f)).

\subsection{Front bumpers with maximized crashworthiness}

In this subsection, we extend the optimization framework to 3D by maximizing the crashworthiness of impact-resisting front bumpers \citep{patel_crashworthiness_2009, sun_crashworthiness_2018, wang_structure_2018, ren_effective_2020, wang_multi-objective_2020} and demonstrate its capability to handle more than two candidate materials.

\subsubsection{Bi-material bumpers in 3D}

As shown in Fig. \ref{Fig: Bumper-Part 1}(a), we consider an arch-shaped design domain with four fixed corners, subjected to quasi-static displacement loading ($u$). The design objective is to optimally distribute titanium, bronze, nickel--chromium, and steel, whose Kirchhoff stress--Lagrangian strain ($\tau_{11}$--$E_{11}$) curves are shown in Fig. \ref{Fig: Bumper-Part 1}(b), to maximize the total energy --- encompassing both elastic energy absorption and plastic energy dissipation --- as shown in Fig. \ref{Fig: Bumper-Part 1}(c).

We begin by optimizing bi-material (titanium and bronze) bumpers in 3D, and FEA and optimization setups are as follows. The design domain is discretized using an unstructured mesh consisting of 138,400 first-order hexahedral elements. The filter radius is set to $R_\zeta = 40$ mm for $\zeta \in \{\rho, \xi_1, \ldots, \xi_{N^\xi} \}$. The Heaviside sharpness parameter, $\beta_\zeta$, is initially set to 1 and is doubled every 40 optimization iterations starting from iteration 41 until it reaches a maximum value of $\beta_\zeta = 512$. The penalty parameters for the density variable are fixed at $p_\kappa = p_\mu = p_h = 3$ and $p_k = 2.5$, while the penalty parameter for the material variables, $p_\xi$, starts at 1 and increases by 0.25 every 40 optimization iterations from iteration 41, up to a maximum of $p_\xi = 3$. The weighting factors in \eqref{Multi-objective Function} are set as $w_\texttt{stiff} = w_\texttt{force} = 0$ and $w_\texttt{energy} = 1$. The constraints include the total material volume ($g_{V0} \leq 0$) with an upper bound of $\overline{V} = 0.2$, and individual material volumes ($g_{V1} \leq 0$ and $g_{V2} \leq 0$) with upper bounds $\overline{V}_1 = \overline{V}_2 = 0.1$ to account for material availability. The maximum number of optimization iterations is set to 600. During the topology optimization process, we use 10 uniform load steps in the FEA. Following optimization, we refine the analysis by using 50 load steps to evaluate the elastoplastic responses of all designs.

\begin{figure}[!htbp]
    \centering
    \includegraphics[height=13.5cm]{Bumper-Part_1.pdf}
    \caption{Design and optimization of 3D bumpers. (a) Design setups: the design domain, boundary conditions, and candidate materials (titanium, bronze, nickel--chromium, steel, and void). The variable $u$ is the applied displacement. (b) Uniaxial Kirchhoff stress--Lagrangian strain ($\tau_{11}$--$E_{11}$) curves of candidate materials. (c) Optimization objective: maximizing the total energy. The variable $F$ is the reaction force. (d) Various views of the intuitive and optimized bi-material bumper designs. (e) Force--displacement ($F$--$u$) curves. The force ($F$) is in kN, and the total energy ($\Pi$) is in kN$\cdot$m. The percentages are the improved values compared to the intuitive design. (f) Elastic/plastic regions in the deformed configuration at the final load step.}
    \label{Fig: Bumper-Part 1}
\end{figure}

The bi-material bumper designs are illustrated in Fig. \ref{Fig: Bumper-Part 1}(d). For comparison, we include an intuitive design as a reference, which shares the same total and individual material volumes as the optimized bumper. The comparison reveals that the optimized design favors an X-shaped part in the middle, in contrast to the I-shaped structure in the intuitive design. This X-shaped topology shortens load paths, thereby increasing structural stiffness and enhancing elastic energy absorption. Additionally, compared to the smooth members of the intuitive design, the optimized bumper features non-smooth, twisted regions that effectively concentrate stress, promoting localized material yielding and greater plastic energy dissipation.

Beyond the advantageous structural geometries, the optimized bumper also strategically distributes the material phases. By analyzing the material distribution in Fig. \ref{Fig: Bumper-Part 1}(d) alongside the elastic/plastic regions shown in Fig. \ref{Fig: Bumper-Part 1}(f), it is evident that the optimized design positions titanium primarily in elastic regions and bronze in plastic regions. This distribution aligns with the material properties illustrated in Fig. \ref{Fig: Bumper-Part 1}(b). Titanium, with its higher yield strength, remains in the elastic deformation regime, providing substantial elastic energy absorption. In contrast, bronze, with its lower yield strength, undergoes plastic deformation. Despite its lower initial strength, bronze exhibits isotropic hardening, allowing its Kirchhoff stress to increase progressively under loading. This behavior contrasts with the perfect plasticity (no hardening) of titanium, enabling bronze to dissipate certain plastic energy.

The combined effects of tailored structural geometries and material phases contribute to the better performance of the optimized bumper compared to the intuitive design. As shown in Fig. \ref{Fig: Bumper-Part 1}(e), the optimized bumper achieves a 105.41\% increase in total energy ($\Pi$), rising from 1.74 to 3.57 kN$\cdot$m, and a 129.24\% improvement in end force, increasing from 44.49 to 101.98 kN. These performance gains are a direct result of the simultaneous optimization of density and material variables enabled by the proposed framework.

\subsubsection{Tri-material and four-material bumpers}

After demonstrating the effectiveness of the proposed framework in optimizing bi-material bumpers in 3D, we now extend the approach to optimize bumpers with more than two candidate materials. Specifically, we optimize tri-material bumpers composed of titanium, bronze, and nickel--chromium. The constraints include the total material volume ($g_{V0} \leq 0$) with an upper bound of $\overline{V} = 0.2$, and individual material volumes ($g_{V1} \leq 0$, $g_{V2} \leq 0$, and $g_{V3} \leq 0$) with equal upper bounds of $\overline{V}_1 = \overline{V}_2 = \overline{V}_3 = \overline{V}/3$.

The intuitive and optimized tri-material bumper designs are illustrated in Fig. \ref{Fig: Bumper-Part 2}(a). Similar to the optimized bi-material design shown in Fig. \ref{Fig: Bumper-Part 1}(d), the optimized tri-material bumper retains the X-shaped structure in the middle, which shortens load paths to enhance structural stiffness. It also exhibits twisted regions that concentrate on plastic deformation. These features enhance both elastic energy absorption and plastic energy dissipation, contributing to the lifted force--displacement ($F$--$u$) curve in Fig. \ref{Fig: Bumper-Part 2}(b) and energy--displacement ($\Pi$--$u$) curve in Fig. \ref{Fig: Bumper-Part 2}(c). Ultimately, the optimized design achieves a 159.72\% increase in end force and a 160.49\% increase in total energy compared to the intuitive design.

\begin{figure}[!htbp]
    \centering
    \includegraphics[height=9.0cm]{Bumper-Part_2.pdf}
    \caption{Design and optimization of tri-material bumpers. (a) Various views of the intuitive and optimized designs. (b)--(c) Force--displacement ($F$--$u$) and energy--displacement ($\Pi$--$u$) curves, respectively. The force ($F$) is in kN, and the total energy ($\Pi$) is in kN$\cdot$m. The percentages are the improved values compared to the intuitive design.}
    \label{Fig: Bumper-Part 2}
\end{figure}

We now optimize a four-material bumper by incorporating steel in addition to titanium, bronze, and nickel--chromium. The constraints include the total material volume ($g_{V0} \leq 0$) with an upper bound of $\overline{V} = 0.2$, and individual material volumes ($g_{V1} \leq 0$, $g_{V2} \leq 0$, $g_{V3} \leq 0$, and $g_{V4} \leq 0$) with equal upper bounds of $\overline{V}_1 = \overline{V}_2 = \overline{V}_3 = \overline{V}_4 = 0.05$. The intuitive and optimized designs are shown in Fig. \ref{Fig: Bumper-Part 3}(a), which share the same usage for each material. Compared to the intuitive design, the optimized four-material bumper features a bulky structure in the middle, connected to four twisted regions. This highly non-intuitive configuration strategically exploits material properties: bulk steel material yields around the displacement loading area, while bronze yields near the fixed ends (similar to the elastic/plastic distribution shown in Fig. \ref{Fig: Bumper-Part 1}(f)), which provide most plastic energy dissipation. Meanwhile, titanium and nickel--chromium primarily deform elastically (except in the twisted regions) due to their higher yield strengths, which mainly contribute to elastic energy absorption. Eventually, as shown in Figs. \ref{Fig: Bumper-Part 3}(b) and (c), the end force of the optimized bumper increases by 300.27\% compared to the intuitive design, while the total energy improves by 259.31\%.

Taking a broader view of the optimized bi-material, tri-material, and four-material bumpers, we draw several observations as follows. First, the proposed topology optimization framework is highly versatile, accommodating a wide range of design problems involving finite strain elastoplasticity, regardless of dimensionality, material hardening types, or the number of candidate materials. Second, as the number of materials increases and the design space expands, it is increasingly difficult to create effective elastoplastic structures based on intuition or experience alone. This is evident from the progressively larger total energy increments achieved by the optimized designs compared to the intuitive ones --- 105.41\% in the bi-material case, 160.49\% in the tri-material case, and 259.31\% in the four-material case. In contrast, the proposed topology optimization framework consistently delivers high-performance designs by leveraging rigorous mechanics-based elastoplastic analysis and gradient-based optimization, effectively navigating the complexities of large design spaces.

\begin{figure}[!htbp]
    \centering
    \includegraphics[height=9.0cm]{Bumper-Part_3.pdf}
    \caption{Design and optimization of four-material bumpers. (a) Various views of the intuitive and optimized designs. (b)--(c) Force--displacement ($F$--$u$) and energy--displacement ($\Pi$--$u$) curves, respectively. The force ($F$) is in kN, and the total energy ($\Pi$) is in kN$\cdot$m. The percentages are the improved values compared to the intuitive design.}
    \label{Fig: Bumper-Part 3}
\end{figure}

\subsection{Cold working profiled sheets with maximized load-bearing capacity and multiple engineering constraints}

In this final example, we highlight the full potential of the proposed framework in optimizing elastoplastic responses involving ultra-large deformations and complex load histories. Furthermore, we demonstrate the versatility and generality of the framework in bridging the gap between mechanical design and practical considerations such as cost, lightweight, and sustainability.

\subsubsection{Multi-stage topology optimization of profiled sheets}

To illustrate the optimization of large elastoplastic deformations and complex load histories, we design profiled sheets as depicted in Fig. \ref{Fig: Sheet-Part 1}. These sheets are manufactured by bending a flat metallic plate at room temperature, a process commonly known as cold working or metal-forming \citep{gearing_plasticity_2001, cvitanic_finite_2008}. The resulting corrugated shapes of the profiled sheets provide both mechanical (e.g., increased area moment of inertia and enhanced material strength due to strain hardening) and non-mechanical (e.g., improved rainwater drainage and better architectural aesthetics) advantages. These attributes make profiled sheets widely applicable in various domains \citep{wright_use_1987}, such as rooftops and walls in structural engineering.

To design such profiled sheets, we begin with a raw flat sheet, as shown in Fig. \ref{Fig: Sheet-Part 1}(a). Our goal is to optimally distribute titanium, bronze, nickel--chromium, and steel (material properties are provided in Table \ref{Table: Material Properties} and Fig. \ref{Fig: Bumper-Part 1}(b)) in the processing (metal-forming) stage while maximizing the end force during the service (load-carrying) stage (Fig. \ref{Fig: Sheet-Part 1}(b)).

The processing and service stages of the profiled sheets are detailed in Fig. \ref{Fig: Sheet-Part 1}(c). During the processing stage, the flat metallic sheet is bent along the short edge to achieve the desired corrugated profile. In the service stage, the sheet undergoes bending along its long edge under new boundary conditions to simulate practical usage. These stages together form the whole-life analysis of the profiled sheet. Notably, unlike traditional topology optimization problems where design variables and objectives belong to the same stage, in this case, the design variables (material distribution) are defined in the processing stage, while the design objective (maximizing load-carrying capacity) belongs to the service stage. Additionally, the irreversible nature of elastoplasticity introduces history dependence both within each stage (where only boundary values, $\overline{\mathbf{q}}$, $\overline{\mathbf{t}}$, and $\overline{\mathbf{u}}$, are updated) and across the stages (where Dirichlet and Neumann boundaries, $\partial \Omega_0^\mathcal{D}$ and $\partial \Omega_0^\mathcal{N}$, respectively, are updated).

To address this complex design problem, we propose a multi-stage topology optimization approach. Within each optimization iteration, boundary conditions are updated to account for different stages of the profiled sheet --- process and service stages; or more precisely, raw, cold-worked, undeformed, and deformed stages in Fig. \ref{Fig: Sheet-Part 1}(c). Note that the initial state variables ($\mathbf{u}$, $\overline{\mathbf{b}}^\texttt{e}$, $\overline{\boldsymbol{\beta}}$, and $\alpha$) of each stage inherit the converged values from the previous stage to capture history dependence. This multi-stage topology optimization approach is naturally accessible through the proposed framework thanks to the comprehensive history-dependent sensitivity analysis in \ref{Sec: Sensitivity Analysis and Verification}. Consequently, no modifications to the framework are required, and we proceed directly to solving the design problem using the following FEA and optimization parameters.

The design domain is discretized using a structured mesh consisting of 28,800 first-order hexahedral elements. The filter radius is set to $R_\zeta = 20$ mm for $\zeta \in \{\rho, \xi_1, \ldots, \xi_{N^\xi} \}$. The Heaviside sharpness parameter, $\beta_\zeta$, is initially set to 1 and doubles every 40 optimization iterations starting from iteration 41 until it reaches a maximum value of $\beta_\zeta = 256$. The penalty parameters for the density variable are fixed at $p_\kappa = p_\mu = p_h = p_k = 3$, while the penalty parameter for the material variables, $p_\xi$, starts at 1 and increases by 0.5 every 60 optimization iterations from iteration 41, up to a maximum of $p_\xi = 4$. The weighting factors in \eqref{Multi-objective Function} are set as $w_\texttt{stiff} = w_\texttt{energy} = 0$ and $w_\texttt{force} = 1$. The constraints include the individual material volumes ($g_{V1} \leq 0$, $g_{V2} \leq 0$, $g_{V3} \leq 0$, and $g_{V4} \leq 0$) with equal upper bounds $\overline{V}_1 = \overline{V}_2 = \overline{V}_3 = \overline{V}_4 = 0.25$ to account for material availability. The maximum number of optimization iterations is set to 400.

During the topology optimization process, 12 non-uniform load steps are used for the processing stage, 16 uniform load steps for the service stage, and 13 steps for transitioning from the processing to service stages to simulate the removal of support and load blocks and restore the sheet to a stress-free state. Following optimization, refined analyses are conducted with 40 load steps for the processing stage, 61 load steps for the service stage, and 13 load steps for the transition stage.

\begin{figure}[!htbp]
    \centering
    \includegraphics[width=17.5cm]{Sheet-Part_1-V2.pdf}
    \caption{Design and optimization of profiled sheets. (a) Design setups: the design domain and candidate materials (titanium, bronze, nickel--chromium, steel, and void). (b) Optimization objective: maximizing the end force. The variables $F$ and $u$ are the reaction force and applied displacement, respectively. (c) Processing and service stages of profiled sheets. (d) Multi-stage topology optimization of profiled sheets. (e) Force--displacement ($F$--$u$) curves of the two stages.}
    \label{Fig: Sheet-Part 1}
\end{figure}

After multi-stage topology optimization, we present the optimized profiled sheet in Fig. \ref{Fig: Sheet-Part 1}(d), alongside an intuitive design with identical material usage for reference. For each design, four snapshots are shown, corresponding to the key stages depicted in Fig. \ref{Fig: Sheet-Part 1}(c). Unlike the serial arrangement of the four candidate materials in the intuitive design, the optimized profiled sheet features an intertwined material distribution that leverages the complementary properties of the materials shown in Fig. \ref{Fig: Bumper-Part 1}(b). Additionally, both designs exhibit large deformations in two directions, which necessitates the proposed framework to account for finite strain elastoplasticity --- an advancement beyond the reach of the infinitesimal strain version in \citet{jia_multimaterial_2025}.

The force--displacement ($F$--$u$) curves for both processing and service stages are compared in Fig. \ref{Fig: Sheet-Part 1}(e). In the processing stage, the force--displacement curves of the two designs are similar; however, in the service stage, the optimized structure exhibits a force--displacement curve that surpasses the intuitive design, achieving a 75.56\% increase in the end force. This performance improvement remarks the effectiveness of the proposed framework in tackling optimization challenges involving large elastoplastic deformations and multiple stages of loading.

\subsubsection{Optimized profiled sheets with practical constraints}

Despite the superior mechanical performance of the optimized profiled sheet endowed by the proposed framework, practical constraints --- such as cost, lightweight, and sustainability --- typically need to be integrated to create a useful mechanical product \citep{kundu2025sustainability}. In this case study, we incorporate these practical constraints into the framework, thereby bridging the gap between elastoplastic design optimization and broader real-world considerations.

To achieve this goal, we replace the individual material volume constraints ($g_{V1} \leq 0$, $g_{V2} \leq 0$, $g_{V3} \leq 0$, and $g_{V4} \leq 0$) imposed on the optimized design in Fig. \ref{Fig: Sheet-Part 2}(d) with practical constraints on price ($g_P \leq 0$), mass density ($g_M \leq 0$), and CO$_2$ footprint ($g_C \leq 0$). These constraints are defined with upper bounds set as the averages of the corresponding properties of the four candidate materials listed in Table \ref{Table: Material Properties}: $\overline{P} = 17.31$ USD/kg, $\overline{M} = 7,340$ kg/m$^3$, and $\overline{C} = 17.64$ kg/kg.

Under these updated constraints, we present the optimized profiled sheets in Fig. \ref{Fig: Sheet-Part 2}(a). Among the four optimized designs, Dsgs. 1--3 are constrained individually by price, mass density, and CO$_2$ footprint, respectively, while Dsg. 4 considers all three constraints simultaneously. These designs feature distinct material distributions (Fig. \ref{Fig: Sheet-Part 2}(a)) and usage patterns (Fig. \ref{Fig: Sheet-Part 2}(b)) as follows.
\begin{itemize}
    \item The price-constrained design (Dsg. 1) primarily incorporates titanium and steel due to their superior initial-strength-to-price ratios of 34.96 and 34.24 MPa$\cdot$kg/USD, respectively, compared to bronze (10.90 MPa$\cdot$kg/USD) and nickel--chromium (17.86 MPa$\cdot$kg/USD). A small amount of bronze appears sparingly, likely due to the strength enhancement achieved through linear isotropic hardening. On the other hand, nickel--chromium usage is minimized due to its highest absolute price.
    \item The weight-constrained design (Dsg. 2) is overwhelmingly dominated by titanium, a result of its optimal combination of the highest initial yield strength and the lowest mass density among the four materials.
    \item In contrast, the CO$_2$-constrained design (Dsg. 3) is largely composed of nickel--chromium. This material is favored because of its medium initial-strength-to-CO$_2$ ratio of 27.11 MPa$\cdot$kg/kg, which lies between the ratios of titanium (21.11 MPa$\cdot$kg/kg), bronze (24.17 MPa$\cdot$kg/kg), and steel (30.54 MPa$\cdot$kg/kg). Furthermore, with nonlinear isotropic hardening, the strength-to-CO$_2$ ratio of nickel--chromium improves to 43.07 MPa$\cdot$kg/kg, making it a suitable option.
    \item Finally, the balanced design (Dsg. 4) leverages similar amounts of all four candidate materials. This distribution reflects the fact that the upper bounds for the price, mass density, and CO$_2$ footprint constraints are averages of the properties of all four materials. As a result, the design naturally balances the contributions of titanium, bronze, nickel--chromium, and steel.
\end{itemize}

\begin{remark}
In addition to the mechanical performance improvements demonstrated in Fig. \ref{Fig: Damper-Part 1}, the incorporation of practical constraints also leads to the automatic inclusion of multiple materials in the optimized designs. This outcome further shows the necessity of a multimaterial topology optimization framework.
\end{remark}

\begin{figure}[!htbp]
    \centering
    \includegraphics[width=18.0cm]{Sheet-Part_2.pdf}
    \caption{Multi-stage topology optimization of profiled sheets with practical constraints. (a) Optimized profiled sheets with various practical constraints. (b) Material usage of the four optimized profiled sheets. (c) Force--displacement ($F$--$u$) curves of the service stage. (d) Performance metrics of the optimized profiled sheets.}
    \label{Fig: Sheet-Part 2}
\end{figure}
 
We assess the performance of these optimized designs in Fig. \ref{Fig: Sheet-Part 2}(c) and evaluate their performance metrics in Fig. \ref{Fig: Sheet-Part 2}(d). The load-carrying capacity is measured by the end force of the structures, while economic consideration, lightweight, and sustainability are quantified as the reciprocals of the price, mass density, and CO$_2$ footprint, respectively. All metrics are normalized on a 0--1 scale for comparative purposes.

Interestingly, all four designs achieve similar load-carrying capacities (Fig. \ref{Fig: Sheet-Part 2}(c)). This result highlights the non-convex nature of the design problem, where multiple local minima exist. Therefore, practical constraints can be incorporated to tailor the non-mechanical performance of designs without compromising mechanical performance, as illustrated in Fig. \ref{Fig: Sheet-Part 2}(d). With this final generalization of the proposed framework to practical constraints, we conclude all example demonstrations.


\section{Conclusions}
\label{Sec: Conclusions}

In this study, we introduced the theory, method, and application of a multimaterial topology optimization approach for programming elastoplastic responses of structures under large deformations. The framework simultaneously determines the optimal structural geometries and material phases, leveraging a mechanics-based finite strain elastoplasticity theory that rigorously ensures isochoric plastic flow. Furthermore, the framework integrates a comprehensive path-dependent sensitivity analysis using the reversed adjoint method and automatic differentiation, enabling gradient-based optimization of design variables.

To demonstrate the effectiveness of the proposed framework, we presented four real-world application examples and uncovered the mechanisms to achieve target behaviors. First, we optimized energy-dissipating dampers, demonstrating the superior energy dissipation performance of the optimized designs compared to intuitive configurations under various loading conditions, including half-cycle, full-cycle, and multiple-cycle scenarios. This example also highlighted the transition from kinematic to isotropic hardening with increasing displacement amplitudes to maximize energy dissipation. Next, we explored the synergistic use of hyperelastic and elastoplastic materials for achieving diverse stiffness--strength interplays of double-clamped beams, showcasing the framework's versatility in handling different material types --- purely hyperelastic, purely elastoplastic, or mixed. In a further example, we extended the optimization to 3D by designing impact-resistant bumpers, while illustrating the capability to handle more than two candidate materials. Finally, we demonstrated multi-stage topology optimization for profiled sheets, focusing on maximizing load-carrying capacity under ultra-large deformations. This example also incorporated practical constraints, including cost, lightweight, and sustainability, bridging the gap between elastoplastic design and real-world considerations.

Across these examples, the proposed framework demonstrated its ability to optimize stiffness, strength, and effective structural toughness for elastoplastic structures in 2D and 3D across various spatial geometries, material types, hardening behaviors, and candidate material combinations. By fully exploiting the potential of elastoplasticity, this framework represents a step forward in designing the next generation of engineering structures. Looking ahead, we aspire to further generalize this framework to account for rate dependence and pressure dependence, which remain an ongoing focus of our research.


\section*{CRediT authorship contribution statement}
\textbf{Yingqi Jia}: Conceptualization, Methodology, Software, Validation, Formal analysis, Investigation, Data curation, Writing--original draft, Writing--review \& editing, Visualization. \textbf{Xiaojia Shelly Zhang}: Conceptualization, Methodology, Investigation, Resources, Writing--original draft, Writing--review \& editing, Supervision, Project administration, Funding acquisition.


\section*{Declaration of competing interest}
The authors declare that they have no known competing financial interests or personal relationships that could have appeared to influence the work reported in this paper.


\section*{Acknowledgments}
Authors X.S.Z. and Y.J. are grateful for the support from the U.S. Defense Advanced Research Projects Agency (DARPA) Award HR0011-24-2-0333. The information provided in this paper is the sole opinion of the authors and does not necessarily reflect the view of the sponsoring agencies.

\section*{Distribution statement}
Approved for public release; distribution is unlimited.

\section*{Data availability}
Data will be made available on request.


\appendix

\section[]{Updating formulae of $\overline{\mathbf{b}}_{n+1}^\texttt{e}$ that enforce the isochoric plastic flow}
\label{Sec: Derivation of Updating Formula of be_bar}

In this section, we introduce the updating formulae of $\mathbf{b}_{n+1}^\texttt{e}$ in \eqref{Updated be_bar} that enforce the isochoric plastic flow ($J_{n+1}^\texttt{p} = 1$). We recall the definitions of $\mathbf{b}_{n+1}^\texttt{e}$ in \eqref{Elastic Part of b} and $\mathbf{C}_{n+1}^\texttt{p}$ in \eqref{Plastic Part of C} and derive
\begin{equation*}
    \mathbf{b}_{n+1}^\texttt{e} = \mathbf{F}_{n+1} (\mathbf{C}_{n+1}^\texttt{p})^{-1} \mathbf{F}_{n+1}^\top.
\end{equation*}
The isochoric plastic flow then requires
\begin{equation} \label{Isochoric Plastic Flow}
    J_{n+1}^\texttt{p} = 1
    \quad \Longleftrightarrow \quad
    \det(\mathbf{b}_{n+1}^\texttt{e}) = J_{n+1}^2
    \quad \Longleftrightarrow \quad
    \det \left( J_{n+1}^{-2/3} \mathbf{b}_{n+1}^\texttt{e} \right) = 1
    \quad \Longleftrightarrow \quad
    \det \left( \overline{\mathbf{b}}_{n+1}^\texttt{e} \right) = 1.
\end{equation}
Additionally, we rewrite $\eqref{Local Governing Equations}_1$ as
\begin{equation*}
    \overline{\mathbf{b}}_{n+1}^\texttt{e}
    = \overline{\mathbf{b}}_{n+1}^\texttt{e,tr}
    - \dfrac{2 \overline{\overline{\mu}}_{n+1}^\texttt{tr}}{\mu} \widehat{\gamma}_{n+1} \mathbf{n}_{n+1}
\end{equation*}
based on Proposition \ref{Compute mu_bar and mu_bar_bar}. This relationship determines the deviatoric part of $\overline{\mathbf{b}}_{n+1}^\texttt{e}$ as
\begin{equation} \label{Deviatoric be_bar}
    \text{dev} \left( \overline{\mathbf{b}}_{n+1}^\texttt{e} \right) = \text{dev} \left( \overline{\mathbf{b}}_{n+1}^\texttt{e,tr} \right) - \dfrac{2 \overline{\overline{\mu}}_{n+1}^\texttt{tr}}{\mu} \widehat{\gamma}_{n+1} \mathbf{n}_{n+1}
\end{equation}
where we remark $\mathbf{n}_{n+1}$ is deviatoric based on its definition in $\eqref{Local Governing Equations}_6$. Based on \eqref{Isochoric Plastic Flow} and \eqref{Deviatoric be_bar}, the enforcement of the isochoric plastic flow amounts to identifying the volumetric part of $\overline{\mathbf{b}}_{n+1}^\texttt{e}$ --- or more specifically, its first invariant, $\mathcal{I}_1 = \text{tr} (\overline{\mathbf{b}}_{n+1}^\texttt{e})$ --- such that $\mathcal{I}_3 = \det(\overline{\mathbf{b}}_{n+1}^\texttt{e}) = 1$ for given $\text{dev} (\overline{\mathbf{b}}_{n+1}^\texttt{e})$. We now determine $\mathcal{I}_1$ by extending the analysis in \citet{simo_associative_1992}.

Note that the invariants $\mathcal{I}_1$, $\mathcal{I}_3$, $\mathcal{J}_2$, and $\mathcal{J}_3$ of a symmetric second-order tensor satisfy
\begin{equation*}
    \dfrac{\mathcal{I}_1^3}{27} - \dfrac{\mathcal{I}_1 \mathcal{J}_2}{3} + \mathcal{J}_3 - \mathcal{I}_3 = 0
\end{equation*}
where $\mathcal{J}_2$ and $\mathcal{J}_3$ are defined as
\begin{equation*}
    \mathcal{J}_2 = \dfrac{1}{2} \left\lVert \text{dev} \left( \overline{\mathbf{b}}_{n+1}^\texttt{e} \right) \right\rVert^2
    \quad \text{and} \quad
    \mathcal{J}_3 = \det \left[ \text{dev} \left( \overline{\mathbf{b}}_{n+1}^\texttt{e} \right) \right].
\end{equation*}
Consequently, a necessary condition of $\mathcal{I}_3 = \det(\overline{\mathbf{b}}_{n+1}^\texttt{e}) = 1$ reads as
\begin{equation} \label{Cubic Equation}
    t^3 + \mathcal{P} t + \mathcal{Q} = 0
\end{equation}
where
\begin{equation*}
    t = \dfrac{\mathcal{I}_1}{3} > 0, \quad
    \mathcal{P} = - \mathcal{J}_2 \leq 0, \quad \text{and} \quad
    \mathcal{Q} = \mathcal{J}_3 - 1.
\end{equation*}
Our objective then boils down to compute the positive real root from the depressed cubic equation in \eqref{Cubic Equation}.

Following a standard procedure, we compute the discriminant of \eqref{Cubic Equation} as
\begin{equation*}
    \Delta = - \left( \dfrac{\mathcal{P}^3}{27} + \dfrac{\mathcal{Q}^2}{4} \right)
\end{equation*}
and discuss the solutions of \eqref{Cubic Equation} based on the sign of $\Delta$ as follows.
\begin{itemize}
    \item The case of $\Delta < 0$. The equation in \eqref{Cubic Equation} has one positive real root and two non-real complex conjugate roots. We should take the real root as
    \begin{equation*}
        t_1 = \left( -\dfrac{\mathcal{Q}}{2} + \sqrt{-\Delta} \right)^{1/3} + \left( -\dfrac{\mathcal{Q}}{2} - \sqrt{-\Delta} \right)^{1/3}.
    \end{equation*}

    \item The case of $\Delta = 0$. The equation in \eqref{Cubic Equation} has a triple root of zero (when $\mathcal{P} = 0$) or the combination of a simple root of $3\mathcal{Q} / \mathcal{P}$ and a double root of $- 3\mathcal{Q} / 2\mathcal{P}$ (when $\mathcal{P} < 0$). We should take the only positive real root as
    \begin{equation*}
        t_2 = \max \left\{ \dfrac{3 \mathcal{Q}}{\mathcal{P}}, - \dfrac{3 \mathcal{Q}}{2 \mathcal{P}} \right\}.
    \end{equation*}

    \item The case of $\Delta > 0$. The equation in \eqref{Cubic Equation} has three distinct real roots expressed in trigonometric forms as shown in \eqref{Trigonometric Solutions}. Based on Descartes' rule of signs, only one solution among $r_1$, $r_2$, and $r_3$ is positive when $\mathcal{Q} \leq 0$, and we should set the solution to \eqref{Cubic Equation} as
    \begin{equation*}
        t_3 = \max\{ r_1, r_2, r_3\}.
    \end{equation*}
    Otherwise ($\mathcal{Q} > 0$), two solutions among $r_1$, $r_2$, and $r_3$ are positive, and we should select the solution such that $\overline{\mathbf{b}}_{n+1}^\texttt{e}$ is positive definite by ensuring
    \begin{equation*}
        \left( \overline{\mathbf{b}}_{n+1}^\texttt{e} \right)_{11} > 0
        \quad \text{and} \quad
        \left( \overline{\mathbf{b}}_{n+1}^\texttt{e} \right)_{11} \left( \overline{\mathbf{b}}_{n+1}^\texttt{e} \right)_{22} - \left( \overline{\mathbf{b}}_{n+1}^\texttt{e} \right)_{12} \left( \overline{\mathbf{b}}_{n+1}^\texttt{e} \right)_{21} > 0
    \end{equation*}
    based on the Sylvester’s criterion.
\end{itemize}

Finally, combining the solutions from all scenarios ($\Delta < 0$, $\Delta = 0$, and $\Delta > 0$), we derive the updating formulae of $\overline{\mathbf{b}}_{n+1}^\texttt{e}$ in \eqref{Updated be_bar}. We remark that \citet{simo_associative_1992} only consider the scenario of $\Delta < 0$ by assuming $\mathcal{J}_2 \ll |1-\mathcal{J}_3|$. Here we remove this assumption and incorporate the scenarios of $\Delta = 0$ and $\Delta > 0$ to account for a broader type of material responses.


\section{Derivation of the second elastoplastic moduli}
\label{Sec: Second Elastoplastic Moduli}

This section derives the second elastoplastic moduli in \eqref{Second Elastoplastic Moduli Final}. Taking the derivative of the second Piola--Kirchhoff stress ($\mathbf{S}_{n+1}$) in \eqref{Second PK Stress} with respect to the Lagrangian strain tensor ($\mathbf{E}_{n+1}$) in $\eqref{Finite Strain Tensors}$ yields
\begin{equation} \label{Second Elastoplastic Moduli Initial}
    \begin{array}{ll}
        \mathbb{C}_{n+1}^\texttt{ep}
        = \dfrac{\partial \mathbf{S}_{n+1}}{\partial \mathbf{E}_{n+1}}
        = 2 \dfrac{\partial \mathbf{S}_{n+1}}{\partial \mathbf{C}_{n+1}}
        = & 2 \dfrac{\partial \varphi^*(\boldsymbol{\tau}_{n+1}^\texttt{vol})}{\partial \mathbf{C}_{n+1}}
        + 2 \dfrac{\partial \varphi^*(\mathbf{s}_{n+1}^\texttt{tr})}{\partial \mathbf{C}_{n+1}}
        - 4 \widehat{\gamma}_{n+1} \varphi^*(\mathbf{n}_{n+1}) \otimes \dfrac{\partial \overline{\overline{\mu}}_{n+1}^\texttt{tr}}{\partial \mathbf{C}_{n+1}} \\[12pt]

        &- 4 \overline{\overline{\mu}}_{n+1}^\texttt{tr} \varphi^* (\mathbf{n}_{n+1}) \otimes \dfrac{\partial \widehat{\gamma}_{n+1}}{\partial \mathbf{C}_{n+1}}
        - 4 \overline{\overline{\mu}}_{n+1}^\texttt{tr} \widehat{\gamma}_{n+1} \dfrac{\partial \varphi^*(\mathbf{n}_{n+1})}{\partial \mathbf{C}_{n+1}}.
    \end{array}
\end{equation}
One immediate next step is to compute the partial derivatives involved in $\mathbb{C}_{n+1}^\texttt{ep}$, which are laid out below.

\subsection[]{The computation of $\partial \varphi^*(\boldsymbol{\tau}_{n+1}^\texttt{vol}) / \partial \mathbf{C}_{n+1}$}

Note that
\begin{equation} \label{Partial tau Partial C Initial}
    \dfrac{\partial \varphi^*(\boldsymbol{\tau}_{n+1}^\texttt{vol})}{\partial \mathbf{C}_{n+1}}
    = \dfrac{\partial \left( J_{n+1} U'_{n+1} \mathbf{C}_{n+1}^{-1} \right)}{\partial \mathbf{C}_{n+1}}
    = \mathbf{C}_{n+1}^{-1} \otimes \dfrac{\partial (J_{n+1} U'_{n+1})}{\partial \mathbf{C}_{n+1}} 
    + (J_{n+1} U'_{n+1}) \dfrac{\partial \mathbf{C}_{n+1}^{-1}}{\partial \mathbf{C}_{n+1}}.
\end{equation}
Substituting
\begin{equation*}
    \dfrac{\partial (J_{n+1} U'_{n+1})}{\partial \mathbf{C}_{n+1}}
    = \dfrac{\partial (J_{n+1} U'_{n+1})}{\partial J_{n+1}}
    \dfrac{\partial J_{n+1}}{\partial \mathbf{C}_{n+1}}
    = (J_{n+1} U'_{n+1})' \left( \dfrac{1}{2} J_{n+1} \mathbf{C}_{n+1}^{-1} \right)
\end{equation*}
and
\begin{equation*}
    \dfrac{\partial \mathbf{C}_{n+1}^{-1}}{\partial \mathbf{C}_{n+1}}
    = - \mathbb{I}_{\mathbf{C}_{n+1}^{-1}}
    \quad \text{with} \quad
    \left( \mathbb{I}_{\mathbf{C}_{n+1}^{-1}} \right)_{ijkl} = \dfrac{1}{2} \left[ \left( C_{n+1}^{-1} \right)_{ik} \left( C_{n+1}^{-1} \right)_{jl} + \left( C_{n+1}^{-1} \right)_{il} \left( C_{n+1}^{-1} \right)_{jk} \right]
\end{equation*}
into \eqref{Partial tau Partial C Initial} renders
\begin{equation} \label{Partial tau Partial C Final}
    \dfrac{\partial \varphi^*(\boldsymbol{\tau}_{n+1}^\texttt{vol})}{\partial \mathbf{C}_{n+1}}
    = \dfrac{1}{2} J_{n+1} (J_{n+1} U'_{n+1})' \mathbf{C}_{n+1}^{-1} \otimes \mathbf{C}_{n+1}^{-1}
    - J_{n+1} U'_{n+1} \mathbb{I}_{\mathbf{C}_{n+1}^{-1}}.
\end{equation}

\subsection[]{The computation of $\partial \varphi^*(\mathbf{s}_{n+1}^\texttt{tr}) / \partial \mathbf{C}_{n+1}$} 

Note that
\begin{equation} \label{be-bar Trial}
    \begin{array}{ll}
        \overline{\mathbf{b}}_{n+1}^\texttt{e,tr}
        &= \overline{\mathbf{f}}_{n+1} \overline{\mathbf{b}}_n^\texttt{e} \overline{\mathbf{f}}_{n+1}^\top
        = \left[ \left( J_{n+1}^f \right)^{-1/3} \mathbf{f}_{n+1} \right]
        \left[ \left( J_{n}^\texttt{e} \right)^{-2/3} \mathbf{b}_n^\texttt{e} \right]
        \left[ \left( J_{n+1}^f \right)^{-1/3} \mathbf{f}_{n+1}^\top \right] \\[10pt]

        &= \left( \dfrac{J_{n+1} J_n^\texttt{e}}{J_n} \right)^{-2/3} \left( \mathbf{F}_{n+1} \mathbf{F}_n^{-1} \right) 
        \left( \mathbf{F}_n^\texttt{e} (\mathbf{F}_n^\texttt{e})^\top \right)
        \left( \mathbf{F}_n^{-\top}  \mathbf{F}_{n+1}^\top \right) \\[10pt]

        & = J_{n+1}^{-2/3} \mathbf{F}_{n+1} \left( \mathbf{F}_n^\texttt{p} \right)^{-1} \left( \mathbf{F}_n^\texttt{p} \right)^{-\top} \mathbf{F}_{n+1}^\top
        = J_{n+1}^{-2/3} \mathbf{F}_{n+1} \left( \mathbf{C}_n^\texttt{p} \right)^{-1} \mathbf{F}_{n+1}^\top
    \end{array}
\end{equation}
due to $J_n = J_n^\texttt{e}$ (assuming isochoric plastic flow) and $J_{n+1} = J_{n+1}^f J_n$. We recall
\begin{equation*}
    \begin{array}{ll}
        \mathbf{s}_{n+1}^\texttt{tr} &= \mu\ \text{dev} \left( \overline{\mathbf{b}}_{n+1}^\texttt{e,tr} \right)
        = \mu J_{n+1}^{-2/3} \text{dev} \left[ \mathbf{F}_{n+1} \left( \mathbf{C}_n^\texttt{p} \right)^{-1} \mathbf{F}_{n+1}^\top \right] \\[10pt]
        &= \mu J_{n+1}^{-2/3} \left\{ \mathbf{F}_{n+1} \left( \mathbf{C}_n^\texttt{p} \right)^{-1} \mathbf{F}_{n+1}^\top
        - \dfrac{1}{3} \text{tr} \left[ \mathbf{F}_{n+1} \left( \mathbf{C}_n^\texttt{p} \right)^{-1} \mathbf{F}_{n+1}^\top \right] \mathbf{I} \right\} \\[10pt]
        &= \mu J_{n+1}^{-2/3} \left\{ \mathbf{F}_{n+1} \left( \mathbf{C}_n^\texttt{p} \right)^{-1} \mathbf{F}_{n+1}^\top
        - \dfrac{1}{3} \left[ \left( \mathbf{C}_n^\texttt{p} \right)^{-1} : \mathbf{C}_{n+1} \right] \mathbf{I} \right\}
    \end{array}
\end{equation*}
and derive
\begin{equation*}
    \varphi^*(\mathbf{s}_{n+1}^\texttt{tr})
    = \mathbf{F}_{n+1}^{-1} \mathbf{s}_{n+1}^\texttt{tr} \mathbf{F}_{n+1}^{-\top}
    = \mu J_{n+1}^{-2/3} \left\{ \left( \mathbf{C}_n^\texttt{p} \right)^{-1}
    - \dfrac{1}{3} \left[ \left( \mathbf{C}_n^\texttt{p} \right)^{-1} : \mathbf{C}_{n+1} \right] \mathbf{C}_{n+1}^{-1} \right\}.
\end{equation*}

Next, we compute
\begin{equation} \label{Partial s-trial-forward Partial C Initial}
    \dfrac{\partial \varphi^*(\mathbf{s}_{n+1}^\texttt{tr})}{\partial \mathbf{C}_{n+1}}
    = \mu \left\{ \left( \mathbf{C}_n^\texttt{p} \right)^{-1}
    - \dfrac{1}{3} \left[ \left( \mathbf{C}_n^\texttt{p} \right)^{-1} : \mathbf{C}_{n+1} \right] \mathbf{C}_{n+1}^{-1} \right\} \dfrac{\partial J_{n+1}^{-2/3}}{\partial \mathbf{C}_{n+1}}
    - \dfrac{\mu}{3} J_{n+1}^{-2/3} \dfrac{\partial}{\partial \mathbf{C}_{n+1}} \left\{ \left[ \left( \mathbf{C}_n^\texttt{p} \right)^{-1} : \mathbf{C}_{n+1} \right] \mathbf{C}_{n+1}^{-1} \right\}.
\end{equation}
Substituting
\begin{equation} \label{Partial J^(-2/3) Partial C}
    \dfrac{\partial J_{n+1}^{-2/3}}{\partial \mathbf{C}_{n+1}}
    = - \dfrac{2}{3} J_{n+1}^{-5/3} \dfrac{\partial J_{n+1}}{\partial \mathbf{C}_{n+1}}
    = - \dfrac{2}{3} J_{n+1}^{-5/3} \left( \dfrac{1}{2} J_{n+1} \mathbf{C}_{n+1}^{-1} \right)
    = - \dfrac{1}{3} J_{n+1}^{-2/3} \mathbf{C}_{n+1}^{-1}
\end{equation}
and
\begin{equation*}
    \begin{array}{ll}
        \dfrac{\partial}{\partial \mathbf{C}_{n+1}} \left\{ \left[ \left( \mathbf{C}_n^\texttt{p} \right)^{-1} : \mathbf{C}_{n+1} \right] \mathbf{C}_{n+1}^{-1} \right\}
        &= \mathbf{C}_{n+1}^{-1} \otimes \dfrac{\partial \left[ \left( \mathbf{C}_n^\texttt{p} \right)^{-1} : \mathbf{C}_{n+1} \right]}{\partial \mathbf{C}_{n+1}}
        + \left[ \left( \mathbf{C}_n^\texttt{p} \right)^{-1} : \mathbf{C}_{n+1} \right] \dfrac{\partial \mathbf{C}_{n+1}^{-1}}{\partial \mathbf{C}_{n+1}} \\[12pt]
        & = \mathbf{C}_{n+1}^{-1} \otimes \left( \mathbf{C}_n^\texttt{p} \right)^{-1}
        - \left[ \left( \mathbf{C}_n^\texttt{p} \right)^{-1} : \mathbf{C}_{n+1} \right] \mathbb{I}_{\mathbf{C}_{n+1}^{-1}}
    \end{array}
\end{equation*}
into \eqref{Partial s-trial-forward Partial C Initial} renders
\begin{equation*}
    \begin{array}{ll}
        \dfrac{\partial \varphi^*(\mathbf{s}_{n+1}^\texttt{tr})}{\partial \mathbf{C}_{n+1}}
        = &- \dfrac{\mu}{3} J_{n+1}^{-2/3} \left\{ \left( \mathbf{C}_n^\texttt{p} \right)^{-1}
        - \dfrac{1}{3} \left[ \left( \mathbf{C}_n^\texttt{p} \right)^{-1} : \mathbf{C}_{n+1} \right] \mathbf{C}_{n+1}^{-1} \right\} \otimes \mathbf{C}_{n+1}^{-1} \\[12pt]

         & - \dfrac{\mu}{3} J_{n+1}^{-2/3}
        \left\{ \mathbf{C}_{n+1}^{-1} \otimes \left( \mathbf{C}_n^\texttt{p} \right)^{-1}
        - \left[ \left( \mathbf{C}_n^\texttt{p} \right)^{-1} : \mathbf{C}_{n+1} \right] \mathbb{I}_{\mathbf{C}_{n+1}^{-1}} \right\}.
    \end{array}
\end{equation*}
Regrouping terms yields
\begin{equation} \label{Partial s-trial-forward Partial C Middle}
    \dfrac{\partial \varphi^*(\mathbf{s}_{n+1}^\texttt{tr})}{\partial \mathbf{C}_{n+1}}
    = - \dfrac{\mu}{3} J_{n+1}^{-2/3} \left\{ \left( \mathbf{C}_n^\texttt{p} \right)^{-1} \otimes \mathbf{C}_{n+1}^{-1} + \mathbf{C}_{n+1}^{-1} \otimes \left( \mathbf{C}_n^\texttt{p} \right)^{-1}
    - \left[ \left( \mathbf{C}_n^\texttt{p} \right)^{-1} : \mathbf{C}_{n+1} \right]
    \left( \mathbb{I}_{\mathbf{C}_{n+1}^{-1}} + \dfrac{1}{3} \mathbf{C}_{n+1}^{-1} \otimes \mathbf{C}_{n+1}^{-1} \right) \right\}.
\end{equation}

Based on \eqref{be-bar Trial}, we derive $J_{n+1}^{-2/3} \left( \mathbf{C}_n^\texttt{p} \right)^{-1} = \varphi^* \left( \overline{\mathbf{b}}_{n+1}^\texttt{tr} \right)$ and further compute
\begin{equation*}
    \left\{ \begin{array}{l}
        \mu \varphi^* \left( \overline{\mathbf{b}}_{n+1}^\texttt{tr} \right) \otimes \mathbf{C}_{n+1}^{-1}
        = \varphi^* \left( \mathbf{s}_{n+1}^\texttt{tr} \right) \otimes \mathbf{C}_{n+1}^{-1}
        + \overline{\mu}_{n+1}^\texttt{tr} \mathbf{C}_{n+1}^{-1} \otimes \mathbf{C}_{n+1}^{-1}, \\[12pt]
        
        \mu \mathbf{C}_{n+1}^{-1} \otimes \varphi^* \left( \overline{\mathbf{b}}_{n+1}^\texttt{tr} \right)
        = \mathbf{C}_{n+1}^{-1} \otimes \varphi^* \left( \mathbf{s}_{n+1}^\texttt{tr} \right)
        + \overline{\mu}_{n+1}^\texttt{tr} \mathbf{C}_{n+1}^{-1} \otimes \mathbf{C}_{n+1}^{-1}, \\[12pt]
        
        \dfrac{\mu}{3} \varphi^* \left( \overline{\mathbf{b}}_{n+1}^\texttt{tr} \right) : \mathbf{C}_{n+1} = \dfrac{\mu}{3} \text{tr} \left( \overline{\mathbf{b}}_{n+1}^\texttt{tr} \right) = \overline{\mu}_{n+1}^\texttt{tr}.
    \end{array} \right.
\end{equation*}
We then reduce \eqref{Partial s-trial-forward Partial C Middle} to
\begin{equation} \label{Partial s-trial-forward Partial C Final}
    \dfrac{\partial \varphi^*(\mathbf{s}_{n+1}^\texttt{tr})}{\partial \mathbf{C}_{n+1}}
    = \overline{\mu}_{n+1}^\texttt{tr} \left( \mathbb{I}_{\mathbf{C}_{n+1}^{-1}} - \dfrac{1}{3} \mathbf{C}_{n+1}^{-1} \otimes \mathbf{C}_{n+1}^{-1} \right)
    - \dfrac{1}{3} \left[ \varphi^* \left( \mathbf{s}_{n+1}^\texttt{tr} \right) \otimes \mathbf{C}_{n+1}^{-1}
    + \mathbf{C}_{n+1}^{-1} \otimes \varphi^* \left( \mathbf{s}_{n+1}^\texttt{tr} \right) \right].
\end{equation}

\subsection[]{The computation of $\partial \overline{\overline{\mu}}_{n+1}^\texttt{tr} / \partial \mathbf{C}_{n+1}$} 

Note that
\begin{equation} \label{Partial mu-bar-bar Partial C}
    \dfrac{\partial \overline{\overline{\mu}}_{n+1}^\texttt{tr}}{\partial \mathbf{C}_{n+1}} = \dfrac{\partial \overline{\mu}_{n+1}^\texttt{tr}}{\partial \mathbf{C}_{n+1}}
    - \dfrac{1}{3} \dfrac{\partial}{\partial \mathbf{C}_{n+1}}
    \text{tr} \left( \overline{\mathbf{f}}_{n+1} \overline{\boldsymbol{\beta}}_n \overline{\mathbf{f}}_{n+1}^\top \right).
\end{equation}
According to \eqref{be-bar Trial}, we derive
\begin{equation*}
    \overline{\mu}_{n+1}^\texttt{tr}
    = \dfrac{\mu}{3} \text{tr} \left( \overline{\mathbf{b}}_{n+1}^\texttt{e,tr} \right)
    = \dfrac{\mu}{3} J_{n+1}^{-2/3} \text{tr} \left( \mathbf{F}_{n+1} \left( \mathbf{C}_n^\texttt{p} \right)^{-1} \mathbf{F}_{n+1}^\top \right)
    = \dfrac{\mu}{3} J_{n+1}^{-2/3} \left( \mathbf{C}_n^\texttt{p} \right)^{-1} : \mathbf{C}_{n+1}.
\end{equation*}
Taking the derivative of $\overline{\mu}_{n+1}^\texttt{tr}$ yields
\begin{equation} \label{Partial mu-bar Partial C}
    \begin{array}{ll}
        \dfrac{\partial \overline{\mu}_{n+1}^\texttt{tr}}{\partial \mathbf{C}_{n+1}}
        & = \dfrac{\mu}{3} \left[ \left( \mathbf{C}_n^\texttt{p} \right)^{-1} : \mathbf{C}_{n+1} \right] \dfrac{\partial J_{n+1}^{-2/3}}{\partial \mathbf{C}_{n+1}}
        + \dfrac{\mu}{3} J_{n+1}^{-2/3} \dfrac{\partial}{\partial \mathbf{C}_{n+1}} \left[ \left( \mathbf{C}_n^\texttt{p} \right)^{-1} : \mathbf{C}_{n+1} \right] \\[12pt]
        & = \dfrac{\mu}{3} \left[ \left( \mathbf{C}_n^\texttt{p} \right)^{-1} : \mathbf{C}_{n+1} \right]
        \left( - \dfrac{1}{3} J_{n+1}^{-2/3} \mathbf{C}_{n+1}^{-1} \right)
        + \dfrac{\mu}{3} J_{n+1}^{-2/3} \left( \mathbf{C}_n^\texttt{p} \right)^{-1} \\[12pt]
        &= \dfrac{\mu}{3} J_{n+1}^{-2/3}
        \left\{ \left( \mathbf{C}_n^\texttt{p} \right)^{-1}
        - \dfrac{1}{3} \left[ \left( \mathbf{C}_n^\texttt{p} \right)^{-1} : \mathbf{C}_{n+1} \right] \mathbf{C}_{n+1}^{-1} \right\}
    \end{array}
\end{equation}
based on \eqref{Partial J^(-2/3) Partial C}.

Additionally, we rewrite
\begin{equation} \label{J^(-2/3) F Gamma F}
    \begin{array}{ll}
        \overline{\mathbf{f}}_{n+1} \overline{\boldsymbol{\beta}}_n \overline{\mathbf{f}}_{n+1}^\top &= \left[ \left( J_{n+1}^f \right)^{-1/3} \mathbf{f}_{n+1} \right] \left( J_n^{-2/3} \boldsymbol{\beta}_n \right) \left[ \left( J_{n+1}^f \right)^{-1/3} \mathbf{f}_{n+1}^\top \right] \\[12pt]

        & = J_{n+1}^{-2/3} \left( \mathbf{F}_{n+1} \mathbf{F}_n^{-1} \right)
        \boldsymbol{\beta}_n \left( \mathbf{F}_n^{-\top} \mathbf{F}_{n+1}^\top \right) = J_{n+1}^{-2/3} \mathbf{F}_{n+1} \boldsymbol{\Gamma}_n \mathbf{F}_{n+1}^\top
    \end{array}
\end{equation}
where we define $\boldsymbol{\Gamma}_n = \mathbf{F}_n^{-1} \boldsymbol{\beta}_n \mathbf{F}_n^{-\top}$. Next, we derive
\begin{equation*}
    \text{tr}\left( \overline{\mathbf{f}}_{n+1} \overline{\boldsymbol{\beta}}_n \overline{\mathbf{f}}_{n+1}^\top \right)
    = J_{n+1}^{-2/3} \boldsymbol{\Gamma}_n : \mathbf{C}_{n+1}
\end{equation*}
and
\begin{equation} \label{Partial f-beta-f Partial C}
    \dfrac{\partial}{\partial \mathbf{C}_{n+1}}
    \text{tr} \left( \overline{\mathbf{f}}_{n+1} \overline{\boldsymbol{\beta}}_n \overline{\mathbf{f}}_{n+1}^\top \right) = 
    J_{n+1}^{-2/3}
    \left[ \boldsymbol{\Gamma}_n
    - \dfrac{1}{3} \left( \boldsymbol{\Gamma}_n : \mathbf{C}_{n+1} \right) \mathbf{C}_{n+1}^{-1} \right]
\end{equation}
in a similar vein as \eqref{Partial mu-bar Partial C}. Substituting \eqref{Partial mu-bar Partial C} and \eqref{Partial f-beta-f Partial C} into \eqref{Partial mu-bar-bar Partial C} renders
\begin{equation*}
    \dfrac{\partial \overline{\overline{\mu}}_{n+1}^\texttt{tr}}{\partial \mathbf{C}_{n+1}} = 
    \dfrac{\mu}{3} J_{n+1}^{-2/3}
    \left\{ \left( \mathbf{C}_n^\texttt{p} \right)^{-1}
    - \dfrac{1}{3} \left[ \left( \mathbf{C}_n^\texttt{p} \right)^{-1} : \mathbf{C}_{n+1} \right] \mathbf{C}_{n+1}^{-1} \right\}
    - \dfrac{1}{3} J_{n+1}^{-2/3}
    \left[ \boldsymbol{\Gamma}_n
    - \dfrac{1}{3} \left( \boldsymbol{\Gamma}_n : \mathbf{C}_{n+1} \right) \mathbf{C}_{n+1}^{-1} \right].
\end{equation*}

For simplification, we further apply the push-forward operator ($\varphi_*$) and compute
\begin{equation*}
    \begin{array}{ll}
        \varphi_* \left( \dfrac{\partial \overline{\overline{\mu}}_{n+1}^\texttt{tr}}{\partial \mathbf{C}_{n+1}} \right)
        &= \dfrac{\mu}{3} \left[ \overline{\mathbf{f}}_{n+1} \overline{\mathbf{b}}_n^\texttt{e} \overline{\mathbf{f}}_{n+1}^\top
        - \dfrac{1}{3} \text{tr} \left( \overline{\mathbf{f}}_{n+1} \overline{\mathbf{b}}_n^\texttt{e} \overline{\mathbf{f}}_{n+1}^\top \right) \mathbf{I} \right]
        - \dfrac{1}{3} \left[ \overline{\mathbf{f}}_{n+1} \overline{\boldsymbol{\beta}}_n \overline{\mathbf{f}}_{n+1}^\top - \dfrac{1}{3} \text{tr} \left( \overline{\mathbf{f}}_{n+1} \overline{\boldsymbol{\beta}}_n \overline{\mathbf{f}}_{n+1}^\top \right) \mathbf{I} \right] \\[12pt]
         & = \dfrac{\mu}{3} \text{dev} \left( \overline{\mathbf{f}}_{n+1} \overline{\mathbf{b}}_n^\texttt{e} \overline{\mathbf{f}}_{n+1}^\top \right)
         - \dfrac{1}{3} \text{dev} \left( \overline{\mathbf{f}}_{n+1} \overline{\boldsymbol{\beta}}_n \overline{\mathbf{f}}_{n+1}^\top \right)
         = \dfrac{\mu}{3} \text{dev} \left( \overline{\mathbf{b}}_{n+1}^\texttt{e,tr} \right)
         - \dfrac{1}{3} \text{dev} \left( \overline{\boldsymbol{\beta}}_{n+1}^\texttt{tr} \right) \\[12pt]
         & = \dfrac{1}{3} \left[ \mathbf{s}_{n+1}^\texttt{tr}
         - \text{dev} \left( \overline{\boldsymbol{\beta}}_{n+1}^\texttt{tr} \right) \right]
         = \dfrac{1}{3} \boldsymbol{\xi}_{n+1}^\texttt{tr}
    \end{array}
\end{equation*}
by using \eqref{be-bar Trial} and \eqref{J^(-2/3) F Gamma F}. Finally, we derive
\begin{equation} \label{Partial mu-bar-bar Partial C Final}
    \dfrac{\partial \overline{\overline{\mu}}_{n+1}^\texttt{tr}}{\partial \mathbf{C}_{n+1}} = \dfrac{1}{3} \varphi^* \left( \boldsymbol{\xi}_{n+1}^\texttt{tr} \right). 
\end{equation}

\subsection[]{The computation of $\partial \widehat{\gamma}_{n+1} / \partial \mathbf{C}_{n+1}$} 

Recall the consistency parameter, $\widehat{\gamma}_{n+1}$, is governed by the algebraic equation, $\mathcal{G}(\widehat{\gamma}_{n+1}) = 0$ in \eqref{Function of gamma}. Taking the derivatives on \eqref{Function of gamma} yields
\begin{equation} \label{Partial gamma Partial C Initial}
    \dfrac{\partial \lVert \boldsymbol{\xi}_{n+1}^\texttt{tr} \rVert}{\partial \mathbf{C}_{n+1}}
    - 2 \left( 1 + \dfrac{h}{3\mu} \right) \widehat{\gamma}_{n+1}
    \dfrac{\partial \overline{\overline{\mu}}_{n+1}^\texttt{tr}}{\partial \mathbf{C}_{n+1}}
    - 2 \overline{\overline{\mu}}_{n+1}^\texttt{tr}
    \left( 1 + \dfrac{h}{3\mu} \right) \dfrac{\partial \widehat{\gamma}_{n+1}}{\partial \mathbf{C}_{n+1}}
    - \dfrac{2 k'}{3} \dfrac{\partial \widehat{\gamma}_{n+1}}{\partial \mathbf{C}_{n+1}} = 0,
\end{equation}
and one apparent next step is to compute $\partial \lVert \boldsymbol{\xi}_{n+1}^\texttt{tr} \rVert / \partial \mathbf{C}_{n+1}$. To this end, we consider
\begin{equation*}
    \lVert \boldsymbol{\xi}_{n+1}^\texttt{tr} \rVert^2 = \boldsymbol{\xi}_{n+1}^\texttt{tr}:\boldsymbol{\xi}_{n+1}^\texttt{tr}
    = \varphi_* \left[ \varphi^* (\boldsymbol{\xi}_{n+1}^\texttt{tr}) \right] : \varphi_* \left[ \varphi^* (\boldsymbol{\xi}_{n+1}^\texttt{tr}) \right]
    = \left[ \mathbf{C}_{n+1} \varphi^* (\boldsymbol{\xi}_{n+1}^\texttt{tr}) \right] : \left[ \mathbf{C}_{n+1} \varphi^* (\boldsymbol{\xi}_{n+1}^\texttt{tr}) \right]
\end{equation*}
and derive
\begin{equation} \label{Partial xi-trial-norm-square Partial C}
    \dfrac{\partial \lVert \boldsymbol{\xi}_{n+1}^\texttt{tr} \rVert^2}{\partial \mathbf{C}_{n+1}}
    = 2 \left[ \mathbf{C}_{n+1} \varphi^* (\boldsymbol{\xi}_{n+1}^\texttt{tr}) \mathbf{C}_{n+1} \right] : \dfrac{\partial \varphi^* (\boldsymbol{\xi}_{n+1}^\texttt{tr})}{\partial \mathbf{C}_{n+1}}
    + 2 \varphi^* (\boldsymbol{\xi}_{n+1}^\texttt{tr}) \mathbf{C}_{n+1} \varphi^* (\boldsymbol{\xi}_{n+1}^\texttt{tr}).
\end{equation}

Following the derivation of $\partial \varphi^* (\mathbf{s}_{n+1}^\texttt{tr}) / \partial \mathbf{C}_{n+1}$ from \eqref{Partial s-trial-forward Partial C Final}, we compute
\begin{equation*}
    \dfrac{\partial \varphi^*(\overline{\boldsymbol{\beta}}_{n+1}^\texttt{tr})}{\partial \mathbf{C}_{n+1}}
    = \dfrac{1}{3}\text{tr} \left( \overline{\mathbf{f}}_{n+1} \overline{\boldsymbol{\beta}}_n \overline{\mathbf{f}}_{n+1}^\top \right)
    \left( \mathbb{I}_{\mathbf{C}_{n+1}^{-1}} - \dfrac{1}{3} \mathbf{C}_{n+1}^{-1} \otimes \mathbf{C}_{n+1}^{-1} \right)
    - \dfrac{1}{3} \left[ \varphi^* \left( \overline{\boldsymbol{\beta}}_{n+1}^\texttt{tr} \right) \otimes \mathbf{C}_{n+1}^{-1}
    + \mathbf{C}_{n+1}^{-1} \otimes \varphi^* \left( \overline{\boldsymbol{\beta}}_{n+1}^\texttt{tr} \right) \right]
\end{equation*}
and further derive
\begin{equation} \label{Partial xi-trial-forward Partial C}
    \begin{array}{ll}
        \dfrac{\partial \varphi^*(\boldsymbol{\xi}_{n+1}^\texttt{tr})}{\partial \mathbf{C}_{n+1}} &= \dfrac{\partial \varphi^*(\mathbf{s}_{n+1}^\texttt{tr})}{\partial \mathbf{C}_{n+1}} -\dfrac{\partial \varphi^*(\overline{\boldsymbol{\beta}}_{n+1}^\texttt{tr})}{\partial \mathbf{C}_{n+1}} \\[12pt]
        &= \overline{\overline{\mu}}_{n+1}^\texttt{tr} \left( \mathbb{I}_{\mathbf{C}_{n+1}^{-1}} - \dfrac{1}{3} \mathbf{C}_{n+1}^{-1} \otimes \mathbf{C}_{n+1}^{-1} \right) - \dfrac{1}{3} \left[ \varphi^* \left( \boldsymbol{\xi}_{n+1}^\texttt{tr} \right) \otimes \mathbf{C}_{n+1}^{-1}
        + \mathbf{C}_{n+1}^{-1} \otimes \varphi^* \left( \boldsymbol{\xi}_{n+1}^\texttt{tr} \right) \right].
    \end{array}
\end{equation}
Substituting \eqref{Partial xi-trial-forward Partial C} into \eqref{Partial xi-trial-norm-square Partial C} and applying the push-forward operator ($\varphi_*$) yield
\begin{equation*}
    \varphi_* \left( \dfrac{\partial \lVert \boldsymbol{\xi}_{n+1}^\texttt{tr} \rVert^2}{\partial \mathbf{C}_{n+1}} \right)
    = 2 \overline{\overline{\mu}}_{n+1}^\texttt{tr} \boldsymbol{\xi}_{n+1}^\texttt{tr}
    - \dfrac{2}{3} \lVert \boldsymbol{\xi}_{n+1}^\texttt{tr} \rVert^2 \mathbf{I}
    + 2 \left( \boldsymbol{\xi}_{n+1}^\texttt{tr} \right)^2.
\end{equation*}
Considering
\begin{equation*}
    \dfrac{\partial \lVert \boldsymbol{\xi}_{n+1}^\texttt{tr} \rVert^2}{\partial \mathbf{C}_{n+1}}
    = \dfrac{\partial \lVert \boldsymbol{\xi}_{n+1}^\texttt{tr} \rVert^2}{\partial \lVert \boldsymbol{\xi}_{n+1}^\texttt{tr} \rVert}
    \dfrac{\partial \lVert \boldsymbol{\xi}_{n+1}^\texttt{tr} \rVert}{\partial \mathbf{C}_{n+1}}
    = 2 \lVert \boldsymbol{\xi}_{n+1}^\texttt{tr} \rVert \dfrac{\partial \lVert \boldsymbol{\xi}_{n+1}^\texttt{tr} \rVert}{\partial \mathbf{C}_{n+1}}
    \quad \Rightarrow \quad
    \dfrac{\partial \lVert \boldsymbol{\xi}_{n+1}^\texttt{tr} \rVert}{\partial \mathbf{C}_{n+1}} 
    = \dfrac{1}{2 \lVert \boldsymbol{\xi}_{n+1}^\texttt{tr} \rVert}
    \dfrac{\partial \lVert \boldsymbol{\xi}_{n+1}^\texttt{tr} \rVert^2}{\partial \mathbf{C}_{n+1}},
\end{equation*}
we derive
\begin{equation*}
    \varphi_* \left( \dfrac{\partial \lVert \boldsymbol{\xi}_{n+1}^\texttt{tr} \rVert}{\partial \mathbf{C}_{n+1}} \right)
    = \overline{\overline{\mu}}_{n+1}^\texttt{tr} \mathbf{n}_{n+1}
    - \dfrac{1}{3} \lVert \boldsymbol{\xi}_{n+1}^\texttt{tr} \rVert \left( \mathbf{n}_{n+1} : \mathbf{n}_{n+1} \right) \mathbf{I}
    + \lVert \boldsymbol{\xi}_{n+1}^\texttt{tr} \rVert \mathbf{n}_{n+1}^2.
\end{equation*}

We remark that the term $\varphi_* \left( \partial \lVert \boldsymbol{\xi}_{n+1}^\texttt{tr} \rVert / \partial \mathbf{C}_{n+1} \right)$ is deviatoric by noticing
\begin{equation*}
    \text{tr} \left[ \varphi_* \left( \dfrac{\partial \lVert \boldsymbol{\xi}_{n+1}^\texttt{tr} \rVert}{\partial \mathbf{C}_{n+1}} \right) \right]
    = \overline{\overline{\mu}}_{n+1}^\texttt{tr} \text{tr}(\mathbf{n}_{n+1})
    - \dfrac{1}{3} \lVert \boldsymbol{\xi}_{n+1}^\texttt{tr} \rVert \mathbf{n}_{n+1} : \mathbf{n}_{n+1} \text{tr}(\mathbf{I})
    + \lVert \boldsymbol{\xi}_{n+1}^\texttt{tr} \rVert \text{tr} (\mathbf{n}_{n+1}^2) = 0.
\end{equation*}
Consequently, we can rewrite
\begin{equation*}
    \varphi_* \left( \dfrac{\partial \lVert \boldsymbol{\xi}_{n+1}^\texttt{tr} \rVert}{\partial \mathbf{C}_{n+1}} \right)
    = \text{dev} \left[ \varphi_* \left( \dfrac{\partial \lVert \boldsymbol{\xi}_{n+1}^\texttt{tr} \rVert}{\partial \mathbf{C}_{n+1}} \right) \right]
    = \overline{\overline{\mu}}_{n+1}^\texttt{tr} \mathbf{n}_{n+1} + \lVert \boldsymbol{\xi}_{n+1}^\texttt{tr} \rVert \text{dev} (\mathbf{n}_{n+1}^2),
\end{equation*}
which renders
\begin{equation} \label{Partial xi-trial-norm Partial C}
    \dfrac{\partial \lVert \boldsymbol{\xi}_{n+1}^\texttt{tr} \rVert}{\partial \mathbf{C}_{n+1}}
    = \varphi^* \left[ \overline{\overline{\mu}}_{n+1}^\texttt{tr} \mathbf{n}_{n+1} + \lVert \boldsymbol{\xi}_{n+1}^\texttt{tr} \rVert \text{dev} (\mathbf{n}_{n+1}^2) \right]
    = \overline{\overline{\mu}}_{n+1}^\texttt{tr} \varphi^*(\mathbf{n}_{n+1})
    + \lVert \boldsymbol{\xi}_{n+1}^\texttt{tr} \rVert \varphi^*\left[ \text{dev} (\mathbf{n}_{n+1}^2) \right].
\end{equation}
Finally, substituting \eqref{Partial mu-bar-bar Partial C Final} and \eqref{Partial xi-trial-norm Partial C} into \eqref{Partial gamma Partial C Initial} yields
\begin{equation} \label{Partial gamma Partial C Final}
    \begin{array}{ll}
        \dfrac{\partial \widehat{\gamma}_{n+1}}{\partial \mathbf{C}_{n+1}} &= \dfrac{1}{2 c_0 \overline{\overline{\mu}}_{n+1}^\texttt{tr}} 
        \left[ \dfrac{\partial \lVert \boldsymbol{\xi}_{n+1}^\texttt{tr} \rVert}{\partial \mathbf{C}_{n+1}}
        - 2 \left( 1 + \dfrac{h}{3\mu} \right) \widehat{\gamma}_{n+1}
        \dfrac{\partial \overline{\overline{\mu}}_{n+1}^\texttt{tr}}{\partial \mathbf{C}_{n+1}} \right] \\[12pt]
        
        &= \dfrac{1}{2 c_0 \overline{\overline{\mu}}_{n+1}^\texttt{tr}}  
        \left\{ \overline{\overline{\mu}}_{n+1}^\texttt{tr} \varphi^*(\mathbf{n}_{n+1})
        + \lVert \boldsymbol{\xi}_{n+1}^\texttt{tr} \rVert \varphi^*\left[ \text{dev} (\mathbf{n}_{n+1}^2) \right]
        - 2 \left( 1 + \dfrac{h}{3\mu} \right) \widehat{\gamma}_{n+1}
        \left[ \dfrac{1}{3} \varphi^* \left( \boldsymbol{\xi}_{n+1}^\texttt{tr} \right) \right] \right\} \\[12pt]

        &= \dfrac{1}{2 c_0 \overline{\overline{\mu}}_{n+1}^\texttt{tr}} 
        \left\{ \left[ \overline{\overline{\mu}}_{n+1}^\texttt{tr} - \dfrac{2}{3} \left( 1 + \dfrac{h}{3\mu} \right) \widehat{\gamma}_{n+1} \lVert \boldsymbol{\xi}_{n+1}^\texttt{tr} \rVert \right] \varphi^* (\mathbf{n}_{n+1})
        + \lVert \boldsymbol{\xi}_{n+1}^\texttt{tr} \rVert \varphi^*\left[ \text{dev} (\mathbf{n}_{n+1}^2) \right] \right\}
    \end{array}
\end{equation}
where we define
\begin{equation*}
    c_0 =  1 + \dfrac{h}{3\mu} + \dfrac{k'}{3 \overline{\overline{\mu}}_{n+1}^\texttt{tr}}.
\end{equation*} 

\subsection[]{The computation of $\partial \varphi^*(\mathbf{n}_{n+1}) / \partial \mathbf{C}_{n+1}$} 

Based on the chain rules, we write out
\begin{equation} \label{Partial n-forward Partial C Initial}
    \dfrac{\partial \varphi^*(\mathbf{n}_{n+1})}{\partial \mathbf{C}_{n+1}} = \dfrac{\partial}{\partial \mathbf{C}_{n+1}} \left[ \dfrac{\varphi^* (\boldsymbol{\xi}_{n+1}^\texttt{tr})}{\lVert \boldsymbol{\xi}_{n+1}^\texttt{tr} \rVert} \right]
    = \dfrac{1}{\lVert \boldsymbol{\xi}_{n+1}^\texttt{tr} \rVert^2}
    \left[ \lVert \boldsymbol{\xi}_{n+1}^\texttt{tr} \rVert \dfrac{\partial \varphi^* (\boldsymbol{\xi}_{n+1}^\texttt{tr})}{\partial \mathbf{C}_{n+1}}
    - \varphi^*(\boldsymbol{\xi}_{n+1}^\texttt{tr}) \otimes \dfrac{\partial \lVert \boldsymbol{\xi}_{n+1}^\texttt{tr} \rVert}{\partial \mathbf{C}_{n+1}} \right].
\end{equation}
Substituting \eqref{Partial xi-trial-norm Partial C} and \eqref{Partial xi-trial-forward Partial C} into \eqref{Partial n-forward Partial C Initial} renders
\begin{equation} \label{Partial n-forward Partial C Final}
    \begin{array}{ll}
        \dfrac{\partial \varphi^*(\mathbf{n}_{n+1})}{\partial \mathbf{C}_{n+1}}
        = &\dfrac{\overline{\overline{\mu}}_{n+1}^\texttt{tr}}{\lVert \boldsymbol{\xi}_{n+1}^\texttt{tr} \rVert}
        \left( \mathbb{I}_{\mathbf{C}_{n+1}^{-1}} - \dfrac{1}{3} \mathbf{C}_{n+1}^{-1} \otimes \mathbf{C}_{n+1}^{-1} \right) - \dfrac{1}{3} \left[ \varphi^* \left( \mathbf{n}_{n+1} \right) \otimes \mathbf{C}_{n+1}^{-1}
        + \mathbf{C}_{n+1}^{-1} \otimes \varphi^* \left( \mathbf{n}_{n+1} \right) \right] \\[12pt]
        
        &- \dfrac{\overline{\overline{\mu}}_{n+1}^\texttt{tr}}{\lVert \boldsymbol{\xi}_{n+1}^\texttt{tr} \rVert}
        \varphi^*(\mathbf{n}_{n+1}) \otimes
        \varphi^*(\mathbf{n}_{n+1})
        - \varphi^*(\mathbf{n}_{n+1}) \otimes \varphi^*\left[ \text{dev} (\mathbf{n}_{n+1}^2) \right].
    \end{array}
\end{equation}

\subsection[]{Complete expression of the second elastoplastic moduli}

Substituting $\partial \varphi^*(\boldsymbol{\tau}_{n+1}^\texttt{vol}) / \partial \mathbf{C}_{n+1}$ from \eqref{Partial tau Partial C Final}, $\partial \varphi^*(\mathbf{s}_{n+1}^\texttt{tr}) / \partial \mathbf{C}_{n+1}$ from \eqref{Partial s-trial-forward Partial C Final}, $\partial \overline{\overline{\mu}}_{n+1}^\texttt{tr} / \partial \mathbf{C}_{n+1}$ from \eqref{Partial mu-bar-bar Partial C Final}, $\partial \widehat{\gamma}_{n+1} / \partial \mathbf{C}_{n+1}$ from \eqref{Partial gamma Partial C Final}, and $\partial \varphi^*(\mathbf{n}_{n+1}) / \partial \mathbf{C}_{n+1}$ from \eqref{Partial n-forward Partial C Final} into \eqref{Second Elastoplastic Moduli Initial}, we finally derive
\begin{equation*}
    \begin{array}{ll}
        \mathbb{C}_{n+1}^\texttt{ep} = &J_{n+1} (J_{n+1} U'_{n+1})' \mathbf{C}_{n+1}^{-1} \otimes \mathbf{C}_{n+1}^{-1}
        - 2 J_{n+1} U'_{n+1} \mathbb{I}_{\mathbf{C}_{n+1}^{-1}} \\[12pt]
        
        &+ 2 \overline{\mu}_{n+1}^\texttt{tr} \left( \mathbb{I}_{\mathbf{C}_{n+1}^{-1}} - \dfrac{1}{3} \mathbf{C}_{n+1}^{-1} \otimes \mathbf{C}_{n+1}^{-1} \right)
        - \dfrac{2}{3} \left[ \varphi^* \left( \mathbf{s}_{n+1}^\texttt{tr} \right) \otimes \mathbf{C}_{n+1}^{-1}
        + \mathbf{C}_{n+1}^{-1} \otimes \varphi^* \left( \mathbf{s}_{n+1}^\texttt{tr} \right) \right] \\[12pt]

        &- \dfrac{4}{3} \widehat{\gamma}_{n+1} \lVert \boldsymbol{\xi}_{n+1}^\texttt{tr} \rVert \varphi^*(\mathbf{n}_{n+1}) \otimes \varphi^* (\mathbf{n}_{n+1}) \\[12pt]

        &- \dfrac{2}{c_0} \varphi^* (\mathbf{n}_{n+1}) \otimes
        \left\{ \left[ \overline{\overline{\mu}}_{n+1}^\texttt{tr} - \dfrac{2}{3} \left( 1 + \dfrac{h}{3\mu} \right) \widehat{\gamma}_{n+1} \lVert \boldsymbol{\xi}_{n+1}^\texttt{tr} \rVert \right] \varphi^* (\mathbf{n}_{n+1})
        + \lVert \boldsymbol{\xi}_{n+1}^\texttt{tr} \rVert \varphi^*\left[ \text{dev} (\mathbf{n}_{n+1}^2) \right] \right\} \\[12pt]

        &- \dfrac{4 \left( \overline{\overline{\mu}}_{n+1}^\texttt{tr} \right)^2 \widehat{\gamma}_{n+1}}{\lVert \boldsymbol{\xi}_{n+1}^\texttt{tr} \rVert}
        \left( \mathbb{I}_{\mathbf{C}_{n+1}^{-1}} - \dfrac{1}{3} \mathbf{C}_{n+1}^{-1} \otimes \mathbf{C}_{n+1}^{-1} \right) \\[12pt] 
        
        &+ \dfrac{4 \overline{\overline{\mu}}_{n+1}^\texttt{tr} \widehat{\gamma}_{n+1}}{3} \left[ \varphi^* \left( \mathbf{n}_{n+1} \right) \otimes \mathbf{C}_{n+1}^{-1}
        + \mathbf{C}_{n+1}^{-1} \otimes \varphi^* \left( \mathbf{n}_{n+1} \right) \right] \\[12pt] 

        &+ \dfrac{4 \left( \overline{\overline{\mu}}_{n+1}^\texttt{tr} \right)^2 \widehat{\gamma}_{n+1}}{\lVert \boldsymbol{\xi}_{n+1}^\texttt{tr} \rVert}
        \varphi^*(\mathbf{n}_{n+1}) \otimes
        \varphi^*(\mathbf{n}_{n+1})
        
        + 4 \overline{\overline{\mu}}_{n+1}^\texttt{tr} \widehat{\gamma}_{n+1} \varphi^*(\mathbf{n}_{n+1}) \otimes \varphi^*\left[ \text{dev} (\mathbf{n}_{n+1}^2) \right].
    \end{array}
\end{equation*}
Regrouping terms renders
\begin{equation*}
    \begin{array}{ll}
        \mathbb{C}_{n+1}^\texttt{ep} = &\left( 2 \overline{\mu}_{n+1}^\texttt{tr} - 2 c_1 \overline{\overline{\mu}}_{n+1}^\texttt{tr} - 2 J_{n+1} U'_{n+1} \right) \mathbb{I}_{\mathbf{C}_{n+1}^{-1}}
        
        + \left[ J_{n+1} (J_{n+1} U'_{n+1})' - \dfrac{2 \overline{\mu}_{n+1}^\texttt{tr}}{3} + \dfrac{2 c_1 \overline{\overline{\mu}}_{n+1}^\texttt{tr}}{3} \right] \mathbf{C}_{n+1}^{-1} \otimes \mathbf{C}_{n+1}^{-1} \\[12pt]
        
        &- \dfrac{2}{3} \left[ \varphi^*(\mathbf{s}_{n+1}^\texttt{tr}) \otimes \mathbf{C}_{n+1}^{-1} + \mathbf{C}_{n+1}^{-1} \otimes \varphi^*(\mathbf{s}_{n+1}^\texttt{tr}) \right]

        + \dfrac{2 c_1}{3} \left[ \varphi^*(\boldsymbol{\xi}_{n+1}^\texttt{tr}) \otimes \mathbf{C}_{n+1}^{-1} + \mathbf{C}_{n+1}^{-1} \otimes \varphi^*(\boldsymbol{\xi}_{n+1}^\texttt{tr}) \right] \\[12pt]

        &- c_3 \varphi^*(\mathbf{n}_{n+1}) \otimes \varphi^*(\mathbf{n}_{n+1})

        - c_4 \varphi^*(\mathbf{n}_{n+1}) \otimes \varphi^*\left[ \text{dev} (\mathbf{n}_{n+1}^2) \right].
    \end{array}
\end{equation*}
with $c_1$, $c_3$, and $c_4$ defined in \eqref{Parameters in Elasticity}. To ensure the major symmetry of $\mathbb{C}_{n+1}^\texttt{ep}$ for computational benefits, we manually symmetrize its last term by following \citet{simo_framework_1988} and derive the expression in \eqref{Second Elastoplastic Moduli Final}.


\section{Verification of FEA implementation for finite strain elastoplasticity}
\label{Sec: FEA Verification}

In this section, we compare the FEA and semi-analytical solutions for a 3D column under uniaxial cyclic loadings. Through this comparison, we aim to demonstrate that the proposed formulae for updating $\overline{\mathbf{b}}^\texttt{e}$ in $\eqref{Updated be_bar}$ ensure the isochoric plastic flow ($J^\texttt{p}=1$), and consequently, the FEA solution matches the semi-analytical solution exactly.

\subsection{Construction of the semi-analytical solution}

For the simplicity of the semi-analytical solution, we assume no hardening occurs ($k(\alpha) \equiv \sigma_y$ and $h=0$) when the material yields. At any time, the deformation gradients can be expressed as
\begin{equation*}
    \mathbf{F} = \left[ \begin{array}{ccc}
        \lambda & & \\
        & \lambda^\texttt{l} & \\
        & & \lambda^\texttt{l}
    \end{array} \right], \quad
    \mathbf{F}^\texttt{p} = \left[ \begin{array}{ccc}
        \lambda^\texttt{p} & & \\
        & \dfrac{1}{\sqrt{\lambda^\texttt{p}}} & \\
        & & \dfrac{1}{\sqrt{\lambda^\texttt{p}}}
    \end{array} \right], \quad \text{and} \quad
    \mathbf{F}^\texttt{e} = \left[ \begin{array}{ccc}
        \dfrac{\lambda}{\lambda^\texttt{p}} & & \\
        & \lambda^\texttt{l} \sqrt{\lambda^\texttt{p}} & \\
        & & \lambda^\texttt{l} \sqrt{\lambda^\texttt{p}}
    \end{array} \right].
\end{equation*}
Here, the variables $\lambda$, $\lambda^\texttt{l}$, and $\lambda^\texttt{p}$ represent the total applied stretch, total lateral stretch, and plastic stretch (Fig. \ref{Fig: Isochoric Plastic Flow}(a)), respectively. Note that the above deformation gradients satisfy the requirements of $\mathbf{F} = \mathbf{F}^\texttt{e} \mathbf{F}^\texttt{p}$ and $\det(\mathbf{F}^\texttt{p}) = 1$. Based on the elastic deformation gradient ($\mathbf{F}^\texttt{e}$), we compute
\begin{equation*}
    \text{dev} (\overline{\mathbf{b}}^\texttt{e}) = \dfrac{1}{3} \left[ \lambda^\texttt{p} \left( \dfrac{\lambda^\texttt{l}}{\lambda} \right)^{2/3}
    - \dfrac{1}{(\lambda^\texttt{p})^2} \left( \dfrac{\lambda}{\lambda^\texttt{l}} \right)^{4/3} \right]
    \left[ \begin{array}{ccc}
        -2 & & \\
        & 1 & \\
        & & 1
    \end{array} \right]
\end{equation*}
and derive the components of the Kirchhoff stress tensor ($\boldsymbol{\tau}$) as
\begin{equation*}
    \left\{ \begin{array}{l}
        \tau_{11} = J U'(J) - \dfrac{2\mu}{3} \left[ \lambda^\texttt{p} \left( \dfrac{\lambda^\texttt{l}}{\lambda} \right)^{2/3} - \dfrac{1}{(\lambda^\texttt{p})^2} \left( \dfrac{\lambda}{\lambda^\texttt{l}} \right)^{4/3} \right], \\[12pt]
        \tau_{22} = \tau_{33} = J U'(J) + \dfrac{\mu}{3} \left[ \lambda^\texttt{p} \left( \dfrac{\lambda^\texttt{l}}{\lambda} \right)^{2/3} - \dfrac{1}{(\lambda^\texttt{p})^2} \left( \dfrac{\lambda}{\lambda^\texttt{l}} \right)^{4/3}\right],
    \end{array} \right.
\end{equation*}
where $J = \lambda (\lambda^\texttt{l})^2$ is the total volume change. 

\begin{figure}[!htbp]
    \centering
    \includegraphics[width=17cm]{Isochoric_Plastic_Flow.pdf}
    \caption{Comparison between the FEA and semi-analytical solutions. (a) Tested mechanical setups. (b) Kirchhoff stress--stretch ($\tau_{11}$--$\lambda_1$) curves. (c) Evolution of the total lateral stretch ($\lambda^\texttt{l}$). (d) Evolution of the determinant of $\overline{\mathbf{b}}^\texttt{e}$.}
    \label{Fig: Isochoric Plastic Flow}
\end{figure}

During the elastic loading/unloading process, the Kirchhoff stress conforms to $\tau_{22} = \tau_{33} = 0$ for given $\lambda$ and $\lambda_p$. We can then solve $\lambda_l$ from
\begin{equation} \label{Solve for lambda_l for Elastic Process}
   \dfrac{\kappa}{2} \left[ \lambda^2 (\lambda^\texttt{l})^4-1 \right] + \dfrac{\mu}{3} \left[ \lambda^\texttt{p} \left( \dfrac{\lambda^\texttt{l}}{\lambda} \right)^{2/3} - \dfrac{1}{(\lambda^\texttt{p})^2} \left( \dfrac{\lambda}{\lambda^\texttt{l}} \right)^{4/3}\right] = 0
\end{equation}
with the choice of $U(J)$ in \eqref{Volumetric and Deviatoric Engery}. When the material yields, the Kirchhoff stress satisfies $\tau_{11} = \pm \sigma_y$ (the sign depends on the loading direction) and $\tau_{22} = \tau_{33} = 0$ for given $\lambda$. We can then analytically compute $\lambda^\texttt{l}$ as
\begin{equation} \label{Solve for lambda_l for Plastic Process}
    \lambda^\texttt{l} = \dfrac{1}{\sqrt{\lambda}} \left( 1 \pm \dfrac{2 \sigma_y}{3 \kappa} \right)^{1/4}
\end{equation}
and solve for $\lambda^\texttt{p}$ from
\begin{equation} \label{Solve for lambda_p for Plastic Process}
    \left( \dfrac{\lambda^\texttt{l}}{\lambda} \right)^{2/3} (\lambda^\texttt{p})^3 \pm \dfrac{\sigma_y}{\mu} (\lambda^\texttt{p})^2 - \left( \dfrac{\lambda}{\lambda^\texttt{l}} \right)^{4/3} = 0.
\end{equation}
Note that the expressions in \eqref{Solve for lambda_l for Elastic Process}, \eqref{Solve for lambda_l for Plastic Process}, and \eqref{Solve for lambda_p for Plastic Process} construct the semi-analytical solutions for predicting the finite strain elastoplastic responses of materials under uniaxial cyclic loadings. 

\subsection{Comparison between the FEA and semi-analytical solutions}
\label{Sec: Comparison with Analytical Solution}

Next, we compare the semi-analytical solutions described above with the FEA predictions based on the theories in Section \ref{Sec: Finite Strain Elastoplasticity}. As a reference, we also include FEA predictions derived from the classical theories in \citet{simo_framework_1988-1, simo_framework_1988} that utilize $\eqref{Local Governing Equations}_1$ to update $\overline{\mathbf{b}}^\texttt{e}$. Without loss of generality, we adopt dummy material constants, $E=1$ MPa, $\nu=0.3$, and $\sigma_y=0.2$ MPa, where $E$ is the initial Young's modulus and $\nu$ is the initial Poisson's ratio.

The comparison results are shown in Figs. \ref{Fig: Isochoric Plastic Flow}(b)--(d), which illustrate the Kirchhoff stress component ($\tau_{11}$), lateral stretch ($\lambda^\texttt{l}$), and the determinant of $\overline{\mathbf{b}}^\texttt{e}$, respectively. Based on the comparison, the FEA predictions in this work ensure isochoric plastic flow and demonstrate good agreement with the semi-analytical solutions, as evidenced by the small absolute 2-norm errors: $1.56 \times 10^{-10}$ for $\tau_{11}$, $7.76 \times 10^{-11}$ for $\lambda^\texttt{l}$, and $7.95 \times 10^{-10}$ for $\text{det}(\overline{\mathbf{b}}^\texttt{e})$. Conversely, the classical FEA initially aligns well during the loading stage but fails to ensure isochoric plastic flow. It also leads to significant deviations from the analytical solutions after unloading occurs. This limitation renders the classical FEA unsuitable for structures experiencing non-monotonic loadings, such as the dampers under cyclic loadings discussed in Section \ref{Sec: Dampers}.


\section{Convergence, precision, and computational time of FEA}
\label{Sec: FEA Convergence}

In this section, we examine the computational aspects of FEA, including convergence, precision, and computational time. Using the optimized damper under multiple-cycle loadings in Fig. \ref{Fig: Damper-Part 3} as an example, we present the required Newton iterations for convergence in Fig. \ref{Fig: Computational Cost}(a). For most load steps in FEA, whether during loading or unloading, convergence is achieved in fewer than 10 Newton iterations. However, at critical load steps corresponding to transitions between loading and unloading, it requires 10--25 iterations to converge.

To further illustrate the convergence and precision of FEA, we highlight four representative load steps in Fig. \ref{Fig: Computational Cost}(a) and show the evolution of the relative residual (right-hand side of $\eqref{Global Governing Equation}_1$) at these steps in Fig. \ref{Fig: Computational Cost}(b). For most steps, such as steps 60 and 100, the residual converges quadratically due to the algorithmic elastoplastic moduli in \eqref{Second Algorithmic Tangent Moduli Final} and the proposed interpolation schemes in \eqref{Interpolated First PK Stress} and \eqref{Interpolated Left-hand Side}. For transitional steps, like steps 66 and 156, a line search (Algorithm \ref{Algorithm of Line Search}) is activated in the initial iterations to reduce the residual. Once the residual reduction trend is established, quadratic convergence resumes. Notably, as shown in Fig. \ref{Fig: Computational Cost}(b), the residuals for all load steps eventually converge to the prescribed tolerance (set to $10^{-8}$ here).

\begin{figure}[!htbp]
    \centering
    \includegraphics[width=16cm]{Computational_Cost.pdf}
    \caption{Convergence and precision of FEA. (a) Required Newton iterations for convergence at each load step in FEA. (b) Evolution of the relative residuals at representative load steps marked in (a).}
    \label{Fig: Computational Cost}
\end{figure}

After discussing convergence and precision, we now focus on the computational time of FEA. For representativeness, we analyze the most complex design case in each example presented in Section \ref{Sec: Sample Examples}. Computational performance is tested on a workstation equipped with an AMD Ryzen Threadripper PRO 3995WX CPU featuring 64 cores and 256 GB of memory. The results are summarized in Table \ref{Table: Computational Time}, where we provide the finite element, degree of freedom (DOF), and parallel process counts, along with the average computational time per Newton iteration in the final load step.

It is evident that the 2D designs (damper and beam) require less than 0.1 seconds per Newton iteration due to the parallel computing capabilities provided by FEniTop \citep{jia_fenitop_2024}. The profiled sheet takes around 2 seconds per iteration, while the bumper design takes around 12 seconds per iteration due to a large DOF count. This high computational demand for the bumper can be reduced by exploiting symmetry and analyzing one quarter of the domain.

\begin{table}[!htbp]
    \caption{Computational time of FEA for representative design cases}
    \label{Table: Computational Time}
    \centering
    \footnotesize
    \begin{tabular}{lllll}
        \hline
        \textbf{Optimized designs} & \textbf{Element counts} & \textbf{DOF counts} & \textbf{Process counts} & \textbf{Time per iteration} \\
        \hline
        Damper in Fig. \ref{Fig: Damper-Part 3} & 15,000 & 30,502 & 8 & 0.07 s\\
        Beam (Dsg. 4) in Fig. \ref{Fig: Beam} & 14,400 & 29,402 & 8 & 0.06 s\\
        Bumper in Fig. \ref{Fig: Bumper-Part 3} & 138,400 & 469,491 & 16 & 11.82 s\\
        Profiled sheet (Dsg. 4) in Fig. \ref{Fig: Sheet-Part 2} & 28,800 & 110,715 & 16 & 1.94 s\\
        \hline
    \end{tabular}
\end{table}


\section{History-dependent sensitivity analysis and verification}
\label{Sec: Sensitivity Analysis and Verification}

In this section, we present the sensitivity analysis for finite strain elastoplasticity. Following the blueprint provided in \citet{jia_multimaterial_2025}, we begin by defining the global vectors for the design and state variables as well as the residuals. After that, we derive the sensitivity expressions with the reversed adjoint method and automatic differentiation. Finally, we verify the derived sensitivity expressions through a comparison with the forward finite difference scheme.

\subsection{Global vectors of design variables, state variables, and residuals}

In this subsection, we define the residual vectors used in the sensitivity analysis. Based on the local and global governing equations in Section \ref{Sec: Finite Strain Elastoplasticity}, we identify five groups of independent state variables as $\overline{\mathbf{b}}_1^\texttt{e}, \ldots, \overline{\mathbf{b}}_N^\texttt{e}$, $\overline{\boldsymbol{\beta}}_1, \ldots, \overline{\boldsymbol{\beta}}_N$, $\alpha_1, \ldots, \alpha_N$, $\widehat{\gamma}_1, \ldots, \widehat{\gamma}_N$, and $\mathbf{u}_1, \ldots, \mathbf{u}_N$. Correspondingly, we define five groups of residual expressions as
\begin{equation} \label{Residual Expresssions}
    \left\{ \begin{array}{l}
        \mathbf{r}_{n+1}^{\overline{\mathbf{b}}^\texttt{e}}(\{\overline{\zeta}\}, \overline{\mathbf{b}}_{n+1}^\texttt{e}, \widehat{\gamma}_{n+1}, \mathbf{u}_{n+1}, \overline{\mathbf{b}}_n^\texttt{e}, \overline{\boldsymbol{\beta}}_n, \mathbf{u}_n) = \overline{\mathbf{b}}_{n+1}^\texttt{e} - \text{dev} \left( \overline{\mathbf{b}}_{n+1}^\texttt{e,tr} \right)
        + \dfrac{2 \overline{\overline{\mu}}_{n+1}^\texttt{tr}}{\mu} \widehat{\gamma}_{n+1} \mathbf{n}_{n+1} - \dfrac{1}{3} \mathcal{I}_1 \mathbf{I} = \mathbf{0}, \\[12pt]

        \mathbf{r}_{n+1}^{\overline{\boldsymbol{\beta}}}(\{\overline{\zeta}\}, \overline{\boldsymbol{\beta}}_{n+1}, \widehat{\gamma}_{n+1}, \mathbf{u}_{n+1}, \overline{\mathbf{b}}_n^\texttt{e}, \overline{\boldsymbol{\beta}}_n, \mathbf{u}_n)
        = \overline{\boldsymbol{\beta}}_{n+1} - \overline{\boldsymbol{\beta}}_{n+1}^\texttt{tr}
        - \dfrac{2 h \overline{\overline{\mu}}_{n+1}^\texttt{tr}}{3 \mu} \widehat{\gamma}_{n+1} \mathbf{n}_{n+1} = \mathbf{0}, \\[12pt]

        r_{n+1}^{\alpha}(\alpha_{n+1}, \widehat{\gamma}_{n+1}, \alpha_n) = \alpha_{n+1} - \alpha_{n+1}^\texttt{tr} - \sqrt{\dfrac{2}{3}} \widehat{\gamma}_{n+1} = 0, \\[12pt]
        
        r_{n+1}^{\widehat{\gamma}}(\{\overline{\zeta}\}, \widehat{\gamma}_{n+1}, \mathbf{u}_{n+1}, \overline{\mathbf{b}}_n^\texttt{e}, \overline{\boldsymbol{\beta}}_n, \alpha_n, \mathbf{u}_n) \\[12pt]
        \quad = \widehat{\gamma}_{n+1} \mathcal{G}(\widehat{\gamma}_{n+1})
        = \widehat{\gamma}_{n+1} \left[ \lVert \boldsymbol{\xi}_{n+1}^\texttt{tr} \rVert
        - 2 \overline{\overline{\mu}}_{n+1}^\texttt{tr}
        \left( 1 + \dfrac{h}{3\mu} \right) \widehat{\gamma}_{n+1}
        - \sqrt{\dfrac{2}{3}} k \left( \alpha_{n+1}^\texttt{tr} + \sqrt{\dfrac{2}{3}} \widehat{\gamma}_{n+1} \right) \right] = 0, \\[12pt]
        
        \displaystyle r_{n+1}^\mathbf{u}(\{\overline{\zeta}\}, \widehat{\gamma}_{n+1}, \mathbf{u}_{n+1}, \overline{\mathbf{b}}_n^\texttt{e}, \overline{\boldsymbol{\beta}}_n, \mathbf{u}_n) = \int_{\Omega_0} \mathbf{P}_{n+1} : \nabla \mathbf{v}\ \text{d} \mathbf{X} - \int_{\Omega_0} \overline{\mathbf{q}}_{n+1} \cdot \mathbf{v}\ \text{d} \mathbf{X} 
        - \int_{\partial \Omega_0^\mathcal{N}} \overline{\mathbf{t}}_{n+1} \cdot \mathbf{v}\ \text{d} \mathbf{X} = 0,
    \end{array} \right.
\end{equation}
where $\{\overline{\zeta}\} = \{ \overline{\rho}, \overline{\xi}_1, \ldots, \overline{\xi}_{N^\texttt{mat}} \}$ is a collection of the physical design variables. We remark that, in contrast to \citet{jia_multimaterial_2025}, here we treat $\widehat{\gamma}_1, \ldots, \widehat{\gamma}_N$ as independent state variables and incorporate its residual in $\eqref{Residual Expresssions}_4$. This treatment allows us to consider nonlinear isotropic hardening laws such as \eqref{Nonlinear Isotropic Hardening Law} where $\widehat{\gamma}_1, \ldots, \widehat{\gamma}_N$ typically have no analytical expressions.

In the context of FEA, we reformulate \eqref{Residual Expresssions} into global residual vectors expressed as 
\begin{equation*}
    \left\{ \begin{array}{l}
        \mathbf{R}_{n+1}^{\overline{\mathbf{b}}^\texttt{e}}
        (\{\overline{\boldsymbol{\zeta}}\}, \mathbf{V}_{n+1}^{\overline{\mathbf{b}}^\texttt{e}}, \mathbf{V}_{n+1}^{\widehat{\gamma}}, \mathbf{U}_{n+1}, \mathbf{V}_n^{\overline{\mathbf{b}}^\texttt{e}}, \mathbf{V}_n^{\overline{\boldsymbol{\beta}}}, \mathbf{U}_n) = \mathbf{0}, \\[8pt]
        
        \mathbf{R}_{n+1}^{\overline{\boldsymbol{\beta}}}(\{\overline{\boldsymbol{\zeta}}\}, \mathbf{V}_{n+1}^{\overline{\boldsymbol{\beta}}}, \mathbf{V}_{n+1}^{\widehat{\gamma}}, \mathbf{U}_{n+1}, \mathbf{V}_n^{\overline{\mathbf{b}}^\texttt{e}}, \mathbf{V}_n^{\overline{\boldsymbol{\beta}}}, \mathbf{U}_n) = \mathbf{0}, \\[8pt]
        
        \mathbf{R}_{n+1}^{\alpha}(\mathbf{V}_{n+1}^\alpha, \mathbf{V}_{n+1}^{\widehat{\gamma}}, \mathbf{V}_n^\alpha) = \mathbf{0}, \\[8pt]
        
        \mathbf{R}_{n+1}^{\widehat{\gamma}}(\{\overline{\boldsymbol{\zeta}}\}, \mathbf{V}_{n+1}^{\widehat{\gamma}}, \mathbf{U}_{n+1}, \mathbf{V}_n^{\overline{\mathbf{b}}^\texttt{e}}, \mathbf{V}_n^{\overline{\boldsymbol{\beta}}}, \mathbf{V}_n^\alpha, \mathbf{U}_n) = \mathbf{0}, \\[8pt]
        
        \mathbf{R}_{n+1}^\mathbf{u}(\{\overline{\boldsymbol{\zeta}}\}, \mathbf{V}_{n+1}^{\widehat{\gamma}}, \mathbf{U}_{n+1}, \mathbf{V}_n^{\overline{\mathbf{b}}^\texttt{e}}, \mathbf{V}_n^{\overline{\boldsymbol{\beta}}}, \mathbf{U}_n) = \mathbf{0}.
    \end{array} \right.
\end{equation*}
In these expressions, we define $\{\overline{\boldsymbol{\zeta}}\} = \{ \overline{\boldsymbol{\rho}}, \overline{\boldsymbol{\xi}}_1, \ldots, \overline{\boldsymbol{\xi}}_{N^\texttt{mat}} \}$ as a collection of global vectors of physical design variables, where $\overline{\zeta}_e$ is the physical design variable of finite element $e$. Additionally, the variable $\mathbf{V}_n^{\overline{\mathbf{b}}^\texttt{e}}$ for $n=1,\ldots,N$ is the global vector of $\overline{\mathbf{b}}_n^\texttt{e}$ defined as
\begin{equation*}
    \mathbf{V}_n^{\overline{\mathbf{b}}^\texttt{e}} = \left[ \left(\overline{\mathbf{b}}_{n,1}^\texttt{e,v}\right)^\top, \ldots, \left(\overline{\mathbf{b}}_{n,N^\texttt{pt}}^\texttt{e,v}\right)^\top \right]^\top
\end{equation*}
where $\overline{\mathbf{b}}_{n,i}^\texttt{e,v}$ for $i = 1, \ldots, N^\texttt{pt}$ is the vector form of $\overline{\mathbf{b}}_{n,i}^\texttt{e}$ expressed as
\begin{equation*}
    \overline{\mathbf{b}}_{n,i}^\texttt{e,v} = \left[ \left( \overline{\mathbf{b}}_{n,i}^\texttt{e} \right)_{11}, \left( \overline{\mathbf{b}}_{n,i}^\texttt{e} \right)_{12}, \left( \overline{\mathbf{b}}_{n,i}^\texttt{e} \right)_{13}, \left( \overline{\mathbf{b}}_{n,i}^\texttt{e} \right)_{22}, \left( \overline{\mathbf{b}}_{n,i}^\texttt{e} \right)_{23}, \left( \overline{\mathbf{b}}_{n,i}^\texttt{e} \right)_{33} \right]^\top.
\end{equation*}
Here, subscript $i$ is the count of integration points used in FEA, and $N^\texttt{pt}$ is the number of all integration points. Similarly, the variable $\mathbf{V}_n^{\overline{\boldsymbol{\beta}}}$ for $n=1,\ldots,N$ is the global vector of $\overline{\boldsymbol{\beta}}_n$ defined as
\begin{equation*}
    \mathbf{V}_n^{\overline{\boldsymbol{\beta}}} = \left[ \left(\overline{\boldsymbol{\beta}}_{n,1}^\texttt{v}\right)^\top, \ldots, \left(\overline{\boldsymbol{\beta}}_{n,N^\texttt{pt}}^\texttt{v}\right)^\top \right]^\top
\end{equation*}
where $\overline{\boldsymbol{\beta}}_{n,i}^\texttt{v}$ for $i = 1, \ldots, N^\texttt{pt}$ is the vector form of $\overline{\boldsymbol{\beta}}_{n,i}$ expressed as
\begin{equation*}
    \overline{\boldsymbol{\beta}}_{n,i}^\texttt{v} = \left[ \left( \overline{\boldsymbol{\beta}}_{n,i} \right)_{11},
    \left( \overline{\boldsymbol{\beta}}_{n,i} \right)_{12},
    \left( \overline{\boldsymbol{\beta}}_{n,i} \right)_{13},
    \left( \overline{\boldsymbol{\beta}}_{n,i} \right)_{22},
    \left( \overline{\boldsymbol{\beta}}_{n,i} \right)_{23},
    \left( \overline{\boldsymbol{\beta}}_{n,i} \right)_{33} \right]^\top.
\end{equation*}
The variables $\mathbf{V}_n^\alpha$ and $\mathbf{V}_n^{\widehat{\gamma}}$ for $n=1,\ldots,N$ are the global vectors of $\alpha_n$ and $\widehat{\gamma}_n$, respectively, which are defined as
\begin{equation*}
    \mathbf{V}_n^\alpha = \left[ \alpha_{n,1}, \ldots, \alpha_{n,N^\texttt{pt}} \right]^\top
    \quad \text{and} \quad
    \mathbf{V}_n^{\widehat{\gamma}} = \left[ \widehat{\gamma}_{n,1}, \ldots, \widehat{\gamma}_{n,N^\texttt{pt}} \right]^\top.
\end{equation*}
Finally, the variable $\mathbf{U}_n$ for $n=1,\ldots,N$ is the global vector of $\mathbf{u}_n$ defined as
\begin{equation*}
    \mathbf{U}_n = \left[ (\mathbf{u}_n)_{1,x}, (\mathbf{u}_n)_{1,y}, (\mathbf{u}_n)_{1,z}, \ldots, (\mathbf{u}_n)_{N^\texttt{node},x}, (\mathbf{u}_n)_{N^\texttt{node},y}, (\mathbf{u}_n)_{N^\texttt{node},z} \right]^\top
\end{equation*}
where subscripts $x$, $y$, and $z$ represent the three components of $(\mathbf{u}_n)_j$ for $j=1,\ldots,N^\texttt{node}$, and $N^\texttt{node}$ is the number of nodes in FEA.

\subsection{Sensitivity analysis}
\label{Sec: Sensitivity Analysis}

After defining the global vectors of design variables ($\overline{\boldsymbol{\rho}}$ and $\overline{\boldsymbol{\xi}}_1, \ldots, \overline{\boldsymbol{\xi}}_{N^\texttt{mat}}$), state variables ($\mathbf{V}_n^{\overline{\mathbf{b}}^\texttt{e}}$, $\mathbf{V}_n^{\overline{\boldsymbol{\beta}}}$, $\mathbf{V}_n^\alpha$, $\mathbf{V}_n^{\widehat{\gamma}}$, and $\mathbf{U}_n$), and residuals ($\mathbf{R}_{n+1}^{\overline{\mathbf{b}}^\texttt{e}}$, $\mathbf{R}_{n+1}^{\overline{\boldsymbol{\beta}}}$, $\mathbf{R}_{n+1}^{\alpha}$, $\mathbf{R}_{n+1}^{\widehat{\gamma}}$, and $\mathbf{R}_{n+1}^\mathbf{u}$), we proceed with the sensitivity analysis for finite strain elastoplasticity. We consider a general function
\begin{equation*}
    \mathcal{F}(\overline{\boldsymbol{\rho}}, \overline{\boldsymbol{\xi}}_1, \ldots, \overline{\boldsymbol{\xi}}_{N^\texttt{mat}}; \mathbf{V}_1^{\overline{\mathbf{b}}^\texttt{e}}, \ldots, \mathbf{V}_N^{\overline{\mathbf{b}}^\texttt{e}}; \mathbf{V}_1^{\overline{\boldsymbol{\beta}}}, \ldots, \mathbf{V}_N^{\overline{\boldsymbol{\beta}}}; \mathbf{V}_1^\alpha, \ldots, \mathbf{V}_n^\alpha; \mathbf{V}_N^{\widehat{\gamma}}, \ldots, \mathbf{V}_1^{\widehat{\gamma}}; \mathbf{U}_1, \ldots, \mathbf{U}_N)
\end{equation*}
and write out its Lagrangian expression as
\begin{equation*}
    \widehat{\mathcal{F}} = \mathcal{F} + \sum_{n=1}^N \left( \boldsymbol{\lambda}_n^{\overline{\mathbf{b}}^\texttt{e}} \cdot \mathbf{R}_n^{\overline{\mathbf{b}}^\texttt{e}}
    + \boldsymbol{\lambda}_n^{\overline{\boldsymbol{\beta}}} \cdot \mathbf{R}_n^{\overline{\boldsymbol{\beta}}}
    + \boldsymbol{\lambda}_n^\alpha \cdot \mathbf{R}_n^\alpha
    + \boldsymbol{\lambda}_n^{\widehat{\gamma}} \cdot \mathbf{R}_n^{\widehat{\gamma}}
    + \boldsymbol{\lambda}_n^\mathbf{u} \cdot \mathbf{R}_n^\mathbf{u} \right)
    = \mathcal{F} + \sum_{n=1}^N \boldsymbol{\lambda}_n \cdot \mathbf{R}_n,
\end{equation*}
where we define
\begin{equation*}
    \boldsymbol{\lambda}_n = \left[
    \left( \boldsymbol{\lambda}_n^{\overline{\mathbf{b}}^\texttt{e}} \right)^\top,
    \left( \boldsymbol{\lambda}_n^{\overline{\boldsymbol{\beta}}} \right)^\top,
    \left( \boldsymbol{\lambda}_n^\alpha \right)^\top,
    \left( \boldsymbol{\lambda}_n^{\widehat{\gamma}} \right)^\top,
    \left( \boldsymbol{\lambda}_n^\mathbf{u} \right)^\top
    \right]^\top
    \quad \text{for} \quad
    n = 1,\ldots,N
\end{equation*}
and
\begin{equation*}
    \mathbf{R}_n = \left[
    \left( \mathbf{R}_n^{\overline{\mathbf{b}}^\texttt{e}} \right)^\top,
    \left( \mathbf{R}_n^{\overline{\boldsymbol{\beta}}} \right)^\top,
    \left( \mathbf{R}_n^\alpha \right)^\top,
    \left( \mathbf{R}_n^{\widehat{\gamma}} \right)^\top,
    \left( \mathbf{R}_n^\mathbf{u} \right)^\top
    \right]^\top
    \quad \text{for} \quad
    n = 1,\ldots,N.
\end{equation*}
The variables $\boldsymbol{\lambda}_n^{\overline{\mathbf{b}}^\texttt{e}}$, $\boldsymbol{\lambda}_n^{\overline{\boldsymbol{\beta}}}$, $\boldsymbol{\lambda}_n^\alpha$, $\boldsymbol{\lambda}_n^{\widehat{\gamma}}$, and $\boldsymbol{\lambda}_n^\mathbf{u}$ for $n=1,\ldots,N$ are the adjoint vectors that take arbitrary real values for now.

Taking the derivative of $\widehat{\mathcal{F}}$ with respect to $\overline{\boldsymbol{\zeta}} \in \{ \overline{\boldsymbol{\rho}}, \overline{\boldsymbol{\xi}}_1, \ldots, \overline{\boldsymbol{\xi}}_{N^\texttt{mat}} \}$ derives
\begin{equation*}
    \dfrac{\text{d} \widehat{\mathcal{F}}}{\text{d} \overline{\boldsymbol{\zeta}}}
    = \dfrac{\partial \mathcal{F}}{\partial \overline{\boldsymbol{\zeta}}}
    + \sum_{n=1}^N \left( \dfrac{\partial \mathbf{V}_n}{\partial \overline{\boldsymbol{\zeta}}} \right)^\top \dfrac{\partial \mathcal{F}}{\partial \mathbf{V}_n}
    + \sum_{n=1}^N \left[ \left( \dfrac{\partial \mathbf{R}_n}{\partial \overline{\boldsymbol{\zeta}}} \right)^\top
    + \left( \dfrac{\partial \mathbf{V}_{n-1}}{\partial \overline{\boldsymbol{\zeta}}} \right)^\top \left( \dfrac{\partial \mathbf{R}_n}{\partial \mathbf{V}_{n-1}} \right)^\top
    + \left( \dfrac{\partial \mathbf{V}_n}{\partial \overline{\boldsymbol{\zeta}}} \right)^\top \left( \dfrac{\partial \mathbf{R}_n}{\partial \mathbf{V}_n} \right)^\top
    \right] \boldsymbol{\lambda}_n.
\end{equation*}
Here we define
\begin{equation*}
    \mathbf{V}_n = \left[
    \left( \mathbf{V}_n^{\overline{\mathbf{b}}^\texttt{e}} \right)^\top,
    \left( \mathbf{V}_n^{\overline{\boldsymbol{\beta}}} \right)^\top,
    \left( \mathbf{V}_n^\alpha \right)^\top,
    \left( \mathbf{V}_n^{\widehat{\gamma}} \right)^\top,
    \left( \mathbf{U}_n \right)^\top
    \right]^\top
    \quad \text{for} \quad
    n = 0,\ldots,N,
\end{equation*}
and note that $\mathbf{V}_0$ represents the initial conditions.

Regrouping the terms in $\text{d} \widehat{\mathcal{F}}/\text{d} \overline{\boldsymbol{\zeta}}$ renders
\begin{equation} \label{Regrouping Terms for Sensitivity}
    \begin{array}{ll}
        \dfrac{\text{d} \widehat{\mathcal{F}}}{\text{d} \overline{\boldsymbol{\zeta}}}
        = &\displaystyle \dfrac{\partial \mathcal{F}}{\partial \overline{\boldsymbol{\zeta}}}
        + \sum_{n=1}^N \left( \dfrac{\partial \mathbf{R}_n}{\partial \overline{\boldsymbol{\zeta}}} \right)^\top \boldsymbol{\lambda}_n
        + \sum_{n=1}^{N-1} \left( \dfrac{\partial \mathbf{V}_n}{\partial \overline{\boldsymbol{\zeta}}} \right)^\top 
        \left[ \left( \dfrac{\partial \mathbf{R}_n}{\partial \mathbf{V}_n} \right)^\top \boldsymbol{\lambda}_n
        + \left( \dfrac{\partial \mathbf{R}_{n+1}}{\partial \mathbf{V}_n} \right)^\top \boldsymbol{\lambda}_{n+1}
        + \dfrac{\partial \mathcal{F}}{\partial \mathbf{V}_n} \right] \\[15pt]
        &+ \displaystyle \left( \dfrac{\partial \mathbf{V}_N}{\partial \overline{\boldsymbol{\zeta}}} \right)^\top
        \left[ \left( \dfrac{\partial \mathbf{R}_N}{\partial \mathbf{V}_N} \right)^\top \boldsymbol{\lambda}_N + \dfrac{\partial \mathcal{F}}{\partial \mathbf{V}_N} \right],
    \end{array}
\end{equation}
where we use $\partial \mathbf{V}_0/\partial \overline{\boldsymbol{\zeta}} = \mathbf{0}$. Note again that $\boldsymbol{\lambda}_1, \ldots, \boldsymbol{\lambda}_N$ are arbitrary, and we select their values to eliminate computationally expensive term, $\partial \mathbf{V}_n/\partial \overline{\boldsymbol{\zeta}} = \mathbf{0}$ for $n = 1,\ldots,N$. This step leads to the adjoint equations expressed as
\begin{equation} \label{Adjoint Equations}
    \left\{ \begin{array}{l}
        \left( \dfrac{\partial \mathbf{R}_n}{\partial \mathbf{V}_n} \right)^\top \boldsymbol{\lambda}_n
        + \left( \dfrac{\partial \mathbf{R}_{n+1}}{\partial \mathbf{V}_n} \right)^\top \boldsymbol{\lambda}_{n+1}
        = - \dfrac{\partial \mathcal{F}}{\partial \mathbf{V}_n} \quad \text{for} \quad n = 1,\ldots,N-1, \\[12pt]
        
        \left( \dfrac{\partial \mathbf{R}_N}{\partial \mathbf{V}_N} \right)^\top \boldsymbol{\lambda}_N = - \dfrac{\partial \mathcal{F}}{\partial \mathbf{V}_N},
    \end{array} \right.
\end{equation}
where
\begin{equation*}
    \dfrac{\partial \mathbf{R}_n}{\partial \mathbf{V}_n} = \left[ \begin{array}{ccccc}
        \mathbf{I} & \mathbf{0} & \mathbf{0} & \dfrac{\partial \mathbf{R}_n^{\overline{\mathbf{b}}^\texttt{e}}}{\partial \mathbf{V}_n^{\widehat{\gamma}}} & \dfrac{\partial \mathbf{R}_n^{\overline{\mathbf{b}}^\texttt{e}}}{\partial \mathbf{U}_n} \\[12pt]
        
        \mathbf{0} & \mathbf{I} & \mathbf{0} & \dfrac{\partial \mathbf{R}_n^{\overline{\boldsymbol{\beta}}}}{\partial \mathbf{V}_n^{\widehat{\gamma}}} & \dfrac{\partial \mathbf{R}_n^{\overline{\boldsymbol{\beta}}}}{\partial \mathbf{U}_n} \\[12pt]

        \mathbf{0} & \mathbf{0} & \mathbf{I} & \dfrac{\partial \mathbf{R}_n^\alpha}{\partial \mathbf{V}_n^{\widehat{\gamma}}} & \mathbf{0} \\[12pt]

        \mathbf{0} & \mathbf{0} & \mathbf{0} & \dfrac{\partial \mathbf{R}_n^{\widehat{\gamma}}}{\partial \mathbf{V}_n^{\widehat{\gamma}}} & \dfrac{\partial \mathbf{R}_n^{\widehat{\gamma}}}{\partial \mathbf{U}_n} \\[12pt]

        \mathbf{0} & \mathbf{0} & \mathbf{0} & \dfrac{\partial \mathbf{R}_n^\mathbf{u}}{\partial \mathbf{V}_n^{\widehat{\gamma}}} & \dfrac{\partial \mathbf{R}_n^\mathbf{u}}{\partial \mathbf{U}_n}
    \end{array} \right]
    \quad \text{and} \quad
        \dfrac{\partial \mathbf{R}_{n+1}}{\partial \mathbf{V}_n} = \left[ \begin{array}{ccccc}
        \dfrac{\partial \mathbf{R}_{n+1}^{\overline{\mathbf{b}}^\texttt{e}}}{\partial \mathbf{V}_n^{\overline{\mathbf{b}}^\texttt{e}}} & \dfrac{\partial \mathbf{R}_{n+1}^{\overline{\mathbf{b}}^\texttt{e}}}{\partial \mathbf{V}_n^{\overline{\boldsymbol{\beta}}}} & \mathbf{0} & \mathbf{0} & \dfrac{\partial \mathbf{R}_{n+1}^{\overline{\mathbf{b}}^\texttt{e}}}{\partial \mathbf{U}_n} \\[12pt]
        
        \dfrac{\partial \mathbf{R}_{n+1}^{\overline{\boldsymbol{\beta}}}}{\partial \mathbf{V}_n^{\overline{\mathbf{b}}^\texttt{e}}} & \dfrac{\partial \mathbf{R}_{n+1}^{\overline{\boldsymbol{\beta}}}}{\partial \mathbf{V}_n^{\overline{\boldsymbol{\beta}}}} & \mathbf{0} & \mathbf{0} & \dfrac{\partial \mathbf{R}_{n+1}^{\overline{\boldsymbol{\beta}}}}{\partial \mathbf{U}_n} \\[12pt]

        \mathbf{0} & \mathbf{0} & \dfrac{\partial \mathbf{R}_{n+1}^\alpha}{\partial \mathbf{V}_n^\alpha} & \mathbf{0} & \mathbf{0} \\[12pt]

        \dfrac{\partial \mathbf{R}_{n+1}^{\widehat{\gamma}}}{\partial \mathbf{V}_n^{\overline{\mathbf{b}}^\texttt{e}}} & \dfrac{\partial \mathbf{R}_{n+1}^{\widehat{\gamma}}}{\partial \mathbf{V}_n^{\overline{\boldsymbol{\beta}}}} & \dfrac{\partial \mathbf{R}_{n+1}^{\widehat{\gamma}}}{\partial \mathbf{V}_n^\alpha} & \mathbf{0} & \dfrac{\partial \mathbf{R}_{n+1}^{\widehat{\gamma}}}{\partial \mathbf{U}_n} \\[12pt]

        \dfrac{\partial \mathbf{R}_{n+1}^\mathbf{u}}{\partial \mathbf{V}_n^{\overline{\mathbf{b}}^\texttt{e}}} & \dfrac{\partial \mathbf{R}_{n+1}^\mathbf{u}}{\partial \mathbf{V}_n^{\overline{\boldsymbol{\beta}}}} & \mathbf{0} & \mathbf{0} & \dfrac{\partial \mathbf{R}_{n+1}^\mathbf{u}}{\partial \mathbf{U}_n}
    \end{array} \right].
\end{equation*}

Based on \eqref{Adjoint Equations}, we compute the adjoint variables ($\boldsymbol{\lambda}_n^{\overline{\mathbf{b}}^\texttt{e}}$, $\boldsymbol{\lambda}_n^{\overline{\boldsymbol{\beta}}}$, $\boldsymbol{\lambda}_n^\alpha$, $\boldsymbol{\lambda}_n^{\widehat{\gamma}}$, and $\boldsymbol{\lambda}_n^\mathbf{u}$ for $n = 1,\ldots,N$) in a reversed order as follows. At load step $N$, we directly write out $\boldsymbol{\lambda}_N^{\overline{\mathbf{b}}^\texttt{e}}$, $\boldsymbol{\lambda}_N^{\overline{\boldsymbol{\beta}}}$, and $\boldsymbol{\lambda}_N^\alpha$ as
\begin{equation*}
    \boldsymbol{\lambda}_N^{\overline{\mathbf{b}}^\texttt{e}} = - \dfrac{\partial \mathcal{F}}{\partial \mathbf{V}_N^{\overline{\mathbf{b}}^\texttt{e}}}, \quad
    \boldsymbol{\lambda}_N^{\overline{\boldsymbol{\beta}}} = - \dfrac{\partial \mathcal{F}}{\partial \mathbf{V}_N^{\overline{\boldsymbol{\beta}}}}, \quad \text{and} \quad
    \boldsymbol{\lambda}_N^\alpha = - \dfrac{\partial \mathcal{F}}{\partial \mathbf{V}_N^\alpha}.
\end{equation*}
We then solve for $\boldsymbol{\lambda}_N^{\widehat{\gamma}}$ and $\boldsymbol{\lambda}_N^\mathbf{u}$ from 
\begin{equation} \label{Reduced Adjoint Equation at Final Step}
    \left[ \begin{array}{cc}
        \left( \dfrac{\partial \mathbf{R}_N^{\widehat{\gamma}}}{\partial \mathbf{V}_N^{\widehat{\gamma}}} \right)^\top
        & \left( \dfrac{\partial \mathbf{R}_N^\mathbf{u}}{\partial \mathbf{V}_N^{\widehat{\gamma}}} \right)^\top \\[12pt]
        \left( \dfrac{\partial \mathbf{R}_N^{\widehat{\gamma}}}{\partial \mathbf{U}_N} \right)^\top
        & \left( \dfrac{\partial \mathbf{R}_N^\mathbf{u}}{\partial \mathbf{U}_N} \right)^\top
    \end{array} \right]
    \left[ \begin{array}{c}
        \boldsymbol{\lambda}_N^{\widehat{\gamma}} \\[12pt]
        \boldsymbol{\lambda}_N^\mathbf{u}
    \end{array} \right]
    = - \left[ \begin{array}{c}
         \dfrac{\partial \mathcal{F}}{\partial \mathbf{V}_N^{\widehat{\gamma}}}
         + \left( \dfrac{\partial \mathbf{R}_N^{\overline{\mathbf{b}}^\texttt{e}}}{\partial \mathbf{V}_N^{\widehat{\gamma}}} \right)^\top \boldsymbol{\lambda}_N^{\overline{\mathbf{b}}^\texttt{e}}
         + \left( \dfrac{\partial \mathbf{R}_N^{\overline{\boldsymbol{\beta}}}}{\partial \mathbf{V}_N^{\widehat{\gamma}}} \right)^\top \boldsymbol{\lambda}_N^{\overline{\boldsymbol{\beta}}}
         + \left( \dfrac{\partial \mathbf{R}_N^\alpha}{\partial \mathbf{V}_N^{\widehat{\gamma}}} \right)^\top \boldsymbol{\lambda}_N^\alpha \\[12pt]
         \dfrac{\partial \mathcal{F}}{\partial \mathbf{U}_N}
         + \left( \dfrac{\partial \mathbf{R}_N^{\overline{\mathbf{b}}^\texttt{e}}}{\partial \mathbf{U}_N} \right)^\top \boldsymbol{\lambda}_N^{\overline{\mathbf{b}}^\texttt{e}}
         + \left( \dfrac{\partial \mathbf{R}_N^{\overline{\boldsymbol{\beta}}}}{\partial \mathbf{U}_N} \right)^\top \boldsymbol{\lambda}_N^{\overline{\boldsymbol{\beta}}}
    \end{array} \right].
\end{equation}
At the remaining load steps, $n = N-1, \ldots, 1$, we write out $\boldsymbol{\lambda}_n^{\overline{\mathbf{b}}^\texttt{e}}$, $\boldsymbol{\lambda}_n^{\overline{\boldsymbol{\beta}}}$, and $\boldsymbol{\lambda}_n^\alpha$ as
\begin{equation*}
    \left\{ \begin{array}{l}
        \boldsymbol{\lambda}_n^{\overline{\mathbf{b}}^\texttt{e}}
        = - \left[ \dfrac{\partial \mathcal{F}}{\partial \mathbf{V}_n^{\overline{\mathbf{b}}^\texttt{e}}} 
        + \left( \dfrac{\partial \mathbf{R}_{n+1}^{\overline{\mathbf{b}}^\texttt{e}}}{\partial \mathbf{V}_n^{\overline{\mathbf{b}}^\texttt{e}}} \right)^\top \boldsymbol{\lambda}_{n+1}^{\overline{\mathbf{b}}^\texttt{e}}
        + \left( \dfrac{\partial \mathbf{R}_{n+1}^{\overline{\boldsymbol{\beta}}}}{\partial \mathbf{V}_n^{\overline{\mathbf{b}}^\texttt{e}}} \right)^\top \boldsymbol{\lambda}_{n+1}^{\overline{\boldsymbol{\beta}}}
        + \left( \dfrac{\partial \mathbf{R}_{n+1}^{\widehat{\gamma}}}{\partial \mathbf{V}_n^{\overline{\mathbf{b}}^\texttt{e}}} \right)^\top
        \boldsymbol{\lambda}_{n+1}^{\widehat{\gamma}}
        + \left( \dfrac{\partial \mathbf{R}_{n+1}^\mathbf{u}}{\partial \mathbf{V}_n^{\overline{\mathbf{b}}^\texttt{e}}} \right)^\top \boldsymbol{\lambda}_{n+1}^\mathbf{u}
        \right], \\[18pt]

        \boldsymbol{\lambda}_n^{\overline{\boldsymbol{\beta}}}
        = - \left[ \dfrac{\partial \mathcal{F}}{\partial \mathbf{V}_n^{\overline{\boldsymbol{\beta}}}}
        + \left( \dfrac{\partial \mathbf{R}_{n+1}^{\overline{\mathbf{b}}^\texttt{e}}}{\partial \mathbf{V}_n^{\overline{\boldsymbol{\beta}}}} \right)^\top \boldsymbol{\lambda}_{n+1}^{\overline{\mathbf{b}}^\texttt{e}}
        + \left( \dfrac{\partial \mathbf{R}_{n+1}^{\overline{\boldsymbol{\beta}}}}{\partial \mathbf{V}_n^{\overline{\boldsymbol{\beta}}}} \right)^\top \boldsymbol{\lambda}_{n+1}^{\overline{\boldsymbol{\beta}}}
        + \left( \dfrac{\partial \mathbf{R}_{n+1}^{\widehat{\gamma}}}{\partial \mathbf{V}_n^{\overline{\boldsymbol{\beta}}}} \right)^\top \boldsymbol{\lambda}_{n+1}^{\widehat{\gamma}}
        + \left( \dfrac{\partial \mathbf{R}_{n+1}^\mathbf{u}}{\partial \mathbf{V}_n^{\overline{\boldsymbol{\beta}}}} \right)^\top \boldsymbol{\lambda}_{n+1}^\mathbf{u}
        \right], \\[18pt]

        \boldsymbol{\lambda}_n^\alpha = - \left[ \dfrac{\partial \mathcal{F}}{\partial \mathbf{V}_n^\alpha}
        + \left( \dfrac{\partial \mathbf{R}_{n+1}^\alpha}{\partial \mathbf{V}_n^\alpha} \right)^\top \boldsymbol{\lambda}_{n+1}^\alpha
        + \left( \dfrac{\partial \mathbf{R}_{n+1}^{\widehat{\gamma}}}{\partial \mathbf{V}_n^\alpha} \right)^\top \boldsymbol{\lambda}_{n+1}^{\widehat{\gamma}}
        \right],
    \end{array} \right.
\end{equation*}
and then solve for $\boldsymbol{\lambda}_n^{\widehat{\gamma}}$ and $\boldsymbol{\lambda}_n^\mathbf{u}$ from 
\begin{equation} \label{Reduced Adjoint Equation at Remaining Steps}
    \left[ \begin{array}{cc}
        \left( \dfrac{\partial \mathbf{R}_n^{\widehat{\gamma}}}{\partial \mathbf{V}_n^{\widehat{\gamma}}} \right)^\top
        & \left( \dfrac{\partial \mathbf{R}_n^\mathbf{u}}{\partial \mathbf{V}_n^{\widehat{\gamma}}} \right)^\top \\[12pt]
        \left( \dfrac{\partial \mathbf{R}_n^{\widehat{\gamma}}}{\partial \mathbf{U}_n} \right)^\top
        & \left( \dfrac{\partial \mathbf{R}_n^\mathbf{u}}{\partial \mathbf{U}_n} \right)^\top
    \end{array} \right]
    \left[ \begin{array}{c}
        \boldsymbol{\lambda}_n^{\widehat{\gamma}} \\[12pt]
        \boldsymbol{\lambda}_n^\mathbf{u}
    \end{array} \right]
    = - \left[ \begin{array}{c}
         \dfrac{\partial \mathcal{F}}{\partial \mathbf{V}_n^{\widehat{\gamma}}}
         + \left( \dfrac{\partial \mathbf{R}_n^{\overline{\mathbf{b}}^\texttt{e}}}{\partial \mathbf{V}_n^{\widehat{\gamma}}} \right)^\top \boldsymbol{\lambda}_n^{\overline{\mathbf{b}}^\texttt{e}}
         + \left( \dfrac{\partial \mathbf{R}_n^{\overline{\boldsymbol{\beta}}}}{\partial \mathbf{V}_n^{\widehat{\gamma}}} \right)^\top \boldsymbol{\lambda}_n^{\overline{\boldsymbol{\beta}}}
         + \left( \dfrac{\partial \mathbf{R}_n^\alpha}{\partial \mathbf{V}_n^{\widehat{\gamma}}} \right)^\top \boldsymbol{\lambda}_n^\alpha \\[12pt]
         \dfrac{\partial \mathcal{F}}{\partial \mathbf{U}_n}
         + \dfrac{\partial \widecheck{\mathcal{F}}}{\partial \mathbf{U}_n}
         + \left( \dfrac{\partial \mathbf{R}_n^{\overline{\mathbf{b}}^\texttt{e}}}{\partial \mathbf{U}_n} \right)^\top \boldsymbol{\lambda}_n^{\overline{\mathbf{b}}^\texttt{e}}
         + \left( \dfrac{\partial \mathbf{R}_n^{\overline{\boldsymbol{\beta}}}}{\partial \mathbf{U}_n} \right)^\top \boldsymbol{\lambda}_n^{\overline{\boldsymbol{\beta}}}
    \end{array} \right]
\end{equation}
where
\begin{equation*}
    \dfrac{\partial \widecheck{\mathcal{F}}}{\partial \mathbf{U}_n}
    := \left( \dfrac{\partial \mathbf{R}_{n+1}^{\overline{\mathbf{b}}^\texttt{e}}}{\partial \mathbf{U}_n} \right)^\top \boldsymbol{\lambda}_{n+1}^{\overline{\mathbf{b}}^\texttt{e}}
    + \left( \dfrac{\partial \mathbf{R}_{n+1}^{\overline{\boldsymbol{\beta}}}}{\partial \mathbf{U}_n} \right)^\top \boldsymbol{\lambda}_{n+1}^{\overline{\boldsymbol{\beta}}}
    + \left( \dfrac{\partial \mathbf{R}_{n+1}^{\widehat{\gamma}}}{\partial \mathbf{U}_n} \right)^\top  \boldsymbol{\lambda}_{n+1}^{\widehat{\gamma}}
    + \left( \dfrac{\partial \mathbf{R}_{n+1}^\mathbf{u}}{\partial \mathbf{U}_n} \right)^\top \boldsymbol{\lambda}_{n+1}^\mathbf{u}.
\end{equation*}

After solving for the adjoint variables, we substitute them into \eqref{Regrouping Terms for Sensitivity} and derive the sensitivity expression as
\begin{equation*}
    \begin{array}{ll}
        \dfrac{\text{d} \mathcal{F}}{\text{d} \overline{\boldsymbol{\zeta}}}
        = \dfrac{\partial \mathcal{F}}{\partial \overline{\boldsymbol{\zeta}}}
        &+ \displaystyle \sum_{n=1}^N \left[ \left( \dfrac{\partial \mathbf{R}_n^{\overline{\mathbf{b}}^\texttt{e}}}{\partial \overline{\boldsymbol{\zeta}}} \right)^\top \boldsymbol{\lambda}_n^{\overline{\mathbf{b}}^\texttt{e}}
        + \left( \dfrac{\partial \mathbf{R}_n^{\overline{\boldsymbol{\beta}}}}{\partial \overline{\boldsymbol{\zeta}}} \right)^\top \boldsymbol{\lambda}_n^{\overline{\boldsymbol{\beta}}}
        + \left( \dfrac{\partial \mathbf{R}_n^{\widehat{\gamma}}}{\partial \overline{\boldsymbol{\zeta}}} \right)^\top  \boldsymbol{\lambda}_n^{\widehat{\gamma}}
        + \left( \dfrac{\partial \mathbf{R}_n^\mathbf{u}}{\partial \overline{\boldsymbol{\zeta}}} \right)^\top \boldsymbol{\lambda}_n^\mathbf{u} \right],
    \end{array}
\end{equation*}
where we use $\partial \mathbf{R}_n^\alpha / \partial \overline{\boldsymbol{\zeta}} = \mathbf{0}$ and $\text{d} \mathcal{F}/\text{d} \overline{\boldsymbol{\zeta}} = \text{d} \widehat{\mathcal{F}}/\text{d} \overline{\boldsymbol{\zeta}}$ by noticing 
\begin{equation} \label{Sensitivity Expression}
    \boldsymbol{\lambda}_n^{\overline{\mathbf{b}}^\texttt{e}} \cdot \mathbf{R}_n^{\overline{\mathbf{b}}^\texttt{e}}
    = \boldsymbol{\lambda}_n^{\overline{\boldsymbol{\beta}}} \cdot \mathbf{R}_n^{\overline{\boldsymbol{\beta}}}
    = \boldsymbol{\lambda}_n^\alpha \cdot \mathbf{R}_n^\alpha
    = \boldsymbol{\lambda}_n^{\widehat{\gamma}} \cdot \mathbf{R}_n^{\widehat{\gamma}}
    = \boldsymbol{\lambda}_n^\mathbf{u} \cdot \mathbf{R}_n^\mathbf{u} = 0.
\end{equation}
We remark that, by using the modern computer method of automatic differentiation, one can effortlessly compute all the above partial derivatives without deriving their explicit (and typically complex) expressions.

\subsection{Sensitivity verification}

In this subsection, we verify the accuracy of the sensitivity analysis described in \ref{Sec: Sensitivity Analysis} by comparing it against the forward finite difference method. Using the finite difference scheme, the sensitivity of any function $\mathcal{F}$ with respect to the physical design variable $\overline{\zeta}_e$ of element $e$ for $\overline{\zeta} \in \{ \overline{\rho}, \overline{\xi}_1, \ldots, \overline{\xi}_{N^\xi} \}$ is computed as
\begin{equation*}
    \dfrac{\text{d} \mathcal{F}}{\text{d} \overline{\zeta}_e} = \dfrac{\mathcal{F}(\overline{\zeta}_e + \varepsilon) - \mathcal{F}(\overline{\zeta}_e)}{\varepsilon}
\end{equation*}
where $\varepsilon \in [10^{-6}, 10^{-4}]$ is a small perturbation parameter.

According to this formula, we compare the sensitivities of all objective functions ($J_\texttt{stiff}$, $J_\texttt{force}$, and $J_\texttt{energy}$) and constraint functions ($g_{V0}$, $g_{V1}$, $g_{V2}$, $g_{V3}$, $g_{V4}$, $g_P$, $g_M$, and $g_C$) employed in Section \ref{Sec: Sample Examples}. For representativeness, the comparisons are focused on the most complex cases: $J_\texttt{energy}$ for the optimized damper in Fig. \ref{Fig: Damper-Part 3}; $J_\texttt{stiff}$ and $g_{V0}$ for Dsg. 4 in Fig. \ref{Fig: Beam}; $g_{V1}$, $g_{V2}$, $g_{V3}$, and $g_{V4}$ for the optimized bumper in Fig. \ref{Fig: Bumper-Part 3}; and $J_\texttt{force}$, $g_P$, $g_M$, and $g_C$ for Dsg. 4 in Fig. \ref{Fig: Sheet-Part 2}.

To perform the comparison, we use stratified sampling of all design variables and present the results in Fig. \ref{Fig: Sensitivity Verification}. The proposed sensitivity analysis in \ref{Sec: Sensitivity Analysis} shows good agreement with the finite difference scheme. The observed absolute errors ($e_\texttt{abs}$) are in the range of $[10^{-11}, 10^{-6}]$, while the relative errors ($e_\texttt{rel}$) are in the range of $[10^{-8}, 10^{-4}]$. This verification demonstrates the reliability of the proposed sensitivity analysis, irrespective of the complexity introduced by the history dependence of finite strain elastoplasticity.

\begin{figure}[!htbp]
    \centering
    \includegraphics[width=18cm]{Sensitivity_Verification.pdf}
    \caption{Sensitivity verification for various functions used in Section \ref{Sec: Sample Examples}. (a) Initial energy of Dsg. 4 in Fig. \ref{Fig: Beam}. (b) End compliance of Dsg. 4 in Fig. \ref{Fig: Sheet-Part 2}. (c) Total energy of the optimized damper in Fig. \ref{Fig: Damper-Part 3}. (d) Total material volume of Dsg. 4 in Fig. \ref{Fig: Beam}. (e)--(h) Volumes of materials 1--4 of the optimized bumper in Fig. \ref{Fig: Bumper-Part 3}. (i)--(k) Price, mass density, and CO$_2$ footprint of Dsg. 4 in Fig. \ref{Fig: Sheet-Part 2}.}
    \label{Fig: Sensitivity Verification}
\end{figure}


\section{Comparisons among various mesh and element combinations}
\label{Sec: Comparison of Meshes}

In this section, we compare the elastoplastic responses of structures defined with various meshes and elements. The proposed topology optimization framework favors a voxel-based mesh, where the grid in the undeformed configurations of structures is fixed during optimization, and solid and void elements coexist. However, for practical manufacturing purposes, a body-fitted mesh is more suitable for generating printable files (e.g., the STereoLithography format), as the mesh boundaries closely align with structural geometries and only contain solid elements.

To evaluate the accuracy loss when converting optimized designs from voxel-based to body-fitted meshes, we take Dsg. 5 in Fig. \ref{Fig: Beam} as an example and compare the elastoplastic responses in Fig. \ref{Fig: Mesh Comparisons} and Table \ref{Table: Mesh Comparisons}. The results show that, compared to the design defined on a voxel-based mesh with quadrilateral elements, the structure defined on the body-fitted mesh with the same elements exhibits negligible relative errors: 4.59\% in initial stiffness, 1.20\% in end force, and 0.84\% in total energy. These results justify the use of voxel-based meshes in the proposed framework.

\begin{figure}[!htbp]
    \centering
    \includegraphics[width=18cm]{Mesh_Comparisons.pdf}
    \caption{Comparisons among various meshes and elements. (a) Compared meshes and the associated finite elements. (b) Material phases. (c) Elastic/plastic regions. (d) Force--displacement ($F$--$u$) curves.}
    \label{Fig: Mesh Comparisons}
\end{figure}

\begin{table}[!htbp]
    \caption{Comparisons of the performance metrics among various meshes and elements}
    \label{Table: Mesh Comparisons}
    \centering
    \footnotesize
    \begin{tabular}{llllll}
        \hline
        \textbf{Meshes} & \textbf{Elements} & \textbf{Usage}
        & \begin{tabular}[x]{@{}l@{}}\textbf{Initial stiffness}\\(N/mm)\end{tabular}
        & \begin{tabular}[x]{@{}l@{}}\textbf{End force}\\(N)\end{tabular}
        & \begin{tabular}[x]{@{}l@{}}\textbf{Total energy}\\(N$\cdot$mm)\end{tabular} \\
        \hline
        Voxel-based, structured & Quadrilateral & Topology optimization & 10.25 & 10.03 & 14.31 \\
        Body-fitted, unstructured & Quadrilateral & Manufacturing & 10.72 (4.59\%) & 10.15 (1.20\%) & 14.43 (0.84\%) \\
        Voxel-based, structured & Triangular & Checking stiffening & 9.88 ($-$3.61\%) & 9.26 ($-$7.68\%) & 13.39 ($-$6.43\%) \\
        \hline
    \end{tabular}
\end{table}

In the context of finite strain elastoplasticity, we also need to examine the element types, as certain mesh and element combinations may exhibit over-stiffening behaviors in structures \citep{sloan_numerical_1982, wells_p-adaptive_2002} due to the isochoric plastic flow, despite the limited compressibility from elasticity. To mitigate these over-stiffening behaviors, higher-order and crossed linear triangular elements \citep{wells_p-adaptive_2002} have been identified as simple and effective strategies. However, both strategies are computationally expensive.

In this study, we opt for first-order quadrilateral and hexahedral elements to reduce computational time. As shown in Fig. \ref{Fig: Mesh Comparisons} and Table \ref{Table: Mesh Comparisons}, the performance metrics of the current setup remain close to the ``golden setup" of crossed linear triangular elements for plasticity \citep{wells_p-adaptive_2002}, with relative errors below 10\%. Additionally, we emphasize that the proposed design framework is independent of mesh and element types, especially with the versatile implementation of FEniTop \citep{jia_fenitop_2024}. Users have the flexibility to balance solution accuracy and computational cost based on their specific requirements.


\bibliographystyle{elsarticle-harv}
\biboptions{authoryear}
\bibliography{References}

\end{document}


\end{document}