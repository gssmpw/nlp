\begin{thebibliography}{99}

\bibitem{AIin5G}
Y. Arjoune and S. Faruque, “Artificial intelligence for 5g wireless systems: Opportunities, challenges, and future research direction,” in \textit{2020 10th Annual Computing and Communication Workshop and Conference (CCWC)}. IEEE, Mar 2020.

\bibitem{ORAN}
A. S. Abdalla, P. S. Upadhyaya, V. K. Shah, and V. Marojevic, “Toward next generation open radio access networks: What o-ran can and cannot do!” in \textit{IEEE Network}. IEEE, Jul 2022, pp. 206–213.

\bibitem{ORAN_Tommaso}
M. Polese, L. Bonati, S. D’Oro, S. Basagni, and T. Melodia, “Understanding o-ran: Architecture, interfaces, algorithms, security, and research challenges,” in \textit{IEEE Communications Surveys \& Tutorials}. IEEE, Jan 2023, pp. 1376–1411.

\bibitem{AIinORAN}
P. H. Masur, J. H. Reed, and N. K. Tripathi, “Artificial intelligence in open-radio access network,” in \textit{IEEE Aerospace and Electronic Systems Magazine}. IEEE, Sep 2022, pp. 6–15.

\bibitem{villa2023x5g}
L. Bonati, F. Restuccia, R. Gangula, S. D’Oro, M. Polese, D. Villa, I. Khan, P. Dini, S. Buzzi, G. Caso, T. Désert, M. Bennis, and T. Melodia, “XS5g: A modular, open-source, programmable 5g platform for o-ran research,” \textit{arXiv preprint arXiv:2406.15935}, 2023. [Online]. Available: \url{https://arxiv.org/abs/2406.15935}.

\bibitem{cheng2024oranslice}
H. Cheng, S. D’Oro, R. Gangula, S. Velumani, D. Villa, L. Bonati, M. Polese, G. Arrobo, C. Maciocco, and T. Melodia, “Oranslice: An open-source 5g network slicing platform for o-ran,” \textit{arXiv preprint arXiv:2410.12978}, 2024. [Online]. Available: \url{https://arxiv.org/abs/2410.12978}.

\bibitem{coloran}
M. Polese, L. Bonati, S. D’Oro, S. Basagni, and T. Melodia, “Colo-ran: Developing machine learning-based xapps for open ran closed-loop control on programmable experimental platforms,” \textit{IEEE Transactions on Mobile Computing}, vol. 22, no. 10, pp. 5787–5800, 2023.

\bibitem{tsampazi2024pandora}
M. Tsampazi, S. D’Oro, M. Polese, L. Bonati, G. Poitau, M. Healy, M. Alavirad, and T. Melodia, “Pandora: Automated design and comprehensive evaluation of deep reinforcement learning agents for open ran,” \textit{arXiv preprint arXiv:2407.11747}, 2024.

\bibitem{tsampazi2023}
M. Tsampazi, S. D’Oro, M. Polese, L. Bonati, G. Poitau, M. Healy, and T. Melodia, “A comparative analysis of deep reinforcement learning-based xapps in o-ran,” in \textit{2023 IEEE Global Communications Conference (GLOBECOM)}. IEEE, Dec 2023, pp. 1–6.

\bibitem{marojevic2022actor}
V. Marojevic, M. Shirvanimoghaddam, and H. Dai, “Actor-critic network for o-ran resource allocation: xapp design, deployment, and analysis,” \textit{arXiv preprint arXiv:2210.04604}, Oct 2022. [Online]. Available: \url{https://arxiv.org/abs/2210.04604}.

\bibitem{Anand2023xApp}
D. Anand, M. A. Togou, and G.-M. Muntean, “A machine learning-based xapp for 5g o-ran to mitigate co-tier interference and improve QoE for various services in a hetnet environment,” in \textit{IEEE International Symposium on Broadband Multimedia Systems and Broadcasting (BMSB 2023)}. IEEE, June 2023, pp. 1–6.

\bibitem{ns3}
“ns-3 network simulator,” \url{https://www.nsnam.org}, 2024, accessed: 2024-10-12.

\bibitem{Colosseum2023}
L. Bonati, M. Polese, S. D’Oro, and T. Melodia, “Colosseum: The open ran digital twin,” in \textit{2023 IEEE International Symposium on Dynamic Spectrum Access Networks (DySPAN)}. IEEE, 2023, pp. 1–10.

\bibitem{srsran}
“srsran 5g,” \url{https://github.com/srsran/srsRAN_Project}, 2024, accessed: 2024-10-12.

\bibitem{amarisoft}
“amarisoft,” \url{https://www.amarisoft.com}.

\bibitem{5062197}
S.-B. Lee, I. Pefkianakis, A. Meyerson, S. Xu, and S. Lu, “Proportional fair frequency-domain packet scheduling for 3gpp lte uplink,” in \textit{IEEE INFOCOM 2009}, 2009, pp. 2611–2615.


\end{thebibliography}
