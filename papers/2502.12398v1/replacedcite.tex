\section{Related Work}
\subsection{Cold Start Problem}

The cold start problem is a fundamental problem in recommender systems. It arises when a new user (or a new item) enters the system, and the system lacks sufficient information to provide accurate recommendations. Although many methods have been proposed to address the cold start problem ____, all of them require the service providers to implement the method. In contrast, our approach is unique in that it can be applied directly by an end user without requiring any modifications to the service itself. This characteristic broadens the applicability of our method, allowing it to be used in services that lack built-in functionalities to address the cold start problem.


\subsection{Quadrature}

Quadrature is a technique to approximate the integral of a function by summing the function values at a finite number of points ____. This is essentially equivalent to finding a discrete measure that approximates the given measure. Quadrature is widely used in numerical analysis and machine learning. One of the common applications is coresets ____, which is a small set of training points that approximates the loss of the model on the entire training set. Our approach can be seen as a quadrature of $\int \ell_T \, d\mu_S$ with points $\mathcal{D}_T$. The main difference is that standard quadrature methods use arbitrary points and/or weights and sidestep combinatorial problems ____, while our approach uses only the items in the target set, which naturally lead to the combinatorial optimization problem. For example, methods based on the Frank-Wolfe algorithm ____ output sparse weights, but they may choose the same items repeatedly in general, and the resulting weights are not uniform. Such output cannot be applied to our setting because users cannot thumbs up the same item multiple times in most services. Some approaches ____ such as Kernel thinning ____ output a subset of the input points with uniform weights, which is similar to our approach. However, these methods assume that the candidate points are sampled according to the distribution being approximated. If this assumption does not hold, as in our case, these methods cannot be directly applied. Our proposed method can be used even when the input points are arbitrary. Other methods ____ employ submodular optimization and greedy algorithms, achieving a $(1 - \frac{1}{e})$-approximation ratio. However, the gap of $(1 - \frac{1}{e})$ does not vanish as the number of items increases. By contrast, our approach can achieve the vanishing error as Corollary \ref{cor: mmd_loss_bound} shows thanks to the continuous optimization approach and the careful rouding process. To the best of our knowledge, our work is the first to provide a theoretical guarantee for such a general and combinatorial setting. We believe that this result is of independent interest.

\subsection{User-side Realization}

User-side realization refers to the concept in which end users implement desired functionalities on their own without requiring modifications to the service itself. Many users experience dissatisfaction with services. Even if they want some functinoalities and request them to the service provider, the provider may not implement them due to various reasons such as cost, complexity, and simple negligence. After all, service providers are not volunteers but businesses. In such cases, the only options users have are not satisfactory, keep using the service despite their dissatisfaction or leave the service. User-side realization provides a proactive alternative to this dilemma. This concept has been explored in various fields such as recommender systems ____, search engines ____, and privacy ____. The main advantage of the user-side realization is that it can be used in services that do not have special functionalities to address the problem, and it broadens the scope of the applicability of the solution. For a more comprehensive discussion on user-side realization, we refer the reader to the Ph.D. thesis by ____. Our approach constitutes a novel application of user-side realization to the cold start problem in recommender systems.