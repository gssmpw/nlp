This study analyses AI policies from ten news organisations and ten universities to identify gaps in the EU AI Act that could be clarified for its implementation and point out possible research directions. All data used in this research is derived from publicly available policy documents, ensuring transparency and compliance with ethical research standards. We strived to present any interpretations or critiques of the policies in a constructive manner to inform policymakers, AI practitioners, and institutional stakeholders.

This study strives to respects intellectual property rights by citing all sources and representing policy content appropriately. Since the analysis pertains to institutional policies rather than individual data, no personally identifiable information is processed or collected.

Finally, we acknowledge that AI governance is an evolving field. To the best of our knowledge, our findings reflect the state of AI policies at the time of analysis and should be interpreted in light of ongoing regulatory and institutional developments. We encourage further interdisciplinary dialogue to refine AI governance frameworks in alignment with ethical, legal, and societal expectations.







