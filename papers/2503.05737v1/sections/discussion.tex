% \section{Discussion: top-down vs bottom-up policies}
% In the following, we explore how organisational policies, which have been in practice and address lived challenges in the use of AI, can inform finding coverage gaps in the EU AI Act. While the organisational policies cannot be used to compare directly with the EU AI Act, they serve as a good start to inform possible improvements that can be made to the legislation in upcoming iterations based on their existing experience and help point out possible points of confusion for practitioners applying the EU AI Act in their domain.
% \label{sec:discusssion}
% \subsection{Transparency \& Accountability}

% \textbf{AIA.} Labeling and making transparent what parts of the content are created by AI and when users are interacting with an AI or AI-created content is part of Article 50 of the EU AI Act.\footnote{\url{https://artificialintelligenceact.eu/article/50/}}

% \textbf{Universities.} The emphasis on academic integrity and disclosure of AI use in coursework mirrors Article 50 of the EU AI Act,\footnote{\url{https://artificialintelligenceact.eu/article/50/}} which requires AI-generated content to be clearly labelled. Many universities, such as the University of Lisbon, enforce transparency by requiring students to declare AI-assisted work, similar to the AI Act’s obligations for AI system deployers. (TODO add ref to enforcement section)

% \textbf{News.} We find that news organizations arrive at a similar requirement? ...

% \textbf{Gaps.} 

% \subsection{Enforceability}

% \textbf{AIA.} ? requires disclosure

% \textbf{Universities.} The emphasis on academic integrity and disclosure of AI use in coursework mirrors Article 50 of the EU AI Act,\footnote{\url{https://artificialintelligenceact.eu/article/50/}} which requires AI-generated content to be clearly labelled. Many universities, such as the University of Lisbon, enforce transparency by requiring students to declare AI-assisted work, similar to the AI Act’s obligations for AI system deployers.

% \textbf{News.}

% \textbf{Gaps.} 

% \subsection{OLD TEXT - to be cannibalized}