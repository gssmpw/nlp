%%
%% This is file `sample-sigconf-authordraft.tex',
%% generated with the docstrip utility.
%%
%% The original source files were:
%%
%% samples.dtx  (with options: `all,proceedings,bibtex,authordraft')
%% 
%% IMPORTANT NOTICE:
%% 
%% For the copyright see the source file.
%% 
%% Any modified versions of this file must be renamed
%% with new filenames distinct from sample-sigconf-authordraft.tex.
%% 
%% For distribution of the original source see the terms
%% for copying and modification in the file samples.dtx.
%% 
%% This generated file may be distributed as long as the
%% original source files, as listed above, are part of the
%% same distribution. (The sources need not necessarily be
%% in the same archive or directory.)
%%
%%
%% Commands for TeXCount
%TC:macro \cite [option:text,text]
%TC:macro \citep [option:text,text]
%TC:macro \citet [option:text,text]
%TC:envir table 0 1
%TC:envir table* 0 1
%TC:envir tabular [ignore] word
%TC:envir displaymath 0 word
%TC:envir math 0 word
%TC:envir comment 0 0
%%
%%
%% The first command in your LaTeX source must be the \documentclass
%% command.
%%
%% For submission and review of your manuscript please change the
%% command to \documentclass[manuscript, screen, review]{acmart}.
%%
%% When submitting camera ready or to TAPS, please change the command
%% to \documentclass[sigconf]{acmart} or whichever template is required
%% for your publication.
%%
%%
%\documentclass[sigconf,authordraft]{acmart}
%\documentclass[manuscript,screen,review,anonymous]{acmart}
\documentclass[manuscript]{acmart}
\setcopyright{none}
%%
%% \BibTeX command to typeset BibTeX logo in the docs
\AtBeginDocument{%
  \providecommand\BibTeX{{%
    Bib\TeX}}}

\usepackage{graphicx}
%\usepackage{emoji}
\usepackage{xurl}
\usepackage{xspace}
%\newcommand{\use}{\emoji{white-check-mark}}
%\newcommand{\risk}{\emoji{warning}}
%\newcommand{\rules}{\emoji{scroll}}
\def\aia{\textsc{\texttt{EU-AIA}}\xspace}

\usepackage{scalerel,graphicx,xparse}

\NewDocumentCommand\use{}{\scalerel*{\includegraphics{check}}{X}}
\NewDocumentCommand\risk{}{\scalerel*{\includegraphics{warn.png}}{X}}
\NewDocumentCommand\rules{}{\scalerel*{\includegraphics{rules}}{X}}
\NewDocumentCommand\lock{}{\scalerel*{\includegraphics{lock.png}}{X}}
\NewDocumentCommand\student{}{\scalerel*{\includegraphics{student.png}}{X}}
\NewDocumentCommand\newspaper{}{\scalerel*{\includegraphics{newspaper.png}}{X}}
% \usepackage{fontspec}

%% Rights management information.  This information is sent to you
%% when you complete the rights form.  These commands have SAMPLE
%% values in them; it is your responsibility as an author to replace
%% the commands and values with those provided to you when you
%% complete the rights form.
\setcopyright{acmlicensed}
\copyrightyear{2025}
\acmYear{2025}
\acmDOI{XXXXXXX.XXXXXXX}

%% These commands are for a PROCEEDINGS abstract or paper.
% \acmConference[Conference acronym 'XX]{Make sure to enter the correct
%   conference title from your rights confirmation emai}{June 03--05,
%   2018}{Woodstock, NY}
%%
%%  Uncomment \acmBooktitle if the title of the proceedings is different
%%  from ``Proceedings of ...''!
%%
%%\acmBooktitle{Woodstock '18: ACM Symposium on Neural Gaze Detection,
%%  June 03--05, 2018, Woodstock, NY}
%\acmISBN{978-1-4503-XXXX-X/18/06}


%%
%% Submission ID.
%% Use this when submitting an article to a sponsored event. You'll
%% receive a unique submission ID from the organizers
%% of the event, and this ID should be used as the parameter to this command.
%%\acmSubmissionID{123-A56-BU3}

%%
%% For managing citations, it is recommended to use bibliography
%% files in BibTeX format.
%%
%% You can then either use BibTeX with the ACM-Reference-Format style,
%% or BibLaTeX with the acmnumeric or acmauthoryear sytles, that include
%% support for advanced citation of software artefact from the
%% biblatex-software package, also separately available on CTAN.
%%
%% Look at the sample-*-biblatex.tex files for templates showcasing
%% the biblatex styles.
%%

%%
%% The majority of ACM publications use numbered citations and
%% references.  The command \citestyle{authoryear} switches to the
%% "author year" style.
%%
%% If you are preparing content for an event
%% sponsored by ACM SIGGRAPH, you must use the "author year" style of
%% citations and references.
%% Uncommenting
%% the next command will enable that style.
%%\citestyle{acmauthoryear}


%%
%% end of the preamble, start of the body of the document source.
\begin{document}

%%
%% The "title" command has an optional parameter,
%% allowing the author to define a "short title" to be used in page headers.
\title{AI Policy in Practice: Comparing Active Policies with International Legislation}
\title{AI Policy in Practice: Comparing Organisation Policies with Coming International Legislation (EU AI Act)}
\title{Local Differences, Global Lessons: Insights from Organisation Policies for International Legislation}
%%
%% The "author" command and its associated commands are used to define
%% the authors and their affiliations.
%% Of note is the shared affiliation of the first two authors, and the
%% "authornote" and "authornotemark" commands
%% used to denote shared contribution to the research.
% \author{Ben Trovato}
% \authornote{Both authors contributed equally to this research.}
% \email{trovato@corporation.com}
% \orcid{1234-5678-9012}
% \author{G.K.M. Tobin}
% \authornotemark[1]
% \email{webmaster@marysville-ohio.com}
% \affiliation{%
%   \institution{Institute for Clarity in Documentation}
%   \city{Dublin}
%   \state{Ohio}
%   \country{USA}
% }

\author{Lucie-Aim\'{e}e Kaffee}
\authornote{Both authors contributed equally to this research.}
\affiliation{%
  \institution{Hugging Face}
  \country{\_}
  }
\email{lucie.kaffee@huggingface.co}

\author{Pepa Atanasova}
\authornotemark[1]
\affiliation{%
  \institution{University of Copenhagen}
  %\city{Rocquencourt}
  \country{Denmark}
}
\email{pepa@di.ku.dk}


\author{Anna Rogers}
\affiliation{%
  \institution{IT University of Copenhagen}
  %\city{Rocquencourt}
  \country{Denmark}
}
\email{arog@itu.dk}

%%
%% By default, the full list of authors will be used in the page
%% headers. Often, this list is too long, and will overlap
%% other information printed in the page headers. This command allows
%% the author to define a more concise list
%% of authors' names for this purpose.
%\renewcommand{\shortauthors}{Trovato et al.}

%%
%% The abstract is a short summary of the work to be presented in the
%% article.


\begin{abstract}
The rapid adoption of AI across diverse domains has led to the development of organisational guidelines that vary significantly, even within the same sector. This paper examines AI policies in two domains, news organisations and universities, to understand how bottom-up governance approaches shape AI usage and oversight. By analysing these policies, we identify key areas of convergence and divergence in how organisations address risks such as bias, privacy, misinformation, and accountability. We then explore the implications of these findings for international AI legislation, particularly the EU AI Act, highlighting gaps where practical policy insights could inform regulatory refinements. Our analysis reveals that organisational policies often address issues such as AI literacy, disclosure practices, and environmental impact, areas that are underdeveloped in existing international frameworks. We argue that lessons from domain-specific AI policies can contribute to more adaptive and effective AI governance at the global level. This study provides actionable recommendations for policymakers seeking to bridge the gap between local AI practices and international regulations.\looseness=-1
%
% In the recent years, organisations in diverse domains had to quickly adapt to the challenges posed by generative AI. Yet even within a single domain, such as news organisations, guidelines can diverge significantly or even conflict, reflecting varied interpretations of risk, accountability, and ethical considerations. Moreover, there is often a mismatch between the prescribed uses of AI and the real functionality scope of these models, which further complicates the regulatory landscape. 
% Meanwhile, broader frameworks like the EU AI Act seek to govern AI at an international level, but many key details for its implementation in the local organizations are currently unspecified. 
% We raise the question: could the analysis of local bottom-up guidelines help to identify key points to be discussed and eventually adapted in international legislation?
% %could analysis of local bottom-up guidelines help to identify the key points that need more clarity from the legislators? %Addressing this issue is key to shaping a more coherent, context-sensitive AI governance that balances real-world needs with overarching legislative goals. 
% We survey organisational AI guidelines from two domains (universities and news organisations) and identify the key points on which the local organisations agree and diverge, which may inform the development of both local guidelines and international legislation.
\end{abstract}

%%
%% The code below is generated by the tool at http://dl.acm.org/ccs.cfm.
%% Please copy and paste the code instead of the example below.
%%
% \begin{CCSXML}
% <ccs2012>
%  <concept>
%   <concept_id>00000000.0000000.0000000</concept_id>
%   <concept_desc>Do Not Use This Code, Generate the Correct Terms for Your Paper</concept_desc>
%   <concept_significance>500</concept_significance>
%  </concept>
%  <concept>
%   <concept_id>00000000.00000000.00000000</concept_id>
%   <concept_desc>Do Not Use This Code, Generate the Correct Terms for Your Paper</concept_desc>
%   <concept_significance>300</concept_significance>
%  </concept>
%  <concept>
%   <concept_id>00000000.00000000.00000000</concept_id>
%   <concept_desc>Do Not Use This Code, Generate the Correct Terms for Your Paper</concept_desc>
%   <concept_significance>100</concept_significance>
%  </concept>
%  <concept>
%   <concept_id>00000000.00000000.00000000</concept_id>
%   <concept_desc>Do Not Use This Code, Generate the Correct Terms for Your Paper</concept_desc>
%   <concept_significance>100</concept_significance>
%  </concept>
% </ccs2012>
% \end{CCSXML}

% \ccsdesc[500]{Do Not Use This Code~Generate the Correct Terms for Your Paper}
% \ccsdesc[300]{Do Not Use This Code~Generate the Correct Terms for Your Paper}
% \ccsdesc{Do Not Use This Code~Generate the Correct Terms for Your Paper}
% \ccsdesc[100]{Do Not Use This Code~Generate the Correct Terms for Your Paper}

%%
%% Keywords. The author(s) should pick words that accurately describe
%% the work being presented. Separate the keywords with commas.
\keywords{EU AI Act, AI Legislation, Governance, AI Policy}
%% A "teaser" image appears between the author and affiliation
%% information and the body of the document, and typically spans the
%% page.
%\begin{teaserfigure}
%  \includegraphics[width=\textwidth]{sampleteaser}
%  \caption{Seattle Mariners at Spring Training, 2010.}
%  \Description{Enjoying the baseball game from the third-base
%  seats. Ichiro Suzuki preparing to bat.}
%  \label{fig:teaser}
%\end{teaserfigure}

\received{20 February 2007}
\received[revised]{12 March 2009}
\received[accepted]{5 June 2009}

%%
%% This command processes the author and affiliation and title
%% information and builds the first part of the formatted document.
\maketitle


\section{Introduction}


\begin{figure}[t]
\centering
\includegraphics[width=0.6\columnwidth]{figures/evaluation_desiderata_V5.pdf}
\vspace{-0.5cm}
\caption{\systemName is a platform for conducting realistic evaluations of code LLMs, collecting human preferences of coding models with real users, real tasks, and in realistic environments, aimed at addressing the limitations of existing evaluations.
}
\label{fig:motivation}
\end{figure}

\begin{figure*}[t]
\centering
\includegraphics[width=\textwidth]{figures/system_design_v2.png}
\caption{We introduce \systemName, a VSCode extension to collect human preferences of code directly in a developer's IDE. \systemName enables developers to use code completions from various models. The system comprises a) the interface in the user's IDE which presents paired completions to users (left), b) a sampling strategy that picks model pairs to reduce latency (right, top), and c) a prompting scheme that allows diverse LLMs to perform code completions with high fidelity.
Users can select between the top completion (green box) using \texttt{tab} or the bottom completion (blue box) using \texttt{shift+tab}.}
\label{fig:overview}
\end{figure*}

As model capabilities improve, large language models (LLMs) are increasingly integrated into user environments and workflows.
For example, software developers code with AI in integrated developer environments (IDEs)~\citep{peng2023impact}, doctors rely on notes generated through ambient listening~\citep{oberst2024science}, and lawyers consider case evidence identified by electronic discovery systems~\citep{yang2024beyond}.
Increasing deployment of models in productivity tools demands evaluation that more closely reflects real-world circumstances~\citep{hutchinson2022evaluation, saxon2024benchmarks, kapoor2024ai}.
While newer benchmarks and live platforms incorporate human feedback to capture real-world usage, they almost exclusively focus on evaluating LLMs in chat conversations~\citep{zheng2023judging,dubois2023alpacafarm,chiang2024chatbot, kirk2024the}.
Model evaluation must move beyond chat-based interactions and into specialized user environments.



 

In this work, we focus on evaluating LLM-based coding assistants. 
Despite the popularity of these tools---millions of developers use Github Copilot~\citep{Copilot}---existing
evaluations of the coding capabilities of new models exhibit multiple limitations (Figure~\ref{fig:motivation}, bottom).
Traditional ML benchmarks evaluate LLM capabilities by measuring how well a model can complete static, interview-style coding tasks~\citep{chen2021evaluating,austin2021program,jain2024livecodebench, white2024livebench} and lack \emph{real users}. 
User studies recruit real users to evaluate the effectiveness of LLMs as coding assistants, but are often limited to simple programming tasks as opposed to \emph{real tasks}~\citep{vaithilingam2022expectation,ross2023programmer, mozannar2024realhumaneval}.
Recent efforts to collect human feedback such as Chatbot Arena~\citep{chiang2024chatbot} are still removed from a \emph{realistic environment}, resulting in users and data that deviate from typical software development processes.
We introduce \systemName to address these limitations (Figure~\ref{fig:motivation}, top), and we describe our three main contributions below.


\textbf{We deploy \systemName in-the-wild to collect human preferences on code.} 
\systemName is a Visual Studio Code extension, collecting preferences directly in a developer's IDE within their actual workflow (Figure~\ref{fig:overview}).
\systemName provides developers with code completions, akin to the type of support provided by Github Copilot~\citep{Copilot}. 
Over the past 3 months, \systemName has served over~\completions suggestions from 10 state-of-the-art LLMs, 
gathering \sampleCount~votes from \userCount~users.
To collect user preferences,
\systemName presents a novel interface that shows users paired code completions from two different LLMs, which are determined based on a sampling strategy that aims to 
mitigate latency while preserving coverage across model comparisons.
Additionally, we devise a prompting scheme that allows a diverse set of models to perform code completions with high fidelity.
See Section~\ref{sec:system} and Section~\ref{sec:deployment} for details about system design and deployment respectively.



\textbf{We construct a leaderboard of user preferences and find notable differences from existing static benchmarks and human preference leaderboards.}
In general, we observe that smaller models seem to overperform in static benchmarks compared to our leaderboard, while performance among larger models is mixed (Section~\ref{sec:leaderboard_calculation}).
We attribute these differences to the fact that \systemName is exposed to users and tasks that differ drastically from code evaluations in the past. 
Our data spans 103 programming languages and 24 natural languages as well as a variety of real-world applications and code structures, while static benchmarks tend to focus on a specific programming and natural language and task (e.g. coding competition problems).
Additionally, while all of \systemName interactions contain code contexts and the majority involve infilling tasks, a much smaller fraction of Chatbot Arena's coding tasks contain code context, with infilling tasks appearing even more rarely. 
We analyze our data in depth in Section~\ref{subsec:comparison}.



\textbf{We derive new insights into user preferences of code by analyzing \systemName's diverse and distinct data distribution.}
We compare user preferences across different stratifications of input data (e.g., common versus rare languages) and observe which affect observed preferences most (Section~\ref{sec:analysis}).
For example, while user preferences stay relatively consistent across various programming languages, they differ drastically between different task categories (e.g. frontend/backend versus algorithm design).
We also observe variations in user preference due to different features related to code structure 
(e.g., context length and completion patterns).
We open-source \systemName and release a curated subset of code contexts.
Altogether, our results highlight the necessity of model evaluation in realistic and domain-specific settings.






\section{Background}
\subsection{Sociotechnical Challenges for Organisations Known from AI Research}
\label{sec:challenges}
% \subsection{Sociotechnological challenges known from research on Generative AI}

\begin{table}[t]
%\footnotesize
\begin{tabular}{p{11.7cm}p{2.7cm}}
\toprule
\textbf{Challenge} \& \textbf{Summary} & \textbf{Risk for the org.} \\
\midrule 
\textbf{What is regulated:} what kind of models even fall under the policy? Definitions can be based on training compute \cite{2024-ai-act}, data \cite{RogersLuccioni_2024_Position_Key_Claims_in_LLM_Research_Have_Long_Tail_of_Footnotes}, performance \cite{anderljungFrontierAIRegulation2023} etc. & Guidelines not scoped appropriately \\
\textbf{Detectability \& enforceability}: can we detect when AI models' usage violates the policy? Particularly, when generated content is used without disclosure? At present, no \cite{PuccettiRogersEtAl_2024_AI_News_Content_Farms_Are_Easy_to_Make_and_Hard_to_Detect_Case_Study_in_Italian}. & Guidelines not enforceable \\
\textbf{Factual errors}: the current models cannot reliably reject queries for which they do not have enough information \cite{amayuelas-etal-2024-knowledge}, and may output plausible-sounding but false results that are hard to identify and check \cite{zhang2023sirenssongaiocean,hicksChatgptBullshit2024}. Retrieval-augmented generation still has this problem \cite{mehrotraPerplexityBullshitMachine}. & Losing credibility and reputation \\
\textbf{Unsafe models}: in spite of attempts to force the models to follow certain content policies \cite{ouyang2022training}, the models can still violate them \cite{DerczynskiGalinkinEtAl_2024_garak_Framework_for_Security_Probing_Large_Language_Models}, and this training can even decrease the quality in some aspects \cite{10.1093/polsoc/puae020,casper2023open}. & Exposing employees to toxic outputs \\
\textbf{Privacy and security risks}: employees using non-local generative AI models may expose sensitive data from themselves and their organizations to the entity controlling such models \cite{Kim_2023_Amazon_warns_employees_not_to_share_confidential_information_with_ChatGPT_after_seeing_cases_where_its_answer_closely_matches_existing_material_from_inside_company} or third-party attackers \cite{WuZhangEtAl_2024_New_Era_in_LLM_Security_Exploring_Security_Concerns_in_Real-World_LLM-based_Systems}. Platform plugins may also increase vulnerabilities \cite{iqbal2024llm}. & Exposing sensitive data \\
\textbf{Misleading marketing claims}:  %With the training data too-big-to-inspect \cite{bender2021dangers}, the benchmark results may be compromised by test set contamination \cite{RogersLuccioni_2024_Position_Key_Claims_in_LLM_Research_Have_Long_Tail_of_Footnotes}. %But based on public benchmark numbers and especially claims of ``emergence'' \cite{RogersLuccioni_2024_Position_Key_Claims_in_LLM_Research_Have_Long_Tail_of_Footnotes}, 
employees may believe the claims of AI  model ``capabilities'' and trust the machine too much \cite{KheraSimonEtAl_2023_Automation_Bias_and_Assistive_AI_Risk_of_Harm_From_AI-Driven_Clinical_Decision_Support}, even though the benchmark results may be compromised by methodological problems and test set contamination \cite{RogersLuccioni_2024_Position_Key_Claims_in_LLM_Research_Have_Long_Tail_of_Footnotes}. %Automation also tends to make easy tasks easier and harder tasks harder \cite{SimkuteTankelevitchEtAl_Ironies_of_Generative_AI_Understanding_and_Mitigating_Productivity_Loss_in_Human-AI_Interaction}, and . 
& Degradation in the outputs of the organization \\
\textbf{Transparency \& accountability}: the social and legal norms on disclosing AI ``assistance'' and taking responsibility for the resulting text have not yet settled. The popular providers of these models do not accept responsibility for any faults in the output\footnote{}. & Public blame for any missteps \\
\textbf{Bias \& inequity}: The social biases in AI systems are well-documented  \cite{bolukbasi2016man,nadeem-etal-2021-stereoset,marchiori-manerba-etal-2024-social,stanczak2023quantifying,hutchinson-etal-2020-social,bender2021dangers,sharma2024generative}, %LLM training data may overrepresent some groups and underrepresent others \cite{bender2021dangers,sharma2024generative}. 
and the use of such models may reinforce misrepresentations in the society. & Discrimination, ethical code violation \\%, and the resulting models may fail to adequately represent minorities even when explicitly steered to do so \cite{santurkar2023whose}. \\ 
\textbf{Explainability}: checking model outputs would be easier if they were accompanied by rationales, the current interpretability methods are not faithful to the model’s actual decision-making \cite{lanham2023measuring,atanasova-etal-2023-faithfulness}. & Trusting unreliable solutions \\
\textbf{Brittleness}: Generative models perform worse outside of their training distribution \cite{McCoyYaoEtAl_2024_Embers_of_autoregression_show_how_large_language_models_are_shaped_by_problem_they_are_trained_to_solve,McCoyYaoEtAl_2024_When_language_model_is_optimized_for_reasoning_does_it_still_show_embers_of_autoregression_analysis_of_OpenAI_o1}. For language models, this includes changes in both language (idiolects, dialects, diachronic changes), content (e.g. evolving world knowledge), and slight variations in prompt formulation and examples \cite{zhu2023promptbench,LuBartoloEtAl_2022_Fantastically_Ordered_Prompts_and_Where_to_Find_Them_Overcoming_Few-Shot_Prompt_Order_Sensitivity}. & Employees wasting time and/or getting poor results \\
\textbf{Risks to creativity}: AI systems may generate unseen sequences of words, but their ``creativity'' is combinatorial, often lacking diversity, feasibility, and depth \cite{si2024can,padmakumar2024does}, and further degraded in languages other than English \cite{marco-etal-2024-pron}. Exposure to AI assistance could \textit{decrease} human creativity and diversity of ideas in non-assisted tasks \cite{kumar2024humancreativityagellms}. & Degradation in the outputs of the organization and its existing human resources \\
\textbf{Credit \& Attribution}: AI systems are commonly trained on copyrighted texts\cite{Gray_2024_OpenAI_Claims_it_is_Impossible_to_Train_AI_Without_Using_Copyrighted_Content} without author consent%, and memorization of high-quality human-authored texts is known to correlate with model performance \cite{liangHolisticEvaluationLanguage2022}
. This practice triggered multiple lawsuits  \cite{BrittainBrittain_2023_Lawsuits_accuse_AI_content_creators_of_misusing_copyrighted_work,Vynck_2023_Game_of_Thrones_author_and_others_accuse_ChatGPT_maker_of_theft_in_lawsuit,JosephSaveriLawFirmButterick_2022_GitHub_Copilot_investigation,GrynbaumMac_2023_Times_Sues_OpenAI_and_Microsoft_Over_AI_Use_of_Copyrighted_Work,Panwar_2025_Generative_AI_and_Copyright_Issues_Globally_ANI_Media_OpenAI_TechPolicyPress}, protests from the creators \cite{Heikkila_2022_This_artist_is_dominating_AI-generated_art_And_hes_not_happy_about_it,More_than_15000_Authors_Sign_Authors_Guild_Letter_Calling_on_AI_Industry_Leaders_to_Protect_Writers}, and questions about the credit for the author of an ``assisted'' text \cite{FormosaBankinsEtAl_2024_Can_ChatGPT_be_author_Generative_AI_creative_writing_assistance_and_perceptions_of_authorship_creatorship_responsibility_and_disclosure}. & Legal exposure, violating plagiarism policies/principles \\
\textbf{Carbon emissions:} The current AI systems are environmentally costly for both training and inference \cite{luccioniCountingCarbonSurvey2023,dodge2022measuring,bouza2023estimate,liMakingAILess2023}, and workflows that significantly rely on them would have adverse effect on climate action. & Not meeting sustainability goals\\
\bottomrule
\end{tabular}
\caption{Major sociotechnical challenges for organizations relying on the current AI technology}
\label{fig:sociotech-challenges}
\end{table}
%Generative models are currently rushed into many applications and have been adopted by many users. This poses a range of sociotechnical challenges, stemming from the technical imperfections of these models, legal uncertainty, questions about the authorship in ``AI-assisted'' content, misleading marketing, and the issues related to how people use such models. This work does not aim to provide a comprehensive survey, but we list some of the key issues previously identified in academic literature in \autoref{fig:sociotech-challenges}, together with the possible risks to organizations whose employees rely on this technology.\looseness=-1
% \subsection{Implications from Policies for Research}
%
AI research literature point to numerous sociotechnical challenges for organisations relying on the current AI technology. Our work does not aim to provide a comprehensive survey, but we list the issues that we identified through literature review in \autoref{fig:sociotech-challenges}, together with the possible risks to organisations whose employees rely on this technology. This list serves as background to the types of problems that organisational policy or regulatory frameworks may identify as issues that need addressing.


%The advancements in generative artificial intelligence (GenAI) models, particularly through the pre-training of large-scale architectures with extensive parameters on vast datasets, have significantly enhanced their capabilities across a broad range of domains. In particular, these models demonstrate promising strengths in contextual understanding; text, image, and video generation; multilingual applications; and zero-shot or few-shot adaptability to new tasks. Such abilities have facilitated their widespread adoption in diverse areas.

%Nevertheless, GenAI models still exhibit limited capabilities in certain tasks, especially those that demand deeper expertise or domain-specific knowledge \cite{rein2023gpqa}. While ongoing developments can improve such limited capabilities and make GenAI models applicable to an ever-growing set of tasks, their adoption also brings considerable challenges and risks. Addressing these issues is essential to ensure alignment with existing active policies (e.g., those within educational institutions or news organizations) and international regulations (e.g., the EU AI Act). Below, we outline the critical challenges associated with the use of GenAI models, as identified by existing research and which should be considered in the development of comprehensive and responsible policies and regulations.

%\paragraph{Inaccuracy and Hallucination}
%One of the most pressing challenges of GenAI models is their tendency to generate inaccurate or ``hallucinated'' content, where outputs are not grounded in training data or established facts \cite{liu-etal-2023-evaluating,10.1145/3703155}. Because their pre-training corpora often come from automatically scraped web sources--many of which include fabricated, outdated, previously generated, or biased information models can produce responses that \textit{conflict with real-world knowledge, user-provided input, or even their own previously generated content}. This issue poses particular risks in domains that demand high accuracy, such as legal practices, where considering the latest changes in the law is of critical importance \cite{cheong2024not}. Moreover, GenAI models may offer \textit{plausible-sounding but false information}, making it difficult for both automated systems and human reviewers to detect these subtle errors \cite{zhang2023sirenssongaiocean}. Furthermore, models often \textit{struggle to classify whether a question is within their scope of knowledge}, whereas even smaller or open-source models perform near random on such tasks \cite{amayuelas-etal-2024-knowledge}. Additionally, the reinforcement learning from human feedback (RLHF) technique \cite{ouyang2022training} for aligning model responses with human preferences, the vague knowledge boundary \cite{ren2024investigatingfactualknowledgeboundary}, and the inherently black-box nature of many GenAI models \cite{sun2022black} further complicate the detection, explanation, and mitigation of hallucinations. While plugins or vector databases that store validated, up-to-date domain information (e.g., legal statutes) could alleviate these issues, their integration must be both reliable and user-friendly. Finally, as GenAI models are expected to excel in multi-task, multi-lingual, and multi-domain environments \cite{bang-etal-2023-multitask}, evaluating and mitigating their hallucinations becomes even more challenging.

%\paragraph{Safety and Alignment}
%The \textit{misalignment between the training objectives of GenAI models and their desired behaviour} poses significant safety risks. For instance, while a large language model (LLM) may be optimized to predict the next word from a massive corpus--regardless of the content quality of that corpus--users typically want the model to generate factual, useful information rather than harmful or false content. This gap underscores the challenge of ``scalable oversight'', where evaluating and controlling highly capable models in complex tasks remains unsolved  \cite{amodei2016concreteproblemsaisafety,leike2017aisafetygridworlds}. 
% More importantly, simply enumerating known or planned use cases of foundation models is not sufficient to capture the full range of ways they might be deployed. 
%Techniques such as RLHF have been introduced to align models more closely with human values, yet these methods face fundamental limitations \cite{casper2023open} such as increased hallucination, ideological favouritism, sycophancy, or even resistance being shut down. In addition, they may inadvertently incentivize ``reward hacking'', in which the model learns to exploit the reward system to achieve high scores without actually solving the desired objectives \cite{10.1093/polsoc/puae020}. These risks are unintentional but become more concerning if the alignment process is corrupted by adversarial actors. Overall, there is concern that the alignment of GenAI models could instead lead to prioritizing simple engagement metrics at the expense of broader societal or consumer well-being.

%\paragraph{Security and Privacy Concerns}
%Generative models raise significant security and privacy issues \cite{iqbal2024llm, yao2024survey}. Interactions with large language models (LLMs) often \textit{lack privacy protections}, making their contents vulnerable to subsequent discovery or adversarial exploitation. Even when an LLM operates locally, chat records typically remain unprotected unless explicitly shielded from disclosure.
%Moreover, the training data for these models often comes from uncurated web sources \textit{susceptible to malicious ``poisoning''} \cite{10.5555/3042573.3042761}, potentially yielding harmful outputs. For instance, adversaries may inject hateful speech into just a few online posts to manipulate a foundation model’s training data, and even small-scale injections can significantly corrupt outputs \cite{schuster2021you}. Compounding these risks, \textit{the dual-use nature of LLMs allows them to be adapted for malicious purposes}, such as disinformation campaigns or targeted extortion attempts \cite{kaffee-etal-2023-thorny}.
% , and raises concerns about dual use, wherein models are repurposed beyond their originally foreseen tasks, potentially enabling overlearning or adversarial reprogramming \cite{elsayed2018adversarial}. 
% Furthermore, multimodality can expand a model’s attack surface by allowing cross-modal inconsistencies to be exploited, as shown when an apple labeled “iPod” was misclassified by CLIP. 
%Beyond data poisoning, the \textit{confidentiality, integrity, and availability} of LLM-based systems can also be undermined by inference or reconstruction attacks, adversarial examples \cite{biggio2013evasion, szegedy2013intriguing}, and resource-depletion exploits \cite{shumailov2021sponge, hong2021a}.

%\paragraph{Transparency and Accountability}
%The opacity characterising the development and application of generative models creates significant challenges for transparency and accountability. Even when GenAI model weight parameters are made public, \textit{many providers withhold critical information} regarding their training and fine-tuning procedures, preventing effective inspection and regulation of these models \cite{bender2021dangers}. Furthermore, beyond merely determining whether models produce correct outcomes, it is equally important to assess whether they do so for the right reasons. Current efforts to align model outputs with human preferences, for instance, have shown that fake, misleading alignment and sycophantic behaviour can emerge, raising concerns about the motivation and capabilities of GenAI models \cite{greenblatt2024alignmentfakinglargelanguage,vanderweij2024aisandbagginglanguagemodels}. Additionally, models have been found to produce increasingly persuasive yet potentially unfaithful and misleading content compounds the difficulty of explaining its behaviour \cite{rogiers2024persuasion,bommasani2021opportunities}. While models can supply explanations of their decisions, such as chain-of-thought responses \cite{wei2022chain}, these rationales are often found to be unfaithful to the model’s actual internal decision-making \cite{lanham2023measuring,atanasova-etal-2023-faithfulness}. Exploring the underlying reasons behind a model’s outputs thus remains an open research challenge, critical to ensure accountability among model providers fostering transparent usage of GenAI systems and creating a regulatory environment in which scientists, policymakers, and end-users can confidently employ GenAI technologies.

%\paragraph{Biases and Inequity}
% , including cultural and linguistic minorities, neurodivergent individuals, and other nontypical users. 
%GenAI systems can perpetuate biases that can often manifest as stereotypes or attitudes \cite{bolukbasi2016man, nadeem-etal-2021-stereoset, marchiori-manerba-etal-2024-social, stanczak2023quantifying,hutchinson-etal-2020-social}, and which can propagate through downstream models and reinforce misrepresentations in society. Overrepresentation of majority voices is another concern, resulting in the homogenization and the exclusion of minority viewpoints -- an echo chamber effect that may be further exacerbated if reliance on generative AI becomes widespread \cite{sharma2024generative}. Furthermore, studies have shown that current large language models (LLMs) often overlook certain demographic groups, such as individuals aged 65+ or widowed, even when explicitly steered to represent them \cite{santurkar2023whose}. 
% Notably, technical vulnerabilities, such as ``jailbreak'' attacks, reveal the fragility of safety training protocols, suggesting that merely scaling up these methods without fundamentally revising optimization objectives may exacerbate mismatched generalization. 
%Mitigating these inequities and biases requires (1) ensuring equitable access and skill development for using AI responsibly, (2) advancing technical research to reliably attribute and address the root sources of bias, and (3) implementing effective safeguards so that generative AI promotes fairness rather than reinforcing discrimination. Finally,  GenAI systems risk exacerbating existing inequities by privileging those with greater access and technical skills required to use GenAI models, thereby widening the divide between privileged and underprivileged populations.



%\paragraph{Known GenAI Limitations} Recent evaluations of GenAI models highlight a range of shortcomings that \textit{limit their performance in real-world scenarios}. Although LLMs can generate ideas that sometimes surpass human experts in perceived novelty, their \textit{creativity largely remains combinatorial rather than genuinely groundbreaking, often lacking diversity, feasibility, and depth} \cite{si2024can, padmakumar2024does}. LLM models further exhibit worse creativity skills in languages different from English \cite{marco-etal-2024-pron}. Additionally, exposure to LLM assistance in creative tasks has been found to lead to decreased creativity and diversity of ideas in subsequent non-assisted tasks \cite{kumar2024humancreativityagellms}. The latter reflects an increasing concern regarding the over-reliance on GenAI, leading to the erosion of critical thinking, specialized skills, and homogenisation of individual voices, especially in educational settings \cite{zhai2024effects}.
%LLMs face additional technical limitations. They exhibit limited robustness to distribution shifts, such as evolving world knowledge or language drift, which jeopardises the reliability of their outputs in high-stakes environments (e.g., in legal or medical applications) \cite{yuan2023revisiting}. Furthermore, LLMs have been increasingly used by being assignment personas, which has been found to unexpectedly skew their performance and introduce biases, highlighting broader concerns about stereotyping and privacy and personalisation applications of GenAI \cite{zhang2024personalization}. Existing evaluations also reveal subpar performance, among others, in abstract reasoning tasks \cite{gendron2023large}, event semantics \cite{tao2023eveval}, non-Latin script contexts \cite{lai-etal-2023-chatgpt}, and multimodal data. LLMs are also highly sensitive to adversarial or even slight prompt variations \cite{zhu2023promptbench}, calling for further research into prompt engineering and model robustness. Overall, while LLMs offer considerable promise, substantial work remains to address these limitations, among others, and improve their performance, particularly regarding creative potential, interpretative flexibility, and ethical concerns.

%The challenges and limitations outlined above underscore the complexity of aligning generative models with both local and global guidelines. 
\subsection{EU AI Act}
\label{sec:aia}
Likewise, the scope of this work does not allow for a detailed discussion of \aia, but for our purposes a key factor is that it implements a risk-based approach to regulating AI, in which systems are categorised systems by their potential threats. AI systems deemed ``unacceptable'' for their \textit{potential risks to EU values and fundamental rights}, such as AI systems performing social scoring, are outright banned. Systems considered ``high-risk'', including those used in critical infrastructure, law enforcement, employment, and education, face stringent ex-ante rules and are also subject to post-market monitoring. Universities, one of our case policy case studies, are in the education sector. Perhaps surprisingly, news is not identified as a high-risk application, given its possible consequences in election cycles. For general- and minimal-risk AI, the \aia primarily relies on transparency obligations, requiring that users be informed when they are interacting with or viewing outputs from certain AI systems.

There is also an important distinction between ``providers'' and ``deployers'', each bearing distinct responsibilities. Providers are entities that develop an AI model, place it on the market under their name or trademark (e.g., `Llama 3' by Meta), or substantially modify an existing AI model. They must ensure the model’s compliance with relevant standards, document design choices and data governance procedures, and, where required, undergo conformity assessments. If the model is a general purpose AI model (as Llama), it falls under additional regulations. Deployers, on the other hand, are the organisations or individuals that integrate and use the AI models in their systems (e.g. a university that creates an AI system using Llama to provide access to it to its staff and students). Their obligations typically focus on correct implementation, ensuring that the system is used within the scope of its intended purposes and monitoring its real-world performance for safety, accuracy, and potential harm.
%
In the case of universities, they can be both deployers of AI systems, if they offer their own AI systems, as well as end-users, subscribing to other AI deployers services, such as `ChatGPT'. 
%
For news organisations, in a majority of cases, the policies assume that they subscribe to an external AI deployer's systems.
%
Downstream users and large-scale distributors of AI generated content, as could be the case for news organisations, do not currently have obligations under the \aia.
%
In this article, we focus on the \aia and exclude discussions about related EU regulation, such as the \textit{Directive on Copyright in the Digital Single Market}.


\putsec{related}{Related Work}

\noindent \textbf{Efficient Radiance Field Rendering.}
%
The introduction of Neural Radiance Fields (NeRF)~\cite{mil:sri20} has
generated significant interest in efficient 3D scene representation and
rendering for radiance fields.
%
Over the past years, there has been a large amount of research aimed at
accelerating NeRFs through algorithmic or software
optimizations~\cite{mul:eva22,fri:yu22,che:fun23,sun:sun22}, and the
development of hardware
accelerators~\cite{lee:cho23,li:li23,son:wen23,mub:kan23,fen:liu24}.
%
The state-of-the-art method, 3D Gaussian splatting~\cite{ker:kop23}, has
further fueled interest in accelerating radiance field
rendering~\cite{rad:ste24,lee:lee24,nie:stu24,lee:rho24,ham:mel24} as it
employs rasterization primitives that can be rendered much faster than NeRFs.
%
However, previous research focused on software graphics rendering on
programmable cores or building dedicated hardware accelerators. In contrast,
\name{} investigates the potential of efficient radiance field rendering while
utilizing fixed-function units in graphics hardware.
%
To our knowledge, this is the first work that assesses the performance
implications of rendering Gaussian-based radiance fields on the hardware
graphics pipeline with software and hardware optimizations.

%%%%%%%%%%%%%%%%%%%%%%%%%%%%%%%%%%%%%%%%%%%%%%%%%%%%%%%%%%%%%%%%%%%%%%%%%%
\myparagraph{Enhancing Graphics Rendering Hardware.}
%
The performance advantage of executing graphics rendering on either
programmable shader cores or fixed-function units varies depending on the
rendering methods and hardware designs.
%
Previous studies have explored the performance implication of graphics hardware
design by developing simulation infrastructures for graphics
workloads~\cite{bar:gon06,gub:aam19,tin:sax23,arn:par13}.
%
Additionally, several studies have aimed to improve the performance of
special-purpose hardware such as ray tracing units in graphics
hardware~\cite{cho:now23,liu:cha21} and proposed hardware accelerators for
graphics applications~\cite{lu:hua17,ram:gri09}.
%
In contrast to these works, which primarily evaluate traditional graphics
workloads, our work focuses on improving the performance of volume rendering
workloads, such as Gaussian splatting, which require blending a huge number of
fragments per pixel.

%%%%%%%%%%%%%%%%%%%%%%%%%%%%%%%%%%%%%%%%%%%%%%%%%%%%%%%%%%%%%%%%%%%%%%%%%%
%
In the context of multi-sample anti-aliasing, prior work proposed reducing the
amount of redundant shading by merging fragments from adjacent triangles in a
mesh at the quad granularity~\cite{fat:bou10}.
%
While both our work and quad-fragment merging (QFM)~\cite{fat:bou10} aim to
reduce operations by merging quads, our proposed technique differs from QFM in
many aspects.
%
Our method aims to blend \emph{overlapping primitives} along the depth
direction and applies to quads from any primitive. In contrast, QFM merges quad
fragments from small (e.g., pixel-sized) triangles that \emph{share} an edge
(i.e., \emph{connected}, \emph{non-overlapping} triangles).
%
As such, QFM is not applicable to the scenes consisting of a number of
unconnected transparent triangles, such as those in 3D Gaussian splatting.
%
In addition, our method computes the \emph{exact} color for each pixel by
offloading blending operations from ROPs to shader units, whereas QFM
\emph{approximates} pixel colors by using the color from one triangle when
multiple triangles are merged into a single quad.



\section{Research Methodology}~\label{sec:Methodology}

In this section, we discuss the process of conducting our systematic review, e.g., our search strategy for data extraction of relevant studies, based on the guidelines of Kitchenham et al.~\cite{kitchenham2022segress} to conduct SLRs and Petersen et al.~\cite{PETERSEN20151} to conduct systematic mapping studies (SMSs) in Software Engineering. In this systematic review, we divide our work into a four-stage procedure, including planning, conducting, building a taxonomy, and reporting the review, illustrated in Fig.~\ref{fig:search}. The four stages are as follows: (1) the \emph{planning} stage involved identifying research questions (RQs) and specifying the detailed research plan for the study; (2) the \emph{conducting} stage involved analyzing and synthesizing the existing primary studies to answer the research questions; (3) the \emph{taxonomy} stage was introduced to optimize the data extraction results and consolidate a taxonomy schema for REDAST methodology; (4) the \emph{reporting} stage involved the reviewing, concluding and reporting the final result of our study.

\begin{figure}[!t]
    \centering
    \includegraphics[width=1\linewidth]{fig/methodology/searching-process.drawio.pdf}
    \caption{Systematic Literature Review Process}
    \label{fig:search}
\end{figure}

\subsection{Research Questions}
In this study, we developed five research questions (RQs) to identify the input and output, analyze technologies, evaluate metrics, identify challenges, and identify potential opportunities. 

\textbf{RQ1. What are the input configurations, formats, and notations used in the requirements in requirements-driven
automated software testing?} In requirements-driven testing, the input is some form of requirements specification -- which can vary significantly. RQ1 maps the input for REDAST and reports on the comparison among different formats for requirements specification.

\textbf{RQ2. What are the frameworks, tools, processing methods, and transformation techniques used in requirements-driven automated software testing studies?} RQ2 explores the technical solutions from requirements to generated artifacts, e.g., rule-based transformation applying natural language processing (NLP) pipelines and deep learning (DL) techniques, where we additionally discuss the potential intermediate representation and additional input for the transformation process.

\textbf{RQ3. What are the test formats and coverage criteria used in the requirements-driven automated software
testing process?} RQ3 focuses on identifying the formulation of generated artifacts (i.e., the final output). We map the adopted test formats and analyze their characteristics in the REDAST process.

\textbf{RQ4. How do existing studies evaluate the generated test artifacts in the requirements-driven automated software testing process?} RQ4 identifies the evaluation datasets, metrics, and case study methodologies in the selected papers. This aims to understand how researchers assess the effectiveness, accuracy, and practical applicability of the generated test artifacts.

\textbf{RQ5. What are the limitations and challenges of existing requirements-driven automated software testing methods in the current era?} RQ5 addresses the limitations and challenges of existing studies while exploring future directions in the current era of technology development. %It particularly highlights the potential benefits of advanced LLMs and examines their capacity to meet the high expectations placed on these cutting-edge language modeling technologies. %\textcolor{blue}{CA: Do we really need to focus on LLMs? TBD.} \textcolor{orange}{FW: About LLMs, I removed the direct emphase in RQ5 but kept the discussion in RQ5 and the solution section. I think that would be more appropriate.}

\subsection{Searching Strategy}

The overview of the search process is exhibited in Fig. \ref{fig:papers}, which includes all the details of our search steps.
\begin{table}[!ht]
\caption{List of Search Terms}
\label{table:search_term}
\begin{tabularx}{\textwidth}{lX}
\hline
\textbf{Terms Group} & \textbf{Terms} \\ \hline
Test Group & test* \\
Requirement Group & requirement* OR use case* OR user stor* OR specification* \\
Software Group & software* OR system* \\
Method Group & generat* OR deriv* OR map* OR creat* OR extract* OR design* OR priorit* OR construct* OR transform* \\ \hline
\end{tabularx}
\end{table}

\begin{figure}
    \centering
    \includegraphics[width=1\linewidth]{fig/methodology/search-papers.drawio.pdf}
    \caption{Study Search Process}
    \label{fig:papers}
\end{figure}

\subsubsection{Search String Formulation}
Our research questions (RQs) guided the identification of the main search terms. We designed our search string with generic keywords to avoid missing out on any related papers, where four groups of search terms are included, namely ``test group'', ``requirement group'', ``software group'', and ``method group''. In order to capture all the expressions of the search terms, we use wildcards to match the appendix of the word, e.g., ``test*'' can capture ``testing'', ``tests'' and so on. The search terms are listed in Table~\ref{table:search_term}, decided after iterative discussion and refinement among all the authors. As a result, we finally formed the search string as follows:


\hangindent=1.5em
 \textbf{ON ABSTRACT} ((``test*'') \textbf{AND} (``requirement*'' \textbf{OR} ``use case*'' \textbf{OR} ``user stor*'' \textbf{OR} ``specifications'') \textbf{AND} (``software*'' \textbf{OR} ``system*'') \textbf{AND} (``generat*'' \textbf{OR} ``deriv*'' \textbf{OR} ``map*'' \textbf{OR} ``creat*'' \textbf{OR} ``extract*'' \textbf{OR} ``design*'' \textbf{OR} ``priorit*'' \textbf{OR} ``construct*'' \textbf{OR} ``transform*''))

The search process was conducted in September 2024, and therefore, the search results reflect studies available up to that date. We conducted the search process on six online databases: IEEE Xplore, ACM Digital Library, Wiley, Scopus, Web of Science, and Science Direct. However, some databases were incompatible with our default search string in the following situations: (1) unsupported for searching within abstract, such as Scopus, and (2) limited search terms, such as ScienceDirect. Here, for (1) situation, we searched within the title, keyword, and abstract, and for (2) situation, we separately executed the search and removed the duplicate papers in the merging process. 

\subsubsection{Automated Searching and Duplicate Removal}
We used advanced search to execute our search string within our selected databases, following our designed selection criteria in Table \ref{table:selection}. The first search returned 27,333 papers. Specifically for the duplicate removal, we used a Python script to remove (1) overlapped search results among multiple databases and (2) conference or workshop papers, also found with the same title and authors in the other journals. After duplicate removal, we obtained 21,652 papers for further filtering.

\begin{table*}[]
\caption{Selection Criteria}
\label{table:selection}
\begin{tabularx}{\textwidth}{lX}
\hline
\textbf{Criterion ID} & \textbf{Criterion Description} \\ \hline
S01          & Papers written in English. \\
S02-1        & Papers in the subjects of "Computer Science" or "Software Engineering". \\
S02-2        & Papers published on software testing-related issues. \\
S03          & Papers published from 1991 to the present. \\ 
S04          & Papers with accessible full text. \\ \hline
\end{tabularx}
\end{table*}

\begin{table*}[]
\small
\caption{Inclusion and Exclusion Criteria}
\label{table:criteria}
\begin{tabularx}{\textwidth}{lX}
\hline
\textbf{ID}  & \textbf{Description} \\ \hline
\multicolumn{2}{l}{\textbf{Inclusion Criteria}} \\ \hline
I01 & Papers about requirements-driven automated system testing or acceptance testing generation, or studies that generate system-testing-related artifacts. \\
I02 & Peer-reviewed studies that have been used in academia with references from literature. \\ \hline
\multicolumn{2}{l}{\textbf{Exclusion Criteria}} \\ \hline
E01 & Studies that only support automated code generation, but not test-artifact generation. \\
E02 & Studies that do not use requirements-related information as an input. \\
E03 & Papers with fewer than 5 pages (1-4 pages). \\
E04 & Non-primary studies (secondary or tertiary studies). \\
E05 & Vision papers and grey literature (unpublished work), books (chapters), posters, discussions, opinions, keynotes, magazine articles, experience, and comparison papers. \\ \hline
\end{tabularx}
\end{table*}

\subsubsection{Filtering Process}

In this step, we filtered a total of 21,652 papers using the inclusion and exclusion criteria outlined in Table \ref{table:criteria}. This process was primarily carried out by the first and second authors. Our criteria are structured at different levels, facilitating a multi-step filtering process. This approach involves applying various criteria in three distinct phases. We employed a cross-verification method involving (1) the first and second authors and (2) the other authors. Initially, the filtering was conducted separately by the first and second authors. After cross-verifying their results, the results were then reviewed and discussed further by the other authors for final decision-making. We widely adopted this verification strategy within the filtering stages. During the filtering process, we managed our paper list using a BibTeX file and categorized the papers with color-coding through BibTeX management software\footnote{\url{https://bibdesk.sourceforge.io/}}, i.e., “red” for irrelevant papers, “yellow” for potentially relevant papers, and “blue” for relevant papers. This color-coding system facilitated the organization and review of papers according to their relevance.

The screening process is shown below,
\begin{itemize}
    \item \textbf{1st-round Filtering} was based on the title and abstract, using the criteria I01 and E01. At this stage, the number of papers was reduced from 21,652 to 9,071.
    \item \textbf{2nd-round Filtering}. We attempted to include requirements-related papers based on E02 on the title and abstract level, which resulted from 9,071 to 4,071 papers. We excluded all the papers that did not focus on requirements-related information as an input or only mentioned the term ``requirements'' but did not refer to the requirements specification.
    \item \textbf{3rd-round Filtering}. We selectively reviewed the content of papers identified as potentially relevant to requirements-driven automated test generation. This process resulted in 162 papers for further analysis.
\end{itemize}
Note that, especially for third-round filtering, we aimed to include as many relevant papers as possible, even borderline cases, according to our criteria. The results were then discussed iteratively among all the authors to reach a consensus.

\subsubsection{Snowballing}

Snowballing is necessary for identifying papers that may have been missed during the automated search. Following the guidelines by Wohlin~\cite{wohlin2014guidelines}, we conducted both forward and backward snowballing. As a result, we identified 24 additional papers through this process.

\subsubsection{Data Extraction}

Based on the formulated research questions (RQs), we designed 38 data extraction questions\footnote{\url{https://drive.google.com/file/d/1yjy-59Juu9L3WHaOPu-XQo-j-HHGTbx_/view?usp=sharing}} and created a Google Form to collect the required information from the relevant papers. The questions included 30 short-answer questions, six checkbox questions, and two selection questions. The data extraction was organized into five sections: (1) basic information: fundamental details such as title, author, venue, etc.; (2) open information: insights on motivation, limitations, challenges, etc.; (3) requirements: requirements format, notation, and related aspects; (4) methodology: details, including immediate representation and technique support; (5) test-related information: test format(s), coverage, and related elements. Similar to the filtering process, the first and second authors conducted the data extraction and then forwarded the results to the other authors to initiate the review meeting.

\subsubsection{Quality Assessment}

During the data extraction process, we encountered papers with insufficient information. To address this, we conducted a quality assessment in parallel to ensure the relevance of the papers to our objectives. This approach, also adopted in previous secondary studies~\cite{shamsujjoha2021developing, naveed2024model}, involved designing a set of assessment questions based on guidelines by Kitchenham et al.~\cite{kitchenham2022segress}. The quality assessment questions in our study are shown below:
\begin{itemize}
    \item \textbf{QA1}. Does this study clearly state \emph{how} requirements drive automated test generation?
    \item \textbf{QA2}. Does this study clearly state the \emph{aim} of REDAST?
    \item \textbf{QA3}. Does this study enable \emph{automation} in test generation?
    \item \textbf{QA4}. Does this study demonstrate the usability of the method from the perspective of methodology explanation, discussion, case examples, and experiments?
\end{itemize}
QA4 originates from an open perspective in the review process, where we focused on evaluation, discussion, and explanation. Our review also examined the study’s overall structure, including the methodology description, case studies, experiments, and analyses. The detailed results of the quality assessment are provided in the Appendix. Following this assessment, the final data extraction was based on 156 papers.

% \begin{table}[]
% \begin{tabular}{ll}
% \hline
% QA ID & QA Questions                                             \\ \hline
% Q01   & Does this study clearly state its aims?                  \\
% Q02   & Does this study clearly describe its methodology?        \\
% Q03   & Does this study involve automated test generation?       \\
% Q04   & Does this study include a promising evaluation?          \\
% Q05   & Does this study demonstrate the usability of the method? \\ \hline
% \end{tabular}%
% \caption{Questions for Quality Assessment}
% \label{table:qa}
% \end{table}

% automated quality assessment

% \textcolor{blue}{CA: Our search strategy focused on identifying requirements types first. We covered several sources, e.g., ~\cite{Pohl:11,wagner2019status} to identify different formats and notations of specifying requirements. However, this came out to be a long list, e.g., free-form NL requirements, semi-formal UML models, free-from textual use case models, UML class diagrams, UML activity diagrams, and so on. In this paper, we attempted to primarily focus on requirements-related aspects and not design-level information. Hence, we generalised our search string to include generic keywords, e.g., requirement*, use case*, and user stor*. We did so to avoid missing out on any papers, bringing too restrictive in our search strategy, and not creating a too-generic search string with all the aforementioned formats to avoid getting results beyond our review's scope.}


%% Use \subsection commands to start a subsection.



%\subsection{Study Selection}

% In this step, we further looked into the content of searched papers using our search strategy and applied our inclusion and exclusion criteria. Our filtering strategy aimed to pinpoint studies focused on requirements-driven system-level testing. Recognizing the presence of irrelevant papers in our search results, we established detailed selection criteria for preliminary inclusion and exclusion, as shown in Table \ref{table: criteria}. Specifically, we further developed the taxonomy schema to exclude two types of studies that did not meet the requirements for system-level testing: (1) studies supporting specification-driven test generation, such as UML-driven test generation, rather than requirements-driven testing, and (2) studies focusing on code-based test generation, such as requirement-driven code generation for unit testing.




\section{AI Policies in News Organisations}
\label{sec:policies:news}
%
\begin{table*}[!t]
\footnotesize
\centering
\begin{tabular}{lcccccccccc}
\toprule
\textbf{Code} & \textbf{FT} & \textbf{ANP} & \textbf{Guardian} & \textbf{Parisien} & \textbf{Spiegel} & \textbf{SZ} & \textbf{BBC} & \textbf{Mediahuis} & \textbf{Ringier} & \textbf{VG} \\
\midrule
\multicolumn{11}{c}{\textbf{Suggested Uses}} \\
Illustrations/Graphics &  \use &   \use &        - &        \use &       \use &  \use &   - &         - &       - &  \use \\
Image generation &  \rules &   \use &        - &        \use &       \use &  - &   - &         - &       - &  \rules \\
Article generation &  \rules &   \use &        - &        \rules &       \use &  \rules &   - &         - &       - &  - \\
Data analysis &  \use &   - &        \use &        - &       - &  \use &   - &         - &       - &  - \\
Language tool &  - &   \use &        \use &        \use &       - &  - &   - &         - &       - &  - \\
Transcription/Translation &  \use &   \use &        - &        - &       - &  \use &   - &         - &       - &  - \\
Ideas (Content) &  - &   \use &        - &        - &       - &  - &   - &         - &       - &  - \\
Ideas (Marketing) &  - &   - &        \use &        - &       - &  - &   - &         - &       - &  - \\
Content Moderation &  - &   - &        - &        - &       - &  \use &   - &         - &       - &  - \\
\midrule
\multicolumn{11}{c}{\textbf{Issued Warning and Rules}} \\
Human oversight &  \rules &   \rules &        \rules &        \rules &       \rules &  \rules &   \rules &         \rules &       \rules &  \rules \\
Declaration of use &  \rules &   - &        \rules &        \rules &       \rules &  \rules &   \rules &         \rules &       \rules &  \rules \\
Factual accuracy & \risk &  \risk &       \risk &        - &      \risk &  - &   \rules &         - &      \risk &  - \\
Bias in AI & \risk &  \risk &        \rules &        - &       - &  - &   - &         \rules &      \risk &  - \\
Privacy and sensitive data &  - &   - &        - &        - &       \rules &  - &   \rules &         \rules &       \rules &  \rules \\
Copyright &  - &   - &        \rules &       \risk &       - &  - &   \rules &         \rules &      \risk &  - \\
AI literacy training &  \rules &   - &        - &        - &       - &  - &   \rules &         \rules &       - &  - \\
Document AI use &  \rules &   - &        - &        - &       \rules &  - &   \rules &         - &       - &  - \\
\bottomrule
\end{tabular}

\caption{Suggested uses, and issued \textit{warnings and rules regarding the use of AI systems} in news organisations. For each element, we denote whether the corresponding guidelines have suggested using AI for its purposes (\use) (with the requirement for human oversight in most cases); have issued warnings regarding the use of AI (\risk), or have issued official rules regarding the use of AI(\rules). See \autoref{tab:coding} for the meaning of the individual codes.}
\label{table-news}
\end{table*}
%
We describe the policies of 10 news organisations, as summarised in Table~\ref{table-news} towards their suggested uses of AI and issued warnings and rules w.r.t. AI use in the newsroom.
%
\subsection{Suggested Uses}
All news outlets surveyed encourage the use of AI in their work; however, they propose different degrees of use and applications. Where in Table~\ref{table-news} outlets do not mention any of the listed codes, they still encourage AI usage but do not explicitly list suggested uses in their policy.
%
Further, all news outlets require human oversight for all or most AI use as well as labelling the output as AI generated.

The use of AI in news organisations has two directions. First, tooling to help automated processes in the day to day work of journalists such as \textbf{data analysis}, \textbf{language tools}, e.g., grammar correction, and \textbf{transcription and translation} of, e.g., interviews. %For these applications there have been automated tools for a long time, but with the improvements of AI technology, the news organisations seem to embrace the AI tools specific for these applications. 
Here, there seems to be very little differences in policies -- either, these use cases are mentioned in the policies as allowed or encouraged, or they are not explicitly mentioned, none of the news organisations forbids this type of AI use. In a similar vein, Süddeutsche Zeitung (SZ) explicitly mention the use of AI for \textbf{content moderation}, showing the wide range of possible AI application in news organisations.
%
The second set of proposed use cases of AI in news organisations is around the content of the published news, i.e., the use of AI for content production. Here policies diverge.
%
Interestingly, while many news organisations explicitly allow the use of AI for the generation of \textbf{illustrations/graphics}, Financial Times (FT) and VG prohibit the use of AI for \textbf{image generation}, i.e., the creation and publication of photorealistic images whereas ANP, Le Parisien, and Der Spiegel allow use for image generation.
%
Likewise, policies differ for \textbf{article generation}, i.e., the use of AI to create full or long parts of texts for news articles. While ANP and Der Spiegel allow article creation under human supervision, Financial Times (FT), Le Parisien, and Süddeutsche Zeitung (SZ) explicitly state that their articles are exclusively written by humans.
%
In the realm of ideation, only ANP encourages the use of AI to support journalists in finding ideas on headlines, articles, and even sources. The Guardian does not mention idea generation for news content, but they do encourage the use of AI for marketing campaign ideas. \looseness=-1
%

%AI has a significant influence on the news environment. While the emphasis on human oversight and labeling of generated content is a positive step, establishing common policies across media outlets is equally crucial. Standardised guidelines can help readers better understand AI's role in news production, fostering transparency and reducing distrust that may arise from assumptions about AI involvement.
%
\subsection{Issued Warning and Rules}

All news organisations mandate human involvement, emphasising \textbf{human oversight} before any generated content is used or published. Outlets like ANP and Der Spiegel require explicit editorial approval for publishing generated content, while BBC and ANP highlight that AI is a supportive tool rather than an independent generator.
%
Further, all news organisations agree on the \textbf{declaration of use}, setting standards about the transparency of AI use with policies mandating clear labeling of generated content.

Topics of broader society implications, such as \textbf{factual accuracy} and \textbf{bias in AI}, are often mentioned across the news organisations' policies, but few specific rules are proposed.
%
Generated text may contain factual inaccuracies (see \autoref{fig:sociotech-challenges}). Many of the news organisations warn about this, but only the BBC addresses this issue by requiring that generated content is assessed for ``accuracy and reliability''.
%
Similarly, inherent bias of AI systems is mentioned across multiple news organisations' policies, with only The Guardian and Mediahuis proposing concrete measures. The Guardian discusses the need to ``guard against the danger of bias'' in both AI models and their training data, while Mediahuis more generally ``watch out for biases in AI systems and work to address them'', emphasising the need to balance the different objectives of journalists, commercial interests, and the audience.

For journalists, sensitive data handling can be a crucial issue. Four of the news organisations have explicit policies on handling \textbf{privacy and sensitive data}. These include the ban of entering into AI systems sensitive data of, as Ringier states in their policy, ``journalistic sources, employees, customers or business partners or other natural persons''. For the BBC, this extends to using AI systems that respect privacy rights.

Given the issues with copyrighted texts used for AI training (see \autoref{fig:sociotech-challenges}), \textbf{copyright} is a central topic for news organisations. Five out of ten analysed news organisations' policies address this topic. 
%
Rigier only warns about copyright infringement in the context of entering programming code whereas Le Parisien only warns about copyright infringement in the context of publishing AI generated images. 
%
Unique to The Guardian is the explicit commitment to fair compensation for data creators whose works contribute to AI models, showcasing a progressive view on data usage and ownership rights.

\textbf{AI literacy training} can better equip journalists to deal with AI. Mediahuis, BBC, and Financial Times (FT) support internal training, fostering newsroom awareness and responsible AI deployment. Those trainings include ensuring accountability for AI development and use, train and qualify staff responsible for AI decision, and newsroom awareness (Mediahuis); clear communication on where AI is used, what data is collected and how it works, and affects both staff and audience (BBC); training on the use of AI for story discovery (FT).       

Some news organisations' policies advocate for documenting all instances of AI usage and experimentation. Der Spiegel uses this information to ``exchange information within the company, with other media and with other partners''. The BBC calls for ``clear accountability'' for the use of AI more generally. However, only the Financial Times (FT) explicitly mentions an internal register to track AI use, highlighting an additional layer of accountability not seen in other policies. %This approach to documentation aligns closely with transparency goals under the EU AI Act.

\section{AI Policies in University Education}
\label{sec:policies:uni}

% \begin{table*}[ht]
% \centering
% \begin{tabular}{lcccccccccc}
% \toprule
%                     & TUM & TUD & KTH           & Aalto & DTU  & KUL  & ETH            & CUNI & Vie  & UiO    \\
% \midrule
% course design       & \use & -   & -             & \use  & \use & -    & \use           & \use & -    & \use   \\
% assessment          & -    & \use& -             & -     & -    & -    & \lock  & \use & -    & \use   \\
% self-learning       & \use & \use& \lock & \use  & \use & -    & -              & -    & \use & \use   \\
% personalisation     & \use & -   & -             & \use  & \use & -    & \use           & -    & \use & \use   \\
% feedback            & -    & \use& -             & \use  & -    & -    & \use           & -    & \use & \use   \\
% coding              & \use & \use& -             & \use  & -    & \use & -              & \use & -    & \use   \\
% gen. images         & -    & -   & -             & -     & -    & \use & \risk          & \use & -    & -      \\
% language tool       & -    & \use& \lock & \use  & -    & \use & \use           & -    & -    & -      \\
% ideation            & -    & \use& -             & \use  & -    & \use & \use           & -    & -    & \use   \\
% topic knowledge     & -    & \use& \use          & \use  & -    & \use & -              & -    & \use & \use   \\
% summarisation       & -    & \use& -             & \use  & -    & \use & -              & \use & \use & -      \\
% translation         & -    & -   & -             & \use  & -    & -    & -              & \use & \risk& \use   \\
% search              & -    & -   & -             & -     & -    & \use & -              & \use & -    & \rules \\
% \bottomrule
% \end{tabular}
% \caption{Suggested \textit{uses of AI systems} in teaching and learning activities. For each element, we denote whether the corresponding guidelines have suggested using AI for its purposes (\use), have enforced restrictions on the use of AI for it (\emoji{lock}), have issued warnings regarding the use of AI for it (\risk), or has issued official rules regarding the use of AI for it (\rules).}
% \label{table-uni-uses}
% \end{table*}


% \begin{table*}[ht]
% \begin{tabular}{llllllllllllllllllll}
% \toprule
% \rotatebox{90}{Institution} & \rotatebox{90}{knowledge cut-off} & \rotatebox{90}{persuasiveness} & \rotatebox{90}{factual accuracy} & \rotatebox{90}{no prioritisation} & \rotatebox{90}{source attribution} & \rotatebox{90}{assess skills} & \rotatebox{90}{not learning} & \rotatebox{90}{privacy} & \rotatebox{90}{copywright} & \rotatebox{90}{biased output} & \rotatebox{90}{plagiarism} & \rotatebox{90}{digital inequity} & \rotatebox{90}{environment} & \rotatebox{90}{exploited labor} & \rotatebox{90}{owner power} & \rotatebox{90}{declaration of use} \\
% \midrule
% TUM & \risk & \risk & \risk & \risk & \risk & \risk & - & \risk & - & \risk & \risk & - & - & - & - & - \\
% TUD & - & - & - & - & - & \risk & - & \risk & - & - & \risk & \risk & \risk & \risk & \risk & \risk/\rules \\
% KTH & - & - & \risk & \risk & - & - & - & - & - & - & - & - & - & - & - & \rules \\
% Aalto & \risk & \risk & \risk & - & \risk & - & - & \risk/\rules & - & \risk & - & - & - & - & - & \rules \\
% DTU & - & - & \risk & - & - & - & - & \risk/\rules & \risk & \risk & - & - & - & - & - & \rules \\
% KUL & - & - & \risk & \risk & - & - & - & \risk/\rules & \risk/\rules & \risk & - & - & \risk & - & - & \rules \\
% ETH & - & - & \risk & - & \risk & \risk & - & \rules & \rules & \risk & - & - & - & - & - & \rules \\
% CUNI & - & - & - & - & - & - & - & \risk/\rules & \rules & - & - & \risk & - & - & - & \rules \\
% Vie & - & - & \risk & \risk/\rules & - & - & \risk & \risk & \risk/\rules & - & \risk & - & - & - & - & \rules \\
% % UdL & - & - & - & - & - & - & - & - & \risk & - & - & \rules & - & - & - & - \\
% UiO & \risk & \risk & \risk & \risk & - & \risk & \risk & \rules & \risk/\rules & - & - & - & \risk & - & - & - \\
% \bottomrule
% \end{tabular}
% \caption{Issued \textit{warnings and rules regarding the use of AI systems} in teaching and learning activities. For each element, we denote whether the corresponding guidelines have enforced restrictions on the use of AI for it (\emoji{lock}), have issued warnings regarding the use of AI for it (\risk), or have issued official rules regarding the use of AI for it (\rules).}
% \label{table-uni-risks-rules}
% \end{table*}

\begin{table*}[!t]
\centering
\footnotesize
\begin{tabular}{lcccccccccc}
\toprule
 \textbf{Code} & \textbf{UiO} & \textbf{Aalto} & \textbf{TUD} & \textbf{KUL} & \textbf{ETH} & \textbf{TUM} & \textbf{Vie} & \textbf{CUNI} & \textbf{DTU} & \textbf{KTH} \\
\midrule
\multicolumn{11}{c}{\textbf{Suggested Uses}} \\
Self-learning      & \use & \use & \use & -    & -    & \use & \use & -    & \use & \lock \\ 
Course design      & \use & \use & -    & -    & \use & \use & -    & \use & \use & -            \\ 
Personalisation    & \use & \use & -    & -    & \use & \use & \use & -    & \use & -            \\
Coding             & \use & \use & \use & \lock & -    & \use & -    & \use & -    & -            \\
Topic knowledge    & \use & \use & \use & \use & -    & -    & \use & -    & -    & \use         \\
Feedback           & \use & \use & \use & -    & \use & -    & \use & -    & -    & -            \\
Language tool      & -    & \use & \use & \use & \use & -    & -    & -    & -    & \lock \\
Ideas           & \use & \use & \use & \use & \use & -    & -    & -    & -    & -            \\
Summarisation      & -    & \use & \use & \use & -    & -    & \use & \use & -    & -            \\
Translation        & \use & \use & -    & -    & -    & -    & \risk& \use & -    & -            \\
Assessment         & \use & -    & \use & -    & \lock & -   & -    & \use & -    & -        \\
Search             & \lock & -   & -    & \use & -    & -    & -    & \use & -    & -            \\
Image generation       & -    & -    & -    & \use & \risk& -    & -    & \use & -    & -            \\

\midrule
\multicolumn{11}{c}{\textbf{Issued Warning and Rules}} \\
Teacher restrictions & \rules & \rules & - & \rules & \rules & \rules & \rules & \rules & - & - \\
AI literacy training & \risk \use      & - & \use & \risk \use & \use & \risk \use & \risk \use & \use & - & -            \\
Human oversight   & \rules     & \rules & - & \rules & \rules & - & \rules & \rules & \rules & \rules \\
Privacy and sensitive data            & \rules      & \risk/\rules & \risk & \risk/\rules & \rules & \risk & \risk & \risk/\rules & \risk/\rules & -            \\
Factual accuracy   & \risk       & \risk    & \risk     & \risk        & \risk  & \risk & \risk & -            & \risk        & \risk        \\
Declaration of use & -           & \rules       & \risk/\rules & \rules & \rules & -     & \rules & \rules & \rules & \rules  \\
Copyright          & \risk/\rules& -            & \risk & \risk/\rules & \rules & \risk     & \risk/\rules & \risk/\rules & \risk & -      \\
Bias in AI      & -           & \risk        & -     & \risk        & \risk  & \risk & -     & -     & \risk & -      \\
No prioritisation  & \risk       & -            & -     & \risk        & -      & \risk & \risk/\rules & -     & -     & \risk  \\
Digital inequity   & \risk       & -            & \risk & \risk        & -      & -     & -     & \risk & -     & -      \\
Knowledge cut-off  & \risk       & \risk        & \risk     & \risk            & -      & \risk & -     & -     & -     & -      \\
Persuasiveness     & \risk       & \risk        & \risk     & -            & -      & \risk & -     & -     & -     & \risk      \\
Source attribution & -           & \risk        & \risk     & -            & \risk  & \risk & \risk     & -     & -     & -      \\
Skills assessment     & \risk       & -            & \risk & -            & \risk  & \risk & -     & -     & -     & -      \\
% Plagiarism         & -           & -            & \risk & -            & -      & \risk & \risk & -     & -     & -      \\
Environment        & \risk       & -            & \risk & \risk        & -      & -     & -     & -     & -     & -      \\
AI over-reliance       & \risk       & -            & -     & -            & -      & -     & \risk & -     & -     & -      \\
% exploited labor    & -           & -            & \risk & -            & -      & -     & -     & -     & -     & -      \\
% owner power        & -           & -            & \risk & -            & -      & -     & -     & -     & -     & -      \\
Teacher load & -           & -            & \risk  & -            & -      & -     & -     & -     & \risk     & -     \\  
\bottomrule
\end{tabular}
\caption{Suggested uses, and issued \textit{warnings and rules regarding the use of AI systems} in teaching and learning activities. For each element, we denote whether the corresponding guidelines have suggested using AI for its purposes (\use), have enforced restrictions on the use of AI for it (\lock), have issued warnings regarding the use of AI (\risk), or have issued official rules regarding the use of AI(\rules). See \autoref{tab:coding} for the meaning of individual codes.}
\label{table-uni}
\end{table*}

% how many give the teacher the opportunity to decide on the use
% how many mention it's a moving field
Table \ref{table-uni} presents a summary of the suggested use cases of AI in teaching and learning as well as the risks and rules enacted in the corresponding institutions. The table presents the points raised by at least two universities, we also discuss points raised by individual universities in the following section.

\subsection{Suggested Uses}
% In contrast, KTH enforces limitations to using AI systems for self-learning by not directly asking the AI to generate specific answers or complete their assignments.

The suggested uses of AI systems in education are framed from two primary perspectives: that of the teacher and the student. The most commonly proposed application is for \textbf{self-learning}, mentioned in seven organisational guidelines. Examples of self-learning activities supported by AI systems include providing students with learning materials and resources, assisting in planning and monitoring their learning, encouraging exploration of covered topics (TUM), and offering alternative perspectives (Aalto).

From the student perspective, AI systems are also frequently viewed as tools for enabling \textbf{personalised learning}, e.g. by recommending individualised learning plans, presenting material with explanations of varying difficulty levels, and enhancing accessibility for students with disabilities (ETH). Additionally, AI is suggested as highly effective for \textbf{coding tasks}, such as understanding concepts, breaking down problems into smaller parts, practising debugging skills, and receiving feedback on code. University guidelines further note that students can leverage AI to \textbf{gain an initial overview of a topic}, terms, or concepts. Moreover, AI can help students monitor their progress and \textbf{provide targeted feedback} on written work or ideas.

AI systems are also commonly suggested as \textbf{language tools}, particularly for grammar, spelling, and reference checks—often considered the safest and most permissible use of AI. For example, KTH identifies this as one of the three approved uses of AI, typically not requiring documentation in the declaration of use. Additional proposed applications for students include \textbf{idea generation, summarising academic literature, translation, student assignment assessment, search engine functionality, and image generation}. However, regarding translation, Vie cautions that AI may produce inaccuracies, particularly with new terminology or less common language combinations.

From the teacher's perspective, suggested uses of AI extend to \textbf{course design} activities, such as formulating learning objectives, drafting rubrics, planning workshops, designing assignments, creating questions, preparing courses, and even simulating a test student. Regarding assessment, ETH explicitly disallows the fully automated grading of student work, but most universities grant teachers the discretion to decide whether \textbf{AI use is restricted} in assignments and exams. \looseness=-1

While many universities provide lists of potential AI applications in teaching and learning, some impose restrictions on those applications. For instance, KTH limits AI usage to predefined prompts and prohibits directly asking the system to generate specific answers or complete assignments. Similarly, KUL restricts the use of AI in coding tasks to generating components of larger assignments, and only if explicitly approved by the teacher.

\subsection{Issued Warnings and Rules}
The most common warnings and rules in university policies regarding AI include considerations of \textbf{privacy} and \textbf{sensitive data}, \textbf{copyright}, \textbf{factual accuracy}, and the \textbf{declaration of use}. While some universities encourage students and teachers to consider privacy and copyright concerns and warn of potential violations, others enforce strict rules regarding the types of data that can be input into AI systems to prevent such issues. For example, Aalto specifies that teachers may only submit student work to university-approved systems and prohibits entering other students' answers or personal information into external systems. To support these policies, Aalto, along with DTU, KUL, and ETH, provides access to Microsoft Copilot for both teachers and students, which is meant to ensure that submitted data is not stored or used for future model training.
%
Regarding copyright, universities caution that AI can reproduce copyrighted material without proper acknowledgement. They also require users to avoid inputting proprietary information, such as the university’s intellectual property, into AI tools.

Other warnings address risks associated with the quality of generated content, including \textbf{biases} in the content, \textbf{lack of prioritisation} of arguments, \textbf{absence of source attribution} (making verification of accuracy, plagiarism, etc., difficult), \textbf{limited knowledge due to cut-off dates}, and a \textbf{persuasive tone} that can mislead users about the correctness of the information. Risks also arise in educational scenarios where AI integration might \textbf{necessitate course reorganisation or additional assessments} to ensure learning objectives are met and student skills are accurately evaluated. AI usage may also contribute to \textbf{digital inequity}, stemming from disparities in access to paid versus free tools and variations in the quality of generated content based on user skills. \textbf{Over-reliance on AI tools} is another identified risk. \looseness=-1

KUL explicitly highlights the lack of reproducibility as a concern, noting that output can vary between attempts. UiO is unique in warning that AI can produce inappropriate or offensive content. KUL also advises against ``humanising'' AI, emphasising that it is merely a tool.

Many universities stress the \textbf{importance of asking the right questions} and not settling for the first answer provided by AI. To support this, they offer guides for crafting effective prompts, and some institutions even provide dedicated courses on this topic. Teachers are encouraged to \textbf{set clear restrictions on AI use} within their courses and must communicate allowed uses transparently. Finally, most guidelines underscore that both teachers and students remain \textbf{fully responsible} for the content they incorporate into their work, regardless of AI use.


% All universities emphasize the importance of academic honesty, particularly regarding AI tools' potential to generate content that students, teachers, and researchers might pass off as their own. All universities classify the unacknowledged use of AI in assessments as a form of cheating and underscore the need for students to cite AI-assisted work, aiming to uphold standards of academic integrity. Universities acknowledge that there are no fully effective tools to detect the use of generative AI, which underlines the importance of academic integrity and the ethical academic standards of behaviour that are expected of all members of the academic community.

% AI as a supportive teaching tool is highlighted across institutions. Universities like \textit{Aalto} and \textit{University of Lisbon} encourage AI tools to support personalized learning and assist with activities like summarizing, idea generation, and coding assistance. \textit{ETH Zurich} focuses on AI to facilitate accessibility, particularly for students with disabilities, showing how AI can diversify educational support.

% Data privacy and security are central concerns, especially regarding sensitive student and university information. \textit{University of Oslo} and \textit{KU Leuven} explicitly restrict the use of personal data in AI tools and advise students to use incognito modes or avoid uploading sensitive content to maintain data security.

% Universities have mixed approaches to AI use in assessments. \textit{Technical University of Munich} and \textit{University College Dublin} present two main models: prohibiting AI in exams versus integrating it with a focus on transparency, critical engagement, and sometimes even oral evaluations.

% From the selected universities, guidelines regarding the use of AI in academia were available mostly in English, either for universities in English speaking countries or for universities that had programs in English. These universities had further courses to educate teachers in the use of AI for teaching and learning. For other universities there were no guidelines in place in the respective language (e.g., ). This indicates that countries where the AI tools would have lower performance due to lower training resources for non-English languages, there is also less guidance on the responsible use of AI and the increased risk of potential flaws in the respective languages.

% While many institutions point the use of AI for personalising the learning process both from the student and teacher perspective, universities where English is not the main language, like the University of Barcelona, emphasize that AI tools lack the cultural or learning situation contextualization necessary for personalisation of the learning experience.

% \textit{Charles University} in the Czech Republic emphasizes critical engagement, encouraging students to balance AI use with independent skill development. Conversely, \textit{University of Lisbon} endorses AI as a creative brainstorming aid but requires “declarations of honor” for assignments where AI is used, underscoring an ethical approach to its adoption.

% Unique to \textit{Imperial College} is an emphasis on the environmental and social impacts of AI, addressing concerns about the carbon footprint and potential biases from training data. This perspective aligns with a growing trend in academia to consider AI's broader ethical implications.

% Overall, the responsible use of AI and the responsibility regarding the generated texts are assigned to the AI user. Teachers are also assigned the responsibility to educate the students regarding the principles of academic integrity.

% \subsection{Further Technological Challenges}
% We find the following technological challenges not addressed in the reviewed guidelines. They are concerned with the suitability of LLMs for evaluation, which is a critical aspect of the educational process. Existing work finds that the current state-of-the-art LLM may not be suitable to serve as LLM TAs for all tasks.

% \paragraph{Unfit for Assessment} \citet{chiang-etal-2024-large} recently conducted a course with AI as an evaluator. Students involved in the survey reported that the LLM did not correctly follow the instructions for the output format (51.3\%), did not properly follow the evaluation criteria (21.5\%), yielding scores too low or too high, and yielded different ratings for the same assignment due to the randomness of LLM decoding. The output of LLMs is also known to be brittle to small changes in the input, even though the meaning of the text remains unchanged \cite{}. Finally, the authors advocate for transparency of the prompts used for automated evaluation, which was not specifically pointed across the reviewed university guidelines.

% \paragraph{Fooling LLM Assessment} 
% Guidelines have also to outline that students are prone to try different manipulation techniques that would fool the LLM to evaluate them with a high score, e.g., by goal hijacking \cite{}, where the malicious prompt aims to make the LLM generate the highest score or by using known vulnerabilities of LLMs such as that LLMs prefer longer responses \cite{}, assertive language \cite{}, etc. This is further completed by the known bias of LLMs used for evaluation to assign higher scores to text generated by the same model \cite{}. This can further introduce assessment inequality for students that, e.g., have access to some closed/paid models used also for assessment.


%\section{Background}
%\subsection{EU AI Act}

% \section{Top-down Perspectives on Generative AI Policy}
\section{Insights from Organisation Policies for the EU AI Act}\label{sec:insights}
We now put the above findings from organisational policies in the perspective of EU AI Act \cite{2024-ai-act} (\aia). We recognise that these efforts towards AI governance are fundamentally different in scope and process through which they were created, and they are complementary. However, given the ongoing effort\footnote{\url{https://artificialintelligenceact.eu/ai-act-implementation-next-steps/}} to develop a more specific implementation guidelines for \aia, we believe that the insights from the bottom-up policies could help to identify areas where more clarity would be appreciated.

%While the organisational policies cannot be used to compare directly with the \aia, they serve as a good start to inform possible improvements that can be made to the legislation in upcoming iterations based on their existing experience and help point out possible points of confusion for practitioners applying the \aia in their domain.
%

\subsection{News Organisations' Policies}
%\textcolor{red}{TODO: Check which risk category news things fall under!}
%\textcolor{red}{TODO: Reference to the selling of news outlet data as training data}
\textit{\textbf{Similarities.}}
News organisations' policies align with the provisions introduced by the \aia for example on the emphasises on \textit{human oversight} with the \aia Article 14.
%
Article 50 of the \aia describes transparency obligations for providers and deployers, aligning with the requirements to \textit{disclose AI-generated content} by news organisations.
%
For high-risk AI systems, Article 10 of the \aia regulates \textit{data governance}, a question that also is relevant to downstream users such as journalists. In the context of the \aia this is limited to training, validation, and test data, whereas for news organisations the question of input of sensitive data into AI systems is crucial to preserve privacy of possible sources, reflecting the broader GDPR compliance required under the \aia.
%
%
%The proposal of documenting and sharing , similar to the technical documentation the \aia, Article 11 requires for high-risk systems.\looseness=-1

\textit{\textbf{Gaps.}}
The news' AI policies cover several points not covered by the \aia. In particular, The Guardian emphasises \textit{compensating those whose data is used for AI}, while the \aia lacks explicit provisions for data creators' compensation outside of existing copyright regulations. 
%
%
Financial Times and Mediahuis include newsroom \textit{AI training} in their policies. The EU Act currently only requires AI literacy training for developers and deployers of AI systems. If news organisations as downstream users of these systems are not considered as deployers, this requirement will not cover this type of distribution of AI-generated content. 
%
Multiple policies require addressing \textit{bias in the AI systems} used by journalists, a topic that is yet to be comprehensively covered by legislation.
%
The \textit{internal AI usage register policy} by Financial Times is an additional documentation practice not specified by the Act but useful for accountability. It could enable retroactive checks on which content was created with which AI system and where the systems where used, which would improve transparency and accountability of this outlet, and hence potentially increase trust in it. 

\subsection{University Education's Policies}
%
\textit{\textbf{Similarities.}}
The university AI guidelines align with several provisions in the \aia, particularly in their focus on \textit{transparency, human oversight, and data privacy}. Concerns about privacy and data protection, emphasised by universities like KUL and the UiO, align with Article 10 of the \aia, which mandates data governance and safeguards for personal data used for training, testing, and validation in AI systems. Universities advise students and teachers against uploading sensitive data to AI tools, reflecting the broader GDPR compliance required under the \aia.\looseness=-1

The \textit{human oversight of AI} in decision-making, particularly in \textit{assessments and exams}, is echoed in Article 14 of the \aia, which mandates human monitoring of AI decision-making processes in high-risk AI systems. Some universities, such as the TUM, prohibit AI in exams, while others integrate AI tools under strict human oversight, ensuring that AI does not replace independent student evaluation.

The \aia defines education as one of the high-risk areas for AI applications in Article 6 Annex III, enforcing \textit{bias mitigation and fairness obligations}. Universities also raise concerns about AI’s potential to reinforce grading biases and manipulation tactics in AI-driven assessments (e.g., response length bias, goal hijacking). Similarly, universities explore AI-driven adaptive learning and dropout risk prediction (though the latter is clearly a high-risk application that comes with extra considerations for the trade-off between improved performance of Transformer-based systems and the difficulty of regulatory compliance \cite{NielsenRaaschou-PedersenEtAl_2024_Trading_off_performance_and_human_oversight_in_algorithmic_policy_evidence_from_Danish_college_admissions}).

\textit{\textbf{Gaps.}}
University AI policies introduce measures that are either absent or not explicitly covered by the \aia. For example, institutions like UiO, TUD, and KUL highlight the \textit{environmental impact} of AI, focusing on sustainability and ethical AI use. 
Furthermore, university AI policies highlight the risk of increasing \textit{digital inequity} among students, which stems both from disparities in access to paid versus free tools and variations in the quality of generated content based on user skills. The \aia, while regulating AI safety and robustness, does not directly address AI’s carbon footprint, resource consumption, or exacerbated digital inequity.

While the \aia, Article 4 mandates \textit{AI literacy training }for developers and deployers, university policies extend this responsibility to educators and students. Educators are encouraged to equip students with the skills needed to critically evaluate AI outputs. Additionally, both educators and students are provided with resources to learn how to craft effective AI inputs to achieve optimal results. This proactive approach in universities contrasts with the \aia’s narrower focus on professional AI users rather than general AI literacy. %\looseness=-1
%These issues remain largely unaddressed in AI regulation but are critical for ensuring fairness in education.

Overall, while the \aia provides a legal framework for AI safety, transparency, and human oversight, universities take a context-specific approach to AI governance, addressing academic integrity, assessment reliability, and pedagogical challenges in ways that current EU regulation does not yet fully capture.

\subsection{Known research challenges not covered in \aia or organisational policies}
\label{sec:challenges:discuss}

Finally, let us consider the set of sociotechnical challenges that is more broadly known from the existing academic literature (\autoref{fig:sociotech-challenges}), but that we have not found to be addressed in sufficient detail in either organisational policies we considered or \aia:

\textit{\textbf{Definition.}} The organisational policies do not typically define what kind of `AI' is being addressed, and the definition proposed in \aia has been criticized by researchers \cite{Hooker_2024_On_Limitations_of_Compute_Thresholds_as_Governance_Strategy}.

\textit{\textbf{Enforceability.}} Many policies we considered require \textbf{declaration of AI use}, yet there are no robust detection mechanisms to verify compliance \cite{PuccettiRogersEtAl_2024_AI_News_Content_Farms_Are_Easy_to_Make_and_Hard_to_Detect_Case_Study_in_Italian}.

\textit{\textbf{Unsafe outputs.}} Only one university in our sample (UiO) mentioned the possibility of exposing students to inappropriate outputs from AI models.

\textit{\textbf{Misleading marketing claims.}} AI providers are constantly advertising new models claimed to be ever better at `reasoning', `understanding' and other constructs of questionable validity for the current AI \cite{Mitchell_2021_Why_AI_is_Harder_Than_We_Think}. Many policies we examined seem to be influenced by `fear of missing out', manufactured by such narratives. More stringent requirements of transparency for claimed benchmark results could alleviate this problem.

\textit{\textbf{Explainability.}} We saw no policies directly addressing the fact that the current AI systems are not interpretable, which has implications for their use (especially where decisions could have significant consequences, e.g. student grading or news fact-checking).

\textit{\textbf{Brittleness.}} Some university policies mention the need for AI literacy training, but we did not find that in news, and \aia also does not discuss that for the users of AI systems.

\textit{\textbf{Creativity.}} It is possible that over-reliance on AI systems could damage basic competences or creativity of its users, but most policies we examined do not seriously consider this factor.

\textit{\textbf{Carbon emissions.}} Only 3 universities and no news organisations considered this point, and it is not addressed in \aia beside documentation.



%
\section{Policy Recommendations Based on Gaps Identified in This Work}
%
To reiterate, while the \aia and organisational policies from universities and news organisations operate within different scopes and serve distinct purposes, they can still inform each other, and insights from academic research can further identify areas not sufficiently addressed by either efforts. This section lists the areas where we believe further governance and research efforts are needed the most. %The \aia establishes a legal framework for AI governance, focusing on risk-based regulation, transparency, and accountability. In contrast, university and newsroom policies address sector-specific challenges, such as academic integrity, journalistic standards, and AI literacy. These policy areas, shaped by on-the-ground experiences in academia and journalism, provide valuable insights for strengthening AI governance beyond the existing provisions of the \aia.

%Despite these differences, the practical implementation of AI policies in universities and news organisations reveals gaps in the \aia that could be clarified for its implementation. These gaps are not necessarily regulatory oversights but rather areas where organisational policies have evolved to meet real-world challenges that may not yet be fully reflected in AI legislation.
%

\textbf{\textit{Expanding AI literacy training.}} The \aia mandates AI literacy training for developers and deployers (Article 4), but it does not mandate AI literacy for students, teachers, journalists, the general public who generate or interact with AI-generated content, or even the media professionals or other users of AI models distributing their outputs on a large scale. Universities integrate AI literacy into academic policies, requiring students and educators to develop critical engagement with AI tools. Institutions like CUNI make a proactive step in this direction emphasising the education in AI ethics and responsible use. Newsrooms such as Financial Times and Mediahuis provide AI training for journalists, ensuring that AI-generated content is fact-checked and responsibly handled. Such training should include also critical reflection on the real functionality of the current AI models vs the marketing hype. While it is important to keep responsibility with the AI developers and deployers, supporting users on how to approach AI will be critical. %, which would also ideally help to shape the future policies. %The \aia does not mandate AI literacy training for media professionals or other users of AI models distributing their outputs on a large scale.

\textit{\textbf{Policies addressing Digital Inequity.}} The \aia does not explicitly address digital inequity, despite its potential to exacerbate social and economic disparities (e.g. due to unequal access to AI and different quality of the models available for different socio-economic and linguistic groups). %
This is particularly relevant in education, where university AI policies have highlighted concerns about disparities in access to paid vs. free AI tools, as well as differences in the quality of AI-generated content based on user skills. Other concerns include the temptation to use AI `education' as a low-cost solution substituting human teachers for the already underpriviledged groups, and siphoning of public resources to for-profit AI providers instead of building public AI infrastructure. All this requires more thought to develop more equitable education infrastructure and policies that consider socio-economic impact on various population groups. 

\textbf{\textit{Improved Transparency for Generated Content.}} While \aia Article 52 mandates disclosure of AI-generated content, it does not specify how AI-assisted work should be attributed or audited by downstream users, or how the record of AI assistance should be kept. Universities enforce strict AI citation rules, requiring students to disclose AI-generated content to uphold academic integrity. However, there is no standardised framework for disclosing AI use, which could aid AI literacy across sectors. For example, in student submissions (e.g. should it be a brief description, or a full prompt+output? How should the source system be specified?) An interesting practice is the internal AI usage registers in Financial Times, which allows editors to track which articles were AI-assisted. %The general guidelines for downstream distribution of AI output would ideally specify how to indicate where AI assistance (or its amount and nature) was used for the readers of the content.

\textbf{\textit{Getting specific about `bias'.}} Both news and university policies sometimes warn about the possibility of `bias', but it is not clear what kinds of bias should be addressed, or how. This is a gap legislation could address by providing a more comprehensive guidance (e.g. based on existing human rights and non-discrimination laws) for model providers, deployers, as well as downstream users and content distributors.

\textbf{\textit{Attribution and compensation of sources of AI training data.}} EU has copyright laws, but AI training data poses new challenges currently tested in both US \cite{Vynck_2023_Game_of_Thrones_author_and_others_accuse_ChatGPT_maker_of_theft_in_lawsuit,GrynbaumMac_2023_Times_Sues_OpenAI_and_Microsoft_Over_AI_Use_of_Copyrighted_Work} and in Europe\cite{Smith_2024_GEMA_Files_Copyright_Lawsuit_Against_OpenAI_in_Germany}. In our sample, only The Guardian advocates for compensation of content creators whose data is used as part of AI training. %, and there are creator groups\footnote{\url{https://www.thedpa.ai/}} advocating for licensing, compensation and transparency. 
In education, an equally important factor is source attribution, without which the students could be unwittingly plagiarising existing scholarly work. %This has plagiarism implications for students using AI-generated summaries or analyses based on uncredited scholarly work. 
%Given the huge implications for jobs and social norms, EU position on this should be clear. 
The question of data governance and compensation should be further investigated, taking into account concepts such as data collectives \cite{DBLP:conf/cscw/HsiehZKRDMGEZ24}.


\textbf{\textit{Disclosing Environmental Impact of AI}} The Act does not explicitly address the carbon footprint of AI models besides documentation, despite researchers' concerns about large-scale computational demands \cite{dodge2022measuring,bouza2023estimate,luccioniCountingCarbonSurvey2023,liMakingAILess2023}. Some universities and news outlets highlight the environmental costs of training and running large models, yet there are no regulatory incentives to optimise for sustainability. One ongoing research direction is developing more efficient models \cite{trevisoEfficientMethodsNatural2023}, but if the more efficient models just get used more, this will not help. Mandating a visible disclosure to the users of how much water and energy each AI query costs, and where the resources come from, could help to discourage excessive use. Organisations could also have AI use by their employees as a part of their sustainability reporting. % Legislative incentives for the responsible use of AI  

\textbf{\textit{Detection and enforceability.}} There are currently no reliable methods of detecting generated text, which makes any policies unenforceable. A promising solution is watermarking \cite{jiang2024watermark,roman2024proactive}, but the providers of commercial LLMs have no incentive to provide a mechanism that could reduce the usage of their services \cite{davisOpenAIWontWatermark2024,gloaguenBlackBoxDetectionLanguage2024}. This is where the considerations of social impact \cite{solaiman2023evaluating} should guide regulation mandating such transparency from the popular AI service providers. %An exchange between practitioners, policy makers, and downstream users can ensure that the right approach is chosen to fulfil all requirements.

\textbf{\textit{Clarifying liability.}} In compliance with \aia, providers of AI models may attempt to build in ``safeguards'' to avoid e.g. toxic outputs, and they will have to pass some evaluations to put the model on the market. But should something go wrong, e.g. seriously impacting the mental health of the user, it is currently not clear how much legal recourse the affected users would have. The question of AI liability \cite{Liability_Rules_for_Artificial_Intelligence_European_Commission} will get more pressing with the amount of cases that pose the question of responsibility for the consequences of AI use \cite{awfulai}.
% 

Finally, we would list the following public-interest areas with potential regulatory significance, which need much more research: detection of synthetic content, model interpretability, source attribution to training data, and long-term effects of AI `assistance' on productivity and skills of the workforce. Besides the above suggestions for regulation, these directions should be among priority areas for academic research funding allocation, as the incentives for working on them are just not present in industry.

%\section{Discussion of Assumptions}\label{sec:discussion}
In this paper, we have made several assumptions for the sake of clarity and simplicity. In this section, we discuss the rationale behind these assumptions, the extent to which these assumptions hold in practice, and the consequences for our protocol when these assumptions hold.

\subsection{Assumptions on the Demand}

There are two simplifying assumptions we make about the demand. First, we assume the demand at any time is relatively small compared to the channel capacities. Second, we take the demand to be constant over time. We elaborate upon both these points below.

\paragraph{Small demands} The assumption that demands are small relative to channel capacities is made precise in \eqref{eq:large_capacity_assumption}. This assumption simplifies two major aspects of our protocol. First, it largely removes congestion from consideration. In \eqref{eq:primal_problem}, there is no constraint ensuring that total flow in both directions stays below capacity--this is always met. Consequently, there is no Lagrange multiplier for congestion and no congestion pricing; only imbalance penalties apply. In contrast, protocols in \cite{sivaraman2020high, varma2021throughput, wang2024fence} include congestion fees due to explicit congestion constraints. Second, the bound \eqref{eq:large_capacity_assumption} ensures that as long as channels remain balanced, the network can always meet demand, no matter how the demand is routed. Since channels can rebalance when necessary, they never drop transactions. This allows prices and flows to adjust as per the equations in \eqref{eq:algorithm}, which makes it easier to prove the protocol's convergence guarantees. This also preserves the key property that a channel's price remains proportional to net money flow through it.

In practice, payment channel networks are used most often for micro-payments, for which on-chain transactions are prohibitively expensive; large transactions typically take place directly on the blockchain. For example, according to \cite{river2023lightning}, the average channel capacity is roughly $0.1$ BTC ($5,000$ BTC distributed over $50,000$ channels), while the average transaction amount is less than $0.0004$ BTC ($44.7k$ satoshis). Thus, the small demand assumption is not too unrealistic. Additionally, the occasional large transaction can be treated as a sequence of smaller transactions by breaking it into packets and executing each packet serially (as done by \cite{sivaraman2020high}).
Lastly, a good path discovery process that favors large capacity channels over small capacity ones can help ensure that the bound in \eqref{eq:large_capacity_assumption} holds.

\paragraph{Constant demands} 
In this work, we assume that any transacting pair of nodes have a steady transaction demand between them (see Section \ref{sec:transaction_requests}). Making this assumption is necessary to obtain the kind of guarantees that we have presented in this paper. Unless the demand is steady, it is unreasonable to expect that the flows converge to a steady value. Weaker assumptions on the demand lead to weaker guarantees. For example, with the more general setting of stochastic, but i.i.d. demand between any two nodes, \cite{varma2021throughput} shows that the channel queue lengths are bounded in expectation. If the demand can be arbitrary, then it is very hard to get any meaningful performance guarantees; \cite{wang2024fence} shows that even for a single bidirectional channel, the competitive ratio is infinite. Indeed, because a PCN is a decentralized system and decisions must be made based on local information alone, it is difficult for the network to find the optimal detailed balance flow at every time step with a time-varying demand.  With a steady demand, the network can discover the optimal flows in a reasonably short time, as our work shows.

We view the constant demand assumption as an approximation for a more general demand process that could be piece-wise constant, stochastic, or both (see simulations in Figure \ref{fig:five_nodes_variable_demand}).
We believe it should be possible to merge ideas from our work and \cite{varma2021throughput} to provide guarantees in a setting with random demands with arbitrary means. We leave this for future work. In addition, our work suggests that a reasonable method of handling stochastic demands is to queue the transaction requests \textit{at the source node} itself. This queuing action should be viewed in conjunction with flow-control. Indeed, a temporarily high unidirectional demand would raise prices for the sender, incentivizing the sender to stop sending the transactions. If the sender queues the transactions, they can send them later when prices drop. This form of queuing does not require any overhaul of the basic PCN infrastructure and is therefore simpler to implement than per-channel queues as suggested by \cite{sivaraman2020high} and \cite{varma2021throughput}.

\subsection{The Incentive of Channels}
The actions of the channels as prescribed by the DEBT control protocol can be summarized as follows. Channels adjust their prices in proportion to the net flow through them. They rebalance themselves whenever necessary and execute any transaction request that has been made of them. We discuss both these aspects below.

\paragraph{On Prices}
In this work, the exclusive role of channel prices is to ensure that the flows through each channel remains balanced. In practice, it would be important to include other components in a channel's price/fee as well: a congestion price  and an incentive price. The congestion price, as suggested by \cite{varma2021throughput}, would depend on the total flow of transactions through the channel, and would incentivize nodes to balance the load over different paths. The incentive price, which is commonly used in practice \cite{river2023lightning}, is necessary to provide channels with an incentive to serve as an intermediary for different channels. In practice, we expect both these components to be smaller than the imbalance price. Consequently, we expect the behavior of our protocol to be similar to our theoretical results even with these additional prices.

A key aspect of our protocol is that channel fees are allowed to be negative. Although the original Lightning network whitepaper \cite{poon2016bitcoin} suggests that negative channel prices may be a good solution to promote rebalancing, the idea of negative prices in not very popular in the literature. To our knowledge, the only prior work with this feature is \cite{varma2021throughput}. Indeed, in papers such as \cite{van2021merchant} and \cite{wang2024fence}, the price function is explicitly modified such that the channel price is never negative. The results of our paper show the benefits of negative prices. For one, in steady state, equal flows in both directions ensure that a channel doesn't loose any money (the other price components mentioned above ensure that the channel will only gain money). More importantly, negative prices are important to ensure that the protocol selectively stifles acyclic flows while allowing circulations to flow. Indeed, in the example of Section \ref{sec:flow_control_example}, the flows between nodes $A$ and $C$ are left on only because the large positive price over one channel is canceled by the corresponding negative price over the other channel, leading to a net zero price.

Lastly, observe that in the DEBT control protocol, the price charged by a channel does not depend on its capacity. This is a natural consequence of the price being the Lagrange multiplier for the net-zero flow constraint, which also does not depend on the channel capacity. In contrast, in many other works, the imbalance price is normalized by the channel capacity \cite{ren2018optimal, lin2020funds, wang2024fence}; this is shown to work well in practice. The rationale for such a price structure is explained well in \cite{wang2024fence}, where this fee is derived with the aim of always maintaining some balance (liquidity) at each end of every channel. This is a reasonable aim if a channel is to never rebalance itself; the experiments of the aforementioned papers are conducted in such a regime. In this work, however, we allow the channels to rebalance themselves a few times in order to settle on a detailed balance flow. This is because our focus is on the long-term steady state performance of the protocol. This difference in perspective also shows up in how the price depends on the channel imbalance. \cite{lin2020funds} and \cite{wang2024fence} advocate for strictly convex prices whereas this work and \cite{varma2021throughput} propose linear prices.

\paragraph{On Rebalancing} 
Recall that the DEBT control protocol ensures that the flows in the network converge to a detailed balance flow, which can be sustained perpetually without any rebalancing. However, during the transient phase (before convergence), channels may have to perform on-chain rebalancing a few times. Since rebalancing is an expensive operation, it is worthwhile discussing methods by which channels can reduce the extent of rebalancing. One option for the channels to reduce the extent of rebalancing is to increase their capacity; however, this comes at the cost of locking in more capital. Each channel can decide for itself the optimum amount of capital to lock in. Another option, which we discuss in Section \ref{sec:five_node}, is for channels to increase the rate $\gamma$ at which they adjust prices. 

Ultimately, whether or not it is beneficial for a channel to rebalance depends on the time-horizon under consideration. Our protocol is based on the assumption that the demand remains steady for a long period of time. If this is indeed the case, it would be worthwhile for a channel to rebalance itself as it can make up this cost through the incentive fees gained from the flow of transactions through it in steady state. If a channel chooses not to rebalance itself, however, there is a risk of being trapped in a deadlock, which is suboptimal for not only the nodes but also the channel.

\section{Conclusion}
This work presents DEBT control: a protocol for payment channel networks that uses source routing and flow control based on channel prices. The protocol is derived by posing a network utility maximization problem and analyzing its dual minimization. It is shown that under steady demands, the protocol guides the network to an optimal, sustainable point. Simulations show its robustness to demand variations. The work demonstrates that simple protocols with strong theoretical guarantees are possible for PCNs and we hope it inspires further theoretical research in this direction.
% 

%\subsection{Sociotechnical Challenges for Organisations Known from AI Research}
\label{sec:challenges}
% \subsection{Sociotechnological challenges known from research on Generative AI}

\begin{table}[t]
%\footnotesize
\begin{tabular}{p{11.7cm}p{2.7cm}}
\toprule
\textbf{Challenge} \& \textbf{Summary} & \textbf{Risk for the org.} \\
\midrule 
\textbf{What is regulated:} what kind of models even fall under the policy? Definitions can be based on training compute \cite{2024-ai-act}, data \cite{RogersLuccioni_2024_Position_Key_Claims_in_LLM_Research_Have_Long_Tail_of_Footnotes}, performance \cite{anderljungFrontierAIRegulation2023} etc. & Guidelines not scoped appropriately \\
\textbf{Detectability \& enforceability}: can we detect when AI models' usage violates the policy? Particularly, when generated content is used without disclosure? At present, no \cite{PuccettiRogersEtAl_2024_AI_News_Content_Farms_Are_Easy_to_Make_and_Hard_to_Detect_Case_Study_in_Italian}. & Guidelines not enforceable \\
\textbf{Factual errors}: the current models cannot reliably reject queries for which they do not have enough information \cite{amayuelas-etal-2024-knowledge}, and may output plausible-sounding but false results that are hard to identify and check \cite{zhang2023sirenssongaiocean,hicksChatgptBullshit2024}. Retrieval-augmented generation still has this problem \cite{mehrotraPerplexityBullshitMachine}. & Losing credibility and reputation \\
\textbf{Unsafe models}: in spite of attempts to force the models to follow certain content policies \cite{ouyang2022training}, the models can still violate them \cite{DerczynskiGalinkinEtAl_2024_garak_Framework_for_Security_Probing_Large_Language_Models}, and this training can even decrease the quality in some aspects \cite{10.1093/polsoc/puae020,casper2023open}. & Exposing employees to toxic outputs \\
\textbf{Privacy and security risks}: employees using non-local generative AI models may expose sensitive data from themselves and their organizations to the entity controlling such models \cite{Kim_2023_Amazon_warns_employees_not_to_share_confidential_information_with_ChatGPT_after_seeing_cases_where_its_answer_closely_matches_existing_material_from_inside_company} or third-party attackers \cite{WuZhangEtAl_2024_New_Era_in_LLM_Security_Exploring_Security_Concerns_in_Real-World_LLM-based_Systems}. Platform plugins may also increase vulnerabilities \cite{iqbal2024llm}. & Exposing sensitive data \\
\textbf{Misleading marketing claims}:  %With the training data too-big-to-inspect \cite{bender2021dangers}, the benchmark results may be compromised by test set contamination \cite{RogersLuccioni_2024_Position_Key_Claims_in_LLM_Research_Have_Long_Tail_of_Footnotes}. %But based on public benchmark numbers and especially claims of ``emergence'' \cite{RogersLuccioni_2024_Position_Key_Claims_in_LLM_Research_Have_Long_Tail_of_Footnotes}, 
employees may believe the claims of AI  model ``capabilities'' and trust the machine too much \cite{KheraSimonEtAl_2023_Automation_Bias_and_Assistive_AI_Risk_of_Harm_From_AI-Driven_Clinical_Decision_Support}, even though the benchmark results may be compromised by methodological problems and test set contamination \cite{RogersLuccioni_2024_Position_Key_Claims_in_LLM_Research_Have_Long_Tail_of_Footnotes}. %Automation also tends to make easy tasks easier and harder tasks harder \cite{SimkuteTankelevitchEtAl_Ironies_of_Generative_AI_Understanding_and_Mitigating_Productivity_Loss_in_Human-AI_Interaction}, and . 
& Degradation in the outputs of the organization \\
\textbf{Transparency \& accountability}: the social and legal norms on disclosing AI ``assistance'' and taking responsibility for the resulting text have not yet settled. The popular providers of these models do not accept responsibility for any faults in the output\footnote{}. & Public blame for any missteps \\
\textbf{Bias \& inequity}: The social biases in AI systems are well-documented  \cite{bolukbasi2016man,nadeem-etal-2021-stereoset,marchiori-manerba-etal-2024-social,stanczak2023quantifying,hutchinson-etal-2020-social,bender2021dangers,sharma2024generative}, %LLM training data may overrepresent some groups and underrepresent others \cite{bender2021dangers,sharma2024generative}. 
and the use of such models may reinforce misrepresentations in the society. & Discrimination, ethical code violation \\%, and the resulting models may fail to adequately represent minorities even when explicitly steered to do so \cite{santurkar2023whose}. \\ 
\textbf{Explainability}: checking model outputs would be easier if they were accompanied by rationales, the current interpretability methods are not faithful to the model’s actual decision-making \cite{lanham2023measuring,atanasova-etal-2023-faithfulness}. & Trusting unreliable solutions \\
\textbf{Brittleness}: Generative models perform worse outside of their training distribution \cite{McCoyYaoEtAl_2024_Embers_of_autoregression_show_how_large_language_models_are_shaped_by_problem_they_are_trained_to_solve,McCoyYaoEtAl_2024_When_language_model_is_optimized_for_reasoning_does_it_still_show_embers_of_autoregression_analysis_of_OpenAI_o1}. For language models, this includes changes in both language (idiolects, dialects, diachronic changes), content (e.g. evolving world knowledge), and slight variations in prompt formulation and examples \cite{zhu2023promptbench,LuBartoloEtAl_2022_Fantastically_Ordered_Prompts_and_Where_to_Find_Them_Overcoming_Few-Shot_Prompt_Order_Sensitivity}. & Employees wasting time and/or getting poor results \\
\textbf{Risks to creativity}: AI systems may generate unseen sequences of words, but their ``creativity'' is combinatorial, often lacking diversity, feasibility, and depth \cite{si2024can,padmakumar2024does}, and further degraded in languages other than English \cite{marco-etal-2024-pron}. Exposure to AI assistance could \textit{decrease} human creativity and diversity of ideas in non-assisted tasks \cite{kumar2024humancreativityagellms}. & Degradation in the outputs of the organization and its existing human resources \\
\textbf{Credit \& Attribution}: AI systems are commonly trained on copyrighted texts\cite{Gray_2024_OpenAI_Claims_it_is_Impossible_to_Train_AI_Without_Using_Copyrighted_Content} without author consent%, and memorization of high-quality human-authored texts is known to correlate with model performance \cite{liangHolisticEvaluationLanguage2022}
. This practice triggered multiple lawsuits  \cite{BrittainBrittain_2023_Lawsuits_accuse_AI_content_creators_of_misusing_copyrighted_work,Vynck_2023_Game_of_Thrones_author_and_others_accuse_ChatGPT_maker_of_theft_in_lawsuit,JosephSaveriLawFirmButterick_2022_GitHub_Copilot_investigation,GrynbaumMac_2023_Times_Sues_OpenAI_and_Microsoft_Over_AI_Use_of_Copyrighted_Work,Panwar_2025_Generative_AI_and_Copyright_Issues_Globally_ANI_Media_OpenAI_TechPolicyPress}, protests from the creators \cite{Heikkila_2022_This_artist_is_dominating_AI-generated_art_And_hes_not_happy_about_it,More_than_15000_Authors_Sign_Authors_Guild_Letter_Calling_on_AI_Industry_Leaders_to_Protect_Writers}, and questions about the credit for the author of an ``assisted'' text \cite{FormosaBankinsEtAl_2024_Can_ChatGPT_be_author_Generative_AI_creative_writing_assistance_and_perceptions_of_authorship_creatorship_responsibility_and_disclosure}. & Legal exposure, violating plagiarism policies/principles \\
\textbf{Carbon emissions:} The current AI systems are environmentally costly for both training and inference \cite{luccioniCountingCarbonSurvey2023,dodge2022measuring,bouza2023estimate,liMakingAILess2023}, and workflows that significantly rely on them would have adverse effect on climate action. & Not meeting sustainability goals\\
\bottomrule
\end{tabular}
\caption{Major sociotechnical challenges for organizations relying on the current AI technology}
\label{fig:sociotech-challenges}
\end{table}
%Generative models are currently rushed into many applications and have been adopted by many users. This poses a range of sociotechnical challenges, stemming from the technical imperfections of these models, legal uncertainty, questions about the authorship in ``AI-assisted'' content, misleading marketing, and the issues related to how people use such models. This work does not aim to provide a comprehensive survey, but we list some of the key issues previously identified in academic literature in \autoref{fig:sociotech-challenges}, together with the possible risks to organizations whose employees rely on this technology.\looseness=-1
% \subsection{Implications from Policies for Research}
%
AI research literature point to numerous sociotechnical challenges for organisations relying on the current AI technology. Our work does not aim to provide a comprehensive survey, but we list the issues that we identified through literature review in \autoref{fig:sociotech-challenges}, together with the possible risks to organisations whose employees rely on this technology. This list serves as background to the types of problems that organisational policy or regulatory frameworks may identify as issues that need addressing.


%The advancements in generative artificial intelligence (GenAI) models, particularly through the pre-training of large-scale architectures with extensive parameters on vast datasets, have significantly enhanced their capabilities across a broad range of domains. In particular, these models demonstrate promising strengths in contextual understanding; text, image, and video generation; multilingual applications; and zero-shot or few-shot adaptability to new tasks. Such abilities have facilitated their widespread adoption in diverse areas.

%Nevertheless, GenAI models still exhibit limited capabilities in certain tasks, especially those that demand deeper expertise or domain-specific knowledge \cite{rein2023gpqa}. While ongoing developments can improve such limited capabilities and make GenAI models applicable to an ever-growing set of tasks, their adoption also brings considerable challenges and risks. Addressing these issues is essential to ensure alignment with existing active policies (e.g., those within educational institutions or news organizations) and international regulations (e.g., the EU AI Act). Below, we outline the critical challenges associated with the use of GenAI models, as identified by existing research and which should be considered in the development of comprehensive and responsible policies and regulations.

%\paragraph{Inaccuracy and Hallucination}
%One of the most pressing challenges of GenAI models is their tendency to generate inaccurate or ``hallucinated'' content, where outputs are not grounded in training data or established facts \cite{liu-etal-2023-evaluating,10.1145/3703155}. Because their pre-training corpora often come from automatically scraped web sources--many of which include fabricated, outdated, previously generated, or biased information models can produce responses that \textit{conflict with real-world knowledge, user-provided input, or even their own previously generated content}. This issue poses particular risks in domains that demand high accuracy, such as legal practices, where considering the latest changes in the law is of critical importance \cite{cheong2024not}. Moreover, GenAI models may offer \textit{plausible-sounding but false information}, making it difficult for both automated systems and human reviewers to detect these subtle errors \cite{zhang2023sirenssongaiocean}. Furthermore, models often \textit{struggle to classify whether a question is within their scope of knowledge}, whereas even smaller or open-source models perform near random on such tasks \cite{amayuelas-etal-2024-knowledge}. Additionally, the reinforcement learning from human feedback (RLHF) technique \cite{ouyang2022training} for aligning model responses with human preferences, the vague knowledge boundary \cite{ren2024investigatingfactualknowledgeboundary}, and the inherently black-box nature of many GenAI models \cite{sun2022black} further complicate the detection, explanation, and mitigation of hallucinations. While plugins or vector databases that store validated, up-to-date domain information (e.g., legal statutes) could alleviate these issues, their integration must be both reliable and user-friendly. Finally, as GenAI models are expected to excel in multi-task, multi-lingual, and multi-domain environments \cite{bang-etal-2023-multitask}, evaluating and mitigating their hallucinations becomes even more challenging.

%\paragraph{Safety and Alignment}
%The \textit{misalignment between the training objectives of GenAI models and their desired behaviour} poses significant safety risks. For instance, while a large language model (LLM) may be optimized to predict the next word from a massive corpus--regardless of the content quality of that corpus--users typically want the model to generate factual, useful information rather than harmful or false content. This gap underscores the challenge of ``scalable oversight'', where evaluating and controlling highly capable models in complex tasks remains unsolved  \cite{amodei2016concreteproblemsaisafety,leike2017aisafetygridworlds}. 
% More importantly, simply enumerating known or planned use cases of foundation models is not sufficient to capture the full range of ways they might be deployed. 
%Techniques such as RLHF have been introduced to align models more closely with human values, yet these methods face fundamental limitations \cite{casper2023open} such as increased hallucination, ideological favouritism, sycophancy, or even resistance being shut down. In addition, they may inadvertently incentivize ``reward hacking'', in which the model learns to exploit the reward system to achieve high scores without actually solving the desired objectives \cite{10.1093/polsoc/puae020}. These risks are unintentional but become more concerning if the alignment process is corrupted by adversarial actors. Overall, there is concern that the alignment of GenAI models could instead lead to prioritizing simple engagement metrics at the expense of broader societal or consumer well-being.

%\paragraph{Security and Privacy Concerns}
%Generative models raise significant security and privacy issues \cite{iqbal2024llm, yao2024survey}. Interactions with large language models (LLMs) often \textit{lack privacy protections}, making their contents vulnerable to subsequent discovery or adversarial exploitation. Even when an LLM operates locally, chat records typically remain unprotected unless explicitly shielded from disclosure.
%Moreover, the training data for these models often comes from uncurated web sources \textit{susceptible to malicious ``poisoning''} \cite{10.5555/3042573.3042761}, potentially yielding harmful outputs. For instance, adversaries may inject hateful speech into just a few online posts to manipulate a foundation model’s training data, and even small-scale injections can significantly corrupt outputs \cite{schuster2021you}. Compounding these risks, \textit{the dual-use nature of LLMs allows them to be adapted for malicious purposes}, such as disinformation campaigns or targeted extortion attempts \cite{kaffee-etal-2023-thorny}.
% , and raises concerns about dual use, wherein models are repurposed beyond their originally foreseen tasks, potentially enabling overlearning or adversarial reprogramming \cite{elsayed2018adversarial}. 
% Furthermore, multimodality can expand a model’s attack surface by allowing cross-modal inconsistencies to be exploited, as shown when an apple labeled “iPod” was misclassified by CLIP. 
%Beyond data poisoning, the \textit{confidentiality, integrity, and availability} of LLM-based systems can also be undermined by inference or reconstruction attacks, adversarial examples \cite{biggio2013evasion, szegedy2013intriguing}, and resource-depletion exploits \cite{shumailov2021sponge, hong2021a}.

%\paragraph{Transparency and Accountability}
%The opacity characterising the development and application of generative models creates significant challenges for transparency and accountability. Even when GenAI model weight parameters are made public, \textit{many providers withhold critical information} regarding their training and fine-tuning procedures, preventing effective inspection and regulation of these models \cite{bender2021dangers}. Furthermore, beyond merely determining whether models produce correct outcomes, it is equally important to assess whether they do so for the right reasons. Current efforts to align model outputs with human preferences, for instance, have shown that fake, misleading alignment and sycophantic behaviour can emerge, raising concerns about the motivation and capabilities of GenAI models \cite{greenblatt2024alignmentfakinglargelanguage,vanderweij2024aisandbagginglanguagemodels}. Additionally, models have been found to produce increasingly persuasive yet potentially unfaithful and misleading content compounds the difficulty of explaining its behaviour \cite{rogiers2024persuasion,bommasani2021opportunities}. While models can supply explanations of their decisions, such as chain-of-thought responses \cite{wei2022chain}, these rationales are often found to be unfaithful to the model’s actual internal decision-making \cite{lanham2023measuring,atanasova-etal-2023-faithfulness}. Exploring the underlying reasons behind a model’s outputs thus remains an open research challenge, critical to ensure accountability among model providers fostering transparent usage of GenAI systems and creating a regulatory environment in which scientists, policymakers, and end-users can confidently employ GenAI technologies.

%\paragraph{Biases and Inequity}
% , including cultural and linguistic minorities, neurodivergent individuals, and other nontypical users. 
%GenAI systems can perpetuate biases that can often manifest as stereotypes or attitudes \cite{bolukbasi2016man, nadeem-etal-2021-stereoset, marchiori-manerba-etal-2024-social, stanczak2023quantifying,hutchinson-etal-2020-social}, and which can propagate through downstream models and reinforce misrepresentations in society. Overrepresentation of majority voices is another concern, resulting in the homogenization and the exclusion of minority viewpoints -- an echo chamber effect that may be further exacerbated if reliance on generative AI becomes widespread \cite{sharma2024generative}. Furthermore, studies have shown that current large language models (LLMs) often overlook certain demographic groups, such as individuals aged 65+ or widowed, even when explicitly steered to represent them \cite{santurkar2023whose}. 
% Notably, technical vulnerabilities, such as ``jailbreak'' attacks, reveal the fragility of safety training protocols, suggesting that merely scaling up these methods without fundamentally revising optimization objectives may exacerbate mismatched generalization. 
%Mitigating these inequities and biases requires (1) ensuring equitable access and skill development for using AI responsibly, (2) advancing technical research to reliably attribute and address the root sources of bias, and (3) implementing effective safeguards so that generative AI promotes fairness rather than reinforcing discrimination. Finally,  GenAI systems risk exacerbating existing inequities by privileging those with greater access and technical skills required to use GenAI models, thereby widening the divide between privileged and underprivileged populations.



%\paragraph{Known GenAI Limitations} Recent evaluations of GenAI models highlight a range of shortcomings that \textit{limit their performance in real-world scenarios}. Although LLMs can generate ideas that sometimes surpass human experts in perceived novelty, their \textit{creativity largely remains combinatorial rather than genuinely groundbreaking, often lacking diversity, feasibility, and depth} \cite{si2024can, padmakumar2024does}. LLM models further exhibit worse creativity skills in languages different from English \cite{marco-etal-2024-pron}. Additionally, exposure to LLM assistance in creative tasks has been found to lead to decreased creativity and diversity of ideas in subsequent non-assisted tasks \cite{kumar2024humancreativityagellms}. The latter reflects an increasing concern regarding the over-reliance on GenAI, leading to the erosion of critical thinking, specialized skills, and homogenisation of individual voices, especially in educational settings \cite{zhai2024effects}.
%LLMs face additional technical limitations. They exhibit limited robustness to distribution shifts, such as evolving world knowledge or language drift, which jeopardises the reliability of their outputs in high-stakes environments (e.g., in legal or medical applications) \cite{yuan2023revisiting}. Furthermore, LLMs have been increasingly used by being assignment personas, which has been found to unexpectedly skew their performance and introduce biases, highlighting broader concerns about stereotyping and privacy and personalisation applications of GenAI \cite{zhang2024personalization}. Existing evaluations also reveal subpar performance, among others, in abstract reasoning tasks \cite{gendron2023large}, event semantics \cite{tao2023eveval}, non-Latin script contexts \cite{lai-etal-2023-chatgpt}, and multimodal data. LLMs are also highly sensitive to adversarial or even slight prompt variations \cite{zhu2023promptbench}, calling for further research into prompt engineering and model robustness. Overall, while LLMs offer considerable promise, substantial work remains to address these limitations, among others, and improve their performance, particularly regarding creative potential, interpretative flexibility, and ethical concerns.

%The challenges and limitations outlined above underscore the complexity of aligning generative models with both local and global guidelines. 


% TODOs:
% education sections polish up - Pepa
% write comparison socio technological challenges (6) - Pepa&Lucie
% conclusion - Pepa
% polishing (Intro, Abstract Pepa) - Pepa&Lucie
% author consent - Pepa

% \section{Bottom-up Perspectives on Generative AI Policy}


%
%
%\section{Implications for Future Research}

\section{Conclusion}
In conclusion, the rapid adoption of AI technologies across diverse domains has exposed significant gaps in governance, with multiple organisations scrambling to quickly develop their policies. Our comparative analysis of AI guidelines in universities and news organisations highlights both shared and domain-specific challenges, such as the need for clear accountability mechanisms, addressing biases, and managing domain-specific applications like personalised learning and content generation. We have also identified multiple challenges that are known in academic research, but not addressed by the current policies. These findings underscore the fragmented nature of current governance efforts and the critical need for cohesive policies that balance local organisational needs with broader societal and technological imperatives, while recognizing and supporting areas where more research is needed for better policies. %Furthermore, our study reveals a gap between the intended uses of AI outlined in policies and the actual capabilities and limitations of these technologies. Misaligned expectations not only complicate the implementation of responsible AI practices but also risk eroding trust in AI systems. 
By identifying these gaps and challenges, this paper offers actionable insights for refining international legislation and informs the critical future directions of research. Ultimately, bridging the disconnect between local practices, academic research, and global standards is essential for ensuring the safe, fair, and effective deployment of AI across diverse contexts.

\section*{Ethical Considerations Statement}
This study analyses AI policies from ten news organisations and ten universities to identify gaps in the EU AI Act that could be clarified for its implementation and point out possible research directions. All data used in this research is derived from publicly available policy documents, ensuring transparency and compliance with ethical research standards. We strived to present any interpretations or critiques of the policies in a constructive manner to inform policymakers, AI practitioners, and institutional stakeholders.

This study strives to respects intellectual property rights by citing all sources and representing policy content appropriately. Since the analysis pertains to institutional policies rather than individual data, no personally identifiable information is processed or collected.

Finally, we acknowledge that AI governance is an evolving field. To the best of our knowledge, our findings reflect the state of AI policies at the time of analysis and should be interpreted in light of ongoing regulatory and institutional developments. We encourage further interdisciplinary dialogue to refine AI governance frameworks in alignment with ethical, legal, and societal expectations.









\section*{Acknowledgements}
\section{Acknowledgements}


\bibliographystyle{ACM-Reference-Format}
\bibliography{main}


\appendix

\section*{Appendix}

\begin{table*}[ht]
\begin{tabular}{lp{10cm}l}
\toprule
\textbf{News Organisation} & \textbf{Policy Link} & \textbf{Version} \\
\midrule
Guardian & \url{https://www.theguardian.com/help/insideguardian/2023/jun/16/the-guardians-approach-to-generative-ai} & June 2023 \\
ANP & \url{https://drive.google.com/file/d/1-3sAJtkOJrdIGw-gZFqYDEGQQNw13e0U/view} & April 2023 \\
Mediahuis & \url{https://www.independent.ie/editorial/editorial/aiframeworkguide140623.pdf} & May 2023 \\
VG & \url{https://www.vg.no/informasjon/redaksjonelle-avgjorelser/188}& April 2023 \\
Parisien & \url{https://www.cbnews.fr/medias/image-groupe-echos-parisien-s-engage-face-intelligence-artificielle-generative-76799} & May 2023 \\
FT & \url{https://www.ft.com/content/18337836-7c5f-42bd-a57a-24cdbd06ec51}& May 2023 \\
SZ & \url{www.ethz.ch/en/the-eth-zurich/education/ai-in-education.html} & Dec. 2024 \\
Spiegel & \url{https://www.sueddeutsche.de/projekte/artikel/kolumne/kuenstliche-intelligenz-ki-e903507/} & June 2023 \\
BBC & \url{https://www.bbc.co.uk/supplying/working-with-us/ai-principles/} & Feb. 2024 \\
Ringier & \url{https://www.ringier.com/ringier-introduces-clear-guidelines-for-the-use-of-artificial-intelligence/} & May 2023 \\
\bottomrule
\end{tabular}
\caption{Links to AI policies and their versions for each news organisation.}
\label{table-app-news-links}
\end{table*}


\begin{table*}[ht]
\begin{tabular}{lp{10cm}l}
\toprule
\textbf{University} & \textbf{Guidelines Link} & \textbf{Version} \\
\midrule
TUM & \url{www.prolehre.tum.de/fileadmin/w00btq/www/Angebote_Broschueren_Handreichungen/ProLehre_Handreichung_ChatGPT_EN.pdf} & Feb. 2023 \\
TU Delft & \url{www.tudelft.nl/teaching-support/educational-advice/assess/guidelines/ai-chatbots-in-unsupervised-assessment} & June 2023 \\
KTH & \url{www.kth.se/profile/wouter/page/chatgpt-pragmatic-guidelines-for-students-september-2023} & Sep. 2023 \\
Aalto & \url{www.aalto.fi/en/services/guidance-for-the-use-of-artificial-intelligence-in-teaching-and-learning-at-aalto-university}& Aug. 2023 \\
DTU & \url{www.dtu.dk/english/newsarchive/2024/01/dtu-opens-up-for-the-use-of-artificial-intelligence-in-teaching} & Jan. 2024 \\
KUL & \url{www.kuleuven.be/english/genai}& - \\
ETH & \url{www.ethz.ch/en/the-eth-zurich/education/ai-in-education.html} & Dec. 2024 \\
CUNI & \url{www.ai.cuni.cz/AI-12-version1-ai_elearning_en.pdf} & June 2023 \\
Vie & \url{www.studieren.univie.ac.at/en/studying-exams/ai-in-studies-and-teaching/} & Sep. 2024 \\
UdL & \url{www.conselhopedagogico.tecnico.ulisboa.pt/files/sites/32/ferramentas-de-ai-no-ensino-v8-1.pdf} & Nov. 2023 \\
UiO & \url{www.uio.no/english/services/ai/} & - \\
\bottomrule
\end{tabular}
\caption{Links to AI guidelines and their versions for each university.}
\label{table-app-uni-links}
\end{table*}

\end{document}
\endinput
%%
%% End of file `sample-sigconf-authordraft.tex'.
