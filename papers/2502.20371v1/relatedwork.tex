\section{Related Work}
A framework for constraints on the sample space of generative models was first theoretically described in \cite{hanneke2018actively}. In their framework, constraints are represented through a black-box oracle function labeling samples as valid or invalid. This problem has been practically explored for generative adversarial networks (GANs) \cite{kong2023data} and diffusion models \cite{naderiparizi2024dont}.
More recently, \citet{christopher2024constrained} proposed a method for generating constraint satisfying samples from pre-trained diffusion models. It requires a projection operator the constraint set, which is generally intractable for complex constraints. Moreover, they employ a Langevin dynamics-based \citep{welling2011bayesian} sampler which is slow to converge.
The main idea in all of these methods was to improve generative models by incorporating information from the constraints. In this paper, however, the goal is to construct a model family that does not generate invalid samples by design.

Recognizing the expressivity of diffusion models, various approaches to incorporating pre-defined constrains into them have emerged in the literature. \citet{lou2023reflected} proposed reflected diffusion models that enforce the whole diffusion sampling trajectory to remain bounded in a convex set. \citet{fishman2023diffusion,fishman2024metropolis} extended reflected diffusion models to support more general constraints. However, their approaches are only evaluated on low-dimensional problems with simple constraints. Moreover, reflected diffusion makes the forward process, and consequently training, expensive.  \citet{fishman2023diffusion} also proposed a barrier function based approach for constrained diffusion models. \citet{liu2024mirror} used barrier functions to transform constrained domains into unconstrained dual ones. Both these methods only support convex constraints.

Another closely related body of work is diffusion bridges, stochastic processes that are guaranteed to end in a given constraint set. \citet{wu2022diffusion} developed a set of mathematically sufficient conditions for designing diffusion bridges to a given constraint set.  The follow-up work of \citet{liu2023learning} used bridges to formulate diffusion models on discrete sets. They also provided closed-form bridges for a restricted set of constraints such as product of intervals. These closed form bridges quickly become intractable as the constraint set gets more complex.