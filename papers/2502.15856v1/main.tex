\documentclass[sn-mathphys]{sn-jnl}

\DeclareMathOperator{\EX}{\mathbb{E}}% expected value
\usepackage{gensymb} 
\usepackage{textcomp} 
\usepackage{pifont} 
\usepackage{numprint}
\usepackage{adjustbox}
%\usepackage{listings}
%\usepackage{fixme}
%\usepackage{float}
%\fxsetup{status=draft}
\usepackage{amsmath,amsthm,amssymb,amsfonts, scrextend}
\usepackage{algorithm}
\usepackage{tabularx} % For table with automatic column width management
\usepackage{array} % For better vertical alignment

\usepackage{hyperref}

\begin{document}

\title[Critical Assesment]{%A critical assessment of the artistic style replication capabilities of modern generative tools
A Critical Assessment of Modern Generative Models' Ability to Replicate Artistic Styles}

\newcommand{\E}{\mathbb{E}}

\author*[1]{\fnm{Andrea} \sur{Asperti}}\email{andrea.asperti@unibo.it}

\author[2]{\fnm{Franky} \sur{George}}\email{f.george-2021@hull.ac.uk}

\author[1]{\fnm{Tiberio} \sur{Marras}}\email{tiberio.marras@studio.unibo.it}

\author[1]{\fnm{Razvan Ciprian} \sur{Stricescu}}\email{razvancipr.stricescu@studio.unibo.it}

\author[1]{\fnm{Fabio} \sur{Zanotti}}\email{fabio.zanotti2@studio.unibo.it}

\affil*[1]{\orgdiv{Department of Informatics: Science and Engineering (DISI)}, \orgname{University of Bologna}, \orgaddress{\street{Mura Anteo Zamboni 7}, \city{Bologna}, \postcode{40126},
\country{Italy}
}}

\affil*[2]{\orgdiv{DAIM, Computer Science}, \orgname{University of Hull}, \orgaddress{\street{Cottingham Rd}, \city{Hull}, \postcode{HU6 7RX},
\country{United Kingdom}
}}

\abstract{
In recent years, advancements in generative artificial intelligence have led to the development of sophisticated tools capable of mimicking diverse artistic styles, opening new possibilities for digital creativity and artistic expression. This paper presents a critical assessment of the style replication capabilities of contemporary generative models, evaluating their strengths and limitations across multiple dimensions. We examine how effectively these models reproduce traditional artistic styles while maintaining structural integrity and compositional balance in the generated images.

The analysis is based on a new large dataset of 
AI-generated works imitating artistic styles of the past,
holding potential for a wide range of applications:
the ``AI-pastiche" dataset.

The study is supported by extensive user surveys, collecting diverse opinions on the dataset 
and investigation both technical and aesthetic challenges, including the ability to generate outputs that are realistic and visually convincing, the versatility of models in handling a wide range of artistic styles, and the extent to which they adhere to the content and stylistic specifications outlined in prompts.

This paper aims to provide a comprehensive overview of the current state of generative tools in style replication, offering insights into their technical and artistic limitations, potential advancements in model design and training methodologies, and emerging opportunities for enhancing digital artistry, human-AI collaboration, and the broader creative landscape.}


\keywords{Generative AI, style replication, creativity, authenticy, versatility, accuracy.}

\maketitle

\section{Introduction}
Generative AI has rapidly expanded into creative fields, transforming how visual art is produced, modified, and experienced. Early breakthroughs, such as StyleGAN \cite{StyleGAN2,StyleGAN-T}, laid the foundation for high-quality image synthesis, but the field has since been revolutionized by the rapid rise of diffusion-based models \cite{DDPM,DDIM,stable-diffusion}. These newer techniques have significantly enhanced the ability to generate realistic images, mimic artistic styles \cite{SGMD,ArtBank}, and even create entirely new visual compositions \cite{Dalle2,Imagen}, establishing diffusion models as the dominant paradigm in generative artistry. Among these capabilities, style replication has emerged as a key area of interest, allowing users to apply diverse historical and modern artistic styles to AI-generated images \cite{FeaST,Style_injection,Step-aware}. This technology enables greater artistic expression and personalization, bridging the gap between computational creativity and traditional artistry and empowering artists, designers, and hobbyists to explore and reinterpret visual styles in ways that were previously highly specialized or time-intensive \cite{designing-barros,GameDesign,inspiring-Haase,craft-edu}.

%Recent advancements in conditioning techniques, supported by robust text-image embedding methods such as CLIP \cite{CLIP}, improved adaptation strategies like LoRA \cite{LoRA} and QA-LoRA \cite{QA-LoRA}, as well as enhanced fine-tuning mechanisms \cite{DreamBooth,Multi-concept}, have significantly increased the precision and flexibility of replicating artistic styles. These methods enable models to respond to more nuanced prompts, effectively aligning generated outputs with highly specific artistic styles or content directives. Such advancements have expanded the creative potential of generative AI, providing unprecedented levels of control and sophistication in style-replication tasks. They empower artists, designers, and hobbyists to explore and reinterpret visual styles in ways that were previously highly specialized or time-intensive \cite{designing-barros,GameDesign,inspiring-Haase,craft-edu}.

The purpose of this study is to provide a critical assessment of the capabilities and limitations of current generative tools in effectively replicating styles. By examining both the technical performance and aesthetic outcomes of these tools, the study aims to highlight their strengths, identify areas where they fall short, and offer insights into the potential improvements needed to enhance their application in creative fields. 

Specifically, we compared twelve modern generative models comprising \href{DallE 3}{https://openai.com/index/dall-e-3/}, \href{Stable Diffusion 1.5}{https://huggingface.co/stable-diffusion-v1-5/stable-diffusion-v1-5}, \href{Stable Diffusion 3.5 large}{https://stabledifffusion.com/tools/sd-3-5-large}, \href{Flux 1.1 Pro}{https://flux1.ai/flux1-1}, \href{Flux 1 Schnell}{https://fluxaiimagegenerator.com/flux-schnell}, \href{Omnigen}{https://omnigenai.org/}, \href{Ideogram}{https://ideogram.ai/login}, \href{Kolors 1.5}{https://klingai.com/text-to-image/new}, \href{Firefly Image 3}{https://firefly.adobe.com/}, \href{Leonardo Pheonix}{https://leonardo.ai}, \href{Midjourney V6.1}{https://www.midjourney.com/imagine} and \href{Auto-Aesthetics v1}{https://neural.love/blog/auto-aesthetics-v1-ai-art-revolution}. 

The models were compared using 73 uniform prompts that span a broad range of painting styles from the past five centuries. This resulted in the creation of a
large supervised dataset of AI-generated artworks: the
AI-pastiche dataset. This dataset not only provides labeled examples of AI-generated images but also offers a valuable resource for advancing research in areas such as deepfake detection, digital forensics, and the ethical study of AI-generated content. By supplying a controlled, high-quality set of deepfakes, the dataset aids in training and testing models for improved detection accuracy, robustness against manipulation, and broader exploration of generative AI capabilities across fields ranging from security to digital art.

The quality of the generated images was evaluated based on two criteria: the ability of the models to faithfully replicate human-crafted artwork and their capacity to adhere to the style and content specified in the prompts. The first criterion was assessed through a public survey in which participants were asked to distinguish between human-created and AI-generated images. The second criterion, which involved a per-prompt comparison of the samples generated by different models, was evaluated directly by members of our team along with a few additional volunteers.

Our results reveal that while modern generative models demonstrate remarkable artistic capabilities, they still encounter significant challenges in faithfully replicating historical styles. Rather than a lack of detail, hyperrealism emerges as the primary obstacle—AI-generated images often display excessive sharpness and unnatural precision, making them visually striking but historically inconsistent. According to our evaluation, state-of-the-art models successfully produce images that non-expert users misidentify as human-created in less than 30\% of cases, highlighting the persistent gap between AI-generated and traditionally crafted artworks.

This work is part of a larger and ambitious project that aims to assess whether Large Language Models possess an aesthetic sense and, if so, to identify the aesthetic principles that guide their preferences. This investigation represents a significant advancement in understanding the emergent abilities of LLMs \cite{emergent,emergent_mirage,loss_perspective,Angelo_TOM} and their social implications.
In evaluating the aesthetic sense of LLMs, it is essential to bypass any potential familiarity the models may have with specific artworks, as this could allow them to draw on pre-existing evaluations or learned data. By using a dataset of fictional or AI-generated artworks, such as the one created as part of this study, we can ensure that LLMs rely solely on the information provided in the dataset, thus offering a more controlled evaluation of their aesthetic judgment.

Summing up, our work makes two major contributions:
\begin{itemize}
\item we created a large, richly annotated dataset of AI-generated images, suitable for a wide range of applications;
\item we conducted an in-depth evaluation of some of the most widely used generative models, assessing their style-transfer capabilities through suitable surveys.
\end{itemize}

The article has the following structure. In Section~\ref{sec:methodology}, we describe our methodology, the way the dataset has been created, the selection of models and the way surveys have been
formulated and conducted. Section~\ref{sec:dataset} gives
a detailed description of the dataset, and the relative metadata.
In Section~\ref{sec:surveys} we give a detailed description 
about the surveys, the target audience, and the frameworks used to publish and collect data.
Section~\ref{sec:results} describes the results of the
evaluation. An in-depth discussion of some of the main critical aspects of the style-transfer capabilities of generative tools is given in Section~\ref{sec:discussion}. 
In Section~\ref{sec:conclusions}, we offer a few ideas for
future developments, and outlies some possible applications 
of our dataset.

\section{Related works}

AI-driven artistic style transfer has grown significantly in recent years, driven by advances in deep learning and generative models. Several works have explored the capabilities, limitations, and applications of AI-generated imagery. Our work contributes with a comprehensive evaluation of multiple generative models, emphasizing their adherence to artistic style and prompt fidelity.

Early works such as Gatys et al. \cite{Gatys_neural_representation} laid the foundation for neural style transfer, introducing methods that blend content and style representations from convolutional neural networks. Subsequent research expanded on these concepts, improving efficiency and control over style application \cite{Huang_style_transfer_normalization}. More recently diffusion-based models have demonstrated superior results in high-fidelity artistic synthesis, allowing for more nuanced style adaptation. Our study build upon these advancements but diverges in its focus on evaluating multiple state-of-the-art models across diverse artistic styles and historical periods. This allows for a broader assessment of model performance. 

One major area of focus has been figuring out how to evaluate and detect AI-generated images. For instance, studies like CIFAKE by Bird and Lotfi \cite{cifake} and GenImage by Zhu et al. \cite{genimage} have worked on measuring how realistic synthetic images are and developing techniques to tell them apart form human-made art. Similarly, Li et al. \cite{AIart} explored the world of adversarial AI-generated art, shedding light on the challenges of authentication and detection. These efforts are vital for assessing the authenticity of generated works, particularly in contexts where human perception plays a critical role. 

To support this kind of research, several large scale datasets have been created.
\begin{itemize}
    \item \textbf{ArtiFact Dataset:}\cite{artifact_dataset} This is a diverse mix of real and synthetic images, covering everything from human faces to animals to landscapes, vehicles and artworks. It includes images synthesized by 25 different methods, including 13 GAN-based models and 7 diffusion models.
    \item \textbf{WildFake Dataset:}\cite{wildfake_dataset} A dataset designed to assess the generalizability of AI-generated image detection models. It contains fake images sourced from the open-source community, covering various styles and synthesis methods.
    \item \textbf{TWIGMA Dataset:}\cite{twigma_dataset} A large-scale collection of AI-generated images scraped from Twitter, from 2021 to 2023, including metadata such as tweet text, engagement metrics and associated hashtags.
\end{itemize}
While these studies focus on detecting AI-generated images we focus on examining how convincingly these images replicate human-created art. Through public perception surveys, we assess whether generated paintings can be mistaken for human artwork providing insights into the models' ability to deceive the viewer aesthetically rather than algorithmically. 

Beyond detection, generated images are increasingly used as data sources for synthetic training and research applications. The work of Yang et al. \cite{aigen}
discusses the implications of using AI-generated images for training machine learning models. They explore the potential of synthetic datasets to enhance machine learning capabilities while also addressing concerns related to biases, authenticity, and ethical challenges.

Another direction in the field is the use of diffusion models for artistic style transfer. Researchers such as Chung et al. \cite{Style_injection} and Zhang et al. \cite{Step-aware} \cite{ArtBank} have introduced training-free methods and pre-trained diffusion models specifically designed for style adaptation. These works highlight the effectiveness of modern diffusion-based architectures in achieving high-fidelity artistic synthesis while maintaining flexibility for style injection. Furthermore, the work of Png et al. \cite{FeaST} proposes a feature-guided approach that improves control over the stylistic aspects of the generated output. 

The creative applications of generative AI has also been widely discussed. Haase et al. \cite{inspiring-Haase} explore the role of generated imagery in inspiring human creativity, particularly in design workflows. Similarly Barros and Ai \cite{designing-barros} investigate the integration of text-to-image models in industrial design, while Vartiainen and Tedre \cite{craft-edu} examine their use in craft education. We complement these works by examining the limitations of generative tools in artistic fidelity, particularly their struggle with maintaining compositional balance, avoiding anachronisms, and ensuring stylistic coherence. We highlight critical shortcomings such as overuse of hyperrealism, anatomical distortions and misinterpretations of historical context, which could be key obstacles to seamless integration into professional artistic workflow.  

Furthermore, a growing body of work focuses on understanding the emergent capabilities of large language models and their application in aesthetic evaluation. Studies such as those by Wei et al. \cite{emergent} and Du et al. \cite{loss_perspective} discuss how LLMs develop new abilities, such as the preference for certain artistic styles. Wang et al. \cite{evaluation-metrics} analyze evaluation metrics for generative images, offering insights on how to assess AI-generated art both quantitatively and qualitatively. These studies can be expanded with our proposed dataset, which unlike other existing ones is a controlled dataset of synthetic artworks. 

\section{Methodology}
\label{sec:methodology}
In this section we outline our methodology for the creation of
the dataset, the selection of models, and their evaluation.

\subsection{Creation of the dataset, aims, methodology used for data acquisition}

The most delicate point in the creation of the dataset was
the definition of the prompts. The importance of providing well-structured prompts for style-transfer operations is well known,
due to their direct impact on the quality and relevance of the generated outputs \cite{prompts-Oppenlaendr,prompts-Sanchez23,optimizing-prompts}. A clear and well-defined prompt eliminates ambiguity, ensuring that the model has a precise understanding of the desired style and content. Without this clarity, models can produce inconsistent or irrelevant results, making it difficult to achieve the intended artistic effect. 

In our case, prompts were generated with the assistance of ChatGPT, iteratively fine-tuning its output until acceptable results were obtained across different generators. A sample was deemed ``acceptable" if ChatGPT could recognize the required style in the generated image based on the prompt. 
%According to our findings, using ChatGPT for prompt generation did not provide any noticeable advantage specifically for DALL·E. 
Once finalized, the same prompt was passed to all models, and the results were compiled into a database accompanied by a rich set of metadata (see Section~\ref{sec:dataset}).

All prompts followed a common structure. They typically began with an indication of the style and historical period to imitate, sometimes reinforced by referencing a specific painter. This was followed by a detailed description of the subject, including suggestions for colors and tones. Finally, each prompt concluded with a hint about the overall sentiment or emotion the artwork was intended to convey. 

Here are a couple of examples:
\begin{itemize}
\item ``Generate a detailed winter landscape painting in the Flemish renaissance style of the second half of the XVI century. Depict a snow-covered village with small, rustic houses nestled into a hilly landscape. Include bare, slender trees in the foreground with hunters walking through the snow, accompanied by dogs. The scene should feature frozen lakes or ponds in the background, where villagers are skating and engaging in winter activities. The sky is a muted, wintry blue-gray, and the overall tone of the painting should evoke a peaceful, yet somewhat melancholic atmosphere, with intricate details showing rural life during winter."
\item ``Generate a view of Venice in the vedutism style of the first half of the XVIII century, focusing on a scene along the Grand Canal. The composition features detailed classical architecture with grand domes and facades, and gondolas moving along the canal. Add soft clouds to the sky and ensure there is little fading in the horizon, providing clear visibility of distant buildings. The color palette should include very soft blues and warm earth tones, avoiding saturated colors. The atmosphere remains calm and luminous, with minimal light-and-shadow effects, capturing the beauty and grandeur of Venice from a broad perspective."
\end{itemize}


\subsection{Models}
\label{sec:models}
% Generative models for imaging are a class of machine learning algorithms designed to create new images by learning patterns from existing data. By approximating the underlying distribution of visual data, these models generate novel outputs, forming a cornerstone of creative AI applications.

% Within this domain, models are broadly categorized as text-to-image (Text2Img) or image-to-image (Img2Img), each serving distinct purposes. Text2Img models generate entirely new images based on textual descriptions, effectively translating linguistic cues into visual content. Img2Img models, on the other hand, modify or enhance existing images, using the input as a foundational reference while applying stylistic or contextual transformations.

% Our research primarily focused on the Text2Img class of models, though we also explored Img2Img approaches such as Instruct-Pix2Pix~\cite{instruct_pix2pix_i2i}, and several Stable Diffusion such as Img2Img-Turbo \cite{img2img_turbo_i2i}. A key challenge with Img2Img models lies in their treatment of the input image: it is typically adopted as a strong content reference refined through textual guidance, rather than as a flexible stylistic guide. In cases where highly specific and strong stylistic blending is required, Img2Img models often fail to deliver convincing results. 
% %A more detailed discussion of these limitations can be found in the thesis of some of the authors \cite{}.


% The models taken into consideration for our investigation have been already listed in the introduction. In Table \ref{tab:generative_models_table}, we summarize a few relevant information. Unfortunately, many models are proprietary and their architecture or training technique are not fully disclosed.
% % As far as we know, all models are based on some variant of diffusion techniques. 

%A more detailed discussion of each model is given in the Appendix A.

Image generative models are a class of machine learning algorithms designed to synthesize novel images by learning the underlying patterns in existing data. By approximating the underlying distribution of visual data, these models generate outputs that form the foundation of various creative AI applications.

Within the domain of image generation, models are broadly categorized into Text-to-Image (Text2Img) and Image-to-Image (Img2Img) frameworks~\cite{instruct_pix2pix_i2i, img2img_turbo_i2i}, although hybrid and specialized approaches also exist. Text2Img models generate entirely new images based on textual descriptions, effectively translating linguistic cues into visual representations. In contrast, Img2Img models modify or enhance existing images by leveraging an input image as a reference while applying stylistic or contextual transformations. This study primarily focuses on Text2Img models due to their ability to create images purely from descriptive text prompts, making them particularly suited for analysing artistic style recreation.

To systematically evaluate the artistic fidelity and limitations of state-of-the-art (SOTA) commercial generative models, we selected 12 diffusion-based models, which are among the most widely used and highly regarded in the field. These models were identified based on their popularity and performance, as detailed in the Introduction. The selection was motivated by three key considerations:
\begin{enumerate}
    \item Benchmarking Established Models: Using well-established models enables the creation of a high-quality AI-generated art dataset, which could serve as a valuable resource for future research.
    \item Avoiding Training and Fine-Tuning Biases: Training a model from scratch or fine-tuning an existing open-source model would not provide a fair assessment of the out-of-the-box capabilities of these models. Our goal was to evaluate their pre-trained performance rather than their adaptability to new training objectives.
    \item Computational Constraints: Training or fine-tuning diffusion models is highly resource-intensive. Proprietary models, in particular, are trained on vast datasets with ongoing refinements by dedicated research teams, making them the most suitable candidates for assessing the current peak capabilities of image generative AI.
\end{enumerate}

Initially, 15 diffusion models were considered. 
Each model was tested using three standardized prompts to evaluate its ability to generate visually coherent and stylistically accurate images. Five researchers independently assessed the outputs based on realism, artifact minimization, and adherence to the prompt. A model was discarded if all five unanimously agreed it failed to meet these criteria.
For example, DeepFloyd IF \cite{stabilityai2023} was among the initial 15 models considered but was excluded from further experimentation. Its generated outputs frequently failed to align with the described artistic movements, particularly struggling with facial features and even simple object shapes (e.g., dogs and other animals).

The final selection of 12 models used in our study is listed in the Introduction, with key specifications summarized in Table 1. It is important to note that many of these models are proprietary, and as a result, their architectural details and training methodologies remain undisclosed.

\begin{table}[h]
\label{tab:models}
\centering
\resizebox{\columnwidth}{!}{%
\begin{tabular}{|p{0.20\textwidth}|p{0.14\textwidth}|p{0.28\textwidth}|p{0.16\textwidth}|p{0.13\textwidth}|p{0.17\textwidth}|}
\hline
\textbf{Model}             & \textbf{Creator}    & \textbf{Architecture}   &
\textbf{Conditioning}   
& \textbf{Resolution (default)}                                        & \textbf{Configurable output shape} \\ \hline

\href{Ideogram 2.0}{https://ideogram.ai/login} 
    & Ideogram AI 
    & Diffusion based architecture, not fully disclosed.
    & text-to-image 
    & 1024x1024 
    & Yes \\ \hline

\href{Flux 1.1 Pro}{https://blackforestlabs.ai/1-1-pro/}              
    & Black Forest Labs          
    & 
    Rectified FLow Transformer \cite{peebles2023scalablediffusionmodelstransformers,esser2024scalingrectifiedflowtransformers}       
    & text-to-image                                                         & 2048x2048 
    & Yes \\ \hline
    
\href{Dall-E 3}{https://openai.com/index/dall-e-3/} (via ChatGPT-4o) 
    & \href{OpenAI}{https://openai.com/research/}              
    & Diffusion based architecture, not fully disclosed.
    & text-to-image                                              
    & 1024x1024                                                        
    & No  \\ \hline
    
\href{Firefly Image 3}{https://firefly.adobe.com/}           
    &  Adobe               
    &  Diffusion based architecture, not fully disclosed.\cite{Adobe}
    & text-to-image, image-to-image                                        
    & 2048x2048         
    &  Yes \\ \hline
    
\href{OmniGen}{https://github.com/VectorSpaceLab/OmniGen}
    & Beijing Academy 
    & Latent Diffusion Model \cite{omnigen_t2i}.
    & multimodal-to-image
    & 2048x2048 
    & Yes \\ \hline
\href{Leonardo Phoenix}{https://leonardo.ai}          
    & Leonardo Interactive Pty          
    & Diffusion based architecture, not fully disclosed.
    & test-to-image, image-to-image                                         & 1024x1024                                                         
    & Yes \\ \hline
\href{Midjourney V6.1}{https://www.midjourney.com/imagine}
    & Midjourney          
    & Diffusion + Transformeers framework, not fully disclosed.                    & text-to-image, image-to-image                                         & 1024x1024      
    & Yes \\ \hline
\href{Stable Diffusion 1.5}{https://huggingface.co/stable-diffusion-v1-5/stable-diffusion-v1-5}       
    & Stability AI (dismissed)         
    & Latent Diffusion Model (LDM) \cite{stable-diffusion}
    & text-to-image, image-to-image                                         & 512x512                                                          
    & Yes \\ \hline
\href{Stable Diffusion 3.5-large}{https://stability.ai/news/introducing-stable-diffusion-3-5} 
    & Stability AI      
    & Advanced LDM framework with CLIP and T5 text encoders  \cite{stable_diffusion_3_5_large_t2i}. 
    & text-to-image, image-to-image \cite{QK_normalization} 
    & 1024x1024 
    & Yes \\ \hline
\href{Flux.1 Schnell}{https://blackforestlabs.ai/announcing-black-forest-labs/}            
    & Black Forest Labs 
    & Fast version of Flux.1.1 \cite{flux_schnell}, trained using latent adversarial diffusion distillation
    \cite{diffusion_distillation}.
    & text-to-image 
    & 2048x2048 
    & Yes \\ \hline
\href{Kolors 1.5}{https://klingai.com/text-to-image/new}  
    & Kuaishou Kolors team - Kling AI 
    & Large-scale latent diffusion based model.\cite{kolors}
    & text-to-image, image-to-image
    & 1024x1024
    & Yes \\ \hline
\href{Auto-Aesthetics v1}{https://neural.love/blog/auto-aesthetics-v1-ai-art-revolution}  
    & Neural.love 
    & Not disclosed \cite{NeuralLoveAI_2024} & text-to-image
    & 1024x1024 
    & Yes \\ \hline
\end{tabular}%
}
\caption{Comparison between the used Models.}
\label{tab:generative_models_table}
\end{table}

% In addition to the previously mentioned models, we evaluated DeepFloydIF \cite{stabilityai2023}. However, this model did not produce convincing results. Its outputs struggled to accurately reflect the artistic movement described in the prompts and encountered difficulties in generating realistic paintings. Common issues included problems with faces and even simpler shapes, such as dogs and other animals. 

%A more detailed discussion of DeepFloydIF can be found in \cite{}. %citare una delle vostre tesi

\subsection{Evaluation}

Models are evaluated based on two distinct and orthogonal criteria, each addressing a crucial aspect of their performance.

\begin{itemize}
\item Authenticity. The first criterion evaluates the model's ability to generate samples that are sufficiently realistic and convincing, such that they could be mistaken for artifacts created by a human. This involves assessing the quality of the generated output in terms of visual coherence, attention to detail, and overall believability. A high score in this area indicates that the model produces outputs that closely mimic human creativity and craftsmanship.

\item Adherence to Prompt Instructions. The second criterion focuses on the model's capacity to accurately follow the detailed instructions specified in the prompt. This involves assessing how well the generated outputs align with the intended artistic style, thematic elements, or any specific requirements outlined. Success in this area demonstrates the model's ability to interpret and faithfully execute complex and nuanced instructions.
\end{itemize}

These two evaluation criteria are deliberately designed to be independent. While a model may excel in producing outputs faithfully mimicking human art crafts, it might still fail to accurately adhere to the stylistic constraints of the prompt, or vice versa. By assessing these dimensions separately, we aim to obtain a comprehensive understanding of the model's strengths and weaknesses across both realism and prompt alignment.

The way we addressed these criteria in our surveys will be 
described in Section~\ref{sec:surveys}.



\section{The AI-pastiche Dataset}
\label{sec:dataset}

% AI-pastiche comprises 953 entries, each representing a unique AI-generated artwork. Every entry is characterized by a set of attributes, including the generative model, subject, style, period, prompt, and generated image, which are described in detail below.

% It is important to note that metadata such as subject, style, and period correspond to the prompt’s intended description rather than the actual characteristics of the generated image.

% A brief overview of these attributes is provided below.

% \begin{itemize}
%     \item generative model. This refers to the model used to generate the image. The list of models was provided in Section \ref{sec:models}.
%     \item subject. A list of tags describing the content based on the prompt. There are approximately 50 different tags, including categories such as ``landscape", ``animals", ``trees", and more. Additionally, some tags represent color schemes or tonal impressions, such as ``gold", ``soft tones", and ``vibrant tones". The number of tags associated to each entry is variable, and encoded as a comma separated list.
%     \item style. A synthetic label describing the style. Current entries comprise: renaissance, baroque, rococo, classicism, romanticism, realism, satirical, impressionsim, art nouveau, naive, expressionism, futurism, cubism, dadaism, fauvism, abstractionism, 
%     symbolism and surrealism. 
%     \item period. An indication of the century to which the desired painting should belong, as specified in the prompt.
%     \item prompt. The complete prompt provided to the generators. The methodology used to define the prompts, along with a few examples, was discussed in Section \ref{sec:methodology}.
%     \item generated image. The identificator of the image.    
% \end{itemize}

AI-Pastiche is a carefully curated dataset comprising 953 AI-generated paintings in famous artistic styles. These images were produced using manually crafted text prompts and include comprehensive metadata describing their generation details. The dataset was created using 73 carefully crafted prompts, with over 20 images generated per prompt across the selected generative models (Section \ref{sec:models}). From this pool, the highest-quality images were manually selected, ensuring fidelity to artistic styles and overall visual appeal.

\subsection{Dataset Objectives}
The two primary purposes of the dataset are:

\begin{enumerate}
\item Analyzing the capabilities and limitations of SOTA generative models in accurately recreating well-known painting styles.
\item Providing a high-quality AI-generated painting dataset for the research community, facilitating future studies on generative AI in artistic domains.
\end{enumerate}

While the current dataset consists of 953 carefully selected images, we plan to expand it in future iterations, incorporating additional artistic styles and more diverse prompts to further evaluate model performance and limitations.

\subsection{Metadata and Composition}
\label{sec:metadata}
The AI-Pastiche dataset includes detailed metadata for each generated painting, summarized in Table 2. It is important to note that attributes such as subject, style, and period correspond to the intended description in the prompt rather than a direct analysis of the generated image itself.

At present, the dataset exhibits some stylistic imbalances, particularly in terms of artistic periods and movements. As shown in Table 3, the majority of paintings emulate XIX-th and the XX-th century styles, with Renaissance, Impressionism, Romanticism, and Baroque being the most represented artistic movements. In future expansions, we aim to mitigate these imbalances by incorporating a broader range of historical styles and more diverse prompts.

Upon publication of this study, the AI-Pastiche dataset will be made publicly available on Kaggle to facilitate open research and further exploration of generative AI in artistic applications.

\begin{table}[h]
\label{tab:metadata}
    \centering
    \renewcommand{\arraystretch}{1.3} % Adjust row height for better readability
    \begin{tabularx}{\textwidth}{|>{\centering\arraybackslash}m{3.5cm}|X|}
        \hline
        \textbf{Attribute} & \textbf{Description} \\ 
        \hline
        \textbf{Generative Model} & The model used to generate the image. The list of models is provided in Section \ref{sec:models}. \\ 
        \hline
        \textbf{Subject} & A list of tags describing the image content based on the prompt. There are approximately 50 different tags, including categories such as “landscape,” “animals,” and “trees.” Some tags also represent colour schemes or tonal impressions, such as “gold,” “soft tones,” and “vibrant tones.” Tags are stored as a comma-separated list. \\ 
        \hline
        \textbf{Style} & A synthetic label describing the artistic style of the image. Styles include Renaissance, Baroque, Rococo, Classicism, Romanticism, Realism, Satirical, Impressionism, Art Nouveau, Naïve, Expressionism, Futurism, Cubism, Dadaism, Fauvism, Abstractionism, Symbolism, and Surrealism. \\ 
        \hline
        \textbf{Period} & The historical period or century to which the intended painting style belongs, as specified in the prompt (e.g., 18th century, Renaissance). \\ 
        \hline
        \textbf{Prompt} & The full text prompt used to generate the image. \\ 
        \hline
        \textbf{Generated Image} & The identifier of the image. \\ 
        \hline
    \end{tabularx}
    \caption{Description of the attributes used in the dataset.}
    \label{tab:attributes}
\end{table}

\begin{table}[h]
    \centering
    \renewcommand{\arraystretch}{1.2} % Improve readability
    \makebox[\textwidth]{ % Forces the first two minipages to stay in the same row
     \begin{minipage}{0.45\textwidth}
            \centering
            \small
        \begin{tabular}{lcc}
            \toprule
            \textbf{Period} & \textbf{Total} & \textbf{\%} \\
            \midrule
            XX century & 307 & 32.2 \\
            XIX century & 289 & 30.3 \\
            XVI century & 153 & 16.1 \\
            XVII century & 117 & 12.3 \\
            XV century & 49 & 5.1 \\	
            XVIII century & 38 & 4 \\	
            \bottomrule
        \end{tabular}
        \caption{AI-pastiche statistics by Period.}
        \label{tab:composition_by_period}
        \end{minipage}
       \hfill
        \begin{minipage}{0.45\textwidth}
            \centering
            \small
            \begin{tabular}{lcc}
                \toprule
                \textbf{Style} & \textbf{Total} & \textbf{\%} \\
                \midrule
                Renaissance & 202 & 21.2 \\
                Impressionism & 136 & 14.3 \\
                Romanticism & 92 & 9.7 \\
                Baroque & 86 & 9.0 \\
                Realism & 60 & 6.3 \\
                Surrealism & 50 & 5.2 \\
                Dadaism & 44 & 4.6 \\
                % Symbolism & 42 & 4.4 \\
                %Classicism & 39 & 4.1 \\
                \bottomrule
            \end{tabular}
            \caption{Most represented Styles.}
        \end{minipage}
        }
\end{table}

\iffalse
        \begin{table}[h]
           \centering
            \small 
            \begin{tabular}{lcc}
                \toprule
                \textbf{Subject Tag} & \textbf{Total} & \textbf{\%} \\
                \midrule
                soft\textunderscore tones & 375 & 9.0 \\
                vibrant\textunderscore tones & 297 & 7.1 \\
                landscape & 292 & 7.0 \\
                persons & 283 & 6.8 \\
                brushstrokes & 231 & 5.5 \\
                person & 230 & 5.5 \\
                water & 204 & 4.9 \\
                dramatic\textunderscore tones & 188 & 4.5 \\
                animals & 165 & 4.0 \\
                portrait & 162 & 3.9 \\
                \bottomrule
            \end{tabular}
            \caption{Dataset statistics by most represented Subject Tag.}
        \end{table}
\fi

\section{The surveys}
\label{sec:surveys}
In order to evaluate the performance of the models along the criteria discussed in the methodology section \ref{sec:methodology}. we implemented and collected data from two distinct surveys.

\subsection{Authenticity}
With authenticity, we refer to the extent to which a model generates outputs that convincingly resemble human-made creations.

The evaluation was conducted using a survey-based approach, where participants were asked to classify images as either AI-generated or human-made. For the human-made paintings, we used a subset of open-access images from the \href{National_Gallery_of_Art}{https://www.nga.gov/open-access-images.html} in Washington. Participants were shown a set of 20 images, one at a time and in sequence, comprising a random mix of genuine and AI-generated works, and were asked to classify each image individually.

To ensure unbiased and reliable responses, the survey presented the images in randomized order, without any metadata or contextual information that could hint at their origin. This design encouraged participants to base their judgments solely on the visual and stylistic qualities of the images.

The survey was conducted anonymously. For privacy reasons, no personal information was collected. However, given the focus on European painting, we recognized that cultural background might influence participants' perceptions. To assess this, we asked participants whether they identified with a European cultural background, with the option to decline to answer.

The survey reached approximately 600 participants, selected from a diverse pool to capture a wide range of perspectives. Most of the  participants were students and colleagues, suggesting a relatively high level of education and some familiarity with artistic aesthetics, though typically without formal training in art critique. This selection was intentional, as it reflected the anticipated audience for AI-generated art in real-world scenarios. The study aimed to evaluate the perceptual authenticity as it might be experienced by the general public.

The survey is still accessible at the following page: \href{AI-pastiche_Survey}{https://script.google.com/macros/s/AKfycbzEnn5jTRW5sU7U98h8nyz6ufyk0uJjhAeOhe2LlbFk7PfpKVmtUuHFKVyyLwAOwIwl/exec}.

 \subsubsection{Adherence to Prompt Instructions}
The purpose of this evaluation is to assess each generated image based on its alignment with the requirements specified in the given prompt.

This classification task is significantly more complex than the previous one, as it requires a careful reading and thorough understanding of the prompt, as well as a comparative evaluation of outputs from different models. 
%While the previous survey was engaging and relatively straightforward, this evaluation was much more problematic tedious, especially for individuals with no specific interest in the research. 
For this reason, we decided to limit participation to a selected number
of members, comprising people of our research group, colleagues of
the department of fine arts, and some of their students. While the
collective number of participants was sensibly smaller than for the first survey, each person
evaluated multiple prompts, resulting in 5706 entries with an average of about 475 assessments for each model.

We also considered the possibility of conducting a fully automated evaluation using techniques like CLIP \cite{CLIP} or similar models. However, the ability of such embeddings to accurately capture nuanced factors—such as artistic style, historical period, or other subtle attributes—remains uncertain. The data collected through this survey can also serve to clarify this issue.
This relates to our broader research program focused on investigating the aesthetic capabilities of Large Language Models, and we plan to address CLIP, among other models, in our future research.

Our evaluation metric is based on the subjective assessment of how well each image reflects the requirements of the prompt. While it is theoretically possible to rank the generated images along a continuous scale, the inherent complexity of the task and the subjective nature of the evaluations led us to simplify the process. Instead, images are categorized into three broad classes: low, medium, and high alignment with the prompt.

These classifications —low, medium, and high— are not absolute or universal but are defined relative to the specific set of images generated for each prompt. This relative approach ensures that the evaluation accounts for the context and inherent variability within each batch of images.



%\begin{figure}[H] % [h] = posizione "here" (vicino al testo)
%    \centering
%    \includegraphics[width=1\textwidth]{images/histogramsurvey2.png} % Cambia la dimensione dell'immagine
%    \caption{Descrizione dell'immagine}
%    \label{fig:miafigura} % Per riferimenti incrociati
%\end{figure}

\section{Results}
\label{sec:results} 
In this section, we report and analyze the results of our survey. It is important to emphasize that our goal is not to compare the performance of different models, but
rather to provide a clearer understanding of the current state of the field. Our focus is on identifying the persistent challenges faced by generative models, highlighting specific problem areas, and discussing potential directions for improvement. By examining these limitations, we aim to contribute to the broader discourse on how these models can be refined and enhanced for more reliable and aesthetically convincing outputs.

\subsection{Authenticity}
In Figure \ref{fig:confusion}, we show the confusion matrix
resulting from the survey: overall, around 28\% of AI-generated images were mistakenly attributed to Humans.
Interestingly enough, a slighly lower but still relevant number of Human-generated images were attributed to AI: in this case, the misclassification percentage is around 20\%.

\begin{figure}[h]
    \centering
    \includegraphics[width=0.5\linewidth]{images/confusion.png}
    \caption{Confusion matrix}
    \label{fig:confusion}
\end{figure}


\begin{figure}[h]
    \centering
    \includegraphics[width=0.9\linewidth]{images/frequency_misclassification_on_images.png}
    \caption{Distribution of Misclassification Percentages on AI-Generated Images}
    \label{fig:distribution_misclassification_on_images}
\end{figure}
Figure \ref{fig:distribution_misclassification_on_images} illustrates the frequency distribution of misclassification percentages for AI-generated images in the dataset. The distribution is skewed toward lower misclassification percentages, with a small subset of images achieving a perfect authenticity score.

Since it is not in the purpose of this work to make a
ranking of models, but merely to understand the overall
state-of-the-art, we only provide a syntetic evaluation
for the six best models, as evinced from  our survey. 
The results are summarized in Table \ref{tab:authenticity}.


\begin{table}[h]
    \centering
    \begin{tabular}{|c|c|c|c|}\hline
         {\bf model} & {\bf total count} & {\bf misclassified} & {\bf ratio}\\\hline
        Ideogram & 532  & 263  & 0.49\\\hline
        Midjourney  & 598  & 257 & 0.43 \\\hline
        Stable-diffusion-3.5-large & 572 & 204 & 0.36 \\\hline
        Stable Diffusion 1.5 & 600 & 195 & 0.33 \\\hline 
        OmniGen & 642 & 206 & 0.32 \\\hline
        Dall$\cdot$E 3 & 562 & 169 & 0.30 \\\hline
        %Flux 1.1 Pro & 514 & 137 & 0.27 \\\hline
        %Leonardo Phoenix & 589 & 151 & 0.26\\\hline
        %Firefly image 3  & 612  & 137 & 0.22 \\\hline
        %Kolors by KlingAI  & 549 & 103 & 0.19 \\\hline        FLUX.1-schnell  & 628 & 104  & 0.17 \\\hline
        %Auto-Aesthetics V1 & 455 & 47 & 0.10 \\\hline\hline
        {\bf Total} & 6868  & 1980 & 0.29 \\\hline
    \end{tabular}
    \caption{Performance of Models in terms of the perceived 
    autheticity. We only list the six most perfomant models, according to our survey. The total refer to all models.}
    \label{tab:authenticity}
\end{table}
The best performing model appears to be Ideogram, achieving an impressive authenticity rate close to 50\%. It is also noteworthy that relatively older models, such as Stable Diffusion 1.5 and Omnigen, perform comparatively well against more recent competitors. As we will see from the results of the second survey (see Section \ref{sec:adherence}), this is partly due to these models adopting a more liberal interpretation of the prompt, often sacrificing strict prompt adherence in favor of aesthetic quality.

Some examples of AI-generated artifacts of different styles and periods among the most frequently classified as human-made according to our survey are shown in Figure \ref{fig:most_convincing}.

\begin{figure}[h]
    \centering
    {\footnotesize
    \begin{tabular}{ccc}
         \includegraphics[height=0.28\linewidth]{images/0931.png} & 
          \includegraphics[height=0.28\linewidth]{images/0222.png} &  
          \includegraphics[height=0.28\linewidth]{images/0825.jpg}
          \\
          (a) Ideogram & (b) Stable-diffusion-3.5-large & (c) Midjourney\\
          \includegraphics[height=0.28\linewidth]{images/0206.jpg} 
          &
          \includegraphics[height=0.28\linewidth]{images/0469.png} &  \includegraphics[height=0.28\linewidth]{images/0846.png} 
          \\
          (e) Firefly & (f) Stable-diffusion-1.5 & (g) Ideogram  
    \end{tabular}
    }
    \caption{Examples of convincing AI-generated examples of different Styles and Periods, according to the results of our survey.}
    \label{fig:most_convincing}
\end{figure}

A per-period investigation (see Table \ref{tab:authenticity_per_period}) show that, not surprisingly, generative
models perform particularly well in mimicking art of the
last century, and (some styles) of the XIX century. 
They clearly seem to be in much more trouble in producing convincing artifacts of previous periods.

\begin{table}[h]
    \centering
    \begin{tabular}{|c|c|c|c|}\hline
         {\bf period} & {\bf total count} & {\bf misclassified} & {\bf ratio}\\\hline
XX century     &     2095    &   697  & 0.33\\\hline
XIX century    & 1955    & 541    & 0.28\\\hline
XVII century   & 786     & 208    & 0.26 \\\hline
XV century     & 331     & 84     & 0.25 \\\hline
XVI century    & 1083 &  265 & 0.24 \\\hline
XVIII century   & 261 & 57 & 0.22 \\\hline
    \end{tabular}
    \caption{Caption}
    \label{tab:authenticity_per_period}
\end{table}

Analyzing results according to artistic styles is complicated by the current underrepresentation of certain movements. For instance, as reported in Table \ref{tab:authenticity_per_style}, the style that the models have been most comfortable with is `Art Nouveau'. 

\begin{table}[h]
    \centering
    \begin{tabular}{|c|c|c|c|}\hline
         {\bf period} & {\bf total count} & {\bf misclassified} & {\bf ratio}\\\hline
art nouveau & 104  & 49 & 0.47 \\\hline
cubism  & 232 & 92 & 0.40 \\\hline
satirical & 74 & 29 & 0.39 \\\hline
impressionism & 922 & 350 & 0.38 \\\hline
dadaism & 320 & 118 & 0.37 \\\hline
futurism & 114 & 42 & 0.37 \\\hline
classicism & 273 & 99 & 0.36 \\\hline
fauvism & 119 & 40 & 0.34 \\\hline
expressionism  & 170  & 57 & 0.34 \\\hline
symbolism  & 302  & 98 & 0.32 \\\hline
vedutism  & 92  & 26 & 0.28 \\\hline
renaissance  & 1458  & 355  & 0.24 \\\hline
romanticism  & 635  & 154  & 0.24 \\\hline
abstractionism  & 91 & 20 & 0.22 \\\hline
baroque & 574 & 123  & 0.21  \\\hline
realism & 402 & 85   & 0.21 \\\hline
surrealism & 334 & 70 & 0.21 \\\hline
rococo & 157 & 30 & 0.20 \\\hline
naive & 201 & 33 & 0.16 \\\hline
    \end{tabular}
    \caption{Caption}
    \label{tab:authenticity_per_style}
\end{table}

However, we have only a single prompt associated with this label, depicting a pencil sketch of a seated man in a pensive attitude. A few examples are shown in Figure \ref{fig:art_nouveau}. Due to the schematic simplicity of both the subject and the technique, it is not surprising that many of the AI-generated artifacts have been mistakenly perceived as human-made.

\begin{figure}[h]
    \centering
    {\footnotesize
    \begin{tabular}{ccc}
         \includegraphics[height=0.3\linewidth]{images/0060.png} & 
          \includegraphics[height=0.3\linewidth]{images/0596.jpg} &  \includegraphics[height=0.3\linewidth]{images/0527.jpg} \\
          (a) Stable-diffusion-3.5-large & (b) OmniGen & (c) Midjourney 
    \end{tabular}
    }
    \caption{``Art nouveau" examples. The sketchy nature of the subject
    specified by the prompt adapted particularly well to the capacities of generative models.}
    \label{fig:art_nouveau}
\end{figure}
A similar problem arises with the ``satirical" style. Again, we
have only one prompt relative to this category, referring to a 
caricature of Otto Von Bismark in the sytle of the satirical magazine ``La Lune", of the end of the XIX century. Many models created convincing 
artifacts, as illustrated in Figure \ref{fig:satirical}.
\begin{figure}[h]
    \centering
    {\footnotesize
    \begin{tabular}{ccc}
         \includegraphics[height=0.3\linewidth]{images/bismark_gpt.png} & 
          \includegraphics[height=0.3\linewidth]{images/Bismark_ideogram.png} &  \includegraphics[height=0.3\linewidth]{images/Bismark_midjourney.png} \\
          (a) Dalle$\cdot$E & (b) Ideogram & (c) Midjourney 
    \end{tabular}
    }
    \caption{Satirical examples.}
    \label{fig:satirical}
\end{figure}

Apart from these cases, generative models appear to be more adept at imitating modern artistic styles, such as Impressionism, Cubism, Dadaism, Futurism, and similar movements. These styles often emphasize abstraction, bold shapes, and expressive brushwork, which align well with the strengths of generative models.

Conversely, models face greater challenges when attempting to replicate older artistic styles, such as Renaissance, Baroque, and Rococo. These styles are characterized by intricate details, realistic depictions, and complex compositions, which demand a level of precision and 
semantic interpretation that many models struggle to achieve.

Interestingly, the worst performance is observed when models attempt to imitate naïve art. One key reason for this difficulty is the challenge most models face in handling the ``flat" perspective typical of this style, as discussed in Section \ref{sec:landscapes}. Unlike classical or modern styles, naïve art often employs a lack of depth, disproportionate figures, and an intuitive rather than rule-based approach to composition. This contradicts the implicit biases of generative models, which are often trained to prioritize realism, shading, and perspective consistency.


\subsubsection{Distinction of results for cultural background}
As mentioned in the introduction, we asked participants to disclose their cultural background to assess its potential impact on the perception of European paintings.

In this regard, the collected data are highly unbalanced, with European participants outnumbering non-European participants by approximately six to one. As a result, any analysis of this factor must be approached with caution, as the sample distribution may limit the reliability of our findings.

The only interesting result is relative to 
the misclassification rate for different historical
periods, shown in Figure~\ref{fig:barplotperiods}.

%Ometteri questa figura, che mi sembra poco informatica
%\begin{figure}[H] % [h] = posizione "here" (vicino al testo)
%    \centering
%    \includegraphics[width=1\textwidth]{images/model.png} % Cambia la dimensione dell'immagine
%    \caption{Misclassification rate between models.}
%    \label{fig:miafigura} % Per riferimenti incrociati
%\end{figure}

\begin{figure}[h] % [h] = posizione "here" (vicino al testo)
    \centering
    \includegraphics[width=1\textwidth]{images/period.png} % Cambia la dimensione dell'immagine
    \caption{Misclassification rates by periods.}
    \label{fig:barplotperiods} % Per riferimenti incrociati
\end{figure}

Not surprisingly, non-European participants tend to misclassify images from the 15th, 16th, and 17th centuries more frequently, likely due to a lower level of familiarity with the artistic movements of those periods. European art from these centuries is deeply rooted in specific cultural and historical contexts, with stylistic conventions that may not be as immediately recognizable to those who have not been extensively exposed to them. 

A similar analysis across different artistic styles did not reveal any additional trends significant enough to report. 

\subsubsection{Influence of the subject}
\label{sec:subject_influence}
Our final investigation examines the influence of subject matter on the model's ability to generate artifacts that can be mistaken for human-made creations. For this analysis, we use the tags described in Section~\ref{sec:metadata}. Specifically, each prompt is represented as a multilabel binarization over its associated set of tags. We then perform a linear regression to predict the average degree of ``authenticity", as determined by the survey, for all entries associated with those tags. The analysis is restricted to tags occurring in at least two different prompts.

Naturally, we do not expect to obtain a highly accurate estimation, as additional — such as the required style and historical period — also play a role. However, our focus is not on the predicted output itself but rather on the weights assigned by the model to different tags, particularly negative tags, which may indicate categories that present challenges for the models.

After normalizing the output using Gaussian normalization, we obtain a prediction error of approximately 0.4 (compared to the unit standard deviation). As expected, the prediction accuracy is not particularly high, but it is sufficient to demonstrate a correlation between tags and perceived authenticity. 

In Figure~\ref{fig:tag-weights} we show the weights associated with the different tags. We do the investigation for all models (blue) and for a restricted subset of models comprising Ideogram, Midjourney, Stable-Diffusion-3.5-large and Dall$\cdot$E, obtaining high scores both in authenticity and prompt adherence.

\begin{figure}
    \centering
    \includegraphics[width=1.\linewidth]{images/auhtenticity_by_tags.png}
    \caption{Impact of the tags on the perceived authenticity of the AI-generate artifact, estimated through linear regression. The analysis is restricted to tags occurring in at least two different prompts. We perform the investigation for all models (blue) and for 
    a restricted subset of models with higher authenticity 
    scores (red).}
    \label{fig:tag-weights}
\end{figure}

%Some negative tags, such as "machinery," "mixed\_media," "mythology," and "moon," appear in only a single prompt, making their statistical significance marginal. 
Looking at the negative scores, a notable group is composed
by tags related to humans: ``crowd", ``person", ``persons", ``child", and ``portrait". 
This provides strong evidence that generative models still struggle to represent humans convincingly when mimicking artistic painting. In addition, portraits of women tend to present more challenges compared to those of men.

This difficulty may arise from several, sometimes contrasting, factors. For example, generative models may fail to achieve realism in highly complex and dynamic scenes involving multiple people or crowds, while at the same time, they may adopt an exaggerated hyperrealism in portraits. We discuss these issues in more detail in Section \ref{sec:discussion}.

From the naturalistic point of view, ``clouds", ``flowers" and ``water" seem to have a negative impact. The tag ``flower" contrasts with ``still\_life", which, by comparison, has a significantly more positive score. In our dataset, the negative perception associated with flowers seems to be primarily linked to paintings in naïf style — one of the styles where generative models, as observed in the previous section, tend to perform the worst. 
The negative scores for ``clouds" and ``water" appear to stem from the inherent complexity of rendering these elements in a way that aligns with the stylistic and historical constraints specified in the prompt.  It is 
also interesting to observe that while the ``best" models seem to be able to cope with water in an acceptable way, their performance on ``clouds" is even worse than average.
We shall discuss this subject in more detail, in Section \ref{sec:landscapes},
where we shall also provide a few examples.

Other tags related to nature, such as ``fog", ``snow", and ``trees", do not appear to pose significant challenges for generative models. These elements are often rendered convincingly, likely due to their relatively uniform structures and the abundance of high-quality reference images available in training datasets. However, the situation changes when considering specific moments of the day. Night scenes can easily suffer from inconsistencies in lighting and contrast or from the hyperrealistic rendering of specific elements, such as the moon. More notably, dawn and sunset present particular difficulties, as generative models often struggle to capture the complex interplay of warm and cool tones, the gradual transitions in atmospheric lighting, and the way natural and artificial light sources interact during these times. These shortcomings can lead to unnatural gradients, misplaced highlights, or an overall loss of realism, making these scenarios more challenging than other elements related to nature.

The explicit request in the prompt to add visible brushstrokes frequently increases the perception of authenticity. In addition, 
models generally perform better when prompted to adopt soft, muted tones rather than vibrant or dramatic color schemes. When working with softer tones, the model is more likely to produce balanced, harmonious compositions that align well with a wide range of artistic styles. In contrast, when tasked with generating highly saturated or dramatic lighting effects, models often tend to over-interpret the request, leading to exaggerated contrasts, unnatural color blending, or an overuse of artificial-looking highlights and shadows.

Finally, models appear to struggle with subjects related to mythology and religion, due to a combination of the inherent complexity of these themes and content moderation filters that may constrain or influence their performance.



\subsection{Adherence to Prompt Instructions}
\label{sec:adherence}
This survey measured user satisfaction based on the alignment of generated images with the requirements specified in the given prompt, considering both content and style. Users rated their evaluations in three categories: ``Good", ``Medium", and ``Low." The evaluation was not intended to be absolute, but rather comparative, assessing how each model's output performed relative to others.

For instance, if a particular image was unanimously classified as "Good," this does not necessarily imply that it was a highly satisfactory interpretation of the prompt. Rather, it simply indicates that, in the collective judgment of the reviewers, it outperformed the outputs of other models.

Reviewers were encouraged to give a balanced repartition in the three categories,
to reduce the impact of the prompt complexity, and inherent variability within each batch of images.

To derive a summary score, we computed a weighted average, assigning a value of 1 to ``Good", 0 to ``Medium", and -1 to ``Low". 

The results are summarized in Figure \ref{fig:plot_survey2} and Table \ref{tab:adherence_scores}. Again, in the Table we only list the most performant models, according to out investigation.

\begin{figure}[h] % [h] = posizione "here" (vicino al testo)
    \centering
    \includegraphics[width=1\textwidth]{images/plot_survey2.png} % Cambia la dimensione dell'immagine %histograms
    \caption{Descrizione dell'immagine}
    \label{fig:plot_survey2} % Per riferimenti incrociati
\end{figure}

\begin{table}[h]
    \centering
    \begin{tabular}{c|c}
    {\bf model} & {\bf average score}\\\hline\hline
  Leonardo Phoenix & 0.37 \\\hline
  Dall$\cdot$E 3 & 0.36\\\hline
  Midjourney & 0.32\\\hline
  Ideogram & 0.29 \\\hline
  stable-diffusion-3.5-large & 0.22 \\\hline
  Flux 1.1 Pro & 0.08 \\\hline
  %Kolors by KlingAI & -0.19\\\hline
  %OmniGen & -0.27 \\\hline
  %Firefly image 3 &  -0.29 \\\hline
  %FLUX.1-schnell & -0.30  \\\hline
  %Stable Diffusion 1.5 & -0.45 \\\hline
  %Auto-Aesthetics V1 & -0.62 \\\hline
    \end{tabular}
    \caption{Average ``adherence" score. The score was computed as a weighted average of the survey results, associating value 1 to ``Good", 0 to ``Medium" and -1 to ``Low". We only list models with a better than average
    behaviour, according to our investigation.}
    \label{tab:adherence_scores}
\end{table}

It is worth noting that prompt adherence leads to a substantially different ranking compared to authenticity scores, which were evaluated without knowledge of the corresponding prompts. A model that was instructed to generate a Renaissance painting but instead produced a convincing Cubist artwork would likely receive a high authenticity score, despite failing to follow the intended artistic style.

This suggests that some models prioritize aesthetic quality over strict prompt adherence, opting for visually compelling outputs even at the expense of accuracy. This tendency is particularly evident in early-generation generative models, such as Stable Diffusion 1.5 and Omnigen, which frequently take creative liberties with prompt instructions.

Despite their loose interpretation of prompts and their occasional introduction of artifacts and distortions, these models remain among the most creative and surprising in our tests. Their ability to produce unexpected yet visually engaging results highlights a trade-off in generative AI: while newer models may achieve higher precision in style replication, earlier models often exhibit a greater degree of unpredictability and artistic exploration, which can sometimes lead to unexpectedly compelling outputs.


\section{Critical aspects of artificial generation}
\label{sec:discussion}
In this section, we highlight some of the most common and critical challenges observed in the generative models under consideration. These insights stem both from our direct experience in dataset creation and from the results of our surveys.

We structure the discussion around three major problem areas: Artifacting and Distortion (Section \ref{sec:artifacts}), Hyperrealism (Section \ref{sec:hyperrealism})
%Texture, Light, and Material Inconsistencies (Section \ref{sec:texture}) 
and Anachronisms (Section \ref{sec:anachronisms}).

\subsection{Artifacting and Distorsion}
\label{sec:artifacts}
One of the most evident problems is artifacting and distortion, where models fail to maintain anatomical coherence or structural integrity.
A few major instances are discussed below.

\subsubsection{Fingers, Hands and Limbs}
Correctly rendering hands and fingers remains one of the major challenges in generative image synthesis. 
The problem is common to most of the models: some examples are given in  Figure~\ref{fig:fingers}

\begin{figure}[h]
    \centering
    {\footnotesize
    \begin{tabular}{cccc}
          \includegraphics[height=0.16\linewidth]{images/leonardo_mani.png} & \includegraphics[height=0.16\linewidth]{images/mani_ideogram.png} &
          \includegraphics[height=0.16\linewidth]{images/flux1_1.png} &
          \includegraphics[height=0.16\linewidth]{images/dalli_mani2.png} \\
          Leonardo Phoenix & Ideogram 
         & Flux1-Schnell & Dall$\cdot$E 3 \\
            \includegraphics[height=0.16\linewidth]
          {images/dalli_mani.png} &
          \includegraphics[height=0.16\linewidth]{images/piedi_0258.png} &
          \includegraphics[height=0.16\linewidth]{images/piedi_ideogram.png} &
          \includegraphics[height=0.16\linewidth]{images/extra_foot.png} 
           \\
          Dall$\cdot$E 3 & Leonardo Phoenix & Ideogram &  Auto-Aesthetics v1  \\
           \includegraphics[height=0.16\linewidth]{images/flux1_2} &
          \includegraphics[height=0.16\linewidth]{images/mani_0012.png} &
          \includegraphics[height=0.16\linewidth]{images/exta_hand_0888.png} &
           \includegraphics[height=0.16\linewidth]{images/extra_mano.png} 
           \\
          Flux1-Schnell & Leonardo Phoenix & Auto-Aesthetics v1  & Auto-Aesthetics v1 
    \end{tabular}
    }
    \caption{Problems with fingers in hands and foots}
    \label{fig:fingers}
\end{figure}
The problem is well known and stems from several factors. The primary challenge is that hands frequently interact with objects or other parts of the body, leading to complex occlusions and overlapping regions. This poses difficulties both during training, where the model must abstract hands and fingers from their specific context, and during generation, where the model must realistically render them within the context of these interactions.

The problem is not limited to human figures. Animals are frequently depicted with an innatural number of legs, heads, or similar distorsions.

\begin{figure}[h]
    \centering
    {\footnotesize
    \begin{tabular}{cccc}
          \includegraphics[height=0.16\linewidth]{images/firefly_beast.png} &
          \includegraphics[height=0.16\linewidth]{images/cani_leonardo.png} &
          \includegraphics[height=0.16\linewidth]{images/three_legs_flux.png} &
          \includegraphics[height=0.16\linewidth]{images/0561.png} \\
          (a) Firefly & (b) Leonardo Phoenix & 
          (c) Flux1-Schnell  & (d) Stable Diff.1.5
    \end{tabular}
    }
    \caption{Caption}
    \label{fig:legs}
\end{figure}

\subsubsection{Distorsions in complex scenarios}
Distortions become more pronounced in complex scenarios, such as groups of people, highly dynamic scenes, or intricate architectural compositions. In these cases, maintaining a natural balance between stylistic accuracy and structural coherence remains a challenging task for most models.
A few typical examples are shown in Figure~\ref{fig:dynamic}, but nearly all models struggled with these specific prompts: (a) a traditional rural festival in the 19th-century realism style, (b) a music lesson in the Rococo style, and (c) a battle between knights in the early Renaissance style.

\begin{figure}[h]
    \centering
    {\footnotesize
    \begin{tabular}{ccc} 
          \includegraphics[height=0.24\linewidth]{images/0634.jpg} &
          \includegraphics[height=0.24\linewidth]{images/0811.png} &
          \includegraphics[height=0.24\linewidth]{images/0605.png} \\
          (a) Firefly & (b) Stable Diffusion 1.5 & (c) Omnigen
    \end{tabular}
    }
    \caption{Caption}
    \label{fig:dynamic}
\end{figure}
One frustrating limitation of current generative models is their inability to dynamically adjust generation time based on the complexity of the task. Unlike human artists, who naturally dedicate more time to intricate compositions while completing simpler ones more quickly, these models follow a fixed computational budget, regardless of the difficulty of the image being generated.

For instance, diffusion-based models operate within a predefined number of denoising steps, meaning they do not inherently ``realize" when an image requires additional refinement to resolve ambiguities in structure, perspective, or stylistic details. Whether generating a minimalist still life or a highly detailed historical battle scene, the model performs the same number of steps, often leading to overprocessing in simple cases and underdeveloped details in complex ones.


%Firefly su festa paesana 0634.jpg\\
%Stable diffusion 1.5 sulla lezione di musica 0811.png \\
%Omnigen cavalieri medievali 0605.webp\\
%autoaestethic persone sulla spiaggia 0753.jpg


\subsection{Hyperrealism}
\label{sec:hyperrealism}
Most generative models, often optimized for photorealism, struggle to reproduce the unique nuances and distinctive qualities characteristic of artistic styles from the past. We shall discuss the issues in 
three paradigmatic cases: portraits, still lifes, and landscapes.

\subsubsection{Portraits}
Modern generative models often exhibit a hyperrealistic tendency when replicating facial details, often in contrast with 
the historical artistic style they were supposed to mimic according to
the prompt. This excessive sharpness and detail can create a fundamental mismatch between the expected stylistic conventions and the generated output, leading to images that feel anachronistic or unconvincing. A few examples are given in 
Figure~\ref{fig:hyperrealism_faces}

\begin{figure}[h]
    \centering
    {\footnotesize
    \begin{tabular}{ccc}
          \includegraphics[height=0.29\linewidth]{images/hyper_0748.jpg} &  \includegraphics[height=0.29\linewidth]{images/hyper_0299.png} & 
          \includegraphics[height=0.29\linewidth]{images/hyper_528.png}\\  (a) Auto-Aesthetics v1 & (b) Midjourney & (c) Kolors by KlingAI \\
          \includegraphics[height=0.29\linewidth]{images/hyper_0364.png} &  \includegraphics[height=0.29\linewidth]{images/hyper_0762.png} & 
          \includegraphics[height=0.29\linewidth]{images/hyper_0531.png}\\  
          (d) Flux 1 - Schnell & (e) Flux 1.1 - Pro & (f) Kolors by KlingAI
    \end{tabular}
    }
    \caption{Hyperrealism examples. The required styles were: (a) baroque, (b) baroque, (c) romanticism, (d) XIXth Century realism,  (e) impressionism, (f) romanticism.}
    \label{fig:hyperrealism_faces}
\end{figure}

The problem becomes even more apparent when considering historical limitations in artistic materials and techniques. Painters working with oil or tempera could not achieve the pore-level skin textures or ultra-sharp reflections that modern models tend to generate by default. When a generative model introduces such hyperrealistic details into a 
Rinascimental painting or a 17th-century Dutch portrait, the output no longer aligns with the stylistic expectations of that period.

\subsubsection{Still lifes}
Another common subject where generative models struggle to restrain their tendency toward excessive realism is still life painting. While still lifes often contain highly detailed depictions of objects, traditional artistic styles—especially those from historical periods—frequently employ soft lighting, controlled textures, and a painterly touch that distinguishes them from hyperrealistic renderings.

Generative models, however, tend to overemphasize surface details, reflections, and textures, producing results that lean toward photographic realism rather than adhering to the stylistic characteristics of classical still life compositions. This issue becomes particularly noticeable in flower arrangements, fruit compositions, and table settings, where the AI-generated images may include overly sharp edges, unnatural glossiness, or exaggerated depth-of-field effects that are inconsistent with traditional oil painting techniques. 

A few typical examples are shown in Figure~\ref{fig:hyperrealism_still_lifes}.

\begin{figure}[h]
    \centering
    {\footnotesize
    \begin{tabular}{ccc}
          \includegraphics[height=0.45\linewidth]{images/0849.png} &  \includegraphics[height=0.45\linewidth]{images/0212.png} & 
          \includegraphics[height=0.45\linewidth]{images/0668.jpg}\\  
          (a) Ideogram & (b) Stable-Diffusion-3.5-large & (c) Auto-Aesthetic V1
    \end{tabular}
    }
    \caption{Hyperrealism examples. The required styles were: (a) surrealism, (b) classicism, (c) renaissance }
    \label{fig:hyperrealism_still_lifes}
\end{figure}

\subsubsection{Landscapes and cityscapes}
\label{sec:landscapes}
As evidenced by the tag analysis in Section \ref{sec:subject_influence}, the most challenging elements in the representation of naturalistic scenes are clouds and water, particularly when combined with specific times of the day—such as dawn or sunset—or when the prompt demands highly dramatic atmospheric effects. These conditions require a delicate interplay of light, color gradients, and reflections, which can be difficult for generative models to reproduce in a way that remains both visually coherent and stylistically faithful.

Successfully depicting clouds and water often requires a nuanced understanding of texture, movement, and atmospheric perspective. While most models have made significant progress in generating these elements with a high degree of realism, they often struggle when tasked with deviating from photorealism in favor of a specific artistic technique. Instead of adapting to the  rushed brushwork of Impressionism or the soft, fairy-like contrast of Baroque landscapes, models frequently default to overly detailed or artificially blended textures, resulting in images that feel technically proficient but stylistically inconsistent with historical painting traditions.
Some examples are given in Figure~\ref{fig:hyperrealism_cluouds_water}.

\begin{figure}[h]
    \centering
    {\footnotesize
    \begin{tabular}{ccc}
          \includegraphics[height=0.29\linewidth]{images/0220.png} &  \includegraphics[height=0.29\linewidth]{images/0359.png} & 
          \includegraphics[height=0.29\linewidth]{images/0080.jpg}\\  
          (a) Stable-Diffusion-3.5 & (b) Flux 1.1 -Schnell & (c) Midjourney
    \end{tabular}
    }
    \caption{Hyperrealism examples. The required styles were: (a) impressionis,, (b) baroque, (c) rococo vedutism }
    \label{fig:hyperrealism_cluouds_water}
\end{figure}

%Many generative models adopt the annoying practice to add a pronounced fading effect at the horizon; some examples are Figure~\ref{fig:anachronisms}(b-c), Figures~\ref{fig:fading} or Figure~\ref{fig:drapery}. 

%\begin{figure}[h]
%    \centering
%    \begin{tabular}{ccc}
%        \includegraphics[height=0.29\linewidth]{images/fading_0334.png} 
        % & \includegraphics[height=0.29\linewidth]{images/fading_0425.png} 
      %   &  \includegraphics[height=0.29\linewidth]{images/fading_0308.png}\\
     %    Flux1.1 - Schnell & Omnigen & Stable Diffusion 3.5
  %  \end{tabular}
  %  \caption{Caption}
   % \label{fig:fading}
%\end{figure}

%This technique is often recommended to amateur painters to add depth to the painting. However, it is not often adopted by artists: the canvas is too limited to miss the opportunity to use the descriptive potential of each part. Especially when multiple narrative planes are layered, what happens in the background has the same significance and clarity as the scenes in the foreground. 

The tension between hyperrealism and stylistic accuracy becomes particularly evident in the case of naïve art. While hyperrealism is defined by extreme attention to detail, the naïve style is intentionally simplified, often characterized by flat perspectives, bold colors, and a disregard for proportional accuracy. Figures and objects may appear distorted or childlike, resembling works created without adherence to formal artistic training.

An example of this contrast is shown in Figure~\ref{fig:naive_fading}(a), where the model introduces a strong fading effect at the horizon. While this technique enhances depth and improves realism, it is fundamentally at odds with the flat rendering style typical of naïve painting. When prompted to generate a flatter sky with reduced fading and perspective effects, the model tends to overcompensate, resulting in an oversimplified artifact that still fails to fully capture the intended aesthetic, as illustrated in Figures~\ref{fig:naive_fading}(b-c).

\begin{figure}[h]
    \centering
    {\footnotesize
    \begin{tabular}{ccc}
        \includegraphics[width=0.3\linewidth]{images/naive_fading.png} 
         & \includegraphics[width=0.3\linewidth]{images/naive_flat2.png} 
         &  \includegraphics[width=0.3\linewidth]{images/naive_flat1.png}\\
         (a) & (b) & (c)
    \end{tabular}
    }
    \caption{(a) A na\"if style sample generated by Dall$\cdot$E, with an excessive sense of prospective. (b-c) Prompting 
    the model to reduce the fading effect at the horizon and produce a more flat perspective results in an oversimplification of the produced artifact.}
    \label{fig:naive_fading}
\end{figure}


%\subsection{Texture, Light and Material inconsistencies}
%\label{sec:texture} 
%
%Dalle 0117.webp (fiume all'alba)
%DallE 0443.webp (onde sulle rocce)
%Stable DIffusion 3.5 00.56.webp (onde sulle rocce)



\subsection{Anachronisms}
\label{sec:anachronisms}
Not infrequently, models may add anachronistic elements in the painting,
completely disrupting the historical setting, and often creating unintentionally humorous effects. A typical example is
the van in the middle of the scene of Figure~\ref{fig:anachronisms}.a, inspired to the style of Pieter Bruegel the Elder.
\begin{figure}[h]
    \centering
    {\footnotesize
    \begin{tabular}{ccc}
          \includegraphics[width=0.27\linewidth]{images/bruegel_winter2.png} &
          \includegraphics[width=0.27\linewidth]{images/cava620.jpg} &
          \includegraphics[width=0.27\linewidth]{images/raffaello_gen.png} \\
          (a) \parbox[t]{3.2cm}{Dall$\cdot$E 3: Winter landscape in ``Bruegel's style''.}& (b) \parbox[t]{3.2cm}{Firefly: Realism Art of the
          Industrial Revolution period.} &
          (c) \parbox[t]{3.2cm}{Dall$\cdot$E 3: Expulsion from Eden in ``Raffaello's style''.}
    \end{tabular}
    }
    \caption{Caption}
    \label{fig:anachronisms}
\end{figure}

The artwork of Figure~\ref{fig:anachronisms}~(b) was supposed to be a watercolor inspired by the Realism Art of the Industrial Revolution period, depicting an open-cut mine with industrial activity. Almost all models filled the scene with modern crane-like machines and trucks. 

Another example is the iron fence in Figure~\ref{fig:anachronisms}~(c). The work is intended to represent the Expulsion from the Garden of Eden, supposedly mimicking the style of Raphael. The composition is rather confused and the style is essentially neoclassical; however, the iron fence enclosing the ``garden" is particularly jarring. Wrought iron fences of this type became common in Europe only in the 19th century, primarily for gates or balconies, so its presence in a painting that is meant to belong to the Renaissance period is entirely unjustified.

While the clothing is generally accurate for the specified period and style, there are still some noticeable mistakes. For example, in 
Figure~\ref{fig:hyperrealism_faces}(d), the girl washing garments, ostensibly from the 19th century, is wearing sneakers — a clear anachronism. Similarly, the group of people lounging by the seaside in Figure~\ref{fig:anachronisms2}, intended to reflect the Impressionist movement, includes women wearing bikinis — an anachronism that is clearly out of place.

\begin{figure}[h]
    \centering
    {\footnotesize
    \begin{tabular}{cc}

          \includegraphics[width=0.3\linewidth]{images/0504.png}\hspace{.5cm} &  \hspace{.5cm}\includegraphics[width=0.3\linewidth]{images/0094.jpg} \\
          (a) Kolors: impressionist seaside & 
          (b) Midjourney: Baptism of Christ
    \end{tabular}
    }
    \caption{Caption}
    \label{fig:anachronisms2}
\end{figure}

The most intriguing and complex example of anachronism is Midjourney's depiction of the Baptism of Christ, shown in 
Figure~\ref{fig:anachronisms2}. The issue lies in the wound on Jesus' side, traditionally associated with the crucifixion.
This example is fascinating as it highlights a fundamental challenge in generative models: the ability to semantically interpret and contextually place visual elements. The inclusion of the wound suggests the model has conflated distinct aspects of Christian iconography, likely due to overlapping representations of Christ within its training data drawn from various narratives.

This mistake underscores the difficulty of ensuring historical and theological accuracy in generative outputs, particularly when representing complex religious or cultural symbols. It reflects a lack of nuanced understanding, with the model treating all depictions of Jesus as interchangeable rather than context-specific. Addressing such issues would require either more carefully curated training data or the development of advanced mechanisms for context-aware generation. This example serves as a compelling case study in the importance of aligning generative outputs with both stylistic fidelity and semantic coherence.


\section{Conclusions}
\label{sec:conclusions}
In this work, we have explored the capabilities and limitations of modern generative models in replicating historical artistic styles. Our analysis is structured around two main contributions: (1) the creation of a large, supervised dataset of AI-generated artworks— the AI-Pastiche dataset— and (2) a comprehensive evaluation of generative models through user surveys assessing perceptual authenticity and prompt adherence.

The AI-Pastiche dataset is a richly annotated collection of AI-generated images, categorized by model, style, period, and subject matter. It serves as a valuable resource for analyzing the strengths and weaknesses of different generative approaches, holding potential for a wide range of applications and providing a benchmark for future research on AI-driven artistic replication.

Using the AI-Pastiche dataset, we conducted a systematic evaluation of generative models based on extensive user surveys. We separately assessed perceptual authenticity—how convincingly an artwork mimics human-created paintings—and prompt adherence—how faithfully the output aligns with the given instructions. The results reveal a key trade-off: some models prioritize aesthetic quality over strict adherence to the prompt, while others sacrifice visual refinement for greater accuracy. This discrepancy underscores the challenges in balancing creative flexibility and control in generative image synthesis.

Our study highlights both the progress and ongoing challenges in generative AI for artistic style replication. While models can produce visually compelling outputs, a major obstacle remains their tendency toward hyperrealism. In attempting to reproduce historical styles, these models focus on surface-level details, such as textures and brushwork, yet fail to capture the deeper artistic principles that define each period. Artistic style is more than a sum of textures—it involves composition, narrative intent, spatial relationships, and cultural context. Given the limited availability of training data for many historical styles, achieving a truly contextually accurate AI-generated artwork remains a difficult task.

Another fundamental limitation is the rigid inference time of generative models. Unlike human artists, who naturally allocate more effort to complex compositions, these models operate under fixed computational budgets, leading to missed opportunities for adaptive refinement. Future improvements may involve confidence-based step adjustments, allowing the model to extend or shorten the generation process depending on the complexity of the scene. More advanced conditioning mechanisms could also enable models to better integrate structural coherence and artistic intent rather than simply mimicking surface features.

Ultimately, our findings point to the next frontier in generative AI for art: moving beyond simple visual reproduction toward models that can understand and interpret artistic traditions in a more holistic and historically grounded way. While significant challenges remain, improvements in training strategies, dataset curation, and adaptive inference methods could help bridge the gap between style imitation and true artistic coherence, bringing generative AI closer to meaningful contributions in digital artistry.

\section*{Declarations}


\textbf{Funding.} Research partially supported by the Future AI Research (FAIR) project of the National Recovery and Resilience Plan (NRRP), Mission 4 Component 2 Investment 1.3 funded from the European Union - NextGenerationEU.\smallskip

\noindent
\textbf{Conflict of Interest.} On behalf of all authors, the corresponding author states that there is no conflict of interest.


\documentclass{MITstyle}

%\usepackage[table]{xcolor}
\usepackage{chngcntr}
\usepackage{hyperref}
\usepackage{microtype}

\title{A Lightweight and Extensible Cell Segmentation and Classification Model for Whole Slide Images}

\author{Nikita Shvetsov~$^{1, }$\footnote{Correspondence e-mail: nikita.shvetsov@uit.no}, Thomas K. Kilvaer~$^{2, 3}$, Masoud Tafavvoghi~$^{4}$, Anders Sildnes~$^{1}$, \\ Kajsa Møllersen~$^{4}$, Lill-Tove Rasmussen Busund~$^{5, 6}$, Lars Ailo Bongo~$^{1}$ \\
%
\vspace{1em} % Space between authors and afilliations
%
\normalfont{\small $^{1}$Department of Computer Science, UiT The Arctic University of Norway}\\
\normalfont{\small $^{2}$Department of Oncology, University Hospital of North Norway}\\
\normalfont{\small $^{3}$Department of Clinical Medicine, UiT The Arctic University of Norway}\\
\normalfont{\small $^{4}$Department of Community Medicine, UiT The Arctic University of Norway}\\
\normalfont{\small $^{5}$Department of Medical Biology, UiT The Arctic University of Norway} \\
\normalfont{\small $^{6}$Department of Clinical Pathology, University Hospital of North Norway} %\vspace{2em}
}

\begin{document}
\maketitle

\section*{Abstract}

% \begin{abstract}
% Developing clinically useful cell-level analysis tools in digital pathology remains challenging due to limitations in dataset granularity, inconsistent annotations, computational demands of advanced models, and difficulties in integrating new technologies into clinical workflows. To address these challenges, we propose a multi-faceted solution that enhances data quality, model performance, and usability to create a lightweight and extensible cell segmentation and classification model.

% First, we update data labels by employing a cross-relabeling process that refines the labels of two existing datasets, PanNuke and MoNuSAC, to create a new unified dataset with enhanced granularity, encompassing seven distinct cell types. Second, we leverage the H-Optimus foundation model as a fixed encoder to improve feature representation for simultaneous cell segmentation and classification tasks. Third, to address the computational demands of foundation models, we employ knowledge distillation to reduce model size and complexity while maintaining comparable performance. Finally, to facilitate integration into clinical workflows, we integrate the distilled model into the QuPath software, a widely used open-source platform in digital pathology.

% Our results demonstrate improvements in cell segmentation and classification performance using the H‑Optimus-based model compared to a CNN-based model. Specifically, the average $R^2$ improved from 0.575 to 0.871, and the average $PQ$ score improved from 0.450 to 0.492, indicating better alignment with actual cell counts and enhanced segmentation and classification quality. Furthermore, the distilled student model maintains performance comparable to the larger foundation model while reducing the parameter count by a factor of 48.
% Overall, by reducing computational complexity and integrating it into existing workflows, the proposed approach may significantly impact diagnostic processes, reduce the workload of pathologists, and contribute to improved patient outcomes. Though our approach shows potential enhancements in efficiency and usability of cell segmentation and classification models in digital pathology, extensive validation is needed to deploy these models in clinical practice.
% \end{abstract}

%%% shortened abstract
\begin{abstract}
Developing clinically useful cell-level analysis tools in digital pathology remains challenging due to limitations in dataset granularity, inconsistent annotations, high computational demands, and difficulties integrating new technologies into workflows. To address these issues, we propose a solution that enhances data quality, model performance, and usability by creating a lightweight, extensible cell segmentation and classification model. 

First, we update data labels through cross-relabeling to refine annotations of PanNuke and MoNuSAC, producing a unified dataset with seven distinct cell types. Second, we leverage the H-Optimus foundation model as a fixed encoder to improve feature representation for simultaneous segmentation and classification tasks. Third, to address foundation models' computational demands, we distill knowledge to reduce model size and complexity while maintaining comparable performance. Finally, we integrate the distilled model into QuPath, a widely used open-source digital pathology platform. 

Results demonstrate improved segmentation and classification performance using the H-Optimus-based model compared to a CNN-based model. Specifically, average $R^2$ improved from 0.575 to 0.871, and average $PQ$ score improved from 0.450 to 0.492, indicating better alignment with actual cell counts and enhanced segmentation quality. The distilled model maintains comparable performance while reducing parameter count by a factor of 48. By reducing computational complexity and integrating into workflows, this approach may significantly impact diagnostics, reduce pathologist workload, and improve outcomes. Although the method shows promise, extensive validation is necessary prior to clinical deployment.
\end{abstract}
\clearpage

\section{Introduction}
In digital pathology, accurate segmentation and classification of cells are crucial for many diagnostic, prognostic, and predictive analyses \cite{Jaber_Beziaeva_etal._2019,Lin_Pan_etal._2022,Park_Ock_etal._2022,Shen_Choi_etal._2024}. Nowadays, developments in computational pathology offer multiple solutions \cite{H._Qu_P._Wu_etal._2020,Javed_Mahmood_etal._2020} to utilize cell-level datasets to train machine learning models that solve these problems. The quality and specificity of training datasets are critical for robust and accurate models. Adhering to the principle of "garbage in, garbage out", it is essential to ensure that these datasets are extensively and accurately labeled with distinct classes that reflect the diverse biological characteristics of different cell types. Unfortunately, the number of open-source datasets comprising such high-quality annotations is limited. Existing cell segmentation datasets \cite{Gamper_Koohbanani_etal._2019,Graham_Vu_etal._2019,Verma_Kumar_etal._2021} may offer extensive annotations for certain cell types while providing more general labels for others. For example, in PanNuke, which is one of the largest open-source datasets comprising labeled cells, various types of morphologically and functionally different inflammatory cells like macrophages and lymphocytes are clustered in a broad "inflammatory" class. Consequently, these classes are frequently omitted from analyses or aggregated into broader meta-classes \cite{Gamper_Koohbanani_etal._2020} and likely interfere with other cell classes included in the dataset. This and similar inconsistencies in annotation granularity limit the ability of machine learning models to learn the comprehensive and nuanced features necessary for accurate cell segmentation and classification. To address these challenges, methods for refining and standardizing dataset annotations are essential to enhance the quality of training data.

A complementary approach to mitigate the absence of high-quality training data is the use of foundation models. Foundation models as encoders are defined as large-scale, versatile networks pre-trained on vast, diverse datasets using self-supervised learning, contrasting with convolutional neural network (CNN) pre-trained encoders that rely on supervised learning with labeled data. In practice, foundation models leverage enormous amounts of weakly or unlabeled data from millions of whole slide images (WSIs) and employ self-attention mechanisms to capture long-range dependencies and global context \cite{Chen_Ding_etal._2024,Saillard_Jenatton_etal._2024,Vorontsov_Bozkurt_etal._2024,Xu_Usuyama_etal._2024}. As a consequence, foundation models are able to produce transferable feature representations across different cell types and tissue environments. The feature representations can be leveraged by decoder networks to produce segmentation masks and pixel-level classifications. Because foundation models have comprehensive feature representations, they can be effectively fine-tuned using much smaller amounts of cell-level data compared to the large datasets needed to train models from scratch. Furthermore, foundation models incorporate adversarial training elements or contrastive learning \cite{Chen_Ding_etal._2024,Xu_Usuyama_etal._2024}, enhancing their resilience and adaptability by exposing them to challenging and varied scenarios during training. This may result in more generalizable models, often making them well-suited for diverse and complex tasks in digital pathology.

Despite the inherent advantages of foundation models, their deployment for practical use faces its own obstacles. In particular, they require substantial computational power, financial investments and rigorous testing to ensure reliability and efficacy for a given task \cite{Akkus_Dangott_etal._2022,Dragomir_Cocuz_etal._2022,Go_2022,Jafri_Farooqui_etal._2024}. Moreover, while foundation models enhance feature representation and performance, they depend on the quality of available annotations for decoder fine-tuning and, like any other model, cannot resolve existing inconsistencies or ambiguities in data labels. Therefore, there remains a critical need for solutions that address both data quality and practical deployment considerations.
Further, integrating new technologies into existing clinical workflows often encounters resistance, as it necessitates adjustments to established diagnostic processes. So, there is a need to develop solutions that could be integrated into current practices, minimizing the burden on medical professionals to adopt new tools \cite{King_Williams_etal._2023}.

Existing solutions \cite{Goldsborough_Philps_etal._2024,Hörst_Rempe_etal._2024}, while addressing some aspects of these challenges, fall short in providing a comprehensive approach. To address the data quality and clinical deployment issues, we propose a multi-faceted solution that encompasses data refinement, model optimization, and integration with existing pathology tools (\hyperref[fig:fig1]{Figure 1}). The outcome is a lightweight cell segmentation and classification model that can be integrated into digital pathology workflows for practical clinical use.

\begin{figure}[h!]
    \centering
    \includegraphics[width=\textwidth, height=0.82\textheight, keepaspectratio]{images/Figure_1.pdf}
    \caption{Overview of the proposed solution, including 1) Data refinement using cross-relabeling, 2) Teacher model development and fine tuning, 3) Student model optimization with knowledge distillation and 4) Student model and QuPath integration}
    \label{fig:fig1}
\end{figure}
\clearpage

Our approach begins with preparing the data for the fine-tuning and training of the machine learning models. We create a refined dataset, acquired via cross-relabeling two cell-level datasets, enhancing annotation specificity and consistency of the labeled data. Subsequently, we create a cell segmentation and classification model based on the foundation model. We leverage the foundation model as a fixed encoder and fine-tune a decoder using the refined dataset to improve generalization across diverse tissue- and cell types.
To ensure that the model remains lightweight and deployable in a possibly resource-constrained environment, we employ knowledge distillation to approximate the functionality of the foundation model. Finally, to facilitate the practical application of our model in digital pathology workflows, we integrate it with the QuPath \cite{Bankhead_Loughrey_etal._2017} application. Each methodological component contributes to the overarching goal of enhancing model performance, generalizability, and usability in clinical settings.

The primary contributions of this paper are:
\begin{enumerate}
    \item \textit{Data labels refinement through cross-relabeling:}
    
    We propose a new method for refining labels of cell-level datasets through cross-relabeling. This method employs classification models to re-label broad and ambiguous instances, resulting in a more diverse dataset. Our evaluation demonstrates that these classification models achieve high accuracy on test subsets, indicating the reliability of the method for label refinement.

    \item \textit{Enhanced model performance via foundation models:}
    
    We employ a foundation model as a feature extractor for the cell segmentation and classification task. In comparison with training a CNN model from scratch, the foundation model backbone only needs fine-tuning, which significantly reduces training time, computational resources and data requirements. We show that using a foundation model encoder leads to better performance in cell segmentation and classification networks than using a CNN-based encoder. This improvement may enable the model to generalize more effectively across various tissue types and imaging methods.
    
    \item \textit{Model optimization through knowledge distillation:}
    
    We show that a smaller student model trained using knowledge distillation on the refined dataset obtained via our cross-relabeling approach from a foundation model achieves comparable performance in cell segmentation and quantification tasks. As a result, this model is more suitable for deployment in environments without high-performance computing resources.
    
    \item \textit{Integration with QuPath:}
    
    We integrate the distilled cell segmentation and classification model into QuPath, a widely used open-source digital pathology platform, to accelerate clinical adaptation by enabling pathologists to more easily incorporate advanced computational tools into their existing workflows.
\end{enumerate}

Through these methodological steps, we aim to bridge the gap between advanced machine learning techniques and practical clinical applications, making accurate and efficient digital pathology accessible in a broader range of healthcare settings.

\section{Refining Existing Datasets Using Cross-Relabeling}
To address the limitations of sparse and ambiguous labeling of cell-level datasets, we propose a generalizable cross-relabeling strategy that can be applied to any dataset containing broadly categorized or imprecisely labeled cell types. This approach involves training and subsequently leveraging classification models to refine broad categories into more specific or biologically relevant classes.
When applied to cell-level data, the methodology includes extracting individual cell images from the dataset patches, preprocessing these images to standardize the size and accommodate partial cells, and then training deep learning classifiers capable of distinguishing between the finer cell subtypes within the coarser categories. 
To illustrate our approach, we focus on the PanNuke \cite{Gamper_Koohbanani_etal._2020, Gamper_Koohbanani_etal._2019} and MoNuSAC \cite{Verma_Kumar_etal._2021} datasets that we have used to train models for cell quantification in our previous works \cite{Shvetsov_Grønnesby_etal._2022,Shvetsov_Sildnes_etal._2024}. We find that for better cell differentiation we have to introduce more granular labels. PanNuke includes a broad classification of "inflammatory" cells, encompassing lymphocytes, macrophages, and neutrophils. Each cell type differs significantly in structure, function, and clinical relevance. Conversely, MoNuSAC uses the label "epithelial" for a class that comprises both benign epithelial cells and malignant neoplastic cells. This practice makes it challenging to differentiate between benign and malignant epithelial cells in the dataset, which is a critical distinction when identifying tumor areas within tissue samples. To address these issues, we implement a cross-relabeling strategy as shown in \hyperref[fig:fig2]{Figure 2}. The key components are two classification models: one is trained on singular cell images from PanNuke data to classify the epithelial meta-class into epithelial and neoplastic classes. The other is trained on MoNuSAC to refine the inflammatory class into lymphocytes, neutrophils, and macrophages.

\begin{figure}[h!]
    \centering
    \includegraphics[width=\textwidth]{images/Figure_2.pdf}
    \caption{Refined dataset generation via cross relabeling}
    \label{fig:fig2}
\end{figure}

The refining approach consists of three consecutive steps. The first is the preprocessing step, in which we extract individual cells from both datasets (\hyperref[fig:fig3]{Figure 3}). The specifics of PanNuke and MoNuSAC patch preparation before cell preprocessing are provided in \hyperref[chap:S1]{Appendix S1}.

\begin{figure}[h!]
    \centering
    \includegraphics[width=\textwidth]{images/Figure_3.pdf}
    \caption{Cell instances preprocessing including (1) cell map extraction, (2) bounding box delineation, (3) adjusting cell boxes and (4) cropping and resizing of cell images}
    \label{fig:fig3}
\end{figure}

During preprocessing, we extract cell type maps from the ground truth label mask and calculate bounding boxes around each cell instance. To accommodate partial cells at patch borders, a common issue in cropped patch images, we employ mirror padding and extend the field of view of the cell label by 15 pixels to capture adjacent cells. We then crop and resize the identified regions to $64 \times 64$ pixels using bicubic interpolation.

The preprocessed PanNuke dataset comprises 68,031 neoplastic and 23,207 epithelial cell images, while MoNuSAC comprises  33,104 lymphocytes, 1,252 neutrophils, and 1,695 macrophages, which we subsequently use in training cell classification models and classifying the cell image data \hyperref[fig:S2]{Appendix Figure S2 (1)}. 

The next step is to train two distinct ResNet50-based classifiers tailored to address the specific labeling challenges inherent in each dataset. We use ResNet50 for classification models due to its proven effectiveness for image classification tasks in histopathology \cite{pan2022reviewmachinelearningapproaches}, and its compatibility with small images. For the PanNuke dataset, we design the classifier, trained on MoNuSAC data, to disaggregate the heterogeneous "inflammatory" cell category into distinct subtypes: lymphocytes, macrophages, and neutrophils. Similarly, for the MoNuSAC dataset, the classifier is trained on PanNuke data and distinguishes between benign and malignant epithelial cells within the overarching "epithelial" label. By applying these targeted classifiers to their respective datasets, we assign more specific labels to individual cell instances, thus enabling us to create a unified dataset.
To ensure a balanced representation of classes, we train both models on datasets that had been equalized to match the size of the least represented class. Thus, we obtain datasets comprising 23,207 samples per class for PanNuke and 1,252 samples per class for MoNuSAC data. Next, we partition both of them into training (70\%), validation (20\%), and testing (10\%) subsets. To mitigate the risk of overfitting, we use a single dropout layer with a rate of p=0.5 in both models and data augmentation using randomized color perturbations, rotation, and horizontal and vertical flipping. We employ AdamW optimizer and the cross-entropy loss function for the training criterion.

To evaluate the two trained models, we measure the classification accuracy on the respective test subsets. The accuracies on the test subset for both classifiers are presented in \hyperref[tab:1]{Table 1}. The PanNuke model achieves an average accuracy of 93.57\%, with higher accuracy for neoplastic cells (96.06\%) compared to epithelial cells (86.26\%). The confusion matrix in Figure A3.1 shows that the model predominantly distinguishes accurately between epithelial and neoplastic tissues, with a substantial number of correct classifications and relatively few misclassifications. The MoNuSAC model demonstrates an average accuracy of 98.92\%, excelling in classifying lymphocytes (99.67\%) and macrophages (94.12\%), with lower performance for neutrophils (85.71\%). The confusion matrix in Figure A3.2 shows that the model identifies lymphocytes and performs reasonably well with macrophages and neutrophils.

\begin{table}[h!]
\renewcommand{\arraystretch}{1.5}
  \centering
  \caption{Cell classification results for PanNuke and MoNuSAC trained models (CI 95\%).}
  \label{tab:1}
  \begin{tabular}{|l|c|c|}
   \hline
   %\rowcolor{gray!30}
    Accuracy               & PanNuke model              & MoNuSAC model              \\
    \hline
    Average      & 0.936 (0.931--0.941)         & 0.989 (0.986--0.993)        \\
    \hline
    Neoplastic   & 0.961 (0.956--0.965)         & -                          \\
    \hline
    Epithelial   & 0.863 (0.849--0.877)         & -                          \\
    \hline
    Lymphocytes  & -                          & 0.997 (0.995--0.999)        \\
    \hline
    Neutrophils  & -                          & 0.857 (0.796--0.918)        \\
    \hline
    Macrophages  & -                          & 0.941 (0.906--0.976)        \\
    \hline
  \end{tabular}
\end{table}

Finally, during the last step, we use the model trained on PanNuke data for epithelial cells in MoNuSAC and the model trained on MoNuSAC for the inflammatory cells class in PanNuke. Specifically, we use classifier models to relabel epithelial cells in MoNuSAC and inflammatory cells in PanNuke data. Then we combine cells with refined labels and the rest of the cells in both datasets to create a refined dataset (\hyperref[fig:S2]{Appendix Figure S2 (2)}). The process of relabeling cells and visualizing them on a patch is shown in \hyperref[fig:fig4]{Figure 4}. The cell counts in the refined dataset are provided in \hyperref[tab:S4]{Appendix Table S4}.

\begin{figure}[h!]
    \centering
    \includegraphics[width=\textwidth, height=0.42\textheight, keepaspectratio]{images/Figure_4.pdf}
    \caption{Cell relabeling procedure for epithelial and inflammatory cell classes}
    \label{fig:fig4}
\end{figure}

%\hfill

Relabeling and combining datasets have been explored in a prior study \cite{Parulekar_Kanwat_etal._2023}, where consecutive fine-tuning on multiple datasets was employed to account for hierarchical class label structures. While the method presented in \cite{Parulekar_Kanwat_etal._2023} is intuitive, it often lacks consistency and requires multiple fine-tuning runs, which can be cumbersome and time-consuming. 
In contrast, cross-relabeling simplifies this process by using specialized classification models tailored to each dataset's specific labeling challenges. This approach provides better transparency and produces a unified dataset encompassing seven distinct cell types across multiple tissue samples, enhancing data diversity for further model training or fine-tuning.

Despite these improvements, cross-relabeling does not entirely resolve issues related to poor labeling quality or the amount of labeled data. Specifically, our results show lower accuracies persist for underrepresented classes, such as macrophages, which may stem from a limited sample availability and intrinsic challenges in distinguishing these cells based solely on H\&E staining. Furthermore, while our method enhances label specificity, it relies on the initial quality of the broad labels; thus, any fundamental inaccuracies in the original annotations can propagate through the relabeling process. Addressing the overall problem of limited data labels may require integrating additional data sources or utilizing complementary immunohistochemical staining methods.
Although the reported performance metrics are obtained from evaluations on the native test sets of each dataset, it is important to note that the primary application of these classifiers is to perform cross-relabeling, where a model trained on one dataset (e.g., PanNuke) is applied to another (e.g., MoNuSAC) and vice versa. We acknowledge that a more systematic evaluation of cross-dataset generalization is needed and could be performed in future work.

Overall, the refined dataset produced by our approach can enhance the supervised training or fine-tuning of cell segmentation and classification models, especially those that utilize pre-trained foundation models to improve feature extraction robustness. In addition, these models can detect nuanced classes that enable researchers to conduct more detailed analyses of biological processes in computational pathology.

\section{Foundation models for robust cell segmentation and classification}

Accurate cell segmentation and classification in digital pathology are hindered by limited labeled data and the fact that conventional CNNs are unable to capture global contextual information due to their local receptive field constraints \cite{Gheflati_Rivaz_2022,Yang_Marcus_etal.}. Traditional approaches in cell quantification have predominantly relied on CNN encoders, such as ResNet50, given their proven effectiveness in semantic segmentation tasks \cite{Deshmane_2023,Graham_Vu_etal._2019,Mukasheva_Koishiyeva_etal._2024,Stringer_Wang_etal._2021}. However, approaches that include fine-tuning of pretrained CNNs, data augmentation, and stain normalization to partially increase data variability and address staining differences often fail to achieve the necessary generalization and robustness across diverse tissue types and staining conditions \cite{G._Wang_W._Li_etal._2018,Gao_Bagci_etal._2018,Karim_El_Khoury_Martin_Fockedey_etal._2021}.

To overcome these challenges, we leverage an encoder-decoder network that uses a foundation model as the encoder and a CNN upsampling decoder (\hyperref[fig:fig5]{Figure 5}) for simultaneous cell segmentation and classification in 2D patches extracted from WSIs. Foundation models with transformer-based architectures are viable alternatives to CNN-based encoders \cite{Shamshad_Khan_etal._2023,Sourget_2023}. They enable the creation of more advanced architectures that can decode or transform learned features more effectively \cite{Chen_Duan_etal._2023,Cheng_Misra_etal._2022,Xie_Wang_etal._2021}.

\begin{figure}[h!]
    \centering
    \includegraphics[width=\textwidth]{images/Figure_5.pdf}
    \caption{UNETR-like model with foundational model as backbone}
    \label{fig:fig5}
\end{figure}

By utilizing a transformer-based encoder, we incorporate global contextual information into the feature extraction process, which is a key advantage of such architectures \cite{Chen_Lu_etal._2021}. This foundation model integration facilitates accurate pixel-wise segmentation and classification without the need for extensive encoder training, thereby potentially improving generalization across varied cellular structures and tissue types.
In our implementation, we employ a modified UNETR \cite{Hatamizadeh_Tang_etal._2021} architecture that combines a vision transformer (ViT) \cite{Dosovitskiy_Beyer_etal._2021} encoder with a CNN-based decoder. The encoder utilizes the pretrained H-Optimus foundation model, which contains 1.1 billion parameters and is trained on over 500,000 H\&E stained WSIs \cite{Saillard_Jenatton_etal._2024}. We extract outputs from four evenly spaced transformer blocks $Z_i$, where $i \in [1, 14, 26, 38]$, to serve as residual connections for the CNN decoder. We select these blocks based on our observation that features from non-adjacent levels of the encoder lead to better overall performance on the test subset.

The CNN decoder upsamples the feature representations, acquired from the transformer blocks, to generate an intermediate vector that is handled by two task-specific layers that generate cell segmentation and classification masks. The first task-specific layer is the ‘Cellpose head’,  which is used to delineate cell instances. The layer generates horizontal and vertical gradient maps to form vector fields that are refined through gradient tracking in a post-processing step using the Cellpose algorithm \cite{Stringer_Wang_etal._2021}, known for its efficacy in cell segmentation tasks and generalizability across multiple domains \cite{Pachitariu_Stringer_2022,Stringer_Pachitariu_2024}. The second task-specific layer is the "Cell type head", which assigns labels to individual pixels. In the post-processing step, we determine the output classification label of each segmented cell instance by majority voting over the labeled pixels that comprise the cell in the segmentation map.

To evaluate model performance and measure the impact of adding a foundation model as backbone, we compare it to a ResNet50-based model. ResNet50 is a widely used solution for encoders in segmentation architectures in the medical domain \cite{Deshmane_2023,Graham_Vu_etal._2019,Mukasheva_Koishiyeva_etal._2024,Stringer_Wang_etal._2021}. For the H-Optimus-based model, we utilize frozen weights for the encoder and only fine-tune the decoder to take advantage of the extensive pre-training of the foundation model. For the ResNet50-based model we start with ImageNet \cite{Deng_Dong_etal.} weights and train both encoder and decoder parts. Hyperparameters for the training step are set to be identical, where possible, for comparable evaluation. 
For this evaluation, we deliberately use the PanNuke dataset to provide a standardized and controlled comparison between the H‑Optimus and ResNet50-based models (\hyperref[fig:S2]{Appendix Figure S2 (3)}). Specifically, we use two of the default PanNuke dataset splits (66\%) for training and validation, and reserve the third split (33\%) for testing.

To address the challenge of cell class imbalance in the PanNuke dataset, which is a common characteristic in most cell-level H\&E patch datasets, both models’ training processes employ a weighted loss function comprising cross-entropy and focal loss \cite{Lin_Goyal_etal._2018}. The focal loss component is adjusted with coefficients derived from each cell class' instance frequency, emphasizing learning from underrepresented classes and enhancing the model's sensitivity to rare but significant cellular patterns. The cross-entropy loss is augmented with spectral decoupling regularization \cite{Pezeshki_Kaba_etal._2021,Pohjonen_Stürenberg_etal._2022} and spatially varying label smoothing \cite{Islam_Glocker_2021}, which potentially stabilizes training and improves generalization in case of complex tissue morphologies. For optimization, we employ the \textit{AdamW} \cite{Loshchilov_Hutter_2019} to counter unbalanced class scenarios, with cosine annealing learning rate scheduler.

We utilize the scikit-learn library \cite{Van_der_Walt_Schönberger_etal._2014} and HoVer-Net \cite{Graham_Vu_etal._2019} implementations of $R^2$ (the coefficient of determination) and $PQ$ (panoptic quality) to evaluate our experiments. Complete mathematical formulations and detailed explanations of these metrics are provided in \hyperref[chap:S5]{Appendix S5}. To compute confidence intervals, we use nonparametric bootstrapping, where after calculating the metric on the full sample, we generated 1000 bootstrap replicates by resampling with replacement and then determined the 95\% confidence intervals as the 2.5th and 97.5th percentiles of the resulting empirical distribution.

%\hfill

The model comparisons are summarized in \hyperref[tab:2]{Table 2}. The H‑Optimus-based model achieves higher $R^2$ across all cell classes compared to the ResNet50-based model, which means that its predictions are more closely aligned with the PanNuke cell counts, indicating a stronger correlation with the observed data. Notably, the improvement of $R^2_{dead}$ may be an indicator of better global contextual representations provided by the foundation model backbone. In terms of segmentation and classification quality combined, measured by the PQ score, the H‑Optimus-based model demonstrates notable improvements across most cell classes. Overall, the average $R^2$ improved from 0.575 to 0.871, while the average $PQ$ score improved from 0.450 to 0.492, demonstrating better performance of the H-Optimus-based model.

\begin{table}[h!]
\renewcommand{\arraystretch}{1.5}
  \centering
  \caption{Cell quantification metrics for baseline and proposed models (CI 95\%).}
  \label{tab:2}
  \begin{tabular}{|l|c|c|}
    \hline
    %\rowcolor{gray!30}
    Metric             & Resnet50-based            & H-optimus-based              \\
    \hline
    $R^2_{neoplastic}$    & 0.681 (0.576--0.769)       & \textbf{0.941 (0.917--0.960)} \\
    \hline
    $R^2_{inflammatory}$  & 0.863 (0.778--0.903)       & \textbf{0.949 (0.918--0.966)} \\
    \hline
    $R^2_{connective}$    & 0.600 (0.488--0.698)       & 0.609 (0.436--0.772)          \\
    \hline
    $R^2_{dead}$          & 0.097 (-11.389--0.669)     & 0.925 (0.404--0.982)          \\
    \hline
    $R^2_{epithelial}$    & 0.635 (0.490--0.747)       & \textbf{0.930 (0.886--0.964)} \\
    \hline
    $PQ_{neoplastic}$       & 0.517 (0.499--0.535)       & \textbf{0.589 (0.575--0.604)} \\
    \hline
    $PQ_{inflammatory}$     & 0.455 (0.429--0.482)       & \textbf{0.528 (0.507--0.549)} \\
    \hline
    $PQ_{connective}$       & 0.416 (0.400--0.431)       & \textbf{0.451 (0.436--0.465)} \\
    \hline
    $PQ_{dead}$             & 0.374 (0.342--0.408)       & 0.292 (0.209--0.365)          \\
    \hline
    $PQ_{epithelial}$       & 0.488 (0.460--0.519)       & \textbf{0.599 (0.579--0.618)} \\
    \hline
  \end{tabular}
\end{table}

Our results  show that integrating the H‑Optimus foundation model within the UNETR architecture enhances the model's ability to segment and classify cells across diverse tissues from PanNuke data. The pretrained transformer encoder provides robust feature representations, resulting in higher average $R^2$ and $PQ$ scores compared to the CNN-based model. This leads to more reliable cell quantification and more accurate downstream analysis. Additionally, the streamlined fine-tuning process reduces computational overhead and training time, making the model more adaptable for new data.

Despite these advancements, the foundation model-based approach does not fully resolve all challenges related to cell segmentation and classification. We observe lower metric scores for underrepresented classes in the training data. Furthermore, foundation models typically encompass billions of parameters, resulting in substantial computational and memory requirements. It therefore poses challenges for deployment in resource-constrained environments, limiting their practical applicability in certain clinical settings.

\section{Model optimization via Knowledge Distillation}

To address the limitations posed by the extensive size of foundation models, we implement knowledge distillation — a model compression technique that leverages the teacher-student paradigm \cite{Hinton_Vinyals_etal._2015}. By training a smaller, more efficient student model to replicate the output of a larger, pre-trained teacher model, we retain performance while significantly reducing the model's complexity and resource requirements (\hyperref[fig:fig6]{Figure 6}).

\begin{figure}[h!]
    \centering
    \includegraphics[width=\textwidth, height=0.45\textheight, keepaspectratio]{images/Figure_6.pdf}
    \caption{Knowledge distillation framework for training a student model using a pre-trained teacher}
    \label{fig:fig6}
\end{figure}

We employ knowledge distillation to compress the H‑Optimus-based teacher model into a more efficient student model. The teacher model is the modified UNETR architecture with the H‑Optimus foundation model described in the previous chapter. The student model is based on a UNet architecture augmented with residual connections and incorporates a smaller ViT encoder with 9 million parameters \cite{Steiner_Kolesnikov_etal._2022,Wightman_2019}. 

First, we fine-tune the teacher model using the refined dataset from the cross-relabeling procedure (Section 2). Initially we train the decoder of the teacher model while keeping the encoder weights frozen. We split the refined dataset into train (70\%), validation (20\%) and test (10\%) subsets (\hyperref[fig:S2]{Appendix Figure S2 (4)}). During fine-tuning, we use the train and validation subsets, while leaving the test subset for model evaluation. We set the training procedure and model hyperparameters to be identical to those that were used to demonstrate the utility of foundation models for the simultaneous cell segmentation and classification task.

Next, we perform knowledge distillation from teacher to student using the refined dataset used to fine-tune the teacher model. The student model is trained to replicate the teacher model's outputs. We utilize a specialized loss function that aligns the student's predicted probability distribution with the teacher's, incorporating the teacher's class probability distribution derived from the output. Following the methodology of Hinton et al. \cite{Hinton_Vinyals_etal._2015}, we experiment with various hyperparameter settings for the temperature ($T$) and the balancing coefficients ($\alpha$ and $\beta$) in the loss function. We vary $T$ from 1 to 20 and adjust $\alpha$ and $\beta$ to balance the distillation and student losses. Through iterative tuning and evaluation, we identify that setting $T=14$, $\alpha=0.3$, and $\beta=0.7$ yields a configuration that converges and closely approximates the teacher model's performance during training.

Finally, we assess the performance of both models using the $R^2$ and $PQ$ (defined in \hyperref[chap:S5]{Appendix S5}) on the test set of the refined dataset (\hyperref[tab:3]{Table 3}). We observe that the 95\% confidence intervals overlap for most cell types, so we cannot claim statistically significant performance differences between the teacher and student models. One exception appears in the neoplastic class. The teacher model produces an $R^2$ of 0.919, while the student model shows an $R^2$ of 0.852. In addition, the student model achieves higher $PQ$ values for the neoplastic and connective classes, though the confidence intervals show overlap.

\begin{table}[h!]
\renewcommand{\arraystretch}{1.5}
  \centering
  \caption{Cell quantification metrics for teacher and distilled student models (CI 95\%).}
  \label{tab:3}
  \begin{tabular}{|l|c|c|}
    \hline
    %\rowcolor{gray!30}
    Metric & Teacher & Student \\
    \hline
    $R^2_{neoplastic}$    & \textbf{0.919} (0.898--0.939) & 0.852 (0.800--0.891) \\
    \hline
    $R^2_{lymphocyte}$    & 0.969 (0.956--0.977)         & 0.969 (0.956--0.978) \\
    \hline
    $R^2_{connective}$    & 0.694 (0.548--0.809)         & 0.618 (0.469--0.741) \\
    \hline
    $R^2_{dead}$          & 0.755 (0.400--0.908)         & 0.424 (0.100--0.731) \\
    \hline
    $R^2_{epithelial}$    & 0.922 (0.870--0.958)         & 0.843 (0.738--0.917) \\
    \hline
    $R^2_{macrophage}$    & 0.384 (-0.369--0.724)        & 0.704 (0.352--0.859) \\
    \hline
    $R^2_{neutrofil}$     & 0.854 (0.578--0.929)         & 0.833 (0.502--0.925) \\
    \hline
    $PQ_{neoplastic}$       & 0.581 (0.569--0.593)         & 0.601 (0.588--0.613) \\
    \hline
    $PQ_{lymphocyte}$       & 0.536 (0.520--0.553)         & 0.563 (0.544--0.579) \\
    \hline
    $PQ_{connective}$       & 0.436 (0.421--0.451)         & 0.457 (0.441--0.474) \\
    \hline
    $PQ_{dead}$             & 0.272 (0.235--0.315)         & 0.279 (0.201--0.369) \\
    \hline
    $PQ_{epithelial}$       & 0.522 (0.500--0.545)         & 0.530 (0.506--0.555) \\
    \hline
    $PQ_{macrophage}$       & 0.524 (0.459--0.588)         & 0.474 (0.405--0.543) \\
    \hline
    $PQ_{neutrofil}$        & 0.541 (0.490--0.592)         & 0.565 (0.522--0.607) \\
    \hline
  \end{tabular}
\end{table}


We further decompose the $PQ$ metric into its $SQ$ and $DQ$ components (\hyperref[tab:S6]{Appendix Table S6}). Both models produce nearly identical $SQ$ values, which indicates that they predict instance boundaries with similar precision. Although the student model shows some improvement in $DQ$ scores for certain classes, the confidence intervals overlap and do not confirm a statistically significant difference.

We observe that the student and teacher models yield comparable detection performance despite the student model using a much smaller and simpler architecture. A model with fewer parameters reduces the risk of overfitting when training data are scarce relative to the model’s complexity \cite{Farias_Ludermir_etal._2022}. The knowledge distillation process also encourages the student model to focus on the most generalizable detection features learned from the teacher. These factors enable the student model to achieve similar detection performance across different cell types.

Additionally, considering the model sizes reported in \hyperref[tab:4]{Table 4}, the distilled model achieves a significant reduction compared to the teacher model, with a 48-fold decrease in parameter count and a 5.5-fold reduction in on-disk size. In inference mode, the teacher model requires 16 GB of VRAM for a batch size of 32, while the distilled model only needs 3 GB of VRAM for the same batch size. These reductions make the distilled model significantly more practical for fine-tuning and deployment in resource-constrained environments.

\begin{table}[h!]
\renewcommand{\arraystretch}{1.5}
  \centering
  \caption{Parameter counts and size of teacher and distilled model}
  \label{tab:4}
  \adjustbox{max width=\textwidth}{%
  \begin{tabular}{|l|c|c|c|}
    \hline
    %\rowcolor{gray!30}
    Metric & H-optimus-based (Teacher) & mobileViT-based (Student) & Magnitude of difference \\
    \hline
    Parameters count       & 1,158,917,906   & \textbf{24,093,393}   & \textbf{48x}  \\
    \hline
    Estimated Total Size (MB) & 87,912       & \textbf{15,935}    & \textbf{5.5x} \\
    \hline
  \end{tabular}%
}
\end{table}

%\hfill

With recent advancements in complex network architectures and the use of pretrained encoders to achieve state-of-the-art performance \cite{Baumann_Dislich_etal._2024,Hörst_Rempe_etal._2024} in cell segmentation and classification tasks, model size, computational complexity, and processing times have increased. This limits the scalability and accessibility of these models. As we demonstrate, this may be mitigated using knowledge distillation. Studies in the field of natural language processing have demonstrated the efficacy of knowledge distillation in retaining the capabilities of the teacher model while achieving significant reductions in size and complexity \cite{Huangpu_Gao_2024,Sun_Yu_etal.}. 

We demonstrate the feasibility of knowledge distillation in digital pathology, specifically for cell segmentation and classification tasks. Moreover, we achieve this performance while also significantly reducing the parameter count. In addressing the challenge of knowledge transfer, we found that distillation from a transformer-based model to a smaller transformer is more straightforward than attempting to map transformer features to CNN blocks. In our experiments, using a CNN-based network as a student results in worse cell quantification performance due to the structural constraints of CNN feature space dimensions. 

Although our primary approach relies on a transformer-based student model that performs well, it can be further optimized to incorporate advantages from CNN architectures. For example, employing alternative techniques such as using ViT adapters \cite{Chen_Duan_etal._2023} or $1 \times 1$ convolutions to adjust feature map sizes may be beneficial for harnessing CNN advantages like enhanced local feature extraction. Moreover, if additional performance improvements are desired, the process can be further enhanced by applying supplementary knowledge distillation techniques, such as self-distillation \cite{Zhang_Song_etal._2019} or online distillation \cite{Houyon_Cioppa_etal._2023}.

Despite these promising results, further validation on independent datasets is necessary to fully understand the model's limitations. Underrepresented classes may pose challenges when addressing complex cases. Pathologists need to validate these models to adopt them in clinical settings. While the distilled models are smaller and more deployable, a technological gap persists because pathologists traditionally rely on established methods for inspecting WSIs and diagnosing diseases. Addressing the complexities involved in deploying models for inference and supporting pathologists in adopting new tools is essential for integrating these models into clinical workflows.

\section{Model integration with QuPath}
Digital pathology tools with graphical user interfaces are essential for visualizing and analyzing WSIs. To make our student model useful in clinical pathology workflows, it needs to be integrated into a tool that enables inspecting regions, creating annotations, and providing quantitative analyses of biomarkers. Therefore, we integrate the trained student model from the previous chapter into the QuPath open‑source platform \cite{Bankhead_Loughrey_etal._2017}. QuPath provides the required annotation, visualization, and analysis tools to interpret complex histological data, including workflows for cell segmentation, classification, and quantification (\hyperref[fig:fig7]{Figure 7}). 

\begin{figure}[h!]
    \centering
    \includegraphics[width=\textwidth]{images/Figure_7.pdf}
    \caption{Visualization of model-generated cell quantification annotations (left) and the corresponding unannotated slide (right) in QuPath}
    \label{fig:fig7}
\end{figure}

To identify the regions in a WSI critical for prognosticating tumor development, such as specific tumor areas or border regions without overlapping healthy tissue, the pathologist uses QuPath to outline these regions. Then, the pathologist initiates a cell segmentation and classification script through the QuPath interface for the selected regions. The resulting annotations and quantified cell information are then directly overlaid onto the WSI in the QuPath interface. Additional design and implementation details are in \hyperref[chap:S7]{Appendix S7}. 

Two common approaches for integrating deep learning models into QuPath are Java‑based native QuPath extensions \cite{Goldsborough_Philps_etal._2024} and the execution of RESTful API requests to a model server coupled with handling the response via an extension, as demonstrated in the application of cell segmentation models applied to immunofluorescence images \cite{Sugawara_2023}. While the community is actively working on these integration strategies, there is currently no universal solution that fully addresses all integration and performance requirements.

Extensions may offer better integration with QuPath, allowing slightly improved performance and more widespread usage of the built-in QuPath models, but they lack the flexibility to customize models and modify their behavior. For example, the newest version of QuPath includes models such as StarDist \cite{Weigert_Schmidt} and InstanSeg \cite{Goldsborough_Philps_etal._2024} that can perform cell segmentation. Both models pose limitations when applied to simultaneous cell segmentation and classification. StarDist performs well only on convex, round shapes by design, whereas some neoplastic, inflammatory, and connective cells exhibit complex and non-convex shapes. InstanSeg provides only semantic segmentation without assigning classes to the segmented cells.

%\hfill

In contrast, our approach offers an alternative integration strategy. It utilizes the paquo library to directly interact with QuPath’s internal application programming interface from within Python. This enables data exchange and processing without the need for intermediate conversion steps and provides greater control over model customization, retraining, and the incorporation of custom processing steps.

The integration of our custom model with QuPath underscores its potential to significantly enhance the diagnostic process by reducing the time burden on pathologists and enabling them to focus on more complex interpretative tasks using familiar software. Leveraging a tool that is already well-established among pathologists increases the likelihood of its adoption into daily clinical workflows. The quantitative data generated through the automated workflow is critical for both clinical decision-making and research, facilitating more accurate biomarker analysis, enabling robust statistical evaluations, and supporting hypothesis generation and testing. Additionally, by streamlining cell segmentation and classification, the tool enhances the scalability and reproducibility of pathological assessments, ultimately contributing to improved diagnostic accuracy and patient outcomes.

\section{Conclusion and future work}

In this study, we address critical challenges in digital pathology and tackle the usability and deployment issues of the developed models in standard computing environments without the need for high-performance computing systems. Our multi-faceted approach encompasses data refinement through cross-relabeling, leveraging foundation models for robust cell segmentation and classification, optimizing model performance via knowledge distillation, and integrating the optimized model into the QuPath software for practical application. This approach is used to construct a capable, versatile, and adjustable model for cell segmentation and classification, with enhanced performance and usability.

\begin{sloppypar}
While our approach shows potential in the field of computational pathology, certain limitations persist. 
For example, our implementation currently exhibits lower performance in detecting macrophages. 
This serves as an instance of the broader challenge of accurately identifying complex cell types. In order to address this issue, extending our approach to incorporate additional data sources, exploring alternative modeling approaches, and integrating other imaging modalities such as immunohistochemical staining may help improve detection accuracy. Moreover, although the distilled model reduces computational demands, integrating advanced deep learning models into clinical practice requires addressing technological gaps and potential resistance to adopting new tools within established diagnostic processes.
\end{sloppypar}

Future work could focus on several key areas to refine the proposed approach and facilitate its adoption in clinical environments. Enhancing the cell-relabeling process with additional datasets \cite{Graham_Jahanifar_etal._2021} could improve the representation of underrepresented cell types and enhance overall model performance. Also, incorporating additional data sources, such as multi-modal imaging or complementary staining methods, may address limitations related to cell type differentiation and class imbalance. Exploring other foundation models \cite{Vorontsov_Bozkurt_etal._2024,Zimmermann_Vorontsov_etal._2024} or introducing additional modalities \cite{Ding_Wagner_etal._2024,Vaidya_Zhang_etal._2025} may provide alternative architectures better suited to specific tasks or offer improved efficiency. Implementing more complex knowledge distillation techniques \cite{Houyon_Cioppa_etal._2023,Zhang_Song_etal._2019} could further optimize the model's performance and adaptability. Additionally, deeper integration with QuPath or other digital pathology software could provide pathologists more control over cell quantification analysis directly within the QuPath interface, thereby increasing accessibility and usability. Such enhancements would not only refine model performance but also ensure greater adaptability and scalability within various clinical environments. Finally, extensive validation of the model by pathologists and benchmarking against independent datasets are essential steps toward establishing the model's reliability and fostering confidence in its clinical utility.

\section*{Acknowledgments} 
This work was funded in part by the Research Council of Norway grant no. 309439 SFI Visual Intelligence, and the North Norwegian Health Authority grant no. HNF1521-20.

\bibliographystyle{IEEEtran}
\begin{sloppypar}
\begin{thebibliography}{99}

\bibitem{chaplot2020neural} Chaplot, Devendra Singh, et al. "Neural topological slam for visual navigation." Proceedings of the IEEE/CVF conference on computer vision and pattern recognition. 2020.

\bibitem{maksymets2021thda} Maksymets, Oleksandr, et al. "Thda: Treasure hunt data augmentation for semantic navigation." Proceedings of the IEEE/CVF International Conference on Computer Vision. 2021.

\bibitem{mezghan2022memory} Mezghan, Lina, et al. "Memory-augmented reinforcement learning for image-goal navigation." 2022 IEEE/RSJ International Conference on Intelligent Robots and Systems (IROS). IEEE, 2022.

\bibitem{al2022zero} Al-Halah, Ziad, Santhosh Kumar Ramakrishnan, and Kristen Grauman. "Zero experience required: Plug \& play modular transfer learning for semantic visual navigation." Proceedings of the IEEE/CVF Conference on Computer Vision and Pattern Recognition. 2022.

\bibitem{ye2021auxiliary} Ye, Joel, et al. "Auxiliary tasks and exploration enable objectgoal navigation." Proceedings of the IEEE/CVF international conference on computer vision. 2021.

\bibitem{chaplot2020object} Chaplot, Devendra Singh, et al. "Object goal navigation using goal-oriented semantic exploration." Advances in Neural Information Processing Systems 33 (2020)

\bibitem{ramakrishnan2022poni} Ramakrishnan, Santhosh Kumar, et al. "Poni: Potential functions for objectgoal navigation with interaction-free learning." Proceedings of the IEEE/CVF Conference on Computer Vision and Pattern Recognition. 2022.

\bibitem{ramrakhya2022habitat} Ramrakhya, Ram, et al. "Habitat-web: Learning embodied object-search strategies from human demonstrations at scale." Proceedings of the IEEE/CVF Conference on Computer Vision and Pattern Recognition. 2022.

\bibitem{mousavian2019visual} Mousavian, Arsalan, et al. "Visual representations for semantic target driven navigation." 2019 International Conference on Robotics and Automation (ICRA). IEEE, 2019.

\bibitem{dhariwal2021diffusion} Dhariwal, Prafulla, and Alexander Nichol. "Diffusion models beat gans on image synthesis." Advances in neural information processing systems 34 (2021)

\bibitem{ho2022classifier} Ho, Jonathan, and Tim Salimans. "Classifier-free diffusion guidance." arXiv preprint arXiv:2207.12598 (2022).

\bibitem{nichol2021glide} Nichol, Alex, et al. "Glide: Towards photorealistic image generation and editing with text-guided diffusion models." arXiv preprint arXiv:2112.10741 (2021)

\bibitem{brooks2023instructpix2pix} Brooks, Tim, Aleksander Holynski, and Alexei A. Efros. "Instructpix2pix: Learning to follow image editing instructions." Proceedings of the IEEE/CVF Conference on Computer Vision and Pattern Recognition. 2023.

\bibitem{fu2023guiding} Fu, Tsu-Jui, et al. "Guiding instruction-based image editing via multimodal large language models." arXiv preprint arXiv:2309.17102 (2023).

\bibitem{geng2024instructdiffusion} Geng, Zigang, et al. "Instructdiffusion: A generalist modeling interface for vision tasks." Proceedings of the IEEE/CVF Conference on Computer Vision and Pattern Recognition. 2024.

\bibitem{zhou2024minedreamer} Zhou, Enshen, et al. "Minedreamer: Learning to follow instructions via chain-of-imagination for simulated-world control." arXiv preprint arXiv:2403.12037 (2024).

\bibitem{zhou2023esc} Zhou, Kaiwen, et al. "Esc: Exploration with soft commonsense constraints for zero-shot object navigation." International Conference on Machine Learning. PMLR, 2023.

\bibitem{yu2023l3mvn} Yu, Bangguo, Hamidreza Kasaei, and Ming Cao. "L3mvn: Leveraging large language models for visual target navigation." 2023 IEEE/RSJ International Conference on Intelligent Robots and Systems (IROS). IEEE, 2023.

\bibitem{gadre2023cows} Gadre, Samir Yitzhak, et al. "Cows on pasture: Baselines and benchmarks for language-driven zero-shot object navigation." Proceedings of the IEEE/CVF Conference on Computer Vision and Pattern Recognition. 2023.

\bibitem{shah2023navigation} Shah, Dhruv, et al. "Navigation with large language models: Semantic guesswork as a heuristic for planning." Conference on Robot Learning. PMLR, 2023.

\bibitem{cai2024bridging} Cai, Wenzhe, et al. "Bridging zero-shot object navigation and foundation models through pixel-guided navigation skill." 2024 IEEE International Conference on Robotics and Automation (ICRA). IEEE, 2024.

\bibitem{yu2023co} Yu, Bangguo, Hamidreza Kasaei, and Ming Cao. "Co-NavGPT: Multi-robot cooperative visual semantic navigation using large language models." arXiv preprint arXiv:2310.07937 (2023).

\bibitem{wu2024voronav} Wu, Pengying, et al. "Voronav: Voronoi-based zero-shot object navigation with large language model." arXiv preprint arXiv:2401.02695 (2024).

\bibitem{qin2023mp5} Qin, Yiran, et al. "Mp5: A multi-modal open-ended embodied system in minecraft via active perception." arXiv preprint arXiv:2312.07472 (2023).

\bibitem{du2024learning} Du, Yilun, et al. "Learning universal policies via text-guided video generation." Advances in Neural Information Processing Systems 36 (2024).

\bibitem{ajay2024compositional} Ajay, Anurag, et al. "Compositional foundation models for hierarchical planning." Advances in Neural Information Processing Systems 36 (2024).

\bibitem{liang2024skilldiffuser} Liang, Zhixuan, et al. "Skilldiffuser: Interpretable hierarchical planning via skill abstractions in diffusion-based task execution." Proceedings of the IEEE/CVF Conference on Computer Vision and Pattern Recognition. 2024.

\bibitem{heusel2017gans} Heusel, Martin, et al. "Gans trained by a two time-scale update rule converge to a local nash equilibrium." Advances in neural information processing systems 30 (2017).

\bibitem{zhang2018unreasonable} Zhang, Richard, et al. "The unreasonable effectiveness of deep features as a perceptual metric." Proceedings of the IEEE conference on computer vision and pattern recognition. 2018.

\bibitem{brown2020language} Brown, Tom B. "Language models are few-shot learners." arXiv preprint arXiv:2005.14165 (2020).

\bibitem{podell2023sdxl} Podell, Dustin, et al. "Sdxl: Improving latent diffusion models for high-resolution image synthesis." arXiv preprint arXiv:2307.01952 (2023).

\bibitem{brohan2022rt} Brohan, Anthony, et al. "Rt-1: Robotics transformer for real-world control at scale." arXiv preprint arXiv:2212.06817 (2022).

\bibitem{brohan2023rt} Brohan, Anthony, et al. "Rt-2: Vision-language-action models transfer web knowledge to robotic control." arXiv preprint arXiv:2307.15818 (2023).

\bibitem{li2024manipllm} Li, Xiaoqi, et al. "Manipllm: Embodied multimodal large language model for object-centric robotic manipulation." Proceedings of the IEEE/CVF Conference on Computer Vision and Pattern Recognition. 2024.

\bibitem{shah2023vint} Shah, Dhruv, et al. "ViNT: A foundation model for visual navigation." arXiv preprint arXiv:2306.14846 (2023).

\bibitem{liu2024visual} Liu, Haotian, et al. "Visual instruction tuning." Advances in neural information processing systems 36 (2024).

\bibitem{hu2021lora} Hu, Edward J., et al. "Lora: Low-rank adaptation of large language models." arXiv preprint arXiv:2106.09685 (2021).

\bibitem{qin2023supfusion} Qin, Yiran, et al. "SupFusion: Supervised LiDAR-camera fusion for 3D object detection." Proceedings of the IEEE/CVF International Conference on Computer Vision. 2023.

\bibitem{qin2024worldsimbench} Qin, Yiran, et al. "Worldsimbench: Towards video generation models as world simulators." arXiv preprint arXiv:2410.18072 (2024).

\bibitem{yu2025gamefactory} Yu, Jiwen, et al. "GameFactory: Creating New Games with Generative Interactive Videos." arXiv preprint arXiv:2501.08325 (2025).

\bibitem{zhou2024code} Zhou, Enshen, et al. "Code-as-Monitor: Constraint-aware Visual Programming for Reactive and Proactive Robotic Failure Detection." arXiv preprint arXiv:2412.04455 (2024).

\bibitem{zhang2024ad} Zhang, Zaibin, et al. "AD-H: Autonomous Driving with Hierarchical Agents." arXiv preprint arXiv:2406.03474 (2024).

\bibitem{wang2024toward} Wang, Chaoqun, et al. "Toward Accurate Camera-based 3D Object Detection via Cascade Depth Estimation and Calibration." arXiv preprint arXiv:2402.04883 (2024).

\bibitem{huang2024story3d} Huang, Yuzhou, et al. "Story3d-agent: Exploring 3d storytelling visualization with large language models." arXiv preprint arXiv:2408.11801 (2024).

\bibitem{savinov2018semi} Savinov, Nikolay, Alexey Dosovitskiy, and Vladlen Koltun. "Semi-parametric topological memory for navigation." arXiv preprint arXiv:1803.00653 (2018).

\bibitem{majumdar2022zson} Majumdar, Arjun, et al. "Zson: Zero-shot object-goal navigation using multimodal goal embeddings." Advances in Neural Information Processing Systems 35 (2022): 32340-32352.

\bibitem{yadav2023offline} Yadav, Karmesh, et al. "Offline visual representation learning for embodied navigation." Workshop on Reincarnating Reinforcement Learning at ICLR 2023. 2023.

\bibitem{yadav2023ovrl} Yadav, Karmesh, et al. "Ovrl-v2: A simple state-of-art baseline for imagenav and objectnav." arXiv preprint arXiv:2303.07798 (2023).

\bibitem{sun2024fgprompt} Sun, Xinyu, et al. "FGPrompt: fine-grained goal prompting for image-goal navigation." Advances in Neural Information Processing Systems 36 (2024).

\bibitem{zhu2017target} Zhu, Yuke, et al. "Target-driven visual navigation in indoor scenes using deep reinforcement learning." 2017 IEEE international conference on robotics and automation (ICRA). IEEE, 2017.

\bibitem{koh2024generating} Koh, Jing Yu, Daniel Fried, and Russ R. Salakhutdinov. "Generating images with multimodal language models." Advances in Neural Information Processing Systems 36 (2024).

\bibitem{krantz2022instance} Krantz, Jacob, et al. "Instance-specific image goal navigation: Training embodied agents to find object instances." arXiv preprint arXiv:2211.15876 (2022).

\bibitem{schulman2017proximal} Schulman, John, et al. "Proximal policy optimization algorithms." arXiv preprint arXiv:1707.06347 (2017).

\bibitem{anderson2018evaluation} Anderson, Peter, et al. "On evaluation of embodied navigation agents." arXiv preprint arXiv:1807.06757 (2018).

\bibitem{lin2024navcot} Lin, Bingqian, et al. "NavCoT: Boosting LLM-Based Vision-and-Language Navigation via Learning Disentangled Reasoning." arXiv preprint arXiv:2403.07376 (2024).

\bibitem{NavGPT} Zhou, Gengze, Yicong Hong, and Qi Wu. "Navgpt: Explicit reasoning in vision-and-language navigation with large language models." Proceedings of the AAAI Conference on Artificial Intelligence.

\bibitem{hahn2021no} Hahn, Meera, et al. "No rl, no simulation: Learning to navigate without navigating." Advances in Neural Information Processing Systems 34 (2021): 26661-26673.

\bibitem{li2025t2isafety} Li, Lijun, et al. "T2ISafety: Benchmark for Assessing Fairness, Toxicity, and Privacy in Image Generation." arXiv preprint arXiv:2501.12612 (2025).

\bibitem{an2024agfsync} An, Jingkun, et al. "AGFSync: Leveraging AI-Generated Feedback for Preference Optimization in Text-to-Image Generation." arXiv preprint arXiv:2403.13352 (2024).


\end{thebibliography}
\end{sloppypar}

\clearpage
\beginsupplement
\section*{Appendix}
\renewcommand{\thesubsection}{S\arabic{subsection}}

\subsection{\label{chap:S1}PanNuke and MoNuSAC preprocessing}
The PanNuke dataset comprises a set of 7,901 RGB patches, each with dimensions of $256 \times 256$ pixels, which we set as the standard patch size for our analysis. In contrast, the MoNuSAC dataset encompasses 294 images of heterogeneous dimensions. To standardize the MoNuSAC images with our experiments, we implement a standardization protocol. Specifically, for images exceeding the dimensions of $256 \times 256$ pixels, we segment them into equal-sized patches and apply mirror padding to the remaining portions to avoid information loss at the peripherals. Patches with dimensions less than $128 \times 128$ pixels are excluded from the dataset due to the insufficient resolution to capture relevant cellular details. For patches where either dimension falls between 128 and 256 pixels, we employ upsampling to achieve the standard patch size. As a result, we obtain a total of 2,823 RGB patches derived from the MoNuSAC dataset for subsequent analysis. For additional details on the MoNuSAC data preparation process, refer to the source code \cite{Shvetsov_2025a}.
\clearpage

\subsection{\label{chap:S2}Data usage for the methodology}

\counterwithin{figure}{subsection}
\renewcommand{\thefigure}{S\arabic{subsection}}

\begin{figure}[h!]
    \centering
    \includegraphics[width=\textwidth, height=0.85\textheight, keepaspectratio]{images/A2.pdf}
    \caption{Overview of the methodology for cross-labeling, dataset refinement, and model comparison. (1) Cross-relabeling - training and testing cell classification models, (2) Cross-relabeling - using cell classification models to create refined dataset, (3) Fine-tuning and training models for comparison, (4) Student knowledge distillation with refined dataset}
    \label{fig:S2}
\end{figure}
\clearpage

\subsection{\label{chap:S3}Confusion matrices for classification models}
\counterwithin{figure}{subsection}
\renewcommand{\thefigure}{S\arabic{subsection}.\arabic{figure}}

\begin{figure}[h!]
    \centering
    \includegraphics[width=\textwidth, height=0.4\textheight, keepaspectratio]{images/A3_1.pdf}
    \caption{Confusion matrix for PanNuke trained model}
    \label{fig:S3.1}
\end{figure}

\begin{figure}[h!]
    \centering
    \includegraphics[width=\textwidth, height=0.4\textheight, keepaspectratio]{images/A3_2.pdf}
    \caption{Confusion matrix for MoNuSAC trained model}
    \label{fig:S3.2}
\end{figure}

\clearpage

\subsection{\label{chap:S4}Datasets cell counts}

\counterwithin{table}{subsection}
\renewcommand{\thetable}{S\arabic{subsection}}

\begin{table}[h!]
\renewcommand{\arraystretch}{2.0}
\centering
\caption{\label{tab:S4}Cell counts for PanNuke, MoNuSAC and refined datasets. Numbers in parentheses indicate preprocessed cell counts for cell classifier models training and testing.}
%\adjustbox{max width=\textwidth}{%
\begin{tabular}{|l|c|c|c|}
\hline
%\rowcolor{gray!30}
Cell type & PanNuke & MoNuSAC & Refined \\
\hline
Neoplastic & 77,403 (68,031) & - & 105,451 \\
\hline
Epithelial & 26,572 (23,207) & - & 29,926 \\
\hline
Epithelial (benign and malignant) & - & 31,402 & - \\
\hline
Inflammatory & 32,276 & - & - \\
\hline
Lymphocytes & - & 37,045 (33,104) & 65,275 \\
\hline
Neutrophils & - & 1,355 (1,252) & 3,833 \\
\hline
Macrophage & - & 1,842 (1,695) & 3,410 \\
\hline
Dead & 2,908 & - & 2,908 \\
\hline
Connective & 50,585 & - & 50,585 \\
\hline
\end{tabular}
%
%}
\end{table}



\clearpage

\subsection{\label{chap:S5}Definition of validation metrics}
\counterwithin{equation}{subsection}
\renewcommand{\theequation}{\arabic{equation}}

\subsubsection{\label{chap:S5.1}R\textsuperscript{2}}
The coefficient of determination, denoted as $R^2$, is a statistical measure that represents the proportion of variance in the dependent variable that is predictable from the independent variables. In the context of cell quantification in pathology, $R^2$ is used to assess how well the predicted quantities of different cell types in a patch align with the actual quantities observed in the ground truth data, with higher values representing more accurate quantification. $R^2$ is defined as
\begin{equation*}
R^2 = 1 - \frac{\sum_{i=1}^n (y_i - \hat{y}_i)^2}{\sum_{i=1}^n (y_i - \bar{y})^2},
\end{equation*}
where $y_i$ represents the actual number of cells of a specific type in the $i$-th image, $\hat{y}_i$ represents the predicted number of cells of that type in the $i$-th image, $\bar{y}$ is the mean of the actual numbers across all images, and $n$ is the total number of images in the dataset.

The $R^2$ metric has a range of $(-\infty, 1]$. An $R^2$ of 1 indicates perfect prediction, where all predicted values exactly match the actual values. An $R^2$ of 0 suggests that the model explains none of the variability of the response data around its mean. If $R^2$ is negative, it indicates that the model performs worse than a model that simply predicts the mean of the actual values for all observations.

\subsubsection{\label{chap:S5.2}PQ}
Panoptic Quality ($PQ$) is a comprehensive metric used to evaluate the performance of segmentation models in tasks that require both instance segmentation and classification. $PQ$ provides a single score that encapsulates both the detection accuracy (i.e., how many objects were correctly identified) and the segmentation quality (i.e., how accurately the objects' boundaries were delineated). This metric is particularly useful in multiclass scenarios where each pixel is classified into distinct categories, such as different cell types in pathology images.

$PQ$ is calculated as the product of two terms: Detection Quality ($DQ$) and Segmentation Quality ($SQ$). It can be expressed as
\begin{equation*}
PQ = DQ \cdot SQ,
\end{equation*}
where
\begin{equation*}
DQ = \frac{TP}{TP + 0.5\, FP + 0.5\, FN},
\end{equation*}
\begin{equation*}
SQ = \frac{\sum_{(p, g) \in \mathcal{M}} IoU(p, g)}{TP}.
\end{equation*}
In these formulas, $TP$ denotes the number of correctly matched instances between ground truth and prediction, $FP$ denotes the predicted instances that have no corresponding ground truth, $FN$ denotes the ground truth instances that were not detected, $IoU(p, g)$ is the Intersection over Union for a pair of matched instances $p$ (prediction) and $g$ (ground truth), and $\mathcal{M}$ is the set of matched pairs.

The $PQ$ metric is calculated for each class and is averaged across classes to provide a global performance measure.

The $PQ$ score has a range of $[0, 1.0]$, where a higher score indicates better performance in both detecting and segmenting the instances correctly. A $PQ$ of 1 signifies perfect identification and segmentation of all instances, whereas a $PQ$ of 0 indicates that no instances were correctly identified and segmented.

\clearpage

\subsection{\label{chap:S6}Segmentation and Detection quality metrics for teacher and student models}

\begin{table}[h!]
\renewcommand{\arraystretch}{2.0}
\centering
\caption{Segmentation and detection quality for student and teacher models (CI 95\%)}
\label{tab:S6}
%\adjustbox{max width=\textwidth}{%
\begin{tabular}{|l|c|c|}
\hline
%\rowcolor{gray!30}
Metric & Teacher & Student \\
\hline
$SQ_{neoplastic}$ & 0.819 (0.815--0.823) & 0.824 (0.819--0.828) \\
\hline
$SQ_{lymphocyte}$ & 0.795 (0.788--0.802) & 0.790 (0.783--0.796) \\
\hline
$SQ_{connective}$ & 0.770 (0.762--0.776) & 0.780 (0.772--0.786) \\
\hline
$SQ_{dead}$ & 0.659 (0.623--0.688) & 0.657 (0.624--0.695) \\
\hline
$SQ_{epithelial}$ & 0.780 (0.770--0.790) & 0.788 (0.779--0.797) \\
\hline
$SQ_{macrophage}$ & 0.788 (0.760--0.810) & 0.757 (0.730--0.783) \\
\hline
$SQ_{neutrofil}$ & 0.782 (0.761--0.801) & 0.775 (0.759--0.792) \\
\hline
$DQ_{neoplastic}$ & 0.706 (0.692--0.719) & 0.727 (0.712--0.741) \\
\hline
$DQ_{lymphocyte}$ & 0.675 (0.656--0.698) & 0.713 (0.691--0.734) \\
\hline
$DQ_{connective}$ & 0.566 (0.546--0.584) & 0.583 (0.565--0.602) \\
\hline
$DQ_{dead}$ & 0.410 (0.361--0.465) & 0.435 (0.306--0.561) \\
\hline
$DQ_{epithelial}$ & 0.668 (0.639--0.694) & 0.673 (0.644--0.702) \\
\hline
$DQ_{macrophage}$ & 0.657 (0.583--0.727) & 0.615 (0.531--0.703) \\
\hline
$DQ_{neutrofil}$ & 0.691 (0.625--0.753) & 0.729 (0.679--0.778) \\
\hline
\end{tabular}
%
%}
\end{table}

\clearpage

\subsection{\label{chap:S7}QuPath integration method}
We adopt an integration strategy leveraging the paquo \cite{Bayer_AG} library, a Python package that enables direct interaction with QuPath’s internal API, thereby facilitating seamless data exchange without intermediate conversion steps. The data processing pipeline (\hyperref[fig:S7]{Appendix Figure S7}) begins with the acquisition of WSIs and their associated annotations from QuPath, which are represented as Shapely \cite{Gillies_Wel_etal._2024} polygons. Utilizing paquo, we directly read, create, and modify these annotations and detections within a QuPath project in the Python environment. Images are then cropped using these polygons and processed by cell segmentation and classification models employing standard vision processing toolkits such as OpenCV, pyvips, and PyTorch. Additionally, QuPath employs Groovy scripts to initiate a Python process that starts the entire pipeline from QuPath graphical interface: fetching polygons, extracting images from them, and running deep learning model inference on the cropped images. 
The results are returned to QuPath, leveraging paquo's Python bindings to manipulate QuPath data while minimizing the computational overhead typically associated with cross-environment communication.

\counterwithin{figure}{subsection}
\renewcommand{\thefigure}{S\arabic{subsection}}

\begin{figure}[h!]
    \centering
    \includegraphics[width=\textwidth]{images/A7.pdf}
    \caption{QuPath integration workflow using Python environment}
    \label{fig:S7}
\end{figure}

Compared to traditional workflows that involve exporting annotations as GeoJSON, classifying them in Python, and reimporting them into QuPath, our approach offers several advantages. We eliminate the need to switch between programming languages, providing a cohesive and streamlined development process entirely within QuPath software and removing the necessity to use other tools. Meanwhile, we avoid storing annotations as intermediate JSON files unless required for external use or archiving. By conducting the entire inference and post-processing workflow within the Python environment, we leverage the power and flexibility of Python libraries for image processing and machine learning. This approach also enables adjustments to any set of labels and models, thereby improving its applicability.

%\hfill

The distilled model and QuPath integration code are packaged into a Docker container, enabling streamlined execution with the Docker engine. Detailed integration code and deployment instructions can be found in the GitHub repository \cite{Shvetsov_2025b}.

Despite these benefits, we acknowledge that the paquo library is a proof‑of‑concept project in its early development stage and has not been tested across all versions of QuPath.

\clearpage

\subsection{\label{chap:S8}Data and code availability statement}
All datasets, models, and code used in this study are publicly available and can be obtained from the repositories listed below. 
The PanNuke \cite{Gamper_Koohbanani_etal._2019} and MoNuSAC \cite{Verma_Kumar_etal._2021} datasets are publicly accessible, and download information along with detailed descriptions can be found in their respective articles. Preprocessing scripts for PanNuke and MoNuSAC data, as well as individual cell extraction scripts, are available on GitHub \cite{Shvetsov_2025a}. The H-Optimus foundation model used in our experiments can be downloaded from the HuggingFace repository \cite{hoptimus2024}, and model information is available on GitHub \cite{Saillard_Jenatton_etal._2024}. In addition, the integration code for QuPath and the distilled model packaged in a Docker container are provided in the repository \cite{Shvetsov_2025b}, and paquo Python library is available from the authors GitHub repository \cite{Bayer_AG}.
\clearpage

\end{document}

%\bibliography{aesthetic,background}

\end{document}
