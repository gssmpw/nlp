% 3.0%\documentclass[journal,comsoc]{IEEEtran}
%\documentclass[10pt,journal,compsoc]{IEEEtran}


\documentclass[lettersize,journal]{IEEEtran}
\usepackage{cite}
%\fi
%\else
%\fi
\usepackage[T1]{fontenc}% optional T1 font encoding
\usepackage{bm}
\usepackage{amsmath}
\interdisplaylinepenalty=2500


\hyphenation{op-tical net-works semi-conduc-tor}
\usepackage[table,xcdraw]{xcolor}

\DeclareMathOperator{\Cov}{Cov}
\usepackage{amsthm}
%\usepackage{subcaption}
\usepackage{float}
\usepackage{setspace}
\usepackage{amssymb}
\usepackage{stfloats}
\usepackage{cite}
\usepackage{ragged2e}
% \usepackage{algorithm}
\usepackage[ruled,vlined,linesnumbered]{algorithm2e}
% \usepackage{algorithmic}
\usepackage{amsfonts}
\usepackage{mathrsfs}
\usepackage{amsmath,amsthm}
\usepackage{array,booktabs}
\usepackage{subfigure}
\usepackage{multirow}
\usepackage{cuted}
\usepackage{multicol}
\usepackage{graphicx}
\usepackage{subfigure}
\usepackage{graphicx,xcolor,bm}
\usepackage{hyperref}
\usepackage{threeparttable}
\usepackage{dcolumn}
\usepackage{setspace}
\usepackage{makecell}
\usepackage{lipsum}
\usepackage{enumerate}
\usepackage{mathrsfs}
\usepackage{bbm}


%%%%%%% ADDED BY ALI
\usepackage[protrusion=true,expansion=true]{microtype}
\pdfoutput=1
%%%%%%%%



\newtheorem{myDef}{Definition}
\newtheorem{myLam}{Lemma}
\newtheorem*{myPro}{Proof}
\newtheorem{Defn}{Definition}
\newtheorem{lem}{Lemma}
\newtheorem{col}{Corollary}
\newtheorem{Prop}{Proposition}
\newtheorem{thm}{Theorem}
\newtheorem{rek}{Remark}
\linespread{0.955}
\usepackage{xfrac}
%\usepackage{ftnright}
%\pagecolor[rgb]{0.9, 0.99, 0.9}
%\usepackage{minipage}
\DeclareMathOperator*{\argmax}{argmax}
\renewcommand{\qedsymbol}{$\blacksquare$}
\usepackage{cite,bm}
\graphicspath{{figures/}}
%\allowdisplaybreaks
%\linespread{0.965}
\hyphenation{op-tical net-works semi-conduc-tor IEEE-Xplore}
\def\BibTeX{{\rm B\kern-.05em{\sc i\kern-.025em b}\kern-.08em
		T\kern-.1667em\lower.7ex\hbox{E}\kern-.125emX}}
\usepackage{balance}
\setlength{\skip\footins}{3.5pt}
\setlength{\footnotesep}{0pc}
\everymath{\footnotesize} % Reduce font size of inline equations
\everydisplay{\small} % Reduce font size of display equations
\setstretch{0.94}
\usepackage{nicefrac}
\allowdisplaybreaks
\begin{document}
\setlength{\abovedisplayskip}{2pt}
\setlength{\belowdisplayskip}{2pt}
\newcommand\semiHuge{\fontsize{21.7}{31.38}\selectfont}
	\title{\semiHuge Future Resource Bank for ISAC: Achieving Fast and Stable Win-Win Matching for Both Individuals and Coalitions	\vspace{-1mm}}
\author{Houyi Qi, Minghui Liwang, \IEEEmembership{Member}, \IEEEmembership{IEEE}, Seyyedali Hosseinalipour, \IEEEmembership{Member}, \IEEEmembership{IEEE}, \\Liqun Fu, \IEEEmembership{Senior Member}, \IEEEmembership{IEEE}, Sai Zou, \IEEEmembership{Senior Member}, \IEEEmembership{IEEE}, and Wei Ni, \IEEEmembership{Fellow}, \IEEEmembership{IEEE}
 \vspace{-4.5mm}
	\thanks{H. Qi (qihouyi@stu.xmu.edu.cn) and L. Fu (liqun@xmu.edu.cn)
		are with  School of Informatics, Xiamen University, Fujian, China. M. Liwang (minghuiliwang@tongji.edu.cn) is with  Department of Control Science and Engineering and  Shanghai Research Institute for Intelligent Autonomous Systems, Tongji University, Shanghai, China. 
		S. Hosseinalipour (alipour@buffalo.edu) is with Department of Electrical Engineering, University at Buffalo--SUNY, NY, USA.
%		X. Wang (xianbin.wang@uwo.ca) is with the Department of Electrical and Computer Engineering, Western University, Ontario, Canada. 
		Sai Zou (drzousai@foxmail.com) is with  College of Big Data and Information Engineering, Guizhou University. Wei Ni (Wei.Ni@ieee.org) is with Data61, CSIRO. (Corresponding author: M. Liwang)}}

	
	\IEEEtitleabstractindextext{
		\begin{abstract}
			\justifying
Future wireless networks must support emerging applications where environmental awareness is as critical as data transmission. Integrated Sensing and Communication (ISAC) enables this vision by allowing base stations (BSs) to allocate bandwidth and power to mobile users (MUs) for communications and cooperative sensing. However, this resource allocation is highly challenging due to: \textit{(i)} dynamic resource demands from MUs and resource supply from BSs, and \textit{(ii)} the selfishness of MUs and BSs. To address these challenges, existing solutions rely on either real-time (online) resource trading, which incurs high overhead and failures, or static long-term (offline) resource contracts, which lack flexibility. To overcome these limitations, we propose the \textit{Future Resource Bank for ISAC}, a hybrid trading framework that integrates offline and online resource allocation through a level-wise client model, where MUs and their coalitions negotiate with BSs. We introduce two mechanisms: \textit{(i)} Offline Role-Friendly Win-Win Matching (offRFW$^2$M), leveraging overbooking to establish risk-aware, stable contracts, and \textit{(ii)} Online Effective Backup Win-Win Matching (onEBW$^2$M), which dynamically reallocates unmet demand and surplus supply. We theoretically prove stability, individual rationality, and weak Pareto optimality of these mechanisms. Through simulations, we show that our framework improves social welfare, latency, and energy efficiency compared to existing methods.
		\end{abstract}
		
		\vspace{-1.85mm}
		% Note that keywords are not normally used for peerreview papers.
		\begin{IEEEkeywords}
			Integrated Sensing and Communication (ISAC), Resource Trading, Stable Matching, Collaborative Sensing.
		\end{IEEEkeywords}
}
	
	%}
\maketitle
 \IEEEdisplaynontitleabstractindextext

\IEEEpeerreviewmaketitle

	\vspace{-3.5mm}
\section{Introduction}
	\vspace{-.5mm}
\IEEEPARstart{T}{he} rapid advancement of wireless communications and artificial intelligence (AI) is driving innovations in autonomous driving, smart manufacturing, and remote healthcare, increasing the demand for real-time environmental awareness\cite{SURVEY 1}. Integrated Sensing and Communication (ISAC) has emerged as a key enabler of these innovations, seamlessly integrating sensing, positioning, and communication to support these complex applications\cite{SURVEY 2, SURVEY 3}.
Despite its transformative potential, ISAC faces critical resource allocation challenges. A key obstacle is the need to meet diverse and stringent requirements of various applications despite limited resource availability. Also, the inherently dynamic nature of ISAC networks, driven by mobile devices and evolving service demands, calls for frequent interactions between base stations (BSs), acting as resource providers, and mobile users (MUs), serving as requestors. These continuous interactions impose substantial overhead,  complicating the delivery of communication and sensing services. 
Moreover, the selfish behavior of BSs and MUs complicates coordination, as both parties prioritize individual gains over overall network efficiency\cite{SURVEY 1}. To address these challenges, integration of a \textit{resource trading market} within ISAC networks is promising. By incorporating economic incentives, such as pricing and trading mechanisms\cite{price1,price2,price3}, this trading-based approach can foster a mutually beneficial ecosystem, where providers (sellers) maximize their revenues while requestors (buyers) optimize their service-related profits.


% Moreover, the self-interested behavior of BSs and MUs further complicates coordination, as both prioritize individual gains over overall network efficiency. This misalignment leads to inefficient resource allocation and service disruptions. A promising solution lies in integrating a resource trading market within ISAC networks. By incorporating economic incentives such as pricing and trading mechanisms, this approach encourages resource providers (sellers) to maximize revenue while enabling requestors (buyers) to optimize service-related profits, fostering an efficient and balanced ecosystem.

%Beyond these challenges, selfish behaviors of both resource providers and requestors make coordination even more difficult, as BSs and MUs often prioritize their individual gains rather than overall network efficiency\cite{SURVEY 1}. This misalignment can result in inefficient resource allocation, service disruptions, and missed opportunities for collaboration. To address these challenges, integration of a resource trading market within ISAC networks is promising. By incorporating economic incentives, as discussed in\cite{price1,price2,price3}, which integrate economic concepts like pricing and trading together with the allocation of communication resources, such a market can motivate resource providers (sellers) to optimize their revenue while enabling resource requestors (buyers) to actively participate and maximize their service-related profits. This trading-based approach fosters a mutually beneficial ecosystem, where sellers enhance their revenue, buyers secure essential resources, and the overall network operates efficiently. 
%Nevertheless, realizing this market requires overcoming significant challenges, as we discuss next.

\vspace{-3mm}
\subsection{Challenges and Motivations}
While market-driven ISAC networks hold promise, their practical design present several unresolved challenges. To address them, we identify key research questions that shape our approach and define the core motivations driving this work.

\textit{Research Question 1: What is Online Resource Trading in A Market-Driven Network, 	and How Can We Overcome Its Limitations?}
Online resource trading, widely adopted in existing studies \cite{Spot 1,Spot 2,Spot 3}, refers to a real-time procedure where buyers and sellers establish agreements in real-time based on current resource supply/demand and network conditions. Despite being flexible and adaptive, this approach has notable drawbacks in dynamic ISAC environments.
A key limitation is its high operational overhead \cite{future 1,future 2}, as negotiating resource quantities, pricing, and agreements consumes time and resources, diverting attention from actual service delivery. Also, fluctuating resource availability increases the risk of buyers failing to secure services despite prolonged negotiations \cite{future 1}.

To mitigate these limitations, a promising alternative is offline trading, which leverages historical/predictive data to establish long-term contracts, reducing the overhead of real-time decisions\cite{RW Matching3}. These contracts improve time efficiency by enabling instant execution when needed. However, a key challenge remains: ensuring their applicability in time-varying ISAC networks, leading to our next research question.

\textit{Research Question 2: Can Offline Trading Fit into Dynamic ISAC Networks? If Not, How Can Its Advantages Be Integrated into a Broader Framework for Greater Adaptability?}
Offline trading involves pre-signing long-term contracts ahead of actual resource allocation. However, ISAC networks are inherently dynamic and subject to uncertainties arising from mobile device movement, time-varying channel conditions, and fluctuating resource supply and demand. These factors can limit the effectiveness of pre-established/long-term contracts\cite{future 1}.
To address this, one solution is to assess key sources of uncertainty in ISAC networks, and then employ advanced risk management techniques to ensure that these contracts remain viable. 
Additionally, overbooking\footnote{Overbooking is a widely used economic strategy in airlines \cite{overbook1}, hotels \cite{overbook2}, and telecommunications \cite{overbook3} industries that addresses the limitations of booking methods in handling fluctuating resource demand and supply.} --- a widely used strategy in other industries\cite{RW Matching3}--- represents a promising tool in resource allocation. It allows sellers to allocate more resources than their theoretical supply, thereby improving resource availability and mitigating fluctuations in supply and demand. Subsequently, \textit{combining online and risk-controlled, overbooking-empowered offline resource allocation in ISAC networks can form a hybrid market with significant advantages.}

Nevertheless, when offline and online modes coexist, a significant challenge arises: the overlapping service demands across buyers can complicate resource allocation and create redundancies, e.g., multiple buyers may target the same sensing objective (e.g., air quality monitoring or hazard detection). Therefore, efficient coordination --- whether through competition or collaboration --- is crucial for optimizing resource utilization. Further complicating this issue is the lack of well-defined revenue distribution mechanisms in collaborative settings. In particular, if buyers form sensing coalitions to share resources, ensuring fair cost allocation and incentive structures becomes a critical yet complex issue. These considerations make designing a framework that balances cooperation, fairness, and efficiency while adapting to dynamic ISAC conditions an open research problem, leading to our next research question.

\textit{Research Question 3: In a Dynamic ISAC Network, Should Resource Buyers Compete for Resources Individually or Collaborate for Mutual Benefits?}
In a resource trading market over ISAC networks, maximizing utility for both buyers and sellers while ensuring efficient resource utilization is essential. Existing efforts primarily focus on individual MUs acquiring resources (e.g., bandwidth and power) from BSs to fulfill their communication and sensing demands \cite{ISAC}. However, in many scenarios, multiple MUs share the same sensing objectives, where performing separate sensing tasks results in redundancy and inefficient resource utilization. To this end, in this work, we propose \textit{sensing coalitions} allowing for MUs with identical sensing targets to form cooperative groups: instead of competing for resources individually, a coalition collectively requests a sensing service from BSs. The sensing results can then be shared among all coalition members. This cooperation benefits both MUs and BSs: MUs receive higher-quality sensing services, while BSs optimize resource utilization by handling aggregated requests rather than redundant demands, establishing a win-win paradigm for profit enhancement for both parties.

Addressing the above three key research questions serves as the core motivation of this work. In a nutshell, we study the resource trading problem in dynamic ISAC networks, where multiple BSs act as resource sellers and multiple MUs serve as buyers, each requiring two types of services --- communication and sensing --- both demanding bandwidth and power resources. To optimize resource allocation and service efficiency, we propose the \textit{Future Resource Bank for ISAC Networks}, a novel framework that \textit{interprets resource trading as a structured financial system with both offline and online modes}. In this model, BSs function as bank presidents, while MUs are classified into two levels of clients. For communication services, each MU acts as an individual client, directly requesting resources from the appropriate BS. For sensing services, MUs with a shared sensing objective can form a coalition, functioning as a team client that collectively requests resources. 
% Detailed contributions are provided by Section I.C.
%This coalition-based approach eliminates redundant sensing tasks and enhances overall efficiency by ensuring that multiple users with the same needs share resources instead of competing for them. This structured, level-wise bank-client model further enables the seamless integration of offline and online trading modes. The offline trading mode facilitates the negotiation of long-term contracts between BSs and both individual and team clients, thereby reducing real-time decision-making overhead and preemptively securing resource commitments. Also, given the dynamic nature of ISAC networks, online trading mode serves as a real-time fallback mechanism, ensuring adaptability by addressing unmet demands and surplus resources through temporary contracts when long-term agreements fail to accommodate instantaneous resource-demand fluctuations. To ensure efficient and stable resource allocation, we further develop a matching-based method that satisfies key properties such as stability, fairness, non-wastefulness, and individual rationality. By leveraging this approach, our framework not only enhances service reliability and resource utilization but also establishes an adaptable, incentive-driven ISAC trading market capable of functioning in dynamic environments.

\vspace{-4mm}
\subsection{Related Works}
This section explores existing research, with a particular focus on resource sharing/allocation in ISAC networks and the application of matching theory in broader networking contexts.


\subsubsection{Resource Allocation in ISAC Networks} Efficient resource allocation in ISAC networks has recently emerged as a critical research area \cite{RW ISAC1,RW ISAC2,RW ISAC3,RW ISAC4}. \textit{Li et al.} \cite{RW ISAC1} introduced a value-of-service-oriented resource allocation scheme for multi-MU collaborative ISAC networks, facilitating concurrent heterogeneous service provisioning. \textit{Du et al.} \cite{RW ISAC2} developed an ISAC resource allocation framework, integrating sensing quality of service (QoS) to optimize resources for diverse applications. \textit{Hu et al.}\cite{RW ISAC3} proposed an ISAC-aided edge computing framework, addressing the rapid proliferation of vehicles and the growing demand for integrating vehicular communications with computing services. \textit{Wang et al.}\cite{RW ISAC4} explored an environment-aware ISAC architecture for green ad-hoc networks, examining the intricate interplay between environmental sensing and data transmission performance.

These above efforts have primarily relied on onsite (online) decision-making approaches, which, while offering valuable contributions, often lead to extended delays, higher energy consumption, and potential service disruptions. Subsequently, our work introduces a hybrid resource trading framework that integrates offline and online trading modes, leveraging long-term contracts to reduce real-time decision overhead, while employing online trading as a backup mechanism to handle dynamic fluctuations in resource demand and supply. 
%Furthermore, we introduce overbooking strategies to enhance resource availability and mitigate inefficiencies caused by uncertainty in ISAC networks. To further optimize resource utilization, we propose a collaborative sensing model, where MUs with shared sensing objectives form coalitions, reducing redundant resource consumption and ensuring a more efficient and equitable resource allocation process. Collectively, these innovations establish a fast, resilient, and adaptable ISAC resource trading scheme that ensures efficient and reliable operations in dynamic network environments.

\subsubsection{Matching-Based Resource Provisioning}
This work develops one of the first efficient matching/assignment mechanisms between clients (MUs and coalitions) and BSs to ensure responsive and cost-effective resource provisioning in ISAC networks. Similar methods have gained attention in other domains such as edge computing and the Internet of Things (IoT), where matching-driven resource allocation is used for balancing diverse demands with available resource supplies \cite{RW Matching1,RW Matching2,RW Matching3,RW Matching4,RW Matching5}. \textit{Ye et al.} \cite{RW Matching1} combined deep reinforcement learning with stable matching to enable adaptive resource allocation in mobile crowdsensing. \textit{Xu et al.} \cite{RW Matching2} introduced a three-sided stable matching with an optimal pricing scheme for distributed vehicular networks. \textit{Qi et al.} \cite{RW Matching3} explored cross-layer pre-matching mechanisms to achieve cost-effective resource trading in cloud-aided edge networks. \textit{Sharghivand et al.} \cite{RW Matching4} considered QoS in terms of service response time and proposed a matching model between cloudlets and IoT applications. \textit{Du et al.} \cite{RW Matching5} developed a matching-based approach for computing resource management in small-cell networks, optimizing resource allocation and service pricing.

While these studies have made certain efforts, they primarily focus on isolated resource types, such as bandwidth, computing capacity, or specific service-related assets. However, ISAC networks require a nuanced approach, as they simultaneously support both communication and sensing services. In particular, unlike conventional wireless networking systems, ISAC necessitates the joint allocation of multiple resource types, particularly bandwidth and power, to balance the dual objectives of reliable communication and accurate sensing. Subsequently, by incorporating \textit{(i)} joint resource allocation strategies for stable matching between service demands and available resources, \textit{(ii)} a hybrid framework that integrates offline and online trading modes, and \textit{(iii)} optimizing both communication and sensing capabilities, our framework establishes one of the first scalable, resilient, and economically viable ISAC trading markets.

\vspace{-4mm}
\subsection{Overview and Summary of Contributions}
% This paper establishing a Future Resource Bank framework for ISAC, between two key parties: \textit{(i)} MUs as requestors, where each MU requires both communication and sensing services but participates intermittently, leading to unpredictability in resource allocation, and \textit{(ii)} BSs as providers, offering bandwidth and power resources, which must be allocated efficiently to support ISAC services. To address the challenges of demand uncertainty, inefficient allocation, and real-time decision-making overhead, we propose a fast and stable win-win matching method that effectively bridges resource demand and supply. Our approach introduces a two-sided trading mechanism that integrates offline and online modes, ensuring long-term stability through pre-established contracts while maintaining real-time flexibility to accommodate demand-supply fluctuations via online contracts when pre-established agreements fail to meet real-time market conditions. 

% Unlike existing approaches that treat MUs as independent requestors, our framework incorporates \textit{both} individual and coalition-based resource allocation. In this framework, MUs with shared sensing objectives can form cooperative coalitions, reducing redundant resource consumption and enhancing overall resource utilization.
% Additionally, we theoretically prove key properties, including individual rationality and weak Pareto optimality for our win-win matching methodology, ensuring that it remains fair, efficient, and stable under dynamic network conditions.
The core principle of our proposed approach in this work is to develop one of the first \textit{time-efficient} and \textit{mutually beneficial} matching mechanisms between BSs and both individual and coalition clients in dynamic ISAC networks. In this regard, our key contributions can be summarized as follows:

\noindent
~$\bullet$ To enable efficient service delivery in dynamic ISAC networks, we introduce the \textit{Future Resource Bank for ISAC}, a novel framework that integrates both offline and online trading modes to balance long-term stability with real-time adaptability. 
Further, unlike existing approaches that treat MUs as independent requestors, our framework incorporates \textit{both} individual and coalition-based resource allocation.


\noindent
~$\bullet$ Our proposed offline trading mode enables BSs and clients to \textit{pre-sign risk-aware long-term contracts} for both communication and sensing services, reducing real-time decision-making overhead. Further, in this mode, we introduce the concept of \textit{overbooking} to the ISAC literature, designed to address the uncertainty in MU resource demands. This concept allows BSs to allocate more resources than their theoretical supply, effectively mitigating fluctuations in resource demand.

\noindent
~$\bullet$
To establish feasible and mutually beneficial long-term  contracts between BSs and clients, we propose the \textit{Role-Friendly Win-Win Matching for Offline Trading} (offRFW$^2$M). We further prove that it satisfies key properties of \textit{stability}, \textit{individual rationality}, and \textit{weak Pareto optimality}, ensuring sustainability, incentivization, and efficiency in ISAC networks.

\noindent
~$\bullet$ Noting that the practical implementation of long-term contracts may suffer from performance degradations due to the intermittent participation of MUs and unpredictable resource demands, we introduce an online trading mode as a \textit{complementary backup mechanism to ensure resource allocation adaptability to the real-time ISAC network conditions}. 

\noindent
~$\bullet$ For the online trading mode, we first 
design a greedy-based algorithm that selects \textit{volunteer clients} willing to forgo their services in exchange for compensation, balancing excessive real-time demand without disrupting overall network performance.
We then enable individual MUs or coalitions with unmet demands to compete for available resources at BSs with surplus supply in real-time through proposing the Effective Backup Win-Win Matching for Online Trading (onEBW$^2$M).
We theoretically prove that onEBW$^2$M retains key properties of stability, individual rationality, and weak Pareto optimality.

\noindent
~$\bullet$ We conduct extensive experiments using both synthetic simulations and real-world datasets. The results demonstrate the superior performance of our method across key evaluation metrics, including social welfare, time efficiency, and energy efficiency compared to the existing baselines.


\noindent~ \textbf{\textit{Notations:}} $(\cdot)^\mathsf{(c)}$ and $(\cdot)^\mathsf{(s)}$ represent super-scripts related to communication services and sensing services, respectively;
$(\cdot)^\top$ and $(\cdot)^\mathsf{H}$ describe transpose and Hermitian, respectively. $j = \sqrt{-1}$; $\|\mathbf{a}\|$ is the Euclidean 2-norm of vector $\mathbf{a}$; $\{\cdot\}^\text{Re}$ is the real component of complex variable. $\operatorname{tr}(\mathbf{A})$ denotes the trace of matrix $\mathbf{A}$ and $\operatorname{diag}\{\mathbf{a}\}$ indicates the diagonal matrix built upon the elements of vector $\mathbf{a}$.

\vspace{-2.5mm}
\section{Overview and Key System Components}
\noindent In this section, we first present an overview of our methodology in Sec.~\ref{sec:over}. We then present the modeling of MUs and BSs in
Sec.~\ref{sec:baseModel}.
We finally present the ISAC model in
Sec.~\ref{sec:modelISAC}.


\begin{figure*}[]
	\centering
    \vspace{-6.5mm}
	\setlength{\abovecaptionskip}{-1.5 mm}
	\includegraphics[width=2\columnwidth]{images/SYSTEM_MODEL1.pdf}
    \vspace{-1mm}
	\caption{Schematic of our future resource bank for ISAC networks, illustrating two types of services (communication and sensing) and two levels of clients (individual MUs and sensing coalitions).}
	\label{SYSTEM MODEL}
	\vspace{-0.65cm}
\end{figure*}
\vspace{-3.5mm}\subsection{Overview of Our Methodology}\label{sec:over}
We consider a dynamic ISAC network, where resource provisioning involves two key parties: \textit{(i)} multiple MUs, denoted by $\bm{\mathcal{U}} = \{u_1, \dots, u_{\bm{|\mathcal{U}|}}\}$, each requiring both sensing and communication services, and \textit{(ii)} multiple BSs, denoted by $\bm{\mathcal{S}} = \{s_1, \dots, s_{\bm{|\mathcal{S}|}}\}$, which provide bandwidth and power resources to MUs.
We further define $\bm{\mathcal{Q}}$ as the set of sensing targets/objectives. When multiple MUs share the same sensing objective, they can form coalitions for cooperative sensing, represented as $\bm{\mathcal{C}} = \{\bm{c}_1, \dots, \bm{c}_{\bm{|\mathcal{C}|}}\}$ (each sensing coalition may contain one MU or more). Our \textit{future resource bank for ISAC} classifies clients into two levels: for communication services, each MU acts as an \textit{individual client}, directly requesting resources from BSs; for sensing services, MUs with common sensing targets form a coalition, functioning as a \textit{team client} that collectively requests resources, reducing redundancy.

Our future resource bank for ISAC further integrates both \textit{offline} and \textit{online} resource trading modes. The offline mode allows individual MUs and coalitions to establish long-term contracts with BSs for future communication and sensing services, securing resource commitments in advance of actual demand requests. In particular, for communication services, a contract between an individual MU $u_i\in\bm{\mathcal{U}}$ and a BS $s_j\in\bm{\mathcal{S}}$ is represented as {\footnotesize $\mathbb{C}^\mathsf{(c)}_{i,j} = \{\mathbbm{c}^\mathsf{(c),B}_{i,j}, \mathbbm{c}^\mathsf{(c),Pow}_{i,j}, \mathbbm{c}^\mathsf{(c),Pay}_{i,j}, \mathbbm{c}^\mathsf{(c),PelU}_{i,j}, \mathbbm{c}^\mathsf{(c),PelS}_{i,j}\}$}, where {\footnotesize$\mathbbm{c}^\mathsf{(c),B}_{i,j}$} and {\footnotesize$\mathbbm{c}^\mathsf{(c),Pow}_{i,j}$} denote the allocated bandwidth and power, {\footnotesize$\mathbbm{c}^\mathsf{(c),Pay}_{i,j}$} represents the unit price, and {\footnotesize$\mathbbm{c}^\mathsf{(c),PelU}_{i,j}$} and {\footnotesize$\mathbbm{c}^\mathsf{(c),PelS}_{i,j}$} are  clauses specifying penalties if the MU or BS breaches the contract.
Similarly, for sensing services, a long-term contract between a sensing coalition $\bm{c}_k\in \bm{\mathcal{C}}$ and a BS $s_j$ as {\footnotesize$\mathbb{C}^\mathsf{(s)}_{k,j} = \{\mathbbm{c}^\mathsf{(s),B}_{k,j}, \mathbbm{c}^\mathsf{(s),Pow}_{k,j}, \mathbbm{c}^\mathsf{(s),Pay}_{k,j}, \mathbbm{c}^\mathsf{(s),PelU}_{k,j}, \mathbbm{c}^\mathsf{(s),PelS}_{k,j}\}$}, where {\footnotesize$\mathbbm{c}^\mathsf{(s),B}_{k,j}$} and {\footnotesize$\mathbbm{c}^\mathsf{(s),Pow}_{k,j}$} specify the bandwidth and power demands, {\footnotesize$\mathbbm{c}^\mathsf{(s),Pay}_{k,j}$} represents the unit price, and {\footnotesize$\mathbbm{c}^\mathsf{(s),PelU}_{k,j}$} and {\footnotesize$\mathbbm{c}^\mathsf{(s),PelS}_{k,j}$} define penalty clauses for contract violations. We presume that MUs within a coalition 
{\footnotesize$\bm{c}_k$} share the same pricing scheme, where the cost per MU is determined as: {\footnotesize$p^\mathsf{(s)}_{i,j} = \mathbbm{c}^\mathsf{(s),Pay}_{k,j} / |\bm{c}_k|$}, where {\footnotesize$|\bm{c}_k|$} is the number of MUs within the coalition. These long-term contracts are directly executed during practical resource transactions\footnote{A practical transaction is a resource trading event between clients and BSs.}.

Since long-term contracts may experience performance degradation during execution, we introduce an online trading mode as a backup mechanism. This mode addresses two key scenarios: when resource demand at a BS exceeds its available supply (due to overbooking as we will  formalize in Sec.~\ref{sec:probForm}), certain clients (called \textit{volunteers}) voluntarily relinquish their commitments in exchange for compensation. Additionally, for MUs with unmet demands, online trading enables temporary contracts with BSs that have surplus resources.
\textit{Our approach to designing the offline and online trading procedures is to establish an effective and stable win-win matching between BSs and both individual and coalition clients, enabling seamless and fast service delivery across dynamic ISAC networks.}

Fig. \ref{SYSTEM MODEL} illustrates the timeline of our methodology. The timeline is divided into two segments: \textit{(i)} offline trading, where long-term contracts are established before practical demands arise, and \textit{(ii)} online trading, which manages practical transactions. Also, MUs are categorized into two client levels: \textit{(i)} individual MUs request communication services directly from BSs, and \textit{(ii)} sensing coalitions, formed by MUs with shared sensing objectives, request sensing services collectively.
In the offline trading mode (gray box in Fig. \ref{SYSTEM MODEL}), long-term contracts are signed between individual MUs, sensing coalitions, and BSs. During online trading (blue box), these contracts are executed, while MUs with unmet demands negotiate temporary contracts with BSs that have surplus resources.
The figure illustrates a scenario with two BSs (\(s_1, s_2\)) and five MUs (\(u_1\) to \(u_5\)). Initially, MUs form sensing coalitions based on shared sensing objectives, where \(\bm{c}_1 = \{u_1, u_2, u_3\}\) and \(\bm{c}_2 = \{u_4, u_5\}\). Each coalition participates in the resource trading market as a single entity, while individual MUs also negotiate contracts for communication services. Using our later offline matching strategy, called offRFW$^2$M, both individual MUs and coalitions establish long-term contracts with BSs, securing bandwidth and power allocations in advance. 
During practical Transaction 1, \(s_1\) selects \(u_1\) as a volunteer, relinquishing its reserved service to balance demand and supply. In practical Transaction 2, \(s_1\) selects \(u_4\) as a volunteer, since total demand exceeds its available resources. This highlights the effectiveness of overbooking, which enables BSs to handle resource demand fluctuations efficiently. For instance, in Transaction 1, when \(u_2\) and \(u_4\) are absent, \(s_1\) still fulfills contracts with \(u_3\) and coalition \(\bm{c}_1\), while \(s_2\) serves \(u_5\) and \(\bm{c}_2\), preventing resource waste. 
Nevertheless, dynamics of client participation can still lead to surplus resources. For example, in Transaction 2, \(s_1\) is unable to fulfill its contract with \(u_4\) due to an imbalance in supply and demand. To resolve this, our later online matching strategy, called onEBW$^2$M, efficiently reassigns resources, ensuring optimal resource utilization. For example, in Transaction 2, \(s_2\) reallocates communication resources to serve \(u_4\).

%where we show a timeline as an example to facilitate the analysis. Specifically, the timeline is divided into two segments: the offline trading (the time before practical demands) and the online trading (involving multiple practical transactions). Also, MUs are classified into two levels of clients based on their required services: each individual MU focuses on communication service, and each coalitions formed by MUs with the same sensing targets pays attention to sensing services\footnote{Note that for offline trading, the MU with the maximum expected sensing value within each coalition will be selected as a \textit{representative} to negotiate with BSs and sign a long-term contract on behalf of this coalition. Differently, in each practical transaction during online trading, the coalition selects the MU with the maximum practical sensing value $V_{i,j}^\mathsf{(s)}$ who participates in the transaction as the representative to negotiate with the BS according to the long-term contract and \textit{profit and cost sharing}. Specifically, the profit can include positioning information such as PEB, with all members of the coalition; while the cost mainly involves the payment to corresponding to the traded BS.}. The left box of Fig. 1 refers to offline trading mode, where long-term contracts are signed between individual MUs, sensing coalitions, and BSs.
%Then, during each actual transaction (the right big box, online mode), long-term contracts can be executed directly, while MUs those with unmet resource demands can negotiate temporary contracts with BSs with surplus resources (if any). For example, in Fig. \ref{SYSTEM MODEL}, with two BSs ($s_1$ and $s_2$) and five MUs ($u_1$ to $u_5$), MUs firstly form proper sensing coalitions (e.g., $u_1$, $u_2$, and $u_3$ form $\bm{c}_1$, while $u_4$ and $u_5$ form $\bm{c}_2$), where each sensing coalition participates in the resource trading market as a single client. Then, individual clients $u_1-u_5$, and the two coalitions will sign long-term contracts with proper BSs for communication and sensing services respectively (e.g., by using offRFW$^2$M.). 
%Then, in Transaction 1, $s_1$ selects individual client $u_1$ as a volunteer, while in Transaction 2, $s_1$ decides individual client $u_4$ as a volunteer, since the overall resource demand exceeds the supply of this BSs.
%Here, we can observe an interesting phenomenon where overbooking allows BSs to effectively handle fluctuations in resource demand. For example, in Transaction 1, when $u_2$ and $u_4$ have not attended, $s_1$ can still trade with $u_3$ and $\bm{c}_1$ directly according to pre-signed contracts, while $s_2$ can trade with $u_5$ and $\bm{c}_2$, thus avoiding resource waste. However, the uncertain resource demand caused by dynamic participation of MUs can still lead to the case where there exist surplus resources, e.g., in Transaction 2, $s_1$ fails to fulfill its contract with $u_4$. To cope with this, our well-designed EBW$^2$M can also gather efficient assignment between remaining demand and supply. For example, in Transaction 2, $s_2$ can offer communication service to $u_4$.

\vspace{-3mm}
\subsection{Modeling of MUs and BSs}\label{sec:baseModel}
\subsubsection{Modeling of MUs} We model each MU $u_i \in \bm{\mathcal{U}}$ as a 6-tuple $\{Q_i, R_i^\mathsf{req}, S_i^\mathsf{req}, N_i^\mathsf{R}, l_i^\mathsf{U},\alpha_i\}$, where $Q_i\in\mathcal{Q}$ represents its sensing target, $R_i^\mathsf{req}$ and $S_i^\mathsf{req}$ denote its requirement on data rate (bits/s) and sensing accuracy, respectively, and $N_i^\mathsf{R}$ represents the number of receiving antennas on $u_i$, while $l_i^\mathsf{U} = [x_i^\mathsf{U}, y_i^\mathsf{U}]^\top$ denotes the $(x,y)$ location of $u_i$. To reflect the dynamic nature of ISAC networks, we model MU participation uncertainty using a Bernoulli random variable $\alpha_i$, $\alpha_i \sim {\bf{B}} \Big((1, 0), (\mathbbm{a}_i, 1-\mathbbm{a}_i)\Big)$, where $\alpha_i=1$ which occurs with probability $\mathbbm{a}_i$ implies that the MU joins the market during a practical transaction; otherwise $\alpha_i=0$. MUs are formally categorized as follows.
\begin{Defn}(Individual MU) An individual MU  participates independently in resource trading for communication services.
\end{Defn}
\begin{Defn}(Sensing Coalition) MUs with a shared sensing target form a coalition, where each coalition acts as a team client in resource trading. A sensing coalition is defined through a two-way mapping/matching function $\mu$ between the MU set $ \bm{\mathcal{U}} $ and coalition set $ \bm{\mathcal{C}} $, which satisfies the following properties:
	
	\noindent
	$\bullet$ For each MU $ u_i \in \bm{\mathcal{U}}, \mu(u_i) \subseteq \bm{\mathcal{C}} $, and $|\mu(u_i)|=1$, meaning each MU belongs to exactly one coalition.
	
\noindent 
$\bullet$ For each coalition $\bm{c}_k \in \bm{\mathcal{C}}$, its assigned MUs satisfy $\mu(\bm{c}_k) \subseteq \bm{\mathcal{U}}$, and every MU $u_i \in \bm{c}_k$ shares the same sensing target $Q_i$.
	
	\noindent
	$\bullet$ An MU \( u_i \) belongs to coalition \( \bm{c}_k \) if and only if \( \bm{c}_k \) includes \( u_i \), i.e., \( u_i \in \mu(\bm{c}_k) \) if and only if \( \bm{c}_k \in \mu(u_i) \).
\end{Defn}

\subsubsection{Modeling of BSs} We model each BS $s_j \in \bm{\mathcal{S}}$ as a 4-tuple $\{B_j, P_j, l_j^\mathsf{S}, N_j^\mathsf{T}\}$, where $B_j$ and $P_j$ refer to the total bandwidth and power resources of $s_j$, respectively, $l_i^\mathsf{S} = [x_i^\mathsf{S}, y_i^\mathsf{S}]^\top$ represents the geographic location of $s_j$, and $N_j^\mathsf{T}$ indicates the number of transmitting antennas of BS $s_j$. We assume that both transmitting (Tx) and receiving (Rx) antennas are implemented using Uniform Linear Arrays (ULA) \cite{RW ISAC1}.

\vspace{-2mm}
\subsection{Modeling of ISAC}\label{sec:modelISAC}
\vspace{-.5mm}
\subsubsection{Communication Model}
We assume that the total bandwidth of BS \( s_j \) is divided into \( N^\mathsf{B}_j \) sub-channels, represented as $\mathcal{B}_j = \{1, \dots, N^\mathsf{B}_j\}$, where each sub-channel has an equal bandwidth of \( B_0 \). In the communication process, when the \( n \)-th sub-channel (\( 1 \leq n \leq N^\mathsf{B}_j \)) of BS \( s_j \) is allocated to MU \( u_i \), the downlink transmitted symbols are denoted as \( s_{i,j,n} = [s_{i,j,n,1}, s_{i,j,n,2}, \dots, s_{i,j,n,N^\star}]^\top \in \mathbb{C}^{N^\star \times 1} \) where \( N^\star \) represents the number of transmitted symbols. Subsequently, the beamforming matrix is  {\footnotesize \( F_{i,j,n} = F^\mathsf{RF}_{i,j,n} F^\mathsf{BB}_{i,j,n} \in \mathbb{C}^{N_j^\mathsf{T} \times N^\star} \)} where {\footnotesize\( F^\mathsf{RF}_{i,j,n} \in \mathbb{C}^{N_j^\mathsf{T} \times N^\mathsf{RF}_i} \)} is the analog precoding matrix, and {\footnotesize\( F^\mathsf{BB}_{i,j,n} \in \mathbb{C}^{N^\mathsf{RF}_j \times N^\star} \)} is the digital beamforming matrix. Here, {\footnotesize\( N^\mathsf{RF}_j \)} denotes the number of radio frequency (RF) chains at the transmitter of BS \( s_j \)\cite{beamforming 1,beamforming 2,beamforming 3}.
%Considering communications, when the \( n \)-th subchannel (\(1 \leq n \leq N_j\)) of BS \( s_j \) is allocated to MU \( u_i \), we denote the downlink transmitted symbols as \( \mathbf{s}_n = [\mathbbm{s}_{n,1}, \mathbbm{s}_{n,2}, \dots, \mathbbm{s}_{n,N^\star}]^\mathsf{T} \in {\bf C}^\mathsf{N^\star \times 1} \), where \( N^\star \) denotes the number of symbols. The beamforming matrix can be defined as \( \mathbf{F}_{i,j,n} = \mathbf{F}^\mathsf{RF}_{i,j,n} \mathbf{F}^\mathsf{BB}_{i,j,n} \in {\bf C}^\mathsf{N_j^\mathsf{T} \times N^\star} \), where \( \mathbf{F}^\mathsf{RF}_{i,j,n} \in {\bf C}^\mathsf{N_j^\mathsf{T} \times N_i^\mathsf{RF}} \) represents the analog precoding matrix. Besides, \( \mathbf{F}^\mathsf{BB}_{i,j,n} \in {\bf C}^\mathsf{N_j^\mathsf{RF} \times N^\star} \) is the digital beamforming matrix, and \( N_j^\mathsf{RF} \) denotes the number of radio frequency (RF) chains at the transmitter of BS $s_j$. It is assumed that BS \(s_j\) is unaware of the precise location information of MUs, which can still be utilized for beamforming schemes \cite{beamforming 1,beamforming 2,beamforming 3}.
For the communication channel, we assume \( L \) distinct propagation paths between BSs and MUs, indexed by $\ell \in\{0, 1, 2, \dots, L - 1\} $ where \( \ell = 0 \) represents the Line-of-Sight (LoS) path, and the remaining paths correspond to Non-LoS (NLoS) components, originating from single-bounce reflections\cite{NLoS}. Consequently, the MU-BS channel matrix \( \mathbf{H}_{i,j,n} \) of size \( N_i^\mathsf{R} \times N_j^\mathsf{T} \) can be expressed as 
\begin{equation}\label{equ.1}
	\mathbf{H}_{i,j,n} = \mathbf{A}_{i,j}^\mathsf{r} \boldsymbol{\Gamma}_{i,j,n} \left(\mathbf{A}_{i,j}^\mathsf{t}\right)^\mathsf{H},
\end{equation}
where
\begin{equation}\label{equ.2}
	\begin{aligned}
		&\boldsymbol{\Gamma}_{i,j,n} = \sqrt{N_i^\mathsf{R} N_j^\mathsf{T} p_{i,j,n}} \, \text{diag} \Bigg\{ \frac{h_{i,j,n,0}}{\sqrt{\rho_{i,j,n,0}}} e^{-j \frac{2\pi n \tau_{i,j,0}}{N^\mathsf{B}_j T_s}}, \dots,\\& \frac{h_{i,j,n,\ell}}{\sqrt{\rho_{i,j,n,\ell}}} e^{-j \frac{2\pi n \tau_{i,j,\ell}}{N^\mathsf{B}_j T_s}}, \dots,\frac{h_{i,j,n,L-1}}{\sqrt{\rho_{i,j,n,L-1}}} e^{-j \frac{2\pi n \tau_{i,j,L-1}}{N^\mathsf{B}_j T_s}} \Bigg\}.
	\end{aligned}
\end{equation}
Here, $p_{i,j,n}$ is the power allocated to MU \( u_i \) on sub-channel \( n \) of BS \( s_j \), while \( h_{i,j,n,\ell} \) and \( \rho_{i,j,n,\ell} \) denote the channel gain and path loss for the \( \ell \)-th path, respectively. The sampling period is represented by \( T_s \), and \(\tau_{i,j,\ell} \) is the time delay of the \( \ell \)-th path. Also, the steering vector \( \mathbf{A}^\mathsf{t}_{i,j} \) and response vector \( \mathbf{A}^\mathsf{r}_{i,j} \) for a ULA are given by:
\begin{equation}\label{AOD}
	     \hspace{-8mm}
\resizebox{0.465\textwidth}{!}{$ \mathbf{A}_{i,j}^\mathsf{t}(\theta_{i,j}) {=} \frac{1}{\sqrt{N_j^\mathsf{T}}} 
	\begin{bmatrix}
		e^{-j \frac{N_j^\mathsf{T} - 1}{2} \frac{2\pi}{\lambda_n} d \sin(\theta_{i,j})} ,\cdots,
		e^{j \frac{N_j^\mathsf{T} - 1}{2} \frac{2\pi}{\lambda_n} d \sin(\theta_{i,j})}
	\end{bmatrix}^\top\hspace{-3mm},
    $}\hspace{-5mm}
\end{equation}
\begin{equation}\label{AOA}
\hspace{-8mm}
\resizebox{0.465\textwidth}{!}{$ 
	 \mathbf{A}_{i,j}^\mathsf{r}(\theta_{i,j}) {=} \frac{1}{\sqrt{N_i^\mathsf{R}}} 
	\begin{bmatrix}
		e^{-j \frac{N_i^\mathsf{R} - 1}{2} \frac{2\pi}{\lambda_n} d \sin(\theta_{i,j})} ,\cdots,
		e^{j \frac{N_i^\mathsf{R} - 1}{2} \frac{2\pi}{\lambda_n} d \sin(\theta_{i,j})}
	\end{bmatrix}^\top\hspace{-3mm},
    $}\hspace{-5mm}
\end{equation}
where \( \theta_{i,j} \) is the angle of departure (AoD), \( d \) is the distance between antenna elements (\( d = \lambda_n / 2 \)), and \( \lambda_n \) is the signal wavelength. Assuming BS \( s_j \) and MU \( u_i \) are synchronized, they can compensate for the Doppler effect and time delay. Subsequently, the received  communication signal is given by
\begin{equation}
	\mathbf{y}_{i,j,n}^\mathsf{(c)} = \mathbf{W}_{i,j,n}^\mathsf{H} \mathbf{H}_{i,j,n} \mathbf{F}_{i,j,n} \mathbf{s}_n + \mathbf{W}_{i,j,n}^\mathsf{H} \mathbf{n}_{i,j,n},
\end{equation}
where \(\mathbf{W}_{i,j,n} \) is the receiver combiner, and \( \mathbf{n}_{i,j,n} \) is the zero-mean additive white Gaussian noise (AWGN) with power \( \sigma_c^2 \). 

Using Shannon’s capacity formula, the achievable data rate for MU \( u_i \) on sub-channel \( n \) of BS \( s_j \) is thus given by
\begin{equation}
	R_{i,j,n} = B_0 \log_2 \left( 1 + \text{SNR}_{i,j,n} \right).
\end{equation}
where interference between different MUs is omitted \cite{SURVEY 3} and the signal-to-noise ratio (SNR) is defined as 
	\begin{equation}
	    \text{SNR}_{i,j,n} = p_{i,j,n} \xi_{i,j,n}
	\end{equation}
with the effective channel gain-to-noise ratio of
\begin{equation}
	\xi_{i,j,n} = {\left| \mathbf{W}_{i,j,n}^\mathsf{H} \mathbf{A}_{i,j}^\mathsf{r} \boldsymbol{\Gamma}_{i,j,n} \left(\mathbf{A}_{i,j}^\mathsf{t}\right)^\mathsf{H} \mathbf{F}_{i,j,n} \right|^2}\Big/{\sigma_c^2}.
\end{equation}
Subsequently, the overall achievable data rate between MU \( u_i \) and BS \( s_j \) is given by
\vspace{-2mm}
\begin{equation}\label{equ. 9}
	R_{i,j}^\mathsf{(c)} = \sum_{n=1}^{N^\mathsf{B}_j} a_{i,j,n} R_{i,j,n},
\end{equation}
where \( a_{i,j,n} \in \{0,1\} \) represents sub-channel allocation, with \( a_{i,j,n} = 1 \) indicating that sub-channel \( n \) of BS \( s_j \) is allocated to MU \( u_i \). 
Let \( \mathcal{N}^\mathsf{(c)}_{i,j}=\{n| a_{i,j,n}=1 \} \) with size ${N}^\mathsf{(c)}_{i,j}=|\mathcal{N}^\mathsf{(c)}_{i,j}|$ denote the set of allocated subchannels to MU \( u_i \) from BS \( s_j \). Also, let \(P^\mathsf{(c)}_{i,j} \)
denote the
total power allocated by BS \( s_j \) to MU \( u_i \), with per-sub-channel power allocation of \( {P^\mathsf{(c)}_{i,j}}/{N^\mathsf{(c)}_{i,j}} \). Assuming identical channel gains and noise powers across sub-channels \cite{RW ISAC1}, the achievable data rate in (\ref{equ. 9}) simplifies to
\begin{equation}\label{equ. 10}
	R_{i,j}^\mathsf{(c)} = N_{i,j}^\mathsf{(c)} B_0 \log_2 \left( 1 + {P_{i,j}^\mathsf{(c)} \xi_{i,j}}\Big/{N_{i,j}^\mathsf{(c)}} \right),
\end{equation}
where \( \xi_{i,j}=\xi_{i,j,n}~\forall n, \) is the channel gain-to-noise power ratio across all sub-channels allocated to MU \( u_i \). Defining \( B^\mathsf{(c)}_{i,j} = N^\mathsf{(c)}_{i,j} B_0 \), the achievable data rate can be rewritten as
\begin{equation}\label{equ. 11}
	R_{i,j}^\mathsf{(c)} = B_{i,j}^\mathsf{(c)} \log_2 \left( 1 + {B_0 P_{i,j}^\mathsf{(c)} \xi_{i,j}}\Big/{B_{i,j}^\mathsf{(c)}} \right).
\end{equation}

Finally, we define the \textit{value} that MU \( u_i \) obtains from utilizing the communication service provided by BS \( s_j \) as
\begin{equation}\label{comm value}
	V_{i,j}^\mathsf{(c)} =\omega_1 R_{i,j}^\mathsf{(c)}= \omega_1 B_{i,j}^\mathsf{(c)} \log_2 \left( 1 + {B_0 P_{i,j}^\mathsf{(c)} \xi_{i,j}}\Big/{B_{i,j}^\mathsf{(c)}} \right),
\end{equation}
where \( \omega_1>0 \) is a weighting coefficient, scaling the contribution of data rate \(R^\mathsf{(c)}_{i,j} \) to the perceived service value.
\begin{figure*}
\vspace{-9mm}
\begin{equation}\label{equ. 21}
 \hspace{-3mm}
\resizebox{0.975\textwidth}{!}{$
\begin{aligned}
    &f(\mathbf{y}_{i,j}^\mathsf{\mathsf{(s)}} | \boldsymbol{\eta}_{i,j}) = \frac{1}{(2\pi \sigma_s^2)^{N^\mathsf{(s)}_{i,j}}} \exp \left\{ \sum_{n=\mathbbm{l}_{i,j}^\mathsf{\mathsf{(s)}}}^{\mathbbm{l}_{i,j}^\mathsf{\mathsf{(s)}}+N^\mathsf{(s)}_{i,j}} \left[ \frac{1}{2\sigma_s^2} \left( \mathbf{H}_{i,j,n}^\mathsf{(s)} \mathbf{F}_{i,j,n}\mathbf{s}_\text{ref} \right)^\mathsf{H} \mathbf{y}_{i,j}^\mathsf{\mathsf{(s)}} - \frac{1}{\sigma_s^2} \left| \mathbf{H}_{i,j,n}^\mathsf{(s)} \mathbf{F}_{i,j,n}\mathbf{s}_\text{ref} \right|^2 \right]^\mathsf{Re} \right\},~
\mathbf{H}_{i,j,n}^\mathsf{(s)} = \sqrt{N_j^\mathsf{T} N_i^\mathsf{R} p_{i,j,n}} \frac{h_{i,j,n}}{\sqrt{\rho_{i,j,n}}}
		e^{-j \frac{2 \pi n \tau_{i,j}}{N_j^\mathsf{B} T_s}} \mathbf{A}_{i,j}^\mathsf{r} \phi_{i,j,k}^\mathsf{(s)} 	\left(\mathbf{A}_{i,j}^\mathsf{t}\right)^\mathsf{H} \theta_{i,j}^\mathsf{(s)}
\end{aligned}
$}
\hspace{-3mm}
\vspace{-0.4mm}
\end{equation}
\vspace{-.5mm}
\hrulefill
\vspace{-5.9mm}
\end{figure*}

\subsubsection{Sensing Model}
Sensing services encompass various applications, including detection, localization, and tracking, with this paper primarily focusing on the localization service as a representative example. In the localization service, in a semi-static sensing environment\footnote{We assume that the sensing target moves slowly, ensuring that the channel environment experienced by the signal remains approximately constant\cite{RW ISAC1}.}, each BS transmits sensing reference signals to the target, while MUs decode the target's position by processing the reflected signals. 
We assume that the approximate positions of targets within the region of interest are known \cite{known}.
We also presume that sensing services operate in two modes. In \textit{device-free sensing}, the targets are objects or entities that do not carry electronic devices \cite{RW ISAC1, device sensing}. In contrast, \textit{device-based sensing} refers to scenarios where the sensing targets are MUs equipped with electronic devices \cite{RW ISAC1}. %In the following, we model these two operational modes.
%Sensing services general encompass different types such as detection, localization, and tracking, while our paper mainly focuses on the localization service of sensing targets. In a semi-static sensing environment, BS \( s_j \) transmits sensing reference signals to the target, and each MU decodes the target's position by processing the reflected signals\footnote{We assume that the sensing target moves slowly, thereby ensuring that the channel environment experienced by the signal remains approximately constant \citen{RW ISAC1}}. Here, we make a reasonable assumption that the approximate positions of targets within a given range of interest are known\cite{known}. 
%Under the assumption of good time-frequency synchronization, the signal received by MU \( m_j \) can be expressed as:
%\begin{equation}
%	\mathbf{y}_{i,j,n,k}=\mathbf{y}_{i,j}^\mathsf{(c)}+\sum_{Q_k\in\mathcal{Q}}\mathbf{y}_{i,j,n,k}^\mathsf{(s)} 
%\end{equation}
%Also, we consider two sensing modes: device-based sensing and device-free sensing\cite{RW ISAC1, device sensing}. In particular, device-free sensing refers to scenarios where the sensing targets are objects or entities that do not carry any electronic devices. In contrast, device-based sensing involves sensing targets that are the MUs themselves equipped with electronic devices\cite{RW ISAC1}. 

\noindent~\textit{(a) Device-free sensing:}
% In device-free sensing, 
% the sensing target of MU \( u_i \) can be any object except itself. 
Suppose that MU \( u_i \) has a device-free sensing task and is assigned to subchannel \( n \) of BS \( s_j \). Accordingly, the signal received by \( u_i \) from \( s_j \), after being \textit{reflected} by the target \( Q_i \), can be expressed as
\begin{equation}
	\begin{aligned}\label{sensing sign}
		&\mathbf{y}_{i,j,n}^\mathsf{(s)} =
		\sqrt{N_j^\mathsf{T} N_i^\mathsf{R} p_{i,j,n}} \frac{h_{i,j,n}}{\sqrt{\rho_{i,j,n}}}
		e^{-j \frac{2 \pi n \tau_{i,j}}{N_j^\mathsf{B} T_s}}
		\\&~~~~\times \mathbf{W}_{i,j,n}^\mathsf{H} \mathbf{A}_{i,j}^\mathsf{r} \phi_{i,j,k}^\mathsf{(s)} 	\left(\mathbf{A}_{i,j}^\mathsf{t}\right)^\mathsf{H} \theta_{i,j}^\mathsf{(s)} \times\mathbf{F}_{i,j,n}\mathbf{s}_\text{ref}
		+ \mathbf{W}_{i,j,n}^\mathsf{H} \mathbf{w}_{i,j,n},
	\end{aligned}
    \hspace{-4mm}
\end{equation}
where $p_{i,j,n}$, $h_{i,j,n}$, $\rho_{i,j,n}$ are the transmit power, channel gain, and path loss, respectively. Also, $\mathbf{w}_{i,j,n}$ denotes the AWGN with power $\sigma_s^2$, $\mathbf{s}_\text{ref}$ is the sensing reference signal, and $\tau_{i,j}$ is the transmission time from BS $s_j$ to MU $u_i$, expressed as
\begin{equation}\label{equ. transmission time}
	\tau_{i,j} = ({d_{i}^\mathsf{Q} + d_{i,j}^\mathsf{Q}})\big/{c},
\end{equation}
where $c$ is the speed of light, $d_{i}^\mathsf{Q}$ and $d_{i,j}^\mathsf{Q}$ are the distances between target $Q_i$ and MU $u_i$ and BS $s_j$, which are given by
\begin{equation}\label{equ. 15}
	d_{i}^\mathsf{Q} = \sqrt{(x_i^\mathsf{U} - x_i^\mathsf{Q})^2 + (y_i^\mathsf{U} - y_i^\mathsf{Q})^2},
\end{equation}
\begin{equation}\label{equ. 16}
	d_{i,j}^\mathsf{Q} = \sqrt{(x_{i}^\mathsf{Q} - x_{j}^\mathsf{S})^2 + (y_i^\mathsf{Q} - y_{j}^\mathsf{S})^2}.
\end{equation}
Here, the position of MU $u_i$, BS $s_j$, and $u_i$'s sensing target $Q_i$ are denoted by $l_i^\mathsf{U} = [x_i^\mathsf{U}, y_i^\mathsf{U}]^\top$, $l^\mathsf{S}_{j} = [x^\mathsf{S}_{j}, y^\mathsf{S}_{j}]^\top$, $l_i^\mathsf{Q} = [x_i^\mathsf{Q}, y_i^\mathsf{Q}]^\top$, respectively. Moreover, the steering and response vector functions in (\ref{sensing sign}) follow the same structures as (\ref{AOD}) and (\ref{AOA}), where \( \theta^\mathsf{(s),Q}_{i,j} \) represents the AoD from BS \( s_j \) to \( Q_i \), and \( \phi^\mathsf{(s),Q}_{i} \) denotes the AoA from \( Q_i \) to BS \( s_j \), which are given by
\begin{equation}
	\theta_{i}^\mathsf{\mathsf{(s),Q}} = \arccos \left( ({x_i^\mathsf{U} - x_{i}^\mathsf{Q}})\big/{\| l_{i}^\mathsf{Q} - l^\mathsf{U}_i \|_2} \right),
\end{equation}
\begin{equation}\label{equ. 18}
	\phi_{i,j}^\mathsf{\mathsf{(s),Q}} = \pi - \arccos \left( {(x_i^\mathsf{Q} - x_{j}^\mathsf{S})}\big/{\| l_i^\mathsf{Q} - l_{j}^\mathsf{S} \|_2} \right).
\end{equation}

\noindent~\textit{(b) Device-based sensing:}
When considering MU \( u_i \) as the sensing target \( Q_i \), the expression of the received signal at MU \( u_i \) follows a similar form to (\ref{sensing sign}), with the angles \( \theta^\mathsf{(s),Q}_i \) and \( \phi^\mathsf{(s),Q}_{i,j} \) replaced by \( \theta_{i,j} \), which represents the AoD and AoA from BS \( s_j \) to MU \( u_i \). 

To quantify the quality of sensing in both device-free and device-based sensing scenarios, we use a common  sensing metric, called Position Error Bound (PEB), which is derived from the Fisher Information Matrix (FIM, \( \mathbf{J}(l^\mathsf{Q}_i) \)) as follows:
\begin{equation}
	\mathbb{E}\left[ \| \hat{l}_{i}^\mathsf{Q} - l_{i}^\mathsf{Q} \|^2 \right] \geq \text{tr} \{ \mathbf{J}^\mathsf{-1}(l_{i}^\mathsf{Q}) \},
\end{equation}
where $\hat{l}_{i}^\mathsf{Q}$ and $ l_{i}^\mathsf{Q}$ are the estimation and true location of sensing target  $Q_i \in\bm{\mathcal{Q}}$, while $\mathbb{E}[\cdot]$ denotes the expectation operator.
Instead of deriving the PEB directly, a more practical approach is to compute the Cramér-Rao Lower Bound (CRLB) for \( \tau_{i,j} \), \( \theta^\mathsf{(s),Q}_i \), and \( \phi^\mathsf{(s)}_{i,j} \) in device-free sensing (or \(\tau_{i,j} \), \( \theta_{i,j} \) in device-based sensing)\cite{RW ISAC1}. We define \( {\boldsymbol{\eta}_{i,j}} = [\tau_{i,j}, \theta^\mathsf{(s),Q}_i, \phi^\mathsf{(s)}_{i,j}]^\mathsf{T} \), and the CRLB (i.e., the variance of the estimate: \( \text{var}(\bm{\hat{\eta}}_{i,j}) \), \( {\boldsymbol{\hat{\eta}}_{i,j}} = [\hat{\tau}_{i,j}, \hat{\theta}^\mathsf{(s),Q}_i, \hat{\phi}^\mathsf{(s)}_{i,j}]^\mathsf{T} \)) serves as a lower bound for the FIM of \( {\boldsymbol{\eta}_{i,j}} \). Specifically, we have \( \text{var}(\bm{\hat{\eta}}_{i,j}) \leq \mathbf{J}_{\boldsymbol{\eta}_{i,j}} \), where $\mathbf{J}_{\boldsymbol{\eta}_{i,j}}$ denotes the FIM of \( {\boldsymbol{\eta}_{i,j}} \),  given by
\vspace{-1mm}
\begin{equation}
	\mathbf{J}_{\boldsymbol{\eta}_{i,j}} = \mathbb{E}_{\boldsymbol{\eta}_{i,j}} \left[ \nicefrac{- \partial^2 \ln f(\mathbf{y}_{i,j}^\mathsf{\mathsf{(s)}} | \boldsymbol{\eta}_{i,j})}{\partial \boldsymbol{\eta}_{i,j} \partial (\boldsymbol{\eta}_{i,j})^\mathsf{T}} \right],
\end{equation}
where $f(\mathbf{y}_{i,j}^\mathsf{\mathsf{(s)}} | \boldsymbol{\eta}_{i,j})$ is the likelihood function of $\mathbf{y}_{i,j }^\mathsf{\mathsf{(s)}}$ with respect to $\boldsymbol{\eta}_{i,j}$, shown in (\ref{equ. 21}).
Let $\mathcal{N}_{i,j}^\mathsf{\mathsf{(s)}} = \{\mathbbm{l}_{i,j}^\mathsf{\mathsf{(s)}}, \dots, \mathbbm{l}_{i,j}^\mathsf{\mathsf{(s)}} + N_{i,j}^\mathsf{\mathsf{(s)}} \}$ with size \( N^\mathsf{(s)}_{i,j} \) contain the subchannels allocated by BS \( s_j \) to MU \( u_i \) for sensing target \( Q_i \), where \( \mathbbm{l}^\mathsf{(s)}_{i,j} \) is the initial subchannel index\cite{RW ISAC1}, and $B_{i,j}^\mathsf{(s)} = N_{i,j}^\mathsf{(s)} B_0$ capture the total bandwidth allocated for sensing.
% By definition, both \( B^\mathsf{(c)}_{i,j} \) and \( B^\mathsf{(s)}_{i,j} \) must be integer multiples of \( B_0 \).
The derivation of \(\mathbf{J}_{\boldsymbol{\eta}_{i,j}} \) with respect to \( B^\mathsf{(s)}_{i,j} \) and the  power allocated for sensing \( P^\mathsf{(s)}_{i,j} \), is provided in Appendix A, where the PEB for the MU-BS pair \( i,j \) can finally be approximated as
\begin{equation}
	\text{PEB}_{i,j}=\sqrt{ \text{tr} \left\{ \mathbf{J}_{\boldsymbol{\eta}_{i,j}}^{-1} \right\}} \approx {\zeta_{i,j}}\Big/{\sqrt{N_{i,j}^\mathsf{(s)} P_{i,j}^\mathsf{(s)}}},
\end{equation}
where \( \zeta_{i,j} \) is influenced by multiple factors, including bandwidth, antennas configuration, channel gain, noise power, and the relative positions of the sensing target and the MU (detailed derivations are provided in Appendix A.
We further introduce \( \omega_2,\omega_3 \in[0, 1] \) as the power and bandwidth coefficients to simply model the PEB as follows:
\begin{equation}
	\frac{1}{\text{PEB}_{i,j}} \approx \kappa_{i,j} \left( P_{i,j}^\mathsf{(s)} \right)^\mathsf{\omega_2} \left( B_{i,j}^\mathsf{(s)} \right)^\mathsf{\omega_3},
\end{equation}
where \( \kappa_{i,j} = {\vartheta_{i,j}}/{\zeta_{i,j}} \) and \( \vartheta_{i,j} \) represents the relative sensing capability of MU \( u_i \). The coefficient \( \kappa_{i,j} \) is influenced by system configurations, imperfections in beamforming gain, filtering gain, and the signal processing algorithm\cite{sensing factor}. 

Finally, we define the \textit{value} of sensing for each MU \( u_i \) as
\begin{equation}\label{sensing value}
	V_{i,j}^\mathsf{(s)}=\omega_4\frac{1}{\text{PEB}_{i,j}}\approx\omega_4\kappa_{i,j} \left( P_{i,j}^\mathsf{(s)} \right)^\mathsf{\omega_2} \left( B_{i,j}^\mathsf{(s)} \right)^\mathsf{\omega_3},
\end{equation}
where \( \omega_4 >0\) is a weighting coefficient, determining the relative importance of sensing accuracy in the utility function.

\vspace{-2mm}
\section{Role-Friendly Win-Win Matching for Offline Trading (offRFW$^2$M)}
\vspace{-.5mm}
\noindent In this section, we introduce offRFW$^2$M, which achieves mutually beneficial and risk-aware long-term contracts for both communication and sensing services. 
% For communication services, we formulate long-term contracts \( \mathbb{C}^\mathsf{(c)}_{i,j} \) between individual MUs and BSs. Similarly, for sensing services, we establish contracts \( \mathbb{C}^\mathsf{(s)}_{k,j} \) between sensing coalitions and BSs. 

% By integrating a role-aware approach into the matching framework, offRFW$^2$M ensures that both individual and coalition-based clients engage in resource trading under stable and incentive-compatible agreements, fostering long-term efficiency in dynamic ISAC networks.

\vspace{-3.5mm}
\subsection{Key Definitions}
We first introduce the key definitions of our many-to-many (M2M) matching framework between MUs and BSs, designed for the offline trading mode. Unlike conventional matching mechanisms, this framework is crafted to mitigate potential risks, ensuring the robustness of long-term contracts. 
% By incorporating risk-aware strategies, the proposed M2M matching model enhances decision-making for both communication and sensing resource allocation.

\vspace{-1.5mm}
\begin{Defn}(M2M Matching for Communication Services in offRFW$^2$M)
	An M2M matching \( \varphi^\mathsf{(c)} \)  for communication services in offRFW$^2$M constitutes a two-way function/mapping between BSs \( \bm{\mathcal{S}} \) and MUs \(\bm{\mathcal{U}} \), satisfying the following properties:
	
	\noindent
	$\bullet$ For each BS $ s_{j} \in \bm{\mathcal{S}},\varphi^\mathsf{(c)}\left( s_j \right) \subseteq \bm{\mathcal{U}} $, meaning that a BS can potentially provide communication services to multiple MUs.
	
	\noindent
	$\bullet$ For each MU $ u_i \in \bm{\mathcal{U}}, \varphi^\mathsf{(c)}\left( u_i \right) \subseteq \bm{\mathcal{S}} $, and $|\varphi^\mathsf{(c)}\left( u_i \right)|=1$, ensuring that each MU is assigned to exactly one BS.
	
	\noindent
	$\bullet$ For each BS $ s_j $ and MU $ u_i $, $ s_j\in\varphi^\mathsf{(c)}(u_i)$ if and only if $ u_i\in\varphi^\mathsf{(c)}\left(s_j\right) $, indicating that a valid matching occurs only when both the MU and the BS mutually accept the contract.
\end{Defn}

\vspace{-3.5mm}
\begin{Defn}(M2M Matching for Sensing Services in offRFW$^2$M)
	An M2M matching \( \varphi^\mathsf{(s)} \)  for sensing services in offRFW$^2$M constitutes a two-way function/mapping between BSs  \( \bm{\mathcal{S}} \) and coalitions \( \bm{\mathcal{C}} \), satisfying the following properties:
	
	\noindent
	$\bullet$ For each BS $ s_{j} \in \bm{\mathcal{S}},\varphi^\mathsf{(s)}\left( s_j \right) \subseteq \bm{\mathcal{C}} $, meaning that a BS can provide sensing resources to multiple sensing coalitions.
	
	\noindent
	$\bullet$ For each MUs' coalition $ \bm{c}_k \in \bm{\mathcal{C}}, \varphi^\mathsf{(s)}\left( \bm{c}_k \right) \subseteq \bm{\mathcal{S}} $ and $|\varphi^\mathsf{(s)}\left( \bm{c}_k \right)|=1$, ensuring that each coalition is assigned to one or more BSs, allowing cooperative resource provisioning.
	
	\noindent
	$\bullet$ For each $ s_j $ and $ \bm{c}_k $, $ s_j\in\varphi^\mathsf{(s)}(\bm{c}_k)$ if and only if $ \bm{c}_k\in\varphi^\mathsf{(s)}\left(s_j\right) $, ensuring that a matching occurs only when both parties mutually accept the 
    resource-sharing agreement.
\end{Defn}
%We next define the \textit{blocking pair}, which is a crucial factor that can make the matching unstable.
% \subsection{Utility, Expected Utility and Risk of Clients/MUs and BSs}

\vspace{-6mm}
\subsection{Utility, Expected Utility, and Risk of Clients}\label{sec:UtilityClients}
\subsubsection{Utility of Individual MUs for Communication Services}
We obtain the utility of communication services for an MU \( u_i \) through three key components: \textit{(i)} The valuation derived from utilizing communication services provided by BS \( s_j \), subtracting the payment made by \( u_i \) to \( s_j \). \textit{(ii)} The penalty incurred when \( u_i \) breaks the contract (e.g., when \( \alpha_i = 0 \) in a practical transaction). \textit{(iii)} The compensation received if \( u_i \) is selected as a volunteer. We formulate the utility function as
\begin{equation}\label{equ. comm utility}
	\begin{aligned}
		u^\mathsf{(c),U}\left(u_i,\varphi^\mathsf{(c)}(u_i),\mathbb{C}^\mathsf{(c)}_{i,j}\right)&=\alpha_i(1-\mathbbm{v}^\mathsf{(c)}_{i,j})\left(V_{i,j}^\mathsf{(c)}-\mathbbm{c}^\mathsf{(c),Pay}_{i,j}\right)\\[-.1em]&\hspace{-4mm}-(1-\alpha_i)\mathbbm{c}^\mathsf{(c),PelU}_{i,j}+\alpha_i\mathbbm{v}^\mathsf{(c)}_{i,j}\mathbbm{c}^\mathsf{(c),PelS}_{i,j},
	\end{aligned}
\end{equation}
where \( \mathbbm{v}^\mathsf{(c)}_{i,j} = 1 \) indicates that \( u_i \) is selected by \( s_j \) as a volunteer in a practical transaction, and \(\mathbbm{v}^\mathsf{(c)}_{i,j} = 0 \) otherwise. In~\eqref{equ. comm utility}, \( V_{i,j}^\mathsf{(c)} \), defined in (\ref{comm value}), is related to the long-term contract \( \mathbb{C}^\mathsf{(c)}_{i,j} \)  between MU $u_i$ and BS \( s_j\in\varphi^\mathsf{(c)}(u_i) \), where the  bandwidth \( \mathbbm{c}^\mathsf{(c),B}_{i,j} \) and power \( \mathbbm{c}^\mathsf{(c),Pow}_{i,j} \) traded are specified (see Sec.~\ref{sec:over}).


\subsubsection{Utility of Coalitions for Sensing Services} We obtain the utility that a sensing coalition \( \bm{c}_k \) obtains through three key components: \textit{(i)} The valuation of sensing services provided by BS \( s_j \), subtracting the payment made by \( \bm{c}_k \) to \( s_j \). \textit{(ii)} The penalty incurred when \( \bm{c}_k \) breaks the contract. \textit{(iii)} The compensation received if \( \bm{c}_k \) is selected as a volunteer. We thus formulate the utility function for sensing coalitions as
\begin{equation}\label{equ. sensing utility}
	\hspace{-3mm}\begin{aligned}
 &u^\mathsf{(s),U}\left(\bm{c}_k,\varphi^\mathsf{(s)}(\bm{c}_k),\mathbb{C}^\mathsf{(s)}_{k,j}\right)=\beta_k\mathbbm{v}^\mathsf{(s)}_{k,j}\mathbbm{c}^\mathsf{(s),PelS}_{k,j}
        \\[-.2em]
        &-(1-\beta_k)\mathbbm{c}^\mathsf{(s),PelU}_{k,j}+(1-\mathbbm{v}^\mathsf{(s)}_{k,j})\beta_k\left(|\bm{c}_k|V^\mathsf{(s),max}_{k,j}-\mathbbm{c}^\mathsf{(s),Pay}_{k,j}\right),
	\end{aligned}
    \hspace{-3mm}
\end{equation}
where $\mathbbm{v}^\mathsf{(s)}_{k,j} = 1$ implies that  $\bm{c}_k$ is selected by $s_j$ as a volunteer in a practical transaction, and $\mathbbm{v}^\mathsf{(s)}_{k,j} = 0$ otherwise.
In \eqref{equ. sensing utility}, \( V^\mathsf{(s),max }_{k,j} \) is the maximum sensing value of MUs in coalition \( \bm{c}_k \), defined as $ V^\mathsf{(s),max }_{k,j}=\max_{u_i\in \bm{c}_k}\{ V^\mathsf{(s)}_{i,j}\}$,\footnote{In each coalition \(\bm{c}_k\), the MU with the highest sensing value is selected as representative to report resource request to BSs. Therefore, the total sensing value obtained by the coalition is \(|\bm{c}_k|V^\mathsf{(s),max }_{k,j}\).} and $V^\mathsf{(s)}_{i,j}$, defined in (\ref{sensing value}), is related to the traded bandwidth \( \mathbbm{c}^\mathsf{(s),B}_{k,j} \) and power \( \mathbbm{c}^\mathsf{(s),Pow}_{k,j} \)  through the long-term contract $\mathbb{C}^\mathsf{(s)}_{k,j}$ (see Sec.~\ref{sec:over}). Also, in \eqref{equ. sensing utility}, \( \beta_k = 1 \) captures that \( \bm{c}_k \) participates in a practical transaction (i.e.,  $\sum_{u_i\in \bm{c}_k}\alpha_i > 0$), and \( \beta_k = 0 \) otherwise\footnote{A coalition $\bm{c}_k$ participates in a practical transaction if and only if any $u_i$ in $\bm{c}_k$ participates: $\bm{c}_k$ will be absent from the transaction if $\mu(\bm{c}_k)=\emptyset$.}. Since uncertainties introduce challenges in directly maximizing the values of (\ref{equ. comm utility}) and (\ref{equ. sensing utility}) in the offline trading mode, we use their expected values over the uncertainties as follows:\footnote{It is assumed that $\alpha_i$ and $\mathbbm{v}^\mathsf{(c)}_{i,j}$, and $\beta_k$ and $\mathbbm{v}^\mathsf{(s)}_{k,j}$ are independent.}
\begin{equation}\label{equ. expected comm utility}\hspace{-3mm}
\resizebox{0.405\textwidth}{!}{$
	\begin{aligned}
		&\mathbb{E}\left [u^\mathsf{(c),U}(u_i,\varphi^\mathsf{(c)}(u_i),\mathbb{C}^\mathsf{(c)}_{i,j})\right ]=\mathbb{E}[\alpha_i]\mathbb{E}[\mathbbm{v}^\mathsf{(c)}_{i,j}]\mathbbm{c}^\mathsf{(c),PelS}_{i,j}-\\[-.2em]
        &(1-\mathbb{E}[\alpha_i])\mathbbm{c}^\mathsf{(c),PelU}_{i,j}+\mathbb{E}[\alpha_i](1-\mathbb{E}[\mathbbm{v}^\mathsf{(c)}_{i,j}])(V_{i,j}^\mathsf{(c)}-\mathbbm{c}^\mathsf{(c),Pay}_{i,j}),
	\end{aligned}
    $}
    \vspace{-0.1mm}
\end{equation}
\begin{equation}\label{equ. expected sensing utility}
 \hspace{-3mm}
\resizebox{0.46\textwidth}{!}{$
	\begin{aligned}
		&\mathbb{E}\left [u^\mathsf{(s),U}(\bm{c}_k,\varphi^\mathsf{(s)}(\bm{c}_k),\mathbb{C}^\mathsf{(s)}_{k,j})\right ]=-(1-\mathbb{E}[\beta_k])\mathbbm{c}^\mathsf{(s),PelU}_{k,j}\\[-.2em]
        &+(1-\mathbb{E}[\mathbbm{v}^\mathsf{(s)}_{k,j}])\mathbb{E}[\beta_k](\mathbb{E}[V^\mathsf{(s),max }_{k,j}]-\mathbbm{c}^\mathsf{(s),Pay}_{k,j})+\mathbb{E}[\beta_k]\mathbb{E}[\mathbbm{v}^\mathsf{(s)}_{k,j}]\mathbbm{c}^\mathsf{(c),PelS}_{i,j},
	\end{aligned}
    $}
    \hspace{-3.5mm}
\end{equation}
\vspace{-1mm}

\noindent where expressions of $\mathbb{E}[V^\mathsf{(s),max }_{k,j}]$, $\mathbb{E}[\alpha_i]$, $\mathbb{E}[\beta_k]$, $\mathbb{E}[\mathbbm{v}^\mathsf{(c)}_{i,j}]$, and $\mathbb{E}[\mathbbm{v}^\mathsf{(s)}_{k,j}]$  are detailed in Appendix B.

Carefully investigating offline trading unveils potential risks due to the uncertainties involved. To evaluate these risks on the client side, we define two key risk assessment aspects:

\noindent~1) An individual MU \( u_i \) faces the risk of obtaining an undesired utility. We quantify this risk as the probability that the utility of \( u_i \) falls below a threshold \(u^\mathsf{(c)}_{\min} \), formulated as
\begin{equation}
\hspace{-3mm}
\resizebox{0.47\textwidth}{!}{$
	\begin{aligned}
			&R_1^\mathsf{U}\big(u_i,\varphi^\mathsf{(c)}(u_i),\mathbb{C}^\mathsf{(c)}_{i,j}\big)=\Pr\left(u^\mathsf{(c),U}\left(u_i,\varphi^\mathsf{(c)}(u_i),\mathbb{C}^\mathsf{(c)}_{i,j}\right)\leq u^\mathsf{(c)}_\mathsf{\min}\right).
	\end{aligned}
    $}\hspace{-5mm}
\end{equation}

\noindent~2) Similarly, a sensing coalition \( \bm{c}_k \) faces the risk of an unsatisfactory utility. We formulate this risk as the probability that the utility of \( \bm{c}_k \) falls below a tolerable threshold \( u^\mathsf{(s)}_{\min} \):
\begin{equation}
\hspace{-3mm}
\resizebox{0.47\textwidth}{!}{$
	\begin{aligned}			&R_2^\mathsf{U}(\bm{c}_k,\varphi^\mathsf{(s)}(\bm{c}_k),\mathbb{C}^\mathsf{(s)}_{k,j}))=\Pr\left(u^\mathsf{(s),U}\left(\bm{c}_k,\varphi^\mathsf{(s)}(\bm{c}_k),\mathbb{C}^\mathsf{(s)}_{k,j})\right)\leq u^\mathsf{(s)}_\mathsf{\min}\right).
	\end{aligned}
    $}\hspace{-5mm}
\end{equation}

The above risks should be managed when designing long-term contracts. Otherwise, clients may prefer online trading, opting out of long-term agreements with any BS.

\vspace{-4.5mm}
\subsection{Utility, Expected Utility, and Risk of BSs}
\subsubsection{Utility of BSs for Communication Services} We obtain the utility of a BS for providing communication services through three components: \textit{(i)} payments received from MUs, \textit{(ii)} Compensation received from MUs that break their contracts, and \textit{(iii)} Compensation paid to MUs selected as volunteers. Accordingly, the utility function of BS \( s_j \) is formulated by
\begin{equation}\label{equ. comm BS U}
	\begin{aligned}
		u^\mathsf{(c),S}\left (s_j,\varphi^\mathsf{(c)}(s_j),\mathbb{C}^\mathsf{(c)}_{i,j}\right )&=\sum_{u_i\in\varphi^\mathsf{(c)}(s_j)}(1-\mathbbm{v}^\mathsf{(c)}_{i,j})\alpha_i\mathbbm{c}^\mathsf{(c),Pay}_{i,j}\\[-.1em]
        &\hspace{-4mm}+(1-\alpha_i)\mathbbm{c}^\mathsf{(c),PelU}_{i,j}-\alpha_i\mathbbm{v}^\mathsf{(c)}_{i,j}\mathbbm{c}^\mathsf{(c),PelS}_{i,j}.
	\end{aligned}
\end{equation}

\subsubsection{Utility of BSs for Sensing Services} Similarly, the utility of a BS for providing sensing services comprises: 
\textit{(i)} payments received from sensing coalitions, \textit{(ii)} compensation received from coalitions that break their contracts, and \textit{(iii)} compensation paid to coalitions selected as volunteers. Mathematically, 
\begin{equation}\label{equ. sensing BS U}
	 \hspace{-4mm}\begin{aligned}
	u^\mathsf{(s),S}\left(s_j,\varphi^\mathsf{(s)}(s_j), \mathbb{C}^\mathsf{(s)}_{k,j}\right)&=\sum_{\bm{c}_k\in\varphi^\mathsf{(c)}(s_j)}(1-\mathbbm{v}^\mathsf{(s)}_{k,j})\beta_k\mathbbm{c}^\mathsf{(s),Pay}_{k,j}\\[-.1em]
    &\hspace{-4mm}+(1-\beta_k)\mathbbm{c}^\mathsf{(s),PelU}_{k,j}-\beta_k\mathbbm{v}^\mathsf{(s)}_{k,j}\mathbbm{c}^\mathsf{(s),PelS}_{k,j}.
	\end{aligned}
    \hspace{-4mm}
\end{equation}
Due to uncertainties that prevent the direct computation of values in (\ref{equ. comm BS U}) and (\ref{equ. sensing BS U}), we consider their expected values:
\begin{equation}
 \hspace{-4mm}
	\begin{aligned}
		&\mathbb{E}\left [u^\mathsf{(c),S}(s_j,\varphi^\mathsf{(c)}(s_j),\mathbb{C}^\mathsf{(c)}_{i,j})\right ]=\hspace{-3mm}\sum_{u_i\in\varphi^\mathsf{(c)}(s_j)}(1-\mathbb{E}[\mathbbm{v}^\mathsf{(c)}_{i,j}])\mathbb{E}[\alpha_i]\mathbbm{c}^\mathsf{(c),Pay}_{i,j}\\[-.1em]
        &+(1-\mathbb{E}[\alpha_i])\mathbbm{c}^\mathsf{(c),PelU}_{i,j}-\mathbb{E}[\alpha_i]\mathbb{E}[\mathbbm{v}^\mathsf{(c)}_{i,j}]\mathbbm{c}^\mathsf{(c),PelS}_{i,j},
	\end{aligned}
     \hspace{-4mm}
\end{equation}
\begin{equation} \hspace{-4mm}
	\begin{aligned}
		&\mathbb{E}\left [u^\mathsf{(s),S}(s_j,\varphi^\mathsf{(s)}(s_j),\mathbb{C}^\mathsf{(s)}_{k,j})\right ]=\hspace{-3mm}\sum_{\bm{c}_k\in\varphi^\mathsf{(c)}(s_j)}(1-\mathbb{E}[\mathbbm{v}^\mathsf{(s)}_{k,j}])\mathbb{E}[\beta_k]\mathbbm{c}^\mathsf{(s),Pay}_{k,j}\\[-.1em]&+(1-\mathbb{E}[\beta_k])\mathbbm{c}^\mathsf{(s),PelU}_{k,j}-\mathbb{E}[\beta_k]\mathbb{E}[\mathbbm{v}^\mathsf{(s)}_{k,j}]\mathbbm{c}^\mathsf{(s),PelS}_{k,j},
	\end{aligned}
     \hspace{-4mm}
\end{equation}
where the expected values are obtained in Appendix B. Similar to Sec.~\ref{sec:UtilityClients}, we define two key risk assessment aspects for BSs:
% Meanwhile, due to the dynamic resource demands from MUs, each BS \( s_j \) faces two key risks, defined as follows:
% we define two key risk assessment aspects

\noindent~1) A BS \( s_j \) risks not being able to provide the committed bandwidth during actual transactions, quantified as
\begin{equation}
	\begin{aligned}		&R_1^\mathsf{S}\big(s_j,\varphi^\mathsf{(c)}(s_j),\mathbb{C}^\mathsf{(c)}_{i,j},\varphi^\mathsf{(s)}(s_j),\mathbb{C}^\mathsf{(s)}_{k,j}\big)=\\[-.1em]
    &~~~~\Pr\Bigg(\sum_{u_i\in\varphi^\mathsf{(c)}(s_j)}\alpha_i\mathbbm{c}^\mathsf{(c),B}_{i,j}+\sum_{\bm{c}_k\in\varphi^\mathsf{(s)}(s_j)}\beta_k\mathbbm{c}^\mathsf{(s),B}_{k,j}> B_j\Bigg).
	\end{aligned}
\end{equation}
% where $\mathbbm{c}^\mathsf{(s),B}_{k,j}$ denotes the amount of bandwidth resources traded by coalition $\bm{c}_k$, and the value of \( \mathbbm{c}^\mathsf{(s),B}_{k,j} \) is determined by the bandwidth resource traded on representative.

%defined as $\mathbbm{c}^\mathsf{(s),B}_{k,j}=\left\{B^\mathsf{(s)}_{i,j}|\max V^\mathsf{(s)}_{i,j}, \forall u_i\in \bm{c}_k, \bm{c}_k\in\varphi^\mathsf{(s)}(s_j)\right\} $.

\noindent~2) A BS \( s_j \) risks failing to supply the stipulated power resources during actual transactions, quantified as
\begin{equation}
	\begin{aligned}		&R_2^\mathsf{S}(s_j,\varphi^\mathsf{(c)}(s_j),\mathbb{C}^\mathsf{(c)}_{i,j},\varphi^\mathsf{(s)}(s_j),\mathbb{C}^\mathsf{(s)}_{k,j})=\\[-.1em]
    &~~~~\Pr\Bigg(\sum_{u_i\in\varphi^\mathsf{(c)}(s_j)}\alpha_i\mathbbm{c}^\mathsf{(c),Pow}_{i,j}+\sum_{\bm{c}_k\in\varphi^\mathsf{(s)}(s_j)}\beta_k\mathbbm{c}^\mathsf{(s),Pow}_{k,j}> P_j\Bigg).
	\end{aligned}
\end{equation}
% where $\mathbbm{c}^\mathsf{(s),Pow}_{k,j}$ describes the amount of power resources traded with the sensing coalition $\bm{c}_k$, and its value is determined by the power resource traded on representative.
%defined as $\mathbbm{c}^\mathsf{(s),Pow}_{k,j}=\left\{P^\mathsf{(s)}_{i,j}|\max V^\mathsf{(s)}_{i,j}, \forall u_i\in \bm{c}_k, \bm{c}_k\in\varphi^\mathsf{(s)}(s_j)\right\} $.

Managing these risks is crucial; otherwise, a BS may prefer online trading over committing to long-term contracts.

\vspace{-4mm}
\subsection{Problem Formulation of Offline Trading Mode}\label{sec:probForm}
We formulate the resource trading in the offline trading mode to optimize the M2M matching and long-term contracts between clients and BSs. Accordingly, the objective of each BS \( s_j \in \bm{\mathcal{S}} \) is to \textit{maximize its overall utility} as follows:
\begin{subequations}
	\begin{align}
\hspace{-3mm}\bm{\mathcal{F}^\mathsf{S}} \hspace{-0.1mm}{:}&\hspace{-1.5mm}\underset{{\mathbb{C}^\mathsf{(c)}_{i,j},\mathbb{C}^\mathsf{(s)}_{k,j}}}{\max}\hspace{-1.25mm}\mathbb{E}\hspace{-0.75mm}\left[\hspace{-0.4mm}u^\mathsf{(c),S}(s_j,\hspace{-0.25mm}\varphi^\mathsf{(c)}(s_j),\hspace{-0.25mm}\mathbb{C}^\mathsf{(c)}_{i,j}){+}u^\mathsf{(s),S}(s_j,\hspace{-0.25mm}\varphi^\mathsf{(s)}(s_j),\hspace{-0.25mm}\mathbb{C}^\mathsf{(s)}_{k,j})\hspace{-0.45mm}\right] \label{equ. PF BS} \tag{37}\hspace{-3mm}\\
&\hspace{-4mm}\text{\textbf{s.t.}}~~~\hspace{-0mm}\varphi^\mathsf{(c)}\left(s_j\right)\subseteq\bm{\mathcal{U}},\varphi^\mathsf{(s)}\left(s_j\right)\subseteq\bm{\mathcal{C}}, \mu\left(\bm{c}_k\right)\subseteq\bm{\mathcal{U}}, \tag{37a}\label{equ. PF BS C1}\\
		&\hspace{-4mm}u_i\in \varphi^\mathsf{(c)}(s_j), \bm{c}_k\in \varphi^\mathsf{(s)}(s_j), u_i\in\mu(\bm{c}_k), \tag{37b}\label{equ. PF BS C2}\\
		&\hspace{-4mm}\sum_{u_i\in\varphi^\mathsf{(c)}(s_j)}B_{i,j}^\mathsf{(c)}+\sum_{\bm{c}_k\in\varphi^\mathsf{(s)}(s_j)}\mathbbm{c}^\mathsf{(s),B}_{k,j}\leq (1+O_j^\mathsf{B})B_j, \tag{37c}\label{equ. PF BS C3}\\
		&\hspace{-4mm}\sum_{u_i\in\varphi^\mathsf{(c)}(s_j)}P_{i,j}^\mathsf{(c)}+\sum_{\bm{c}_k\in\varphi^\mathsf{(s)}(s_j)}\mathbbm{c}^\mathsf{(s),Pow}_{k,j}\leq (1+O_j^\mathsf{Pow})P_j, \tag{37d}\label{equ. PF BS C4}\\
		&\hspace{-4mm}R_1^\mathsf{S}(s_j,\varphi^\mathsf{(c)}(s_j),\mathbb{C}^\mathsf{(c)}_{i,j},\varphi^\mathsf{(s)}(s_j),\mathbb{C}^\mathsf{(s)}_{k,j})\leq \rho_1,\label{equ. PF BS C5}\tag{37e}
		\\&\hspace{-4mm}R_2^\mathsf{S}(s_j,\varphi^\mathsf{(c)}(s_j),\mathbb{C}^\mathsf{(c)}_{i,j},\varphi^\mathsf{(s)}(s_j),\mathbb{C}^\mathsf{(s)}_{k,j})\leq \rho_2,\label{equ. PF BS C6}\tag{37f}
	\end{align}
\end{subequations}

\noindent where $ \rho_1 $ and $\rho_2 $ are risk thresholds falling in interval $ (0, 1] $, and $O_j^\mathsf{B}$ and $O_j^\mathsf{Pow}$ represent the overbooking rate of BS $s_j$ for bandwidth and power resource, respectively. In $ \bm{\mathcal{F}^\mathsf{S}} $, constraints (\ref{equ. PF BS C1}) and (\ref{equ. PF BS C2}) enforce the feasibility of communication and sensing M2M matchings  $\varphi^\mathsf{(c)}$ and $\varphi^\mathsf{(s)}$. 
 Constraints (\ref{equ. PF BS C3}) and (\ref{equ. PF BS C4}) ensure that the bandwidth and power resources sold by BS $s_j$ do not exceed its supply $(1+O_j^\mathsf{B})B_j$ and $(1+O_j^\mathsf{Pow})P_j$ after overbooking. Constraints (\ref{equ. PF BS C5}) and (\ref{equ. PF BS C6}) specify the risks of BSs with their tractable forms obtained in Appendix B through probabilistic analysis.
Further, each client (i.e., $u_i $ or $\bm{c}_k$) aims \textit{to maximize its utility} as follows:
\vspace{-1.05mm}
\begin{subequations}
	\begin{align}
	\bm{\mathcal{F}^\mathsf{U}}:~&
	\left\{ \begin{matrix}
		\underset{{\mathbb{C}^\mathsf{(c)}_{i,j}}}{\max}~\mathbb{E}\left[u^\mathsf{(c),U}(u_i,\varphi^\mathsf{(s)}(u_i),\mathbb{C}^\mathsf{(c)}_{i,j})\right] \\[-.2em]
		\underset{{\mathbb{C}^\mathsf{(s)}_{k,j}}}{\max}~\mathbb{E}\left[u^\mathsf{(s),U}(\bm{c}_k,\varphi^\mathsf{(s)}(\bm{c}_k),\mathbb{C}^\mathsf{(s)}_{k,j})\right]
	\end{matrix}\right\}, \tag{38}\label{equ. PF MU}\\[-.1em]
		\text{\textbf{s.t.}}~~~
		&\varphi^\mathsf{(c)}\left(u_i\right)\subseteq\bm{\mathcal{S}}, \mu\left(u_i\right)\subseteq\bm{\mathcal{C}},\varphi^\mathsf{(s)}\left(\bm{c}_k\right)\subseteq\bm{\mathcal{S}} ,\label{equ. PF MU C1}\tag{38a}\\[-.2em]
		&s_j\in \varphi^\mathsf{(c)}(u_i), \bm{c}_k\in \mu(u_i), s_j\in\varphi^\mathsf{(s)}(\bm{c}_k), \tag{38b}\label{equ. PF MU C2}\\[-.2em]
		&V^\mathsf{(c)}_{i,j}\ge \mathbbm{c}^\mathsf{(c),Pay}_{i,j},V^\mathsf{(s)}_{i,j}=V^\mathsf{(s),max }_{k,j}\ge \mathbbm{c}^\mathsf{(s),Pay}_{i,j}, \tag{38c}\label{equ. PF MU C3}\\[-.2em]
		&V^\mathsf{(c)}_{i,j}\geq R^\mathsf{req},\tag{38d}\label{equ. PF MU C4}\\[-.2em]
		&V^\mathsf{(s),max }_{k,j}\geq S^\mathsf{req},\tag{38e}\label{equ. PF MU C5}\\[-.2em]
            & B_{\min} \leq B^\mathsf{(c)}_{i,j}, B^\mathsf{(s)}_{k,j}\leq B_{\max},\tag{38f}\label{equ. PF MU C8}\\[-.2em]
		&P_{\min} \leq P^\mathsf{(c)}_{i,j}, P^\mathsf{(s)}_{k,j}\leq P_{\max},\tag{38g}\label{equ. PF MU C9}\\[-.2em] 
		&R_1^\mathsf{U}(u_i,\varphi^\mathsf{(c)}(u_i),\mathbb{C}^\mathsf{(c)}_{i,j})\leq \rho_3,\tag{38h}\label{equ. PF MU C6}\\[-.2em]
		&R_2^\mathsf{U}(\bm{c}_k,\varphi^\mathsf{(s)}(\bm{c}_k),\mathbb{C}^\mathsf{(s)}_{k,j})\leq \rho_4,\tag{38i}\label{equ. PF MU C7}
	\end{align}
\end{subequations}

\noindent where \( B_{\min} \) and \( B_{\max} \) represent the lower and upper bounds of bandwidth resources that each MU can request, while \( P_{\min} \) and \( P_{\max} \) denote the minimum and maximum power resources that can be allocated to each MU. $ \rho_3,\rho_4\in (0, 1] $ are risk thresholds. In $ \bm{\mathcal{F}^\mathsf{U}} $, constraints (\ref{equ. PF MU C1}) and (\ref{equ. PF MU C2}) guarantee the feasibility of communication and sensing M2M matchings  $\varphi^\mathsf{(c)}$ and $\varphi^\mathsf{(s)}$. Constraint (\ref{equ. PF MU C3}) ensures that the obtained valuation of $u_i$ exceeds its payments, while constraints (\ref{equ. PF MU C4}) and (\ref{equ. PF MU C5}) ensure that the communication and sensing service quality of each MU or sensing coalition meets the corresponding requirements. Constraints (\ref{equ. PF MU C8}) and (\ref{equ. PF MU C9}) guarantee the bandwidth and power resources requested by each client for services are constrained within a certain range. Also, constraints (\ref{equ. PF MU C6}) and (\ref{equ. PF MU C7}) specify the risks of each MU with their tractable forms obtained in Appendix B.

The offline trading mode thus presents a \textit{multi-objective optimization (MOO) problem} involving both $\bm{\mathcal{F}^\mathsf{S}}$ and $\bm{\mathcal{F}^\mathsf{U}}$, where the conflicting utilities of different parties make designing a win-win solution a complex task. Furthermore, the probabilistic nature of risks further complicates the problem. To address this MOO problem, we propose offRFW$^2$M, which obtains long-term contracts while achieving mutually beneficial expected utilities and managing risks for both parties. 

 \vspace{-4mm}
\subsection{Solution Design: Structure of offRFW$^2$M}

In a nutshell, offRFW$^2$M enables BSs and clients to negotiate the quantity and pricing of bandwidth and power resources for two distinct service types: (i) individual MUs engage in resource trading for communication services; (ii) coalitions participate in resource trading for sensing services. Given the interdependencies between resource demands and pricing strategies, offRFW$^2$M is an iterative method that we describe below, with its details outlined in Alg.~\ref{Alg:1}.
\begin{algorithm}[t!] %其中这里面不能有H不然会报错,不过不影响结果
	{\scriptsize \setstretch{0.4}\caption{{Proposed Role-Friendly Win-Win Matching for Offline Trading}\label{Alg:1}}%算法名字
		\LinesNumbered %要求显示行号
		\textbf{Initialization:} $ \mathcal{X} \leftarrow 1 $, $ \mathbbm{c}^\mathsf{(c),Pay}_{i,j}\left\langle 1 \right\rangle \leftarrow p^\mathsf{\min}_{i,j}$, $ \mathbbm{c}^\mathsf{(s),Pay}_{k,j}\left\langle 1 \right\rangle \leftarrow p^\mathsf{\min}_{k,j}$, $\mathbbm{c}^\mathsf{(c),PelU}_{i,j}$, $\mathbbm{c}^\mathsf{(c),PelS}_{i,j}$, $\mathbbm{c}^\mathsf{(s),PelU}_{k,j}$, $\mathbbm{c}^\mathsf{(s),PelS}_{k,j}$, ${flag}_{j} \leftarrow 1 $, $\mathbb{Y}^\mathsf{(c)}\left( u_i \right)\leftarrow \varnothing$, $\mathbb{Y}^\mathsf{(c)}\left( s_{j} \right)\leftarrow \varnothing$, $\mathbb{Y}^\mathsf{(s)}\left( \bm{c}_k \right)\leftarrow \varnothing$, $\mathbb{Y}^\mathsf{(s)}\left( s_{j} \right)\leftarrow \varnothing$\ %\;用于换行
		
		\For{$\forall u_i\in\bm{\mathcal{U}}$}{
		$\bm{c}_k\leftarrow u_i$ forms coalitions based on shared sensing target, where $\bm{c}_k\in\bm{\mathcal{C}}$
		}
		\While{$ \sum_{u_i\in\bm{\mathcal{U}}}{flag}_{i} $ and $\sum_{\bm{c}_k\in\bm{\mathcal{C}}}{flag}_{k}$}{
			\textbf{$ {flag}_{i} \leftarrow {\bf False} $, $ {flag}_{k} \leftarrow {\bf False} $}
			
			\textbf{Calculate:} $\overrightarrow{L^\mathsf{(c)}_i}$ and $\overrightarrow{L^\mathsf{(s)}_k}$ under constraints (\ref{equ. PF MU C4}) - (\ref{equ. PF MU C7})
            
            $ F^\mathsf{(c),\star}_i\left\langle \mathcal{X} \right\rangle\leftarrow \overrightarrow{L^\mathsf{(c)}_i}$, $ F^\mathsf{(s),\star}_k\left\langle \mathcal{X} \right\rangle\leftarrow \overrightarrow{L^\mathsf{(s)}_k}$      
			 $\mathbb{Y}^\mathsf{(c)}\left( u_i \right), B_{i,j}^\mathsf{(c)}\left\langle \mathcal{X} \right\rangle, P_{i,j}^\mathsf{(c)}\left\langle \mathcal{X} \right\rangle, \mathbbm{c}^\mathsf{(c),Pay}_{i,j}\left\langle \mathcal{X} \right\rangle  \leftarrow F^\mathsf{(c),\star}_i\left\langle \mathcal{X} \right\rangle $, $ \mathbb{Y}^\mathsf{(s)}\left( \bm{c}_k \right), B_{k,j}^\mathsf{(s)}\left\langle \mathcal{X} \right\rangle, P_{k,j}^\mathsf{(s)}\left\langle \mathcal{X} \right\rangle, \mathbbm{c}^\mathsf{(s),Pay}_{k,j}\left\langle \mathcal{X} \right\rangle \} \leftarrow F^\mathsf{(s),\star}_k\left\langle \mathcal{X} \right\rangle $

			
			\If{$ \forall\mathbb{Y}^\mathsf{(c)}\left( u_i \right) \neq \varnothing $ or $ \forall\mathbb{Y}^\mathsf{(s)}\left( \bm{c}_k \right) \neq \varnothing $}{
				\For{$\forall u_i \in \bm{\mathcal{U}}$ }{
				$ u_i $ sends a proposal to $ s_j $, where $s_j\in\mathbb{Y}^\mathsf{(c)}\left( u_i \right)$}
				\For{$\forall \bm{c}_k \in \bm{\mathcal{C}}$ }{
				$ \bm{c}_k $ sends a proposal  to $ s_j $, where $s_j\in\mathbb{Y}^\mathsf{(s)}\left( \bm{c}_k \right)$}
				
				\While{
					$ \Sigma_{u_i\in \bm{\mathcal{U}}}{flag}_{i} > 0 $}{
					$ {\widetilde{\mathbb{Y}}}\left(s_j\right) \leftarrow$ collect proposals from clients
					
					$ \mathbb{Y}^\mathsf{(c)}(s_j)$, $\mathbb{Y}^\mathsf{(s)}(s_j) \leftarrow $ choose MUs from $ {\widetilde{\mathbb{Y}}}\left(s_j\right) $ that can achieve the maximization of the expected utility of BS $s_j$ (i.e., (\ref{equ. PF BS})) by using DP under (\ref{equ. PF BS C3}), (\ref{equ. PF BS C4}), (\ref{equ. PF BS C5}), and (\ref{equ. PF BS C6})
					
					$ s_j $ temporally accepts the clients in $ \mathbb{Y}^\mathsf{(c)}(s_j) $ and $ \mathbb{Y}^\mathsf{(s)}(s_j) $, and rejects the others
				}
				
				\For{
					$ \forall u_i \in \mathbb{Y}^\mathsf{(c)}\left( s_j \right) $
				}{
					\If{$ u_i $ is rejected by $ s_j $, $V^\mathsf{(c)}_{i,j}\ge \mathbbm{c}^\mathsf{(c),Pay}_{i,j}$ and constraints (\ref{equ. PF MU C4}) and (\ref{equ. PF MU C6}) are met}{
						$ \mathbbm{c}^\mathsf{(c),Pay}_{i,j}\left\langle {\mathcal{X} + 1} \right\rangle \leftarrow \min\left\{ \mathbbm{c}^\mathsf{(c),Pay}_{i,j}\left\langle \mathcal{X} \right\rangle + \mathrm{\Delta}p~,{ V}^\mathsf{(c)}_{i,j} \right\} $}
					\Else{$ \mathbbm{c}^\mathsf{(c),Pay}_{i,j}\left\langle {\mathcal{X} + 1} \right\rangle \leftarrow \mathbbm{c}^\mathsf{(c),Pay}_{i,j}\left\langle \mathcal{X} \right\rangle $}
				}
				
				\For{
					$ \forall \bm{c}_k \in \mathbb{Y}^\mathsf{(s)}\left( s_j \right) $
				}{
					\If{$ u_i $ is rejected by $ s_j $, $V^\mathsf{(s)}_{k,j}\ge \mathbbm{c}^\mathsf{(s),Pay}_{k,j}$ and constraints (\ref{equ. PF MU C5}) and (\ref{equ. PF MU C7}) are met}{
						$ \mathbbm{c}^\mathsf{(s),Pay}_{k,j}\left\langle {\mathcal{X} + 1} \right\rangle \leftarrow \min\left\{ \mathbbm{c}^\mathsf{(s),Pay}_{k,j}\left\langle \mathcal{X} \right\rangle + \mathrm{\Delta}p~,{ V}^\mathsf{(s)}_{k,j} \right\} $}
					\Else{$ \mathbbm{c}^\mathsf{(s),Pay}_{k,j}\left\langle {\mathcal{X} + 1} \right\rangle \leftarrow \mathbbm{c}^\mathsf{(s),Pay}_{k,j}\left\langle \mathcal{X} \right\rangle $}
				}
                $ p_{i,\mathbbm{n}}^\mathsf{(c)}\leftarrow \mathbbm{c}^\mathsf{(c),Pay}_{i,j}\left\langle \mathcal{X}+1 \right\rangle, p_{i,\mathbbm{n}}^\mathsf{(c)} \in F^\mathsf{(c),\star}_{i}\left\langle \mathcal{X} \right\rangle$, $
                    p_{k,\mathbbm{m}}^\mathsf{(s)}\leftarrow \mathbbm{c}^\mathsf{(c),Pay}_{i,j}\left\langle \mathcal{X}+1 \right\rangle, p_{i,\mathbbm{n}}^\mathsf{(c)} \in F^\mathsf{(c),\star}_{i}\left\langle \mathcal{X} \right\rangle$
                    
					\If{$\mathcal{X}\le2$ and there exists $F^\mathsf{(c),\star}_{i}\left\langle \mathcal{X}-1 \right\rangle \neq F^\mathsf{(c),\star}_{i}\left\langle \mathcal{X} \right\rangle $ or $F^\mathsf{(s),\star}_{k}\left\langle \mathcal{X}-1 \right\rangle \neq F^\mathsf{(s),\star}_{k}\left\langle \mathcal{X} \right\rangle $}{
					$ {flag}_{i} \leftarrow {\bf True} $, $ {flag}_{k} \leftarrow {\bf True} $	}	\      
				$ \mathcal{X}\leftarrow \mathcal{X}+1 $
			}
		
		
		}		 
	

		$\varphi^\mathsf{(c)}(s_j)\leftarrow\mathbb{Y}^\mathsf{(c)}(s_j)$, $\varphi^\mathsf{(c)}(u_i)\leftarrow \mathbb{Y}^\mathsf{(c)}(u_i)$,
		$\varphi^\mathsf{(s)}(s_j)\leftarrow\mathbb{Y}^\mathsf{(s)}(s_j)$, $\varphi^\mathsf{(s)}(\bm{c}_k)\leftarrow \mathbb{Y}^\mathsf{(s)}(\bm{c}_k)$ , $\mathcal{X} \leftarrow \mathcal{X}-1$

		
		\textbf{Return:} $\mathbb{C}_{i,j}^\mathsf{(c)} =\{ B_{i,j}^\mathsf{(c)}\left\langle \mathcal{X} \right\rangle, P_{i,j}^\mathsf{(c)}\left\langle \mathcal{X} \right\rangle, \mathbbm{c}^\mathsf{(c),Pay}_{i,j}\left\langle \mathcal{X} \right\rangle, 		\mathbbm{c}^\mathsf{(c),PelU}_{i,j}, \mathbbm{c}^\mathsf{(c),PelS}_{i,j} \}$, $\mathbb{C}_{k,j}^\mathsf{(s)} =\{ B_{k,j}^\mathsf{(s)}\left\langle \mathcal{X} \right\rangle, P_{k,j}^\mathsf{(s)}\left\langle \mathcal{X} \right\rangle, \mathbbm{s}^\mathsf{(s),Pay}_{k,j}\left\langle \mathcal{X} \right\rangle, \mathbbm{c}^\mathsf{(s),PelU}_{k,j}, \mathbbm{c}^\mathsf{(s),PelS}_{k,j} \} $}
\end{algorithm}

\noindent~\textbf{Step 1. Initialization} (line 1, Alg.~\ref{Alg:1}): The negotiation process involves multiple rounds, indexed by $\mathcal{X}$. The initial (i.e., $\mathcal{X}=1$) payment of communication services by an individual MU $u_i$ is set as $p^\mathsf{(c)}_{i,j}\left\langle 1 \right\rangle = p^\mathsf{(c),\min}_{i,j}$, while the initial payment for sensing services by each coalition $\bm{c}_k$ is set as $p^\mathsf{(s)}_{k,j}\left\langle 1 \right\rangle = p^\mathsf{(s),\min}_{k,j}$ (line 1). We also define the set $\mathbb{Y}^\mathsf{(c)}(u_i)$ to include the BSs that $u_i$ is interested in for communication service, and $\mathbb{Y}^\mathsf{(c)}(s_j)$ to capture the MUs temporarily selected by $s_j$ for communication service. Similarly, set $\mathbb{Y}^\mathsf{(s)}(\bm{c}_k)$ comprises the BSs that $\bm{c}_k$ is interested in for sensing service, while $\mathbb{Y}^\mathsf{(s)}(s_j)$ covers the sensing coalitions temporarily selected by $s_j$. 


% \vspace{-14mm}
\noindent~\textbf{Step 2. Establishment of MU coalitions and preference lists} (lines 2-6, Alg.~\ref{Alg:1}): Before the matching process, MUs form sensing coalitions based on their sensing targets. At the beginning of each round $\mathcal{X}$, each individual MU announces its resource requests for communication service to BSs according to its preference list, defined in Definition \ref{def 5}.

\vspace{-2mm}
\begin{Defn}(Preference List of each MU in offRFW$^2$M)\label{def 5}
Consider the set of feasible long-term contract solutions for MU $ u_i$ that satisfy (\ref{equ. PF MU C8}), (\ref{equ. PF MU C9}), and (\ref{equ. PF MU C4}) as {\footnotesize \( \bm{\mathcal{C}^\mathsf{(c),F}}_{i} = \{ \bm{F^{(c)}}_{i,1}, \dots, \bm{F^{(c)}}_{i,\mathbbm{n}}, \dots, \bm{F^{(c)}}_{i,|\bm{\mathcal{C}^\mathsf{(c),F}_{i}}|} \} \)}, where {\footnotesize\( \bm{F^{(c)}}_{i,\mathbbm{n}} = \{s^\mathsf{(c),F}_j, \mathbb{C}^\mathsf{(c),F}_{i,\mathbbm{n}} \} \)} represents a feasible solution, which consists of: (i) the BS {\footnotesize\( s^\mathsf{(c),F}_j \in \bm{\mathcal{S}} \)} to which the request should be sent, (ii) the feasible five-tuple contract item {\footnotesize\( \mathbb{C}^\mathsf{(c),F}_{i,n}= \{ B_{i,\mathbbm{n}}^\mathsf{(c)}, P_{i,\mathbbm{n}}^\mathsf{(c)}, p^\mathsf{(c)}_{i,\mathbbm{n}}, \mathbbm{c}^\mathsf{(c),PelS}_{i,j}, \mathbbm{c}^\mathsf{(c),PelU}_{i,j} \} \)}, where {\footnotesize\( B_{i,\mathbbm{n}}^\mathsf{(c)} \)} and {\footnotesize\( P_{i,\mathbbm{n}}^\mathsf{(c)} \)} represent the required bandwidth and power resources, and {\footnotesize\( p^\mathsf{(c)}_{i,\mathbbm{n}} \)} represents the bid for the current solution, initialized with {\footnotesize$p_{i,j}^\mathsf{(c),min}$}. The preference list \( \overrightarrow{L^\mathsf{(c)}_i} \) of an MU \( u_i \) regarding feasible solutions {\footnotesize\( \bm{F^{(c)}}_{i,\mathbbm{n}} \in \bm{\mathcal{C}^\mathsf{(c),F}}_{i} \)} is a vector of tuples, sorted in non-ascending order based on their expected utility:
% value {\footnotesize\( u^\mathsf{(c),U}(u_i, s^\mathsf{(c),F}_j, \mathbb{C}^\mathsf{(c),F}_{i,\mathbbm{n}}) \)}.
%The preference list \( \overrightarrow{L^\mathsf{(c)}_i} \) of an MU \( u_i \) regarding BSs is a vector of tuples representing contract terms \( \{s_j, B_{i,j}^\mathsf{(c)}, P_{i,j}^\mathsf{(c)}, p_{i,j}^\mathsf{(c)}\} \) under constraint (\ref{equ. PF MU C4}), sorted in non-ascending order based on the expected utility value \( u^\mathsf{(c),U}(u_i, \varphi^\mathsf{(c)}(u_i)) \): 
\begin{equation}
\hspace{-3mm}
	\begin{aligned}
			\overrightarrow{L^\mathsf{(c)}_i} = [\bm{F^{(c)}}_{i,\mathbbm{n}}\in \bm{\mathcal{C}^\mathsf{(c),F}}_{i}, \text{sorted in non-ascending order  based on (\ref{equ. expected comm utility})}].
	\end{aligned}
    \hspace{-3mm}
\end{equation}
%\noindent where \( B_{\min} \) and \( B_{\max} \) represent the lower and upper bounds of bandwidth resources that each MU can request, while \( P_{\min} \) and \( P_{\max} \) denote the minimum and maximum power resources that can be allocated to each MU. 
\end{Defn}
\vspace{-1mm}
Meanwhile, each coalition announces its resource requests to BSs according to its preference list, defined in Definition \ref{def 6}.

\vspace{-2mm}
\begin{Defn}(Preference List of Sensing Coalition in offRFW$^2$M)\label{def 6} Consider the set of feasible long-term contract solutions for coalition $ \bm{c}_k$ that satisfy  (\ref{equ. PF MU C8}), (\ref{equ. PF MU C9}), and (\ref{equ. PF MU C5})  as {\footnotesize\( \bm{\mathcal{C}^\mathsf{(s),F}}_{k} = \{ \bm{F^{(s)}}_{k,1}, \dots, \bm{F^{(s)}}_{k,\mathbbm{m}}, \dots, \bm{F^{(s)}}_{k,|\bm{\mathcal{C}^\mathsf{(s),F}_{k}}|} \} \)}, where {\footnotesize\( \bm{F^{(s)}}_{k,\mathbbm{m}} = \{s^\mathsf{(s),F}_j, \mathbb{C}^\mathsf{(s),F}_{k,\mathbbm{m}} \} \)} represents a feasible solution, which consists of: (i) the BS {\footnotesize\( s^\mathsf{(s),F}_j \in \bm{\mathcal{S}} \)} to which the request should be sent, 
(ii) the feasible five-tuple contract item {\footnotesize\( \mathbb{C}^\mathsf{(s),F}_{k,\mathbbm{m}} = \{ B_{k,\mathbbm{m}}^\mathsf{(s)}, P_{k,\mathbbm{m}}^\mathsf{(s)}, p^\mathsf{(s)}_{i,\mathbbm{m}}, \mathbbm{c}^\mathsf{(s),PelS}_{k,j}, \mathbbm{c}^\mathsf{(s),PelU}_{k,j} \} \)}, where {\footnotesize\( B_{k,\mathbbm{m}}^\mathsf{(s)} \)} and {\footnotesize\( P_{k,\mathbbm{m}}^\mathsf{(s)} \)} represent the required bandwidth and power resources, and {\footnotesize\( p^\mathsf{(s)}_{k,\mathbbm{m}} \)} represents the bid for the current solution, initialized with $p_{k,j}^\mathsf{(s),min}$. The preference list \( \overrightarrow{L^\mathsf{(s)}_k} \) of an coalition \( \bm{c}_k \) regarding feasible solutions {\footnotesize\( \bm{F^{(s)}}_{k,\mathbbm{m}} \in \bm{\mathcal{C}^\mathsf{(s),F}}_{k} \)} is a vector of tuples, sorted in non-ascending order based on their expected utility:
\begin{equation}
\hspace{-2.5mm}
	\begin{aligned}
			\overrightarrow{L^\mathsf{(s)}_k} = [\bm{F^{(s)}}_{k,\mathbbm{m}}\in \bm{\mathcal{C}^\mathsf{(s),F}}_{k},  \text{sorted in non-ascending order  based on (\ref{equ. expected sensing utility})}].
	\end{aligned}
    \hspace{-2mm}
\end{equation}
\end{Defn}
\vspace{-1mm}
\noindent~\textbf{Step 3. Proposal of clients} (lines 7-12, Alg.~\ref{Alg:1}): At round \( \mathcal{X} \), each individual MU \( u_i \) and sensing coalition \( \bm{c}_k \) select their \textit{preferred solutions} {\footnotesize\( F^\mathsf{(c),\star}_{i}\left\langle \mathcal{X} \right\rangle \)} and {\footnotesize\( F^\mathsf{(s),\star}_k\left\langle \mathcal{X} \right\rangle \)} from the first elements of their preference lists {\footnotesize\( \overrightarrow{L_i^\mathsf{(c)}} \)} and {\footnotesize\( \overrightarrow{L_k^\mathsf{(s)}} \)}, respectively. They then record the selected BS \( s_j \) in {\footnotesize\( \mathbb{Y}^\mathsf{(c)}(u_i) \)} for communication and {\footnotesize\( \mathbb{Y}^\mathsf{(s)}(\bm{c}_k) \)} for sensing service, and obtain the request information regarding the required bandwidth, power resources, and payments (e.g., for communication service, {\footnotesize$B_{i,j}^\mathsf{(c)}\left\langle \mathcal{X} \right\rangle\leftarrow B_{i,\mathbbm{n}}^\mathsf{(c)}, P_{i,j}^\mathsf{(c)}\left\langle \mathcal{X} \right\rangle\leftarrow P_{i,\mathbbm{n}}^\mathsf{(c)}, \mathbbm{c}^\mathsf{(c),Pay}_{i,j}\left\langle \mathcal{X} \right\rangle\leftarrow p_{i,\mathbbm{n}}^\mathsf{(c)}$}). Each client then transmits its solution to the BSs in {\footnotesize\( \mathbb{Y}^\mathsf{(c)}(u_i)\)} and {\footnotesize\( \mathbb{Y}^\mathsf{(s)}(\bm{c}_k) \)}, initiating the resource negotiations.
%In the $\mathcal{X}$-th round, each individual MU $u_i$ and sensing coalition $\bm{c}_k$ select their \textit{preferred solutions} $F^\mathsf{(c),\star}_{i}\left\langle \mathcal{X} \right\rangle$ and $F^\mathsf{(s),\star}_k\left\langle \mathcal{X} \right\rangle$ from their preference lists $\overrightarrow{L_i^\mathsf{(c)}}$ and $\overrightarrow{L_j^\mathsf{(s)}}$, and record the selected BS \( s_j \) in \( \mathbb{Y}^\mathsf{(c)}(u_i) \) for communication service and \( \mathbb{Y}^\mathsf{(s)}(\bm{c}_k) \) for sensing service. 
%Specifically, a solution consists of \textit{(i)} the BS to which the request should be sent,
%\textit{(ii)} the bandwidth and power requested by the client, and 
%\textit{(iii)} the payment and cost for the MU. Each client then transmits its solution to the BSs in \( \mathbb{Y}^\mathsf{(c)}(u_i)\) and \( \mathbb{Y}^\mathsf{(s)}(\bm{c}_k) \), initiating the resource negotiations.

\noindent~\textbf{Step 4. Client selection on BSs' side} (lines 13-16, Alg.~\ref{Alg:1}): After collecting the information of individual MUs and sensing coalitions in set ${\widetilde{\mathbb{Y}}}\left(s_j\right)$, each BS $s_j$ solves a two-dimensional 0-1 knapsack problem, which can be solved using dynamic programming (DP) \cite{MY tsc}, to determine a temporary selection of MUs denoted by {\footnotesize$\mathbb{Y}^\mathsf{(c)}(s_j)$} and sensing coalitions {\footnotesize$\mathbb{Y}^\mathsf{(s)}(s_j)$}, where {\footnotesize$\mathbb{Y}^\mathsf{(c)}(s_j)$} and {\footnotesize$\mathbb{Y}^\mathsf{(s)}(s_j)$} belong to {\footnotesize$ {\widetilde{\mathbb{Y}}}\left(s_j\right)$}, maximizing the expected utility of BS \( s_j \) while satisfying constraints (\ref{equ. PF BS C3}), (\ref{equ. PF BS C4}), (\ref{equ. PF BS C5}), and (\ref{equ. PF BS C6}). Then, each $s_j$ reports its decisions to the  MUs and sensing coalitions for the current round.

\noindent~\textbf{Step 5. Decision-making on clients' side} (lines 17-27, Alg.~\ref{Alg:1}): After receiving decisions from BSs, each MU {\footnotesize\( u_i \in \mathbb{Y}^\mathsf{(c)}(s_j) \)} and each coalition{\footnotesize \( \bm{c}_k \in \mathbb{Y}^\mathsf{(s)}(s_j) \)} evaluate their current solutions {\footnotesize$F^\mathsf{(c),\star}_i\left\langle \mathcal{X} \right\rangle$} and {\footnotesize$F^\mathsf{(s),\star}_k\left\langle \mathcal{X} \right\rangle$} . The payment for an individual MU \( u_i \) and coalition \( \bm{c}_k \) remains unchanged if any of the following conditions are met:
\textit{(i)} \( u_i \) or \( \bm{c}_k \) is accepted by \( s_j \); 
\textit{(ii)} the current payment {\footnotesize\( \mathbbm{c}^\mathsf{(c),Pay}_{i,j} \)} or {\footnotesize\( \mathbbm{c}^\mathsf{(s),Pay}_{k,j} \)} equals its valuation {\footnotesize\( V^\mathsf{(c)}_{i,j} \)}; 
\textit{(iii)} constraints (\ref{equ. PF MU C4}), (\ref{equ. PF MU C5}), (\ref{equ. PF MU C6}), and (\ref{equ. PF MU C7}) are not met. 
Otherwise, \( u_i \) or \( \bm{c}_k \) will increase its bid associated with the current solution {\footnotesize$F^\mathsf{(c),\star}_i\left\langle \mathcal{X} \right\rangle$ and $F^\mathsf{(s),\star}_k\left\langle \mathcal{X} \right\rangle$} for \( s_j \) in the next round by $\Delta p$ to enhance its competitiveness in the market.

\noindent~\textbf{Step 6. Repeat} (lines 4-30, Alg.~\ref{Alg:1}): If all the preferred solutions stay unchanged in two consecutive rounds, the matching process terminates. We use $ \sum_{u_i\in\bm{\mathcal{U}}}{flag}_{i}={\bf False} $ and $\sum_{\bm{c}_k\in\bm{\mathcal{C}}}{flag}_{k}={\bf False} $ to capture this (line 5). Otherwise, the next round starts, re-iterating the above steps (lines 4-30).

\noindent~\textbf{Computational complexity:} The computational complexity of our proposed offRFW$^2$M depends on the total number of rounds involved in Alg. 1 (denoted by \( \mathcal{X}^{\mathsf{max}} \)), the overbooked resources {\footnotesize\( (1 + O_j^\mathsf{B}) B_j \)} and {\footnotesize\( (1 + O_j^\mathsf{Pow}) P_j \)}, as well as the number of clients sending requests to BS \( s_j \) in the $ \mathcal{X}^\mathsf{\text{th}} $ round, denoted as {\footnotesize\( |{\widetilde{\mathbb{Y}}}\left(s_j\right)|_{\mathcal{X}} \)}. In particular, the overall complexity of offRFW$^2$M for each BS \( s_j \) is
{\footnotesize $\sum_{\mathcal{X}=1}^\mathsf{\mathcal{X}^{\mathsf{max}}} \mathcal{O}\left( |{\widetilde{\mathbb{Y}}}\left(s_j\right)|_{\mathcal{X}} \times (1 + O_j^\mathsf{B})B_j \times (1 + O_j^\mathsf{Pow}) P_j \right)$}.

% \noindent~\textbf{Solution Characteristics:} This work provides a novel perspective on offline trading-driven long-term contract determination by designing a \textit{role-friendly and win-win matching mechanism that achieves mutually beneficial expected utilities} for both parties, while ensuring acceptable risk levels. From the clients' perspective, the preference list for each client is determined using Definitions \ref{def 5} and \ref{def 6}, under constraints (\ref{equ. PF MU C4}) and (\ref{equ. PF MU C5}), ensuring that the selected solution maximizes their expected utility. From the BSs' perspective, the appropriate contract terms are selected from the feasible solutions reported by clients, utilizing a DP algorithm to maximize the expected utility of the BS (line 15, Alg. 1).

\vspace{-4.25mm}
\subsection{Solution Characteristics and Key Properties}
\vspace{-.55mm}
As offRFW$^2$M is deployed prior to practical transactions, our focus is on maximizing the utilities of clients and BSs while controlling potential risks. This differentiates our approach significantly from conventional matching mechanisms, which primarily emphasize immediate resource allocation without considering long-term contractual stability and risks. We next analyze the key properties of our matching mechanism.

\vspace{-2mm}
\begin{Defn}(Blocking Pairs for Communication Services in offRFW$^2$M)
	Under a given matching $ \varphi^\mathsf{(c)} $, an MU $ u_i $, a BS set $ \mathbb{S} \subseteq \bm{\mathcal{S}}$ and a contract $\mathbb{C}^\prime$, denoted by $ \left(u_i; \mathbb{S}; \mathbb{C}^\prime\right) $, may form one of the two types of blocking pairs.
	
	\noindent \textbf{Type 1 blocking pair:} Type 1 blocking pair satisfies the following two conditions:
	
	\noindent
	$\bullet$ MU $ u_i $ prefers the BS set $ \mathbb{S} \subseteq \bm{\mathcal{S}} $ over its currently matched BS set $ \varphi^\mathsf{(c)}(u_i) $, meaning 
    \vspace{-.2mm}
	\begin{equation}\label{key}
    \hspace{-3mm}
\resizebox{0.36\textwidth}{!}{$
		\begin{aligned}
			\mathbb{E}\left[{u^\mathsf{(c),U}}(u_i,\mathbb{S},\mathbb{C}^\prime)\right ]>\mathbb{E}\left [{u^\mathsf{(c),U}}(u_i,\varphi^\mathsf{(c)}(u_i),\mathbb{C}^\mathsf{(c)}_{i,j})\right]. 
		\end{aligned} 
        $}
	\end{equation}
	
	
	\noindent
	$\bullet$ Every BS in $ \mathbb{S} $ prefers to serve MU \( u_i \) rather than its matched MU set. That is, for any $ s_j\in \mathbb{S} $, there exists a set $ \varphi^\mathsf{(c)\prime}(s_j) $ that constitutes the MUs that need to be evicted, satisfying
	\begin{equation}\label{equ. 42}
    \hspace{-3mm}
    \resizebox{0.46\textwidth}{!}{$
		\begin{aligned}
\mathbb{E}\hspace{-.5mm}\left[\hspace{-.4mm}u^\mathsf{(c),S}\hspace{-.5mm}\left(\hspace{-.5mm}s_j,\hspace{-.5mm}\left\{\hspace{-.5mm}\varphi\left( s_j \right)\hspace{-.5mm}\backslash\varphi^\mathsf{(c)\prime}\left( s_j \right)\hspace{-.5mm}\right\} {\cup} \left\{ u_i \right\}\hspace{-.5mm},\hspace{-.5mm}\mathbb{C}^\prime\hspace{-.5mm} \right)\hspace{-.5mm}\right ]\hspace{-.5mm} {>} \mathbb{E}\hspace{-.5mm}\left [\hspace{-.4mm}u^\mathsf{(c),S}\hspace{-.5mm}\left(\hspace{-.5mm}s_j,\varphi\hspace{-.5mm}\left( s_j,\mathbb{C}^\mathsf{(c)}_{i,j} \right)\hspace{-.5mm} \right)\hspace{-.5mm}\right]\hspace{-.85mm}.
		\end{aligned}
        $}
         \hspace{-3mm}
	\end{equation} 
	
	\noindent \textbf{Type 2 blocking pair:} Type 2 blocking pair satisfies the following two conditions:
	
	\noindent
	$\bullet$ MU $ u_i $ prefers the BS set $ \mathbb{S} \subseteq \bm{\mathcal{S}} $ over its currently matched BS set $ \varphi^\mathsf{(c)}(u_i) $, meaning
    \vspace{-.2mm}
	\begin{equation}\label{key}
    \hspace{-3mm}
\resizebox{0.37\textwidth}{!}{$
		\begin{aligned}
			\mathbb{E}\left[{u^\mathsf{(c),U}}(u_i,\mathbb{S}, \mathbb{C}^\prime )\right]>\mathbb{E}\left [{u^\mathsf{(c),U}}(u_i,\varphi^\mathsf{(c)}(u_i),\mathbb{C}^\mathsf{(c)}_{i,j} )\right].
		\end{aligned}
        $}
	\end{equation} 
	
	\noindent
	$\bullet$ Every BS in $ \mathbb{S} $ prefers to further serve MU $ u_i $ in conjunction to its matched MU set. That is, for any BS $ s_j\in \mathbb{S} $, we have
	\begin{equation}\label{equ. 44}
     \hspace{-3mm}
\resizebox{0.475\textwidth}{!}{$
		\begin{aligned}
			&\mathbb{E}\left[u^\mathsf{(c),S}(s_j,\varphi^\mathsf{(c)}(s_j)\cup\left\{ u_i \right\}, \mathbb{C}^\prime)\right ]>\mathbb{E}\left [u^\mathsf{(c),S}(s_j,\varphi^\mathsf{(c)}(s_j),\mathbb{C}^\mathsf{(c)}_{i,j} )\right ] .
		\end{aligned} 
        $}\hspace{-5mm}
	\end{equation}
\end{Defn}

\vspace{-3.75mm}
\begin{Defn}(Blocking Pairs for Sensing Services in offRFW$^2$M)
	Under a given matching $ \varphi^\mathsf{(s)} $, coalition $ \bm{c}_k $, BS set $ \mathbb{S} \subseteq \bm{\mathcal{S}}$, and a contract $\mathbb{C}^\prime$, denoted by $ \left(\bm{c}_k; \mathbb{S}; \mathbb{C}^\prime\right) $, may form one of the following two types of blocking pairs.
	
	\noindent \textbf{Type 1 blocking pair:} Type 1 blocking pair satisfies the following two conditions:
	
	\noindent
	$\bullet$ Coalition $ \bm{c}_k $ prefers the BS set $ \mathbb{S} \subseteq \bm{\mathcal{S}} $ rather than its currently matched BS set $ \varphi^\mathsf{(s)}(\bm{c}_k) $, meaning 
    \vspace{-.2mm}
	\begin{equation}\label{key}
    \hspace{-3mm}
\resizebox{0.38\textwidth}{!}{$
		\begin{aligned}
			\mathbb{E}\left [{u^\mathsf{(s),U}}(\bm{c}_k,\mathbb{S},\mathbb{C}^\prime)\right ]>\mathbb{E}\left [{u^\mathsf{(s),U}}(\bm{c}_k,\varphi^\mathsf{(s)}(\bm{c}_k),\mathbb{C}^\mathsf{(s)}_{k,j})\right ]. 
		\end{aligned} 
        $}
	\end{equation}
	
	\noindent
	$\bullet$ Every BS in $ \mathbb{S} $ prefers to serve  \( \bm{c}_k \) rather than its matched coalitions. That is, for any $ s_j\in \mathbb{S} $, there exists a set $ \varphi^\mathsf{(s)\prime}(s_j) $ that constitutes the coalitions that need to be evicted, satisfying
	\begin{equation}\label{equ. 46}
         \hspace{-3mm}
\resizebox{0.480\textwidth}{!}{$
		\begin{aligned}
			&\mathbb{E}\hspace{-.6mm}\left [\hspace{-.4mm} u^\mathsf{(s),S}\hspace{-.7mm}\left(\hspace{-.5mm}s_j,\hspace{-.5mm}\left\{\hspace{-.5mm}\varphi\left( s_j \right)\hspace{-.5mm}\backslash\varphi^\mathsf{(s)\prime}\left( s_j \right)\hspace{-.5mm}\right\} \hspace{-.6mm}\cup\hspace{-.6mm} \left\{ \bm{c}_k \right\}\hspace{-.5mm},\hspace{-.5mm}\mathbb{C}^\prime \right)\hspace{-.5mm}\right ] \hspace{-.8mm}>\hspace{-.6mm} \mathbb{E}\hspace{-.5mm}\left [\hspace{-.4mm}u^\mathsf{(s),S}\hspace{-.5mm}\left(s_j,\varphi\hspace{-.7mm}\left(\hspace{-.5mm} s_j,\mathbb{C}^\mathsf{(s)}_{k,j},\mathbb{C}^\mathsf{(s)}_{k,j} \right) \hspace{-.5mm}\right)\hspace{-.5mm}\right ]\hspace{-.8mm}.
		\end{aligned}
        $}
           \hspace{-5.5mm}
	\end{equation} 
	
	\noindent \textbf{Type 2 blocking pair:} Type 2 blocking pair satisfies the following two conditions:
	
	\noindent
	$\bullet$ Coalition $ \bm{c}_k $ prefers the BS set $ \mathbb{S} \subseteq \bm{\mathcal{S}} $ to its currently matched BS set $ \varphi^\mathsf{(s)}(u_i) $, meaning
	\begin{equation}\label{key}
    \hspace{-3mm}
\resizebox{0.358\textwidth}{!}{$
		\begin{aligned}
			\mathbb{E}\left [{u^\mathsf{(s),U}}(\bm{c}_k,\mathbb{S},\mathbb{C}^\prime )\right ]>\mathbb{E}\left [{u^\mathsf{(s),U}}(\bm{c}_k,\varphi^\mathsf{(s)}(\bm{c}_k),\mathbb{C}^\mathsf{(s)}_{k,j} )\right ].
		\end{aligned}
        $}
	\end{equation} 
	
	\noindent
	$\bullet$ Every BS in $ \mathbb{S} $ prefers to further serve $ \bm{c}_k $ in addition to its matched coalition set. That is, for any BS $ s_j\in \mathbb{S} $, we have
	\begin{equation}\label{equ. 48}
    \hspace{-3mm}
\resizebox{0.47\textwidth}{!}{$
		\begin{aligned}
			&\mathbb{E}\left [u^\mathsf{(s),S}(s_j,\varphi^\mathsf{(s)}(s_j)\cup\left\{ \bm{c}_k \right\},\mathbb{C}^\prime)\right ]>\mathbb{E}\left [u^\mathsf{(s),S}(s_j,\varphi^\mathsf{(s)}(s_j),\mathbb{C}^\mathsf{(s)}_{k,j} )\right ] .
		\end{aligned} 
        $}    \hspace{-5mm}
	\end{equation}
\end{Defn}
In essence, a Type 1 blocking pair undermines the stability of the matching by incentivizing a BS to reallocate its resources to a different set of clients that yield a higher  utility. Similarly, a Type 2 blocking pair introduces instability, as the BS possesses residual resources that could be allocated to additional clients, further increasing its utility. These blocking pairs are used in the following to define the major characteristics of offRFW$^2$M.

\vspace{-2mm}
\begin{Prop}\label{Prop 1}(Individual Rationality of offRFW$^2$M) The proposed offRFW$^2$M mechanism ensures individual rationality for BSs, individual MUs, and sensing coalitions as follows:
	
	\noindent
	$\bullet$ For each BS: \textit{(i)} the bandwidth and power resources of BS $s_j$ booked to matched clients $\varphi^\mathsf{(c)}\left(s_j\right)$ and coalitions $\varphi^\mathsf{(s)}\left(s_j\right)$ does not exceed $(1+O_j^\mathsf{B})B_j$ and $(1+O_j^\mathsf{Pow})P_j$ after applying overbooking, i.e., constraints (\ref{equ. PF BS C3}) and (\ref{equ. PF BS C4}) are met; \textit{(ii)} the risks associated with each BS are maintained within a certain acceptable range, i.e., constraint (\ref{equ. PF BS C5}) and (\ref{equ. PF BS C6}) are satisfied.
	
	\noindent
	$\bullet$ For each client (i.e., each MU and each coalition): \textit{(i)} The value obtained by each client is at least equal to the payment it makes, ensuring that (\ref{equ. PF MU C3}) is met; \textit{(ii)} the risks associated with each client are  acceptable, satisfying  (\ref{equ. PF MU C4})-(\ref{equ. PF MU C7}).
\end{Prop}

\vspace{-4mm}
\begin{Prop}(Fairness of offRFW$^2$M): The proposed offRFW$^2$M ensures fairness by preventing the formation of Type 1 blocking pairs, ensuring that clients are satisfied with their matched BSs and no BS is incentivized to reallocate its resources to a different set of clients.\end{Prop}

\vspace{-4mm}
\begin{Prop}(Non-wastefulness of offRFW$^2$M): offRFW$^2$M guarantees non-wastefulness by preventing the formation of Type 2 blocking pairs, ensuring that BSs efficiently utilize their resources without leaving a surplus.\end{Prop}
\vspace{-4mm}
\begin{Prop}(Strong Stability of offRFW$^2$M)\label{Prop 4} The proposed offRFW$^2$M achieves strong stability by ensuring that the matching remains individually rational, fair, and non-wasteful.
\end{Prop}

\vspace{-4mm}
\begin{Prop}(Stability of Sensing Coalitions in offRFW$^2$M)
In offRFW$^2$M, each sensing coalition $\bm{c}_k$ is stable, that is 
	
\noindent $\bullet$ For each MU $u_i$ in $\bm{c}_k$, its expected utility is above $u^\mathsf{(s)}_\mathsf{\min}$. 

\noindent $\bullet$ For each MU $u_i$ in sensing coalition $\bm{c}_k$, the expected utility obtained by joining the coalition \( \bm{c}_k \) (shown in (\ref{equ. expected sensing utility})) is lower than the expected utility when trading as an individual.
\end{Prop}


%Note that competitive equilibrium represents a conventional concept in economic behaviors, playing an important role in analyzing the performance of commodity markets upon having flexible prices and multiple players. When the considered market arrives at the competitive equilibrium, there exists a price at which the number of BS owners that will pay is equal to the number of MUs that will sell\cite{RW Matching5}. Correspondingly, the competitive equilibrium of FT-M2M matching is defined below.
%
%\begin{Defn}(Competitive equilibrium associated with trading between MUs and BS owners in offline trading mode) The trading between MUs and BS reaches a competitive equilibrium if the following conditions are satisfied:
%	
%	\noindent
%	$\bullet$ For each MU $ u_i \in \bm{\mathcal{U}} $, if $ u_i $ is associated with a BS $ s_j\in \bm{\mathcal{S}} $, then $ \mathbb{E}[V^\mathsf{(c)}_{i,j}]\geq \mathbbm{c}^\mathsf{(c),Pay}_{i,j} $ and $ \mathbb{E}[V^\mathsf{(s)}_{i,j}]\geq p^\mathsf{(s)}_{i,j} $,
%	
%	\noindent
%	$\bullet$ For each BS $ s_j\in \bm{\mathcal{S}} $, $ s_j $ is willing to trade with the MU that can bring it with the maximum expected utility,
%	
%	\noindent
%	$\bullet$ For each BS \( s_j \) in set \( \bm{\mathcal{S}} \), when \( s_j \) does not serve more MUs, it indicates that the remaining resource after deducting the payments made to matched MUs is insufficient to serve an additional MU.
%\end{Defn}

Note that for the MOO problem defined by $ \bm{\mathcal{F}^\mathsf{U}} $ and $ \bm{\mathcal{F}^\mathsf{S}} $, a Pareto improvement occurs when the expected social welfare (i.e., the summation of expected utilities of clients and BSs)\cite{RW Matching3} can be increased with another matching result. A matching is thus weakly Pareto optimal when no further Pareto improvement is possible, which is a desired property.

%For the MOO problem collectively given by $ \bm{\mathcal{F}^\mathsf{U}} $ and $ \bm{\mathcal{F}^\mathsf{S}} $, a Pareto improvement occurs when the \textit{expected social welfare (referring to a summation of expected utilities of clients and BSs in our considered market)}\footnote{The expected utilities of MUs are actually composed of two parts: \textit{(i)} the expected utility of individual MUs for resource trading in communication services, and \textit{(ii)} the expected utility of each MU within a coalition for resource trading in communication services. For easy analysis, we collectively refer to these as the expected utilities of MUs.} can be increased with another feasible matching result\cite{RW Matching3}. Thus, a matching is weak Pareto optimal when there is no Pareto improvement.

\vspace{-2mm}
\begin{Prop}(Weak Pareto Optimality of offRFW$^2$M)  offRFW$^2$M is weakly Pareto optimal, ensuring that no other matching can increase the social welfare of the system.
\end{Prop}
The detailed proofs of all the above results are presented in Appendix C.

\vspace{-2mm}
\section{Effective Backup Win-Win Matching for Online Trading (onEBW$^2$M)}
\vspace{-1mm}
\noindent The intermittent participation of MUs may hinder the seamless execution of pre-signed long-term contracts during practical transactions, potentially resulting in financial losses for both parties. To address this, we introduce an online trading mode as a complementary mechanism, handling two scenarios: \textit{(i)} When the resource demand at a BS exceeds its available supply, the BS strategically designates certain long-term contract clients as \textit{voluntary contributors} using a greedy-based approach. These clients forgo their services in exchange for compensations. \textit{(ii)} When there exist clients with unmet resource demands ---including voluntary clients and those without long-term contracts --- as well as BSs with surplus resources, we implement the onEBW$^2$M mechanism, which obtains two types of \textit{temporary contracts}: 
\textit{(i)} for communication service, a contract between an individual MU $u_i$ and a BS $s_j$, represented as {\footnotesize$\dot{\mathbb{C}}^\mathsf{(c)}_{i,j}$}, and 
\textit{(ii)} for sensing service, a contract between a sensing coalition $\bm{c}_k$ and a BS $s_j$, represented as {\footnotesize$\dot{\mathbb{C}}^\mathsf{(s)}_{k,j}$}. 
These temporary contracts consist of two terms: the quantity of trading resources {\footnotesize$\dot{\mathbbm{c}}^\mathsf{(c),B}_{i,j}, \dot{\mathbbm{c}}^\mathsf{(c),Pow}_{i,j}, \dot{\mathbbm{c}}^\mathsf{(s),B}_{k,j}, \dot{\mathbbm{c}}^\mathsf{(s),Pow}_{k,j}$}, and the service price {\footnotesize$\dot{\mathbbm{c}}^\mathsf{(c),Pay}_{i,j}, \dot{\mathbbm{c}}^\mathsf{(s),Pay}_{k,j}$} (super-scripts have the same meaning as those in Sec.~\ref{sec:over}).


% This dynamic adjustment mechanism further enhances the overall utilities of both BSs and clients, optimizing the efficiency of the trading process.

%  In particular, for communication services, a contract between an individual MU $u_i$ and a BS $s_j$ is represented as $\mathbb{C}^\mathsf{(c)}_{i,j} = \{\mathbbm{c}^\mathsf{(c),B}_{i,j}, \mathbbm{c}^\mathsf{(c),Pow}_{i,j}, \mathbbm{c}^\mathsf{(c),Pay}_{i,j}, \mathbbm{c}^\mathsf{(c),PelU}_{i,j}, \mathbbm{c}^\mathsf{(c),PelS}_{i,j}\}$, where $\mathbbm{c}^\mathsf{(c),B}_{i,j}$ and $\mathbbm{c}^\mathsf{(c),Pow}_{i,j}$denote the allocated bandwidth and power, $\mathbbm{c}^\mathsf{(c),Pay}_{i,j}$ represents the unit price, and $\mathbbm{c}^\mathsf{(c),PelU}_{i,j}$ and $\mathbbm{c}^\mathsf{(c),PelS}_{i,j}$are default clauses specifying penalties if either party breaches the contract.
%Similarly, for sensing services, a long-term contract between a sensing coalition $\bm{c}_k$ and a BS $s_j$ as $\mathbb{C}^\mathsf{(s)}_{k,j} = \{\mathbbm{c}^\mathsf{(s),B}_{k,j}, \mathbbm{c}^\mathsf{(s),Pow}_{k,j}, \mathbbm{c}^\mathsf{(s),Pay}_{k,j}, \mathbbm{c}^\mathsf{(s),PelU}_{k,j}, \mathbbm{c}^\mathsf{(s),PelS}_{k,j}\}$, where $\mathbbm{c}^\mathsf{(s),B}_{k,j}$ and $\mathbbm{c}^\mathsf{(s),Pow}_{k,j}$

To identify the MUs and BSs that participate in online trading (i.e., in onEBW$^2$M), we introduce the following notations:

\noindent $\bullet$ $\bm{\mathcal{U}^\prime}$: The set of MUs that trade with BSs during online trading (MUs without long-term contracts and volunteers), $\bm{\mathcal{U}^\prime}\subseteq \bm{\mathcal{U}}$;

\noindent $\bullet$ $\bm{\mathcal{C}^\prime}$: The set of coalitions (without long-term contracts and volunteers) that trade with BSs during online trading, $\bm{\mathcal{C}^\prime}\subseteq \bm{\mathcal{C}}$;

\noindent $\bullet$ $\bm{\mathcal{S}^\prime}$:The set of BSs with available resources beyond those specified in long-term contracts,  $\bm{\mathcal{S}^\prime}\subseteq \bm{\mathcal{S}}$;

\noindent $\bullet$ $\nu^\mathsf{(c)}(s_j)$ and $\nu^\mathsf{(s)}(s_j)$: The set of MUs and coalitions served by BS \( s_j \) for communication and sensing services in online trading mode, where $\nu^\mathsf{(c)}(s_j)$ and $\nu^\mathsf{(s)}(s_j)$$ \subseteq \bm{\mathcal{U}^\prime}$;

\noindent $\bullet$ $\nu^\mathsf{(c)}(u_i)$ and $\nu^\mathsf{(s)}(\bm{c}_k)$: The BS that serves MU \( u_i \) and coalition \( \bm{c}_k \) for communication and sensing services in online trading mode, where $\nu^\mathsf{(c)}(u_i)\subseteq \bm{\mathcal{S}^\prime}$ and $\nu^\mathsf{(s)}(u_i)$ $ \subseteq \bm{\mathcal{S}^\prime}$.

In the online trading mode, we define the practical utility of a client in communication and sensing services as the difference between its obtained valuation and its payments as follows:
\begin{equation}\label{eq:49_n}
	u^\mathsf{U^\prime,(c)}(u_i,\nu^\mathsf{(c)}(u_i),\dot{\mathbb{C}}^\mathsf{(c)}_{i,j})=V^\mathsf{(c)}_{i,j}-\dot{\mathbbm{c}}^\mathsf{(c),Pay}_{i,j},
\end{equation}
\begin{equation}\label{eq:49_n2}
	u^\mathsf{U^\prime,{(s)}}(\bm{c}_{k},\nu^\mathsf{(s)}(\bm{c}_{k}),\dot{\mathbb{C}}^\mathsf{(s)}_{k,j})=V^\mathsf{(s)}_{k,j}-\dot{\mathbbm{c}}^\mathsf{(s),Pay}_{k,j}.
\end{equation}
Similarly, the utility of BS $s_j \in \bm{\mathcal{S}^\prime}$ for communication and sensing services is calculated via its received payments as 
\begin{equation}\label{eq:50_n}
	u^\mathsf{S^\prime,{(c)}}(s_j,\nu^\mathsf{(c)}(s_j),\dot{\mathbb{C}}^\mathsf{(c)}_{i,j})=\sum_{u_i\in\nu^\mathsf{(c)}(s_j)}\dot{\mathbbm{c}}^\mathsf{(c),Pay}_{i,j},
\end{equation}
\begin{equation}\label{eq:50_n2}
	u^\mathsf{S^\prime,{(s)}}(s_j,\nu^\mathsf{(s)}(s_j),\dot{\mathbb{C}}^\mathsf{(s)}_{k,j})=\sum_{u_i\in\nu^\mathsf{(s)}(s_j)}\dot{\mathbbm{c}}^\mathsf{(s),Pay}_{k,j}.
\end{equation}
\vspace{-3.7mm}

During online trading, the objective of each client and BS is \textit{to maximize its practical utility (e.g., \eqref{eq:49_n},\eqref{eq:49_n2} similar to~\eqref{equ. PF MU}, and summing \eqref{eq:50_n} and \eqref{eq:50_n} similar to \eqref{equ. PF BS}), resampling the MOO problem in Sec.~\ref{sec:probForm}.} Subsequently, for brevity, since onEBW$^2$M is similar to offRFW$^2$M, the detailed formulation of the optimization for onEBW$^2$M, the definitions of matchings $\nu^\mathsf{(c)}(.)$ and $\nu^\mathsf{(s)}(.)$, the solution design and its characteristics are provided in Appendix D.

\vspace{-3mm}
\section{Numerical Evaluations}
\noindent We conduct experiments under both synthetic settings (Sec.~\ref{sec:Synt}) and a real-world dataset (Sec.~\ref{sec:Synt}) to evaluate the performance of our framework. For clarity, our future resource bank framework for ISAC is abbreviated as "FRBank." For synthetic simulations, we adopt the following parameters: $|\bm{\mathcal{Q}}|=8$, $N^\mathsf{T}_j\in[8,16]$, $N^\mathsf{R}_i\in[4,8]$, $B_j\in[80,120]$MHz, $P_j\in[10, 20]$dBW, $R_i^\mathsf{req}\in[0.01,10]$bit/s, $S_i^\mathsf{req}\in[1,100]$, $B_0=180$KHz\cite{RW ISAC1}, and $\mathbbm{a}_i\in[0.64,0.96]$\cite{MY tsc}.
For real-world experiments, we employ the EUA dataset\cite{EUA dataset}, which provides information about BSs and MUs in Melbourne metropolitan area of Australia, covering over 9,000 km$^2$. We utilize the Monte-Carlo method, where each value in the figures is the average of over 100 independent tests.

\vspace{-4mm}
\subsection{Benchmark Methods and Evaluation Metrics}
\vspace{-.5mm}
% We incorporate relevant benchmark methods. 
We first consider two conventional resource trading benchmarks, each relying on a single trading mode.

\noindent $\bullet$ \textbf{Conventional online resource trading (ConOnline)}\cite{RW Matching3}: This method relies solely on online resource trading. It conducts M2M matching between BSs and MUs in every practical transaction for communication and sensing services.

\noindent $\bullet$ \textbf{Conventional offline resource trading (ConOffline)}\cite{future 1}: This method relies on offline resource trading. It conducts M2M matching to obtain long-term contracts between BSs and MUs for communication and sensing services, which are executed during practical transactions without any backup mechanisms.

% Incorporating offline and online trading modes into a hybrid trading approach is a unique aspect of our work. Thus, 
We also consider two hybrid trading benchmarks from the broader networking area  to verify the necessity of integrating different trading modes, sensing coalitions, and overbooking in our method, tailored for dynamic ISAC networks.

\noindent $\bullet$ \textbf{Hybrid trading for resource provision (Hybrid)}\cite{MY tsc}: This method integrates the ConOffline and ConOnline methods but does not incorporate overbooking or sensing coalitions.

\noindent $\bullet$ \textbf{Hybrid trading for resource provision with overbooking (HybridO)}: This method is similar to the above method but considers overbooking when determining long-term contracts.
% , ensuring a more adaptable resource allocation strategy.

We also incorporate a greedy resource provisioning method.

\noindent $\bullet$ \textbf{Greedy-driven resource provisioning (Greedy)}\cite{MY tsc}: This method is an online trading where BSs select MUs offering  highest payments, while satisfying its resource constraints.

To quantify the performance, we consider the metrics below:

\noindent $\bullet$ \textbf{Social welfare:} The summation of utilities of MUs and BSs for both communication and sensing services (i.e., summing (\ref{equ. comm utility}), (\ref{equ. sensing utility}), (\ref{equ. comm BS U}), and (\ref{equ. sensing BS U})). Here, the utility of MUs for sensing services captures the profit obtained by each MU in its coalition.

\noindent $\bullet$ \textbf{Utility of MUs and BSs}: The respective utilities received by MUs and BSs in the trading process.

\noindent $\bullet$ \textbf{Running time (RT)}: The delay cost (in ms) incurred in searching for matching results, reflecting the time efficiency.

\noindent $\bullet$ \textbf{Number of interactions (NI)}: The number of clients-to-BSs interactions to obtain matchings, measuring the  overhead.

\noindent $\bullet$ \textbf{Delay caused by interactions between BSs and clients (DIBC)}: The duration of decision-making (in ms). We presume the delay of each interaction to be $[1, 15]$ ms\cite{MY tsc,E2E1,E2E2}.

\noindent $\bullet$ \textbf{Energy consumption incurred by interactions between BSs and clients (ECIBC)}: The energy (in Watts) consumed during the decision-making process. To obtain it, we assume that the transmit power of MUs lies in $[0.2, 0.4]$ watts, while the transmit power of BSs lies in $[6, 20]$ Watts \cite{MY tsc,E2E1,E2E2}.



\vspace{-4mm}
\subsection{Synthetic Experiments}\label{sec:Synt}
\vspace{-0.5mm}
\subsubsection{Social Welfare and Utility of MUs/BSs}
\begin{figure}[]
	\vspace{-0.45cm}
	\centering
	\setlength{\abovecaptionskip}{-1 mm}
	\includegraphics[width=1\columnwidth]{images/1.pdf}
	\caption{Performance comparisons in terms of social welfare, utility of BSs, and utility of MUs. We consider 5 BSs in (a)-(c), and 7 BSs in (d)-(f).}
	\label{SW}
    \vspace{-6mm}
\end{figure}
Social welfare, along with the utilities of MUs and BSs that constitute it, serve as crucial metrics for assessing the effectiveness of our method,  depicted in Fig. \ref{SW} (Figs. \ref{SW}(a)-(c) and \ref{SW}(d)-(f) consider 5 and 7 BSs, respectively, to capture different market scales).

In Fig. \ref{SW}(a), the curves of benchmark methods such as ConOnline, ConOffline, Hybrid, and HybridO, show an upward trend when the number of MUs is below 60. This is because an increase in the number of MUs leads to higher resource demand, thereby improving the utilities of both MUs and BSs (also evident in Figs. \ref{SW}(b) and \ref{SW}(c)). However, when the number of MUs exceeds 60, these curves stabilize as bidding approaches the maximum utility that MUs can achieve. This phenomenon is particularly evident in Fig. \ref{SW}(c), where MU utilities decline and stabilize once the number of MUs surpasses 60.  
In contrast, the curves of our FRBank demonstrate mostly an upward trend, attributed to the ability of additional MUs to derive sensing value by joining sensing coalitions. Notably, in Fig. \ref{SW}(c), the MU utility curve of FRBank declines when the number of MUs is around 70. This decline results from intensified competition driving up bids, thereby reducing MU utilities. However, as the number of MUs continues to grow and bidding reaches its upper limit, newly participating MUs benefit from sensing coalitions, leading to an increase in MU utilities. 
Moreover, the overbooking strategy embedded in HybridO mitigates competition among MUs compared to Hybrid, leading to a slight reduction in social welfare. However, this strategy effectively handles resource demand fluctuations, making HybridO superior to Hybrid in terms of interaction overhead (later revealed in Fig. \ref{NI}). The well-designed online trading mode, acting as a complementary mechanism, enables both Hybrid and HybridO to achieve higher social welfare than ConOffline by accommodating more MUs.
Also, Greedy exhibits the weakest performance, as its design prioritizes maximizing BS revenues over efficient resource utilization.
Further, qualitatively similar results are obtained for a larger market scale with 7 BSs in Figs. \ref{SW}(d)-(f).
% demonstrate that our FRBank significantly outperforms other benchemark methods in terms of social welfare performance. This further underscores the effectiveness of incorporating sensing coalitions as team clients in the trading market, enabling more efficient resource utilization and fostering a more competitive and cooperative trading environment. 



\subsubsection{Evaluation on Interaction Overhead}
\begin{figure*}[]
	\centering
    \vspace{-9mm}
	\setlength{\abovecaptionskip}{-2 mm}
	\includegraphics[width=2\columnwidth]{images/NI.pdf}
	\caption{Performance comparisons in terms of RT, NI, DIBC, and ECIBC metrics.}
	\label{NI}
	\vspace{-0.5cm}
\end{figure*}
% The efficiency of time and energy consumption during interactions before finalizing a transaction is a critical factor in evaluating service delivery performance, particularly in dynamic ISAC networks. 
To provide a quantitative assessment of time and energy efficiency, we consider four key metrics illustrated in Fig.~\ref{NI}, each reflecting the decision-making overhead, including: RT (Figs. \ref{NI}(a) and (e)), NI (Figs. \ref{NI}(b) and (f)), DIBC (Figs. \ref{NI}(c) and (g)), and ECIBC (Figs. \ref{NI}(d) and (h)). The y-axis of subfigures in Fig. \ref{NI} is presented on a logarithmic scale to enhance the visualizations. Also, ConOffline is excluded from the comparison here, as it operates exclusively in the offline trading mode and therefore does not incur real-time decision-making delays or overhead. 

Figs. \ref{NI}(a) and \ref{NI}(e) depict RT performance across different market scales. ConOnline exhibits significantly higher RT compared to other methods since BSs and MUs require substantial time to determine matching results during each transaction. This issue becomes more pronounced as resource demand intensifies with an increasing number of MUs.
Our FRBank significantly reduces RT compared to ConOnline, as many MUs bypass online trading due to the integration of the overbooking strategy, the sensing coalitions, and the pre-established long-term contracts. Also, the well-structured overbooking mechanism in HybridO enables it to achieve a lower RT compared to Hybrid, as the majority of contracts are successfully fulfilled, thereby minimizing the number of participants involved in real-time transactions. In addition, while Greedy achieves a RT comparable to Hybrid due to the absence of price negotiations, its overall social welfare performance remains suboptimal (see Fig. \ref{SW}).


Figs. \ref{NI}(b)-(d) and Figs. \ref{NI}(f)-(h) illustrate the interaction overhead between BSs and clients, encompassing NI, as well as the time (DIBC) and energy consumption (ECIBC) incurred in reaching matching decisions. The results demonstrate that ConOnline exhibits the worst performance due to its reliance on online trading, leading to significant interaction overhead. In contrast, the structured offline trading mode in Hybrid, HybridO, and FRBank significantly reduces interaction overhead by limiting the number of active participants in each practical transaction. Notably, FRBank and HybridO surpass Hybrid due to the overbooking mechanism, which proactively mitigates the impact of uncertain MU availability and enhances the probability of contract fulfillment.
Moreover, FRBank outperforms all other methods by incorporating sensing coalitions as strategic market participants. These coalitions act as single decision-making entities, effectively streamlining negotiation complexity and reducing the volume of interactions required in both offline and online trading phases. This coalition-based structure not only enhances trading efficiency but also ensures better resource allocation and task execution.
While Greedy exhibits minimal interaction overhead (as seen in Fig. \ref{NI}(a)), this efficiency comes at the expense of weakened social welfare and reduced BS revenue. In summary, FRBank establishes a well-balanced framework, minimizing interaction costs while maintaining economic viability for both BSs and MUs, ultimately leading to superior social welfare and system-wide efficiency.



\subsubsection{Individual Rationality}
\begin{figure} \centering 
%	\vspace{-1.9999cm}
%	\subfigtopskip=2pt
%	\subfigbottomskip=10pt
%	\subfigcapskip=-2.0cm
\vspace{-3mm}
	\setlength{\abovecaptionskip}{0.1 cm}
	\subfigure[] {
		\label{fig:a} 
		\includegraphics[width=0.45\columnwidth]{images/IR_MU.pdf} 
	} \hspace{-4mm}
	\subfigure[] { 
		\label{fig:b} 
		\includegraphics[width=0.45\columnwidth]{images/IR_BS.pdf} 
	} 
    \vspace{-2mm}
	\caption{Demonstration of individual rationality of MUs and BSs.} 
	\label{fig} 
	\vspace{-0.6 cm}
\end{figure} 
 Fig. \ref{fig:a} quantifies the \textit{value} accrued by MUs (15 MUs are randomely selected from a pool of 50) from accessing both communication and sensing services, along with their corresponding payment obligations. As can be seen, the paid payment of MUs for communication and sensing services never exceed the values they receive. This validates that our designed offRFW$^2$M and onEBW$^2$M in FRBank effectively uphold the individual rationality of MUs.
Additionally, Fig. \ref{fig:b} validates the individual rationality of the 5 BSs, demonstrating that the allocated bandwidth and power for each BS never exceed their total available resources. This is ensured by the \textit{voluntary} selection mechanism during online trading in FRBank, which prevents resource shortages. 
% These findings further reinforce the individual rationality of BSs within our FRBank framework, ensuring sustainable and incentive-compatible resource allocation.

\subsubsection{Performance Analysis of Our Unique Considerations}

This paper introduces several novel aspects that are unexplored in existing ISAC research. Specifically, we highlight three key innovations: sensing coalition, overbooking, and risk analysis. To systematically evaluate their advantages and impact, we conduct dedicated analysis as follows.


\noindent~\textit{(a) Analysis of sensing coalitions (Figs. \ref{SW}-\ref{NI}):} 
% The sensing coalition is formed by grouping MUs with aligned sensing objectives, enabling them to share both profits and costs. 
As illustrated in Fig. \ref{SW}, incorporating sensing coalitions allows FRBank to achieve the highest MU utility and social welfare. Moreover, considering sensing coalitions can significantly improve interaction efficiency, as evidenced by superior performance in NI, time, and energy consumption (see Fig. \ref{NI}) during the bargaining and decision-making process. 
This improvement stems from the fact that each coalition appoints a representative MU to engage in negotiations, effectively reducing the number of active participants and simplifying the matching process. 
% Consequently, this not only optimizes resource allocation but also minimizes negotiation overhead, reinforcing the efficiency of our proposed FRBank framework.

\begin{figure}[t]
\centering 
\vspace{-0mm}
	%	\vspace{-1.9999cm}
	%	\subfigtopskip=2pt
	%	\subfigbottomskip=10pt
	%	\subfigcapskip=-2.0cm
	\setlength{\abovecaptionskip}{0 cm}
	\subfigure[] {
		\label{overbook} 
		\includegraphics[width=0.45\columnwidth]{images/overbook.pdf} 
	} \hspace{-4mm}
	\subfigure[] { 
		\label{risk} 
		\includegraphics[width=0.45\columnwidth]{images/risk.pdf} 
	} 
    \vspace{-0.5mm}
	\caption{Performance comparisons in terms of RDSLC and DRLC.} 
	\label{fig} 
	\vspace{-0.6 cm}
\end{figure}
\noindent~\textit{(b) Analysis of overbooking (Fig. \ref{overbook}):} In our FRBank,  overbooking enables BSs to mitigate fluctuations in MU demand during practical transactions by strategically overbooking resources in long-term contracts.
To evaluate its impact, we analyze the \textbf{r}atio of BSs' resource \textbf{d}emand to resource \textbf{s}upply in practical transactions under \textbf{l}ong-term \textbf{c}ontracts (RDSLC) across different overbooking levels. Specifically, Fig. \ref{overbook} presents a comparative analysis between HybridO and FRBank for different resources, highlighting how overbooking optimizes resource allocation stability and adaptability in dynamic ISAC environments. In Fig. \ref{overbook}, as the overbooking rate increases, we observe a rise in the RDSLC curves for both HybridO and FRBank in terms of bandwidth utilization. This trend highlights the effectiveness of the overbooking strategy in mitigating resource underutilization caused by the uncertainty of MU participation in transactions. Nevertheless, it is critical to carefully calibrate the overbooking rate, as excessive overbooking can lead to resource shortages. For instance, when the overbooking rate exceeds 0.3, the RDSLC curve for HybridO in terms of bandwidth surpasses 1, indicating that the BS should select voluntary clients to compensate because of the resource deficit. 
This underscores the importance of a proper overbooking rate, preventing both  resource underutilization and overcommitment in ISAC networks.

\noindent~\textit{(c) Analysis of risks (Fig. \ref{risk}):} To illustrate the importance of risk assessment and management, we introduce a metric called \textbf{d}efault \textbf{r}ate on \textbf{l}ong-term \textbf{c}ontract (DRLC)\footnote{The ratio of the number of failed transactions under long-term contracts to the total number of transactions specified in the contract during practical operations. This metric quantifies the contract violation rate.} upon having 5 BSs and different number of MUs. In Fig.~\ref{risk}, we conduct ablation experiments by introducing three comparative methods called FRBankNoR, HybridONoR, and ConOfflineNoR,  indicating FRBank, HybridO, and ConOffline methods excluding risk analysis. Note that if the utility of either the client or the BS becomes negative, or the compensation received is insufficient to offset the incurred loss, the affected party will opt to break contracts to minimize its loss and maximize its own utility. Additionally, due to overbooking and fluctuations in MUs' resource demand, BSs may encounter situations where resource supply is insufficient. In such cases, the BS may choose to breach long-term contracts and compensate volunteers to ensure sufficient resource availability. As shown in the figure, the value of DRLC of ConOfflineNoR, HybridONoR and FRBankNoR are significantly higher than those of ConOffline, HybridO and FRBank. This improvement is attributed to the implementation of effective risk management, which mitigates the risks of clients or BSs encountering unsatisfactory transactions and the risk of insufficient resource supply from the BS. These risk constraints allow our design to better adapt to the challenges posed by dynamic and uncertain ISAC network environments.

\vspace{-3mm}
\subsection{Experiments on a Real-World Dataset}\label{sec:real}
\begin{figure}[t]
	\vspace{-0.25cm}
	\centering
	%	\setlength{\abovecaptionskip}{-2 mm}
	\includegraphics[width=1\columnwidth]{images/EUA.png}
        \vspace{-8mm}
	\caption{Locations of BSs and MUs in Melbourne  area as per EUA Dataset.}
	\label{EUA}
    \vspace{-6mm}
\end{figure}

To further evaluate the performance, we leverage the real-world EUA dataset\cite{EUA dataset}, focusing on Melbourne Central Business District  (see Fig. \ref{EUA}).
Within this region, we randomly selected 140 MUs, identified the geographic locations of 11 BSs, and defined 28 sensing targets. The results are summarized in Table \ref{table EUA}, which confirm that FRBank delivers the best balance across key metrics, including social welfare, RT, NI, DIBC, and ECIBC. Note that ConOffline operates exclusively in the offline mode and therefore does not incur real-time decision-making delays or overhead, reflected by N/A cells in the table.
	


  \begin{table}[h]
\vspace{-4mm}
	{\small 
		\caption{Performance Evaluations for EUA Dataset \\(Alg. 1: FRBank, Alg. 2: ConOnline, Alg. 3: ConOffline, Alg. 4: Hybrid, Alg. 5: HybridO, Alg. 6: Greedy)}
        \vspace{-3mm}
		\begin{center}\vspace{-0.2cm}\label{table EUA}
			\setlength{\tabcolsep}{0.5mm}{
				\begin{tabular}{|c||c|c|c|c|c|c|}
					\hline \hline\rowcolor{gray!20}
					\textbf{Performance} & \textbf{Alg. 1}& \textbf{Alg. 2}&\textbf{Alg. 3}&\textbf{Alg. 4} & \textbf{Alg. 5} & \textbf{Alg. 6}\\ \hline\hline
					\textbf{Social welfare} &5025.82&3108.15&2867.33&3304.16&3084.43&1940.89
					\\ \hline
					\textbf{RT (ms)} &6705.7&389286.8&0.7 &27832.5&11712.7&1505.6\\ \hline
					\textbf{NI} &2878&37820&N/A&11803&8207&280
					\\ \hline
					\textbf{DIBC (ms)} &23291&302678&N/A&94909&64740 &2231	\\ \hline
					\textbf{ECIBC (W)} &304.53&3972.40&N/A&1251.95&868.92&29.89	\\ \hline \hline
			\end{tabular}}
	\end{center}}
        \vspace{-5mm}
\end{table}

\vspace{-2mm}
\section{Conclusion and Future Work}
\noindent This paper tackled the challenge of resource allocation in multi-MU, multi-BS ISAC networks. We proposed a novel framework called the \textit{future resource bank for ISAC (FRBank)}, which models individual MUs and sensing coalitions as clients while integrating offline and online resource trading modes to enable efficient and mutually beneficial interactions. 
FRBank incorporates offRFW$^2$M with overbooking strategies to obtain long-term contracts between clients and BSs while mitigating potential risks. It also incorporates onEBW$^2$M to address challenges arising from intermittent MU participation.
Extensive experiments demonstrated the superior performance of FRBank in terms of time efficiency and utility. 
Future research can target optimizing overbooking rates to better adapt to fluctuating market conditions. Additionally, further exploration is warranted in game-theoretic modeling of value and cost distribution within sensing coalitions in ISAC environments.

\vspace{-1mm}
{\tiny
\begin{thebibliography}{1}
	\bibitem{SURVEY 1} X. Wang, J. Mei, S. Cui, C.-X. Wang, and X. S. Shen, "Realizing 6G: The operational goals, enabling technologies of future networks, and value-oriented intelligent multi-dimensional multiple access," \textit{IEEE Netw.}, vol. 37, no. 1, pp. 10-17, Jan. 2023.
	
	\bibitem{SURVEY 2} F. Liu, Y. Cui, C. Masouros, J. Xu, T. X. Han, and Y. C. Eldar, "Integrated Sensing and Communications: Toward Dual-Functional Wireless Networks for 6G and Beyond," \textit{IEEE J. Sel. Areas Commun.}, vol. 40, no. 6, pp. 1728-1767, Jun. 2022.
	
	\bibitem{SURVEY 3} F. Dong, F. Liu, Y. Cui, W. Wang, K. Han, and Z. Wang, "Sensing as a service in 6G perceptive networks: A unified framework for ISAC resource allocation," \textit{IEEE Trans. Wireless Commun.}, vol. 22,  pp. 3522-3536, 2023.
	
	\bibitem{price1} Z. Zhou, B. Wang, B. Gu, B. Ai, S. Mumtaz, J. Rodriguez, M. Guizani, "Time-Dependent Pricing for Bandwidth Slicing Under Information Asymmetry and Price Discrimination," \textit{IEEE Trans. Commun.}, vol. 68, no. 11, pp. 6975-6989, Nov. 2020, 
	
	\bibitem{price2} L. Xie, S. Meng, W. Yao and X. Zhang, "Differential Pricing Strategies for Bandwidth Allocation With LFA Resilience: A Stackelberg Game Approach," \textit{IEEE Trans. Inf. Foren. Secur.}, vol. 18, pp. 4899-4914, 2023.
	
	\bibitem{price3} W. Fan, X. Li, B. Tang, Y. Su and Y. Liu, "MEC Network Slicing: Stackelberg-Game-Based Slice Pricing and Resource Allocation With QoS Guarantee," \textit{IEEE Trans. Netw. Serv. Manag.}, vol. 21, no. 4, pp. 4494-4509, Aug. 2024.
	
	\bibitem{Spot 1} D. Wang, Y. Jia, L. Liang, K. Ota, and M. Dong, "Resource allocation in blockchain integration of UAV-enabled MEC networks: A Stackelberg differential game approach," \textit{IEEE Trans. Serv. Comput.}, pp. 1-1, 2024.
	
	\bibitem{Spot 2} X. Liu and W. Li, "A truthful randomized mechanism for heterogeneous resource allocation with multi-minded in mobile edge computing," \textit{IEEE Trans. Netw. Serv. Manag.}, pp. 1-1, 2024.
	
	\bibitem{Spot 3} Z. Yang, R. Zhang, J. Yao, and M. Zhao, "Price-based offloading for time-sensitive and thermal-aware MEC networks," \textit{IEEE Wireless Commun. Netw. Conf. (WCNC)}, Dubai, United Arab Emirates, 2024, pp. 1-6.
	
	\bibitem{future 1} M. Liwang, Z. Gao, and X. Wang, "Let's trade in the future! A futures-enabled fast resource trading mechanism in edge computing-assisted UAV networks," \textit{IEEE J. Sel. Areas Commun.}, vol. 39, pp. 3252-3270, 2021.
	
	\bibitem{future 2} S. Sheng, R. Chen, P. Chen, X. Wang, and L. Wu, "Futures-based resource trading and fair pricing in real-time IoT networks," \textit{IEEE Wireless Commun. Lett.}, vol. 9, no. 1, pp. 125-128, 2020.
	
	\bibitem{RW Matching3} H. Qi, M. Liwang, X. Wang, L. Li, W. Gong, J. Jin, and Z. Jiao, "Bridge the Present and Future: A Cross-Layer Matching Game in Dynamic Cloud-Aided Mobile Edge Networks," \textit{IEEE Trans. Mobile Comput.}, vol. 23, no. 12, pp. 12522-12539, 2024.
	
	\bibitem{overbook1} J. Ma, Y. K. Tse, X. Wang, and M. Zhang, "Examining customer perception and behaviour through social media research: An empirical study of the United Airlines overbooking crisis," \textit{Transp. Res. Part E: Logistics Transp. Rev.}, vol. 127, pp. 192-205, 2019.
	
	\bibitem{overbook2} N. Haynes and D. Egan, "The perceptions of frontline employees towards hotel overbooking practices: Exploring ethical challenges," \textit{J. Revenue Pricing Manag.}, vol. 19, pp. 119-128, 2020.
	
	\bibitem{overbook3} A. Adebayo, D. B. Rawat, and M. Song, "Prediction-based adaptive RF spectrum reservation in wireless virtualization," \textit{IEEE Int. Conf. Commun. (ICC)}, Dublin, Ireland, 2020, pp. 1-6.
	
	\bibitem{ISAC} B. Li, X. Wang, S. Ahn, S. -I. Park and Y. Wu, "Successive Resource Allocation in Multi-User ISAC System through Deep Reinforcement Learning," \textit{IEEE Int. Conf. Commun. (ICC)},  2024, pp. 5577-5583.
	
	\bibitem{RW ISAC1} B. Li, X. Wang, and F. Fang, "Maximizing the value of service provisioning in multi-MU ISAC networks through fairness guaranteed collaborative resource allocation," \textit{IEEE J. Sel. Areas Commun.}, vol. 42, no. 9, pp. 2243-2258, 2024.
	
	\bibitem{RW ISAC2} F. Dong, F. Liu, Y. Cui, W. Wang, K. Han, and Z. Wang, "Sensing as a service in 6G perceptive networks: A unified framework for ISAC resource allocation," \textit{IEEE Trans. Wireless Commun.}, vol. 22, no. 5, pp. 3522-3536, 2023.
	
	\bibitem{RW ISAC3} B. Hu, W. Zhang, Y. Gao, J. Du, and X. Chu, "Multiagent deep deterministic policy gradient-based computation offloading and resource allocation for ISAC-aided 6G V2X networks," \textit{IEEE Internet of Things J.}, vol. 11, no. 20, pp. 33890-33902, 2024.
	
	\bibitem{RW ISAC4} J. Wang, L. Bai, J. Chen, and J. Wang, "Starling flocks-inspired resource allocation for ISAC-aided green ad hoc networks," \textit{IEEE Trans. Green Commun. Netw.}, vol. 7, no. 1, pp. 444-454, 2023.
	
	\bibitem{RW Matching1} X. Ye and L. Fu, "Joint MCS adaptation and RB allocation in cellular networks based on deep reinforcement learning with stable matching," \textit{IEEE Trans. Mobile Comput.}, vol. 23, no. 1, pp. 549-565, 2024.
	
	\bibitem{RW Matching2} Y. Xu, S. Zhang, C. Lyu, J. Liu, T. Taleb, and S. Norio, "TRIMP: Three-sided stable matching for distributed vehicle sharing system using Stackelberg game," \textit{IEEE Trans. Mobile Comput.}, vol. 24, no. 2, pp. 1132-1148, 2025.
	
	\bibitem{RW Matching4} N. Sharghivand, F. Derakhshan, L. Mashayekhy, and L. Mohammadkhanli, "An edge computing matching framework with guaranteed quality of service," \textit{IEEE Trans. Cloud Comput.}, vol. 10, pp. 1557-1570, 2022.
	
	\bibitem{RW Matching5} Y. Du, J. Li, L. Shi, T. Liu, F. Shu, and Z. Han, "Two-tier matching game in small cell networks for mobile edge computing," \textit{IEEE Trans. Serv. Comput.}, vol. 15, no. 1, pp. 254-265, 2022.
	
	
	
	\bibitem{beamforming 1} G. Kwon, A. Conti, H. Park, and M. Z. Win, "Joint communication and localization in millimeter wave networks," \textit{IEEE J. Sel. Topics Signal Process.}, vol. 15, no. 6, pp. 1439-1454, 2021.
	
	\bibitem{beamforming 2} X. Mu, Y. Liu, L. Guo, J. Lin, and L. Hanzo, "NOMA-aided joint radar and multicast-unicast communication systems," \textit{IEEE J. Sel. Areas Commun.}, vol. 40, no. 6, pp. 1978-1992, 2022.
	
	\bibitem{beamforming 3} Z. Yifan, Z. Huilin, and Z. Fuhui, "Resource allocation for a wireless powered integrated radar and communication system," \textit{IEEE Wireless Commun. Lett.}, vol. 8, no. 1, pp. 253-256, 2019.
	
	\bibitem{NLoS} R. Yang, C.-X. Wang, J. Huang, E. M. Aggoune, and Y. Hao, "A novel 6G ISAC channel model combining forward and backward scattering," \textit{IEEE Trans. Wireless Commun.}, vol. 22, no. 11, pp. 8050-8065, 2023.
	
	\bibitem{known} Z. He, W. Xu, H. Shen, D. W. K. Ng, Y. C. Eldar, and X. You, "Full-duplex communication for ISAC: Joint beamforming and power optimization," \textit{IEEE J. Sel. Areas Commun.}, vol. 41, no. 9, pp. 2920-2936, 2023.
	
	\bibitem{device sensing} J. Wang, Q. Gao, M. Pan, and Y. Fang, "Device-free wireless sensing: Challenges, opportunities, and applications," \textit{IEEE Netw.}, vol. 32, no. 2, pp. 132-137, 2018.
	
	\bibitem{sensing factor} F. Liu, W. Yuan, C. Masouros, and J. Yuan, "Radar-assisted predictive beamforming for vehicular links: Communication served by sensing," \textit{IEEE Trans. Wireless Commun.}, vol. 19, no. 11, pp. 7704-7719, 2020.
	
	\bibitem{MY tsc} H. Qi, M. Liwang, S. Hosseinalipour, X. Xia, Z. Cheng, X. Wang, and Z. Jiao, “Matching-based hybrid service trading for task assignment over dynamic mobile crowdsensing networks,” \textit{IEEE Trans. Serv. Comput.}, pp. 1-14, 2023.
	
	\bibitem{EUA dataset} P. Lai, Q. He, M. Abdelrazek, F. Chen, J. Hosking, J. Grundy, and Y. Yang, "Optimal edge MU allocation in edge computing with variable sized vector bin packing," \textit{Int. Conf. Service-Oriented Comput. (ICSOC)}, Springer, 2018, pp. 230-245.
	
	
	\bibitem{E2E1} 5G Americas, "New services \& applications with 5G ultra-reliable low latency communications," White Paper, Nov. 2018.
	
	\bibitem{E2E2} 3GPP, "5G; Study on scenarios and requirements for next generation access technologies," 3GPP TR 38.913 version 17.0.0 Release 17, 2022.
	
	\bibitem{46} H. V. Poor, \textit{An Introduction To Signal Detection and Estimation,} 2nd ed. New York, NY, USA: Springer-Verlag, 1994.
	\bibitem{47} M. Braun, "OFDM radar algorithms in mobile communication networks," Ph.D. dissertation, Karlsruhe, Karlsruher Institut fur Technologie, 2014
	
	\bibitem{48} J. Chen, X. Wang, and Y.-C. Liang, "Impact of channel aging on dual-function radar-communication systems: Performance analysis and resource allocation," \textit{IEEE Trans. Commun.}, vol. 71, pp. 4972-4987, Aug. 2023.
	\bibitem{49} B. Li, X. Wang, E. Au, and Y. Xin, "Joint localization and environment sensing of rigid body with 5G millimeter wave MIMO," \textit{IEEE Open J. Signal Process.}, vol. 4, pp. 117-131, 2023.
	
	
\end{thebibliography}
	}
  
\newpage
\clearpage
\appendices
\section{Derivations associated with ISAC}
\subsection{Derivation of CRLB of ${\boldsymbol{\eta}_{i,j}}$}
Here, we first note that the device-free sensing model and the device-based sensing model are similar in structure, with the latter being derivable in a similar manner. As an example, we present the detailed derivation of the device-free sensing model. Denote
\begin{equation}
	\mathbf{H}_{i,j,n} = \Omega(\tau_{i,j,n}) \mathbf{A}_{i,j}^\mathsf{r} (\phi^\mathsf{(s)}_{i,j}) (\mathbf{A}_{i,j}^\mathsf{t})^\mathsf{H} \theta^\mathsf{(s)}_{i,j},
\end{equation}
where
\begin{equation}
	\Omega(\tau_{i,j}) = \frac{\sqrt{N_j^\mathsf{T} N_i^\mathsf{R} p_{i,j,n} h_{i,j,n}}}{\sqrt{p_{i,j,n}}} e^{j\frac{2\pi n \tau_{i,j,n}}{NT_s}}.
\end{equation}
Based on \cite{RW ISAC1,46}, when dealing with AWGN, the $(\mathbbm{p},\mathbbm{q})$-th elements in $\mathbf{J}_{\boldsymbol{\eta}_{i,j}}(\mathbbm{p},\mathbbm{q})$ can be derived as
\begin{equation}
	\mathbf{J}_{\boldsymbol{\eta}_{i,j}}(\mathbbm{p},\mathbbm{q}) = \sum_n \frac{2}{\sigma_s^2} \left\{ \frac{\partial (\mathbf{H}_{i,j,n} s_{ref})^\mathsf{H}}{\partial {\boldsymbol{\eta}_{i,j}}(\mathbbm{p})} \cdot \frac{\partial (\mathbf{H}_{i,j,n} s_{ref})}{\partial {\boldsymbol{\eta}_{i,j}}(\mathbbm{q})} \right\}^\mathsf{Re},
\end{equation}
where ${\boldsymbol{\eta}_{i,j}}(\mathbbm{q})$ is the $\mathbbm{q}$-th element of ${\boldsymbol{\eta}_{i,j}}$. By decomposing $\mathbf{H}_{i,j,n}$ across antennas, we have
\begin{equation}
	\begin{aligned}
		&\tilde{h}_{i,j,n,l_j^\mathsf{T},l_i^\mathsf{R}} =\\& \Omega(\tau_{i,j,n}) \times \exp\left( \frac{2\pi n}{\lambda_n} j \left( - \frac{N_j^\mathsf{T}-1}{2} + l_j^\mathsf{T} \right) \right)
		\times d \sin \theta_{i,j} \\&+ \frac{2\pi n}{\lambda_n} j \left( - \frac{N_i^\mathsf{R}-1}{2} + l_i^\mathsf{R} \right) d \sin \phi_{i,j},
	\end{aligned}
\end{equation}
where $l_j^\mathsf{T}\in \{1,2,...,N_j^\mathsf{T}\}$ and $l_i^\mathsf{R} \in \{1,2,...,N_i^\mathsf{R}\}$. Inspired by the method presented in \cite{46}, under AWGN, the $(\mathbbm{p},\mathbbm{q})$-th elements in $\mathbf{J}_{\boldsymbol{\eta}_{i,j}}$ can then be derived as
\begin{equation}{\small
	\begin{aligned}
			&\mathbf{J}_{\boldsymbol{\eta}_{i,j}}(\mathbbm{p},\mathbbm{q}) =\\& \frac{2}{\sigma_s^2} \sum_{n = \mathbbm{l}^\mathsf{(s)}_{i,j}}^{\mathbbm{l}^\mathsf{(s)}_{i,j}+N^\mathsf{(s)}_{i,j}} \sum_{l_j^\mathsf{T}=1}^{N_j^\mathsf{T}} \sum_{l_i^\mathsf{R}=1}^{N_i^\mathsf{R}} \left\{ \frac{\partial (\tilde{h}_{i,j,n,l_j^\mathsf{T},l_i^\mathsf{R}})^\mathsf{H}}{\partial {\boldsymbol{\eta}_{i,j}}(\mathbbm{p})} \cdot \frac{\partial \tilde{h}_{i,j,n,l_j^\mathsf{T},l_i^\mathsf{R}}}{\partial {\boldsymbol{\eta}_{i,j}}(\mathbbm{q})} \right\}^\mathsf{Re},
	\end{aligned}}
\end{equation}
where the first summation is among all the subcarriers in the subchannels. Due to the difficulty of deriving a closed-form with inverse processing of $\mathbf{J}_{\boldsymbol{\eta}_{i,j}}$, we adopt a similar approximation approach as also applied in \cite{RW ISAC1,47,48}. For the three variables in ${\boldsymbol{\eta}_{i,j}}$, assuming that the error source comes from the white noise, the estimation of each measurement is considered as independent, then, we have
\begin{equation}
	\text{var}(\bm{\hat{\eta}}_{i,j}(i)) \geq \mathbf{J}_{\boldsymbol{\eta}_{i,j}}^{-1}(i,i), \quad i \in \{1, 2, 3\},
\end{equation}
as shown in (\ref{equ. 59}).
\begin{figure*}[b!]
\vspace{-4mm}
	\hrulefill
	\begin{equation}\label{equ. 59}
		\begin{aligned}
			\text{var}(\hat{\theta}_{i,j}) &\geq \frac{\zeta_1}{N^\mathsf{(s)}_{i,j} P_{i,j}} = \frac{3\sigma^2 \rho_{i,j}}{16\pi^2 h^2_{i,j}N^\mathsf{(s)}_{i,j} N_i^\mathsf{R} P^\mathsf{(s)}_{i,j} \cos^2 \theta_{i,j}N_j^\mathsf{T} (N_j^\mathsf{T} + 1)(2N_j^\mathsf{T} + 1)} \\
			\text{var}(\hat{\phi}_{i,j}) &\geq \frac{\zeta_2}{N^\mathsf{(s)}_{i,j} P_{i,j}} = \frac{3\sigma^2 \rho_{i,j}}{16\pi^2 h^2_{i,j}N^\mathsf{(s)}_{i,j} N_j^\mathsf{T} P^\mathsf{(s)}_{i,j} \cos^2 \theta_{i,j}N_i^\mathsf{R} (N_i^\mathsf{R} + 1)(2N_i^\mathsf{R} + 1)} \\
			\text{var}(\hat{\tau}_{i,j}) &\geq  \frac{\zeta_3}{N^\mathsf{(s)}_{i,j}(N^\mathsf{(s)}_{i,j}+1)(2N^\mathsf{(s)}_{i,j}+1)P^\mathsf{(s)}_{i,j}}, \zeta_3 = \frac{3\sigma^2 \rho_{i,j}(N_{i,j}^\mathsf{(s)})^2 T_s^2}{4\pi^2 h^2_{i,j} N_j^\mathsf{T} N_i^\mathsf{R}} 
		\end{aligned}
	\end{equation}
    \vspace{-7mm}
\end{figure*}

 

\subsection{Derivation of PEB from CRLB of ${\boldsymbol{\eta}_{i,j}}$}
With the FIM of channel parameters, the PEB$_{i,j}$ can be derived through the variable transformation tensor $\mathbb{T}$ from ${\boldsymbol{\eta}_{i,j}}$ to $l^\mathsf{Q}_{i}$\cite{RW ISAC1,49}, as given by
\begin{equation}\label{equ. 60}
	\mathbf{J}_{l^\mathsf{Q}_{i}} = \mathbb{T} \mathbf{J}_{\boldsymbol{\eta}_{i,j}} \mathbb{T}^\mathsf{T}, \quad \mathbb{T} = \frac{\partial {\boldsymbol{\eta}^\mathsf{T}_{i,j}}}{\partial l^\mathsf{Q}_{i}},
\end{equation}
where $\mathbb{T} \in \mathbb{R}^{3 \times 2}$ can be written as
\begin{equation}\label{equ. 61}
	\mathbb{T} = \left[ \frac{\partial \tau_{i,j}}{\partial l^\mathsf{Q}_{i}}, \frac{\partial \theta_{i,j}}{\partial l^\mathsf{Q}_{i}}, \frac{\partial \phi_k}{\partial l^\mathsf{Q}_{i}} \right].
\end{equation}

From (\ref{equ. transmission time})-(\ref{equ. 18}), we have
\begin{equation}\label{equ. 62}
	\begin{aligned}
			&\frac{\partial \tau_{i,j}}{\partial l^\mathsf{Q}_{i}} = \frac{ \left[ \cos(\theta_{i,j}) + \cos(\phi_{i,j}), \sin(\theta_{i,j}) + \sin(\phi_{i,j}) \right]^\mathsf{T}}{c},
		\\&\frac{\partial \theta_{i,j}}{\partial l^\mathsf{Q}_{i}} = \frac{\left[ - \sin(\theta_{i,j}), \cos(\theta_{i,j}) \right]^\mathsf{T}}{d^\mathsf{Q}_{i}} ,
		\\&\frac{\partial \phi_{i,j}}{\partial l^\mathsf{Q}_{i}} = \frac{-\left[ \cos(\pi - \phi_{i,j}), \cos(\pi - \phi_{i,j}) \right]^\mathsf{T}}{d^\mathsf{Q}_{i,j}} .
	\end{aligned}
\end{equation}
\begin{figure*}[t!]
\vspace{-6mm}
	\begin{equation}\label{equ. 63}
		\begin{aligned}
			&\mathbf{J}_{l^\mathsf{Q}_{i,j}}(1,1) = P^\mathsf{(s)}_{i,j} N^\mathsf{(s)}_{i,j} \left( \frac{\sin^2 \theta_{i,j} }{\zeta_1 (d^\mathsf{Q}_{i,j})^2}+\frac{\cos^2 \theta_{i,j} }{\zeta_2 (d^\mathsf{Q}_{i})^2} + \frac{\left( \cos \theta_{i,j} + \cos \phi_{i,j} \right)^2}{\zeta_3 c^2} \left( N^\mathsf{(s)}_{i,j}+1 \right) \left( 2N^\mathsf{(s)}_{i,j}+1 \right)\right)
			\\&\mathbf{J}_{l^\mathsf{Q}_{i,j}}(2,2) = P^\mathsf{(s)}_{i,j} N^\mathsf{(s)}_{i,j} \left( \frac{\cos^2 \theta_{i,j} }{\zeta_1 (d^\mathsf{Q}_{i,j})^2}+\frac{\sin^2 \theta_{i,j} }{\zeta_2 (d^\mathsf{Q}_{i})^2} + \frac{\left( \sin \theta_{i,j} + \sin \phi_{i,j} \right)^2}{\zeta_3 c^2} \left( N^\mathsf{(s)}_{i,j}+1 \right) \left( 2N^\mathsf{(s)}_{i,j}+1 \right) \right)
		\end{aligned}
	\end{equation}\hrulefill
    \vspace{-6mm}
\end{figure*}
With (\ref{equ. 59}) and (\ref{equ. 62}), $\mathbf{J}_{l^\mathsf{Q}_{i}}$ can be expressed as (\ref{equ. 63}). Then, we can get
\begin{equation}
	\begin{aligned}
			&\text{PEB}_{i,j} = \sqrt{ \text{tr} \left\{ \mathbf{J}_{l^\mathsf{Q}_{i}}^{-1} \right\}} \\&\approx \sqrt{ \frac{1}{\mathbf{J}_{l^\mathsf{Q}_{i}}(1,1)} + \frac{1}{\mathbf{J}_{l^\mathsf{Q}_{i}}(2,2)} }
		= \frac{\zeta_{i,j}}{\sqrt{N^\mathsf{(s)}_{i,j}P^\mathsf{(s)}_{i,j}}} ,
	\end{aligned}
\end{equation}
where the expression for \( \zeta_{i,j} \), in terms of bandwidth, antennas, channel gain, noise power, and the position of sensing targets/MUs, can be derived by substituting (\ref{equ. 60}) into (\ref{equ. 61}).




\section{Derivations associated with offRFW$^2$M}
\noindent\textbf{Mathematical expectation of $ \alpha_i $.} $ \alpha_i $ refers to the MU participation uncertainty, following a Bernoulli random variable $\alpha_i$, $\alpha_i \sim {\bf{B}} \{(1, 0), (\mathbbm{a}_i, 1-\mathbbm{a}_i)\}$. The expectation of $\alpha_i$ can simply computed as $ \mathbb{E}[\alpha_i]=1\times\mathbbm{a}_i+0\times (1-\mathbbm{a}_i))=\mathbbm{a}_i $.

\noindent\textbf{Mathematical expectation of $ \beta_k $.} The variable \( \beta_k \) represents the uncertainty associated with the participation of \( \bm{c}_k \) in a practical transaction.
 The expectation of $\beta_k$ can simply computed as
\begin{equation}
	\begin{aligned}
	&\mathbb{E}[\beta_k]=1\times\Pr(\sum_{u_i\in \bm{c}_k}\alpha_i > 0)+0\times\Pr(\sum_{u_i\in \bm{c}_k}\alpha_i = 0)\\&=1-\Pr(\sum_{u_i\in \bm{c}_k}\alpha_i = 0)=1-\prod_{u_i\in\bm{c}_k}(1-\mathbbm{a}_i)
	\end{aligned}
\end{equation}

\noindent\textbf{Mathematical expectation of $ \mathbbm{v}^\mathsf{(c)}_{i,j} $ and $ \mathbbm{v}^\mathsf{(s)}_{k,j} $.}
As resource overbooking can lead to the case where a contractual client being selected as a volunteer, we use $ \mathbbm{v}^\mathsf{(c)}_{i,j} $ and $ \mathbbm{v}^\mathsf{(s)}_{k,j} $ to indicate whether MU $ u_i $ and $ \bm{c}_k$ is determined as a volunteer by BS $ s_j $ in each practical transaction.

Due to intermittent participation of MUs, assessing the close-form of the probability of an MU being a volunteer needs large calculations. We first use notation $\bm{\mathcal{M}}_j=\left\{\bm{M}_1,..,\bm{M}_n,...,\bm{M_{|\mathcal{M}_j|}}\right\}$ to collect all the possible cases of the participation of clients participation in the online trading market, where $\bm{M}_n=\left\{\alpha_1,...,\alpha_i,...,\alpha_{|\varphi^\mathsf{(c)}\left(s_j\right)|}, \beta_1,...,\beta_k, ..., \beta_{|\varphi^\mathsf{(s)}\left(s_j\right)|}\right\}$ is a vector and denoted as the ${n}^\text{th}$ case of clients' participation in a transaction. For example, suppose that $s_j$ has pre-signed long-term contracts with two clients, all the possible cases of these clients taking part in a practical transaction can be expressed as $\bm{\mathcal{M}_j}=\left\{\bm{M}_1,\bm{M}_2,\bm{M}_3,\bm{M}_4\right\}=\left\{\left\{0,0\right\},\left\{0,1\right\},\left\{1,0\right\},\left\{1,1\right\}\right\}$. For notational simplicity, we use a set $\bm{\mathcal{M}}^\mathsf{U}_{i,j}$ to denote all the cases that $u_i$ attends in a transaction, which, unfortunately, is selected as a volunteer by $s_j$.
Accordingly, the probability of MU $ u_i $ being determined by $s_j$ as a volunteer is given by
\begin{equation}\label{key}{\small
		\begin{aligned}
			\text{Pr}(\mathbbm{v}^\mathsf{(c)}_{i,j}=1)=\sum_{\bm{M}_n\in\bm{\mathcal{M}}^\mathsf{U}_{i,j}}{\prod_{\alpha_i,\beta_k\in \bm{M}_n}}{\mathbb{E}[\alpha_i]\mathbb{E}[\beta_k]},		
	\end{aligned}}
\end{equation}
and the expected value of $\mathbbm{v}^\mathsf{(c)}_{i,j}$ can be calculated by
\begin{equation}{\small
		\begin{aligned}
			&\mathbb{E}[\mathbbm{v}^\mathsf{(c)}_{i,j}]=\text{Pr}\left(\mathbbm{v}^\mathsf{(c)}_{i,j}=0\right)\times 0+\text{Pr}\left(\mathbbm{v}^\mathsf{(c)}_{i,j}=1\right)\times 1\\&=\text{Pr}(\mathbbm{v}^\mathsf{(c)}_{i,j}=1)=\sum_{\bm{M}_n\in\bm{\mathcal{M}}^\mathsf{U}_{i,j}}{\prod_{\alpha_i,\beta_k\in \bm{M}_n}}{{\mathbb{E}[\alpha_i]\mathbb{E}[\beta_k]}}.
	\end{aligned}}
\end{equation}
Similarly, we use the set \( \bm{\mathcal{M}}^\mathsf{C}_{k,j} \) to represent all the cases where \( \bm{c}_k \) participates in a transaction, which, however, has become a volunteer by \( s_j \). Therefore, the expected value of \( \mathbbm{v}^\mathsf{(s)}_{i,j} \) is given by
\begin{equation}{\small
		\begin{aligned}
			&\mathbb{E}[\mathbbm{v}^\mathsf{(s)}_{k,j}]=\text{Pr}\left(\mathbbm{v}^\mathsf{(s)}_{k,j}=0\right)\times 0+\text{Pr}\left(\mathbbm{v}^\mathsf{(s)}_{k,j}=1\right)\times 1\\&=\text{Pr}(\mathbbm{v}^\mathsf{(s)}_{k,j}=1)=\sum_{\bm{M}_n\in\bm{\mathcal{M}}^\mathsf{C}_{k,j}}{\prod_{\alpha_i,\beta_k\in \bm{M}_n}}{{\mathbb{E}[\alpha_i]\mathbb{E}[\beta_k]}}.
	\end{aligned}}
\end{equation}


\noindent\textbf{Mathematical expectation of $ V^\mathsf{(s),max }_{k,j} $.} Due to the uncertainty in the participation of MUs in the practical transaction, and \( V^{\mathsf{(s), max}}_{k,j} = \max \{V^{\mathsf{(s)}}_{i,j}\} , u_i \in \bm{c}_k \), the expected value of \( V^{\mathsf{(s), max}}_{k,j} \) can be defined as \( \mathbb{E}[V^{\mathsf{(s), max}}_{k,j}] = \max\{ \mathbbm{a}_i V^{\mathsf{(s)}}_{i,j} \}, u_i \in \bm{c}_k \).


\noindent\textbf{Derivations associated with (\ref{equ. PF BS C5}) and (\ref{equ. PF BS C6}).} We obtain an upper bound for the left-hand side of equation (\ref{equ. PF BS C5}) using the Markov inequality \cite{RW Matching3}, as given by
\begin{equation}\label{key}{\small
		\begin{aligned}
			&R_1^\mathsf{S}(s_j,\varphi^\mathsf{(c)}(s_j),\mathbb{C}^\mathsf{(c)}_{i,j},\varphi^\mathsf{(s)}(s_j),\mathbb{C}^\mathsf{(s)}_{k,j}) \leq \rho_1 \Rightarrow \\& \Pr\left(\sum_{u_i\in\varphi^\mathsf{(c)}(s_j)}\alpha_i\mathbbm{c}^\mathsf{(c),B}_{i,j}+\sum_{\bm{c}_k\in\varphi^\mathsf{(s)}(s_j)}\beta_k\mathbbm{c}^\mathsf{(s),B}_{k,j}> B_j\right)\leq\\& \frac{\mathbb{E}\left[\sum_{u_i\in\varphi^\mathsf{(c)}(s_j)}\alpha_i\mathbbm{c}^\mathsf{(c),B}_{i,j}+\sum_{\bm{c}_k\in\varphi^\mathsf{(s)}(s_j)}\beta_k\mathbbm{c}^\mathsf{(s),B}_{k,j}\right]}{B_j} \leq\rho_1 \Rightarrow\\&  
            \frac{\sum_{u_i\in\varphi^\mathsf{(c)}(s_j)}\mathbb{E}[\alpha_i]\mathbbm{c}^\mathsf{(c),B}_{i,j}+\sum_{\bm{c}_k\in\varphi^\mathsf{(s)}(s_j)}\mathbb{E}[\beta_k]\mathbbm{c}^\mathsf{(s),B}_{k,j}}{B_j} \leq\rho_1.
	\end{aligned}}
\end{equation}
Similarly, constraint (\ref{equ. PF BS C6}) can be reformulated as
\begin{equation}\label{key}{\small
		\begin{aligned}
			&R_2^\mathsf{S}(s_j,\varphi^\mathsf{(c)}(s_j),\mathbb{C}^\mathsf{(c)}_{i,j},\varphi^\mathsf{(s)}(s_j),\mathbb{C}^\mathsf{(s)}_{k,j}) \leq \rho_2 \Rightarrow \\& \Pr\left(\sum_{u_i\in\varphi^\mathsf{(c)}(s_j)}\alpha_i\mathbbm{c}^\mathsf{(c),Pow}_{i,j}+\sum_{\bm{c}_k\in\varphi^\mathsf{(s)}(s_j)}\beta_k\mathbbm{c}^\mathsf{(s),Pow}_{k,j}> P_j\right)\leq \rho_2 \\& \Rightarrow  
            \frac{\sum_{u_i\in\varphi^\mathsf{(c)}(s_j)}\mathbb{E}[\alpha_i]\mathbbm{c}^\mathsf{(c),Pow}_{i,j}+\sum_{\bm{c}_k\in\varphi^\mathsf{(s)}(s_j)}\mathbb{E}[\beta_k]\mathbbm{c}^\mathsf{(s),Pow}_{k,j}}{P_j} \leq\rho_2.
	\end{aligned}}
\end{equation}

\noindent\textbf{Derivations associated with (\ref{equ. PF MU C6}) and (\ref{equ. PF MU C7}).} 
Constraint (\ref{equ. PF MU C6}) represents a probabilistic expression, making its close form non-trivial to be obtained. Therefore, we define the variable $ \hat{u}^\mathsf{(c),U}\left(u_i,\varphi^\mathsf{(c)}(u_i),\mathbb{C}^\mathsf{(c)}_{i,j}\right)=u^\mathsf{(c)}_\mathsf{\max}-u^\mathsf{(c),U}\left(u_i,\varphi^\mathsf{(c)}(u_i),\mathbb{C}^\mathsf{(c)}_{i,j}\right) $, where $u^\mathsf{(c)}_\mathsf{\max}$ represents the maximum value of $ u^\mathsf{(c),U}\left(u_i,\varphi^\mathsf{(c)}(u_i),\mathbb{C}^\mathsf{(c)}_{i,j}\right)$, and transform (\ref{equ. PF MU C6}) into a tractable one by exploiting a set of bounding techniques. First, (\ref{equ. PF MU C6}) can be rewritten as
\begin{equation}\label{DR 38d}{\small
		\begin{aligned}
			&R_1^\mathsf{U}(u_i,\varphi^\mathsf{(c)}(u_i),\mathbb{C}^\mathsf{(c)}_{i,j})
            \\&=\Pr\left(u^\mathsf{(c),U}\left(u_i,\varphi^\mathsf{(c)}(u_i),\mathbb{C}^\mathsf{(c)}_{i,j}\right)\leq u^\mathsf{(c)}_\mathsf{\min}\right)
            \\&=\Pr\left(\hat{u}^\mathsf{(c),U}\left(u_i,\varphi^\mathsf{(c)}(u_i),\mathbb{C}^\mathsf{(c)}_{i,j}\right)\geq u^\mathsf{(c)}_\mathsf{\max}-u^\mathsf{(c)}_\mathsf{\min}\right)\leq\rho_3 .
	\end{aligned}}
\end{equation}

To obtain a tractable form for (\ref{DR 38d}), we can have the upper-bound of its left-hand side by using Markov inequality \cite{RW Matching3}:
\begin{equation}\label{DR 38d1}{\small
		\begin{aligned}
		&\Pr\left(\hat{u}^\mathsf{(c),U}\left(u_i,\varphi^\mathsf{(c)}(u_i),\mathbb{C}^\mathsf{(c)}_{i,j}\right)\geq u^\mathsf{(c)}_\mathsf{\max}-u^\mathsf{(c)}_\mathsf{\min}\right)
        \\&\leq \frac{\mathbb{E}\left[\hat{u}^\mathsf{(c),U}\left(u_i,\varphi^\mathsf{(c)}(u_i),\mathbb{C}^\mathsf{(c)}_{i,j}\right)\right]}{u^\mathsf{(c)}_\mathsf{\max}-u^\mathsf{(c)}_\mathsf{\min}}
        \\&=\frac{u^\mathsf{(c)}_\mathsf{\max}-\mathbb{E}\left[u^\mathsf{(c),U}\left(u_i,\varphi^\mathsf{(c)}(u_i),\mathbb{C}^\mathsf{(c)}_{i,j}\right)\right]}{u^\mathsf{(c)}_\mathsf{\max}-u^\mathsf{(c)}_\mathsf{\min}}.
	\end{aligned}}
\end{equation}
Combining (\ref{DR 38d}) and (\ref{DR 38d1}), we can then get a tractable form for (\ref{equ. PF MU C6}):
\begin{equation}\label{key}{\small
		\begin{aligned}
			\frac{u^\mathsf{(c)}_\mathsf{\max}-\mathbb{E}\left[u^\mathsf{(c),U}\left(u_i,\varphi^\mathsf{(c)}(u_i),\mathbb{C}^\mathsf{(c)}_{i,j}\right)\right]}{u^\mathsf{(c)}_\mathsf{\max}-u^\mathsf{(c)}_\mathsf{\min}}\leq\rho_3	,
	\end{aligned}}
\end{equation}
where the value of $\mathbb{E}\left [u^\mathsf{(c),U}\left(u_i,\varphi^\mathsf{(c)}(u_i),\mathbb{C}^\mathsf{(c)}_{i,j}\right)\right ]$ is given by (\ref{equ. expected comm utility}).
Similarly, we can get a tractable form for (\ref{equ. PF MU C7}):
\begin{equation}\label{key}{\small
		\begin{aligned}
			\frac{u^\mathsf{(s)}_\mathsf{\max}-\mathbb{E}\left [u^\mathsf{(s),U}\left(\bm{c}_k,\varphi^\mathsf{(s)}(\bm{c}_k),\mathbb{C}^\mathsf{(s)}_{k,j}\right)\right ]}{u^\mathsf{(s)}_\mathsf{\max}-u^\mathsf{(s)}_\mathsf{\min}}\leq\rho_4	,
	\end{aligned}}
\end{equation}
where $u^\mathsf{(s)}_\mathsf{\max}$ represents the maximum value of $ u^\mathsf{(s),U}\left(\bm{c}_k,\varphi^\mathsf{(s)}(\bm{c}_k),\mathbb{C}^\mathsf{(s)}_{k,j}\right)$, and the value of $\mathbb{E}\left [u^\mathsf{(s),U}\left(\bm{c}_k,\varphi^\mathsf{(s)}(\bm{c}_k),\mathbb{C}^\mathsf{(s)}_{k,j}\right)\right ]$ is given by (\ref{equ. expected sensing utility}).







\section{Property Analysis of offRFW$^2$M}
\begin{Prop}\label{Prop 7}
	(Convergence of a set of matching in offRFW$^2$M) Alg. 1 converges within finite rounds.
\end{Prop}
\begin{proof}
	As the offRFW$^2$M refers to a set of M2M matching (matching between BSs and individual MUs, as well as matching between BSs and coalitions), we utilize the DP algorithm to transform the problem into a two-dimensional 0-1 knapsack problem \cite{RW Matching3}. After a finite number of rounds, each client's payment can either be accepted or reach its maximum value while considering constraints (\ref{equ. PF BS C3}), (\ref{equ. PF BS C4}), (\ref{equ. PF BS C5}), and (\ref{equ. PF BS C6}) (e.g., lines 17-26, Alg. 1), ensuring the convergence.
\end{proof}

\begin{Prop}
	(Individual rationality of offRFW$^2$M) The proposed offRFW$^2$M mechanism ensures individual rationality for All the BSs, individual MUs, and sensing coalitions are individual rational in the offRFW$^2$M.
\end{Prop}
\begin{proof}
	We offer the analysis on proving the individual rationality of both BSs and clients.
	
		\textbf{Individual rationality of BSs.} Owing to overbooking, each BS $s_j$ regards $(1+O_j^\mathsf{B})B_j$ and $(1+O_j^\mathsf{Pow})P_j$ as up limit of resources for serving MUs, and the actual number of matched clients of $s_j$ will definitely not exceed its overbooked resource supply (e.g., line 15, Alg. 1). In addition, thanks to the risk analysis, the risk of BS \(s_{j}\) having actual resource demand exceeding supply is controlled within a reasonable range (e.g., ensuring that (\ref{equ. PF BS C5}) and (\ref{equ. PF BS C6}) are satisfied, see line 15, Alg. 1).
	
	\textbf{Individual rationality of clients.} Lines 17-26 of Alg. 1 ensure that the value obtained by each client is at least equal to the payment it makes, thereby satisfying constraint (\ref{equ. PF MU C3}). Furthermore, lines 6, 18 and 23 of Alg. 1 guarantee that risks associated with each client are controlled within acceptable limits, satisfying constraints (\ref{equ. PF MU C4}), (\ref{equ. PF MU C5}), (\ref{equ. PF MU C6}), and (\ref{equ. PF MU C7}).
	 
	As a summary, clients and BSs are individual rationality in our proposed offRFW$^2$M.
\end{proof}


\begin{Prop}
	No blocking pair can exist in the Resource Trading for Communication Services in offRFW$^2$M.
\end{Prop}
\begin{proof}
	We show there is no blocking pair of either Type 1 or Type 2, as following:
	
	\noindent 
	$\bullet$ \textbf{There is no Type 1 blocking pair related to communication services of offRFW$^2$M.} We offer the proof by considering contradiction.
	
	Under a given matching $ \varphi^\mathsf{(c)} $, MU $ u_i $ and BS $ s_j $ form a Type 1 blocking pair $ \left(u_i;s_j;\mathbb{C}^\prime\right) $.
	If MU $ u_i $ does not sign a long-term contract with BS $ s_j $, when any of the following conditions is met: \textit{(i)} the final payment offered by MU $ u_i $ equals to its expected valuation; and \textit{(ii)} the risk is out of control (e.g., constraint (\ref{equ. PF MU C6})). For analytical simplicity, we use $ p_{i,j}^\mathsf{(c),max} $ to denoted the maximum payment from $u_i$ to $s_j$ under an accepted risk $ R_1^\mathsf{U} $. Thus, the final payment $ \mathbbm{c}^\mathsf{(c),Pay}_{i,j}$ can only refer to $\mathbb{E}[V^\mathsf{(c)}_{i,j}]$ or $p_{i,j}^\mathsf{(c),max} $, shown by (\ref{59}) and (\ref{60}).
	\begin{equation}\label{59}{\small
			\begin{aligned}
				\mathbbm{c}^\mathsf{(c),Pay}_{i,j} = \text{min}\left\{\mathbb{E}[V^\mathsf{(c)}_{i,j}],p_{i,j}^\mathsf{(c),max}\right\},
		\end{aligned}}
	\end{equation}
		\begin{equation}\label{60}
		\begin{aligned}
			&\mathbb{E}\left[u^\mathsf{(c),S}\left(s_j,\left\{\varphi\left( s_j \right)\backslash\widetilde{\varphi^\mathsf{(c)\prime}}\left( s_j \right)\right\} \cup \left\{ u_i \right\},\mathbb{C}^\prime \right)\right ] \\&< \mathbb{E}\left [u^\mathsf{(c),S}\left(s_j,\varphi\left( s_j\right),\mathbb{C}^\mathsf{(c)}_{i,j} \right)\right].\\
		\end{aligned}
	\end{equation} 
	
	If BS $ s_j $ selects MU $ u_i $, we have $ \mathbbm{c}^\mathsf{(c),Pay}_{i,j}\left\langle \mathcal{X}^\mathsf{*} \right\rangle\leq \mathbbm{c}^\mathsf{(c),Pay}_{i,j}\left\langle \mathcal{X} \right\rangle = \text{min}\left\{\mathbb{E}[V^\mathsf{(c)}_{i,j}],p_{i,j}^\mathsf{(c),max}\right\} $ and the following (\ref{81})
	\begin{equation}\label{81}{\small
			\begin{aligned}
				&\mathbb{E}\left[u^\mathsf{(c),S}\left(s_j,\left\{\varphi\left( s_j \right)\backslash\widetilde{\varphi^\mathsf{(c)\prime}}\left( s_j \right)\right\} \cup \left\{ u_i \right\},\mathbb{C}^\prime \right)\right ] \geq\\& \mathbb{E}\left[u^\mathsf{(c),S}\left(s_j,\left\{\varphi\left( s_j \right)\backslash\widetilde{\varphi^\mathsf{(c)\prime\prime}}\left( s_j \right)\right\} \cup \left\{ u_i \right\},\mathbb{C}^\prime \right)\right ],\\
		\end{aligned}}
	\end{equation}
	where $ 
	\widetilde{\varphi^\mathsf{(c)\prime\prime}}\left(s_j\right) \subseteq \widetilde{\varphi^\mathsf{(c)\prime}}\left(s_j\right) $. From (\ref{60}) and (\ref{81}), we can get
	\begin{equation}\label{key}\small{
			\begin{aligned}
				&\mathbb{E}\left [u^\mathsf{(c),S}\left(s_j,\varphi\left( s_j\right),\mathbb{C}^\mathsf{(c)}_{i,j} \right)\right]> \\&\mathbb{E}\left[u^\mathsf{(c),S}\left(s_j,\left\{\varphi\left( s_j \right)\backslash\widetilde{\varphi^\mathsf{(c)\prime\prime}}\left( s_j \right)\right\} \cup \left\{ u_i \right\},\mathbb{C}^\prime \right)\right ],
		\end{aligned}}
	\end{equation}
	which is contrary to (\ref{equ. 42}), thus ensuring the inexistence of Type 1 blocking pairs.
	
	\noindent 
	$\bullet$ \textbf{There is no Type 2 blocking pair related to communication services of offRFW$^2$M.}
	We conduct the proof by considering cases of contradiction. 
	
	Under a given matching $ \varphi^\mathsf{(s)} $, MU $ u_i $ and BS $ s_j $ form a Type 2 blocking pair $ \left(u_i;s_j;\mathbb{C}^\prime\right) $, as shown by (\ref{equ. 44}).
	If MU $ u_i $ is rejected by BS $ s_j $, the final payment of $ u_i $ can be set by $ \mathbbm{c}^\mathsf{(c),Pay}_{i,j} = \text{min}\left\{\mathbb{E}[V^\mathsf{(c)}_{i,j}],p_{i,j}^\mathsf{(c),max}\right\} $, where the only reason of such a rejection is that $ s_j $ has no surplus resources. However, the coexistence of (\ref{equ. 44}) shows that BS $ s_j $ has adequate resource supply to serve MUs, which contradicts our previous assumption. Therefore, we prove that there is no Type 2 blocking pair.
	
	As a summary, no blocking pair can exist during the matching related to communication services in offRFW$^2$M. 
\end{proof}


\begin{Prop}\label{Prop 10}
	No blocking pair can exist in the Resource Trading for Sensing Services in offRFW$^2$M
\end{Prop}
\begin{proof}
	We show there is no blocking pair of either Type 1 or Type 2, as following:
	
	\noindent 
	$\bullet$ \textbf{There is no Type 1 blocking pair related to sensing services of offRFW$^2$M.} We offer the proof by considering contradiction.
	
	Under a given matching $ \varphi^\mathsf{(s)} $, coalition $ \bm{c}_k $ and BS $ s_j $ form a Type 1 blocking pair $ \left(\bm{c}_k;s_j;\mathbb{C}^\prime\right) $.
	If coalition $ \bm{c}_k $ does not sign a long-term contract with BS $ s_j $, when any of the following conditions is met: \textit{(i)} the final payment offered by coalition $ \bm{c}_k $ equals to its expected valuation; and \textit{(ii)} the risk is out of control (e.g., constraint (\ref{equ. PF MU C7})). For analytical simplicity, we use $ p_{k,j}^\mathsf{(s),max} $ to denoted the maximum payment from $\bm{c}_k$ to $s_j$ under an accepted risk $ R_2^\mathsf{U} $. Thus, the final payment $ \mathbbm{c}^\mathsf{(s),Pay}_{k,j}$ can only refer to $\mathbb{E}[V^\mathsf{(s)}_{k,j}]$ or $p_{k,j}^\mathsf{(s),max} $, shown by (\ref{59a}) and (\ref{60a}).
	\begin{equation}\label{59a}{\small
			\begin{aligned}
				\mathbbm{c}^\mathsf{(s),Pay}_{k,j} = \text{min}\left\{\mathbb{E}[V^\mathsf{(s)}_{k,j}],p_{k,j}^\mathsf{(s),max}\right\},
		\end{aligned}}
	\end{equation}
	\begin{equation}\label{60a}
		\begin{aligned}
			&\mathbb{E}\left[u^\mathsf{(s),S}\left(s_j,\left\{\varphi\left( s_j \right)\backslash\widetilde{\varphi^\mathsf{(s)\prime}}\left( s_j \right)\right\} \cup \left\{ \bm{c}_k \right\},\mathbb{C}^\prime \right)\right ] \\&< \mathbb{E}\left [u^\mathsf{(s),S}\left(s_j,\varphi\left( s_j\right),\mathbb{C}^\mathsf{(s)}_{k,j} \right)\right].\\
		\end{aligned}
	\end{equation} 
	
	If BS $ s_j $ selects coalition $ \bm{c}_k $, we have $ \mathbbm{c}^\mathsf{(s),Pay}_{k,j}\left\langle \mathcal{X}^\mathsf{*} \right\rangle\leq \mathbbm{c}^\mathsf{(s),Pay}_{k,j}\left\langle \mathcal{X} \right\rangle = \text{min}\left\{\mathbb{E}[V^\mathsf{(s)}_{k,j}],p_{k,j}^\mathsf{(s),max}\right\} $ and the following (\ref{81a})
	\begin{equation}\label{81a}{\small
			\begin{aligned}
				&\mathbb{E}\left[u^\mathsf{(s),S}\left(s_j,\left\{\varphi\left( s_j \right)\backslash\widetilde{\varphi^\mathsf{(s)\prime}}\left( s_j \right)\right\} \cup \left\{ \bm{c}_k \right\},\mathbb{C}^\prime \right)\right ] \geq\\& \mathbb{E}\left[u^\mathsf{(s),S}\left(s_j,\left\{\varphi\left( s_j \right)\backslash\widetilde{\varphi^\mathsf{(s)\prime\prime}}\left( s_j \right)\right\} \cup \left\{ \bm{c}_k \right\},\mathbb{C}^\prime \right)\right ],\\
		\end{aligned}}
	\end{equation}
	where $ 
	\widetilde{\varphi^\mathsf{(s)\prime\prime}}\left(s_j\right) \subseteq \widetilde{\varphi^\mathsf{(s)\prime}}\left(s_j\right) $. From (\ref{60a}) and (\ref{81a}), we can get
	\begin{equation}\label{key}\small{
			\begin{aligned}
				&\mathbb{E}\left [u^\mathsf{(s),S}\left(s_j,\varphi\left( s_j\right),\mathbb{C}^\mathsf{(s)}_{k,j} \right)\right]> \\&\mathbb{E}\left[u^\mathsf{(s),S}\left(s_j,\left\{\varphi\left( s_j \right)\backslash\widetilde{\varphi^\mathsf{(s)\prime\prime}}\left( s_j \right)\right\} \cup \left\{ \bm{c}_k \right\},\mathbb{C}^\prime \right)\right ],
		\end{aligned}}
	\end{equation}
	which is contrary to (\ref{equ. 46}), and thus proving the inexistence of Type 1 blocking pairs.
	
	\noindent 
	$\bullet$ \textbf{There is no Type 2 blocking pair related to sensing services of offRFW$^2$M.}
	We conduct the proof by considering cases of contradiction. 
	
	Under a given matching $ \varphi^\mathsf{(s)} $, coalition $ \bm{c}_k $ and BS $ s_j $ form a Type 2 blocking pair $ \left(\bm{c}_k;s_j;\mathbb{C}^\prime\right) $, as shown by (\ref{equ. 48}).
	If coalition $ \bm{c}_k $ is rejected by BS $ s_j $, the final payment of $ \bm{c}_k $ can be set by $ \mathbbm{c}^\mathsf{(s),Pay}_{k,j} = \text{min}\left\{\mathbb{E}[V^\mathsf{(s)}_{k,j}],p_{k,j}^\mathsf{(s),max}\right\} $, where the only reason of such a rejection is that $ s_j $ has no surplus resources. However, the coexistence of (\ref{equ. 48}) shows that BS $ s_j $ has adequate resource supply to serve coalitions, which contradicts our previous assumption. Therefore, we prove that there is no Type 2 blocking pair.
	
	As a summary, no blocking pair can exist during the matching related to sensing services in offRFW$^2$M. 
\end{proof}

\begin{Prop}\label{Prop 11}
	(Fairness, Non-wastefulness, Strong Stability of offRFW$^2$M) offRFW$^2$M is fair, non-wasteful, strong stable.
\end{Prop}
\begin{proof}
	Since the matching result of Alg. 1 holds Propositions \ref{Prop 7}-\ref{Prop 10}, according to Propositions \ref{Prop 1}-\ref{Prop 4}, our proposed offRFW$^2$M is strongly fairness, non-wastefulness, strong stability.
\end{proof}

\begin{Prop}(Stability of Sensing Coalitions in offRFW$^2$M) The proposed offRFW$^2$M ensures that each sensing coalition $\bm{c}_k$ is stable.\end{Prop}
\begin{proof}
Due to line 24 in Alg. 1, each MU \( u_i \) in the sensing coalition \( \bm{c}_k \) ensures its expected utility to exceed \( u^\mathsf{(s)}_\mathsf{\min} \). Furthermore, MUs within a coalition share both costs and profits, leading to a lower expected utility per MU compared to trading individually. Also, even in the extreme case where only one MU from the coalition engages in a transaction, the incurred cost remains equal to the expected utility when trading independently. Therefore, we can conclude that joining coalition $\bm{c}_k$ will not result in a lower expected utility than trading as an individual.
\end{proof}



\begin{Prop}
	(Weak Pareto optimality of offRFW$^2$M) The proposed offRFW$^2$M provides a weak Pareto optimality.
\end{Prop}
\begin{proof}
	Reviewing our design of offRFW$^2$M, each participant (e.g., client, BS) makes decisions according to its preference list to determine the trading counterpart and the specific terms of the long-term contract (e.g., resource trading volume, transaction price, compensation price). If the alternative choice ranks higher in the participant's preference list, they will switch their matching target and long-term contract in the following round. Such a switch indicates that returning to the previous choice would not result in a higher expected utility. For an MU \( u_i \), if there exists a BS \( s_j \) that can offer a higher expected utility than its currently matched BS, \( u_i \) and \( s_j \) are more inclined to establish a matching relationship. This, however, forms a blocking pair. Since Proposition \ref{Prop 11} confirms that our proposed offRFW$^2$M is stable and free of blocking pairs, there is no possibility of Pareto improvement when the procedure of matching \( \varphi^\mathsf{(c)} \) terminates. Similarly, we can infer that there is no Pareto improvement in matching \( \varphi^\mathsf{(s)} \) (e.g., Propositions \ref{Prop 10} and \ref{Prop 11}). In conclusion, the offRFW$^2$M we study is said to be weak Pareto optimal.
\end{proof}

\section{Details of onEBW$^2$M}
\subsection{Key Definitions}
\begin{Defn}(M2M Matching for Communication Services in onEBW$^2$M)
	An M2M matching \( \nu^\mathsf{(c)} \) designed for communication services in onEBW$^2$M constitutes a two-way function/mapping between the BS set \( \bm{\mathcal{S}^\prime} \) and the MU set \(\bm{\mathcal{U}^\prime} \), satisfying the following properties:
	
	\noindent
	$\bullet$ For each BS $ s_{j} \in \bm{\mathcal{S}^\prime},\nu^\mathsf{(c)}\left( s_j \right) \subseteq \bm{\mathcal{U}^\prime} $, meaning that a BS can provide communication services to multiple MUs simultaneously based on its available resources.
	
	\noindent
	$\bullet$ For each MU $ u_i \in \bm{\mathcal{U}^\prime}, \nu^\mathsf{(c)}\left( u_i \right) \subseteq \bm{\mathcal{S}^\prime} $, and $|\nu^\mathsf{(c)}\left( u_i \right)|=1$, ensuring that each MU is assigned to exactly one BS, maintaining a structured association for stable temporary contracts.
	
	\noindent
	$\bullet$ For each BS $ s_j $ and MU $ u_i $, $ s_j\in\nu^\mathsf{(c)}(u_i)$ if and only if $ u_i\in\nu^\mathsf{(c)}\left(s_j\right) $, indicating that a valid matching occurs only when both the MU and the BS mutually accept the contract, ensuring reciprocal agreement.
\end{Defn}
\begin{Defn}(M2M Matching for Sensing Services in onEBW$^2$M)
	An M2M matching \( \nu^\mathsf{(s)} \) designed for sensing services in onEBW$^2$M constitutes a two-way function/mapping between the BS set \( \bm{\mathcal{S}^\prime} \) and the sensing coalition set \( \bm{\mathcal{C}^\prime} \), satisfying the following properties:
	
	\noindent
	$\bullet$ For each BS $ s_{j} \in \bm{\mathcal{S}^\prime},\nu^\mathsf{(s)}\left( s_j \right) \subseteq \bm{\mathcal{C}^\prime} $, meaning that a BS can provide sensing resources to multiple sensing coalitions simultaneously based on its available bandwidth and power.
	
	\noindent
	$\bullet$ For each MUs' coalition $ \bm{c}_k \in \bm{\mathcal{C}^\prime}, \nu^\mathsf{(s)}\left( \bm{c}_k \right) \subseteq \bm{\mathcal{S}^\prime} $, and $|\nu^\mathsf{(s)}\left( \bm{c}_k \right)|=1$, ensuring that each coalition is assigned to one or more BSs, allowing cooperative resource provisioning for enhanced sensing accuracy.
	
	\noindent
	$\bullet$ For each BS $ s_j $ and sensing coalition $ \bm{c}_k $, $ s_j\in\nu^\mathsf{(s)}(\bm{c}_k)$ if and only if $ \bm{c}_k\in\nu^\mathsf{(s)}\left(s_j\right) $, ensuring that a valid matching occurs only when both the BS and the coalition mutually accept the resource-sharing agreement, fostering stable and efficient sensing operations.
\end{Defn}
\subsection{Problem Formulation}
We formulate the bandwidth and power resource trading in the designed online mode as obtaining M2M matching between level-wise clients with unmet demands (individual MUs and sensing coalitions) and BSs with surplus supply, while simultaneously determining their temporary contracts. Similar to offline trading mode, the objective of each BS \( s_j \in \bm{\mathcal{S^\prime}} \) is to maximize its overall practical utility, as formulated by
\begin{subequations}{\small
		\begin{align}	\hspace{-3mm}\bm{\mathcal{F}^\mathsf{S^\prime}}\hspace{-1mm}{:}&\hspace{-2mm}\underset{{\dot{\mathbb{C}}^\mathsf{(c)}_{i,j},\dot{\mathbb{C}}^\mathsf{(s)}_{k,j}}}{\max}\hspace{-1mm}u^\mathsf{(c),S}(s_j,\nu^\mathsf{(c)}(s_j),\dot{\mathbb{C}}^\mathsf{(c)}_{i,j})\hspace{-1mm}+ \hspace{-1mm}u^\mathsf{(s),S}(s_j,\nu^\mathsf{(s)}(s_j),\dot{\mathbb{C}}^\mathsf{(s)}_{k,j}) \label{equ. PF BSa} \tag{83}\\
			\text{s.t.}~~~
			&\nu^\mathsf{(c)}\left(s_j\right)\subseteq\bm{\mathcal{U}^\prime},\nu^\mathsf{(s)}\left(s_j\right)\subseteq\bm{\mathcal{C}^\prime}, \mu^\prime\left(\bm{c}_k\right)\subseteq\bm{\mathcal{U}^\prime}, \tag{83a}\label{equ. PF BS C1a}\\
			&u_i\in \nu^\mathsf{(c)}(s_j), \bm{c}_k\in \nu^\mathsf{(s)}(s_j), u_i\in\mu^\prime(\bm{c}_k), \tag{83b}\label{equ. PF BS C2a}\\
			&\sum_{u_i\in\nu^\mathsf{(c)}(s_j)}B_{i,j}^\mathsf{(c)}+\sum_{\bm{c}_k\in\nu^\mathsf{(s)}(s_j)}\dot{\mathbbm{c}}^\mathsf{(s),B}_{k,j}\leq B^\prime_j, \tag{83c}\label{equ. PF BS C3a}\\
			&\sum_{u_i\in\nu^\mathsf{(c)}(s_j)}P_{i,j}^\mathsf{(c)}+\sum_{\bm{c}_k\in\nu^\mathsf{(s)}(s_j)}\dot{\mathbbm{c}}^\mathsf{(s),Pow}_{k,j}\leq P^\prime_j, \tag{83d}\label{equ. PF BS C4a}
	\end{align}}
\end{subequations}
In $ \bm{\mathcal{F}^\mathsf{S^\prime}} $, constraint (\ref{equ. PF BS C1a}) and (\ref{equ. PF BS C2a}) enforce that the set of MUs $\nu^\mathsf{(c)}(s_j)$ should belong to set $\bm{\mathcal{U}^\prime}$, the coalition set $\nu^\mathsf{(s)}(s_j)$ for sensing services must be covered by $\bm{\mathcal{C}^\prime}$, and the MU set $\mu^\prime(\bm{c}_k)$ of coalition $\bm{c}_k$ has to be within $\bm{\mathcal{U}^\prime}$. Constraints (\ref{equ. PF BS C3a}) and (\ref{equ. PF BS C4a}) ensure that the bandwidth and power resources sold by BS $s_j$ do not exceed its available supply $B_j^\prime$ and $P_j^\prime$. 
Furthermore, each client (i.e., $u_i $ or $\bm{c}_k$) also aims \textit{to maximize its utility}, as described by the following optimization problem
\begin{subequations}
		\begin{align}
			\bm{\mathcal{F}^\mathsf{U^\prime}}:~&
			\left\{ \begin{matrix}
				\underset{{\dot{\mathbb{C}}^\mathsf{(c)}_{i,j}}}{\max}~u^\mathsf{(c),U}(u_i,\nu^\mathsf{(s)}(u_i),\dot{\mathbb{C}}^\mathsf{(c)}_{i,j}) \\
				\underset{{\dot{\mathbb{C}}^\mathsf{(s)}_{k,j}}}{\max}~u^\mathsf{(s),U}(\bm{c}_k,\nu^\mathsf{(s)}(\bm{c}_k),\dot{\mathbb{C}}^\mathsf{(s)}_{k,j})
			\end{matrix}\right\}, \tag{84}\label{equ. PF MUa}\\
			\text{s.t.}~~~
			&\nu^\mathsf{(c)}\left(u_i\right)\subseteq\bm{\mathcal{S}^\prime}, \mu^\prime\left(u_i\right)\subseteq\bm{\mathcal{C}^\prime},\nu^\mathsf{(s)}\left(\bm{c}_k\right)\subseteq\bm{\mathcal{S}^\prime} ,\label{equ. PF MU C1a}\tag{84a}\\
			&s_j\in \nu^\mathsf{(c)}(u_i), \bm{c}_k\in \mu^\prime(u_i), s_j\in\nu^\mathsf{(s)}(\bm{c}_k), \tag{84b}\label{equ. PF MU C2a}\\
			&V^\mathsf{(c)}_{i,j}\ge \dot{\mathbbm{c}}^\mathsf{(c),Pay}_{i,j},V^\mathsf{(s)}_{i,j}=V^\mathsf{(s),max }_{k,j}\ge \dot{\mathbbm{c}}^\mathsf{(s),Pay}_{i,j}, \tag{84c}\label{equ. PF MU C3a}\\
			&V^\mathsf{(c)}_{i,j}\geq R^\mathsf{req},\tag{84d}\label{equ. PF MU C4a}\\
			&V^\mathsf{(s),max }_{k,j}\geq S^\mathsf{req},\tag{84e}\label{equ. PF MU C5a}\\
                        & B_{\min} \leq B^\mathsf{(c)}_{i,j}, B^\mathsf{(s)}_{k,j}\leq B_{\max},\tag{84f}\label{equ. PF MU C8a}\\
		&P_{\min} \leq P^\mathsf{(c)}_{i,j}, P^\mathsf{(s)}_{k,j}\leq P_{\max},\tag{84g}\label{equ. PF MU C9a}
	\end{align}
\end{subequations}
In $ \bm{\mathcal{F}^\mathsf{U^\prime}} $, constraints (\ref{equ. PF MU C1a}) and (\ref{equ. PF MU C2a}) are similar to constraints (\ref{equ. PF BS C1a}) and (\ref{equ. PF BS C2a}). Constraint (\ref{equ. PF MU C3a}) ensures that the obtained valuation of $u_i$ benefit from $s_j$ or $\bm{c}_k$ can cover its individual payment, while constraints (\ref{equ. PF MU C4a}) and (\ref{equ. PF MU C5a}) guarantee that the communication and sensing service quality of each MU or sensing coalition meets the corresponding requirements. Constraints (\ref{equ. PF MU C8a}) and (\ref{equ. PF MU C9a}) guarantee the bandwidth and power resources requested by each client for services are constrained within a certain range.

The online trading mode thus presents an MOO problem involving both $\bm{\mathcal{F}^\mathsf{S^\prime}}$ and $\bm{\mathcal{F}^\mathsf{U^\prime}}$, where the conflicting utilities of different parties make designing a win-win solution for them a complex task. To address this MOO problem, we propose onEBW$^2$M, which facilitates temporary contracts while achieving mutually -- beneficial practical utilities for both parties. The following sections discuss the detailed implementation of onEBW$^2$M.


\subsection{Solution Design}
\begin{algorithm}[t!] %其中这里面不能有H不然会报错,不过不影响结果
	{\footnotesize\setstretch{0.4}\caption{{Proposed Effective Backup Win-Win Matching for Online Trading}}%算法名字
		\LinesNumbered %要求显示行号
		\textbf{Initialization:} $ \mathcal{X} \leftarrow 1 $, $ \dot{\mathbbm{c}}^\mathsf{(c),Pay}_{i,j}\left\langle 1 \right\rangle \leftarrow p^\mathsf{\min}_{i,j}$, $ \dot{\mathbbm{c}}^\mathsf{(s),Pay}_{k,j}\left\langle 1 \right\rangle \leftarrow p^\mathsf{\min}_{k,j}$, ${flag}_{j} \leftarrow 1 $, $\mathbb{Y}^\mathsf{(c)}\left( u_i \right)\leftarrow \varnothing$, $\mathbb{Y}^\mathsf{(c)}\left( s_{j} \right)\leftarrow \varnothing$, $\mathbb{Y}^\mathsf{(s)}\left( \bm{c}_k \right)\leftarrow \varnothing$, $\mathbb{Y}^\mathsf{(s)}\left( s_{j} \right)\leftarrow \varnothing$\ %\;用于换行
		
		\For{$\forall u_i\in\bm{\mathcal{U}^\prime}$}{
			$\bm{c}_k\leftarrow u_i$ forms coalitions based on shared sensing target, where $\bm{c}_k\in\bm{\mathcal{C}^\prime}$
		}
		\While{$ \sum_{u_i\in\bm{\mathcal{U}^\prime}}{flag}_{i} $ and $\sum_{\bm{c}_k\in\bm{\mathcal{C}^\prime}}{flag}_{k}$}{
			\textbf{$ {flag}_{i} \leftarrow {\bf False} $, $ {flag}_{k} \leftarrow {\bf False} $}
			
			\textbf{Calculate:} $\overrightarrow{L^\mathsf{(c)}_i}$ and $\overrightarrow{L^\mathsf{(s)}_k}$ under constraints (\ref{equ. PF MU C4a}) and (\ref{equ. PF MU C5a})
			
			$ F^\mathsf{(c),\star}_i\left\langle \mathcal{X} \right\rangle\leftarrow \overrightarrow{L^\mathsf{(c)}_i}$, $ F^\mathsf{(s),\star}_k\left\langle \mathcal{X} \right\rangle\leftarrow \overrightarrow{L^\mathsf{(s)}_k}$      
			 $\mathbb{Y}^\mathsf{(c)}\left( u_i \right), B_{i,j}^\mathsf{(c)}\left\langle \mathcal{X} \right\rangle, P_{i,j}^\mathsf{(c)}\left\langle \mathcal{X} \right\rangle, \dot{\mathbbm{c}}^\mathsf{(c),Pay}_{i,j}\left\langle \mathcal{X} \right\rangle  \leftarrow F^\mathsf{(c),\star}_i\left\langle \mathcal{X} \right\rangle $, $ \mathbb{Y}^\mathsf{(s)}\left( \bm{c}_k \right), B_{k,j}^\mathsf{(s)}\left\langle \mathcal{X} \right\rangle, P_{k,j}^\mathsf{(s)}\left\langle \mathcal{X} \right\rangle, \dot{\mathbbm{c}}^\mathsf{(s),Pay}_{k,j}\left\langle \mathcal{X} \right\rangle \} \leftarrow F^\mathsf{(s),\star}_k\left\langle \mathcal{X} \right\rangle $
			
			
			\If{$ \forall\mathbb{Y}^\mathsf{(c)}\left( u_i \right) \neq \varnothing $ or $ \forall\mathbb{Y}^\mathsf{(s)}\left( \bm{c}_k \right) \neq \varnothing $}{
				\For{$\forall u_i \in \bm{\mathcal{U}^\prime}$ }{
					$ u_i $ sends a proposal about its information to $ s_j $, where $s_j\in\mathbb{Y}^\mathsf{(c)}\left( u_i \right)$}
				\For{$\forall \bm{c}_k \in \bm{\mathcal{C}^\prime}$ }{
					$ \bm{c}_k $ sends a proposal about its information to $ s_j $, where $s_j\in\mathbb{Y}^\mathsf{(s)}\left( \bm{c}_k \right)$}
				
				\While{
					$ \Sigma_{u_i\in \bm{\mathcal{U}^\prime}}{flag}_{i} > 0 $}{
					$ {\widetilde{\mathbb{Y}}}\left(s_j\right) \leftarrow$ collect proposals from clients
					
					$ \mathbb{Y}^\mathsf{(c)}(s_j)$, $\mathbb{Y}^\mathsf{(s)}(s_j) \leftarrow $ choose MUs from $ {\widetilde{\mathbb{Y}}}\left(s_j\right) $ that can achieve the maximization of the utility of BS $s_j$ (i.e., (\ref{equ. PF BSa})) by using DP under constraints (\ref{equ. PF BS C3a}) and (\ref{equ. PF BS C4a})
					
					$ s_j $ temporally accepts the clients in $ \mathbb{Y}^\mathsf{(c)}(s_j) $ and $ \mathbb{Y}^\mathsf{(s)}(s_j) $, and rejects the others
				}
				
				\For{
					$ \forall u_i \in \mathbb{Y}^\mathsf{(c)}\left( s_j \right) $
				}{
					\If{$ u_i $ is rejected by $ s_j $, $V^\mathsf{(c)}_{i,j}\ge \dot{\mathbbm{c}}^\mathsf{(c),Pay}_{i,j}$ and constraint (\ref{equ. PF MU C4a}) is met}{
						$ \dot{\mathbbm{c}}^\mathsf{(c),Pay}_{i,j}\left\langle {\mathcal{X} + 1} \right\rangle \leftarrow \min\left\{ \dot{\mathbbm{c}}^\mathsf{(c),Pay}_{i,j}\left\langle \mathcal{X} \right\rangle + \mathrm{\Delta}p~,{ V}^\mathsf{(c)}_{i,j} \right\} $}
					\Else{$ \dot{\mathbbm{c}}^\mathsf{(c),Pay}_{i,j}\left\langle {\mathcal{X} + 1} \right\rangle \leftarrow \dot{\mathbbm{c}}^\mathsf{(c),Pay}_{i,j}\left\langle \mathcal{X} \right\rangle $}
				}
				
				\For{
					$ \forall \bm{c}_k \in \mathbb{Y}^\mathsf{(s)}\left( s_j \right) $
				}{
					\If{$ u_i $ is rejected by $ s_j $, $V^\mathsf{(s)}_{k,j}\ge \dot{\mathbbm{c}}^\mathsf{(s),Pay}_{k,j}$ and constraint (\ref{equ. PF MU C5a}) is met}{
						$ \dot{\mathbbm{c}}^\mathsf{(s),Pay}_{k,j}\left\langle {\mathcal{X} + 1} \right\rangle \leftarrow \min\left\{ \dot{\mathbbm{c}}^\mathsf{(s),Pay}_{k,j}\left\langle \mathcal{X} \right\rangle + \mathrm{\Delta}p~,{ V}^\mathsf{(s)}_{k,j} \right\} $}
					\Else{$ \dot{\mathbbm{c}}^\mathsf{(s),Pay}_{k,j}\left\langle {\mathcal{X} + 1} \right\rangle \leftarrow \dot{\mathbbm{c}}^\mathsf{(s),Pay}_{k,j}\left\langle \mathcal{X} \right\rangle $}
				}

                $ p_{i,\mathbbm{n}}^\mathsf{(c)}\leftarrow \dot{\mathbbm{c}}^\mathsf{(c),Pay}_{i,j}\left\langle \mathcal{X}+1 \right\rangle, p_{i,\mathbbm{n}}^\mathsf{(c)} \in F^\mathsf{(c),\star}_{i}\left\langle \mathcal{X} \right\rangle$, $
                    p_{k,\mathbbm{m}}^\mathsf{(s)}\leftarrow \dot{\mathbbm{c}}^\mathsf{(c),Pay}_{i,j}\left\langle \mathcal{X}+1 \right\rangle, p_{i,\mathbbm{n}}^\mathsf{(c)} \in F^\mathsf{(c),\star}_{i}\left\langle \mathcal{X} \right\rangle$
                
				\If{$\mathcal{X}\le2$ and there exists $F^\mathsf{(c),\star}_{i}\left\langle \mathcal{X}-1 \right\rangle \neq F^\mathsf{(c),\star}_{i}\left\langle \mathcal{X} \right\rangle $ or $F^\mathsf{(s),\star}_{k}\left\langle \mathcal{X}-1 \right\rangle \neq F^\mathsf{(s),\star}_{k}\left\langle \mathcal{X} \right\rangle $}{
					$ {flag}_{i} \leftarrow {\bf True} $, $ {flag}_{k} \leftarrow {\bf True} $,}	\
					$ \mathcal{X}\leftarrow \mathcal{X}+1 $	
			}
			
			
		}		 
		
		
		$\nu^\mathsf{(c)}(s_j)\leftarrow\mathbb{Y}^\mathsf{(c)}(s_j)$, $\nu^\mathsf{(c)}(u_i)\leftarrow \mathbb{Y}^\mathsf{(c)}(u_i)$,
		$\nu^\mathsf{(s)}(s_j)\leftarrow\mathbb{Y}^\mathsf{(s)}(s_j)$, $\nu^\mathsf{(s)}(\bm{c}_k)\leftarrow \mathbb{Y}^\mathsf{(s)}(\bm{c}_k)$ , $\mathcal{X} \leftarrow \mathcal{X}-1$
		
		
		\textbf{Return:} $\dot{\mathbb{C}}_{i,j}^\mathsf{(c)} =\{ B_{i,j}^\mathsf{(c)}\left\langle \mathcal{X} \right\rangle, P_{i,j}^\mathsf{(c)}\left\langle \mathcal{X} \right\rangle, \dot{\mathbbm{c}}^\mathsf{(c),Pay}_{i,j}\left\langle \mathcal{X} \right\rangle \}$, $\dot{\mathbb{C}}_{k,j}^\mathsf{(s)} =\{ B_{k,j}^\mathsf{(s)}\left\langle \mathcal{X} \right\rangle, P_{k,j}^\mathsf{(s)}\left\langle \mathcal{X} \right\rangle, \mathbbm{s}^\mathsf{(s),Pay}_{k,j}\left\langle \mathcal{X} \right\rangle \} $}
\end{algorithm}
Our proposed onEBW$^2$M mechanism enables BSs with surplus resources and clients to negotiate the quantity and pricing of bandwidth and power resources for two distinct service types, similar with the offline mode: individual MUs engage in resource trading for communication services, while coalitions participate in resource trading for sensing services. Note that when there exist clients with unmet resource demands—including voluntary clients and those without long-term contracts—as well as BSs with surplus resources, we implement the onEBW$^2$M mechanism to establish temporary contracts for real-time resource allocation. Specifically, onEBW$^2$M is similar to the offRFW$^2$M mechanism we proposed. The main difference is that in onEBW$^2$M, transactions use current market/resource information to negotiate acceptable terms for temporary contracts, without considering expected utilities and risks introduced by dynamic networks. Thus, we omit its details here as constrained by limited space.

%The details are outlined in Alg. 2.
%
%\noindent
%\textbf{Step 1. Initialization} (line 1): At the beginning for Alg. 2, the initial payment of communication services by an individual MU $u_i$ is set as $p^\mathsf{(c)}_{i,j}\left\langle 1 \right\rangle = p^\mathsf{(c),\min}_{i,j}$, while the initial payment for sensing services by each coalition $\bm{c}_k$ is set as $p^\mathsf{(s)}_{k,j}\left\langle 1 \right\rangle = p^\mathsf{(s),\min}_{k,j}$ (line 1). We define the set $\mathbb{Y}^\mathsf{(c)}(u_i)$ to include the BSs that $u_i$ is interested in for communication service, while $\mathbb{Y}^\mathsf{(c)}(s_j)$ includes the MUs temporarily selected by $s_j$ for communication service. Similarly, set $\mathbb{Y}^\mathsf{(s)}(\bm{c}_k)$ comprises the BSs that $\bm{c}_k$ is interested in for sensing service, while $\mathbb{Y}^\mathsf{(s)}(s_j)$ covers the sensing coalitions temporarily selected by $s_j$. The entire negotiation process involves multiple rounds, indexed by $\mathcal{X}$, to achieve final convergence on temporary contracts.
%
%\noindent
%\textbf{Step 2. Establishment of MU coalitions and preference lists} (lines 2-6): Before the matching process, MUs form sensing coalitions based on their sensing targets. At the beginning of each round, each individual MU announces its resource requests for communication service to BSs according to its preference list, as defined in Definition \ref{def 5a}).
%\begin{Defn}(Preference List of MU in onEBW$^2$M)\label{def 5a}
%	The preference list \( \overrightarrow{L^\mathsf{(c)}_i} \) of an MU \( u_i \) regarding BSs is a vector of tuples representing contract terms \( \{s_j, B_{i,j}^\mathsf{(c)}, P_{i,j}^\mathsf{(c)}, p_{i,j}^\mathsf{(c)}\} \) under constraint (\ref{equ. PF MU C4a}), sorted in non-ascending order based on the utility value \( u^\mathsf{(c),U}(u_i, \nu^\mathsf{(c)}(u_i)) \): 
%	\begin{equation}{\small
%			\begin{aligned}
%				&\overrightarrow{L^\mathsf{(c)}_i} = [\{s_j, B_{i,j}^\mathsf{(c)}, P_{i,j}^\mathsf{(c)}, p_{i,j}^\mathsf{(c)}\} | \text{non-ascending on (\ref{equ. comm utility})} \\&\text{ under constraint (\ref{equ. PF MU C4a})}, \forall s_j \in \bm{\mathcal{S}^\prime}, \forall B^\mathsf{(c)}_{i,j} \in [B^\mathsf{\min}, B^\mathsf{\max}],\\& \forall P^\mathsf{(c)}_{i,j} \in [P^\mathsf{\min}, P^\mathsf{\max}]]
%		\end{aligned}}
%	\end{equation}
%	
%	\noindent where \( B_{\min} \) and \( B_{\max} \) represent the lower and upper bounds of bandwidth resources that each MU can request, while \( P_{\min} \) and \( P_{\max} \) denote the minimum and maximum power resources that can be allocated to each MU. 
%\end{Defn}
%Meanwhile, each coalition announces its resource requests for sensing services to BSs according to its preference list, as defined in Definition \ref{def 6a}).
%\begin{Defn}(Preference List of Sensing Coalition in onEBW$^2$M)\label{def 6a} The preference list \( \overrightarrow{L^\mathsf{(s)}_k} \) of a coalition \( \bm{c}_k \) regarding BSs is a vector of feasible contracts \( \{s_j, B_{k,j}^\mathsf{(s)}, P_{k,j}^\mathsf{(s)}, p_{k,j}^\mathsf{(s)}\} \) under constraint (\ref{equ. PF MU C5a}), sorted in non-ascending order based on the utility value of \( u^\mathsf{(s),U}(\bm{c}_k, \nu^\mathsf{(s)}(\bm{c}_k)) \):
%	\begin{equation}{\small
%			\begin{aligned}
%				&\overrightarrow{L^\mathsf{(s)}_k} = [\{s_j, B_{k,j}^\mathsf{(s)}, P_{k,j}^\mathsf{(s)}, p_{k,j}^\mathsf{(s)}\} | \text{non-ascending on (\ref{equ. sensing utility})} \\&\text{ under constraints (\ref{equ. PF MU C5a})}, \forall s_j \in \bm{\mathcal{S}^\prime}, \forall \dot{\mathbbm{c}}^\mathsf{(s),B}_{i,j} \in [B^\mathsf{\min}, B^\mathsf{\max}],\\& \forall \dot{\mathbbm{c}}^\mathsf{(s),Pow}_{i,j} \in [P^\mathsf{\min}, P^\mathsf{\max}]]
%		\end{aligned}}
%	\end{equation}
%\end{Defn}
%
%\noindent
%\textbf{Step 3: Proposal of clients} (lines 7-12): In the $\mathcal{X}$-th round, each individual MU $u_i$ and sensing coalition $\bm{c}_k$ select their preferred solutions $F^\mathsf{(c),\star}_{i}\left\langle \mathcal{X} \right\rangle$ and $F^\mathsf{(s),\star}_k\left\langle \mathcal{X} \right\rangle$ from their preference lists $\overrightarrow{L_i^\mathsf{(c)}}$ and $\overrightarrow{L_j^\mathsf{(s)}}$, and recording the selected BS \( s_j \) in \( \mathbb{Y}^\mathsf{(c)}(u_i) \) for communication services and \( \mathbb{Y}^\mathsf{(s)}(\bm{c}_k) \) for sensing services. Specifically, a feasible solution consists of the following three components:
%\textit{(i)} determining the BS to which the request should be sent,
%\textit{(ii)} deciding the amount of bandwidth and power resources requested by the client, and 
%\textit{(iii)} calculating the payment and cost for the MU. Each client then transmits its selected optimal solution to the BSs in \( \mathbb{Y}^\mathsf{(c)}(u_i)\) and \( \mathbb{Y}^\mathsf{(s)}(\bm{c}_k) \), initiating the resource negotiation process.
%
%\noindent
%\textbf{Step 5. Client selection on BSs' side} (lines 13-16): After collecting the information of individual MUs and sensing coalitions in set ${\widetilde{\mathbb{Y}}}\left(s_j\right)$, each BS $s_j$ solves a two-dimensional 0-1 knapsack problem, which can be solved using DP \cite{MY tsc}, to determine a temporary selection of MUs denoted by $\mathbb{Y}^\mathsf{(c)}(s_j)$ and of sensing coalitions $\mathbb{Y}^\mathsf{(s)}(s_j)$, where $\mathbb{Y}^\mathsf{(c)}(s_j)$ and $\mathbb{Y}^\mathsf{(s)}(s_j)$ belong to $ {\widetilde{\mathbb{Y}}}\left(s_j\right)$, maximizing the utility of BS \( s_j \) while satisfying constraints (\ref{equ. PF BS C3a}) and (\ref{equ. PF BS C4a}). Then, each $s_j$ reports its decisions to the corresponding MUs and sensing coalitions for the current round.
%
%\noindent
%\textbf{Step 6. Decision-making on clients' side} (lines 17-26): After receiving decisions from BSs, each MU \( u_i \in \mathbb{Y}^\mathsf{(c)}(s_j) \) and each coalition \( \bm{c}_k \in \mathbb{Y}^\mathsf{(s)}(s_j) \), evaluate their current solutions $F^\mathsf{(c),\star}_i\left\langle \mathcal{X} \right\rangle$ and $F^\mathsf{(s),\star}_k\left\langle \mathcal{X} \right\rangle$ . The payment for an individual MU \( u_i \) and coalition \( \bm{c}_k \) remains unchanged if any of the following conditions are met:
%\textit{(i)} \( u_i \) or \( \bm{c}_k \) is accepted by \( s_j \); 
%\textit{(ii)} the current payment \( \dot{\mathbbm{c}}^\mathsf{(c),Pay}_{i,j} \) or \( \dot{\mathbbm{c}}^\mathsf{(s),Pay}_{k,j} \) equals its valuation \( V^\mathsf{(c)}_{i,j} \); 
%\textit{(iii)} constraints (\ref{equ. PF MU C4a}) and (\ref{equ. PF MU C5a}) are not met. 
%Otherwise, \( u_i \) or \( \bm{c}_k \) will increase its bid associated with the current solution $F^\mathsf{(c),\star}_i\left\langle \mathcal{X} \right\rangle$ and $F^\mathsf{(s),\star}_k\left\langle \mathcal{X} \right\rangle$ for \( s_j \) in the next round to enhance its competitiveness in the market.
%
%\noindent
%\textbf{Step 7. Repeat} (lines 4-28): If all the preferred solutions stay unchanged from the $ (\mathcal{X}-1)^\mathsf{\text{th}} $ round to the $ \mathcal{X}^\mathsf{\text{th}} $ round, the matching process terminates at round $ \mathcal{X} $. We use $ \sum_{u_i\in\bm{\mathcal{U}^\prime}}{flag}_{i}={\bf False} $ and $\sum_{\bm{c}_k\in\bm{\mathcal{C}^\prime}}{flag}_{k}={\bf False} $ to denote this situation (line 5, Alg. 2). Otherwise, the process iterates through the previous steps (e.g., lines 4-28, Alg. 2) in the next round until convergence is achieved.

\noindent\textbf{Computational complexity of onEBW$^2$M:} The computational complexity of our proposed onEBW$^2$M depends on the number of rounds involved in Alg. 2, denoted by \( \mathcal{X}^{\mathsf{max}} \), the remain resources \( B^\prime_j \) and \( P^\prime_j \), as well as the number of clients sending requests to BS \( s_j \) in the $ \mathcal{X}^\mathsf{\text{th}} $ round, denoted as \( |\widetilde{\mathbb{Y}}(s_j)|_{\mathcal{X}} \). In particular, the overall complexity of onEBW$^2$M for each BS \( s_j \) is:
$\sum_{\mathcal{X}=1}^\mathsf{\mathcal{X}^{\mathsf{max}}} \mathcal{O}\left( |\widetilde{\mathbb{Y}}(s_j)|_{\mathcal{X}} \times B^\prime_j \times P^\prime_j \right)$.

\noindent \textbf{Solution Characteristics:} This work provides a novel perspective on online trading-driven temporary contract determination by designing a \textit{effective backup win-win matching mechanism that achieves mutually beneficial utilities} for both parties. From the clients' perspective, the preference list in practical transactions for each client is similar to  Definitions \ref{def 5} and \ref{def 6}, under constraints (\ref{equ. PF MU C4a}) and (\ref{equ. PF MU C5a}), ensuring that the selected solution maximizes their utility. From the BSs' perspective, the appropriate contract terms are selected from the feasible solutions reported by clients, utilizing a DP algorithm to maximize the utility of the BS (line 15, Alg. 2).


\subsection{Design Targets and Property Analysis}
In the following, we analyze the key properties of our unique matching mechanism within this online trading framework.

\begin{Defn}(Blocking Pair of Resource Trading for Communication Services in onEBW$^2$M)
	Under a given matching $ \nu^\mathsf{(c)} $, an MU $ u_i $, a BS set $ \mathbb{S} \subseteq \bm{\mathcal{S}^\prime}$ and a contract $\dot{\mathbb{C}}^\prime$ may form one of the following two types of blocking pairs, denoted by $ \left(u_i; \mathbb{S}; \dot{\mathbb{C}}^\prime\right) $.
	
	\noindent \textbf{Type 1 blocking pair:} Type 1 blocking pair satisfies the following two conditions:
	
	\noindent
	$\bullet$ MU $ u_i $ prefers the BS set $ \mathbb{S} \subseteq \bm{\mathcal{S}^\prime} $ over its currently matched BS set $ \nu^\mathsf{(c)}(u_i) $, meaning 
	\begin{equation}\label{key}
		\begin{aligned}
			{u^\mathsf{(c),U}}\left(u_i,\mathbb{S},\dot{\mathbb{C}}^\prime\right)>{u^\mathsf{(c),U}}\left (u_i,\nu^\mathsf{(c)}(u_i),\dot{\mathbb{C}}^\mathsf{(c)}_{i,j}\right ). 
		\end{aligned} 
	\end{equation}
	
	
	\noindent
	$\bullet$ Every BS in $ \mathbb{S} $ prefers to serve MU \( u_i \) rather than its currently matched MU set. That is, for any BS $ s_j\in \mathbb{S} $, there exists an MU set $ \nu^\mathsf{(c)\prime}(s_j) $ that constitutes the MUs that need to be evicted, satisfying
	\begin{equation}\label{88}
		\begin{aligned}
			&u^\mathsf{(c),S}\left(s_j,\left\{\nu\left( s_j \right)\backslash\nu^\mathsf{(c)\prime}\left( s_j \right)\right\} \cup \left\{ u_i \right\},\dot{\mathbb{C}}^\prime \right) \\&> u^\mathsf{(c),S}\left(s_j,\nu\left( s_j,\dot{\mathbb{C}}^\mathsf{(c)}_{i,j} \right) \right).\\
		\end{aligned}
	\end{equation} 
	
	\noindent \textbf{Type 2 blocking pair:} Type 2 blocking pair satisfies the following two conditions:
	
	\noindent
	$\bullet$ MU $ u_i $ prefers the BS set $ \mathbb{S} \subseteq \bm{\mathcal{S}^\prime} $ over its currently matched BS set $ \nu^\mathsf{(c)}(u_i) $, meaning
	\begin{equation}\label{key}
		\begin{aligned}
			{u^\mathsf{(c),U}}\left (u_i,\mathbb{S}, \dot{\mathbb{C}}^\prime \right )>{u^\mathsf{(c),U}}\left (u_i,\nu^\mathsf{(c)}(u_i),\dot{\mathbb{C}}^\mathsf{(c)}_{i,j} \right ).
		\end{aligned}
	\end{equation} 
	
	\noindent
	$\bullet$ Every BS in $ \mathbb{S} $ prefers to further serve MU $ u_i $ in conjunction to its currently matched/assigned MU set. That is, for any BS $ s_j\in \mathbb{S} $, we have
	\begin{equation}\label{90}
		\begin{aligned}
			\hspace{-3mm}u^\mathsf{(c),S}\left (s_j,\nu^\mathsf{(c)}(s_j)\cup\left\{ u_i \right\}, \dot{\mathbb{C}}^\prime\right )\hspace{-1mm}>\hspace{-1mm}u^\mathsf{(c),S}\left (s_j,\nu^\mathsf{(c)}(s_j),\dot{\mathbb{C}}^\mathsf{(c)}_{i,j} \right ) .
		\end{aligned} 
	\end{equation}
\end{Defn}

\begin{Defn}(Blocking Pair of Resource Trading for Sensing Services in onEBW$^2$M)
	Under a given matching $ \nu^\mathsf{(s)} $, coalition $ \bm{c}_k $, BS set $ \mathbb{S} \subseteq \bm{\mathcal{S}^\prime}$ and a contract $\dot{\mathbb{C}}^\prime$ may form one of the following two types of blocking pairs, denoted by $ \left(\bm{c}_k; \mathbb{S};\dot{\mathbb{C}}^\prime\right) $.
	
	\noindent \textbf{Type 1 blocking pair:} Type 1 blocking pair satisfies the following two conditions:
	
	\noindent
	$\bullet$ Coalition $ \bm{c}_k $ prefers the BS set $ \mathbb{S} \subseteq \bm{\mathcal{S}^\prime} $ rather than its currently matched BS set $ \nu^\mathsf{(s)}(\bm{c}_k) $, meaning 
	\begin{equation}\label{key}
		\begin{aligned}
			{u^\mathsf{(s),U}}(\bm{c}_k,\mathbb{S},\dot{\mathbb{C}}^\prime)>{u^\mathsf{(s),U}}\left (\bm{c}_k,\nu^\mathsf{(s)}(\bm{c}_k),\dot{\mathbb{C}}^\mathsf{(s)}_{k,j}\right ). 
		\end{aligned} 
	\end{equation}
	
	\noindent
	$\bullet$ Every BS in $ \mathbb{S} $ prefers to serve coalition \( \bm{c}_k \) rather than its currently matched coalitions. That is, for any BS $ s_j\in \mathbb{S} $, there exists a coalition set $ \nu^\mathsf{(s)\prime}(s_j) $ that constitutes the coalitions that need to be evicted, satisfying
	\begin{equation}\label{92}
		\begin{aligned}
			&u^\mathsf{(s),S}\left(s_j,\left\{\nu\left( s_j \right)\backslash\nu^\mathsf{(s)\prime}\left( s_j \right)\right\} \cup \left\{ \bm{c}_k \right\},\dot{\mathbb{C}}^\prime \right) \\&> u^\mathsf{(s),S}\left(s_j,\nu\left( s_j,\dot{\mathbb{C}}^\mathsf{(s)}_{k,j},\dot{\mathbb{C}}^\mathsf{(s)}_{k,j} \right) \right).\\
		\end{aligned}
	\end{equation} 
	
	\noindent \textbf{Type 2 blocking pair:} Type 2 blocking pair satisfies the following two conditions:
	
	\noindent
	$\bullet$ Coalition $ \bm{c}_k $ prefers the BS set $ \mathbb{S} \subseteq \bm{\mathcal{S}^\prime} $ to its currently matched BS set $ \nu^\mathsf{(s)}(u_i) $, i.e.,
	\begin{equation}\label{key}
		\begin{aligned}
		{u^\mathsf{(s),U}}(\bm{c}_k,\mathbb{S},\dot{\mathbb{C}}^\prime )>{u^\mathsf{(s),U}}\left (\bm{c}_k,\nu^\mathsf{(s)}(\bm{c}_k),\dot{\mathbb{C}}^\mathsf{(s)}_{k,j} \right ).
		\end{aligned}
	\end{equation} 
	
	\noindent
	$\bullet$ Every BS in $ \mathbb{S} $ prefers to further serve coalition $ \bm{c}_k $ in addition to its currently matched/assigned coalition set. That is, for any BS $ s_j\in \mathbb{S} $, we have
	\begin{equation}\label{94}
		\begin{aligned}
			&u^\mathsf{(s),S}(s_j,\nu^\mathsf{(s)}(s_j)\cup\left\{ \bm{c}_k \right\},\dot{\mathbb{C}}^\prime)>u^\mathsf{(s),S}(s_j,\nu^\mathsf{(s)}(s_j),\dot{\mathbb{C}}^\mathsf{(s)}_{k,j} ).
		\end{aligned} 
	\end{equation}
\end{Defn}
Building on the above two definitions, a Type 1 blocking pair undermines the stability of the matching by incentivizing a BS to reallocate its resources to a different set of clients that yield a higher utility. Similarly, a Type 2 blocking pair introduces instability, as the BS possesses residual resources that could be allocated to additional clients, thereby further maximizing its utility. These blocking pairs are used in the following to define the major characteristics of our proposed matching methodology.

\begin{Prop}\label{Prop 14}(Individual rationality of onEBW$^2$M) The proposed onEBW$^2$M mechanism ensures individual rationality for BSs, individual MUs, and sensing coalitions under the following conditions:
	
	\noindent
	$\bullet$ For each BS: the bandwidth and power resources of a BS $s_j$ booked to matched clients $\nu^\mathsf{(c)}\left(s_j\right)$ and coalitions $\nu^\mathsf{(s)}\left(s_j\right)$ does not exceed $B^\prime_j$ and $P^\prime_j$, i.e., constraints (\ref{equ. PF BS C3a}) and (\ref{equ. PF BS C4a}) are met.
	
	\noindent
	$\bullet$ For each client (i.e., each MU and each coalition): \textit{(i)} The value obtained by each client is at least equal to the payment it makes, ensuring that constraint (\ref{equ. PF MU C3a}) is met; \textit{(ii)} each client are satisfying constraints (\ref{equ. PF MU C4a}) and (\ref{equ. PF MU C5a}).
\end{Prop}


\begin{Prop}(Fairness of onEBW$^2$M): The proposed onEBW$^2$M mechanism ensures fairness by preventing the formation of Type 1 blocking pairs, ensuring that clients are satisfied with their matched BSs and no BS is incentivized to reallocate its resources to a different set of clients at the expense of existing agreements.\end{Prop}
\begin{Prop}(Non-wastefulness of onEBW$^2$M): onEBW$^2$M guarantees non-wastefulness by preventing the formation of Type 2 blocking pairs, ensuring that BSs efficiently utilize their resources without leaving surplus allocations that could accommodate additional clients.\end{Prop}

\begin{Prop}(Strong Stability of onEBW$^2$M) \label{Prop 17}The proposed onEBW$^2$M mechanism achieves strong stability by ensuring that the matching remains individually rational, fair, and non-wasteful.
\end{Prop}

\begin{Prop}(Stability of Sensing Coalitions in onEBW$^2$M)
	In onEBW$^2$M, each sensing coalition $\bm{c}_k$ is stable when the following conditions are satisfied: 
	
	\noindent $\bullet$ For each MU $u_i$ in sensing coalition $\bm{c}_k$, its utility onEBW$^2$M above $u^\mathsf{(s)}_\mathsf{\min}$; 
	
	\noindent $\bullet$ For each MU $u_i$ in sensing coalition $\bm{c}_k$, the utility obtained by joining the coalition \( \bm{c}_k \) is greater than the utility when trading as an individual.
\end{Prop}

Further, for the MOO problem collectively defined by $ \bm{\mathcal{F}^\mathsf{U^\prime}} $ and $ \bm{\mathcal{F}^\mathsf{S^\prime}} $, a Pareto improvement occurs when the social welfare (i.e., the summation of utilities of clients and BSs in the considered market)\cite{RW Matching3} can be increased with another feasible matching result. A matching is thus weakly Pareto optimal when no further Pareto improvement is possible, which is a desired property.

%For the MOO problem collectively given by $ \bm{\mathcal{F}^\mathsf{U}} $ and $ \bm{\mathcal{F}^\mathsf{S}} $, a Pareto improvement occurs when the \textit{social welfare (referring to a summation of utilities of clients and BSs in our considered market)}\footnote{The utilities of MUs are actually composed of two parts: \textit{(i)} the utility of individual MUs for resource trading in communication services, and \textit{(ii)} the utility of each MU within a coalition for resource trading in communication services. For easy analysis, we collectively refer to these as the utilities of MUs.} can be increased with another feasible matching result\cite{RW Matching3}. Thus, a matching is weak Pareto optimal when there is no Pareto improvement.

\begin{Prop}(Weak Pareto optimality of onEBW$^2$M) The proposed onEBW$^2$M mechanism ensures weak Pareto optimality by preventing further Pareto improvements, ensuring that no alternative matching configuration can increase the social welfare of the system.
\end{Prop}

We next examine the aforementioned property of onEBW$^2$M, as outlined below:

\begin{Prop}\label{Prop 19}
	(Convergence of a set of matching in onEBW$^2$M) Alg. 2 converges within finite rounds.
\end{Prop}
\begin{proof}
	As the onEBW$^2$M refers to a set of M2M matching (matching between BSs and individual MUs, as well as matching between BSs and coalitions), we utilize the DP algorithm to transform the problem into a two-dimensional 0-1 knapsack problem \cite{RW Matching3}. After a finite number of rounds, each client's payment can either be accepted or reach its maximum value while considering constraints (\ref{equ. PF BS C3a}) and (\ref{equ. PF BS C4a}) (e.g., lines 17-26, Alg. 2), which ensuring the convergence.
\end{proof}

\begin{Prop}
	(Individual rationality of onEBW$^2$M) The proposed onEBW$^2$M mechanism ensures individual rationality for All the BSs, individual MUs, and sensing coalitions are individual rational in the onEBW$^2$M.
\end{Prop}
\begin{proof}
	We offer the analysis on proving the individual rationality of both BSs and clients.
	
	\textbf{Individual rationality of BSs.} each BS $s_j$ regards $B^\prime_j$ and $P^\prime_j$ as up limit of resources for serving MUs, and the actual number of matched clients of $s_j$ will definitely not exceed its remain resource supply (e.g., line 15, Alg. 2).
	
	\textbf{Individual rationality of clients.} Lines 17-26 of Alg. 2 ensure that the value obtained by each client is at least equal to the payment it makes, thereby satisfying constraint (\ref{equ. PF MU C3a}). Furthermore, lines 6, 18 and 23 of Alg. 2 guarantee that each client are satisfying constraints (\ref{equ. PF MU C4a}) and (\ref{equ. PF MU C5a}),.
	
	As a summary, clients and BSs are individual rationality in our proposed onEBW$^2$M.
\end{proof}


\begin{Prop}
	No blocking pair can exist in the Resource Trading for Communication Services in onEBW$^2$M.
\end{Prop}
\begin{proof}
	We show there is no blocking pair of either Type 1 or Type 2, as following:
	
	\noindent 
	$\bullet$ \textbf{There is no Type 1 blocking pair related to communication services of onEBW$^2$M.} We offer the proof by considering contradiction.
	
	Under a given matching $ \nu^\mathsf{(c)} $, MU $ u_i $ and BS $ s_j $ form a Type 1 blocking pair $ \left(u_i;s_j;\dot{\mathbb{C}}^\prime\right) $.
	If MU $ u_i $ does not trading with BS $ s_j $, the payment of MU $ u_i $ during the last round can only be its valuation $V^\mathsf{(c)}_{i,j}$, as shown by (\ref{59A}) and (\ref{60A}).
	\begin{equation}\label{59A}{\small
			\begin{aligned}
				\dot{\mathbbm{c}}^\mathsf{(c),Pay}_{i,j} = V^\mathsf{(c)}_{i,j},
		\end{aligned}}
	\end{equation}
	\begin{equation}\label{60A}
		\begin{aligned}
			&u^\mathsf{(c),S}\left(s_j,\left\{\nu\left( s_j \right)\backslash\widetilde{\nu^\mathsf{(c)\prime}}\left( s_j \right)\right\} \cup \left\{ u_i \right\},\dot{\mathbb{C}}^\prime \right) \\&< u^\mathsf{(c),S}\left(s_j,\nu\left( s_j\right),\dot{\mathbb{C}}^\mathsf{(c)}_{i,j} \right).\\
		\end{aligned}
	\end{equation} 
	
	If BS $ s_j $ selects MU $ u_i $, we have $ \dot{\mathbbm{c}}^\mathsf{(c),Pay}_{i,j}\left\langle \mathcal{X}^\mathsf{*} \right\rangle\leq \dot{\mathbbm{c}}^\mathsf{(c),Pay}_{i,j}\left\langle \mathcal{X} \right\rangle =V^\mathsf{(c)}_{i,j} $ and the following (\ref{81aa})
	\begin{equation}\label{81aa}{\small
			\begin{aligned}
				&u^\mathsf{(c),S}\left(s_j,\left\{\nu\left( s_j \right)\backslash\widetilde{\nu^\mathsf{(c)\prime}}\left( s_j \right)\right\} \cup \left\{ u_i \right\},\dot{\mathbb{C}}^\prime \right) \geq\\& u^\mathsf{(c),S}\left(s_j,\left\{\nu\left( s_j \right)\backslash\widetilde{\nu^\mathsf{(c)\prime\prime}}\left( s_j \right)\right\} \cup \left\{ u_i \right\},\dot{\mathbb{C}}^\prime \right),\\
		\end{aligned}}
	\end{equation}
	where $ 
	\widetilde{\nu^\mathsf{(c)\prime\prime}}\left(s_j\right) \subseteq \widetilde{\nu^\mathsf{(c)\prime}}\left(s_j\right) $. From (\ref{60a}) and (\ref{81aa}), we can get
	\begin{equation}\label{key}\small{
			\begin{aligned}
				&u^\mathsf{(c),S}\left(s_j,\nu\left( s_j\right),\dot{\mathbb{C}}^\mathsf{(c)}_{i,j} \right)> \\&u^\mathsf{(c),S}\left(s_j,\left\{\nu\left( s_j \right)\backslash\widetilde{\nu^\mathsf{(c)\prime\prime}}\left( s_j \right)\right\} \cup \left\{ u_i \right\},\dot{\mathbb{C}}^\prime \right),
		\end{aligned}}
	\end{equation}
	which is contrary to (\ref{88}),thus ensuring the inexistence of Type 1 blocking pairs.
	
	\noindent 
	$\bullet$ \textbf{There is no Type 2 blocking pair related to communication services of onEBW$^2$M.}
	We conduct the proof by considering cases of contradiction. 
	
	Under a given matching $ \nu^\mathsf{(s)} $, MU $ u_i $ and BS $ s_j $ form a Type 2 blocking pair $ \left(u_i;s_j;\dot{\mathbb{C}}^\prime\right) $, as shown by (\ref{94}).
	If MU $ u_i $ is rejected by BS $ s_j $, the final payment of $ u_i $ can be set by $ \dot{\mathbbm{c}}^\mathsf{(c),Pay}_{i,j} = V^\mathsf{(c)}_{i,j} $, where the only reason of such a rejection is that $ s_j $ has no surplus resources. However, the coexistence of (\ref{90}) shows that BS $ s_j $ has adequate resource supply to serve MUs, which contradicts our previous assumption. Therefore, we prove that there is no Type 2 blocking pair.
	
	As a summary, no blocking pair can exist during the matching related to communication services in onEBW$^2$M. 
\end{proof}


\begin{Prop}\label{Prop 22}
	No blocking pair can exist in the Resource Trading for Sensing Services in onEBW$^2$M.
\end{Prop}
\begin{proof}
	We show there is no blocking pair of either Type 1 or Type 2, as following:
	
	\noindent 
	$\bullet$ \textbf{There is no Type 1 blocking pair related to sensing services of onEBW$^2$M.} We offer the proof by considering contradiction.
	
	Under a given matching $ \nu^\mathsf{(s)} $, coalition $ \bm{c}_k $ and BS $ s_j $ form a Type 1 blocking pair $ \left(\bm{c}_k;s_j;\dot{\mathbb{C}}^\prime\right) $.
	If $ \bm{c}_k $ does not trading with BS $ s_j $, the payment of MU $ u_i $ during the last round can only be its valuation $V^\mathsf{(s)}_{i,j}$, as shown by (\ref{59A1}) and (\ref{60A1}).
	\begin{equation}\label{59A1}{\small
			\begin{aligned}
				\dot{\mathbbm{c}}^\mathsf{(s),Pay}_{k,j} = V^\mathsf{(s)}_{k,j},
		\end{aligned}}
	\end{equation}
	\begin{equation}\label{60A1}
		\begin{aligned}
			&u^\mathsf{(s),S}\left(s_j,\left\{\nu\left( s_j \right)\backslash\widetilde{\nu^\mathsf{(s)\prime}}\left( s_j \right)\right\} \cup \left\{ \bm{c}_k \right\},\dot{\mathbb{C}}^\prime \right) \\&<u^\mathsf{(s),S}\left(s_j,\nu\left( s_j\right),\dot{\mathbb{C}}^\mathsf{(s)}_{k,j} \right).\\
		\end{aligned}
	\end{equation} 
	
	If BS $ s_j $ selects coalition $ \bm{c}_k $, we have $ \dot{\mathbbm{c}}^\mathsf{(s),Pay}_{k,j}\left\langle \mathcal{X}^\mathsf{*} \right\rangle\leq \dot{\mathbbm{c}}^\mathsf{(s),Pay}_{k,j}\left\langle \mathcal{X} \right\rangle =V^\mathsf{(s)}_{k,j} $ and the following (\ref{81aaa})
	\begin{equation}\label{81aaa}{\small
			\begin{aligned}
				&u^\mathsf{(s),S}\left(s_j,\left\{\nu\left( s_j \right)\backslash\widetilde{\nu^\mathsf{(s)\prime}}\left( s_j \right)\right\} \cup \left\{ \bm{c}_k \right\},\dot{\mathbb{C}}^\prime \right) \geq\\& u^\mathsf{(s),S}\left(s_j,\left\{\nu\left( s_j \right)\backslash\widetilde{\nu^\mathsf{(s)\prime\prime}}\left( s_j \right)\right\} \cup \left\{ \bm{c}_k \right\},\dot{\mathbb{C}}^\prime \right),\\
		\end{aligned}}
	\end{equation}
	where $ 
	\widetilde{\nu^\mathsf{(s)\prime\prime}}\left(s_j\right) \subseteq \widetilde{\nu^\mathsf{(s)\prime}}\left(s_j\right) $. From (\ref{60A1}) and (\ref{81aaa}), we can get
	\begin{equation}\label{key}\small{
			\begin{aligned}
				&u^\mathsf{(s),S}\left(s_j,\nu\left( s_j\right),\dot{\mathbb{C}}^\mathsf{(s)}_{k,j} \right)> \\&u^\mathsf{(s),S}\left(s_j,\left\{\nu\left( s_j \right)\backslash\widetilde{\nu^\mathsf{(s)\prime\prime}}\left( s_j \right)\right\} \cup \left\{ \bm{c}_k \right\},\dot{\mathbb{C}}^\prime \right),
		\end{aligned}}
	\end{equation}
	which is contrary to (\ref{92}), and thus proving the inexistence of Type 1 blocking pairs.
	
	\noindent 
	$\bullet$ \textbf{There is no Type 2 blocking pair related to sensing services of onEBW$^2$M.}
	We conduct the proof by considering cases of contradiction. 
	
	Under a given matching $ \nu^\mathsf{(s)} $, coalition $ \bm{c}_k $ and BS $ s_j $ form a Type 2 blocking pair $ \left(\bm{c}_k;s_j;\dot{\mathbb{C}}^\prime\right) $, as shown by (\ref{94}).
	If coalition $ \bm{c}_k $ is rejected by BS $ s_j $, the final payment of $ \bm{c}_k $ can be set by $ \dot{\mathbbm{c}}^\mathsf{(s),Pay}_{k,j} =V^\mathsf{(s)}_{k,j}$, where the only reason of such a rejection is that $ s_j $ has no surplus resources. However, the coexistence of (\ref{94}) shows that BS $ s_j $ has adequate resource supply to serve coalitions, which contradicts our previous assumption. Therefore, we prove that there is no Type 2 blocking pair.
	
	As a summary, no blocking pair can exist during the matching related to sensing services in onEBW$^2$M. 
\end{proof}

\begin{Prop}\label{Prop 24}
	(Fairness, Non-wastefulness, Strong Stability of onEBW$^2$M) onEBW$^2$M is fair, non-wasteful, and strongly stable.
\end{Prop}
\begin{proof}
	Since the matching result of Alg. 2 holds Propositions \ref{Prop 19}-\ref{Prop 22}, according to Propositions \ref{Prop 14}-\ref{Prop 17}, our proposed onEBW$^2$M is strongly fairness, non-wastefulness, strong stability.
\end{proof}

\begin{Prop}(Stability of Sensing Coalitions in onEBW$^2$M) The proposed onEBW$^2$M ensures that each sensing coalition $\bm{c}_k$ is stable.\end{Prop}
\begin{proof}
	Due to line 24 in Alg. 2, each MU \( u_i \) in the sensing coalition \( \bm{c}_k \) ensures its utility to exceed \( u^\mathsf{(s)}_\mathsf{\min} \). Furthermore, MUs within a coalition share both costs and profits, leading to a lower utility per MU compared to trading individually. Therefore, we can conclude that joining coalition $\bm{ c}_k$ will not result in a lower utility than trading as an individual.
\end{proof}



\begin{Prop}
	(Weak Pareto optimality of onEBW$^2$M) The proposed onEBW$^2$M provides a weak Pareto optimality.
\end{Prop}
\begin{proof}
Reviewing our design of onEBW$^2$M, each participant (e.g., client, BS) makes decisions according to its preference list to determine the trading counterpart and the specific terms of the temporary contract. If the alternative choice ranks higher in the participant's preference list, they will switch their matching target and contract in the following round. Such a switch indicates that returning to the previous choice would not result in a higher utility. For an MU \( u_i \), if there exists a BS \( s_j \) that can offer a higher utility than its currently matched BS, \( u_i \) and \( s_j \) are more inclined to establish a matching relationship. This, however, forms a blocking pair. Since Proposition \ref{Prop 24} confirms that our proposed onEBW$^2$M is stable and free of blocking pairs, there is no possibility of Pareto improvement when the procedure of matching \( \nu^\mathsf{(c)} \) terminates. Similarly, we can infer that there is no Pareto improvement in matching \( \nu^\mathsf{(s)} \) (e.g., Propositions \ref{Prop 22} and \ref{Prop 24}). In conclusion, the onEBW$^2$M game we study is said to be weak Pareto optimal.
\end{proof}


\end{document}