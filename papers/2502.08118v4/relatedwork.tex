\section{Related Works}
This section explores existing research, with a particular focus on resource sharing/allocation in ISAC networks and the application of matching theory in broader networking contexts.


\subsubsection{Resource Allocation in ISAC Networks} Efficient resource allocation in ISAC networks has recently emerged as a critical research area \cite{RW ISAC1,RW ISAC2,RW ISAC3,RW ISAC4}. \textit{Li et al.} \cite{RW ISAC1} introduced a value-of-service-oriented resource allocation scheme for multi-MU collaborative ISAC networks, facilitating concurrent heterogeneous service provisioning. \textit{Du et al.} \cite{RW ISAC2} developed an ISAC resource allocation framework, integrating sensing quality of service (QoS) to optimize resources for diverse applications. \textit{Hu et al.}\cite{RW ISAC3} proposed an ISAC-aided edge computing framework, addressing the rapid proliferation of vehicles and the growing demand for integrating vehicular communications with computing services. \textit{Wang et al.}\cite{RW ISAC4} explored an environment-aware ISAC architecture for green ad-hoc networks, examining the intricate interplay between environmental sensing and data transmission performance.

These above efforts have primarily relied on onsite (online) decision-making approaches, which, while offering valuable contributions, often lead to extended delays, higher energy consumption, and potential service disruptions. Subsequently, our work introduces a hybrid resource trading framework that integrates offline and online trading modes, leveraging long-term contracts to reduce real-time decision overhead, while employing online trading as a backup mechanism to handle dynamic fluctuations in resource demand and supply. 
%Furthermore, we introduce overbooking strategies to enhance resource availability and mitigate inefficiencies caused by uncertainty in ISAC networks. To further optimize resource utilization, we propose a collaborative sensing model, where MUs with shared sensing objectives form coalitions, reducing redundant resource consumption and ensuring a more efficient and equitable resource allocation process. Collectively, these innovations establish a fast, resilient, and adaptable ISAC resource trading scheme that ensures efficient and reliable operations in dynamic network environments.

\subsubsection{Matching-Based Resource Provisioning}
This work develops one of the first efficient matching/assignment mechanisms between clients (MUs and coalitions) and BSs to ensure responsive and cost-effective resource provisioning in ISAC networks. Similar methods have gained attention in other domains such as edge computing and the Internet of Things (IoT), where matching-driven resource allocation is used for balancing diverse demands with available resource supplies \cite{RW Matching1,RW Matching2,RW Matching3,RW Matching4,RW Matching5}. \textit{Ye et al.} \cite{RW Matching1} combined deep reinforcement learning with stable matching to enable adaptive resource allocation in mobile crowdsensing. \textit{Xu et al.} \cite{RW Matching2} introduced a three-sided stable matching with an optimal pricing scheme for distributed vehicular networks. \textit{Qi et al.} \cite{RW Matching3} explored cross-layer pre-matching mechanisms to achieve cost-effective resource trading in cloud-aided edge networks. \textit{Sharghivand et al.} \cite{RW Matching4} considered QoS in terms of service response time and proposed a matching model between cloudlets and IoT applications. \textit{Du et al.} \cite{RW Matching5} developed a matching-based approach for computing resource management in small-cell networks, optimizing resource allocation and service pricing.

While these studies have made certain efforts, they primarily focus on isolated resource types, such as bandwidth, computing capacity, or specific service-related assets. However, ISAC networks require a nuanced approach, as they simultaneously support both communication and sensing services. In particular, unlike conventional wireless networking systems, ISAC necessitates the joint allocation of multiple resource types, particularly bandwidth and power, to balance the dual objectives of reliable communication and accurate sensing. Subsequently, by incorporating \textit{(i)} joint resource allocation strategies for stable matching between service demands and available resources, \textit{(ii)} a hybrid framework that integrates offline and online trading modes, and \textit{(iii)} optimizing both communication and sensing capabilities, our framework establishes one of the first scalable, resilient, and economically viable ISAC trading markets.

\vspace{-4mm}