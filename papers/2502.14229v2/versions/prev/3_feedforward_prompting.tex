\section{Feedforward Prompting}
\label{section:feedforward_prompting}


% Features and various applications of this prompting method

% - Textual Outline Preview (Update: this should be a more direct representation, not just literally an outline component)
% - Preview of multiple possible interpretations (instead of asking, did you mean Apple the fruit or company, it just gives you two possible outputs, and lets you choose by either clicking which one, or keep typing to add more context to your prompt)
% - Outline is Directly Expandable/Modifiable -- slowly can become the response itself
% - Directly revies the previous output (you can also choose to ignore this if you'd like)
% - Select + Feedforward = 

% Types of Feedforward
% - Outline (Anticipated content and structure it will generate)
% - Where you will reivse content (hint by creating a gap/shade at a location)
% - How much content you will revise (hint by visually expanding something some length)
% - What the tone may sound like (hint with concrete phrases)
% - What kinds of representations you might want (hint at showing a bullet point list, but you switch to table)
% - 

% Feedforward + Chain of Thought
% Chain of Thought begins as the user types. These "chains" then can be further expanded by the user, where the user can explore each step of the chain that the LLM anticipates will be the change...
% The end-user is then able to adjust the stages

% The core idea about feedforward prompting is an interaction paradigm that prompts you to add more context to your prompt. It's this very rapid back and forth, incremental understanding of one another to reach this ideal prompt that has a good idea of what you might want to get....




% Maybe this workshop paper will explore three case studies, maybe:
% 1) Conversational Interface
% 2) Writing System
% 3) Generative UI


% Maybe with Graphic Art, you have heavier loads of generation, like selecting and autofill. and the feedforward is showing the actual scope it might change as well...
% Or maybe what will be replaced, what won't, what might shift -- these results also affords users to make more granular changes, since they have the control to do so now.
% I think I can still explore this kind of pattern on a writing interface as well. First making broad strokes and changes, but feedforward provides hints at what text will be revised, and basically prompts users to make more granular changes.

% Maybe the exploration is really the design space of understanding the representations necessary for diverse kinds of LLM feedforward info.

% Various applications: Generative and Malleable UI, Graphics/Design, Direct Text Editing