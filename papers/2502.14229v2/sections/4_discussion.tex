\section{Future Work}
\label{section:discussion}

%  implement feedforward in GenAI systems across all contexts and applications, making it
% Our goal is to make feedforward a fundamental design component in all GenAI systems. While this paper demonstrates several feedforward designs for GenAI, we must further consider how we can effectively design and implement feedforward for GenAI systems in all applications.

Our goal is to establish feedforward as a fundamental design component in all GenAI systems. While this paper presents several feedforward designs, we envision a more comprehensive design space to represent and guide feedforward design across all GenAI applications. Based on our four prototypes, we identify three potential design dimensions for GenAI feedforward: representation, level of detail, and manipulability.

First, our examples implemented feedforward representations in the form of outlines, minimaps, lists of operations, example phrases, and multiple cursors. We may identify categories of representations that help distinguish which feedforward representations are more useful for different use cases. For instance, while a list of operations might inform users about the type of UI a GenAI system will generate, a wireframe might better communicate the structure and layout of the UI.

Second, we explored different ways to present varying levels of detail in feedforward. For example, the conversational UI displays an outline summarizing key topics, while the minimap omits textual details and instead presents blocks of paragraphs. This dimension aligns with previous research on feedforward \cite{bau2008octopocus}. A more rigorous investigation could explore optimal levels of detail for different types of feedforward representations in GenAI, as well as allowing users to define the level and type of detail themselves.

Lastly, we explored how users can manipulate feedforward content, either by revising their prompts or by directly resizing, repositioning, or selecting elements. We aim to investigate additional interaction techniques that enhance user control and engagement with feedforward designs.

Future work should expand on this preliminary design space by gathering, analyzing, and critiquing a broader range of GenAI systems to identify variations of feedforward across diverse contexts.

% Second, we must streamline the development of feedforward for GenAI.
% In our conceptual implementation of feedforward within a conversational LLM interface, the AI generates feedforward components when the user pauses typing.
% These components then serve as contextual input for the LLM upon submitting the prompt, helping it structure and guide the full response.
% While this approach is straightforward for a single conversational interface, this may not be the most cost-efficient approach for implementing feedforward. Perhaps other strategies may be to support feedforward algorithmically without the use of LLMs \cite{zhutian2024sketchgenerate}, but this might constrain the generative capabilities of the LLM.
% Furthermore, repeatedly implementing the same feedforward-GenAI pipeline across multiple applications, regardless of its implementation details, can become tedious.
% Future work could also provide a developer toolkit---perhaps a web-based UI library---that provides a catalog of feedforward components for developers to seamlessly integrate them into any GenAI web application.





% We want to work towards a design space for feedforward in genAI.
% From our prototype explorations in four applications: conversation UIs, document editors, malleable interfaces, and agent automations, we surface preliminary dimensions: 1) representation, 2) level of detail, 3) manipulability
% providing outline, minimap, list of operations, cut-outs of examples, these representations are forms of abstractions that are useful for feedforward in genai.
% level of detail is another, balancing the level of detail in feedforward is crucial in making sure users are able to understand what's ahead without being overwhelmed. Researchers have identified instances where there was cognitive burden in code completions and image generation. While balancing this can be a fine line, we believe the right level of detail can be identified for the many possible representations for feedforward in genAI systems.
% level of detail seemed to also be basically how far into the generation it goes, for instnace, the highest level of detail for a feedforward outline may be the full response itself.
% We explored various ways users can revise their prompt or steer AI upon finding out misalignment through feedforward. We explored that while users can revise their prompt to disambiguate their intents, they can also directly manipualte the feedforward content or specify their prompt by making AI-generated adjustments on the feedforward content.......

% we can take further steps to integrate feedforward into more GenAI systems.

% First, we can streamline the design process of feedforward for GenAI, we must solidify a comprehensive design space. From our prototype explorations in four applications: conversation UIs, document editors, malleable interfaces, and agent automations, we surfaced three preliminary dimensions: 1) representation, 2) level of detail, and 3) manipulability.
% We designed feedforward to be represented in an outline, minimap, list of operations, example phrases, and multiple cursors. We anticipate a rigorous investigation of feedforward will surface more representations suitable for each of their unique contexts.
% We also explored ways to present various levels of detail of feedforward. The conversational UI presents an outline detailing the key topics, while the minimap presents the topics organized into paragraphs.
% We aim to further investigate the ``right'' levels of detail across various feedforward representations to unsure users can anticipate AI's response without experiencing too much cognitive burden. 
% We also explored various ways users can revise their prompt or steer AI upon identifying misalignments between the user and AI. We explored how users can disambiguate their intents by either revising the text of their prompt and also directly manipulating the feedforward content via resizing, repositioning, and selecting.
% Future work should expand on this preliminary design space by gathering, analyzing, and critiquing a comprehensive collection of GenAI systems and identifying potential variations of feedforward across different contexts.
% We need to work on the design space
% We need to make it implementable

% Second, we can streamline the development of feedforward for GenAI.
% Our conceptual implementation of feedforward in our conversational LLM interface prompts AI to generate feedforward components once the user pauses their typing. These feedforward components then feed into the LLM as context to structure and guide the full response. 
% While this implementation strategy is straightforward for a single conversational interface, repeatedly applying the same structure across all applications can become tedious.
% Future work can build a toolkit for developers, perhaps in the form of a web-based UI library, that provides a catalog of feedforward components to integrate into any GenAI web application.


% \cite{dang2022ganslider, zhutian2024sketchgenerate}.
% While balancing this can be a fine line, we believe the right level of detail can be identified for the many possible representations for feedforward in genAI systems.
% level of detail seemed to also be basically how far into the generation it goes, for instnace, the highest level of detail for a feedforward outline may be the full response itself.


% What are other dimensions of feedforward? Maybe when the feedforward is presented
% Future work should rigorously investigate the various applications of GenAI potential uses of feedforward to develop a full design space. This design space could then be used to evaluate in various systems to surface design guidelines for future implementations

% If we do establish this, one potential avenue is to develop a framework, potentially a UI library that contains a catalog of feedforward components with various representations. These can plug into GenAI systems to streamline the process of designing and implementing feedforward for GenAI.


 

% \subsection{Towards a Design Space}

% Through iterative prototyping of feedforward designs in generative AI systems, we came up with three dimensions for designing feedforward in GenAI.

% \begin{enumerate}
%     \item Representation
%     \item Level of Detail
%     \item Manipulation
% \end{enumerate}

% A unique dimension of feedforward in GenAI is the ability and need to manipulate and revise the feedforward directly. Ultimately, the provided feedforward is not only a medium for AI to communicate its intents to the user, but also an opportunity for the user to communicate theirs back to AI.

% \subsection{How to support developing feedforward easily?}

% How do we make it easier to build feedforward in these interfaces?

% At least in web applications, we can build ui packages for developers to plug in into their system.

% We took the design route of making each feedforward representation modular with feedforward components. Developers could plug in certain feedforward components into their application. Communities would also be able to contribute other components to the package. 

% \subsection{Future Work}

% What are the kinds of feedforward information to always show? Length of response?

% What are kinds of feedforward information to show dynamically depending on context and task? Task specific feedforward given the prompt and conversation?

% What are the right representations for these kinds of feedforward?