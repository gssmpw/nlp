%%
%% This is file `sample-authordraft.tex',
%% generated with the docstrip utility.
%%
%% The original source files were:
%%
%% samples.dtx (with options: `authordraft')
%% 
%% IMPORTANT NOTICE:
%% 
%% For the copyright see the source file.
%% 
%% Any modified versions of this file must be renamed
%% with new filenames distinct from sample-authordraft.tex.
%% 
%% For distribution of the original source see the terms
%% for copying and modification in the file samples.dtx.
%% 
%% This generated file may be distributed as long as the
%% original source files, as listed above, are part of the
%% same distribution. (The sources need not necessarily be
%% in the same archive or directory.)
%%
%% The first command in your LaTeX source must be the \documentclass command.

%% Document format
\documentclass[sigconf]{acmart}


% Puts line numbers on the left of each line
% \usepackage{lineno}
% \linenumbers
% \renewcommand\linenumberfont{\normalfont\footnotesize\color{red}}
% % Set line numbers to appear on the left side only
% \setlength{\linenumbersep}{1cm}

\AtBeginDocument{%
  \providecommand\BibTeX{{%
    \normalfont B\kern-0.5em{\scshape i\kern-0.25em b}\kern-0.8em\TeX}}}


% Rights management information.
\copyrightyear{2025}
\acmYear{2025}
\setcopyright{acmcopyright}
\acmConference[CHI '25 Workshop on Tools for Thought]{CHI Conference on Human Factors in Computing Systems}{April 26-May 1, 2025}{Yokohama, Japan}
\acmBooktitle{CHI Conference on Human Factors in Computing Systems (CHI '25), April 26-May 1, 2025, Yokohama, Japan}
% \acmDOI{0000000.0000000}
\acmISBN{978-1-4503-XXXX-X/18/06}



% Submission ID.
% \acmSubmissionID{123-A56-BU3}


% Switches to the "author year" style for citations and references.
% \citestyle{acmauthoryear}

\usepackage{lipsum}
\usepackage{subcaption}
\usepackage{macros}
\usepackage{float}
\usepackage{wrapfig}

\newcommand{\changed}[1]{\textcolor{black}{#1}}

\newcommand{\etal}{et al.}
\newcommand{\subsubsubsection}[1]{\paragraph{\textbf{#1}}}


%%
%% end of the preamble, start of the body of the document source.
\begin{document}


% \title[Feedforward Prompting]{Exploring how to show users what's ahead of the LLM response with Feedforward Prompting}
% Exploring how to show users what to expect from LLM outputs
% \title[Feedforward Prompting]{Feedforward Prompting: Can the interface help users know what to expect from LLM outputs?}
% \title[Feedforward in GenAI]{Feedforward in GenAI Interaction: Opportunities and Challenges}
% \title[Feedforward in GenAI]{Feedforward in GenAI Interaction: Exploring Opportunities for a Design Space}
% \title[Feedforward in GenAI]{The Need for Feedforward in GenAI Interaction: Exploring Opportunities for a Design Space}
% \title[Feedforward in Generative AI]{Feedforward as the Medium for Human-AI Communication}
% \title[Feedforward in Generative AI]{Feedforward as a Shared Abstraction for Human-AI Intent Communication}
% \title[Feedforward for Generative AI]{The Feedforward Challenge of Generative AI}
% \title[Feedforward for Generative AI]{Feedforward for Generative AI: Exploring Opportunities for a Design Space}
% \title[Feedforward for Generative AI]{Feedforward in Generative AI: Toward a Defined Design Space}
% \title[Feedforward for Generative AI]{Feedforward in Generative AI: Opportunities for a Design Space}
\title{Feedforward in Generative AI: Opportunities for a Design Space}



%%
\author{Bryan Min}
\email{bdmin@ucsd.edu}
\affiliation{%
  \institution{University of California San Diego}
  \streetaddress{9500 Gilman Dr}
  \city{La Jolla}
  \state{California}
  \country{USA}
  \postcode{92093}
}

\author{Haijun Xia}
\email{haijunxia@ucsd.edu}
\affiliation{%
  \institution{University of California San Diego}
  \streetaddress{9500 Gilman Dr}
  \city{La Jolla}
  \state{California}
  \country{USA}
  \postcode{92093}
}

%%
\renewcommand{\shortauthors}{Min et al.}

%%
% Abstract.
\begin{abstract}
% GenAI is very capable and is being integrated into many diverse systems
% However, a fundamental challenge in GenAI interaction is that it's difficult to anticipate what AI will actually generate, respond with, and do.
% This is because of feedback leads to ...
% Therefore, this paper aims to advance the perspective that feedforward is important, not just feedback.
% This perspective to advance is to reach a design space, figure out ways to make feedforward effective and se how it can be designed and stuff.
% In order to advnace this investigation, we explored the design of four prototype systems that implement feedforward in GenAI---X, X, X, and X.
% This paper aims to spark discussion on how we can advance this perspective.
% We invite researchers....
% 

% Although research has developed approaches for providing effective feedback and facilitating a high quality experimenting experience to help users gain more experience with GenAI systems, these approaches focus on providing users with feedback, which requires users to see what GenAI produces from their prompt after submitting their prompt.
% While feedback helps users develop an understanding of the GenAI system through exploration and experience, users cannot always anticipate the outcome of their prompts and must rely on iterative trial-and-error. 

% Generative AI (GenAI) models are more than capable than ever at augmenting productivity and cognition in a diverse range of contexts. However, a fundamental challenge is that it is difficult to anticipate exactly what AI will generate.
% As a result, users must repeatedly exchange messages with AI to concretely clarify their intents.
% This process can incur significant cognitive load and time investment, especially for heavy tasks such as generating software interfaces on the fly or automating web tasks with agents.
% If we want users to easily leverage the potential of GenAI systems across diverse contexts seamlessly, they must be able to anticipate AI outputs without learning from feedback, and instead look ahead at what AI will generate.
% This paper aims to advance the perspective that GenAI must not only provide informative feedback, but also informative feedforward in GenAI.
% In other words, we must design GenAI systems that will tell users what the AI will generate before the user submits their prompt.
% To spark the discussion of feedforward in GenAI, we designed diverse instantiations of feedforward in four different GenAI applications---conversational UIs, document editors, malleable interfaces, and automation agents. We discuss how these design explorations can lead to a rigorous investigation of a design space for feedforward in GenAI as well as design guidelines that can be applied to all forms of GenAI applications.

% Instead, they should be able to look ahead and understand what the AI will generate before submitting their prompts.

% Generative AI (GenAI) models have become more capable than ever at augmenting productivity and cognition across diverse contexts.
% However, a fundamental challenge remains for users in that it is difficult to anticipate what AI will generate.
% As a result, users must repeatedly exchange messages with AI to clarify their intent concretely, incurring significant cognitive load and time investment.
% In order for users to seamlessly leverage the full potential of GenAI systems across various contexts, they must be able to anticipate AI outputs without relying solely on feedback.

Generative AI (GenAI) models have become more capable than ever at augmenting productivity and cognition across diverse contexts. However, a fundamental challenge remains as users struggle to anticipate what AI will generate. As a result, they must engage in excessive turn-taking with the AI's feedback to clarify their intent, leading to significant cognitive load and time investment. Our goal is to advance the perspective that in order for users to seamlessly leverage the full potential of GenAI systems across various contexts, we must design GenAI systems that not only provide informative feedback but also informative feedforward—designs that tell users what AI will generate before the user submits their prompt. To spark discussion on feedforward in GenAI, we designed diverse instantiations of feedforward across four GenAI applications: conversational UIs, document editors, malleable interfaces, and agent automations, and discussed how these designs can contribute to a more rigorous investigation of a design space and a set of guidelines for feedforward in all GenAI systems.

% We discuss how these design implementations can contribute to a rigorous investigation of the design space for feedforward in GenAI and design guidelines applicable across diverse GenAI applications.



% these AI models are challenging to properly prompt for most users.
% Although systems have focused on mitigating these challenges through a variety of solutions/approaches of providing feedback, users must learn this feedback like newcomers for each new generative AI system.
% In addition to feedback, we argue that we must also provide informative feedforward to users. Feedforward will enable newcomers of any generative AI system to be able to anticipate what it will generate, thus helping them prompt more efficiently and reduce frictions and intention-gaps.
% This workshop paper aims to spark discussion on how we can design and implement feedforward for generative AI, and we showcase a few methods for understanding this design space.

\end{abstract}

%%
% ACM Computing Classification System.
% http://dl.acm.org/ccs.cfm
\begin{CCSXML}
<ccs2012>
   <concept>
       <concept_id>10003120.10003121.10003124</concept_id>
       <concept_desc>Human-centered computing~Interaction paradigms</concept_desc>
       <concept_significance>500</concept_significance>
       </concept>
   <concept>
       <concept_id>10003120.10003121.10003126</concept_id>
       <concept_desc>Human-centered computing~HCI theory, concepts and models</concept_desc>
       <concept_significance>500</concept_significance>
       </concept>
 </ccs2012>
\end{CCSXML}

\ccsdesc[500]{Human-centered computing~Interaction paradigms}
\ccsdesc[500]{Human-centered computing~HCI theory, concepts and models}

%%
% Keywords.
\keywords{Feedforward, Human-AI Interaction, Generative AI}

%%
% \begin{teaserfigure}
%     \includegraphics[trim=0cm 0cm 0cm 0cm, clip=true, width=\textwidth]{figures/teaser.png}
%     \caption{(A) We perform a content analysis of overview-detail interfaces (blue: overview; orange: detail view) to understand its variations in the wild, finding three key dimensions: \textit{content}, \textit{composition}, and \textit{layout}. (B) Based on these dimensions, we provide interactions for end-users to customize the overview-detail interface. Among them, we contribute a novel technique, \textit{Fluid Attributes}, enabling users to (C) surface attributes to the overview that only exist in the detail view and (D) operate on attributes they surface.}
%     \label{fig:teaser}
%     \Description{The figure depicts four images that encompass the primary contributions of this paper. The first image depicts three variations of overview-detail interfaces from our content analysis - highlighting key dimensions of content, compositions, and layout. The second image is taken from our design probe, showing the various interactions for end-users to customize the overview-detail interface to match their own needs and preferences. Among these interactions, we contribute a novel technique, Fluid Attributes, shown in the third and fourth images, enabling users to directly interact with and manipulate content attributes between the overview and the detail view. The third image shows, a user selecting attributes to be surfaced in the overview while the fourth image shows a user sorting the overview via a selected attribute.}
% \end{teaserfigure}


%%
\maketitle{}

%%
\section{Introduction}

Video generation has garnered significant attention owing to its transformative potential across a wide range of applications, such media content creation~\citep{polyak2024movie}, advertising~\citep{zhang2024virbo,bacher2021advert}, video games~\citep{yang2024playable,valevski2024diffusion, oasis2024}, and world model simulators~\citep{ha2018world, videoworldsimulators2024, agarwal2025cosmos}. Benefiting from advanced generative algorithms~\citep{goodfellow2014generative, ho2020denoising, liu2023flow, lipman2023flow}, scalable model architectures~\citep{vaswani2017attention, peebles2023scalable}, vast amounts of internet-sourced data~\citep{chen2024panda, nan2024openvid, ju2024miradata}, and ongoing expansion of computing capabilities~\citep{nvidia2022h100, nvidia2023dgxgh200, nvidia2024h200nvl}, remarkable advancements have been achieved in the field of video generation~\citep{ho2022video, ho2022imagen, singer2023makeavideo, blattmann2023align, videoworldsimulators2024, kuaishou2024klingai, yang2024cogvideox, jin2024pyramidal, polyak2024movie, kong2024hunyuanvideo, ji2024prompt}.


In this work, we present \textbf{\ours}, a family of rectified flow~\citep{lipman2023flow, liu2023flow} transformer models designed for joint image and video generation, establishing a pathway toward industry-grade performance. This report centers on four key components: data curation, model architecture design, flow formulation, and training infrastructure optimization—each rigorously refined to meet the demands of high-quality, large-scale video generation.


\begin{figure}[ht]
    \centering
    \begin{subfigure}[b]{0.82\linewidth}
        \centering
        \includegraphics[width=\linewidth]{figures/t2i_1024.pdf}
        \caption{Text-to-Image Samples}\label{fig:main-demo-t2i}
    \end{subfigure}
    \vfill
    \begin{subfigure}[b]{0.82\linewidth}
        \centering
        \includegraphics[width=\linewidth]{figures/t2v_samples.pdf}
        \caption{Text-to-Video Samples}\label{fig:main-demo-t2v}
    \end{subfigure}
\caption{\textbf{Generated samples from \ours.} Key components are highlighted in \textcolor{red}{\textbf{RED}}.}\label{fig:main-demo}
\end{figure}


First, we present a comprehensive data processing pipeline designed to construct large-scale, high-quality image and video-text datasets. The pipeline integrates multiple advanced techniques, including video and image filtering based on aesthetic scores, OCR-driven content analysis, and subjective evaluations, to ensure exceptional visual and contextual quality. Furthermore, we employ multimodal large language models~(MLLMs)~\citep{yuan2025tarsier2} to generate dense and contextually aligned captions, which are subsequently refined using an additional large language model~(LLM)~\citep{yang2024qwen2} to enhance their accuracy, fluency, and descriptive richness. As a result, we have curated a robust training dataset comprising approximately 36M video-text pairs and 160M image-text pairs, which are proven sufficient for training industry-level generative models.

Secondly, we take a pioneering step by applying rectified flow formulation~\citep{lipman2023flow} for joint image and video generation, implemented through the \ours model family, which comprises Transformer architectures with 2B and 8B parameters. At its core, the \ours framework employs a 3D joint image-video variational autoencoder (VAE) to compress image and video inputs into a shared latent space, facilitating unified representation. This shared latent space is coupled with a full-attention~\citep{vaswani2017attention} mechanism, enabling seamless joint training of image and video. This architecture delivers high-quality, coherent outputs across both images and videos, establishing a unified framework for visual generation tasks.


Furthermore, to support the training of \ours at scale, we have developed a robust infrastructure tailored for large-scale model training. Our approach incorporates advanced parallelism strategies~\citep{jacobs2023deepspeed, pytorch_fsdp} to manage memory efficiently during long-context training. Additionally, we employ ByteCheckpoint~\citep{wan2024bytecheckpoint} for high-performance checkpointing and integrate fault-tolerant mechanisms from MegaScale~\citep{jiang2024megascale} to ensure stability and scalability across large GPU clusters. These optimizations enable \ours to handle the computational and data challenges of generative modeling with exceptional efficiency and reliability.


We evaluate \ours on both text-to-image and text-to-video benchmarks to highlight its competitive advantages. For text-to-image generation, \ours-T2I demonstrates strong performance across multiple benchmarks, including T2I-CompBench~\citep{huang2023t2i-compbench}, GenEval~\citep{ghosh2024geneval}, and DPG-Bench~\citep{hu2024ella_dbgbench}, excelling in both visual quality and text-image alignment. In text-to-video benchmarks, \ours-T2V achieves state-of-the-art performance on the UCF-101~\citep{ucf101} zero-shot generation task. Additionally, \ours-T2V attains an impressive score of \textbf{84.85} on VBench~\citep{huang2024vbench}, securing the top position on the leaderboard (as of 2025-01-25) and surpassing several leading commercial text-to-video models. Qualitative results, illustrated in \Cref{fig:main-demo}, further demonstrate the superior quality of the generated media samples. These findings underscore \ours's effectiveness in multi-modal generation and its potential as a high-performing solution for both research and commercial applications.
% \section{Critique of GenAI Interfaces Through the Lens of Feedforward}
% \section{Critique Through the Lens of Feedforward}
\label{section:critique}

In order to understand a wider range of needs and use cases for feedforward in GenAI, we must reflect and critique past and current GenAI systems and explore how we may improve on them.
We use a critical lens \cite{MBL2021generativetheoriesinteraction} with our perspective that GenAI systems need sufficient feedforward to assess and critique three GenAI systems the first author contributed to designing and implementing: Sensecape \cite{sensecape}, Luminate \cite{luminate}, and Malleable Overview-Detail Interfaces \cite{malleableODI}.

% In order to understand a wider range of needs and use cases for feedforward in GenAI, we must reflect and critique past and current systems. 
% This critical lens is to help us concretely understand

\paragraph{Malleable Overview-Detail Interfaces}


\paragraph{Luminate}
% 


\paragraph{Sensecape}
% Describe what the system is and what it's for. Then describe key features.
% Then, go into several of those features and say, however, the user may not know X or Y.
% Leads them to wonder, what will this be?

\section{Feedforward in Generative AI}
\label{section:feedforward}


This section aims to explore possible ways to provide feedforward in generative AI systems. We view that effective feedforward in generative AI interfaces should enable users to:
\begin{itemize}
    \item Anticipate what the AI's response or action may be.
    \item Disambiguate their prompt without excessive exchanges in conversation or interaction.
    \item Directly engage with the feedforward content.
    \item  Reflect on their prompting practices with less friction.
\end{itemize}

To explore various ways to design feedforward in GenAI systems, we first explore feedforward in an LLM-powered conversational interface. We then describe scenarios for three other designs of feedforward in GenAI applications.


\begin{figure*}
    \centering
    \includegraphics[width=1\linewidth]{figures/applications.pdf}
    \caption{We explored three different applications for designing feedforward in GenAI systems: (A) document editors, (B) malleable interfaces, and (C) agent automation systems.}
    \label{fig:applications}
    \vspace{6pt}
\end{figure*}


\subsection{Feedforward in Conversational LLM UIs}

\subsubsection{Feedforward Components}

Research has shown that users benefit from multiple representations when interacting with LLMs \cite{graphologue, sensecape}.
Therefore, we developed feedforward components that each aim to provide a specific representation that would help users anticipate the LLM's response. 
As the user types a prompt, the conversational UI generates feedforward content inside two feedforward components: 1) a key topic phrase alongside a high-level outline of the anticipated response (Fig. \ref{fig:feedforward-components}.1a) and 2) a minimap view with paragraph blocks (Fig. \ref{fig:feedforward-components}.1b). These representations help users anticipate the topics and subtopics the LLM will generate, as well as whether its response might be too short or too long.

%  to help users anticipate what kinds of topics and subtopics the LLM will share 
% of the anticipated length of the response to help users know whether the LLM will generate a short snippet of text or a detailed explanation about a topic


\subsubsection{Manipulating Feedforward Content}

We enable users to manipulate feedforward content in two ways.
First, when users edit their prompt, the feedforward content updates to reflect the changes.
This allows users to identify misalignments between their intent and the feedforward content, encouraging them to add details to the prompt before submitting it.
For instance, they can clarify an ambiguous term (Fig. \ref{fig:feedforward-components}.2) or request a shorter response (Fig. \ref{fig:feedforward-components}.3).
These instant updates to the feedforward components may also spark reflection, such as realizing that it may be useful to understand the advantages and limitations of Wizard of Oz Prototyping rather than just its definition.
% Nevertheless, feedforward components give users the anticipatory insights and granular control over what information they will receive from the LLM.


% In addition to revising their prompt to adjust the feedforward content, 
Second, users can directly edit content inside the feedforward components. For example, rather than prompting the LLM to remove subtopics from the outline, users can directly delete the subtopic title in the outline component or delete the paragraph in the minimap (Fig. \ref{fig:feedforward-manipulation}A). Users can also drag the edge of the minimap component to expand the feedforward content's level of details (Fig. \ref{fig:feedforward-manipulation}B). This can allow users control ``how far ahead'' they can look into the LLM's response. Additionally, users can highlight parts of the outline, revealing a tooltip that generates a list of possible actions and outputs based on that selection, such as adding examples about a subtopic or organizing the content in a bullet-point list. Users can either select one of these actions or request their own, which will update the feedforward outline (Fig. \ref{fig:feedforward-manipulation}C).


\subsection{Application Scenarios}

Our goal is to explore opportunities for designing feedforward in all GenAI interfaces---beyond conversational UIs.
% This can include video creation platforms, programming tools, and more.
In this paper, we explore three applications to showcase various feedforward designs across diverse contexts: document editors, malleable interfaces, and agent automations.


\subsubsection{Document Editors}

Developers and researchers have begun to integrate AI into document editors and writing canvases to support various aspects of writing tasks, such as fixing grammar \cite{laban2024inksync}, presenting continuously synchronized outlines \cite{dang2022beyondtextgen}, and providing widgets to adjust the tone of writing along various dimensions \cite{masson2024textoshop, chatgptcanvas2025}.
However, it can be difficult for users to anticipate what the LLM might produce based on their writing and revisions. For instance, moving a slider up to change the tone from ``informal'' to ``formal'' does not indicate exactly how ``formal'' they are making their writing. It is only after submitting the change and reviewing the final result that users realize whether they have achieved the desired outcome. To reduce the amount trial-and-error to achieve the desired result, document editors can provide feedforward that presents example phrases as the user moves the slider up and down, helping the user quickly develop a clear understanding of the level of formality they are adjusting to (Fig. \ref{fig:applications}A). This feedforward mechanism can be expanded to let users control how much detail is shown in each example phrase, allowing them to reveal more or fewer words as needed. This customization helps align the feedforward component with user preferences, reducing cognitive load.
% Additionally, users may have the option to increase the level of detail in the feedforward content, revealing the full sentences from which the phrases are derived.


% it's hard to anticipate what these changes might be, and where they might edit your writing. results in users having to repeatedly switch between writing and revising with AI.
% Feedforward in AI writing tools can support this process by helping users anticipate what and where AI will modify their writing.
% For instance, instead of making a full complete change/rewritten snippet, can instead provide feedforward through example phrases, if user slides to write more formal, then it will show example phrases that represent that level of formality. (Fig. X)
% Other instance is where you edit. it's going to change where the edit is made before you accept the change...

% If you have a slider for tone changes, or a spectrum of tone changes, instead of saying more X or less X, feedforward can provide concrete example phrases it will use.

% Also, requesting and explicit changes will prompt AI to provide feedforward of which paragraphs it will look at. Increasing the level of detail will highlight which sentences, which phrases, and even potentially what AI will replace the content with.


\subsubsection{Malleable Interfaces}

GenAI is additionally enabling software interfaces to become increasingly malleable, enabling users to generate custom functional applications on the fly \cite{litt2025workout, onetwoval2025calculator}, personalized interfaces that blends user activity \cite{cao2025jelly}, and user-defined abstractions in overview-detail interfaces \cite{malleableODI}.
In current generated malleable interfaces, the user prompts for a custom application and provides specifications, and after rounds of clarifying, AI generates and compiles the requested application. However, it can be unclear what kinds of software components the AI might produce based on their prompts alone. For example, if the user asks AI to show only linen-textured couches from a shopping website, the user cannot be sure whether AI will add a filtering operation, perform another search query, or create a brand new list linen-textured couches. Feedforward can present a list of system operations the AI will perform before submitting the prompt, allowing them to quickly assess and revise the operations without unnecessary exchanges with the AI (Fig. \ref{fig:applications}B).

% Ask for generating interface. before you submit your prompt, you get a wireframe? You get attributes about what kind of interface it will generate? You get

% Brief context of overview-detail interfaces
% Asking for details about certain items, Asking for an attribute at a high level. Will it surface attributes? will it generate a new one? Will it combine several attributes?

% What kinds of data will the AI focus on looking at? Will AI look at a set of attributes among all of the data provided? If so, which ones?


\subsubsection{Agent Automations}
% Being able to control the "speed" of generating content, which would be controlling the "depth of thought" on how much the model should break down the task.

The AI community has also presented demonstrations of AI agents that can automate diverse tasks on the web \cite{operator2025, dia2025}. However, there lacks clear communication of exactly what the agent is doing on the screen. For instance, if a user asks an AI agent to look for tickets to a basketball game, the agent will break the task into steps and perform them by opening and navigating a webpage. During this phase, the user may be unaware of which buttons or components the agent plans to click to find the best prices, which portions of the webpage it will scan for context, or where it will navigate next. Agent automation systems can integrate feedforward visualizations by providing area selections to show where the AI agent is focusing on for context and opaque cursors to tell the user where the agent will click next (Fig. \ref{fig:applications}C). These feedforward representations help users anticipate the agent’s future actions and provide visual affordances to intervene in potential errors by adjusting the selection area or dragging the cursor.

% Users can anticipate what an AI agent will do before they perform the task. Chain-of-thought and surrounding implementations instantiate feedforward by visualizing stages of actions the agent plans to perform. During each step, users can then stop and adjust stage by stage, providing steerability.

% AI agents should also present feedforward for various actions they aim to perform before submitting the prompt. For instance, users should see depth of thought AI aims to use from that prompt, which then users can then manually adjust.

% During automation tasks, agents currently show an AI-controlled cursor indicating clicks AI aims to perform. We view that this form of communication can be extended to all forms of selection. AI can place shaded rectangles over areas they are currently taking in as context. AI provides feedforward before performing the action. Because of this, the user is able to confirm that is the action the user wants AI to perform, and if it isn't can adjust the selection to steer AI to use the correct selection.



% \subsection{Implementing Feedforward}

% We should also investigate what is the best way to implement feedofrward.
% We implemented such that the feedforward is generated live, then the feedforward feeds into the context for the real content to generate as a skeleton.


% Preliminary Design Space for Feedforward in Generative AI systems
% Representations
% Levels of Detail
% Prompt Revision and Redirection Mechanisms
% 


% , or reflecting on the minimap feedforward component's information  that they must explicitly ask for ``just the definition'' if they want a shorter response.

% The outline will produce feedforward about content, but for the length of the expected response, feedforward will represent as a box to visualize the length in size.

% \subsubsection{Adjusting the Levels of Feedforward Detail}


% As you type your prompt to an LLM, you will see a short outline of text that preview what the LLM will output. This gives you an idea of what you can anticipate given your prompt. As you adjust your prompt, the preview also updates, allowing you to closer align your prompt to your desired outcome.

% Problem chatgpt, ask a question, don't know what it will respond with. Especially can't gauge if your prompt is ambiguous or concrete enough.
% Sending, responds different topic, or generate too much
% Result is you need to submit your prompt again, tuning it, might miss another aspect, or the prompt might not go in your favor again
% gets repetitive
% Instead feedforward in convo LLMs can provide previews of what might happen.


% Feedforward components do not only serve the purpose of providing information before submitting a prompt, but also enables users to interact with the content inside directly. 

% Users can adjust a slider that would adjust the levels of detail the feedforward content will provide. Users can move this slider to control how far ahead they see into what might be generated as they construct their prompt and also look further ahead into what their prompt might lead to.

% As the user slides to view more details, it adjusts all aspects of information of the feedforward content. For instance, text headers and outlines will expand to show subheaders and more details in the outline, while a label ``table'', previewing that the LLM's response will generate a table, will present a low fidelity table. Upon increasing the level of detail more the table will populate with more details as its column and row titles and text inside the cells.

% You can slide to adjust the levels of detail the feedforward will output. for instance, if anticipated to provide a table, the highest level could be the word "table", then describe number of columns and rows + visualize that table in the feedforward space, then populate content in that table little by little, until you go to the end and it ultimately becomes the fully generated response

% You can interact with feedforward content directly. You can select any content in the feedforward content, ask to expand on it, expanding on that level of detail. Selecting on content will also automatically generate recommended prompts for actions to perform, such as if you select the header ``Examples'' a dropdown menu will generate prompts such as ``Remove this header'', ``More details on examples'', and ``Replace with Application Scenarios''.


% Users can ... 
% When incorporating the slider to adjust the level of detail alongside manipulating the feedforward content, users can for instance select a table preview, increase or reduce the level of detail of that table specifically. If the user feels the table does not represent the contained information adequately, they can hold shift while reducing the level of detail to the term ``table''. Upon changing ``table'' to ``two columns'', the feedforward content updates to provide a preview of the two columns arrangement of the information.

% You can interact with any feedforward representation.

% You can also interact with the feedforward content at any level of detail. For instance, you can grow more detailed, and if you don't like where it's going, you can slide back, adjust some content, and see what that would generate.
% Maybe when you choose to go back and then change, you will see a feedforward of the feedforward, showing what that adjustment you made might look like.

% \begin{figure*}
%     \centering
%     \includegraphics[width=1\linewidth]{figures/feedforward-workflow.png}
%     \caption{Enter Caption}
%     \label{fig:enter-label}
% \end{figure*}

\section{Future Work}
\label{section:discussion}

%  implement feedforward in GenAI systems across all contexts and applications, making it
% Our goal is to make feedforward a fundamental design component in all GenAI systems. While this paper demonstrates several feedforward designs for GenAI, we must further consider how we can effectively design and implement feedforward for GenAI systems in all applications.

Our goal is to establish feedforward as a fundamental design component in all GenAI systems. While this paper presents several feedforward designs, we envision a more comprehensive design space to represent and guide feedforward design across all GenAI applications. Based on our four prototypes, we identify three potential design dimensions for GenAI feedforward: representation, level of detail, and manipulability.

First, our examples implemented feedforward representations in the form of outlines, minimaps, lists of operations, example phrases, and multiple cursors. We may identify categories of representations that help distinguish which feedforward representations are more useful for different use cases. For instance, while a list of operations might inform users about the type of UI a GenAI system will generate, a wireframe might better communicate the structure and layout of the UI.

Second, we explored different ways to present varying levels of detail in feedforward. For example, the conversational UI displays an outline summarizing key topics, while the minimap omits textual details and instead presents blocks of paragraphs. This dimension aligns with previous research on feedforward \cite{bau2008octopocus}. A more rigorous investigation could explore optimal levels of detail for different types of feedforward representations in GenAI, as well as allowing users to define the level and type of detail themselves.

Lastly, we explored how users can manipulate feedforward content, either by revising their prompts or by directly resizing, repositioning, or selecting elements. We aim to investigate additional interaction techniques that enhance user control and engagement with feedforward designs.

Future work should expand on this preliminary design space by gathering, analyzing, and critiquing a broader range of GenAI systems to identify variations of feedforward across diverse contexts.

% Second, we must streamline the development of feedforward for GenAI.
% In our conceptual implementation of feedforward within a conversational LLM interface, the AI generates feedforward components when the user pauses typing.
% These components then serve as contextual input for the LLM upon submitting the prompt, helping it structure and guide the full response.
% While this approach is straightforward for a single conversational interface, this may not be the most cost-efficient approach for implementing feedforward. Perhaps other strategies may be to support feedforward algorithmically without the use of LLMs \cite{zhutian2024sketchgenerate}, but this might constrain the generative capabilities of the LLM.
% Furthermore, repeatedly implementing the same feedforward-GenAI pipeline across multiple applications, regardless of its implementation details, can become tedious.
% Future work could also provide a developer toolkit---perhaps a web-based UI library---that provides a catalog of feedforward components for developers to seamlessly integrate them into any GenAI web application.





% We want to work towards a design space for feedforward in genAI.
% From our prototype explorations in four applications: conversation UIs, document editors, malleable interfaces, and agent automations, we surface preliminary dimensions: 1) representation, 2) level of detail, 3) manipulability
% providing outline, minimap, list of operations, cut-outs of examples, these representations are forms of abstractions that are useful for feedforward in genai.
% level of detail is another, balancing the level of detail in feedforward is crucial in making sure users are able to understand what's ahead without being overwhelmed. Researchers have identified instances where there was cognitive burden in code completions and image generation. While balancing this can be a fine line, we believe the right level of detail can be identified for the many possible representations for feedforward in genAI systems.
% level of detail seemed to also be basically how far into the generation it goes, for instnace, the highest level of detail for a feedforward outline may be the full response itself.
% We explored various ways users can revise their prompt or steer AI upon finding out misalignment through feedforward. We explored that while users can revise their prompt to disambiguate their intents, they can also directly manipualte the feedforward content or specify their prompt by making AI-generated adjustments on the feedforward content.......

% we can take further steps to integrate feedforward into more GenAI systems.

% First, we can streamline the design process of feedforward for GenAI, we must solidify a comprehensive design space. From our prototype explorations in four applications: conversation UIs, document editors, malleable interfaces, and agent automations, we surfaced three preliminary dimensions: 1) representation, 2) level of detail, and 3) manipulability.
% We designed feedforward to be represented in an outline, minimap, list of operations, example phrases, and multiple cursors. We anticipate a rigorous investigation of feedforward will surface more representations suitable for each of their unique contexts.
% We also explored ways to present various levels of detail of feedforward. The conversational UI presents an outline detailing the key topics, while the minimap presents the topics organized into paragraphs.
% We aim to further investigate the ``right'' levels of detail across various feedforward representations to unsure users can anticipate AI's response without experiencing too much cognitive burden. 
% We also explored various ways users can revise their prompt or steer AI upon identifying misalignments between the user and AI. We explored how users can disambiguate their intents by either revising the text of their prompt and also directly manipulating the feedforward content via resizing, repositioning, and selecting.
% Future work should expand on this preliminary design space by gathering, analyzing, and critiquing a comprehensive collection of GenAI systems and identifying potential variations of feedforward across different contexts.
% We need to work on the design space
% We need to make it implementable

% Second, we can streamline the development of feedforward for GenAI.
% Our conceptual implementation of feedforward in our conversational LLM interface prompts AI to generate feedforward components once the user pauses their typing. These feedforward components then feed into the LLM as context to structure and guide the full response. 
% While this implementation strategy is straightforward for a single conversational interface, repeatedly applying the same structure across all applications can become tedious.
% Future work can build a toolkit for developers, perhaps in the form of a web-based UI library, that provides a catalog of feedforward components to integrate into any GenAI web application.


% \cite{dang2022ganslider, zhutian2024sketchgenerate}.
% While balancing this can be a fine line, we believe the right level of detail can be identified for the many possible representations for feedforward in genAI systems.
% level of detail seemed to also be basically how far into the generation it goes, for instnace, the highest level of detail for a feedforward outline may be the full response itself.


% What are other dimensions of feedforward? Maybe when the feedforward is presented
% Future work should rigorously investigate the various applications of GenAI potential uses of feedforward to develop a full design space. This design space could then be used to evaluate in various systems to surface design guidelines for future implementations

% If we do establish this, one potential avenue is to develop a framework, potentially a UI library that contains a catalog of feedforward components with various representations. These can plug into GenAI systems to streamline the process of designing and implementing feedforward for GenAI.


 

% \subsection{Towards a Design Space}

% Through iterative prototyping of feedforward designs in generative AI systems, we came up with three dimensions for designing feedforward in GenAI.

% \begin{enumerate}
%     \item Representation
%     \item Level of Detail
%     \item Manipulation
% \end{enumerate}

% A unique dimension of feedforward in GenAI is the ability and need to manipulate and revise the feedforward directly. Ultimately, the provided feedforward is not only a medium for AI to communicate its intents to the user, but also an opportunity for the user to communicate theirs back to AI.

% \subsection{How to support developing feedforward easily?}

% How do we make it easier to build feedforward in these interfaces?

% At least in web applications, we can build ui packages for developers to plug in into their system.

% We took the design route of making each feedforward representation modular with feedforward components. Developers could plug in certain feedforward components into their application. Communities would also be able to contribute other components to the package. 

% \subsection{Future Work}

% What are the kinds of feedforward information to always show? Length of response?

% What are kinds of feedforward information to show dynamically depending on context and task? Task specific feedforward given the prompt and conversation?

% What are the right representations for these kinds of feedforward?
% \section{Conclusion}
\label{section:conclusion}



%%
% Acknowledgments.
% \begin{acks}
% Thanks to the anonymous reviewers for their helpful comments.
% \end{acks}

%%
\bibliographystyle{ACM-Reference-Format}
\bibliography{main}

\appendix
% \clearpage

% \section{Appendix}
% \label{section:appendix}


\end{document}
