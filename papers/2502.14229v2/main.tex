%%
%% This is file `sample-authordraft.tex',
%% generated with the docstrip utility.
%%
%% The original source files were:
%%
%% samples.dtx (with options: `authordraft')
%% 
%% IMPORTANT NOTICE:
%% 
%% For the copyright see the source file.
%% 
%% Any modified versions of this file must be renamed
%% with new filenames distinct from sample-authordraft.tex.
%% 
%% For distribution of the original source see the terms
%% for copying and modification in the file samples.dtx.
%% 
%% This generated file may be distributed as long as the
%% original source files, as listed above, are part of the
%% same distribution. (The sources need not necessarily be
%% in the same archive or directory.)
%%
%% The first command in your LaTeX source must be the \documentclass command.

%% Document format
\documentclass[sigconf]{acmart}


% Puts line numbers on the left of each line
% \usepackage{lineno}
% \linenumbers
% \renewcommand\linenumberfont{\normalfont\footnotesize\color{red}}
% % Set line numbers to appear on the left side only
% \setlength{\linenumbersep}{1cm}

\AtBeginDocument{%
  \providecommand\BibTeX{{%
    \normalfont B\kern-0.5em{\scshape i\kern-0.25em b}\kern-0.8em\TeX}}}


% Rights management information.
\copyrightyear{2025}
\acmYear{2025}
\setcopyright{acmcopyright}
\acmConference[CHI '25 Workshop on Tools for Thought]{CHI Conference on Human Factors in Computing Systems}{April 26-May 1, 2025}{Yokohama, Japan}
\acmBooktitle{CHI Conference on Human Factors in Computing Systems (CHI '25), April 26-May 1, 2025, Yokohama, Japan}
% \acmDOI{0000000.0000000}
\acmISBN{978-1-4503-XXXX-X/18/06}



% Submission ID.
% \acmSubmissionID{123-A56-BU3}


% Switches to the "author year" style for citations and references.
% \citestyle{acmauthoryear}

\usepackage{lipsum}
\usepackage{subcaption}
\usepackage{macros}
\usepackage{float}
\usepackage{wrapfig}

\newcommand{\changed}[1]{\textcolor{black}{#1}}

\newcommand{\etal}{et al.}
\newcommand{\subsubsubsection}[1]{\paragraph{\textbf{#1}}}


%%
%% end of the preamble, start of the body of the document source.
\begin{document}


% \title[Feedforward Prompting]{Exploring how to show users what's ahead of the LLM response with Feedforward Prompting}
% Exploring how to show users what to expect from LLM outputs
% \title[Feedforward Prompting]{Feedforward Prompting: Can the interface help users know what to expect from LLM outputs?}
% \title[Feedforward in GenAI]{Feedforward in GenAI Interaction: Opportunities and Challenges}
% \title[Feedforward in GenAI]{Feedforward in GenAI Interaction: Exploring Opportunities for a Design Space}
% \title[Feedforward in GenAI]{The Need for Feedforward in GenAI Interaction: Exploring Opportunities for a Design Space}
% \title[Feedforward in Generative AI]{Feedforward as the Medium for Human-AI Communication}
% \title[Feedforward in Generative AI]{Feedforward as a Shared Abstraction for Human-AI Intent Communication}
% \title[Feedforward for Generative AI]{The Feedforward Challenge of Generative AI}
% \title[Feedforward for Generative AI]{Feedforward for Generative AI: Exploring Opportunities for a Design Space}
% \title[Feedforward for Generative AI]{Feedforward in Generative AI: Toward a Defined Design Space}
% \title[Feedforward for Generative AI]{Feedforward in Generative AI: Opportunities for a Design Space}
\title{Feedforward in Generative AI: Opportunities for a Design Space}



%%
\author{Bryan Min}
\email{bdmin@ucsd.edu}
\affiliation{%
  \institution{University of California San Diego}
  \streetaddress{9500 Gilman Dr}
  \city{La Jolla}
  \state{California}
  \country{USA}
  \postcode{92093}
}

\author{Haijun Xia}
\email{haijunxia@ucsd.edu}
\affiliation{%
  \institution{University of California San Diego}
  \streetaddress{9500 Gilman Dr}
  \city{La Jolla}
  \state{California}
  \country{USA}
  \postcode{92093}
}

%%
\renewcommand{\shortauthors}{Min et al.}

%%
% Abstract.
\begin{abstract}
% GenAI is very capable and is being integrated into many diverse systems
% However, a fundamental challenge in GenAI interaction is that it's difficult to anticipate what AI will actually generate, respond with, and do.
% This is because of feedback leads to ...
% Therefore, this paper aims to advance the perspective that feedforward is important, not just feedback.
% This perspective to advance is to reach a design space, figure out ways to make feedforward effective and se how it can be designed and stuff.
% In order to advnace this investigation, we explored the design of four prototype systems that implement feedforward in GenAI---X, X, X, and X.
% This paper aims to spark discussion on how we can advance this perspective.
% We invite researchers....
% 

% Although research has developed approaches for providing effective feedback and facilitating a high quality experimenting experience to help users gain more experience with GenAI systems, these approaches focus on providing users with feedback, which requires users to see what GenAI produces from their prompt after submitting their prompt.
% While feedback helps users develop an understanding of the GenAI system through exploration and experience, users cannot always anticipate the outcome of their prompts and must rely on iterative trial-and-error. 

% Generative AI (GenAI) models are more than capable than ever at augmenting productivity and cognition in a diverse range of contexts. However, a fundamental challenge is that it is difficult to anticipate exactly what AI will generate.
% As a result, users must repeatedly exchange messages with AI to concretely clarify their intents.
% This process can incur significant cognitive load and time investment, especially for heavy tasks such as generating software interfaces on the fly or automating web tasks with agents.
% If we want users to easily leverage the potential of GenAI systems across diverse contexts seamlessly, they must be able to anticipate AI outputs without learning from feedback, and instead look ahead at what AI will generate.
% This paper aims to advance the perspective that GenAI must not only provide informative feedback, but also informative feedforward in GenAI.
% In other words, we must design GenAI systems that will tell users what the AI will generate before the user submits their prompt.
% To spark the discussion of feedforward in GenAI, we designed diverse instantiations of feedforward in four different GenAI applications---conversational UIs, document editors, malleable interfaces, and automation agents. We discuss how these design explorations can lead to a rigorous investigation of a design space for feedforward in GenAI as well as design guidelines that can be applied to all forms of GenAI applications.

% Instead, they should be able to look ahead and understand what the AI will generate before submitting their prompts.

% Generative AI (GenAI) models have become more capable than ever at augmenting productivity and cognition across diverse contexts.
% However, a fundamental challenge remains for users in that it is difficult to anticipate what AI will generate.
% As a result, users must repeatedly exchange messages with AI to clarify their intent concretely, incurring significant cognitive load and time investment.
% In order for users to seamlessly leverage the full potential of GenAI systems across various contexts, they must be able to anticipate AI outputs without relying solely on feedback.

Generative AI (GenAI) models have become more capable than ever at augmenting productivity and cognition across diverse contexts. However, a fundamental challenge remains as users struggle to anticipate what AI will generate. As a result, they must engage in excessive turn-taking with the AI's feedback to clarify their intent, leading to significant cognitive load and time investment. Our goal is to advance the perspective that in order for users to seamlessly leverage the full potential of GenAI systems across various contexts, we must design GenAI systems that not only provide informative feedback but also informative feedforward—designs that tell users what AI will generate before the user submits their prompt. To spark discussion on feedforward in GenAI, we designed diverse instantiations of feedforward across four GenAI applications: conversational UIs, document editors, malleable interfaces, and agent automations, and discussed how these designs can contribute to a more rigorous investigation of a design space and a set of guidelines for feedforward in all GenAI systems.

% We discuss how these design implementations can contribute to a rigorous investigation of the design space for feedforward in GenAI and design guidelines applicable across diverse GenAI applications.



% these AI models are challenging to properly prompt for most users.
% Although systems have focused on mitigating these challenges through a variety of solutions/approaches of providing feedback, users must learn this feedback like newcomers for each new generative AI system.
% In addition to feedback, we argue that we must also provide informative feedforward to users. Feedforward will enable newcomers of any generative AI system to be able to anticipate what it will generate, thus helping them prompt more efficiently and reduce frictions and intention-gaps.
% This workshop paper aims to spark discussion on how we can design and implement feedforward for generative AI, and we showcase a few methods for understanding this design space.

\end{abstract}

%%
% ACM Computing Classification System.
% http://dl.acm.org/ccs.cfm
\begin{CCSXML}
<ccs2012>
   <concept>
       <concept_id>10003120.10003121.10003124</concept_id>
       <concept_desc>Human-centered computing~Interaction paradigms</concept_desc>
       <concept_significance>500</concept_significance>
       </concept>
   <concept>
       <concept_id>10003120.10003121.10003126</concept_id>
       <concept_desc>Human-centered computing~HCI theory, concepts and models</concept_desc>
       <concept_significance>500</concept_significance>
       </concept>
 </ccs2012>
\end{CCSXML}

\ccsdesc[500]{Human-centered computing~Interaction paradigms}
\ccsdesc[500]{Human-centered computing~HCI theory, concepts and models}

%%
% Keywords.
\keywords{Feedforward, Human-AI Interaction, Generative AI}

%%
% \begin{teaserfigure}
%     \includegraphics[trim=0cm 0cm 0cm 0cm, clip=true, width=\textwidth]{figures/teaser.png}
%     \caption{(A) We perform a content analysis of overview-detail interfaces (blue: overview; orange: detail view) to understand its variations in the wild, finding three key dimensions: \textit{content}, \textit{composition}, and \textit{layout}. (B) Based on these dimensions, we provide interactions for end-users to customize the overview-detail interface. Among them, we contribute a novel technique, \textit{Fluid Attributes}, enabling users to (C) surface attributes to the overview that only exist in the detail view and (D) operate on attributes they surface.}
%     \label{fig:teaser}
%     \Description{The figure depicts four images that encompass the primary contributions of this paper. The first image depicts three variations of overview-detail interfaces from our content analysis - highlighting key dimensions of content, compositions, and layout. The second image is taken from our design probe, showing the various interactions for end-users to customize the overview-detail interface to match their own needs and preferences. Among these interactions, we contribute a novel technique, Fluid Attributes, shown in the third and fourth images, enabling users to directly interact with and manipulate content attributes between the overview and the detail view. The third image shows, a user selecting attributes to be surfaced in the overview while the fourth image shows a user sorting the overview via a selected attribute.}
% \end{teaserfigure}


%%
\maketitle{}

%%
\section{Introduction}\label{sec:intro}

In computational finance, Monte Carlo simulations are used extensively to estimate the expected value of financial payoffs based on the solution of stochastic differential equations (SDEs) which model the evolution of stock prices, interest rates, exchange rates and other quantities \cite{glasserman04}.  Monte Carlo methods are very general and flexible, but for high accuracy it requires generating a large number of costly SDE path approximations, which has motivated research into a number of variance reduction or, equivalently, cost reduction techniques. One such method is
Multilevel Monte Carlo (MLMC), which was proposed in \cite{GILES2008} and was adapted for various applications that are summarised in \cite{Giles_overview17} and successfully combined with other methods such as quasi-Monte Carlo methods. The main idea of MLMC is to approximate the payoff using different time stepping resolutions when numerically solving the underlying SDE and to generate an optimal number of samples on each level, such that the overall computational cost is minimised subject to the desired bound on the variance. %, such that the total computational cost is minimised. 
The computational savings come from the fact that most samples are computed on the coarser levels and hence are less expensive while only a few samples from the finest levels are required \cite{GILES2008}.


Among the directions in which the computational cost 
of MLMC methods could further be reduced, an important avenue is the use of lower precision calculations, especially for the first Monte Carlo levels where the targeted accuracy is relatively low. 
 An overview of the research on mixed precision for the standard Monte Carlo (MC) framework is provided in \cite{ChowMixedPrecisionStandardMC} but only a few references study the potential of low precision computation in the MLMC framework \cite{Rounding_error_oliver}. To the best of our knowledge, the only MLMC framework with customised precision in the literature is \cite{brugger2014mixed}, but they use a uniform precision for all operations on each Monte Carlo level instead of optimising 
 the precision of each intermediary variable to reduce as much as possible the cost of path generation.
 
An important motivation for an MLMC framework with variable precision would be performing the low precision computations on reconfigurable hardware devices such as Field Programmable Gate Arrays (FPGAs). FPGAs contain customizable logic blocks and connectors that make it easy to adapt the digital circuit architecture for a specific application, leading to a highly parallel and optimised implementation. Therefore they are successfully exploited in applications that require high speed and have high computational workload, such as signal processing \cite{woods2008fpga}, and real time applications like high frequency trading \cite{HFT1,HFT2}. That is why a number of previous works in hardware architecture design implemented the MLMC algorithm to price financial options using FPGAs as accelerators, which resulted in improved speed and power efficiency compared to full CPU architectures \cite{Schryver2013AMM}. The paper \cite{lindsey2016domain} also proposed 
a Domain Specific Language to automate the configuration of FPGAs for this specific application. However, only \cite{brugger2014mixed} proposed a heuristic to reduce the precision in calculations.

In addition, all aforementioned works considered that the random number generation (RNG) is performed in single or double precision. Yet in most cases an important portion of the workload in the overall MLMC simulation comes from the RNG and in \cite{brugger2014mixed} this limited the total computational savings.
To reduce the cost of MLMC simulations in particular those based on the Geometric Brownian Motion (GBM), \cite{approximateICDF_Oliver, NestedOliver} have proposed to use approximate random numbers that are generated by applying an approximation of the inverse CDF to uniform random numbers. In \cite{NestedOliver}, the authors proposed a way to integrate these lower precision random variables into a \textit{nested} MLMC framework and completed a numerical analysis to bound the resulting error at each MC level by a product of the time step and the error in the random number approximation. The same authors show in \cite{approximateICDF_Oliver} that using approximate random variables reduces the cost of path generation by a factor 7.


In this paper we propose a nested MLMC framework that combines the use of approximate random normal variables and lower precision calculations to reduce the computational cost of MLMC even further than \cite{brugger2014mixed,NestedOliver}. We illustrate the efficiency of our framework in Matlab, after making several assumptions on the cost of operations and size of the errors that we carefully justify. We focus on the case of GBM and use the approximate RNG methods presented in \cite{approximateICDF_Oliver} as well as a new slightly modified method that combines CDF inversion and the central limit theorem. To choose the precision of the variables in the low precision path generation, we introduce a novel method to optimise the bit-widths. This optimisation is performed before the main path generation loop is executed and is based on a linear model of the payoff error  
due to rounding when computing in low precision. The error model relies on algorithmic differentiation in a similar manner to \cite{unifying-bwoptim,bitwidth-AD,ADAPT}. The bit-width optimisation procedure can be performed off-line, so this stage can be excluded from the on-line time complexity of our framework. The user specified desired accuracy is then enforced by calculating on-line the number of samples that need to be generated.

In terms of hardware design, we suggest implementing the low precision path generation on FPGAs and the full-precision ones on a CPU or GPU. 
The FPGA offers enough flexibility to define a separate bit-width for every variable in the low precision path generation, and can be reconfigured periodically to update the bit-widths when the market parameters have changed considerably. 


The paper is organized as follows : \Cref{sec:MLMC} introduces MLMC and nested MLMC to make clear the estimator that is implemented in our framework. Then in \Cref{sec:RNG} we detail the methods that could be used to obtain approximate random normally distributed numbers very cheaply for the low precision path generation. In \Cref{sec:error_model} and \Cref{sec:costModel} we propose an error model and a cost model (resp.) that we then use to formulate the optimisation problem that is solved to obtain the optimal bit-widths of fixed point variables in \Cref{sec:optimisation}. Finally we summarise our results and future directions in \Cref{sec:conclusion}.



% \section{Critique of GenAI Interfaces Through the Lens of Feedforward}
% \section{Critique Through the Lens of Feedforward}
\label{section:critique}

In order to understand a wider range of needs and use cases for feedforward in GenAI, we must reflect and critique past and current GenAI systems and explore how we may improve on them.
We use a critical lens \cite{MBL2021generativetheoriesinteraction} with our perspective that GenAI systems need sufficient feedforward to assess and critique three GenAI systems the first author contributed to designing and implementing: Sensecape \cite{sensecape}, Luminate \cite{luminate}, and Malleable Overview-Detail Interfaces \cite{malleableODI}.

% In order to understand a wider range of needs and use cases for feedforward in GenAI, we must reflect and critique past and current systems. 
% This critical lens is to help us concretely understand

\paragraph{Malleable Overview-Detail Interfaces}


\paragraph{Luminate}
% 


\paragraph{Sensecape}
% Describe what the system is and what it's for. Then describe key features.
% Then, go into several of those features and say, however, the user may not know X or Y.
% Leads them to wonder, what will this be?

\section{Feedforward in Generative AI}
\label{section:feedforward}


This section aims to explore possible ways to provide feedforward in generative AI systems. We view that effective feedforward in generative AI interfaces should enable users to:
\begin{itemize}
    \item Anticipate what the AI's response or action may be.
    \item Disambiguate their prompt without excessive exchanges in conversation or interaction.
    \item Directly engage with the feedforward content.
    \item  Reflect on their prompting practices with less friction.
\end{itemize}

To explore various ways to design feedforward in GenAI systems, we first explore feedforward in an LLM-powered conversational interface. We then describe scenarios for three other designs of feedforward in GenAI applications.


\begin{figure*}
    \centering
    \includegraphics[width=1\linewidth]{figures/applications.pdf}
    \caption{We explored three different applications for designing feedforward in GenAI systems: (A) document editors, (B) malleable interfaces, and (C) agent automation systems.}
    \label{fig:applications}
    \vspace{6pt}
\end{figure*}


\subsection{Feedforward in Conversational LLM UIs}

\subsubsection{Feedforward Components}

Research has shown that users benefit from multiple representations when interacting with LLMs \cite{graphologue, sensecape}.
Therefore, we developed feedforward components that each aim to provide a specific representation that would help users anticipate the LLM's response. 
As the user types a prompt, the conversational UI generates feedforward content inside two feedforward components: 1) a key topic phrase alongside a high-level outline of the anticipated response (Fig. \ref{fig:feedforward-components}.1a) and 2) a minimap view with paragraph blocks (Fig. \ref{fig:feedforward-components}.1b). These representations help users anticipate the topics and subtopics the LLM will generate, as well as whether its response might be too short or too long.

%  to help users anticipate what kinds of topics and subtopics the LLM will share 
% of the anticipated length of the response to help users know whether the LLM will generate a short snippet of text or a detailed explanation about a topic


\subsubsection{Manipulating Feedforward Content}

We enable users to manipulate feedforward content in two ways.
First, when users edit their prompt, the feedforward content updates to reflect the changes.
This allows users to identify misalignments between their intent and the feedforward content, encouraging them to add details to the prompt before submitting it.
For instance, they can clarify an ambiguous term (Fig. \ref{fig:feedforward-components}.2) or request a shorter response (Fig. \ref{fig:feedforward-components}.3).
These instant updates to the feedforward components may also spark reflection, such as realizing that it may be useful to understand the advantages and limitations of Wizard of Oz Prototyping rather than just its definition.
% Nevertheless, feedforward components give users the anticipatory insights and granular control over what information they will receive from the LLM.


% In addition to revising their prompt to adjust the feedforward content, 
Second, users can directly edit content inside the feedforward components. For example, rather than prompting the LLM to remove subtopics from the outline, users can directly delete the subtopic title in the outline component or delete the paragraph in the minimap (Fig. \ref{fig:feedforward-manipulation}A). Users can also drag the edge of the minimap component to expand the feedforward content's level of details (Fig. \ref{fig:feedforward-manipulation}B). This can allow users control ``how far ahead'' they can look into the LLM's response. Additionally, users can highlight parts of the outline, revealing a tooltip that generates a list of possible actions and outputs based on that selection, such as adding examples about a subtopic or organizing the content in a bullet-point list. Users can either select one of these actions or request their own, which will update the feedforward outline (Fig. \ref{fig:feedforward-manipulation}C).


\subsection{Application Scenarios}

Our goal is to explore opportunities for designing feedforward in all GenAI interfaces---beyond conversational UIs.
% This can include video creation platforms, programming tools, and more.
In this paper, we explore three applications to showcase various feedforward designs across diverse contexts: document editors, malleable interfaces, and agent automations.


\subsubsection{Document Editors}

Developers and researchers have begun to integrate AI into document editors and writing canvases to support various aspects of writing tasks, such as fixing grammar \cite{laban2024inksync}, presenting continuously synchronized outlines \cite{dang2022beyondtextgen}, and providing widgets to adjust the tone of writing along various dimensions \cite{masson2024textoshop, chatgptcanvas2025}.
However, it can be difficult for users to anticipate what the LLM might produce based on their writing and revisions. For instance, moving a slider up to change the tone from ``informal'' to ``formal'' does not indicate exactly how ``formal'' they are making their writing. It is only after submitting the change and reviewing the final result that users realize whether they have achieved the desired outcome. To reduce the amount trial-and-error to achieve the desired result, document editors can provide feedforward that presents example phrases as the user moves the slider up and down, helping the user quickly develop a clear understanding of the level of formality they are adjusting to (Fig. \ref{fig:applications}A). This feedforward mechanism can be expanded to let users control how much detail is shown in each example phrase, allowing them to reveal more or fewer words as needed. This customization helps align the feedforward component with user preferences, reducing cognitive load.
% Additionally, users may have the option to increase the level of detail in the feedforward content, revealing the full sentences from which the phrases are derived.


% it's hard to anticipate what these changes might be, and where they might edit your writing. results in users having to repeatedly switch between writing and revising with AI.
% Feedforward in AI writing tools can support this process by helping users anticipate what and where AI will modify their writing.
% For instance, instead of making a full complete change/rewritten snippet, can instead provide feedforward through example phrases, if user slides to write more formal, then it will show example phrases that represent that level of formality. (Fig. X)
% Other instance is where you edit. it's going to change where the edit is made before you accept the change...

% If you have a slider for tone changes, or a spectrum of tone changes, instead of saying more X or less X, feedforward can provide concrete example phrases it will use.

% Also, requesting and explicit changes will prompt AI to provide feedforward of which paragraphs it will look at. Increasing the level of detail will highlight which sentences, which phrases, and even potentially what AI will replace the content with.


\subsubsection{Malleable Interfaces}

GenAI is additionally enabling software interfaces to become increasingly malleable, enabling users to generate custom functional applications on the fly \cite{litt2025workout, onetwoval2025calculator}, personalized interfaces that blends user activity \cite{cao2025jelly}, and user-defined abstractions in overview-detail interfaces \cite{malleableODI}.
In current generated malleable interfaces, the user prompts for a custom application and provides specifications, and after rounds of clarifying, AI generates and compiles the requested application. However, it can be unclear what kinds of software components the AI might produce based on their prompts alone. For example, if the user asks AI to show only linen-textured couches from a shopping website, the user cannot be sure whether AI will add a filtering operation, perform another search query, or create a brand new list linen-textured couches. Feedforward can present a list of system operations the AI will perform before submitting the prompt, allowing them to quickly assess and revise the operations without unnecessary exchanges with the AI (Fig. \ref{fig:applications}B).

% Ask for generating interface. before you submit your prompt, you get a wireframe? You get attributes about what kind of interface it will generate? You get

% Brief context of overview-detail interfaces
% Asking for details about certain items, Asking for an attribute at a high level. Will it surface attributes? will it generate a new one? Will it combine several attributes?

% What kinds of data will the AI focus on looking at? Will AI look at a set of attributes among all of the data provided? If so, which ones?


\subsubsection{Agent Automations}
% Being able to control the "speed" of generating content, which would be controlling the "depth of thought" on how much the model should break down the task.

The AI community has also presented demonstrations of AI agents that can automate diverse tasks on the web \cite{operator2025, dia2025}. However, there lacks clear communication of exactly what the agent is doing on the screen. For instance, if a user asks an AI agent to look for tickets to a basketball game, the agent will break the task into steps and perform them by opening and navigating a webpage. During this phase, the user may be unaware of which buttons or components the agent plans to click to find the best prices, which portions of the webpage it will scan for context, or where it will navigate next. Agent automation systems can integrate feedforward visualizations by providing area selections to show where the AI agent is focusing on for context and opaque cursors to tell the user where the agent will click next (Fig. \ref{fig:applications}C). These feedforward representations help users anticipate the agent’s future actions and provide visual affordances to intervene in potential errors by adjusting the selection area or dragging the cursor.

% Users can anticipate what an AI agent will do before they perform the task. Chain-of-thought and surrounding implementations instantiate feedforward by visualizing stages of actions the agent plans to perform. During each step, users can then stop and adjust stage by stage, providing steerability.

% AI agents should also present feedforward for various actions they aim to perform before submitting the prompt. For instance, users should see depth of thought AI aims to use from that prompt, which then users can then manually adjust.

% During automation tasks, agents currently show an AI-controlled cursor indicating clicks AI aims to perform. We view that this form of communication can be extended to all forms of selection. AI can place shaded rectangles over areas they are currently taking in as context. AI provides feedforward before performing the action. Because of this, the user is able to confirm that is the action the user wants AI to perform, and if it isn't can adjust the selection to steer AI to use the correct selection.



% \subsection{Implementing Feedforward}

% We should also investigate what is the best way to implement feedofrward.
% We implemented such that the feedforward is generated live, then the feedforward feeds into the context for the real content to generate as a skeleton.


% Preliminary Design Space for Feedforward in Generative AI systems
% Representations
% Levels of Detail
% Prompt Revision and Redirection Mechanisms
% 


% , or reflecting on the minimap feedforward component's information  that they must explicitly ask for ``just the definition'' if they want a shorter response.

% The outline will produce feedforward about content, but for the length of the expected response, feedforward will represent as a box to visualize the length in size.

% \subsubsection{Adjusting the Levels of Feedforward Detail}


% As you type your prompt to an LLM, you will see a short outline of text that preview what the LLM will output. This gives you an idea of what you can anticipate given your prompt. As you adjust your prompt, the preview also updates, allowing you to closer align your prompt to your desired outcome.

% Problem chatgpt, ask a question, don't know what it will respond with. Especially can't gauge if your prompt is ambiguous or concrete enough.
% Sending, responds different topic, or generate too much
% Result is you need to submit your prompt again, tuning it, might miss another aspect, or the prompt might not go in your favor again
% gets repetitive
% Instead feedforward in convo LLMs can provide previews of what might happen.


% Feedforward components do not only serve the purpose of providing information before submitting a prompt, but also enables users to interact with the content inside directly. 

% Users can adjust a slider that would adjust the levels of detail the feedforward content will provide. Users can move this slider to control how far ahead they see into what might be generated as they construct their prompt and also look further ahead into what their prompt might lead to.

% As the user slides to view more details, it adjusts all aspects of information of the feedforward content. For instance, text headers and outlines will expand to show subheaders and more details in the outline, while a label ``table'', previewing that the LLM's response will generate a table, will present a low fidelity table. Upon increasing the level of detail more the table will populate with more details as its column and row titles and text inside the cells.

% You can slide to adjust the levels of detail the feedforward will output. for instance, if anticipated to provide a table, the highest level could be the word "table", then describe number of columns and rows + visualize that table in the feedforward space, then populate content in that table little by little, until you go to the end and it ultimately becomes the fully generated response

% You can interact with feedforward content directly. You can select any content in the feedforward content, ask to expand on it, expanding on that level of detail. Selecting on content will also automatically generate recommended prompts for actions to perform, such as if you select the header ``Examples'' a dropdown menu will generate prompts such as ``Remove this header'', ``More details on examples'', and ``Replace with Application Scenarios''.


% Users can ... 
% When incorporating the slider to adjust the level of detail alongside manipulating the feedforward content, users can for instance select a table preview, increase or reduce the level of detail of that table specifically. If the user feels the table does not represent the contained information adequately, they can hold shift while reducing the level of detail to the term ``table''. Upon changing ``table'' to ``two columns'', the feedforward content updates to provide a preview of the two columns arrangement of the information.

% You can interact with any feedforward representation.

% You can also interact with the feedforward content at any level of detail. For instance, you can grow more detailed, and if you don't like where it's going, you can slide back, adjust some content, and see what that would generate.
% Maybe when you choose to go back and then change, you will see a feedforward of the feedforward, showing what that adjustment you made might look like.

% \begin{figure*}
%     \centering
%     \includegraphics[width=1\linewidth]{figures/feedforward-workflow.png}
%     \caption{Enter Caption}
%     \label{fig:enter-label}
% \end{figure*}

\section{Discussions}

In this section, we carry out some discussions about the relationship of STAIR with the techniques applied in proprietary LLMs. For StrongReject, we report the goodness scores on three types of data, including PAIR, PAP-Misrepresentation, and None for queries without jailbreak.



\begin{table}[ht]
\vspace{-1ex}
    \centering
    \caption{Comparison with open-source reasoning LLMs and those trained with Deliberative Alignment on multiple benchmarks.}
    \scriptsize
    % \renewcommand{\arraystretch} % Increase row height
\resizebox{\linewidth}{!}{%
    \begin{tabular}{l|c@{\;\,}c@{\;\,}cc@{\;\,}c}
    \toprule[1.5pt]
       \multirow{2}{*}{o1-Like Models}  & \multicolumn{3}{c}{StrongReject} & \multirow{2}{*}{XsTest}  & \multirow{2}{*}{GSM8k}  \\ \cmidrule(lr){2-4}
         & None & PAIR & PAP-Mis \\\midrule
      LLaMA-o1  & 0.5771 & 0.4441 & 0.5272 & 27.00\% &  79.38\%  \\
      Skywork-o1  & 0.6865 & 0.4034 & 0.4397 & 27.50\% &  91.28\%  \\
      OpenO1 & 0.6837 & 0.3367 & 0.3522 & 34.00\% & 87.41\% \\
      DeepSeek-r1-Dist. & 0.5551 & 0.2987 & 0.3590 & 26.00\% & 91.28\%\\
      QwQ-32B-Preview & 0.8800 & 0.3195 &  0.5978 & 88.50\% & \bf 95.22\%\\\midrule
      \multicolumn{6}{c}{+ \sc Deliberative Alignment}\\\midrule
      Open-o1 & 0.9030 & 0.3782 & 0.4400 & 79.00\% & 86.58\%\\
      DeepSeek-r1-Dist. & 0.9756 & 0.5759 & 0.5895 & 78.00\% & 91.13\%\\\midrule
      STAIR-DPO-3 & \bf 1.0000 & \bf 0.7919 & \bf 0.9677 & \bf 99.00\% & 87.64\%   \\\bottomrule[1.5pt]
    \end{tabular}}
    \label{tab:reasoning}
    \vspace{-2ex}
\end{table}



\subsection{Reasoning for Alignment}

Alongside the release of o-family models by OpenAI~\cite{jaech2024openai}, they proposed the technique of Deliberative Alignment~\cite{guan2024deliberative}, which benefits safety alignment from the existing powerful reasoning foundation models. Our method, in contrast, does not rely on this prerequisite and can make normal instruction-tuned LLMs better aligned by integrating safety-aware reasoning. 

We reproduce deliberative alignment to our best on open-source o1-like LLMs and compare the results. To guarantee a fair comparison, we select models inheriting LLaMA-8B, including LLaMA-o1~\cite{zhang2024accessing}, Skywork-o1-Open-LLaMA-3.1-8B~\cite{skyworkopeno12024}, OpenO1-LLaMA-8B\footnote{https://huggingface.co/O1-OPEN/OpenO1-LLama-8B-v0.1}, and DeepSeek-r1-Distilled-LLaMA-8B~\cite{deepseekai2025deepseekr1incentivizingreasoningcapability} with an exception of QwQ-32B-Preview~\cite{qwq-32b-preview}. We first test the safety of these models and find that most of them cannot resist even simple harmful queries, as shown by the results of StrongReject-None and XsTest in~\cref{tab:reasoning}. Then, we combine the 25k safety-related prompts in the seed dataset with some safety policies, which are generated by OpenAI o1-preview and manually organized, and ask the model to reason according to the provided terms and decide whether to refuse the queries. After filtering the responses with successful refusals, we use the prompts and responses to train the model using SFT. This procedure is conducted on Open-o1 and DeepSeek-r1-Distilled. We can notice the increasing refusal rates on straightforward questions, but the vulnerability to jailbreak attacks still remains. This might be attributed to the limited reasoning capability, the lack of more complex data, or the absence of further RL training. By comparison, the model trained after three iterations with STAIR has better resilience against jailbreak while preserving comparable performance on GSM8k.







\subsection{Comparison with Commercial LLMs}

Besides the publicly released technique, commercial LLMs, which are more broadly used by society, usually have their own safety guardrails against malicious jailbreak attacks. We select a group of popular commercial LLMs from different institutions and compare their performance on StrongReject with our method.

\cref{tab:proprietary} lists the results on diverse commercial LLMs. We can see that most LLMs can correctly refuse straightforward harmful questions, with goodness scores all over 0.95. However, some of them demonstrate worrying vulnerability to modern jailbreak attacks, while Claude-3.5 from Anthropic has the best defense. o1, reported to be much better than GPT-4o~\cite{jaech2024openai}, is not included because of the frequent warnings of jailbreak attempts during API calls. Through iterative self-improvement of safety-aware reasoning, we consolidate LLaMA to a comparable level to Claude, even surpassing it when we apply test-time scaling.


\begin{table}[t]
    \centering
    \caption{Comparison with Proprietary LLMs on StrongReject}
    \scriptsize
    \newcommand{\degree}{90}
    \resizebox{\linewidth}{!}{%
    \begin{tabular}{l@{\;\,}|@{\;\;}c@{\;\;}c@{\;\;}c@{\;\;}c@{\;\;}c@{\;\;}c@{\;\;}|@{\;\;}c@{\;\;}c}
    \toprule[1.5pt]
         & \rotatebox{\degree}{GPT-4o} & \rotatebox{\degree}{Claude-3} & \rotatebox{\degree}{Claude-3.5} & \rotatebox{\degree}{Qwen-Max} & \rotatebox{\degree}{Gemini-1.5} & \rotatebox{\degree}{DeepSeek-R1}& \rotatebox{\degree}{STAIR-DPO-3} & \rotatebox{\degree}{+Beam Search}\\\midrule
     None    & 0.9796 & 0.9968 & \bf 1.0000 & 0.9844 & 0.9952 & 0.9633 & \bf 1.0000 & \bf 1.0000\\\midrule
     PAIR    & 0.3327 & 0.8710 & \bf 0.9129 &  0.3187 & 0.5791 & 0.2069 & 0.7919 & 0.8994\\
     PAP-Mis & 0.4217 & 0.9601 & 0.9589 & 0.4269 & 0.7504 & 0.4034 & 0.9677  &\bf 0.9788 \\\midrule
     Average &  0.3772 & 0.9156 & 0.9359 & 0.3728 & 0.6648 & 0.3052 & 0.8798  & \bf 0.9391 \\
     \bottomrule[1.5pt]
    \end{tabular}}
    \label{tab:proprietary}
    \vspace{-3ex}
\end{table}


% \input{sections/99_conclusion}

%%
% Acknowledgments.
% \begin{acks}
% Thanks to the anonymous reviewers for their helpful comments.
% \end{acks}

%%
\bibliographystyle{ACM-Reference-Format}
\bibliography{main}

\appendix

\section{Pre-annotation and classifiers}\label{app:pre}
\paragraph{Pre-annotation}
We start with the \textbf{Second Reading debates of Bills},\footnote{\url{https://www.parliament.uk/about/how/laws/passage-bill/commons/coms-commons-second-reading/}} where the members debate the main principles of a certain Bill. The advantages of using such debates are: (i) the stance of an argument can be easily identified based on whether they support %for 
the Bills; (ii) debates can be paired with brief Bill introductions,\footnote{e.g., the `long title' on page \url{https://bills.parliament.uk/bills/3858}} providing clear argument topics; and (iii) the arguments 
focus on Bill principles, with fewer discussions on specific amendments and clauses, which require less contextual awareness than other Bill debates like the ones for the Committee Stage.\footnote{\url{https://www.parliament.uk/about/how/laws/passage-bill/commons/coms-commons-comittee-stage/}}
We choose five Bills, including topics relevant to animal welfare and parental leave (see Table \ref{tab:bill} for the Bill introductions), 
which may be easier to annotate and more likely to have emotional arguments.

Three annotators label 245 texts from these debates for \textbf{three layers}: (\emph{L1}) 
whether the text evokes emotions, (\emph{L2}) whether the text contains standalone arguments, and (\emph{L3}) the stance of the text toward the Bill. \emph{L1} and \emph{L2} are labeled `0' (for answer `no') or `1' (for `yes'). If \emph{L2} is labeled `1', annotators proceed to label \emph{L3}, which has four options: `0' for support, `1' for opposition, `2' for inability to identify stance without additional context, and `3' for a neutral stance suggesting additional amendments or policies. Besides, 40 texts from the pilot annotation are also annotated for \emph{L1} and \emph{L2}.  
To potentially speed up the annotation process, the 285 texts are selected from those judged as both emotional and argumentative by GPT4o. Here, we prompt GPT4o with simple questions such as \emph{Does this text try to convince readers something?} and \emph{Is this text emotional?'}.

40 of the outputs are jointly labeled by all annotators, achieving average Cohen's Kappa of 0.622 for \emph{L1}, 0.674 for \emph{L2}, and 0.762 for \emph{L3} across annotator pairs. 
As shown in the `Question' column of Table \ref{tab:pre}, GPT4o already achieves a high precision of 0.82 in detecting argumentative texts using simple prompts. However, its precision for emotional text classification is still low (0.53).

We then convert the annotations for \emph{L3} to \emph{L3$^{*}$}, where we pair argument pairs based on their topics and stances. The categories include: `different topic' for pairs with different topics (from different Bills), `different stance' for pairs with the same topic but different stances, and `same' for pairs with the same topic and stance.

The number of texts annotated for each layer and the corresponding label distribution\se{s} are summarized in Table \ref{tab:pre} (left). 

\paragraph{Automatic Pipeline}
We develop three classifiers based on GPT4o 
to automatically identify the argument pairs needed. The pipeline is as follows: 
\begin{enumerate}[]
    \item \textbf{Argumentative text classification}: our goal is to have a \textbf{high precision} classifier since we have sufficient candidate texts. We find that when we ask GPT-4o to provide the major claim, evidence, and reasoning connecting the evidence to the major claim in the text, its precision increases from 0.82 to 0.96, as shown in the `Argumentative' row of Table \ref{tab:pre}. 
    
    We then retain texts judged as argumentative for \hansard{} using this prompt, while for \deuparl{}, we use a German translation of the same prompt. The overall performance of GPT4o on German data is assessed after completing the stance agreement classification task (see below).

    \item \textbf{Stance agreement classification}: 
    To enable the flexible selection of classifiers with specific performance characteristics (e.g., high recall, high precision), we introduce a parameter into the prompt, with its threshold optimized to achieve different specialized performance levels.
    To do so, we ask GPT4o to rate the likelihood that two given arguments address the same topic and share the same stance on a Likert scale from 0 to 100. We randomly sample 600 argument pairs (with a 2:1:1 ratio for the three categories of \emph{L3$^{*}$}) from the dataset, ‘optimize’ the threshold of ratings for the `same’ category 
    using argument pairs from two Bills, and test the performance on the remaining three Bills to prevent data leakage. We evaluate all possible combinations of Bills for the training and test sets.
    We observe that as the threshold increases, precision on the `same’ category ($P_{same}$) consistently improves, while macro F1 begins to decrease beyond certain thresholds. With a threshold of 100, $P_{same}$ reaches 0.92, but F1 is very low at 0.45. Therefore, we select a threshold of 90 as a more balanced trade-off, achieving $P_{same}=0.81$ and $\textit{F1}=0.76$, to obtain more candidates that are still highly likely to be true positives. 
    
    For \hansard{}, we retain the argument pairs labeled as belonging to the `same' category using this threshold. For \deuparl{}, we apply the German translation of the prompt with the same threshold to identify argument pairs. One annotator evaluates 50 candidates from the outputs of steps 1 and 2: no argument is labeled as non-argumentative, while 12 argument pairs are identified as false positives in the stance agreement task, yielding $P_{same}=0.76$. This value is only 4 percent points lower than the result on English data. Consequently, we retain these prompt settings for the German data.
    
    \item  \textbf{Emotional text classification}: we aim for a \textbf{balanced} classifier because we also need non-emotional arguments. Since this is a subjective task, we ask GPT4o to rate how likely it can feel the emotions 
    in the texts on a \se{L}ikert scale of 0-100, and then `optimize' the threshold of the rates for the `emotional' category on 70\% of the data and check how it performs on the remaining 30\%. Overall, with this step, we can improve the macro F1 to 0.74-0.81 (averaged over three rounds of data splitting), depending on the gold from different annotators. The best threshold for two annotators is 75, while that for the other is 85, so we use the threshold 75 to represent the majority, which has a macro F1 of 0.75, averaged across the three annotators. 

    We use this threshold to select the argument pairs for \hansard{}. For \deuparl{}, we further optimize the threshold using a small-scale set of human annotations and adjust it to 85. This setting is then used to label the binary emotions of arguments. 

\end{enumerate}

\begin{table}[!ht]
\resizebox{\linewidth}{!}{%
\begin{tabular}{@{}lcccc@{}}
\toprule
                               & \multicolumn{2}{c}{Pre-Annotation} & \multicolumn{2}{c}{Automatic Pipeline}   
                               \\
                               & \#                 & \%  & Question  & `Optimized' \\ \midrule
\multicolumn{3}{l}{\emph{L1 - emotion} }                                             \\ \midrule
Emotional                      & 151                & 53.0 & 0.53 (P)  & \multirow{2}{*}{0.75 (F1)} \\ 
Non-emotional                  & 134                & 47.0 & -  \\ \bottomrule
\multicolumn{3}{l}{\emph{L2 - argument}}                                         \\ \midrule
Argumentative                  & 234                & 82.1 & 0.82 (P) & 0.96 (P)  \\
Non-argumentative              & 51                 & 17.9 & - & - \\ \bottomrule
\multicolumn{3}{l}{\emph{L3 - stance} }                                           \\ \midrule    
Support                        & 170                & 72.6 & - \\
Opposition                     & 2                  & 0.9 & -  \\
Neutral                        & 29                 & 12.4  & -\\
Irrelevant                     & 16                 & 14.1& - \\ \midrule
\multicolumn{3}{l}{\emph{L3$^{*}$ - pair stance}}                                            \\ \midrule   
Same           & 2,905            & 8.9  & -   & \multirow{3}{*}{\makecell{0.80 ($P_{same}$) \\ 0.75 (F1)}} \\
Different stance & 3,325              & 10.2 & - \\
Different topic                & 26,486             & 81.0 & - \\ \midrule
Total                          & 32,716             & 100  & - \\ \bottomrule
\end{tabular}}
\caption{Number of texts annotated for each layer and category (\#) and the corresponding label distribution (\%). Performance of GPT4o on the binary emotion classification, argument identification, and stance agreement detection tasks used for automatically identifying the target argument pairs.}\label{tab:pre}
\end{table}


\begin{table*}[!ht]
\resizebox{\linewidth}{!}{
\begin{tabular}{@{}l@{}}
\toprule
\emph{Introduction}                                                                                                                                                                         \\ \midrule
A Bill to Prohibit the export of certain livestock from Great Britain for slaughter.                                                                                                 \\ \midrule
\makecell[l]{A Bill to create offences of dog abduction and cat abduction and to confer a power to make corresponding provision  \\ relating to the abduction of other animals commonly kept as pets.} \\ \midrule
A Bill to make provision about leave and pay for employees with responsibility for children receiving neonatal care.                                                                   \\ \midrule
A Bill to prohibit the import and export of shark fins and to make provision relating to the removal of fins from sharks.                                                            \\ \midrule
A Bill to prohibit the sale and advertising of activities abroad which involve low standards of welfare for animals.                                                                 \\ \bottomrule
\end{tabular}}
\caption{The introductions of the five Bills selected in \protect\bill{}.}\label{tab:bill}
\end{table*}


% Please add the following required packages to your document preamble:
% \usepackage{booktabs}
\begin{table*}[]
\resizebox{\linewidth}{!}{
\begin{tabular}{@{}l|l@{}}
\toprule
English                                                                                                                                                                                  & German                                                                                                                                                                                     \\ \midrule
\makecell[l]{iran, integrat, ukraine, russia, asylum,\\ deportation, israel, gaza, expulsion, \\ displacement, migration, migrant, \\immigrant, refugee, palestine,invasion,\\ repatriation, hamas, hisbollah} & \makecell[l]{ukraine, russland, migrant, \\ immigrant, flüchtling, asyl,\\ gaza, iran, palästina, \\israel, krieg, invasion, \\sanktionen, waffenlieferungen, friedensverhandlungen, \\kriegsverbrechen, flüchtlingskrise, nato,\\ energieversorgung, vertreibung, migrationspolitik,\\ asylverfahren, grenzsicherung, integration, \\abschiebung, aufenthaltsgenehmigung, menschenhandel, \\seenotrettung, rückführung, schutzstatus, \\waffenstillstand, raketenangriffe, besatzung, \\zwei-staaten-lösung, friedensprozess, intifada, \\ hamas, hisbollah, menschenrechte, un-resolution
} \\ \bottomrule
\end{tabular}
}
\caption{Keywords used to filter debates for \hansard{} and \deuparl{}.}\label{tab:keywords}
\end{table*}



% Please add the following required packages to your document preamble:
% \usepackage{booktabs}
\begin{table*}[]
\centering
%\resizebox{!}{\linewidth}{
\begin{tabularx}{\linewidth}{X}
\toprule
\emph{Remove Emotion Prompt}  \\ \midrule
====\textbf{System Prompt}=====\\ I will give you an argumentative text that **can** appeal to emotion.    \\ \\ Your task is to generate an argument with the same stance for the same topic **without emotional language**, by rephrasing the text but maintaining a similar style and length. \\ \\ Briefly explain why the rewritten argument no longer evokes emotions.\\ \\ Answer in the following way:\\ Generated argument: \\ Explanation:\\ ====\textbf{User Prompt}=====\\ Text: \{original argument\}  \\ \midrule
\emph{Add Emotion Prompt}  \\ \midrule
====\textbf{System Prompt}=====\\ I will give you an argumentative text that **cannot** appeal to emotion.\\     \\ Your task is to generate an argument with the same stance on the same topic **with emotions**, by rephrasing the text but maintaining a similar style and length. \\ \\ Briefly explain why the rewritten argument can evoke emotions now.\\ \\ Answer in the following way:\\ Generated argument: \\ Explanation:\\ ====\textbf{User Prompt}=====\\ Text: \{original argument\}                \\ \bottomrule
\end{tabularx}
%}
\caption{Prompts used to remove/add emotions for synthetic arguments.}\label{tab:prompt_synthetic}
\end{table*}


\section{Arguments from others}\label{app:other}
\paragraph{\dagstuhl{}} 
\citet{wachsmuth-etal-2017-computational} collected human ratings on a Likert scale of 1–3 for multiple dimensions of argument quality, including argument effectiveness (convincingness)\footnote{“Argumentation is effective if it persuades the target audience of (or corroborates agreement with) the author’s stance on the issue.” — \citet{wachsmuth-etal-2017-computational}} and emotional appeal. These ratings were applied to 304 argumentative texts from \citet{habernal-gurevych-2016-argument}, which were sourced from a textual debate portal in \textbf{English}. We retain only those arguments whose average convincingness rating (across the three annotators) exceeds 1.5. 
Next, we pair arguments that share the same stance on the same topics and calculate the absolute differences in their emotional appeal ratings. From these pairs, we randomly select 10 topics and then retain the 5 argument pairs with the largest absolute differences in emotional appeal for each topic.

\paragraph{\lynn{}}
\citet{greschner2024fearfulfalconsangryllamas} collected
discrete emotion labels from a reader respective (e.g. joy, disgust etc.) for 300 \textbf{German} arguments associated with 30 statements, drawn from \citet{velutharambath_wuehrl_klinger_2024a}. Each argument was annotated by three annotators. We interpret the number of annotations marking the argument as containing specific emotions (rather than `no emotion') as its emotion score. E.g., if three annotators identify specific emotions in the argument, its emotion score would be 3. Using a procedure similar to the one employed for \dagstuhl{}, we pair arguments referencing the same statement, randomly select 25 statements, and then retain the 
two argument pairs per statement that exhibit the greatest differences in emotion scores. 


\begin{table}[]
\centering
\resizebox{\linewidth}{!}{
\begin{tabular}{@{}llllll@{}}
\toprule
     & \dagstuhl{} & \bill{} & \hansard{} & \lynn{} & \deuparl{} \\ \midrule
\multicolumn{6}{l}{\emph{Increase}}                               \\ \midrule
\rz{}   & \textbf{-0.06}   & \textbf{0.15}         & \textbf{0.05}   & \textbf{-0.38}  & \textbf{0.32}   \\
\rth{} & -0.18    & -0.21        & -0.31  & -0.46   & -0.38  \\ \midrule
\multicolumn{6}{l}{\emph{Decrease}    }                        \\ \midrule
\ro{} & -0.12 & -0.21 & -0.03 & 0.08 & -0.19    \\
\rtw{} & \textbf{0.36}    & \textbf{0.27}         & \textbf{0.29}   & \textbf{0.76}   & \textbf{0.25} \\ \bottomrule
\end{tabular}}
\caption{BWS scores for the 4 argument groups: \rz{}, \ro{}, \rtw{} and \rth{}, derived from the majority votes of the annotation for pairwise comparisons of emotional intensity. `Increase'/`Decrease' denotes the direction to increase/decrease the perceived emotional intensity.}\label{tab:bws}
\vspace{-.3cm}
\end{table}

\section{Prompts}\label{app:prompt}
Table \ref{tab:prompt_synthetic} presents the prompts used to introduce/remove emotions. Table \ref{tab:promptc} illustrates the prompts used for evaluating argument convincingness.

% Please add the following required packages to your document preamble:
% \usepackage{booktabs}
\begin{table*}
\footnotesize
\resizebox{\linewidth}{!}{
\begin{tabular}{@{}cl@{}}
\toprule
\multicolumn{2}{l}{Prompt Template}                                                                                                                                                                                                                                                                                                                                                                                                                                                                                                                                                                                                                                                                                     \\ \midrule
Shared & \begin{tabular}[c]{@{}l@{}}Below, you will find one pair of argumentative texts discussing the same topic with the same stance. The topic may be a binary \\choice, a bill from UK parliamentary debates, or a simple statement. Both arguments either support or oppose the topic, or they \\favor one side if the topic involves a binary choice.\\ \\ Your task is to evaluate each pair to determine **which argumentative text you find more convincing**. There are three label options:\\ 0 (Both arguments are equally convincing.)\\ 1 (Argument 1 is more convincing.)\\ 2 (Argument 2 is more convincing.)\\ \\ **Note**: Truncated sentences or grammatical errors should be **ignored**.\end{tabular} \\ \midrule
1      & \begin{tabular}[c]{@{}l@{}}Please answer your label option **without** any explanations.\\ \\ \{text\}\end{tabular}                                                                                                                                                                                                                                                                                                                                                                                                                                                                                                                                                                                            \\ \midrule
2      & \begin{tabular}[c]{@{}l@{}}Please answer your label option and briefly explain why you choose this label.\\ \\ \{text\}\\ \\ Below is an example answer for you; please follow this format in your response.\\ Label: 2\\ Explanation: because Argument 2 provides more statistics supporting the claim, while Argument 1 contains logical fallacies.\end{tabular}                                                                                                                                                                                                                                                                                                                                             \\ \midrule
3      & \begin{tabular}[c]{@{}l@{}}Please answer your label option and briefly explain why you choose this label.\\ \\ \{text\}\\ \\Below is an example answer for you; please follow this format in your response.\\ Label: 1\\ Explanation: Argument 1 is more convincing, because I totally agree with its point and it evokes my empathy.\end{tabular}                                                                                                                                                                                                                                                                                                                                                                                                                                                            \\ \bottomrule
\end{tabular}}
\caption{Prompt templates for comparing the convincingness of an argument pair. The {text} field contains the two arguments and their topic. The complete prompt is formed by combining the text in the `Shared' row with the text in the corresponding indexed row. For example, Prompt 1 consists of the text from both the `Shared' row and row `1'.}\label{tab:promptc}
\end{table*}



\begin{table}[]
\vspace{-.2cm}
\centering
\setlength\tabcolsep{2pt} 
\resizebox{\linewidth}{!}{%
\begin{tabular}{@{}lcc|cccccc@{}}
\toprule
& \multicolumn{2}{c|}{\textbf{\#Annotators}} & \multicolumn{6}{c}{\textbf{Agreements}} \\
              & \textbf{S} &  \textbf{C} & \multicolumn{3}{c}{\textbf{EMO}} & \multicolumn{3}{c}{\textbf{CONV}} \\ %\midrule
              &&& $\alpha$ & Full & Maj. & $\alpha$ & Full & Maj. \\ \midrule
\dagstuhl{}      & 1        & 4         &  0.506  & 6.5\% & 74.5\%   & 0.540  & 14.0\% & 80.0\%   \\
\bill{} & 1        & 4        &  0.449  & 7.0\% & 76.5\%   & 0.463  & 10.5\% & 78.0\%   \\
\hansard{}       & 1        & 4        &  0.361 & 0.5\% & 68.0\%   & 0.371  & 6.0\% & 75.0\%   \\
\lynn{}       & 2        & 3       &  0.729 & 13.5\% & 87.5\%   & 0.607  & 16.0\% & 82.0\%   \\
\deuparl{}       & 3        & 2       & 0.352  & 8.0\% & 80.5\%   & 0.364  & 4.5\% & 74.5\%   \\ \midrule
Avg    & -        & - & 0.479 & 7.1\% & 77.4\% & 0.469 & 10.2\% & 77.9\% \\ 
\bottomrule
\end{tabular}}
\caption{\textbf{Left}: Number of student (S) and crowdsourcing (C) annotators per batch. \textbf{Right}: Krippendorf's $\alpha$ for the most agreeing annotator pairs (\textbf{$\alpha$}), the percentages of annotation instances where all annotators agree on a certain label (\textbf{Full}),  and the percentage of annotation instances where at least three annotators agree on a certain label (\textbf{Maj.}). 
}
\label{tab:annotator}
\vspace{-.6cm}
\end{table}


\section{Annotation Interface}\label{app:anno}
Figure \ref{fig:anno} shows the screenshots of the annotation interface for convincingness (top) and emotion (bottom) comparisons. We collect the annotations via Google Forms\footnote{\url{https://docs.google.com/forms/}} for crowdsourcing annotators.

\begin{figure*}
    \includegraphics[width=\linewidth]{structure/figs/anno/conv_form.pdf}
    \includegraphics[width=\linewidth]{structure/figs/anno/emo_form.pdf}
    \caption{Screenshots of the annotation interface for convincingness (top) and emotion (bottom) comparison.}\label{fig:anno}
\end{figure*}


\section{Examples}\label{app:exa}
Table \ref{tab:pos} and \ref{tab:neg} provide example instances from \hansard{} and \lynn{}, where emotions have a positive and negative impact, respectively. 

\begin{table*}[!ht]
\centering
\begin{tabularx}{\textwidth}{ X | X }
\toprule
\multicolumn{2}{l}{\textbf{Topic}: The public supports the UK's aid for Ukrainian refugees} \\ \midrule
\rz{}  & \ro{}  \\ \midrule
Members across this House are determined that we, as a country, should open our arms to these people, and this determination has been on full display today. The scenes of devastation and human misery inflicted by President Putin’s barbarous assault on what he calls “Russia’s cousins” in Ukraine have unleashed a tidal wave of solidarity and generosity across the country. British people always step forward and step up in these moments, and since the first tanks rolled into Ukraine, they have come forward in droves with offers of help: community centres have been flooded with critical supplies; the Association of Ukrainians in Great Britain has received millions in donations; and charities such as the Red Cross have been overwhelmed with people giving whatever they can. The outpouring of public support has been nothing short of remarkable. & While this Government, and this whole House, have risen to the occasion with our offer of support to Ukrainians fleeing war, our lethal aid and our stranglehold on economic sanctions on Russia have clearly shown that we will keep upping the ante to ensure that Putin fails. As Members have argued today, it has been abundantly clear in recent days that we can and must do more. It is exactly right, therefore, that my right hon. Friend the Secretary of State for Levelling Up, Housing and Communities set out on Monday the new and uncapped sponsorship scheme, Homes for Ukraine. It is a scheme to allow Ukrainians with no family ties to the UK to be sponsored by individuals or organisations that can offer them a home. It is a scheme that draws not only on the exceptional good will and generosity of the British people, but one that gives them the opportunity to help make a difference.                                                                                                                                                        \\ \midrule
\rth{}   & \rtw{} \\ \midrule
Members of this House have expressed a commitment to welcoming individuals from Ukraine. The recent conflict initiated by President Putin has resulted in significant destruction in Ukraine, prompting a substantial response of support across the country. British citizens have actively contributed since the conflict began, with community centers collecting essential supplies, the Association of Ukrainians in Great Britain receiving financial contributions, and charities like the Red Cross witnessing increased donations.  & In these trying times, the Government and this entire House have demonstrated unwavering courage and compassion by extending our support to Ukrainians escaping the horrors of war. Our determined provision of lethal aid and the relentless imposition of economic sanctions on Russia are powerful affirmations that we will stop at nothing to ensure Putin's defeat. As Members have passionately discussed today, the urgency to do even more has never been clearer. That is why it is so heartening that my right hon. Friend the Secretary of State for Levelling Up, Housing and Communities announced on Monday the new and limitless Homes for Ukraine sponsorship scheme. This initiative opens its arms to Ukrainians without family connections in the UK, allowing them to be warmly embraced by individuals or organizations ready to offer them a sanctuary. It is a testament not only to the extraordinary kindness and generosity of the British people but also to their deep desire to make a meaningful impact in the lives of those in desperate need. \\ \bottomrule
\end{tabularx}
\caption{An example instance from \hansard{} where emotions have a \textbf{positive} impact on argument convincingness.
}\label{tab:pos}
\end{table*}

\begin{table*}[!ht]
\centering
\begin{tabularx}{\textwidth}{ X | X }
\toprule
\multicolumn{2}{l}{Topic: Haie können Krebs bekommen.}    \\ \midrule
\rz{}  & \ro{}  \\  \midrule
Haie sind mehrzellige Lebewesen, wie auch der Mensch. Die Beonderheit von mehrzelligen Lebewesen ist, dass die Zellen sich sowohl stark spezialisieren und untereinander vernetz kommunizieren. Damit werden sie anfällig für bestimmte Zelldefekte, die sich über die genannte Struktur fortpflanzen und den Krebs ausmachen. Haie verfügen, wie auch der Mensch und überhaupt alle mehrzelligen Lebewesen, über nur eine sehr eingeschränkte Möglichkeit diese Defekte zu korrigieren und aufzuhalten, damit können beide gleichermaßen Krebs bekommen & Da auch Fische Krebs bekommen können, ist es auch möglich, dass Haie Krebs bekommen können. Dieser wird durch mutierte Zellen ausgelöst, weshalb dies auch bei Fischarten ausgelöst werden kann. Krebs ist eine weit verbreitete und häufige Krankheit, weshalb Krebs durch Wissenschaftler auch bereits bei Haien festgestellt werden konnte.\\ Krebs kann außerdem auch durch verschiedene Umweltfaktoren wie Umweltverschmutzung ausgelöst werden, diesem Risiko sind Haie ja durchaus ausgesetzt. Deshalb ist die Gefahr einer Erkrankung auch nicht gerade gering.  \\ \midrule
\rth{}  & \rtw{}  \\ \midrule
Haie, ebenso wie Menschen, sind mehrzellige Organismen. Eine charakteristische Eigenschaft solcher Organismen ist die Spezialisierung und Vernetzung ihrer Zellen. Diese Struktur macht sie anfällig für Zellfehler, die sich ausbreiten und zu Krebs führen können. Haie und Menschen besitzen nur begrenzte Mechanismen zur Korrektur und Kontrolle dieser Defekte, was bedeutet, dass beide Arten gleichermaßen anfällig für Krebs sind.  & Die Vorstellung, dass Haie - diese majestätischen und oft missverstandenen Kreaturen der Meere - an Krebs erkranken können, ist zutiefst beunruhigend. Diese Krankheit, die durch die heimtückische Mutation von Zellen verursacht wird, hat bereits viele Fischarten heimgesucht. Die Tatsache, dass auch Haie, die Könige der Ozeane, nicht sicher vor dieser grausamen Krankheit sind, ist erschütternd. Angesichts der weit verbreiteten Umweltverschmutzung, die unsere Ozeane verschlingt, sind Haie einem erheblichen Risiko ausgesetzt, an Krebs zu erkranken. Es ist traurig und alarmierend, dass diese beeindruckenden Tiere, die seit Millionen von Jahren die Meere durchstreifen, nun durch menschliche Einflüsse bedroht sind.
\\ \bottomrule
\end{tabularx}
\caption{An example instance from \lynn{} where emotions have a \textbf{negative} impact on argument convincingness.
}\label{tab:neg}
\end{table*}




\section{LLM}\label{app:llm}
Figure \ref{fig:dis_prompt} illustrates the consistency, positivity and negativity rates of LLMs with different prompts, averaged across instances in all datasets. Table \ref{tab:llm_ranking} displays macro F1 scores and model rankings for LLMs in predicting convincingness rankings of argument pairs ('Static') and the resulting categories of emotional effect (`Dynamic') in English and German.

\begin{figure*}[]
    \centering
    \includegraphics[width=\linewidth]{structure/figs/llm/dis_prompt1.pdf}
    \includegraphics[width=\linewidth]{structure/figs/llm/dis_prompt2.pdf}
    \includegraphics[width=\linewidth]{structure/figs/llm/dis_prompt3.pdf}
    \caption{Consistency, positivity and negativity rates of LLMs with different prompts, averaged across instances in all datasets.}\label{fig:dis_prompt}
\end{figure*}


\begin{table*}[]
\resizebox{\linewidth}{!}{
\begin{tabular}{lcccc|cccc}
\toprule
                           & \multicolumn{4}{c|}{\textbf{EN}}               & \multicolumn{4}{c}{\textbf{DE}}               \\ 
\textbf{Model}                      & Static & Ranking & Dynamic & Ranking & Static & Ranking & Dynamic & Ranking \\ \midrule
gpt-4o-2024-08-06          & \textbf{0.486}  & 1       & 0.411   & 2       & \textbf{0.443}  & 1       & \textbf{0.447}   & 1       \\
Llama-3.3-70B-Instruct     & 0.417  & 2       & \textbf{0.415}   & 1       & 0.372  & 2       & 0.392   & 4       \\
gpt-4o-mini                & 0.416  & 3       & 0.392   & 5       & 0.35   & 4       & 0.394   & 3       \\
Qwen2.5-72B-Instruct       & 0.398  & 4       & 0.398   & 4       & 0.357  & 3       & 0.41    & 2       \\
gpt-3.5-turbo              & 0.39   & 5       & 0.382   & 6       & 0.338  & 6       & 0.381   & 6       \\
Mixtral-8x7B-Instruct-v0.1 & 0.368  & 6       & 0.376   & 7       & 0.35   & 5       & 0.387   & 5       \\
Mistral-7B-Instruct-v0.3   & 0.367  & 7       & 0.407   & 3       & 0.288  & 8       & 0.36    & 9       \\
Llama-3.2-3B-Instruct      & 0.322  & 8       & 0.32    & 10      & 0.281  & 10      & 0.367   & 8       \\
Qwen2.5-0.5B-Instruct      & 0.308  & 9       & 0.342   & 9       & 0.284  & 9       & 0.344   & 10      \\
Qwen2.5-7B-Instruct        & 0.304  & 10      & 0.346   & 8       & 0.319  & 7       & 0.373   & 7       \\
Llama-3.2-1B-Instruct      & 0.286  & 11      & 0.309   & 11      & 0.274  & 11      & 0.343   & 11 \\ \bottomrule     
\end{tabular}}
\caption{Macro F1 scores and model rankings for LLMs in predicting convincingness rankings of argument pairs ('Static') and the resulting categories of emotional effect (`Dynamic') in English and German. For each model, we present the best prompt result to highlight its potential. Human and LLM labels are determined by majority votes from different annotators and rounds, respectively.}\label{tab:llm_ranking}
\end{table*}



\end{document}
