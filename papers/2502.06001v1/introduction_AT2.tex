
\section{Introduction}
The dissemination of information to disparate participants is a fundamental problem in both the construction and theory of distributed systems. A common strategy for solving this problem is to ``broadcast'', i.e.~to transmit a piece of information initially held by one agent to all other agents in the system. In fact, broadcast is not merely a fundamental communication primitive in many models, but also underlies \anew{solutions to} other fundamental problems such as leader election and wake-up. Given this essential role in the operation of distributed computer systems and the potential volume of broadcasts, an important consideration is \anew{simplifying the algorithms and minimizing the overhead} required for each broadcast.

\anew{Within a synchronous setting, \textit{Amnesiac Flooding} as introduced by Hussak and Trehan in 2019~\cite{HussakT-AF-PODC19,HussakT-AF-STACS20} eliminates the need of the standard flooding algorithm to store historical messages.} The algorithm terminates in asymptotically optimal $O(D)$ time (for $D$ the diameter of the network) and is stateless as agents are not required to hold any information between communication rounds. \anew{The algorithm in the fault-free synchronous message passing model is defined as follows:}

\anew{
\begin{definition}{\textbf{Amnesiac flooding algorithm.}} (adapted from~\cite{hussak2023termination})
Let $G = (V,E)$ be an undirected graph, with vertices $V$ and edges $E$ (representing a network where the vertices represent the nodes of the network and edges represent the connections between the nodes).  Computation proceeds in synchronous `rounds' where each round consists of nodes receiving messages sent from their neighbours. A receiving node then sends messages to some neighbours in the next round. No messages are lost in transit. The algorithm is defined by the following rules:
\begin{itemize}
\item[(i)]
All  nodes from a subset of sources or {\it initial nodes} $I \subseteq V$  send a message $M$ to all of their neighbours in round 1. 
\item[(ii)]
In subsequent rounds, every node that received $M$ from a neighbour in the previous round, sends $M$ to all, and only, those nodes from which it did not receive $M$. Flooding \emph{terminates} when $M$ is no longer sent to any node in the network.
\end{itemize}
\end{definition}
}

%Simply stated, the rule is: ``If an agent receives a message in a given round, it forwards that message to all neighbours it did not receive it from in this round''.

\anew{These rules} imply several other desirable properties. Firstly, the algorithm only requires the ability to forward the messages, but does not read the content (or even the header information) of any message to make routing decisions. Secondly, the algorithm only makes use of local information and does not require knowledge of a unique identifier. Thirdly, once the broadcast has begun, the initial broadcaster may immediately forget that they started it.

However, extending Amnesiac Flooding and other stateless flooding algorithms (such as those proposed in~\cite{turau2020stateless,turau2021synchronous,bayramzadeh2021weak}) beyond synchronous fault-free scenarios is challenging. This is due to the fragility of these algorithms and their inability to build in complex fault-tolerance due to the absence of state and longer term memory. It has subsequently been shown that no stateless flooding protocol can terminate under moderate asynchrony, unless it is allowed to perpetually modify a super-constant number (i.e.~$\omega(1)$) of bits in each message~\cite{turau2020stateless}. Yet, given the fundamental role of broadcast in distributed computing, the resilience of these protocols is extremely important even on synchronous networks.

%However, 
Outside of a partial robustness to crash failures, the fault sensitivity of Amnesiac Flooding under synchrony has not been explored in the literature. This omission is further compounded by the use of Amnesiac Flooding as an underlying subroutine for the construction of other broadcast protocols.
In particular, multiple attempts have been made to extend Amnesiac Flooding to new settings (for example routing multiple concurrent broadcasts \cite{bayramzadeh2021weak} or flooding networks without guaranteed edge availability \cite{turau2021synchronous}), while maintaining its desirable properties. 
However, none have been entirely successful, typically requiring some state-fulness.
It has not in fact been established that \textit{any other} protocol can retain all of Amnesiac Flooding's remarkable properties even in its original setting.
These gaps stem fundamentally from the currently limited knowledge of the dynamics of Amnesiac Flooding beyond the fact of its termination and its speed to do so.
In particular, both of the existing techniques (parity arguments such as in~\cite{hussak2023termination} or auxiliary graph constructions such as in~\cite{turau2021amnesiac}) used to obtain termination results for Amnesiac Flooding are unable to consider faulty executions of the protocol and fail to capture the underlying structures driving terminating behaviour.\\

We address these gaps through the application of novel analysis and by considering the structural properties of Amnesiac Flooding directly. 
By considering the sequence of message configurations, we are able to identify the structures underlying Amnesiac Flooding's termination and use these to reason about the algorithm in arbitrary configurations.
The resulting dichotomy gives a comprehensive and structured understanding of termination in Amnesiac Flooding.
For example, we apply this to investigate the sensitivity of Amnesiac Flooding with respect to several forms of fault and find it to be quite fragile.
Furthermore, we show that under reasonable assumptions on the properties of a synchronous network, any strictly stateless deterministic terminating broadcast algorithm oblivious to the content of messages, must produce the exact same sequence of message configurations as Amnesiac Flooding on any network from any initiator.
We therefore argue that Amnesiac Flooding is unique.
However, we show that if any of these restrictions are relaxed, even slightly, distinct terminating broadcast algorithms can be obtained.
% We propose several algorithms for these relaxed cases, each adapting Amnesiac Flooding to exploit the relaxation and produce distinct behaviour.
As a result of this uniqueness and simplicity, we argue that Amnesiac Flooding represents a prototypical broadcast algorithm.
This leaves open the natural question: do there exist fundamental stateless algorithms underlying solutions to other canonical distributed network problems?
% Given the uniqueness of Amnesiac Flooding and the fact that our proposed algorithms all make use of it as a core, this leaves open the question of whether there a fundamentally different approach is even possible.

% It has not in fact been established that \textit{any other} algorithm can retain all of Amnesiac Flooding's remarkable properties even in its original setting.
% These gaps stem fundamentally from the currently limited knowledge of the dynamics of Amnesiac Flooding beyond the fact of its termination and its speed to do so.
% In particular, both of the existing techniques: \anew{(parity arguments such as in~\cite{hussak2023termination} or auxillary graph constructions such as in~\cite{turau2021amnesiac})} used to obtain termination results for Amnesiac Flooding  are unable to consider faulty executions of the protocol and fail to capture the underlying structures driving terminating behaviour. 

% \anew{In this work, we present a fresh approach by studying the underlying structural properties of Amnesiac Flooding itself. This leads us to derive the surprising result that under the reasonable additional condition (beyond the standard assumptions of a fault-free synchronous model) of the algorithm being oblivious to the message content, Amnesiac Flooding is the only strictly stateless deterministic algorithm that can achieve terminating broadcast. Stating our four conditions explicitly, we derive four additional truly stateless algorithms on minimal relaxation of the conditions. For example, we introduce \emph{Random Flooding} which has each node randomly choose in every round between Amnesiac Flooding and total flooding to all neighbours. This can be shown to successfully implement terminating broadcast with probability $1$ but with potentially exponential termination time.  Another protocol which we call \emph{Parrot Flooding} needs a single readable bit in the messages to be terminating from all sources. This landscape of additional protocols needs to be further explored but all the protocols we could devise build on Amnesiac Flooding leaving open the question of whether there is a fundamentally different approach possible.

% \anew{However, 
% extending Amnesiac Flooding and other stateless Flooding algorithms (such as those proposed in~\cite{turau2020stateless,turau2021synchronous,bayramzadeh2021weak}) beyond synchronous fault-free scenarios is challenging. This is due to the fragility of these algorithms and their inability to build in complex fault-tolerance due to the absence of state and longer term memory.}
% %are particularly practical. This is due in large part to their extreme fragility; 
% Amnesiac Flooding has been shown to not terminate under even a weak form of asynchrony~\cite{hussak2023termination}. \anew{Subsequently, it has also been shown that no stateless flooding algorithm can terminate under moderate asynchrony, unless it is allowed to perpetually modify a super-constant number (i.e.~$\omega(1)$) of bits in each message~\cite{turau2020stateless}. Yet, given the fundamental role of broadcast in distributed computing, understanding the resilience of these algorithms is extremely important even on synchronous networks.}


% %In practice, any fault sensitive broadcast algorithm has only limited practical utility. However, 
% Outside of a partial robustness to crash failures, the fault sensitivity of Amnesiac Flooding \anew{under synchrony seems to be} %as remained
% largely \anew{unexplored}. 
% %unaddressed 
% %in the literature. This omission is further aggravated by the use of 
% \anew{Since Amnesiac Flooding is already being used as an underlying subroutine for the construction of other broadcast algorithms, we believe a systematic study of fault sensitivity and the structural properties of Amnesiac Flooding will inform further understanding and applicability of these algorithms}.
% In particular, multiple attempts have been made to extend Amnesiac Flooding to new settings (for example routing multiple concurrent broadcasts \anew{\cite{HT-AFCases-Arxiv20,bayramzadeh2021weak}} or flooding networks without guaranteed edge availability \cite{turau2021synchronous}), while maintaining its desirable properties. 
% However, none have been entirely successful, typically requiring some state-fulness.

 % }

%\ha{Talk about the importance of related protocols for a few sentences}
% 


\subsection{Our Contributions}

%Given the fragility of both Amnesiac Flooding and the Stateless Flooding protocols proposed in~\cite{turau2020stateless}, 
In this work,
%we attempt a fresh approach studying the underlying structural requirements for success of Amnesiac Flooding while restricting the power of the algorithm/model.
we investigate the existence of other protocols possessing the following four desirable properties of Amnesiac Flooding:
\begin{enumerate}
  %  \item \StrongTrueStatelessness: 
    \item \anew{\StrictStatelessness:}
    Nodes maintain no information other than their port labellings between rounds. This includes whether or not they were in the initiator set.
    \item \Obliviousness: Routing decisions may not depend on the contents of received messages.
    \item \Determinism: All decisions made by a node must be deterministic.
    \item \Bandwidth: Each node may send at most one message per edge per round.
\end{enumerate}

Our main technical results regarding the existence of alternative protocols to Amnesiac Flooding are given in the next two theorems (reworded in \cref{sec: Uniqueness,Relaxing}).

\begin{*theorem}[Uniqueness of Amnesiac Flooding]
    Any terminating broadcast algorithm possessing all of \StrictStatelessness, \Obliviousness, \Determinism\ and \Bandwidth{} behaves identically to Amnesiac Flooding on all graphs under all valid labellings.
\end{*theorem}
Note that the last theorem allows, but does not require, that nodes have access to unique identifiers labelling themselves and their ports. 
However, we enforce the condition that these identifiers, should they exist, may be drawn adversarially from some super set of $[n+\kappa]$ where $n$ is the number of nodes on the networks and $\kappa=R(9,8)$ where $R(9,8)$ is a Ramsey Number.
It is important to stress here that this result holds even if the space of unique identifiers is only greater than $n$ by an additive constant.
In contrast to the last theorem, the next one does not assume that agents have access to unique identifiers.
\begin{*theorem}[Existence of relaxed Algorithms]
    There exist terminating broadcast algorithms which behave distinctly from Amnesiac Flooding on infinitely many networks possessing any three of: \anew{\StrictStatelessness}, \Obliviousness, \Determinism\ and \Bandwidth.
\end{*theorem}

\anew{We derive four of these relaxed algorithms which all build upon Amnesiac Flooding:
\begin{enumerate}
    \item \anew{\textsc{Neighbourhood-2 Flooding:} \StrictStatelessness\  is relaxed.}
    \item \anew{\textsc{Random Flooding:} \Determinism\  is relaxed.}
    \item \textsc{1-Bit Flooding (Message Dependent)}: \Obliviousness\  is relaxed.
    \item \anew{\textsc{1-Bit Flooding (High Bandwidth):} \Bandwidth\ is relaxed.}
\end{enumerate}

}
We note that despite being a non-deterministic algorithm \textsc{Random Flooding} achieves broadcast with certainty and terminates almost surely in finite time.\\

\anew{We also} perform a comprehensive investigation of the fault sensitivity of Amnesiac Flooding in a synchronous setting. 
Through the use of a method of invariants, we obtain much stronger characterizations of termination than were previously known, \anew{for} both Amnesiac Flooding, and a \anew{subsequently} proposed Stateless Flooding protocol~\cite{turau2020stateless}.  This allows us to provide precise characterizations of the behaviour of Amnesiac Flooding under the loss of single messages, uni-directional link failure, and time bounded Byzantine failures. 
The above invariants may be of independent interest, beyond fault sensitivity, as they provide strong intuition for how asynchrony interferes with the termination of both Amnesiac Flooding and the Stateless Flooding proposed in~\cite{turau2020stateless}.



%Our first

\anew{The main technical result concerning fault sensitivity}  is a \textit{dichotomy} characterizing the configurations, starting from which, Amnesiac Flooding terminates. 
As the rigorous statement of the result requires some additional notation and terminology, we will state it only informally here. 
We show in \cref{thm:balance} that, whether a configuration terminates or not, is determined exclusively by the parity of messages distributed around cycles and the so-called faux-even cycles (FEC). 
This result implies the following three theorems which demonstrate the fragility of Amnesiac Flooding under three increasingly strong forms of fault.  We give our fault model explicitly in \cref{sec:faults}.
    \begin{*theorem}[Single Message Failure]
    \label{thm:message drop}
        If single-source Amnesiac Flooding experiences a \emph{single message drop failure} for the message $(u,v)$ then it fails to 
       terminate.
        if either:
        \begin{itemize}
            \item $uv$ is not a bridge
            \item $uv$ lies on a FEC subgraph
        \end{itemize}
        Moreover, it fails to broadcast if and only if this is the first message sent along $uv$, $uv$ is a bridge, and the side of the cut containing $u$ does not contain an odd cycle.
    \end{*theorem}
\anew{We, also, consider the possibility of a link/edge failing in one direction (or if we consider an undirected edge as two directed links in opposite directions, only one of the directed links fails).}
    \begin{*theorem}[Uni-directional link failure]
    \label{thm: directed}
        For any graph $G=(V,E)$ and any initiator set $I\subsetneq V$ there exists an edge $e \in E$ such that a uni-directional link failure at $e$ will cause Amnesiac Flooding to either fail to broadcast or fail to terminate when initiated from $I$ on $G$. Furthermore, for any non-empty set of uni-directional link failures there exists $v \in V$ such that, when Amnesiac Flooding is initiated at $v$, it will either fail to broadcast or fail to terminate.
    \end{*theorem}
\anew{We, finally, consider a set of Byzantine nodes, who know the original message, are free to collude among themselves and may decide to forward this message in any arbitrary pattern to their neighbours. However, for a discussion of termination to be meaningful, we require that the nodes have Byzantine behaviour for only a finite number of rounds.}
    \begin{*theorem}[Byzantine Failure]\label{thm: Byzantine}
        If Amnesiac Flooding on $G=(V,E)$ initiated from $I\subsetneq V$ experiences a \emph{weak Byzantine failure} at $J\subseteq V\setminus I$, then the adversary can force:
        \begin{itemize}
            \item Failure to broadcast if and only if $J$ contains a cut vertex set.
            \item Non-termination if and only if at least one member of $J$ lies on either a cycle or FEC subgraph. 
        \end{itemize}
    \end{*theorem}
Two natural corollaries of the Single Message Failure Theorem we wish to highlight here are: (i) on any network from any initial node there exists a single message, the dropping of which, will produce either non-termination or non-broadcast and (ii) dropping any message on a bipartite network will cause either non-termination or non-broadcast.
Similarly, the Byzantine Failure Theorem implies that any Byzantine set containing a non-leaf node on any network can force either non-termination or non-broadcast for Amnesiac Flooding with any initial set.\\

% Given the fragility of both Amnesiac Flooding and the Stateless Flooding protocols proposed in~\cite{turau2020stateless}, we also investigate the existence of other protocols possessing the following four desirable properties of Amnesiac Flooding:
% \begin{enumerate}
%     \item \StrongTrueStatelessness: Nodes maintain no information other than their port labellings between rounds. This includes whether or not they were in the initiator set.
%     \item \Obliviousness: Routing decisions may not depend on the contents of received messages.
%     \item \Determinism: All decisions made by a node must be deterministic.
%     \item \Bandwidth: Each node may send at most one message per edge per round.
% \end{enumerate}


% Our main technical results regarding the existence of alternative protocols to Amnesiac Flooding are given the next two theorems (reworded in \cref{Uniqueness,Relaxing}).

% \begin{*theorem}[Uniqueness of Amnesiac Flooding]
%     Any terminating broadcast algorithm possessing all of Strong True Statelessness, Obliviousness, Determinism and Singular Bandwidth behaves identically to Amnesiac Flooding on all graphs under all valid labellings.
% \end{*theorem}
% Note that the last theorem allows nodes to have access to unique identifiers labelling themselves and their ports which may be drawn from any super set of $[n+\kappa]$ where $n$ is the number of nodes on the networks and $\kappa>R(9,8)$ where $R(9,8)$ is a Ramsey Number.
% It is important to stress here that this result holds even if the space of unique identifiers is only greater than $n$ by an additive constant.
% In contrast to the last theorem, the next one does not assume that agents have access to unique identifiers.
% \begin{*theorem}[Existence of relaxed Algorithms]
%     There exist terminating broadcast algorithms which behave distinctly from Amnesiac Flooding on infinitely many networks possessing any three of: \StrongTrueStatelessness, \Obliviousness, \Determinism and \Bandwidth.
% \end{*theorem}


%On the other hand, Amnesiac Flooding is known to be extremely fragile and depends heavily on the synchrony of the communication model. For example, non-termination can occur easily if only a single message is delayed.

%In this work, we show that up to reasonable assumptions on a synchronous network of distributed agents, Amnesiac Flooding uniquely possesses the aforementioned desirable properties, even if only finite termination time is required. Furthermore, this holds even if agents are given knowledge of their own and their neighbours' unique ids.  We additionally investigate the fault tolerance of Amnesiac Flooding in a synchronous setting, via the dropping of messages. On bipartite graphs, we demonstrate that the dropping of a single message will always lead to either a failure to broadcast or terminate, as well as providing precise conditions for failure on non-bipartite graphs. Counterintuitively, these conditions are more easily met on highly-connected graphs. Further, by a method of conserved quantities, we provide exact conditions for the dropping of multiple messages to produce non-termination on any graph and in doing so capture the fundamental mechanism determining (non-)termination of Amnesiac Flooding from any configuration of messages. This underscores the centrality of parity to the termination of Amnesiac Flooding, and provides greater evidence to the impossibility of a similar result for asynchronous settings. However, finally and positively, we show that even slightly relaxing the above conditions allows for the existence of other stateless broadcast protocols. In particular, permitting even a single bit of readable information on the messages or allowing the initial source to remember this fact.