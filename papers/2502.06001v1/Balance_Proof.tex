\section{Proof of theorem \ref{thm:balance}}
    \label{apx: balance}
    In this appendix we prove theorem \cref{thm:balance}. We begin by establishing that (im)balance is preserved by the operation of Amnesiac Flooding,
    \begin{lemma}
        For any set of initiators $I\subseteq V$ and configuration $S$ on $G=(V,E)$, $A_{I,G}(S)$ is balanced if and only if $S$ is balanced.
    \end{lemma}
    \begin{proof}
        We will show this for each subgraph type in the tetrachotomy of definition \ref{def: balance}. The first case is trivial. 
        
        Then, let $H$ be an odd cycle. If the number of messages on $H$ in $A_G(S)$ increases, this can only be due to a node $u$ on $H$ receiving a message from $v$ outside of $H$. In this case, if $u$ already received a message on $H$ it will have no effect, otherwise it will introduce two new messages, one clockwise and one anti-clockwise. If the number of messages decreases this means that two messages collided on $H$ removing both, this removes one clockwise and one anti-clockwise. Therefore, the balance on $H$ does not change.

        Now let $H$ be an even cycle. The parity of the distance between messages is preserved on an even cycle when $A$ is applied. Therefore, changing the balance would require the introduction or removal of messages. The argument is then the same as in the odd cycle case, with the caveat that any pair of messages added (removed) by the external activation of (collision at) a node have the same parity of distance between their heads and any other message. Therefore, the balance on $H$ is preserved.

        Finally, let $H$ be a faux-even cycle. If no nodes of $H$ receive a message from outside of $H$, then the even cycle configuration $A_H(S_H)_{2,H}$ is the same as $A_{H_2}(S_{2,H})$, unless there is a message from a cycle onto a shared node with the path and no other messages arriving at that node. This case is equivalent to an external activation of the node on the other node in $H_2$ associated with the shared path node. If new messages are added to $H$ from outside, these will produce either $2$ or $4$ messages on $H_2$ in the even cycle configuration, each pair in opposite directions and the same parity of distance apart. Therefore, the balance on $H_2$ is preserved.
    \end{proof}
    This gives the following immediate corollary, as the empty configuration is trivially balanced.
    \begin{corollary}
    \label{corr: Forward}
        If $S$ is imbalanced on $G=(V,E)$, then for all $k>0$, $A^k_G(S)\neq\emptyset$.
    \end{corollary}
    This gives us the forward direction of \cref{thm:balance}. For the other direction, we need the notion of message paths and their recurrence.
    \begin{definition}
    \label{def: message paths}
    A message $m=(v_0,v_1)$ in configuration $S\subset V^2$ has a \emph{message path} $v_0v_1$.
    We define the rest of its paths recursively, i.e. $m=(v_0,v_1)$ has a message path $v_0...v_{k+1}$ from $S$ on $G$ if:
    \begin{itemize}
    \item 
        \item $v_0v_1...v_{k}$ is a message path of $m$ in $S$
        \item The message $(v_k,v_{k+1})$ exists in $A_G^k(S)$
    \end{itemize}
     We say that a message $m=(v_0,v_1)$ is \emph{recurrent} on $G=(V,E)$ from $S$ if $m$ has a message path of the form $v_0v_1...v_0v_1$ on $G$ from $S$.
    \end{definition}
    We obtain the following property relating message paths and termination immediately.
    \begin{lemma}
    \label{lemma: recurrence}
        Let $S$ be a non-empty configuration on $G=(V,E)$ such that $A_{G}^k(S)=S$, then $S$ contains a recurrent message on $G$.
    \end{lemma}
    \begin{proof}
        If $m$ is a message in $A_{G}^k(S)$ there must exist a message $m'$ in $S$, such that there is a message path from $m'$ to $m$. By the same argument there must exist another message $m''$ such that there is a message path from $m''$ to $m'$ and therefore to $m$. However, this cannot extend back infinitely with unique messages as there is a finite number of messages in $S$. Thus, there must exist some repeated message in the sequence, which in turn is by definition recurrent.
    \end{proof}
    The crucial observation is that the number of consecutive steps that can be spent on certain subgraphs by a message path is bounded if beginning from a balanced configuration. We will need four such results.
    \begin{lemma}
    Let $G=(V,E)$ with $H \subseteq G$ an odd cycle of length $2k+1$. If $m$ is a message of $S$ on $H$ with a message path of length $2k+2$ restricted only to $H$, then $S$ is imbalanced on $G$.
        \label{lemma: odd cycles}
\end{lemma}
\begin{proof}
    For the sake of contradiction assume that $S$ is balanced on $G$ and $m\in S$ has a message path from $S$ of length $2k+2$ restricted only to $H$. 
    Without loss of generality, assume that $m$ is travelling clockwise.
    On a cycle a message can collide with exactly one other message, upon doing which all of its message paths must terminate. 
    If there is a message $m'$ on $H$ travelling anti-clockwise in $S$, then either it collides with $m$ within $2k$ steps or it collides with another message travelling clockwise first. 
    Since $S$ is balanced on $G$ there must be an equal number of messages travelling clockwise and anti-clockwise on $H$, and a collision removes one of each.
    Therefore, unless new messages are added to $H$ in subsequent steps, $m$ must collide with a message $m'$ travelling anti-clockwise before it takes $2k+1$ steps.
    Thus, new messages must be added to $H$ that collide with the message $m$ would otherwise collide with. 
    The introduction of a new message to $H$ must produce one message travelling clockwise and one travelling anti-clockwise, $m_c$ and $m_a$ respectively. 
    Thus, $m_c$ must collide with $m'$ before $m$ does and so must be introduced between $m$ and $m'$. 
    However, this means that $m$ will collide with $m_a$ before it would have reached $m'$ giving our contradiction.
\end{proof}
\begin{lemma}
    Let $G=(V,E)$ with $H \subseteq G$ an even cycle of length $2k$. If $m$ is a message of $S$ on $H$ with a message path of length $k+1$ restricted only to $H$, then $S$ is imbalanced on $G$.
    \label{lemma: even cycles}
\end{lemma}
\begin{proof}
    The argument is essentially the same as for the odd case with the caveat that in messages with an even distance between their heads travelling in opposite directions collide in at most $k$ steps on an even cycle. 
\end{proof}
\begin{lemma}
    Let $G=(v,E)$ with $H\subseteq G$ be a $FEC_{x,y,z}$. If $m$ is a message of $S$ on $H$ with a message path of length $x+y+z+2$ restricted only to $H$ but not only to one of its cycles, then $S$ is imbalanced on $G$.
    \label{lemma: FEC}
\end{lemma}
\begin{proof}
    For the sake of contradiction assume that $S$ is balanced on $G$ and such a message path exists. Then consider, a path of length $x+y+z+2$. Since $A_{H}(S)_{2,H}$ can only have additional messages relative to $A_{H_2}(S)$, a message path must deviate from $H_2$ in order for the Lemma \ref{lemma: even cycles} not to apply. Thus, such a message path must either remain on a cycle upon crossing the path point, or cross from one cycle onto the path and then to the other cycle in the wrong direction. In the former case this violates Lemma \ref{lemma: odd cycles}. In the latter case we can simply take the other even cycle representation and the result follows from Lemma \ref{lemma: even cycles}.
\end{proof}
\begin{lemma}
    Let $G=(V,E)$ with $H\subseteq G$ an even cycle of length $2k$ and let $m \in S$ be a message on $H$ where $S$ is balanced on $G$. If $m$ has a message path restricted to $H$ on $S$ of length $k$, then there exists $m'$ in $S$ on $H$ such that $m$ and $m'$ have the same start point and $m'$ also has a message path of length $k$ restricted to $H$.
    \label{lemma: EvenSub}
\end{lemma}
\begin{proof}
    Since $S$ is balanced on $G$ there must exist a message colliding with $m$ on $H$ exactly $k$ steps after $S$. Therefore, either this message exists in $S$ or is added to $H$ after $S$. If it is added after $S$, then there must already have existed a message travelling counter to $m$ in $S$ which would have collided with $m$ sooner. Further, any new message added to block this one would collide with $m$ even sooner. Iterating this argument, the statement holds.
\end{proof}
This gives us the wedge we will need to obtain our main result, which we now reframe according to Lemma \ref{lemma: recurrence}.
\begin{theorem}
    Let $S$ be a configuration on $G$, $S$ is imbalanced if and only if it contains a recurrent message.
\end{theorem}
The forward direction is an immediate consequence of Corollary \ref{corr: Forward}. For the reverse we take $G=(V,E)$ to be a communication graph and $S \subseteq V^2$ to be a balanced configuration of messages. For contradiction we assume that $S$ contains a recurrent message $m$ which has a message path $W$ performing exactly one excursion and return to $m$. We will view $W$ as both a walk on $G$ and a word. Further, $W$ must obey the Conditions of the lemmas defined previously re-framed here as:
\begin{enumerate}
    \item $W$ cannot return to the node it just came from, i.e. no sequence of the form $uvu$. \label{cond: No leaves}
    \item $W$ cannot take $2x+2$ consecutive steps around an odd cycle of length $2x+1$ (Lemma \ref{lemma: odd cycles}). \label{cond: odd cycle}
    \item $W$ cannot take $x+1$ consecutive steps around an even cycle of length $2x$ (Lemma \ref{lemma: even cycles}). \label{cond: even cycle}
    \item $W$ cannot take more than $x+y+z+2$ steps on an $FEC$ unless it sticks purely to one cycle (Lemma \ref{lemma: FEC}) .\label{cond: FEC}
    \item If $W$ takes $x$ steps around an even cycle of length $2x$, then there exists $W'$ which takes the opposite path of equal length and is otherwise identical (Lemma \ref{lemma: EvenSub}). \label{cond: even replacement}
\end{enumerate}
The fifth of these has the following useful interpretation, if the existence of $W'$ implies imbalance then the existence of $W$ implies imbalance. Therefore, we will use this rule as a substitution allowing us to "modify" $W$ to take the alternate path. We can now begin in earnest:
\begin{lemma}[Repetition Lemma]
    If $W$ contains a factor $u...u$ then that factor contains an odd cycle as a subfactor.
\end{lemma}
\begin{proof}
    By Condition \ref{cond: No leaves}, we cannot return immediately to the same node. If the whole factor is not a cycle we must return to a node before $u$ and so we instead consider the innermost cycle. This cycle must be consecutive and traversed fully, thus for $S$ to be balanced this cycle must be odd by Condition \ref{cond: even cycle}.
\end{proof}
\begin{lemma}[Odd Cycle Lemma]
\label{lemma: Odd Cycle Lemma}
    There can only be one factor of $W$ that forms an odd cycle.
\end{lemma}
\begin{proof}
    Assume there are at least two for contradiction. We take the first two cycles of odd length in $W$, they are either sequential ($u...u...v...v$), interleaved ($u...v...u...v$) or repetitive ($u...u...u$). In the sequential case this is a fully traversed $FEC$ and so in violation of Condition $4$. In the interleaved case, we have three subsequences $w_1,w_2,w_3$ of length $a,b,c$ respectively such that $a+b$ is odd and $b+c$ is odd (see Figure \ref{fig: Odd Cycle Lemma Diagram}). Thus, $a+c$ is even. Since $w_1$ and $w_3$ induce an even cycle, either $a=c$ or $W$ is in violation of Condition $3$. Therefore, we have that $W$ travels exactly half way around an even cycle along $w_3$. Thus there must exist a message path $W'$ that contains the factor $w_1w_2w_1$ which is more than $a+b$ steps around an odd cycle of length $a+b$ and so is in violation of Condition $2$. In the repetitive case, this is multiple traversals of an odd cycle in full and so also in violation of Condition $2$. Thus, there exists at most one odd cycle in $W$.
\end{proof}
\begin{figure}
    \centering
    \includegraphics[width=0.6\linewidth]{images/OddCycleTwist.pdf}
    \caption{An illustration of the walk described in the interleaved case in the proof of \ref{lemma: Odd Cycle Lemma}}
    \label{fig: Odd Cycle Lemma Diagram}
\end{figure}
This immediately gives the following results:
\begin{lemma}
   Every node appears at most twice in $W$. \label{lemma: Three is a magic number}
\end{lemma}
\begin{lemma}
    There exists at most one node that is both a member of a consecutive odd cycle and appears twice. Furthermore, such a node must be the start and end of the cycle.
\end{lemma}
Therefore, there must exist a cycle $C$ containing $m$ such that $W$ fully traverses $C$ before returning to $m$, with possible excursions.
To be clear, there exists a sequence of pairs $(w_0,w_1),(w_1,w_2)...(w_k,w_0),(w_0,w_1)$ such that $C=w_0w_1..w_kw_0$ is a cycle of $G$, each pair appears in $W$ in consecutive order (i.e. $W=w_0w_1...w_1w_2...w_kw_0..w_0w_1$) and $m=(w_0,w_1)$. 
With this established we are now ready to prove \cref{thm:balance}.
We demonstrate that no matter how we construct $W$ it must take too many steps around some structure and therefore cannot exist.
\begin{proof}[Proof of \cref{thm:balance}]
    ~\paragraph{Claim: $C$ is an odd cycle. }
    If $C$ is of even length (say $2k$), then there must be an excursion from $C$ otherwise Condition $3$ would be violated. However, there can be at most one such excursion as it must contain a consecutive odd cycle and so the two subsequences on either side must be factors of $W$. Therefore, since some subsequence of $W$ traverses $C$ fully with one additional step, one of the two factors must take $k+1$ steps around $C$. This violates Condition $3$ and so $C$ must be an odd cycle.

    ~\paragraph{Claim: $W$ consists of two odd cycles $C$ and $\hat{C}$ connected by a path. }
    If $C$ is an odd cycle there must be an excursion from it or Condition $2$ would be violated. This excursion must contain exactly one consecutive odd cycle which we denote by $\hat{C}$. If $\hat{C}$ does not share its starting node with $C$, $W$ either forms a path between $C$ and $\hat{C}$ or some chain of cycles. We can use Condition $5$ to eliminate all even cycles of this chain, then if there are any odd cycles in the chain they correspond to a fully traversed $FEC$ when paired with $\hat{C}$ and so violate Condition $4$. Thus, we have a $W$ that takes a simple path from $C$ to $\hat{C}$ and back, although possibly intersecting $C$ along the way.
    \paragraph{Claim: Neither $C$ nor $\hat{C}$ intersect with the path between them. }
    Assume that the path to $\hat{C}$ does intersect $C$, since we are taking the same path in both directions any node shared between the path and $C$ appear in $W$ three times. 
    This violates Lemma \ref{lemma: Three is a magic number} and so $C$ must be disjoint from the path to $\hat{C}$.
    Similarly $\hat{C}$ must be disjoint from the path otherwise it would violate the same lemma.
    ~\paragraph{Claim: $C$ and $\hat{C}$ intersect.}
    If $C$ is disjoint from $\hat{C}$ the pair would form a fully traversed $FEC$, thereby violating Condition $4$. Thus, $C$ and $\hat{C}$ must intersect.
    ~\paragraph{Claim: $C$ and $\hat{C}$ do not intersect. }
    Let $W=uv..wx_1,...,x_kw..uv$ where $C=u...w...uv$ and the excursion to $\hat{C}$ is given by $wx_1...x_kw$.
    Now assume that $\hat{C}$ contains a node from $C$ which occurs strictly before $w$ in $W$. This node is on a consecutive odd cycle and appears twice.
    There must exist a latest such node in the ordering of $C$, we denote it $y$.
    Since $y$ is on a consecutive odd cycle and appears twice in $W$ it must be the start and end point of $\hat{C}$. 
    However, then either $y=w$ which is impossible or $y$ appears three times, violating Lemma \ref{lemma: Three is a magic number}. 
    The same argument holds taking the earliest node shared by $C$ and $\hat{C}$ strictly after $w$. Thus, the only node that can be shared by $C$ and $\hat{C}$ is $W$, implying that $W$ forms two odd cycles sharing a single node.
    However, this is an $FEC$ which is fully traversed and so violates Condition $4$.\\
    Thus, $W$ cannot exist and so there can be no recurrent message in $S$.
\end{proof}
We can now immediately obtain a proof of \cref{corr: balance is fast}.
\begin{proof}
    If there does not exist $k\leq 2|E|$ such that $A^k(S)=\emptyset$ then there must be a message path of length $2|E|+1$ for some message in $S$.
    By the pigeon hole principle this message path must visit the same edge twice travelling in the same direction.
    Thus, this requires a recurrent message.
    Therefore, $S$ cannot be balanced.
\end{proof}