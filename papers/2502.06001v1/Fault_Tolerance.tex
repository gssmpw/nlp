\section{Fault Sensitivity}
\label{sec: faults}
    In order to consider the fault sensitivity of Amnesiac Flooding, we need to be able to determine its behaviour outside of correct broadcasts. 
    Unfortunately, neither of the existing termination proofs naturally extend to the case of arbitrary message configurations.
    Fortunately, we can derive an invariant property of message configurations when restricted to subgraphs that exactly captures non-termination, which we will call "balance". 
    \begin{theorem}
        \label{thm:balance}
        Let $S$ be a configuration on $G=(V,E)$ then there exists $k\geq 0$ such that $A^{k}_{G}(S)=\emptyset$ if and only if $S$ is balanced on $G$.
    \end{theorem}
    In fact we obtain that not only do balanced configurations terminate, they terminate quickly.
    \begin{corollary}
    \label{corr: balance is fast}
        Let $S$ be a balanced configuration on $G=(V,E)$ then there exists $k\leq 2|E|$ such that $A^{k}_G(S)=\emptyset$.
    \end{corollary}
    Intuitively, for the protocol not to terminate, we require that a message is passed around forever and since it is impossible for a message to be passed back from a leaf node the message must traverse either a cycle or system of interconnected cycles. 
    As we will demonstrate in the rest of the section we need only consider systems of at most two cycles. Specifically, we can introduce an invariant property determined by parity constraints on the number of messages travelling in each direction and their spacing around: odd-cycles, even cycles and what we will dub, faux-even cycles.
    \begin{definition}
        A \emph{faux-even cycle} (FEC) is a graph comprised of either two node disjoint odd cycles connected by a path or two odd cycles sharing only a single node. We denote by $FEC_{x,y,z}$ the $FEC$ with one cycle of length $2x+1$, one of length $2z+1$ and a path containing $y$ edges between them.
        We emphasize that if $y=0$ the two cycles share a common node and if $y=1$ the two cycles are connected by a single edge.
    \end{definition}
    FECs get their name from behaving like even cycles with respect to the operator $A$. In order to capture this we can perform a transformation to convert them into an equivalent even cycle.
    \begin{definition}
        Let $F=(V,E)$ be $FEC_{x,y,z}$, the \emph{even cycle representation} of $F$ denoted $F_2$ is the graph constructed by splitting the end points of the interconnecting path in two, and duplicating the path to produce an even cycle. Formally, if the two cycles are of the form $a_0...a_{2x}a_0$ and $c_0...c_{2z}c_0$, with \anew{$a_0$ and $c_0$ connected by the path} $b_1...b_{y-1}$, we construct the following large even cycle from four paths $$a_0...a_{2x}a_{-1}d_1...d_{y-1}c_{-1}c_{2z}...c_0b_{y-1}...b_1a_0.$$ Here $a_{-1}$ is a copy of $a_0$, $c_{-1}$ is a copy of $c_0$, \anew{and the path $d_1...d_{y-1}$ is a copy of the path $b_1...b_{y-1}$} (See figure \ref{fig: Even Cycle Representation}). Note that if $y=0$, $a_0=c_0$ and so we do not include any nodes from $b$ or $d$. Similarly, if $y=1$, $a_0$ and $c_0$ are connected by a single edge, as are $a_{-1}$ and $c_{-1}$.  
    \end{definition}
    \begin{figure}
        \centering
        %\includegraphics[width=0.5\linewidth]{images/Even Cycle Representation.pdf}
        \includegraphics[scale=0.25]{images/Even_Cycle_Representation.pdf}
        \caption{Left: An $FEC_{x,y,z}$. Right: The corresponding even cycle representation. Please note that this depiction only holds for $y\geq 2$. For $y=1$: $a_0$ and $c_0$ are connected directly by an edge in both sub figures (as are $a_{-1}$ and $c_{-1}$). For $y=0$: $a_0=c_0$ and $a_{-1}=c_{-1}$. }
        \label{fig: Even Cycle Representation}
    \end{figure}
    \begin{definition}
        Let $F=(V,E)$ be $FEC_{x,y,z}$ and $S$ a configuration of Amnesiac Flooding on $F$, the \emph{even cycle representation} of $S$ on $F$ denoted $S_{2,F}$ is determined as follows. For each $m \in S$
    \begin{itemize}
        \item If $m=(a_{2x},a_0)$ (resp. $(a_0,a_{2x})$) we add $(a_{2x},a_{-1})$ (resp. $(a_{-1},a_{2x})$) to $S_{2,F}$.
        \item If $m=(b_{i},b_{j})$ for some $i,j \in \{1,...,y-1\}$, we add both $(b_{i},b_{j})$ and $(d_i,d_j)$ to $S_{2,F}$.
        \item If $m=(c_{2z},c_0)$ (resp. $(c_0,c_{2z})$) we add $(c_{2z},c_{-1})$ (resp. $(c_{-1},c_{2z})$) to $S_{2,F}$.
        \item If $m=(a_0, b_1)$ (resp. $(b_1,a_0)$) we add both $(a_0,b_1)$ and $(a_{-1},d_1)$ (resp. $(b_1,a_0)$ and $(d_1,a_{-1})$) to $S_{2,F}$.
        \item If $m=(c_0, b_{y-1})$ (resp. $(b_{y-1},c_0)$) we add both $(c_0,b_{y-1})$ and $(c_{-1},d_{y-1})$ (resp. $(b_{y-1},c_0)$ and $(d_{y-1},c_{-1})$) to $S_{2,F}$.
        \item If $m=(a_0,c_0)$ (resp. $(c_0,a_0)$) we add both $(a_0,c_0)$ and $(a_{-1},c_{-1})$ (resp. $(c_{0},a_{0})$ and $(c_{-1},a_{-1})$) to $S_{2F}$.
        \item Otherwise we add $m$ to $S_{2F}$.
    \end{itemize}
    \end{definition}
    With this established we can now define the notion of balance.
    \begin{definition}
    \label{def: balance}
        A configuration $S$ is \emph{balanced} on $G=(V,E)$ if for all subgraphs $H$ of $G$ one of the following holds:
        \begin{itemize}
            \item $H$ is not a cycle or FEC.
            \item $H$ is an odd cycle and $S_H$ contains an equal number of messages travelling clockwise and anti-clockwise on $H$.
            \item $H$ is an even cycle and for any given message $m$ in $S_H$, there is an equal number of messages travelling clockwise and anti-clockwise on $H$ such that their heads are an even distance from $m$'s.
            \item $H$ is an $FEC$ and $S_{2,H}$ is balanced on $H_2$.
        \end{itemize}
    \end{definition}
    With these definitions established, we can present the intuition behind the proof of \cref{thm:balance}.
    \begin{proof}[Sketch of the proof of \cref{thm:balance}]
    We first establish that balance is conserved by Amnesiac flooding and, as the empty configuration is balanced, Amnesiac Flooding cannot terminate from any imbalanced configuration.
    For Amnesiac Flooding not to terminate it requires that some message travels around the communication graph and returns to the same edge, in the same direction. We show that if a configuration is balanced, the trajectory of any message can spend only a bounded number of consecutive steps on any given cycle or FEC. These results allow us to show that no message's trajectory that includes the same edge in the same direction twice could have begun from a balanced configuration. Thus, Amnesiac Flooding started from any balanced configuration must terminate.
    The full proof is given in appendix \ref{apx: balance}.
    \end{proof}
    \subsection{Consequences of Theorem~\ref{thm:balance}}
    \label{sec:faults}
    In this work we consider three key forms of fault of increasing severity: message dropping, uni-directional link failure and weak-Byzantine failures. Intuitively, these correspond to a set of messages failing to send in a specific round, a link failing in one direction creating a directed edge and a set of nodes becoming transiently controlled by an adversary.

    More precisely, let $\mathbf{S}=(S_{i})_{i \in \mathbb{N}}$ be the sequence of actual message configurations on our network. 
    We say that $\mathbf{S}$ is \emph{fault free} for $G=(V,E)$ \textbf{if} $S_{i+1}=A_{G}(S_{i})$ for all $i \in \mathbb{N}$.
    Otherwise, we say it experienced a \emph{fault}. We say $\mathbf{S}$ has suffered from,
     
    \begin{itemize}
        \item Message dropping, if there exists $T\subseteq V^2$ and $k\geq 1$ such that $S_{k+1}=A(S_k)\setminus T$ and for all $i\neq k$, $S_{i+1}=A(S_i)$. This corresponds to all messages in $T$ being dropped on round $k$.
        \item Uni-directional link failure, if there exists $X\subseteq V^2$ such that for all $i\geq 1$, $S_{i+1}=A(S_i)\setminus X$. This corresponds to all oriented links in $X$ failing.
        \item Weak-Byzantine, if there exists $Y \subseteq V$ such that for some $k$ at least twice the diameter, for all $i<k$, $S_{i+1}\setminus\{(u,v)|u \in Y\}=A(S_i)\setminus\{(u,v)|u \in Y\}$. This corresponds to a possible failure where an adversary determines the forwarding decisions of the nodes in $Y$ until round $k$.
    \end{itemize}
    Note that we refer to the Byzantine failures as weak, since they are only transient and only interfere with the forwarding of the message, not its content. It is obvious to see that in a stateless setting there is no way to deal with a Byzantine fault that changes the message as there is no way to verify which message is authentic. Intuitively, in our setting, Weak-Byzantine agents may choose to send messages to an arbitrary set of neighbours in each round and they are all controlled by a single coordinated adversary. We say that a Weak-Byzantine adversary with control of a given set of nodes can force some behaviour if there exists any weak byzantine failure on that set of nodes producing the forced behaviour. We can now express our fault sensitivity results, the proofs of which we defer to appendix \ref{apx: faults} and begin with an extreme case of single message dropping.
        \begin{*theorem}[Single Message Failure]
    \label{thm:message drop}
        If single-source Amnesiac Flooding experiences a \emph{single message drop failure} for the message $(u,v)$ then it fails to 
       terminate.
        if either:
        \begin{itemize}
            \item $uv$ is not a bridge
            \item $uv$ lies on a FEC subgraph
        \end{itemize}
        Moreover, it fails to broadcast if and only if this is the first message sent along $uv$, $uv$ is a bridge, and the side of the cut containing $u$ does not contain an odd cycle.
    \end{*theorem}
    Thus, Amnesiac Flooding is extremely non-fault tolerant with respect to message dropping.\\
    Secondly, considering uni-directional link failures we obtain the following.
        \begin{*theorem}[Uni-directional link failure]
    \label{thm: directed}
        For any graph $G=(V,E)$ and any initiator set $I\subsetneq V$ there exists an edge $e \in E$ such that a uni-directional link failure at $e$ will cause Amnesiac Flooding to either fail to broadcast or fail to terminate when initiated from $I$ on $G$. Furthermore, for any non-empty set of uni-directional link failures there exists $v \in V$ such that, when Amnesiac Flooding is initiated at $v$, it will either fail to broadcast or fail to terminate.
    \end{*theorem}
     Finally for the weak-Byzantine case.
        \begin{*theorem}[Byzantine Failure]\label{thm: Byzantine}
        If Amnesiac Flooding on $G=(V,E)$ initiated from $I\subsetneq V$ experiences a \emph{weak Byzantine failure} at $J\subseteq V\setminus I$, then the adversary can force:
        \begin{itemize}
            \item Failure to broadcast if and only if $J$ contains a cut vertex set.
            \item Non-termination if and only if at least one member of $J$ lies on either a cycle or FEC subgraph. 
        \end{itemize}
    \end{*theorem} 
