\section{Proof of \cref{thm: AF is unique}}
\label{apx: Uniqueness}
In this appendix we prove \cref{thm: AF is unique}.
We begin with the following immediate observations:
\begin{observation}
    \;
    \label{obs: basic uniqueness properties}
    \begin{enumerate}
        \item For all $\emptyset\neq S \subset \mathbb{N}$ and $u \in \mathbb{N}$, $b(u,S)\neq \emptyset$. Otherwise no new nodes will be informed after the first round, violating correctness.
        \item For $u, v, w \in \mathbb{N}$, $f(v,\{u,w\},\{u\})\in \{\{w\},\{u,w\}\}$. Otherwise if $v$ is a bridge on the graph the message can never reach the other side.
        \item For $u, v \in [\kappa]$, if $f(u,\{v\},\{v\})=v$ then $f(v,\{u\},\{u\})=\emptyset$. Otherwise $P$ will not terminate on the two node path labelled with $u$ and $v$.
        \item For $u,v,w \in [\kappa]$ if $f(u,\{v\},\{v\})=\{v\}$ and $f(w,\{v\},\{v\})=\{v\}$ then either $f(v,\{u,w\},\{u\})=\{u,w\}$, $f(v,\{u,w\},\{w\})=\{u,w\}$ or both. Additionally, $f(v,\{u,w\},\{u,w\})=\emptyset$. Otherwise the protocol would not terminate on the three node path with labels $uvw$. \label{obs:p3 behaviour}
    \end{enumerate}
\end{observation}
Our strategy will be to show the existence of a small set of labels which when restricted to $P$ behaves identically to Amnesiac Flooding for subcubic graphs. From there we can inductively argue that this holds for all labels up to degree $3$ and then in turn that this holds for all graphs. We begin with the following lemma, where for $u,v \in [\kappa]$, we denote $f(u,\{v\},\{v\})=\{v\}$ by $u>_*v$ for notational compactness, however please note that this relation is not transitive. 
\begin{lemma}
\label{lemma: ramsey}
    For any $c'$ if $\kappa$ is sufficiently large $\exists S \subseteq [\kappa]$ such that $\forall u,v \in S$ $f(u,\{v\},\{v\})=\emptyset$ and $|S|=c'$.
\end{lemma}
\begin{proof}
    We begin by considering $u,v,w,x,y,z \in [\kappa]$ such that $u>_*v<_*w$ and $x>_*y<_*z$. By  Observation \ref{obs: basic uniqueness properties} (4), wlog. we can assume that $f(v,\{u,w\},\{u\})=\{u,w\}$ and that $f(y,\{x,z\},\{z\})=\{x,z\}$. Now consider the path on 6 nodes labelled $uvwxyz$, as well as two paths on three nodes labeled $uvw$ and $xyz$. Since $u,v,y,z$ have the same neighbourhood on $P_6$ as on their corresponding $P_3$ they must be obeying the same policy. Thus, only $w$ and $x$ have a different policy. However, by exhaustive search it can be shown that there exists no protocol that will terminate from every node of the $P_6$, as well as each connected sub-graph. Thus there cannot exist such $u,v,w,x,y,z$. \\

    Consider the digraph $D$ with node set $V=[\kappa]$ and edges $\{(u,v)\in V\times V|u>_*v\}$, the set $S$ in the statement of the lemma corresponds to an independent set of size $c'$. The condition derived in the previous section implies that $D$ is $H$-subgraph free for $H$ the digraph in figure \ref{fig: ramsey}. There exists a $\kappa$ (explicitly $R(8,c')$) such that $D$ must contain either a tournament on $8$ nodes or an independent set on $c'$ nodes. In the latter case the statement holds, in the former consider that there is no way to orient the edges of $K_4$ such that no node has two predecessors. Therefore, if $D$ contains a tournament on $8$ vertices then it contains $H$ as a sub-digraph. Thus, the claim holds for $\kappa>R(8,c')$.
\end{proof}
\begin{figure}
    \centering
    \includegraphics[width=0.7\textwidth]{images/Forbidden_graph.pdf}
    \caption{The forbidden sub-digraph used in the argument of lemma \ref{lemma: ramsey}.}
    \label{fig: ramsey}
\end{figure}
With this established, we can strengthen this argument to obtain Amnesiac like behaviour for degree 2.
\begin{lemma}
    \label{lemma:degree 2}
    If there exists $S\subseteq [\kappa]$ such that $|S|>6$ where $\forall u,v \in S$ $u\not>_*v$, then there exists $T\subseteq S$ with $|T|=|S|-3$ such that $\forall u,v,w \in T$, $f(v, \{u,w\},\{u\})=\{w\}$.
\end{lemma}
\begin{proof}
    There are several cases to consider. If there are no triples of the form $u,v,w \in S$ such that $f(v,\{u,w\},\{u\})=\{u,w\}$ then the claim is trivial. Now assume that all such triples include one of the three identifiers $x,y,z \in S$. In this case, we take the remaining $|S|-3$ to be $T$ and the claim holds. Finally, assume that there are two disjoint triples $a,b,c$ and $g,h,i$ with this property. Without loss of generality we can assume that $f(b,\{a,c\},\{a\})=\{a,c\}$ and $f(h,\{g,i\},\{g\})=\{g,i\}$. Taking the path on $6$ nodes labelled $abcghi$, we have that either $b,c,g$ or $c,g,h$ must also be such a triple, or the protocol will not terminate on our path. 
    However, should $bcg$ or $cgh$ be such triples, we can shorten the path by deleting vertices from each end to obtain a labelling that once again does not terminate.
    More concretely, while ever there are only two such triples on the path that are ``facing each other'' they will bounce the message back and forth between each other.
    If there are more than two triples we just remove vertices until only two remain.
    Thus, as $P$ terminates for all graphs and labellings there of, we have that all such triples share one of three identifiers, and so we can take $T$ to be $S\setminus\{x,y,z\}$.
\end{proof}
\begin{corollary}
    If $\kappa\geq R(9,8)$ then $P$ is AF on the set $T$ described in the statement of lemma \ref{lemma:degree 2} up to degree $2$.
\end{corollary}
\begin{proof}
    There are two remaining cases to consider, the behaviour of $b$ and the case where a node receives messages from both of its neighbours.
    In the former case, consider the path on three nodes and broadcast from the central node.
    Since both of the broadcasting node's neighbours are leaves and by assumption will not return the message, for broadcast to be correct the initial node must sent to both of its neighbours.\\
    On the other hand, take $C_3$ labelled with $u,v,w \in T$.
    If $u$ initiates the broadcast, it will receive the message from both $v$ and $w$ three rounds later.
    If it sends to both $v$ and $w$, the process will repeat indefinitely.
    However, if it sends to only one of the two, the message will then circle forever.
    Thus, it must send to neither $v$ or $w$.\\
    Therefore, the claim holds.
\end{proof}
We now show that this behaviour extends to subcubic graphs.
\begin{lemma}
    \label{lemma: degree 2 to 3}
    Let $T \subseteq [\kappa]$ be a set of labels such that $|T|\geq 6$ and $P$ is AF up to degree $2$ on $T$, then $P$ is AF up to degree $3$ on $T$.
\end{lemma}
\begin{proof}
    We will construct a set of subcubic graphs such that only AF behaviour will lead to broadcast and termination for $P$ when they are labelled from $T$.\\
    Our first contestant and the base for much of this argument is the star $S_3$, labelled with $u,v,w,x \in T$ with $u$ the centre.
    Since $P$ is AF on $T$ up to degree $2$, neither $v$, $w$ nor $x$ will ever return a message to $u$.
    Thus for broadcast at $u$ to be successful, we must have that $b(u, \{v,w,x\})=\{v,w,x\}$.
    Furthermore, if broadcast is initiated at a leaf, $u$ must pass the message to both other leaves, thus $\{w,x\}\subseteq f(u,\{v,w,x\},\{v\})$.
    We can then repeat this argument substituting all possible arrangements of $T$ to obtain this rule for all quadruples.
    In fact we will assume this occurs for each graph labelling pair we consider for the remainder of this proof.\\
    Now consider, the same $S_3$ but with $v$ and $w$ connected by a new edge to form a paw graph.
    Should broadcast be initiated at $x$, then after four rounds $u$ will receive a message from both $v$ and $w$.
    If $\{v,w\} \subseteq f(u,\{v,w,x\},\{v,w\})$ then the same state will be reached three rounds later leading to non-termination.
    However, this does not directly preclude an asymmetric policy where $u$ sends a message back to either $v$ or $w$, but not both.  
    Without loss of generality, assume $v \in f(u,\{v,w,x\},\{v,w\})$ but $w \notin f(u,\{v,w,x\},\{v,w\})$. 
    There are now two possibilities, either $f(u,\{v,w,x\},\{w\})=\{v,w,x\}$ or $f(u,\{v,w,x\},\{w\})=\{v,x\}$. 
    In both cases we obtain a repeating sequence and so neither terminate. 
    Thus, $v,w, \notin f(u,\{v,w,x\},\{v,w\})$.\\
    We now take our original $S_3$ and attach a new node $y \in T$ to $v$ and $w$.
    Then if the broadcast is initiated at $y$, $u$ will only receive the message once. Thus as it is a bridge, it must pass the message to $x$ the first time it receives the message. Thus, $f(u,\{v,w,x\},\{v,w\})=\{x\}$ and we only need to determine the policy when receiving from one or all of its neighbours.\\

    Unfortunately, the case for receiving a single message is slightly more complex and will require we make use of a communication graph with two nodes of degree $3$.
    Fortunately, the previous policies apply to all quadruples and so the behaviour for both nodes is partially determined.
    Let $u,v,w,x,y,z \in T$, we then wlog. have three possibilities to consider.
    \begin{itemize}
        \item $x \notin f(u,\{v,w,x\},\{x\})$ and $u\notin f(x,\{u,y,z\},\{u\})$: This is the policy of Amnesiac Flooding.
        \item $x \in f(u,\{v,w,x\},\{x\})$ and $u\notin f(x,\{u,y,z\},\{u\})$: Consider the first graph and labelling in ~\cref{fig: diamond and h}, the protocol will not terminate broadcasting from $u$.
        \item $x \in f(u,\{v,w,x\},\{x\})$ and $u\in f(x,\{u,y,z\},\{u\})$: Consider the first graph and labelling in ~\cref{fig: diamond and h}, the protocol will not terminate broadcasting from $u$ or $x$.
    \end{itemize}
    \begin{figure}
        \centering
        \includegraphics[width=\linewidth]{images/Hs.pdf}
        \caption{Left: The graph used in the proof of lemma \ref{lemma: degree 2 to 3} to exclude the case $x \in f(u,\{v,w,x\},\{x\})$ and $u\notin f(x,\{u,y,z\},\{u\})$. Right: The H graph used in the proof of lemma \ref{lemma: degree 2 to 3} to exclude the case $x \in f(u,\{v,w,x\},\{x\})$ and $u\in f(x,\{u,y,z\},\{u\})$.}
        \label{fig: diamond and h}
    \end{figure}
    Finally, we must deal with the case where a degree $3$ node receives from all of its neighbours.
    First, consider $K_{2,3}$ with the nodes of degree $3$ labelled $u$ and $y$, with the degree $2$ nodes labelled $v,w,x$.
    If $f(u,\{v,w,x\},\{v,w,x\}) \notin \{\emptyset, \{v,w,x\}\}$ then the protocol will not terminate when broadcast is initiated at $y$.
    However, if $f(u,\{v,w,x\},\{v,w,x\})=\{v,w,x\}$ then the protocol will not terminate on the diamond graph (degree 2 nodes labelled $v$ and $w$, while degree $3$ nodes are labelled $u$ and $x$) when broadcast is initiated at $u$.
    Therefore, $f(u,\{v,w,x\},\{v,w,x\})=\emptyset$.
\end{proof}
For ease of reference we will combine the following results into the more readable statement below.
\begin{lemma}
    There exists $T\subseteq [\kappa]$ such that $|T|\geq 6$ and $P$ is AF on $T$ up to degree 3.
\end{lemma}
With this established we can extend the argument to all labels.
While technically the $T$ referred to in the previous lemma is not necessarily $[|T|]$, all labels within $[\kappa]$ can be reordered arbitrarily as they are available for all graphs.
For this reason and ease of presentation we will simply take $[T]=[|T|]$.
\begin{lemma}
    \label{lemma: id induction}
    $P$ is AF on $\mathbb{N}$ up to degree $3$.
\end{lemma}
\begin{proof}
    Our argument will be by induction, we will show that if $P$ is AF on $[m]$ up to degree $3$, it is AF on $[m+1]$ up to degree $3$. We shall do this by constructing communication graphs where only a few agents may deviate from an AF policy and then showing that none of these policies are correct and terminating. Throughout we will focus on the identifiers $v=m+1$ and $w,x,y,z,a,b,c,d \in [m]$ however our result relies on considering all possible permutations of the $[m]$ labels on our graphs. We note here that $c$ and $d$ are essentially dummy nodes to bound a path of arbitrary size (and possibly trivial) which contains all of the unused labels, thus they do not violate our base case of 6 nodes.

    First consider the extended paw graph depicted and labelled as in figure \ref{fig:PawBase}. The only two nodes that can have non-AF policies in this scenario are $v$ and $w$ and all others must be using AF policies. Since, the long path starting at $y$ contains only identifiers from $[m]$ and has maximum degree $3$, any broadcast initiated outside of the path will never lead to messages leaving that path back into the main graph. Thus, we can essentially ignore it. Then we have eight main cases to consider, namely each combination of $f(v,\{w\},\{w\})=\emptyset$ or $\{w\}$, $f(w,\{v,x\},\{v\})=\{x\}$ or $\{v,x\}$, and $f(w,\{v,x\},\{x\})=\{v\}$ or $\{v,x\}$. If $f(v,\{w\},\{w\})=\{w\}$ and the other two are as in Amnesiac Flooding, then the message will simply bounce back along the extended paw forever. Similarly, if $f(w,\{v,x\},\{x\})=\{v,x\}$ and $f(v,\{w\},\{w\})=\emptyset$, the same will occur. 
    There are in fact only two cases where termination will occur. Firstly, all three can be as in Amnesiac Flooding. Secondly, $f(v,\{w\},\{w\})=\emptyset$, $f(w,\{v,x\},\{v\})=\{v,x\}$, and $f(w,\{v,x\},\{x\})=\{v\}$ also terminates. Full sketches of non-termination are given in figure \ref{fig:Paw Non-Termination} for each of the remaining three cases. By permuting the identifiers other than $v$, we obtain that for all $\alpha,\beta \in [m]$ we must have that $f(v,\{\alpha\},\{\alpha\})=\emptyset$ and $f(\alpha,\{v,\beta\},\{\beta\})=\{v\}$.\\ \\
    \begin{figure}
        \centering
        \includegraphics[width=0.6\linewidth]{images/Paws/PawBase.pdf}
        \caption{The extended paw graph used in the proof of lemma \ref{lemma: id induction}. Note that the path from $c$ to $d$ represents all identifiers from $[m]$ that are not included in the main body of the paw graph.}
        \label{fig:PawBase}
    \end{figure}
    \begin{figure}
        \centering
        The case $f(w,\{v,x\},\{v\})=\{v,x\}$:\\ \includegraphics[width=\linewidth]{images/Paws/PawPolicy2.pdf}\\
        The case $f(w,\{v,x\},\{x\})=\{v,x\}$:\\ \includegraphics[width=\linewidth]{images/Paws/PawPolicy1.pdf}\\
    \end{figure}
    \begin{figure}
    \ContinuedFloat
    \centering
        The case $f(w,\{v,x\},\{v\})=\{v,x\}$ and $f(w,\{v,x\},\{x\})=\{v,x\}$:\\ \includegraphics[width=\linewidth]{images/Paws/PawPolicy3.pdf}\\
        \caption{A non-terminating broadcast for the cases $f(v,\{w\},\{w\})=\{w\}$ and either: $f(w,\{v,x\},\{v\})=\{v,x\}$, $f(w,\{v,x\},\{x\})=\{v,x\}$ or both, respectively. Message traces are given up to either the final unique state or the first repeat.}
        \label{fig:Paw Non-Termination}
    \end{figure}
    We now need to show that $f(\alpha,\{v,\beta\},\{v\})=\{\beta\}$ as well. Consider a path on $m+1$ nodes labelled $wxvyzc...d$, where the section $c...d$ contains all remaining labels from $[m]$. Since, only $v,w,y$ could have policies different to those of Amnesiac Flooding in this setting, a message can only be sent from outside the subsection $v,w,y$ once, and so we can treat this structure like $P_3$. Similar to the path on three nodes, without loss of generality, there are essentially 12 scenarios to consider, which produce behaviours falling into 5 general cases expressed in the table in figure \ref{fig:table}.
    \begin{figure}
        \centering
        \resizebox{\textwidth}{!}{
        \begin{tabular}{|c|c|c|c|}
             \hline
             Terminating & $f(v,\{w,y\},\{w\})=\{y\}$ & $f(v,\{w,y\},\{w\})=\{y\}$ & $f(v,\{w,y\},\{w\})=\{w,y\}$ \\
             Policies & $f(v,\{w,y\},\{y\})=\{w\}$ & $f(v,\{w,y\},\{y\})=\{w,y\}$ & $f(v,\{w,y\},\{y\})=\{w,y\}$\\
             \hline
             $f(w,\{v,x\},\{v\})=\{x\}$& Case 0& Case 1 & Case 1 \\
             $f(y,\{v,z\},\{v\})=\{z\}$& & &\\
             \hline
             $f(w,\{v,x\},\{v\})=\{v,x\}$& Case 1 & Case 2 & Non-Terminating \\
             $f(y,\{v,z\},\{v\})=\{v\}$& & & \\
             \hline 
             $f(w,\{v,x\},\{v\})=\{v\}$& Case 1 & Non-Terminating & Non-Terminating\\
             $f(y,\{v,z\},\{v\})=\{v,z\}$& & & \\
             \hline 
             $f(w,\{v,x\},\{v\})=\{v,x\}$&Non-Terminating & Case 3 & Case 4 \\
             $f(y,\{v,z\},\{v\})=\{v,z\}$& & & \\
             \hline
        \end{tabular}}
        \caption{The possible terminating cases for the policy discussed in theorem \ref{lemma: id induction} for the path on $m+1$ nodes}
        \label{fig:table}
    \end{figure}
    It remains to construct a communication graph and labelling such that for each case $1$-$4$ termination fails.\\
    For case $1$, there is exactly one of $x,v,y$ which implements a policy different to Amnesiac Flooding, we denote it $\gamma$.
    Furthermore, without loss of generality there exists at least one of its neighbours $\beta$ such that upon receiving a message from $\beta$ it sends messages to both of its neighbours.
    We take the path $wxvyz$, which contains either the segment $\beta \gamma$ or $\gamma \beta$.
    In the former case, we attach $a$ and $b$ to $w$ to form a $C_3$ on $abw$ and the path $c...d$ to $z$, forming an extended paw graph.
    In the latter case, we instead attach $a$ and $b$ to $z$ and the path on $c...d$ to $w$ to again form an extended paw.
    Now consider broadcast initiated at the solitary leaf node.
    The message will reach $\gamma$ and then bounce back and forth between $\gamma$ and the $C_3$ for ever.

    In case $2$, upon receiving from $v$, $w$ will send to both of its neighbours, as will $v$ upon receiving from $y$.
    This means that they both respond in the same direction on the path.
    We are therefore able to construct an extended $C_5$ as in figure \ref{fig:C5}, on which broadcast will not terminate when initiated at $z$.\\
    \begin{figure}
        \centering
        \includegraphics[width=0.4\linewidth]{images/ExtendedC5.pdf}
        \includegraphics[width=0.5\linewidth]{images/C5Policy.pdf}
        \caption{A communication graph for Case 2 where the path $c...d$ contains all remaining identifiers from $[m]$ and a non-terminating message sequence initiated at node $z$. The message sequence is given to the final unique configuration.}
        \label{fig:C5}
    \end{figure}
    In case $3$, both $w$ and $y$ will respond to $v$ however $v$ responds only to $y$.
    We construct a paw graph as in Case $1$. 
    In fact, we can use the same paw graph for case $4$, as both will fail to terminate when broadcast is initiated at $z$.
    For, the precise messages see figure \ref{fig: Cases3 and 4}.\\
    \begin{figure}
        \centering
        \includegraphics[width=\linewidth]{images/Case3_policy.pdf}
        \caption{Non-terminating messages sequences for cases $3$ and $4$ in lemma \ref{lemma: id induction}. Both behave identically for the first 9 iterations, however case $3$ takes two further iterations to repeat a step (indicated by dashed lines). Note that path $c...d$ has been omitted here but would be attached to $z$.}
        \label{fig: Cases3 and 4}
    \end{figure}
    Thus as we have obtained non-termination for all cases other than case $0$, we must have for all $\alpha,\beta,\gamma \in [m+1]$ that $f(\alpha, \{\beta, \gamma\}, \{\beta\})=\{\gamma\}$.
    Furthermore, we must have that $b(\alpha, \{\beta, \gamma\})=\{\beta,\gamma\}$ as otherwise the protocol would not terminate on a cycle.
    This also implies that $f(\alpha, \{\beta,\gamma\},\{\beta,\gamma\})=\emptyset$ to ensure termination on odd cycles when broadcast is initiated at $\alpha$.
    Finally, we must have that $f(\alpha,\{\beta\},\{\beta\})=\emptyset$ as $\alpha$ could be on the end of an extended paw.
    Thus, we have that $P$ is AF on $[m+1]$ up to degree $2$.\\

    Fortunately, we can reuse much of the previous argument for obtaining degree $3$ behaviour.
    Excluding the diamond graph and the $K_{3,2}$ graph, we can use all of the constructions in \cref{lemma: degree 2 to 3} by simply adding a path $c...d$ to a node of degree at most $2$ which is not adjacent to $v$.
    However, we can slightly alter the diamond and $K_{3,2}$ graph to permit the addition of such path to a node of degree at most $2$ which is not adjacent to $v$ no matter where $v$ appears in the labelling.
    These are given explicitly in figure \cref{fig:degree 3 but longer}.
    Note that while, these do require $8$ nodes to construct, we can use the original constructions from \cref{lemma: degree 2 to 3} until we have more than $\kappa$ nodes.
    \begin{figure}
        \centering
        \includegraphics[width=\linewidth]{images/Degree3ButLonger.pdf}
        \caption{The two graphs used to determine the policy for identifiers of degree 3 in lemma \ref{lemma: id induction}. Left: The analogue to the diamond graph. Right: The analogue to $K_{3,2}$}
        \label{fig:degree 3 but longer}
    \end{figure}
    Thus, repeating the argument from \cref{lemma: degree 2 to 3}, we have that $P$ is AF on $[m+1]$ up to degree $3$.
    Since, $P$ is AF on $T$ up to degree $3$, by induction we have that $P$ is AF on $\mathbb{N}$ up to degree $3$.
\end{proof}
Thus painfully we have that $P$ is indistinguishable from Amnesiac Flooding on subcubic graphs.
It may reassure the reader that we will not have to repeat this process for every maximum degree as $3$ is something of a magic number. 
In particular, once nodes behave up to degree $3$ we can construct binary tree like structures to simulate high degree nodes with lower ones.
With this tool under our belt, we are ready to prove theorem \ref{thm: AF is unique}.
\begin{proof}[Proof of Theorem \ref{thm: AF is unique}]
    By lemma \ref{lemma: id induction} we have that $P$ is AF up to degree $3$ on $\mathbb{N}$.
    It is immediate that for all $u \in \mathbb{N}$, $S \subset \mathbb{N}$ $b(u,S)=S$ as $u$ could be at the centre of a star with one of its leaves replaced by a long path.
    In this case, $u$ will never receive a message again and so must send to all of its neighbours in the first round.

    
    For an arbitrary label $u \in \mathbb{N}$ and degree $k\in \mathbb{N}$ we can show that $u$ must behave as though it is implementing Amnesiac Flooding if it is at a node of degree $k$.
    Take $S \subset \mathbb{N}$ such that $|S|=k$, we will show that for all non-empty $T\subseteq S$, $f(u,S,T)=S \setminus T$ via induction on the size of $T$.

    Beginning with our base case.
    If $u$ receives from a single neighbour we can construct a pair of graphs (special cases of our general construction) that enforce the AF policy. 
    Consider the tree in figure \ref{fig:OneMessageDegreeK}, if $u$ receives a message from $v$ it must send to at least $x_1,...,x_{k-1}$ as $u$ will receive messages in only a single round. 
    This follows as the rest of tree will use AF policies and since the graph is bipartite each node will be active only once. 
    Thus, the only two options for $f(u,S,\{v\})$ are $S$ or $S\setminus \{v\}$.\\ \\
    Now consider the second graph from figure \ref{fig:OneMessageDegreeK} and a broadcast initiated at $u$. 
    We can see that if $f(u,S,\{v\})=S$ then the cycle and $u$ will simply pass a message back and forth forever. 
    Thus, $f(u,S,\{v\})=S\setminus\{v\}$.
    \begin{figure}
        \centering
        \includegraphics[width=\linewidth]{images/OneMessageDegreek.pdf}
        \caption{Two graphs used in the proof of theorem \ref{thm: AF is unique} to determine identifiers' response to receiving only a single message. In both cases $c...d$ is a path containing all unused identifiers from $[m]$ where $m$ is the highest id used in labelling the body of the graph. Left: A tree that forbids sending to too small a subset of neighbours. Right: A graph that forbids sending a message to all neighbours.}
        \label{fig:OneMessageDegreeK}
    \end{figure}
        We can now generalize this construction and perform our induction. 
    Assume that for any non-empty subset $T$ of $S$ of size at most $q$, $f(u,S,T)=S\setminus T$. 
    For the sake of contradiction assume that this is not true for $V$ where $T\subset V \subseteq S$ with $|V|=q+1$. 
    Thus either $f(u,S,V)\cap V\neq \emptyset$ or $f(u,S,V)\cup V\neq S$. \\ 
    In the first, case let $W=V\cap f(u,S,V)$, and for some ordering label the elements of $W$: $w_1,...,w_r$ and the elements of $V\setminus W$: $v_1,...,v_{q+1-r}$.\\ 
    If $W=V$, then we construct a communication graph of the form in figure \ref{fig: BaseMany-MessagesDegreeK} and consider a broadcast initiated at $u$. 
    As every node other than $u$ in the graph has maximum degree $3$, the rest of the graph will use policies indistinguishable from Amnesiac Flooding. 
    Thus, the messages will travel in only one direction through the binary tree and so at the time $w^*$ receives a message it will be the only message on the graph. 
    This message will then be passed onto the cycle where it will circulate, before being passed back onto the binary tree in the opposite direction. 
    Again the policy of all nodes other than $u$ is indistinguishable from Amnesiac Flooding and so when $u$ next receives a message, it receives from all of $V$ and these are the only messages (outside of the path from $c$ to $d$).
    We therefore have a repeating sequence as $u$ will forward the message back to every node of $W$.
    \begin{figure}
        \centering
        \includegraphics[width=\linewidth]{images/BaseDegreeKManyReceive.pdf}
        \caption{A graph used in the proof of theorem \ref{thm: AF is unique}. The graph consists of a star centred at $u$ with its leaves partitioned into two sets of size $r$ and $k-r$. The leaves in the set of size $r$ are connected by a binary tree of depth $\lceil \log_2{r}\rceil$ with root $w^*$ to a cycle, which in turn is connected to a path. Note that the path $c...d$ contains all identifiers from $[m]$ not used in the labelling of the rest of the graph, where $m$ is the largest $id$ present.}
        \label{fig: BaseMany-MessagesDegreeK}
    \end{figure}\\
        If $W\subset V$, then we construct a slightly different communication graph (see figure \ref{fig:Many-MessagesDegreeK}).
    This graph instead partitions $S$ into $W$, $V \setminus W$ and $S \setminus V$, with separate binary trees for $W$ and $V \setminus W$.
    Here we consider a broadcast initiated at $a$.
    As every node other than $u$ has maximum degree $3$ it must make the same forwarding decisions as Amnesiac Flooding. 
    Thus, when $u$ first receives a message it will receive a message from all identifiers in $V$ and there will be no other messages in the body of the graph.
    Then by assumption $u$ will send a message to some portion of $\{l_1,...,l_{k-q-1}\}$ as well as all of $W$. 
    The leaves will not respond and messages will travel only upwards in the binary tree with $W$ as leaves and $w^*$ as its root. 
    Thus, $a$ will next receive a message only from $w^*$ and since it makes the same decision of Amnesiac Flooding it will send to $v^*$ and $c$. 
    Until $u$ receives a message the only messages in the body of the graph will be those travelling down the binary tree with $V\setminus W$ as its leaves and they will all arrive at $u$ simultaneously. 
    Thus, $u$ will then receive only from $V\setminus W$. Since $|V\setminus W|<|V|=q+1$ by assumption $u$ must make the same decision as Amnesiac Flooding and so will send messages to all of $W$ and its leaves. 
    This creates a repeating sequence and so we have non-termination. 
    Thus, since $f(u,S,V)\cap V\neq\emptyset$ always allows us to construct a communication graph with a non-terminating broadcast, we must have that $f(u,S,V)\cap V =\emptyset$.
    \begin{figure}
        \centering
        \includegraphics[width=\linewidth]{images/DegreeKManyReceive.pdf}
        \caption{A graph used in the proof of \cref{thm: AF is unique}. The graph consists of a star centred at $u$ with its leaves partitioned into three sets of size $r$, $q-r+1$ and $k-q-1$. The  leaves in the first two sets are then each joined to a single node labelled by $w*$ and $v*$ respectively by binary trees of depth $\lceil \log_2{q}\rceil$. The single nodes are connected to a node labelled $a$ which is the start of a path $ac...d$ that contains all identifiers in $[m]$ not used in the labelling of the body of the graph where $m$ is the largest id used in that labelling.}
        \label{fig:Many-MessagesDegreeK}
    \end{figure}
    Now consider the communication graph from figure \ref{fig:Many-MessagesDegreeK} again but with $W$ an arbitrary strict subset of $V$.
    Since $f(u,S,V)$ does not contain any id from $V$ and none of $\{l_1,...,l_{k-q-1}\}$ will send a message back to $u$, $u$ sends messages in only a single round. 
    Therefore, $u$ must send to all of $\{l_1,...,l_{k-q-1}\}$ otherwise some would not receive the message (and so the protocol would not implement broadcast correctly). 
    Thus, $f(u,S,V)=S\setminus V$ and so we have our contradiction.

    Therefore, if $f(u,S,T)=S\setminus T$ for $S\subset \mathbb{N}$ where $|S|=k$ and all $T\subseteq S$ such that $0<|T|\leq q$, then $f(u,S,V)=S\setminus V$ for $V\subset S$ such that $|V|=q+1$. 
    Thus, by induction since we know this to be true for all $u$ and $S$ when $q=1$ it must hold for all $q<|S|$.

    This gives our claim, as for every $u$ and $k$ we can apply this argument and show that for any $k \in \mathbb{N}$, $P$ is AF up to degree $k$.
\end{proof}