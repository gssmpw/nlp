
\section{Related Work}
The literature surrounding both the broadcast problem and fault sensitivity is vast, and a summary of even their intersection is well beyond the scope of this work. Instead, we shall focus on work specifically concerning stateless, or nearly stateless broadcast.

The termination of Amnesiac Flooding and its derivatives has been the focus of several works. Two independent proofs of the algorithm's termination have been presented, using either parity arguments over message return times~\cite{hussak2023termination} or auxillary graph constructions~\cite{turau2021amnesiac}. The latter technique has further been used to establish tight diameter independent bounds on the termination of multicast using Amnesiac Flooding, complementing the eccentricity based bounds of~\cite{hussak2023termination}. The techniques we develop in this work combine notions of both and allows us to consider arbitrary configurations of messages, rather than those resulting purely from correct broadcast. 

There have, further, been multiple variants of the algorithm introduced. It was observed by~\cite{turau2020stateless} in a result reminiscent of the BASIC protocol proposed by~\cite{gopal1999fast}, that sending a second wave of messages from a subset of the initial nodes could reduce the worst case $2D+1$ termination time to the optimal $D$ in all but a specific subset of bipartite graphs. We note that our fault sensitivity results extend naturally to this algorithm as well, as the same invariants apply to this setting. Beyond this, there have been several approaches to deal with the flooding of multiple messages simultaneously. In unpublished work the original authors of~\cite{hussak2023termination} show that under certain conditions termination can be retained, even when conflicting floods occur. Since then two partially stateless algorithms have been proposed, both making use of message buffers and a small amount of local memory~\cite{turau2021synchronous, bayramzadeh2021weak}. We will not be directly concerned with these approaches, as we assume a single concurrent broadcast throughout. However, the mechanism employed in~\cite{turau2021synchronous} should be highlighted as it rather cleverly exploits the underlying parity properties we identify as driving termination. Furthermore, as reduction to Amnesiac Flooding is used as technique for proofs in many of these works, the comprehensive understanding of its termination we present here could prove a powerful tool for future work in these areas.

While the robustness of Amnesiac flooding and its variants have been previously studied, this has been focused on two forms of fault. The first is the disappearance and reappearance of nodes and links. The termination of Amnesiac Flooding is robust to the former and vulnerable to the latter~\cite{hussak2023termination}, as we will observe this is a necessary consequence of the invariants driving termination and their relation to cyclicity. A pseudo-stateless extension to Amnesiac Flooding has been proposed to circumvent this~\cite{turau2021synchronous}, implicitly exploiting the parity conditions of~\cite{hussak2023termination}. The second are faults that violate synchrony. Under a strong form of asynchrony, truly stateless and terminating broadcast is impossible~\cite{turau2020stateless}. However under a weaker form of asynchrony (the case of fixed delays on communication links) is more fine-grained. Although, termination results have been obtained for cycles, as well as the case of single delayed edges in bipartite graphs~\cite{hussak2023termination}, there is no clear understanding of the impact of fixed channel delays. While we do not directly address this, we believe that techniques mirroring our invariant characterizations may prove fruitful in this area. To our knowledge, this work is the first to consider both the uniqueness of Amnesiac Flooding, as well as its fault sensitivity beyond node/link unavailability in a synchronous setting.

Beyond Amnesiac Flooding and its extensions, the role of memory in information dissemination is well studied in a variety of contexts. Frequently, stateful methods obtain faster termination time, such as in the phone-call model where the ability to remember one's communication partners and prevent re communication dramatically improves termination time and message efficiency~\cite{RandomisedBroadcastFirst,RandomisedBroadcastFinal,SocialNetworksSpreadRumors}. 
Similarly, for bit dissemination in the passive communication model the addition of only $\log{\log{n}}$ bits of memory is sufficient to break the near linear time convergence lower bound of~\cite{LimitsofInformationSpread} and achieve polylogarithmic time~\cite{korman2022early}. 
Even more strongly, a recent work~\cite{MemoryLowerBoundsDISC} has shown that in the context of synchronous anonymous dynamic networks, stabilizing broadcast from an idle start is impossible with $O(1)$ memory and even with $o(\log{n})$ memory if termination detection is required.
Despite this low memory and even stateless broadcasts remain desirable~\cite{gopal1999fast}. 

Several stateless broadcast schemes exist, to give an example, for mobile ad hoc networks which due to the lower power and rapid movement of devices see diminishing returns from maintaining information about the network~\cite{manfredi2011understanding}. 
However, given a lack of synchronisation as well as the wish to avoid so-called broadcast storms~\cite{ni1999broadcast}, these techniques typically rely on either some form of global knowledge (such as the direction or distance to the initiator) or the ability to sample network properties by eavesdropping on communications over time~\cite{manetsArentReallyStateless,ruiz2015survey}.
It should be noted, that in contrast to many models, such as anonymous dynamic networks, radio networks and many manets, the typical framework for studying stateless flooding ("true statelessness" as defined by~\cite{turau2020stateless} restricting the model of~\cite{dolev2017stateless}) permits the knowledge and distinguishing of neighbours in both broadcasting and receiving.


Similarly, the resilience of stateful broadcast methods is well studied.
<Random message adversaries> <Message dropping><Amitabh?>