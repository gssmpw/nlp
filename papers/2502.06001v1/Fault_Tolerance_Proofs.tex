
\section{Proof of the Fault Sensitivity results}
\label{apx: faults}

In this appendix, we prove the fault sensitivity results of section \cref{sec: faults}, using \cref{thm:balance} as our main technical tool.
We begin with the single message failure case, restated below.
    \begin{*theorem}[Single Message Failure]
    \label{thm:message drop}
        If single-source Amnesiac Flooding experiences a \emph{single message drop failure} for the message $(u,v)$ then it fails to 
       terminate.
        if either:
        \begin{itemize}
            \item $uv$ is not a bridge
            \item $uv$ lies on a FEC subgraph
        \end{itemize}
        Moreover, it fails to broadcast if and only if this is the first message sent along $uv$, $uv$ is a bridge, and the side of the cut containing $u$ does not contain an odd cycle.
    \end{*theorem}
\begin{proof}[Proof of Single Message Failure Theorem]
    For the non-termination, this is an immediate consequence of theorem \ref{thm:balance} as the dropping of the message must produce an imbalance.\\
    For correctness, let $S_u$ be the configuration of Amnesiac Flooding on $G=(V,E)$ in the first round of broadcast from $u$. 
    A node $v$ will become informed via the shortest message path out of any message from $S$ ending at $v$, since this is the shortest path it cannot cross the same edge twice and so removing the second use of an edge will not harm correctness. 
    On the other hand if $uv$ is not a bridge, then the state reached by dropping a message on $uv$ the configuration becomes imbalanced and so all nodes will receive messages infinitely often. 
    Finally, if the message $(u,v)$ is dropped the first time $uv$ is used and $uv$ is a bridge, either there is an odd-cycle on $u$'s side of the cut or there isn't. 
    If there isn't then $u$ will be activated only once, as the component the messages can access is bipartite. 
    Otherwise, $u$ will be activated a second time and so the message will cross $uv$ and flood the other half as normal.
\end{proof}
We now turn to the result for Uni-directional link failure.
We will first need the following lemma.
\begin{lemma}\label{lemma: directed}
    Let $G=(V,E,A)$ be a simple-mixed communication graph such that $(u,v) \in A \implies (v,u) \notin A$, $C$ a cycle on $G$ containing at least one arc from $A$ and $S\subseteq V^2$ a message configuration on $G$. If $S_C$ contains more messages travelling in the same direction as an arc from $A$ than counter to it, $S_C$ is not a terminating configuration of Amnesiac Flooding on $G$.
\end{lemma}
\begin{proof}
    Since $C$ is a cycle on $G$, all arcs are oriented in the same direction with respect to $C$. 
    The number of messages on $C$ can change for three reasons: either messages are added due to external activation, messages are removed as a result of a collision between two messages, or a message reaches an arc oriented in the opposite direction. 
    The first creates either two messages, one travelling in each direction, or only a single message travelling in the same direction as the arcs. 
    The second removes one message travelling in both directions. 
    The third can remove only messages travelling counter to the arcs. 
    Therefore, there is no way for the number of messages travelling with the arcs to decrease relative to the number travelling counter to the arcs. 
    Thus, if there are more travelling with the arcs, the protocol can never reach termination.
\end{proof}
We can now prove the result for the uni-directional link failure case.
    \begin{*theorem}[Uni-directional link failure]
    \label{thm: directed}
        For any graph $G=(V,E)$ and any initiator set $I\subsetneq V$ there exists an edge $e \in E$ such that a uni-directional link failure at $e$ will cause Amnesiac Flooding to either fail to broadcast or fail to terminate when initiated from $I$ on $G$. Furthermore, for any non-empty set of uni-directional link failures there exists $v \in V$ such that, when Amnesiac Flooding is initiated at $v$, it will either fail to broadcast or fail to terminate.
    \end{*theorem}
\begin{proof}[Proof of Uni-directional Link Failure Theorem]
    For the first claim, there are two cases either the graph contains a bridge or it does not. 
    If the graph contains a bridge we take $e$ to be a bridge edge and orient it towards the broadcasting node. 
    This prevents broadcast as it disconnects the graph.\\
    On the other hand, if the graph does not contain a bridge it must contain a cycle. 
    We take the closest node $v$ of the cycle to our initiating node $u$ and orient one of its cycle edges $(w,v)$ towards it (please note that $v$ may in fact be $u$ or even non-unique).
    As $v$ is the closest node of the cycle to $u$ and nodes are always first informed over the shortest path from the initiating node, $v$ first receives a message it must do so from outside of the cycle.
    Furthermore, there can either be no other messages on the cycle when $v$ first receives the message.
    Then $v$ will attempt to send a message to both of its neighbours on the cycle, however $w$ will not receive the message as it is sent counter to the arc $(w,v)$.
    Any other node of the cycle being activated will cause the addition of one message in each direction, where as $v$ will only contribute a single message in the direction of the arc.
    Thus, there will be more messages travelling in the direction of the arc around the cycle and so by lemma \ref{lemma: directed} the protocol will not terminate.\\

    For the second claim, we observe that either the set of edge failures does not leave a strongly connected mixed-graph, or it contains a cycle with an arc. 
    In the former case there must be two nodes $u$ and $v$ such that no message from $u$ can reach $v$, so we take $u$ to be our initial node and the process fails to broadcast. 
    In the latter case, if the arc is $(u,v)$ we take the node $v$ to be our initial node. 
    In this case, immediately we have only a single message travelling in the direction of the arc around the cycle and none travelling counter to it. 
    Therefore, by application of lemma \ref{lemma: directed} the protocol will not terminate.
\end{proof}
Finally, we prove the Byzantine case.
    \begin{*theorem}[Byzantine Failure]\label{thm: Byzantine}
        If Amnesiac Flooding on $G=(V,E)$ initiated from $I\subsetneq V$ experiences a \emph{weak Byzantine failure} at $J\subseteq V\setminus I$, then the adversary can force:
        \begin{itemize}
            \item Failure to broadcast if and only if $J$ contains a cut vertex set.
            \item Non-termination if and only if at least one member of $J$ lies on either a cycle or FEC subgraph. 
        \end{itemize}
    \end{*theorem}
\begin{proof}[Proof of Byzantine Failure Theorem]
    Since the Byzantine nodes are time bounded, for them to force the protocol not terminate they must selectively forward messages in order to force a non-terminating configuration at their final step. 
    By theorem \ref{thm:balance}, this requires them to force an imbalanced configuration.
    If the Byzantine set contains a node on a cycle or FEC, this is trivial as in a single step they can either send an additional message or fail to send a message, which will lead to imbalance.
    Conversely, if the Byzantine set does not contain such a node there is no way they can produce an imbalance on a cycle or FEC, as they can only externally activate (or deny external activations to) such a structure.
    But external activations do not affect the balance of a graph.
    Thus, the Byzantine agents can force non-termination if and only if the set contains a node on a cycle or FEC.\\

    The Byzantine agents can trivially prevent a broadcast if their set contains a cut node set, as Byzantine nodes on the cut can simply not send messages, effectively removing the nodes from the communication graph.
    However, if the set of nodes does not contain a cut node set, this strategy cannot disconnect the graph and so Amnesiac Flooding will still be successful.
    Assume for the sake of contradiction there exists a forwarding strategy such that the Byzantine agents can prevent broadcast.
    Let $B\subset V$ be the set of Byzantine nodes, $u \not \in B$ our initial node and $v\in V$ a node that will not be informed under this strategy.
    Since $B$ does not contain a cut node set there must be a shortest path $P$ from $u$ to $v$ in $G\setminus B$.
    As $v$ is not informed, the Byzantine nodes must prevent a message from travelling along from $u$ to $v$ along $P$.
    However, no member of $B$ lies on $P$ and so the only way to block a message on $P$ is to send a message in the other direction to collide with it.
    This can only be achieved by one of the Byzantine nodes $b_0$ sending a message that will reach a node $w$ on $P$ before the message from $u$ would.
    However, since $w$ has not yet received a message it must forward the message from $b_0$ in both directions along the path.
    Thus the message from $b_0$ will now inform $v$, and so must be prevented from reaching it.
    Iterating this argument, any message sent by the Byzantine nodes that blocks the message from $u$ along $P$ to $v$ will only lead to some other message reaching $v$ first.
    Thus, there is no strategy that can force the broadcast to be incorrect.
    
\end{proof}