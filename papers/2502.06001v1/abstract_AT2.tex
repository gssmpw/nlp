\begin{abstract}
   %   Broadcast is a central problem in distributed computing. Recently, Hussak and Trehan [PODC'19/STACS'20/DC'23] proposed a truly stateless broadcasting protocol (Amnesiac Flooding), which  even has optimal termination time (O(diameter)) in a synchronous network. This spawned a series of work examining the algorithm and variants and the role of memory and state in broadcast in static and dynamic networks [Parzych and Damude, DISC 2024].
   %  %, the algorithm terminates with optimal termination time in a deterministic network.
   % % a truly  was proposed (Amnesiac Flooding), which was surprisingly proven to terminate within time linear in the diameter of the network.
   %   Despite this it has remained unclear: (i) What conditions are necessary for stateless broadcast algorithms? (ii) Is Amnesiac Flooding the only terminating stateless broadcast algorithm under certain distributed models? (iii) How robust is Amnesiac Flooding with respect to \textit{faults} and variations to the network models.
    Broadcast is a central problem in distributed computing. Recently, Hussak and Trehan [PODC'19/DC'23] proposed a stateless broadcasting protocol (Amnesiac Flooding), which was surprisingly proven to terminate in asymptotically optimal time (linear in the diameter of the network). However, it remains unclear: (i) Are there other stateless terminating broadcast algorithms with the desirable properties of Amnesiac Flooding, (ii) How robust is Amnesiac Flooding with respect to \emph{faults}?

    In this paper we make progress on both of these fronts. Under a reasonable restriction (obliviousness to message content) additional to the fault-free synchronous model, we prove that Amnesiac Flooding is the \emph{only} strictly stateless deterministic protocol that can achieve terminating broadcast. We identify four natural properties of a terminating broadcast protocol that Amnesiac Flooding uniquely satisfies. In contrast, we prove that even minor relaxations of \textit{any} of these four criteria allow the construction of other terminating broadcast protocols.
    
    On the other hand, we prove that Amnesiac Flooding can become non-terminating or non-broadcasting, even if we allow just one node to drop a single message on a single edge in a single round. As a tool for proving this, we focus on the set of all \textit{configurations} of transmissions between nodes in the network, and obtain a \textit{dichotomy} characterizing the configurations, starting from which, Amnesiac Flooding terminates. 
    Additionally, we characterise the structure of sets of Byzantine agents capable of forcing non-termination or non-broadcast of the protocol on arbitrary networks.
    
    %condition (beyond the standard assumptions of a fault-free synchronous model) of the algorithm being oblivious to the message content, Amnesiac Flooding is the only strictly stateless deterministic algorithm that can achieve terminating broadcast.

   % On the other hand, we identify four natural properties of a network broadcasting protocol (strong true statelessness, determinism, obliviousness to the content of the message, and each node sending at most one message per edge per round), such that Amnesiac Flooding is the \textit{only} terminating \anew{broadcast} protocol satisfying all of them. In contrast, we prove that, by dropping \textit{any} of these four criteria, there exist other terminating broadcast protocols. \anew{The other stateless protocols we have discovered are all variations of Amnesiac Flooding leaving open the question of whether a different approach even exists in these settings. }


    
    
    
    %Despite this, it remained unclear (i)~how robust this protocol is with respect to \textit{faults} and (ii)~whether other \textit{variations} exist retaining its desirable properties?

    
    %In this paper we make progress on both of these fronts. On the one hand, we prove that, on any network, the protocol can become non-terminating or non-broadcasting, even if we allow just one node to drop a single message on a single edge in a single round. As a tool for proving this, we focus on the set of all \textit{configurations} of transmissions between nodes in the network, and obtain a \textit{dichotomy} characterizing the configurations, starting from which, Amnesiac Flooding terminates. 
   % Second, we characterise the structure of sets of Byzantine agents capable of forcing non-termination or non-broadcast of the algorithm on arbitrary networks. 
    
    
   % On the other hand, we identify four natural properties of a network broadcasting protocol (strong true statelessness, determinism, obliviousness to the content of the message, and each node sending at most one message per edge per round), such that Amnesiac Flooding is the \textit{only} terminating \anew{broadcast} protocol satisfying all of them. In contrast, we prove that, by dropping \textit{any} of these four criteria, there exist other terminating broadcast protocols. \anew{The other stateless protocols we have discovered are all variations of Amnesiac Flooding leaving open the question of whether a different approach even exists in these settings. }
\end{abstract}
