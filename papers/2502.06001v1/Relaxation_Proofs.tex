\section{\anew{Uniqueness: } Existence of Protocols in the relaxed cases}
\label{apx: relaxed}
In this appendix, we present protocols for each of the relaxations discussed in \cref{sec: relaxation}
\subsection{Protocol with limited statelessness}
For the protocol with limited statelessness, we simply exploit the ability of the protocol to identify whether it is on a star.
The protocol can be stated as follows:
\begin{definition}[Neighbourhood 2 flooding]
    \emph{Neighbourhood 2 Flooding} is the protocol defined by the following rule:
    upon receiving a message, if a node has a neighbour of degree $2$ or higher, \emph{or} has only a single neighbour, it implements Amnesiac Flooding.
    Otherwise, upon receiving a message it sends to all of its neighbours.
\end{definition}
\begin{proposition}
    Neighbourhood 2 flooding is correct and terminates in finite time.
\end{proposition}
\begin{proof}
    The protocol is automatically correct and terminating for non-star graphs by the correctness of Amnesiac Flooding.
    If the protocol is on a star, then either the broadcast is initiated at the centre or a leaf.
    On either the first round (if at the centre) or the second (if at a leaf), the centre sends to all of the leaves.
    The leaves all have a neighbour of degree $2$ or more and so do not respond.
    Thus all nodes have received the message and no further messages are sent.
    In the special case of $K_2$ we simply treat the initiating node as the centre, and the same argument holds.
\end{proof}
This implies that while the ability to retain information about the identifiers in a nodes neighbourhood, does not permit anything other than Amnesiac Flooding, even slightly more information does as Neighbourhood 2 flooding produces a different sequence of messages on star graphs.
However distinct behaviour is obtained only for graphs of constant diameter.
In general, a similar construction (Neighbourhood k flooding) can be produced if nodes know the structure of their $k$th order neighbourhood and send to all neighbours upon receiving a message if and only if they are on a tree, can see all of its leaves and have the highest ID out of all nodes.
This still obtains only a linear maximum diameter in the number of hops nodes aware are of.
It remains open whether an protocol in this setting can be found with behaviour distinct from Amnesiac Flooding on graphs of unbounded diameter.
\subsection{Protocol with non-determinism}
For a non-deterministic relaxation we assume each agent has access to a single bit of uniform randomness per round, and otherwise work in the setting of \cref{def: functions}.
We note that while these bits must be independent between rounds, they can be either shared or independent between agents.
With this established we present the following protocol which is correct and terminates in finite time with probability $1$.
\begin{definition}[Random Flooding]
    \emph{Random Flooding} is the protocol defined by the following rule: on each round, if a node has a random bit equal to $0$ it implements Amnesiac Flooding. Otherwise upon receiving a message it sends to all neighbours.
\end{definition}
\begin{proposition}
    Random Flooding correctly achieves broadcast in diameter rounds and terminates in finite time with probability $1$.
\end{proposition}
\begin{proof}
    Regardless of the random bits, messages will propagate down all edges and all agents will receive a message along the shortest path from the initiator node.
    Thus, the protocol is correct and achieves broadcast in exactly $D$ rounds.\\
    Consider the random variable $X(t)$ which for $t \in \mathbb{N}$ takes the value $1$ if on rounds $t, t+1,...,t+D$ all nodes receive $1$ as a random bit, but at round $t+D+1$ all nodes receive $0$, and $0$ otherwise.\\
    We now make the following claim, if for some $t \in \mathbb{N}$, $X(t)=1$, then the protocol has terminated by time $t+D+2$. 
    If there are no messages at time $t$ then, Random Flooding cannot send any new messages and so the protocol has already terminated.
    Otherwise, there must exist at least one node receiving a message at time $t$.
    Since this node has a random bit of $1$, it will send to all of its neighbours.
    Repeating this notion, at time $t+D$ all nodes will be sending to all of their neighbours.
    However, at time $t+D+1$ all agents implement Amnesiac Flooding, and so upon receiving from all of their neighbours terminate, giving our claim.\\
    The event $X(t)=1$ occurs with positive probability $\Omega(2^{-n^2})$ and $X(t)$ is independent of $X(t+D+2)$ for all $t \in \mathbb{N}$.
    Let $Y=\cap_{i\geq 1}\{X(i(D+1))=0\}$, since $Y$ is the limit event of a sequence of monotonically decreasing events and $\lim_{m\to\infty}\mathbb{P}(\cap_{1\leq i \leq m}\{X(i(D+1))=0\})\to 0$, $\mathbb{P}(Y)=0$.
    Thus, Random Flooding terminates in finite time with probability $1$.
\end{proof}
In fact, we can weaken the setting further such that the ``random'' bits are drawn from some sequence with only the fairness guarantee that $X(t)=1$ occurs almost surely for finite $t$.
\subsection{Protocols that can read messages/send multiple messages}
We lump these two together, as we will demonstrate that only one bit of information need be propagated with the message.
This could be in the form of readable header information, but can equally be encoded in the decision to send one or two messages.
Without specifying, we take the setting of \cref{def: functions} but where agents have access to a single bit of information determined by the initiator when broadcast begins and which is available to them on any round where they receive a message.
We will exploit this single bit of information to encode whether the initiator node is a leaf or not, which permits us to make use of the following subroutine,
\begin{definition}[Parrot Flooding]
    \emph{Parrot Flooding} is the protocol defined by the following rule: if a node is a leaf upon receiving a message it returns it to its neighbour. Otherwise, all non-leaf nodes implement Amnesiac Flooding.
\end{definition}
Parrot flooding has the following useful property, the proof of which we defer.
\begin{lemma}
    \label{lemma: parrot flooding works}
    On any graph $G=(V,E)$ with initiator node $v \in V$ such that $v$ is not a leaf node, parrot flooding is correct and terminates in finite time.
\end{lemma}
This allows us to obtain the following protocol which can be adapted to either the higher bandwidth or readable header relaxation.
\begin{definition}[1-Bit Flooding]
    \emph{1-Bit Flooding} is the protocol defined according to the following rule: When broadcast begins, the initiator picks $0$ for the shared bit if it is a leaf and $1$ otherwise.
    Upon receiving a message, if the shared bit is a $1$ the node implements parrot flooding, and implements Amnesiac Flooding otherwise.
\end{definition}
Therefore, we immediately have.
\begin{proposition}
    On any graph $G=(V,E)$ 1-Bit flooding is correct and terminates in finite time.
\end{proposition}
If we have readable header information, then the single bit is placed in the header.
On the other hand, if the nodes are able to send multiple messages per edge per round, then the initiator may choose to send $2$ messages in the case where it wishes to encode a $1$ and a single one otherwise.
In this case, if nodes always forward the same number of messages then the single bit is propagated and maintained.\\

The remainder of the section is devoted to the proof of \cref{lemma: parrot flooding works}.
First, we formalise parrot flooding in the same manner as Amnesiac Flooding.
\begin{definition}[Parrot Flooding Redefinition]
    Let $S$ be a configuration on $G=(V,E)$ as in Amnesiac Flooding. The \emph{Parrot Flooding} protocol functions as follows: $P_{G,I}(S)=A_{G,I}(S)\cup T$ where $T=\{(v,u)|(u,v) \in S \wedge deg(v)=1\}$.
\end{definition}
We begin with the following observation.
\begin{observation}
    Let $S$ be a configuration of messages on $G=(V,E)$. Since messages are only reintroduced at leaves rather than on cycles or FECs, $P(S)$ is balanced on $G$ if and only if $S$ is balanced on $G$. Formally, this follows from \cref{thm: Byzantine}.
\end{observation}
However, if parrot flooding is initiated at the terminal vertex of a path this configuration is balanced but will not terminate.
Therefore, we require a stronger condition than balance to capture termination under parrot flooding.
Consider a message path of parrot flooding (the definition of which is obtained by simply replacing the operator $A$ with $P$ in definition \ref{def: message paths}).
We immediately inherit \cref{lemma: recurrence,lemma: odd cycles,lemma: even cycles,lemma: even cycles,lemma: EvenSub} as none of them depend on the behaviour of leaves.
Thus any recurrent message $(u,v) \in S$ must have a message path $uv...wlw...uv$ where $l$ is a leaf node or it would be captured by the argument used in the proof of \cref{thm:balance}.
Therefore, in fact there exists a configuration $\hat{S}=P^k(S)$ for some $k \in \mathbb{N}$ such that $(w,l)\in \hat{S}$ where $(w,l)$ is recurrent on $\hat{S}$.
The rest of the argument is similar to that used in the proof of \cref{thm:balance}, and we will construct a suitable notion of balance.
First, however, we must define the additional structure that produces non-termination.
\begin{definition}[Leaf paths]
    A \emph{leaf path} $L=l_1...l_k$ on $G=(V,E)$ is a sequence of nodes from $V$ starting and ending at leaves such that $l_il_{i+1} \in E$ and $l_i\neq l_{i+2}$ unless $i=1$ and $k=3$.\\
    The path representation of $L$ denoted $L_p$ is given by the path $l_1,...,l_k$ where copies of the same node are treated as distinct (if $l_i$ is the $j$th appearance of some node $v \in V$ we replace it with a new node $v_j$).\\
    Let $S\subseteq V^2$ be a configuration on $G$, then the path representation of $S$ with respect to $L$ denoted $S_{L_{p}}$ is given by taking all messages from $S$ only between nodes of $L$ and for each message $(u,v)$ adding the set $\{(u_i,v_j)|u_i \in L_p\wedge v_j \in L_p\}$.
\end{definition}
\begin{definition}
    A config $S$ is \emph{$P$-Balanced} on a leaf path $L$ if for all messages $m$ in $S_{L_p}$, the subset of messages that have an even distance between their head and $m$'s along $L_{p}$ is of even cardinality.
    Let $G=(V,E)$ be a graph and $S\subseteq V^2$ a configuration on $G$. Then $S$ is $P$-Balanced on $G$ if $S$ is balanced on $G$ and $P$-Balanced on all leaf paths of $G$.
\end{definition}
This property is conserved by parrot flooding.
\begin{lemma}
    $P$-Balance is preserved under Parrot flooding with non-leaf external node activation.
\end{lemma}
\begin{proof}
    We have already established that balance is preserved by parrot flooding, so it remains to show that $P$-Balance is conserved on all leaf paths.
    Let $Q$ be a leaf path starting at $u$ and ending at $v$ on $G=(V,E)$ and let $S$ be a configuration on $G$.
    Then $P_G(S)_{Q_{p}}=P_{Q_p,I}(S)$ for some $I \subset V$ such that $u,v \not\in I$, i.e. a step of parrot flooding on $G$ is equivalent to a step of parrot flooding on the leaf path representation with some possible external node activation.
    After a step of parrot flooding, two messages on $Q_p$ will either have the distance between their heads: stay the same, increase by two or decrease by two.
    Upon externally activating a non-leaf node two messages are created, with their heads an even distance apart on $Q_p$.
    Symmetrically, a collision between two messages can only be between two messages with an even distance (explicitly $0$) between their heads and removes both.
    Thus, $P$-Balance is maintained.
\end{proof}
\begin{corollary}
    If $S$ is not $P$-Balanced on $G$ then $\not\exists k \in \mathbb{N}$ such that $P_G^k(S)=\emptyset$.
\end{corollary}
We now show that from a $P$-Balanced configuration no message can take too many steps along a leaf path.
\begin{lemma}
    Let $G=(V,E)$ be a graph and $m$ a message in $S\subseteq V^2$. If $m$ has a message path from $S$ on $G$ that takes $k+1$ consecutive steps along a leaf path of length $k$ then $S$ is not $P$-Balanced on $G$.
\end{lemma}
\begin{proof}
    For contradiction assume such a message path exists on the leaf-path $L$ and $S$ is $P$-Balanced on $G$. Then since $m$ takes a step on $L$ there must exist an odd number of other messages on $L_p$ with an even head distinct to $m$ as $S$ is $P$-Balanced on $L$.
    These messages will collide with $m$ the first time they cross over, which unless they are all blocked will occur before $k+1$ steps occur. 
    However, for them to all be blocked, new messages must be added to the $L_p$ with an even head distance to $m$. 
    But, any new message $m_1$ blocking a potential blocker $m'$ of $m$ must introduce $m_2$ with an even head distance to $m$ which will reach $m$ before $m'$ would have.
    Thus, $m$ will collide with another message on $L$ before taking $k+1$ steps along $L$.
\end{proof}
This gives us what we need to mirror the result of \cref{thm:balance}.
\begin{lemma}
    Let $G=(V,E)$ be a graph and $S\subseteq V^2$ be a configuration of it, $S$ has a recurrent message in parrot flooding if and only if $S$ is not $P$-Balanced on $G$.
\end{lemma}
\begin{proof}
    If $S$ is balanced on $G$ we must have that any recurrent message uses a leaf node in its message-path. Therefore, that message must traverse a full leaf path and then take an additional step.
    By the previous lemma this cannot happen if the configuration is $P$-Balanced on $G$. 
    Therefore, a recurrent message does not exist if $S$ is $P$-Balanced on $G$.

    However, if $S$ is not $P$-Balanced on $G$ then the configuration is non-terminating and so must have a recurrent message.
\end{proof}
\begin{corollary}
    \label{corr: Parrots}
    Parrot flooding terminates on $G$ from a configuration $S$ if and only if $S$ is $P$-Balanced on $G$.
\end{corollary}
\begin{proof}[Proof of \ref{lemma: parrot flooding works}]
    Parrot flooding correctly broadcasts as every node receives a message along the shortest path from a member of the initiator set.
    In the first round, every node in the initiator set sends to all of its neighbours, corresponding to an external activation of any leaf paths that it may lie on.
    This cannot break $P$-Balance and since $\emptyset$ is $P$-balanced, by corollary \ref{corr: Parrots} the protocol must terminate in finite time. 
\end{proof}