\section{\anew{Uniqueness: } Relaxing the constraints}
\label{sec: relaxation}
\label{Relaxing}
Despite the uniqueness established in the previous section.
By even slightly relaxing any of the $4$ properties presented in Conditions \ref{obs:properties}, we are able to obtain correct and terminating protocols distinct from Amnesiac Flooding.
We summarise this in the following result:
\begin{theorem}
    \label{thm: Relaxation}
    For each of the constraints in conditions \ref{obs:properties}, there exists a relaxation permitting a correct and terminating protocol distinct from Amnesiac Flooding. 
\end{theorem}
While the majority of these protocols are not practical, and most are similarly fault sensitive to Amnesiac Flooding, they demonstrate that a broad spectrum of protocols exist.
Specifically, the relaxations we make use of are as follows:
\begin{itemize}
    \item \StrongTrueStatelessness: Several relaxations of this already exist, such as Stateless Flooding (initiator retains information for one round) or even classical non-Amnesiac Flooding (nodes are able to retain 1-bit for one round). We present a third relaxation, where nodes know the ID of their neighbours and their neighbours' neighbours.\label{relax: stateless}
    \item \Obliviousness: We permit one bit of read only header information to be included with the message and used in routing decisions. \label{relax: blind}
    \item \Determinism: We give each node access to one bit of randomness per round. \label{relax: determinism}
    \item \Bandwidth: We allow each agent to send up to two messages per edge per round. \label{relax: bandwidth}
\end{itemize}
Since each of these constitutes only a minor relaxation of the restrictions, we argue that the uniqueness of Amnesiac Flooding is in some sense tight.
We will present protocols and demonstrate their correctness and termination for each of these cases independently in appendix \ref{apx: relaxed}.


