
\section{Uniqueness}
\label{sec: Uniqueness}
% As there have been several attempts to obtain terminating broadcast protocols outside of the fault-free synchronous setting while maintain Amnesiac Flooding's desirable properties, a natural question is whether other such algorithms even exist in the original setting.
Given the known sensitivity of Amnesiac Flooding to even minor faults and instances of asynchrony, a natural question is whether there exists a more robust protocol which preserves its desirable properties. Specifically, does there exist a terminating protocol for broadcast which obeys all of the following for all graphs and port labellings:
\begin{condition}\;
    \begin{enumerate}
        \label{obs:properties}
        \item  \anew{\StrongTrueStatelessness: }
        %\StrongTrueStatelessness:
        Nodes maintain no information other than their port labellings between rounds. This includes whether or not they were in the initiator set.
        \item \Obliviousness: Routing decisions may not depend on the contents of received messages.
        \item \Determinism: All decisions made by a node must be deterministic.
        \item \Bandwidth: Each node may send at most one message per edge per round.
    \end{enumerate}
\end{condition}

The answer is strongly negative, in fact there is no other terminating broadcast protocol preserving these properties at all. We will actually prove the slightly stronger case, that this holds even if agents are provided with unique identifiers and are aware of the identifiers of their neighbours. Any broadcast protocol meeting the constraints of conditions \ref{obs:properties} must be expressible in the following form:
\begin{definition}
\label{def: functions}
    A \emph{protocol} $P=(b,f)$ is a pair of functions, an \emph{initial function} $b$ and a \emph{forwarding function} $f$, where $b:\mathbb{N}\times 2^{\mathbb{N}}\to 2^{\mathbb{N}}$ and $f:\mathbb{N}\times 2^{\mathbb{N}}\times 2^\mathbb{N}\to 2^\mathbb{N}$. 
    The protocol is implemented as follows. On the first round the initiator node $s$ with neighbourhood $N(s)$ sends messages to $b(s,N(s))$.
    On future rounds, each node $u$ sends messages to every node in $f(u,N(u), S)$ where $S$ is the set of nodes they received messages from in the previous round.
    Further, we require that $b(u,N(u)), f(u,N(u),S)\subseteq N(u)$ and that $f(u,N(u),\emptyset)=\emptyset$, enforcing that agents can only communicate over edges of the graph and can only forward messages they have actually received respectively.\\
    If there exists a graph $G=(V,E)$ with a unique node labelling and initiator node $s$ such that protocols $P$ and $Q$ send different messages in at least one round when implementing broadcast on $G$ initiated at $s$, then we say that $P$ and $Q$ are \emph{distinct}.
    Otherwise, we consider them the same protocol.
\end{definition}
In this setting, achieving broadcast is equivalent to every node receiving a message at least once and terminating in finite time corresponds to there existing a finite round after which no messages are sent.
For example, Amnesiac Flooding is defined by the following functions:
\begin{definition}[Amnesiac Flooding Redefinition]
    $P_{AF}=(b_{AF},f_{AF})$, where $b(u,S)=S$ and $f(u,S,T)=S\setminus T$ if $T\neq \emptyset$ and $\emptyset$ otherwise.
\end{definition}
For a protocol $P=(b,f)$, a set $S \subset \mathbb{N}$ and a degree $k\in \mathbb{N}$ we describe $P$ as AF up to degree $k$ on $S$ if there is no graph of maximum degree $k$ or less labelled with only members of $S$ which distinguishes $P$ from $P_{AF}$.
From here on we will assume that all unique labels are drawn from $[n+\kappa]$ where $\kappa$ is a sufficiently large constant and $n$ is the number of nodes in the graph. We obtain the following result, the proof of which makes up the remainder of this section, provided $\kappa\geq R(9,8)$.
\begin{theorem}
\label{thm: AF is unique}
    Let $P=(b,f)$ be a terminating broadcast protocol that meets the conditions of observation \ref{obs:properties}, then $P$ is not distinct from Amnesiac Flooding.
\end{theorem}
\begin{proof}[Proof sketch for \cref{thm: AF is unique}]
The basic argument is to show that any terminating broadcast protocol that meets the criteria of \cref{obs:properties} is identical to Amnesiac Flooding.
Our core technique is to construct a set of network topologies such that any policy distinct from Amnesiac Flooding fails on at least one of them.\\
Through the use of a combinatoric argument we establish that for $\kappa\geq R(9,8)$, there must exist a set $T$ of at least $9$ identifiers with the following property: for all $u,v \in T$ if $u$ has only the neighbour $v$, upon receiving a message $u$ will not return the message to $v$.
From here, we show that there must exist $U\subseteq T$ containing at least $6$ identifiers such that $P$ is AF up to degree $2$ on $U$.
By constructing a set of small sub-cubic graphs, we are able to extend this to degree $3$.\\
This forms the base case of a pair of inductive arguments. First, we construct a progression of sub-cubic graphs which enforce that if $P$ is AF on $[m]$ up to degree $3$ it must be AF on $[m+1]$ up to degree $3$.
We then construct a family of graphs which have a single node of high-degree and all other nodes have a maximum degree of $3$. Therefore, by the previous construction, all other nodes must behave as though running Amnesiac Flooding.
These graphs permit an inductive argument showing that this unique high degree node must also behave as through running Amnesiac Flooding.
In combination, these two constructions enforce that $P$ behaves like Amnesiac Flooding in all possible cases.
The full proof is given in appendix \ref{apx: Uniqueness}.
\end{proof}

