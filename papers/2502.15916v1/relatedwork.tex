\section{Related Work}
\label{sec:related}

The growing demand for speech data sets in underrepresented languages has driven various resource creation efforts. This section reviews existing community-centered initiatives and African speech datasets.

Multiple projects have aimed to address resource gaps in African languages through community involvement. \citet{nekotoparticipatory} introduced the Oshiwambo data set project, which created more than 5000 sentences in the Oshindonga dialect along with their English translations. This participatory and cost-effective approach highlighted the potential for collaborative resource creation in underrepresented African languages, but they were not able to release the dataset due to the lack of an African-centered data license that would benefit the data creators \cite{african_licenses}. Similarly, AfroDigit presented the first audio digit dataset in 38 African languages enabling speech recognition applications such as telephone and street number recognition \citep{emezue2023afrodigits}. BibleTTS provides 86 hours of high-fidelity Bible recordings in 10 African languages~\citep{meyer2022bibletts}, and AfricanVoices focuses on collecting high-quality speech datasets for African languages while also providing speech synthesizers~\citep{ogun20241000}. AfriSpeech-200 extends these efforts by providing a 200-hour pan-African speech corpus for English-accented ASR in the clinical and general domains~ \citep{olatunji2023afrispeech}. Meanwhile, Kencorpus features 5.6 million words and 177 hours of speech of 3 Kenyan languages; Swahili, Dholuo, and Luhya~\citep{wanjawa2022kencorpus}.

Several datasets have been developed to increase ASR resources for isiXhosa. \citet{louw2001african} collected a spontaneous monolingual corpus of isiXhosa and five other South African languages. This data, sourced from first-language speakers via an annotated telephone-based database. Although this work demonstrated the potential of African language resources, its scale and scope were limited. \citet{strom2018linguistic} captured linguistic diversity of isiXhosa by emphasizing the importance of including regional accents and tonal variations. Lastly, \citet{van2017synthesising} curated a multilingual dataset of four South African languages, including isiXhosa, in a code-mixed setting with English. This resource advanced acoustic modeling and multilingual ASR systems \cite{biswas2003semi}, demonstrating the potential of code-mixed datasets. However, its scope remained task-specific, limiting its utility for downstream applications.

% , however resource constraints limited the broader applicability of their dataset