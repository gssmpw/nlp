



\section{Application to the Motivating Example}



The Mecanum wheel robot system \eqref{eq:4w} presented in Section \ref{sec:MotExmp} can be cast into the required form \eqref{eq:sys} by using a dynamic extension of the linear velocities $v_1, v_2$.
We thus define the state variables as $\xv_1 = \begin{bmatrix} x & y & \theta & v_1 \end{bmatrix}^\top$ and $\xv_2 = v_2$.
This extension incorporates the energy-intense input as part of the state, and is necessary so that the resulting system satisfies the assumptions \ref{asmpt_1} and \ref{asmpt_2}.

\subsection{Derivation of the controller equations}

When operating the platform in \textit{energy-saving mode}, the objective is to reduce the transversal velocity $v_2$ to zero, thereby minimizing energy consumption. 
Conversely, in \textit{dexterous mode}, the goal is to achieve trajectory tracking of the full configuration of the vehicle  (position and orientation).
To address these objectives, we define the output vector as $\yv = \begin{bmatrix} \yv_1^\top & y_2 & y_3 \end{bmatrix}^\top$, where $\yv_1 = \begin{bmatrix} x & y \end{bmatrix}^\top$, $y_2 = \theta$, and $y_3 = v_2$. The input vector is given by $\uv = \begin{bmatrix} u_1 & u_2 & u_3 \end{bmatrix}^\top$, with
    $\dot{v}_1=u_1$, 
    $\dot{v}_2=u_2$ 
 and $u_3 := v_3$.
The system exhibits a \emph{(vector) relative degree} $\rv = \{\rhov_1, \rhov_2\}$, where $\rhov_1 = \{2, 2\}$ and $\rhov_2 = 1$, relative to the pair $(\uv, \yv)$. Additionally, it exhibits a  \emph{(vector) relative degree}  $\bar{\rv} = \{\rhov_1, \rhov_3\}$, with $\rhov_3 = 1$, relative to the pair  $(\uv, \bar{\yv})$. Consequently, Assumptions~\ref{asmpt_1} and~\ref{asmpt_2} hold, allowing the application of Theorem~\ref{thm:1}.
Consider now the output ${\yv}_{\sigma}$ defined  in \eqref{y_tilde}. At the $\tilde{\rv}$-th differentation, the dynamics are given by $$
{\yv}^{({\rv}_{\sigma})}_{\sigma}={\Am}_{\sigma}(\xv)\uv 
$$ where the interaction matrix defined in~\eqref{eq:Atilde} is 
$${\Am}_{\sigma}(\xv) = \begin{pmatrix}
    c_\theta  & -s_\theta &-v_1s_\theta-v_2c_\theta   \\ s_\theta& c_\theta& v_1c_\theta-v_2s_\theta  \\
    0 & 1-\sigma & \sigma
\end{pmatrix}$$% 
and $\bv(\xv)=\boldsymbol{0}$. Then, using \eqref{eq:control}, the control law  becomes 
\begin{equation}
    \uv = {\Am}_{\sigma}(\xv)^{-1}\vv_{\sigma}
\label{eq:c_un}
\end{equation}
 with $\vv_{\sigma}$ as in \eqref{eq:v} and the $\vv_j$ as in \eqref{eq:control2} solves the Problem~\ref{prob:main_problem} for the four mecanum wheel vehicle.


\begin{figure}[t]
\centering
\includegraphics[width=0.54\columnwidth]{mizzo3.jpg}
\hfill
\includegraphics[width=0.44\columnwidth]{mizzo4.jpg}
      \caption{Stroboscopic highlights of two simulations.
      In \textbf{Simulation~1}  (left), the robot converges to and follows a circular trajectory.
      In \textbf{Simulation~2} (right), the robot converges to a straight-line trajectory  while carrying a load (depicted in orange) and avoiding hanging obstacles (shown in red).
      The robot operates in dexterity mode only when necessary (as determined by the switching signal $\sigma$), prioritizing energy-saving mode when far from obstacles.}
\label{fig:animation}
\end{figure}

\begin{rem}
The dynamic extension of the sagittal velocity leads to a singularity in the resulting controller.
Specifically, the decoupling matrix ${\Am}_{\sigma}(\xv)$ becomes singular when $\sigma = 0$ and $v_1 = 0$.
Such a singularity must be carefully addressed and avoided during trajectory planning, particularly when employing interpolation techniques. This can typically be achieved by appropriately selecting the initialization of the state $v_1$—an additional degree of freedom available in the design.
   
\end{rem}





\subsection{Numerical Simulations}
To demonstrate the capabilities of the proposed controller in a realistic scenario, we present two simulations where a four-Mecanum-wheel omnidirectional robot is tasked with transporting a load while following two different trajectories in the presence of obstacles (see Fig.~\ref{fig:animation}). 


The first trajectory is a circular path defined by:
\[
\yv^d_1(t) = \begin{bmatrix}
r\sin(\omega t) &
-r\cos(\omega t)
\end{bmatrix}^\top, 
\]
where $ r = 8 \, \text{m} $ and $ \omega = 0.15 \, \text{rad/s} $. 

Two hanging obstacles are positioned along the trajectory. When the robot encounters these obstacles, a reorientation of $\pi/2 \, \text{rad}$ is required to avoid collisions between the transported load and the obstacles. The desired orientation for obstacle avoidance is defined as:
\[
y_2^d(t) = \omega t + \frac{\pi}{2}.
\]

For the majority of the trajectory, the switching signal $\sigma$ is set to zero, meaning that the robot operates in energy-saving mode. However, when passing beneath the obstacles, $\sigma$ switches to one, activating the dexterous mode. In this mode, the robot adjusts its orientation to avoid collisions while continuing to follow the position trajectory. 

\begin{figure}[t]
    \centering
    \includegraphics[trim={0.5cm 0cm 0cm 0cm},clip,scale=0.174]{mizzo5.pdf}
    \caption{\textbf{Simulation~1.}  A circular input reference trajectory for the position of the CoM and a square form switching signal ${\sigma}$ are given to the control system. The gray areas correspond to $\sigma=0$ whereas the orange areas correspond to $\sigma=1$. The top row shows the output variables $\yv_1,\yv_2$, and the bottom row shows the sagittal velocity $v_1$, the third output $v_2$ and the control input $u_3$.  }
    \label{fig:circle}
\end{figure}

\begin{figure}[t]
    \centering
    \includegraphics[trim={0.6cm 0cm 0cm 0cm},clip,scale=0.174]{mizzo6.pdf}
\caption{\textbf{Simulation~2.} A ramp input reference trajectory with a square form switching signal $\sigma$ are given to the control system.
The gray areas correspond to $\sigma=0$ whereas the orange areas correspond to $\sigma=1$. The top row shows the output variables $\yv_1,\yv_2$, and the bottom row shows the sagittal velocity $v_1$, the third output $v_2$ and the control input $u_3$. }
    \label{fig:ramp} 
\end{figure}


Fig.~\ref{fig:circle} illustrates the simulation results. The gray areas correspond to $\sigma=0$ (energy-saving mode), while the orange areas represent $\sigma=1$ (dexterous mode).
The plots show the output variables $\yv_1$ and $\yv_2$ at the top, and the sagittal velocity $v_1$, the third output $v_2$, and the control input $u_3$ at the bottom.



 
The second trajectory is a straight line defined by:
\[
\yv^d_1(t) = \begin{bmatrix}
5 + \frac{t}{4} &
5 + \frac{t}{4}
\end{bmatrix}^\top.
\]

Similarly, a hanging obstacle is placed along this trajectory. To avoid a collision, the desired orientation is given by:
\[
y_2^d(t) = \frac{3\pi}{4}.
\]


The initial conditions for both simulations were set far from the trajectory to showcase transient behaviors. The gain matrices used were:
\[
\Lm^1_{1} = \mathbf{I}_2, \quad
\Lm^2_{1} = \mathbf{I}_2, \quad
L^1_2 = 0.75, \quad
L^1_3 = 0.65.
\]
This simulation also incorporated a low-pass filtered Gaussian noise $\nv \in \mathbb{R}^3$ in the actuation inputs to enhance realism. 
Precisely, to the control input \eqref{eq:c_un} we add the noise resulting from the solution of 
\begin{align}\label{eq:lowpass-noise}
    \dot{\nv}=-k\nv+\boldsymbol{\mu}, && \boldsymbol{\mu} \in \mathcal{N}(\boldsymbol{0}, q^2 \mathbf{I}_3),
\end{align}
where $q = 0.4$ and, $k=0.1$. 
Figure \ref{fig:noise} shows the components of the noise $\nv$ added to the actuators during Simulation~1.





%\begin{comment}
\begin{figure}[t]
    \centering
    \includegraphics[trim={0.04cm 0cm 0cm 0cm},clip,scale=0.171]{mizzo7.pdf}
    \caption{A realization of the Gaussian noise resulting from \eqref{eq:lowpass-noise}.}
    \label{fig:noise}
\end{figure}
%\end{comment}
% The robot begins from an initial state far from the trajectory to highlight its transient behavior. Fig.~\ref{fig:ramp} shows the results, with the same conventions for $\sigma=0$ and $\sigma=1$ as in Simulation~1.

 
The plots for both simulations (Figs.~\ref{fig:circle} and~\ref{fig:ramp}) demonstrate the controller's ability to solve the trajectory tracking problem effectively. When the switching signal $\sigma=1$, the system transitions into dexterous mode, where the smooth tracking of both $\yv_1$ and $\yv_2$ is ensured.
When the switching signal $\sigma=0$, the system transitions into energy saving mode, where
the velocity $v_2$ is brought exponentially to zero while ensuring smooth tracking of $\yv_1$.
It is noteworthy that while the trajectory $y_2^d(t)$ is defined at all times, it is only enforced when $\sigma=1$. When $\sigma=0$, the angle variable $\theta$ ceases to follow the prescribed trajectory. Instead, the robot moves in a unicycle-like fashion, with $\theta(t)$ evolving according to the platform’s flat outputs $x$ and $y$, i.e., $\theta(t) = \mathrm{atan2}\{\dot{y}, \dot{x}\}$. Despite this deviation in $\theta$, the position coordinates $x$ and $y$, and hence $\yv_1$, continue to perfectly track the desired trajectory.


The videos corresponding to the simulations of Figs.~\ref{fig:circle} and~\ref{fig:ramp} are available at \mbox{{\small\url{https://youtu.be/Wn4hVNXEjmc}}}. 








