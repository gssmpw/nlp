 \documentclass[letterpaper, 10 pt, conference]{ieeeconf}
\IEEEoverridecommandlockouts    
\overrideIEEEmargins 
\usepackage[dvipsnames]{xcolor}
%\usepackage[letterpaper, left=1in, right=1in, bottom=1in, top=0.75in]{geometry}
 % Comment this line out if you need a4\pagestyle{plain}paper


\usepackage{lettrine}
\usepackage{balance}
\usepackage{graphicx}
%%%% source %%%%%%%%
% DLR, 2005; later edits ADL %
%%%%%%%%%%%%%%%%

%%%%  requirements %%%%

\usepackage{amsmath}
\usepackage{amssymb}
%usepackage{accents} 
% improves spacing (e.g., of \dddots)

%%%%%%%%%%%%%%%%
%%%%%%  general  %%%%%
%%%%%%%%%%%%%%%%

\newcommand{\vect}[1]{\mathbf{#1}}
\newcommand{\mat}[1]{\mathbf{#1}}

\newcommand{\diff}[2]{\frac{\partial #1}{\partial #2}}
\newcommand{\diffs}[3]{\frac{\partial^2 #1}{
\ifx#2#3 
\partial #2^2
\else
\partial #2 \partial #3
\fi
}}
%\newcommand{\norm}[1]{{\left \| {#1} \right \|}}
\newcommand{\de}[1]{{\left | {#1} \right |}}
% Unit matrix
\newcommand{\IIm}{\mat{I}}
% Zero matrix
\newcommand{\zerom}{\mat{O}}


\newcommand{\alphav}{\mathbf{\alpha}}

\newcommand{\betav}{\mathbf{\beta}}
\newcommand{\chiv}{\mathbf{\chi}}
\newcommand{\dchiv}{\mathbf{\dot{\chiv}}}
%\newcommand{\dddott}[1]{{#1}^{[3]}}
\newcommand{\dddott}[1]{{\stackrel{\mathbf{...}}{#1}}}
%\newcommand{\ddddott}[1]{{#1}^{[4]}}
\newcommand{\ddddott}[1]{{\stackrel{\mathbf{....}}{#1}}}

\newcommand{\normm}[1]{{\left \| {#1} \right \|}}

\DeclareMathOperator*{\argmax}{arg\,max}
\DeclareMathOperator*{\argmin}{arg\,min}
\DeclareMathOperator*{\sign}{sign}

%%%%%%%%%%%%%%%%
%%% vectors (low case)  %%%
%%%%%%%%%%%%%%%%

% Zero vector
\newcommand{\zerov}{\vect{0}}

\newcommand{\av}{\vect{a}}
\newcommand{\bv}{\vect{b}}
\newcommand{\cv}{\vect{c}}
\newcommand{\dcv}{\dot{\vect{c}}}
\newcommand{\ddcv}{\ddot{\vect{c}}}
\newcommand{\dv}{\vect{d}}
\newcommand{\ev}{\vect{e}}
\newcommand{\dev}{\dot{\vect{e}}}
\newcommand{\ddev}{\ddot{\vect{e}}}
\newcommand{\fv}{\vect{f}}
\newcommand{\gv}{\vect{g}}
\newcommand{\gbv}{\bar{\vect{g}}}
\newcommand{\dgv}{\dot{\vect{g}}}
\newcommand{\ddgv}{\ddot{\vect{g}}}
\newcommand{\hv}{\vect{h}}
\newcommand{\kv}{\vect{k}}
\newcommand{\lv}{\vect{l}}
\newcommand{\mv}{\vect{m}}
\newcommand{\nv}{\vect{n}}
\newcommand{\dnv}{\dot{\vect{n}}}


\newcommand{\rhov}{\mathbf{\rho}}


\newcommand{\ddnv}{\ddot{\vect{n}}}
\newcommand{\ov}{\vect{o}}
\newcommand{\pv}{\vect{p}}
\newcommand{\dpv}{\dot{\vect{p}}}


\newcommand{\xiv}{\mathbf{\xi}}


\newcommand{\ddpv}{\ddot{\vect{p}}}
\newcommand{\qv}{{\vect{q}}}
\newcommand{\dqv}{\dot{\vect{q}}}
\newcommand{\ddqv}{\ddot{\vect{q}}}
\newcommand{\dddqv}{\dddott{\vect{q}}}
\newcommand{\ddddqv}{\ddddott{\vect{q}}}
\newcommand{\qbv}{\bar{\vect{q}}}
\newcommand{\dqbv}{\dot{\bar{\vect{q}}}}
\newcommand{\ddqbv}{\ddot{\bar{\vect{q}}}}
\newcommand{\dddqbv}{\dddott{\bar{\vect{q}}}}
\newcommand{\ddddqbv}{\ddddott{\bar{\vect{q}}}}
\newcommand{\qhv}{\hat{\vect{q}}}
\newcommand{\qtbv}{\tilde{\bar{\vect{q}}}}
\newcommand{\rv}{{\vect{r}}}
\newcommand{\drv}{\dot{\vect{r}}}
\newcommand{\sv}{\vect{s}}
\newcommand{\dsv}{\dot{\vect{s}}}
\newcommand{\uv}{\vect{u}}
\newcommand{\vv}{\vect{v}}
\newcommand{\dvv}{\dot{\vect{v}}}
\newcommand{\wv}{\vect{w}}
\newcommand{\dwv}{\dot{\vect{w}}}
\newcommand{\xv}{\vect{x}}
\newcommand{\dxv}{\dot{\vect{x}}}
\newcommand{\ddxv}{\ddot{\vect{x}}}
\newcommand{\dddxv}{\dddott{\vect{x}}}
\newcommand{\ddddxv}{\ddddott{\vect{x}}}
\newcommand{\txv}{\vect{\tilde{x}}}
\newcommand{\dtxv}{\dot{\tilde{\vect{x}}}}
\newcommand{\ddtxv}{\ddot{\tilde{\vect{x}}}}
\newcommand{\dddtxv}{\dddott{\tilde{\vect{x}}}}
\newcommand{\ddddtxv}{\ddddott{\tilde{\vect{x}}}}
\newcommand{\yv}{\vect{y}}
\newcommand{\ybv}{\bar{\vect{y}}}
\newcommand{\dyv}{\dot{\vect{y}}}
\newcommand{\ytv}{\tilde {\vect{y}}}
\newcommand{\zv}{\vect{z}}
\newcommand{\dzv}{\dot{\vect{z}}}
\newcommand{\ddzv}{\ddot{\vect{z}}}

%%%%%%%%%%%%%%%%
% vectors (low case) in Greek %
%%%%%%%%%%%%%%%%

\newcommand{\gammav}{\vect{\gamma}}
\newcommand{\lambdav}{\vect{\lambda}}
\newcommand{\muv}{\vect{\mu}}
\newcommand{\etav}{\vect{\eta}}
\newcommand{\deltav}{\vect{\delta}}
\newcommand{\ddeltav}{\dot{\vect{\delta}}}

\newcommand{\phiv}{\vect{\phi}}
\newcommand{\dphiv}{\dot{\vect{\phi}}}
\newcommand{\ddphiv}{\ddot{\vect{\phi}}}

\newcommand{\psiv}{\vect{\psi}}

\newcommand{\sigmav}{\vect{\sigma}}
\newcommand{\dsigmav}{\dot{\vect{\sigma}}}

\newcommand{\piv}{\vect{\pi}}

\newcommand{\tauv}{\vect{\tau}}
\newcommand{\dtauv}{\dot{\vect{\tau}}}
\newcommand{\ddtauv}{\ddot{\vect{\tau}}}
\newcommand{\thetav}{\vect{\theta}}
\newcommand{\dthetav}{\dot{\vect{\theta}}}
\newcommand{\ddthetav}{\ddot{\vect{\theta}}}
\newcommand{\thetavt}{\tilde{\vect{\theta}}}
\newcommand{\nuv}{\vect{\nu}}
\newcommand{\omegav}{\vect{\omega}}
\newcommand{\dxiv}{\dot{\vect{\xi}}}

%%%%%%%%%%%%%%%%
%%   vectors (capitalized)   %%
%%%%%%%%%%%%%%%%

\newcommand{\Fv}{\vect{F}}
\newcommand{\Mv}{\vect{M}}
\newcommand{\Tv}{\vect{T}}
\newcommand{\Vv}{\vect{V}}
\newcommand{\Wv}{\vect{W}}

%%%%%%%%%%%%%%%%%
% vectors (capitalized) in Greek %
%%%%%%%%%%%%%%%%%

\newcommand{\Alphav}{\vect{\Alpha}}
\newcommand{\Betav}{\vect{\Beta}}
\newcommand{\Gammav}{\vect{\Gamma}}
\newcommand{\Thetam}{\vect{\Theta}}
\newcommand{\dThetam}{\dot{\Thetam}}
\newcommand{\ddThetam}{\ddot{\Thetam}}

%%%%%%%%%%%%%%%%
%%  matrices (capitalized)  %%
%%%%%%%%%%%%%%%%

% Unit matrix

% Zero matrix


\newcommand{\Am}{\mat{A}}
\newcommand{\Bm}{\mat{B}}
\newcommand{\Cm}{\mat{C}}
\newcommand{\dCm}{\dot{\Cm}}
\newcommand{\ddCm}{\ddot{\Cm}}
\newcommand{\dddCm}{\dddott{\Cm}}
\newcommand{\Dm}{\mat{D}}
\newcommand{\Em}{\mat{E}}
\newcommand{\Fm}{\mat{F}}
\newcommand{\Gm}{\mat{G}}
\newcommand{\Hm}{\mat{H}}
\newcommand{\Jm}{\mat{J}}
\newcommand{\Lm}{\mat{L}}
\newcommand{\Jbm}{\bar{\mat{J}}}
\newcommand{\dJm}{\dot{\Jm}}
\newcommand{\ddJm}{\ddot{\Jm}}
\newcommand{\dddJm}{\dddott{\Jm}}
\newcommand{\Km}{\mat{K}}
\newcommand{\Mm}{\mat{M}}
\newcommand{\dMm}{\dot{\Mm}}
\newcommand{\ddMm}{\ddot{\Mm}}
\newcommand{\dddMm}{\dddott{\Mm}}
\newcommand{\Nm}{\mat{N}}
\newcommand{\Pm}{\mat{P}}
\newcommand{\Qm}{\mat{Q}}
\newcommand{\Rm}{\mat{R}}
\newcommand{\Sm}{\mat{S}}
\newcommand{\Tm}{\mat{T}}
\newcommand{\Vm}{\mat{V}}
\newcommand{\Um}{\mat{U}}
\newcommand{\Wm}{\mat{W}}
\newcommand{\Xm}{\mat{X}}
\newcommand{\Ym}{\mat{Y}}
\newcommand{\Zm}{\mat{Z}}

%%%%%%%%%%%%%%%%%%
% matrices (capitalized) in Greek %
%%%%%%%%%%%%%%%%%%

\newcommand{\Deltam}{\mat{\Delta}}
\newcommand{\Sigmam}{\mat{\Sigma}}
\newcommand{\Gammam}{\mat{\Gamma}}
\newcommand{\Lambdam}{\mat{\Lambda}}

%%%%%%%%%%%%%%%%%%
%%% matrices (calligraphic)  %%%
%%%%%%%%%%%%%%%%%%

\newcommand{\calAm}{\mat{\cal A}}
\newcommand{\calCm}{\mat{\cal C}}
\newcommand{\calMm}{\mat{\cal M}}
\newcommand{\dcalMm}{\dot{\calMm}}

%%%%%%%%%%%%%%%%
%%% matrices with indices %%
%%%%%%%%%%%%%%%%

\newcommand{\Jqbm}{\Jm_{\bar q}}
\newcommand{\Jgm}{\Jm_{g}}
\newcommand{\Jhm}{\Jm_{h}}

%etc.

\newcommand{\plottwofigures}[8]{
\begin{figure}[H]
  \begin{subfigure}[t]{#4\textwidth}

    \centering

\includegraphics[width=\linewidth]{#1}
    \caption{#2}
      \label{#3}
  \end{subfigure}
  \hfill
  \begin{subfigure}[t]{#8\textwidth}
    \centering
    \includegraphics[width=\linewidth]{#5}
    \caption{#6}
     \label{#7}
  \end{subfigure}
\end{figure}
}

%%%%% end %%%%%%%

\usepackage{svg}
\usepackage{float}
\usepackage{hyperref}
\usepackage{academicons}
\usepackage{xcolor}

% updated version 
\newcommand{\update}[1]{{\color{black}  #1}}
% to be done command
\newcommand{\tbd}[1]{{\color{Maroon}  \textbf{to correct - } #1}}

\newcommand{\pieter}[1]{{\color{orange}\footnotesize PvG: #1}}
\newcommand{\mirko}[1]{{\color{WildStrawberry}\footnotesize MM: #1}}
\newcommand{\antonio}[1]{{\color{magenta} \footnotesize AF: #1}}
\let\proof\relax
\let\endproof\relax
\usepackage{amsthm}
\theoremstyle{plain}% default
\newtheorem{result}{\textbf{Result}}[section]
\newtheorem{exmp}{\textbf{Example}}[section]
\newtheorem{proposition}{Proposition}
\newtheorem{thm}{Theorem}
\newtheorem{lem}{Lemma}
\newtheorem{assumpt}{Assumption}
\newtheorem{problem}{Problem}
\newtheorem{corollary}{Corollary}[thm]
\usepackage{comment}
\usepackage[font=small]{caption}
\usepackage{subcaption}
\usepackage{calc}
\newtheorem{rem}{\textbf{Remark}}[section]
\usepackage{mathtools} 
%\usepackage{subfig}
% \title{\LARGE \bf
% A unified control method for a vehicle morphing between fully actuated and under-actuated propulsion modes
% }
\title{\LARGE \bf
Unified Feedback Linearization for Nonlinear Systems with\\ Dexterous and Energy-Saving Modes}
\usepackage{orcidlink}

\author{Mirko Mizzoni$^1$\orcidlink{0009-0006-2165-3475}, Pieter van Goor$^1$\orcidlink{0000-0003-4391-7014}, and Antonio Franchi$^{1,2}$\orcidlink{0000-0002-5670-1282}
\thanks{$^1$Robotics and Mechatronics group, Faculty of Electrical Engineering,  Mathematics, and Computer Science (EEMCS), University of Twente, 7500 AE Enschede, The Netherlands. {\footnotesize \tt m.mizzoni@utwente.nl}, {\footnotesize \tt p.c.h.vangoor@utwente.nl}, {\footnotesize \tt a.franchi@utwente.nl}}
\thanks{$^2$Department of Computer, Control and Management Engineering, Sapienza University of Rome, 00185 Rome, Italy, {\footnotesize \tt antonio.franchi@uniroma1.it}}\thanks{This work was partially funded by the Horizon Europe research agreement no. 101120732 (AUTOASSESS).}}


\begin{document}


\maketitle
\thispagestyle{empty}
\pagestyle{empty}

% takeawya lesson
% abstract >-> statement of a thereom 
% the paper will represent the proof 
% of my statement

\begin{abstract}
Systems with a high number of inputs compared to the degrees of freedom (e.g. a mobile robot with Mecanum wheels) often have a minimal set of \textit{energy-efficient} inputs needed to achieve a \textit{main task} (e.g. position tracking) and a set of \text{energy-intense} inputs needed to achieve an additional \textit{auxiliary task} (e.g. orientation tracking). 
This letter presents a unified control scheme, derived through feedback linearization, that can switch between two modes: 
an \textit{energy-saving mode}, which tracks the main task using only the energy-efficient inputs while forcing the \textit{energy-intense} inputs to zero, and a \textit{dexterous mode}, which also uses the energy-intense inputs to track the auxiliary task as needed.
The proposed control guarantees the exponential tracking of the main task and that the dynamics associated with the main task evolve independently of the a priori unknown switching signal.
When the control is operating in dexterous mode, the exponential tracking of the auxiliary task is also guaranteed.
Numerical simulations on an omnidirectional Mecanum wheel robot validate the effectiveness of the proposed approach and demonstrate the effect of the switching signal on the exponential tracking behavior of the main and auxiliary tasks.
\end{abstract}


\section{Introduction}
In many mechatronics systems, for example robotic or transportation systems involving high-dimensional and omnidirectional motion~\cite{TAHERI2020103958}, the full control input is not always necessary to complete a task. 
\begin{figure}[t]
    \centering
\includegraphics[trim={3.9cm 3.6cm 2.8cm 2.8cm},clip,scale=0.35]{mizzo1.pdf}
    \caption{A four Mecanum wheels omnidirectional vehicle. This system is capable of lateral movement (dexterous mode), but this comes at the cost of higher power consumption compared to the more efficient forward-only motion (energy-saving mode). The application of the generic nonlinear controller introduced in this work to this system ensures independent and decoupled exponential stabilization of the position output, along with either the orientation (in dexterous mode) or the lateral speed (in energy-saving mode), depending on the a priori unknown operation mode selected by an external source.}
    \label{fig:model}
\end{figure} 
For instance, Mecanum wheel robots (see Fig.~\ref{fig:model}) are renowned for their omnidirectional mobility, allowing seamless movement in all directions on a flat surface. 
Nevertheless, the sideways movement is so energy-demanding that sometimes it is preferred to avoid or limit it, especially for tasks in which it is not strictly required. 
Another example is flying morphing platforms, such as those described in~\cite{aboudorra2023omnimorph,Ryll2022FASTHex}, which allow additional lateral forces that can be deactivated to conserve energy during point-to-point translation tasks.
This redundancy allows for selective input constraints that can optimize energy consumption by deactivating unnecessary degrees of freedom. 
Such strategies are especially relevant in systems where task requirements are confined to a lower-dimensional subspace, as observed in robotic manipulators and mobile platforms.
The principle of exploiting redundancy for efficiency ~\cite{siciliano_redound,10008952,deluca_tp} is inspired by biological motor control systems, which prioritize tasks-relevant actions while minimizing unnecessary effort.
Methods such as null space projection
~\cite{7097068,zhang2021improved,9326869} have demonstrated the feasibility of reducing actuator usage without compromising task performance.
For example, in \cite{7017672}, auxiliary tasks were achieved by prescribing motion in the null space of the quadrotor-arm Jacobian while preserving the primary task.
Nevertheless, these methods are tailored to the specific structure of the systems and are not general or easy to extend.
Moreover, the auxiliary tasks are not typically perfectly achieved since they are framed as the minimization of the argument of a given cost function, without the guarantee that this minimum is achieved.
Nonlinear Model Predictive Control (nMPC) offers a powerful alternative~\cite{ref_mpc}, enabling the prioritization of specific inputs through a carefully designed energy cost function ~\cite{2020l-BicMazFarCarFra,10591286}. While it has shown strong practical performance, nMPC is inherently susceptible to local minima, which can compromise system robustness.
Additionally, the presence of conflicting objectives may introduce instability, further limiting its reliability ~\cite{He2015OnSO}.


The present work addresses a class of system that are suitable to be controlled in two possible modes: \textit{energy-saving mode} and \textit{dexterous mode}.
These are system which have a primary \textit{main task} that has to be fulfilled at any time, and in parallel to that, an exogenous system (e.g., a human operator) decides at any time if (1) a secondary \textit{auxiliary task} must be also executed (\textit{dexterous mode}) or (2) a minimum number of control inputs should be used (i.e., have to be non-zero) in order to  minimize energy consumption (\textit{energy-saving mode}).


The main contribution is the development of a generic control framework for nonlinear systems with \textit{theoretical guarantees}.
This framework employs a mode-switching signal to transition between an \textit{energy-saving mode}, where unnecessary inputs are disabled to prioritize the \textit{main task}, and a \textit{dexterous mode}, where the remaining inputs are utilized to address the \textit{auxiliary task}. 
Crucially, we demonstrate that the satisfaction of the \textit{main task} is preserved throughout the process—before, during, and after the switching event.



The paper is organized as follows. Section~\ref{sec:MotExmp} introduces a motivating example. 
Section~\ref{subs:preliminaries}  presents the required notions for describing our framework.
In Sections~\ref{subs:sect_4} and~\ref{sect:method}, we formulate the problem and present the proposed solution.
The paper concludes with a continuation of the motivating example and corresponding simulations.


%\textit{Notations}: In what follows, $c_\theta$ and $s_\theta$ are used as a shorthand for $\cos\theta$ and $\sin\theta$, respectively.

\section{Notation}
We use the shorthand notation \( c_\theta \) for \( \cos\theta \) and \( s_\theta \) for \( \sin\theta \). 
We denote the identity matrix of size \( n \times n \) by \( \mathbf{I}_n \).
Given a vector $\av_m$, we denote its  $i$-th entry by  $a_{m,i}$.

\section{Motivating Example}\label{sec:MotExmp}

As a simple motivating example, consider a four Mecanum wheel omnidirectional vehicle (see Fig.~\ref{fig:model}) whose configuration is given
by the variables $(x,y,\theta)\in \mathbb{R}^2\times \mathcal{S}^1$ representing the
coordinates of the center point and the orientation angle of the vehicle in an inertial
frame, respectively.



The system has three inputs: the sagittal ($v_1$) and transversal ($v_2$) velocities, and the turning rate of the vehicle ($v_3$). 
These inputs are mapped one-to-one to a three dimensional subspace of the four-wheel angular velocities through a simple kinematic relation that is omitted here for simplicity, see~\cite{Muir1990} for the details. 
The dynamical model is given by
\begin{equation}
\Sigma_{\text{4W}}: \left\{ 
\begin{split}
    \dot{x}&=v_1c_\theta -v_2s_\theta,  \\
    \dot{y}&=v_1s_\theta + v_2 c_\theta, \\
     \dot{\theta}&=v_3,
\end{split}
\right.
\label{eq:4w}
\end{equation}
This type of vehicle is more dexterous than non-holonomic systems, like differential drive robots and car-like vehicles, thanks to its ability to also move laterally.
This makes it an ideal solution for precise maneuvering in restricted spaces and motivates its wide use in logistics~\cite{logistics4W,logistics4W_2}.
The downside, as has been shown in~\cite{zhou2016experimental}, is that this vehicle consumes twice the energy when moving in the transversal direction at a given speed as when moving in the sagittal direction at the same speed.

Our goal is obtain a single control scheme that allows the vehicle to seamlessly switch between a \emph{dexterous mode}, where the transversal velocity input $v_2$ is used and each degree of freedom of the configuration $(x,y,\theta)$ is controlled independently, and an \emph{energy-saving mode}, where only the position variables $(x,y)$ are controlled independently (as in a non-holonomic vehicle) and $v_2$ is nullified to minimize the energy consumed. 
We treat the decision about when to switch between the two modes as an a priori unknown exogenous signal $\sigma\in\{0,1\}$, which could be provided by, e.g., a high level operator, depending on the task at hand.
For example, dexterous mode might be requested for accurate maneuvering and manipulation in narrow spaces, and energy saving mode for long distance transportation in large spaces. 

Model identification followed by feedback linearization control schemes are both effective and widely used in practice in industrial robotic systems due to their precise knowledge of system model parameters. 
A naive approach to obtain the sought feedback linearizing controller would be to switch between two different control schemes, one designed  for the dexterous mode and one designed for the energy-saving mode.
In dexterous mode the vector relative degree of the outputs $(x,y,\theta)$ is $\{1,1,1\}$ and a suitable controller is given by
%
\begin{equation}
\begin{split}
   \begin{pmatrix}
v_1\\
v_2\\
v_3
\end{pmatrix} = \begin{pmatrix}
c_\theta & -s_\theta & 0\\
s_\theta & c_\theta & 0\\
0 & 0 & 1
\end{pmatrix}^{-1}\begin{pmatrix}
    \dot{x}^d + k_{p1}(x^d-x) \\
    \dot{y}^d + k_{p2}(y^d-y)  \\
    \dot{\theta}^d + k_{p3}(\theta^d-\theta)
\end{pmatrix}\ ,
\end{split}
\label{eq:contrUFull}
\end{equation}
where $k_{p1},k_{p2},k_{p3} > 0$ are chosen gains.
In energy-saving mode, when $v_2=0$, one can add a dynamic extension of the input $v_1$ which leads to
a vector relative degree of the outputs $(x,y)$ equal to $\{2,2\}$, and a suitable controller is given by

\begin{equation}
\left\{
     \begin{split}
        v_2 &= 0 , \quad v_1(0)\neq 0,\\
        \begin{pmatrix}
        \dot{v}_1\\
        v_3
        \end{pmatrix} 
        &=
             \begin{pmatrix}
        c_\theta & -v_1 s_\theta  \\
        s_\theta & v_1 c_\theta
        \end{pmatrix}^{-1}
                \begin{pmatrix}
        w_1\\
        w_2
        \end{pmatrix}\\ 
        w_1&=  \ddot{x}^d + k_{p1}(x^d - x) + k_{d1}(\dot{x}^d -\dot{x}),\\
        w_2&=\ddot{y}^d + k_{p2}(y^d - y) + k_{d2}(\dot{y}^d -\dot{y}),
    \end{split}
    \right.
    \label{eq:contrUUnd}
\end{equation}
where $k_{p1},k_{p2},k_{d1},k_{d2} > 0$ are chosen gains~\cite{DELUCA2000687}.


The first issue with simply switching between two unrelated control schemes is in the behavior of the output error dynamics.
In both cases, the position $(x,y)$ is controlled as an output of the system, but the relative degree is changed.
This means that it may not be possible to switch between the controllers without inducing a transient behavior in the error dynamics, and in the worst case, it may not even be possible to guarantee stability or convergence when switching occurs.
Another issue in practice is that when $\sigma$ switches, for example, from $1$ to $0$, the transversal velocity $v_2$ is instantaneously set to zero and therefore it is subject to a discontinuous jump, which is undesirable in practice due to the fact that no real vehicle can change its velocity instantaneously due to its inertia. 
Attempting to abruptly halt the vehicle's lateral motion would likely result in slippage or, worse, damage to components of the actual vehicle.
To address this challenge, our goal is to develop a unified controller which receives as input a desired trajectory of the position, a desired orientation angle trajectory, and an exogenous switching signal $\sigma$ which dictates the mode of our controller. 
We require that:
\begin{enumerate}
    \item The position tracking error is exponentially stable and evolves independently of the value of $\sigma$.
    \item The orientation tracking error is exponentially stable whenever $\sigma = 1$ (dexterous mode).
    \item The transversal velocity is brought exponentially to zero whenever $\sigma = 0$ (energy-saving mode) without discontinuity. 
\end{enumerate}

\section{Preliminaries}\label{subs:preliminaries}

For a comprehensive introduction to feedback linearization, the interested reader is referred to \cite{Isidori1995}.

Consider a multivariable nonlinear system
    \begin{equation}
    \left\{
\begin{split}
     \dot{\xv}&= \fv(\xv)+\Gm(\xv)\uv,\\
    \yv &= \hv(\xv),
    \label{eq:sys}
    \end{split}
    \right.
    \end{equation}
where \mbox{$\xv\in \mathbb{R}^n$} is the state, the input matrix is \mbox{$\Gm(\xv)=\begin{bmatrix}
    \gv_1(\xv)  \;\cdots \; \gv_{p}(\xv)
\end{bmatrix}\in \mathbb{R}^{n \times p}$}, $\fv(\xv)$,  $\gv_1(\xv), \ldots, \gv_{p}(\xv)$ are smooth vector fields, and $\hv(\xv)=\begin{bmatrix} h_1(\xv)  \cdots h_{p}(\xv)\end{bmatrix}^\top$ is a smooth function defined on an open set of $\mathbb{R}^n$.
The  system (\ref{eq:sys}) is said to have \emph{(vector) relative degree} $\rv = \{
    r_1, \ldots, r_{p}
\}$ at a point $\xv^\circ$ w.r.t. the input-output pair  $(\uv,\yv)$ if  
\begin{flalign}
&\text{\textrm{(i)}}    & L_{\gv_j}L^{k}_{\fv} h_i(\xv) &=0,&
\end{flalign}
for all $1\leq j \leq p$, for all $k\leq r_i-1$, for all $1\leq i \leq p$ and for all $\xv$ in a neighborhood of $\xv^\circ$, and\\
\textrm{(ii)}\;\;  the $p\times p$ matrix 
   \begin{align}
    \Am(\xv) &:= 
         \begin{pmatrix}
            L_{\gv_1}L^{r_1-1}_{\fv}h_1(\xv) & \cdots&  L_{\gv_{p}}L^{r_1-1}_{\fv}h_1(\xv) \\ 
            L_{\gv_1}L^{r_2-1}_{\fv}h_2(\xv) & \cdots&  L_{\gv_{p}}L^{r_2-1}_{\fv}h_2(\xv) \\  
            \vdots & & \vdots \\
            L_{\gv_1}L^{r_{p}-1}_{\fv}h_{p}(\xv) & \cdots&  L_{\gv_{p}}L^{r_{p}-1}_{\fv}h_{p}(\xv) 
        \end{pmatrix}
    \label{eq:intbmatrix}
\end{align}  
is nonsingular at $\xv = \xv^\circ$. 
The output array at the $\rv$-th derivative may then be written as an affine system of the form
\begin{equation}
{\yv}^{(\rv)} :=\begin{bmatrix}
    y_1^{(r_1)} \;\cdots\;  y_{p}^{(r_{p})}
\end{bmatrix}^\top =\bv(\xv)+\Am(\xv)\uv,
\label{eq:y_r_now}
\end{equation}
with 
\begin{equation}
 \bv(\xv):=\begin{bmatrix}
L_{\boldsymbol{f}}^{(r_1)}{h_1(\xv)} \; \cdots \; 
L_{\boldsymbol{f}}^{(r_{p})}{h_{p}(\xv)}
\end{bmatrix}^\top.
\label{eq:b}
\end{equation}


Suppose the system \eqref{eq:sys} has some \emph{(vector) relative degree} $\rv:=\{r_1,\ldots,r_p\}$ at $\xv^\circ$ and that the matrix $\Gm(\xv^\circ)$ has rank $p$ in a  neighborhood $\mathcal{U}$ of $\xv^\circ$. Suppose also that  \mbox{$r_1+r_2+\ldots+r_p=n$}, and choose the  control input to be $$\uv = \Am^{-1}(\xv)[-\bv(\xv)+\vv],
$$
where $\vv \in \mathbb{R}^{p}$ can be assigned freely and $\Am(\xv), \bv(\xv)$ are defined as in~(\ref{eq:intbmatrix}) and~(\ref{eq:b}).
Then the output dynamics \eqref{eq:y_r_now} become
$$
\yv^{(\rv)} = \vv.
$$
We refer to \(\yv\) as a \textit{linearizing output array}, which possesses the property that the entire state and input of the system can be expressed in terms of \(\yv\) and its time derivatives. 


\section{Learning Method}
As a base architecture for RL, the actor-critic model is used,
in which the actor outputs do not represent probabilities for actions
but instead represent continuous motor commands.
Dynamic RL is applied solely to the actor, while the critic is trained by conventional RL using BPTT \citep{PDP}
although ideally, all learning should be dynamic.
For clarity, each actor and critic consists of a separate RNN with sensor signals as inputs.
%Q-learning is more widely used.
%However, it is the learning for discrete actions, and some more process is required
%before getting the final motor commands.
%From the view of building autonomous learning agents,
%there remains the problem how the process is acquired through RL.
%On the other hand, the outputs of the actor in actor-critic can be dealt with as continuous motion signals.

%Figure \ref{fig:neuron_forward} shows a general static-type neuron model with $m$ inputs.
In each dynamic neuron, its internal state $u$ at time $t$ is derived
as the first-order lag of the inner product of the connection weight vector ${\bf w}=(w_1, ... , w_m)^\mathrm{T}$
and input vector ${\bf x}_t=(x_{1t}, ... , x_{mt})^\mathrm{T}$ where $m$ is the number of inputs as
\begin{equation}
u_t = \left(1-\frac{\Delta t}{\tau}\right)u_{t-1}+\frac{\Delta t}{\tau}{\bf w}\cdot{\bf x}_t
\label{Eq:internal_state}
\end{equation}
where $\tau$ is a time constant and $\Delta t$ is the step width, which is 1.0 in this paper.
For static-type neurons, the internal state $u$ is just the inner product as
\begin{equation}
u_t = {\bf w}\cdot{\bf x}_t.
\label{Eq:internal_state_static}
\end{equation}
%by setting $\tau=\Delta t$.
The inputs ${\bf x}_t$ can be the external inputs or the pre-synaptic neuron outputs at time $t$,
%which may be outputs of neurons.
but for the feedback connections, where the inputs come from the same or an upper layer,
they are the outputs of the pre-synaptic neuron at time $t-1$. 
The output $o_t$ is derived from the internal state $u_t$ as
\begin{equation}
o_t = f(U_t)=f(u_t+\theta)
\label{Eq:output}
\end{equation}
where $U_t=u_t+\theta$, $\theta$ is the bias, and $f(\cdot)$ is an activation function,
which is a hyperbolic tangent in this paper.

Dynamic RL controls the dynamics of the system, including RNN, directly by adjusting the sensitivity \citep{Sensitivity} in each neuron.
Sensitivity is an index for each neuron that is the Euclidian norm of the output gradient
with respect to the input vector ${\bf x}$.
It is defined as
%how a neuron is sensitive to a small change in its inputs.
%It is defined as the Euclidean norm of the output gradient with respect to the input vector ${\bf x}$ as
\begin{equation}
s(U; {\bf w}) = \|\nabla_{\bf x} o\| = f'(U)\|{\bf w}\|.
\label{Eq:sensitivity}
\end{equation}
Here, $\| {\bf v} \| = \sqrt{\sum_i^mv_i^2}$ for a vector ${\bf v}=(v_1, ..., v_m)^\mathrm{T}$.
In the form of a vector elements, the sensitivity is represented as
\begin{equation}
s(U; {\bf w}) = \sqrt{\sum_i^m \left( \frac{\partial o}{\partial x_i} \right)^2} = f'(U)\sqrt{\sum_i^m w_i^2}\ .
\label{Eq:sensitivity_non_vector}
\end{equation}
Sensitivity refers to the maximum ratio of the absolute value of the output deviation $do$
to the magnitude of the infinitesimal variation $d{\bf x}$ in the input vector space.
It represents the degree of contraction or expansion from the neighborhood around the current inputs
to the corresponding neighborhood around the current output through the neuron's processing.
In the previous work \citep{Sensitivity}, it was defined only for static-type neurons
(Eq.~(\ref{Eq:internal_state_static})).
In this study, the same definition is also applied to dynamic neurons (Eq.~(\ref{Eq:internal_state})),
assuming that the infinitesimal variation $d{\bf x}$ of the input ${\bf x}$
changes slowly enough compared to the time constant $\tau$.
%it is assumed that the infinitesimal deviation $d{\bf x}$ of the input ${\bf x}$
%is a constant vector near the time $t$.
%By solving the linear asymptotic equation as in Eq.~(\ref{Eq:internal_state}),
%the deviation $du$ of the internal state can be represented as
%\begin{equation}
%du_t \approx \left\{1+\left(1-\alpha\right)+\left(1-\alpha\right)^2+...\right\}\alpha{\bf w}\cdot d{\bf x}_t 
%= \sum_{i=0}^\infty (1-\alpha)^i\alpha{\bf w}\cdot d{\bf x}_t = {\bf w}\cdot d{\bf x}_t
%\end{equation}
%where $0.0 < \alpha = \frac{\Delta t}{\tau} \leq 1.0$.
%Then the gradient of the internal state $u$ with respect to the input ${\bf x}$ becomes
%\begin{equation}
% \|\nabla_{{\bf x}_t} u_t\| = \|{\bf w}\|,
%\end{equation}
%and we can derive Eq.~(\ref{Eq:sensitivity}) as well also for the dynamic neurons.
%if the activation function $f$ is a monotonically increasing function.

In the previous research \citep{Sensitivity},  the author's group proposed sensitivity adjustment learning (SAL).
SAL was applied to ensure the sensitivity of each neuron in parallel with gradient-based supervised learning.
This approach is beneficial not only for maintaining sensitivity during forward computation in the neural network
but also for avoiding diverging or vanishing gradients during backward computation.
Because Dynamic RL incorporates SAL and sensitivity-controlled RL (SRL), which is an extension of SAL for RL,
SAL will be explained first.

In SAL, the moving average of sensitivity $\bar{s}$ is computed first as
\begin{equation}
 \bar{s}_t \leftarrow (1-\alpha) \bar{s}_{t-1}  + \alpha s_t
 \label{Eq:Ave_sen}
\end{equation}
where $\alpha$ is a small constant, and this computation is performed across episodes.
When the average sensitivity $\bar{s}$ is below a predetermined constant $s_{th}$,
the weights and bias in each neuron are updated locally to the gradient direction of the sensitivity as
\begin{align}
\Delta {\bf w}_t &= \eta_{SAL}\frac{\Delta t}{\tau} \nabla_{\bf w} s(U_t; {\bf w})
%                        = \eta_{SAL}\frac{\Delta t}{\tau} \nabla_{\bf w} \{f'(U_t)\|{\bf w}\|\}
                        = \eta_{SRL}\frac{\Delta t}{\tau} \left( f'(U_t)\frac{\bf w}{\| {\bf w} \|} + \| {\bf w} \| \nabla_{\bf w} f'(U_t) \right)
\label{Eq:SAL_ORG}\\
%\end{equation}
%and
%\begin{equation}
\Delta {\theta}_t &= \eta_{SAL} \frac{\Delta t}{\tau}\frac{\partial s(U_t; {\bf w})}{\partial \theta}
%                          = \eta_{SAL} \frac{\Delta t}{\tau}\frac{\partial \{f'(U_t)\|{\bf w}\|\}}{\partial \theta}
                          = \eta_{SRL}\frac{\Delta t}{\tau} \| {\bf w} \| \frac{\partial f'(U_t)}{\partial \theta}.
\label{Eq:SAL_Bias_ORG}
\end{align}
where $\eta_{SAL}$ is the learning rate for SAL.
$\Delta t / \tau$ is multiplied to adjust the update to the neuron's time scale.
By expanding the equation with the activation function being hyperbolic tangent,
\begin{align}
\Delta {\bf w}_t &= \eta_{SAL}\frac{\Delta t}{\tau} (1-o_t^2) \left( \frac{\bf w}{\| {\bf w}\|} - 2o_t\|{\bf w}\|{\bf x}_t \right)
\label{Eq:SAL} \\
%\end{equation}
%\begin{equation}
\Delta {\theta}_t &= -2 \eta_{SAL}\frac{\Delta t}{\tau} o_t(1-o_t^2) ||{\bf w}||
\label{Eq:SAL_Bias}
\end{align}
are derived.
%, where
%\begin{equation}
%\frac{do}{dU} = f'(U) = \frac{dtanh(U)}{dU} =\frac{1}{\cosh^2(U)} = 1- o^2.
%\end{equation}

\begin{figure}[t]
\centerline{\includegraphics[scale=0.35]{DynamicRL.pdf}}
%\centerline{\includegraphics[scale=0.5, pagebox=cropbox, clip]{Task1.pdf}} 
\caption{Dynamic RL applies either SAL or SRL depending on the condition in each neuron.}
\label{fig:DynamicRL}
\end{figure}
In Dynamic RL proposed here, as shown in Fig.\ref{fig:DynamicRL},
SAL is applied when the moving average of the sensitivity $\overline{s}$
is less than a constant $s_{th}$, otherwise sensitivity-controlled RL (SRL) is applied in each neuron.
%When not less
SAL always tries to increase the sensitivity in each neuron,
but whether SRL tries to increase or decrease the sensitivity depends on the temporal difference (TD) error ${\hat r}$ as
\begin{align}
\Delta {\bf w}_t &= -\eta_{SRL}\frac{\Delta t}{\tau} \hat{r}_t \nabla_{\bf w} s(U_t; {\bf w})\\
%\label{Eq:SRL} \\
%\end{equation}
%\begin{equation}
\Delta \theta_t &= -\eta_{SRL}\frac{\Delta t}{\tau} \hat{r}_t \frac{\partial s(U_t; {\bf w})}{\partial \theta}
%\label{Eq:SRL_Bias}
\end{align}
where $\eta_{SRL}$ is the learning rate for SRL.
TD error is computed as
\begin{equation}
\hat{r}_t = \gamma C_{t+1} + r_{t+1} - C_t = \gamma\left(C_{t+1}-\frac{C_t-r_{t+1}}{\gamma}\right)
\label{Eq:TDerr}
\end{equation}
where $\gamma\ (0.0<\gamma<1.0)$ is the discount factor, $C_t$ is the critic output (state value),
and $r_t$ is the reinforcement signal, which can be a reward or a penalty, at time $t$.
As the basic concept summarized in Fig.\ref{fig:BasicConcept},
when TD error is positive, in other words, the new critic (state value) $C_{t+1}$ is greater than the expected value
$\frac{C_t-r_{t+1}}{\gamma}$,
RL reduces the sensitivity to reinforce the reproducibility.
When it is negative, i.e., the new state value is less than expected,
RL makes the sensitivity greater to reinforce the exploratory nature.
This is expected to control the local convergence or divergence, depending on how good or bad the state is.

\begin{figure}[ht]
\centerline{\includegraphics[scale=0.28]{BasicConcept.pdf}}
%\centerline{\includegraphics[scale=0.5, pagebox=cropbox, clip]{Task1.pdf}} 
\caption{Basic concept of Dynamic RL (or more specifically, SRL) proposed in this paper.}
\label{fig:BasicConcept}
\end{figure}

Upon expansion, we obtain,
\begin{align}
\Delta {\bf w}_t &= -\eta_{SRL}\frac{\Delta t}{\tau} \hat{r}_t \left( f'(U_t)\frac{\bf w}{\| {\bf w} \|} + \| {\bf w} \| \nabla_{\bf w} f'(U_t) \right)\\
\Delta \theta_t &= -\eta_{SRL}\frac{\Delta t}{\tau} \hat{r}_t \| {\bf w} \| \frac{\partial f'(U_t)}{\partial \theta}.
\end{align}
By further expanding as the activation function $f(\cdot)$ being $\tanh$,
\begin{align}
\Delta {\bf w}_t &= - \eta_{SRL}\frac{\Delta t}{\tau} \hat{r}_t (1-o_t^2) \left( \frac{\bf w}{\| {\bf w}\|} - 2o_t\|{\bf w}\|{\bf x}_t \right)
\label{Eq:SRL}\\
\Delta \theta_t &= 2\eta_{SRL}\frac{\Delta t}{\tau} \hat{r}_t o_t (1-o_t^2) \|{\bf w}\|.
\label{Eq:SRL_Bias}
\end{align}
%The equation is rewritten in the element form as
%\begin{equation}
%\Delta w_i = \eta_{SAL} \frac{(1-o^2) \left\{ w_i -2ox_i \sum_k w_k^2 \right\}}{\sqrt{\sum_k w_k^2}}.
%\end{equation}
%The author calls the first term $-\eta \hat{r} (1-o^2){\bf w}/\left|{\bf w}\right|$ the linear term,
%which is originated from $|{\bf w}|$ in Eq.~(\ref{Eq:sensitivity}).
%The second term $2 \eta \hat{r} (1-o^2) o|{\bf w}|{\bf x}$ is called non-linear term,
%which is originated from $f'(x)$ in Eq.~(\ref{Eq:sensitivity}).
%Different from the case of weight, bias $\theta$ cannot increase the sensitivity directly, but
%can increase it indirectly by updating the bias so that the value $U$ becomes closer to 0.0.
Notably, this computation can be done locally in each neuron except for receiving the TD errors.
Furthermore, since the dynamics are generated not only by the loops inside the RNN
but also influenced by the loops that are formed with the outside world,
this learning can be applied to all the neurons, including those outside the loop in the RNN, including the output neurons.

In the following simulations, the proposed RL is compared to the conventional RL using BPTT.
%Then, the conventional RL used here is explained next.
Now many techniques have been proposed to improve the performance, but for a pure comparison of base methods,
simple learning using gradient-based BPTT is employed.
In Dynamic RL, the motor command vector ${\bf M}_t$
is a function ${\bf M}(\cdot)$ of the actor output vector ${\bf A}_t$ as ${\bf M}_t = {\bf M}({\bf A}_t)$,
%is identical to the actor output vector ${\bf A}_t$,
but in the conventional RL, since a random noise vector ${\bf \epsilon}_t$ is added to the actor output vector
as explorations, the actual motor command vector ${\bf M}_t$ is expressed as
\begin{equation}
%{\bf M}_t = {\bf A}_t + {\bf \epsilon}_t
{\bf M}_t = {\bf M}({\bf A}_t + {\bf \epsilon}_t)
\end{equation}
For conventional RL, training signals for the actor network are derived as
\begin{equation}
{\bf A}_{train,t} = {\bf A}_t + \hat{r}_t {\bf \epsilon}_t .
\label{Eq:ConvRL}
\end{equation}
Then, the actor network is trained based on the BPTT method by these training signals.
In this paper, it learned 10 or 20 steps backward in time, depending on the task.
While, in the Dynamic RL, since no calculation going back through time is necessary,
the computational cost is considerably smaller than in the case of conventional RL.

\begin{figure}[t]
\centerline{\includegraphics[scale=0.31]{ConvRL.pdf}}
%\centerline{\includegraphics[scale=0.5, pagebox=cropbox, clip]{Task1.pdf}} 
\caption{A conceptual diagram of conventional RL.
RL aims to control the actor output vector based on the TD error.
It does not utilize information about time changes in the RNN's state and is closed only at each step.}
\label{fig:ConvRL}
\vspace{5mm}
\centerline{\includegraphics[scale=0.31]{DYN_RL.pdf}}
%\centerline{\includegraphics[scale=0.5, pagebox=cropbox, clip]{Task1.pdf}} 
\caption{A conceptual diagram of Dynamic Reinforcement Learning (RL).
RL aims to control the convergence or divergence of the flow around state transitions according to the TD error
by controlling the sensitivity in each neuron.}
\label{fig:DYN_RL}
\end{figure}
%As described in the Introduction, 
Dynamic RL has a significant difference in the way of learning
from conventional RL.
For better understanding, the author attempts to illustrate their differences with diagrams at the expense of accuracy.
In the conventional RL, external noise ${\bf \epsilon}$  is added to the actor output vector ${\bf A}$.
As shown in Fig.~\ref{fig:ConvRL}, according to Eq.~(\ref{Eq:ConvRL}),
if the value function is better than expected, i.e., if the TD error $\hat{r}$ is positive,
the network is trained to move the output vector ${\bf A}$ to the direction of the noise ${\bf \epsilon}$.
By contrast, if the TD error $\hat{r}$ is negative, the network is trained to move the output vector ${\bf A}$ to the opposite direction.
This RL does not use the temporal change in the outputs or network states; rather, it considers only the outputs at that moment in time.
All the weights and biases are updated to move the output vector with the help of the gradient method
using error backpropagation even through time.

On the other hand, Dynamic RL does not aim to move the state or output directly,
but as shown in Fig.~\ref{fig:DYN_RL}, it aims to control the convergence or divergence of the neighborhood
around the state transition by changing each neuron's sensitivity depending on the TD error.
The concept of controlling dynamics can also be applied to other types of learning, such as supervised learning.
The author refers to it as Dynamic Learning from a broader perspective and will discuss it in the subsection
\ref{subsec:Future}.
%Therefore, the learning in the neurons that are not included in any loop in the RNN
%also influences the dynamics.

%In this paper, to improve the performance further,
%another gradient-based learning is applied to the output neurons referring to \citep{Hoerzer,Matsuki}.
%Here, the deviation from the moving average is computed.
%\begin{equation}
%\tilde{o}_t = {o}_t - \bar{o}_t
%\end{equation}
%where $\bar{o}_t = 0.8 \bar{o}_{t-1} + 0.2 {o}_t$, and the weight vector is updated
%by the product of it and TD error as
%\begin{equation}
%\Delta {\bf w}_t = \eta_{grad} \hat{r}_t \tilde{o}_t {\bf x}_t.
%\label{Eq:GradL}
%\end{equation}
%The biases are updated as
%\begin{equation}
%\Delta {\bf \theta}_t = \eta_{grad} \hat{r}_t \tilde{o}_t.
%\label{Eq:GradL_Bias}
%\end{equation}
This concept should also be introduced to the critic network,
but here, conventional learning is used for the critic, regardless of how the actor network is trained.
The training signal is derived as
\begin{equation}
C_{train,t} = \gamma C_{t+1} + r_{t+1} = C_t + \hat{r}_t,
\label{Eq:C_train}
\end{equation}
and the critic network is always trained with BPTT using this training signal.

In Dynamic RL, the network outputs were often saturated (close to $1.0$ or $-1.0$ in hyperbolic tangent),
and it is difficult to perform fine and smooth control.
To avoid saturation, the regularization was applied only to the output layer's connection weights in the actor network as
\begin{equation}
  \Delta {\bf W} = -\eta_{reg} {\bf W}.
  \label{Eq:Regularize}
\end{equation} 
This learning was applied in both Dynamic and conventional RL cases.
% for fair comparison.
 
Furthermore, one more technique used in this paper is ``critic raising''.
When an agent cannot reach its goal for a long time,
since the critic output becomes small, the gradient of the critic also becomes small.
Therefore, referring to the ``optimistic initial value'' \citep{Sutton1998},
when the moving average $\bar{C}$ of the critic output $C$ is less than a constant $C_{th}$,
the bias of the output layer in the critic network is increased to raise the critic value as
\begin{equation}
  \Delta \theta_t = \eta_{raise} (C_{th}-\bar{C}_t)
  \label{Eq:Raise_Critic}
\end{equation} 
where $\bar{C}$ is the moving average of the critic output $C$ as 
\begin{equation}
 \bar{C}_t \leftarrow (1-\beta) \bar{C}_{t-1}  + \beta C_t
 \label{Eq:Ave_C}
\end{equation}
where $\beta$ is a small constant, and this computation is performed across episodes.
%, but except for the preparation steps,
%in which only the RNN was computed without actually moving for preparation.
%This was also applied in both Dynamic and conventional RL cases.





\section{Application to the Motivating Example}



The Mecanum wheel robot system \eqref{eq:4w} presented in Section \ref{sec:MotExmp} can be cast into the required form \eqref{eq:sys} by using a dynamic extension of the linear velocities $v_1, v_2$.
We thus define the state variables as $\xv_1 = \begin{bmatrix} x & y & \theta & v_1 \end{bmatrix}^\top$ and $\xv_2 = v_2$.
This extension incorporates the energy-intense input as part of the state, and is necessary so that the resulting system satisfies the assumptions \ref{asmpt_1} and \ref{asmpt_2}.

\subsection{Derivation of the controller equations}

When operating the platform in \textit{energy-saving mode}, the objective is to reduce the transversal velocity $v_2$ to zero, thereby minimizing energy consumption. 
Conversely, in \textit{dexterous mode}, the goal is to achieve trajectory tracking of the full configuration of the vehicle  (position and orientation).
To address these objectives, we define the output vector as $\yv = \begin{bmatrix} \yv_1^\top & y_2 & y_3 \end{bmatrix}^\top$, where $\yv_1 = \begin{bmatrix} x & y \end{bmatrix}^\top$, $y_2 = \theta$, and $y_3 = v_2$. The input vector is given by $\uv = \begin{bmatrix} u_1 & u_2 & u_3 \end{bmatrix}^\top$, with
    $\dot{v}_1=u_1$, 
    $\dot{v}_2=u_2$ 
 and $u_3 := v_3$.
The system exhibits a \emph{(vector) relative degree} $\rv = \{\rhov_1, \rhov_2\}$, where $\rhov_1 = \{2, 2\}$ and $\rhov_2 = 1$, relative to the pair $(\uv, \yv)$. Additionally, it exhibits a  \emph{(vector) relative degree}  $\bar{\rv} = \{\rhov_1, \rhov_3\}$, with $\rhov_3 = 1$, relative to the pair  $(\uv, \bar{\yv})$. Consequently, Assumptions~\ref{asmpt_1} and~\ref{asmpt_2} hold, allowing the application of Theorem~\ref{thm:1}.
Consider now the output ${\yv}_{\sigma}$ defined  in \eqref{y_tilde}. At the $\tilde{\rv}$-th differentation, the dynamics are given by $$
{\yv}^{({\rv}_{\sigma})}_{\sigma}={\Am}_{\sigma}(\xv)\uv 
$$ where the interaction matrix defined in~\eqref{eq:Atilde} is 
$${\Am}_{\sigma}(\xv) = \begin{pmatrix}
    c_\theta  & -s_\theta &-v_1s_\theta-v_2c_\theta   \\ s_\theta& c_\theta& v_1c_\theta-v_2s_\theta  \\
    0 & 1-\sigma & \sigma
\end{pmatrix}$$% 
and $\bv(\xv)=\boldsymbol{0}$. Then, using \eqref{eq:control}, the control law  becomes 
\begin{equation}
    \uv = {\Am}_{\sigma}(\xv)^{-1}\vv_{\sigma}
\label{eq:c_un}
\end{equation}
 with $\vv_{\sigma}$ as in \eqref{eq:v} and the $\vv_j$ as in \eqref{eq:control2} solves the Problem~\ref{prob:main_problem} for the four mecanum wheel vehicle.


\begin{figure}[t]
\centering
\includegraphics[width=0.54\columnwidth]{mizzo3.jpg}
\hfill
\includegraphics[width=0.44\columnwidth]{mizzo4.jpg}
      \caption{Stroboscopic highlights of two simulations.
      In \textbf{Simulation~1}  (left), the robot converges to and follows a circular trajectory.
      In \textbf{Simulation~2} (right), the robot converges to a straight-line trajectory  while carrying a load (depicted in orange) and avoiding hanging obstacles (shown in red).
      The robot operates in dexterity mode only when necessary (as determined by the switching signal $\sigma$), prioritizing energy-saving mode when far from obstacles.}
\label{fig:animation}
\end{figure}

\begin{rem}
The dynamic extension of the sagittal velocity leads to a singularity in the resulting controller.
Specifically, the decoupling matrix ${\Am}_{\sigma}(\xv)$ becomes singular when $\sigma = 0$ and $v_1 = 0$.
Such a singularity must be carefully addressed and avoided during trajectory planning, particularly when employing interpolation techniques. This can typically be achieved by appropriately selecting the initialization of the state $v_1$—an additional degree of freedom available in the design.
   
\end{rem}





\subsection{Numerical Simulations}
To demonstrate the capabilities of the proposed controller in a realistic scenario, we present two simulations where a four-Mecanum-wheel omnidirectional robot is tasked with transporting a load while following two different trajectories in the presence of obstacles (see Fig.~\ref{fig:animation}). 


The first trajectory is a circular path defined by:
\[
\yv^d_1(t) = \begin{bmatrix}
r\sin(\omega t) &
-r\cos(\omega t)
\end{bmatrix}^\top, 
\]
where $ r = 8 \, \text{m} $ and $ \omega = 0.15 \, \text{rad/s} $. 

Two hanging obstacles are positioned along the trajectory. When the robot encounters these obstacles, a reorientation of $\pi/2 \, \text{rad}$ is required to avoid collisions between the transported load and the obstacles. The desired orientation for obstacle avoidance is defined as:
\[
y_2^d(t) = \omega t + \frac{\pi}{2}.
\]

For the majority of the trajectory, the switching signal $\sigma$ is set to zero, meaning that the robot operates in energy-saving mode. However, when passing beneath the obstacles, $\sigma$ switches to one, activating the dexterous mode. In this mode, the robot adjusts its orientation to avoid collisions while continuing to follow the position trajectory. 

\begin{figure}[t]
    \centering
    \includegraphics[trim={0.5cm 0cm 0cm 0cm},clip,scale=0.174]{mizzo5.pdf}
    \caption{\textbf{Simulation~1.}  A circular input reference trajectory for the position of the CoM and a square form switching signal ${\sigma}$ are given to the control system. The gray areas correspond to $\sigma=0$ whereas the orange areas correspond to $\sigma=1$. The top row shows the output variables $\yv_1,\yv_2$, and the bottom row shows the sagittal velocity $v_1$, the third output $v_2$ and the control input $u_3$.  }
    \label{fig:circle}
\end{figure}

\begin{figure}[t]
    \centering
    \includegraphics[trim={0.6cm 0cm 0cm 0cm},clip,scale=0.174]{mizzo6.pdf}
\caption{\textbf{Simulation~2.} A ramp input reference trajectory with a square form switching signal $\sigma$ are given to the control system.
The gray areas correspond to $\sigma=0$ whereas the orange areas correspond to $\sigma=1$. The top row shows the output variables $\yv_1,\yv_2$, and the bottom row shows the sagittal velocity $v_1$, the third output $v_2$ and the control input $u_3$. }
    \label{fig:ramp} 
\end{figure}


Fig.~\ref{fig:circle} illustrates the simulation results. The gray areas correspond to $\sigma=0$ (energy-saving mode), while the orange areas represent $\sigma=1$ (dexterous mode).
The plots show the output variables $\yv_1$ and $\yv_2$ at the top, and the sagittal velocity $v_1$, the third output $v_2$, and the control input $u_3$ at the bottom.



 
The second trajectory is a straight line defined by:
\[
\yv^d_1(t) = \begin{bmatrix}
5 + \frac{t}{4} &
5 + \frac{t}{4}
\end{bmatrix}^\top.
\]

Similarly, a hanging obstacle is placed along this trajectory. To avoid a collision, the desired orientation is given by:
\[
y_2^d(t) = \frac{3\pi}{4}.
\]


The initial conditions for both simulations were set far from the trajectory to showcase transient behaviors. The gain matrices used were:
\[
\Lm^1_{1} = \mathbf{I}_2, \quad
\Lm^2_{1} = \mathbf{I}_2, \quad
L^1_2 = 0.75, \quad
L^1_3 = 0.65.
\]
This simulation also incorporated a low-pass filtered Gaussian noise $\nv \in \mathbb{R}^3$ in the actuation inputs to enhance realism. 
Precisely, to the control input \eqref{eq:c_un} we add the noise resulting from the solution of 
\begin{align}\label{eq:lowpass-noise}
    \dot{\nv}=-k\nv+\boldsymbol{\mu}, && \boldsymbol{\mu} \in \mathcal{N}(\boldsymbol{0}, q^2 \mathbf{I}_3),
\end{align}
where $q = 0.4$ and, $k=0.1$. 
Figure \ref{fig:noise} shows the components of the noise $\nv$ added to the actuators during Simulation~1.





%\begin{comment}
\begin{figure}[t]
    \centering
    \includegraphics[trim={0.04cm 0cm 0cm 0cm},clip,scale=0.171]{mizzo7.pdf}
    \caption{A realization of the Gaussian noise resulting from \eqref{eq:lowpass-noise}.}
    \label{fig:noise}
\end{figure}
%\end{comment}
% The robot begins from an initial state far from the trajectory to highlight its transient behavior. Fig.~\ref{fig:ramp} shows the results, with the same conventions for $\sigma=0$ and $\sigma=1$ as in Simulation~1.

 
The plots for both simulations (Figs.~\ref{fig:circle} and~\ref{fig:ramp}) demonstrate the controller's ability to solve the trajectory tracking problem effectively. When the switching signal $\sigma=1$, the system transitions into dexterous mode, where the smooth tracking of both $\yv_1$ and $\yv_2$ is ensured.
When the switching signal $\sigma=0$, the system transitions into energy saving mode, where
the velocity $v_2$ is brought exponentially to zero while ensuring smooth tracking of $\yv_1$.
It is noteworthy that while the trajectory $y_2^d(t)$ is defined at all times, it is only enforced when $\sigma=1$. When $\sigma=0$, the angle variable $\theta$ ceases to follow the prescribed trajectory. Instead, the robot moves in a unicycle-like fashion, with $\theta(t)$ evolving according to the platform’s flat outputs $x$ and $y$, i.e., $\theta(t) = \mathrm{atan2}\{\dot{y}, \dot{x}\}$. Despite this deviation in $\theta$, the position coordinates $x$ and $y$, and hence $\yv_1$, continue to perfectly track the desired trajectory.


The videos corresponding to the simulations of Figs.~\ref{fig:circle} and~\ref{fig:ramp} are available at \mbox{{\small\url{https://youtu.be/Wn4hVNXEjmc}}}. 









\begin{comment}
    \section{Practical Remarks}

\begin{rem}
If one wants to include also input constraints in the problem, a practical possibility is to replace the implementation of the controller via matrix inversion in~\eqref{eq:control} with the solution of the following (convex) optimization problem 
\begin{equation}
   \begin{split}
       \uv^* = \argmin &||\vv - \bv(\xv)-\tilde{\Am}(\xv,\sigma)\uv||_\Wm ^2 \\
       \mathrm{s.t.}& \uv \in \mathcal{U} 
   \end{split} 
\end{equation}
with $\bv(\xv)$ as in \eqref{eq:b}, $\vv$ as in \eqref{eq:v}, $\tilde{\Am}(\xv,\sigma)$  as in \eqref{eq:Atilde} and \textit{eventually} a  positive-definite weight matrix $\Wm$.
\end{rem}
\end{comment}
We present RiskHarvester, a risk-based tool to compute a security risk score based on the value of the asset and ease of attack on a database. We calculated the value of asset by identifying the sensitive data categories present in a database from the database keywords. We utilized data flow analysis, SQL, and Object Relational Mapper (ORM) parsing to identify the database keywords. To calculate the ease of attack, we utilized passive network analysis to retrieve the database host information. To evaluate RiskHarvester, we curated RiskBench, a benchmark of 1,791 database secret-asset pairs with sensitive data categories and host information manually retrieved from 188 GitHub repositories. RiskHarvester demonstrates precision of (95\%) and recall (90\%) in detecting database keywords for the value of asset and precision of (96\%) and recall (94\%) in detecting valid hosts for ease of attack. Finally, we conducted an online survey to understand whether developers prioritize secret removal based on security risk score. We found that 86\% of the developers prioritized the secrets for removal with descending security risk scores.

\balance
\bibliographystyle{IEEEtran}
\bibliography{bibAlias,bibAF,bibCustom,ref2}



\end{document}
