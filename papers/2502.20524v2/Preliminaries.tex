\section{Preliminaries}\label{subs:preliminaries}

For a comprehensive introduction to feedback linearization, the interested reader is referred to \cite{Isidori1995}.

Consider a multivariable nonlinear system
    \begin{equation}
    \left\{
\begin{split}
     \dot{\xv}&= \fv(\xv)+\Gm(\xv)\uv,\\
    \yv &= \hv(\xv),
    \label{eq:sys}
    \end{split}
    \right.
    \end{equation}
where \mbox{$\xv\in \mathbb{R}^n$} is the state, the input matrix is \mbox{$\Gm(\xv)=\begin{bmatrix}
    \gv_1(\xv)  \;\cdots \; \gv_{p}(\xv)
\end{bmatrix}\in \mathbb{R}^{n \times p}$}, $\fv(\xv)$,  $\gv_1(\xv), \ldots, \gv_{p}(\xv)$ are smooth vector fields, and $\hv(\xv)=\begin{bmatrix} h_1(\xv)  \cdots h_{p}(\xv)\end{bmatrix}^\top$ is a smooth function defined on an open set of $\mathbb{R}^n$.
The  system (\ref{eq:sys}) is said to have \emph{(vector) relative degree} $\rv = \{
    r_1, \ldots, r_{p}
\}$ at a point $\xv^\circ$ w.r.t. the input-output pair  $(\uv,\yv)$ if  
\begin{flalign}
&\text{\textrm{(i)}}    & L_{\gv_j}L^{k}_{\fv} h_i(\xv) &=0,&
\end{flalign}
for all $1\leq j \leq p$, for all $k\leq r_i-1$, for all $1\leq i \leq p$ and for all $\xv$ in a neighborhood of $\xv^\circ$, and\\
\textrm{(ii)}\;\;  the $p\times p$ matrix 
   \begin{align}
    \Am(\xv) &:= 
         \begin{pmatrix}
            L_{\gv_1}L^{r_1-1}_{\fv}h_1(\xv) & \cdots&  L_{\gv_{p}}L^{r_1-1}_{\fv}h_1(\xv) \\ 
            L_{\gv_1}L^{r_2-1}_{\fv}h_2(\xv) & \cdots&  L_{\gv_{p}}L^{r_2-1}_{\fv}h_2(\xv) \\  
            \vdots & & \vdots \\
            L_{\gv_1}L^{r_{p}-1}_{\fv}h_{p}(\xv) & \cdots&  L_{\gv_{p}}L^{r_{p}-1}_{\fv}h_{p}(\xv) 
        \end{pmatrix}
    \label{eq:intbmatrix}
\end{align}  
is nonsingular at $\xv = \xv^\circ$. 
The output array at the $\rv$-th derivative may then be written as an affine system of the form
\begin{equation}
{\yv}^{(\rv)} :=\begin{bmatrix}
    y_1^{(r_1)} \;\cdots\;  y_{p}^{(r_{p})}
\end{bmatrix}^\top =\bv(\xv)+\Am(\xv)\uv,
\label{eq:y_r_now}
\end{equation}
with 
\begin{equation}
 \bv(\xv):=\begin{bmatrix}
L_{\boldsymbol{f}}^{(r_1)}{h_1(\xv)} \; \cdots \; 
L_{\boldsymbol{f}}^{(r_{p})}{h_{p}(\xv)}
\end{bmatrix}^\top.
\label{eq:b}
\end{equation}


Suppose the system \eqref{eq:sys} has some \emph{(vector) relative degree} $\rv:=\{r_1,\ldots,r_p\}$ at $\xv^\circ$ and that the matrix $\Gm(\xv^\circ)$ has rank $p$ in a  neighborhood $\mathcal{U}$ of $\xv^\circ$. Suppose also that  \mbox{$r_1+r_2+\ldots+r_p=n$}, and choose the  control input to be $$\uv = \Am^{-1}(\xv)[-\bv(\xv)+\vv],
$$
where $\vv \in \mathbb{R}^{p}$ can be assigned freely and $\Am(\xv), \bv(\xv)$ are defined as in~(\ref{eq:intbmatrix}) and~(\ref{eq:b}).
Then the output dynamics \eqref{eq:y_r_now} become
$$
\yv^{(\rv)} = \vv.
$$
We refer to \(\yv\) as a \textit{linearizing output array}, which possesses the property that the entire state and input of the system can be expressed in terms of \(\yv\) and its time derivatives. 

