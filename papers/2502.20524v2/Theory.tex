 
\section{Problem Formulation}\label{subs:sect_4}

In this section, we generalise our motivating example and state our formal problem definition.

Consider a system $\Sigma$ of the form
\begin{equation}
\Sigma:\left\{
    \begin{split}
\dot{\xv}&=\fv(\xv)+\Gm(\xv)\uv, \\
\yv_1&=\hv_1(\xv_1),\\
\yv_2&=\hv_2(\xv_1),\\
\yv_3&=\hv_3(\xv_2),
\label{eq:Sigma12}
\end{split}
\right.
\end{equation} with $\xv=\begin{bmatrix}
    \xv_1^\top & \xv_2^\top
\end{bmatrix}^\top\in\mathbb{R}^n$, where $\uv\in \mathbb{R}^{p} $ is the control input, and $\yv_i \in \mathbb{R}^{p_i}$ are the output arrays, with $i=1,2,3$, $p_1+p_2=p$, and $p_3=p_2$. We
refer to  $\xv_2$ as the state of the \emph{energy-intense} actuation part and we assume that $\Sigma$ possesses the property that $\xv_2 = \boldsymbol{0}$ whenever $\yv_3 = \boldsymbol{0}$, $\dot{\yv}_3 = \boldsymbol{0}$, and all higher-order derivatives of $\yv_3$ up to a finite order are $\boldsymbol{0}$.

Denote with $\yv = \begin{bmatrix}
    \yv_1^\top & \yv_2^\top
\end{bmatrix}^\top$ and with $\bar{\yv} = \begin{bmatrix}
    \yv_1^\top & \yv_3^\top
\end{bmatrix}^\top$.
%We assume
Consider a desired output  trajectory \mbox{$\yv^{d}=\begin{bmatrix}
    (\yv^{d}_1)^\top &  (\yv^{d}_2)^\top & \boldsymbol{0}^\top
\end{bmatrix}^\top$} where \mbox{$\yv^{d}_i: [0,\infty) \to \mathbb{R}^{p_i}$}, $i=1,2$,  a signal $\sigma: [0,\infty) \to \{0,1\}$, and define the error $
\ev = \begin{bmatrix}
    \ev_1^\top &
    \ev_2^\top &
    \ev_3^\top
\end{bmatrix} := \begin{bmatrix}
    \yv^d_1 - \yv_1\\
    \yv^d_2 - \yv_2 \\
    -\yv_3
\end{bmatrix}$
with respect to the desired output trajectory.

\begin{problem}\label{prob:main_problem}
Design a smooth controller which, under suitable assumptions, obtains the following output specifications: 
    \begin{enumerate}
\item The dynamics of the error $\ev_1$ are exponentially stable\label{c:1} and independent of the value of $\sigma$.
    \item When $\sigma = 1$, the dynamics of the error 
    $\ev_2$ are exponentially stable\label{c:2}.
    \item When $\sigma = 0$, the dynamics of the error 
    $\ev_3$ are exponentially stable\label{c:3}.
\end{enumerate}
\end{problem}
We make the following assumptions.
\begin{assumpt}\label{asmpt_1}
    The system  \eqref{eq:Sigma12}  has a \emph{(vector)} relative degree $\rv  =\{\rhov_1,\rhov _2\}$ with $\rhov_1=\{r_1,\ldots,r_{p_1}\}, \rhov_2=\{r_{p_1+1},\ldots,r_{p}\}$  w.r.t. the pair ($\uv,\yv)$ and $\bar{\rv}=\{\rhov_1,\rhov_3\}$ with $\rhov_3=\{\bar{r}_{p_1+1},\ldots,\bar{r}_{p}\}$  w.r.t. the pair ($\uv,\bar{\yv}$) at $\xv^\circ$. 
\end{assumpt}
\begin{assumpt}\label{asmpt_2} 
The system \eqref{eq:Sigma12}
  is such that $\sum_{i=1}^pr_i=n$ and $\sum_{i=1}^{p_1}r_i+\sum_{i=p_1+1}^p\bar{r}_i =n$.
\end{assumpt}


\section{Proposed method}\label{sect:method}
\begin{lem}\label{lem:1}
Let $\sigma:[0,\infty)\to \{0,1\}$ be an exogenous switching signal, and let $\vv_1$, $\vv_2$, and $\vv_3$ be arbitrary assignable virtual inputs. 
If Assumptions~\ref{asmpt_1} and~\ref{asmpt_2} hold, then 
there exists a feedback controller $\uv = \kv_{\sigma}(\xv,\vv_1, \vv_2,\vv_3$) such that
\begin{equation}
\left\{
\begin{split}
    \yv_1^{(\rv')}&=\vv_1 \quad \text{at any time},\\
    \yv_2^{(\rv'')}&=\vv_2  \quad \text{when $\sigma=0$},\\
    {\yv}_3^{(\bar{\rv}'')}&=\vv_3 \quad \text{when $\sigma=1$},
    \end{split}
    \right.
\label{eq:output_dyn_tilde}
\end{equation}
and the system has trivial internal dynamics.
\label{lem_1}
\end{lem}



\begin{proof}
We denote with $\Am(\xv)$ the resulting interaction matrix obtained at the $\rv$-th differentiation of the output vector $\yv$ and with  $\bar{\Am}(\xv)$ the resulting interaction matrix obtained at the $\bar{\rv}$-th differentiation of the output vector $\bar{\yv}$. For clarity, we partition the two matrices into submatrices i.e.
\begin{align*}
 \Am(\xv)=\left[\begin{smallmatrix}
 \Am_{11}(\xv) & \Am_{12}(\xv)\\
   \Am_{21}(\xv) & \Am_{22}(\xv)\\
\end{smallmatrix}\right], 
&&
\bar{\Am}(\xv)=\left[\begin{smallmatrix}
 \bar{\Am}_{11}(\xv) & \bar{\Am}_{12}(\xv)\\
   \bar{\Am}_{21}(\xv) & \bar{\Am}_{22}(\xv)\\
\end{smallmatrix}\right], 
\end{align*}
in which $\bar{\Am}_{11}(\xv)=\Am_{11}(\xv)$ and   $\bar{\Am}_{12}(\xv)=\Am_{12}(\xv)$ since the two vectors $\yv$ and $\bar{\yv}$ share the same first entries $\yv_1$.
We know by assumption that both $\Am(\xv)$ and $\bar{\Am}(\xv)$ are nonsingular matrices.
Consider now the output \begin{equation}
{\yv}_{\sigma}= \begin{bmatrix}
 \yv_1 ^\top &
 \sigma \yv_2^\top + (1-\sigma)\yv_3^\top
\end{bmatrix}^\top.
\label{y_tilde}
\end{equation}
The corresponding interaction matrix obtained when considering the pair ($\uv,{\yv}_{\sigma}$) has the structure
\begin{equation}
{\Am}_{\sigma}(\xv) = \left[\begin{smallmatrix}
 \Am_{11}(\xv) & \Am_{12}(\xv) \\
 \sigma \Am_{21}(\xv)+(1-\sigma) \bar{\Am}_{21}(\xv) &   \sigma \Am_{22}(\xv)+(1-\sigma) \bar{\Am}_{22}(\xv)
\end{smallmatrix}\right].
\label{eq:Atilde}
\end{equation}
The determinant of ${\Am}_{\sigma}(\xv)$ is 
\begin{equation}
    \sigma \operatorname{det}\Am(\xv) + (1-\sigma) \operatorname{det}\bar{\Am}(\xv).
\end{equation}
Hence, independently of the value of $\sigma\in \{0,1\}$, the matrix ${\Am}_{\sigma}(\xv)$ is invertible in a neighborhood of $\xv^\circ$. 
Moreover, from Assumption \ref{asmpt_2}, the sum of the relative degree is always equal to the dimension of the state space $n$. 
It follows that, at the ${\rv}_{\sigma}$-th differentiation of the output ${\yv}_{\sigma}$, (where ${\rv}_{\sigma}=\rv $ when $\sigma=1$ and ${\rv}_{\sigma}=\bar{\rv}$ when $\sigma=0$) we get
\begin{equation}
{\yv}^{({\rv}_{\sigma})}_{\sigma} = \bv(\xv) + {\Am}_{\sigma}(\xv)\uv.
\end{equation}
By choosing the feedback linearizing controller
\begin{equation}
    \uv = {\Am}_{\sigma}(\xv)^{-1}[-\bv(\xv)+{\vv}_{\sigma}],
    \label{eq:control}
\end{equation}
with \begin{equation}
    {\vv}_{\sigma}=\begin{bmatrix}
    \vv_1^\top & \sigma \vv_2^\top+(1-\sigma ){\vv}_3^\top
\end{bmatrix}^\top,
\label{eq:v}
\end{equation}
the dynamics of the switching output ${\yv}_{\sigma}$ is given by~\eqref{eq:output_dyn_tilde}.
The internal dynamics are trivial since $~{\sum_{i}^n{r}_{\sigma,i}=n}$.
\end{proof}

The overall control architecture is shown in Figure~\ref{fig:lemma1}.
\begin{figure}[t]
    \centering
   \includegraphics[trim={3.5cm 2cm 1.3cm 2cm},clip,scale=0.3]{mizzo2.pdf}
    \caption{The Proposed Control Architecture.} 
    \label{fig:lemma1}
\end{figure}



\begin{thm}\label{thm:1}


    If  Assumptions~\ref{asmpt_1} and \ref{asmpt_2},  hold, then  a  solution of Problem~\ref{prob:main_problem}
is given by  the controller~\eqref{eq:control} with $\vv_1,\vv_2,\vv_3$ in~\eqref{eq:v} given by 
\begin{equation}
    \vv_j = \yv_j^{d,(\rhov_j)}+ \sum^{\rhov_{j}-1}_{i=1}\Lm^i_{j} ({\yv}^{d,(i)}_j-{\yv}^{(i)}_j), \quad j=1,2,3
    \label{eq:control2}
\end{equation} where  the $\Lm^i_j$ matrices are chosen arbitrarily, subject only to the constraint that substituting~\eqref{eq:control2} into~\eqref{eq:output_dyn_tilde} results in a linear output error dynamics that is exponentially stable.
\end{thm}

\begin{proof}
    If Assumptions~\ref{asmpt_1} and~\ref{asmpt_2} hold, then Lemma \ref{lem:1} applies. Hence, there exists a feedback linearizing controller \eqref{eq:control}  such that~\eqref{eq:output_dyn_tilde} holds. At this point, if we replace $\vv$ with \eqref{eq:v} then the error dynamics are given by 
    \begin{equation}
\left\{
\begin{split}
\ev_1^{(\rhov_1)} + \sum^{\rhov_{1}-1}_{i=1} \Lm^i_{1} \ev^{(i)}_1 &= \boldsymbol{0}, \quad \text{at any time},\\
\ev_2^{(\rhov_2)} + \sum^{\rhov_{2}-1}_{i=1} \Lm^i_{2} \ev^{(i)}_2 &= \boldsymbol{0}, \quad \text{when $\sigma=0$},\\
\ev_3^{(\rhov_3)} + \sum^{\rhov_{3}-1}_{i=1} \Lm^i_{3} \ev^{(i)}_3 &= \boldsymbol{0}, \quad \text{when $\sigma=1$}.
\end{split}
\right.
\end{equation}
    which solves the Problem \ref{prob:main_problem}.
\end{proof}
One possible choice of $\Lm^i_j$ is given by diagonal positive definite matrices. 








\begin{rem}
    The closed loop system can be seen as a 
     switched system with externally forced switching \cite{Liberzon2003,survey_hybrid}
   of the form $ \dot{\xv}=\fv_{\sigma}(\xv,t)
    $ with output array ${\yv}_{\sigma}$ given in \eqref{y_tilde} and with switching signal $\sigma$.
\end{rem}


