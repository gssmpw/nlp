

\section{Conclusion}

This letter presents a novel control approach that selectively utilizes system inputs,  reducing energy consumption by eliminating unnecessary inputs for the primary task.
Beyond the primary control objective, a secondary task is dynamically assigned based on an input-switching signal.
This signal alternates between tracking additional system variables and driving energy-intense variables to zero. 
A key contribution of this work is demonstrating that the primary task is always fulfilled, irrespective of the switching signal's state. 

The proposed method provides the following advantages compared with state of the art alternatives:
 \begin{enumerate}
     \item Formal guarantees of \textit{exponential} stability of the primary objective with dynamics that are decoupled from the switching signal.
     \item Formal guarantees of \textit{exponential} stability of the secondary task while the switching signal is constant.
     \item A smooth control law with a closed-form expression.
 \end{enumerate}
Since the method is based on feedback linearization, a potential disadvantage is its lack of robustness to uncertainties in system model parameters. 
Additionally, the method requires a specific form of the system in order to be applied, which limits its generality.
Nevertheless, the domain of application of the method is  large and includes several practical systems, such as the Mecanum wheel robot presented in the example.




Future research will focus on robustifying and experimentally validating these theoretical findings,  particular to the Mecanum Wheel. 

