\section{Related Work}
\label{sec: Related Work}

\par In this section, we present the existing work related to ISAC from three perspectives, \textit{i.e.}, scenario, optimization objectives, and optimization methods.

\subsection{IRS-assisted ISAC System}
\par For IRS-assisted DFRC-based system, the authors in____ utilized semi-passive IRS to assist single-antenna users in achieving single-input multiple-output (SIMO) millimeter-wave (mmWave) communication with a BS while simultaneously performing location-aware services. The authors in____ deployed active IRS equipped with amplifiers to assist multiple-input multiple-output (MIMO) DFRC-based ISAC scenario. Moreover, multi-IRS-assisted ISAC systems are explored in____, addressing two distinct sensing scenarios involving point targets and extended targets. In addition, the authors in____ explored the scenario where sensing signals enable communication with multiple users, taking into account the influence of environmental clutter on ISAC systems supported by IRS. However, the aforementioned existing works primarily employ fixed IRS to assist ISAC system. In contrast, mobile IRS can provide more flexible support for ISAC system, which can further enhances the system performance.

\subsection{Optimization Variables in IRS-assisted ISAC System}
\par In terms of communication metrics, the authors in____ optimized resource allocation for the IRS-supported DFRC-based ISAC system to improve the communication rate for users. Moreover, the authors in____ tackled eavesdropping threats in the ISAC scenario by refining BS beamforming matrix and adjusting IRS phase shifts to enhance the secure communication rate. In terms of sensing metrics, the authors in____ explored serving users and targets simultaneously by transmitting superimposed NOMA signals, aiming to maximize the beampattern gain, which measures energy concentration on the target. Based on this, the authors in____ further explored maximizing the beampattern gain for targets in a STAR-RIS-supported ISAC system with environmental clutter. Moreover, the authors in____ explored ISAC technology to tackle precise vehicle positioning and high-quality communication challenges by optimizing IRS phase shifts to minimize the position error bound. However, the aforementioned existing works optimized either communication or sensing metrics individually, without addressing the simultaneous optimization of both metrics.

\subsection{Optimization Methods in IRS-assisted ISAC System}
\par Currently, various optimization methods have been widely used to solve the optimization problems in IRS-assisted ISAC systems. For example, the authors in____ adopted a multi-strategy alternating optimization algorithm based on quadratically constrained quadratic programming and semidefinite relaxation to enhance both the transmission rate and probing power. Moreover, the authors in____ designed an improved particle swarm optimization algorithm to jointly optimize active and passive beamforming for maximizing the echo signal power. In addition, the authors in____ employed the proximal policy optimization (PPO) algorithm to optimize beamforming matrix and the UAV trajectory to maximize the transmission rate in the aerial IRS-assisted ISAC system. Furthermore, the authors in____ explored a STAR-RIS-supported secure ISAC framework and used the soft actor-critic (SAC) and DDPG algorithms to improve the long-term secrecy rate.
\par However, both convex optimization and swarm intelligence methods face limitations in dynamic scenarios where the prior knowledge is difficult to acquire. Moreover, while DRL algorithms show inherent advantages over traditional optimization methods, limitations still persist in analyzing complex environmental features and state-action relationships, which may affect the quality of decision.

% However, in practical scenarios, obtaining prior knowledge about the environment in advance is often challenging. 

\begin{figure}
    \centering
    \includegraphics[width=0.85\linewidth]{System_Model_1.pdf}
    \caption{Aerial IRS-assisted ISAC system.}
    \label{fig: system model}
\end{figure}

%section
%System Model and Problem Formulation
%