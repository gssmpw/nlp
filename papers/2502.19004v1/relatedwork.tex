\section{Related works}
\subsection{Resource Allocation in Metaverse System}
The work in \cite{10250875} investigated resource allocation strategies within an AR-enabled vehicular edge Metaverse, focusing on maximizing the Metaverse operator's reward by jointly optimizing the CPU frequency and transmit power of AR vehicles, the sizes of computation models, and the distribution of computational resources on the Metaverse server.
  The investigation in \cite{10144631} proposed an adaptive edge resource allocation method based on Soft Actor-Critic with GCN (SAC-GCN), which defines the multiuser Metaverse environment as a graph with each agent represented by a node. The study in \cite{9880566} introduced a hierarchical game-theoretic-based coded distributed computing (CDC) framework to provide collaborative computing and Metaverse services for the vehicular Metaverse, where idle resources of vehicles can be allocated to handle intensive computation tasks. In the same study, the authors develop a coalition game to select reliable and resource-rich vehicles based on the reputation values calculated through the subjective logical model; a Stackelberg-based incentive mechanism to motivate a coalition of vehicles to participate in rendering tasks. 
  The authors of \cite{10368052} presented Human Centric (UC) resource allocation, considering joint communication and computational resources as well as VR video resolutions. The authors addressed the non-convex system cost problem, which includes Energy Consumption (EC) and delay, and solved it using the fractional programming technique. The work in \cite{10148094} proposed a Joint Resource Allocation and Metaverse Service (JRAMS) selection strategy that dynamically allocates communication and computation resources to Metaverse services with high QoE. This study proposed \textit{meta-distance} -- i.e., a novel metric for measuring virtual distance in the Metaverse that considers both service latency and social distance among users. JRAMS consists of two steps: i) a one-to-many matching game with externalities to match base stations and Metaverse users using Non-Orthogonal Multiple Access (NOMA) subchannel allocations, and ii) a coalition formation game to solve the Metaverse service selection problem. 
 \subsection{Vehicular Twin Migration}
In addressing Metaverse service interruption due to the movement of vehicles and the limited-service coverage of RSUs, some studies present VT migration solutions, where RSUs contribute resources to host VTs and facilitate efficient VT migration. Since a single RSU is insufficient for supporting large-scale VTs migration. The authors of \cite{10281020} proposed a coalition game approach framework for reliable VT migration in vehicular Metaverses. The coalition game among RSUs is formulated based on the reputation value of RSUs computed through the subjective logic model. 
The RSUs in the coalition game collectively provide bandwidth resources for reliable and large-scale VTs migration, while serving several VTs migrations at the same time.
 The authors of \cite{zhong2023blockchain} presented a blockchain-assisted game approach for VT migration in vehicular Metaverse to maintain continuous service and offer immersive experiences for VMUs as vehicles move. In this framework, the reputation value of RSUs is calculated based on their interaction freshness with VMUs, and the coalition of RSUs is formed based on their reputation value to share bandwidth resources. To provide immersive Metaverse services and efficiently migrate the VTs,  the coalition with the maximum utility is chosen, and a Stackelberg model is adopted to handle the interaction between the RSUs' coalition and VMUs, motivating VMU's active participation. This aims to provide seamless VT migration and enhance Metaverse services for VMUs. 
 
 To address the problems with the VT migration process due to insufficient bandwidth resources from RSUs for timely migration, the study in \cite{10505943} presented a Stackelberg-based incentive mechanism for vehicular Metaverse and proposed a MADRL algorithm, Multi-Agent LSTM-based Proximal Policy Optimization (MALPPO). The RSU that provides the Metaverse service and the RSU that requests the Metaverse interact through this incentive game.  The MALPPO algorithm facilitates learning the \textit{Stackelberg Equilibrium (SE)} without requiring private information from others, relying solely on past experiences. The work in\cite{10415630} presented an avatar task migration approach based on MADRL to address the limited resources available on vehicles and the high mobility of vehicles, where the avatar task is migrated to the nearest RSU or UAV for execution, leading to reduced communication overhead and task processing latency.  The authors utilized the MAPPO algorithm to optimize the avatar migration optimization problem. 
 
 To improve the convergence of MAPPO due to dimensionality and nonstationary problems in sharing parameters, the work in \cite{10415630} applied a Transformer-based MAPPO approach via sequential decision-making models, and a smart contract with blockchain is utilized to ensure trustworthiness in the computation resource-sharing transactions.  To address the challenges of making avatar migration decisions due to vehicle mobility, dynamic workload of RSUs, and RSU heterogeneity, the authors in \cite{10185562} proposed a MADRL-based dynamic avatar task migration framework based on real-time trajectory prediction. Specifically, they proposed a model to predict the future trajectories of vehicles based on their historical data, which could be useful for forecasting the available resources and workload of RSUs. The avatar task migration problem is formulated as a long-term mixed-integer programming problem, which is then transformed into a partially observable MDP, and multiple DRL agents with hybrid continuous and discrete actions are used to address the formulated optimization problem.

\begin{table*}[http]
	\scriptsize
    \caption{The comparison of our work with state-of-the-art works}
	\label{tab.1}
	\centering
	 \begin{tabular}{|p{0.8cm}| p{3.9cm}| p{3.6cm}| p{3.5cm}| p{3.8cm}| p{4.5cm}| }
    \hline
  \textbf{Paper} & \textbf{Scenario}&\textbf{Method}&\textbf{Problem}&\textbf{Objective}\\
		\hline
   \cite{9973630} & Blockchain-based mobile edge computing platform for resource sharing.&RL algorithm for multiple task allocation. & Multiple task allocation.
 & Optimize costs and utility. \\
   \hline
   \cite{10144631} & Adaptive edge resource allocation in the Metaverse using SAC-GCN.&Multiagent SAC-GCN with self-attention mechanism. & Edge resource allocation for multiple MVU. & Optimizing resource allocation and utilization rate. Improves UX. 
\\
		\hline
  \cite{9880566} &Collaborative computing paradigm based on CDC in vehicular Metaverse. & Game-theoretic CDC. Coalition formation game and Stackelberg game.  &Real-time rendering inefficiency
and reputation management.
 &Improving UX and resource utilization rate. \\
  \hline
\cite{10368052} & Human-centric resource allocation in the Metaverse over wireless communications. & Fractional programming technique for non-convex optimization. UC utility measure. &Insufficient bandwidth, power, computing, resolution, and CPU frequency resources.
 &Optimizing resource allocation and maximizing utility-cost ratio.\\
\hline
 \cite{10148094} & QoE analysis and resource allocation for wireless Metaverse services. &Personalized resource and attention-aware rendering capacity allocation. & Metaverse
service selection and resource allocation. & Enhancing QoE in Metaverse services.
\\
		\hline

  \cite{10281020} & Coalition formation game for VT migration in vehicular edge computing. & Double-level coalition formation game. & Limited resource of single RSU for VT migration.& Minimizing expenses and maximizing revenue.\\
		\hline
  \cite{zhong2023blockchain} & Blockchain-assisted twin migration for vehicular Metaverses.
 &MADRL algorithm for migration decisions. & Avatar task migration in vehicular Metaverses for immersive services.& Optimize avatar task migration by integrating learning-based algorithms.
\\
		\hline
  \cite{10505943} & VT migration in vehicular Metaverse. &MADRL with Stackelberg game. & VT migration.&  Optimize resource allocation.\\
		\hline
 \cite{10415630} & UAV-assisted avatar task migration for vehicular Metaverse. &MAPPO algorithm for the avatar task migration.& Avatar task migration from vehicles to RSUs/UAVs. & 
Latency reduction and UX. 
\\
		\hline
  \cite{10185562} & Avatar migration in vehicular Metaverses.& MADRL  for real-time trajectory prediction and avatar task migration. & Inefficient avatar migration decisions. & Minimize avatar service latency with an optimized pre-migration decision. 
\\
		\hline
Our work & Resource allocation and VT migration in the multi-tier vehicular Metaverse.&MO-MADRL and hierarchical Stackelberg game-based incentive. & Inefficient resource allocation and VT migration decisions.  & Minimizing migration costs, EC, and latency while maximizing UX.
\\
		\hline
	\end{tabular}
\end{table*}

We present a comparison and summary of the most relevant existing works with our proposed framework in TABLE \ref{tab.1}. As tabulated, the studies in \cite{9973630,10144631,9880566,10368052,10148094} focused on allocating computation and communication resources to enhance the immersive Metaverse experience and resource utilization efficiency in a variety of Metaverse services application scenarios. Furthermore, the works in \cite{10281020,zhong2023blockchain,10505943,10415630,10185562} investigated avatar tasks and VT migration in a variety of application contexts to provide reliable and efficient Metaverse services. However, the mentioned works considered neither a multi-tier resource allocation nor a VT migration in vehicular Metaverse. Aiming to close this knowledge gap and advance the field meaningfully, we propose a joint resource allocation and VT migration framework that can dynamically allocate resources and relocate the VTs across different layers of the multi-tier vehicular Metaverse so as to optimize performance, reliability, and resource utilization. The proposed framework can dynamically allocate resources and relocate the VTs across different layers of the multi-tier vehicular network to optimize performance, reliability, and resource utilization. \newline 
We now proceed to detail our system model.