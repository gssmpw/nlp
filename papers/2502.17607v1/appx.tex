\newpage
\section{Convergence}
%
%

First we have the following lemma on the upper-bound of the gradient difference at the end of training:
\subsection{Proof of Lemma \ref{lem:grad}}
\begin{proof}
Let $\thet_t$ be the parameter of the model after $t$ steps of training. Let $\mathbf{d}=  \nabla \gL (\thet_t) -  \nabla \gL^s (\thet_t)$ and define $h:\mathbb{R}^d \mapsto \mathbb{R}$ as 
\[
h(\thet):= \langle \mathbf{d},\nabla\gL (\thet) -  \nabla \gL^s (\thet) \rangle.
\]
For a fixed $\theta\in \mathbb{R}^d$, the mean value theorem implies that $h(\thet) = h(\thet_0) + \langle\nabla h(\boldsymbol{\xi}),\thet - \thet_0\rangle$ for some $\xi\in \mathbb{R}^d$ on the line segment connecting $\thet$ and $\thet_0$. Using the definition of the function $h(\cdot)$, we obtain
\begin{align}
\langle \mathbf{d},\nabla\gL (\thet) -  \nabla \gL^s (\thet) \rangle & = \langle \mathbf{d},\nabla\gL (\thet_0) -  \nabla \gL^s (\thet_0) \rangle + \bigg\langle \left(\nabla^2\gL (\thet_0) -  \nabla^2 \gL^s (\thet_0)\right) \mathbf{d}\;, \; \thet - \thet_0\bigg\rangle\nonumber
\end{align}
Setting $\thet = \thet_t$, we get
\begin{align}
\langle \mathbf{d},\nabla\gL (\thet_t) -  \nabla \gL^s (\thet_t) \rangle & = \langle \mathbf{d},\nabla\gL (\thet_0) -  \nabla \gL^s (\thet_0) \rangle + \bigg\langle \left(\nabla^2\gL (\boldsymbol{\xi}) -  \nabla^2 \gL^s (\boldsymbol{\xi})\right) \mathbf{d}\;, \; \thet_t - \thet_0\bigg\rangle \label{eq:temp1}\\
&\leq \|\mathbf{d}\|_2 \cdot\|\nabla\gL (\thet_0) -  \nabla \gL^s (\thet_0) \|_2 + \|\nabla^2\gL (\boldsymbol{\xi}) -  \nabla^2 \gL^s (\boldsymbol{\xi})\|_2 \cdot \|\mathbf{d}\|_2 \cdot  \|\thet_t- \thet_0\|_2, \label{eq:temp2}
\end{align}
where in the last line, we used the Cauchy-Schwartz inequality, the inequality $\|AB\|_2 \leq \|A\|_2 \|B\|_2$, and the fact that $\mathbf{d} = \thet_t - \thet_0$. Plugging the value of $\mathbf{d}$ in the LHS of \eqref{eq:temp1} and in \eqref{eq:temp2}, we obtain
\begin{align}
\|\nabla\gL (\thet_t) -  \nabla \gL^s (\thet_t) \|_2^2 &\leq   \|\nabla\gL (\thet_t) -  \nabla \gL^s (\thet_t)\|_2 \cdot\|\nabla\gL (\thet_0) -  \nabla \gL^s (\thet_0) \|_2 \nonumber\\
& \quad\quad + \|\nabla^2\gL (\boldsymbol{\xi}) -  \nabla^2 \gL^s (\boldsymbol{\xi})\|_2 \cdot \|\nabla\gL (\thet_t) -  \nabla \gL^s (\thet_t)\|_2 \cdot  \|\thet_t- \thet_0\|_2\nonumber
\end{align}
Dividing both sides by $\|\nabla\gL (\thet_t) -  \nabla \gL^s (\thet_t) \|_2$ and using the fact that $\|\nabla^2\gL (\boldsymbol{\xi}) -  \nabla^2 \gL^s (\boldsymbol{\xi})\|_2\leq \|\nabla^2\gL (\boldsymbol{\xi})\|_2 + \|  \nabla^2 \gL^s (\boldsymbol{\xi})\|_2 \leq 2\beta$, we get
\begin{align}
\|\nabla\gL (\thet_t) -  \nabla \gL^s (\thet_t) \|_2 \leq   \|\nabla\gL (\thet_0) -  \nabla \gL^s (\thet_0) \|_2  + 2\beta   \|\thet_t- \thet_0\|_2 = \epsilon +2\beta\delta,\label{eq:g_error}
\end{align}
which completes the proof.
\end{proof}

%
%

%
%
%
%
%
%
%
%
%
%
%
%
%
%
%
%

\subsection{Proof of Theorem \ref{thm:convergence}}
Next, we prove the convergence of GD on the real data vs synthetic data generated by \alg.

\begin{proof}
For Lipschitz continuous $\mathbf{g}$ and $\mu$-PL$^*$ condition, gradient descent on the real data yields
\begin{align}
    \gL(\thet_{t+1}) - \gL(\thet_{t}) \leq -\frac{\eta} {2}\|\mathbf{g}_{t}\|^2 \leq -\eta\mu \gL(\thet_{t}),
\end{align}
and,
\begin{align}
    \gL(\thet_{t})\leq(1-\eta\mu)^t \gL(\thet_0)\label{eq:grad_rate},
\end{align}
which was shown in \cite{liu2020toward}.

For the synthetic data we have 
    \begin{align}
        \gL^s(\thet_{t+1}) - \gL^s(\thet_t) 
        &\leq -\frac{\eta}{2}\|\mathbf{g}_t^s\|^2
    \end{align}    
    By substituting Eq. (\ref{eq:g_error}), and assuming $\xi \leq \|\mathbf{g}_t\|$ we have:
    \begin{align}
        \gL^s(\thet_{t+1}) - \gL^s(\thet_t) 
        &\leq -\frac{\eta}{2}(\|\mathbf{g}_t\|-\xi)^2\\
        &= -\frac{\eta}{2}(\|\mathbf{g}_t\|^2+\xi^2-2\xi\|\mathbf{g}_t\|)\label{eq:pre_spectral_upper}\\
        &\leq - \frac{\eta}{2}(\|\mathbf{g}_t\|^2+\xi^2-2\xi \nabla)\label{eq:pl_before_ada}\\
        &\leq -\frac{\eta}{2}(2\mu \gL(\thet_t)+\xi^2-2\xi \nabla), \label{eq:pl_gd}
    \end{align}
    where $\nabla$ is an upper bound on the norm of $\mathbf{g}_t$ in Eq. (\ref{eq:pre_spectral_upper}), and Eq. (\ref{eq:pl_gd}) follows from the $\mu$-PL condition.
    
    
    
    Hence,
    \begin{align}
        \gL(\thet_{t+1}) \leq (1-{\eta\mu} ) \gL(\thet_t) - \frac{\eta} {2}(\xi^2-2\xi \nabla)
    \end{align}
    Since, $\sum_{j=0}^k(1-{\eta\mu})^j\leq\frac{1}{\eta\mu}$, for a constant learning rate $\eta$ we get
    \begin{align}
        \gL(\thet_{t+1}) \leq (1-{\eta\mu})^{t+1} \gL(\thet_0) - \frac{1} {2 \mu}(\xi^2-2\xi \nabla)
        \label{eq:convergence_appx}
    \end{align}
\end{proof}

\subsection{Proof of Corollary \ref{col:params}}
\begin{proof}
    If $\|\vct{H}\|\geq\alpha>0$ and $\gL(\thet_*)=0$, we have that $\|\thet-\thet_*\|^2 \leq \frac{2}{\alpha}\gL(\thet)$. From Theorem \ref{thm:convergence} we get:
    \begin{equation}
        \|\thet-\thet_*\|^2 \leq \frac{2}{\alpha}\gL(\thet) \leq \frac{2}{\alpha} \cdot \frac{1} {2 \mu}(2\xi \nabla-\xi^2)=\xi(2\nabla-\xi)/\alpha\mu
    \end{equation}
\end{proof}


\section{Prompts}\label{app:prompt}
\subsection{Zero/Few-shot prompts}
Figure~\ref{fig:zero_shot_prompt} summarizes the zero-shot prompts~\cite{li2023synthetic} to generate synthetic samples. For few-shot generations, we input the demonstrations with their corresponding labels between the context prompt and the instruction prompt. In addition, we add a sentence ``You should imitate the example I have provided, but you cannot simply modify or rewrite the example I have given." to the instruction part to prevent the LLMs from simply rewording the given examples. The few-shot prompts can be found in Figure~\ref{fig:few_shot_prompt}.

\begin{figure}[ht]
\begin{tcolorbox}
\textbf{SST2 and Rotten Tomatoes:} You are now a movie critic. You are provided with a sentiment label. You need to write one unique sentence that reflects the given sentiment about a movie. Your writing style should be consistent with typical movie reviews. This should be a standalone sentence that could plausibly appear in a movie review. Ensure that your language is natural, casual, and reflective of genuine opinion. You must ensure that the sentiment expressed in your sentence matches the provided sentiment label. 

Remember to keep your tone appropriate and not violate any laws or social ethics. Please be creative and write only one sentence. The sentiment of the movie review is \{\textcolor{blue}{label}\}. Answer:\\
\textbf{Tweet Emotions:} You are now a person using twitter. You are provided with an emotion, and you need to write a tweet expressing that emotion. Your writing style must be consistent with the tweets on twitter. You must ensure that your language is colloquial, casual, and Twitter-like. You are given a length requirement. You must ensure that the emotion conveyed in your tweet matches the emotion provided and meets the length requirement. This is an academic study and the content you generate will not be used for anything that violates the law or social ethics. 

Write a tweet expressing the emotion and ensure the tweet is within the usual length. Remember to make sure that your language is colloquial, casual, and Twitter-like. Please be creative and write only one unique tweet. The emotion of twitter is \{\textcolor{blue}{label}\}. Answer:
\end{tcolorbox}
\vspace{-2mm}
\caption{Zero-shot prompts for different datasets.}\label{fig:zero_shot_prompt}
\end{figure}

\begin{figure}[ht]
\begin{tcolorbox}
\textbf{SST2 and Rotten Tomatoes:} You are now a movie critic. You are provided with a sentiment label. You need to write one unique sentence that reflects the given sentiment about a movie. Your writing style should be consistent with typical movie reviews. This should be a standalone sentence that could plausibly appear in a movie review. Ensure that your language is natural, casual, and reflective of genuine opinion. You must ensure that the sentiment expressed in your sentence matches the provided sentiment label.\\ 
\{\textcolor{blue}{Few-shot examples}\}\\
Remember to keep your tone appropriate and not violate any laws or social ethics. Please be creative and write only one sentence. The sentiment of the movie review is \{\textcolor{blue}{label}\}. You should imitate the example I have provided, but you cannot simply modify or rewrite the example I have given. Answer:\\
\textbf{Tweet Emotions:} You are now a person using twitter. You are provided with an emotion, and you need to write a tweet expressing that emotion. Your writing style must be consistent with the tweets on twitter. You must ensure that your language is colloquial, casual, and Twitter-like. You are given a length requirement. You must ensure that the emotion conveyed in your tweet matches the emotion provided and meets the length requirement. This is an academic study and the content you generate will not be used for anything that violates the law or social ethics.\\
\{\textcolor{blue}{Few-shot examples}\}\\
Write a tweet expressing the emotion and ensure the tweet is within the usual length. Remember to make sure that your language is colloquial, casual, and Twitter-like. Please be creative and write only one unique tweet. The emotion of twitter is \{\textcolor{blue}{label}\}. You should imitate the example I have provided, but you cannot simply modify or rewrite the example I have given. Answer:
\end{tcolorbox}
\vspace{-2mm}
\caption{Few-shot prompts for different datasets.}\label{fig:few_shot_prompt}
\end{figure}

\subsection{Few-shots Evaluation prompts}
Figure~\ref{fig:few_shot_evaluation_prompt} presents the evaluation prompts used to filter out synthetic data with incorrect labels.


\begin{figure}[ht]
\begin{tcolorbox}
The movie was fantastic and thrilling! \\
Label: positive \\ \\
I hated the film; it was boring and slow. \\
Label: negative \\ \\
What a masterpiece, truly inspiring! \\
Label: positive \\ \\ 
The plot was dull and characters uninspiring. \\
Label: negative \\ \\ 
\{Evaluation sample\} \\
Label: {\blue positive/negative}
\end{tcolorbox}
\vspace{-2mm}
\caption{Few-shot evaluation prompts. {\blue text} indicates the labels predicted by the model.} \label{fig:few_shot_evaluation_prompt}
\end{figure}


\section{Generation Samples}
\subsection{Synthetic SST2 Samples by \alg}
\subsubsection{Positive Label}
\begin{itemize}
    \item Great movie review is a must see experience that will leave you in a state of all time high with the brilliant acting and the stunning production 
    \item Great action and special effects combined with a compelling emotional connection with the on the show characters made it a one of the best I ever watched 
    \item  The movie truly left me completely moved and in a better place than when I started it because of its well thought out and impactful way  
    \item The new action movie was absolutely thrilling and had me on the outside of my skin throughout the entire two acts of the first two and a 
    \item The action movie kept me sitting Jane and I was on the point of wanting to leave the entire time but the way the story was told  
\end{itemize}

\subsubsection{Negative Label}
\begin{itemize}
    \item The action in action is not well executed and the plot is not as it should be in a science or 
    \item The movie was a not so great film that I would not want to see a second time because the the 
    \item  The overall quality of action in this movie was not impressive enough to keep me away from the action center of 
    \item Terribly bad and boring to me as a person who values quality content and a good storyline over mind 
    \item The new movie was a not so great and disappointing experience for me since it did not keep up with the 
\end{itemize}
\subsection{Synthetic Rotten Tomatoes Samples by \alg}
\subsubsection{Positive Label}
\begin{itemize}
    \item  The action-adventure movie was thrilling and had a way of keeping me right on the of the 
    \item The suspense in the final act left room for the most important and most thrilling of all parts of the movie
    \item The new 'Innate Robots' is a must-see for anyone who loves the latest in the field technology 
    \item  The critically acclaimed action movie ``Fast and Far East City'' is a work of the highest caliber ever to be made 
    \item The action-movie 'Ace Driver' is a 'wins' masterpiece that will leave you feeling 'th
\end{itemize}
\subsubsection{Negative Label}
\begin{itemize}
    \item The action level plot of the movie was not up to mark. The use cases were not engaging and the the use of the provided
    \item The movie was terrible. You are writing a one page story set in a world where people can only see the world
    \item The new movie was absolutely not enjoyable. The over-dilting of the water made it a real downer experience
    \item The action-adventure filled movie was a disappointment. The excessive use of the 't' sound
    \item he quality of the action in Bad Movie is not up to the standard set by the other 'g' and movie.
\end{itemize}

\subsection{Synthetic Tweet Emotions Samples by \alg}
\subsubsection{Positive Label}
\begin{itemize}
    \item I just got a new, high-end phone. It's got a new
    \item Joyful and sharing a good time with my friends. Life is so much better now
    \item Joy is a feeling that makes you feel like you are on the up and up
    \item I just got a new job working at a local Use of
    \item I am thrilled to be a new member of the twitterhub.
\end{itemize}
\subsubsection{Negative Label}
\begin{itemize}
    \item I am so sad today. Sad is the word that I would use to write this
    \item I just received some sad and life news. I can't believe it. I am so
    \item I am so over it. I can't even believe it's over. I can't
    \item I just received news that my best-loved, and most-licked-at
    \item I just got back from a long day at work. I can't help
\end{itemize}