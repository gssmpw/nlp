\documentclass[letterpaper,twocolumn,10pt]{article}
\usepackage{include/usenix}
% \usepackage[available,functional,reproduced]{include/usenixbadges}

% Use § symbol with \autoref
\renewcommand{\sectionautorefname}{\S}
\renewcommand{\subsectionautorefname}{\S}
\renewcommand{\subsubsectionautorefname}{\S}

\usepackage{tikz}
\usepackage{float}
\usepackage{amsmath}
\usepackage{balance}
\usepackage{booktabs}
\usepackage{hyperref}
\usepackage[capitalise,nameinlink]{cleveref}
\usepackage{xcolor}
\usepackage{xspace}
\usepackage{multirow}
\usepackage{tabularx}
\usepackage{algorithm}
\usepackage{algpseudocode}
\algtext*{EndWhile}
\algtext*{EndIf}
\algtext*{EndFor}
\usepackage{caption}
\usepackage{subcaption}
\usepackage[compact]{titlesec}

% Inline enumeration
\usepackage[inline]{enumitem}

\hyphenation{machine workflows CentOS Podman}

\usepackage{enumitem}
\setitemize{noitemsep,topsep=0pt,parsep=0pt,partopsep=0pt}

\newcommand{\sys}[0]{CRIUgpu\xspace}
\def\Snospace~{\S{}}

\newcommand{\heading}[1]{\vspace{4pt}\noindent\textbf{#1}}
\newcommand{\topheading}[1]{\noindent\textbf{#1}}

\renewcommand{\algorithmicrequire}{\textbf{Input:}}
\renewcommand{\algorithmicensure}{\textbf{Output:}}

% For comments.
\newcommand\rb[1]{\textcolor{orange}{RB: #1}}
\newcommand\vk[1]{\textcolor{purple}{Viki: #1}}
\newcommand\rs[1]{\textcolor{blue}{RS: #1}}

% Style commands.
\newcommand{\stitle}[1]{\vspace{1.ex}\noindent{\bf #1}}
\newcommand{\noindentstitle}[1]{\noindent{\bf #1}}

\usepackage[compact]{titlesec}
\titleformat*{\section}{\large\bfseries}
\titleformat*{\subsection}{\normalsize\bfseries}
\titleformat*{\subsubsection}{\normalsize\bfseries}
\titlespacing*{\section}{0pt}{*2}{*1}
\titlespacing*{\subsection}{0pt}{*1.5}{*0.8}

\iffalse
\titleformat{\subsection}[runin]% runin puts it in the same paragraph
       {\normalfont\bfseries}% formatting commands to apply to the whole heading
       {\thesubsection}% the label and number
       {0.5em}% space between label/number and subsection title
       {}% formatting commands applied just to subsection title
       [.]% punctuation or other commands following subsection title
\fi

\captionsetup[figure]{labelfont={bf},name={Figure},labelsep=period}
\captionsetup[table]{labelfont={bf},name={Table},labelsep=period}

\usepackage{tikz}
\newcommand*\circled[1]{
    \tikz[baseline=(char.base)]{
        \node[shape=circle,draw,inner sep=.5pt] (char) {#1};
    }
}

% A vertical rule between subfigures
\newcommand{\rulesep}{\unskip\ \vrule\ }

\begin{document}

% \date{}

\title{\sys: Transparent Checkpointing of GPU-Accelerated Workloads}

\author{
    \rm
        Radostin Stoyanov$^{\star\dagger}$ \enskip
        Viktória Spišaková$^{\ddagger}$ \enskip
        Jesus Ramos$^{\diamond}$ \enskip
        Steven Gurfinkel$^{\diamond}$ \enskip
        Andrei Vagin$^{\triangle}$ \enskip
    \\ \rm
        Adrian Reber$^{\dagger}$ \enskip
        Wesley Armour$^{\star}$ \enskip
        Rodrigo Bruno$^{\bullet}$ \enskip
    \\ \\
    {
        $^{\star}$University of Oxford\enskip
        {$^{\ddagger}$Masaryk University\enskip}
        $^{\diamond}$NVIDIA\enskip 
        $^{\triangle}$Google\enskip
        $^{\dagger}$Red Hat\enskip
    }\\
        {$^{\bullet}$INESC-ID, Instituto Superior Técnico, University of Lisbon\enskip}
}

\maketitle

\begin{abstract}  
Test time scaling is currently one of the most active research areas that shows promise after training time scaling has reached its limits.
Deep-thinking (DT) models are a class of recurrent models that can perform easy-to-hard generalization by assigning more compute to harder test samples.
However, due to their inability to determine the complexity of a test sample, DT models have to use a large amount of computation for both easy and hard test samples.
Excessive test time computation is wasteful and can cause the ``overthinking'' problem where more test time computation leads to worse results.
In this paper, we introduce a test time training method for determining the optimal amount of computation needed for each sample during test time.
We also propose Conv-LiGRU, a novel recurrent architecture for efficient and robust visual reasoning. 
Extensive experiments demonstrate that Conv-LiGRU is more stable than DT, effectively mitigates the ``overthinking'' phenomenon, and achieves superior accuracy.
\end{abstract}  

% humans are sensitive to the way information is presented.

% introduce framing as the way we address framing. say something about political views and how information is represented.

% in this paper we explore if models show similar sensitivity.

% why is it important/interesting.



% thought - it would be interesting to test it on real world data, but it would be hard to test humans because they come already biased about real world stuff, so we tested artificial.


% LLMs have recently been shown to mimic cognitive biases, typically associated with human behavior~\citep{ malberg2024comprehensive, itzhak-etal-2024-instructed}. This resemblance has significant implications for how we perceive these models and what we can expect from them in real-world interactions and decisionmaking~\citep{eigner2024determinants, echterhoff-etal-2024-cognitive}.

The \textit{framing effect} is a well-known cognitive phenomenon, where different presentations of the same underlying facts affect human perception towards them~\citep{tversky1981framing}.
For example, presenting an economic policy as only creating 50,000 new jobs, versus also reporting that it would cost 2B USD, can dramatically shift public opinion~\cite{sniderman2004structure}. 
%%%%%%%% 图1:  %%%%%%%%%%%%%%%%
\begin{figure}[t]
    \centering
    \includegraphics[width=\columnwidth]{Figs/01.pdf}
    \caption{Performance comparison (Top-1 Acc (\%)) under various open-vocabulary evaluation settings where the video learners except for CLIP are tuned on Kinetics-400~\cite{k400} with frozen text encoders. The satisfying in-context generalizability on UCF101~\cite{UCF101} (a) can be severely affected by static bias when evaluating on out-of-context SCUBA-UCF101~\cite{li2023mitigating} (b) by replacing the video background with other images.}
    \label{fig:teaser}
\end{figure}


Previous research has shown that LLMs exhibit various cognitive biases, including the framing effect~\cite{lore2024strategic,shaikh2024cbeval,malberg2024comprehensive,echterhoff-etal-2024-cognitive}. However, these either rely on synthetic datasets or evaluate LLMs on different data from what humans were tested on. In addition, comparisons between models and humans typically treat human performance as a baseline rather than comparing patterns in human behavior. 
% \gabis{looks good! what do we mean by ``most studies'' or ``rarely'' can we remove those? or we want to say that we don't know of previous work doing both at the same time?}\gili{yeah the main point is that some work has done each separated, but not all of it together. how about now?}

In this work, we evaluate LLMs on real-world data. Rather than measuring model performance in terms of accuracy, we analyze how closely their responses align with human annotations. Furthermore, while previous studies have examined the effect of framing on decision making, we extend this analysis to sentiment analysis, as sentiment perception plays a key explanatory role in decision-making \cite{lerner2015emotion}. 
%Based on this, we argue that examining sentiment shifts in response to reframing can provide deeper insights into the framing effect. \gabis{I don't understand this last claim. Maybe remove and just say we extend to sentiment analysis?}

% Understanding how LLMs respond to framing is crucial, as they are increasingly integrated into real-world applications~\citep{gan2024application, hurlin2024fairness}.
% In some applications, e.g., in virtual companions, framing can be harnessed to produce human-like behavior leading to better engagement.
% In contrast, in other applications, such as financial or legal advice, mitigating the effect of framing can lead to less biased decisions.
% In both cases, a better understanding of the framing effect on LLMs can help develop strategies to mitigate its negative impacts,
% while utilizing its positive aspects. \gabis{$\leftarrow$ reading this again, maybe this isn't the right place for this paragraph. Consider putting in the conclusion? I think that after we said that people have worked on it, we don't necessarily need this here and will shorten the long intro}


% If framing can influence their outputs, this could have significant societal effects,
% from spreading biases in automated decision-making~\citep{ghasemaghaei2024understanding} to reducing public trust in AI-generated content~\citep{afroogh2024trust}. 
% However, framing is not inherently negative -- understanding how it affects LLM outputs can offer valuable insights into both human and machine cognition.
% By systematically investigating the framing effect,


%It is therefore crucial to systematically investigate the framing effect, to better understand and mitigate its impact. \gabis{This paragraph is important - I think that right now it's saying that we don't want models to be influenced by framing (since we want to mitigate its impact, right?) When we talked I think we had a more nuanced position?}




To better understand the framing effect in LLMs in comparison to human behavior,
we introduce the \name{} dataset (Section~\ref{sec:data}), comprising 1,000 statements, constructed through a three-step process, as shown in Figure~\ref{fig:fig1}.
First, we collect a set of real-world statements that express a clear negative or positive sentiment (e.g., ``I won the highest prize'').
%as exemplified in Figure~\ref{fig:fig1} -- ``I won the highest prize'' positive base statement. (2) next,
Second, we \emph{reframe} the text by adding a prefix or suffix with an opposite sentiment (e.g., ``I won the highest prize, \emph{although I lost all my friends on the way}'').
Finally, we collect human annotations by asking different participants
if they consider the reframed statement to be overall positive or negative.
% \gabist{This allows us to quantify the extent of \textit{sentiment shifts}, which is defined as labeling the sentiment aligning with the opposite framing, rather then the base sentiment -- e.g., voting ``negative'' for the statement ``I won the highest prize, although I lost all my friends on the way'', as it aligns with the opposite framing sentiment.}
We choose to annotate Amazon reviews, where sentiment is more robust, compared to e.g., the news domain which introduces confounding variables such as prior political leaning~\cite{druckman2004political}.


%While the implications of framing on sensitive and controversial topics like politics or economics are highly relevant to real-world applications, testing these subjects in a controlled setting is challenging. Such topics can introduce confounding variables, as annotators might rely on their personal beliefs or emotions rather than focusing solely on the framing, particularly when the content is emotionally charged~\cite{druckman2004political}. To balance real-world relevance with experimental reliability, we chose to focus on statements derived from Amazon reviews. These are naturally occurring, sentiment-rich texts that are less likely to trigger strong preexisting biases or emotional reactions. For instance, a review like ``The book was engaging'' can be framed negatively without invoking specific cultural or political associations. 

 In Section~\ref{sec:results}, we evaluate eight state-of-the-art LLMs
 % including \gpt{}~\cite{openai2024gpt4osystemcard}, \llama{}~\cite{dubey2024llama}, \mistral{}~\cite{jiang2023mistral}, \mixtral{}~\cite{mistral2023mixtral}, and \gemma{}~\cite{team2024gemma}, 
on the \name{} dataset and compare them against human annotations. We find  that LLMs are influenced by framing, somewhat similar to human behavior. All models show a \emph{strong} correlation ($r>0.57$) with human behavior.
%All models show a correlation with human responses of more than $0.55$ in Pearson's $r$ \gabis{@Gili check how people report this?}.
Moreover, we find that both humans and LLMs are more influenced by positive reframing rather than negative reframing. We also find that larger models tend to be more correlated with human behavior. Interestingly, \gpt{} shows the lowest correlation with human behavior. This raises questions about how architectural or training differences might influence susceptibility to framing. 
%\gabis{this last finding about \gpt{} stands in opposition to the start of the statement, right? Even though it's probably one of the largest models, it doesn't correlate with humans? If so, better to state this explicitly}

This work contributes to understanding the parallels between LLM and human cognition, offering insights into how cognitive mechanisms such as the framing effect emerge in LLMs.\footnote{\name{} data available at \url{https://huggingface.co/datasets/gililior/WildFrame}\\Code: ~\url{https://github.com/SLAB-NLP/WildFrame-Eval}}

%\gabist{It also raises fundamental philosophical and practical questions -- should LLMs aim to emulate human-like behavior, even when such behavior is susceptible to harmful cognitive biases? or should they strive to deviate from human tendencies to avoid reproducing these pitfalls?}\gabis{$\leftarrow$ also following Itay's comment, maybe this is better in the dicsussion, since we don't address these questions in the paper.} %\gabis{This last statement brings the nuance back, so I think it contradicts the previous parapgraph where we talked about ``mitigating'' the effect of framing. Also, I think it would be nice to discuss this a bit more in depth, maybe in the discussion section.}






\section{Background on Causal Inference}
\label{sec:background-causal} 



 \newtextold{In this section, we 
 %formalize the notion of {\em Average Treatment Effect and understand the 
 review the basic concepts and key assumptions for inferring the effects of an intervention on the outcome on collected datasets without performing randomized controlled experiments. 
We use {\em Pearl's graphical causal model} for {\em observational causal analysis} \cite{pearl2009causal} to define these concepts.}


\par
\paratitle{Causal Inference and Causal DAGs} The primary goal of causal inference is to model causal dependencies between attributes and evaluate how changing one variable (referred to as intervention) would affect the other.
Pearl's Probabilistic Graphical Causal Model \cite{pearl2009causal} can be written as a tuple $(\exo, \edvar, Pr_{\exo}, \psi)$, where $\exo$ is a set of {\em exogenous} variables, $\Pr_{\exo}$ is the joint distribution of \exo, and $\edvar$ is a set of observed {\em endogenous variables}.
Here $\psi$ is a set of structural equations that encode dependencies among variables. The equation for $A \in \edvar$ takes the following form:
%that encode the dependencies among the variables.  These equations are of the form 
$$\psi_{A}: 
\dom(Pa_{\exo}(A)) {\times} \dom(Pa_{\edvar}(A)) \to \dom(A)$$
Here $Pa_{\exo}(A) {\subseteq} {\exo}$ and $Pa_{\edvar}(A) {\subseteq} \edvar \setminus \{A\}$ respectively denote the exogenous and endogenous parents of $A$. A causal relational model is associated with a directed acyclic graph ({\em causal DAG}) $G$, whose nodes are the endogenous variables $\edvar$ and there is a directed edge from $X$ to $O$ if  $X {\in} Pa_{\edvar}(O)$. The causal DAG obfuscates exogenous variables as they are unobserved. %Any given set of values for the exogenous variables completely determines the values of the endogenous variables by the structural equations (we do not need any known closed-form expressions of the structural equations in this work). 
The probability distribution $\Pr_{\exo}$ on exogenous variables $\exo$ induces a probability distribution  
on the endogenous variables $\edvar$ by the structural equations $\psi$.  A causal DAG can be constructed by a domain expert as in the above example, or using existing {\em causal discovery} algorithms~\cite{glymour2019review}. 



\begin{figure}
    \centering
    \small
    \begin{tikzpicture}[node distance=0.6cm and 1cm, every node/.style={minimum size=0.5cm}]
        \tikzset{vertex/.style = {draw, circle, align=center}}

        \node[vertex] (Ethnicity) {\bf\scriptsize{{Ethnicity}}};
        \node[vertex, right=0.3cm of Ethnicity] (Gender) {\bf{\scriptsize{Gender}}};
        \node[vertex, right=0.3cm of Gender] (Age) {\bf{\scriptsize{Age}}};
        \node[vertex, below=0.3cm of Gender] (Role) {\bf{\scriptsize{Role}}};
        \node[vertex, right=0.3cm of Role] (Education) {\bf{\small{\scriptsize{Education}}}};
        \node[vertex, below=0.3cm of Role] (Salary) {\bf{\scriptsize{Salary}}};

        \draw[->] (Ethnicity) -- (Salary);
        \draw[->] (Gender) -- (Role);
        \draw[->] (Age) -- (Role);
         \draw[->] (Education) -- (Role);
           \draw[->] (Education) -- (Salary);
             \draw[->] (Ethnicity) -- (Education);
                \draw[->] (Ethnicity) -- (Role);
             \draw[->] (Gender) -- (Education);
               \draw[->] (Age) -- (Education);
                 \draw[->] (Role) -- (Salary);
        \draw[->] (Gender) to[bend right] (Salary);
        \draw[->] (Age) -- (Salary);
    \end{tikzpicture}
    \caption{Partial causal DAG for the Stack Overflow dataset.}
    \label{fig:causal_DAG}
\end{figure}



 \begin{example}
Figure \ref{fig:causal_DAG} depicts a partial causal DAG for the SO dataset over the attributes in Table \ref{tab:data} as endogenous variables (we use a larger causal DAG with all 20 attributes in our experiments). 
  Given this causal DAG, we can observe that the role that a coder has in their company depends on their education, age gender and ethnicity.
\end{example}
\par


\par
\paratitle{Intervention} In Pearl's model, a treatment $T = t$ (on one or more variables) is considered as an {\em intervention} to a causal DAG by mechanically changing the DAG such that the values of node(s) of $T$ in $G$ are set to the value(s) in $t$, which is denoted by $\doop(T = t)$. Following this operation, the probability distribution of the nodes in the graph changes as the treatment nodes no longer depend on the values of their parents. Pearl's model gives an approach to estimate the new probability distribution by identifying the confounding factors $Z$ described earlier using conditions such as {\em d-separation} and {\em backdoor criteria} \cite{pearl2009causal}, which we do not discuss in this paper.


\par
\paratitle{Average Treatment Effect} The effects of an intervention are often measured by evaluating
% \par
% \paratitle{Causal inference, Treatment, ATE, and CATE}
% \newtextold{One of the primary goals  of {\em causal inference} is to estimate the effect of making a change in terms of a {\em treatment} $T$ (often referred to as an intervention)
% on the outcome $O$. 
% %A variable that is modified is often referred to as the treatment variable $T$ and the metric used to captures 
% The effect of treatment $T$ on outcome $O$ is measured by 
% %is known as 
{\em Conditional Average treatment effect (CATE)}, 
%a {\em treatment variable} $T$ on an outcome variable $O$ (e.g., what is the effect of higher \verb|Education| on \verb|Salary|). 
measuring the effect of an intervention on a subset of records~\cite{rubin1971use,holland1986statistics} by calculating the difference in average outcomes between the group that receives the treatment and the group that does not (called the {\em control} group), providing an estimate of how the intervention by $T$ influences an outcome $O$ for a given subpopulation. 
% Mathematically,
% \begin{equation}
%     %{\small ATE(T,O) = \mathbb{E}[O \mid \doop(T=1)] -      \mathbb{E}[O \mid \doop(T=0)]}
%     {\small ATE(T, O) = \mathbb{E}[O \mid \doop(T=1)] -  
%     \mathbb{E}[O \mid \doop(T=0)]}
% \label{eq:ate}
% \end{equation}
% In our work, where the treatment with maximum effect may vary among different subpopulations, we are interested in computing the \emph{Conditional Average Treatment Effect} (CATE), which measures the effect of a treatment on an outcome on \emph{a subset of input units}~\cite{rubin1971use,holland1986statistics}. 
Given a subset of the records defined by (a vector of) attributes $B$ and their values $b$, 
%g {\in} \Qagg(\db)$ defined by a predicate $G {=} g$ 
we can compute $CATE(T,O \mid B = b)$ as:
{
\begin{eqnarray}    
    %CATE(T,O \mid G=g) = \mathbb{E}[O \mid \doop(T=1)&, G=g] -  \mathbb{E}[O \mid \doop(T=0), G=g] 
   % CATE(T,O \mid B = b) = 
    \mathbb{E}[O \mid \doop(T=1), B = b] -  
    \mathbb{E}[O \mid \doop(T=0), B = b]\label{eq:cate}
\end{eqnarray}
}
Setting $B=\phi$ is equivalent to the ATE estimate.
The above definitions assumes that the treatment assigned to one unit does not affect the outcome of another unit (called the {Stable Unit Treatment Value Assumption (SUTVA)) \cite{rubin2005causal}}\footnote{This assumption does not hold for causal inference on multiple tables and even on a single table where tuples depend on each other.}. 


The ideal way of estimating the ATE and CATE is through {\em randomized controlled experiments}, 
where the population is randomly divided into two groups (treated and control, for binary treatments): 
%treated group that receives the treatment and control group that does not (denoted by 
%{the \em treated} group 
denoted by 
$\doop(T = 1)$ 
%for a binary treatment)  (the {\em control} group, 
and $\doop(T = 0)$ resp.)~\cite{pearl2009causal}.
%\sr{edited up to here, going to read the rest first, this section should not look like causumx}
%\par
%\par
However, randomized experiments cannot always be performed due to ethical or feasibility issues. In these scenarios, observational data is used to estimate the treatment effect, which requires the following additional assumptions. 
% {\em Observational Causal Analysis} still allows sound causal inference under additional assumptions. Randomization in controlled trials mitigates the effect of {\em confounding factors}, i.e., attributes that can affect the treatment assignment and outcome. Suppose we want to understand the causal effect of \verb|Education| on \verb|Salary| from the SO dataset.  %in Example~\ref{ex:running_example}. 
% We no longer apply Eq. (\ref{eq:ate}) since the values of \verb|Education| were not assigned at random in this data, and obtaining higher education largely depends on other attributes like \verb|Gender|, \verb|Age|, and \verb|Country|. 
% Pearl's model provides ways to account for these confounding attributes $Z$ to get an unbiased causal estimate from observational data under the following assumptions ($\independent$ denotes independence):
% \vspace{-2mm}
\newtextold{
The first assumption is called {\em unconfoundedness} or {\em strong ignorability}  \cite{rosenbaum1983central} says that the independence of outcome $O$ and treatment $T$ conditioning on a set of confounder variables  (covariates) $Z$, i.e.,
%\begin{eqnarray}
 $    O \independent T | Z {=} z$.
 %\label{eq:unconfoundedness}
%\end{eqnarray}
The second assumption called {\em overlap or positivity} says that there is a chance of observing individuals in both the treatment and control groups for every combination of covariate values, i.e., 
%\begin{eqnarray}
   $ 0 < Pr(T {=} 1 ~~|~~Z {=} z)< 1 $.
   %\label{eq:overlap}
%\end{eqnarray}
}
%\sg{Is this overlap or positivity? maybe both are the same?} \sr{yeah - same - from Google AI - The overlap assumption, also known as the positivity assumption, is a key assumption in causal inference that states that there is a chance of observing individuals in both the treatment and control groups for every combination of covariate values.}
% The above conditions are known as {\em Strong Ignorability} in Rubin's model \cite{rubin2005causal}.
The unconfoundedness assumption requires that the treatment $T$ and the outcome $O$ be independent when conditioned on a set of variables $Z$. In SO, assuming that only $Z$ =\{\verb|Gender|, \verb|Age|, \verb|Country|\} affects $T = $ \verb|Education|, if we condition on a fixed set of values of $Z$, i.e., consider people of a given gender, from a given country, and at a given age, then $T = $ \verb|Education| and $O = $ \verb|Salary| are independent. For such confounding factors $Z$,  Eq. (\ref{eq:cate}) reduces to the following form 
(positivity 
gives the feasibility of the expectation difference): 
 \vspace{-1mm}
{\small
\begin{flalign}    
% \begin{eqnarray}
   % % & ATE(T,O) = \mathbb{E}_Z \left[\mathbb{E}[O \mid T=1, Z = z] -  
   %  \mathbb{E}[O \mid T=0, Z = z] \right] \label{eq:conf-ate}\\
 & CATE(T,O {\mid} B {=} b) {=} \nonumber
    \mathbb{E}_Z \left[\mathbb{E}[O {\mid} T{=}1, B {=} b, Z {=} z] {-}  
    \mathbb{E}[O {\mid} T{=}0, B {=} b, Z {=} z]\right]\label{eq:conf-cate}
\end{flalign}
% \end{eqnarray}
}
% \vspace{-4mm}
This equation contains conditional probabilities and not $\doop(T = b)$, which can be estimated from an observed data. 
Pearl's model gives a systematic way to find such a $Z$ when a causal DAG is available. 




%%%%%%%%%%%%%%%%%%%%%%%%%%%%%%%%%%%%%%%%%%%%%%%%
\begin{figure*}[t]
    \centering
    \begin{subfigure}[b]{0.49\textwidth}
        \centering
        \includegraphics[width=\textwidth]{figures/criu-cuda-checkpoint-arch.pdf}
        \caption{Checkpoint/restore with CUDA plugin.}
        \label{fig:criu-cuda-checkpoint-arch}
    \end{subfigure}
    \hspace{0.005em}
    \rulesep
    \hspace{0.005em}
    \begin{subfigure}[b]{0.49\textwidth}
        \centering
        \includegraphics[width=\textwidth]{figures/criu-amdgpu-checkpoint-arch.pdf}
        \caption{Checkpoint/restore with AMD GPU plugin.}
        \label{fig:criu-amdgpu-checkpoint-arch}
    \end{subfigure}
    \vspace{-0.5em}
    \caption{An overview of the transparent checkpoint/restore mechanisms with CUDA and AMD GPU plugins for CRIU.}
    \label{fig:criu-gpu-checkpoint-arch}
    \vspace{-1em}
\end{figure*}
%%%%%%%%%%%%%%%%%%%%%%%%%%%%%%%%%%%%%%%%%%%%%%%%
\section{Transparent GPU Checkpointing}\label{sec:design}%
Several open-source tools enable transparent checkpointing of Linux processes running on the CPU~\cite{hhargrove2006berkeley,ansel2009dmtcp,criu}, of which Checkpoint/Restore in Userspace (CRIU) is the most widely used and actively maintained. However, a key limitation of CRIU is that, out of the box, it does not support saving and restoring the state of external hardware devices such as GPUs~\cite{shukla2022singularity,eiling2022cricket}. To address this limitation, we extend the functionality of CRIU with \textit{plugins} (\autoref{sec:gpu-plugins}) that handle GPU state. In comparison to the previous work (utilizing device-proxy mechanisms to intercept, log, and replay API calls~\cite{eiling2022cricket,eiling2023cricket,shukla2022singularity,gupta2024just}), we leverage recently introduced driver capabilities to enable transparent GPU checkpointing~\cite{bhardwaj2021drm,gurfinkel2024checkpointing}. Our aim is to enable \textit{fully transparent} checkpointing that supports a wide range of GPU devices and avoids the performance overheads and limitations of API interception (\autoref{sec:background}).

%%%%%%%%%%%%%%%%%%%%%%%%%%%%%%%%%%%%%%%%%%%%%%%%%%%%%%%%%
\subsection{GPU Plugins}\label{sec:gpu-plugins}
%
Checkpointing of CUDA~\cite{gurfinkel2024checkpointing} and ROCm~\cite{bhardwaj2021fast} applications is achieved through driver capabilities that capture and restore the GPU state (e.g., memory) associated with the target processes. Since this functionality is specific to GPU-accelerated applications and not required for other (e.g., CPU-only) workloads, we implement it as dynamically loadable shared libraries (\textit{plugins}), which can be optionally installed. When these plugins are installed, they are loaded during CRIU's initialization phase and utilized to handle GPU resources. ~\Cref{fig:criu-gpu-checkpoint-arch} illustrates the checkpoint/restore mechanisms with CUDA (\autoref{sec:cuda-plugin}) and AMD GPU (\autoref{sec:amd-gpu-plugin}) plugins. These plugins implement callbacks that are executed at specific stages (known as \textit{hooks}; \autoref{sec:plugin-hooks}) during the checkpoint and restore operations. In addition, each plugin defines \textit{initialization} and \textit{exit} callback functions. The initialization function is called when the plugin is loaded, with an argument specifying the current CRIU operation (\textit{dump}, \textit{pre-dump}, or \textit{restore}). Similarly, the plugin's exit function is invoked at the end of the checkpoint/restore operation, with an argument indicating whether the operation has been successful. This allows the plugins to perform cleanup tasks or, in the event of an error, restore the target processes to their original state.

%%%%%%%%%%%%%%%%%%%%%%%%%%%%%%%%%%%%%%%%%%%%%%%%%%%%%%%%%
\subsubsection{CUDA Plugin}\label{sec:cuda-plugin}%
The CUDA plugin utilizes a checkpointing utility called \texttt{cuda-checkpoint}~\cite{cuda-checkpoint} to perform a set of actions (\textit{lock}, \textit{checkpoint}, \textit{restore}, \textit{unlock}) for all tasks running on NVIDIA GPUs.
%
In particular, these actions are used to enable transparent GPU checkpointing as follows:
\begin{enumerate}[label=\itshape(\roman*\upshape),nosep]
    \item \textit{Locking} all CUDA APIs affecting the GPU state of the target processes and waiting for active operations (e.g., stream callbacks) to complete. \sys uses a timeout (10 seconds by default) with this action to avoid indefinite blocking. If the timeout expires, \sys attempts to restore all CPU and GPU tasks to their original state.

    \item \textit{Checkpointing} the GPU state of CUDA tasks into host memory allocations managed by the driver, and releasing all GPU resources held by the application.
\end{enumerate}
%
Executing these steps results in the CUDA tasks entering a \textit{checkpointed} state without direct reference to GPU hardware.
%
This allows to perform checkpoint/restore operations with CRIU similar to a CPU-only workloads.
%
It is important to note that a standalone invocation of the \texttt{cuda-checkpoint} tool does not handle the state of processes and threads running on the CPU, which can result in undefined behavior.
%
For example, multi-threaded workloads, such as Ollama~\cite{morgan2023ollama}, use error-handling mechanisms that detect unresponsive GPU tasks and restart them.
% 
To prevent inconsistencies and undefined behavior, \sys ensures that all CPU and GPU tasks are suspended (locked) before checkpointing their state.
% 
This is achieved through the Linux ptrace seize with interrupt mechanism~\cite{linux-ptrace}, which halts the execution of relevant processes and threads running on the CPU, allowing \sys to capture their state in a unified CPU-GPU snapshot.
%
Restoring the state of CUDA applications has the following steps:
%
\begin{enumerate}[label=\itshape(\roman*\upshape),nosep]
    \item \textit{Restore} resources such as device memory back to the GPU, memory mappings to their original addresses, and reconstruct CUDA objects (e.g., streams and contexts).
    \item \textit{Unlock} driver APIs, allowing the CUDA application to resume execution on the GPU.
\end{enumerate}
% 
In addition, a boolean flag is set in the inventory image of the snapshot indicating whether it contains GPU state, allowing for compatibility checks and optimizations during restore.

%%%%%%%%%%%%%%%%%%%%%%%%%%%%%%%%%%%%%%%%%%%%%%%%%%%%%%%%%
\begin{figure*}[t]
    \centering
    \begin{subfigure}[]{0.45\textwidth}
        \centering
        \includegraphics[width=\textwidth]{figures/cuda-checkpoint-flow.pdf}
        \caption{Sequence of interactions between CRIU, cuda-checkpoint, and NVIDIA driver.}
        \label{fig:cuda-checkpoint-flow}
    \end{subfigure}
    \hspace{0.95em}
    \rulesep
    \hspace{0.05em}
    \begin{subfigure}[]{0.45\textwidth}
        \centering
        \includegraphics[width=.85\textwidth]{figures/amdgpu-checkpoint-flow.pdf}
        \caption{Sequence of interactions between CRIU and KFD.}
        \label{fig:amdgpu-checkpoint-flow}
    \end{subfigure}
    \vspace{-0.5em}
    \caption{Sequence diagrams of CRIU interactions with NVIDIA and AMD drivers.}
    \label{fig:plugins-checkpoint-workflow}
    \vspace{-1em}
\end{figure*}
%%%%%%%%%%%%%%%%%%%%%%%%%%%%%%%%%%%%%%%%%%%%%%%%%%%%%%%%%
\subsubsection{AMD GPU Plugin}\label{sec:amd-gpu-plugin}
The AMD GPU plugin enables transparent checkpoint/restore using input/output control (\texttt{ioctl}) operations with the Kernel Fusion Driver (KFD). These operations are used to pause and resume the execution of GPU processes, as well as to capture and restore their state, which consist of memory buffer objects (BOs), queues, events, and topology.

GPU-accessible BOs are kernel-managed device (VRAM) and system (graphics translation table) memory, user-managed memory (userptr), and special apertures for signaling (doorbell) and control registers (MMIO). The saved BO properties include buffer type, handle, size, virtual address, device file offset for CPU mapping, and memory contents.

GPU work is typically submitted through user-mode queues with associated user- and kernel-managed memory buffers.
Checkpointing requires preempting and saving the state of all queues belonging to the process.
This includes queue type (compute or DMA), kernel-managed control stack, memory queue descriptor, read/write pointers, doorbell offset, and architected queueing language (AQL) pointer.
The state stored in user-managed BOs includes ring buffer (commands), AQL queue, completion tracking (end-of-processing) buffer, and context save area (preempted shader state).
%
For checkpoint/restore of GPU-to-host signaling events, the allocated event IDs and their signaling state are saved and restored, while the event slot contents are included in the memory data.
%
During checkpointing, the plugin performs the following \texttt{ioctl} operations:
\begin{enumerate}[label=\itshape(\roman*\upshape), nosep]
    \item \texttt{PROCESS\_INFO} -- collecting metadata about the process, pausing its execution, and evicting all queues

    \item \texttt{CHECKPOINT} -- capturing the GPU state described above

    \item \texttt{UNPAUSE} -- restores the evicted queues
\end{enumerate}
%
For security reasons, KFD allows these \texttt{ioctl} calls to be performed only by the same process that opened the \texttt{/dev/kfd} file descriptor, and requires \texttt{CAP\_CHECKPOINT\_RESTORE} or \texttt{CAP\_SYS\_ADMIN} capability.
% 
The plugin performs the following operations during restore:
\begin{enumerate}[label=\itshape(\roman*\upshape),nosep]
    \item \texttt{RESTORE}: reinstates the checkpointed state of processes
    \item \texttt{RESUME}: resumes execution of processes on the GPU
\end{enumerate}

Checkpointed applications can only be restored on systems with compatible GPU topology with the same number, type, memory size, VRAM accessibility by the host, and connectivity between GPUs.
When restoring on a different system or with different subset of GPUs on the same system, the unique GPU identifiers (GPUIDs) might be different during restore. These identifiers are based on properties like the instruction set and compute units. To address this, the plugin performs a translation of the GPUIDs used by the restored processes that applies to all KFD ioctl calls.


%%%%%%%%%%%%%%%%%%%%%%%%%%%%%%%%%%%%%%%%%%%%%%%%%%%%%%%%%
\subsubsection{Plugin Hooks}\label{sec:plugin-hooks}
CRIU provides a set of hooks for checkpointing external resources such as UNIX sockets, file descriptors, mountpoints, and network devices. These hooks serve as an API that can be used with plugins to extend the existing functionality.

\stitle{AMD GPU Plugin Hooks.} CRIU provides two hooks for handling checkpoint and restore operations with device files: \texttt{DUMP\_EXT\_FILE} and \texttt{RESTORE\_EXT\_FILE}. When checkpointing ROCm applications, these hooks are invoked for the \texttt{/dev/kfd} and \texttt{/dev/dri/renderD*} device nodes. The obtained KFD file descriptor is used by the plugin to perform \texttt{ioctl} calls to manage memory, queues, and signals, while per-GPU device render node files are utilized to handle CPU mapping of VRAM and GTT BOs. Two additional plugin hooks have been introduced to enable checkpoint/restore of AMD GPU device virtual memory areas (VMA): \texttt{HANDLE\_DEVICE\_VMA} and \texttt{UPDATE\_VMA\_MAP}. These hooks allow the plugin to translate device file names and mmap offsets to newly allocated ones during restore. In particular, these offsets identify BOs within a render node device file, and the translation mechanism allows a process to be restored on a different GPU. A \texttt{RESUME\_DEVICES\_LATE} hook has been introduced to finalize the restore of userptr mappings and resume execution on the GPU for each restored process, after CRIU's restorer PIE code has restored all VMAs.

\stitle{CUDA Plugin Hooks.} Similarly, two additional plugin hooks have been introduced to invoke \textit{lock} and \textit{checkpoint} actions with the \texttt{cuda-checkpoint} utility for processes running on NVIDIA GPUs: \texttt{PAUSE\_DEVICES} and \texttt{CHECKPOINT\_DEVICES}. The \textit{pause} hook is called immediately before the target CPU processes are frozen. This hook is used by the CUDA plugin to place the corresponding GPU tasks in a \textit{locked} state, halting any pending work and preparing them to be checkpointed. Following this, the \textit{checkpoint} hook is called after all CPU and GPU processes are frozen/locked state to checkpoint their GPU state into host memory. The CUDA plugin also utilizes the \texttt{RESUME\_DEVICES\_LATE} hook to \textit{restore} the state of processes from host memory to the GPU and perform the \textit{unlock} action to resume their execution.

%%%%%%%%%%%%%%%%%%%%%%%%%%%%%%%%%%%%%%%%%%%%%%%%%%%%%%%%%
\subsection{Checkpoint/Restore Workflow}
The sequence of operations described above for AMD GPU and CUDA plugins is illustrated in \Cref{fig:plugins-checkpoint-workflow}. Each plugin uses a different method for checkpointing the GPU state of applications. For CUDA applications, \circled{1} performs a \textit{lock} action that halts the execution of device API calls. Similarly, the AMD GPU plugin invokes a KFD ioctl call to collect metadata, pause execution, and the evict queues for the target ROCm application. \circled{2} checkpoints the GPU state to host memory for CUDA applications. In contrast, at this stage the AMD GPU plugin saves the GPU state into a set of checkpoint files. \circled{3} continues with traditional checkpoint operations for CUDA applications as the GPU state is included in host memory. The AMD GPU plugin at this stage invokes a KFD ioctl call to resume the state of queues. The restore functionality has analogous sequence operations as described in \Cref{sec:gpu-plugins}.

\begin{figure*}[t]
    \centering
    \includegraphics[width=\linewidth]{figures/local_response_prompting}
    \caption{The text suggestions in the local response widget are flexible:  \textit{(A)} Users get suggestions without any input.  \textit{(B)} \revision{Suggestions} can be adapted and refined by entering text, for example keywords or a draft snippet. In all cases, suggestions are generated with an LLM based on the text of the incoming email and all local responses that the user has entered so far, \revision{even if responses have been} added to later parts of the email \revision{first}. \revision{In \textit{(C)}, for example, the suggested title of the idea pitch is generated based on the information about the project that the user has already entered in local responses below.} Note that suggestions are paginated, with three pages of two suggestions each.}
    \Description{
    This figure shows screenshots of our user interface displaying an incoming email. The figure is divided into three sections:
    A) Left Section: Email-driven Suggestions without User Input
    The selected sentence in the incoming email reads: "Please feel free to tell me any ideas what we could get her!"
    Below the email, there is an empty text field for optional user input.
    Two AI-generated responses are shown beneath the text field: one suggests a piece of jewellery as a gift, while the other does not offer any ideas.
    B) Centre Section: Prompt-driven Suggestions with User Input
    The same sentence from the email is selected: "Please feel free to tell me any ideas what we could get her!"
    This time, the keywords "balloon ride" are entered into the text field below.
    As a result, the AI-generated suggestions include the idea of a balloon ride in both proposed texts.
    C) Right Section: AI Suggestions Respecting Existing Responses
    One sentence from the incoming email is selected, and the text field below remains empty.
    The AI-generated responses incorporate information from existing responses elsewhere in the email.
    Additionally, all AI suggestions are paginated, with three pages of two suggestions each, as indicated by arrows next to the suggestions.}
    \label{fig:local_response_prompting}
\end{figure*}



\section{Implementation}
\label{sec:implementation}
We implemented a frontend and backend, which preprocessed emails, logged user data, and generated responses. 

\subsection{Frontend}
We implemented our web app with the React\footnote{\url{https://react.dev/}} framework.

\subsubsection{Display of the Incoming Email}
This view matches standard mobile email UIs: It includes the sender's name and picture, the email subject, and the main text body (\cref{fig:teaser} left). 
The user can select sentences in the incoming email by tapping on them (cf. design goal \ref{dg:humandecides}: \revision{Human decides, AI supports}; and goal \ref{dg:microtasking}: \revision{Support mobile microtasking}). This opens the local response widget (\cref{sec:impl_local_response_widget}).
The ``Finalize Reply'' button at the bottom of the UI switches to the next screen (\cref{fig:teaser} centre), which we describe in \cref{sec:impl_finalize}.
In accordance with design goal \ref{dg:workflows} \revision{(Support diverse workflows -- with and without AI)}, no interaction with any sentence or AI feature is required before proceeding to this next screen.



\subsubsection{Local Response Widget}\label{sec:impl_local_response_widget}

This UI widget is inserted into the email text below the user's selected sentence. It comprises of a text field (\cref{fig:teaser} C) and a paginated card view that shows text suggestions (\cref{fig:teaser} D). In the text field, users can enter both manual responses or prompts to refine these suggestions (\cref{fig:local_response_prompting}). 

Concretely, the widget offers six suggestions (2 positive, 2 negative, 2 neutral), showing two at once. The system aims to show one positive and one negative response on the first page, if possible. \revision{This was motivated by findings on positivity bias in AI-generated communication text~\cite{Mieczkowski2021} and to increase the chance of offering a response option fitting to the user's intent (cf.~\cite{Kannan2016smartreply}).} \revision{We realised this by prompting the LLM to do so (see \cref{sec:appendix_sentence_without_input_prompt}). Concretely, the variable ``attribute'' in the prompt template was replaced with \textit{accepting}, \textit{declining}, and \textit{neutral} to generate varying suggestions. In our tests, we observed that this simple prompting approach worked well and that it did not negatively impact generated suggestions in cases where these attributes do not apply (e.g. our ``cat'' example in \cref{fig:teaser}D).} %
Users can navigate through suggestions using the adjacent arrow buttons. They can accept a suggestion by tapping on it.

The widget has two states -- open and collapsed (\cref{fig:teaser} A, B): 
It is collapsed by tapping the currently selected sentence again, by selecting a different sentence, by accepting a suggestion, or by clicking on the check mark in the top right corner. When the text field is empty, the check mark transforms into a trash icon to delete the local reply. Multiple widgets can be in the collapsed state throughout the email but only one widget at a time can be open and in focus. %
A widget's text is shown in the collapsed state. This allows users to keep track of all their local replies so far. Tapping on a collapsed widget opens it again for further editing. 






\subsubsection{Finalising the Reply}\label{sec:impl_finalize}
This view  (\cref{fig:teaser} centre) shows the current state of the reply after the local response step. That is, it displays any text entered in response to individually selected sentences together in a single text field.

Users can manually adjust this text and/or tap the ``Improve'' button to request the AI to enhance the email. 
This \imppass{} feature is realised with a prompt \revision{(see \cref{subsec:appendix_improve_email_prompt})} to the underlying LLM to correct spelling and grammar, refine wording, and add missing salutations or regards while adhering to both the incoming email's content and the existing reply text. 

If no text is entered first, the ``Improve'' button acts as message-level support, generating a reply based on the incoming email's text and the current input on this screen. For example, a user could skip the local response and enter a prompt here, effectively realising a message generation workflow similar to the industry default (\cref{sec:related_work_current_products}). This flexibility contributes to our design goal \ref{dg:workflows} \revision{(Support diverse workflows – with and without AI).}

When the user is satisfied with their reply, the email can be sent by tapping the ``Send Email'' button at the bottom of this screen.


\subsubsection{Improved Email Pop-up}\label{sec:impl_imppass}
The \imppass{} feature does not change the user's text directly, in line with our design goal \ref{dg:control} \revision{(The user stays in control).}
Instead, the new text is shown in a pop-up view with formatting familiar from ``track changes'' in text editors (\cref{fig:teaser} right). 
Users can approve these changes, which updates the text, or discard them (cf. design goal \ref{dg:humandecides}: \revision{Human decides, AI supports}.) Further editing after acceptance and/or requesting improvements repeatedly is possible. 


\subsection{Backend}
Our prototype's backend has three purposes: 
(1) It \textit{hosts the web app} on a Next.js server. 
(2) It provides \textit{email preprocessing}, which handles tasks such as sentence-splitting and making API calls to the LLM. 
(3) It \textit{hosts the LLM}. 

We experimented with several models and APIs and discussed factors such as latency, stability of service, and subjective response quality in meetings with all authors. Based on this, we used the Llama 3 8B Instruct \cite{llama3modelcard} model for the main study. 

Similarly, we iterated over several prompting approaches for the text generation features. Overall, this resulted in a few-shot approach, providing the model with several input-output examples to generate fitting responses. 
As an overview, we use the following prompt templates (details in \cref{sec:appendix_prompts}):

\paragraph{Sentence-level support, without user input:}
We prompted six suggestions for the sentence selected in the email (2 positive, 2 neutral, 2 negative). This balanced the options, following related work~\cite{Kannan2016smartreply}, as the LLM favoured positive responses in our tests.

\paragraph{Sentence-level support, with user input:} 
This was identical to the above case but now it included the user's input in addition to the selected sentence. %
We emphasised alignment with the user's sentiment (e.g. no negative suggestions if the user had entered ``yes'').


\paragraph{Message-level support:}
We prompted the LLM to answer to the whole email, also by taking into account any current user input, if available. A variation of this was also used for the \imppass{} feature (\cref{sec:impl_imppass}). That prompt emphasised improving the current state of the reply while closely adhering to the information provided by the user. %

\begin{figure*}[h!]
    \centering %trim titles
     \includegraphics[width=.5\textwidth,trim={0 1.4cm 0 3.4cm},clip]{figures/sae_feature_accuracies_layer31_pre_SQUAD_train.png}%{figures/sae_feature_accuracies_layer31_pre.png}
    ~%
    \includegraphics[width=.5\textwidth,trim={0cm 1.4cm 0 3.4cm},clip]{figures/sae_feature_accuracies_layer31_pre_BoolQ_train.png}
\caption{Out-of-distribution comparison between top SAE features,  pre-activation, and linear probes on layer 31; trained on SQUAD (left) and BoolQ (right).}
    \label{fig:sae-probe_pre31}
\end{figure*}

\begin{figure}[h!]
    \centering %trim titles
    \includegraphics[width=.5\textwidth,trim={0 1.4cm 0 3.4cm},clip]{figures/hierarchical_sae_probe_layer31_pre.png}
    \caption{Combinations of SAE features, displaying the median value across top feature groups with quartile ranges in the error bars.}
    \label{fig:sae-k-pre31}
\end{figure}

% \begin{figure*}[t]
% \centering

% %--------------- First row ---------------%
% \begin{subfigure}[t]{0.49\textwidth}
%   \centering
%   \includegraphics[width=\linewidth,
%                    trim={2cm 1.4cm 0 3.4cm},clip]
%                    {figures/sae_feature_accuracies_layer31_pre_SQUAD_train.png}
%   \caption{Out-of-distribution comparison on layer 31, trained on SQuAD.}
%   \label{fig:sae-probe_pre31-squad}
% \end{subfigure}
% \hfill
% \begin{subfigure}[t]{0.49\textwidth}
%   \centering
%   \includegraphics[width=\linewidth,
%                    trim={2cm 1.4cm 0 3.4cm},clip]
%                    {figures/sae_feature_accuracies_layer31_pre_BoolQ_train.png}
%   \caption{Out-of-distribution comparison on layer 31, trained on BoolQ.}
%   \label{fig:sae-probe_pre31-boolq}
% \end{subfigure}

% \vspace{0.5em}  % Tighten or remove if desired

% %--------------- Second row ---------------%
% \begin{subfigure}[t]{0.49\textwidth}
%   \centering
%   \includegraphics[width=\linewidth,
%                    trim={0 1.4cm 0 3.4cm},clip]
%                    {figures/hierarchical_sae_probe_layer31_pre.png}
%   \caption{Combinations of SAE features.}
%   \label{fig:sae-k-pre31}
% \end{subfigure}
% \hfill
% \begin{subfigure}[t]{0.30\textwidth}
%   \centering
%   \includegraphics[width=\linewidth,
%                    trim={5.2cm 4.1cm 0 3.4cm},clip]
%                    {figures/feature_similarities.png}
%   \caption{Cosine similarities of top SAE features and probes.}
%   \label{fig:similarities}
% \end{subfigure}

% \caption{
%   Four related plots: 
%   (a) and (b) show out-of-distribution comparisons on layer 31 
%   for SQuAD vs.\ BoolQ training;
%   (c) demonstrates combinations of SAE features;
%   (d) shows feature/probe cosine similarities.
% }
% \label{fig:all_four}
% \end{figure*}




\section{Evaluation}



In the following, we present of our main experiments; see the appendix for additional findings.

% Todo
% Get some formatted dataset examples for all datasets
% Try question in celeb dataset (how old)
% Check probe accuracy and generalization for different training data sizes
% Averaging linear probes as baseline?
% Use higher k
% Prioritize over weekend with higher number of features + OOD eval
% Baseline attention probe
% Change cosine similarity analysis to average absolute
% Add clean reconstruction result of residual stream probe with SAE
% Cosine sim of residual stream probes with top SAE features
% Probes with each other
% Probes with top sae features
% Check pre vs post relu
% Compare features to narrow SAE
% Compare to Layer 31 SAE
% Is it more sparse? Less combinations needed? Then it is a feature that is not fully computed in L20?




\myparagraph{Linear vs SAE Probes: Generalization, Figure~\ref{fig:sae-probe_pre31}} 
We focus first on layer 31.
In domain, the best SAE features reach an accuracy of around 0.8 while the linear probes reach 0.9. Note that this is not surprising, since the probes have more parameters that are actually trained and thus optimized for this data. Nevertheless, it shows some advantage of probes in case in-domain data is available. 

We see rather great variation across our out-of-distribution datasets. Our custom Equation data stands out in that several SAE features and also the probe reach high performance. While this seems to show that the mathematical context makes answerability easier to detect, observe that the performance is considerably worse on layer 20, see Figure~\ref{fig:sae-probe_pre20} in the appendix. %This shows that it is a complex concept only fully represented later in the model.

Some, but few, top SAE features reach considerable performance out of distribution on IDK - matching the performance of the linear probe - and Celeb. Yet, the performance on BoolQ is considerably bad. %\vt{REASONS?} 
On the other hand, the linear probe performs bad on Celeb. Figure~\ref{fig:res-probe-all} in the Appendix shows the median value over 10 bootstrap samples including quartiles in the error bars; overall it correlates with performance.

For layer 20 (Figure~\ref{fig:sae-probe_pre20}), we see generally worse performance. 
Interestingly, the numbers for Celeb are significantly worse than all others for both the SAE and the linear probes. Since we see one exception (an SAE feature with higher than random performance), we hypothesize that there are special features encoding knowledge about celebrities which do not happen to be among our top features. % ie rather than the knowledge is not present on layer 20
%this might also explain why the probe performs bad on celeb that this would be more about people features not answerability
In fact, a closer investigation reveals that there are good features for BoolQ and our domain-specific Equation and Celeb datasets on layer 20 already (see Figure~\ref{fig:top-in-domain} in the appendix), but they are not the same features as the ones found by training on SQuAD. %TODO

Finally, we confirmed our findings by training on  BoolQ (also 2k samples) and evaluating on the other datasets. We mainly see that varying the training data can make everything considerably worse, even with the same task and seemingly similar, but potentially lower-quality data. The unanswerable samples in BoolQ were constructed by combining contexts and questions of similar dataset samples, hence capture only one type of unanswerability.

Overall, our experiments demonstrate one main critical issue with OOD data: \emph{the standard procedure for finding good SAE features can easily fail, even if good features are available}. 
The fact that good features exist while the linear probes also fail shows some potential of SAEs. Yet finding good, generalizing features represents an open challenge.

%A key issue for SAE probing could be feature splitting, a phenomenon where SAEs of different sizes learn features in different granularities, often splitting more general features into multiple more specialized features \citep{bricken2023monosemanticity, chanin2024absorption}. If abstract concepts like answerability are split into many separate features, this can cause problems for feature-based practical applications.

\myparagraph{Top Features, %Distributions, 
Figure~\ref{fig:sae-probe_pre31}} 
Interestingly, the top three features on the in-domain SQuAD data happen to also generalize better here. This does not turn out be the case beyond the top-1 feature more generally, see Appendix~\ref{app:eval-other-layers}. For BoolQ, the variability of the results precludes clear conclusions.
%

\myparagraph{SAE Feature Combinations, Figure~\ref{fig:sae-k-pre31}} 
Given the partly domain-specific nature of our out-of-distribution datasets, we hypothesized that combinations of features might work better as general probes. %Observe that such combinations have also been considered in previous works \cite{lecun,oneother}.
However, while increasing the number of SAE features improves the in-domain performance, OOD performance doesn't improve upon the best performing individual feature (top of blue error bars) here; layer 31, pre-activation. Other examples in Appendix~\ref{app:feature-combinations} show similar trends, and even some degradation. This underlines our above finding that the ood setting requires better methods for SAE feature search.

% \begin{wrapfigure}
% {r}{0.25\textwidth}

\begin{figure}[h!]
    \centering %trim titles
    \includegraphics[width=.18\textwidth,trim={5.2cm 4.1cm 0 3.4cm},clip]{figures/feature_similarities.png}
    \caption{Cosine similarities of top SAE features and linear probes for different seeds; the blue square shows high similarity between linear probes.}
\label{fig:similarities}
\end{figure}
  
% \end{wrapfigure}
\myparagraph{Feature Similarity, Figures~\ref{fig:similarities} \& \ref{fig:similarity_k}}

We find great similarity between different linear probes but only slight similarity between SAE features and individual probes, and it's even less between SAE features. Interestingly, the best SAE feature turns out to have highest (though low) similarity with the probes. Figure~\ref{fig:similarity_k} shows that combining SAE features yields greater similarity with linear probes.




% \myparagraph{Other experiments}
% We validated our setup by searching for bias-related features as it was done in related works.
% We also experimented with (inofficial) SAEs for an instruction-tuned Llama model, but the quality of the SAEs was not good enough for further experimentation. Finally, Gemma 2 2B and also the base models \vt{did not yield good enough performance on the question answering task itself}. 

%%%%%%%%%%%%%%%%%%%%%%%%%%%%%%%%%%%%%%%%%%%
\section{Discussion}~\label{sec:discussion}%
\sys is a novel technique for fully transparent checkpointing of GPU applications implemented with CUDA and ROCm. Our evaluation results show that \sys can create unified CPU-GPU snapshots for large models in single- and multi-GPU setups without steady-state performance overhead. Our evaluation results further demonstrate that \sys scales linearly with the number of GPU devices for data-parallel workloads, enabling efficient checkpointing with large-scale applications.

\stitle{Deterministic Restore.}
In contrast to previous work, which requires validation of the replayed GPU device API calls~\cite{gupta2024just}, \sys relies on a locking mechanism that ensure the execution of tasks is suspended before creating a checkpoint. This guarantees consistent CPU-GPU snapshots and deterministic restore operations, during which both CPU and GPU state are restored before the application resumes execution.

\stitle{Real-World Deployments.}
The proposed GPU checkpointing mechanism has been successfully implemented in several production systems, including MemVerge~\cite{wu2025memory} and Modal~\cite{belotti2025memory}.


%%%%%%%%%%%%%%%%%%%%%%%%%%%%%%%%%%%%%%%%%%%%%%%
\section{Related Work}~\label{sec:related-work}%
Checkpoint/Restore has recently received significant attention in the context of AI workloads as a way to quickly recover from faults during model training (either in the GPU or in the CPU), and to better utilize hardware when multiple users content for the same hardware. The latter is becoming increasingly relevant with the introduction of LLM Agents in which GPU devices may need to wait for external agents to return before the LLM running on the GPU could proceed~\cite{nvidia2023introduction}. This section summarizes recent research efforts and how they have focused on improving different aspects of Checkpoint Restore, and how it compares to \sys.

\stitle{Checkpoint Types.} Distributed runtime engines such as Ray~\cite{moritz2018ray} and deep learning frameworks and libraries such as PyTorch~\cite{paszke2019pytorch}, TensorFlow~\cite{abadi2016tensorflow}, and MXNet~\cite{chen2015mxnet} allow users to specify checkpoint and restore logic including how and when to checkpoint. Such techniques are specific to each runtime engine and require developer involvement, thus being considered \textit{application-aware checkpoints}.

Many other works proposed and improved designs for transparent checkpoints. Singularity~\cite{shukla2022singularity} proposed a device proxy server design that relies on API interception to log all runtime and drive API calls. This log would be used to reconstruct the GPU state upon a restore operation. Such technique has been used widely as a mechanism for semi-transparent checkpoint restore support~\cite{chaudhary2020balancing,gupta2024just}. However, a design based on a device proxy server leads to a number of challenges (as discussed in \autoref{sec:background}), resulting in limited support for large and complex training and inference workloads. \sys, on the other hand, introduces a new design for checkpoint and restore by offering a fully-transparent and unified checkpoint mechanism to save the state of the application running on the CPU (including the engine/framework/library running the user application), and its corresponding state on the GPU. \sys does not rely on API interception and supports different GPU families with both CUDA and ROCm applications.

\stitle{Deciding when to Checkpoint.} Transparent GPU checkpointing is often used to enable error recovery for training jobs by periodically saving the model parameters and optimizer state to persistent storage. State-of-the-art error recovery solutions today implement \textit{periodic} or \textit{just-in-time} checkpointing support where training jobs can be resumed from any prior checkpoint. When GPU workloads experience an error, they either hang or crash, and cause the training jobs to be unexpectedly terminated. These errors are often detected with a series of diagnostic tests and problematic nodes are cordoned off, allowing the training job to resume on a different node.

Recovery mechanisms used with periodic checkpoints are often optimized to balance low running time overhead and high checkpoint frequency~\cite{mohan2021checkfreq}. These optimizations aim to minimize both the runtime overhead and the amount of work the job has to redo on recovery (recovery cost).
%
In contrast, just-in-time checkpointing solves this problem by creating a checkpoint only after a failure has occurred by leveraging the fact that the GPU state is only updated during a short interval and multiple replicas are holding the same state. As a result, this approach is able to recover from errors by recomputing at most one mini-batch~\cite{gupta2024just}.

\sys can be integrated with both periodic and just-in-time checkpointing policies to i) accelerate the checkpoint and restore operations, and ii) create a unified and consistent snapshot of the CPU and GPU state. Furthermore, \sys does not rely on specific workload characteristics such as relying on replicas having an exact copy of the model, thus being a general approach for GPU state checkpointing.

\stitle{Checkpoint/Restore Optimizations.} There have been many other works that optimize checkpoint/restore operations for GPU-accelerated workloads.
CheckFreq~\cite{mohan2021checkfreq} reduces the overhead of periodic checkpointing by overlapping communication with snapshotting of the model state in the GPU. 
Gemini~\cite{team2024gemini} checkpoints the GPU state to local and remote host memory, and interleaves checkpointing traffic with training traffic to reduce the overheads of checkpointing. 
Nebula~\cite{microsoft2024boost} copies model state asynchronously reducing the time during which the training job is paused for checkpoint. 
DeepFreeze~\cite{nicolae2020deepfreeze} proposes a fine-grain asynchronous checkpointing of deep learning models and shards the checkpointing effort across multiple workers. However, it only considers CPU clusters and does not take into account the cost of snapshotting the model state in memory when trained with state-of-the-art GPUs.
Check-N-Run~\cite{eisenman2022check} proposes incremental/differential checkpointing to speedup recommendation model training and uses quantization techniques to reduce the snapshot size.
ZeRo~\cite{rajbhandari2020zero} shards model parameters and optimizer state across data-parallel GPUs, parallelizing the checkpoint effort.

We consider our work to be complementary to these previous contributions. Optimizations based on sharding, compression, asynchronous, or incremental checkpointing could be incorporated into \sys. In summary, \sys presents a novel checkpoint/restore technique on which many of previously proposed optimizations could be integrated.

Another recent line of related work explores running GPU-accelerated inference workloads on serverless platforms, focusing on optimizing resource efficiency, scheduling, and caching policies~\cite{10.5555/3433701.3433792,280768,ishakian2018serving,280704,10.1145/3620678.3624664,infaas,10.1145/3503222.3507709,234998,yang2024on-demand}. Yang et al.~\cite{yang2024on-demand} also leverages CRIU to accelerate the restore time for GPU workloads, but, unlike \sys, does not explore unified container snapshots. Besides, the paper~\cite{yang2024on-demand} focuses on AMD-only devices and does not explore the performance of checkpoint and restore operations on large ML models.
Nevertheless, most of these systems focus on reducing model load time rather than on a general approach for checkpoint and restore GPU-accelerated applications. Many additional optimizations could be combined with \sys to further facilitate model loading. These optimizations are outside the scope of this paper and will be addressed in future work.

\stitle{Framework-level checkpointing} is supported by many popular machine learning libraries, such as PyTorch~\cite{paszke2019pytorch} and TensorFlow~\cite{abadi2016tensorflow}. This approach requires the application, framework or library to correctly capture and restore the runtime state of applications. However, implementing and maintaining this checkpointing mechanism introduces additional development and testing burdens. In addition, it has been shown to be both error-prone and inefficient, often leading to checkpoint file loss or corruption due to job interruptions, as well as significant recovery times~\cite{mohan2021checkfreq}. Thus, in this work, we focus on developing a reliable and efficient system-level checkpointing that is fully transparent.

\section{Conclusion}\label{sec:conclusion}
This work introduces a novel approach to TOT query elicitation, leveraging LLMs and human participants to move beyond the limitations of CQA-based datasets. Through system rank correlation and linguistic similarity validation, we demonstrate that LLM- and human-elicited queries can effectively support the simulated evaluation of TOT retrieval systems. Our findings highlight the potential for expanding TOT retrieval research into underrepresented domains while ensuring scalability and reproducibility. The released datasets and source code provide a foundation for future research, enabling further advancements in TOT retrieval evaluation and system development.

\section*{Acknowledgments}
The authors gratefully acknowledge Felix Kuehling, Rajneesh Bardwaj, David Yat Sin, David Francis, and Ramesh Errabolu for their invaluable help and support with implementation of AMD GPU plugin. We acknowledge the use of the University of Oxford Advanced Research Computing (ARC) facility~\cite{richards2015university} in conducting parts of the evaluation experiments. This work is partly supported by the EPSRC Doctoral Training Partnership (DTP) [grant number EP/T517811/1].

\balance
\bibliographystyle{plain}
\bibliography{main}

\end{document}