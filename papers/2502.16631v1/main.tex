\documentclass[letterpaper,twocolumn,10pt]{article}
\usepackage{include/usenix}
% \usepackage[available,functional,reproduced]{include/usenixbadges}

% Use § symbol with \autoref
\renewcommand{\sectionautorefname}{\S}
\renewcommand{\subsectionautorefname}{\S}
\renewcommand{\subsubsectionautorefname}{\S}

\usepackage{tikz}
\usepackage{float}
\usepackage{amsmath}
\usepackage{balance}
\usepackage{booktabs}
\usepackage{hyperref}
\usepackage[capitalise,nameinlink]{cleveref}
\usepackage{xcolor}
\usepackage{xspace}
\usepackage{multirow}
\usepackage{tabularx}
\usepackage{algorithm}
\usepackage{algpseudocode}
\algtext*{EndWhile}
\algtext*{EndIf}
\algtext*{EndFor}
\usepackage{caption}
\usepackage{subcaption}
\usepackage[compact]{titlesec}

% Inline enumeration
\usepackage[inline]{enumitem}

\hyphenation{machine workflows CentOS Podman}

\usepackage{enumitem}
\setitemize{noitemsep,topsep=0pt,parsep=0pt,partopsep=0pt}

\newcommand{\sys}[0]{CRIUgpu\xspace}
\def\Snospace~{\S{}}

\newcommand{\heading}[1]{\vspace{4pt}\noindent\textbf{#1}}
\newcommand{\topheading}[1]{\noindent\textbf{#1}}

\renewcommand{\algorithmicrequire}{\textbf{Input:}}
\renewcommand{\algorithmicensure}{\textbf{Output:}}

% For comments.
\newcommand\rb[1]{\textcolor{orange}{RB: #1}}
\newcommand\vk[1]{\textcolor{purple}{Viki: #1}}
\newcommand\rs[1]{\textcolor{blue}{RS: #1}}

% Style commands.
\newcommand{\stitle}[1]{\vspace{1.ex}\noindent{\bf #1}}
\newcommand{\noindentstitle}[1]{\noindent{\bf #1}}

\usepackage[compact]{titlesec}
\titleformat*{\section}{\large\bfseries}
\titleformat*{\subsection}{\normalsize\bfseries}
\titleformat*{\subsubsection}{\normalsize\bfseries}
\titlespacing*{\section}{0pt}{*2}{*1}
\titlespacing*{\subsection}{0pt}{*1.5}{*0.8}

\iffalse
\titleformat{\subsection}[runin]% runin puts it in the same paragraph
       {\normalfont\bfseries}% formatting commands to apply to the whole heading
       {\thesubsection}% the label and number
       {0.5em}% space between label/number and subsection title
       {}% formatting commands applied just to subsection title
       [.]% punctuation or other commands following subsection title
\fi

\captionsetup[figure]{labelfont={bf},name={Figure},labelsep=period}
\captionsetup[table]{labelfont={bf},name={Table},labelsep=period}

\usepackage{tikz}
\newcommand*\circled[1]{
    \tikz[baseline=(char.base)]{
        \node[shape=circle,draw,inner sep=.5pt] (char) {#1};
    }
}

% A vertical rule between subfigures
\newcommand{\rulesep}{\unskip\ \vrule\ }

\begin{document}

% \date{}

\title{\sys: Transparent Checkpointing of GPU-Accelerated Workloads}

\author{
    \rm
        Radostin Stoyanov$^{\star\dagger}$ \enskip
        Viktória Spišaková$^{\ddagger}$ \enskip
        Jesus Ramos$^{\diamond}$ \enskip
        Steven Gurfinkel$^{\diamond}$ \enskip
        Andrei Vagin$^{\triangle}$ \enskip
    \\ \rm
        Adrian Reber$^{\dagger}$ \enskip
        Wesley Armour$^{\star}$ \enskip
        Rodrigo Bruno$^{\bullet}$ \enskip
    \\ \\
    {
        $^{\star}$University of Oxford\enskip
        {$^{\ddagger}$Masaryk University\enskip}
        $^{\diamond}$NVIDIA\enskip 
        $^{\triangle}$Google\enskip
        $^{\dagger}$Red Hat\enskip
    }\\
        {$^{\bullet}$INESC-ID, Instituto Superior Técnico, University of Lisbon\enskip}
}

\maketitle

\begin{abstract}
Retrieval-Augmented Generation (RAG) is often used with Large Language Models (LLMs) to infuse domain knowledge or user-specific information. In RAG, given a user query, a retriever extracts chunks of relevant text from a knowledge base. These chunks are sent to an LLM as part of the input prompt. Typically, any given chunk is repeatedly retrieved across user questions. However, currently, for every question, attention-layers in LLMs fully compute the key values (KVs) repeatedly for the input chunks, as state-of-the-art methods cannot reuse KV-caches when chunks appear at arbitrary locations with arbitrary contexts. Naive reuse leads to output quality degradation.  This leads to potentially redundant computations on expensive GPUs and increases latency. In this work, we propose \sys, a system for managing and reusing precomputed KVs corresponding to the text chunks (we call \textit{chunk-caches}) in RAG-based systems. We present how to identify \hl{\textit{chunk-caches} that are reusable}, how to efficiently perform a small fraction of recomputation to \textit{fix} the cache to maintain output quality, and how to efficiently store and evict \textit{chunk-caches} in the hardware for maximizing reuse while masking any overheads. With real production workloads as well as synthetic datasets, we show that \sys reduces redundant computation by \textbf{51\%} over SOTA prefix-caching and \textbf{75\%} over full recomputation.
\hl{Additionally, with continuous batching on a real production workload, we get a \textbf{1.6$\times$} speedup in throughput and a \textbf{2$\times$} reduction in end-to-end response latency over prefix-caching while maintaining quality, for both the \llama-3-8B and \llama-3-70B models. 
}
\end{abstract}






The increasing reliance on LLMs for multimodal tasks across far-reaching sectors such as healthcare, finance, and manufacturing underscores the need to assess the accuracy and reliability of the information they generate. Vision-Language Models (VLM) have achieved state-of-the-art (SoTA) performance on Visual Question-Answering (VQA) benchmarks, and these models often utilize Retrieval-Augmented Generation (RAG) to maintain factual accuracy and relevance in a dynamic information environment. However, this has led to uncertainty in the information the LLM bases its answer on, as it may choose between parametric memory and retrieved sources. When models rely on memorized information instead of dynamically retrieving information, they may inadvertently propagate outdated or incorrect information, causing serious legal and ethical risks and undermining trust and reliability in AI systems \citep{huang2023survey}.
% The ability to strike a balance between generalization and specialization in AI systems is therefore crucial for ensuring the safe, reliable use of these technologies in real-world applications.

Despite these concerns, the way that Vision-Language models (VLMs) memorize and retrieve information, particularly in complex multimodal tasks, remains under-explored. Current research often focuses on either the general capabilities of large language models (LLMs) or the specialized retrieval mechanisms in retrieval augmented generation systems (RAG) \citep{incontext_rag,chen_murag_2022,liu_universal_2023}. Particularly in the context of multimodal retrieval and multihop reasoning, few studies analyze the tradeoff between finetuning for specialized tasks and zero-shot prompting for general-purpose vision-language capabilities. A lack of consensus on how to approach this tradeoff motivates the development of measures to quantify reliance on parametric memory, as well as metrics for quantifying the potential performance impact of extending LLMs with RAG systems.

To address this gap, we investigate how multimodal QA models balance accuracy with memorization on the WebQA benchmark. We compare finetuned multimodal systems against zero-shot VLMs, analyzing how retrieval performance influences QA accuracy. In particular, we focus on cases where retrieval fails, allowing us to measure reliance on parametric memory through two proposed metrics---the \ppr (\PPR) which quantifies how much model accuracy is influenced by retrieval quality, contrasting performance in best-case versus worst-case retrieval scenarios, and the \ucr (\UCR) which measures how often correct QA responses are generated when the retriever fails, providing a proxy for memorization.

To enable this analysis, we make several methodological contributions. For the finetuned QA models, we investigate Vision-Transformer (ViT) architectures, which allow for multihop reasoning over multiple sources. To investigate the impact of retrieval performance on trained LMs, we propose a variable-input Fusion-in-Decoder (FiD) model \cite{tanaka_slidevqa_2023, nlvr2}, building upon the VoLTA architecture \citep{pramanick_volta_2023}. For the zero-shot case, we build upon previous research on In-Context Retrieval \citep{incontext_rag} by demonstrating that LLMs such as GPT-4o are capable of performing the final ranking step of the retrieval process. In doing so, we find that GPT-4o, a general-purpose LLM, achieves SoTA performance on the WebQA task, outperforming existing finetuned RAG models by a significant margin (7\% higher accuracy). 

Crucially, our results reveal that while retrieval-augmented models reduce memorization, the training paradigm plays an important role. Finetuned models exhibit higher reliance on parametric memory, whereas zero-shot RAG approaches have lower memorization scores at the cost of accuracy. This suggests that while retrieval modules may mitigate the risks associated with outdated or incorrect information, SoTA performance requires that they be coupled with specialized QA models. Our memorization measures contribute to the development of transparent and reliable AI systems, particularly in applications where the sourcing of up-to-date, factual information is critical.



% We investigate the impact of question complexity on the ability of these models to integrate multiple data sources—such as images, text, and external retrievers—and produce coherent and accurate answers. We also explore whether in-context retrieval can be a viable alternative to traditional retrieval-augmented systems, offering a more streamlined approach to multimodal QA.

% To achieve this, we first compare zero-shot prompting multimodal LLMs with finetuned multimodal systems. We evaluate both types of models on the WebQA benchmark, a dataset designed for complex question answering that requires reasoning across both image and text sources. For the finetuned models, we use a Fusion-in-Decoder (FiD) architecture, which allows for multihop reasoning over multiple sources. Additionally, we introduce the concept of In-Context Retrieval Language Modeling (RLM), where the LLM itself performs retrieval tasks without the need for external retrievers. This method builds upon existing research in in-context learning  and aims to explore the viability of LLMs retrieving relevant sources and generating accurate answers directly from their context window.

% In order to investigate source utilization in finetuned multimodal models and LLMs, three lines of inquiry are established; 
% \begin{itemize}
%     \item Study 1: retrieval vs QA performance on webQA (motivating example, does QA answer correctly even with incorrect sources?)
%     \item Study 2: performance on adversarial examples where parametric knowledge would be incorrect by design
%     \item Study 3: improving performance on adversarial examples by fine-tuning (i.e model robustness)
% \end{itemize}

% Note, there is one weakness in this plan which is tying in the work we've already done. 
% If we added something from adversarial generation to the retrieval experiment (like a combination of study 1 + 3) it would be complete. So for instance we could try fine-tuning the retriever with adversarial examples (and not just the QA model)

% \begin{figure}
%     \centering
%     \includegraphics[width=0.95\linewidth]{figures/segmentation/webqa_segment_infill.png}
%     \caption{Example of the segmentation substitution pipeline from the WebQA task.}
%     % d5c76d760dba11ecb1e81171463288e9
%     \label{fig:seg_sub_pipeline}
% \end{figure}



% Retrieval augmented generation (RAG) with zero-shot prompting and fine-tuning Large Language Models (LLMs) have become the go-to methods for tasks relying on information retrieval and text generation. In many cases the LLMs parametric memory can sufficiently generalize to answer questions without being provided with retrieval mechanisms for out-of-domain knowledge. However, LLMs often hallucinate and provide wrong information in certain scenarios. This problem is amplified even further on open-domain Question Answering (QA) tasks involving multiple modalities. Grounded text generation using retrieved sources \citep{lewis2021retrievalaugmented} has been extensively studied for text-to-text QA tasks, but its application in multimodal settings has not been studied as much.


% Multimodal reasoning and question answering have gained prominence in recent research endeavors, with an increasing emphasis on handling various forms of data, particularly text and images. In this study, we address a specific gap in the existing literature by focusing on the development of a versatile multihop model capable of accommodating varying numbers of input images.

% Our motivation for this research lies in the growing complexity of answering questions using information on the web, where the challenge of navigating the open-domain setting is further complicated by the presence of multiple modalities and sometimes requires reasoning over multiple sources. WebQA is an ideal dataset on which to compare performance of finetuned RAG systems against general purpose LLMs; it is multimodal, with correct answers requiring reasoning over image and text sources. It is multihop, requiring a complex reasoning process over multiple sources. Finally, WebQA questions from different categories can be broken down into subdomains to analyze performance over domains of varying cardinality.

% Motivated by the real-world challenges of building retrieval and question answering (QA) systems, we design and finetune a closed domain, multimodal, multihop QA model, that is capable of reasoning over a varying number of sources taken as input from an external retriever module. This research contributes to the relatively underexplored domain of multihop reasoning across various input sources and modalities. Our goal is to explore the challenges posed by these scenarios and develop strategies that enable QA models to retrieve relevant information, conduct logical or numerical reasoning across diverse modalities, and generate coherent responses in natural language. To our knowledge, this is the first application of the Fusion-in-Decoder (FiD) architecture \cite{tanaka_slidevqa_2023, nlvr2} that is shown to work with a variable number of inputs, enabling multi-hop reasoning over sources.

% In-Context Learning refers to the ability of LLMs to perform any task by simply providing examples in the input prompt \citep{dong2022survey,min2022rethinking}. Inspired by this research, we propose a method to use the LLM itself as a multimodal retriever, potentially eschewing the requirement of a distinct retrieval module, thereby allowing the design of simpler retrieval-augmented QA systems. We dub this method In-Context Retrieval Language Modeling (RLM). To the best of the authors knowledge, In-Content RLM is disparate from other retrieval augmented approaches which utilize external retrieval modules \citep{incontext_rag,chen_murag_2022,liu_universal_2023}. Despite being a natural extension of In-Context learning, In-Context RLM has not yet been studied empirically.

% To expand on our contribution of In-Context Retrieval, this stems from the well-researched in-context learning of LLMs. In-context learning is the ability of a model to perform any task given a sufficient context window \citep{dong2022survey,min2022rethinking}. Such tasks could include retrieval and ranking, but typically, the go-to solution for tasks requiring retrieval has been RAG. To the best of the authors knowledge, In-Context Retrieval is distinct from In-Context Retrieval Augmented Language Modelling (RALM), and despite being a natural extension of In-Context learning, In-Context Retrieval has not yet been shown empirically.

% Finally, we explore the tradeoff between using zero-shot prompting LLMs and the fine-tuning approach. While we find that, overall, GPT-4o obtains SoTA performance on the WebQA task, outperforming the accuracy of existing finetuned RAG approaches by 7\%, finetuned approaches still perform better on more restricted subdomains\footnote{``In-Context RLM" @ \url{https://eval.ai/web/challenges/challenge-page/1255/leaderboard/3168}}. Finally, we validate that GPT-4o is relying on retrieval abilities to solve the task; we find that GPT-4o is capable of retrieving relevant sources in the presence of distractors and furthermore, when GPT-4o fails to retrieve correct sources, it answers incorrectly 75\% of the time, meaning that it is not relying on parametric memory for this task.

% \paragraph{Contributions}
% Based on our experimentation and analysis on the WebQA benchmark, we make the following contributions:
% \begin{itemize}
%     \item Propose a new architecture for multimodal multihop QA that takes variable number of input sources inspired by the Fusion-in-Decoder method.
%     \item Comparison of general purpose LLMs vs specialized models on the WebQA benchmark.
%     \item Observation of In-Context Multimodal Retrieval abilities of GPT-4o and that it does not rely on parametric memory for multimodal QA.
%     \item Analysis of relationship between retrieval and QA task performance.
%     \item Analysis of task and query complexity on the performance of retrieval and QA tasks.
% \end{itemize}
















% Throughout this paper, we will present our methodology, experiments, and findings, emphasizing our approach to multihop reasoning over varying numbers of input images. We believe that our work contributes to a deeper understanding of multimodal reasoning and has the potential to enhance the capabilities of question-answering systems in the intricate, multimodal landscape of web-based information.
\section{Background and Motivation}
\label{sec:background}

We introduce the background on serverless workload serving and motivate the use of runtime resource adaptation to address resource inefficiency in existing serverless platforms.

\subsection{Resource Inefficiency with Early Binding}
% In current serverless platforms, developers are required to specify immutable sizes for their deployed functions.
% Then, providers consider functions' runtime workloads  (e.g., concurrency)  and resource usage to scale out/in their instances.
% Moreover, due to high runtime variability, functions must size their functions for worst-case scenarios.
% This, however, incurs considerable resource inefficiency.
Current serverless workflow platforms (e.g., AWS Step Functions~\cite{aws-step-function} and Azure Durable Functions~\cite{azure-durable-function}) offer the opportunity for developers to build various applications with advanced logic like chaining, branching, and parallel execution.
These applications can be defined by JSON-based structured languages (e.g., Amazon States Language) or other programming languages.
Meanwhile, developers require to specify resource configurations, including memory size, CPU cores, and scaling options, for individual functions---an early-binding approach.
The serverless platform is responsible for monitoring the workload intensity and resource usage at runtime and scaling out/in function instances accordingly.
To account for potential runtime variability, developers must size the functions in their application workflow accounting for the worst case in order to provide SLO guarantees over the end-to-end delay of request processing, e.g., the 99th percentile (P99) of the end-to-end delay must be within a given target. 
After deployment, the function sizes become immutable. The worst case is not representative and over-shoots most of the time, leading to resource inefficiency. 


To verify this claim, we conduct a data-driven analysis with a dataset from Microsoft Azure Functions~\cite{azure-dataset} to explicitly demonstrate the resource inefficiency issue. % , deriving from the worst-case based early bind.
To quantify the inefficiency, we define a metric called \emph{slack}---the margin between the actual execution time and the SLO, which is calculated as $1-l/T$ with $l$ and $T$ representing end-to-end latency and SLO, respectively.
Under certain SLO defined with P99 latency as done by existing works (e.g., \cite{osdi22-orion,mac22-wisefuse}),  we can see from Figure \ref{fig:bg:slack} that more than 60\% function invocations have slacks over 60\%.
Particularly, we analyze slacks of the top 100 most popular functions, whose invocations account for 81.6\% of the total function invocations. % (depicted in Figure~\ref{fig:bg:popular_func}) of overall invocations.
The result shows that only 20\% of the invocations of the popular functions (blue line in Figure~\ref{fig:bg:slack}) have slacks less than 40\%.
This means the majority of requests are processed faster than necessary.
Notably, in DAG-based workloads (i.e., Azure Durable Functions), the resource inefficiency further deteriorates wherein the ratio between the 95th percentile and 50th percentile is by up to three times \cite{mac22-wisefuse}.

% \begin{figure}[t!]
% \centering
% \includegraphics[width=0.25\textwidth]{./figure/motivation/Average_P99_cdf_top=100.pdf}
% \vspace{-0.3cm}
% \caption{Sufficient function slacks in production traces.}
% \label{fig:bg:slack}
% \end{figure}

\subsection{Runtime Dynamics}
\label{sec:bg:worst-case}

The resource inefficiency caused by the large slack can be mainly attributed to the over-provisioning of resources by the developer. This is to ensure that the SLO is guaranteed even in the worst case (i.e., P99). However, normal cases deviate from the worst case significantly due to runtime dynamics. 
In particular, we observe that functions face two major dynamic factors at runtime: varying working sets and inevitable performance interference. These two factors contribute significantly to the variance of the function execution time. 
% Functions face two remarkably dynamic factors at runtime: working sets and performance interference, which lead to considerable variance of execution latency.

\begin{figure*}[!t]
	\centering
	\subfloat[]{
		\includegraphics[width=0.24\textwidth]{./figure/motivation/Average_P99_cdf_top=100.pdf}
		\label{fig:bg:slack}
	}
	\hspace{8mm}
	\subfloat[]{
		\includegraphics[width=0.25\textwidth]{./figure/motivation/function-latency-ml-analyze-varying-worksets.pdf}
		\label{fig:bg:ml-func-latency}
	}
	\hspace{8mm}
	\subfloat[]{
	\includegraphics[width=0.28\textwidth]{./figure/motivation/coresident-perf.pdf}   
	\label{fig:bg:perf-inteference}
	}
	%\vspace{-0.1cm}
	\caption{(a) slacks of function invocations in production traces, (b) function latency variance caused by varying input worksets for functions object detection (OD), question answering (QA), and and text-to-speech (TS), respectively,
 (c) performance interference attributed to co-location of homogeneous function with different dominant resource demands.}
 %\vspace{-0.4cm}
\end{figure*}

%'ml-analyze':{'text-to-speech': 'text-to-speech', 'question-answer': 'question answer',
%                      'object-detection': 'object detection'
\textbf{\textit{Varying working sets.}} 
The working set, i.e., input data like videos, audios, and texts, can have varying sizes.
Taking Microsoft Azure Function Blobs (storage service) as an example, their data size difference can be as high as nine orders of magnitude~\cite{azure-function-blob}.
Such a large difference results in substantial variance of the execution time even for the same function~\cite{socc21-faast,eurosys21-ofc}.
Specifically, we measure the execution time of three functions under different working sets (detailed in \S\ref{exp:setup}).
Figure~\ref{fig:bg:ml-func-latency} illustrates the results, where we can observe a variance of up to 3.8 times in function execution caused by varying working set sizes.

% \begin{figure}[t!]
% \centering
% \includegraphics[width=0.25\textwidth]{././figure/motivation/function-latency-ml-analyze-varying-worksets.pdf}
% \vspace{-0.3cm}
% \caption{Function latency variance caused by varying input worksets}
% \label{fig:bg:ml-func-latency}
% \end{figure}	

\textbf{\textit{Performance interference.}}
% On the other hand, function deployment, which decides when and where to deploy functions, is completely undertaken by providers.
For simplicity and security, commercial serverless platforms, such as Alibaba Function Compute, Microsoft Azure, and AWS Lambda, exclusively deploy function instances belonging to the same tenant, or even belonging to the same function, in the same virtual machine~\cite{socc22-owl,atc18-peek-bench}.
For example, the empirical study in~\cite{socc22-owl} shows that in Alibaba Function Compute 65\% of the virtual machines exclusively deploy instances of the same function.
This co-location of homogeneous function instances, however, can incur severe resource contention on the same resource dimensions, particularly for network bandwidth and memory bandwidth of virtual machines~\cite{sc21-gsight,micro19-faaSprofiler,socc22-owl,atc18-peek-bench}.
To verify this observation, we use a virtual machine to run a function increasing the number of co-located instances from one to six while measuring the execution time of four different functions with resource dominance on different dimensions namely computing, I/O, network, and memory, respectively (detailed in \S\ref{exp:setup}). 
As shown in Figure~\ref{fig:bg:perf-inteference}, the co-location of homogeneous functions leads to substantial resource contention and performance interference, prolonging the function execution time up to 8.1 times. The performance interference is often hard to model and predict.

% this co-residency results in substantial increase of execution latency by up to 8.1 times,leading to considerable variance in function execution time.
% when compared with that with concurrency as one.

%for CPU-, IO-, network- and memory-intensive functions as the concurrency rises from one to six.
%Figure shows that significant performance interference can be observed, . 
%compared with the inclusive deployment (concurrency as one), 
% this exclusive deployment (gray bar) results in substantial increase of execution latency by up to 8.1$\times$ for CPU-, IO-, network- and memory-intensive functions as the concurrency rises from one to six.

% this exclusive deployment (gray bar) results in substantial increase of execution latency by up to 8.1$\times$ for CPU-, IO-, network- and memory-intensive functions as the concurrency rises from one to six.
% As depicted in Figure~\ref{fig:bg:concurrent_latency}, with the concurrency rising  from one to six,  the exclusive deployment results in substantial increase of execution latency by up to 8.1$\times$.
% This significantly magnifies execution latency variance.

% \begin{figure}[t!]
% \centering
% \includegraphics[width=0.25\textwidth]{./figure/motivation/coresident-perf.pdf}
% \vspace{-0.3cm}
% \caption{Performance interference attributed to co-residency of homogeneous function.}
% \label{fig:bg:perf-inteference}
% \end{figure}




\subsection{Runtime Resource Adaptation}
\label{sec:bg:adaptive-allocation}
To tackle the aforementioned resource inefficiency issue, we can adopt a late-binding approach through \emph{runtime resource adaptation}, which resizes functions on the fly based on runtime information (e.g., function slacks), achieving higher resource efficiency without violating SLO. For example, given a workflow as a chain of functions, the resource allocation of the downstream functions can be adjusted when the first function finishes execution. This way, the slack from the first function can be exploited to optimize resource efficiency. 

The idea sounds straightforward and has been considered in some existing works \cite{infocom22-stepconf,middleware20-fifer,socc21-llama,socc21-kraken,middleware20-xanadu}.
However, most of these works make an unrealistic assumption that either the developer performs the adaptation decision with access to runtime information or the serverless platform provider performs the adaptation with domain knowledge of the application workflow. These assumptions render these solutions impractical to deploy in real-world serverless systems. The information barrier between the developer and the provider calls for a new solution. 

We identify the following challenges and opportunities for a full-fledged design for runtime resource adaptation. 

\textbf{\textit{Skewed function execution time distribution.}} 
Resource allocation for a serverless workflow is typically done by leveraging performance profiles of all the functions in the workflow. 
During the offline profiling, the execution time distribution for each function is first obtained by running the function with a variety of sample inputs under different resource conditions. Then, given a time budget, existing approaches typically use P99 of the function execution time as a target and calculate the corresponding resource demands. However, due to the high runtime variability, the distribution of the function execution time is highly skewed where the difference between P50 and P99 can be as high as 100 times~\cite{socc23-huawei-cloud}. This means that if only the function execution time at a single percentile (P50 or P99) is used for resource allocation, there will be significant resource under-provisioning and over-provisioning for most requests at runtime. To address this issue, our idea is to allow for the exploration of the function execution time at diverse percentiles during resource allocation. 


% It is a prerequisite to profile execution latency for adaptive resource allocation.  
% As aforementioned, owing to a variety of unexpected runtime dynamics,  execution latency demonstrates skewed distributions, by up to 100$\times$ between 99\% percentile and 50\% percentile on Huawei cloud serverless~\cite{socc23-huawei-cloud} .
% This makes the current a single statistic (e.g., mean) or 99\% percentile distribution based profiling suffer significant under- and over-estimation.
% To fix this issue, our insight is to \textit{introduce more diverse percentiles to profile execution latency}. 

\textbf{\textit{Dependencies of adaptation decisions.}}
As the function execution progresses, a sub-workflow will be generated by removing the finished function(s) from the workflow. Within each sub-workflow, the resource adaptation decisions for remaining functions are dependent on each other due to the constraint imposed by the end-to-end latency SLO. For example, under-provisioning a function will result in a reduction of the time budget for executing its downstream functions, thus calling for more resources for these downstream functions to avoid SLO violations. Meanwhile, the selection of the percentile for the execution time of each function dictates resource-latency tradeoff for that function. For example, a higher percentile means that more resources will be allocated to ensure that more requests processed by the function will finish within the given time budget. On the contrary, a lower percentile means that more requests will risk SLO violation, but at the benefits of reduced resource consumption. To address such complex dependencies, we propose the following ideas: (1) We introduce two metrics (i.e., the timeout metric and the resilience metric detailed in \S\ref{sec:profilier}) to balance the resource adaptation decisions of the head function of the current sub-workflow and those of the remaining downstream functions. These metrics help us connect the decision making across sub-workflows and avoids sub-optimal adaptation decisions in each sub-workflow. 
(2) We explore lower percentiles for the head function and a high percentile (i.e., P99) for other functions in each sub-workflow. Using lower percentiles maximizes the opportunity for resource optimization since any over-time execution of the head function can later be compensated by resource adaptation in the next round. The high percentile ensures that the resource adaptation is not too radical to cause SLO violations. 

% Each workflow generates multiple sub-workflows as the execution moves forwards. 
% Within sub-workflows, the provisioning is inter-corrected.
% For instance, under-provisioning upstream functions may directly shrink the time budget for downstream functions, which dictates more resources required by the latter against (sub-) SLO violation. 
% This makes sub-workflows generally adopted as the basic unit to make adaptation decisions~\cite{socc21-llama,rtas22-fa2}. 
%  Moreover,  due to the high variance of execution performance, runtime adaptation requires to carry out function by function, i.e.,  discrete adaptation.
%  This, however, can easily lead to a sub-optimal (analyzed in~\S~\ref{sec:synthesizer:generate}).
% Our insight is to \emph{introduce a metric (i.e., resilience detailed in \S~\ref{sec:profilier}) to quantify the inter-correlation as well as a heuristic design (i.e., heavier head explained in \S~\ref{sec:synthesizer:generate})  to calibrate the sub-optimal,  such that resource adaptation can explore higher resource efficiency without SLO guarantee}.

% In particular, latency percentiles (introduced by the profiling)  involves resource adaptation as a new knob.
% Specifically, higher percentile earns  stronger guarantees in SLOs but may be highly prone to resource over-allocation because of its latency over-estimation, impairing resource efficiency.
% In contrast, decreasing percentiles offers the opportunity to explore higher resource efficiency, but suffers the risk of timeout, i.e., execution latency beyond specified time budget, and  may thus incur  SLO violations.
% Here, our insight is to \emph{moderately explore percentiles (detailed in~\S~\ref{sec:synthesizer:generate}), where head functions of  (sub-)workflows can explore lower percentiles because this creates the opportunity to reap higher resource efficiency while possible timeout can be recovered by subsequent functions' re-adaptive allocation.
% On the other head, non head functions maintain percentiles as 99\%}.
% This can well keep the trade-off between opportunities of exploring higher resource efficiency and risks of SLO violations. 
% Additionally, it effectively shrinks the searching space, benefiting the adaptation with higher time-efficiency.


\textbf{\textit{Tight resource adaptation window.}}
Runtime resource adaptation requires to calculate a new resource allocation decision for the remaining sub-workflow immediately when a function finishes execution. Since serverless functions are typically short-lived (less than 1s on average)~\cite{atc18-peek-bench,socc22-owl,atc20-serverless-in-the-wild,socc23-huawei-cloud}, the window for resource adaptation is quite tight. Assuming the serverless platform will perform the runtime adaptation on behalf of the developer since the platform has access to full runtime information, the resource adaptation decision making should be fast without involving complex calculations and logic or exploring a large space. As discussed before, the serverless platform provider does not have domain knowledge of the serverless workflow. Hence, the developer must pass the necessary information to the serverless platform for runtime adaptation decision making. Our idea is to let the developer synthesize critical hints containing resource allocation rules and options, which the serverless platform provider utilizes to perform runtime resource adaptation. The hints should be highly condensed so the serverless platform can make adaptation decisions quickly enough. 


% Apart from highly varying execution performance, serverless functions are also short-living (less than 1s on average)~\cite{atc18-peek-bench,socc22-owl,atc20-serverless-in-the-wild,socc23-huawei-cloud}, so is the window for adaptive allocation. 
% This variance and volatility calls for a well-preparation of hints for all possible runtime situations while promising them compact and straightforward enough for providers to easily take action.

% Here, our insight is to \emph{holistically synthesize hints in an offline manner, and then utilize the discreteness of adaptive allocation in both decision-making and decision-executing (detailed in~\S~\ref{sec:synthesizer:condense}) to fully condense the hints.
% Finally, hints are warped into a simple and compact table.
% Base on that, providers can accomplish the runtime adaption promptly and properly}.

To demonstrate the potential of runtime resource adaptation incorporating all the above ideas, we take a real-world serverless workflow (explained in \S\ref{exp:setup}) as an example, and evaluate its end-to-end latency (denoted by E2E) and resource consumption (CPU cores).
As illustrated in Figure~\ref{fig:bg:size}, the late-binding (blue triangle) reduces the resource consumption by up to 42.2\% compared with existing early-binding solutions (orange circle), while ensuring SLO guarantees. This highlights the significant gains from runtime resource adaptation. 


\begin{figure}[t!]
\centering
\includegraphics[width=0.45\textwidth]{./figure/motivation/size_early_bind_vs_ours.pdf}
%\vspace{-0.1cm}
\caption{Performance comparison between early-binding (left)~\cite{eurosys19-grandslam} and late-binding (runtime resource adaptation), where the CPU consumption (right) is normalized by the optimal obtained with exhaustive search.} 
%\vspace{-0.3cm}
\label{fig:bg:size}
\end{figure}

   
	







\begin{figure*}[t!]
    \centering
    \includegraphics[width=\linewidth]{figures/interface.png}
    \caption{\pluto's user interface. The key components include a data panel (A), chart editor (B), chart title (C), main chart canvas (D), and a chart description (E).
    Here, the user has manually entered a description and clicked the {\small{\faIcon[regular]{lightbulb}}} \textbf{Suggest} button to get ideas on improving the chart and text for communication purposes.
    This results in the system suggesting a title and adding a highlight annotation for \annotation{\textit{Single Family}} homes, while also generating a chart design recommendation (F) and a set of description editing recommendations (G).}
    \Description[Pluto's user interface.]{From left to right, the system contains of: 1) a data pane showing the available attributes, 2) a pane to specify encodings, annotations, and filters to create a chart, 3) the chart along with text boxes for the title and description above and below it, respectively, and 4) a right panel where the system presents recommendations to edit the chart and description.}
    \label{fig:interface}
\end{figure*}

\section{Design}

The central idea of our work is exploring a unified visualization system for authoring well-integrated charts and text.
Designing such a system, however, requires considering several open questions about the type of assistance the system should provide, and when and how system suggestions should be surfaced.

In exploring the chart-and-text authoring experience, several questions arise that warrant exploration. We must first discern which chart elements, ranging from axis labels and ticks to titles, descriptions, and annotations, necessitate the most authoring support. Additionally, we would need to determine the appropriate level of system assistance—whether to generate entire descriptions, fill in partially written text, or refine user-authored drafts—and how this assistance might vary with different types of text. Given the non-mutual exclusivity of text types, such as descriptions influencing titles, we must also consider if and how the system should sequence its suggestions. The timing of these suggestions is another critical factor: should they be offered immediately following the creation of a chart or once the user has initiated the authoring process? Deciding whether these suggestions should be proactive or solicited on-demand, along with the specific user actions that should trigger them, is also a direction worth considering. Lastly, we must explore the potential for a synergistic relationship between the chart and text, i.e., how interactions with each can be leveraged to enhance the other and what mechanisms would facilitate this interplay.

\subsection{Design Goals}
\label{sec:design-goals}

With the aforementioned considerations in mind, we iteratively compiled a list of design goals to guide our system's development.
These goals were informed by prior research and systems focusing on authoring text for visualizations (e.g.,~\cite{kim2023emphasischecker,latif2021kori,liu2023autotitle,he2024leveraging,tang2023vistext,singh2024figura11y}), general principles of mixed-initiative user interfaces~\cite{horvitz1999}, as well as formative interviews with two experts on authoring text and charts for data-driven communication.

The two experts were a practitioner and a researcher who both author and critique text for data visualizations. 
Furthermore, they also regularly interact with end-users in the creation of text and charts.
\new{Both experts voluntarily participated in the interviews and were not financially compensated.
We interviewed each expert twice over a span of three weeks.
Each session lasted for 30-60 minutes.
}
\new{During the first set of interviews, we asked the experts about the key challenges users generally encounter during the authoring of text with charts. Additionally, to guide our design and identify critical features, we also presented an early version of our prototype with a basic set of functionality including generating titles and descriptions for a chart and supporting interactive highlighting of chart elements based on the text.
Based on the initial feedback, we incorporated additional types of recommendations and refined the system design before the second meeting where the experts provided feedback on the overall utility and perceived usability of the different features.
We subsequently developed the final version of \pluto~by iterating on this feedback and leveraging findings from related work on text+chart authoring systems~(e.g.,~\cite{latif2021kori,kim2023emphasischecker,sultanum2023datatales,lin2023inksight,choi2022intentable}).
}

\vspace{.5em}
\noindent\textbf{DG1. Leverage the textual narrative to guide chart design.}
In line with prior work~\cite{stokes2022striking,ottley2019curious,kim2021towards}, the experts also stressed that the text should not only convey the right levels of information but also be well-aligned with the chart for a smooth reading experience.
As text has an inherent narrative flow, we noted that the system should \new{incorporate techniques from prior work on updating chart specification based on narrative text~\cite{wang2022towards,chen2022crossdata,shen2024data} to} inspect the flow of information in the text and leverage it to augment the chart.
This augmentation could involve making data transformation changes (e.g., sorting) or adding annotations to highlight portions of the chart that are emphasized in the text.

\vspace{.5em}
\noindent\textbf{DG2. Support direct manipulation interactions with the chart for text generation.}
Visualizations make it easy to perceive trends in the data and identify points of interest.
Phrasing something visually interesting as text can be challenging, however. For instance, one of the experts noted, ``\textit{sometimes I notice something potentially interesting on the chart and want some quick text to verify what I'm seeing and get ideas for how to talk about it.}'' Given this multimodal nature of charts and text, \new{in line with prior chart-and-text authoring systems (e.g.,~\cite{chen2022crossdata,lin2023inksight})}, we noted that the system should allow leveraging direct interactions with the chart (e.g., brushing a region or mark selection) to generate corresponding text.

\vspace{.5em}
\noindent\textbf{DG3. Provide varying levels of assistance for text authoring.}
Both experts noted that users need different levels of assistance when writing text depending on their goals and experience level.
For instance, novice and intermediate users may need auto-generated text to jump-start their authoring process, whereas domain experts may benefit from fine-tuning suggestions to improve manually written text.
Combining this comment with prior work on text-chart authoring~\cite{stokes2022striking,kim2023emphasischecker}, we noted that the system should not only recommend text for chart authors to add but also recommend editing actions (e.g., reordering sentence) or flag potential factual errors in the text (e.g., incorrectly stated trends).

\vspace{.5em}
\noindent\textbf{DG4. Incorporate context-sensitive recommendations near their relevant targets to facilitate easier interpretation.}
System recommendations in the context of a unified text and chart authoring process could apply to different targets (e.g., chart, title, or description) and focus on either adding new content or editing existing content.
Interpreting this broad set of recommendations can be challenging, however.
For instance, in our early prototypes, we explored listing all recommendations in a side panel, but both experts noted that this was overwhelming and distracted them from the main content.
Iterating on the designs, we noted that for improved usability, the system recommendations should be placed close to the targets they apply to and should also be presented differently (e.g., in-place overlays vs. suggested actions) based on the type of recommendation.

\vspace{.5em}
\noindent\textbf{DG5. Recommendations should be unobtrusive during targeted authoring.}
While the recommendations are designed to help craft cohesive text and charts, there may be instances where the chart authors have clear authoring goals in mind.
In such targeted authoring scenarios, the recommendations should not interfere with the users' flow but still be available on demand if users want ideas for text content or chart design.
Authors should have full control over the final content, however, and should be able to edit/update any suggestions made by the system.
\newline

\noindent{}Note that these goals are not exhaustive or mutually exclusive, nor are they meant to be prescriptive.
For instance, we primarily focus on content suggestions and do not deeply consider operations like formatting as part of the recommendation space. Rather, \textbf{DG1}-\textbf{DG5} are only meant to be an initial set of goals to help ground our design and enable us to develop and test a viable prototype.

\begin{figure*}[t]
    \centering
    \includegraphics[width=\linewidth]{figures/local_response_prompting}
    \caption{The text suggestions in the local response widget are flexible:  \textit{(A)} Users get suggestions without any input.  \textit{(B)} \revision{Suggestions} can be adapted and refined by entering text, for example keywords or a draft snippet. In all cases, suggestions are generated with an LLM based on the text of the incoming email and all local responses that the user has entered so far, \revision{even if responses have been} added to later parts of the email \revision{first}. \revision{In \textit{(C)}, for example, the suggested title of the idea pitch is generated based on the information about the project that the user has already entered in local responses below.} Note that suggestions are paginated, with three pages of two suggestions each.}
    \Description{
    This figure shows screenshots of our user interface displaying an incoming email. The figure is divided into three sections:
    A) Left Section: Email-driven Suggestions without User Input
    The selected sentence in the incoming email reads: "Please feel free to tell me any ideas what we could get her!"
    Below the email, there is an empty text field for optional user input.
    Two AI-generated responses are shown beneath the text field: one suggests a piece of jewellery as a gift, while the other does not offer any ideas.
    B) Centre Section: Prompt-driven Suggestions with User Input
    The same sentence from the email is selected: "Please feel free to tell me any ideas what we could get her!"
    This time, the keywords "balloon ride" are entered into the text field below.
    As a result, the AI-generated suggestions include the idea of a balloon ride in both proposed texts.
    C) Right Section: AI Suggestions Respecting Existing Responses
    One sentence from the incoming email is selected, and the text field below remains empty.
    The AI-generated responses incorporate information from existing responses elsewhere in the email.
    Additionally, all AI suggestions are paginated, with three pages of two suggestions each, as indicated by arrows next to the suggestions.}
    \label{fig:local_response_prompting}
\end{figure*}



\section{Implementation}
\label{sec:implementation}
We implemented a frontend and backend, which preprocessed emails, logged user data, and generated responses. 

\subsection{Frontend}
We implemented our web app with the React\footnote{\url{https://react.dev/}} framework.

\subsubsection{Display of the Incoming Email}
This view matches standard mobile email UIs: It includes the sender's name and picture, the email subject, and the main text body (\cref{fig:teaser} left). 
The user can select sentences in the incoming email by tapping on them (cf. design goal \ref{dg:humandecides}: \revision{Human decides, AI supports}; and goal \ref{dg:microtasking}: \revision{Support mobile microtasking}). This opens the local response widget (\cref{sec:impl_local_response_widget}).
The ``Finalize Reply'' button at the bottom of the UI switches to the next screen (\cref{fig:teaser} centre), which we describe in \cref{sec:impl_finalize}.
In accordance with design goal \ref{dg:workflows} \revision{(Support diverse workflows -- with and without AI)}, no interaction with any sentence or AI feature is required before proceeding to this next screen.



\subsubsection{Local Response Widget}\label{sec:impl_local_response_widget}

This UI widget is inserted into the email text below the user's selected sentence. It comprises of a text field (\cref{fig:teaser} C) and a paginated card view that shows text suggestions (\cref{fig:teaser} D). In the text field, users can enter both manual responses or prompts to refine these suggestions (\cref{fig:local_response_prompting}). 

Concretely, the widget offers six suggestions (2 positive, 2 negative, 2 neutral), showing two at once. The system aims to show one positive and one negative response on the first page, if possible. \revision{This was motivated by findings on positivity bias in AI-generated communication text~\cite{Mieczkowski2021} and to increase the chance of offering a response option fitting to the user's intent (cf.~\cite{Kannan2016smartreply}).} \revision{We realised this by prompting the LLM to do so (see \cref{sec:appendix_sentence_without_input_prompt}). Concretely, the variable ``attribute'' in the prompt template was replaced with \textit{accepting}, \textit{declining}, and \textit{neutral} to generate varying suggestions. In our tests, we observed that this simple prompting approach worked well and that it did not negatively impact generated suggestions in cases where these attributes do not apply (e.g. our ``cat'' example in \cref{fig:teaser}D).} %
Users can navigate through suggestions using the adjacent arrow buttons. They can accept a suggestion by tapping on it.

The widget has two states -- open and collapsed (\cref{fig:teaser} A, B): 
It is collapsed by tapping the currently selected sentence again, by selecting a different sentence, by accepting a suggestion, or by clicking on the check mark in the top right corner. When the text field is empty, the check mark transforms into a trash icon to delete the local reply. Multiple widgets can be in the collapsed state throughout the email but only one widget at a time can be open and in focus. %
A widget's text is shown in the collapsed state. This allows users to keep track of all their local replies so far. Tapping on a collapsed widget opens it again for further editing. 






\subsubsection{Finalising the Reply}\label{sec:impl_finalize}
This view  (\cref{fig:teaser} centre) shows the current state of the reply after the local response step. That is, it displays any text entered in response to individually selected sentences together in a single text field.

Users can manually adjust this text and/or tap the ``Improve'' button to request the AI to enhance the email. 
This \imppass{} feature is realised with a prompt \revision{(see \cref{subsec:appendix_improve_email_prompt})} to the underlying LLM to correct spelling and grammar, refine wording, and add missing salutations or regards while adhering to both the incoming email's content and the existing reply text. 

If no text is entered first, the ``Improve'' button acts as message-level support, generating a reply based on the incoming email's text and the current input on this screen. For example, a user could skip the local response and enter a prompt here, effectively realising a message generation workflow similar to the industry default (\cref{sec:related_work_current_products}). This flexibility contributes to our design goal \ref{dg:workflows} \revision{(Support diverse workflows – with and without AI).}

When the user is satisfied with their reply, the email can be sent by tapping the ``Send Email'' button at the bottom of this screen.


\subsubsection{Improved Email Pop-up}\label{sec:impl_imppass}
The \imppass{} feature does not change the user's text directly, in line with our design goal \ref{dg:control} \revision{(The user stays in control).}
Instead, the new text is shown in a pop-up view with formatting familiar from ``track changes'' in text editors (\cref{fig:teaser} right). 
Users can approve these changes, which updates the text, or discard them (cf. design goal \ref{dg:humandecides}: \revision{Human decides, AI supports}.) Further editing after acceptance and/or requesting improvements repeatedly is possible. 


\subsection{Backend}
Our prototype's backend has three purposes: 
(1) It \textit{hosts the web app} on a Next.js server. 
(2) It provides \textit{email preprocessing}, which handles tasks such as sentence-splitting and making API calls to the LLM. 
(3) It \textit{hosts the LLM}. 

We experimented with several models and APIs and discussed factors such as latency, stability of service, and subjective response quality in meetings with all authors. Based on this, we used the Llama 3 8B Instruct \cite{llama3modelcard} model for the main study. 

Similarly, we iterated over several prompting approaches for the text generation features. Overall, this resulted in a few-shot approach, providing the model with several input-output examples to generate fitting responses. 
As an overview, we use the following prompt templates (details in \cref{sec:appendix_prompts}):

\paragraph{Sentence-level support, without user input:}
We prompted six suggestions for the sentence selected in the email (2 positive, 2 neutral, 2 negative). This balanced the options, following related work~\cite{Kannan2016smartreply}, as the LLM favoured positive responses in our tests.

\paragraph{Sentence-level support, with user input:} 
This was identical to the above case but now it included the user's input in addition to the selected sentence. %
We emphasised alignment with the user's sentiment (e.g. no negative suggestions if the user had entered ``yes'').


\paragraph{Message-level support:}
We prompted the LLM to answer to the whole email, also by taking into account any current user input, if available. A variation of this was also used for the \imppass{} feature (\cref{sec:impl_imppass}). That prompt emphasised improving the current state of the reply while closely adhering to the information provided by the user. %

%%%%%%%%%%%%%%%%%%%%%%%%%%%%%%%%%%%%%%%%%%%%
\begin{table*}[t]
\centering
\begin{tabular}{@{}l|ccc|ccc|ccc@{}}
\toprule
\multicolumn{1}{c} {\textbf{Time} (\textit{s})}&
  \multicolumn{1}{l}{\textbf{4xA100}} &
  \multicolumn{1}{l}{\textbf{2xA100}} &
  \multicolumn{1}{l}{\textbf{A100}} &
  \multicolumn{1}{l}{\textbf{4xA6000}} &
  \multicolumn{1}{l}{\textbf{2xA6000}} &
  \multicolumn{1}{l}{\textbf{A6000}} &
  \multicolumn{1}{l}{\textbf{4xV100}} &
  \multicolumn{1}{l}{\textbf{2xV100}} &
  \multicolumn{1}{l}{\textbf{V100}} \\ \midrule
CPU Freezing (\textit{s}) & 21.49 & 10.32 & 4.96  & 15.23 & 9.11  & 3.35  & 29.41  & 14.50 & 6.90  \\
CPU Frozen (\textit{s}) & 33.58 & 16.15 & 7.79  & 43.69 & 29.48 & 11.24 & 74.96  & 38.56 & 19.23 \\
CPU Mem. dump (\textit{s}) & 31.30 & 15.02 & 7.28  & 42.1  & 28.59 & 10.88 & 70.30  & 36.17 & 18.10 \\
CPU Mem. write (\textit{s}) & 28.62 & 13.99 & 6.80  & 40.4  & 27.70 & 10.49 & 66.30  & 34.40 & 17.26 \\ \midrule
%
\sys Checkpoint (\textit{s}) & 55.09 & 26.49 & 12.78 & 58.93 & 38.61 & 14.60 & 104.40 & 53.08 & 26.18 \\
\sys Restore (\textit{s}) & 35.13 & 17.22 & 8.32  & 24.1  & 13.83 & 5.50  & 43.14  & 21.69 & 10.61 \\ \midrule
Checkpoint size (GB) & 41.01 & 20.46 & 9.94 & 39.98 & 19.97 & 9.75 & 40.03  & 19.97 & 9.81 \\ \bottomrule
\end{tabular}\par
\vspace{-0.5em}
\captionsetup{justification=centering}
\caption{Checkpoint and restore performance (in seconds) when scaling training of GPT-2 Small (124M) to multiple GPUs.\\Times are not comparable across GPU families.}
\label{tab:multi-gpu-checkpointing}
\end{table*}

%%%%%%%%%%%%%%%%%%%%%%%%%%%%%%%%%%%%%%%%%%%%%%%%
\begin{figure}[t]
    \centering
    \includegraphics[width=.9\columnwidth]{figures/H100_Lock_Checkpoint_vs_Restore_Unlock.pdf}
    \vspace{-.5em}
    \caption{In-memory GPU checkpoint/restore with H100. Similar results are observed with A100.}
    \label{fig:in-memory-checkpoint-restore}
\end{figure}
%%%%%%%%%%%%%%%%%%%%%%%%%%%%%%%%%%%%%%%%%%%%%%%%
\begin{figure*}[t]
  \centering
  \begin{subfigure}[b]{\columnwidth}
    \includegraphics[width=.9\textwidth]{figures/H100_Unified_Restore_Times.pdf}
    \label{fig:h100-unified-restore-times}
  \end{subfigure}
  \hfill
  \begin{subfigure}[b]{\columnwidth}
    \includegraphics[width=.9\textwidth]{figures/A100_Unified_Restore_Times.pdf}
    \label{fig:a100-unified-restore-times}
  \end{subfigure}
  \vspace{-1em}
  \caption{Time to restore for model training from a checkpoint with \sys for H100 and A100 GPUs.}
  \label{fig:unified-restore-times}
\end{figure*}
%%%%%%%%%%%%%%%%%%%%%%%%%%%%%%%%%%%%%%%%%%%%%%%%

\section{Evaluation} \label{sec:evaluation}%
%
Our evaluation seeks to answer the following questions:
\begin{itemize}[leftmargin=*,leftmargin=15pt,itemindent=0pt]
    \item How does \sys perform when checkpointing and restoring large language models? (\textsection{\ref{sec:eval:diff-models}})

    \item What are the scalability implications of using checkpointing and restoring with multiple GPU devices? (\textsection{\ref{sec:eval:scalability}})

    \item What are the dominant factors affecting the latency of checkpointing and restore operations? (\textsection{\ref{sec:eval:overhead}})

    \item Can \sys support checkpoint and restore with both CUDA and ROCm applications? (\textsection{\ref{sec:eval:rocm}})
\end{itemize}

\subsection{Experimental Methodology}%
%
\stitle{Evaluation Setup.}
We evaluate \sys on 10 servers with specifications described in \Cref{tab:server-configs}, running Ubuntu 22.04 with kernel version 6.2.0 (A100 and H100), 5.15.0 (V100 and A6000), and NVIDIA driver 565.57.01, CUDA 12.7, and CentOS Stream 9 with kernel 5.14 with ROCm 5.6 (MI210).

\stitle{Performance Measurements.} We measure the performance of checkpoint and restore operations using detailed statistics generated by CRIU~\cite{criu-statistics} about the time spent in different stages, and with external tools like \texttt{perf stat} to gather more detailed performance data. We run each experiment 10 times and calculate the mean and standard deviation of each value in the collected data. To analyze the overhead of checkpointing with \sys, we measure the following performance metrics for models of different sizes:

\begin{itemize}[leftmargin=*,leftmargin=10pt,itemindent=0pt]
    \item \textbf{Checkpoint time:} The total time to create a snapshot of the running GPU application.
    \item \textbf{Freezing time:} The time to suspend the application using \texttt{ptrace} seize and interrupt.
    \item \textbf{Frozen time:} The time during checkpointing when the application is not running.
    \item \textbf{Memory dump time:} The time to collect the CPU memory pages of running processes. This does not include the time to write this memory to storage.
    \item \textbf{Memory write time:} The time to save the memory state to persistent storage.
    \item \textbf{Restore time:} The time to restore both CPU and GPU state from storage, and to resume the application.
\end{itemize}

\stitle{Workloads and micro-benchmarks.} We evaluate the proposed checkpoint/restore mechanisms for multiple models of different sizes (listed below) and a set of ROCm micro-benchmarks~\cite{amd2024rocm} representing common HPC workloads for AMD GPU~(\textsection{\ref{sec:eval:rocm}}). For NVIDIA GPUs we use the following models:

\begin{itemize}[leftmargin=*,leftmargin=10pt,itemindent=0pt]
    \item \textbf{LLaMA} \textbf{3.2} (1B, 3B) and \textbf{3.1} (8B)

    \item \textbf{GPT-2} with 124M, 355M, 774M, 1.5B parameters

    \item \textbf{BERT} Base (110M) and Large (340M) models
\end{itemize}
%
\subsection{\sys Performance with Deep Learning Models}
\label{sec:eval:diff-models}
%
We evaluate the performance of GPU checkpoint and restore operations with multiple model training workloads of different sizes. The results in~\Cref{fig:in-memory-checkpoint-restore} show that the time required to checkpoint the GPU state into host memory increases significantly for models with large number of parameters.
For instance, checkpointing and locking operations for the GPT-2 Small model (124M parameters) take an average of 4.9 seconds and 240 ms, respectively. In comparison, for a larger model such as GPT-2 XL (1.5B parameters), these operations require an average of 28 seconds and 500 ms, respectively.
The time to restore the GPU state from host memory increases gradually, with 2.5 seconds for GPT-2 Small and 11 seconds for GPT-2 XL, while the unlock time remains consistent for both models at approximately 160 ms.
We observe similar results with both A100 and H100 GPUs, highlighting the crucial impact memory bandwidth has on the performance of GPU checkpointing operations.
Several techniques have been proposed to address this problem, such as data compression and on-demand parallelism~\cite{yang2024on-demand}. Incorporating such techniques could further improve the efficiency of checkpoint/restore operations, especially for large-scale models.

~\Cref{fig:unified-restore-times} shows the unified restore time (the time to restore the combined CPU-GPU state) for models with different sizes with both H100 and A100 GPUs. The time required to restore the GPU state makes up a significant portion of the total restore time for small models, but becomes a relatively lesser portion for larger models. These results demonstrate that the restore time is also affected by the available bandwidth for CPU-GPU memory transfers, as well as the speed at which checkpoint data is loaded from disk into host memory.
The performance results and checkpoint sizes shown in \Cref{tab:combined-checkpoint-restore-times} suggest that the differences in performance between the experiments with H100 and A100 GPUs can be attributed not only to advancements in GPU hardware architecture but also to the critical role of the available CPU resources.

%%%%%%%%%%%%%%%%%%%%%%%%%%%%%%%%%%%%%%%%%%%%%%%%
\begin{table}[t]
\centering
\renewcommand{\arraystretch}{.9} % Reduce row height for tighter spacing
\begin{tabular}{@{}c|rrr@{}}
\toprule
\textbf{Model} & \textbf{Total (GB)} & \textbf{GPU (\%)} & \textbf{CPU (\%)} \\ \midrule
BERT-B~~(110M)  & 5.22  & 82.38\%  & 17.62\% \\
GPT2-S~~(124M)  & 10.32 & 89.15\%  & 10.85\% \\
BERT-L~~(340M)  & 9.90  & 90.91\%  & 9.09\%   \\
GPT2-M~(355M)  & 19.57 & 91.31\% & 8.69\%   \\
GPT2-L~~(774M)  & 35.39 & 90.99\% & 9.01\%  \\
LLaMA 3.2~~~(1B) & 14.81 & 92.37\% & 7.63\%  \\
LLaMA 3.2~~~(3B) & 29.54 & 95.70\%  & 4.30\%   \\
LLaMA 3.1~~~(8B) & 55.89 & 97.35\% & 2.65\%  \\
GPT2-XL~(1.5B) & 60.12 & 96.02\% & 3.98\%  \\ \bottomrule
\end{tabular}
\vspace{-.5em}
\caption{The total unified checkpoint size (in GB) and the corresponding proportions of GPU memory and CPU state, respectively, for various models running on H100 GPU. We observe similar results with A100 GPU.}
\label{tab:cpu-to-gpu-state-comparison}
\vspace{-1.5em}
\end{table}
%%%%%%%%%%%%%%%%%%%%%%%%%%%%%%%%%%%%%%%%%%%%%%%%

\Cref{tab:cpu-to-gpu-state-comparison}  shows the total \sys checkpoint sizes for various models on both A100 and H100 GPUs, along with a breakdown of GPU memory and CPU state proportions. A key insight is the dominance of GPU memory in overall checkpoint size, consistently exceeding 80\% and often surpassing 90\% for larger models like LLaMA 3.1 (8B) and GPT2-XL (1.5B). These results further emphasize the importance of efficient CPU-GPU memory transfers in optimizing the performance of checkpointing and restore operations.

%%%%%%%%%%%%%%%%%%%%%%%%%%%%%%%%%%%%%%%%%%
\subsection{Multi-GPU Checkpointing Performance}
\label{sec:eval:scalability}
%
We evaluate the scalability of \sys by running an experiment designed to train a large language model (GPT-2) across 1x, 2x, and 4x GPUs of V100, A6000, and A100 types, using data parallelism to distribute the workload.
\Cref{tab:multi-gpu-checkpointing} shows that the checkpoint size increases with the number of GPUs, as each GPU stores its own copy of the model parameters.
For instance, the checkpoint size for 1, 2, and 4 A100 GPUs is $\approx10$, $\approx20$, and $\approx40$ GB, respectively.
This increase in checkpoint size reflects the increasing amount of intermediate model state that is saved as the number of GPUs increases.
The freezing, frozen, memory dump and memory write times also increase with the number of GPUs, likely because more time is spend on handling the larger checkpoint data with additional GPUs.
For example, as the number of A100 GPUs increases, creating a unified CPU-GPU snapshot requires scanning more memory pages.
A single GPU requires $\approx7$ million pages scanned, two GPUs require $\approx15$ million, and four GPUs require $\approx28$ million. Our experiments demonstrate that \sys efficiently scales as the number of GPUs and data-parallel replicas increase. We observe that checkpointing and restore times scale near linearly as we increase the number of GPUs from 1 to 4 across different GPU types.

\subsection{Checkpoint and Restore Latency}%
\label{sec:eval:overhead}%
We analyze the overhead of checkpoint and restore operations of \sys by measuring the latency during these processes for LLaMA 3.1 and GPT2-XL model training workloads on H100 and A100 GPUs. The results in ~\Cref{tab:combined-checkpoint-restore-times} highlight the performance differences between the H100 and A100 GPUs, as well as the impact of additional CPU and memory resources on reducing overhead during the checkpoint and restore operations with \sys. The primary factors affecting \sys's checkpoint and restore performance are:

\begin{itemize}
    \item \textbf{GPU count}: Increasing the number of GPUs leads to increased checkpoint size and latency;
    \item \textbf{CPU-GPU bandwidth}: The speed of data transfers between the CPU and GPUs directly affects checkpointing and restoring speed;
    \item \textbf{GPU memory usage}: Larger models with more parameters have higher checkpoint and restore latencies.
\end{itemize}

\begin{figure}[t]
  \centering
  \includegraphics[width=\columnwidth]{figures/MI210_Unified_Checkpointing_Times.pdf}
  \vspace{-2.5em}
  \caption{Breakdown of the \sys checkpointing time for HPC benchmarks running on AMD MI210 GPU.}
  \label{fig:amd-gpu-checkpointing}
  \vspace{-.5em}
\end{figure}
%%%%%%%%%%%%%%%%%%%%%%%%%%%%%%%%%%%%%%%%%
\begin{table}[t]
\centering
\begin{tabular}{lr}
\toprule
\textbf{Benchmark} & \textbf{Checkpoint Size}
\\ \midrule
Binomial Option Pricing       & 305~MB \\
Bitonic Sort                  & 614~MB \\
Discrete Cosine Transform     & 1.2~GB \\
1D Haar Wavelet Decomposition & 333~MB \\
Fast Walsh Transform          & 307~MB \\
Floyd Warshall                & 484~MB \\
Prefix Sum                    & 306~MB \\
Recursive Gaussian            & 311~MB \\
Histogram                     & 16.64 GB \\
Matrix Multiplication         & 19.88 GB \\
Convolution                   & 13.83 GB \\
\bottomrule
\end{tabular}
\caption{\sys checkpoint sizes of ROCm benchmarks.}
\label{tab:rocm-benchmark-checkpoints}
\vspace{-1em}
\end{table}

%%%%%%%%%%%%%%%%%%%%%%%%%%%%%%%%%%%%%%%%%
\subsection{\sys Support for ROCm Devices}
\label{sec:eval:rocm}

In addition to support for CUDA, to demonstrate the checkpointing functionality of \sys for AMD GPUs, we evaluate its performance using a set of ROCm HPC micro-benchmarks. These benchmarks provide a set of workloads representative of typical HPC applications, allowing us to analyse \sys's ability to effectively checkpoint and restore GPU state across different computational patterns. \Cref{fig:amd-gpu-checkpointing} shows the frozen, memory dump, and memory write times during checkpointing for each benchmark. While most of the evaluated benchmarks have relatively small checkpoint size, typically ranging from under 500~MB to 1.2~GB, a few have significantly larger checkpoint sizes (Histogram, Matrix Multiplication, and Convolution), shown in \Cref{tab:rocm-benchmark-checkpoints}. This increase in checkpoint size directly correlates with longer freezing and memory dump times, as \sys must checkpoint larger amount of data. An interesting observation is the contrasting distribution of checkpoint data between host memory and GPU memory across these benchmarks. While more than half of the checkpoint size for the Convolution benchmark is attributed to AMD GPU state, Histogram and Matrix Multiplication have the majority of their state residing in host memory.


%%%%%%%%%%%%%%%%%%%%%%%%%%%%%%%%%%%%%%%%%%%
\section{Discussion}~\label{sec:discussion}%
\sys is a novel technique for fully transparent checkpointing of GPU applications implemented with CUDA and ROCm. Our evaluation results show that \sys can create unified CPU-GPU snapshots for large models in single- and multi-GPU setups without steady-state performance overhead. Our evaluation results further demonstrate that \sys scales linearly with the number of GPU devices for data-parallel workloads, enabling efficient checkpointing with large-scale applications.

\stitle{Deterministic Restore.}
In contrast to previous work, which requires validation of the replayed GPU device API calls~\cite{gupta2024just}, \sys relies on a locking mechanism that ensure the execution of tasks is suspended before creating a checkpoint. This guarantees consistent CPU-GPU snapshots and deterministic restore operations, during which both CPU and GPU state are restored before the application resumes execution.

\stitle{Real-World Deployments.}
The proposed GPU checkpointing mechanism has been successfully implemented in several production systems, including MemVerge~\cite{wu2025memory} and Modal~\cite{belotti2025memory}.


%%%%%%%%%%%%%%%%%%%%%%%%%%%%%%%%%%%%%%%%%%%%%%%
\section{Related Work}~\label{sec:related-work}%
Checkpoint/Restore has recently received significant attention in the context of AI workloads as a way to quickly recover from faults during model training (either in the GPU or in the CPU), and to better utilize hardware when multiple users content for the same hardware. The latter is becoming increasingly relevant with the introduction of LLM Agents in which GPU devices may need to wait for external agents to return before the LLM running on the GPU could proceed~\cite{nvidia2023introduction}. This section summarizes recent research efforts and how they have focused on improving different aspects of Checkpoint Restore, and how it compares to \sys.

\stitle{Checkpoint Types.} Distributed runtime engines such as Ray~\cite{moritz2018ray} and deep learning frameworks and libraries such as PyTorch~\cite{paszke2019pytorch}, TensorFlow~\cite{abadi2016tensorflow}, and MXNet~\cite{chen2015mxnet} allow users to specify checkpoint and restore logic including how and when to checkpoint. Such techniques are specific to each runtime engine and require developer involvement, thus being considered \textit{application-aware checkpoints}.

Many other works proposed and improved designs for transparent checkpoints. Singularity~\cite{shukla2022singularity} proposed a device proxy server design that relies on API interception to log all runtime and drive API calls. This log would be used to reconstruct the GPU state upon a restore operation. Such technique has been used widely as a mechanism for semi-transparent checkpoint restore support~\cite{chaudhary2020balancing,gupta2024just}. However, a design based on a device proxy server leads to a number of challenges (as discussed in \autoref{sec:background}), resulting in limited support for large and complex training and inference workloads. \sys, on the other hand, introduces a new design for checkpoint and restore by offering a fully-transparent and unified checkpoint mechanism to save the state of the application running on the CPU (including the engine/framework/library running the user application), and its corresponding state on the GPU. \sys does not rely on API interception and supports different GPU families with both CUDA and ROCm applications.

\stitle{Deciding when to Checkpoint.} Transparent GPU checkpointing is often used to enable error recovery for training jobs by periodically saving the model parameters and optimizer state to persistent storage. State-of-the-art error recovery solutions today implement \textit{periodic} or \textit{just-in-time} checkpointing support where training jobs can be resumed from any prior checkpoint. When GPU workloads experience an error, they either hang or crash, and cause the training jobs to be unexpectedly terminated. These errors are often detected with a series of diagnostic tests and problematic nodes are cordoned off, allowing the training job to resume on a different node.

Recovery mechanisms used with periodic checkpoints are often optimized to balance low running time overhead and high checkpoint frequency~\cite{mohan2021checkfreq}. These optimizations aim to minimize both the runtime overhead and the amount of work the job has to redo on recovery (recovery cost).
%
In contrast, just-in-time checkpointing solves this problem by creating a checkpoint only after a failure has occurred by leveraging the fact that the GPU state is only updated during a short interval and multiple replicas are holding the same state. As a result, this approach is able to recover from errors by recomputing at most one mini-batch~\cite{gupta2024just}.

\sys can be integrated with both periodic and just-in-time checkpointing policies to i) accelerate the checkpoint and restore operations, and ii) create a unified and consistent snapshot of the CPU and GPU state. Furthermore, \sys does not rely on specific workload characteristics such as relying on replicas having an exact copy of the model, thus being a general approach for GPU state checkpointing.

\stitle{Checkpoint/Restore Optimizations.} There have been many other works that optimize checkpoint/restore operations for GPU-accelerated workloads.
CheckFreq~\cite{mohan2021checkfreq} reduces the overhead of periodic checkpointing by overlapping communication with snapshotting of the model state in the GPU. 
Gemini~\cite{team2024gemini} checkpoints the GPU state to local and remote host memory, and interleaves checkpointing traffic with training traffic to reduce the overheads of checkpointing. 
Nebula~\cite{microsoft2024boost} copies model state asynchronously reducing the time during which the training job is paused for checkpoint. 
DeepFreeze~\cite{nicolae2020deepfreeze} proposes a fine-grain asynchronous checkpointing of deep learning models and shards the checkpointing effort across multiple workers. However, it only considers CPU clusters and does not take into account the cost of snapshotting the model state in memory when trained with state-of-the-art GPUs.
Check-N-Run~\cite{eisenman2022check} proposes incremental/differential checkpointing to speedup recommendation model training and uses quantization techniques to reduce the snapshot size.
ZeRo~\cite{rajbhandari2020zero} shards model parameters and optimizer state across data-parallel GPUs, parallelizing the checkpoint effort.

We consider our work to be complementary to these previous contributions. Optimizations based on sharding, compression, asynchronous, or incremental checkpointing could be incorporated into \sys. In summary, \sys presents a novel checkpoint/restore technique on which many of previously proposed optimizations could be integrated.

Another recent line of related work explores running GPU-accelerated inference workloads on serverless platforms, focusing on optimizing resource efficiency, scheduling, and caching policies~\cite{10.5555/3433701.3433792,280768,ishakian2018serving,280704,10.1145/3620678.3624664,infaas,10.1145/3503222.3507709,234998,yang2024on-demand}. Yang et al.~\cite{yang2024on-demand} also leverages CRIU to accelerate the restore time for GPU workloads, but, unlike \sys, does not explore unified container snapshots. Besides, the paper~\cite{yang2024on-demand} focuses on AMD-only devices and does not explore the performance of checkpoint and restore operations on large ML models.
Nevertheless, most of these systems focus on reducing model load time rather than on a general approach for checkpoint and restore GPU-accelerated applications. Many additional optimizations could be combined with \sys to further facilitate model loading. These optimizations are outside the scope of this paper and will be addressed in future work.

\stitle{Framework-level checkpointing} is supported by many popular machine learning libraries, such as PyTorch~\cite{paszke2019pytorch} and TensorFlow~\cite{abadi2016tensorflow}. This approach requires the application, framework or library to correctly capture and restore the runtime state of applications. However, implementing and maintaining this checkpointing mechanism introduces additional development and testing burdens. In addition, it has been shown to be both error-prone and inefficient, often leading to checkpoint file loss or corruption due to job interruptions, as well as significant recovery times~\cite{mohan2021checkfreq}. Thus, in this work, we focus on developing a reliable and efficient system-level checkpointing that is fully transparent.

\section{Conclusion}\label{sec:conclusion}
This work introduces a novel approach to TOT query elicitation, leveraging LLMs and human participants to move beyond the limitations of CQA-based datasets. Through system rank correlation and linguistic similarity validation, we demonstrate that LLM- and human-elicited queries can effectively support the simulated evaluation of TOT retrieval systems. Our findings highlight the potential for expanding TOT retrieval research into underrepresented domains while ensuring scalability and reproducibility. The released datasets and source code provide a foundation for future research, enabling further advancements in TOT retrieval evaluation and system development.

\section*{Acknowledgments}
The authors gratefully acknowledge Felix Kuehling, Rajneesh Bardwaj, David Yat Sin, David Francis, and Ramesh Errabolu for their invaluable help and support with implementation of AMD GPU plugin. We acknowledge the use of the University of Oxford Advanced Research Computing (ARC) facility~\cite{richards2015university} in conducting parts of the evaluation experiments. This work is partly supported by the EPSRC Doctoral Training Partnership (DTP) [grant number EP/T517811/1].

\balance
\bibliographystyle{plain}
\bibliography{main}

\end{document}