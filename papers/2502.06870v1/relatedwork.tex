\section{Related Work}
\textbf{Road Network Representation Learning} aims to transform the road network into a general low-dimensional representation matrix. Since graph representation learning methods can model the topological structure of the road network~\cite{DeepWalk,LINE,node2vec,GraphSAGE}, some existing methods~\cite{IRN2Vec,chen,SARN} take the spatial correlations of road segments into account by using graph representation learning methods. 
A more recent method~\cite{HRNR, yu2024bigcity} takes the transition patterns into account when modeling the road networks. In summary, our proposed method is the first attempt to jointly model the dynamics of road segments based on trajectory and traffic state data.


\textbf{Trajectory Representation Learning} aims to transform the trajectory into a general low-dimensional representation vector. Early TRL studies~\cite{trajectory2vec,t2vec} obtain trajectory representations through the reconstruction task. Recent TRL methods~\cite{TMRN,PIM,CSTRM,JCLRNT,WSCCL,Toast,MMTEC,START} primarily first obtain the road network representations and then derive trajectory representations through sequential models with self-supervised tasks. The trajectory representation is assumed to be static in most methods, with only a few encoding temporal information in trajectories. For example, Trembr~\cite{Trembr} and START~\cite{START} reconstructs timestamps during the decoding process. However, they do not consider the impact of dynamic traffic states on trajectory representations, which is one of our main contributions.