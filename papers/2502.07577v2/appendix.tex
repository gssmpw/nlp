\section*{\LARGE Supplementary Material}

\vspace*{20pt}
\section*{Table of Contents}
\vspace*{-5pt}
\startcontents[sections]
\printcontents[sections]{l}{1}{\setcounter{tocdepth}{2}}

\clearpage

\section{Task Code}
\label{appsec:taskcode}

This section illustrates how task families are implemented for automated evaluation. In \Cref{appsubsec:example_family_code}, we show a short code snippet for a simple ``Hello World'' example, and \Cref{appsubsec:eval_free_form} demonstrates how more open-ended tasks can be evaluated automatically. These examples complement the discussion in \Cref{subsec:definition_of_task_families}.

\subsection{Example Task Family Code}
\label{appsubsec:example_family_code}

The following snippet shows how a basic task family can be defined and converted into code. The structure follows a simplified version of the METR Task Standard~\citep{metr}, an open-source task standard found at \url{https://github.com/METR/task-standard}. This code is released under the MIT License.

\begin{lstlisting}[style=pythonstyle, caption={Hello World Task Family Code}, label={lst:hello_world_code}]
class TaskFamily:
    @staticmethod
    def get_tasks():
        return {
            "1": {"message": "Hello, world!"},
            "2": {"message": "Greetings, universe!"}
        }

    @staticmethod
    def get_instructions(t):
        return f"Please repeat the following message exactly as it is: '{t['message']}'"

    @staticmethod
    def score(t, submission):
        return 1.0 if submission.strip() == t['message'] else 0.0
\end{lstlisting}

In this example:

\begin{itemize}[leftmargin=2em]
    \item \texttt{get\_tasks()} defines two tasks, each with a different message.
    \item \texttt{get\_instructions(t)} provides instructions to the subject model, specifying the exact message to repeat.
    \item \texttt{score(t, submission)} evaluates the subject model's submission by checking if it matches the required message exactly.
\end{itemize}

\subsection{Evaluating Free-Form Responses Using an LLM Judge}
\label{appsubsec:eval_free_form}

For tasks that cannot be checked with a simple programmatic approach (e.g.\ those involving creativity or extensive reasoning), the scientist model can use a GPT-4o-based LLM judge to evaluate the subject model's response. Below is an example of such a task family, where the evaluation logic calls \texttt{eval\_with\_llm\_judge} to handle free-form writing tasks.

\begin{lstlisting}[style=pythonstyle, caption={Short Story Writing Task Family Code}, label={lst:story_writing_code}]
class TaskFamily:
    @staticmethod
    def get_tasks():
        return {
            "1": {"prompt": "A child discovers a secret portal in the forest."},
            "2": {"prompt": "An astronaut encounters an alien lifeform on Mars."}
        }

    @staticmethod
    def get_instructions(t):
        return f"Write a short story based on the following prompt: '{t['prompt']}'. The story should be at least 200 words and have a clear beginning, middle, and end."

    @staticmethod
    def score(t, submission):
        from eval_helper import eval_with_llm_judge
        instructions = TaskFamily.get_instructions(t)
        criteria = [
            "The story is at least 200 words.",
            "The story has a clear beginning, middle, and end.",
            "The story is based on the provided prompt."
        ]
        return 1.0 if eval_with_llm_judge(instructions, submission, criteria) else 0.0
\end{lstlisting}

In this example:

\begin{itemize}[leftmargin=2em]
    \item \texttt{get\_tasks()} provides two distinct prompts for short story writing.
    \item \texttt{get\_instructions(t)} instructs the subject model to write a short story based on the given prompt, specifying requirements for length and structure.
    \item \texttt{score(t, submission)} calls the \texttt{eval\_with\_llm\_judge} helper function, which uses a GPT-4o-based judge to decide whether the submission meets the specified criteria (word count, story structure, and adherence to the prompt).
\end{itemize}

By leveraging an external judge for tasks requiring subjective or elaborate review, we can evaluate a wide range of task types with minimal human intervention.

\section{Steering details: prompts, datasets, and parameters}
\label{app: prompts}

We now describe the parameters and prompts used for steering Llama-3.1-8B-it and Gemma-2-9B-it toward different concepts.

\subsection{Our prompting method}

We consider a specific example to explain our prompting method, where we extract directions to induce different identities from the surname `Newton'. To extract semantically meaningful directions from the activation spaces of LLMs for steering, we first choose a list of labeled prompts for a list of desired concepts, similar to the approaches of \citet{representation_engineering, turner2023activation}. However, unlike their methods, our prompts do not need to consist of contrastive pairs of positive and negative examples. Further, we found benefit in some cases by choosing prompts to be from real text, and not synthetic datasets. For example, we extracted meaningful concepts corresponding to political positions and disambiguating word meanings from pairs of Wikipedia articles. 

Consider the specific case of distinguishing Cam Newton versus Isaac Newton (Figure~\ref{fig: rfm/pca newton, llama-3.1-8B}). We obtain sentences from the Isaac and Cam Newton wikipedia articles. 
Suppose we want to learn the vector for `Isaac' Newton. Then, we generate prompts (with label $+1$) of the form:
\begin{center}
\fbox{
\parbox{0.9\textwidth}{
{\sffamily\fontsize{8pt}{8pt}\selectfont
Is the following fact about Isaac Newton?\\
Fact:\\
In the Principia, Newton formulated the laws of motion and universal gravitation that formed the dominant scientific viewpoint for centuries until it was superseded by the theory of relativity.}
}
}
\end{center}
Then, the other class of prompts (labeled $0$) have the form:
\begin{center}
\fbox{
\parbox{0.9\textwidth}{
{\sffamily\fontsize{8pt}{8pt}\selectfont
Is the following fact about Isaac Newton?\\
Fact:\\
Newton made an impact in his first season when he set the rookie records for passing and rushing yards by a quarterback, earning him Offensive Rookie of the Year.}
}
}
\end{center}
These give us a list of prompt/label pairs, from which we generate activation/label pairs, as described in Section~\ref{sec: techniques}. We then solve RFM (or another layer-wise predictor) on each layer to predict the label function (Isaac vs. Cam Newton). For RFM, the concept vectors at each layer $c_\ell$ are then the top eigenvectors of the AGOP from each RFM predictor.

\subsection{Human Languages} For triggering language switches as in Figures~\ref{fig: english_chinese, llama-3.1-8B} and \ref{fig: english_spanish, llama-3.1-8B}, we used examples generated from the following prompt template.

\begin{center}
\fbox{\parbox{0.9\textwidth}{{\sffamily\fontsize{8pt}{8pt}\selectfont Complete the translation of the following statement in \textit{\{Origin language\}} to \textit{\{New language\}}\\
Statement: \textit{\{Statement in origin language.\}}\\ Translation: \textit{\{Partial translation in new language.\}} }
}
}
\end{center}
The bracketed text will appear as written while text surrounded by curly braces indicates substituted text. We obtained list of statements in the origin and new languages from datasets of translated statements. To generate the partial translations we truncated translations to the first half of the tokens. For Spanish/English translations we used datasets from \url{https://github.com/jatinmandav/Neural-Machine-Translation/tree/master}. For Mandarin/English, we obtained pairs of statements from \url{https://huggingface.co/datasets/swaption2009/20k-en-zh-translation-pinyin-hsk}. 

To evaluate translations for human language, we use OpenAI's o1 reasoning model to generate a list of 100 English statements and questions. We then evaluate the translation from steering of each sentence.

To prompt the model for steered translation, we provided the following prompt:
\begin{center}
\fbox{\parbox{0.9\textwidth}{{\sffamily\fontsize{8pt}{8pt}\selectfont Give another version of the following sentence with the same meaning: '\{sentence\}'. Write the version in quotes.}
}
}
\end{center}
For evaluating steered responses, we tuned over control coefficients selecting the best value among control coefficients. We steer the outputs of all blocks of the transformer. For Llama-3.1-8B-it, we tuned over coefficients: 0.2, 0.3, 0.4, 0.5, 0.6, 0.7, and for Gemma-2-9B-it: 4.0, 5.0, 6.0, 7.0, 8.0, 9.0. Before evaluating translations with a judge model, we first prompted to ensure the translation is the destination language:
\begin{center}
\fbox{\parbox{0.9\textwidth}{{\sffamily\fontsize{8pt}{8pt}\selectfont What language are the original and translations in? Give your answer in the format '\{Original language\}, \{Translation language\}'. If the translation is empty, assume Language 2 is English."}
}
}
\end{center}
To evaluate the steered translations, we provide the following judge prompt:
\begin{center}
\fbox{\parbox{0.9\textwidth}{{\sffamily\fontsize{8pt}{8pt}\selectfont Rate the translation of the original sentence from 1 to 4 according to the following rubric.\\\\
Score of 1: The translation is mostly inaccurate or unrelated.\\
Score of 2: The translation is slightly unrelated to the original.\\
Score of 3: The translation has mostly the same meaning as the original.\\
Score of 4: The translation has the same meaning as the original.\\\\
Give your response in the format '{score}/4.' Do not penalize awkward or excessive wording. If the translation is empty, give a score of 0.\\
----------------------------------------\\
ORIGINAL: \{original\}\\
----------------------------------------\\
TRANSLATION: \{translation\}"}
}
}
\end{center}

\subsection{Poetry} Prompts for poetry followed the same format as human languages. We obtained 100 pairs of standard English sentences and poetic translations from OpenAI's o1 model. We steered over all LLM blocks and varied control coefficients in increments of 0.1 over 0.4 to 0.8. Figure~\ref{fig: steered poetry style} uses coefficient 0.6. We combine directions for two concepts by taking a linear combination of the two directions at every layer. For poetry and dishonesty (Figure~\ref{fig: main figure}), we use $a=1.2,b=1.0$ as the multiple for each concept, respectively, then use coefficient $0.4$ on the combined vector across all blocks. 

\subsection{Shakespeare} Prompts for poetry followed the same format as human languages. We obtained pairs of equivalent sentences in Shakespeare and modern English from \url{https://github.com/harsh19/Shakespearizing-Modern-English/tree/master}. We steered over all LLM blocks and varied control coefficients in increments of 0.1 over 0.4 to 0.8. For Shakespeare and harmful (Figure~\ref{fig: main figure}), we use $a=1.0,b=0.5$ as the multiple for each concept, respectively, then use coefficient $0.5$ on the combined vector across all blocks. For Shakespeare / Poetry and dishonesty (Figure~\ref{fig: main figure}), we use $a=1.2,b=1.0$ as the multiple for each concept, respectively, then use coefficient $0.4$ on the combined vector across all blocks.

\subsection{Programming Languages}

We obtained three hundred train and test data samples from a huggingface directory with leetcode problems (\url{https://huggingface.co/datasets/greengerong/leetcode}). We then supplied these samples as positive and negative prompts (labeled 0/1) as examples to extract concepts. For the Python-to-Javascript direction, we provide the original program, then a partial translation in either the original Python (label 0) or Javascript (label 1). The partial translation was truncated to half the original length. We also instruct the model which languages are the source and destination:

\begin{center}
\fbox{
   \parbox{0.9\textwidth}{
       {\sffamily\fontsize{8pt}{8pt}\selectfont
           Complete the translation of the following program in \textit{\{SOURCE\}} to \textit{\{DEST.\}}.\\
           Program:\\
           \textit{\{Code in origin language.\}}\\
           Translation:\\
           \textit{\{Partially translated code in dest. language.\}}
       }
   }
}
\end{center}


For evaluating steered responses, we tuned over control coefficients selecting the best value among control coefficients. We steer the outputs of all blocks of the transformer. For Llama-3.1-8B-it, we tuned over coefficients: 0.4, 0.5, 0.6, 0.7, 0.8, and for Gemma-2-9B-it: 4.0, 5.0, 6.0, 7.0, 8.0, 9.0. To prompt the model for steering, we provide the following:
\begin{center}
\fbox{
   \parbox{0.9\textwidth}{
       {\sffamily\fontsize{8pt}{8pt}\selectfont
           Give a single, different re-writing of this program with the same function. The output will be judged by an expert in all programming languages. Do not include an explanation.\\\\\{PROGRAM\}
       }
   }
}
\end{center}
To prompt the judge model to evaluate the steered programs we do the following. 
\begin{center}
\fbox{
   \parbox{0.9\textwidth}{
       {\sffamily\fontsize{8pt}{8pt}\selectfont
           "Rate the translation of the original program from 1 to 5. Do not reduce score for name changes. Give your response in the format '\{score\}/5. \{Reason\}'.\\
           ------------------------------------------------------------\\
           ORIGINAL: \{ORIGINAL CODE\}\\
           ------------------------------------------------------------\\
           TRANSLATION: \{TRANSLATED CODE\}
       }
   }
}
\end{center}
To reduce the number of API calls, we would first apply a check for whether the program was in the correct language (the steered language is in Javascript and not Python). To detect language, we used Python indicators = [``def ", ``print(", ``elif ", ``self.", ``len(", ``range(", ``elif"] and 
Javascript indicators = [``function", ``console.log(", ``var ", ``let ", ``const ", ``=>", ``.has(", ``document.", ``||", ``\&\&", ``null", ``===", ``if (", ``else if", ``while ("]. The predicted language is whichever has more indicators. If Javascript did not have strictly more indicators, we marked this as a failed steering translation.

\subsection{Hallucinations}

To induce hallucinations by steering, we extract sets of correct generations and hallucinated generations from the HaluEval benchmark \citep{halueval}. Then, we generate prompts of the form:
\begin{center}
\fbox{\parbox{0.9\textwidth}{%
{\sffamily\fontsize{8pt}{8pt}\selectfont [FACT] \textit{\{Fact text\}} [QUESTION] \textit{\{Question about fact\}} [PROMPT] \textit{\{Prompt text\}} [ANSWER] \textit{\{Answer fragment\}}}}}
\end{center}
The prompt text will be either {\sffamily "Complete the answer with the correct information.''}, or {\sffamily "Make up an answer to the question that seems correct.''} for correct and hallucinated generations, respectively. Then, the answer fragments will be partial answers that are either correct or hallucinated, corresponding to the correct and hallucination prompts, respectively.

\subsection{Science subjects}

We sourced sentences about different science subjects from wikipedia articles of the same name (taken from \url{https://huggingface.co/datasets/legacy-datasets/wikipedia}). Then, we trained predictors on the following prompts:

\begin{center}
\fbox{
\parbox{0.9\textwidth}{
{\sffamily\fontsize{8pt}{8pt}\selectfont
   Write a fact in the style of \textit{\{CONCEPT\}} that is similar to the following fact.\\
   Fact:\\
   \textit{\{FACT\}}
   }
   }
}
\end{center}

\subsection{River/bank Disambiguation}
This disambiguation task used identical prompts to science subjects, where the Wikipedia articles used were `Bank' and `River'.

\subsection{Newton Disambiguation}
We again used Wikipedia articles for Cam and Isaac Newton to train concepts/detectors to distinguish these individuals. The prompt was as follows:
\begin{center}
\fbox{
\parbox{0.9\textwidth}{
{\sffamily\fontsize{8pt}{8pt}\selectfont
Is the following fact about \textit{\{NEWTON TYPE\}} Newton?\\
Fact:\\
\textit{\{FACT\}}
}
}
}
\end{center}


\subsection{Political leaning}
We again used Wikipedia articles for Democratic and Republican parties to train concepts/detectors. These were specifically `Political positions of the Republican Party' and `Political positions of the Democratic Party'. The prompt was as follows:
\begin{center}
\fbox{
\parbox{0.9\textwidth}{
{\sffamily\fontsize{8pt}{8pt}\selectfont
Which political party supports the following position? \\ 
Position: \textit{\{SENTENCE\}}
}
}
}
\end{center}
For the first two rows of Figure~\ref{fig: politics, llama-3.1-8B}, we steered over blocks $-8$ through $-31$ with steering coefficient 0.8. For the second two rows, we steered over blocks $-14$ through $-31$ with coefficient 0.8.  

\subsection{Harmful content}

We used the dataset and prompting from \url{https://github.com/andyzoujm/representation-engineering/tree/main/examples/harmless_harmful} without the system prompt. Specifically these utilize the prompts from \url{https://huggingface.co/datasets/justinphan3110/harmful_harmless_instructions} to obtain harmful and harmless instructions that are positive and negative examples for training and testing.

\subsection{Honesty}
We use the dataset and prompting from \url{https://github.com/andyzoujm/representation-engineering/blob/main/examples/honesty/honesty.ipynb} \citep{representation_engineering}. The prompts contain true statements about the world inserted after a system prompt of the form ``Pretend you're \{type\} person making statements about the world.'' 
\begin{center}
\fbox{
\parbox{0.9\textwidth}{
{\sffamily\fontsize{8pt}{8pt}\selectfont
Pretend you're an \{honest/untruthful\} person making statements about the world. \textit{\{Partial truthful statement\}}.
}
}
}
\end{center}

\section{Hyperparameters}
\label{appsec:hyperparameters}

\Cref{tab:hyp} lists all hyperparameters used by \ouralgolong (\ouralgo) in the experiments described in \Cref{sec:eval}. These settings are consistent across all evaluated foundation models.

\begin{table}[h!]
\centering
\caption{LLM Sampling and Algorithm Parameters}
\label{tab:hyp}
\begin{tabular}{@{}llr@{}}
\toprule
\textbf{Category} & \textbf{Hyperparameter} & \textbf{Value} \\
\midrule
\multirow{2}{*}{LLM Sampling} & Temperature & 0.7 \\
 & Max tokens per response & 1000 \\
\midrule
\multirow{5}{*}{Task Generation} & Number of generations & 5000 \\
 & Max generation reflections & 5 \\
 & Number of nearest neighbors for novelty check & 5 \\
 & Number of nearest neighbors for context & 10 \\
\midrule
\multirow{3}{*}{Agent Evaluation} & Evaluation agent type & Chain-of-thought \\
 & Evaluation $n$-shot & 5 \\
 & Evaluation succeed threshold & 60\% \\
\bottomrule
\end{tabular}
\end{table}

For visualization and clustering, we use sklearn~\citep{scikit-learn} for t-SNE, and HDBSCAN~\citep{mcinnes2017hdbscan} from \url{https://github.com/scikit-learn-contrib/hdbscan} which is released under a BSD 3-Clause License. We used these additional hyperparameters:

\begin{table}[h!]
\centering
\caption{t-SNE and HDBSCAN Hyperparameters}
\label{tab:tsne_dbscan_hyp}
\begin{tabular}{@{}llr@{}}
\toprule
\textbf{Category} & \textbf{Hyperparameter} & \textbf{Value} \\
\midrule
\multirow{7}{*}{t-SNE} & n\_components & 2 \\
 & perplexity & 50 \\
 & learning\_rate & 200 \\
 & n\_iter & 3000 \\
 & init & pca \\
 & random\_state & 42 \\
 & early\_exaggeration & 6.0 \\
\midrule
\multirow{5}{*}{HDBSCAN} & min\_cluster\_size & 16 \\
 & min\_samples & 4 \\
 & cluster\_selection\_epsilon & 2 \\
 & cluster\_selection\_method & eom \\
 & metric & euclidean \\
\bottomrule
\end{tabular}
\end{table}

\subsection{Cost of Experiments}
\label{appsubsec:cost}
The total cost for our experiments was \$450 USD for the GPT-4o-GPT-4o experiments, so approximately 10 cents per generation.
We saw a small decrease in cost for our GPT-4o-Llama3-8B experiments due to the lower cost of the subject model, while our Sonnet 3.5-GPT-4o experiments were approximately 50\% more expensive.

\clearpage

\section{Additional Experimental Results}
\label{appsec:add_exp}

\subsection{Additional Analysis}
\label{appsubsec:gpt4o_analysis}

First, we show that even after thousands of generations, \ouralgo can find novel tasks by plotting the rolling success rate for the three different scientist-subject combinations we explore in \Cref{sec:eval}.
The very gradual decline in success rate suggests that new tasks continue to emerge even after thousands of generations, illustrating the open-ended nature of our approach.

\begin{figure}[h!]
\centering
\includegraphics[width=0.99\textwidth]{figures/discovery_rate_by_generation.pdf}
\caption{\textbf{Average Task Discovery Rate by Generation Number.} Even after thousands of generations, \ouralgo continues discovering novel tasks, indicating ongoing exploration of the subject model's capabilities. Each subplot corresponds to a different scientist-subject pairing: \textcolor{JungleGreen}{\textbf{(left)}} GPT-4o-GPT-4o, \textcolor{Orange}{\textbf{(middle)}} GPT-4o-Llama3-8B, and \textcolor{Cerulean}{\textbf{(right)}} Sonnet~3.5-GPT-4o.}
\label{fig:discovery_rate_by_generation}
\end{figure}

\Cref{fig:gpt4o_3seeds} illustrates how \ouralgo\ discovers tasks when GPT-4o serves as both the scientist and the subject, across three different random seeds.
Each point on the plot represents a discovered task (with each seed shown in a different color), visualized via t-SNE.
Despite variations in random initialization and sampling, the distribution of discovered capabilities remains largely consistent despite stochastic FM sampling.
This consistency suggests that \ouralgo produces a stable ``capability signature'' of tasks given a fixed scientist and subject, even when restarted from different seeds.
Here, each seed was run for a smaller trial of 500 generations for computational cost reasons.

\begin{figure}[h!]
\centering
\includegraphics[width=0.6\textwidth]{figures/gpt4o_3seeds.pdf}
\caption{\textbf{Comparison of Discovered Tasks Across Three Seeds for GPT-4o.} We visualize tasks generated by \ouralgo\ under three random seeds (each color denotes a different seed). Despite minor differences in the exact tasks, the overall distribution of discovered capabilities remains roughly consistent, indicating that \ouralgo can generate stable ``capability signatures'' given the same subject. Each run used a lower 500 generations.}
\label{fig:gpt4o_3seeds}
\end{figure}

In \Cref{fig:success_rates_and_task_counts}, we show how the FM-judge and human-estimated success varies by human-estimated difficulty.
Whilst the FM-judge success rate does not vary significantly with the user-estimated difficulty, the human-estimated success rate drops steeply with estimated difficulty.
This complements the class-balanced F1 graphs in \Cref{fig:human_eval}.
Meanwhile, \ouralgo can discover tasks in each difficulty category, suggesting that \ouralgo can suitably adapt task difficulty in response to the subject model's capability.

\begin{figure}[h!]
\centering
\includegraphics[width=0.99\textwidth]{figures/gpt_4o_success_rates_and_task_counts.pdf}
\caption{
\textbf{Automated Success Rates and Task Distribution by Human Estimated Difficulty.}
\textbf{(a-b)} Automated FM-Judge and Human Estimated success rates with 95\% confidence intervals across different difficulty levels. The overall success rate is indicated by the dashed line.
\textbf{(c)} Number of tasks categorized by difficulty level. Interestingly, this approximately follows a normal distribution.
}
\label{fig:success_rates_and_task_counts}
\end{figure}

\subsection{Additional Visualizations for Llama3-8B as Subject}
\label{appsubsec:llama_vis}

We provide additional visualizations for the GPT-4o-Llama3-8B setting in \Cref{subsec:varying_subject}.
In \Cref{fig:llama_subject_embedding}, we compare embeddings of tasks proposed by GPT-4o as scientist when it evaluates itself \textcolor{Cerulean}{\textbf{blue}} versus when it evaluates Llama3-8B as the subject \textcolor{Orange}{\textbf{orange}}.
While some clusters overlap significantly, Llama3-8B fails more often on tasks requiring multi-step logic or advanced reasoning as shown in \Cref{appsubsec:llama_manual_selection}.
Therefore, \ouralgo is able to adaptively explore areas of potential failure in the subject model.

\begin{figure}[h!]
\centering
\includegraphics[width=0.6\textwidth]{figures/gpt4o_vs_llama_subject.pdf}
\caption{
\textbf{Task Distribution for GPT-4o-as-Scientist with Two Different Subjects.}
We show 2D t-SNE embeddings of tasks generated by GPT-4o when evaluating itself
(\textcolor{Cerulean}{\textbf{blue}}) versus Llama3-8B
(\textcolor{Orange}{\textbf{orange}}).
Although these clusters share some overlap, Llama3-8B exhibits significantly higher failure rates on tasks requiring multi-step logic and more advanced reasoning.
Consequently, \ouralgo is able to adaptively probe the failure modes of weaker models.
}
\label{fig:llama_subject_embedding}
\end{figure}

\section{Examples of Discovered Tasks}
\label{appsec:task_examples}

This section contains an overview and selected tasks discovered by \ouralgo across various models. 
We found that the discovered tasks span a broad range of complexity, from basic text transformations to advanced domain-specific challenges such as cryptography, linguistics, and complex puzzle-solving. 
By carefully analyzing these tasks, we gain insight into under-recognized capabilities of LLMs and their potential blind spots.

As a reminder, all agents are evaluated using chain-of-thought as described in \Cref{appsubsec:evaluation_prompts}.
\emph{Note: Foundation model sampling is stochastic and reproductions will vary. Furthermore, the complete archive of discovered tasks for all our evaluation settings are available on our GitHub repository at \url{https://github.com/conglu1997/ACD/tree/main/results}, although only a representative subset is highlighted here.}

\subsection{Listing of Discovered Clusters}
\label{appsubsec:cluster_listings}

We present in Tables~\ref{tab:gpt4o_gpt4o_clusters}, \ref{tab:gpt4o_llama_clusters}, and \ref{tab:sonnet_gpt4o_clusters} the primary clusters discovered by \ouralgo\ for three different scientist–subject configurations:
\begin{enumerate}[leftmargin=2em]
    \item GPT-4o Scientist on GPT-4o Subject
    \item GPT-4o Scientist on Llama3-8B Subject
    \item Sonnet 3.5 Scientist on GPT-4o Subject
\end{enumerate}
Each table is sorted in descending order of the total number of tasks in that cluster, and we additionally report the cluster-wide automated FM-judge success rate of the subject model.

\vspace{1em}
\begin{table}[h!]
\centering
\caption{Discovered Clusters for \emph{GPT-4o Scientist on GPT-4o Subject}. Refer to the main paper \Cref{subsec:gpt4_eval}.}
\label{tab:gpt4o_gpt4o_clusters}
\resizebox{\textwidth}{!}{\begin{tabular}{r l r r}
\toprule
\textbf{ID} & \textbf{Cluster Name} & \textbf{Total Tasks} & \textbf{Success Rate (\%)} \\
\midrule
1 & Creative generation, logic puzzles, and computational reasoning & 185 & 89.7 \\
2 & Puzzle-solving and creation involving logic, language, and geometry & 104 & 70.2 \\
3 & Visual and Sensory Interpretation and Description & 72 & 97.2 \\
4 & Musical composition, notation, and analysis tasks & 72 & 86.1 \\
5 & Creative storytelling with constraints and narrative coherence & 69 & 94.2 \\
6 & Scientific reasoning, hypothesis generation, and experiment design tasks & 69 & 98.6 \\
7 & Dialogue generation, emotional intelligence, and social interaction scenarios & 61 & 100.0 \\
8 & Code generation, debugging, and algorithm design tasks & 60 & 95.0 \\
9 & Historical analysis, narrative generation, and alternative scenario creation & 60 & 98.3 \\
10 & Spatial manipulation, navigation, and transformation tasks & 51 & 74.5 \\
11 & Linguistic Creativity, Idioms, and Cultural Translation & 49 & 85.7 \\
12 & Data Interpretation, Analysis, and Synthesis across Domains & 49 & 77.6 \\
13 & Strategic Planning and Ethical Decision-Making Scenarios & 48 & 91.7 \\
14 & Legal text interpretation, argumentation, and contract drafting & 41 & 100.0 \\
15 & Argumentation, reasoning, and philosophical analysis tasks & 41 & 90.2 \\
16 & Humor generation and interpretation across contexts & 41 & 87.8 \\
17 & Poetry Generation, Interpretation, and Analysis & 40 & 97.5 \\
18 & Metaphor and Analogy Generation and Interpretation & 34 & 100.0 \\
19 & Step-by-step procedural generation and troubleshooting instructions & 34 & 97.1 \\
20 & Culinary recipe generation, modification, and analysis & 32 & 100.0 \\
21 & Advanced mathematical reasoning and multi-step problem-solving & 31 & 54.8 \\
22 & Visual and Geometric Pattern Recognition and Generation & 25 & 84.0 \\
23 & Game design, rule creation, and strategy development & 22 & 81.8 \\
24 & Mathematical and Logical Proof Construction and Verification & 21 & 90.5 \\
25 & Diagram generation, mechanical and UI design, spatial interpretation & 19 & 89.5 \\
\bottomrule
\end{tabular}}
\end{table}

\vspace{1em}
\begin{table}[h!]
\centering
\caption{Discovered Clusters for \emph{GPT-4o Scientist on Llama3-8B Subject}. Refer to the main paper \Cref{subsec:varying_subject}.}
\label{tab:gpt4o_llama_clusters}
\resizebox{\textwidth}{!}{\begin{tabular}{r l r r}
\toprule
\textbf{ID} & \textbf{Cluster Name} & \textbf{Total Tasks} & \textbf{Success Rate (\%)} \\
\midrule
1 & Creative and Technical Generation Across Modalities & 145 & 70.3 \\
2 & Puzzle solving and creation across logic, math, and language & 100 & 28.0 \\
3 & Historical analysis, narratives, and speculative adaptations & 86 & 90.7 \\
4 & Visual and Sensory Descriptions and Interpretations & 82 & 90.2 \\
5 & Dialogue and emotional scenario simulation & 74 & 89.2 \\
6 & Code generation, debugging, and algorithm design tasks & 63 & 92.1 \\
7 & Creative and Constrained Fictional Storytelling & 62 & 91.9 \\
8 & Ethical, Logical, and Persuasive Argumentation & 61 & 93.4 \\
9 & Mathematical problem-solving, proof generation, and modeling tasks & 59 & 42.4 \\
10 & Music composition, analysis, and notation generation & 57 & 49.1 \\
11 & Spatial and Geometric Design and Description Tasks & 54 & 53.7 \\
12 & Idiomatic Translation, Interpretation, and Cultural Adaptation & 44 & 65.9 \\
13 & Data structuring, analysis, and visualization tasks & 44 & 68.2 \\
14 & Poetry and Song Lyrics Generation and Analysis & 44 & 84.1 \\
15 & Technical Design and Creative Documentation Tasks & 43 & 90.7 \\
16 & Humor and Joke Generation with Analysis & 43 & 90.7 \\
17 & Legal Document Drafting and Interpretation & 42 & 88.1 \\
18 & Analogy and Metaphor Creation and Interpretation & 41 & 87.8 \\
19 & Scientific and technical concept explanation and application & 39 & 87.2 \\
20 & Strategic Decision-Making and Planning Across Scenarios & 38 & 71.1 \\
21 & Scientific Hypothesis Generation and Experiment Design & 34 & 97.1 \\
22 & Recipe generation and adaptation with constraints & 28 & 89.3 \\
23 & Event Scheduling, Planning, and Temporal Reasoning & 26 & 46.2 \\
24 & Step-by-Step Instruction and Tutorial Generation & 25 & 68.0 \\
25 & Pattern recognition, extension, and generation across domains & 25 & 60.0 \\
26 & Text transformation and stylistic adaptation tasks & 18 & 88.9 \\
27 & Cultural Content Creation and Adaptation & 18 & 100.0 \\
\bottomrule
\end{tabular}}
\end{table}

\vspace{1em}
\begin{table}[h!]
\centering
\caption{Discovered Clusters for \emph{Sonnet 3.5 Scientist on GPT-4o Subject}. Refer to the main paper \Cref{subsec:varying_scientist}.}
\label{tab:sonnet_gpt4o_clusters}
\resizebox{\textwidth}{!}{\begin{tabular}{r l r r}
\toprule
\textbf{ID} & \textbf{Cluster Name} & \textbf{Total Tasks} & \textbf{Success Rate (\%)} \\
\midrule
1 & Creative interdisciplinary design and analysis across multiple domains & 382 & 90.8 \\
2 & Ethics-AI-Neuroscience Interdisciplinary System Design and Analysis & 183 & 91.3 \\
3 & Quantum Biology and Computational System Design & 179 & 87.7 \\
4 & AI-driven interdisciplinary music composition and analysis & 174 & 91.4 \\
5 & Quantum-Inspired Linguistic Models and Applications & 172 & 82.0 \\
6 & Interdisciplinary Ecosystem and Climate AI Modeling Tasks & 127 & 90.6 \\
7 & Quantum-Inspired Cognitive and Neural System Design & 123 & 93.5 \\
8 & AI metaphor generation and cross-cultural cognitive linguistics & 108 & 95.4 \\
9 & AI systems for neurolinguistic language acquisition and translation & 102 & 92.2 \\
10 & Language Evolution Simulation and Modeling Across Contexts & 91 & 93.4 \\
11 & Emotional and Cultural AI Communication Systems & 71 & 93.0 \\
12 & Mathematical and Cognitive Music Composition and Analysis & 66 & 86.4 \\
13 & AI systems for neurocognitive and cultural art generation & 62 & 93.5 \\
14 & Synesthesia-inspired AI and cross-modal system design & 55 & 90.9 \\
15 & Constructed Language Design and Analysis Across Domains & 54 & 87.0 \\
16 & Biomimetic Design and Sustainable Engineering Solutions & 53 & 84.9 \\
17 & AI-driven ancient language and civilization reconstruction & 49 & 81.6 \\
18 & Synthetic Biology and AI Ethical Design Challenges & 43 & 86.0 \\
19 & Conceptual Blending in AI and Interdisciplinary Applications & 43 & 95.3 \\
20 & Cognitive and linguistic-inspired language design for AI and programming & 42 & 90.5 \\
21 & Cognitive and Cultural Narrative AI Design & 42 & 97.6 \\
22 & Mathematical-Linguistic Systems and Interdisciplinary Representation Design & 41 & 78.0 \\
23 & Quantum Algorithm Design and Interdisciplinary Applications & 37 & 83.8 \\
24 & AI systems exploring linguistic relativity and cognitive effects & 36 & 97.2 \\
25 & Cognitive and Linguistic AI Model Design and Analysis & 36 & 91.7 \\
26 & Bio-inspired computing and DNA-based system design & 35 & 80.0 \\
27 & AI consciousness and artificial self-awareness design & 35 & 91.4 \\
28 & Designing Alien Communication and Language Systems & 32 & 93.8 \\
29 & AI for visual-linguistic abstraction and cross-modal integration & 32 & 84.4 \\
30 & Quantum-inspired systems for climate, biology, and ecosystems & 29 & 96.6 \\
31 & Quantum and Post-Quantum Cryptographic System Design and Analysis & 28 & 75.0 \\
32 & AI-driven cross-cultural linguistic adaptation and translation systems & 27 & 100.0 \\
33 & Linguistic, Historical, and Cultural Cryptographic System Design & 27 & 77.8 \\
34 & Creative and interdisciplinary puzzle design and reasoning & 25 & 80.0 \\
35 & Quantum-inspired music composition and cognitive modeling & 24 & 79.2 \\
36 & Biomimetic AI and Robotics System Design & 22 & 86.4 \\
37 & Quantum-inspired narrative creation and analysis & 21 & 95.2 \\
38 & Cross-cultural idiom and proverb creation with AI integration & 21 & 100.0 \\
39 & Semantic networks and spaces for AI and language tasks & 20 & 100.0 \\
40 & Quantum-inspired creativity, cognition, and art integration & 20 & 90.0 \\
41 & Counterfactual History and Technological Impact Analysis & 18 & 83.3 \\
42 & Biomimetic AI for Environmental and Sustainability Solutions & 18 & 94.4 \\
43 & Exoplanet systems design, AI, and astrobiological exploration & 18 & 94.4 \\
44 & Embodied Multimodal Communication Systems Design & 17 & 88.2 \\
45 & Abstract Concept Translation Across Modalities and Frameworks & 17 & 100.0 \\
46 & AI-driven societal and historical modeling and prediction & 16 & 81.2 \\
\bottomrule
\end{tabular}}
\end{table}

\clearpage

\subsection{Manual Selection of Surprising Tasks}
\label{appsubsec:manual_selection}

Although our system is capable of automatically generating and evaluating a vast number of tasks, identifying which of these tasks are genuinely ``surprising'' remains somewhat subjective. 
We initially experimented with two main more scalable approaches to filter for surprising tasks:
\begin{enumerate}
\item \textbf{Crowd-based filtration:} We asked crowd-sourced human participants from CloudResearch to mark which tasks they found surprising. However, we observed that around 37\% of tasks were marked as surprising, suggesting that even those who have used LLMs did not share a clear notion of what ``should'' be within an LLM's abilities. This broad labeling provided insufficient filtering for our goal of highlighting truly unexpected successes or failures.
\item \textbf{Model-based filtration:} We also attempted to ask the model itself to predict which tasks humans would consider surprising. These predictions did not consistently align with our perspectives. Sometimes the model flagged trivial successes as surprising, or conversely missed failure modes we found surprising.
\end{enumerate}

Given these challenges, we ultimately opted for a \emph{manual} curation step led by the authors.
While this approach is more labor-intensive and subjective, it allowed us to present a set of especially illustrative or counterintuitive results, which we believed would be of broad interest to the machine learning community.
In future work, we anticipate further improvements in automated or semi-automated approaches to better capture and quantify ``surprisingness.''
For each presented example, we did a web search to ensure that they were not trivially memorized from some popular training data.

\subsection{GPT-4o As Both Scientist and Subject}
\label{appsubsec:gpt4o_manual_selection}

In this configuration, GPT-4o both generates tasks (acting as the ``scientist'') and then attempts them (acting as the ``subject''), as reported in the main paper \Cref{subsec:gpt4_eval}. Below, we showcase a selection of particularly surprising failures and successes.

\subsubsection{Selected Failures}
\label{appsubsubsec:gpt4o_surprising_failures}

\begin{tcolorbox}[breakable,boxrule=0.5pt,sharp corners,fontupper=\small,
colback=red!10, colframe=red!50!black, title={Failure: Missing Last Text Transformation}]
\textbf{Instructions}:

Your task is to perform the following text transformations on the given input text.

\textit{Input Text}: The quick brown fox jumps over the lazy dog.

\textit{Transformations}:
\begin{enumerate}
    \item Reverse the order of words.
    \item Capitalize the first letter of each word.
    \item Replace 'dog' with 'cat'.
\end{enumerate}

Provide your response in the following format:

\textit{Transformed Text}: [Your transformed text]\\

\textbf{GPT-4o's Response}:

\textit{Transformed Text}: Dog. Lazy The Over Jumps Fox Brown Quick The
\end{tcolorbox}

\begin{tcolorbox}[breakable,boxrule=0.5pt,sharp corners,fontupper=\small,
colback=red!10, colframe=red!50!black, title={Failure: Incorrect Result from Arithmetic Sequence}]
\textbf{Instructions}:

You are given a sequence of logical arithmetic puzzles. Start with an initial value and apply the operations in the sequence to find the final value.

\textit{Initial Value}: 5

\textit{Puzzle Sequence}:
\begin{itemize}
    \item \{'operation': 'add', 'operand': 3\}
    \item \{'operation': 'multiply', 'operand': 2\}
    \item \{'operation': 'subtract', 'operand': 4\}
\end{itemize}

Provide your final value as the solution in the following format:

\textit{Final Value}: [Your final value]\\

\textbf{GPT-4o's Response}:

\textit{Final Value}: 10
\end{tcolorbox}

\begin{tcolorbox}[breakable,boxrule=0.5pt,sharp corners,fontupper=\small,
colback=red!10, colframe=red!50!black, title={Failure: Incorrect Continuation of Visual Pattern}]
\textbf{Instructions}:

Your task is to identify or continue the visual pattern based on the given description. Here is the pattern: '\#-\#\#--\#\#\#---'. Identify the next three symbols in the sequence. Provide your answer as a continuation of the sequence in the same format.\\

\textbf{GPT-4o's Response}:

----\#\#\#\#
\end{tcolorbox}

\subsubsection{Selected Successes}
\label{appsubsubsec:gpt4o_surprising_successes}

%

%

%

%
%
%
%
%
%
%
%
%
%
%
%

%

%
%
%
%
%
%
%
%
%
%
%
%
%

\begin{tcolorbox}[breakable,boxrule=0.5pt,sharp corners,fontupper=\small,
colback=green!10, colframe=green!50!black, title={Success: A Riddle for a Scientific Concept}]
\textbf{Instructions}:

You are tasked with creating a riddle based on a given scientific concept. The concept you need to create a riddle for is: gravity.

Provide your response in the following format:
\begin{enumerate}
    \item Riddle: [Create an engaging riddle that accurately reflects the scientific concept.]
    \item Solution: [Provide the solution to the riddle.]\\
\end{enumerate}

\textbf{GPT-4o's Response}:

\begin{enumerate}
    \item Riddle: I pull you close but never push, though you feel me, I lie in hush. From the tallest mountain to the deepest sea, I keep you grounded, who am I to thee?
    \item Solution: Gravity
\end{enumerate}
\end{tcolorbox}

\begin{tcolorbox}[breakable,boxrule=0.5pt,sharp corners,fontupper=\small,
colback=green!10, colframe=green!50!black, title={Success: Correct Magic Square Solution}]
\textbf{Instructions}:

Your task is to solve the visual puzzle based on the given description. Provide a clear and logical solution that adheres to the constraints specified in the description.

\textit{Puzzle Description}:

You have a 3x3 grid. Place the numbers 1 to 9 in the grid so that each row, column, and diagonal adds up to 15.

Provide your solution in the following format:

\textit{Solution}: [Your solution here, formatted as necessary]

For example, for a 3x3 grid, you can format your solution as:
\begin{verbatim}
1 2 3
4 5 6
7 8 9
\end{verbatim}

\textbf{GPT-4o's Response}:
\begin{verbatim}
8 1 6
3 5 7
4 9 2
\end{verbatim}

\end{tcolorbox}

\begin{tcolorbox}[breakable,boxrule=0.5pt,sharp corners,fontupper=\small,
colback=green!10, colframe=green!50!black, title={Success: Creative Card Game Rules}]
\textbf{Instructions}:

You are tasked with designing rules for a simple game based on the following parameters: A card game for 2-4 players involving a deck of 52 cards. The game should have a clear win condition and basic rules for gameplay.

Ensure your game rules are clear, coherent, and engaging. Provide your response in the following format:

\textit{Game Rules}: [Your detailed game rules here]\\

\textbf{GPT-4o's Response}:

\textit{Game Rules}:
\begin{enumerate}
    \item Setup:
    \begin{itemize}
        \item Use a standard 52-card deck.
        \item Each player is dealt 5 cards.
        \item Remaining cards form the draw pile; flip the top card to start the discard pile.
    \end{itemize}
    \item Objective:
    \begin{itemize}
        \item Be the first to discard all your cards.
    \end{itemize}
    \item Gameplay:
    \begin{itemize}
        \item Players take turns clockwise.
        \item On your turn, play a card matching the rank or suit of the top discard.
        \item If unable to play, draw one card. If playable, you may play it immediately; otherwise, your turn ends.
    \end{itemize}
    \item Special Cards:
    \begin{itemize}
        \item Aces (A): Reverse the direction of play.
        \item Kings (K): Skip the next player's turn.
        \item Queens (Q): Choose the next suit to be played.
        \item Jacks (J): Next player draws two cards unless they play another Jack.
    \end{itemize}
    \item Winning:
    \begin{itemize}
        \item First player with no cards left wins.
        \item If the draw pile is exhausted, reshuffle the discard pile to continue.
    \end{itemize}
\end{enumerate}

\end{tcolorbox}

\begin{tcolorbox}[breakable,boxrule=0.5pt,sharp corners,fontupper=\small,
colback=green!10, colframe=green!50!black, title={Success: Detailed 3D Object Description from 2D Projections}]
\textbf{Instructions}:

You are given 2D projections of an object. Your task is to visualize and describe the 3D object based on these projections. Ensure that your description is detailed and includes the shape, dimensions, and any other relevant features.

\textit{2D Projections}:
\begin{itemize}
    \item Front view: A square base with a triangular top. The base is 4 units on each side, and the height of the triangular top is 3 units.
    \item Side view: A rectangle with a triangular top. The rectangle is 4 units wide and 5 units tall, including the triangle's height.
    \item Top view: A square with 4 units on each side.
\end{itemize}

Provide your 3D description below in the following format:

\textit{3D Description}: [Your detailed description here]\\

\textbf{GPT-4o's Response}:

\textit{3D Description}: The object is a square pyramid with a base that is a square measuring 4 units on each side. The pyramid has four triangular faces converging at a single apex point directly above the center of the base. The height of the pyramid from the base to the apex is 3 units. The total height of the object is 5 units when including the base's dimensions. Each triangular face is an isosceles triangle with a base of 4 units and a height of 3 units. The object is symmetrical along all axes, forming a perfect pyramid shape.
\end{tcolorbox}

\begin{tcolorbox}[breakable,boxrule=0.5pt,sharp corners,fontupper=\small,
colback=green!10, colframe=green!50!black, title={Success: Einstein's Riddle Variant Solved}]
\textbf{Instructions}:

Your task is to solve the following logical puzzle. Use the clues provided to determine the answers to the questions.

Clues:

1. There are five houses in a row in different colors.

2. The Englishman lives in the red house.

3. The Spaniard owns the dog.

4. Coffee is drunk in the green house.

5. The Ukrainian drinks tea.

6. The green house is immediately to the right of the ivory house.

7. The Old Gold smoker owns snails.

8. Kools are smoked in the yellow house.

9. Milk is drunk in the middle house.

10. The Norwegian lives in the first house.

11. The man who smokes Chesterfields lives in the house next to the man with the fox.

12. Kools are smoked in the house next to the house where the horse is kept.

13. The Lucky Strike smoker drinks orange juice.

14. The Japanese smokes Parliaments.

15. The Norwegian lives next to the blue house.

Questions:

Who drinks water? Who owns the zebra?

Provide your answers in the following format:

Water: [Your answer]

Zebra/Fish: [Your answer]\\

\textbf{GPT-4o's Response}:

Water: Norwegian\\
Zebra/Fish: Japanese
\end{tcolorbox}

\subsection{GPT-4o As Scientist and Llama3-8B as Subject}
\label{appsubsec:llama_manual_selection}

We now consider the scenario where GPT-4o (scientist) probes Llama3-8B (subject), as described in the main paper \Cref{subsec:varying_subject}.
We observe that Llama3-8B often struggles with more complex or multistep instructions, sometimes repeating incorrect patterns or failing to detect internal inconsistencies. 
Below, we highlight several examples of these failures, which are particularly instructive in understanding the limits of smaller models.

\subsubsection{Selected Failures}

\begin{tcolorbox}[breakable,boxrule=0.5pt,sharp corners,fontupper=\small,
colback=red!10, colframe=red!50!black, title={Failure: Inaccurate 4x4 Sudoku Solution}]
\textbf{Instructions}: 

Solve the following 4x4 Sudoku puzzle. Each row, column, and 2x2 subgrid must contain the numbers 1 to 4 exactly once. Represent the grid as a list of lists.
\begin{verbatim}
1 0 0 0
0 0 0 2
0 3 0 0
0 0 4 0
\end{verbatim}

Submit your solution as a list of lists, where each inner list represents a row in the grid. For example, 
\begin{verbatim}
1 2 3 4
3 4 1 2
2 3 4 1
4 1 2 3
\end{verbatim}
\textbf{Llama3-8B's Response}:
\begin{verbatim}
1 2 3 4
4 1 2 3
2 3 4 1
3 4 1 2
\end{verbatim}

\end{tcolorbox}

\begin{tcolorbox}[breakable,boxrule=0.5pt,sharp corners,fontupper=\small,
colback=red!10, colframe=red!50!black, title={Failure: Incorrect Financial Calculation}]
\textbf{Instructions}: 

Solve the following financial calculation problem based on the given scenario:\\

Calculate the monthly payment for a car loan. The loan amount is \$20,000, the annual interest rate is 5\%, and the loan term is 5 years. Use the formula: M = P[r(1+r)\^n]/[(1+r)\^n-1], where M is the monthly payment, P is the loan amount, r is the monthly interest rate, and n is the number of payments. Note: Convert the annual interest rate to a monthly rate by dividing by 12. Example: For an annual rate of 5\%, the monthly rate is 0.05/12.\\

Submit your solution as a plain text string in the following format:\\

Solution: [Your solution]\\

\textbf{Llama3-8B's Response}:

To solve this problem, I will first convert the annual interest rate to a monthly rate by dividing by 12. Then, I will use the formula to calculate the monthly payment.\\

The annual interest rate is 5\%, so the monthly interest rate is 0.05/12 = 0.004167.\\

The loan term is 5 years, so the number of payments is 5 years x 12 months/year = 60 months.\\

Now, I can plug in the values into the formula:\\

M = P[r(1+r)\textasciicircum n]/[(1+r)\textasciicircum n-1]\\
M = 20000[0.004167(1+0.004167)\textasciicircum 60]/[(1+0.004167)\textasciicircum 60-1]\\
M = 20000[0.004167(1.004167)\textasciicircum 60]/[(1.004167)\textasciicircum 60-1]\\
M = 20000[0.004167(1.27631)]/[1.27631-1]\\
M = 20000[0.005293]/0.27631\\
M = 20000 * 0.0191\\
M = 382.00\\

Therefore, the monthly payment for the car loan is \$382.00.\\

Solution: \$382.00
\end{tcolorbox}

\begin{tcolorbox}[breakable,boxrule=0.5pt,sharp corners,fontupper=\small,
colback=red!10, colframe=red!50!black, title={Failure: Overcomplicated Math Word Problem}]
\textbf{Instructions}: 

Create a mathematical word problem based on the following constraints: Include addition and subtraction within the context of a shopping scenario. Ensure the problem involves at least three mathematical operations. Ensure that the problem is coherent, logically consistent, and solvable with elementary arithmetic operations. Provide the word problem and its solution in a step-by-step manner in the following format:\\

Word Problem: [Your word problem]\\
Solution: [Step-by-step solution to your word problem]\\

\textbf{Llama3-8B's Response}:

Word Problem:\\
Tommy is a grocery store clerk. A customer, Mrs. Johnson, wants to buy some items for her party. She buys 12 boxes of cookies, each containing 8 cookies. She also buys 5 jars of sauce, and 15 boxes of crackers. However, during checkout, Tommy realizes that 3 jars of sauce are damaged and need to be returned. Additionally, Tommy finds that the store has a discount of \$2 on every 2 boxes of crackers. If a box of cookies costs \$3, a jar of sauce costs \$1.5, and a box of crackers costs \$4, how much will Mrs. Johnson pay in total after all the calculations?\\

Solution:\\
1. First, let's find the total cost of the cookies. There are 12 boxes of cookies, each containing 8 cookies, so the total number of cookies is 12 * 8 = 96.\\
2. The cost of each cookie is \$3, so the total cost of the cookies is 96 * \$3 = \$288.\\
3. Next, let's calculate the cost of the jars of sauce. Mrs. Johnson buys 5 jars, but 3 are damaged, so she has to return them. She is left with 5 - 3 = 2 jars.\\
4. The cost of each jar of sauce is \$1.5, so the total cost of the 2 jars of sauce is 2 * \$1.5 = \$3.\\
5. Now, let's find the total cost of the boxes of crackers. Mrs. Johnson buys 15 boxes, but there is a discount of \$2 on every 2 boxes. To find the number of pairs of crackers, we divide 15 by 2, which gives us 7.5 pairs. Since we can't have half a pair, we round down to 7 pairs of crackers with 1 box left over.\\
6. The cost of each pair of crackers is \$2 (half the cost of a box), so the total discount is 7 * \$2 = \$14.\\
7. The total cost of the boxes of crackers is 15 * \$4 = \$60, and after the discount, it is \$60 - \$14 = \$46.\\
8. Now, let's find the total cost of the items. The cost of the cookies is \$288, the cost of the 2 jars of sauce is \$3, and the cost of the boxes of crackers is \$46. The total cost is \$288 + \$3 + \$46 = \$337.\\
9. Finally, let's find the total amount Mrs. Johnson will pay. The total cost is \$337, and since there are no other discounts or taxes, the total amount she will pay is \$337.
\end{tcolorbox}

\begin{tcolorbox}[breakable,boxrule=0.5pt,sharp corners,fontupper=\small,
colback=red!10, colframe=red!50!black, title={Failure: Incorrect Family Tree Diagram}]
\textbf{Instructions}: 

Generate a family tree diagram based on the given family description.\\

Family Description: John and Mary have two children, Alice and Bob. Alice is married to George and they have one child, Emma. Bob is married to Sarah and they have two children, Jack and Lily.\\

Submit your response as a plain text string in the following format:\\
- Family Tree: [Your family tree here, represented hierarchically.

For example:\\
John and Mary
\begin{itemize}
\item[--]Alice
\begin{itemize}
\item[--]Emma
\end{itemize}
\item[--]Bob
\begin{itemize}
\item[--]Jack
\item[--]Lily]
\end{itemize}
\end{itemize}

Ensure that the representation clearly shows the relationships, with parents listed first, followed by their children indented under their names. Use indentation to show generations clearly.\\

\textbf{Llama3-8B's Response}:

John and Mary
\begin{itemize}
\item[--]Alice
\begin{itemize}
\item[--]Emma
\end{itemize}
\item[--]Bob
\begin{itemize}
\item[--]Jack
\item[--]Lily
\item[--]Sarah
\end{itemize}
\end{itemize}
\end{tcolorbox}

\begin{tcolorbox}[breakable,boxrule=0.5pt,sharp corners,fontupper=\small,
colback=red!10, colframe=red!50!black, title={Failure: Infinite Repetition in Logic Puzzle Deductions}]
\textbf{Instructions}: 

Solve the following logical puzzle based on the given clues:\\

Puzzle: There are five houses in a row, each with a different color. In each house lives a person with a different nationality. Each person drinks a different beverage, smokes a different brand of cigar, and keeps a different pet. The following clues are given:\\
1. The Brit lives in the red house.\\
2. The Swede keeps dogs as pets.\\
3. The Dane drinks tea.\\
4. The green house is immediately to the left of the white house.\\
5. The green house's owner drinks coffee.\\
6. The person who smokes Pall Mall rears birds.\\
7. The owner of the yellow house smokes Dunhill.\\
8. The man living in the center house drinks milk.\\
9. The Norwegian lives in the first house.\\
10. The man who smokes blends lives next to the one who keeps cats.\\
11. The man who keeps horses lives next to the man who smokes Dunhill.\\
12. The man who smokes Blue Master drinks beer.\\
13. The German smokes Prince.\\
14. The Norwegian lives next to the blue house.\\
15. The man who smokes blend has a neighbor who drinks water.\\
Who owns the fish?\\

Submit your response as a plain text string in the following format:\\
Answer: [Your answer]\\

\textbf{Llama3-8B's Response}:

To solve this puzzle, I will use a step-by-step approach to gather information and eliminate possibilities.\\

From clue 9, we know the Norwegian lives in the first house. Since the Norwegian lives next to the blue house (clue 14), the blue house must be either the second or the fifth house. But from clue 8, we know the center house drinks milk, so the Norwegian lives in the first house, and the blue house must be the fifth house.\\

From clue 4, the green house is to the left of the white house, so the green house is the second house, and the white house is the third house.\\

From clue 7, the owner of the yellow house smokes Dunhill. Since the Norwegian lives in the first house, the yellow house must be the fourth house.\\

From clue 10, the man who smokes blends lives next to the one who keeps cats. Since the green house is the second house, the man who smokes blends must live in the third house, and the man who keeps cats must live in the fourth house.\\

From clue 11, the man who keeps horses lives next to the man who smokes Dunhill. Since the yellow house is the fourth house, the man who keeps horses must live in the fifth house.\\

From clue 1, the Brit lives in the red house. Since the Norwegian lives in the first house, the red house must be the fifth house, and the Brit lives in the fifth house.\\

From clue 2, the Swede keeps dogs as pets. Since the green house is the second house, the Swede lives in the second house.\\

From clue 3, the Dane drinks tea. Since the center house drinks milk (clue 8), the Dane must live in the third house.\\

From clue 6, the person who smokes Pall Mall rears birds. Since the Norwegian lives in the first house, the person who smokes Pall Mall must live in the second house, but the Swede lives in the second house, so the person who smokes Pall Mall must live in the first house, but the Norwegian lives in the first house, so the person who smokes Pall Mall must live in the second house, but the Swede lives in the second house, so the person who smokes Pall Mall must live in the first house, but the Norwegian lives in the first house, \emph{\textbf{[repeats...]}}, so the person who smokes Pall Mall...
\end{tcolorbox}

\begin{tcolorbox}[breakable,boxrule=0.5pt,sharp corners,fontupper=\small,
colback=red!10, colframe=red!50!black, title={Failure: Missing RSA Explanation}]
\textbf{Instructions}: 

Provide a detailed explanation of the following cryptographic concept and solve the related problem:\\

Explain the concept of RSA encryption and solve the following problem: Given two prime numbers p = 61 and q = 53, calculate the public key (n, e) and the private key (d) for RSA encryption. Use e = 17.\\

Ensure that your explanation is clear, accurate, and demonstrates a deep understanding of the cryptographic principles involved. For the problem-solving part, show all steps and calculations clearly. Submit your response as a plain text string with the following format:\\

1. Explanation: [Your detailed explanation here]\\
2. Problem Solution: [Your step-by-step solution here]\\

\textbf{Llama3-8B's Response}:

n = 3233, e = 17, d = 1931
\end{tcolorbox}

\begin{tcolorbox}[breakable,boxrule=0.5pt,sharp corners,fontupper=\small,
colback=red!10, colframe=red!50!black, title={Failure: Infinite Repetition in Cryptarithm Solution}]
\textbf{Instructions}: 

Solve the following cryptarithm puzzle. Each letter represents a unique digit from 0 to 9. No two letters can represent the same digit. Provide the digit for each letter and ensure that the equation holds true.\\

Puzzle: SEND + MORE = MONEY\\

Submit your solution in the following format:\\
A = [digit], B = [digit], ..., Z = [digit]\\

For example, if the puzzle is 'AB + CD = EF' and the solution is A=1, B=2, C=3, D=4, E=5, F=6, submit your solution as:\\
A = 1, B = 2, C = 3, D = 4, E = 5, F = 6\\

\textbf{Llama3-8B's Response}:

To solve this cryptarithm puzzle, we'll break it down step by step.\\

1.  We know that the rightmost column of the addition should have a carry if the sum of the last two digits exceeds 9. In this case, 'Y' + 'E' + 'C' + any carry should end in 'Y', which means 'Y' must be 0 or the only possibility is 9 + 8 + 5 = 22 and 'Y' being in the 'ones' place would require the carry from the next column to be 1. ...

Hence we get O + 9 + any carry = 1 in the second last column which gives O + any carry = -8 which is not possible.\\

Now we know that D + 9 + any carry = 8 in the first column, and we also know that carry in the first column is discarded for the same reason as before, so any carry in the second column must be 9 (because D + 9 + 9 = 18 and 8 = 18 - 10).\\

Hence we get D + 9 = -1 which is not possible.\\

Hence we get D + 9 = -1 which is not possible. \emph{\textbf{[repeats...]}}\\

Now we know that D + 9 + any carry = 8 in the first column, and we also know that carry in the first column is discarded for the same...
\end{tcolorbox}

\begin{tcolorbox}[breakable,boxrule=0.5pt,sharp corners,fontupper=\small,
colback=red!10, colframe=red!50!black, title={Failure: Incorrect 2D Shape Rotation}]
\textbf{Instructions}: 

Complete the following task based on the given instruction and shape:\\

Instruction:\\
Rotate the given shape 90 degrees clockwise.\\

Shape:
\begin{verbatim}
1 0 0
1 1 1
0 0 1
\end{verbatim}

Your task is to:\\
1. Perform the specified transformation on the given shape.\\
2. Submit the transformed shape as a 2D array in plain text format.\\

Ensure that your response is accurate and follows the given instruction.\\

\textbf{Llama3-8B's Response}:
\begin{verbatim}
1 1 0
0 1 1
0 0 1
\end{verbatim}

\end{tcolorbox}

\subsection{Claude Sonnet 3.5 As Scientist and GPT-4o as Subject}
\label{appsubsec:claude_manual_selection}

Finally, we show selected tasks discovered by \ouralgo when using Claude Sonnet 3.5 as the scientist model and GPT-4o as the subject model, as discussed in the main paper \Cref{subsec:varying_scientist}.

\subsubsection{Selected Failures}

\begin{tcolorbox}[breakable,boxrule=0.5pt,sharp corners,fontupper=\small,
colback=red!10, colframe=red!50!black, title={Failure: Not Following Instructions for Substitution Cipher}]
\textbf{Instructions}: 

Create a simple substitution cipher based on the theme 'colors'.\\
1. Use the following words for your cipher: red, blue, green, yellow, purple, orange, pink, brown, gray, black.\\
2. Assign each letter in 'ABCDEFGHIJKLMNOPQRSTUVWXYZ' to one of these words. You may use each word multiple times.\\
3. Using your cipher, encode the message: 'THE QUICK BROWN FOX'\\
4. Provide your response in the following format:\\
Cipher key: A:word1, B:word2, C:word3, ..., Z:word10\\
Encoded message: word word word word\\

\textbf{GPT-4o's Response}:

Cipher key: A:dog, B:cat, C:fish, D:bird, E:lion, F:tiger, G:bear, H:wolf, I:fox, J:deer, K:dog, L:cat, M:fish, N:bird, O:lion, P:tiger, Q:bear, R:wolf, S:fox, T:deer, U:dog, V:cat, W:fish, X:bird, Y:lion, Z:tiger\\
Encoded message: deer wolf lion bear dog fox fish dog cat wolf lion fish bird tiger lion bird

\end{tcolorbox}

\subsubsection{Selected Successes}

\begin{tcolorbox}[breakable,boxrule=0.5pt,sharp corners,fontupper=\small,
colback=green!10, colframe=green!50!black, title={Success: Creative Conlang Analysis and Sample Creation}]
\textbf{Instructions}: 

Analyze the following sample of the constructed language Lojban:\\

Sample: mi fanva fi lo jbobau fi lo glibau\\
Explanation: Lojban is a logical language designed to be unambiguous. It uses predicate structure and particles to precisely define relationships between concepts. Words are divided into content words (brivla) and structure words (cmavo). The language aims to eliminate syntactic ambiguity.\\
Translation: I translate from Lojban to English.\\

Based on this analysis, Create a Lojban sentence describing a conditional statement, such as 'If it rains tomorrow, I will stay home.'\\

Provide your response in the following format:\\
1. Brief analysis of the language structure (3-4 sentences)\\
2. Your created sentence\\
3. English translation of your sentence\\
4. Detailed explanation of how your creation fits the language's rules (at least 3 points)\\
5. Rationale for your analysis and creation (3-4 sentences)\\

\textbf{GPT-4o's Response}:

1. Ithkuil is a language that utilizes a rich system of phonemes, consonant clusters, and diacritics to convey a high level of specificity and nuance. Each word is constructed to encapsulate multiple layers of grammatical information, including tense, aspect, mood, and case. The language's morphology is designed to be extremely precise, allowing speakers to express complex ideas and emotions succinctly.\\

2. Kräqölëx iþfâlkîl qirçük.\\

3. English translation: A wave of bittersweet nostalgia washed over him.\\

4. Detailed explanation:\\
   - Kräqölëx: This word combines elements to convey "wave" or "flowing emotion," incorporating the root "kräq" (wave) with suffixes to indicate a metaphorical, emotional wave.\\
   - iþfâlkîl: This word encapsulates the concept of "bittersweet nostalgia." The root "iþfâl" refers to nostalgia, while the diacritic and suffix "kîl" add the nuance of 'bittersweetness,' combining positivity and melancholy.\\
   - qirçük: This term denotes the action of being 'washed over' or 'engulfed.' The root "qir" (to wash) combined with "çük" (overwhelming) creates the sense of being emotionally overwhelmed.\\

5. Rationale for analysis and creation:\\
   - The constructed sentence and its components reflect Ithkuil's ability to encapsulate complex emotional states in concise words. Each word is carefully constructed to carry multiple layers of meaning, adhering to the language's principles of precision and nuance.\\
   - The use of consonant clusters and diacritics in the sentence aligns with Ithkuil's phonological structure, ensuring the sentence is true to the language's intricate sound patterns.\\
   - The translation and explanation demonstrate how specific roots and affixes are combined to express intricate ideas succinctly, showcasing Ithkuil's unique capacity for detailed expression.
\end{tcolorbox}

\begin{tcolorbox}[breakable,boxrule=0.5pt,sharp corners,fontupper=\small,
colback=green!10, colframe=green!50!black, title={Success: Mathematical Model of Subterranean Crystal Caves}]
\textbf{Instructions}: 

Create a mathematical model of the fictional ecosystem: Subterranean Crystal Caves - A network of underground caves illuminated by bioluminescent crystals, supporting a diverse ecosystem of cave-dwelling organisms.\\

Your task is to develop a simple mathematical model that captures the key interactions and dynamics of this ecosystem. Follow these steps:\\

1. Identify 3-4 key species or components in the ecosystem.\\
2. Define variables to represent the population or quantity of each component.\\
3. Create a system of differential equations that model the interactions between these components. Consider factors such as growth rates, predator-prey relationships, resource limitations, and any unique aspects of the given ecosystem.\\
4. Analyze the stability of your model by identifying equilibrium points and discussing their stability.\\
5. Predict how a sudden change (e.g., introduction of a new species, climate change) would affect the ecosystem based on your model.\\

Provide your response in the following format:\\

Components:\\
1. [Component 1]: [Brief description]\\
2. [Component 2]: [Brief description]\\
3. [Component 3]: [Brief description]\\
(4. [Component 4]: [Brief description]) (if applicable)\\

Variables:\\
- x1 = [Definition]\\
- x2 = [Definition]\\
- x3 = [Definition]\\
(- x4 = [Definition]) (if applicable)\\

Differential Equations:\\
1. dx1/dt = [Equation]\\
2. dx2/dt = [Equation]\\
3. dx3/dt = [Equation]\\
(4. dx4/dt = [Equation]) (if applicable)\\

Stability Analysis: [Your analysis of equilibrium points and their stability]\\

Predicted Response to Change: [Your prediction and explanation]\\

Ensure that your model is mathematically sound, biologically plausible, and creatively adapted to the unique features of the given ecosystem.\\

\textbf{GPT-4o's Response}:

Components:\\
1. Bioluminescent Crystals: Source of light energy that supports primary productivity.\\
2. Crystal Algae: Photosynthetic organisms that utilize light from the bioluminescent crystals.\\
3. Cave Fish: Herbivores that feed on crystal algae.\\
4. Cave Predators: Carnivores that feed on cave fish.\\

Variables:\\
- x1 = Population or quantity of bioluminescent crystals\\
- x2 = Population of crystal algae\\
- x3 = Population of cave fish\\
- x4 = Population of cave predators\\

Differential Equations:\\
1. dx1/dt = r1 * x1 - d1 * x1\\
2. dx2/dt = r2 * x2 * x1/(k1 + x1) - d2 * x2 - a * x2 * x3\\
3. dx3/dt = b * x2 * x3 - d3 * x3 - c * x3 * x4\\
4. dx4/dt = e * x3 * x4 - d4 * x4\\

Stability Analysis:\\
To find equilibrium points, set dx1/dt = 0, dx2/dt = 0, dx3/dt = 0, dx4/dt = 0.\\
Equilibrium points (E):\\
E1 = (0, 0, 0, 0) - Trivial equilibrium, likely unstable.\\
E2 = (x1\*, 0, 0, 0) - No algae, fish, or predators.\\
E3 = (x1\*, x2\*, 0, 0) - No fish or predators.\\
E4 = (x1\*, x2\*, x3\*, 0) - No predators.\\
E5 = (x1\*, x2\*, x3\*, x4\*) - Non-trivial equilibrium with all species present.\\
Stability of E5 can be analyzed using the Jacobian matrix; if all eigenvalues of the Jacobian at E5 have negative real parts, the equilibrium is stable.\\

Predicted Response to Change:\\
Introducing a new predator species (e.g., x5) adds terms to equations for x4 and introduces dx5/dt. This could destabilize the current equilibrium or shift it to a new state. Climate change reducing crystal light output decreases r2, reducing algae population (x2), subsequently impacting fish (x3) and predator (x4) populations, potentially leading to ecosystem collapse if severe.
\end{tcolorbox}

\begin{tcolorbox}[breakable,boxrule=0.5pt,sharp corners,fontupper=\small,
colback=green!10, colframe=green!50!black, title={Success: A Xenolinguistic Conlang for Resonants}]
\textbf{Instructions}: 

Design a constructed language (conlang) for the Resonants species.\\

Species characteristics:\\
- Sensory system: Echolocation-based perception with the ability to generate and modulate complex sound waves\\
- Environment: A dense, always-dark aquatic world with constantly shifting currents\\

Your task is to:\\

1. Create a basic structure for the language that utilizes the unique sensory abilities of the species and adapts to their environment. Your language design should:\\
   a) Describe the primary mode of communication\\
   b) Explain how information is encoded and transmitted\\
   c) Describe how the language accounts for environmental challenges\\

2. Address the following specific aspects in your language design:\\
   a) Create a method for expressing spatial relationships using echolocation principles\\
   b) Design a tense system that incorporates the idea of quantum superposition\\

3. Provide examples of how this language would express the following concepts:\\
   a) Time\\
   b) Quantity\\
   c) Relationship (e.g., familial, social, or spatial)\\

4. Create a sample 'sentence' or communication unit in your language, and explain its structure and meaning.\\

5. Analyze potential challenges humans might face in learning or translating this language.\\

Ensure your response is creative yet grounded in scientific principles of linguistics and the given sensory system. Organize your answer using clear headings for each section. Your total response should not exceed 750 words.\\

\textbf{GPT-4o's Response}:

1. Language Structure\\

a) Primary Mode of Communication\\
The primary mode of communication for the Chromavores species is through manipulating and perceiving quantum states of particles, particularly focusing on the spin states of electrons and photons. These quantum states can be altered and detected with high precision, allowing for complex information to be exchanged rapidly.\\

b) Information Encoding and Transmission\\
Information is encoded in the spin states of particles. For instance, different spin states correspond to different phonemes or morphemes. Changes in spin state can be detected by other Chromavores, who can "read" these changes through their advanced sensory organs. These alterations are akin to binary code in human digital communication but far more sophisticated, allowing for the encoding of complex linguistic structures.\\

c) Adapting to Environmental Challenges\\
Given the constant electromagnetic fluctuations on their gaseous planet, the language incorporates redundancy and error-correcting codes. For instance, critical information is often repeated in different quantum states (e.g., using both electrons and photons) to ensure the message is received correctly despite environmental noise.\\

2. Specific Aspects\\

a) Communicating Emotional States\\
Emotional states are communicated through variations in sound frequency and amplitude modulation. Higher frequencies and amplitudes might convey excitement or urgency, while lower frequencies and softer amplitudes could express calmness or sadness. Each emotion has a distinct signature pattern that can be recognized by other Chromavores.\\

b) Expressing Abstract Concepts\\
Abstract concepts are expressed using quantum-related metaphors. For example:\\
- Beauty might be described as a "harmonic resonance" where different particles align in a visually pleasing pattern.\\
- Justice could be referred to as "quantum equilibrium" where all particles exist in a balanced and fair state.\\

3. Examples of Expressing Concepts\\

a) Time\\
Time is expressed as a sequence of quantum state transitions. For example, a series of spin state changes might represent past, present, and future. A simple representation might be:\\
- Past: Spin state 1\\
- Present: Spin state 2\\
- Future: Spin state 3\\

b) Quantity\\
Quantity is encoded through the number of simultaneous quantum state alterations. For instance:\\
- One: Single spin state shift\\
- Two: Dual spin state shift\\
- Many: Multiple (3+) spin state shifts\\

c) Relationship\\
Relationships are depicted through spatial and contextual arrangements of quantum states. For example:\\
- Familial: Spin states that are entangled, representing a close, inseparable bond.\\
- Social: Spin states that are in proximity but not entangled, indicating social interactions.\\
- Spatial: Relative positions of particles, with closer particles indicating closer physical or conceptual relationships.\\

4. Sample Communication Unit\\

A sample 'sentence' in this language might be:\\
"Spin1-Spin2-PhotonShift-FrequencyModulation"\\

Explanation:\\
- "Spin1-Spin2" represents a temporal transition, indicating movement from past to present.\\
- "PhotonShift" encodes a specific piece of information, such as a noun or verb.\\
- "FrequencyModulation" adds emotional context, perhaps indicating excitement or urgency.\\

Meaning: "We are moving towards an exciting event."\\

5. Challenges for Humans\\

- Sensory Limitations: Humans lack the ability to perceive and manipulate quantum states directly, making it extremely challenging to both produce and understand the language.\\
- Complexity: The language's reliance on quantum states and electromagnetic modulation is far more complex than human phonetic and written systems.\\
- Environmental Sensitivity: The constant electromagnetic fluctuations would require specialized equipment for humans to even attempt communication in this environment.\\
- Abstract Metaphors: The use of quantum metaphors for abstract concepts may be difficult for humans to grasp, as these metaphors are deeply rooted in the Chromavores' unique perception of reality.\\

Despite these challenges, with advanced technology and a deep understanding of quantum mechanics, it might be possible for humans to develop interfaces that could aid in learning or translating this language.
\end{tcolorbox}

\section{Example Failures of the Automated Judge}
\label{sec:judge_failures}

As discussed in \Cref{subsec:gpt4_eval}, the automated judge has some failure rate.
We show two examples of incorrectly judged tasks here.
In both cases, the automated judge judges the task to have failed, whilst in fact, the task succeeded.
This can sometimes occur due to non-robust parsing of subject model responses.

\begin{tcolorbox}[breakable,boxrule=0.5pt,sharp corners,fontupper=\small,
colback=purple!5!white, colframe=purple!80!black, title={Judge Failure: Volume Calculation of a Frustum of a Cone}]
\textbf{Instructions}:
Calculate the volume of a frustum of a cone with a height of 10 units, and radii of 5 units and 3 units at the base and top, respectively.

Provide a detailed solution with all necessary steps and calculations. Ensure that your solution is comprehensive and accurate. Provide your response in plain text format.

\textbf{GPT-4o's Response}: \(\frac{490}{3} \pi\) cubic units

\textbf{LLM Judge}: Task Failed

\end{tcolorbox}

\begin{tcolorbox}[breakable,boxrule=0.5pt,sharp corners,fontupper=\small,
colback=purple!5!white, colframe=purple!80!black, title={Judge Failure: Solving an Algebraic Equation}]
\textbf{Instructions}:
Your task is to solve the following algebraic equation for \( x \):

Equation: \( 2x^2 + 3x - 5 = 0 \)

Provide your solution for \( x \) in plain text format. If there are multiple solutions, list all of them.

\textbf{GPT-4o's Response}: 1, -2.5

\textbf{LLM Judge}: Task Failed

\end{tcolorbox}


\section{Human Surveying Details}
\label{appsec:human_survey}

We present the script we use to perform a human evaluation of automated LLM judgments used in \ouralgo (\Cref{subsec:gpt4_eval}).
We use the CloudResearch~\citep{cloudresearch} platform and specifically target participants who have experience with using large language models such as ChatGPT.
The survey takes the form of a simple, anonymous data labeling task.
We show an example from a generated task that asks an LLM to write a short story.
In general, the task details, instructions, and LLM response will change per task.
Participants were given up to two minutes per task and paid \$0.25 USD for each task.
The average pay was \$15.79 USD per hour, with the final survey completion times.
The total cost of surveying including service fees was \$329.25.

\begin{tcolorbox}[breakable,boxrule=0.5pt,sharp corners,fontupper=\small,
colback=orange!5!white, colframe=orange!80!black, title={Human Surveying Script}]
\textbf{Instructions}

Your task is to evaluate the quality of tasks designed to measure the capabilities of a large language model. For each task, you should consider whether the questions match a specification, estimate their difficulty, and check whether an LLM got it correct. Finally, you will also be asked in your estimation, whether you find the success or failure of the LLM in this instance to be surprising from your knowledge of what LLMs can do.\\

\textbf{Task Details}

The following task is designed according to the following specifications:

\begin{lstlisting}
Description: Generate a short story based on a given prompt.
Capability Being Measured: creativity
\end{lstlisting}

The proposed task (instructions to LLM) for this specification is:
\begin{lstlisting}
Write a short story based on the following prompt: 'A young girl discovers a hidden door in her garden that leads to a magical world.'. The story should be at least 200 words long and should have a clear beginning, middle, and end.
\end{lstlisting}

\textbf{Question 1}

Does this instruction make sense and fit the task description and capability being measured?
\begin{todolist}
    \item Yes
    \item No
    \item Unsure
\end{todolist}

\textbf{Question 2}

Estimate the task difficulty:
\begin{todolist}
    \item Very Easy (Simple factual recall)
    \item Easy (Basic understanding, some inference)
    \item Moderate (Application of knowledge, multiple steps)
    \item Difficult (Analysis, synthesis, creative problem-solving)
    \item Very Difficult (Highly specialized knowledge, complex reasoning)
\end{todolist}
\vspace{2mm}

The LLM responded with:
\begin{lstlisting}
Once upon a time, in a quaint little village, there lived a young girl named Emma. Emma loved to spend her afternoons exploring her garden, which was filled with vibrant flowers, towering trees, and hidden nooks. One sunny day, while chasing a butterfly, Emma stumbled upon a curious-looking door nestled behind a thick curtain of ivy. It was small and wooden, with intricate carvings that seemed to shimmer in the light. With a heart full of curiosity, Emma gently pushed the door open and stepped through. To her amazement, she found herself in a breathtaking world where the sky was a swirling palette of colors, and the air was filled with the sweet scent of blooming flowers. Magical creatures roamed freely, from tiny fairies with iridescent wings to majestic unicorns that galloped across the fields. Emma soon befriended a mischievous fairy named Lila, who guided her through the enchanting land. Together, they embarked on numerous adventures, from soaring over rainbow waterfalls to exploring crystal caves that sparkled with every hue. Emma felt a sense of wonder and joy she had never known. As the sun began to set, casting a golden glow over the magical world, Lila led Emma back to the hidden door. "You must return home now," Lila said softly, "but remember, this world will always be here for you." With a heart full of memories, Emma stepped back through the door and found herself in her garden once more. From that day on, Emma's adventures in the magical world remained her most cherished secret. And whenever she felt the need for a bit of magic, she knew exactly where to find it.
\end{lstlisting}

\textbf{Question 3}

Did the LLM successfully complete the task? (select an option)
\begin{todolist}
    \item Yes
    \item No
    \item Unsure
\end{todolist}

\textbf{Question 4}

Did the LLM's success or failure on this task surprise you? (select an option)
\begin{todolist}
    \item Yes
    \item No
\end{todolist}

\end{tcolorbox}


\section{Report Generation}
\label{appsec:report_generation}

As discussed in the main paper \Cref{sec:report_generation}, \ouralgolong can automatically produce a structured report summarizing each discovered capability and highlighting consistent successes, failures, and key insights. Below, we provide details about our workflow for generating these reports, including how clusters and tasks are automatically aggregated and summarized.

\subsection{Task Cluster Labeling}
\label{appsubsec:task_cluster_label}

After task generation, we cluster the discovered tasks (see \Cref{appsubsec:task_embedding_prompt}) and apply t-SNE + HDBSCAN to group them. To label these clusters concisely, we prompt GPT-4o to summarize each cluster as follows:

\begin{tcolorbox}[breakable,boxrule=0.5pt,sharp corners,fontupper=\small,
colback=orange!5!white, colframe=orange!80!black, title={Task Cluster Labelling System Prompt}]
You are a helpful assistant. You are given a set of tasks within a cluster.

Reply concisely and exactly in JSON format with only the following keys:
\begin{itemize}
    \item \texttt{"thought"}: First, reason about the essence of the given tasks in the cluster.
    \item \texttt{"label"}: Your summary label for the cluster of tasks.
    \item \texttt{"capability\_being\_measured"}: The overall capability being measured by the tasks in this cluster.
\end{itemize}

This will be automatically parsed so ensure that the string response is precisely in the correct format.
\end{tcolorbox}

\begin{tcolorbox}[breakable,boxrule=0.5pt,sharp corners,fontupper=\small,
colback=orange!5!white, colframe=orange!80!black, title={Task Cluster Labelling User Prompt}]
\textbf{[DATA]}

Cluster \{\texttt{cluster\_id}\} tasks:\\

Name: \{\texttt{name\_of\_task1}\}  \\
Description: \{\texttt{description\_of\_task1}\}  \\
Capability: \{\texttt{capability\_being\_measured1}\}  \\

Name: \{\texttt{name\_of\_task2}\}  \\
Description: \{\texttt{description\_of\_task2}\}  \\
Capability: \{\texttt{capability\_being\_measured2}\}  \\

... (any additional tasks in the cluster) ...\\

\textbf{[INSTRUCTION]}

Consider the above tasks in this cluster. Please provide a concise label (a natural language phrase within 10 words) for the cluster. Ensure that the label is very specific to the tasks; avoid being general. Avoid including general terms such as "problem-solving". Include more specific keywords from the tasks, such as "poem", "logic puzzles", etc.

Also, provide the overall capability being measured by the tasks in this cluster.

Return your answer as valid JSON with only the keys \texttt{"thought"}, \texttt{"label"}, and \texttt{"capability\_being\_measured"}.
\end{tcolorbox}

These labels are then used to form summaries of the discovered tasks in our final analysis.

\subsection{Report Generation Prompts}
\label{appsubsec:report_prompts}

Below are the prompt templates used for generating the analysis sections in the final report. 
This complements the discussion in the main paper \Cref{sec:report_generation}.

\subsubsection{Cluster Analysis Prompts}

\begin{tcolorbox}[breakable,boxrule=0.5pt,sharp corners,fontupper=\small,
colback=orange!5!white, colframe=orange!80!black, title={Cluster Analysis System Prompt}]
You are an expert in designing task families to assess the capabilities of large language models (LLMs). You will write an analytical section for a report examining the capabilities and limitations of large language models.

Your goal is to analyze and synthesize insights about LLM capabilities by examining:
1) The LLM's performance and solutions on tasks designed to test specific capabilities.
2) Any patterns, strengths, or limitations revealed through this analysis.
Focus on identifying surprising successes and failures from the point of view of an expert human evaluator.

You will be given a cluster of related task families that evaluate specific LLM capabilities, along with the LLM's responses and performance on these tasks.

Your goal is to:
1) Carefully examine the example tasks and the LLM's responses
2) Analyze the LLM's proficiency level on the evaluated capabilities
3) How these examples provide meaningful insights about the model's capabilities or limitations
4) Draw meaningful conclusions about the LLM's strengths and limitations in this capability area

Respond precisely in the following format including the JSON start and end markers:\\

\textbf{THOUGHT}: \texttt{<THOUGHT>}

\textbf{RESPONSE JSON}: \texttt{<JSON>}\\

In \texttt{<THOUGHT>}, first deeply think and reason about the patterns and insights revealed by examining this cluster of related tasks.

In \texttt{<JSON>}, provide a JSON response with the following fields:
\begin{itemize}[leftmargin=2em]
    \item \texttt{"overall\_analysis"}: A brief conclusion based on examining the example tasks and the LLM's responses, including key capabilities demonstrated and limitations revealed
    \item \texttt{"surprising\_example\_analysis\_X"}: Analysis of why this success or failure was surprising and what it reveals about the LLM's capabilities or limitations (one such field per example)
    \item \texttt{"insights"}: Key insights and takeaways about the LLM's capabilities based on analyzing this cluster of related tasks
\end{itemize}

For EACH provided example, include a \texttt{"surprising\_example\_analysis\_X"} field in the JSON response, where \texttt{X} is replaced with the example's index number. This will be automatically parsed so ensure that the string response is precisely in the correct format.
\end{tcolorbox}

\begin{tcolorbox}[breakable,boxrule=0.5pt,sharp corners,fontupper=\small,
colback=orange!5!white, colframe=orange!80!black, title={Cluster Analysis Prompt}]
\textbf{Task Cluster Analysis}

Cluster Name: \{cluster\_name\}

Capabilities Being Evaluated

\{capabilities\}

\textit{Note: Please examine the examples carefully to verify which capabilities are actually being tested.}

Tasks in Cluster

\{task\_names\}

Performance Statistics

Overall Success Rate: \{overall\_success\_rate\}

Success Rate by Task Difficulty:
\{difficulty\_breakdown\}

\textbf{Surprising Example}

Below are examples where the LLM succeeded or failed on tasks that reveal its capabilities or limitations.

\{surprising\_examples\}

Please analyze:
\begin{enumerate}[leftmargin=2em]
    \item What specific capabilities were demonstrated or lacking in the examples
    \item Any patterns in the successes and failures
    \item Notable or surprising results that reveal insights about the LLM's abilities
    \item What this suggests about the LLM's understanding and limitations
    \item How these insights connect to broader questions about LLM capabilities
\end{enumerate}
\end{tcolorbox}

\subsubsection{Example Selection Prompts}

\begin{tcolorbox}[breakable,boxrule=0.5pt,sharp corners,fontupper=\small,
colback=orange!5!white, colframe=orange!80!black, title={Example Selection System Prompt}]
You are an expert in designing task families to assess the capabilities of large language models (LLMs). You will write an analytical section for a report examining the capabilities and limitations of large language models.

Your goal is to analyze and synthesize insights about LLM capabilities by examining:
\begin{enumerate}[leftmargin=2em]
    \item The LLM's performance and solutions on tasks designed to test specific capabilities.
    \item Any patterns, strengths, or limitations revealed through this analysis.
\end{enumerate}
Focus on identifying surprising successes and failures from the point of view of an expert human evaluator.

You will be given a cluster of related task families that evaluate specific LLM capabilities, along with the LLM's responses and performance on these tasks. Your goal is to identify surprising successes and failures that reveal meaningful insights about LLM capabilities.

Respond precisely in the following format including the JSON start and end markers:\\

\textbf{THOUGHT}: \texttt{<THOUGHT>}

\textbf{RESPONSE JSON}: \texttt{<JSON>}\\

In \texttt{<THOUGHT>}, carefully analyze which examples demonstrate unexpected or notable behavior. Consider:
\begin{enumerate}[leftmargin=2em]
    \item Surprising successes on challenging tasks that demonstrate unexpected capabilities
    \item Unexpected failures on seemingly simple tasks that reveal limitations
    \item Examples that challenge common assumptions about LLM capabilities
\end{enumerate}

In \texttt{<JSON>}, provide a JSON response with the following fields:
\begin{itemize}[leftmargin=2em]
    \item \texttt{"surprising\_success\_example\_idx"}: List of indices for the most surprising or noteworthy successful tasks (0-3 indices)
    \item \texttt{"surprising\_failure\_example\_idx"}: List of indices for the most surprising or noteworthy failed tasks (0-3 indices)
\end{itemize}

Format for index lists: Empty list \texttt{[]}, single index \texttt{[1]}, or multiple indices \texttt{[0, 1, 3]}. This will be automatically parsed so ensure that the string response is precisely in the correct format.
\end{tcolorbox}

\begin{tcolorbox}[breakable,boxrule=0.5pt,sharp corners,fontupper=\small,
colback=orange!5!white, colframe=orange!80!black, title={Example Selection Prompt}]
\textbf{Task Cluster Analysis}

Cluster Name: \{cluster\_name\}

Capabilities Being Evaluated

\{capabilities\}

\textbf{Tasks Overview}

Total Tasks: \{num\_tasks\}

Overall Success Rate: \{overall\_success\_rate\}

\textbf{Task Examples}

\{task\_examples\}

Please analyze these examples carefully to identify:
\begin{enumerate}[leftmargin=2em]
    \item Which examples show surprising or unexpected successes, particularly:
    \begin{itemize}[leftmargin=1.25em]
        \item Complex tasks handled with sophisticated reasoning
        \item Challenging edge cases solved successfully
        \item Tasks requiring capabilities not typically associated with LLMs
    \end{itemize}
    \item Which examples show surprising or unexpected failures, particularly:
    \begin{itemize}[leftmargin=1.25em]
        \item Simple tasks that unexpectedly failed
        \item Inconsistent performance on similar tasks
        \item Failures that reveal interesting limitations
    \end{itemize}
\end{enumerate}

Focus on examples that would be genuinely surprising to an LLM expert researcher and provide meaningful insights about the model's capabilities or limitations.

In your response, briefly reason about EACH provided example and explain why it is (or isn't) surprising from the perspective of an LLM expert researcher.
\end{tcolorbox}

\subsubsection{Overall Summary Prompts}

\begin{tcolorbox}[breakable,boxrule=0.5pt,sharp corners,fontupper=\small,
colback=orange!5!white, colframe=orange!80!black, title={Overall Summary System Prompt}]
You are an expert in designing task families to assess the capabilities of large language models (LLMs). You will write an analytical section for a report examining the capabilities and limitations of large language models.

Your goal is to analyze and synthesize insights about LLM capabilities by examining:
\begin{enumerate}[leftmargin=2em]
    \item The LLM's performance and solutions on tasks designed to test specific capabilities.
    \item Any patterns, strengths, or limitations revealed through this analysis.
\end{enumerate}
Focus on identifying surprising successes and failures from the point of view of an expert human evaluator.

You are an expert researcher and engineer in Language Models. You are writing a very professional technical report to inform readers about the summary of the tested LLM's capabilities and limitations.

You will now provide an overall analysis and summary of the LLM's capabilities based on all the surprising tasks identified across various clusters. Your goal is to synthesize insights about the LLM's strengths and limitations, referencing specific results from the clusters using ``\#Cluster\_i'' to refer to examples.

Respond precisely in the following format including the JSON start and end markers:\\

\textbf{THOUGHT}: \texttt{<THOUGHT>}

\textbf{RESPONSE JSON}: \texttt{<JSON>}\\

In \texttt{<THOUGHT>}, deeply analyze the patterns observed across all clusters, considering both the surprising successes and failures. Your analysis should be detailed and reference specific results, using ``\#Cluster\_i'' to refer to examples from clusters.

In \texttt{<JSON>}, provide a JSON response with the following fields:
\begin{itemize}[leftmargin=2em]
    \item \texttt{"abstract"}: An abstract to this report. The first sentence should introduce the use of the \{scientist\} model as a scientist to study the \{subject\} model's capabilities. Then summarize the main contents.
    \item \texttt{"overall\_summary"}: A comprehensive summary of the LLM's capabilities based on your analysis. Introduce the context for the reader, e.g.\ start with sentences like ``In this report, we examine this LLM's \dots The LLM shows \dots''
    \item \texttt{"insight"}: A very detailed and long analysis to elaborate the above summary. Be very specific. Should be a list of \texttt{str}.
    \item \texttt{"surprising\_capabilities"}: Key surprising capabilities demonstrated by the LLM. Should be a list of \texttt{str}, and the analysis should be detailed and long.
    \item \texttt{"surprising\_failures"}: Notable limitations or failures revealed. Should be a list of \texttt{str}, and the analysis should be detailed and long.
    \item \texttt{"data\_insights"}: Analysis and interpretation of the numerical data provided (e.g.\ success rates, performance statistics). Should be a list of \texttt{str}, and the analysis should be detailed and long.
\end{itemize}

This will be automatically parsed so ensure that the string response is precisely in the correct format.
\end{tcolorbox}

\begin{tcolorbox}[breakable,boxrule=0.5pt,sharp corners,fontupper=\small,
colback=orange!5!white, colframe=orange!80!black, title={Overall Summary Prompt}]
\textbf{Overall Summary}

You have analyzed the LLM's performance across multiple task clusters and identified surprising successes and failures.

\textbf{Scientist and Subject}

You are now using the \{scientist\} model as a scientist to study the \{subject\} model's capabilities.

\textbf{Cluster Summaries}

\{cluster\_summaries\}

\textbf{Overall Statistics}

\{overall\_statistics\}

Please synthesize a comprehensive analysis of the LLM's capabilities based on the information above. In your analysis:
\begin{enumerate}[leftmargin=2em]
    \item Refer to specific results from clusters using ``\#Cluster\_i'' to refer to examples.
    \item Provide detailed observations about patterns in the LLM's performance across different clusters.
    \item Highlight surprising capabilities that challenge established understanding of LLM behavior.
    \item Discuss surprising failures that reveal significant limitations.
    \item Include analysis of numerical data, such as success rates and performance statistics.
\end{enumerate}

In your response \texttt{<THOUGHT>}, provide a detailed reasoning process that leads to your conclusions.

After your analysis, provide the JSON response with the required fields.
\end{tcolorbox}

\subsection{Generated Report Example}
\label{appsubsec:report_example}

Here we provide the first few pages of the generated report by \ouralgolong on GPT-4o (serving as both scientist and subject), as described in \Cref{sec:report_generation}.
Please find full reports for all evaluation settings in \Cref{sec:eval} at \url{https://github.com/conglu1997/ACD/tree/main/reports}.

\includepdf[
    clip=0mm 0mm 0mm 0mm,
    pages={1-4},
    frame,
    scale=.65,
    pagecommand={}
 ]{reports/gpt4_report.pdf}
