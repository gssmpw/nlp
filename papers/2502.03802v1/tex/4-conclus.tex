\section{Conclusion}

In this work, we proposed multiPCM, allowing us to more effectively distinguish between direct and indirect causal relationships. We integrated multiPCM with bivariate Convergent Cross Mapping (CCM) in a two-phase framework, MXMap, that first establishes an initial causal graph and then prunes indirect connections. Through experiments on simulated species interaction systems and real-world ERA5 meteorological data, we demonstrated that MXMap outperforms traditional methods and exhibits robust performance in complex, high-dimensional dynamical systems.

There are still limits to this current framework (discussed in Appendix~\ref{appsec:mxmap_limits}), demonstrated in runtime complexity, scalability, and possible failure cases in highly noisy environment or non-stationary systems. Future work could focus on enhancing the robustness of cross mapping under noisy conditions~\citep{monster2017causal}. Investigating and incorporating certain noise-handling mechanisms~\citep{zhang2024enhancing} in MXMap could further enhance its applicability in noisy real-world scenarios. Another direction is to explore adaptive parameter selection~\citep{shortreed2017outcome, machlanski2023hyperparameter} for MXMap, such as optimizing embedding dimensions and lags based on data properties and currently outputs. Current grid search methods are computationally expensive for larger datasets, and efficient heuristic or learning-based tuning could improve scalability. Finally, applying MXMap to other real-world domains, such as power systems, larger timescale climate modeling, and epidemiology, could further validate its versatility and reveal complex causal interactions.