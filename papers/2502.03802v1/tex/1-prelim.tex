\section{Preliminaries}
\label{sec:prelim}

\subsection{State-Space Reconstruction (SSR)}
\label{sec:ssr}

In physical continuous-time dynamical systems, the interplay between driving forces and dissipation leads systems to settle into characteristic behaviors, represented by attractor manifolds in state space~\citep{milnor1985concept}. Understanding these attractors is crucial for interpreting the system's dynamics and predicting future behavior. However, real-world measurements are often limited, making it infeasible to observe all variables required to fully characterize the state-space attractor.

% State-Space Reconstruction (SSR) \textcolor{red}{Di: a reference is needed here} addresses this challenge: \textit{Given an $n$-dimensional observed time series from an $N$-dimensional dynamical system ($n < N$), can we recover the attractor manifold, and thereby the higher-dimensional dynamics, from the lower-dimensional observations?} A common approach is to use sequences of lagged observations to reconstruct a delay embedding (DE) that approximates the system's attractor. Whitney's Embedding Theorem~\citep{whitney1936differentiable} and Takens' Embedding Theorem~\citep{takens2006detecting} establish that this reconstruction is diffeomorphic (i.e., a continuously differentiable and invertible mapping) to the true attractor under certain conditions~\citep{sauer1991j}. When these conditions are satisfied, such delay embeddings are termed "shadow manifolds" and serve as low-dimensional approximations of the system~\citep{sugihara2012detecting}.

State-Space Reconstruction (SSR)~\citep{vlachos2008state} addresses this challenge: \textit{Given an $n$-dimensional observed time series from an $N$-dimensional dynamical system ($n < N$), can we recover the attractor manifold, and thereby the higher-dimensional dynamics, from the lower-dimensional observations?} A common approach is to use sequences of lagged observations to reconstruct a delay embedding (DE) that approximates the system's attractor. Whitney's Embedding Theorem~\citep{whitney1936differentiable} and Takens' Embedding Theorem~\citep{takens2006detecting} establish that this reconstruction is diffeomorphic (i.e., a continuously differentiable and invertible mapping) to the true attractor under certain conditions~\citep{sauer1991j}. When these conditions are satisfied, such delay embeddings are termed "shadow manifolds" and serve as low-dimensional approximations of the system~\citep{sugihara2012detecting}.

% Following~\citep{vlachos2010nonuniform, butler2023causal}, we introduce the univariate delay-coordinate embedding used in Takens' Theorem. Suppose an attractor $\mathcal{A}$ exists for the dynamical system, and a time series $\{x_t\}_{t=0}^{T}$ is observed from one state variable $\textbf{X}$. Given delay $\tau$ and embedding dimension $E$, where $\tau$ and $E$ are positive integers, the vector signal $\textbf{m}_{x}(t)$ of lagged values is defined as: 
% % \textcolor{red}{Di: You may need to pay attention for the definition for a variable and a vector}

Following~\citep{vlachos2010nonuniform, butler2023causal}, we introduce the univariate delay-coordinate embedding used in Takens' Theorem. Suppose an attractor $\mathcal{A}$ exists for the dynamical system, and a time series $\{x_t\}_{t=0}^{T}$ is observed from one state variable $\textbf{x}$ sampled at a constant rate. Given delay $\tau$ and embedding dimension $E$, where $\tau$ and $E$ are positive integers, the vector signal $\vv{\textbf{m}}_{x}(t)$ of lagged values is defined as:

\begin{equation}
    % \textbf{m}_{x}(t)= \left[x_{t},  x_{t-\tau},  x_{t-2\tau},  x_{t-3\tau},  \ldots,  x_{t-(E-2)\tau},  x_{t-(E-1)\tau}\right] 
    \vv{\textbf{m}}_{x}(t) \vcentcolon= \left[x_{t},  x_{t-\tau},  x_{t-2\tau},  x_{t-3\tau},  \ldots,  x_{t-(E-2)\tau},  x_{t-(E-1)\tau}\right]
\end{equation}

As time progresses, these vectors form an $E$-dimensional delay embedding $\textbf{M}_x$. The lag $\tau$ determines the observation time scale for reconstruction, while the embedding dimension $E$ defines the complexity of the embedding. Takens' theorem suggests that $E$ should be greater than twice the fractal dimension of the attractor $\mathcal{A}$, i.e., $E > 2 \cdot \text{dim}(\mathcal{A})$~\citep{sauer1991j, kugiumtzis1996state}. In practice, $\tau$ and $E$ are determined empirically. Time-delayed autocorrelation~\citep{kugiumtzis1996state} and delay mutual information~\citep{fraser1986independent, klikova2011reconstruction} are commonly used to select an optimal $\tau$, while the \textit{false nearest neighbors} (FNN) method~\citep{kennel1992determining} is typically used to determine $E$, by tracking changes in nearest neighbors as embedding dimensions increase.

\subsection{Convergent Cross Mapping (CCM)}
\label{sec:uni-ccm}
Causality in a discrete-time dynamical system~\citep{butler2023causal,cummins2015efficacy} can be defined as follows: given two state variables $\textbf{x}$ and $\textbf{y}$, if the future evolution of $\textbf{y}$ depends on $\textbf{x}$, then $\textbf{x}$ is said to cause $\textbf{y}$, denoted as $\textbf{x} \Rightarrow \textbf{y}$. This causal influence can be represented in a state-space equation, as shown in Eq.~\ref{eq:def_cause}:

\begin{equation}
\label{eq:def_cause}
    \textbf{y}_{t+1}=\mathcal{F}_y \left(\textbf{y}_{t}, \textbf{x}_{t} \right)
\end{equation}

The relationship between $\textbf{x}$ and $\textbf{y}$ can be unidirectional ($\textbf{x} \Rightarrow \textbf{y}$ or $\textbf{y} \Rightarrow \textbf{x}$), bidirectional ($\textbf{x} \Leftrightarrow \textbf{y}$), or there may be no causal link at all.

% Convergent Cross Mapping (CCM) \textcolor{red}{Di: In the main body of a paper, CCM should only be defined onece}~\citep{sugihara2012detecting} leverages the diffeomorphism between reconstructed shadow manifolds, as stated in Takens' Theorem. Cross mapping measures how well local neighborhoods in one reconstructed manifold map to the corresponding neighborhoods in another. For delay embeddings $\textbf{M}_x$ and $\textbf{M}_y$ reconstructed from $\textbf{x}$ and $\textbf{y}$, if $\textbf{M}_x$ and $\textbf{M}_y$ are both valid shadow manifolds of the attractor $\mathcal{A}$, they are diffeomorphic to each other via their relationship to $\mathcal{A}$.

CCM~\citep{sugihara2012detecting} leverages the diffeomorphism between reconstructed shadow manifolds, as stated in Takens' Theorem. Cross mapping measures how well local neighborhoods in one reconstructed manifold map to the corresponding neighborhoods in another. For delay embeddings $\textbf{M}_x$ and $\textbf{M}_y$ reconstructed from $\textbf{x}$ and $\textbf{y}$, if $\textbf{M}_x$ and $\textbf{M}_y$ are both valid shadow manifolds of the attractor $\mathcal{A}$, they are diffeomorphic to each other via their relationship to $\mathcal{A}$.

If $\textbf{x}$ causes $\textbf{y}$ ($\textbf{x} \Rightarrow \textbf{y}$), observations of $\textbf{y}$ should contain information about $\textbf{x}$. This allows the reconstruction of the dynamics of $\textbf{x}$ from $\textbf{y}$, but not necessarily vice versa. In this case, the quality of mapping from $\textbf{M}_y$ to $\textbf{M}_x$ should be better compared to the reverse direction, indicating a causal link from $\textbf{x}$ to $\textbf{y}$.

CCM uses a $k$-nearest neighbor ($k$NN) regression approach (also known as \textit{simplex projection}) to evaluate the quality of cross mapping. Given time series $\{x_t\}_{t=0}^{T}$ and $\{y_t\}_{t=0}^{T}$, to verify whether $\textbf{x} \Rightarrow \textbf{y}$, the procedure is as follows:

\begin{enumerate}[label={[\arabic*]}]
    \item Construct delay embeddings $\textbf{M}_x, \textbf{M}_y$ with appropriate delay $\tau$ and embedding dimension $E$.
    \item For each point in $\textbf{M}_y$, identify the $k$-nearest neighbors $\mathcal{N}_y$.
    \item Use the timestamps of $\mathcal{N}_y$ to find corresponding points $\hat{\mathcal{N}}_x$ on $\textbf{M}_x$ and compute a weighted average to form a projected reconstruction $\hat{\textbf{M}}_x$, hence the reconstructed $\hat{\mathbf{x}}$.
    \item Calculate the correlation score $\rho_{x\Rightarrow y}$ between the true $\mathbf{x}$ and the reconstructed $\hat{\mathbf{x}}$.
    \item Repeat these steps with increasing sequence length; if $\textbf{x} \Rightarrow \textbf{y}$, the correlation score should converge, indicating a valid cross map.
\end{enumerate}

The same procedure is repeated for the reverse causal assumption to yield another correlation score $\rho_{y\Rightarrow x}$. The correlation scores $\rho_{x\Rightarrow y}$ and $\rho_{y\Rightarrow x}$ quantify the cross mapping quality, where a higher score in one direction suggests a stronger causal link. In practice, if the input length $L$ is large enough, we consider that the yielded correlation scores are already in the convergence zone, and can be used as final correlation estimates.

\subsection{Partial Cross Mapping (PCM)}
\label{sec:uni-pcm}
The original CCM does not distinguish between direct and indirect causality. For three variables $\textbf{x}$, $\textbf{y}$, and $\textbf{z}$ in a causal chain ($\textbf{x}\Rightarrow\textbf{y}\Rightarrow\textbf{z}$), CCM may incorrectly identify a direct causal link between $\textbf{x}$ and $\textbf{z}$ due to transitivity through $\textbf{y}$. Partial Cross Mapping (PCM)~\citep{leng2020partial,jiang2023partial} was proposed to distinguish between direct and indirect causal relationships. In a causal chain like $\textbf{x}\Rightarrow\textbf{y}\Rightarrow\textbf{z}$, PCM aims to determine whether the causal link between $\textbf{x}$ and $\textbf{z}$ is direct or mediated by $\textbf{y}$.

A PCM test considers three variables: the \textit{potential cause} $\textbf{x}$, the \textit{condition} $\textbf{y}$, and the \textit{potential effect} $\textbf{z}$. The goal is to assess whether there is a direct link between $\textbf{x}$ and $\textbf{z}$. This is done by performing cross mapping between the shadow manifolds of each variable to obtain a reconstruction of $\textbf{x}$, denoted by $\hat{\textbf{x}}^{\textbf{z}}$ (from $\textbf{z}$ to $\textbf{x}$), and another reconstruction of $\textbf{x}$ via $\textbf{y}$, denoted by $\hat{\textbf{x}}^{\hat{\textbf{y}}^{\textbf{z}}}$ (first from $\textbf{z}$ to $\textbf{y}$, then from $\textbf{y}$ to $\textbf{x}$). The correlation scores are defined as follows:
% \begin{equation}
%     \rho_{All}=\left| \text{Corr}(\textbf{x}, \hat{\textbf{x}}^{\textbf{z}}) \right|
% \end{equation}
% \begin{equation}
%     \rho_{Direct}=\left| \text{ParCorr}(\textbf{x}, \hat{\textbf{x}}^{\textbf{z}} | \hat{\textbf{x}}^{\hat{\textbf{y}}^{\textbf{z}}} ) \right|
% \end{equation}

\begin{align}
    \rho_{All}=\left| \text{Corr}(\textbf{x}, \hat{\textbf{x}}^{\textbf{z}}) \right|   &&    \rho_{Direct}=\left| \text{ParCorr}(\textbf{x}, \hat{\textbf{x}}^{\textbf{z}} | \hat{\textbf{x}}^{\hat{\textbf{y}}^{\textbf{z}}} ) \right|
\end{align}

$\rho_{All}$ represents the correlation between the original $\textbf{x}$ and the reconstruction $\hat{\textbf{x}}^{\textbf{z}}$, capturing apparent information transfer through all paths. $\rho_{Direct}$ represents the partial correlation, conditioning on the intermediate variable $\textbf{y}$ to assess direct information transfer between $\textbf{x}$ and $\textbf{z}$. If no direct causal link exists, the direct information transfer should be significantly reduced after conditioning on $\textbf{y}$.

PCM uses an empirical threshold $H \in [0, 1)$ to determine causality:
\begin{itemize}
    \item If $\rho_{All} \geq \rho_{Direct} \geq H$, a direct causal link from $\textbf{x}$ to $\textbf{z}$ is inferred.
    \item If $\rho_{All} \geq H \gg \rho_{Direct}$, only indirect causality is suggested.
    \item If $H > \rho_{All} \geq \rho_{Direct}$, no causal relationship is inferred.
\end{itemize}

To distinguish direct from indirect links, we propose an adaptation in our work, to use the correlation ratio $\gamma$ as an alternative:
\begin{equation}
    \gamma=\frac{\rho_{Direct}}{\rho_{All}}
\end{equation}
A smaller ratio $\gamma$ implies negligible direct information transfer after conditioning, suggesting indirect causality via $\textbf{y}$. Conversely, a larger $\gamma$ indicates strong direct information transfer, suggesting a direct causal link. An empirical ratio threshold $\gamma^* \in (0, 1)$ is used to decide whether to retain or eliminate the direct link based on how important such causal influence is.
