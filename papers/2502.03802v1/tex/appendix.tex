\newpage

\appendix

\section{Validation of multiPCM with More Generated Systems}
\label{appsec:valid_multiPCM}

A more complete grid-search results is presented in Table~\ref{tab:app_multiPCM}, with more cases of simulated systems and additional profiles of correlation scores $\rho_{All}$ and $\rho_{Direct}$, on top of the ratio and label. 

Input length is $L = 3500$, the ranges of lags and embedding dimensions are ($\tau, E \in \{1, 2, 3, \ldots, 8\}$). The predicted labels indicating whether direct causality is rejected based on a PCM threshold of 0.45 (this threshold selection is discussed in Appendix~\ref{appsec:thres_multiPCM}). \textcolor{red}{Red} label indicates rejection, suggesting there is only indirect causality, while \textcolor{blue}{blue} label indicates acceptance, suggesting direct causality exists hence we should not remove the link between the colored nodes.

\begin{table}[htb]
\centering

\makebox[\linewidth]{%

\resizebox{1.2\textwidth}{!}{% Adjust the scaling factor to exceed column width

\begin{tabular}{c|c|cc|cc}
Type         & $Direct$ & \multicolumn{2}{c|}{$Indirect$} & \multicolumn{2}{c}{$Both$} \\ \hline
Causality         &\begin{minipage}{.15\linewidth} \centering \includegraphics[width=0.3\linewidth]{imgs/ValidPCM/4VDirect.png} \end{minipage}& \multicolumn{1}{c|}{\begin{minipage}{.15\linewidth} \centering \includegraphics[width=0.3\linewidth]{imgs/ValidPCM/4VIndirect1.png} \end{minipage}}    & \begin{minipage}{.12\linewidth} \centering \includegraphics[width=\linewidth]{imgs/ValidPCM/4VIndirect2.png} \end{minipage} & \multicolumn{1}{c|}{\begin{minipage}{.15\linewidth} \centering \includegraphics[width=0.5\linewidth]{imgs/ValidPCM/4VBoth1.png} \end{minipage}} & \begin{minipage}{.12\linewidth} \centering \includegraphics[width=\linewidth]{imgs/ValidPCM/4VBoth2.png} \end{minipage} \\ \hline
$\rho_{All}$    &\begin{minipage}{.153\linewidth} \centering \includegraphics[width=\linewidth]{imgs/ValidPCM/4VDirect_sc1.png} \end{minipage}& \multicolumn{1}{c|}{\begin{minipage}{.153\linewidth} \centering \includegraphics[width=\linewidth]{imgs/ValidPCM/4VIndirect1_sc1.png} \end{minipage}}    & \begin{minipage}{.153\linewidth} \centering \includegraphics[width=\linewidth]{imgs/ValidPCM/4VIndirect2_sc1.png} \end{minipage}    & \multicolumn{1}{c|}{\begin{minipage}{.153\linewidth} \centering \includegraphics[width=\linewidth]{imgs/ValidPCM/4VBoth1_sc1.png} \end{minipage}} & \begin{minipage}{.153\linewidth} \centering \includegraphics[width=\linewidth]{imgs/ValidPCM/4VBoth2_sc1.png} \end{minipage}  \\ \hline
$\rho_{Direct}$ &\begin{minipage}{.153\linewidth} \centering \includegraphics[width=\linewidth]{imgs/ValidPCM/4VDirect_sc2.png} \end{minipage}& \multicolumn{1}{c|}{\begin{minipage}{.153\linewidth} \centering \includegraphics[width=\linewidth]{imgs/ValidPCM/4VIndirect1_sc2.png} \end{minipage}}    &  \begin{minipage}{.153\linewidth} \centering \includegraphics[width=\linewidth]{imgs/ValidPCM/4VIndirect2_sc2.png} \end{minipage}  & \multicolumn{1}{c|}{\begin{minipage}{.153\linewidth} \centering \includegraphics[width=\linewidth]{imgs/ValidPCM/4VBoth1_sc2.png} \end{minipage}} & \begin{minipage}{.153\linewidth} \centering \includegraphics[width=\linewidth]{imgs/ValidPCM/4VBoth2_sc2.png} \end{minipage} \\ \hline
Ratio             &\begin{minipage}{.153\linewidth} \centering \includegraphics[width=\linewidth]{imgs/ValidPCM/4VDirect_ratio.png} \end{minipage}& \multicolumn{1}{c|}{\begin{minipage}{.153\linewidth} \centering \includegraphics[width=\linewidth]{imgs/ValidPCM/4VIndirect1_ratio.png} \end{minipage}}    & \begin{minipage}{.153\linewidth} \centering \includegraphics[width=\linewidth]{imgs/ValidPCM/4VIndirect2_ratio.png} \end{minipage}    & \multicolumn{1}{c|}{\begin{minipage}{.153\linewidth} \centering \includegraphics[width=\linewidth]{imgs/ValidPCM/4VBoth1_ratio.png} \end{minipage}} & \begin{minipage}{.153\linewidth} \centering \includegraphics[width=\linewidth]{imgs/ValidPCM/4VBoth2_ratio.png} \end{minipage} \\ \hline
Label    &\begin{minipage}{.153\linewidth} \centering \includegraphics[width=\linewidth]{imgs/ValidPCM/4VDirect_label.png} \end{minipage}& \multicolumn{1}{c|}{\begin{minipage}{.153\linewidth} \centering \includegraphics[width=\linewidth]{imgs/ValidPCM/4VIndirect1_label.png} \end{minipage}}    & \begin{minipage}{.153\linewidth} \centering \includegraphics[width=\linewidth]{imgs/ValidPCM/4VIndirect2_label.png} \end{minipage} & \multicolumn{1}{c|}{\begin{minipage}{.153\linewidth} \centering \includegraphics[width=\linewidth]{imgs/ValidPCM/4VBoth1_label.png} \end{minipage}} & \begin{minipage}{.153\linewidth} \centering \includegraphics[width=\linewidth]{imgs/ValidPCM/4VBoth2_label.png} \end{minipage}
\end{tabular}
}
}

\caption{Performance of multiPCM: Correlation scores, correlation ratio, predicted label ($thres=0.45$) under grid search. Red dot indicates there isn't direct causality between the colored nodes, while blue indicates there is direct causality.}

\label{tab:app_multiPCM}

\end{table}


\section{Empirical Threshold Selection for multiPCM in Simulated Systems}
\label{appsec:thres_multiPCM}

The threshold for correlation ratio used in multiPCM was selected empirically by testing on our simulated systems under three distinct causality scenarios: direct, indirect, and both. The data of these scenarios are generated based on the description of Section~\ref{sec:genData}, varying from 3-variable to 7-variable with increasing complexity. The data generation is noise-free for simplicity.

Here we use lag $\tau=1$ and embedding dimension $E=7$ for multiPCM (since the discrete data generation uses 1 as the lag, and the highest dimension of all the systems is 7, we round the dimension up to the highest value to ensure performance across all simulated systems). We illustrate the output labels with varying ratio thresholds ranging from 0.05 to 0.95, with a step size of 0.05. Lower ratio threshold implies that the edge removal is more lenient and tends to retain more edges that lie between direct and indirect; while higher threshold indicates the edge removal is stricter and will only keep the edges whose direct correlation is strong enough.

For direct and both causality scenarios, the link should be retained (labeled "\textcolor{blue}{direct}"); For indirect causality scenarios, the link should be removed (labeled "\textcolor{red}{indirect}"). 

The selected threshold should achieve a relatively good accuracy across all systems, here we set the tolerance for each threshold value to 2 mistaken labels at each causality senario.


\begin{figure}[htb]
    \centering

    % Subfigure 1: Direct causal link
    \subfigure[Direct (expected "\textcolor{blue}{direct}")]{
        \includegraphics[width=0.31\linewidth]{imgs/threshold/direct_noNoise.png}
        \label{appfig:direct_noNoise}
    }
    \hfill
    % Subfigure 2: Indirect causal link
    \subfigure[Indirect (expected "\textcolor{red}{indirect}")]{
        \includegraphics[width=0.31\linewidth]{imgs/threshold/indirect_noNoise.png}
        \label{appfig:indirect_noNoise}
    }
    \hfill
    % Subfigure 3: Both direct and indirect
    \subfigure[Both (expected "\textcolor{blue}{direct}")]{
        \includegraphics[width=0.31\linewidth]{imgs/threshold/both_noNoise.png}
        \label{appfig:both_noNoise}
    }
    \caption{Different PCM ratio thresholds in three causality scenarios and how they impact the predicted labels.}
    \label{appfig:noNoise-pcmThres}
\end{figure}

Fig.~\ref{appfig:noNoise-pcmThres} illustrates the results in heatmaps. In the "Direct" scenario, the thresholds that satisfy the tolerance are from 0.05 all the way to 0.9 (inclusive). In the "Indirect" scenario, the thresholds that satisfy the tolerance are from 0.45 to 0.95 (inclusive). In the "Both" scenario, the thresholds that satisfy the tolerance are from 0.05 to 0.45 (inclusive). Overall, 0.45 is relatively the best threshold value across all the simulation scenarios. 

Note that the appropriate threshold may vary depending on the system and the specific purpose of the study. For the ERA5 chain $tcw \Rightarrow rad \Rightarrow  T_{2m}$ discussed in Section~\ref{sec:weather_chain}, with MXMap lag $\tau=4$ and dimension $E=6$, the PCM ratio threshold required to remove the likely-indirect link $tcw\Rightarrow T_{2m}$ is 0.7. This higher threshold value may suggest the presence of a weaker but direct causal relationship between total cloud water $tcw$ and near-ground temperature $T_{2m}$, or it could reflect the influence of unaccounted latent variables. If the goal is to establish a causal graph with the correct order while tolerating some additional links, a denser graph may be acceptable for certain applications. In such cases, tolerating the retention of certain edges by multiPCM can still yield useful insights.


\section{ERA5 Meteorological Data: Candidate Testing Systems}
\label{appsec:era5}

\subsection{Cloud ($tcw$) $\Rightarrow$ Radiation ($rad$) $\Rightarrow$ Ground-Level Temperature ($T_{2m}$)}
\label{appsec:weather-chain}

This causal chain used in our work, where clouds impact radiation levels and in turn affect ground-level temperatures, is well-supported in meteorological research, especially concerning winter seasons.

Clouds play a significant role in modulating surface radiation, as they both reflect incoming solar radiation (shortwave) and trap outgoing terrestrial radiation (longwave)~\citep{stephens2005cloud}, hence $tcw\Rightarrow rad$.

During the day, clouds primarily act as a barrier to incoming solar radiation. Thick and extensive cloud cover reflects a significant portion of the solar radiation back into space, thereby lowering the temperature. At night, the longwave terrestrial radiation play a predominant role, which, combined with the presence of cloud cover (like an insulating blanket), leads to a warming effect. 

The warming effect is especially pronounced in winter~\citep{ramanathan1989cloud}, because the reduced daylight and lower solar angles reduce the overall shortwave radiation, rendering cloud cover's longwave effect more prominent. Research also indicates that during winter, the impact of clouds on radiation and temperature is more pronounced in polar regions, or in temperate zones with longer nights and lower solar angle, emphasizing the effect of cloud cover on maintaining warmer near-ground temperatures.

The timescale of this mechanism is usually within an hour or within a couple of hours~\citep{stephens2005cloud,pielke2005should}: the impact of cloud cover on radiation flux is almost instantaneous, and the response of ground-level temperature depends on several factors (e.g. time of the day, humidity, wind). This information is crucial for choosing the lag value for the state-space reconstruction used in cross mapping based methods and the maximal lag $\tau$ used in PCMCI. 


\subsection{Introducing More Variables to the System}
\label{appsec:era5_moreV}

To enhance the complexity and realism of the ERA5 testing system, we introduce additional variables that reflect key meteorological processes. These include the following: 

\subsubsection{Near-Ground Temperature Advection $T_{adv950}$ }

$T_{adv950}$ stands for the near-ground temperature advection at the $950 hPa$ pressure level. It captures the transport of temperature influenced by atmospheric motion, and should be a direct cause of the near-ground temperature $T_{2m}$, hence $T_{adv950} \Rightarrow T_{2m}$.

\subsubsection{Radiation Components: Solar Radiation ($rad_{solar}$) and Terrestrial Radiation ($rad_{terr}$)}

Separating the radiation components into solar and terrestrial radiation enables finer analysis of their individual roles in energy balance and temperature dynamics:
\begin{itemize}
    \item \textbf{Solar Radiation ($rad_{solar}$)} is mainly and directly affected by cloud cover due to its role in shielding or transmitting shortwave radiation, hence $tcw\Rightarrow rad_{solar}$.
    \item \textbf{Terrestrial Radiation ($rad_{terr}$)} and cloud cover have a more complex interaction mechanism, and their causal relationship would be weaker and less prominent compared to $tcw\Rightarrow rad_{solar}$.
\end{itemize}
Both radiation components should be directly impacting the temperature, hence $rad_{solar} \Rightarrow T_{2m}$ and $rad_{terr} \Rightarrow T_{2m}$.

\subsection{Evaluating Causal Relationships}
\label{appsec:era5_eval}
We evaluate the model performance on the following systems:
\begin{enumerate}
    \item \textbf{The 3-variable chain $tcw \Rightarrow rad \Rightarrow  T_{2m}$:} Since the ground truth is known and relatively well justified, the causal discovery of this system will be evaluated in the same way as the simulated systems, using the metrics to compute the differences between truth causal graph and predicted graph;
    \item \textbf{The expanded 5-variable system with $tcw$, $rad_{solar}$, $rad_{terr}$, $T_{adv950}$ and $T_{2m}$:} While establishing the exact ground truth causal graph for this higher-dimensional weather system is challenging due to its inherent complexity, the following bivariate causal relationships are well-understood and serve as a benchmark for evaluation: $rad_{solar} \Rightarrow T_{2m}$, $rad_{terr} \Rightarrow T_{2m}$, $T_{adv950} \Rightarrow T_{2m}$, $tcw\Rightarrow rad_{solar}$; We evaluate how well the causal inference methods detect these known relationships consistently. 
\end{enumerate}

We present the full results along with the predicted graphs in Appendix Table~\ref{apptab:mxmap-era5-5V} to supplement the results presented in main Section~\ref{sec:era5_5V_eval}.

\begin{table}[htb]
\centering
\caption{Full results for detection of Benchmark Causal Relationships in the ERA5 5V System. A checkmark (green) indicates a correctly detected and oriented edge, a half-checkmark (gray) denotes a detected but ambiguously oriented edge, and a crossmark (red) represents an undetected or incorrectly oriented edge.}
\label{apptab:mxmap-era5-5V}

% First segment of the table
\resizebox{\textwidth}{!}{
\begin{tabular}{c|cccc}
Methods                          & tsFCI                            & VARLiNGAM                        & Granger                          & PCMCI                            \\ \hline
Output                           &  
\begin{minipage}{.24\linewidth} \centering \includegraphics[width=\linewidth]{imgs/ERA5_5V/tsfci.png} \end{minipage} &  
\begin{minipage}{.24\linewidth} \centering \includegraphics[width=\linewidth]{imgs/ERA5_5V/varlingam.png} \end{minipage} &  
\begin{minipage}{.24\linewidth} \centering \includegraphics[width=\linewidth]{imgs/ERA5_5V/granger.png} \end{minipage} &  
\begin{minipage}{.24\linewidth} \centering \includegraphics[width=\linewidth]{imgs/ERA5_5V/pcmci.png} \end{minipage} \\ \hline
$rad_{solar} \Rightarrow T_{2m}$ & \cmark                       & \halfcheckmark & \halfcheckmark & \halfcheckmark \\
$rad_{terr} \Rightarrow T_{2m}$  & \halfcheckmark & \halfcheckmark & \cmark                       & \halfcheckmark \\
$T_{adv950} \Rightarrow T_{2m}$  & \xmark                           & \halfcheckmark & \halfcheckmark & \halfcheckmark \\
$tcw\Rightarrow rad_{solar}$     & \xmark                           & \xmark                           & \halfcheckmark & \cmark                       
\end{tabular}
}

\vspace{1em}

% Second segment of the table
\resizebox{0.9\textwidth}{!}{
\begin{tabular}{c|ccc}
Methods                          & DYNOTEARS                        & SLARAC                           & MXMap                            \\ \hline
Output                           &  
\begin{minipage}{.24\linewidth} \centering \includegraphics[width=\linewidth]{imgs/ERA5_5V/dynotears.png} \end{minipage} &  
\begin{minipage}{.3\linewidth} \centering \includegraphics[width=\linewidth]{imgs/ERA5_5V/slarac.png} \end{minipage} &  
\begin{minipage}{.24\linewidth} \centering \includegraphics[width=\linewidth]{imgs/ERA5_5V/mxmap.png} \end{minipage} \\ \hline
$rad_{solar} \Rightarrow T_{2m}$ & \halfcheckmark & \xmark                           & \cmark \\
$rad_{terr} \Rightarrow T_{2m}$  & \xmark                           & \cmark                       & \cmark \\
$T_{adv950} \Rightarrow T_{2m}$  & \cmark                       & \halfcheckmark & \cmark \\
$tcw\Rightarrow rad_{solar}$     & \cmark                       & \cmark                       & \cmark                       
\end{tabular}
}

\end{table}


% \begin{table}[htb]
% \centering
% \caption{Full results for detection of benchmark causal relationships on the ERA5 5V system. Checkmark (green) stands for correctly detected and oriented edge, half-checkmark (gray) stands for detected but unclearly oriented edge, and crossmark (red) stands for non-detected and reversely oriented edge.}
% \label{apptab:mxmap-era5-5V}

% % First sub-table
% \begin{subtable}
% \centering
% % \caption{Methods: tsFCI to PCMCI}
% \resizebox{\textwidth}{!}{
% \begin{tabular}{c|cccc}
% Methods                          & tsFCI                            & VARLiNGAM                        & Granger                          & PCMCI                            \\ \hline
% Output                           &  
% \begin{minipage}{.24\linewidth} \centering \includegraphics[width=\linewidth]{imgs/ERA5_5V/tsfci.png} \end{minipage} &  
% \begin{minipage}{.24\linewidth} \centering \includegraphics[width=\linewidth]{imgs/ERA5_5V/varlingam.png} \end{minipage} &  
% \begin{minipage}{.24\linewidth} \centering \includegraphics[width=\linewidth]{imgs/ERA5_5V/granger.png} \end{minipage} &  
% \begin{minipage}{.24\linewidth} \centering \includegraphics[width=\linewidth]{imgs/ERA5_5V/pcmci.png} \end{minipage} \\ \hline
% $rad_{solar} \Rightarrow T_{2m}$ & \cmark                       & \halfcheckmark & \halfcheckmark & \halfcheckmark \\
% $rad_{terr} \Rightarrow T_{2m}$  & \halfcheckmark & \halfcheckmark & \cmark                       & \halfcheckmark \\
% $T_{adv950} \Rightarrow T_{2m}$  & \xmark                           & \halfcheckmark & \halfcheckmark & \halfcheckmark \\
% $tcw\Rightarrow rad_{solar}$     & \xmark                           & \xmark                           & \halfcheckmark & \cmark                       
% \end{tabular}
% }

% \end{subtable}

% \vspace{1em}

% % Second sub-table
% \begin{subtable}
% \centering
% % \caption{Methods: DYNOTEARS to MXMap}
% \resizebox{0.9\textwidth}{!}{
% \begin{tabular}{c|ccc}
% Methods                          & DYNOTEARS                        & SLARAC                           & MXMap                            \\ \hline
% Output                           &  
% \begin{minipage}{.24\linewidth} \centering \includegraphics[width=\linewidth]{imgs/ERA5_5V/dynotears.png} \end{minipage} &  
% \begin{minipage}{.28\linewidth} \centering \includegraphics[width=\linewidth]{imgs/ERA5_5V/slarac.png} \end{minipage} &  
% \begin{minipage}{.24\linewidth} \centering \includegraphics[width=\linewidth]{imgs/ERA5_5V/mxmap.png} \end{minipage} \\ \hline
% $rad_{solar} \Rightarrow T_{2m}$ & \halfcheckmark & \xmark                           & \cmark \\
% $rad_{terr} \Rightarrow T_{2m}$  & \xmark                           & \cmark                       & \cmark \\
% $T_{adv950} \Rightarrow T_{2m}$  & \cmark                       & \halfcheckmark & \cmark \\
% $tcw\Rightarrow rad_{solar}$     & \cmark                       & \cmark                       & \cmark                       
% \end{tabular}
% }

% \end{subtable}

% \end{table}


\section{RESIT (Regression with Subsequent Independence Test) Framework}
\label{appsec:resit}

\begin{algorithm}[htb]
    \caption{Regression with Subsequent Independence Test (RESIT)}
    \label{alg:RESIT}
    % \scriptsize
      \SetAlgoLined
    \KwIn{i.i.d. samples of a $p$-dimensional distribution on $(X_1, \ldots, X_p)$}
    \KwOut{Set of parents for each variable: $\{\text{pa}(1), \ldots, \text{pa}(p)\}$}

    $S \gets \{1, \ldots, p\}$

    $\pi \gets [\ ]$
    
    \textbf{Phase 1: Determine topological order}
    
    \While{$S \neq \emptyset$}{
        \For{$k \in S$}{
            Regress $X_k$ on $\{X_i\}_{i \in S \setminus \{k\}}$ ;
            
            Measure dependence between residual $r = X_k - \hat{X_k}$ and predictors $\{X_i\}_{i \in S \setminus \{k\}}$ ;
        }
        
        Let $k^*$ be the variable with the weakest dependence ;
        
        $S \gets S \setminus \{k^*\}$ ;
        
        $\text{pa}(k^*) \gets S$ ;
        
        $\pi \gets [k^*, \pi]$ ;
        
    }

    \textbf{Phase 2: Remove superfluous edges}
    
    \For{$k = 2$ \KwTo $p$}{
        \For{$\ell \in \text{pa}(\pi(k))$}{
            Regress $X_{\pi(k)}$ on $\{X_i\}_{i \in \text{pa}(\pi(k)) \setminus \{\ell\}}$ ;
            
            \If{residuals are independent of $\{X_i\}_{i \in \{\pi(1), \ldots, \pi(k-1)\}}$}{
                $\text{pa}(\pi(k)) \gets \text{pa}(\pi(k)) \setminus \{\ell\}$ ;
            }
        }
    }
\end{algorithm}


RESIT is an approach that extends the principles of ANM by iteratively conducting regression followed by an independence test on the residuals \citep{peters2014causal}. The original RESIT framework contains two phases, the causal order determination phase, and the edge elimination phase and is outlined in Algorithm~\ref{alg:RESIT}.

In Phase 1, one variable is selected as a prediction target (alleged effect) in each iteration, and the remaining variables are used as predictors (alleged causes) to fit a regression model. The regression error is computed, and the dependence between the residue and the predictors is measured. The variable with the weakest dependence (most likely effect) is identified and removed from the set. This process is repeated until all variables are ordered, establishing a topological order.

In Phase 2, for each variable with at least one parent, the set of parent variables is retrieved. For each parent variable, a regression model is fitted to predict the target variable using all other parents except the current one. The independence of the regression residue is compared with a threshold, and if it is sufficiently independent, the parent variable is removed from the parent set. This phase refines the causal structure by eliminating unnecessary dependencies, resulting in a clearer causal graph.

The RESIT algorithm iteratively establishes a causal order of variables by evaluating the strength of their causal relationships through regression and independence testing, ultimately refining the identified causal structure.






\section{Overview of the Baseline Methods}
\label{appsec:baseline_methods}

% tsFCI~\citep{entner2010causal}, VAR-LiNGAM~\citep{hyvarinen2010estimation}, PCMCI~\citep{runge2019detecting}, Granger Causality~\citep{granger1969investigating}, DYNOTEARS~\citep{pamfil2020dynotears}, and SLARAC~\citep{weichwald2020causal}

Time-Series Fast Causal Inference (tsFCI)~\citep{entner2010causal} is an extension of the Fast Causal Inference (FCI)~\citep{spirtes2013causal} algorithm for time-series data. It is designed to infer causal relationships from temporal data, taking into account potential latent confounders. The output is a Partial Ancestral Graph (PAG) with repeating structures over time steps to reflect the time-lagged causal impact. The predicted result allows unoriented edges, here we interpret such edge as bidirectional.

VAR-LiNGAM (Vector Autoregressive Linear Non-Gaussian Acyclic Model)~\citep{hyvarinen2010estimation} enhances the LiNGAM~\citep{shimizu2006linear} framework with vector autoregressive (VAR) models, enabling causal discovery in multivariate time series assuming non-Gaussian distribution and acyclic causal graph. 

PCMCI (Peter and Clark Momentary Conditional Independence)~\citep{runge2019detecting}, is a method designed to detect causal links in time series data. PCMCI extends the PC algorithm for temporal data, combining it with momentary conditional independence tests to account for lagged dependencies and reduce the risk of false discoveries in highly autocorrelated data. PCMCI identifies causal links at different levels of time lags, hence its time-unrolled causal graphs will be fully oriented, while the conventional causal graph may have contemporaneous, bidirectional or feedback processes, In our work, such link is interpreted as bidirectional.

Granger Causality~\citep{granger1969investigating} is a statistical hypothesis test used to determine whether one time-series can predict another, implying a causal relationship. It is based on the idea that a time-series $\mathbf{X}$ is said to Granger-cause another series 
$\mathbf{Y}$ if including past values of $\mathbf{X}$ improves the prediction of 
$\mathbf{Y}$ over a model that only uses past values of $\mathbf{Y}$.

DYNOTEARS~\citep{pamfil2020dynotears} is a Bayesian structure learning algorithm for time series data. It extends the DAG-NOTEARS~\citep{zheng2018dags} framework, allowing for detection of contemporaneous and time-lagged relationships between variables in a time-series. The method enforces acyclicity via a differentiable constraint, enabling scalable and accurate causal discovery. 

SLARAC (Subsampled Linear Auto-Regression Absolute Coefficients)~\citep{weichwald2020causal} is among the \texttt{tidybench} time-series causal discovery benchmarks. It is among the best performing models in the NeurIPS 2019 Causality for Climate competition~\citep{runge2020causality}, emphasizes the use of large regression coefficients over traditional small $p$-value thresholds. 

% The PC (Peter-Clark) algorithm~\citep{spirtes2001causation} is a constraint-based method for causal discovery that applies conditional independence tests iteratively to construct a causal graph, assuming no latent confounders and that relationships follow causal Markov and faithfulness assumptions. It first generates an undirected skeleton of variables connected by edges where dependencies are found, then orients edges using rules based on these dependencies. The predicted result is the Markov equivalent structure of the true causal structure, hence allowing unoriented edges.

% Fast Causal Inference (FCI)~\citep{spirtes2013causal}, an extension of the PC algorithm, is designed to handle scenarios with hidden confounders and selection bias, making it suitable for more realistic datasets where these issues are present. FCI combines conditional independence tests with more sophisticated orientation rules to produce a Partial Ancestral Graph (PAG) that indicates potential causal directions while accounting for possible hidden variables. It allows both bidirectional edges and unoriented edges. 

% LiNGAM (Linear Non-Gaussian Acyclic Model)~\citep{shimizu2006linear} takes a different approach by assuming linear, non-Gaussian relationships and utilizing Independent Component Analysis (ICA) to determine causal directions in an acyclic structure. This method is particularly effective when causal relationships are linear and non-Gaussian, as it leverages statistical properties of non-Gaussianity to identify the structure without requiring conditional independence tests. It produces a fully directed, acyclic graph (DAG).

% PCMCI (Peter and Clark Momentary Conditional Independence)~\citep{runge2019detecting}, is a method designed to detect causal links in time series data. PCMCI extends the PC algorithm for temporal data, combining it with momentary conditional independence tests to account for lagged dependencies and reduce the risk of false discoveries in highly autocorrelated data. PCMCI identifies causal links at different levels of time lags, hence its time-unrolled causal graphs will be fully oriented, while the conventional causal graph may have bidirectional or feedback processes, In our work, such link is interpreted as bidirectional.

Causal discovery with observational data is a rich and fast-developing field, offering diverse methods and baselines. For a comprehensive review of recent advances, please refer to~\citep{camps2023discovering}, where methodologies, challenges and applications are discussed in depth.


\section{Experimental Setup}
\label{appsec:exp_setup}

RESIT is implemented with the \texttt{LiNGAM} library and MLP regressor from \texttt{scikit-learn}~\citep{scikit-learn, sklearn_api}. To ensure a good enough fit quality, we choose the following parameters for they MLP regressor: 2 hidden layers of size 100, max number of iteration 1000.

tsFCI is implemented with FCI from \texttt{causal-learn} using the \texttt{fisherz} as independence test and level of significance $\alpha=0.05$; VAR-LiNGAM and Granger implemented with \texttt{causal-learn} with max lag of 1; DYNOTEARS is implemented using the \texttt{Causalnex} library, and SLARAC from \texttt{tidybench} with max lag of 1. PCMCI from the \texttt{tigramite} package is implemented using robust correlation (\texttt{RobustCorr}), with a significance level of $\alpha_{pc}=0.2$ for PC and a maximum lag of 1. As recommended by the original authors~\citep{runge2019detecting}, $\alpha_{pc}$ should not be interpreted as a strict measure of statistical significance. Instead, it is used in Phase 1 (PC) to reduce the search space for independence tests, and a slightly larger value for $\alpha_{pc}$ often provides faster inference while maintaining relatively good accuracy. 
% For MXMap on simulated data, the lag $\tau$ is set to 1, and embedding dimension $E$ is chosen to be the number of variables of each simulated system. We use random seed 97 for the experiments that compares performance against baseline methods.




% For RESIT and LiNGAM, we adopt python implementations in the \texttt{LiNGAM} library~\citep{shimizu2014lingam}; PC and FCI are from the \texttt{causal-learn} library~\citep{zheng2024causal}; PCMCI is from the \texttt{tigramite} library~\citep{runge2023causal}. Detailed experimental setup is provided in Appendix~\ref{appsec:exp_setup}.




\section{Evaluation Metrics for Graph Outputs}
\label{appsec:metrics}
The adjacency matrix of each output graph is defined with entries of 0 and 1, where the row index represents the cause variable and the column index represents the effect variable. An entry of 1 at $(row, column)$ indicates the existence of a causal edge between the two variables. To evaluate the model performance, the following metric scores are calculated on the adjacency matrices of the ground truth and the predicted graphs
% ~\footnote{A tool for manual evaluation with a graphical interface has been implemented and can be found at \url{https://github.com/elisejiuqizhang/DAGComparisonToolbox}.}
:

\begin{itemize}
    \item \textbf{Precision:} Precision measures the proportion of correctly identified causal edges out of all edges predicted by the model. Mathematically, it is defined as the number of true positive edges divided by the total number of predicted edges. In the context of adjacency matrices, it is calculated as:
        \begin{equation}
            \text { Precision }=\frac{\sum_{i, j} \mathbb{I}\left(\hat{A}_{i j}=1 \text { and } A_{i j}=1\right)}{\sum_{i, j} \mathbb{I}\left(\hat{A}_{i j}=1\right)}
        \end{equation}
        where $\hat{A}$ is the predicted adjacency matrix, $A$ is the ground truth adjacency matrix, and $\mathbb{I}$ is the indicator function.
    \item \textbf{Recall:} Recall measures the proportion of correctly identified causal edges out of all actual edges in the ground truth. It is defined as the number of true positive edges divided by the total number of true edges. For adjacency matrices, recall is calculated as:
    \begin{equation}
            \text { Recall }=\frac{\sum_{i, j} \mathbb{I}\left(\hat{A}_{i j}=1 \text { and } A_{i j}=1\right)}{\sum_{i, j} \mathbb{I}\left(A_{i j}=1\right)}
        \end{equation}
        
    \item \textbf{F1:} The F1 score is the harmonic mean of precision and recall, providing a single metric that balances both concerns. It is particularly useful when the class distribution is imbalanced. The F1 score for adjacency matrices is calculated as:
    \begin{equation}
        \text { F1 Score }=2 \times \frac{\text { Precision } \times \text { Recall }}{\text { Precision }+ \text { Recall }}
    \end{equation}
    \item \textbf{Structural Hamming Distance (SHD):} It counts the number of edge additions, deletions, and reversals needed to transform the predicted adjacency matrix into the ground truth adjacency matrix. A lower SHD indicates a closer match to the true causal structure. Formally, SHD is calculated as:
    \begin{equation}
        \mathrm{SHD}=\sum_{i, j} \mathbb{I}\left(\hat{A}_{i j} \neq A_{i j}\right)
    \end{equation}
\end{itemize}

These four metrics provide a comprehensive evaluation of a model's performance in inferring causal structures from multivariate dynamical systems.


% \section{Ground Truth Structures}
% \label{appsec:ground_truth}

% Table~\ref{apptab:comparison_3V}
%  and Table~\ref{apptab:comparison_4V} show the ground truth causal graphs of data used in experiments in Section~\ref{sec:baseline_sim}. The terms "direct, indirect, both, cycle" are used to characterize the type of causal relationship between variables $\mathbf{x}$ and $\mathbf{y}$, In experiments, if a causal inference method cannot determine the direction of a link, we treat the link as bidirectional, reflect the bidirectionality in the adjacency matrix and calculate the metrics accordingly.
 
% \begin{table}[htb]
%     \centering
%     % First part of the table (3V settings)
%     \begin{tabular}{c|c|c|c}
%         3V Direct & 3V Indirect & 3V Both & 3V Cycle \\ \hline
%         \begin{minipage}{.17\linewidth} \centering \includegraphics[width=\linewidth]{imgs/ComparisonBaselinesSim/3V Direct 1.png} \end{minipage} & 
%         \begin{minipage}{.17\linewidth} \centering \includegraphics[width=\linewidth]{imgs/ComparisonBaselinesSim/3V Indirect .png} \end{minipage} &  
%         \begin{minipage}{.16\linewidth} \centering \includegraphics[width=\linewidth]{imgs/ComparisonBaselinesSim/3V Both no Cycle .png} \end{minipage} &  
%         \begin{minipage}{.16\linewidth} \centering \includegraphics[width=\linewidth]{imgs/ComparisonBaselinesSim/3V Both Cycle .png} \end{minipage}
%     \end{tabular}
%     \caption{Ground truth structures for 3V settings}
%     \label{apptab:comparison_3V}
% \end{table}

% \vspace{1em} % Adds some space between the two parts of the table

% \begin{table}[htb]
%     \centering

% \makebox[\linewidth]{%

% \resizebox{1.2\textwidth}{!}{% Adjust the scaling factor to exceed column width

%     % Second part of the table (4V settings)
%     \begin{tabular}{c|c|c|c|c|c}
%         4V Direct 1 & 4V Direct 2 & 4V Indirect 1 & 4V Indirect 2 & 4V Both & 4V Cycle \\ \hline
%         \begin{minipage}{.17\linewidth} \centering \includegraphics[width=\linewidth]{imgs/ComparisonBaselinesSim/4V Direct 1.png} \end{minipage} & \begin{minipage}{.14\linewidth} \centering \includegraphics[width=\linewidth]{imgs/ComparisonBaselinesSim/4V Direct 2 .png} \end{minipage} & 
%         \begin{minipage}{.17\linewidth} \centering \includegraphics[width=\linewidth]{imgs/ComparisonBaselinesSim/4V Indirect 1 .png} \end{minipage} & \begin{minipage}{.135\linewidth} \centering \includegraphics[width=\linewidth]{imgs/ComparisonBaselinesSim/4V Indirect 2.png} \end{minipage} &
%         \begin{minipage}{.165\linewidth} \centering \includegraphics[width=\linewidth]{imgs/ComparisonBaselinesSim/4V Both no Cycle 1 .png} \end{minipage} &
%         \begin{minipage}{.165\linewidth} \centering \includegraphics[width=\linewidth]{imgs/ComparisonBaselinesSim/4V Both Cycle 1 .png} \end{minipage} 
%     \end{tabular}
%     }
%     }
%     \caption{Ground truth structures for 4V settings}
%     \label{apptab:comparison_4V}
% \end{table}





\section{MXMap: Current Limits}
\label{appsec:mxmap_limits}

\subsection{Runtime Complexity}

MXMap has a runtime complexity of $\mathcal{O}(n^2)$ due to its pairwise processing design. To validate the runtime behavior, we conduct additional experiments using the chain causal structure with an increasing number of variables (from 3V chain to 8V chain) and plot the runtime in Fig.~\ref{appfig:mxmap_runtime}, where we can confirm the $\mathcal{O}(n^2)$ complexity.

\begin{figure}[htb]
    \centering
    \includegraphics[width=0.7\linewidth]{imgs/MXMap_runtime.png}
    \caption{MXMap runtime as number of variables increase (chain structure).}
    \label{appfig:mxmap_runtime}
\end{figure}

To compare MXMap with the selected baseline methods, we track their runtime on the 8-variable chain (the highest dimensional system used in this work) with input length 4000. The approximate average runtimes were as follows:
\begin{itemize}
    \item \textbf{MXMap:} 40 seconds
    \item \textbf{tsFCI:} 15 seconds
    \item \textbf{Granger:} 3 seconds
    \item \textbf{VAR-LiNGAM:} 4 seconds
    \item \textbf{DYNOTEARS:} 5 seconds
    \item \textbf{SLARAC:} 15 seconds
\end{itemize}
% \textbf{MXMap:} 40 seconds; \textbf{tsFCI:} 15 seconds; \textbf{Granger:} 3 seconds; \textbf{VAR-LiNGAM:} 4 seconds; \textbf{DYNOTEARS:} 5 seconds; \textbf{SLARAC:} 15 seconds.

Although MXMap is not the fastest method, its causal detection performance remains the most satisfactory, since its assumptions are tailored to nonlinear coupled dynamical systems such as the simulated systems used in the experiments. The $\mathcal{O}(n^2)$ complexity may have limited the method's scalability, it is precisely this design that enables MXMap to reliably detect causal cycles, a capability lacking in many alternative methods. This can be interpreted as a trade-off between computational cost and performance highlights MXMap’s ability to handle cyclic causal structures and achieve strong causal discovery in systems that match its assumptions.

\subsection{Potential Failure Cases}

MXMap, like other cross-mapping-based methods, relies on specific assumptions that may limit its applicability under certain conditions. The following outlines the potential failure cases for MXMap:

\begin{enumerate}
    \item \textbf{System doesn't have an attractor:} The presence of an attractor is fundamental to all cross-mapping-based methods, ensuring that delay embedding reconstruction accurately captures the system’s state-space structure. If such assumption is violated (e.g.,  transient systems, non-stationary time series, or systems dominated by stochasticity), there won't be reliable stable state-space structure to correctly establish time-index correspondence for nearest neighborhoods.
    \item \textbf{Strong synchrony or forcing:} As discussed in the original CCM paper~\citep{sugihara2012detecting}, strong forcing or synchrony can obscure the intrinsic dynamics of the forced (effect) variable, driving it to adopt the almost same manifold topology as the forcing (causal) variable. This compromises the reconstruction of independent state spaces, leading to unreliable causal inference.
    \item \textbf{Improper lag $\tau$ and dimension $E$ when constructing delay embeddings:} Incorrect choices of the delay-embedding parameters can hinder state-space reconstruction and lead to failure in CCM, as discussed by \href{https://www.youtube.com/watch?v=lKj4hr_2-Vg}{the tutorial} with examples.
    \item \textbf{High noise levels:} Observations with high noise may lead to failure in cross maps~\citep{monster2017causal}.
\end{enumerate}





\section{Complete Causal Discovery Evaluation of MXMap and Baseline Methods on Simulated Systems}
\label{appsec:complet}

In this section, we present the complete analysis of performance for each model. The ground truth causal graphs and predicted graphs are visualized for each coupled system, and the four metrics are calculated for evaluation (best score in bold). In most of the cases, MXMap demonstrates optimal or second-best performance in discovering the underlying causal structure.

\begin{table}[htb]
\begin{tabular}{l|c|c|c|c|c|c}
Method    & Ground Truth      & Predicted & Precision     & Recall       & F1            & SHD        \\ \hline
tsFCI     & \multirow{7}{*}[-3em]{\begin{minipage}{.17\linewidth} \centering \includegraphics[width=\linewidth]{imgs/sim_new/gt/3V_chain_gt.png} \end{minipage}} & \begin{minipage}{.17\linewidth} \centering \includegraphics[width=\linewidth]{imgs/sim_new/pred/3V/3V_chain_tsfci_noN.png} \end{minipage}& 0             & 0            & 0             & 4          \\
VARLiNGAM &                   &\begin{minipage}{.17\linewidth} \centering \includegraphics[width=\linewidth]{imgs/sim_new/pred/3V/3V_chain_varlingam_noN.png} \end{minipage}& 0.50          & \textbf{1.0} & 0.67          & 2          \\
Granger   &                   &\begin{minipage}{.17\linewidth} \centering \includegraphics[width=\linewidth]{imgs/sim_new/pred/3V/3V_chain_granger_noN.png} \end{minipage}& 0             & 0            & 0             & 4          \\
PCMCI     &                   &\begin{minipage}{.17\linewidth} \centering \includegraphics[width=\linewidth]{imgs/sim_new/pred/3V/3V_chain_pcmci_noN.png} \end{minipage}& \textbf{0.67} & \textbf{1.0} & \textbf{0.80} & \textbf{1} \\
DYNOTEARS &                   &\begin{minipage}{.17\linewidth} \centering \includegraphics[width=\linewidth]{imgs/sim_new/pred/3V/3V_chain_dynotears_noN.png} \end{minipage}& 0.40          & \textbf{1.0} & 0.57          & 3          \\
SLARAC    &                   &\begin{minipage}{.17\linewidth} \centering \includegraphics[width=\linewidth]{imgs/sim_new/pred/3V/3V_chain_slarac_noN.png} \end{minipage}& 0             & 0            & 0             & 6          \\ \cline{1-1} \cline{3-7} 
MXMap     &                   & \begin{minipage}{.17\linewidth} \centering \includegraphics[width=\linewidth]{imgs/sim_new/pred/3V/3V_chain_mxmap_noN.png} \end{minipage}& \textbf{0.67} & \textbf{1.0} & \textbf{0.80} & \textbf{1}
\end{tabular}
\caption{3V Chain (No Noise)}
\label{tab:3V_chain_noN}
\end{table}

\begin{table}[htb]
\begin{tabular}{l|c|c|c|c|c|c}
Method    & Ground Truth      & Predicted & Precision    & Recall        & F1           & SHD        \\ \hline
tsFCI     & \multirow{7}{*}[-3em]{\begin{minipage}{.17\linewidth} \centering \includegraphics[width=\linewidth]{imgs/sim_new/gt/3V_chain_gt.png} \end{minipage}} & \begin{minipage}{.17\linewidth} \centering \includegraphics[width=\linewidth]{imgs/sim_new/pred/3V/3V_chain_tsfci_gN.png} \end{minipage}& 0.33         & 0.50          & 0.40         & 3          \\
VARLiNGAM &                   & \begin{minipage}{.17\linewidth} \centering \includegraphics[width=\linewidth]{imgs/sim_new/pred/3V/3V_chain_varlingam_gN.png} \end{minipage}& 0            & 0    & 0            & 4          \\
Granger   &                   & \begin{minipage}{.17\linewidth} \centering \includegraphics[width=\linewidth]{imgs/sim_new/pred/3V/3V_chain_granger_gN.png} \end{minipage} & 0            & 0             & 0            & 4          \\
PCMCI     &                   &\begin{minipage}{.17\linewidth} \centering \includegraphics[width=\linewidth]{imgs/sim_new/pred/3V/3V_chain_pcmci_gN.png} \end{minipage}& \textbf{1.0} & \textbf{1.0}  & \textbf{1.0} & \textbf{0} \\
DYNOTEARS &                   &\begin{minipage}{.17\linewidth} \centering \includegraphics[width=\linewidth]{imgs/sim_new/pred/3V/3V_chain_dynotears_gN.png} \end{minipage} & 0.25         & 0.50 & 0.33         & 4          \\
SLARAC    &                   & \begin{minipage}{.17\linewidth} \centering \includegraphics[width=\linewidth]{imgs/sim_new/pred/3V/3V_chain_slarac_gN.png} \end{minipage}& 0            & 0             & 0            & 6          \\ \cline{1-1} \cline{3-7} 
MXMap     &                   &\begin{minipage}{.17\linewidth} \centering \includegraphics[width=\linewidth]{imgs/sim_new/pred/3V/3V_chain_mxmap_gN.png} \end{minipage} & \textbf{1.0} & \textbf{1.0}  & \textbf{1.0} & \textbf{0}
\end{tabular}
\caption{3V Chain (Gaussian Additive Noise, Level 0.01)}
\label{tab:3V_chain_gN}
\end{table}

\begin{table}[htb]
\begin{tabular}{l|c|c|c|c|c|c}
Method    & Ground Truth      & Predicted & Precision    & Recall       & F1           & SHD        \\ \hline
tsFCI     & \multirow{7}{*}[-9.5em]{\begin{minipage}{.17\linewidth} \centering \includegraphics[width=\linewidth]{imgs/sim_new/gt/3V_immorality_gt.png} \end{minipage}} & \begin{minipage}{.17\linewidth} \centering \includegraphics[width=\linewidth]{imgs/sim_new/pred/3V/3V_immorality_tsfci_noN.png} \end{minipage}          & 0.50         & 1.0          & 0.67         & 2          \\
VARLiNGAM &                   & \begin{minipage}{.17\linewidth} \centering \includegraphics[width=\linewidth]{imgs/sim_new/pred/3V/3V_immorality_varlingam_noN.png} \end{minipage}  & 0            & 0            & 0            & 4          \\
Granger   &                   & \begin{minipage}{.17\linewidth} \centering \includegraphics[width=\linewidth]{imgs/sim_new/pred/3V/3V_immorality_granger_noN.png} \end{minipage}  & 0            & 0            & 0            & 4          \\
PCMCI     &                   & \begin{minipage}{.17\linewidth} \centering \includegraphics[width=\linewidth]{imgs/sim_new/pred/3V/3V_immorality_pcmci_noN.png} \end{minipage}  & \textbf{1.0} & \textbf{1.0} & \textbf{1.0} & \textbf{0} \\
DYNOTEARS &                   & \begin{minipage}{.17\linewidth} \centering \includegraphics[width=\linewidth]{imgs/sim_new/pred/3V/3V_immorality_dynotears_noN.png} \end{minipage}  & 0.50         & \textbf{1.0} & 0.67         & 2          \\
SLARAC    &                   & \begin{minipage}{.17\linewidth} \centering \includegraphics[width=\linewidth]{imgs/sim_new/pred/3V/3V_immorality_slarac_noN.png} \end{minipage}  & 0            & 0            & 0            & 6          \\ \cline{1-1} \cline{3-7} 
MXMap     &                   & \begin{minipage}{.17\linewidth} \centering \includegraphics[width=\linewidth]{imgs/sim_new/pred/3V/3V_immorality_mxmap_noN.png} \end{minipage}  & 0.67         & \textbf{1.0} & 0.80         & \textbf{1}
\end{tabular}
\caption{3V Immorality (No Noise)}
\label{tab:3V_immo_noN}
\end{table}

\begin{table}[htb]
\begin{tabular}{l|c|c|c|c|c|c}
Method    & Ground Truth      & Predicted & Precision    & Recall       & F1           & SHD        \\ \hline
tsFCI     & \multirow{7}{*}[-9.5em]{\begin{minipage}{.17\linewidth} \centering \includegraphics[width=\linewidth]{imgs/sim_new/gt/3V_immorality_gt.png} \end{minipage}} & \begin{minipage}{.17\linewidth} \centering \includegraphics[width=\linewidth]{imgs/sim_new/pred/3V/3V_immorality_tsfci_gN.png} \end{minipage}           & 0.50         & \textbf{1.0} & 0.67         & 2          \\
VARLiNGAM &                   & \begin{minipage}{.17\linewidth} \centering \includegraphics[width=\linewidth]{imgs/sim_new/pred/3V/3V_immorality_varlingam_gN.png} \end{minipage}  & 0            & 0            & 0            & 4          \\
Granger   &                   & \begin{minipage}{.17\linewidth} \centering \includegraphics[width=\linewidth]{imgs/sim_new/pred/3V/3V_immorality_granger_gN.png} \end{minipage} & 0            & 0            & 0            & 4          \\
PCMCI     &                   & \begin{minipage}{.17\linewidth} \centering \includegraphics[width=\linewidth]{imgs/sim_new/pred/3V/3V_immorality_pcmci_gN.png} \end{minipage}  & \textbf{1.0} & \textbf{1.0} & \textbf{1.0} & \textbf{0} \\
DYNOTEARS &                   &  \begin{minipage}{.17\linewidth} \centering \includegraphics[width=\linewidth]{imgs/sim_new/pred/3V/3V_immorality_dynotears_gN.png} \end{minipage}  & 0.50         & \textbf{1.0} & 0.67         & 2          \\
SLARAC    &                   & \begin{minipage}{.17\linewidth} \centering \includegraphics[width=\linewidth]{imgs/sim_new/pred/3V/3V_immorality_slarac_gN.png} \end{minipage}  & 0            & 0            & 0            & 6          \\ \cline{1-1} \cline{3-7} 
MXMap     &                   &  \begin{minipage}{.17\linewidth} \centering \includegraphics[width=\linewidth]{imgs/sim_new/pred/3V/3V_immorality_mxmap_gN.png} \end{minipage}  & \textbf{1.0} & \textbf{1.0} & \textbf{1.0} & \textbf{0}
\end{tabular}
\caption{3V Immorality (Gaussian Additive Noise, Level 0.01)}
\label{tab:3V_immo_gN}
\end{table}


\begin{table}[htb]
\begin{tabular}{l|c|c|c|c|c|c}
Method    & Ground Truth      & Predicted & Precision    & Recall       & F1           & SHD        \\ \hline
tsFCI     & \multirow{7}{*}[-6.2em]{\begin{minipage}{.17\linewidth} \centering \includegraphics[width=\linewidth]{imgs/sim_new/gt/3V_both_noCycle_gt.png} \end{minipage}} &\begin{minipage}{.17\linewidth} \centering \includegraphics[width=\linewidth]{imgs/sim_new/pred/3V/3V_noCycle_tsfci_noN.png} \end{minipage}           & 0            & 0            & 0            & 6          \\
VARLiNGAM &                   &  \begin{minipage}{.17\linewidth} \centering \includegraphics[width=\linewidth]{imgs/sim_new/pred/3V/3V_noCycle_varlingam_noN.png} \end{minipage} & 0.25         & 0.33         & 0.29         & 5          \\
Granger   &                   &\begin{minipage}{.17\linewidth} \centering \includegraphics[width=\linewidth]{imgs/sim_new/pred/3V/3V_noCycle_granger_noN.png} \end{minipage}  & 0.25         & 0.33         & 0.29         & 5          \\
PCMCI     &                   & \begin{minipage}{.17\linewidth} \centering \includegraphics[width=\linewidth]{imgs/sim_new/pred/3V/3V_noCycle_pcmci_noN.png} \end{minipage} & 0.60         & \textbf{1.0} & 0.75         & 2          \\
DYNOTEARS &                   & \begin{minipage}{.17\linewidth} \centering \includegraphics[width=\linewidth]{imgs/sim_new/pred/3V/3V_noCycle_dynotears_noN.png} \end{minipage} & 0.50         & 0.67         & 0.57         & 3          \\
SLARAC    &                   &  \begin{minipage}{.17\linewidth} \centering \includegraphics[width=\linewidth]{imgs/sim_new/pred/3V/3V_noCycle_slarac_noN.png} \end{minipage} & 0.25         & 0.33         & 0.29         & 5          \\ \cline{1-1} \cline{3-7} 
MXMap     &                   & \begin{minipage}{.17\linewidth} \centering \includegraphics[width=\linewidth]{imgs/sim_new/pred/3V/3V_noCycle_mxmap_noN.png} \end{minipage} & \textbf{1.0} & \textbf{1.0} & \textbf{1.0} & \textbf{0}
\end{tabular}
\caption{3V No Cycle (No Noise)}
\label{tab:3V_noCycle_noN}
\end{table}


\begin{table}[htb]
\begin{tabular}{l|c|c|c|c|c|c}
Method    & Ground Truth      & Predicted & Precision    & Recall       & F1           & SHD        \\ \hline
tsFCI     & \multirow{7}{*}[-6.2em]{\begin{minipage}{.17\linewidth} \centering \includegraphics[width=\linewidth]{imgs/sim_new/gt/3V_both_noCycle_gt.png} \end{minipage}} & \begin{minipage}{.17\linewidth} \centering \includegraphics[width=\linewidth]{imgs/sim_new/pred/3V/3V_noCycle_tsfci_gN.png} \end{minipage}          & 0.40         & 0.67         & 0.5          & 4          \\
VARLiNGAM &                   & \begin{minipage}{.17\linewidth} \centering \includegraphics[width=\linewidth]{imgs/sim_new/pred/3V/3V_noCycle_varlingam_gN.png} \end{minipage} & 0            & 0            & 0            & 6          \\
Granger   &                   & \begin{minipage}{.17\linewidth} \centering \includegraphics[width=\linewidth]{imgs/sim_new/pred/3V/3V_noCycle_granger_gN.png} \end{minipage} & 0            & 0            & 0            & 6          \\
PCMCI     &                   & \begin{minipage}{.17\linewidth} \centering \includegraphics[width=\linewidth]{imgs/sim_new/pred/3V/3V_noCycle_pcmci_gN.png} \end{minipage} & 0.75         & \textbf{1.0} & 0.86         & 1          \\
DYNOTEARS &                   & \begin{minipage}{.17\linewidth} \centering \includegraphics[width=\linewidth]{imgs/sim_new/pred/3V/3V_noCycle_dynotears_gN.png} \end{minipage}& 0.75         & \textbf{1.0} & 0.86         & 1          \\
SLARAC    &                   & \begin{minipage}{.17\linewidth} \centering \includegraphics[width=\linewidth]{imgs/sim_new/pred/3V/3V_noCycle_slarac_gN.png} \end{minipage} & 0            & 0            & 0            & 6          \\ \cline{1-1} \cline{3-7} 
MXMap     &                   & \begin{minipage}{.17\linewidth} \centering \includegraphics[width=\linewidth]{imgs/sim_new/pred/3V/3V_noCycle_mxmap_gN.png} \end{minipage} & \textbf{1.0} & \textbf{1.0} & \textbf{1.0} & \textbf{0}
\end{tabular}
\caption{3V No Cycle (Gaussian Additive Noise, Level 0.01)}
\label{tab:3V_noCycle_gN}
\end{table}

\begin{table}[htb]
\begin{tabular}{l|c|c|c|c|c|c}
Method    & Ground Truth      & Predicted & Precision    & Recall       & F1           & SHD        \\ \hline
tsFCI     & \multirow{7}{*}[-6.2em]{\begin{minipage}{.17\linewidth} \centering \includegraphics[width=\linewidth]{imgs/sim_new/gt/3V_Cycle_gt.png} \end{minipage}} &  \begin{minipage}{.17\linewidth} \centering \includegraphics[width=\linewidth]{imgs/sim_new/pred/3V/3V_Cycle_tsfci_noN.png} \end{minipage}         & 0.50         & 0.67         & 0.57         & 3          \\
VARLiNGAM &                   & \begin{minipage}{.17\linewidth} \centering \includegraphics[width=\linewidth]{imgs/sim_new/pred/3V/3V_Cycle_varlingam_noN.png} \end{minipage} & 0.50         & 0.67         & 0.57         & 3          \\
Granger   &                   & \begin{minipage}{.17\linewidth} \centering \includegraphics[width=\linewidth]{imgs/sim_new/pred/3V/3V_Cycle_granger_noN.png} \end{minipage} & 0.67         & 0.67         & 0.67         & 2          \\
PCMCI     &                   & \begin{minipage}{.17\linewidth} \centering \includegraphics[width=\linewidth]{imgs/sim_new/pred/3V/3V_Cycle_pcmci_noN.png} \end{minipage} & 0.50         & \textbf{1.0} & 0.67         & 3          \\
DYNOTEARS &                   & \begin{minipage}{.17\linewidth} \centering \includegraphics[width=\linewidth]{imgs/sim_new/pred/3V/3V_Cycle_dynotears_noN.png} \end{minipage} & 0            & 0            & 0            & 5          \\
SLARAC    &                   & \begin{minipage}{.17\linewidth} \centering \includegraphics[width=\linewidth]{imgs/sim_new/pred/3V/3V_Cycle_slarac_noN.png} \end{minipage} & 0.25         & 0.33         & 0.29         & 5          \\ \cline{1-1} \cline{3-7} 
MXMap     &                   & \begin{minipage}{.17\linewidth} \centering \includegraphics[width=\linewidth]{imgs/sim_new/pred/3V/3V_Cycle_mxmap_noN.png} \end{minipage} & \textbf{1.0} & \textbf{1.0} & \textbf{1.0} & \textbf{0}
\end{tabular}
\caption{3V Cycle (No Noise)}
\label{tab:3V_Cycle_noN}
\end{table}

\begin{table}[htb]
\begin{tabular}{l|c|c|c|c|c|c}
Method    & Ground Truth      & Predicted & Precision    & Recall       & F1           & SHD        \\ \hline
tsFCI     & \multirow{7}{*}[-6.2em]{\begin{minipage}{.17\linewidth} \centering \includegraphics[width=\linewidth]{imgs/sim_new/gt/3V_Cycle_gt.png} \end{minipage}} & \begin{minipage}{.17\linewidth} \centering \includegraphics[width=\linewidth]{imgs/sim_new/pred/3V/3V_Cycle_tsfci_gN.png} \end{minipage}    & 0            & 0            & 0            & 5          \\
VARLiNGAM &                   &  \begin{minipage}{.17\linewidth} \centering \includegraphics[width=\linewidth]{imgs/sim_new/pred/3V/3V_Cycle_varlingam_gN.png} \end{minipage}  & 0.33         & 0.33         & 0.33         & 4          \\
Granger   &                   &  \begin{minipage}{.17\linewidth} \centering \includegraphics[width=\linewidth]{imgs/sim_new/pred/3V/3V_Cycle_granger_gN.png} \end{minipage} & 0.33         & 0.33         & 0.33         & 4          \\
PCMCI     &                   &  \begin{minipage}{.17\linewidth} \centering \includegraphics[width=\linewidth]{imgs/sim_new/pred/3V/3V_Cycle_pcmci_gN.png} \end{minipage} & 0.50         & \textbf{1.0} & 0.67         & 3          \\
DYNOTEARS &                   & \begin{minipage}{.17\linewidth} \centering \includegraphics[width=\linewidth]{imgs/sim_new/pred/3V/3V_Cycle_dynotears_gN.png} \end{minipage}& 0.50         & 0.65         & 0.57         & 3          \\
SLARAC    &                   & \begin{minipage}{.17\linewidth} \centering \includegraphics[width=\linewidth]{imgs/sim_new/pred/3V/3V_Cycle_slarac_gN.png} \end{minipage} & 0.40         & 0.67         & 0.50         & 4          \\ \cline{1-1} \cline{3-7} 
MXMap     &                   & \begin{minipage}{.17\linewidth} \centering \includegraphics[width=\linewidth]{imgs/sim_new/pred/3V/3V_Cycle_mxmap_gN.png} \end{minipage} & \textbf{1.0} & \textbf{1.0} & \textbf{1.0} & \textbf{0}
\end{tabular}
\caption{3V Cycle (Gaussian Additive Noise, Level 0.01)}
\label{tab:3V_Cycle_gN}
\end{table}


\begin{table}[htb]
\begin{tabular}{l|c|c|c|c|c|c}
Method    & Ground Truth      & Predicted & Precision    & Recall       & F1           & SHD        \\ \hline
tsFCI     & \multirow{7}{*}[-0.7em]{\begin{minipage}{.17\linewidth} \centering \includegraphics[width=\linewidth]{imgs/sim_new/gt/4V_chain_gt.png} \end{minipage}} &  \begin{minipage}{.17\linewidth} \centering \includegraphics[width=\linewidth]{imgs/sim_new/pred/4V/4V_chain_tsfci_noN.png} \end{minipage}         & 0.50         & \textbf{1.0} & 0.67         & 3          \\
VARLiNGAM &                   &  \begin{minipage}{.17\linewidth} \centering \includegraphics[width=\linewidth]{imgs/sim_new/pred/4V/4V_chain_varlingam_noN.png} \end{minipage}     & 0.40         & 0.67         & 0.50         & 4          \\
Granger   &                   & \begin{minipage}{.17\linewidth} \centering \includegraphics[width=\linewidth]{imgs/sim_new/pred/4V/4V_chain_granger_noN.png} \end{minipage}    & 0.33         & 0.33         & 0.33         & 4          \\
PCMCI     &                   & \begin{minipage}{.17\linewidth} \centering \includegraphics[width=\linewidth]{imgs/sim_new/pred/4V/4V_chain_pcmci_noN.png} \end{minipage}    & 0.75         & \textbf{1.0} & 0.86         & 1          \\
DYNOTEARS &                   & \begin{minipage}{.17\linewidth} \centering \includegraphics[width=\linewidth]{imgs/sim_new/pred/4V/4V_chain_dynotears_noN.png} \end{minipage}     & 0.25         & 0.67         & 0.36         & 7          \\
SLARAC    &                   &           & 0.11         & 0.33         & 0.17         & 10         \\ \cline{1-1} \cline{3-7} 
MXMap     &                   & \begin{minipage}{.17\linewidth} \centering \includegraphics[width=\linewidth]{imgs/sim_new/pred/4V/4V_chain_mxmap_noN.png} \end{minipage}     & \textbf{1.0} & \textbf{1.0} & \textbf{1.0} & \textbf{0}
\end{tabular}
\caption{4V Chain (No Noise)}
\label{tab:4V_Chain_noN}
\end{table}

\begin{table}[htb]
\begin{tabular}{l|c|c|c|c|c|c}
Method    & Ground Truth      & Predicted & Precision    & Recall       & F1           & SHD        \\ \hline
tsFCI     & \multirow{7}{*}[-0.7em]{\begin{minipage}{.17\linewidth} \centering \includegraphics[width=\linewidth]{imgs/sim_new/gt/4V_chain_gt.png} \end{minipage}} & \begin{minipage}{.17\linewidth} \centering \includegraphics[width=\linewidth]{imgs/sim_new/pred/4V/4V_chain_tsfci_gN.png} \end{minipage}          & 0.50         & 0.67         & 0.57         & 3          \\
VARLiNGAM &                   & \begin{minipage}{.17\linewidth} \centering \includegraphics[width=\linewidth]{imgs/sim_new/pred/4V/4V_chain_varlingam_gN.png} \end{minipage} & 0            & 0            & 0            & 6          \\
Granger   &                   & \begin{minipage}{.17\linewidth} \centering \includegraphics[width=\linewidth]{imgs/sim_new/pred/4V/4V_chain_granger_gN.png} \end{minipage} & 0            & 0            & 0            & 6          \\
PCMCI     &                   & \begin{minipage}{.17\linewidth} \centering \includegraphics[width=\linewidth]{imgs/sim_new/pred/4V/4V_chain_pcmci_gN.png} \end{minipage} & \textbf{1.0} & \textbf{1.0} & \textbf{1.0} & \textbf{0} \\
DYNOTEARS &                   & \begin{minipage}{.17\linewidth} \centering \includegraphics[width=\linewidth]{imgs/sim_new/pred/4V/4V_chain_dynotears_gN.png} \end{minipage} & 0.11         & 0.33         & 0.17         & 10         \\
SLARAC    &                   & \begin{minipage}{.17\linewidth} \centering \includegraphics[width=\linewidth]{imgs/sim_new/pred/4V/4V_chain_slarac_gN.png} \end{minipage} & 0.11         & 0.33         & 0.17         & 10         \\ \cline{1-1} \cline{3-7} 
MXMap     &                   &  \begin{minipage}{.17\linewidth} \centering \includegraphics[width=\linewidth]{imgs/sim_new/pred/4V/4V_chain_mxmap_gN.png} \end{minipage}   & \textbf{1.0} & \textbf{1.0} & \textbf{1.0} & \textbf{0}
\end{tabular}
\caption{4V Chain (Gaussian Additive Noise, Level 0.01)}
\label{tab:4V_Chain_gN}
\end{table}

\begin{table}[htb]
\begin{tabular}{l|c|c|c|c|c|c}
Method    & Ground Truth      & Predicted & Precision     & Recall       & F1            & SHD        \\ \hline
tsFCI     & \multirow{7}{*}[-5em]{\begin{minipage}{.17\linewidth} \centering \includegraphics[width=\linewidth]{imgs/sim_new/gt/4V_both_noCycle_gt.png} \end{minipage}} & \begin{minipage}{.17\linewidth} \centering \includegraphics[width=\linewidth]{imgs/sim_new/pred/4V/4V_noCycle_tsfci_noN.png} \end{minipage}   & 0.40          & 0.50         & 0.44          & 5          \\
VARLiNGAM &                   & \begin{minipage}{.17\linewidth} \centering \includegraphics[width=\linewidth]{imgs/sim_new/pred/4V/4V_noCycle_varlingam_noN.png} \end{minipage} & 0             & 0            & 0             & 7          \\
Granger   &                   & \begin{minipage}{.17\linewidth} \centering \includegraphics[width=\linewidth]{imgs/sim_new/pred/4V/4V_noCycle_granger_noN.png} \end{minipage} & 0.40          & 0.50         & 0.44          & 5          \\
PCMCI     &                   & \begin{minipage}{.17\linewidth} \centering \includegraphics[width=\linewidth]{imgs/sim_new/pred/4V/4V_noCycle_pcmci_noN.png} \end{minipage} & 0.80          & \textbf{1.0} & \textbf{0.89} & \textbf{1} \\
DYNOTEARS &                   & \begin{minipage}{.17\linewidth} \centering \includegraphics[width=\linewidth]{imgs/sim_new/pred/4V/4V_noCycle_dynotears_noN.png} \end{minipage} & 0.60          & 0.75         & 0.67          & 3          \\
SLARAC    &                   & \begin{minipage}{.17\linewidth} \centering \includegraphics[width=\linewidth]{imgs/sim_new/pred/4V/4V_noCycle_slarac_noN.png} \end{minipage}  & 0.57          & \textbf{1.0} & 0.72          & 3          \\ \cline{1-1} \cline{3-7} 
MXMap     &                   &  \begin{minipage}{.17\linewidth} \centering \includegraphics[width=\linewidth]{imgs/sim_new/pred/4V/4V_noCycle_mxmap_noN.png} \end{minipage} & \textbf{0.80} & \textbf{1.0} & \textbf{0.89} & \textbf{1}
\end{tabular}
\caption{4V No Cycle (No Noise)}
\label{tab:4V_noCycle_noN}
\end{table}

\begin{table}[htb]
\begin{tabular}{l|c|c|c|c|c|c}
Method    & Ground Truth      & Predicted & Precision     & Recall       & F1            & SHD        \\ \hline
tsFCI     & \multirow{7}{*}[-5em]{\begin{minipage}{.17\linewidth} \centering \includegraphics[width=\linewidth]{imgs/sim_new/gt/4V_both_noCycle_gt.png} \end{minipage}} & \begin{minipage}{.17\linewidth} \centering \includegraphics[width=\linewidth]{imgs/sim_new/pred/4V/4V_noCycle_tsfci_gN.png} \end{minipage} & 0.50          & 0.75         & 0.60          & 4          \\
VARLiNGAM &                   & \begin{minipage}{.17\linewidth} \centering \includegraphics[width=\linewidth]{imgs/sim_new/pred/4V/4V_noCycle_varlingam_gN.png} \end{minipage} & 0             & 0            & 0             & 8          \\
Granger   &                   & \begin{minipage}{.17\linewidth} \centering \includegraphics[width=\linewidth]{imgs/sim_new/pred/4V/4V_noCycle_granger_gN.png} \end{minipage} & 0             & 0            & 0             & 8          \\
PCMCI     &                   & \begin{minipage}{.17\linewidth} \centering \includegraphics[width=\linewidth]{imgs/sim_new/pred/4V/4V_noCycle_pcmci_gN.png} \end{minipage} & \textbf{0.80} & \textbf{1.0} & \textbf{0.89} & \textbf{1} \\
DYNOTEARS &                   & \begin{minipage}{.17\linewidth} \centering \includegraphics[width=\linewidth]{imgs/sim_new/pred/4V/4V_noCycle_dynotears_gN.png} \end{minipage} & \textbf{0.80} & \textbf{1.0} & \textbf{0.89} & \textbf{1} \\
SLARAC    &                   &  \begin{minipage}{.17\linewidth} \centering \includegraphics[width=\linewidth]{imgs/sim_new/pred/4V/4V_noCycle_slarac_gN.png} \end{minipage} & 0.67          & \textbf{1.0} & 0.8           & 2          \\ \cline{1-1} \cline{3-7} 
MXMap     &                   & \begin{minipage}{.17\linewidth} \centering \includegraphics[width=\linewidth]{imgs/sim_new/pred/4V/4V_noCycle_mxmap_gN.png} \end{minipage}  & \textbf{0.80} & \textbf{1.0} & \textbf{0.89} & \textbf{1}
\end{tabular}
\caption{4V No Cycle (Gaussian Additive Noise, Level 0.01)}
\label{tab:4V_noCycle_gN}
\end{table}

\begin{table}[htb]
\begin{tabular}{l|c|c|c|c|c|c}
Method    & Ground Truth      & Predicted & Precision    & Recall       & F1           & SHD        \\ \hline
tsFCI     & \multirow{7}{*}[-4.6em]{\begin{minipage}{.17\linewidth} \centering \includegraphics[width=\linewidth]{imgs/sim_new/gt/4V_Cycle_gt.png} \end{minipage}} & \begin{minipage}{.17\linewidth} \centering \includegraphics[width=\linewidth]{imgs/sim_new/pred/4V/4V_Cycle_tsfci_noN.png} \end{minipage}    & 0.50         & 0.25         & 0.33         & 4          \\
VARLiNGAM &                   & \begin{minipage}{.17\linewidth} \centering \includegraphics[width=\linewidth]{imgs/sim_new/pred/4V/4V_Cycle_varlingam_noN.png} \end{minipage}  & 0.50         & 0.50         & 0.50         & 4          \\
Granger   &                   & \begin{minipage}{.17\linewidth} \centering \includegraphics[width=\linewidth]{imgs/sim_new/pred/4V/4V_Cycle_granger_noN.png} \end{minipage}  & \textbf{1.0} & 0.25         & 0.40         & 3          \\
PCMCI     &                   & \begin{minipage}{.17\linewidth} \centering \includegraphics[width=\linewidth]{imgs/sim_new/pred/4V/4V_Cycle_pcmci_noN.png} \end{minipage}  & 0.33         & \textbf{1.0} & 0.50         & 8          \\
DYNOTEARS &                   & \begin{minipage}{.17\linewidth} \centering \includegraphics[width=\linewidth]{imgs/sim_new/pred/4V/4V_Cycle_dynotears_noN.png} \end{minipage}  & 0.36         & \textbf{1.0} & 0.53         & 7          \\
SLARAC    &                   & \begin{minipage}{.17\linewidth} \centering \includegraphics[width=\linewidth]{imgs/sim_new/pred/4V/4V_Cycle_slarac_noN.png} \end{minipage}   & 0.44         & \textbf{1.0} & 0.62         & 5          \\ \cline{1-1} \cline{3-7} 
MXMap     &                   & \begin{minipage}{.17\linewidth} \centering \includegraphics[width=\linewidth]{imgs/sim_new/pred/4V/4V_Cycle_mxmap_noN.png} \end{minipage}  & \textbf{1.0} & \textbf{1.0} & \textbf{1.0} & \textbf{0}
\end{tabular}
\caption{4V Cycle (No Noise)}
\label{tab:4V_Cycle_noN}
\end{table}

\begin{table}[htb]
\begin{tabular}{l|c|c|c|c|c|c}
Method    & Ground Truth      & Predicted & Precision     & Recall       & F1            & SHD        \\ \hline
tsFCI     & \multirow{7}{*}[-4.6em]{\begin{minipage}{.17\linewidth} \centering \includegraphics[width=\linewidth]{imgs/sim_new/gt/4V_Cycle_gt.png} \end{minipage}} & \begin{minipage}{.17\linewidth} \centering \includegraphics[width=\linewidth]{imgs/sim_new/pred/4V/4V_Cycle_tsfci_gN.png} \end{minipage}   & 0.25          & 0.25         & 0.25          & 6          \\
VARLiNGAM &                   & \begin{minipage}{.17\linewidth} \centering \includegraphics[width=\linewidth]{imgs/sim_new/pred/4V/4V_Cycle_varlingam_gN.png} \end{minipage}   & 0.40          & 0.50         & 0.44          & 5          \\
Granger   &                   &  \begin{minipage}{.17\linewidth} \centering \includegraphics[width=\linewidth]{imgs/sim_new/pred/4V/4V_Cycle_granger_gN.png} \end{minipage}  & 0.25          & 0.25         & 0.25          & 6          \\
PCMCI     &                   & \begin{minipage}{.17\linewidth} \centering \includegraphics[width=\linewidth]{imgs/sim_new/pred/4V/4V_Cycle_pcmci_gN.png} \end{minipage}   & 0.50          & \textbf{1.0} & 0.67          & 4          \\
DYNOTEARS &                   & \begin{minipage}{.17\linewidth} \centering \includegraphics[width=\linewidth]{imgs/sim_new/pred/4V/4V_Cycle_dynotears_gN.png} \end{minipage}    & 0.38          & 0.75         & 0.50          & 6          \\
SLARAC    &                   &   \begin{minipage}{.17\linewidth} \centering \includegraphics[width=\linewidth]{imgs/sim_new/pred/4V/4V_Cycle_slarac_gN.png} \end{minipage}    & 0.38          & 0.75         & 0.50          & 6          \\ \cline{1-1} \cline{3-7} 
MXMap     &                   & \begin{minipage}{.17\linewidth} \centering \includegraphics[width=\linewidth]{imgs/sim_new/pred/4V/4V_Cycle_mxmap_gN.png} \end{minipage}    & \textbf{0.67} & \textbf{1.0} & \textbf{0.80} & \textbf{2}
\end{tabular}
\caption{4V Cycle (Gaussian Additive Noise, Level 0.01)}
\label{tab:4V_Cycle_gN}
\end{table}

\begin{table}[htb]
\begin{tabular}{l|c|c|c|c|c|c}
Method    & Ground Truth      & Predicted & Precision     & Recall       & F1            & SHD        \\ \hline
tsFCI     & \multirow{7}{*}[-9.6em]{\begin{minipage}{.17\linewidth} \centering \includegraphics[width=\linewidth]{imgs/sim_new/gt/5V-1_gt.png} \end{minipage}} &  \begin{minipage}{.17\linewidth} \centering \includegraphics[width=\linewidth]{imgs/sim_new/pred/5V/5V1_tsfci_noN.png} \end{minipage}& 0.29          & 0.50         & 0.36          & 7          \\
VARLiNGAM &                   & \begin{minipage}{.17\linewidth} \centering \includegraphics[width=\linewidth]{imgs/sim_new/pred/5V/5V1_varlingam_noN.png} \end{minipage} & 0.25          & 0.25         & 0.25          & 6          \\
Granger   &                   &  \begin{minipage}{.17\linewidth} \centering \includegraphics[width=\linewidth]{imgs/sim_new/pred/5V/5V1_granger_noN.png} \end{minipage} & 0.25          & 0.25         & 0.25          & 6          \\
PCMCI     &                   &  \begin{minipage}{.17\linewidth} \centering \includegraphics[width=\linewidth]{imgs/sim_new/pred/5V/5V1_pcmci_noN.png} \end{minipage}   & 0.42          & 0.75         & 0.55          & 5          \\
DYNOTEARS &                   &  \begin{minipage}{.17\linewidth} \centering \includegraphics[width=\linewidth]{imgs/sim_new/pred/5V/5V1_dynotears_noN.png} \end{minipage}  & 0.25          & 0.50         & 0.33          & 8          \\
SLARAC    &                   &  \begin{minipage}{.17\linewidth} \centering \includegraphics[width=\linewidth]{imgs/sim_new/pred/5V/5V1_slarac_noN.png} \end{minipage}  & 0.13          & 0.50         & 0.20          & 16         \\ \cline{1-1} \cline{3-7} 
MXMap     &                   &  \begin{minipage}{.17\linewidth} \centering \includegraphics[width=\linewidth]{imgs/sim_new/pred/5V/5V1_mxmap_noN.png} \end{minipage} & \textbf{0.80} & \textbf{1.0} & \textbf{0.89} & \textbf{1}
\end{tabular}
\caption{5V Structure 1 Without Cycle (No Noise)}
\label{tab:5V1_noN}
\end{table}

\begin{table}[htb]
\begin{tabular}{l|c|c|c|c|c|c}
Method    & Ground Truth      & Predicted & Precision     & Recall       & F1            & SHD        \\ \hline
tsFCI     & \multirow{7}{*}[-9.6em]{\begin{minipage}{.17\linewidth} \centering \includegraphics[width=\linewidth]{imgs/sim_new/gt/5V-1_gt.png} \end{minipage}} &  \begin{minipage}{.17\linewidth} \centering \includegraphics[width=\linewidth]{imgs/sim_new/pred/5V/5V1_tsfci_gN.png} \end{minipage} & 0.50          & 0.75         & 0.60          & 4          \\
VARLiNGAM &                   & \begin{minipage}{.17\linewidth} \centering \includegraphics[width=\linewidth]{imgs/sim_new/pred/5V/5V1_varlingam_gN.png} \end{minipage} & 0.25          & 0.25         & 0.25          & 6          \\
Granger   &                   & \begin{minipage}{.17\linewidth} \centering \includegraphics[width=\linewidth]{imgs/sim_new/pred/5V/5V1_granger_gN.png} \end{minipage}  & 0.25          & 0.25         & 0.25          & 6          \\
PCMCI     &                   & \begin{minipage}{.17\linewidth} \centering \includegraphics[width=\linewidth]{imgs/sim_new/pred/5V/5V1_pcmci_gN.png} \end{minipage} & 0.43          & 0.75         & 0.55          & 5          \\
DYNOTEARS &                   & \begin{minipage}{.17\linewidth} \centering \includegraphics[width=\linewidth]{imgs/sim_new/pred/5V/5V1_dynotears_gN.png} \end{minipage} & 0.13          & 0.50         & 0.21          & 15         \\
SLARAC    &                   &  \begin{minipage}{.17\linewidth} \centering \includegraphics[width=\linewidth]{imgs/sim_new/pred/5V/5V1_slarac_gN.png} \end{minipage} & 0.17          & 0.75         & 0.27          & 16         \\ \cline{1-1} \cline{3-7} 
MXMap     &                   & \begin{minipage}{.17\linewidth} \centering \includegraphics[width=\linewidth]{imgs/sim_new/pred/5V/5V1_mxmap_gN.png} \end{minipage}  & \textbf{0.80} & \textbf{1.0} & \textbf{0.89} & \textbf{1}
\end{tabular}
\caption{5V Structure 1 Without Cycle (Gaussian Additive Noise, Level 0.01)}
\label{tab:5V1_gN}
\end{table}

\begin{table}[htb]
\begin{tabular}{l|c|c|c|c|c|c}
Method    & Ground Truth      & Predicted & Precision    & Recall       & F1           & SHD        \\ \hline
tsFCI     & \multirow{7}{*}[-8.6em]{\begin{minipage}{.17\linewidth} \centering \includegraphics[width=\linewidth]{imgs/sim_new/gt/5V-2_gt.png} \end{minipage}} & \begin{minipage}{.17\linewidth} \centering \includegraphics[width=\linewidth]{imgs/sim_new/pred/5V/5V2_tsfci_noN.png} \end{minipage} & 0.50         & 0.83         & 0.64         & 6          \\
VARLiNGAM &                   & \begin{minipage}{.17\linewidth} \centering \includegraphics[width=\linewidth]{imgs/sim_new/pred/5V/5V2_varlingam_noN.png} \end{minipage} & 0            & 0            & 0            & 12         \\
Granger   &                   &  \begin{minipage}{.17\linewidth} \centering \includegraphics[width=\linewidth]{imgs/sim_new/pred/5V/5V2_granger_noN.png} \end{minipage}  & 0            & 0            & 0            & 12         \\
PCMCI     &                   &  \begin{minipage}{.17\linewidth} \centering \includegraphics[width=\linewidth]{imgs/sim_new/pred/5V/5V2_pcmci_noN.png} \end{minipage} & 0.55         & \textbf{1.0} & 0.71         & 5          \\
DYNOTEARS &                   &  \begin{minipage}{.17\linewidth} \centering \includegraphics[width=\linewidth]{imgs/sim_new/pred/5V/5V2_dynotears_noN.png} \end{minipage}  & 0.22         & 0.33         & 0.27         & 11         \\
SLARAC    &                   &  \begin{minipage}{.17\linewidth} \centering \includegraphics[width=\linewidth]{imgs/sim_new/pred/5V/5V2_slarac_noN.png} \end{minipage} & 0.07         & 0.17         & 0.10         & 18         \\ \cline{1-1} \cline{3-7} 
MXMap     &                   &  \begin{minipage}{.17\linewidth} \centering \includegraphics[width=\linewidth]{imgs/sim_new/pred/5V/5V2_mxmap_noN.png} \end{minipage}  & \textbf{1.0} & \textbf{1.0} & \textbf{1.0} & \textbf{0}
\end{tabular}
\caption{5V Structure 2 With Cycle (No Noise)}
\label{tab:5V2_noN}
\end{table}

\begin{table}[htb]
\begin{tabular}{l|c|c|c|c|c|c}
Method    & Ground Truth      & Predicted & Precision     & Recall       & F1            & SHD        \\ \hline
tsFCI     & \multirow{7}{*}[-9.6em]{\begin{minipage}{.17\linewidth} \centering \includegraphics[width=\linewidth]{imgs/sim_new/gt/5V-2_gt.png} \end{minipage}} & \begin{minipage}{.17\linewidth} \centering \includegraphics[width=\linewidth]{imgs/sim_new/pred/5V/5V2_tsfci_gN.png} \end{minipage} & 0.40          & 0.67         & 0.50          & 8          \\
VARLiNGAM &                   &  \begin{minipage}{.17\linewidth} \centering \includegraphics[width=\linewidth]{imgs/sim_new/pred/5V/5V2_varlingam_gN.png} \end{minipage} & 0             & 0            & 0             & 12         \\
Granger   &                   &  \begin{minipage}{.17\linewidth} \centering \includegraphics[width=\linewidth]{imgs/sim_new/pred/5V/5V2_granger_gN.png} \end{minipage}   & 0             & 0            & 0             & 12         \\
PCMCI     &                   & \begin{minipage}{.17\linewidth} \centering \includegraphics[width=\linewidth]{imgs/sim_new/pred/5V/5V2_pcmci_gN.png} \end{minipage} & \textbf{0.75} & \textbf{1.0} & \textbf{0.86} & \textbf{2} \\
DYNOTEARS &                   &  \begin{minipage}{.17\linewidth} \centering \includegraphics[width=\linewidth]{imgs/sim_new/pred/5V/5V2_dynotears_gN.png} \end{minipage} & 0.20          & 0.33         & 0.25          & 12         \\
SLARAC    &                   &  \begin{minipage}{.17\linewidth} \centering \includegraphics[width=\linewidth]{imgs/sim_new/pred/5V/5V2_slarac_gN.png} \end{minipage}  & 0.07          & 0.17         & 0.10          & 18         \\ \cline{1-1} \cline{3-7} 
MXMap     &                   & \begin{minipage}{.17\linewidth} \centering \includegraphics[width=\linewidth]{imgs/sim_new/pred/5V/5V2_mxmap_gN.png} \end{minipage} & \textbf{0.75} & \textbf{1.0} & \textbf{0.86} & \textbf{2}
\end{tabular}
\caption{5V Structure 2 With Cycle (Gaussian Additive Noise, Level 0.01)}
\label{tab:5V2_gN}
\end{table}

\begin{table}[htb]
\begin{tabular}{l|c|c|c|c|c|c}
Method    & Ground Truth      & Predicted & Precision     & Recall       & F1            & SHD        \\ \hline
tsFCI     & \multirow{7}{*}[-16em]{\begin{minipage}{.17\linewidth} \centering \includegraphics[width=\linewidth]{imgs/sim_new/gt/6V_gt.png} \end{minipage}} &  \begin{minipage}{.17\linewidth} \centering \includegraphics[width=\linewidth]{imgs/sim_new/pred/6V/6V_tsfci_noN.png} \end{minipage}  & 0.17          & 0.17         & 0.17          & 10         \\
VARLiNGAM &                   &  \begin{minipage}{.17\linewidth} \centering \includegraphics[width=\linewidth]{imgs/sim_new/pred/6V/6V_varlingam_noN.png} \end{minipage} & 0.29          & 0.33         & 0.31          & 9          \\
Granger   &                   &  \begin{minipage}{.17\linewidth} \centering \includegraphics[width=\linewidth]{imgs/sim_new/pred/6V/6V_granger_noN.png} \end{minipage} & 0             & 0            & 0             & 11         \\
PCMCI     &                   &  \begin{minipage}{.17\linewidth} \centering \includegraphics[width=\linewidth]{imgs/sim_new/pred/6V/6V_pcmci_noN.png} \end{minipage} & 0.35          & \textbf{1.0} & 0.52          & 11         \\
DYNOTEARS &                   &  \begin{minipage}{.17\linewidth} \centering \includegraphics[width=\linewidth]{imgs/sim_new/pred/6V/6V_dynotears_noN.png} \end{minipage}  & 0.16          & 0.50         & 0.24          & 19         \\
SLARAC    &                   & \begin{minipage}{.17\linewidth} \centering \includegraphics[width=\linewidth]{imgs/sim_new/pred/6V/6V_slarac_noN.png} \end{minipage}   & 0             & 0            & 0             & 25         \\ \cline{1-1} \cline{3-7} 
MXMap     &                   &  \begin{minipage}{.17\linewidth} \centering \includegraphics[width=\linewidth]{imgs/sim_new/pred/6V/6V_mxmap_noN.png} \end{minipage} & \textbf{0.75} & \textbf{1.0} & \textbf{0.86} & \textbf{2}
\end{tabular}
\caption{6V Structure Without Cycle (No Noise)}
\label{tab:6V_noN}
\end{table}

\begin{table}[htb]
\begin{tabular}{l|c|c|c|c|c|c}
Method    & Ground Truth      & Predicted & Precision     & Recall       & F1            & SHD        \\ \hline
tsFCI     & \multirow{7}{*}[-16em]{\begin{minipage}{.17\linewidth} \centering \includegraphics[width=\linewidth]{imgs/sim_new/gt/6V_gt.png} \end{minipage}} &  \begin{minipage}{.17\linewidth} \centering \includegraphics[width=\linewidth]{imgs/sim_new/pred/6V/6V_tsfci_gN.png} \end{minipage}   & 0.27          & 0.50         & 0.35          & 11         \\
VARLiNGAM &                   &  \begin{minipage}{.17\linewidth} \centering \includegraphics[width=\linewidth]{imgs/sim_new/pred/6V/6V_varlingam_gN.png} \end{minipage}    & 0.22          & 0.33         & 0.27          & 11         \\
Granger   &                   & \begin{minipage}{.17\linewidth} \centering \includegraphics[width=\linewidth]{imgs/sim_new/pred/6V/6V_granger_gN.png} \end{minipage}   & 0             & 0            & 0             & 12         \\
PCMCI     &                   & \begin{minipage}{.17\linewidth} \centering \includegraphics[width=\linewidth]{imgs/sim_new/pred/6V/6V_pcmci_gN.png} \end{minipage}   & \textbf{0.67} & \textbf{1.0} & \textbf{0.80} & \textbf{3} \\
DYNOTEARS &                   & \begin{minipage}{.17\linewidth} \centering \includegraphics[width=\linewidth]{imgs/sim_new/pred/6V/6V_dynotears_gN.png} \end{minipage}   & 0.17          & 0.50         & 0.25          & 18         \\
SLARAC    &                   & \begin{minipage}{.17\linewidth} \centering \includegraphics[width=\linewidth]{imgs/sim_new/pred/6V/6V_slarac_gN.png} \end{minipage}     & 0             & 0            & 0             & 27         \\ \cline{1-1} \cline{3-7} 
MXMap     &                   &  \begin{minipage}{.17\linewidth} \centering \includegraphics[width=\linewidth]{imgs/sim_new/pred/6V/6V_mxmap_gN.png} \end{minipage}   & 0.60          & \textbf{1.0} & 0.75          & 4         
\end{tabular}
\caption{6V Structure Without Cycle (Gaussian Additive Noise, Level 0.01)}
\label{tab:6V_gN}
\end{table}

\begin{table}[htb]
\begin{tabular}{l|c|c|c|c|c|c}
Method    & Ground Truth      & Predicted & Precision     & Recall       & F1            & SHD        \\ \hline
tsFCI     & \multirow{7}{*}[-9em]{\begin{minipage}{.17\linewidth} \centering \includegraphics[width=\linewidth]{imgs/sim_new/gt/7V_gt.png} \end{minipage}} & \begin{minipage}{.17\linewidth} \centering \includegraphics[width=\linewidth]{imgs/sim_new/pred/7V/7V_tsfci_noN.png} \end{minipage}   & 0.43          & 0.63         & 0.50          & 10         \\
VARLiNGAM &                   &  \begin{minipage}{.17\linewidth} \centering \includegraphics[width=\linewidth]{imgs/sim_new/pred/7V/7V_varlingam_noN.png} \end{minipage}   & 0.17          & 0.25         & 0.20          & 16         \\
Granger   &                   &  \begin{minipage}{.17\linewidth} \centering \includegraphics[width=\linewidth]{imgs/sim_new/pred/7V/7V_granger_noN.png} \end{minipage}   & 0.18          & 0.25         & 0.21          & 15         \\
PCMCI     &                   &  \begin{minipage}{.17\linewidth} \centering \includegraphics[width=\linewidth]{imgs/sim_new/pred/7V/7V_pcmci_noN.png} \end{minipage}   & 0.42          & \textbf{1.0} & 0.59          & 11         \\
DYNOTEARS &                   & \begin{minipage}{.17\linewidth} \centering \includegraphics[width=\linewidth]{imgs/sim_new/pred/7V/7V_dynotears_noN.png} \end{minipage}   & 0.30          & 0.75         & 0.43          & 16         \\
SLARAC    &                   &  \begin{minipage}{.17\linewidth} \centering \includegraphics[width=\linewidth]{imgs/sim_new/pred/7V/7V_slarac_noN.png} \end{minipage}   & 0.06          & 0.13         & 0.09          & 21         \\ \cline{1-1} \cline{3-7} 
MXMap     &                   &  \begin{minipage}{.17\linewidth} \centering \includegraphics[width=\linewidth]{imgs/sim_new/pred/7V/7V_mxmap_noN.png} \end{minipage}   & \textbf{0.67} & \textbf{1.0} & \textbf{0.80} & \textbf{4}
\end{tabular}
\caption{7V Structure With Cycle (No Noise)}
\label{tab:7V_noN}
\end{table}

\begin{table}[htb]
\begin{tabular}{l|c|c|c|c|c|c}
Method    & Ground Truth      & Predicted & Precision     & Recall       & F1            & SHD        \\ \hline
tsFCI     & \multirow{7}{*}[-9em]{\begin{minipage}{.17\linewidth} \centering \includegraphics[width=\linewidth]{imgs/sim_new/gt/7V_gt.png} \end{minipage}} &  \begin{minipage}{.17\linewidth} \centering \includegraphics[width=\linewidth]{imgs/sim_new/pred/7V/7V_tsfci_gN.png} \end{minipage}  & 0.43          & 0.75         & 0.55          & 10         \\
VARLiNGAM &                   &  \begin{minipage}{.17\linewidth} \centering \includegraphics[width=\linewidth]{imgs/sim_new/pred/7V/7V_varlingam_gN.png} \end{minipage}   & 0.18          & 0.25         & 0.21          & 15         \\
Granger   &                   & \begin{minipage}{.17\linewidth} \centering \includegraphics[width=\linewidth]{imgs/sim_new/pred/7V/7V_granger_gN.png} \end{minipage}  & 0.18          & 0.25         & 0.21          & 15         \\
PCMCI     &                   &  \begin{minipage}{.17\linewidth} \centering \includegraphics[width=\linewidth]{imgs/sim_new/pred/7V/7V_pcmci_gN.png} \end{minipage}   & 0.44          & \textbf{1.0} & 0.62          & 10         \\
DYNOTEARS &                   & \begin{minipage}{.17\linewidth} \centering \includegraphics[width=\linewidth]{imgs/sim_new/pred/7V/7V_dynotears_gN.png} \end{minipage}  & 0.18          & 0.50         & 0.27          & 22         \\
SLARAC    &                   &  \begin{minipage}{.17\linewidth} \centering \includegraphics[width=\linewidth]{imgs/sim_new/pred/7V/7V_slarac_gN.png} \end{minipage}  & 0.10          & 0.25         & 0.14          & 25         \\ \cline{1-1} \cline{3-7} 
MXMap     &                   &   \begin{minipage}{.17\linewidth} \centering \includegraphics[width=\linewidth]{imgs/sim_new/pred/7V/7V_mxmap_gN.png} \end{minipage}    & \textbf{0.58} & 0.88         & \textbf{0.70} & \textbf{6}
\end{tabular}
\caption{7V Structure With Cycle (Gaussian Additive Noise, Level 0.01)}
\label{tab:7V_gN}
\end{table}