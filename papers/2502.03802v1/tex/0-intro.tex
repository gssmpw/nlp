\section{Introduction}
% \textcolor{red}{Di: 1) Again, you should not put multiple places in bold, 2) you may need to add five more recently published related papers in the introduction}

Nonlinear dynamical systems are omnipresent across scientific disciplines, and understanding causal relationships in these systems is crucial for unveiling the underlying mechanisms that drive system behaviors. Classic causal inference methods, such as Granger Causality (GC)~\citep{granger1969investigating} and other functional causal models (FCMs), including the Additive Noise Model (ANM)~\citep{hoyer2008nonlinear, liu2024causal} and the Post Nonlinear Model (PNL)~\citep{zhang2015estimation, keropyan2023rank}, struggle with these systems due to their assumption of a predictive relationship from cause to effect, which does not hold in the presence of complex dynamics like coupling and chaos.

% Convergent Cross Mapping (CCM)~\citep{sugihara2012detecting} was proposed as a model-free approach for bivariate causal inference in nonlinear dynamical systems. CCM addresses these limitations by leveraging state-space manifold reconstructions and cross mapping between reconstructed embeddings. Since its introduction, CCM has inspired further developments, including Partial Cross Mapping (PCM)~\citep{leng2020partial}, which aims to distinguish indirect from direct causalities in three-variable systems. However, PCM is limited to mapping operations between univariate delay embeddings, which can be less efficient when dealing with higher-dimensional systems with multiple interconnected variables. \textcolor{red}{Di: are there any related works try to tackle this problem?}

Convergent Cross Mapping (CCM)~\citep{sugihara2012detecting, barraquand2021inferring} was proposed as a model-free approach for bivariate causal inference in coupled dynamical systems. CCM addresses these limitations by leveraging state-space manifold reconstructions and cross mapping between reconstructed embeddings. Since its introduction, CCM has inspired further developments, including Partial Cross Mapping (PCM)~\citep{leng2020partial}, which aims to distinguish indirect from direct causalities in three-variable systems. However, PCM is limited to mapping operations between univariate delay embeddings, which can be less effective or even fail when dealing with complex systems with multiple interconnected variables~\citep{chen2022causation}.

To overcome this limitation, we propose \textbf{multiPCM}, an extension of PCM to the multivariate setting that allows for more effective causal inference by utilizing cross mapping via multivariate embeddings. We further integrate multiPCM with bivariate CCM into a two-phase framework named \textbf{MXMap} (Multivariate Cross Mapping for Causal Discovery). The proposed framework is designed for multivariate causal discovery, and is not only confined to assumptions of directed acyclic graphs (DAGs) but can also handle cycles. In the first phase, bivariate CCM generates an initial, potentially dense causal graph; In the second phase, multiPCM prunes indirect connections, refining the graph to isolate direct causal relationships. We systematically evaluate multiPCM and MXMap on benchmark datasets, including both simulated ecosystems and real-world meteorological data.

The contributions of this work are summarized as follows:

\begin{itemize}
    \item \textbf{Extension of PCM to multivariate settings}: We introduce multiPCM, which extends PCM to utilize multivariate delay embeddings for more robust causal inference in high-dimensional dynamical systems.
    \item \textbf{Two-phase causal discovery framework}: We propose MXMap, combining bivariate CCM with multiPCM to generate and refine causal graphs in nonlinear dynamical systems, which can also detect cycles.
    \item \textbf{Comprehensive evaluation on nonlinear dynamical systems}: We validate multiPCM and MXMap on simulated and real-world datasets. MXMap is compared against multiple baseline methods — including tsFCI, VAR-LiNGAM, PCMCI, Granger Causality, DYNOTEARS, SLARAC — demonstrating advantages in accuracy and refinement capabilities.
\end{itemize}