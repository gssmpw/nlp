% \documentclass[anon,12pt]{clear2025} % Anonymized submission
\documentclass[final,12pt]{clear2025} % Include author names

% The following packages will be automatically loaded:
% amsmath, amssymb, natbib, graphicx, url, algorithm2e
% The following packages will be automatically loaded:
% amsmath, amssymb, natbib, graphicx, url, algorithm2e

\usepackage[utf8]{inputenc} % allow utf-8 input
\usepackage[T1]{fontenc}    % use 8-bit T1 fonts
\usepackage{hyperref}       % hyperlinks
\usepackage{url}            % simple URL typesetting
\usepackage{booktabs}       % professional-quality tables
\usepackage{amsfonts}       % blackboard math symbols
\usepackage{nicefrac}       % compact symbols for 1/2, etc.
\usepackage{microtype}      % microtypography
\usepackage{lipsum}
\usepackage{fancyhdr}       % header
\usepackage{graphicx}       % graphics

\usepackage{multirow}
\usepackage{subcaption}
\usepackage{caption}

\usepackage{array}
\graphicspath{{media/}}     % organize your images and other figures under media/ folder
\usepackage{mathtools}

\usepackage{booktabs}
\usepackage{bbding}
\usepackage{pifont}
\newcommand{\cmark}{\textcolor{teal}{\ding{51}}}%
\newcommand{\xmark}{\textcolor{purple}{\ding{55}}}%

\usepackage{tikz}
\def\halfcheckmark{\tikz\draw[scale=0.3,gray,fill=gray](0,.35) -- (.25,0) -- (1,.7) -- (.25,.15) -- cycle (0.75,0.2) -- (0.77,0.2)  -- (0.6,0.7) -- cycle;}


\usepackage{esvect}

% \usepackage{amsmath}
\newcommand*{\Comb}[2]{{}^{#1}C_{#2}}

\usepackage{algorithm}
% \usepackage{algpseudocode}
% \usepackage{algorithm2e}

\usepackage{enumitem}
\renewcommand{\labelenumi}{\arabic{enumi}.}

%Header
\pagestyle{fancy}
\thispagestyle{empty}
\rhead{ \textit{ }} 

\title[MXMap]{MXMap: A Multivariate Cross Mapping Framework for Causal Discovery in Dynamical Systems}
\usepackage{times}
% Use \Name{Author Name} to specify the name.
% If the surname contains spaces, enclose the surname
% in braces, e.g. \Name{John {Smith Jones}} similarly
% if the name has a "von" part, e.g \Name{Jane {de Winter}}.
% If the first letter in the forenames is a diacritic
% enclose the diacritic in braces, e.g. \Name{{\'E}louise Smith}

% Two authors with the same address
% \clearauthor{\Name{Author Name1} \Email{abc@sample.com}\and
%  \Name{Author Name2} \Email{xyz@sample.com}\\
%  \addr Address}

% Three or more authors with the same address:
% \clearauthor{\Name{Author Name1} \Email{an1@sample.com}\\
%  \Name{Author Name2} \Email{an2@sample.com}\\
%  \Name{Author Name3} \Email{an3@sample.com}\\
%  \addr Address}

% Authors with different addresses:
\clearauthor{%
 \Name{Elise Zhang} \Email{elise.zhang@mail.mcgill.ca}\\
 \addr  McGill University, Montréal, QC Canada
 \AND
 \Name{François Mirallès} \Email{miralles.francois@hydroquebec.com}\\
 \addr Hydro-Québec Research Institute (IREQ), Varennes, QC Canada
 \AND
 \Name{Raphaël Rousseau-Rizzi} \Email{rousseau-rizzi.raphael@hydroquebec.com}\\
 \addr Hydro-Québec Research Institute (IREQ), Varennes, QC Canada
 \AND
 \Name{Di Wu} \Email{di.wu5@mcgill.ca}\\
 \addr McGill University, Montréal, QC Canada
 \AND
 \Name{Arnaud Zinflou} \Email{zinflou.arnaud@hydroquebec.com}\\
 \addr Hydro-Québec Research Institute (IREQ), Varennes, QC Canada
 \AND
 \Name{Benoit Boulet} \Email{benoit.boulet@mcgill.ca}\\
 \addr McGill University, Montréal, QC Canada
 %
}

\begin{document}

\maketitle

\begin{abstract}%

Convergent Cross Mapping (CCM) is a powerful method for detecting causality in coupled nonlinear dynamical systems, providing a model-free approach to capture dynamic causal interactions. Partial Cross Mapping (PCM) was introduced as an extension of CCM to address indirect causality in three-variable systems by comparing cross-mapping quality between direct cause-effect mapping and indirect mapping through an intermediate conditioning variable. However, PCM remains limited to univariate delay embeddings in its cross-mapping processes. In this work, we extend PCM to the multivariate setting, introducing multiPCM, which leverages multivariate embeddings to more effectively distinguish indirect causal relationships. We further propose a multivariate cross-mapping framework (MXMap) for causal discovery in dynamical systems. This two-phase framework combines (1) pairwise CCM tests to establish an initial causal graph and (2) multiPCM to refine the graph by pruning indirect causal connections. Through experiments
\footnote{Implementation at \url{https://github.com/elisejiuqizhang/multiPCM}.}
on simulated data and the \textit{ERA5 Reanalysis} weather dataset, we demonstrate the effectiveness of MXMap. Additionally, MXMap is compared against several baseline methods, showing advantages in accuracy and causal graph refinement.%

  % Convergent Cross Mapping (CCM) is a powerful method for detecting causality in coupled nonlinear dynamical systems, providing a model-free approach to capture dynamic causal interactions. Partial Cross Mapping (PCM) was introduced as an extension of CCM to address indirect causality in three-variable systems by comparing cross-mapping quality between direct cause-effect mapping and indirect mapping through an intermediate conditioning variable. However, PCM remains limited to univariate delay embeddings in its cross-mapping processes. In this work, we extend PCM to the multivariate setting, introducing \textbf{multiPCM}, which leverages multivariate embeddings to more effectively distinguish indirect causal relationships. We further propose a multivariate cross-mapping framework (\textbf{MXMap}) for causal discovery in dynamical systems. This two-phase framework combines (1) pairwise CCM tests to establish an initial causal graph and (2) \textbf{multiPCM} to refine the graph by pruning indirect causal connections. Through experiments on simulated data and the \textit{ERA5 Reanalysis} weather dataset, we demonstrate the effectiveness of \textbf{MXMap}. Additionally, \textbf{MXMap} is compared against several baseline methods, showing advantages in accuracy and causal graph refinement. \textcolor{red}{Di: I think it may not be good to put multiple places in bold  in the abstract}% 
\end{abstract}

\begin{keywords}%
  Causal inference, state-space reconstruction, convergent cross-mapping, partial cross mapping, nonlinear dynamical system %
\end{keywords}

\section{Introduction}
\label{sec:intro}
Treatment non-adherence is a pervasive and persistent challenge in healthcare. Researchers estimate that poor medication adherence leads to 125,000 preventable deaths annually in the U.S. and contributes to \$100-\$300 billion in avoidable healthcare costs \citep{benjamin2012medication}. This issue is particularly prevalent among patients with chronic conditions such as hypertension, with 40-50\% failing to take their medications as prescribed \citep{kleinsinger2018unmet, algabbani2020treatment}. While researchers have extensively documented this problem through surveys and interviews \citep{boratas2018evaluation, fernandez2019adherence, algabbani2020treatment, najjuma2020adherence, schober2021high}, the studies---and ultimately understanding of treatment non-adherence---remain limited by small sample sizes and self-reporting bias \citep{adams1999evidence, stirratt2015self}. Physical solutions to monitor and encourage adherence such as electronic pill caps have shown promise in controlled settings but remain impractical for large-scale deployment due to high costs and implementation challenges~\citep{parker2007adherence,mauro2019effect}.
 
These measurement challenges take on new urgency as healthcare systems increasingly rely on machine learning (ML) models trained on electronic health records (EHRs) to guide treatment decisions~\citep{komorowski2018artificial, brugnara2020multimodal, zheng2021personalized, mroz2024predicting, yi2024development,shen2024data,chen2022clustering}. These machine learning models learn from historical patient data, which assume that prescribed treatments were actually taken. However, this introduces an implicit bias---models trained on non-adherent patients learn patterns that misrepresent true treatment effects. This implicit bias may degrade model performance and disproportionately impact underserved populations, who often face greater barriers to treatment adherence~\citep{bosworth2006racial, schober2021high}.

Recent advances in large language models (LLMs) have shown that LLMs can advance medical understanding by accurately extracting information from EHRs \citep{agrawal2022large, goel2023llmsaccelerateannotationmedical}. Instead of relying on self-reported treatment adherence from questionnaires and interviews, LLMs could serve as a powerful tool for identifying treatment non-adherence directly from EHRs. By analyzing rich but unstructured clinical notes, LLMs can detect documented instances of missed medications, unfilled prescriptions, and patient-reported barriers to adherence, enabling systematic assessment of treatment non-adherence across large patient populations.

In this study, we examine hypertension treatment non-adherence using EHR data from a large academic hospital by leveraging an LLM to analyze clinical notes, and further investigate its impact on causal inference and ML model performance (Figure~\ref{fig:diagram}). With a cohort of 3,623 patients, we identify 786 (21.7\%) cases of non-adherence and extract demographic and clinical factors that are statistically significant. Additionally, we apply topic modeling to clinical notes revealing underlying reasons for non-adherence.

To assess the effect of treatment non-adherence bias on downstream model performance, we perform causal inference and build predictive models using EHR with treatment records.  Our results show that ignoring treatment non-adherence bias could lead to reversed conclusions in treatment effect estimation, significantly degrade the performance of predictive models up to 5\%, and lead to unfair predictions. Furthermore, we highlight the importance of addressing treatment non-adherence bias by showing simply removing patient records with non-adherence, though reducing the size of the training dataset, could improve model performance and lead to fairer predictions.
\\
\newline
The contributions of this work include:
\begin{enumerate}
\item Conducting a large-scale study on treatment non-adherence in hypertension and identifying statistically significant factors associated with non-adherence.

\item Comparing LLM identification against physician annotations, LLMs perform well with 92\% accuracy, precision and recall.


\item Identifying patient-reported reasons for treatment non-adherence including side effects, forgetfulness, difficulties obtaining refills, etc.

\item Demonstrating the harmful impact of ignoring treatment non-adherence bias on causal inference and predictive modeling, leading to poorer performance and exacerbating racial disparities.

\end{enumerate}

\section{Preliminaries}
\label{sec:prelim}

\subsection{State-Space Reconstruction (SSR)}
\label{sec:ssr}

In physical continuous-time dynamical systems, the interplay between driving forces and dissipation leads systems to settle into characteristic behaviors, represented by attractor manifolds in state space~\citep{milnor1985concept}. Understanding these attractors is crucial for interpreting the system's dynamics and predicting future behavior. However, real-world measurements are often limited, making it infeasible to observe all variables required to fully characterize the state-space attractor.

% State-Space Reconstruction (SSR) \textcolor{red}{Di: a reference is needed here} addresses this challenge: \textit{Given an $n$-dimensional observed time series from an $N$-dimensional dynamical system ($n < N$), can we recover the attractor manifold, and thereby the higher-dimensional dynamics, from the lower-dimensional observations?} A common approach is to use sequences of lagged observations to reconstruct a delay embedding (DE) that approximates the system's attractor. Whitney's Embedding Theorem~\citep{whitney1936differentiable} and Takens' Embedding Theorem~\citep{takens2006detecting} establish that this reconstruction is diffeomorphic (i.e., a continuously differentiable and invertible mapping) to the true attractor under certain conditions~\citep{sauer1991j}. When these conditions are satisfied, such delay embeddings are termed "shadow manifolds" and serve as low-dimensional approximations of the system~\citep{sugihara2012detecting}.

State-Space Reconstruction (SSR)~\citep{vlachos2008state} addresses this challenge: \textit{Given an $n$-dimensional observed time series from an $N$-dimensional dynamical system ($n < N$), can we recover the attractor manifold, and thereby the higher-dimensional dynamics, from the lower-dimensional observations?} A common approach is to use sequences of lagged observations to reconstruct a delay embedding (DE) that approximates the system's attractor. Whitney's Embedding Theorem~\citep{whitney1936differentiable} and Takens' Embedding Theorem~\citep{takens2006detecting} establish that this reconstruction is diffeomorphic (i.e., a continuously differentiable and invertible mapping) to the true attractor under certain conditions~\citep{sauer1991j}. When these conditions are satisfied, such delay embeddings are termed "shadow manifolds" and serve as low-dimensional approximations of the system~\citep{sugihara2012detecting}.

% Following~\citep{vlachos2010nonuniform, butler2023causal}, we introduce the univariate delay-coordinate embedding used in Takens' Theorem. Suppose an attractor $\mathcal{A}$ exists for the dynamical system, and a time series $\{x_t\}_{t=0}^{T}$ is observed from one state variable $\textbf{X}$. Given delay $\tau$ and embedding dimension $E$, where $\tau$ and $E$ are positive integers, the vector signal $\textbf{m}_{x}(t)$ of lagged values is defined as: 
% % \textcolor{red}{Di: You may need to pay attention for the definition for a variable and a vector}

Following~\citep{vlachos2010nonuniform, butler2023causal}, we introduce the univariate delay-coordinate embedding used in Takens' Theorem. Suppose an attractor $\mathcal{A}$ exists for the dynamical system, and a time series $\{x_t\}_{t=0}^{T}$ is observed from one state variable $\textbf{x}$ sampled at a constant rate. Given delay $\tau$ and embedding dimension $E$, where $\tau$ and $E$ are positive integers, the vector signal $\vv{\textbf{m}}_{x}(t)$ of lagged values is defined as:

\begin{equation}
    % \textbf{m}_{x}(t)= \left[x_{t},  x_{t-\tau},  x_{t-2\tau},  x_{t-3\tau},  \ldots,  x_{t-(E-2)\tau},  x_{t-(E-1)\tau}\right] 
    \vv{\textbf{m}}_{x}(t) \vcentcolon= \left[x_{t},  x_{t-\tau},  x_{t-2\tau},  x_{t-3\tau},  \ldots,  x_{t-(E-2)\tau},  x_{t-(E-1)\tau}\right]
\end{equation}

As time progresses, these vectors form an $E$-dimensional delay embedding $\textbf{M}_x$. The lag $\tau$ determines the observation time scale for reconstruction, while the embedding dimension $E$ defines the complexity of the embedding. Takens' theorem suggests that $E$ should be greater than twice the fractal dimension of the attractor $\mathcal{A}$, i.e., $E > 2 \cdot \text{dim}(\mathcal{A})$~\citep{sauer1991j, kugiumtzis1996state}. In practice, $\tau$ and $E$ are determined empirically. Time-delayed autocorrelation~\citep{kugiumtzis1996state} and delay mutual information~\citep{fraser1986independent, klikova2011reconstruction} are commonly used to select an optimal $\tau$, while the \textit{false nearest neighbors} (FNN) method~\citep{kennel1992determining} is typically used to determine $E$, by tracking changes in nearest neighbors as embedding dimensions increase.

\subsection{Convergent Cross Mapping (CCM)}
\label{sec:uni-ccm}
Causality in a discrete-time dynamical system~\citep{butler2023causal,cummins2015efficacy} can be defined as follows: given two state variables $\textbf{x}$ and $\textbf{y}$, if the future evolution of $\textbf{y}$ depends on $\textbf{x}$, then $\textbf{x}$ is said to cause $\textbf{y}$, denoted as $\textbf{x} \Rightarrow \textbf{y}$. This causal influence can be represented in a state-space equation, as shown in Eq.~\ref{eq:def_cause}:

\begin{equation}
\label{eq:def_cause}
    \textbf{y}_{t+1}=\mathcal{F}_y \left(\textbf{y}_{t}, \textbf{x}_{t} \right)
\end{equation}

The relationship between $\textbf{x}$ and $\textbf{y}$ can be unidirectional ($\textbf{x} \Rightarrow \textbf{y}$ or $\textbf{y} \Rightarrow \textbf{x}$), bidirectional ($\textbf{x} \Leftrightarrow \textbf{y}$), or there may be no causal link at all.

% Convergent Cross Mapping (CCM) \textcolor{red}{Di: In the main body of a paper, CCM should only be defined onece}~\citep{sugihara2012detecting} leverages the diffeomorphism between reconstructed shadow manifolds, as stated in Takens' Theorem. Cross mapping measures how well local neighborhoods in one reconstructed manifold map to the corresponding neighborhoods in another. For delay embeddings $\textbf{M}_x$ and $\textbf{M}_y$ reconstructed from $\textbf{x}$ and $\textbf{y}$, if $\textbf{M}_x$ and $\textbf{M}_y$ are both valid shadow manifolds of the attractor $\mathcal{A}$, they are diffeomorphic to each other via their relationship to $\mathcal{A}$.

CCM~\citep{sugihara2012detecting} leverages the diffeomorphism between reconstructed shadow manifolds, as stated in Takens' Theorem. Cross mapping measures how well local neighborhoods in one reconstructed manifold map to the corresponding neighborhoods in another. For delay embeddings $\textbf{M}_x$ and $\textbf{M}_y$ reconstructed from $\textbf{x}$ and $\textbf{y}$, if $\textbf{M}_x$ and $\textbf{M}_y$ are both valid shadow manifolds of the attractor $\mathcal{A}$, they are diffeomorphic to each other via their relationship to $\mathcal{A}$.

If $\textbf{x}$ causes $\textbf{y}$ ($\textbf{x} \Rightarrow \textbf{y}$), observations of $\textbf{y}$ should contain information about $\textbf{x}$. This allows the reconstruction of the dynamics of $\textbf{x}$ from $\textbf{y}$, but not necessarily vice versa. In this case, the quality of mapping from $\textbf{M}_y$ to $\textbf{M}_x$ should be better compared to the reverse direction, indicating a causal link from $\textbf{x}$ to $\textbf{y}$.

CCM uses a $k$-nearest neighbor ($k$NN) regression approach (also known as \textit{simplex projection}) to evaluate the quality of cross mapping. Given time series $\{x_t\}_{t=0}^{T}$ and $\{y_t\}_{t=0}^{T}$, to verify whether $\textbf{x} \Rightarrow \textbf{y}$, the procedure is as follows:

\begin{enumerate}[label={[\arabic*]}]
    \item Construct delay embeddings $\textbf{M}_x, \textbf{M}_y$ with appropriate delay $\tau$ and embedding dimension $E$.
    \item For each point in $\textbf{M}_y$, identify the $k$-nearest neighbors $\mathcal{N}_y$.
    \item Use the timestamps of $\mathcal{N}_y$ to find corresponding points $\hat{\mathcal{N}}_x$ on $\textbf{M}_x$ and compute a weighted average to form a projected reconstruction $\hat{\textbf{M}}_x$, hence the reconstructed $\hat{\mathbf{x}}$.
    \item Calculate the correlation score $\rho_{x\Rightarrow y}$ between the true $\mathbf{x}$ and the reconstructed $\hat{\mathbf{x}}$.
    \item Repeat these steps with increasing sequence length; if $\textbf{x} \Rightarrow \textbf{y}$, the correlation score should converge, indicating a valid cross map.
\end{enumerate}

The same procedure is repeated for the reverse causal assumption to yield another correlation score $\rho_{y\Rightarrow x}$. The correlation scores $\rho_{x\Rightarrow y}$ and $\rho_{y\Rightarrow x}$ quantify the cross mapping quality, where a higher score in one direction suggests a stronger causal link. In practice, if the input length $L$ is large enough, we consider that the yielded correlation scores are already in the convergence zone, and can be used as final correlation estimates.

\subsection{Partial Cross Mapping (PCM)}
\label{sec:uni-pcm}
The original CCM does not distinguish between direct and indirect causality. For three variables $\textbf{x}$, $\textbf{y}$, and $\textbf{z}$ in a causal chain ($\textbf{x}\Rightarrow\textbf{y}\Rightarrow\textbf{z}$), CCM may incorrectly identify a direct causal link between $\textbf{x}$ and $\textbf{z}$ due to transitivity through $\textbf{y}$. Partial Cross Mapping (PCM)~\citep{leng2020partial,jiang2023partial} was proposed to distinguish between direct and indirect causal relationships. In a causal chain like $\textbf{x}\Rightarrow\textbf{y}\Rightarrow\textbf{z}$, PCM aims to determine whether the causal link between $\textbf{x}$ and $\textbf{z}$ is direct or mediated by $\textbf{y}$.

A PCM test considers three variables: the \textit{potential cause} $\textbf{x}$, the \textit{condition} $\textbf{y}$, and the \textit{potential effect} $\textbf{z}$. The goal is to assess whether there is a direct link between $\textbf{x}$ and $\textbf{z}$. This is done by performing cross mapping between the shadow manifolds of each variable to obtain a reconstruction of $\textbf{x}$, denoted by $\hat{\textbf{x}}^{\textbf{z}}$ (from $\textbf{z}$ to $\textbf{x}$), and another reconstruction of $\textbf{x}$ via $\textbf{y}$, denoted by $\hat{\textbf{x}}^{\hat{\textbf{y}}^{\textbf{z}}}$ (first from $\textbf{z}$ to $\textbf{y}$, then from $\textbf{y}$ to $\textbf{x}$). The correlation scores are defined as follows:
% \begin{equation}
%     \rho_{All}=\left| \text{Corr}(\textbf{x}, \hat{\textbf{x}}^{\textbf{z}}) \right|
% \end{equation}
% \begin{equation}
%     \rho_{Direct}=\left| \text{ParCorr}(\textbf{x}, \hat{\textbf{x}}^{\textbf{z}} | \hat{\textbf{x}}^{\hat{\textbf{y}}^{\textbf{z}}} ) \right|
% \end{equation}

\begin{align}
    \rho_{All}=\left| \text{Corr}(\textbf{x}, \hat{\textbf{x}}^{\textbf{z}}) \right|   &&    \rho_{Direct}=\left| \text{ParCorr}(\textbf{x}, \hat{\textbf{x}}^{\textbf{z}} | \hat{\textbf{x}}^{\hat{\textbf{y}}^{\textbf{z}}} ) \right|
\end{align}

$\rho_{All}$ represents the correlation between the original $\textbf{x}$ and the reconstruction $\hat{\textbf{x}}^{\textbf{z}}$, capturing apparent information transfer through all paths. $\rho_{Direct}$ represents the partial correlation, conditioning on the intermediate variable $\textbf{y}$ to assess direct information transfer between $\textbf{x}$ and $\textbf{z}$. If no direct causal link exists, the direct information transfer should be significantly reduced after conditioning on $\textbf{y}$.

PCM uses an empirical threshold $H \in [0, 1)$ to determine causality:
\begin{itemize}
    \item If $\rho_{All} \geq \rho_{Direct} \geq H$, a direct causal link from $\textbf{x}$ to $\textbf{z}$ is inferred.
    \item If $\rho_{All} \geq H \gg \rho_{Direct}$, only indirect causality is suggested.
    \item If $H > \rho_{All} \geq \rho_{Direct}$, no causal relationship is inferred.
\end{itemize}

To distinguish direct from indirect links, we propose an adaptation in our work, to use the correlation ratio $\gamma$ as an alternative:
\begin{equation}
    \gamma=\frac{\rho_{Direct}}{\rho_{All}}
\end{equation}
A smaller ratio $\gamma$ implies negligible direct information transfer after conditioning, suggesting indirect causality via $\textbf{y}$. Conversely, a larger $\gamma$ indicates strong direct information transfer, suggesting a direct causal link. An empirical ratio threshold $\gamma^* \in (0, 1)$ is used to decide whether to retain or eliminate the direct link based on how important such causal influence is.

\section{Method}

Our literature search focused on AI-assisted creative research tools in contrast to AI-assisted writing tools, creative research tools assist with the co-creation of concepts and ideas in the research process rather than merely improving stylistic or rhetorical choices in written research. In total, we surveyed 11 systems papers published in top HCI venues (i.e., CHI, CSCW, UIST, and ToCHI) over the last three years (2022-2024)\footnote{We include a relevant CHI 2025 paper made available on arXiv.}; details can be found in Table~\ref{tab:tool_classification}. 

As LLMs became widely used in 2022 with the release of ChatGPT \cite{openai2022chatgpt}, this timeframe was chosen to reflect the period of significant growth in LLM popularity and adoption, allowing us to capture the most relevant and impactful developments in GenAI-driven research tools. Of the systems surveyed, six of the systems integrate LLM-based functionalities, while the other five represent a more traditional AI approach and employ machine learning techniques (e.g., Seq2Seq, BERT, RNN). By examining both GenAI and traditional AI approaches, we aim to understand to what extent GenAI tools represent a fundamental shift in capabilities and design considerations compared to more established AI approaches.

The thematic dimensions presented in Section~\ref{sec:design-space} resulted from an iterative process among the authors. We engaged in extensive internal discussions and consulted with an external expert specializing in knowledge spaces and the role of AI in fostering creativity and sensemaking. This collaborative and iterative approach resulted in the four dimensions presented in the next section.
\section{Experiments}
\label{sec:exp}

% We evaluate the proposed multiPCM and MXMap on simulated and real-world dynamical systems as described in the Section~\ref{sec:data} below. MXMap performance is then compared with several established multivariate causal inference methods, including RESIT~\citep{peters2014causal}, PC~\citep{spirtes2001causation}, Fast Causal Inference (FCI)~\citep{spirtes2013causal}, LiNGAM~\citep{shimizu2006linear}, and PC with Momentary Condition Independence (PCMCI)~\citep{runge2019detecting}. Overview of these methods and how the outputs of each model are interpreted can be found in Appendix~\ref{appsec:baseline_methods}. For RESIT and LiNGAM, we adopt python implementations in the \texttt{LiNGAM} library~\citep{shimizu2014lingam}; PC and FCI are from the \texttt{causal-learn} library~\citep{zheng2024causal}; PCMCI is from the \texttt{tigramite} library~\citep{runge2023causal}. Detailed experimental setup is provided in Appendix~\ref{appsec:exp_setup}.

We evaluate the proposed multiPCM and MXMap on simulated and real-world dynamical systems as described in the Section~\ref{sec:data} below. MXMap performance is then compared with several established multivariate causal inference methods), including RESIT~\citep{peters2014causal}, tsFCI~\citep{entner2010causal}, VAR-LiNGAM~\citep{hyvarinen2010estimation}, PCMCI~\citep{runge2019detecting}, Granger Causality~\citep{granger1969investigating}, DYNOTEARS~\citep{pamfil2020dynotears}, and SLARAC~\citep{weichwald2020causal}. Overview of these methods and how the outputs of each model are interpreted can be found in Appendix~\ref{appsec:baseline_methods}. 

For RESIT and VAR-LiNGAM, we adopt python implementations in the \texttt{LiNGAM} library~\citep{shimizu2014lingam}; tsFCI is adapted based on the implementation of FCI from the \texttt{causal-learn} library~\citep{zheng2024causal}; PCMCI is from the \texttt{tigramite} library~\citep{runge2023causal}; Granger Causality is also from \texttt{causal-learn}; DYNOTEARS is implemented using the \texttt{Causalnex}~\citep{Beaumont_CausalNex_2021} library; SLARAC is from the \texttt{tidybench} repository~\citep{weichwald2020causal}. Experimental setup is provided in Appendix~\ref{appsec:exp_setup}.

\subsection{Data}
\label{sec:data}

\subsubsection{Simulated Data: Species Interaction Systems}
\label{sec:genData}

Following similar data generation scheme as in~\cite{leng2020partial}, we generate the following systems of varying complexity, from 3-variable to 7-variable. These systems are derived and adapted from the Lotka-Volterra competition models~\citep{volterra1931theorie, lotka1925elements, roques2011probing} that characterize species interactions and exhibit chaotic behaviors. Examples of 3-species and 4-species systems are as Eq.~\ref{eq:3var} and Eq.~\ref{eq:4var}.

\begin{equation}
\label{eq:3var}
\begin{aligned}
x_t & =x_{t-1}\left(\alpha_x-\alpha_x x_{t-1}-\beta_{yx} y_{t-1}-\beta_{zx} z_{t-1}\right)\cdot\eta_{x}+\epsilon_{x} \\
y_t & =y_{t-1}\left(\alpha_y-\alpha_y y_{t-1}-\beta_{xy} x_{t-1}-\beta_{zy} z_{t-1}\right)\cdot\eta_{y}+\epsilon_{y} \\
z_t & =z_{t-1}\left(\alpha_z-\alpha_z z_{t-1}-\beta_{xz} x_{t-1}-\beta_{yz} y_{t-1}\right)\cdot\eta_{z}+\epsilon_{z}
\end{aligned}
\end{equation}
For the 3-species system (Eq.~\ref{eq:3var}), the coefficients $\alpha$ for autonomous dynamics are set respectively as $\alpha_x=3.70, \alpha_y=3.78, \alpha_z=3.72$.

\begin{equation}
\label{eq:4var}
\begin{aligned}
w_t & =w_{t-1}\left(\alpha_w-\alpha_w w_{t-1}-\beta_{xw} x_{t-1}-\beta_{yw} y_{t-1}-\beta_{zw} z_{t-1}\right)\cdot\eta_{w}+\epsilon_{w} \\
x_t & =x_{t-1}\left(\alpha_x-\alpha_x x_{t-1}-\beta_{wx} w_{t-1}-\beta_{yx} y_{t-1}-\beta_{zx} z_{t-1}\right)\cdot\eta_{x}+\epsilon_{x} \\
y_t & =y_{t-1}\left(\alpha_y-\alpha_y y_{t-1}-\beta_{wy} w_{t-1}-\beta_{xy} x_{t-1}-\beta_{zy} z_{t-1}\right)\cdot\eta_{y}+\epsilon_{y} \\
z_t & =z_{t-1}\left(\alpha_z-\alpha_z z_{t-1}-\beta_{wz} w_{t-1}-\beta_{xz} x_{t-1}-\beta_{yz} y_{t-1}\right)\cdot\eta_{z}+\epsilon_{z}
\end{aligned}
\end{equation}
For the 4-species system (Eq.~\ref{eq:4var}), the coefficients $\alpha_w=3.70, \alpha_x=3.78, \alpha_y=3.72, \alpha_z=3.70$ are used. For each system, the coupling coefficient $\beta_{ij}$ is either 0 or 0.35, depending on whether the causal interaction is present or absent. 

% An example of a 3-variable chain structure ($x \Rightarrow y \Rightarrow z$) in the 3-species system is as follows in Eq.~\ref{eq:3varChain}, with the interaction coefficients $\beta_{xy}, \beta_{yz}$ activated while all the other $\beta$ set to 0.

% \begin{equation}
% \label{eq:3varChain}
% \begin{aligned}
% x_t & =x_{t-1}\left(3.70-3.70 x_{t-1}\right)\cdot\eta_{x}+\epsilon_{x} \\
% y_t & =y_{t-1}\left(3.78-3.78 y_{t-1}-0.35 x_{t-1}\right)\cdot\eta_{y}+\epsilon_{y} \\
% z_t & =z_{t-1}\left(3.72-3.72 z_{t-1}-0.35 y_{t-1}\right)\cdot\eta_{z}+\epsilon_{z}
% \end{aligned}
% \end{equation}

For higher dimensional systems, we follow a similar logic, where the autonomous dynamics coefficients $\alpha$ are sampled in range $[3.70, 3.80)$, and the coupling coefficients $\beta$ are 0.35 when there is causal interaction between a variable pair (0 if no causal interaction). These systems are used to evaluate the effectiveness of multiPCM and MXMap in capturing both direct and indirect causal links in controlled scenarios. The ground truth causal structures of all used systems (from 3-variable to 7-variable) are demonstrated in Appendix~\ref{appsec:complet}.



\subsubsection{ERA5 Reanalysis Meteorological Data}

We also evaluate MXMap on real-world meteorological data from the ERA5 Global Reanalysis dataset~\citep{hersbach2020era5}, provided by the Copernicus program by ECMWF. The ERA5 dataset offers hourly climate variables over a global scale, allowing us to investigate causality in a practical environmental setting.

% We extract hourly winter data (December to February) for the Montreal region from 1981 to 2023. We evaluate the causal discovery methods on a causal chain: total cloud water ($tcw$) $\Rightarrow$ total radiation ($rad$) $\Rightarrow$ near-ground temperature ($T_{2m}$). This causal chain is particularly pronounced in winter, where cloud coverage directly impacts radiation levels, which in turn affects near-ground temperature (Explanation of this mechanism provided in Appendix~\ref{appsec:weather-chain}). By focusing exclusively on winter data, we capture this chain as the most prominent phenomenon, providing an ideal test case for assessing MXMap's effectiveness in uncovering causal relationships within complex, real-world environmental systems.

We extract hourly winter data (December to February) for the Montreal region from 1981 to 2023. Two experimental setups were designed to assess the effectiveness of causal discovery methods:

\begin{itemize}
    \item \textbf{3V Chain}: 
    A simplified 3-variable system capturing the causal chain: $tcw \Rightarrow rad \Rightarrow  T_{2m}$. 
    This causal chain is particularly pronounced in winter when cloud coverage strongly affects radiation levels, which subsequently modulate ground-level temperatures.
    \item \textbf{5V System}: This system includes solar radiation ($rad_{solar}$), terrestrial radiation ($rad_{terr}$), near-ground temperature advection ($T_{adv950}$), total cloud water ($tcw$) and near-ground temperature ($T_{2m}$). This setup examines whether well-established causal relationships can be detected, namely $rad_{solar} \Rightarrow T_{2m}$, $rad_{terr} \Rightarrow T_{2m}$, $T_{adv950} \Rightarrow T_{2m}$, $tcw\Rightarrow rad_{solar}$. 
    The 5V system builds on the 3V chain by introducing two radiation components as well as a temperature advection term. While the complete causal graph of this system is not fully established due to its complexity, the bivariate relationships outlined above are well-supported in meteorological literature and provide a robust benchmark for evaluation.
\end{itemize}

Detailed explanations of these mechanisms and meteorological contexts are provided in Appendix~\ref{appsec:era5}. By focusing on winter data, when these causal mechanisms are most pronounced, these setups serve as an ideal test case for assessing MXMap’s ability to uncover causal relationships in complex, real-world environmental systems.


\subsection{Validation of multiPCM}
\label{sec:valid_multiPCM}

To validate the effectiveness of multiPCM in distinguishing between direct and indirect causal relationships, we perform experiments on simulated four-variable systems generated without noise. Specifically, we evaluate three scenarios: purely direct causality, purely indirect causality, and combined direct and indirect causality, as illustrated in Table~\ref{tab:multiPCM}. The causal relationships of interest are highlighted in color, while the other variables are shown in gray and grouped together to form a multivariate embedding, used as the condition set ($Conds$) for multiPCM.

We use an input length of $L = 3500$ time steps for all tests and conduct multiPCM on a range of lag and embedding dimension values ($\tau, E \in \{1, 2, 3, \ldots, 8\}$). The results of the grid search are presented in Table~\ref{tab:multiPCM} (complete table with more cases in Appendix~\ref{appsec:valid_multiPCM}), where we analyze the correlation ratio $\gamma=\rho_{Direct}/\rho_{All}$ and the predicted labels indicating whether direct causality between colored nodes is rejected and to be removed based on a PCM threshold of 0.45 (this empirical threshold selection is discussed in Appendix~\ref{appsec:thres_multiPCM}). \textcolor{red}{Red} label indicates rejection, since only indirect causality exists; while \textcolor{blue}{blue} label indicates the existence direct causality, hence the link between colored nodes should be kept.

\begin{table}[htb]
\centering

\makebox[\linewidth]{%

\resizebox{0.85\textwidth}{!}{% Adjust the scaling factor to exceed column width

\begin{tabular}{c|c|c|c}
Type         & $Direct$ & $Indirect$ & $Both$ \\ \hline
Causality         &\begin{minipage}{.15\linewidth} \centering \includegraphics[width=0.3\linewidth]{imgs/ValidPCM/4VDirect.png} \end{minipage}& \begin{minipage}{.15\linewidth} \centering \includegraphics[width=0.3\linewidth]{imgs/ValidPCM/4VIndirect1.png} \end{minipage}   & \begin{minipage}{.15\linewidth} \centering \includegraphics[width=0.5\linewidth]{imgs/ValidPCM/4VBoth1.png} \end{minipage}   \\ \hline
$\rho_{All}$    &\begin{minipage}{.3\linewidth} \centering \includegraphics[width=\linewidth]{imgs/ValidPCM/4VDirect_sc1.png} \end{minipage}& \begin{minipage}{.3\linewidth} \centering \includegraphics[width=\linewidth]{imgs/ValidPCM/4VIndirect1_sc1.png} \end{minipage}    &  \begin{minipage}{.3\linewidth} \centering \includegraphics[width=\linewidth]{imgs/ValidPCM/4VBoth1_sc1.png} \end{minipage}  \\ \hline
$\rho_{Direct}$ &\begin{minipage}{.3\linewidth} \centering \includegraphics[width=\linewidth]{imgs/ValidPCM/4VDirect_sc2.png} \end{minipage}& \begin{minipage}{.3\linewidth} \centering \includegraphics[width=\linewidth]{imgs/ValidPCM/4VIndirect1_sc2.png} \end{minipage}    & \begin{minipage}{.3\linewidth} \centering \includegraphics[width=\linewidth]{imgs/ValidPCM/4VBoth1_sc2.png} \end{minipage} \\ \hline
Ratio             &\begin{minipage}{.3\linewidth} \centering \includegraphics[width=\linewidth]{imgs/ValidPCM/4VDirect_ratio.png} \end{minipage}& \begin{minipage}{.3\linewidth} \centering \includegraphics[width=\linewidth]{imgs/ValidPCM/4VIndirect1_ratio.png} \end{minipage}    & \begin{minipage}{.3\linewidth} \centering \includegraphics[width=\linewidth]{imgs/ValidPCM/4VBoth1_ratio.png} \end{minipage}  \\ \hline
Label    &\begin{minipage}{.3\linewidth} \centering \includegraphics[width=\linewidth]{imgs/ValidPCM/4VDirect_label.png} \end{minipage}& \begin{minipage}{.3\linewidth} \centering \includegraphics[width=\linewidth]{imgs/ValidPCM/4VIndirect1_label.png} \end{minipage}    & \begin{minipage}{.3\linewidth} \centering \includegraphics[width=\linewidth]{imgs/ValidPCM/4VBoth1_label.png} \end{minipage}
\end{tabular}

}

}

\caption{Performance of multiPCM (Full Table in Appendix~\ref{appsec:valid_multiPCM}): Profiles of correlation scores, correlation ratios, predicted label ($thres=0.45$) under grid search. Red dot indicates there isn't direct causality between the colored nodes, while blue indicates there is direct causality between the colored nodes.}

\label{tab:multiPCM}

\end{table}


The experimental results demonstrate the effectiveness of multiPCM in distinguishing direct and indirect causal relationships under different scenarios:
\begin{enumerate}
    \item \textbf{Direct Causality:} For cases involving direct causality ($Direct$ and $Both$), we observe that both correlation scores, $\rho_{All}$ and $\rho_{Direct}$, are significantly higher when the lag $\tau$ is small, and start dropping as lag increases. This trend is also consistent with observations from previous works on cross mapping in nonlinear systems.
    \item \textbf{Indirect Causality:} In the case of purely indirect causality, the correlation profiles tend to be inconsistent across different lags and embedding dimensions, showing fluctuating surfaces rather than a clear decreasing trend. This behavior is characteristic of indirect interactions, as these relationships become weaker and more unstable with increasing delay.
    \item \textbf{Optimal Hyperparameter Range:} 
    The grid search results for predicted labels further illustrate the behavior of multiPCM. There exists an ideal range of lag and embedding dimension values for accurately inferring causality: Here in Table~\ref{tab:multiPCM}, when the multiPCM lag ($\tau$) is less than 2 and the embedding dimension ($E$) is in the interval of $[3, 8]$, multiPCM produces consistent and accurate results across these three causal structures. In practice, if the approximate system dimension and timescale of the lag are known, the appropriate selections for $\tau$ and $E$ would likely align closely with the ground truth values: For the demonstrated 3-variable (3V) and 4-variable (4V) systems, the actual system dimension is 3 or 4, the generating dynamics use a lag of 1 (All these ground-truth $\tau$ and $E$ values fall in the detected ranges above). Notably, the selection of $E$ seems to be more tolerant for slight overestimation when the true state dimension is unknown. Thus, when the exact ground truth is unavailable, a slightly higher dimension for delay embedding may still yield reliable results, and can potentially help account for latent variables.
\end{enumerate}

These observations show that multiPCM is capable of correctly distinguishing direct and indirect causalities in multivariate scenarios. By performing cross mapping with multivariate embeddings, multiPCM achieves consistent causal inference that is robust across different lag and embedding dimension configurations.



\subsection{Prediction Consistency}

Mirage correlations are common in nonlinear dynamical systems. Coupled nonlinear systems often exhibit transient correlations between variables, which may change or disappear entirely when different subsequences are sampled. This presents a significant challenge for causal discovery, particularly for methods relying on consistent predictive relationships, such as Granger causality. The original CCM paper~\cite{sugihara2012detecting} illustrated this phenomenon using a bivariate system of competing species.

\begin{figure}[htb]
    \centering
    \begin{minipage}[b]{0.31\linewidth}
        \centering
        \includegraphics[width=\linewidth]{imgs/PredConsi/4varChain2.png}
        \captionsetup{justification=centering}
        \caption*{(a) Subsequence 1}
    \end{minipage}
    \begin{minipage}[b]{0.31\linewidth}
        \centering
        \includegraphics[width=\linewidth]{imgs/PredConsi/4varChain1.png}
        \captionsetup{justification=centering}
        \caption*{(b) Subsequence 2}
    \end{minipage}
    \begin{minipage}[b]{0.31\linewidth}
        \centering
        \includegraphics[width=\linewidth]{imgs/PredConsi/4varChain3.png}
        \captionsetup{justification=centering}
        \caption*{(c) Subsequence 3}
    \end{minipage}
    % \begin{minipage}[b]{0.245\linewidth}
    %     \centering
    %     \includegraphics[width=\linewidth]{imgs/PredConsi/4varChain4.png}
    %     \captionsetup{justification=centering}
    %     \caption*{(d) Subsequence 4}
    % \end{minipage}
    % \caption{Visualizations of subsequences from a 4-species chain system. \textcolor{red}{Di: please briefly explain the findings of these four figures here} }
    \caption{Visualizations of subsequences from a 4-species chain system: In the same sequence, correlations between variables can be positive, negative or zero when sampling from different start points.}
    \label{fig:4varChainNoNoise}
\end{figure}


To evaluate the robustness of our proposed approach in such scenarios, we first illustrate the chaotic behavior and mirage correlations in the noise-free four-species chain system ($w \Rightarrow x \Rightarrow y \Rightarrow z$), as defined in Section~\ref{sec:genData}. We randomly sample different starting points and visualize subsequences (of length 25) from these sampled points in Fig.~\ref{fig:4varChainNoNoise}. As shown, the correlations between variables vary widely across different subsequences, exhibiting positive, near-zero, and even negative correlations.

\begin{table}[htb]
\centering

\resizebox{1.0\textwidth}{!}{
\begin{tabular}{l|c|ccccc}
\textbf{Model}            & \textbf{  MXMap  } & \multicolumn{5}{c}{\textbf{RESIT-MLP}}                                                                             \\ \hline
\textbf{Prediction} &   
\begin{minipage}{.045\textwidth}
\centering
    \includegraphics[width=\linewidth]{imgs/PredConsi/4varChain_mxmap.png}
\end{minipage} 
& \multicolumn{1}{c|}{
\begin{minipage}{.12\textwidth}
\centering
    \includegraphics[width=\linewidth]{imgs/PredConsi/4varChain_RESIT1.png}
\end{minipage} 
}  & \multicolumn{1}{c|}{
\begin{minipage}{.12\textwidth}
\centering
    \includegraphics[width=\linewidth]{imgs/PredConsi/4varChain_RESIT2.png}
\end{minipage} 
}  & \multicolumn{1}{c|}{
\begin{minipage}{.12\textwidth}
\centering
    \includegraphics[width=\linewidth]{imgs/PredConsi/4varChain_RESIT3.png}
\end{minipage} 
}  & \multicolumn{1}{c|}{
\begin{minipage}{.12\textwidth}
\centering
    \includegraphics[width=\linewidth]{imgs/PredConsi/4varChain_RESIT4.png}
\end{minipage} 
}  & 
\begin{minipage}{.12\textwidth}
\centering
    \includegraphics[width=\linewidth]{imgs/PredConsi/4varChain_RESIT5.png}
\end{minipage} 
\\ \hline
\textbf{Precision}            & \textbf{1.0}    & \multicolumn{1}{c|}{0.5000} & \multicolumn{1}{c|}{0.3333} & \multicolumn{1}{c|}{0.2500} & \multicolumn{1}{c|}{0.5000} & 0.4000
\\ \hline
\textbf{Recall}            & \textbf{1.0}    & \multicolumn{1}{c|}{1.0} & \multicolumn{1}{c|}{0.6667} & \multicolumn{1}{c|}{0.3333} & \multicolumn{1}{c|}{0.6667} & 0.6667
\\ \hline
\textbf{F1}            & \textbf{1.0}    & \multicolumn{1}{c|}{0.6667} & \multicolumn{1}{c|}{0.4444} & \multicolumn{1}{c|}{0.2857} & \multicolumn{1}{c|}{0.5714} & 0.5000
\\ \hline
\textbf{SHD}            & \textbf{0}    & \multicolumn{1}{c|}{3} & \multicolumn{1}{c|}{5} & \multicolumn{1}{c|}{5} & \multicolumn{1}{c|}{3} & 4
\\ \hline
\textbf{Count}            & \textbf{10}    & \multicolumn{1}{c|}{4} & \multicolumn{1}{c|}{3} & \multicolumn{1}{c|}{1} & \multicolumn{1}{c|}{1} & 1
\end{tabular}
}

\caption{Comparison of MXMap and RESIT-MLP on 10 randomly sampled segments from a 4-species chain system $w \Rightarrow x \Rightarrow y \Rightarrow z$.}

\label{tab:PredCons}

\end{table}


To further assess prediction consistency, we apply the MXMap and RESIT (recapitulation in Appendix~\ref{appsec:resit}) frameworks to different segments of the sequences to determine the causal order of variables. Specifically, we generate 10 random positive integers as starting timestamps to sample 10 test sequences from the 4-species chain system (Eq.~\ref{eq:4var}), each with a length of 3500.

For MXMap, we select a $k$-nearest neighbor ($k$NN) size of 10, a PCM correlation ratio threshold of 0.6, and delay embedding parameters $\tau = 2$ and $dim = 6$. For RESIT, we used the scikit-learn MLP regressor with two layers of 32 units each, along with an HSIC threshold ($\alpha = 0.01$) for edge removal. 

Table~\ref{tab:PredCons} shows the results (evaluation metrics listed in Appendix~\ref{appsec:metrics}). MXMap consistently determined the correct causal order across all sampled segments, yielding stable predictions with perfect precision, recall, and F1 scores. In contrast, RESIT-MLP's predictions varied significantly depending on the starting point of each sequence. The causal graph predicted by RESIT changed across different segments, illustrating the sensitivity of its prediction to initial conditions due to the underlying predictive model assumption. Additionally, MXMap achieved better evaluation metrics across all four selected metrics.

\subsection{Comparison with Other Established Causal Inference Methods}

To demonstrate the effectiveness of MXMap in multivariate causal inference for nonlinear dynamical systems, we compare its performance on simulated multivariate dynamical systems (including cycles) with baseline methods (tsFCI, VAR-LiNGAM, PCMCI, Granger Causality, DYNOTEARS, SLARAC). When interpreting the predicted outputs, we consider non-oriented causal edges and bidirectional causal edges equivalent, which is in turn reflected in the metric calculation. 


\subsubsection{Simulated Systems}
\label{sec:baseline_sim}

\begin{table}[htb]
\makebox[\linewidth]{%

\resizebox{0.95\textwidth}{!}{% Adjust the scaling factor to exceed column width

\begin{tabular}{l|cc|cc|cc|cc}
\multicolumn{1}{c|}{\multirow{2}{*}{Structure}} & \multicolumn{2}{c|}{3V Chain}                             & \multicolumn{2}{c|}{3V Immorality}                        & \multicolumn{2}{c|}{3V No Cycle}                          & \multicolumn{2}{c}{3V Cycle}                             \\ \cline{2-9} 
\multicolumn{1}{c|}{}                           & \multicolumn{1}{l}{No Noise} & \multicolumn{1}{l|}{Noise} & \multicolumn{1}{l}{No Noise} & \multicolumn{1}{l|}{Noise} & \multicolumn{1}{l}{No Noise} & \multicolumn{1}{l|}{Noise} & \multicolumn{1}{l}{No Noise} & \multicolumn{1}{l}{Noise} \\ \hline
tsFCI                                           & 4                            & 3                          & 2                            & 2                          & 6                            & 4                          & 3                            & 5                         \\
VARLiNGAM                                       & 2                            & 4                          & 4                            & 4                          & 5                            & 6                          & 3                            & 4                         \\
Granger                                         & 4                            & 4                          & 4                            & 4                          & 5                            & 6                          & 2                            & 4                         \\
PCMCI                                           & \textbf{1}                   & \textbf{0}                 & \textbf{0}                   & \textbf{0}                 & 2                            & 1                          & 3                            & 3                         \\
DYNOTEARS                                       & 3                            & 4                          & 2                            & 2                          & 3                            & 1                          & 5                            & 3                         \\
SLARAC                                          & 6                            & 6                          & 6                            & 6                          & 5                            & 6                          & 5                            & 4                         \\ \hline
MXMap                                           & \textbf{1}                   & \textbf{0}                 & 1                            & \textbf{0}                 & \textbf{0}                   & \textbf{0}                 & \textbf{0}                   & \textbf{0}               
\end{tabular}
} }
\caption{SHD scores of MXMap and baselines for 3V settings on simulated no-noise and noisy dynamical systems.}
\label{tab:mxmap-sim-3V}
\end{table}

\begin{table}[htb]
\makebox[\linewidth]{%

\resizebox{0.72\textwidth}{!}{% Adjust the scaling factor to exceed column width

\begin{tabular}{l|cc|cc|cc}
\multicolumn{1}{c|}{\multirow{2}{*}{Structure}} & \multicolumn{2}{c|}{4V Chain}                             & \multicolumn{2}{c|}{4V No Cycle}                          & \multicolumn{2}{c}{4V Cycle}                             \\ \cline{2-7} 
\multicolumn{1}{c|}{}                           & \multicolumn{1}{l}{No Noise} & \multicolumn{1}{l|}{Noise} & \multicolumn{1}{l}{No Noise} & \multicolumn{1}{l|}{Noise} & \multicolumn{1}{l}{No Noise} & \multicolumn{1}{l}{Noise} \\ \hline
tsFCI                                           & 3                            & 3                          & 5                            & 4                          & 4                            & 6                         \\
VARLiNGAM                                       & 4                            & 6                          & 7                            & 8                          & 4                            & 5                         \\
Granger                                         & 4                            & 6                          & 5                            & 8                          & 3                            & 6                         \\
PCMCI                                           & 1                            & \textbf{0}                 & \textbf{1}                   & \textbf{1}                 & 8                            & 4                         \\
DYNOTEARS                                       & 7                            & 10                         & 3                            & \textbf{1}                 & 7                            & 6                         \\
SLARAC                                          & 10                           & 10                         & 3                            & 2                          & 5                            & 6                         \\ \hline
MXMap                                           & \textbf{0}                   & \textbf{0}                 & \textbf{1}                   & \textbf{1}                 & \textbf{0}                   & \textbf{2}               
\end{tabular}
} }
\caption{SHD scores of MXMap and baselines for 4V settings on simulated no-noise and noisy dynamical systems.}
\label{tab:mxmap-sim-4V}
\end{table}

\begin{table}[htb]
\makebox[\linewidth]{%

\resizebox{0.95\textwidth}{!}{% Adjust the scaling factor to exceed column width

\begin{tabular}{l|cc|cc|cc|cc}
\multicolumn{1}{c|}{\multirow{2}{*}{Structure}} & \multicolumn{2}{c|}{5V  No Cycle}                         & \multicolumn{2}{c|}{5V  Cycle}                            & \multicolumn{2}{c|}{6V  No Cycle}                         & \multicolumn{2}{c}{7V  Cycle}                            \\ \cline{2-9} 
\multicolumn{1}{c|}{}                           & \multicolumn{1}{l}{No Noise} & \multicolumn{1}{l|}{Noise} & \multicolumn{1}{l}{No Noise} & \multicolumn{1}{l|}{Noise} & \multicolumn{1}{l}{No Noise} & \multicolumn{1}{l|}{Noise} & \multicolumn{1}{l}{No Noise} & \multicolumn{1}{l}{Noise} \\ \hline
tsFCI                                           & 7                            & 4                          & 6                            & 8                          & 10                           & 11                         & 10                           & 10                        \\
VARLiNGAM                                       & 6                            & 6                          & 12                           & 12                         & 9                            & 11                         & 16                           & 15                        \\
Granger                                         & 6                            & 6                          & 12                           & 12                         & 11                           & 12                         & 15                           & 15                        \\
PCMCI                                           & 5                            & 5                          & 5                            & \textbf{2}                 & 11                           & \textbf{3}                 & 11                           & 10                        \\
DYNOTEARS                                       & 8                            & 15                         & 11                           & 12                         & 19                           & 18                         & 16                           & 22                        \\
SLARAC                                          & 16                           & 16                         & 18                           & 18                         & 25                           & 27                         & 21                           & 25                        \\ \hline
MXMap                                           & \textbf{1}                   & \textbf{1}                 & \textbf{0}                   & \textbf{2}                 & \textbf{2}                   & 4                          & \textbf{4}                   & \textbf{6}               
\end{tabular}
} }
\caption{SHD scores of MXMap and baselines for 5V-7V settings on simulated no-noise and noisy dynamical systems.}
\label{tab:mxmap-sim-5-7V}
\end{table}


Tables \ref{tab:mxmap-sim-3V}, \ref{tab:mxmap-sim-4V} and \ref{tab:mxmap-sim-5-7V} show the results of SHD scores (best in bold) on simulated systems with varying complexity from 3 to 7 variables. A more complete evaluation with all four metrics ($Prec$, $Rec$, $F1$, and $SHD$), along with visualizations of ground truth graphs, predicted causal graphs, is provided in Appendix~\ref{appsec:complet}. The time series data are generated under both noise-free and noisy settings (Gaussian additive noise, strength 0.01). 
% Visualizations of the ground truth and predicted graphs are provided in the Appendix~\ref{appsec:complet}. 
Overall, MXMap consistently achieves good performance the baselines, yielding lower SHD scores which indicate fewer incorrect edges in the predicted causal graphs.



\subsubsection{ERA5 3-Variable Chain: $tcw \Rightarrow rad \Rightarrow  T_{2m}$}
\label{sec:weather_chain}

\begin{table}[hbt]
\centering
\resizebox{0.7\textwidth}{!}{
\begin{tabular}{c|cccc|c}
Method & \multicolumn{1}{c}{PC} & \multicolumn{1}{c}{FCI} & \multicolumn{1}{c}{LiNGAM} & \multicolumn{1}{c|}{PCMCI}& \multicolumn{1}{c}{MXMap} \\ \hline
Output & \begin{minipage}{.06\linewidth} \centering \includegraphics[width=\linewidth]{imgs/ERA5/pc-fci-era5.png} \end{minipage} & \begin{minipage}{.06\linewidth} \centering \includegraphics[width=\linewidth]{imgs/ERA5/pc-fci-era5.png} \end{minipage} & \begin{minipage}{.12\linewidth} \centering \includegraphics[width=\linewidth]{imgs/ERA5/lingam-era5.png} \end{minipage} & \begin{minipage}{.06\linewidth} \centering \includegraphics[width=\linewidth]{imgs/ERA5/pcmci-era5.png} \end{minipage} & \begin{minipage}{.06\linewidth} \centering \includegraphics[width=\linewidth]{imgs/ERA5/mxmap-era5.png} \end{minipage}  \\ \hline
$Prec$ & 0.33 & 0.33  & 0  & 0.50  & \textbf{0.67} \\ \hline
$Rec$ & \textbf{1.0} & \textbf{1.0}  & 0  & \textbf{1.0}  & \textbf{1.0} \\ \hline
$F1$ & 0.50  & 0.50  & 0  & 0.67  & \textbf{0.80} \\ \hline
$SHD$   & 4  & 4  & 4  & 2  & \textbf{1}                        
\end{tabular}
}
\caption{Causal inference methods on the ERA5 3V system.}
\label{tab:mxmap-era5}
\end{table}

For inferring the chain $tcw \Rightarrow rad \Rightarrow T_{2m}$, we take input sequence length of 6000, and consider the lag $\tau$ value to be 4 (for the lag value of delay embedding formulation, and max lag of the PCMCI method) While other methods either failed to identify causal directions correctly or predicted incorrect causal orders, MXMap consistently maintained the correct causal order and produced results closest to the expected ground truth.

\subsubsection{ERA5 5-Variable System: $tcw$, $rad_{solar}$, $rad_{terr}$, $T_{adv950}$ and $T_{2m}$}
\label{sec:era5_5V_eval}

\begin{table}[htb]
\centering
\resizebox{1\textwidth}{!}{
\begin{tabular}{l|llllll|l}
Methods                          & tsFCI                            & VARLiNGAM                        & Granger                          & PCMCI                            & DYNOTEARS                        & SLARAC                           & MXMap      \\ \hline
% Output                           &                                  &                                  &                                  &                                  &                                  &                                  &            \\ \hline
$rad_{solar} \Rightarrow T_{2m}$ & \cmark                       & \halfcheckmark & \halfcheckmark & \halfcheckmark & \halfcheckmark & \xmark                           & \cmark \\
$rad_{terr} \Rightarrow T_{2m}$  & \halfcheckmark & \halfcheckmark & \cmark                       & \halfcheckmark & \xmark                           & \cmark                       & \cmark \\
$T_{adv950} \Rightarrow T_{2m}$  & \xmark                           & \halfcheckmark & \halfcheckmark & \halfcheckmark & \cmark                       & \halfcheckmark & \cmark \\
$tcw\Rightarrow rad_{solar}$     & \xmark                           & \xmark                           & \halfcheckmark & \cmark                       & \cmark                       & \cmark                       & \cmark
\end{tabular}
}
\caption{Detection of Benchmark Causal Relationships in the ERA5 5V System (full visualizations in Table~\ref{apptab:mxmap-era5-5V}). A checkmark (green) indicates a correctly detected and oriented edge, a half-checkmark (gray) denotes a detected but ambiguously oriented edge, and a crossmark (red) represents an undetected or incorrectly oriented edge.}
\label{tab:mxmap-era5-5V}
\end{table}

The objective is to evaluate the performance via observing how many of these well-established causal relationships are detected by the methods $rad_{solar} \Rightarrow T_{2m}$, $rad_{terr} \Rightarrow T_{2m}$, $T_{adv950} \Rightarrow T_{2m}$, $tcw\Rightarrow rad_{solar}$ (explanations of these mechanisms in Appendix~\ref{appsec:era5}, full table with predicted graphs in Table~\ref{apptab:mxmap-era5-5V} in Appendix~\ref{appsec:era5_eval}). The results in Table~\ref{tab:mxmap-era5-5V} show that MXMap is overall able to correctly identify and orient edges that represent the 4 benchmark causal mechanisms in the 5-variable system, and outperforms the other baseline methods.
\section{Conclusion}

In this work, we proposed multiPCM, allowing us to more effectively distinguish between direct and indirect causal relationships. We integrated multiPCM with bivariate Convergent Cross Mapping (CCM) in a two-phase framework, MXMap, that first establishes an initial causal graph and then prunes indirect connections. Through experiments on simulated species interaction systems and real-world ERA5 meteorological data, we demonstrated that MXMap outperforms traditional methods and exhibits robust performance in complex, high-dimensional dynamical systems.

There are still limits to this current framework (discussed in Appendix~\ref{appsec:mxmap_limits}), demonstrated in runtime complexity, scalability, and possible failure cases in highly noisy environment or non-stationary systems. Future work could focus on enhancing the robustness of cross mapping under noisy conditions~\citep{monster2017causal}. Investigating and incorporating certain noise-handling mechanisms~\citep{zhang2024enhancing} in MXMap could further enhance its applicability in noisy real-world scenarios. Another direction is to explore adaptive parameter selection~\citep{shortreed2017outcome, machlanski2023hyperparameter} for MXMap, such as optimizing embedding dimensions and lags based on data properties and currently outputs. Current grid search methods are computationally expensive for larger datasets, and efficient heuristic or learning-based tuning could improve scalability. Finally, applying MXMap to other real-world domains, such as power systems, larger timescale climate modeling, and epidemiology, could further validate its versatility and reveal complex causal interactions.

% Acknowledgments---Will not appear in anonymized version
\acks{This work is supported by Mitacs Accelerate Research Fellowship in collaboration with Hydro-Québec Research Institute (IREQ).}

% \section{Introduction}

% This is where the content of your paper goes.
% \begin{itemize}
%   \item Limit the main text (not counting references and appendices) to 12 PMLR-formatted pages, using this template. Please add any additional appendix to the same file after references - there is no page limit for the appendix.
%   \item Include, either in the main text or the appendices, \emph{all} details, proofs and derivations required to substantiate the results.
%   \item The contribution, novelty and significance of submissions will be judged primarily based on
% \textit{the main text of 12 pages}. Thus, include enough details, and overview of key arguments, 
% to convince the reviewers of the validity of result statements, and to ease parsing of technical material in the appendix.
%   \item Use the \textbackslash documentclass[anon,12pt]\{clear2025\} option during submission process -- this automatically hides the author names listed under \textbackslash clearauthor. Submissions should NOT include author names or other identifying information in the main text or appendix. To the extent possible, you should avoid including directly identifying information in the text. You should still include all relevant references, discussion, and scientific content, even if this might provide significant hints as to the author identity. But you should generally refer to your own prior work in third person. Do not include acknowledgments in the submission. They can be added in the camera-ready version of accepted papers.
  
%   Please note that while submissions must be anonymized, and author names are withheld from reviewers, they are known to the area chair overseeing the paper’s review.  The assigned area chair is allowed to reveal author names to a reviewer during the rebuttal period, upon the reviewer’s request, if they deem such information is needed in ensuring a proper review.  
%   \item Use \textbackslash documentclass[final,12pt]\{clear2025\} only during camera-ready submission.
% \end{itemize}


\bibliography{ref}

\newpage
%  no headers for the rest of the document
\pagestyle{plain}
\subsection{Lloyd-Max Algorithm}
\label{subsec:Lloyd-Max}
For a given quantization bitwidth $B$ and an operand $\bm{X}$, the Lloyd-Max algorithm finds $2^B$ quantization levels $\{\hat{x}_i\}_{i=1}^{2^B}$ such that quantizing $\bm{X}$ by rounding each scalar in $\bm{X}$ to the nearest quantization level minimizes the quantization MSE. 

The algorithm starts with an initial guess of quantization levels and then iteratively computes quantization thresholds $\{\tau_i\}_{i=1}^{2^B-1}$ and updates quantization levels $\{\hat{x}_i\}_{i=1}^{2^B}$. Specifically, at iteration $n$, thresholds are set to the midpoints of the previous iteration's levels:
\begin{align*}
    \tau_i^{(n)}=\frac{\hat{x}_i^{(n-1)}+\hat{x}_{i+1}^{(n-1)}}2 \text{ for } i=1\ldots 2^B-1
\end{align*}
Subsequently, the quantization levels are re-computed as conditional means of the data regions defined by the new thresholds:
\begin{align*}
    \hat{x}_i^{(n)}=\mathbb{E}\left[ \bm{X} \big| \bm{X}\in [\tau_{i-1}^{(n)},\tau_i^{(n)}] \right] \text{ for } i=1\ldots 2^B
\end{align*}
where to satisfy boundary conditions we have $\tau_0=-\infty$ and $\tau_{2^B}=\infty$. The algorithm iterates the above steps until convergence.

Figure \ref{fig:lm_quant} compares the quantization levels of a $7$-bit floating point (E3M3) quantizer (left) to a $7$-bit Lloyd-Max quantizer (right) when quantizing a layer of weights from the GPT3-126M model at a per-tensor granularity. As shown, the Lloyd-Max quantizer achieves substantially lower quantization MSE. Further, Table \ref{tab:FP7_vs_LM7} shows the superior perplexity achieved by Lloyd-Max quantizers for bitwidths of $7$, $6$ and $5$. The difference between the quantizers is clear at 5 bits, where per-tensor FP quantization incurs a drastic and unacceptable increase in perplexity, while Lloyd-Max quantization incurs a much smaller increase. Nevertheless, we note that even the optimal Lloyd-Max quantizer incurs a notable ($\sim 1.5$) increase in perplexity due to the coarse granularity of quantization. 

\begin{figure}[h]
  \centering
  \includegraphics[width=0.7\linewidth]{sections/figures/LM7_FP7.pdf}
  \caption{\small Quantization levels and the corresponding quantization MSE of Floating Point (left) vs Lloyd-Max (right) Quantizers for a layer of weights in the GPT3-126M model.}
  \label{fig:lm_quant}
\end{figure}

\begin{table}[h]\scriptsize
\begin{center}
\caption{\label{tab:FP7_vs_LM7} \small Comparing perplexity (lower is better) achieved by floating point quantizers and Lloyd-Max quantizers on a GPT3-126M model for the Wikitext-103 dataset.}
\begin{tabular}{c|cc|c}
\hline
 \multirow{2}{*}{\textbf{Bitwidth}} & \multicolumn{2}{|c|}{\textbf{Floating-Point Quantizer}} & \textbf{Lloyd-Max Quantizer} \\
 & Best Format & Wikitext-103 Perplexity & Wikitext-103 Perplexity \\
\hline
7 & E3M3 & 18.32 & 18.27 \\
6 & E3M2 & 19.07 & 18.51 \\
5 & E4M0 & 43.89 & 19.71 \\
\hline
\end{tabular}
\end{center}
\end{table}

\subsection{Proof of Local Optimality of LO-BCQ}
\label{subsec:lobcq_opt_proof}
For a given block $\bm{b}_j$, the quantization MSE during LO-BCQ can be empirically evaluated as $\frac{1}{L_b}\lVert \bm{b}_j- \bm{\hat{b}}_j\rVert^2_2$ where $\bm{\hat{b}}_j$ is computed from equation (\ref{eq:clustered_quantization_definition}) as $C_{f(\bm{b}_j)}(\bm{b}_j)$. Further, for a given block cluster $\mathcal{B}_i$, we compute the quantization MSE as $\frac{1}{|\mathcal{B}_{i}|}\sum_{\bm{b} \in \mathcal{B}_{i}} \frac{1}{L_b}\lVert \bm{b}- C_i^{(n)}(\bm{b})\rVert^2_2$. Therefore, at the end of iteration $n$, we evaluate the overall quantization MSE $J^{(n)}$ for a given operand $\bm{X}$ composed of $N_c$ block clusters as:
\begin{align*}
    \label{eq:mse_iter_n}
    J^{(n)} = \frac{1}{N_c} \sum_{i=1}^{N_c} \frac{1}{|\mathcal{B}_{i}^{(n)}|}\sum_{\bm{v} \in \mathcal{B}_{i}^{(n)}} \frac{1}{L_b}\lVert \bm{b}- B_i^{(n)}(\bm{b})\rVert^2_2
\end{align*}

At the end of iteration $n$, the codebooks are updated from $\mathcal{C}^{(n-1)}$ to $\mathcal{C}^{(n)}$. However, the mapping of a given vector $\bm{b}_j$ to quantizers $\mathcal{C}^{(n)}$ remains as  $f^{(n)}(\bm{b}_j)$. At the next iteration, during the vector clustering step, $f^{(n+1)}(\bm{b}_j)$ finds new mapping of $\bm{b}_j$ to updated codebooks $\mathcal{C}^{(n)}$ such that the quantization MSE over the candidate codebooks is minimized. Therefore, we obtain the following result for $\bm{b}_j$:
\begin{align*}
\frac{1}{L_b}\lVert \bm{b}_j - C_{f^{(n+1)}(\bm{b}_j)}^{(n)}(\bm{b}_j)\rVert^2_2 \le \frac{1}{L_b}\lVert \bm{b}_j - C_{f^{(n)}(\bm{b}_j)}^{(n)}(\bm{b}_j)\rVert^2_2
\end{align*}

That is, quantizing $\bm{b}_j$ at the end of the block clustering step of iteration $n+1$ results in lower quantization MSE compared to quantizing at the end of iteration $n$. Since this is true for all $\bm{b} \in \bm{X}$, we assert the following:
\begin{equation}
\begin{split}
\label{eq:mse_ineq_1}
    \tilde{J}^{(n+1)} &= \frac{1}{N_c} \sum_{i=1}^{N_c} \frac{1}{|\mathcal{B}_{i}^{(n+1)}|}\sum_{\bm{b} \in \mathcal{B}_{i}^{(n+1)}} \frac{1}{L_b}\lVert \bm{b} - C_i^{(n)}(b)\rVert^2_2 \le J^{(n)}
\end{split}
\end{equation}
where $\tilde{J}^{(n+1)}$ is the the quantization MSE after the vector clustering step at iteration $n+1$.

Next, during the codebook update step (\ref{eq:quantizers_update}) at iteration $n+1$, the per-cluster codebooks $\mathcal{C}^{(n)}$ are updated to $\mathcal{C}^{(n+1)}$ by invoking the Lloyd-Max algorithm \citep{Lloyd}. We know that for any given value distribution, the Lloyd-Max algorithm minimizes the quantization MSE. Therefore, for a given vector cluster $\mathcal{B}_i$ we obtain the following result:

\begin{equation}
    \frac{1}{|\mathcal{B}_{i}^{(n+1)}|}\sum_{\bm{b} \in \mathcal{B}_{i}^{(n+1)}} \frac{1}{L_b}\lVert \bm{b}- C_i^{(n+1)}(\bm{b})\rVert^2_2 \le \frac{1}{|\mathcal{B}_{i}^{(n+1)}|}\sum_{\bm{b} \in \mathcal{B}_{i}^{(n+1)}} \frac{1}{L_b}\lVert \bm{b}- C_i^{(n)}(\bm{b})\rVert^2_2
\end{equation}

The above equation states that quantizing the given block cluster $\mathcal{B}_i$ after updating the associated codebook from $C_i^{(n)}$ to $C_i^{(n+1)}$ results in lower quantization MSE. Since this is true for all the block clusters, we derive the following result: 
\begin{equation}
\begin{split}
\label{eq:mse_ineq_2}
     J^{(n+1)} &= \frac{1}{N_c} \sum_{i=1}^{N_c} \frac{1}{|\mathcal{B}_{i}^{(n+1)}|}\sum_{\bm{b} \in \mathcal{B}_{i}^{(n+1)}} \frac{1}{L_b}\lVert \bm{b}- C_i^{(n+1)}(\bm{b})\rVert^2_2  \le \tilde{J}^{(n+1)}   
\end{split}
\end{equation}

Following (\ref{eq:mse_ineq_1}) and (\ref{eq:mse_ineq_2}), we find that the quantization MSE is non-increasing for each iteration, that is, $J^{(1)} \ge J^{(2)} \ge J^{(3)} \ge \ldots \ge J^{(M)}$ where $M$ is the maximum number of iterations. 
%Therefore, we can say that if the algorithm converges, then it must be that it has converged to a local minimum. 
\hfill $\blacksquare$


\begin{figure}
    \begin{center}
    \includegraphics[width=0.5\textwidth]{sections//figures/mse_vs_iter.pdf}
    \end{center}
    \caption{\small NMSE vs iterations during LO-BCQ compared to other block quantization proposals}
    \label{fig:nmse_vs_iter}
\end{figure}

Figure \ref{fig:nmse_vs_iter} shows the empirical convergence of LO-BCQ across several block lengths and number of codebooks. Also, the MSE achieved by LO-BCQ is compared to baselines such as MXFP and VSQ. As shown, LO-BCQ converges to a lower MSE than the baselines. Further, we achieve better convergence for larger number of codebooks ($N_c$) and for a smaller block length ($L_b$), both of which increase the bitwidth of BCQ (see Eq \ref{eq:bitwidth_bcq}).


\subsection{Additional Accuracy Results}
%Table \ref{tab:lobcq_config} lists the various LOBCQ configurations and their corresponding bitwidths.
\begin{table}
\setlength{\tabcolsep}{4.75pt}
\begin{center}
\caption{\label{tab:lobcq_config} Various LO-BCQ configurations and their bitwidths.}
\begin{tabular}{|c||c|c|c|c||c|c||c|} 
\hline
 & \multicolumn{4}{|c||}{$L_b=8$} & \multicolumn{2}{|c||}{$L_b=4$} & $L_b=2$ \\
 \hline
 \backslashbox{$L_A$\kern-1em}{\kern-1em$N_c$} & 2 & 4 & 8 & 16 & 2 & 4 & 2 \\
 \hline
 64 & 4.25 & 4.375 & 4.5 & 4.625 & 4.375 & 4.625 & 4.625\\
 \hline
 32 & 4.375 & 4.5 & 4.625& 4.75 & 4.5 & 4.75 & 4.75 \\
 \hline
 16 & 4.625 & 4.75& 4.875 & 5 & 4.75 & 5 & 5 \\
 \hline
\end{tabular}
\end{center}
\end{table}

%\subsection{Perplexity achieved by various LO-BCQ configurations on Wikitext-103 dataset}

\begin{table} \centering
\begin{tabular}{|c||c|c|c|c||c|c||c|} 
\hline
 $L_b \rightarrow$& \multicolumn{4}{c||}{8} & \multicolumn{2}{c||}{4} & 2\\
 \hline
 \backslashbox{$L_A$\kern-1em}{\kern-1em$N_c$} & 2 & 4 & 8 & 16 & 2 & 4 & 2  \\
 %$N_c \rightarrow$ & 2 & 4 & 8 & 16 & 2 & 4 & 2 \\
 \hline
 \hline
 \multicolumn{8}{c}{GPT3-1.3B (FP32 PPL = 9.98)} \\ 
 \hline
 \hline
 64 & 10.40 & 10.23 & 10.17 & 10.15 &  10.28 & 10.18 & 10.19 \\
 \hline
 32 & 10.25 & 10.20 & 10.15 & 10.12 &  10.23 & 10.17 & 10.17 \\
 \hline
 16 & 10.22 & 10.16 & 10.10 & 10.09 &  10.21 & 10.14 & 10.16 \\
 \hline
  \hline
 \multicolumn{8}{c}{GPT3-8B (FP32 PPL = 7.38)} \\ 
 \hline
 \hline
 64 & 7.61 & 7.52 & 7.48 &  7.47 &  7.55 &  7.49 & 7.50 \\
 \hline
 32 & 7.52 & 7.50 & 7.46 &  7.45 &  7.52 &  7.48 & 7.48  \\
 \hline
 16 & 7.51 & 7.48 & 7.44 &  7.44 &  7.51 &  7.49 & 7.47  \\
 \hline
\end{tabular}
\caption{\label{tab:ppl_gpt3_abalation} Wikitext-103 perplexity across GPT3-1.3B and 8B models.}
\end{table}

\begin{table} \centering
\begin{tabular}{|c||c|c|c|c||} 
\hline
 $L_b \rightarrow$& \multicolumn{4}{c||}{8}\\
 \hline
 \backslashbox{$L_A$\kern-1em}{\kern-1em$N_c$} & 2 & 4 & 8 & 16 \\
 %$N_c \rightarrow$ & 2 & 4 & 8 & 16 & 2 & 4 & 2 \\
 \hline
 \hline
 \multicolumn{5}{|c|}{Llama2-7B (FP32 PPL = 5.06)} \\ 
 \hline
 \hline
 64 & 5.31 & 5.26 & 5.19 & 5.18  \\
 \hline
 32 & 5.23 & 5.25 & 5.18 & 5.15  \\
 \hline
 16 & 5.23 & 5.19 & 5.16 & 5.14  \\
 \hline
 \multicolumn{5}{|c|}{Nemotron4-15B (FP32 PPL = 5.87)} \\ 
 \hline
 \hline
 64  & 6.3 & 6.20 & 6.13 & 6.08  \\
 \hline
 32  & 6.24 & 6.12 & 6.07 & 6.03  \\
 \hline
 16  & 6.12 & 6.14 & 6.04 & 6.02  \\
 \hline
 \multicolumn{5}{|c|}{Nemotron4-340B (FP32 PPL = 3.48)} \\ 
 \hline
 \hline
 64 & 3.67 & 3.62 & 3.60 & 3.59 \\
 \hline
 32 & 3.63 & 3.61 & 3.59 & 3.56 \\
 \hline
 16 & 3.61 & 3.58 & 3.57 & 3.55 \\
 \hline
\end{tabular}
\caption{\label{tab:ppl_llama7B_nemo15B} Wikitext-103 perplexity compared to FP32 baseline in Llama2-7B and Nemotron4-15B, 340B models}
\end{table}

%\subsection{Perplexity achieved by various LO-BCQ configurations on MMLU dataset}


\begin{table} \centering
\begin{tabular}{|c||c|c|c|c||c|c|c|c|} 
\hline
 $L_b \rightarrow$& \multicolumn{4}{c||}{8} & \multicolumn{4}{c||}{8}\\
 \hline
 \backslashbox{$L_A$\kern-1em}{\kern-1em$N_c$} & 2 & 4 & 8 & 16 & 2 & 4 & 8 & 16  \\
 %$N_c \rightarrow$ & 2 & 4 & 8 & 16 & 2 & 4 & 2 \\
 \hline
 \hline
 \multicolumn{5}{|c|}{Llama2-7B (FP32 Accuracy = 45.8\%)} & \multicolumn{4}{|c|}{Llama2-70B (FP32 Accuracy = 69.12\%)} \\ 
 \hline
 \hline
 64 & 43.9 & 43.4 & 43.9 & 44.9 & 68.07 & 68.27 & 68.17 & 68.75 \\
 \hline
 32 & 44.5 & 43.8 & 44.9 & 44.5 & 68.37 & 68.51 & 68.35 & 68.27  \\
 \hline
 16 & 43.9 & 42.7 & 44.9 & 45 & 68.12 & 68.77 & 68.31 & 68.59  \\
 \hline
 \hline
 \multicolumn{5}{|c|}{GPT3-22B (FP32 Accuracy = 38.75\%)} & \multicolumn{4}{|c|}{Nemotron4-15B (FP32 Accuracy = 64.3\%)} \\ 
 \hline
 \hline
 64 & 36.71 & 38.85 & 38.13 & 38.92 & 63.17 & 62.36 & 63.72 & 64.09 \\
 \hline
 32 & 37.95 & 38.69 & 39.45 & 38.34 & 64.05 & 62.30 & 63.8 & 64.33  \\
 \hline
 16 & 38.88 & 38.80 & 38.31 & 38.92 & 63.22 & 63.51 & 63.93 & 64.43  \\
 \hline
\end{tabular}
\caption{\label{tab:mmlu_abalation} Accuracy on MMLU dataset across GPT3-22B, Llama2-7B, 70B and Nemotron4-15B models.}
\end{table}


%\subsection{Perplexity achieved by various LO-BCQ configurations on LM evaluation harness}

\begin{table} \centering
\begin{tabular}{|c||c|c|c|c||c|c|c|c|} 
\hline
 $L_b \rightarrow$& \multicolumn{4}{c||}{8} & \multicolumn{4}{c||}{8}\\
 \hline
 \backslashbox{$L_A$\kern-1em}{\kern-1em$N_c$} & 2 & 4 & 8 & 16 & 2 & 4 & 8 & 16  \\
 %$N_c \rightarrow$ & 2 & 4 & 8 & 16 & 2 & 4 & 2 \\
 \hline
 \hline
 \multicolumn{5}{|c|}{Race (FP32 Accuracy = 37.51\%)} & \multicolumn{4}{|c|}{Boolq (FP32 Accuracy = 64.62\%)} \\ 
 \hline
 \hline
 64 & 36.94 & 37.13 & 36.27 & 37.13 & 63.73 & 62.26 & 63.49 & 63.36 \\
 \hline
 32 & 37.03 & 36.36 & 36.08 & 37.03 & 62.54 & 63.51 & 63.49 & 63.55  \\
 \hline
 16 & 37.03 & 37.03 & 36.46 & 37.03 & 61.1 & 63.79 & 63.58 & 63.33  \\
 \hline
 \hline
 \multicolumn{5}{|c|}{Winogrande (FP32 Accuracy = 58.01\%)} & \multicolumn{4}{|c|}{Piqa (FP32 Accuracy = 74.21\%)} \\ 
 \hline
 \hline
 64 & 58.17 & 57.22 & 57.85 & 58.33 & 73.01 & 73.07 & 73.07 & 72.80 \\
 \hline
 32 & 59.12 & 58.09 & 57.85 & 58.41 & 73.01 & 73.94 & 72.74 & 73.18  \\
 \hline
 16 & 57.93 & 58.88 & 57.93 & 58.56 & 73.94 & 72.80 & 73.01 & 73.94  \\
 \hline
\end{tabular}
\caption{\label{tab:mmlu_abalation} Accuracy on LM evaluation harness tasks on GPT3-1.3B model.}
\end{table}

\begin{table} \centering
\begin{tabular}{|c||c|c|c|c||c|c|c|c|} 
\hline
 $L_b \rightarrow$& \multicolumn{4}{c||}{8} & \multicolumn{4}{c||}{8}\\
 \hline
 \backslashbox{$L_A$\kern-1em}{\kern-1em$N_c$} & 2 & 4 & 8 & 16 & 2 & 4 & 8 & 16  \\
 %$N_c \rightarrow$ & 2 & 4 & 8 & 16 & 2 & 4 & 2 \\
 \hline
 \hline
 \multicolumn{5}{|c|}{Race (FP32 Accuracy = 41.34\%)} & \multicolumn{4}{|c|}{Boolq (FP32 Accuracy = 68.32\%)} \\ 
 \hline
 \hline
 64 & 40.48 & 40.10 & 39.43 & 39.90 & 69.20 & 68.41 & 69.45 & 68.56 \\
 \hline
 32 & 39.52 & 39.52 & 40.77 & 39.62 & 68.32 & 67.43 & 68.17 & 69.30  \\
 \hline
 16 & 39.81 & 39.71 & 39.90 & 40.38 & 68.10 & 66.33 & 69.51 & 69.42  \\
 \hline
 \hline
 \multicolumn{5}{|c|}{Winogrande (FP32 Accuracy = 67.88\%)} & \multicolumn{4}{|c|}{Piqa (FP32 Accuracy = 78.78\%)} \\ 
 \hline
 \hline
 64 & 66.85 & 66.61 & 67.72 & 67.88 & 77.31 & 77.42 & 77.75 & 77.64 \\
 \hline
 32 & 67.25 & 67.72 & 67.72 & 67.00 & 77.31 & 77.04 & 77.80 & 77.37  \\
 \hline
 16 & 68.11 & 68.90 & 67.88 & 67.48 & 77.37 & 78.13 & 78.13 & 77.69  \\
 \hline
\end{tabular}
\caption{\label{tab:mmlu_abalation} Accuracy on LM evaluation harness tasks on GPT3-8B model.}
\end{table}

\begin{table} \centering
\begin{tabular}{|c||c|c|c|c||c|c|c|c|} 
\hline
 $L_b \rightarrow$& \multicolumn{4}{c||}{8} & \multicolumn{4}{c||}{8}\\
 \hline
 \backslashbox{$L_A$\kern-1em}{\kern-1em$N_c$} & 2 & 4 & 8 & 16 & 2 & 4 & 8 & 16  \\
 %$N_c \rightarrow$ & 2 & 4 & 8 & 16 & 2 & 4 & 2 \\
 \hline
 \hline
 \multicolumn{5}{|c|}{Race (FP32 Accuracy = 40.67\%)} & \multicolumn{4}{|c|}{Boolq (FP32 Accuracy = 76.54\%)} \\ 
 \hline
 \hline
 64 & 40.48 & 40.10 & 39.43 & 39.90 & 75.41 & 75.11 & 77.09 & 75.66 \\
 \hline
 32 & 39.52 & 39.52 & 40.77 & 39.62 & 76.02 & 76.02 & 75.96 & 75.35  \\
 \hline
 16 & 39.81 & 39.71 & 39.90 & 40.38 & 75.05 & 73.82 & 75.72 & 76.09  \\
 \hline
 \hline
 \multicolumn{5}{|c|}{Winogrande (FP32 Accuracy = 70.64\%)} & \multicolumn{4}{|c|}{Piqa (FP32 Accuracy = 79.16\%)} \\ 
 \hline
 \hline
 64 & 69.14 & 70.17 & 70.17 & 70.56 & 78.24 & 79.00 & 78.62 & 78.73 \\
 \hline
 32 & 70.96 & 69.69 & 71.27 & 69.30 & 78.56 & 79.49 & 79.16 & 78.89  \\
 \hline
 16 & 71.03 & 69.53 & 69.69 & 70.40 & 78.13 & 79.16 & 79.00 & 79.00  \\
 \hline
\end{tabular}
\caption{\label{tab:mmlu_abalation} Accuracy on LM evaluation harness tasks on GPT3-22B model.}
\end{table}

\begin{table} \centering
\begin{tabular}{|c||c|c|c|c||c|c|c|c|} 
\hline
 $L_b \rightarrow$& \multicolumn{4}{c||}{8} & \multicolumn{4}{c||}{8}\\
 \hline
 \backslashbox{$L_A$\kern-1em}{\kern-1em$N_c$} & 2 & 4 & 8 & 16 & 2 & 4 & 8 & 16  \\
 %$N_c \rightarrow$ & 2 & 4 & 8 & 16 & 2 & 4 & 2 \\
 \hline
 \hline
 \multicolumn{5}{|c|}{Race (FP32 Accuracy = 44.4\%)} & \multicolumn{4}{|c|}{Boolq (FP32 Accuracy = 79.29\%)} \\ 
 \hline
 \hline
 64 & 42.49 & 42.51 & 42.58 & 43.45 & 77.58 & 77.37 & 77.43 & 78.1 \\
 \hline
 32 & 43.35 & 42.49 & 43.64 & 43.73 & 77.86 & 75.32 & 77.28 & 77.86  \\
 \hline
 16 & 44.21 & 44.21 & 43.64 & 42.97 & 78.65 & 77 & 76.94 & 77.98  \\
 \hline
 \hline
 \multicolumn{5}{|c|}{Winogrande (FP32 Accuracy = 69.38\%)} & \multicolumn{4}{|c|}{Piqa (FP32 Accuracy = 78.07\%)} \\ 
 \hline
 \hline
 64 & 68.9 & 68.43 & 69.77 & 68.19 & 77.09 & 76.82 & 77.09 & 77.86 \\
 \hline
 32 & 69.38 & 68.51 & 68.82 & 68.90 & 78.07 & 76.71 & 78.07 & 77.86  \\
 \hline
 16 & 69.53 & 67.09 & 69.38 & 68.90 & 77.37 & 77.8 & 77.91 & 77.69  \\
 \hline
\end{tabular}
\caption{\label{tab:mmlu_abalation} Accuracy on LM evaluation harness tasks on Llama2-7B model.}
\end{table}

\begin{table} \centering
\begin{tabular}{|c||c|c|c|c||c|c|c|c|} 
\hline
 $L_b \rightarrow$& \multicolumn{4}{c||}{8} & \multicolumn{4}{c||}{8}\\
 \hline
 \backslashbox{$L_A$\kern-1em}{\kern-1em$N_c$} & 2 & 4 & 8 & 16 & 2 & 4 & 8 & 16  \\
 %$N_c \rightarrow$ & 2 & 4 & 8 & 16 & 2 & 4 & 2 \\
 \hline
 \hline
 \multicolumn{5}{|c|}{Race (FP32 Accuracy = 48.8\%)} & \multicolumn{4}{|c|}{Boolq (FP32 Accuracy = 85.23\%)} \\ 
 \hline
 \hline
 64 & 49.00 & 49.00 & 49.28 & 48.71 & 82.82 & 84.28 & 84.03 & 84.25 \\
 \hline
 32 & 49.57 & 48.52 & 48.33 & 49.28 & 83.85 & 84.46 & 84.31 & 84.93  \\
 \hline
 16 & 49.85 & 49.09 & 49.28 & 48.99 & 85.11 & 84.46 & 84.61 & 83.94  \\
 \hline
 \hline
 \multicolumn{5}{|c|}{Winogrande (FP32 Accuracy = 79.95\%)} & \multicolumn{4}{|c|}{Piqa (FP32 Accuracy = 81.56\%)} \\ 
 \hline
 \hline
 64 & 78.77 & 78.45 & 78.37 & 79.16 & 81.45 & 80.69 & 81.45 & 81.5 \\
 \hline
 32 & 78.45 & 79.01 & 78.69 & 80.66 & 81.56 & 80.58 & 81.18 & 81.34  \\
 \hline
 16 & 79.95 & 79.56 & 79.79 & 79.72 & 81.28 & 81.66 & 81.28 & 80.96  \\
 \hline
\end{tabular}
\caption{\label{tab:mmlu_abalation} Accuracy on LM evaluation harness tasks on Llama2-70B model.}
\end{table}

%\section{MSE Studies}
%\textcolor{red}{TODO}


\subsection{Number Formats and Quantization Method}
\label{subsec:numFormats_quantMethod}
\subsubsection{Integer Format}
An $n$-bit signed integer (INT) is typically represented with a 2s-complement format \citep{yao2022zeroquant,xiao2023smoothquant,dai2021vsq}, where the most significant bit denotes the sign.

\subsubsection{Floating Point Format}
An $n$-bit signed floating point (FP) number $x$ comprises of a 1-bit sign ($x_{\mathrm{sign}}$), $B_m$-bit mantissa ($x_{\mathrm{mant}}$) and $B_e$-bit exponent ($x_{\mathrm{exp}}$) such that $B_m+B_e=n-1$. The associated constant exponent bias ($E_{\mathrm{bias}}$) is computed as $(2^{{B_e}-1}-1)$. We denote this format as $E_{B_e}M_{B_m}$.  

\subsubsection{Quantization Scheme}
\label{subsec:quant_method}
A quantization scheme dictates how a given unquantized tensor is converted to its quantized representation. We consider FP formats for the purpose of illustration. Given an unquantized tensor $\bm{X}$ and an FP format $E_{B_e}M_{B_m}$, we first, we compute the quantization scale factor $s_X$ that maps the maximum absolute value of $\bm{X}$ to the maximum quantization level of the $E_{B_e}M_{B_m}$ format as follows:
\begin{align}
\label{eq:sf}
    s_X = \frac{\mathrm{max}(|\bm{X}|)}{\mathrm{max}(E_{B_e}M_{B_m})}
\end{align}
In the above equation, $|\cdot|$ denotes the absolute value function.

Next, we scale $\bm{X}$ by $s_X$ and quantize it to $\hat{\bm{X}}$ by rounding it to the nearest quantization level of $E_{B_e}M_{B_m}$ as:

\begin{align}
\label{eq:tensor_quant}
    \hat{\bm{X}} = \text{round-to-nearest}\left(\frac{\bm{X}}{s_X}, E_{B_e}M_{B_m}\right)
\end{align}

We perform dynamic max-scaled quantization \citep{wu2020integer}, where the scale factor $s$ for activations is dynamically computed during runtime.

\subsection{Vector Scaled Quantization}
\begin{wrapfigure}{r}{0.35\linewidth}
  \centering
  \includegraphics[width=\linewidth]{sections/figures/vsquant.jpg}
  \caption{\small Vectorwise decomposition for per-vector scaled quantization (VSQ \citep{dai2021vsq}).}
  \label{fig:vsquant}
\end{wrapfigure}
During VSQ \citep{dai2021vsq}, the operand tensors are decomposed into 1D vectors in a hardware friendly manner as shown in Figure \ref{fig:vsquant}. Since the decomposed tensors are used as operands in matrix multiplications during inference, it is beneficial to perform this decomposition along the reduction dimension of the multiplication. The vectorwise quantization is performed similar to tensorwise quantization described in Equations \ref{eq:sf} and \ref{eq:tensor_quant}, where a scale factor $s_v$ is required for each vector $\bm{v}$ that maps the maximum absolute value of that vector to the maximum quantization level. While smaller vector lengths can lead to larger accuracy gains, the associated memory and computational overheads due to the per-vector scale factors increases. To alleviate these overheads, VSQ \citep{dai2021vsq} proposed a second level quantization of the per-vector scale factors to unsigned integers, while MX \citep{rouhani2023shared} quantizes them to integer powers of 2 (denoted as $2^{INT}$).

\subsubsection{MX Format}
The MX format proposed in \citep{rouhani2023microscaling} introduces the concept of sub-block shifting. For every two scalar elements of $b$-bits each, there is a shared exponent bit. The value of this exponent bit is determined through an empirical analysis that targets minimizing quantization MSE. We note that the FP format $E_{1}M_{b}$ is strictly better than MX from an accuracy perspective since it allocates a dedicated exponent bit to each scalar as opposed to sharing it across two scalars. Therefore, we conservatively bound the accuracy of a $b+2$-bit signed MX format with that of a $E_{1}M_{b}$ format in our comparisons. For instance, we use E1M2 format as a proxy for MX4.

\begin{figure}
    \centering
    \includegraphics[width=1\linewidth]{sections//figures/BlockFormats.pdf}
    \caption{\small Comparing LO-BCQ to MX format.}
    \label{fig:block_formats}
\end{figure}

Figure \ref{fig:block_formats} compares our $4$-bit LO-BCQ block format to MX \citep{rouhani2023microscaling}. As shown, both LO-BCQ and MX decompose a given operand tensor into block arrays and each block array into blocks. Similar to MX, we find that per-block quantization ($L_b < L_A$) leads to better accuracy due to increased flexibility. While MX achieves this through per-block $1$-bit micro-scales, we associate a dedicated codebook to each block through a per-block codebook selector. Further, MX quantizes the per-block array scale-factor to E8M0 format without per-tensor scaling. In contrast during LO-BCQ, we find that per-tensor scaling combined with quantization of per-block array scale-factor to E4M3 format results in superior inference accuracy across models. 


\end{document}
