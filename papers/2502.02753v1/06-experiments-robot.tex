
%\subsubsection{Real-GC} 

\subsection{Physical Experimental Results}\label{sec:real_task}
Our robotic test bed (Fig.~\ref{fig:sock_puppet}[Right]) comprises a collaborative manipulator equipped with a customized suction gripper(Fig.~\ref{fig:sock_puppet}[Left]), which is capable of vacuum suction and dexterous contact with its soft tip. Two 5 MP 3D cameras are positioned with one above each tote. 

\begin{wrapfigure}{r}{0.25\textwidth}  % 'r' means the figure is on the right
    \centering
    \includegraphics[width=0.25\textwidth]{figures/station_2_overview.pdf}
    \caption{[Left] A customized suction gripper capable of vacuum suction and dexterous contact. [Right] Physical robotic system.}
    \vspace{-1mm}
    \label{fig:sock_puppet}
\end{wrapfigure}

Similarly, the experimental task (Fig.~\ref{fig:station2_sequence}) consists of four skills: flips down an object from the edge of the picking tote, grasps it with the suction cup, packs it at the proper pose, based on a goal image using a generic brown box(Fig.~\ref{fig:real_images}(c), and pushes it to the corners of the tote.
The first two skills have clear success or failure criteria, while packing succeeds if the object is in the correct quarter of the tote, and pushing succeeds if the object is within 2 cm of the correct corner. 

We evaluate \ours in a open-loop control framework, where \ours takes a single state observation, two images of two totes plus an image of the in-hand object (if any), and outputs the trajectory for the selected skill. The robot then executes the   entire trajectory in an open loop and moves out of the observable environment. We use a set of five objects (Fig.~\ref{fig:real_images} for the physical experimental task. For each object in our training set (Fig.~\ref{fig:real_images}(a)), we collect 15, 15, 24, and 24 human demonstrations for the four skills respectively. 

\begin{figure}[t]
    \centering
    \includegraphics[width=0.4\textwidth]{figures/station_2_sequence.pdf}
    \vspace{-2mm}
    \caption{Task sequence of real robot goal-state conditioned pick-n-pack.}
    \label{fig:station2_sequence}
\end{figure}

%During the model inference, \ours is given three images: overview images of the two totes and an image of the in-hand object.
%Similar to the object-independent goal images in simulation, we use goal images of a brown box(Fig.~\ref{fig:real_images} (c)) which is neither in the training set, nor in the test set.

%The goal-state indicator is only related to the packing tote, the goal images of the picking tote and the in-hand object is zero-padded.

%Based on the observations from the three cameras and the goal image, models decide on the skill to execute and the whole trajectory of the skill execution.
%After the execution, the robot moves out of the environment and updates the observation images.




\begin{figure}
    \centering
    \vspace{-3mm}
    \includegraphics[width=0.4\textwidth]{figures/real_objects.pdf}
    \vspace{-3mm}
    \caption{Training (a) and test (b) object set for physical experiments. (c) Image of the packing tote with a universal object as the goal-state indicator.}
    \label{fig:real_images}
\end{figure}

 




% I think this could have been discussed before this section - so skip
%For the four skills, a test case is labeled as success if the object is flipped down, leaves picking tote, is placed at right quarter of the packing tote, and is packed less than 2cm to the corner respectively. A test case is labeled as a failure if the robot fails to proceed to the next phase after three actions or collides with the environment.

We first evaluate both \ours and Octo on the couscous box with different goal-state images (Tab.~\ref{tab:real_robot_diff_corner_results}).
Both models succeed in the first three skills in all the test cases but Octo only successfully pushes the object to corners in $35\%$ test cases.
In Tab.~\ref{tab:real_robot_diff_obj_results}, we show the experiment results on the diverse object set.
Octo has $0\%$ success rate on the small object (``Instax'' box) and the novel object (Bag).
It also fails in pushing other objects to corners in most test cases.
In contrast, \ours finishes $88\%$ test cases among the objects.
In addition, most of the failures of Octo in pushing are due to collisions with the environment, while two failures of \ours on pushing are inaccurate location with the open loop limitation: the final object pose is slightly outside the goal region (2.1 cm and 2.6 cm away from the corner).
In summary, \ours has much higher success rate in finishing the whole task than Octo, especially on small objects and novel objects.

\begin{table}[H]
\centering
\caption{Four Image Conditioned Goals(Couscous Box)}
\label{tab:real_robot_diff_corner_results}
\begin{tabular}{@{\centering\arraybackslash}m{1.7cm}>{\centering\arraybackslash}m{0.8cm}>{\centering\arraybackslash}m{0.8cm}>{\centering\arraybackslash}m{0.8cm}>{\centering\arraybackslash}m{0.8cm}>{\centering\arraybackslash}m{0.8cm}>{\centering\arraybackslash}m{0.8cm}>{\centering\arraybackslash}m{0.8cm}>{\centering\arraybackslash}m{1cm}@{}}
\toprule
 % & \multicolumn{6}{c}{\textbf{Task Completion}} \\ \cmidrule(lr){2-7} 
 & \multicolumn{2}{c}{\textbf{Flip}} & \multicolumn{2}{c}{\textbf{Pick}} & \multicolumn{2}{c}{\textbf{Pack}} & \multicolumn{2}{c}{\textbf{Push}} \\ \cmidrule(lr){2-3} \cmidrule(lr){4-5} \cmidrule(lr){6-7}  \cmidrule(lr){8-9}
 \textbf{Goal State} & \textbf{\ours} & \textbf{Octo} & \textbf{\ours} & \textbf{Octo} & \textbf{\ours} & \textbf{Octo} & \textbf{\ours} & \textbf{Octo} \\ \midrule
Top Left & \multicolumn{1}{|c}{5/5} & 5/5 & \multicolumn{1}{|c}{5/5} & 5/5 & \multicolumn{1}{|c}{5/5} & 5/5 & \multicolumn{1}{|c}{4/5} & 3/5 \\
Top Right & \multicolumn{1}{|c}{5/5} & 5/5 & \multicolumn{1}{|c}{5/5} & 5/5 & \multicolumn{1}{|c}{5/5} & 5/5 & \multicolumn{1}{|c}{2/5} & 0/5 \\
Bottom Left & \multicolumn{1}{|c}{5/5} & 5/5 & \multicolumn{1}{|c}{5/5} & 5/5 & \multicolumn{1}{|c}{5/5} & 5/5 & \multicolumn{1}{|c}{5/5} & 2/5 \\
Bottom Right & \multicolumn{1}{|c}{5/5} & 5/5 & \multicolumn{1}{|c}{5/5} & 5/5 & \multicolumn{1}{|c}{5/5} & 5/5 & \multicolumn{1}{|c}{5/5} & 2/5 \\ 
 \bottomrule
\end{tabular}
\end{table}


\begin{table}[H]
\centering
\caption{Five Objects Image-Conditioned (Bottom Left Corner)}
\label{tab:real_robot_diff_obj_results}
\begin{tabular}{@{\centering\arraybackslash}m{1.7cm}
>{\centering\arraybackslash}m{0.8cm}
>{\centering\arraybackslash}m{0.8cm}
>{\centering\arraybackslash}m{0.8cm}
>{\centering\arraybackslash}m{0.8cm}
>{\centering\arraybackslash}m{0.8cm}
>{\centering\arraybackslash}m{0.8cm}
>{\centering\arraybackslash}m{0.8cm}
>{\centering\arraybackslash}m{0.8cm}@{}}
\toprule
 % & \multicolumn{6}{c}{\textbf{Task Completion}} \\ \cmidrule(lr){2-7} 
 & \multicolumn{2}{c}{\textbf{Flip}} & \multicolumn{2}{c}{\textbf{Pick}} & \multicolumn{2}{c}{\textbf{Pack}} & \multicolumn{2}{c}{\textbf{Push}} \\ \cmidrule(lr){2-3} \cmidrule(lr){4-5} \cmidrule(lr){6-7}  \cmidrule(lr){8-9}
 \textbf{Object} & \textbf{\ours} & \textbf{Octo} & \textbf{\ours} & \textbf{Octo} & \textbf{\ours} & \textbf{Octo} & \textbf{\ours} & \textbf{Octo} \\ \midrule
``Instax'' Box & \multicolumn{1}{|c}{ 4/5} & 0/5 & \multicolumn{1}{|c}{4/5} & 0/5 & \multicolumn{1}{|c}{4/5} & 0/5 & \multicolumn{1}{|c}{2/5} & 0/5 \\
``Mina'' Box & \multicolumn{1}{|c}{5/5} & 4/5 & \multicolumn{1}{|c}{5/5} & 4/5 & \multicolumn{1}{|c}{5/5} & 4/5 & \multicolumn{1}{|c}{5/5} & 3/5 \\
Couscous Box & \multicolumn{1}{|c}{5/5} & 5/5 & \multicolumn{1}{|c}{5/5} & 5/5 & \multicolumn{1}{|c}{5/5} & 5/5 & \multicolumn{1}{|c}{5/5} & 2/5 \\
Rice Box & \multicolumn{1}{|c}{ 5/5} & 5/5 & \multicolumn{1}{|c}{5/5} & 5/5 & \multicolumn{1}{|c}{5/5} & 4/5 & \multicolumn{1}{|c}{5/5} & 1/5 \\ 
Bag (OOD) &\multicolumn{1}{|c}{ 5/5} & 0/5 & \multicolumn{1}{|c}{5/5} & 0/5 & \multicolumn{1}{|c}{5/5} & 0/5 & \multicolumn{1}{|c}{5/5} & 0/5  \\ \bottomrule
\end{tabular}
\vspace{-3mm}
\end{table}

% Structure
% \begin{enumerate}
% \item Long horizon pick-n-pack.
% \begin{enumerate}
%     \item description with figures and qualitative results.
%     \item language-conditioning, long box, 15 demos on each corner, MuST vs Octo single policy vs non-object-centric MuST vs skill addition.
%     \item Skill skipping and repeating (small table or only show in video).
%     \item Universal goal image conditioning
%     \item Generality: 4 objects * 15 demos * 4 corners $\rightarrow$ in/out-of distribution objects.
% \end{enumerate}
% \item Multi-task sequence pick-n-pack.
% \begin{enumerate}
%     \item description with figures
%     \item Objects in the central region: Need to show a balancing sequence choice to avoid ``modality collapse''
%     \item Object on the boundary: Knows to flip first.
% \end{enumerate}
% \item Station $2$ trajectory planning
% \begin{enumerate}
%     \item Description with figures
%     \item Object-independent goal image conditioning: MuST vs Octo single policy.
% \end{enumerate}
% \end{enumerate}