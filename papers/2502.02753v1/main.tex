\documentclass[letterpaper, 10 pt, conference]{format/ieeeconf}  % Comment this line out if you need a4paper

%\documentclass[a4paper, 10pt, conference]{ieeeconf}      % Use this line for a4 paper

\IEEEoverridecommandlockouts                              % This command is only needed if 
                                                          % you want to use the \thanks command

\overrideIEEEmargins                                      % Needed to meet printer requirements.

%In case you encounter the following error:
%Error 1010 The PDF file may be corrupt (unable to open PDF file) OR
%Error 1000 An error occurred while parsing a contents stream. Unable to analyze the PDF file.
%This is a known problem with pdfLaTeX conversion filter. The file cannot be opened with acrobat reader
%Please use one of the alternatives below to circumvent this error by uncommenting one or the other
%\pdfobjcompresslevel=0
%\pdfminorversion=4

% See the \addtolength command later in the file to balance the column lengths
% on the last page of the document
\usepackage{dsfont}

\usepackage{amsmath}
\usepackage{comment}
\usepackage{nohyperref}
\definecolor{matplotlib0}{HTML}{1f77b4}
\definecolor{matplotlib1}{HTML}{d62728}
\definecolor{matplotlib2}{HTML}{2ca02c}
\definecolor{matplotlib3}{HTML}{ff7f0e}
\definecolor{matplotlib4}{HTML}{9467bd}
\definecolor{matplotlib5}{HTML}{8c564b}
\definecolor{matplotlib6}{HTML}{e377c2}
\definecolor{matplotlib7}{HTML}{7f7f7f}
\definecolor{matplotlib8}{HTML}{bcbd22}
\definecolor{matplotlib9}{HTML}{17becf}

\usepackage{mathtools}

\DeclarePairedDelimiter\abs{\lvert}{\rvert}%
\DeclarePairedDelimiter\norm{\lVert}{\rVert}%

\usepackage{booktabs}
\usepackage{array}
\usepackage{multirow}
\usepackage{colortbl}
\usepackage{tablefootnote}
\usepackage{threeparttable}
\usepackage[acronym, style=super, nonumberlist]{glossaries}
\renewcommand*{\glsgroupskip}{}

\usepackage{pgfplots}
\definecolor{color0}{rgb}{0.12156862745098,0.466666666666667,0.705882352941177} % blue
\definecolor{color1}{rgb}{1,0.498039215686275,0.0549019607843137}
\definecolor{color2}{rgb}{0.172549019607843,0.627450980392157,0.172549019607843} % green
\definecolor{color3}{rgb}{0.83921568627451,0.152941176470588,0.156862745098039} % red
\definecolor{color4}{rgb}{0.580392156862745,0.403921568627451,0.741176470588235}
\definecolor{colorblue}{rgb}{0.12156862745098,0.466666666666667,0.705882352941177} % blue
\definecolor{colorgreen}{rgb}{0.172549019607843,0.627450980392157,0.172549019607843} % green
\definecolor{colorred}{rgb}{0.83921568627451,0.152941176470588,0.156862745098039} % red
\definecolor{colorblack}{rgb}{0,0,0} % black
\definecolor{colororange}{rgb}{1,0.56,0} % orange
\usepgfplotslibrary{fillbetween}
\usepgfplotslibrary{colormaps}
\pgfplotsset{compat=1.16}

\pgfplotscreateplotcyclelist{matplotlib}{
  {matplotlib0},
  {matplotlib1},
  {matplotlib2},
  {matplotlib3},
  {matplotlib4},
  {matplotlib5},
  {matplotlib6},
  {matplotlib7},
  {matplotlib8},
  {matplotlib9}
}

\pgfplotsset{every axis/.append style={
    cycle list name=matplotlib
}}

\usepackage{listings}

\definecolor{code_default}{HTML}{000000}
\definecolor{code_keyword}{HTML}{AC4142}
\definecolor{code_identifier}{HTML}{D28445}

%%% Local Variables:
%%% mode: latex
%%% TeX-master: "report"
%%% End: % put the rest of the preamble here
\usepackage{soul} % only for highlight purposes
\usepackage[a-2b,mathxmp]{pdfx}[2018/12/22]
\newcommand{\TODO}[1]{\textcolor{red}{#1}}
\def\ours{MuST\xspace}

\def\progss{ProGSS\xspace}

%%%%%%%%%%%%%%%%%%%%%%%%%%%%%%%%%%%%%%%%%%%%%%%%%%%%%%%%%%%%%%%%
% Spacing related commands.
%%%%%%%%%%%%%%%%%%%%%%%%%%%%%%%%%%%%%%%%%%%%%%%%%%%%%%%%%%%%%%%%
% Space between figure and caption
\setlength{\abovecaptionskip}{2pt}
\setlength{\belowcaptionskip}{2pt}

% Space between text and figs
\setlength{\dbltextfloatsep}{1.5pt plus .5pt minus .5pt}
\setlength{\textfloatsep}{.15pt plus .5pt minus .5pt}
\setlength{\intextsep}{1.5pt plus .5pt minus .5pt}

% Space between equations and text
\setlength{\belowdisplayskip}{1pt} \setlength{\belowdisplayshortskip}{1pt}
\setlength{\abovedisplayskip}{1pt} 
\setlength{\abovedisplayshortskip}{1pt}

% Paragraph formatting 
\setlength{\parskip}{1.5pt}


\newif\ifarxiv
\arxivfalse

\font\titlefont=ptmb at 15.95pt
\title{\titlefont
MuST: Multi-Head Skill Transformer for Long-Horizon Dexterous Manipulation with Skill Progress}
\author{Kai Gao$^{1,2}$\quad Fan Wang$^{1}$\quad Erica Aduh$^{1}$\quad Dylan Randle$^{1}$\quad Jane Shi$^{1}$
\thanks{$^{1}$Amazon Robotics, MA, USA. Email: {\tt\small { \{kaigaoar, fanwanf, aduheric, dylanran, janeshi\}}@amazon.com}.
}
\thanks{$^{2}$Department of Computer Science, Rutgers University, NJ, USA. Email: {\tt\small { \{kg627\}}@cs.rutgers.edu}. Work done during the internship at Amazon Robotics.
}
}
\begin{document}

\maketitle

% \section{Content List}
% \input{00-contents}
% \clearpage

\begin{abstract}
Robot picking and packing tasks require dexterous manipulation skills, such as rearranging objects to establish a good grasping pose, or placing and pushing items to achieve tight packing. These tasks are challenging for robots due to the complexity and variability of the required actions. To tackle the difficulty of learning and executing long-horizon tasks, we propose a novel framework called the Multi-Head Skill Transformer (MuST). This model is designed to learn and sequentially chain together multiple motion primitives (skills), enabling robots to perform complex sequences of actions effectively. MuST introduces a ``progress value'' for each skill, guiding the robot on which skill to execute next and ensuring smooth transitions between skills. Additionally, our model is capable of expanding its skill set and managing various sequences of sub-tasks efficiently. Extensive experiments in both simulated and real-world environments demonstrate that MuST significantly enhances the robot's ability to perform long-horizon dexterous manipulation tasks.
\end{abstract}

\section{Introduction}\label{sec:intro}
As robotics technology advances, the deployment of robots in everyday tasks is rapidly transitioning from a conceptual idea to a practical reality. Unlike traditional industrial robots, which are typically designed for repetitive, single-purpose tasks, the next generation of robots is expected to perform complex, dexterous manipulation tasks that require the seamless integration of multiple skills and the execution of diverse sub-tasks.

Recently, advances in policy learning~\cite{chi2023diffusion, zhao2023learning, team2024octo} have shown great promise by effectively learning from human demonstrations to address dexterous manipulation tasks that were notoriously difficult to design and program manually. These methods leverage the richness and variety of human demonstrations, capable of capturing multi-modal data, and many also utilize foundation models to improve generalization and learning efficiency~\cite{open_x_embodiment_rt_x_2023, brohan2023rt1roboticstransformerrealworld, brohan2023rt2visionlanguageactionmodelstransfer, ze20243ddiffusionpolicygeneralizable, yang2024equibotsim3equivariantdiffusionpolicy}. However, despite these advances, most systems are designed and tested to handle specific, single tasks and often fall short when faced with long-horizon tasks that require combining and sequencing multiple skills over extended periods~\cite{mandlekar2021learninggeneralizelonghorizontasks, duan2017oneshotimitationlearning}.

\begin{figure}[t]
    \centering
    \includegraphics[width=0.5\textwidth]{figures/prob.pdf}
   \includegraphics[width=0.5\textwidth]{figures/Intro.pdf}
    \caption{[Top] An example of long-horizon dexterous manipulation. The robot executes four skills to manipulate an object from the boundary of the picking tote to the corner of the packing tote. [Bottom] Our proposed imitation learning model \ours with N-skill and skill selector \progss.}
    \label{fig:intro}
\end{figure}

A key hypothesis we have to improve reliability of policy learning for long-horizon task is that many such tasks can be decomposed into multiple heterogeneous sub-tasks, where each sub-task requires a distinct skill, and these skills are highly reusable. This is particularly evident in warehouse robotics, where tasks such as picking and packing involve dinstinct skills like flipping, grasping, and pushing. For example, Fig.~\ref{fig:intro}[Top] shows a robot flipping an object from the boundary of a picking tote and compactly packing it in the corner of a packing tote. %In this paper, we refer to these skill modules interchangeably as sub-tasks or skills.

Decades of research have explored the problem of chaining multiple skills to achieve long-horizon tasks. Task and Motion Planning (TAMP) methods integrate high-level symbolic reasoning with low-level motion planning, enabling robots to sequence skills while ensuring physical feasibility~\cite{Dantam16, Srivastava14, Wolfe10}. Similarly, reinforcement learning (RL), particularly hierarchical RL, decomposes tasks into sub-skills, facilitating more efficient exploration in complex environments~\cite{kulkarni2016hierarchicaldeepreinforcementlearning, eysenbach2018diversityneedlearningskills, vezhnevets2017feudalnetworkshierarchicalreinforcement, bacon2016optioncriticarchitecture}. However, these methods still face challenges. For instance, RL methods struggle with exploration and scalability, as the search space for multi-skill tasks grows exponentially with complexity~\cite{nasiriany2022augmenting}. %Additionally, ensuring smooth transitions between skills is difficult, requiring precise coordination of states.

Instead, we propose a novel approach, \ours (Multi-Head Skill Transformer), designed to enhance the reliability of policy learning by primarily building upon the policy's existing structure, while introducing minimal additional complexity. \ours operates by decomposing long-horizon tasks into sequence of reusable skills. At each timestamp, the appropriate skill is selected based on the current observation and state, enabled by a robust progress estimator for each skill and a skill selector. 

Specifically, \ours extends the policy learning model Octo~\cite{team2024octo} by introducing multiple heads, with each head responsible for a specific skill, along with a progress estimator that tracks the progress of skill execution. A skill selector function, named \progss, maps the progress across all skill heads to determine the appropriate skill to execute at any given state. Both the skill heads and progress estimators are trained simultaneously using the same pre-trained Octo transformer backbone. With the multi-head structure, \ours allows for the training of multiple skills either synchronously or asynchronously, facilitating the integration of a large skill set and the addition of new skills as needed.

The main advantage of \ours is that it provides a clear understanding of each individual skill and the overall task progress through continuous progress estimation. It also offers flexibility in determining when to terminate a skill by setting a termination threshold. Additionally, \ours enhances reliability under disturbance, as the model continuously reasons when to skip or redo certain skills to ensure task completion.

Comprehensive experiments in both simulated and real-world environments demonstrate that \ours effectively addresses challenging long-horizon dexterous manipulation tasks, showing significant improvements over the Octo baseline models. In tasks involving flipping, picking, packing, and pushing, \ours increased the overall task completion rate from 32.5\% with the baseline Octo single policy to 90\% with \ours.


%\section{Introduction}\label{sec:intro}
%\section{Introduction}

Tutoring has long been recognized as one of the most effective methods for enhancing human learning outcomes and addressing educational disparities~\citep{hill2005effects}. 
By providing personalized guidance to students, intelligent tutoring systems (ITS) have proven to be nearly as effective as human tutors in fostering deep understanding and skill acquisition, with research showing comparable learning gains~\citep{vanlehn2011relative,rus2013recent}.
More recently, the advancement of large language models (LLMs) has offered unprecedented opportunities to replicate these benefits in tutoring agents~\citep{dan2023educhat,jin2024teach,chen2024empowering}, unlocking the enormous potential to solve knowledge-intensive tasks such as answering complex questions or clarifying concepts.


\begin{figure}[t!]
\centering
\includegraphics[width=1.0\linewidth]{Figs/Fig.intro.pdf}
\caption{An illustration of coding tutoring, where a tutor aims to proactively guide students toward completing a target coding task while adapting to students' varying levels of background knowledge. \vspace{-5pt}}
\label{fig:example}
\end{figure}

\begin{figure}[t!]
\centering
\includegraphics[width=1.0\linewidth]{Figs/Fig.scaling.pdf}
\caption{\textsc{Traver} with the trained verifier shows inference-time scaling for coding tutoring (detailed in \S\ref{sec:scaling_analysis}). \textbf{Left}: Performance vs. sampled candidate utterances per turn. \textbf{Right}: Performance vs. total tokens consumed per tutoring session. \vspace{-15pt}}
\label{fig:scale}
\end{figure}


Previous research has extensively explored tutoring in educational fields, including language learning~\cite{swartz2012intelligent,stasaski-etal-2020-cima}, math reasoning~\cite{demszky-hill-2023-ncte,macina-etal-2023-mathdial}, and scientific concept education~\cite{yuan-etal-2024-boosting,yang2024leveraging}. 
Most aim to enhance students' understanding of target knowledge by employing pedagogical strategies such as recommending exercises~\cite{deng2023towards} or selecting teaching examples~\cite{ross-andreas-2024-toward}. 
However, these approaches fall short in broader situations requiring both understanding and practical application of specific pieces of knowledge to solve real-world, goal-driven problems. 
Such scenarios demand tutors to proactively guide people toward completing targeted tasks (e.g., coding).
Furthermore, the tutoring outcomes are challenging to assess since targeted tasks can often be completed by open-ended solutions.



To bridge this gap, we introduce \textbf{coding tutoring}, a promising yet underexplored task for LLM agents.
As illustrated in Figure~\ref{fig:example}, the tutor is provided with a target coding task and task-specific knowledge (e.g., cross-file dependencies and reference solutions), while the student is given only the coding task. The tutor does not know the student's prior knowledge about the task.
Coding tutoring requires the tutor to proactively guide the student toward completing the target task through dialogue.
This is inherently a goal-oriented process where tutors guide students using task-specific knowledge to achieve predefined objectives. 
Effective tutoring requires personalization, as tutors must adapt their guidance and communication style to students with varying levels of prior knowledge. 


Developing effective tutoring agents is challenging because off-the-shelf LLMs lack grounding to task-specific knowledge and interaction context.
Specifically, tutoring requires \textit{epistemic grounding}~\citep{tsai2016concept}, where domain expertise and assessment can vary significantly, and \textit{communicative grounding}~\citep{chai2018language}, necessary for proactively adapting communications to students' current knowledge.
To address these challenges, we propose the \textbf{Tra}ce-and-\textbf{Ver}ify (\textbf{\model}) agent workflow for building effective LLM-powered coding tutors. 
Leveraging knowledge tracing (KT)~\citep{corbett1994knowledge,scarlatos2024exploring}, \model explicitly estimates a student's knowledge state at each turn, which drives the tutor agents to adapt their language to fill the gaps in task-specific knowledge during utterance generation. 
Drawing inspiration from value-guided search mechanisms~\citep{lightman2023let,wang2024math,zhang2024rest}, \model incorporates a turn-by-turn reward model as a verifier to rank candidate utterances. 
By sampling more candidate tutor utterances during inference (see Figure~\ref{fig:scale}), \model ensures the selection of optimal utterances that prioritize goal-driven guidance and advance the tutoring progression effectively. 
Furthermore, we present \textbf{Di}alogue for \textbf{C}oding \textbf{T}utoring (\textbf{\eval}), an automatic protocol designed to assess the performance of tutoring agents. 
\eval employs code generation tests and simulated students with varying levels of programming expertise for evaluation. While human evaluation remains the gold standard for assessing tutoring agents, its reliance on time-intensive and costly processes often hinders rapid iteration during development. 
By leveraging simulated students, \eval serves as an efficient and scalable proxy, enabling reproducible assessments and accelerated agent improvement prior to final human validation. 



Through extensive experiments, we show that agents developed by \model consistently demonstrate higher success rates in guiding students to complete target coding tasks compared to baseline methods. We present detailed ablation studies, human evaluations, and an inference time scaling analysis, highlighting the transferability and scalability of our tutoring agent workflow.




\section{Related Works}\label{sec:related}
\section{Background and related work}
% 重点看Artistic data visualization: Beyond visual analytics 和Visualization criticism-the missing link between information visualization and art 的被引


This section reviews the background on artistic data visualization and previous research related to this topic.

\subsection{Artistic Data Visualization in Art History Context}
\label{ssec:contemporary}

Art history has been marked by transformative periods characterized by different aesthetic pursuits, materials, and methods. Since the time of Plato, imitation (or \textit{mimesis}, which views art as a mirror to the world around us) has been an important pursuit~\cite{pooke2021art}. Successful artworks, such as Greek sculptures and the convincing illusions of depth and space in Renaissance paintings, exemplify this tradition.
The advent of modern society and new technology, especially photography, posed a significant challenge to the notion of art as imitation~\cite{perry2004themes}. By the 1850s, modern art began to emerge with the core goal of transcending traditional forms and conventions. Movements like Post Impressionism, Expressionism, and Cubism revolutionized art through innovative uses of form (\eg color, texture, composition), moving art towards abstraction and experimentation. 
After World War II, the Cold War and the surge of various social problems heightened skepticism about the progress narrative of modernity and the superiority of the capitalist system, leading to the rise of postmodernism and the birth of contemporary art~\cite{hopkins2000after,harrison1992art}. One prominent feature of contemporary art is the absence of fixed standards or genres historically defined by the church or the academy. Postmodern design neither defines a cohesive set of aesthetic values nor restricts the range of media used~\cite{pooke2021art}, sparking novel practices such as installations, performances, lens-based media, videos, and land-based art~\cite{hopkins2000after}.
Meanwhile, artists have increasingly drawn energy from various philosophical and critical theories such as gender studies, psychoanalysis, Marxism, and post-structuralism~\cite{pooke2021art}. As a result, as described by Foster~\cite{foster1999recodings}, artists have increasingly become ``manipulators of signs and symbols... and the viewer an active reader of messages rather than a passive contemplator of the aesthetic''. Hopkins~\cite{hopkins2000after} described this shift as the ``death of the object'' and ``the move to conceptualism''. 
% Joseph Kosuth, one of the most important representatives of conceptual artists, also once said that “all art (after Duchamp) is conceptual (in nature) because art only exists conceptually”
% As argued by Danto~\cite{danto2015after}, traditional notions of aesthetics can no longer apply to contemporary art. ``“All there is at the end,” Danto wrote, “is theory, art having finally become vaporized in a dazzle of pure thought about itself, and remaining, as it were, solely as the object of its own theoretical consciousness.''
% The Anti-aesthetic (1983) has described these as ‘anti-aesthetic’ strategies – typified, as we have seen, by a conceptually driven approach to the art object and to the process of its production.

Emerging within the contemporary art historical context, data art has been significantly propelled by the advent of big data. An early milestone was Kynaston McShine's 1970 exhibition \textit{Information} at the Museum of Modern Art (MoMA). 
% In the exhibition catalogue, McShine wrote~\cite{information_moma}: ``Increasingly artists use mail, telegrams, telex machines, etc., for transmission of works themselves—photographs, films, documents—or of information about their activity.'' 
% The millennium era has witnessed substantial growth in this area.
In 2008, Google’s Data Arts Team was founded to explore the advancement of what creativity and technology can do together~\cite{google}.
% data artist Aaron Koblin
In 2012, Viégas and Wattenberg's \textit{Wind Map}, an artwork that visualizes real-time air movement, became the first web-based artwork to be included in MoMA's permanent collection~\cite{wind}.
Since 2013, the academic conference IEEE VIS has included an Arts Program (IEEE VISAP), showcasing artistic data visualizations through accepted papers and curated exhibitions. 
As noted by Barabási~\cite{dataism} (a Fellow of the American Physical Society and the head of a data art lab), data has become a vital medium for artists to deal with the complexities of our society: ``Humanity is facing a complexity explosion. We are confronted with too much data for any of us to make sense of...The traditional tools and mediums of art, be they canvas or chisel, are woefully inadequate for this task...today’s and tomorrow’s artists can embrace new tools and mediums that scale to the challenge, ensuring that their practice can continue to reflect our changing epistemology.''
% a physicist and head of a data art lab, has noted, 

% Artists are now exploring new mediums and methods, incorporating datasets, computer technology, and algorithms into their work.



\subsection{Research on Artistic Data Visualization}
\label{ssec:artisticvis}

Artistic data visualization is also referred to as artistic visualization, data art, or information art~\cite{holmquist2003informative,rodgers2011exploring,few,viegas2007artistic}. Despite the varying terminologies, there is a basic consensus that artistic data visualization must be art practice grounded in real data~\cite{viegas2007artistic}.
Since the early 2000s, a series of papers introduced innovative artistic systems for visualizing everyday data, such as museum visit routes and bus schedule information~\cite{skog2003between,holmquist2003informative,viegas2004artifacts}.
In 2007, Viégas and Wattenberg~\cite{viegas2007artistic} explicitly proposed the concept of \textit{artistic data visualization} and viewed it as a promising domain beyond visual analytics.
% and defined it as ``visualization of data done by artists with the intent of making art''. 
Kosara~\cite{kosara2007visualization} drew a spectrum of visualization design, positioning artistic visualization and pragmatic visualization at opposite ends of this spectrum to demonstrate that the goals of these two types of design often diverge. 
% advocating that analytical visualizations prioritize practicality, while artistic data visualizations emphasize sublime quality, that is, the capacity to inspire awe and grandeur and elicit profound emotional or intellectual responses. 
% In 2011, Rodgers and Bartram~\cite{rodgers2011exploring} utilized artistic data visualization to enhance residential energy use feedback. 
However, overall, research on this subject has been sparse. Among those relevant papers, most have focused on specific applications of artistic data visualization. 
%~\cite{rodgers2011exploring,schroeder2015visualization,perovich2020chemicals}
For instance, Rodgers and Bartram~\cite{rodgers2011exploring} utilized ambient artistic data visualization to enhance residential energy use feedback. Samsel~\etal~\cite{samsel2018art} transferred artistic styles from paintings into scientific visualization.
Artistic practice has also been observed in fields such as data physicalization~\cite{hornecker2023design,perovich2020chemicals,offenhuber2019data} and sonification~\cite{enge2024open}. For example, Hornecker~\etal~\cite{hornecker2023design} found that many artists are practicing transforming data into tangible artifacts or installations. Enge~\etal~\cite{enge2024open} discussed a set of representative artistic cases that combine sonification and visualization.
% dragicevic2020data
% Offenhuber~\cite{offenhuber2019data} created a set of artworks in urban settings that transform air quality data into situated displays, allowing people to encounter visualizations in their daily lives.

% But in contrast, empirical studies that describe the characteristics of artistic visualization and how they are created are extremely scarce. This scarcity forms a stark contrast to the increasingly rich and diverse practices by artists in the field.
% As for the difference between artistic data visualization and traditional visualizations for analytics, Vi{\'e}gas and Wattenberg~\cite{viegas2007artistic} thought that the most salient feature of artistic data visualizations is their forceful expression of viewpoints.
% In Ramirez~\cite{ramirez2008information}'s opinion, functional information visualizations are concerned with usability and performance while aesthetic information visualizations are concerned with visually attractive forms of representation design.
% Donath~\etal~\cite{donath2010data} presented a series of tools developed to integrate artistic expressions in generating unique and customized visualizations based on users' personal data, such as health monitoring data, album records, and e-mail records. 

On the other hand, some studies, while not focusing on artistic data visualization, have explored a key art-related concept: aesthetics. 
% ~\cite{moere2012evaluating,cawthon2007effect,lau2007towards} explored the aesthetics of visualization design in their research. They
For example, Moere~\etal~\cite{moere2012evaluating} compared analytical, magazine, and artistic visualization styles, noting that analytical styles enhance the discovery of analytical insights, while the other two induce more meaning-based insights. Cawthon~\etal~\cite{cawthon2007effect} asked participants to rank eleven visualization types on an aesthetic scale from ``ugly'' to ``beautiful'', finding that some visualizations (\eg sunburst) were perceived as more beautiful than others (\eg beam trees).
% Moere~\etal~\cite{moere2012evaluating} displayed data in three different styles (analytical style, magazine style, artistic style) and found that these styles led to different perceptions of usability and types of insights.
% More importantly, the authors found that the sunburst chart ranks the highest in aesthetics and is one of the top-performing visualizations in both efficiency and effectiveness, thus exemplifying the notion that "beautiful is indeed usable".
Factors such as embellishment~\cite{bateman2010useful}, colorfulness~\cite{harrison2015infographic}, and interaction~\cite{stoll2024investigating} have also been found to influence aesthetics. 
% borkin2013makes,tanahashi2012design
However, most of these studies have simplified aesthetics to hedonic features (\eg beauty), without delving into the nuanced connotations of aesthetics.
% most of these studies have simplified aesthetics to concepts like 'beauty,' 'preference,' or 'pleasing,' without exploring the deeper essence of aesthetics as the core of art.

The value of artistic data visualization to the visualization community is still in controversy. For instance, Few~\cite{few} argued for a clearer distinction between data art and data visualization; he highlighted the negative consequences when data art ``masquerades as data visualization'', such as making visualizations difficult to perceive. Willers~\cite{willers2014show} thought the artistic approach is ``unlikely be appreciated if the intention was for rapid decision making.''
% In an interview, American artist and technologist Harris commented that ``data can be made pretty by design, but this is a superficial prettiness, like a boring woman wearing too much makeup.''~\cite{harris2015beauty} 
To address these gaps, more empirical research needs to be conducted to explore how artistic data visualizations are designed, their underlying pursuits, and how they might inspire our community.




% Examining this field not only helps us understand the latest application of data visualization in various domains but also extends our understanding of the aesthetic and humanistic aspects of data visualization.
% there should be more theoretical investigation into artistic data visualization. 

% These three concepts emphasize placing or installing visualizations at physical places that people will encounter in their daily lives. 

% ~\cite{wang2019emotional}


% gap between art and science~\cite{judelman2004aesthetics}
% constructive visualization~\cite{huron2014constructive}
% data feminism~\cite{d2020data}
% critical infovis~\cite{dork2013critical}
% citizen data and participation~\cite{valkanova2015public}

% \x{Lee~\etal~\cite{lee2013sketchstory}, give users artistic freedom to create their own visualizations.}


% Aesthetics refers to the study of beauty, taste, and sensory perception and is deeply intertwined with art.
% a particular taste for or approach to what is pleasing to the senses and especially sight

% why shouldn't all charts be scatter plot~\cite{bertini2020shouldn}
% aesthetic model~\cite{lau2007towards}
% Aesthetics for Communicative Visualization : a Brief Review
% Stacked graphs--geometry \& aesthetics~\cite{byron2008stacked}
% storyline optimization~\cite{tanahashi2012design}
% graphic designers rate the attractiveness of non-standard and pictorial visualizations higher than standard and abstract ones, whereas the opposite is true for laypeople.~\cite{quispel2014would}
% evaluate aesthetics - wordcloud
% An Evaluation of Semantically Grouped Word Cloud Designs, tag cloud, wordle

% On the other hand, empirical studies conducted with designers have shown that functionality is not the only design goal of visualization. For example, Bigelow~\etal~\cite{bigelow2014reflections} found that designers would frequently fine-tune the non-data elements in their designs, and data encoding was even "a later consideration with respect to other visual elements of the infographic".
% Moere~\cite{moere2011role} - design
% Quispel~\etal~\cite{quispel2018aesthetics} found that for information designers, clarity and aesthetics are both important for making a design attractive.

\section{Problem Formulation}\label{sec:prob}
Let the long-horizon task be represented as a sequence of \emph{skills}, denoted by a finite set \( \mathcal{S} = \{s_1, s_2, \dots, s_N\} \), where \( N \) is the total number of skills. The task execution is governed by a policy \( \pi \), which, at each timestep \( t \), takes as input the observation \( o_t \in \mathcal{O} \) and the current robot state \( x_t \in \mathcal{X} \), and outputs the action \( P_t \in \mathcal{A} \), which will be further detailed below for our problem setting.

\subsection{Single Policy Learning}

The observation \( o_t \) is a representation of the environment at time \( t \), which could include sensor readings or environmental context. The robot state \( x_t \) includes the joint configurations, velocities, and other internal variables defining the robot's configuration.

The policy \( \pi \) learns to map:
\[
\pi : (o_t, x_t) \to P_t
\]
where \( P_t \in \mathcal{A} \) is the predicted action. This action consists of two components: the robot’s 6D pose \( p_t = (p_t^x, p_t^y, p_t^z, \theta_t^x, \theta_t^y, \theta_t^z) \in \mathbb{R}^6 \) in Cartesian space and orientation, and a discrete value \( u_t \in \{-1, 0 , 1\} \) representing whether the suction is turned on (\( u_t = 1 \)), turned off (\( u_t = -1 \)), or remains in its current state (\( u_t = 0 \)).

Thus, the action \( P_t \) predicted by the policy can be expressed as:
$P_t = (p_t, u_t)$, 
where \( p_t \) is the 6D pose and \( u_t \) is the suction indicator.

\subsection{\ours: Decomposition of Skills and Progress}

Unlike learning the long-horizon task in one policy, our method, \ours, decomposes the task into skill-specific predictions. For each skill \( s_i \in \mathcal{S} \), we predict not only the robot’s action \( P_t^{(i)} \in \mathcal{A} \), but also a progress value \( \rho_t^{(i)} \in [0, 1] \) that indicates how much of the skill \( s_i \) has been completed at time \( t \). The progress value evolves over time and reaches \( \rho_t^{(i)} = \theta_i \), where \( \theta_i \) is the termination threshold when a skill is considered fully executed.

Thus, for each skill \( s_i \), MuST outputs a tuple:
$(P_t^{(i)}, \rho_t^{(i)})$
at each timestep \( t \), where \( P_t^{(i)} = (p_t^{(i)}, u_t^{(i)}) \) is the predicted action for skill \( s_i \), and \( \rho_t^{(i)} \) is the progress indicator.

\subsection{Skill Selection and Execution (\progss)}

At each timestep \( t \), a skill selector function \( \sigma \) determines the next skill to execute based on the progress values \( \rho_t^{(i)} \) for all skills:
\[
\sigma: \{\rho_t^{(1)}, \rho_t^{(2)}, \dots, \rho_t^{(N)}\} \to s_j
\]
where \( s_j \in \mathcal{S} \) is the selected skill to be executed at time \( t \).

The robot then executes the action \( P_t^{(j)} \) predicted by MuST for the selected skill \( s_j \):
$a_t = P_t^{(j)}$,
where \( a_t \in \mathcal{A} \) represents the robot’s pose and suction status to execute at timestep \( t \).

\subsection{Goal-Conditioned Policy Learning}

In both the single policy learning and MuST approaches, an alternative model variant can be used where the input also encodes a goal condition. This goal condition can be represented as either a goal image \( I_g \in \mathcal{I} \) or a goal language instruction \( l_g \in \mathcal{L} \), where \( \mathcal{I} \) is the set of possible goal images and \( \mathcal{L} \) is the set of possible language instructions.

The policy can then be learned as a mapping that includes the goal condition:
\[
\pi : (o_t, x_t, I_g \text{ or } l_g) \to P_t
\]
where the goal condition, whether in the form of a goal image \( I_g \) or a language instruction \( l_g \), provides additional information to the policy for determining the action \( P_t \).

In this variant, the policy outputs \( P_t = (p_t, u_t) \), as described earlier, but the decision process is now conditioned on the provided goal.




%\section{Preliminaries}\label{sec:prob}
%\subsection{Problem Formulation: Goal-State Conditioned Manipulation}
%\todo{Jane: I think we need to start with Imitation learning as in Hydra paper, given demo dataset of observation and actions }
\todo{Jane will work on this section - will propose to use notations at three levels per slack message}
Consider a robot manipulating objects in a bounded workspace $\mathcal W$.
Denote the state space of $\mathcal{W}$ by $S$, each state $s\in S$ includes the robot configuration and the poses of all movable objects in $\mathcal W$.
We define our action space as $\mathcal A\subseteq SO(3)\times\{-1,0,1\}$, which includes an end-effector pose in $SO(3)$ and a relative signal of the suction cup with $-1/0/1$ for deactivation/idling/activation respectively. 
For the manipulation task, the robot is given an expert demonstration dataset $D:=\{\tau_i\}_{i=1}^{n}$. 
Each $\tau_i \in D$ is a sequence of state-action pairs $\{(s_j, a_j)\}_{j=1}^{|\tau_i|}$.
Given the termination conditions of the manipulation task, we denote the goal states of the manipulation task as $S_G\subset S$. 
Correspondingly, $D^{S_G} \subseteq D$ is a subset of demonstrations, in which the demonstration sequences terminate in $S_G$, i.e., $\tau[-1]\in S_G \times \mathcal A, \ \forall \tau \in D^{S_G}$.

Inside the workspace, there is skill set $\Sigma:=\{\sigma_i\}_{i=1}^{N}$.
Each skill $\sigma_i:=\{a^i_{j}\}\subset \mathcal A$ is an action set representing a manipulation primitive in $\mathcal W$. 
Let $D^{S_G}_i$ be the demonstration dataset of $\sigma_i$ in $D^{S_G}$.
Each $\tau_i\in D^{S_G}_i$ is a segment of a demonstration in $D^{S_G}$ and all the actions of $\tau_i$ are in $\sigma_i$.

We present the current state $s$ and the goal states $S_G$ as observations $\mathcal O(s)$ and $\mathcal I(S_G)$, where $\mathcal O$ and $\mathcal I$ map $s$ and $S_G$ to multi-modal data, such as RGBD images, proprioceptive data, and/or natural language.
The robot learns a set of policies $\Pi :=\{\pi_i\}_{i=1}^{N}$ for the skill set $\Sigma$.
Each policy $\pi_i:\mathcal O \times \mathcal I \rightarrow \mathcal A$ learns $\sigma_i$ with dataset $D^{S_G}_i$ minimizing an supervised loss 
$$\mathcal L:=\mathbb{E}_{(s,a)\sim p_{D^{S_G}_i}} d(a,\pi_i(\mathcal O(s), \mathcal I(S_G)))$$ 
where $d$ is a distance metric in $\mathcal A$.

During roll-outs of $\pi_i$, at each time step $t$, the next state is obtained by $s_{t+1}=\mathcal T(s_t, a_t)$, where $\mathcal{T}$ is the state transition model in $\mathcal W$ and $a_t=\pi_i(\mathcal O(s_t), \mathcal I(S_G))$ is the computed action of $\pi_i$.

To complete the manipulation task, a skill selector $\Phi:\mathcal{O}\times \mathcal{I} \rightarrow \Pi$ chooses a policy to execute until $S_G$ is reached.

\subsection{Action Space with Dilated Relative Suction}
\todo{Jane: Should we include more than Suction?}
In the action space $\mathcal A$, we use relative representation of suction activity, since it works better in pick and pack scenarios, where the suction cup only activates in the picking tote and deactivates in the packing tote. 
In contrast, the absolute representation has mixed signals in both totes, which results in 
occasional deactivation at the picking tote.
The biggest weakness of the relative representation is the sparsity in each episode\cite{team2024octo}. 
To address this issue, we extend the non-zero entries by applying k-neighborhood dilation in demonstration episodes, which sets the k-neighborhood elements non-zero as well.




\section{Methodology}\label{sec:method}



\begin{table*}[ht!]
\centering
\small % Reduce font size
\setlength{\tabcolsep}{4pt} % Reduce horizontal padding if needed
\begin{tabular}{l p{5.5cm} p{5.5cm}}
    \toprule
    \textbf{Dataset} & \textbf{Answerable} & \textbf{Unanswerable} \\
    \midrule
    SQUAD & 
    \textbf{Passage:} The first beer pump known in England is believed to [\dots]. %have been invented by John Lofting (b. Netherlands 1659--d. Great Marlow, Buckinghamshire 1742), an inventor, manufacturer, and merchant of London.
    \newline 
    \textbf{Question:} When was John Lofting born? 
    & 
    \textbf{Passage:} Starting in 2010/2011, Hauptschulen were merged  [\dots]. %with Realschulen and Gesamtschulen to form a new type of comprehensive school  in the German States of Berlin and Hamburg---called Stadtteilschule in Hamburg and Sekundarschule in Berlin (see: Education in Berlin, Education in Hamburg). 
    \newline 
    \textbf{Question:} In what school year were Hauptschulen last combined with Realschulen and Gesamtschulen? \\
    \midrule
    IDK & 
    \textbf{Passage:} Singapore has reported 16 deaths. \newline 
    \textbf{Question:} Where are the deaths? 
    & 
    \textbf{Passage:} Showed the arrest of the prime suspect. \newline 
    \textbf{Question:} Where was the arrest? \\
    \midrule
    BoolQ & 
    \textbf{Passage:} On April 20, 2018, ABC officially renewed \textit{Grey's Anatomy} for a network primetime drama record-tying fifteenth season. \newline 
    \textbf{Question:} Is season 14 the last of \textit{Grey's Anatomy}? 
    & 
    \textbf{Passage:} Discover is the fourth largest credit card brand in the U.S., behind Visa, MasterCard, and American Express, with nearly 44 million cardholders. \newline 
    \textbf{Question:} Are pasilla chiles and poblano chiles the same? \\
    \midrule
    Equation & 
    \textbf{Given equations:} \newline n = 53 \newline v = 90 \newline 
    \textbf{Final equation:} \newline n / v = 
    & 
    \textbf{Given equations:} \newline n = 17 \newline u = 38 \newline 
    \textbf{Final equation:} \newline n * t = \\
    \midrule
    Celebrity & 
    \textbf{Article:} Yesterday, I saw an article about Gerard Butler. They really are a great actor. \newline 
    \textbf{Question:} Do you know what their age is? 
    & 
    \textbf{Article:} Yesterday, I saw an article about Tania Scott. They really are a great actor. \newline 
    \textbf{Question:} Do you know what their age is? \\
    \bottomrule
\end{tabular}
\caption{Answerable and Unanswerable Examples from Different Datasets}
\label{tab:answerability}
\end{table*}




\section{Methodology}
We evaluate SAE probes for answerability detection with a specific focus on  generalization.

\myparagraph{SAE Probes}
We use the "Gemma Scope" SAEs pretrained by \citet{lieberum2024gemma} for the instruction-tuned model Gemma 2 \citep{team2024gemma}, and specifically the largest available ones with a width (number of dimensions) of 131k. \citet{lieberum2024gemma} provide SAEs trained on layers 20 and 31.
%
Note that answerability more generally (i.e., beyond specific types of answerability) is a rather high-level concept, which we assume to be represented in intermediate and later layers.
Unless otherwise specified, we search for features using 2k samples of SQUAD (balanced, leaving 1.8k for testing). We collect the feature activations on the last token position and then use 5-fold cross validation for finding SAE features that are predictive for answerability, thus obtaining 1-sparse SAE probes \citep{gurnee2023finding}.
%Note that the common naming as ``probes'' can be misleading, with these probes there are no parameters to be trained.
We then train final probes\footnote{We use SAE probes and SAE features synonymously.} (i.e., scale and bias) for best performing features, which are used for the out-of-distribution evaluation. See Appendix~\ref{app:prelims} for details. ``Top'' features are selected based on training set performance.




\myparagraph{Baselines: Linear Probes}
We train simple linear residual stream probes on the (in-domain) training dataset we also use for finding the SAE features. To ensure robustness, we employ bootstrap analysis across different training splits. %, revealing significant variability in out-of-distribution generalization [i don't recall if this true anymore]. % VT commented based on comment
Since we also focus on SAE features for the residual stream, this probing represents an upper bound for the SAE probing performance on in-domain data.
Observe that these probes achieve 85-90\% accuracy on the in-domain SQUAD data, and thus provide a strong benchmark for comparison. 

\myparagraph{Datasets}
We focus on context-based question answering in the English language.
%In the main experiments, we train and evaluate on SQUAD \cite{rajpurkar2018know} and additionally
We use established data as well as datasets specifically constructed  for out-of-distribution evaluation; for examples see Table~\ref{tab:answerability}.
%We use two established answerability datasets (BoolQ \cite{} and IDK \cite{}), and create two synthetic specialized datasets (math equations, celebrity names):
\begin{itemize}[leftmargin=*,topsep=0pt,noitemsep]
    \item \textbf{SQUAD} \citep{rajpurkar2018know}: Established dataset, passages plus questions relating to them. %Dataset consisting of a short context passage and a question relating to the context. We follow the training data split and prompting template provided by \citet{slobodkin2023curious}.
    \item \textbf{IDK} \citep{sulem2021we}: Dataset with questions in the style of SQUAD.
    % , containing both answerable and unanswerable examples. We specifically use the non-competitive and unanswerable subsets of the ACE-whQA dataset.
    \item \textbf{BoolQ\_3L} 
    \citep{sulem2022yes}: Context-based yes/no questions. % with answerable and unanswerable subsets.
    \item \textbf{Math Equations}: Synthetic dataset contrasting solvable equations with equations containing unknown variables.
    \item \textbf{Celebrity Recognition}: Context-based queries requiring background knowledge about celebrities; for construction details, see Appendix~\ref{app:datasets}.
    % For construction, we use a public dataset of actors and movies from IMDB\footnote{\url{https://www.kaggle.com/datasets/darinhawley/imdb-films-by-actor-for-10k-actors}}, and generate a list of the 1000 most popular actors after 1990, as measured by the total number of ratings their movies received. We construct an additional dataset of non-celebrity names by randomly generating first and last name combinations using the most common North American names from Wikipedia\footnote{\url{https://en.wikipedia.org/wiki/Lists_of_most_common_surnames_in_North_American_countries} and \url{https://en.wikipedia.org/wiki/List_of_most_popular_given_names?utm_source=chatgpt.com}}. 
\end{itemize}

% For the question-answering datasets (SQUAD, BoolQ, and IDK) we use the following consistent formatting:
%
% \begin{verbatim}
% Given the following passage and question, answer the question:
% Passage: {passage}
% Question: {question}
% \end{verbatim}
%
% For the equation dataset, we construct ...
%
% \begin{verbatim}
% "Given the following equations, determine the result of the final equation.
% Given equations:
% {eq_1}
% {eq_2}
% Final equation:
% {eq_3}"
% \end{verbatim}
%
% For the celebrity dataset, we use a public list of celebrity names from \cite{} and generate additional non-celebrity names using the most common north american first names and last names from wikipedia. 
%
% \begin{verbatim}
% "Yesterday, I saw an article about {name}. They really are a great actor. Do you know what their age is?"
% \end{verbatim}

% \subsection{Generalization Analysis}

% We evaluate generalization through multiple lenses:
% \begin{itemize}
%     \item Cross-dataset performance comparing SQUAD-trained features on out-of-distribution data
%     \item Prompt variation analysis testing robustness to input formatting
%     \item Analysis of using a combination of up to five SAE features 
%     \item Similarity analysis of top-performing SAE features and learned residual stream probes
% %    \item Feature consistency analysis across domains using attribution scores
% %    \item Ablation studies measuring the impact of individual features versus feature combinations
% \end{itemize}
%This approach reveals variance in feature transfer ability---while some SAE features show impressive generalization, others remain strongly domain-specific. Moreover, despite strong in-domain performance, residual stream probes exhibit inconsistent transfer when it comes to generalisation.

% \vt{brief intro sentence, why we choose the feature, compare (only) to regular probes}

% \myparagraph{Hypothesis} 
% % vs probe esp in terms of generalization hold their promise?

% \myparagraph{Datasets} 
% % mainly context-based QA
% % squad, generaliz on ambigqa, boolq, math celeb
% % figure with some prompt example, rest into appendix


% \myparagraph{Methods}  % if we don't have more about method we can rename it to models
% \vt{gemma scope since... , also experimented briefly with llama scope; @lovis anything what we could mention here specifically?}

% \paragraph{Linear probes}

% \lh{
% - Trained on 2000 samples of SQUAD (balanced, leaving 1800 for testing)
% - last token position
% - 5 fold cross validation
% - sweeping over regularization parameters with 26 logarithmically spaced steps between 0.0001 and 1
% - Fitting the final probe with the best regularization parameter on the whole training set
% - Trained probes are then evaluated on OOD datasets
% - Analysis is repeated 10 times with different random training set splits

% }



% \begin{table*}[h]
%     \centering
%     \begin{tabular}{|l|p{6cm}|p{6cm}|}
%         \hline
%         \textbf{Dataset} & \textbf{Answerable} & \textbf{Unanswerable} \\
%         \hline
%         SQUAD & Given the following passage and question, answer the question: \newline \textbf{Passage:} The first beer pump known in England is believed to have been invented by John Lofting (b. Netherlands 1659-d. Great Marlow Buckinghamshire 1742) an inventor, manufacturer and merchant of London.\newline \textbf{Question:} When was John Lofting born? & Given the following passage and question, answer the question:\newline \textbf{Passage:} Starting in 2010/2011, Hauptschulen were merged with Realschulen and Gesamtschulen to form a new type of comprehensive school in the German States of Berlin and Hamburg, called Stadtteilschule in Hamburg and Sekundarschule in Berlin (see: Education in Berlin, Education in Hamburg). \newline \textbf{Question:} In what school year were Hauptschulen last combined with Realschulen and Gesamtschulen?\\
%         \hline
%         IDK & Given the following passage and question, answer the question: \newline \textbf{Passage:} Singapore has reported 16 deaths.\newline \textbf{Question:} Where are the deaths? & Given the following passage and question, answer the question:\newline \textbf{Passage:} Showed the arrest of the prime suspect.\newline \textbf{Question:} Where was the arrest? \\
%         \hline
%         BoolQ & Given the following passage and question, answer the question: \newline \textbf{Passage:} On April 20, 2018, ABC officially renewed Grey's Anatomy for a network primetime drama record-tying fifteenth season.\newline \textbf{Question:} Is season 14 the last of grey's anatomy? & Given the following passage and question, answer the question:\newline \textbf{Passage:} Discover is the fourth largest credit card brand in the U.S., behind Visa, MasterCard and American Express, with nearly 44 million cardholders. \newline \textbf{Question:} Are pasilla chiles and poblano chiles the same?\\
%         \hline
%         Equation & Given the following equations, determine the result of the final equation.\newline \textbf{Given equations}:\newline n = 53\newline v = 90\newline \textbf{Final equation}:\newline n / v = & Given the following equations, determine the result of the final equation.\newline \textbf{Given equations}: \newline n = 17\newline u = 38\newline \textbf{Final equation}:\newline n * t = \\
%         \hline
%         Celebrity & Yesterday, I saw an article about Gerard Butler. They really are a great actor. Do you know what their age is? & Yesterday, I saw an article about Tania Scott. They really are a great actor. Do you know what their age is? \\
%         \hline
%     \end{tabular}
%     \caption{Answerable and Unanswerable Examples from Different Datasets}
%     \label{tab:answerability}
% \end{table*}









\section{Evaluation}\label{sec:experiments}
In this section, we evaluate the performance of \ours in both simulated and real-robot environments, with a total of six experiments, under the various conditions:
\begin{itemize}
    \item Performance gain of MuST in comparison with the single-head Octo model 
    
   \item  Performance of MuST, conditioned on task goals specified by images or language instructions
    
    \item Performance of MuST across a diverse set of objects
    
    \item Ability for MuST to react to unexpected environment disturbance, based on the skill progress values
\end{itemize}

The performance is measured by two  metrics:
\begin{enumerate}
    \item {\bf Skill completion and task completion:} we report the completion of a single skill, and the task is completed only if all skills in the task have been successful executed. Any failed skill will result in failure of all subsequent skills.
    \item {\bf Execution time:} The execution time is defined as spent time for manipulation until the object consistently stays at the goal pose $g$ for 100 time steps. 
\end{enumerate}

We compare \ours with the single-head Octo model, which learn all skills without progress estimation.
Both models finetune Octo-Base (93M params) checkpoint and use L1 action heads as decoding heads for skills and progress. 
For closed-loop control scenarios, both models carry out inference every 50 time steps and compute the action sequence for the next 50 time steps.
We train the models with an Nvidia V100 16GB GPU.


\subsection{Simulation Experimental Results and Comparison with Octo Baseline  }

Our goal-state conditioned Pick-n-Pack manipulation task Fig.~\ref{fig:intro}[Top] consists of four skills:
%In simulation,  we first test \ours on goal-state conditioned pick-n-pack to evaluate the overall performance of \ours and compare it with Octo.
%Shown in Fig.~\ref{fig:intro}[Top], this manipulation task requires four skills:
First, the robot flips down the object from the tote boundary to enable picking. 
Next, it picks the object from the picking tote.
Based on the goal-state indicator $I_g$ or $l_g$, the robot packs the object near the desired corner of the packing tote.
Finally, the robot pushs the object, both  rotating and translating it, to fit it tightly the desired corner.

%In simulation, we collect x human demonstrations of this task is collected by controlling robot states with keyboard.The models are trained with only 15 human demonstrations for each of the four corners as goal states.

The goal is given by either a language prompt or a goal image Fig.~\ref{fig:end_state_indicator} . 
For the goal images, instead of specific images of objects, we use a blue patch to suggest the goal state which makes it independent of object appearance, enhancing the model's generalization capability across different object types.
%The human demonstrations of this task is collected by controlling robot states with keyboard.
%In experiments, we present the task completion rate for each skill. 
%Test cases fail in earlier skills are labeled as failures in subsequent skills as well.
%To further evaluate the execution efficiency, we also present the average execution time of successful test cases.


\begin{figure}
    \centering
    \includegraphics[width=0.5\textwidth]{figures/end_state_indicator.pdf}
    \caption{We use either language prompts or images as goal state indicators to customize packing poses.}
    \label{fig:end_state_indicator}
\end{figure}

Tab.\ref{tab:single_language_results} reports the performance metrics for both  \ours  and the Single Head Octo. The task is language conditioned Pick-n-Pack with the ``long box''(Fig.~\ref{fig:object_set}), and each task is repeated 10 times. While Octo model succeeded in  $32.5\%$ tasks, \ours maintains $80\%-90\%$ success rate.
In addition, \ours is $23.7\%-38.4\%$ faster than Octo in execution time for the finished tasks.
The results suggest that \ours is more robust than Octo baseline in long horizon manipulation tasks. 
%The models are trained with only 15 human demonstrations for each of the four corners as goal states.
%In this task, each camera has good visibility in some skills but is occluded in others, which makes it hard to learn a shared encoder for all skills.


% \begin{table*}[ht]
% \centering
% \caption{Language Conditioned Pick-n-Pack (Long Box)}
% \label{tab:single_language_results}
% \begin{tabular}{@{\centering\arraybackslash}m{1.7cm}>{\centering\arraybackslash}m{1cm}>{\centering\arraybackslash}m{1cm}>{\centering\arraybackslash}m{1cm}>{\centering\arraybackslash}m{1cm}>{\centering\arraybackslash}m{1cm}>{\centering\arraybackslash}m{1cm}>{\centering\arraybackslash}m{1cm}>{\centering\arraybackslash}m{1cm}>
% {\centering\arraybackslash}m{1cm}>{\centering\arraybackslash}m{1cm}>{\centering\arraybackslash}m{1cm}>{\centering\arraybackslash}m{1cm}>{\centering\arraybackslash}m{1cm}@{}}
% \toprule
%  & \multicolumn{10}{c}{\textbf{Task Completion}} \\ \cmidrule(lr){2-11} 
%  & \multicolumn{2}{c}{\textbf{Flip}} & \multicolumn{2}{c}{\textbf{Pick}} & \multicolumn{2}{c}{\textbf{Pack}} & \multicolumn{2}{c}{\textbf{Push (Orientation)}} & \multicolumn{2}{c}{\textbf{Push (Position)}} & \multicolumn{2}{c}{\textbf{Execution Time}}\\ \cmidrule(lr){2-3} \cmidrule(lr){4-5} \cmidrule(lr){6-7} \cmidrule(lr){8-9} \cmidrule(lr){10-11} \cmidrule(lr){12-13}
%  \textbf{End State} & \textbf{\ours} & \textbf{Octo} & \textbf{\ours} & \textbf{Octo} & \textbf{\ours} & \textbf{Octo} & \textbf{\ours} & \textbf{Octo}& \textbf{\ours} & \textbf{Octo}& \textbf{\ours} & \textbf{Octo} \\ \midrule
% % \textbf{Task} & \textbf{Method 1 (Success)} & \textbf{Method 1 (Time)} & \textbf{Method 2 (Success)} & \textbf{Method 2 (Time)} & \textbf{Method 3 (Success)} & \textbf{Method 3 (Time)} & \textbf{Method 4 (Success)} & \textbf{Method 4 (Time)} & \textbf{Method 5 (Success)} & \textbf{Method 5 (Time)} \\ \midrule
% Top Left & \multicolumn{1}{|c}{10/10} & 10/10 & \multicolumn{1}{|c}{10/10} & 8/10 & \multicolumn{1}{|c}{10/10} & 7/10 & \multicolumn{1}{|c}{10/10} & 4/10 & \multicolumn{1}{|c}{10/10} & 4/10 & \multicolumn{1}{|c}{1324.2} & 1734.3 \\
% Top Right & \multicolumn{1}{|c}{10/10} & 10/10 & \multicolumn{1}{|c}{10/10} & 10/10 & \multicolumn{1}{|c}{9/10} & 9/10 & \multicolumn{1}{|c}{8/10} & 7/10 & \multicolumn{1}{|c}{8/10} & 4/10 & \multicolumn{1}{|c}{1530.0} & 2639.7 \\
% Bottom Left & \multicolumn{1}{|c}{10/10} & 10/10 & \multicolumn{1}{|c}{10/10} & 7/10 & \multicolumn{1}{|c}{10/10} & 7/10 & \multicolumn{1}{|c}{9/10} & 5/10 & \multicolumn{1}{|c}{9/10} & 3/10 & \multicolumn{1}{|c}{1538.9} & 2337.3\\
% Bottom Right & \multicolumn{1}{|c}{10/10} & 10/10 & \multicolumn{1}{|c}{10/10} & 7/10 & \multicolumn{1}{|c}{9/10} & 6/10 & \multicolumn{1}{|c}{9/10} & 4/10 & \multicolumn{1}{|c}{9/10} & 2/10 & \multicolumn{1}{|c}{1571.7} & 2550.5 \\ \bottomrule
% \end{tabular}
% \end{table*}

\scriptsize
\begin{table}[ht]
\centering
\caption{Language Conditioned Pick-n-Pack (Long Box)}
\label{tab:single_language_results}
\begin{tabular}{@{\centering\arraybackslash}m{1.4cm}>{\centering\arraybackslash}m{0.8cm}>{\centering\arraybackslash}m{0.8cm}>{\centering\arraybackslash}m{0.8cm}>{\centering\arraybackslash}m{0.8cm}>
{\centering\arraybackslash}m{0.8cm}>{\centering\arraybackslash}m{0.8cm}>{\centering\arraybackslash}m{0.8cm}>{\centering\arraybackslash}m{0.8cm}>{\centering\arraybackslash}m{0.8cm}@{}}
\toprule
 & \multicolumn{6}{c}{\textbf{Task Completion}} \\ \cmidrule(lr){2-7} 
 & \multicolumn{2}{c}{\scriptsize{\textbf{Flip$\rightarrow$Pack}}} & \multicolumn{2}{c}{\textbf{{\scriptsize Push (Orientation)}}} & \multicolumn{2}{c}{{\scriptsize \textbf{Push \newline (Position)}}} & \multicolumn{2}{c}{{\scriptsize \textbf{Execution \newline Time}}}\\ \cmidrule(lr){2-3} \cmidrule(lr){4-5} \cmidrule(lr){6-7} \cmidrule(lr){8-9}
 \textbf{End State} & \textbf{\ours} & \textbf{Octo} & \textbf{\ours} & \textbf{Octo}& \textbf{\ours} & \textbf{Octo}& \textbf{\ours} & \textbf{Octo} \\ \midrule
% \textbf{Task} & \textbf{Method 1 (Success)} & \textbf{Method 1 (Time)} & \textbf{Method 2 (Success)} & \textbf{Method 2 (Time)} & \textbf{Method 3 (Success)} & \textbf{Method 3 (Time)} & \textbf{Method 4 (Success)} & \textbf{Method 4 (Time)} & \textbf{Method 5 (Success)} & \textbf{Method 5 (Time)} \\ \midrule
{\scriptsize Top Left} & \multicolumn{1}{|c}{10/10} & 7/10 & \multicolumn{1}{|c}{10/10} & 4/10 & \multicolumn{1}{|c}{10/10} & 4/10 & \multicolumn{1}{|c}{1324} & 1734 \\
{\scriptsize Top Right} & \multicolumn{1}{|c}{9/10} & 9/10 & \multicolumn{1}{|c}{8/10} & 7/10 & \multicolumn{1}{|c}{8/10} & 4/10 & \multicolumn{1}{|c}{1530} & 2639 \\
{\scriptsize Bottom Left} & \multicolumn{1}{|c}{10/10} & 7/10 & \multicolumn{1}{|c}{9/10} & 5/10 & \multicolumn{1}{|c}{9/10} & 3/10 & \multicolumn{1}{|c}{1538} & 2337\\
{\scriptsize Bottom Right} & \multicolumn{1}{|c}{9/10} & 6/10 & \multicolumn{1}{|c}{9/10} & 4/10 & \multicolumn{1}{|c}{9/10} & 2/10 & \multicolumn{1}{|c}{1571} & 2550 \\ \bottomrule
\end{tabular}
\end{table}
\normalsize


\begin{figure}
    \centering
    \includegraphics[width=0.37\textwidth]{figures/object_set.pdf}
    \caption{Training object set and test object set in simulation. The 3D model used are open-source models sampled from the YCB Object and Model Set~\cite{YCB}, NVIDIA SimReady assets~\cite{nvidia_simready}, open-source models from SketchFab~\cite{sketchfab_open_source}, and the Google Scanned Objects dataset~\cite{2022googlescannedobjectshighquality}.}
    \label{fig:object_set}
\end{figure}

\subsection{Additional Evaluation of \ours and \progss in Simulation}
In this section, we further evaluate the performance of \ours, conditioned with goal images, on a diverse object set, and on progress estimation.

Tab.~\ref{tab:single_image_results} summarizes  the performance of \ours on image-conditioned Pick-n-Pack task on the same Long Box object. \ours maintains $80\%-90\%$ success rate for 40 evaluation trials.



%In Tab.~\ref{tab:four_obj_language_results}, we test \ours on a diverse object set.
Furthermore, we evaluate \ours on a diverse object set. In this experiment, we train \ours on four different boxes(Fig.~\ref{fig:object_set}). The training dataset include 15 demonstrations for each of training objects at each of the four goal poses.
We test \ours on all objects including four training objects and two novel objects, and results are reported in Tab.~\ref{tab:four_obj_language_results} for 20 trials on each object.
For the first three skills, \ours maintains over $90\%$ completion rate on training object set and that drops to $70\%-75\%$ on objects in the new category.
For the last push skill, \ours solves $80\%$ test cases on the cracker box, $65\%-70\%$ on other training objects.
\ours only finishes around $40\%$ pushes on novel objects. 
In the training dataset with cuboid objects, the push demonstrations use box corners for rotation.  Our hypothesis is that the same push behavior does not generalize well to the tested novel objects with irregular shapes.


\begin{table}[H]
\centering
\caption{Image Conditioned Pick-n-Pack (Long Box)}
\label{tab:single_image_results}
\begin{tabular}{@{\centering\arraybackslash}m{1.7cm}>{\centering\arraybackslash}m{1cm}>{\centering\arraybackslash}m{1cm}>{\centering\arraybackslash}m{1cm}>{\centering\arraybackslash}m{1.5cm}>{\centering\arraybackslash}m{1.2cm}>{\centering\arraybackslash}m{1cm}@{}}
\toprule
 & \multicolumn{5}{c}{\textbf{Task Completion}} \\ \cmidrule(lr){2-6} 
\textbf{Packing \newline Corner} & \textbf{Flip} & \textbf{Pick} & \textbf{Pack} & \textbf{Push (Orientation)} & \textbf{Push (Position)} & \textbf{Execution Time}\\ \midrule
% \textbf{Task} & \textbf{Method 1 (Success)} & \textbf{Method 1 (Time)} & \textbf{Method 2 (Success)} & \textbf{Method 2 (Time)} & \textbf{Method 3 (Success)} & \textbf{Method 3 (Time)} & \textbf{Method 4 (Success)} & \textbf{Method 4 (Time)} & \textbf{Method 5 (Success)} & \textbf{Method 5 (Time)} \\ \midrule
Top Left & 10/10 & 10/10 & 10/10 & 9/10 & 8/10 & 1570 \\
Top Right & 9/10 & 9/10 & 9/10 & 8/10 & 8/10 & 1765 \\
Bottom Left & 10/10 & 10/10 & 10/10 & 9/10 & 9/10 & 1225 \\
Bottom Right & 10/10 & 10/10 & 10/10 & 8/10 & 9/10 & 1158 \\ \bottomrule
\end{tabular}
\end{table}


\begin{table}[ht]
\centering
\caption{Language-Conditioned Pick-n-Pack with Diverse Object Set}
\label{tab:four_obj_language_results}
\begin{tabular}{@{\centering\arraybackslash}m{1.7cm}>{\centering\arraybackslash}m{1cm}>{\centering\arraybackslash}m{1cm}>{\centering\arraybackslash}m{1cm}>{\centering\arraybackslash}m{1.5cm}>{\centering\arraybackslash}m{1.2cm}>{\centering\arraybackslash}m{1cm}@{}}
\toprule
 & \multicolumn{5}{c}{\textbf{Task Completion}} \\ \cmidrule(lr){2-6} 
\textbf{Test Object} & \textbf{Flip} & \textbf{Pick} & \textbf{Pack} & \textbf{Push (Orientation)} & \textbf{Push (Position)} & \textbf{Execution Time}\\ \midrule
Cracker Box & 20/20 & 20/20 & 19/20 & 16/20 & 16/20 & 2073 \\
Liquid Box & 20/20 & 20/20 & 19/20 & 14/20 & 13/20 & 1769 \\
Long Box & 20/20 & 20/20 & 18/20 & 14/20 & 13/20 & 1330 \\
Oil Tin & 20/20 & 20/20 & 19/20 & 14/20 & 13/20 & 1875 \\ \midrule
Bottle1 (OOD) & 20/20 & 20/20 & 15/20 & 12/20 & 9/20 & 1608\\ 
Bottle2 (OOD) & 20/20 & 20/20 & 14/20 & 9/20 & 7/20 & 2009\\ \bottomrule
\end{tabular}
\end{table}

Additionally, we demonstrate that \ours can react to unexpected environment disturbance based on the skill progress values. As an example, when user resets an object on the tote edge, \ours would select the Flip skill repeatedly. Similarly, \ours would skip the first skill, and start with the second Pick skill when the object is in the pick state. We include these demonstrations with the skill progress value graphs in the accompanying video.

\subsection{Handling Multiple Sequences in Simulation}\label{sec:sim_task2}

We designed the task of multi-sequence pick-n-pack to evaluate the performance of \ours when multiple skill sequences are demonstrated.
As shown in Fig.~\ref{fig:multi_sequence}, the robot is tasked to flip down a box and place it at the packing tote. 
Specifically, when the object is located in the central area of the tote, the robot can choose to flip the object before or after pick-n-place.
However, when the object is located at the boundary of the picking tote, the robot cannot execute pick-n-place before a successful flip.
For each of the three cases of skill sequences in Fig.~\ref{fig:multi_sequence}, we made 50 demonstrations. 
% We test \ours on 80 trials for each of the two initial state categories. 

\begin{figure}[ht]
    \centering
    \includegraphics[width=0.35\textwidth]{figures/multi_sequence.pdf}
    \vspace{-3mm}
    \caption{Multi-sequence pick-n-pack. When the object is sampled at the central area of the tote, there are two possible skill orderings; when the object is sampled at the edge, the robot cannot directly pick it up before flipping.}
    \label{fig:multi_sequence}
\end{figure}

Tab.~\ref{tab:multi_sequence_results} shows sequence selection distribution and task completion rate.
When the object is sampled at the central area of the picking tote, out of the 80 trials, \ours chooses ``pick first'' and ``flip first'' sequences in $37.5\%$ and $62.5\%$ trials respectively with around $80\%$ success rate on both skill sequences.
When the object is sampled at the edge of the picking tote, picking directly is impossible. 
\ours chooses the ``flip first'' sequence in $96.2\%$ test cases.
The results suggest that \ours effectively handles multiple skill sequences and avoids ``modality collapses'' in long horizon manipulation tasks.



\begin{table}[H]
\centering
\caption{Multi-Sequence Pick-n-Pack}
\label{tab:multi_sequence_results}
\begin{tabular}{@{\centering\arraybackslash}m{4cm}>{\centering\arraybackslash}m{2cm}>{\centering\arraybackslash}m{2cm}@{}}
\toprule
 % & \multicolumn{2}{c}{\textbf{Task Completion}} \\ \cmidrule(lr){2-3} 
\textbf{Test Cases} & \textbf{Pick First} & \textbf{Flip First}\\ \midrule
Manipulate from central area & 24/30 & 39/50\\
Manipulate from edge & 0/3 & 73/77\\ \bottomrule
\end{tabular}
\end{table}






%\subsubsection{Real-GC} 

\subsection{Physical Experimental Results}\label{sec:real_task}
Our robotic test bed (Fig.~\ref{fig:sock_puppet}[Right]) comprises a collaborative manipulator equipped with a customized suction gripper(Fig.~\ref{fig:sock_puppet}[Left]), which is capable of vacuum suction and dexterous contact with its soft tip. Two 5 MP 3D cameras are positioned with one above each tote. 

\begin{wrapfigure}{r}{0.25\textwidth}  % 'r' means the figure is on the right
    \centering
    \includegraphics[width=0.25\textwidth]{figures/station_2_overview.pdf}
    \caption{[Left] A customized suction gripper capable of vacuum suction and dexterous contact. [Right] Physical robotic system.}
    \vspace{-1mm}
    \label{fig:sock_puppet}
\end{wrapfigure}

Similarly, the experimental task (Fig.~\ref{fig:station2_sequence}) consists of four skills: flips down an object from the edge of the picking tote, grasps it with the suction cup, packs it at the proper pose, based on a goal image using a generic brown box(Fig.~\ref{fig:real_images}(c), and pushes it to the corners of the tote.
The first two skills have clear success or failure criteria, while packing succeeds if the object is in the correct quarter of the tote, and pushing succeeds if the object is within 2 cm of the correct corner. 

We evaluate \ours in a open-loop control framework, where \ours takes a single state observation, two images of two totes plus an image of the in-hand object (if any), and outputs the trajectory for the selected skill. The robot then executes the   entire trajectory in an open loop and moves out of the observable environment. We use a set of five objects (Fig.~\ref{fig:real_images} for the physical experimental task. For each object in our training set (Fig.~\ref{fig:real_images}(a)), we collect 15, 15, 24, and 24 human demonstrations for the four skills respectively. 

\begin{figure}[t]
    \centering
    \includegraphics[width=0.4\textwidth]{figures/station_2_sequence.pdf}
    \vspace{-2mm}
    \caption{Task sequence of real robot goal-state conditioned pick-n-pack.}
    \label{fig:station2_sequence}
\end{figure}

%During the model inference, \ours is given three images: overview images of the two totes and an image of the in-hand object.
%Similar to the object-independent goal images in simulation, we use goal images of a brown box(Fig.~\ref{fig:real_images} (c)) which is neither in the training set, nor in the test set.

%The goal-state indicator is only related to the packing tote, the goal images of the picking tote and the in-hand object is zero-padded.

%Based on the observations from the three cameras and the goal image, models decide on the skill to execute and the whole trajectory of the skill execution.
%After the execution, the robot moves out of the environment and updates the observation images.




\begin{figure}
    \centering
    \vspace{-3mm}
    \includegraphics[width=0.4\textwidth]{figures/real_objects.pdf}
    \vspace{-3mm}
    \caption{Training (a) and test (b) object set for physical experiments. (c) Image of the packing tote with a universal object as the goal-state indicator.}
    \label{fig:real_images}
\end{figure}

 




% I think this could have been discussed before this section - so skip
%For the four skills, a test case is labeled as success if the object is flipped down, leaves picking tote, is placed at right quarter of the packing tote, and is packed less than 2cm to the corner respectively. A test case is labeled as a failure if the robot fails to proceed to the next phase after three actions or collides with the environment.

We first evaluate both \ours and Octo on the couscous box with different goal-state images (Tab.~\ref{tab:real_robot_diff_corner_results}).
Both models succeed in the first three skills in all the test cases but Octo only successfully pushes the object to corners in $35\%$ test cases.
In Tab.~\ref{tab:real_robot_diff_obj_results}, we show the experiment results on the diverse object set.
Octo has $0\%$ success rate on the small object (``Instax'' box) and the novel object (Bag).
It also fails in pushing other objects to corners in most test cases.
In contrast, \ours finishes $88\%$ test cases among the objects.
In addition, most of the failures of Octo in pushing are due to collisions with the environment, while two failures of \ours on pushing are inaccurate location with the open loop limitation: the final object pose is slightly outside the goal region (2.1 cm and 2.6 cm away from the corner).
In summary, \ours has much higher success rate in finishing the whole task than Octo, especially on small objects and novel objects.

\begin{table}[H]
\centering
\caption{Four Image Conditioned Goals(Couscous Box)}
\label{tab:real_robot_diff_corner_results}
\begin{tabular}{@{\centering\arraybackslash}m{1.7cm}>{\centering\arraybackslash}m{0.8cm}>{\centering\arraybackslash}m{0.8cm}>{\centering\arraybackslash}m{0.8cm}>{\centering\arraybackslash}m{0.8cm}>{\centering\arraybackslash}m{0.8cm}>{\centering\arraybackslash}m{0.8cm}>{\centering\arraybackslash}m{0.8cm}>{\centering\arraybackslash}m{1cm}@{}}
\toprule
 % & \multicolumn{6}{c}{\textbf{Task Completion}} \\ \cmidrule(lr){2-7} 
 & \multicolumn{2}{c}{\textbf{Flip}} & \multicolumn{2}{c}{\textbf{Pick}} & \multicolumn{2}{c}{\textbf{Pack}} & \multicolumn{2}{c}{\textbf{Push}} \\ \cmidrule(lr){2-3} \cmidrule(lr){4-5} \cmidrule(lr){6-7}  \cmidrule(lr){8-9}
 \textbf{Goal State} & \textbf{\ours} & \textbf{Octo} & \textbf{\ours} & \textbf{Octo} & \textbf{\ours} & \textbf{Octo} & \textbf{\ours} & \textbf{Octo} \\ \midrule
Top Left & \multicolumn{1}{|c}{5/5} & 5/5 & \multicolumn{1}{|c}{5/5} & 5/5 & \multicolumn{1}{|c}{5/5} & 5/5 & \multicolumn{1}{|c}{4/5} & 3/5 \\
Top Right & \multicolumn{1}{|c}{5/5} & 5/5 & \multicolumn{1}{|c}{5/5} & 5/5 & \multicolumn{1}{|c}{5/5} & 5/5 & \multicolumn{1}{|c}{2/5} & 0/5 \\
Bottom Left & \multicolumn{1}{|c}{5/5} & 5/5 & \multicolumn{1}{|c}{5/5} & 5/5 & \multicolumn{1}{|c}{5/5} & 5/5 & \multicolumn{1}{|c}{5/5} & 2/5 \\
Bottom Right & \multicolumn{1}{|c}{5/5} & 5/5 & \multicolumn{1}{|c}{5/5} & 5/5 & \multicolumn{1}{|c}{5/5} & 5/5 & \multicolumn{1}{|c}{5/5} & 2/5 \\ 
 \bottomrule
\end{tabular}
\end{table}


\begin{table}[H]
\centering
\caption{Five Objects Image-Conditioned (Bottom Left Corner)}
\label{tab:real_robot_diff_obj_results}
\begin{tabular}{@{\centering\arraybackslash}m{1.7cm}
>{\centering\arraybackslash}m{0.8cm}
>{\centering\arraybackslash}m{0.8cm}
>{\centering\arraybackslash}m{0.8cm}
>{\centering\arraybackslash}m{0.8cm}
>{\centering\arraybackslash}m{0.8cm}
>{\centering\arraybackslash}m{0.8cm}
>{\centering\arraybackslash}m{0.8cm}
>{\centering\arraybackslash}m{0.8cm}@{}}
\toprule
 % & \multicolumn{6}{c}{\textbf{Task Completion}} \\ \cmidrule(lr){2-7} 
 & \multicolumn{2}{c}{\textbf{Flip}} & \multicolumn{2}{c}{\textbf{Pick}} & \multicolumn{2}{c}{\textbf{Pack}} & \multicolumn{2}{c}{\textbf{Push}} \\ \cmidrule(lr){2-3} \cmidrule(lr){4-5} \cmidrule(lr){6-7}  \cmidrule(lr){8-9}
 \textbf{Object} & \textbf{\ours} & \textbf{Octo} & \textbf{\ours} & \textbf{Octo} & \textbf{\ours} & \textbf{Octo} & \textbf{\ours} & \textbf{Octo} \\ \midrule
``Instax'' Box & \multicolumn{1}{|c}{ 4/5} & 0/5 & \multicolumn{1}{|c}{4/5} & 0/5 & \multicolumn{1}{|c}{4/5} & 0/5 & \multicolumn{1}{|c}{2/5} & 0/5 \\
``Mina'' Box & \multicolumn{1}{|c}{5/5} & 4/5 & \multicolumn{1}{|c}{5/5} & 4/5 & \multicolumn{1}{|c}{5/5} & 4/5 & \multicolumn{1}{|c}{5/5} & 3/5 \\
Couscous Box & \multicolumn{1}{|c}{5/5} & 5/5 & \multicolumn{1}{|c}{5/5} & 5/5 & \multicolumn{1}{|c}{5/5} & 5/5 & \multicolumn{1}{|c}{5/5} & 2/5 \\
Rice Box & \multicolumn{1}{|c}{ 5/5} & 5/5 & \multicolumn{1}{|c}{5/5} & 5/5 & \multicolumn{1}{|c}{5/5} & 4/5 & \multicolumn{1}{|c}{5/5} & 1/5 \\ 
Bag (OOD) &\multicolumn{1}{|c}{ 5/5} & 0/5 & \multicolumn{1}{|c}{5/5} & 0/5 & \multicolumn{1}{|c}{5/5} & 0/5 & \multicolumn{1}{|c}{5/5} & 0/5  \\ \bottomrule
\end{tabular}
\vspace{-3mm}
\end{table}

% Structure
% \begin{enumerate}
% \item Long horizon pick-n-pack.
% \begin{enumerate}
%     \item description with figures and qualitative results.
%     \item language-conditioning, long box, 15 demos on each corner, MuST vs Octo single policy vs non-object-centric MuST vs skill addition.
%     \item Skill skipping and repeating (small table or only show in video).
%     \item Universal goal image conditioning
%     \item Generality: 4 objects * 15 demos * 4 corners $\rightarrow$ in/out-of distribution objects.
% \end{enumerate}
% \item Multi-task sequence pick-n-pack.
% \begin{enumerate}
%     \item description with figures
%     \item Objects in the central region: Need to show a balancing sequence choice to avoid ``modality collapse''
%     \item Object on the boundary: Knows to flip first.
% \end{enumerate}
% \item Station $2$ trajectory planning
% \begin{enumerate}
%     \item Description with figures
%     \item Object-independent goal image conditioning: MuST vs Octo single policy.
% \end{enumerate}
% \end{enumerate}
% 

\section{Experiments} \label{sec6}
We investigate the empirical performance of our new procedures in various experiments to demonstrate their effectiveness.
%To demonstrate the effectiveness of our new procedures, we investigate their empirical performance in the following experiments. 
Recall that our procedures are developed for two distinct goals, namely estimation of the optimal trade-off curve $T$ (see Section \ref{sec:4}) and auditing a privacy claim $T^{(0)}$ (see Section \ref{sec:goal2}). We will run experiments for both of these objectives. \\
%These goals correspond to Sections \ref{sec:4} and \ref{sec:goal2} respectively. \\
%This section aims to validate the theoretical results presented in Section~\todo{cite section} and Section~\todo{cite section}. \\
\textbf{Experiment Setting:} 
%We have outlined two distinct objectives along with their corresponding methodologies:
%\begin{description}
 %   \item[\textbf{Goal 1: Uniform Estimation of the Privacy Curve $T$}]  
 %   The first objective is to uniformly estimate an unknown privacy curve $T$, as stated in Theorem~\ref{theo:1}. To validate not only the theoretical correctness but also the practical effectiveness of this estimation approach, we conducted a simulation study on all four mechanisms. The results of this study are presented in \todo{Table~\ref{tab:estimation_f_curves} and Figure~\ref{fig:todo}.}
 %   \item[\textbf{Goal 2: Detection of Privacy Violations}] 
 %   The second objective is inferential in nature. As formulated in Theorem~\ref{theo:auditor}, the goal is to detect privacy violations for a predefined false positive rate. To demonstrate the effectiveness of this methodology, we constructed faulty algorithms and analyzed their behavior. The results of this analysis are depicted in Figure~\ref{fig:todo}.
%\end{description}
Throughout the experiments, we consider databases $\DB,\DB' \in [0,1]^r$, where the participant number is always $r=10$. As discussed in Section \ref{sec:overview_techniques}, we first choose a pair of neighboring datasets such that there is a large difference in the output distributions of $\Mech(D)$ and $\Mech(D')$. We can achieve this by simply choosing $D$ and $D'$ to be as far apart as possible (while still remaining neighbors) and we settle on the choice 
%As typical in the privacy validation literature, we consider two neighboring databases that are far apart. On the $r$-dimensional cube $[0,1]^r$ we make the natural choice of
\begin{equation}\label{eq_databases}
    \DB=(0,\hdots, 0)\quad \textnormal{and} \quad \DB'=(1,0,\hdots, 0)
\end{equation}
for all our experiments.
%and notice that similar results as the ones below hold for other pairs of databases. %Our methods do however work just as well for other data bases $D$ and $D'$.
%Additionally, for data lying in the unit cube, this choice is natural, as these two databases are far apart on the unit cube.

\subsection{Mechanisms}\label{sec:algorithms}
In this section, we test our methods on two frequently encountered mechanisms from the auditing literature: the Gaussian mechanism and differentially private Stochastic Gradient Descent (DP-SGD). We study two other prominent DP algorithms -- the Laplace and Subsampling mechanism -- in Appendix \ref{AppB}. \\
%We apply our methods to four mechanisms frequently encountered in the privacy literature: the Gaussian mechanism, the Laplace mechanism, the Subsampling mechanism, and, most notably, the Noisy Stochastic Gradient Descent (DP-SGD) mechanism. These algorithms are quite  heterogeneous and hence collectively form a good benchmark to evaluate our methods. We quickly review these mechanisms and specify parameter settings. \\

\noindent \textbf{Gaussian mechanism.}
We consider the summary statistic $S(x)= \sum_{i=1}^{10} x_i$ and the mechanism
\begin{equation*}
    M(x):= S(x)+Y~,
\end{equation*}
where $Y\sim \mathcal N (0, \sigma^2)$. The statistic $S(x)$ is privatized by the random noise $Y$ if the variance $\sigma^2$ of the Normal distribution is appropriately scaled. We choose $\sigma = 1$ for our experiments and note that - in our setting - the optimal trade-off curve is given by 
\begin{align*}
     T_{Gauss}(\alpha)= \Phi(\Phi^{-1}(1-\alpha)- 1).
\end{align*}
We point the reader to \cite{Dong2022} for more details. \\








%\textbf{Additive Noise Mechanisms}
%The most prominent DP algorithms are the Laplace and the Gaussian Mechanism. If we consider the summary statistic $S(x)= \sum_{i=1}^{10} x_i$, the output can be described by
%\begin{equation*}
%    M(x):= S(x)+Y~,
%\end{equation*}
%where $Y\sim Lap(0,b)$ or $Y\sim \mathcal N (0, \sigma^2)$, respectively. Here $b>0$ denotes the scale parameter in the Laplace distribution and $\sigma^2$ the variance for the normal distribution. Given the structure of $M$, these mechanisms are generically referred to as "additive noise mechanisms". Appropriately scaled, both mechanisms fulfill $f$-DP. For the Gaussian mechanism we set $\sigma=1$, for which \cite{Dong2022} derived the trade-off curve
%\begin{equation*}
%    T_{Gauss}(\alpha)= \Phi(\Phi^{-1}(1-\alpha)-\mu)
%\end{equation*}
%as an explicit expression of the optimal trade-off function between $P = \mathcal N(0,1)$ and $Q = \mathcal N(\mu,1)$. For the Laplace mechanism, we set $b=1$, and note again that \cite{Dong2022} derived an explicit formula given by
%\begin{equation*}
%    T_{Lap}(\alpha)=\begin{cases}
%        1-\exp(\mu)\alpha,  &\alpha<\exp(-\mu)/2~,\\
%        \exp(-\mu)/4 \alpha,  &\exp(-\mu)/2\leq \alpha\leq 1/2~,\\
%        \exp(-\mu)(1-\alpha), &\alpha>1/2
 %   \end{cases}
%\end{equation*}
%for the optimal trade-off function between $P = Lap(0,1)$ and $Q = Lap(\mu,1)$.\\
%After the simpler additive noise mechanisms, we turn to the more advanced mechanisms of subsampling and the  DP-SGD, both of which play a role in the context of private machine learning.\\
%\textbf{Subsampling Mechanism:} Random subsampling provides an effective way to enhance the privacy of a DP mechanism $M$. We only provide a rough outline here and refer for details to \cite{Dong2022}.
%In simple words, we choose an integer $m$ with  $1\leq m< r$, where $r$ is the size of the databases $D$. In subsampling, we extract random subsamples of size $m$ of these databases, giving us the smaller databases $\bar D$. The mechanism $M$ is then applied to  $\bar D$ instead of $D$, providing an additional layer of protection to users. If a user is not part of $\bar D$, their privacy cannot be compromised. The amplifying effect of subsampling is visible in the optimal trade-off curve: If $M$ implies the curve $T$, it turns out that $M(\bar D)$ implies the curve
%\begin{equation*}
%    \bar T(\alpha)=  \frac{m}{r}T(\alpha)+\frac{r-m}{r}(1-\alpha),
%\end{equation*}
%which is strictly more private than $T$ for any $m<r$. A minor technical peculiarity of subsampling is that the resulting curve $\bar T$ is generally not symmetrical, even if $T$ is (see \cite{Dong2022} for details on the symmetry of trade-off functions). Trade-off curves are usually considered to be symmetrical and one can symmetrize $\bar T$ by applying a symmetrizing operator $\mathbf{C}$ (again, see \cite{Dong2022}). In our simulations we will adhere to this procedure and depict the symmetrized version $\mathbf{C}[\bar T]$ together with a symmetrized estimator. Further details on the symmetrization can be found in Appendix \ref{AppB}.
%At first glance, this result may seem complicated, but it is in fact quite natural. Suppose we select $m$ entries randomly for the two neighboring databases $\DB,\DB'$ with each of size $k$. All entries in  $\DB,\DB'$ except one are identical - so the chance of selecting the entry where they differ (in $m$ draws) is $p=m/k$. 
%Conversely the probability of not selecting named entry is $(1-p)$.  
%This characterizes the construction of $T_p$, and the factor $(1-\alpha)$ corresponds to perfect privacy, as in that case $\alpha+\beta=\alpha+1-\alpha=1$. With that in hand, $C_p(T)=\min\{T_p,T_p^{-1}\}^{**}$ (again see in \cite{Dong2022}) is simply a symmetrization of $T_p$ in a sense that it is the greatest convex minorant of the minimum $T_p$ and $T_p^{-1}$ (\todo{maybe cite sth}). Here, for a function $T$, $T^{**}$ denotes the twice convex conjugate of $T$.\\  
%For the following experiments involving subsampling, we use the Gaussian mechanism as $M$ (with $\sigma=1$) and obtain the subsampled version $M'$, by fixing the parameter $m=5$ (recall that $r=10$). \\
%Observing only independent outputs of $M(x)$ will only yield an estimator for $\hat T_p$. As an additional contribution, we approximate $T_{sub}(\alpha)$ by incorporating a numerical approximation based on $\hat T_p$. \todo{should we expand here?}\\

\noindent \textbf{DP-SGD.} The DP-SGD mechanism is designed to (privately) approximate a solution for the empirical risk minimization problem
\begin{equation*}
\theta^*=argmin_{\theta\in \Theta} \mathcal L_x(\theta) \quad \text{with} \quad \mathcal L_x(\theta)=\frac{1}{r}\sum_{i=1}^{r} \ell(\theta, x_i)~.
\end{equation*}
Here, $\ell$ denotes a loss function, $\Theta$ a closed convex set and $\theta^*\in \Theta$ the unique optimizer. For sake of brevity, we provide a description of DP-SGD in the appendix (see Algorithm \ref{alg:noisy_sgd}). In our setting, we consider the loss function $\ell(\theta, x_i)=\frac{1}{2} (\theta-x_i)^2$, initial model $\theta_0=0$ and $\Theta=\mathbb{R}$. The remaining parameters are fixed as $\sigma=0.2, \rho = 0.2, \tau = 10, m=5$. In order to have a theoretical benchmark for our subsequent empirical findings, we also derive the theoretical trade-off curve $T_{SGD}$ analytically for our setting and choice of databases (see Appendix \ref{AppB} for details). Our calculations yield
%For the choice of databases as in equation \eqref{eq_databases}, one can compute the trade-off curve $T_{SGD}$ analytically: 
\begin{equation*}
    T_{SGD}(\alpha)=\sum_{I\subset \{1,\hdots, \tau \}} \frac{1}{2^{\tau}}\Phi\Big(\Phi^{-1} (1-\alpha)-\frac{\mu_I}{\bar\sigma}\Big)~.
\end{equation*}
where $\mu_I$ is chosen as in \eqref{mu_I} and $\bar{\sigma}$ as in \eqref{sigma_bar}.

\subsection{Simulations}
We begin by outlining the parameter settings of our KDE and $k$-NN methods for our simulations. We then discuss the metrics employed to validate our theoretical findings and, in a last step, present and analyze our simulation results.\\
\textbf{Parameter settings:}
%For the subsequent simulations we always use the same parameters across all algorithms, acknowledging the black-box setting. 
For the KDEs, we consider different sample sizes of $n_1=10^2,10^3,10^4,10^5,10^6$ and we fix the perturbation parameter at $h=0.1$. For the bandwidth parameter $b$ (see Sec. \ref{sec:kde}), we use the method of \cite{bandwidth}. To approximate the optimal trade-off curve, we use $1000$ equidistant values for $\eta$ between $0$ and $15$ (see Algorithm \ref{alg:pointwise_KDE_estimator} for details on the procedure). For the $k$-NN, we set the training sample size to $n_2=10^6,10^7,10^8$ and testing sample size to $10^6$. \\
%\todo{Yu: are you sure that this is sufficient?}\\

\noindent \textbf{Estimation}
The first goal of this work is estimation of the optimal trade-off curve $T$. In our experiments, we want to illustrate the uniform convergence of the estimator $\hat T_h$ to the optimal curve $T$, derived in Theorem \ref{theo:1}. Therefore, we consider increasing sample sizes $n_1$ to study the decreasing error. The distance of $\hat T_h$ and $T$ in each simulation run is measured by the  uniform distance\footnote{Of course, one cannot practically maximize over all (infinitely many) arguments $\alpha \in [0,1]$. The estimator $\hat T_h$ is made for a grid of values for $\eta$ (see our parameter settings above) and we maximize over all gridpoints.} %maximum distance on a grid $G$ 
%We repeatedly estimate the respective trade-off curves of the four mechanism introduced in Section \ref{sec:algorithms} and computed 
\[
    Error_T:=\sup_{\alpha \in [0,1]}|\hat T_h(\alpha)-T(\alpha)|.
\]
%on a grid $G$ of $[0,1]$. 
%In our setting, we defined $G$ as the grid given by the KDE. Since, we choose $1000$ $\eta$ equidistant, we will get $1000$ $\alpha$ values. However, they do not have to be equidistant nor unique. 
To study not only the distance in one simulation run, but across many, we calculate $Error_T$ in $1000$ independent runs and take the (empirical) mean squared error
\begin{equation}\label{eq:mse}
    MSE(Error_T):= \Ex{Error_T^2}
    %\mathbb{E}\mathbb Var(Error_G)+\mathbb E[Error_G]^2~.
\end{equation}
The results are depicted in Figure \ref{fig:estimation_mse} for the DP algorithms described in this section and the appendix. On top of that, we also construct figures that upper and lower bound the worst case errors for the Gaussian mechanism and DP-SGD over the $1000$ simulation runs. These plots visually show how the error of the estimator $\hat T_h$ shrinks as $n_1$ grows. 
% for the sample size $n_1=1000$. 
%For that, we computed the worst estimation point wise on an equidistant discretization of $[0,1]$ and interpolated the curves linearly. 
The results are summarized in Figures \ref{fig:gaussian}-\ref{fig:sgd}.\\
\begin{figure}
\centering\includegraphics[width=0.75\linewidth]{Figures/plot_table.png}
    \caption{\centering
    Empirical MSE defined in \eqref{eq:mse} to empirically validate Theorem \ref{theo:1} for varying sample sizes $n_1$ and over $1000$ simulation runs each.}\label{fig:estimation_mse}
\end{figure}
\begin{figure*}[h!]
    \centering
    \subfloat[$n_1=10^3$]{\includegraphics[width=0.3\textwidth]{Figures/Gaussian_shade_1000.png}}
    \hfill
    \subfloat[$n_1=10^4$]{\includegraphics[width=0.3\textwidth]{Figures/Gaussian_shade_10000.png}}
    \hfill
    \vspace{-0.2cm}
    \subfloat[$n_1=10^5$]{\includegraphics[width=0.3\textwidth]{Figures/Gaussian_shade_100000.png}}
    \caption{Estimation of the Gaussian Trade-off curve $T_{Gauss}$ for varying sample sizes and $\mu=1$. Min- and Max Curve lower- and upper bound the worst point-wise deviation from the true curve $T_{Gauss}$ over $1000$ simulations.}
    \label{fig:gaussian}
\vspace{-0.1cm}
\centering
    \subfloat[$n_1=10^3$]{\includegraphics[width=0.3\textwidth]{Figures/SGD_shade_1000.png}}
    \hfill
    \subfloat[$n_1=10^4$]{\includegraphics[width=0.3\textwidth]{Figures/SGD_shade_10000.png}}
    \hfill
    \vspace{-0.2cm}
    \subfloat[$n_1=10^5$]{\includegraphics[width=0.3\textwidth]{Figures/SGD_shade_100000.png}}
    \caption{Estimation of the DP-SGD Trade-off curve $T_{SGD}$ for varying sample sizes. Min- and Max Curve lower- and upper bound the worst point-wise deviation from the true curve $T_{SGD}$ over $1000$ simulations.}
    \label{fig:sgd}
\end{figure*}


\noindent {\textbf{Inference}\label{Inference}}
Next, we turn to the second goal of this work: Auditing a $T^{(0)}$-DP claim for a postulated trade-off curve $T^{(0)}$. 
The theoretical foundations of our auditor can be found in Theorem \ref{theo:auditor}. The theorem makes two guarantees: First, that for a mechanism $M$ satisfying $T^{(0)}$-DP the auditor will (correctly) not detect a violation, except with low, user-determined probability $\gamma$. Second, if $M$ violates  $T^{(0)}$-DP, the auditor will (correctly) detect the violation for sufficiently large sample sizes $n_1,n_2$. Together, these results mean that if a violation of $T^{(0)}$-DP is detected by the auditor, the user can have high confidence that $M$ does indeed not satisfy $T^{(0)}$-DP. 
%To begin, we examine the first result, which ensures, informally speaking that the auditor will not generate more than $\gamma>0$ false positives, when auditing a mechanism $M$. The second result guarantees that if the claimed privacy does not hold, the auditor will eventually identify this with probability $1$ as the sample size increases.
For the first part, we consider a scenario, where the claimed trade-off curve $T^{(0)}$ is the correct one $T^{(0)}=T$ ($M$ does not violate $T^{(0)}$-DP). For the second part, we choose a function $T^{(0)}$ above the true curve $T$ ($M$ violates $T^{(0)}$-DP). We will consider both scenarios for the Gaussian mechanism and DP-SGD.
%We will use two of the four mechanism for illustration: First, the standard Gaussian mechanism, as an example of an additive noise mechanism and second the DP-SGD mechanism, as an example of a machine learning mechanism.
%We start with auditing correctly claimed curves $T^{(0)}$. For that purpose, 
We run our auditor (Algorithm \ref{alg:auditor}) with parameters $n_1=10^4$ and $\gamma=0.05$ fixed. The choice of $\gamma=0.05$ is standard for confidence regions in statistics and we further explore the impact of $n_1$ and $\gamma$ in additional experiments in Appendix \ref{AppB}. Here, we focus on the most impactful parameter, the sample size $n_2$ and study values of  $n_2 = 10^6,10^7,10^8$. \\
Technically, the auditor only outputs a binary response that indicates whether a violation is detected or not. However, in our below experiments, we depict the inner workings of the auditor and geometrically illustrate how a decision is reached. More precisely, in Figure \ref{fig:not_faulty_sgd_gauss} we depict the claimed trade-off curve $T^{(0)}$ as a blue line. The auditor makes an estimate for the true trade-of curve $T$, namely $\hat T_h$ depicted as the orange line. The location, where the orange line (estimated DP) and the blue line (claimed DP) are the furthest apart is indicated by the vertical, dashed green line. This position is associated with the threshold $\hat \eta^*$ in Algorithm \ref{alg:pointwise_KDE_estimator}. As a second step, $\hat \eta^*$ is used in the $k$NN method to make a confidence region, depicted as a purple square (this is $\square_\gamma$ from \eqref{e:defsq}). If the square is fully below the claimed curve $T^{(0)}$, a violation is detected (Figure \ref{fig:faulty_sgd_gauss}) and if not, then no violation is detected (Figures \ref{fig:gaussian} and \ref{fig:sgd}). As we can see, detecting violations requires $n_2$ to be large enough, especially when $T^{(0)}$ and $T$ are close to each other. \\
For the incorrect $T^{(0)}$-DP claims, we have done the following: For the Gaussian case (Figure \ref{fig:faulty_sgd_gauss}), we have used a trade-off curve with parameter $\mu=0.5$ instead of the true $\mu=1$. For DP-SGD, we have used the trade-off curve corresponding to $\tau = 5$ instead of the true $\tau =10$ iterations (Figure \ref{fig:faulty_sgd_gauss}). 

%the trade-off curve of DP-SGD with a correctly claimed privacy curve and false claim. The correctly claimed can be found in Figure \ref{fig:not_faulty_sgd_gauss}. For the incorrectly claimed curve, it was stated that DP-SGD ran for only $t_{-}=5$ iterations, accessing the data just five times and thereby reinforcing the privacy guarantee. However, in reality it ran $t_{-}=10$ times, leading to a privacy breach. The results are depicted in Figure \ref{fig:faulty_sgd_gauss}.
%In this algorithm, we Algorithm \ref{alg:KDE_estimator} as a subroutine to derive an estimate $\hat \eta^*$ and we use the sample size $n_1=10^4$. Second, we run the $k$-NN with different sample sizes $n_2=10^6,10^7,10^8$. For the confidence level, we set $\gamma=0.05$, which yields confidence squares induced by $w(\gamma)$ defined in equation \eqref{e:wgamma}. 
%For the first audit in displayed in Figure \ref{fig:not_faulty_sgd_gauss}, we have audited a correctly claimed trade-off curve $T_0$. We detect a faulty mechanism, whenever the purple square $\square_\gamma=\square_{0.05}$ is disjoint from the claimed curve $T_0$. For a faulty implementation, we have use $\mu=0.5$ for the claimed curve $T_0$, while in reality the true curve only fulfills $\mu=1$. The results are displayed in Figure \ref{fig:faulty_sgd_gauss}. To complement these results with the DP-SGD, we also considered DP-SGD with a correctly claimed privacy curve and false claim. The correctly claimed can be found in Figure \ref{fig:not_faulty_sgd_gauss}. For the incorrectly claimed curve, it was stated that DP-SGD ran for only $t_{-}=5$ iterations, accessing the data just five times and thereby reinforcing the privacy guarantee. However, in reality it ran $t_{-}=10$ times, leading to a privacy breach. The results are depicted in Figure \ref{fig:faulty_sgd_gauss}.
\begin{figure*}[h!]
    \centering
    \subfloat[\centering $n_2=10^6$,\textbf{Ground Truth:} No Violation; \newline \textbf{Decision:} \textcolor{green}{"No Violation"}{\textcolor{green}{\scalebox{1.5}{\ding{51}}}}]{\includegraphics[width=0.3\textwidth]{Figures/gauss_100.png}}
    \hfill
    \subfloat[\centering $n_2=10^7$,\textbf{ Ground truth:} No Violation; \newline \textbf{Decision:} \textcolor{green}{"No Violation"}{\textcolor{green}{\scalebox{1.5}{\ding{51}}}}]{\includegraphics[width=0.3\textwidth]{Figures/gauss_100_7.png}}
    \hfill
    \subfloat[\centering $n_2=10^8$, \textbf{ Ground truth:}No Violation; \newline \textbf{Decision:} \textcolor{green}{"No Violation"}{\textcolor{green}{\scalebox{1.5}{\ding{51}}}}]{\includegraphics[width=0.3\textwidth]{Figures/gauss_100_8.png}}
   \vspace{-1em}
    \subfloat[\centering $n_2=10^6$, \textbf{Ground truth:} No Violation; \newline \textbf{Decision:} \textcolor{green}{"No Violation"}{\textcolor{green}{\scalebox{1.5}{\ding{51}}}}]{\includegraphics[width=0.3\textwidth]{Figures/sgd_100.png}}
    \hfill
    \subfloat[\centering $n_2=10^7$, \textbf{Ground truth:} No Violation; \newline \textbf{Decision:} \textcolor{green}{"No Violation"}{\textcolor{green}{\scalebox{1.5}{\ding{51}}}}]{\includegraphics[width=0.3\textwidth]{Figures/sgd_100_7.png}}
    \hfill
    \subfloat[\centering $n_2=10^8$, \textbf{Ground truth:} No Violation; \newline \textbf{Decision:} \textcolor{green}{"No Violation"}{\textcolor{green}{\scalebox{1.5}{\ding{51}}}}]{\includegraphics[width=0.3\textwidth]{Figures/sgd_100_8.png}} \caption{\textbf{Auditing a correct Mechanism:} Claimed curve $\textcolor{blue}{T^{(0)}} = T_{Gauss}$ (a,b,c) and $\textcolor{blue}{T^{(0)}} = T_{SGD}$ (d,e,f). Obtain critical vertical line with step 3 in Algorithm \ref{alg:auditor} with intercept $(\hat\alpha(\hat\eta^*), \hat \beta(\hat \eta^*))$, $k$-NN point estimator \small{\textcolor{purple}{\ding{108}}} $(\tilde\alpha(\hat\eta^*), \tilde \beta(\hat\eta^*))$ and confidence region $\textcolor{purple}{\square}$. The sample size for the KDE is $n_1=10^4$ and the confidence parameter is $\gamma=0.05$.}
    \label{fig:not_faulty_sgd_gauss}
\end{figure*}
\begin{figure*}[h!]
    \centering
    \subfloat[\centering $n_2=10^6$, \textbf{Ground truth:} Violation; \newline \textbf{Decision:} \textcolor{red}{"No Violation"}{\textcolor{red}{\scalebox{1.5}{\ding{55}}}}]{\includegraphics[width=0.3\textwidth]{Figures/gauss_faulty_100.png}}
    \hfill
    \subfloat[\centering $n_2=10^7$, \textbf{Ground truth:} Violation; \newline \textbf{Decision:} \textcolor{green}{"Violation"}{\textcolor{green}{\scalebox{1.5}{\ding{51}}}}]{\includegraphics[width=0.3\textwidth]{Figures/gauss_faulty_100_7.png}}
    \hfill
    \subfloat[\centering $n_2=10^8$, \textbf{Ground truth:} Violation; \newline \textbf{Decision:} \textcolor{green}{"Violation"}{\textcolor{green}{\scalebox{1.5}{\ding{51}}}}]{\includegraphics[width=0.3\textwidth]{Figures/gauss_faulty_100_8.png}}
    \vspace{-1em}
    \subfloat[\centering $n_2=10^6$, \textbf{Ground truth:} Violation; \newline \textbf{Decision:} \textcolor{red}{"No Violation"}{\textcolor{red}{\scalebox{1.5}{\ding{55}}}}]
    {\includegraphics[width=0.3\textwidth]{Figures/sgd_faulty_100.png}}
    \hfill
    \subfloat[\centering $n_2=10^7$, \textbf{Ground truth:} Violation; \newline \textbf{Decision:} \textcolor{red}{"No Violation"}{\textcolor{red}{\scalebox{1.5}{\ding{55}}}}]{\includegraphics[width=0.3\textwidth]{Figures/sgd_faulty_100_7.png}}
    \hfill
    \subfloat[\centering $n_2=10^8$, \textbf{Ground truth:} Violation; \newline \textbf{Decision:} \textcolor{green}{"Violation"}{\textcolor{green}{\scalebox{1.5}{\ding{51}}}}]{\includegraphics[width=0.3\textwidth]{Figures/sgd_faulty_100_8.png}}
     \caption{\textbf{Auditing a faulty Mechanism:} Claimed Curve $\textcolor{blue}{T^{(0)}} = T_{Gauss}$ (a,b,c) with $\mu=0.5$ and $\textcolor{blue}{T^{(0)}} = T_{SGD}$ (d,e,f) with $t_{-}=5$. Both mechanisms assume stronger privacy ($\mu=0.5<1$ and $t_{-}=5<10$). Critical vertical line derived by KDEs using step 3 in Algorithm \ref{alg:auditor} with intercept $(\hat\alpha(\hat\eta^*), \hat \beta(\hat \eta^*))$, $k$-NN point estimator {\textcolor{purple}{\ding{108}}} $(\tilde\alpha(\hat\eta^*), \tilde \beta(\hat\eta^*))$ and confidence region $\textcolor{purple}{\square}$. The sample size for KDE is $n_1=10^4$ and the confidence parameter is $\gamma=0.05$.}
    \label{fig:faulty_sgd_gauss}
\end{figure*}

\noindent\textbf{Implementation Details} The implementation is done using python and R. \footnote{\scriptsize{\url{https://github.com/stoneboat/fdp-estimation}}}. For the simulations, we have used a local device and a server. All runtimes were collected on a local device with an Intel Core i5-1135G7 processor (2.40 GHz), 16 GB of memory, and running Ubuntu 22.04.5, averaged over $10$ simulations. Thus, we demonstrate fast runtimes even on a standard personal computer.
Additionally, we used a server with four AMD EPYC 7763 64-Core (3.5 GHz) processors and 2 TB of memory and running Ubuntu 22.04.4 was used for repetitive simulations. For python, we have used Python 3.10.12 and the libraries "numpy" \cite{2020NumPy-Array}, "scikit-learn" \cite{pedregosa2011scikit} and "scipy" \cite{2020SciPy-NMeth}. For R, we used R version 4.3.1 and the libraries "fdrtool" \cite{fdrtool} and "Kernsmooth" \cite{Kernsmooth}. 
\begin{table}[h!]
\centering
\begin{tabular}{|l|c|}
\hline
\textbf{Algorithm}                           & \textbf{Runtime in seconds} \\ \hline
Gaussian mechanism              &  26.3                                                    \\ \hline
Laplace mechanism            &    30.51                                                     \\ \hline
Subsampling mechanism         &   27.82                                                      \\ \hline
DP-SGD           &              61.1                                         \\ \hline
 
\end{tabular}
\caption{Average runtimes of Algorithm \ref{alg:pointwise_KDE_estimator} for $n_1=10^5$ over $10$ runs to obtain the full trade-off curve $T$.}
\label{tab:running_times_KDE}
\end{table}
\begin{table}[h!]
\centering
\begin{tabular}{|l|c|}
\hline
\textbf{Algorithm}                           & \textbf{Runtime in seconds} \\ \hline
Gaussian mechanism              &    62.63                                                  \\ \hline
Laplace mechanism            &        67.04                                                 \\ \hline
Subsampling mechanism         &     66.98                                                  \\ \hline
DP-SGD           &    114.86                                                 \\ \hline
 
\end{tabular}
\caption{Average runtimes of Algorithm \ref{alg: general BayBox estimator} for $n_2=10^6$ over $5$ runs to obtain one point of the trade-off curve $T$ with confidence region.} %\\[-8ex]} 
\label{tab:running_times_kNN}
\end{table}

\subsection{Interpretation of the results}
Our experiments empirically showcase details of our methods' concrete performance. 
%refine our understanding of certain details of our methods. 
For Goal 1 (estimation), we see in Figure \ref{fig:estimation_mse} the fast decay of the estimation error of $\hat T_h$ for the optimal trade-off curve. The estimation error decays fast in $n_1$, regardless of whether there are plateau values in the sense of Assumption \ref{ass1} (e.g. Laplace Mechanism) or not (e.g. Gaussian Mechanism).
These quantitative results are supplemented by the visualizations in  
Figures~\ref{fig:gaussian}--\ref{fig:sgd}, where we depict the largest distance of $\hat T_h$ and $T$ in $1000$ simulation runs (captured by the red band). Even for the modest sample size of $n_1 = 10^3$, this band is fairly tight and for $n_1 = 10^5$ the estimation error is almost too minute to plot. We find this convergence astonishingly fast. It may be partly explained by the estimator $\hat T_h$ being structurally similar to $T$ -  after all $\hat T_h$ is also designed to be a trade-off curve for an almost optimal LR test.
The approximation over the entire unit interval corresponds to the uniform convergence guarantee in Theorem~\ref{theo:1}. 
%demonstrate that even with relatively small sample sizes, such as $n_1 = 10^3$, the worst global error across 1000 simulations remains notably small. For that observe that the combined worst deviation from the true curve $T_0$ across $1000$ simulations is already for $n_1=10^4$ almost negligible. As the sample size increases even further, the error converges to zero for all $\alpha\in[0,1]$ as visible in Figures~\ref{fig:gaussian}--\ref{fig:sgd} (c). This aligns with the uniform convergence established in Theorem~\ref{theo:1}. In addition to that, throughout all mechanisms, Figure \ref{fig:estimation_mse} illustrates the convergence for the same set of parameters, highlighting the robustness and adaptability necessary in a black-box setting. Here, we also emphasize that the MSEs are computed on an equidistant grid evaluated on the KDEs. This distinction is important because, in principle, for an equidistant $\eta$, the distribution of the $\alpha$ values could theoretically concentrate on a few points. This behavior is especially observable whenever the quotient of the densities is constant for a non-negligible subset of $[0,1]$. An example of that would be the Laplace mechanism.
%To address this potential issue, we have evaluated the error on both the grid implied by the KDE and an equidistant grid. Through this comparison, we have observed that such concentration does not impact the estimation. If this phenomenon arises, a linear interpolation is sufficient, as the trade-off curve is also linear on that subset, and if it does not arise, then the $\alpha$'s are also evenly distributed. \todo{@Tim: do you agree with me? I think it is important to make this remark, as someone could be a bit confused when observing this clustered results for the Laplacian algorithm}
% \\

For Goal 2 (inference), we recall that a  $T^{(0)}$-DP violation is detected if the box $\square_\gamma$ (purple) lies completely below the postulated curve $T^{(0)}$ (blue). In Figure \ref{fig:not_faulty_sgd_gauss} we consider the case of no violation where $T=T^{(0)}$, and we expect not to detect a violation. This is indeed what happens, since $\square_\gamma$ intersects with the curve $T^{(0)}$ in all considered cases. Interestingly, we observe that $\square_\gamma$ has a center close to $\alpha=0$ in the cases where no violation occurs (such a behavior might give additional visual evidence to users that no violation occurs).
%In principal, one would reject the privacy curve, whenever the purple square $\textcolor{purple}{\square}$ is disjoint from the blue \textcolor{blue}{curve}, i.e.
%\begin{equation*}
%    \textcolor{purple}{\square} \cap \textcolor{blue}{\textnormal{curve}}=\emptyset~.
%\end{equation*}
%In Figure \ref{fig:not_faulty_sgd_gauss} and \ref{fig:not_faulty_sgd_gauss} we have displayed the case where the claimed privacy indeed holds, so we expect to not detect a violation. For both mechanism, we can observe that for any sample size, we correctly do not reject that claim, as the $\textcolor{purple}{\square}$ and the $\textcolor{blue}{curve}$ are not disjoint. 
In Figure \ref{fig:faulty_sgd_gauss}, we display the case of faulty claims, where the privacy breach is caused by a smaller variance for both mechanisms under investigation. In accordance with Theorem \ref{theo:auditor}, we expect a detection of a violation if $n_2$ is large enough. This is indeed what happens, at a sample size of $n_2=10^7$ for the Gaussian mechanism and at  $n_2=10^8$ for DP-SGD. As we can see, larger samples $n_2$ are needed to expose claims $T^{(0)}$ that are closer to the truth $T$ (as for DP-SGD in our example). For larger $n_2$ the square $\square_\gamma$ shrinks (see eq. \eqref{e:defsq}) leading to a higher resolution of the auditor. 
%While for both mechanisms, we do not detect a statistical significant deviation for $n_2=10^6$ (since the $\textcolor{purple}{\square}$ is not disjoint from the blue \textcolor{blue}{curve}), already for $n_2=10^7$, the auditor detects the privacy violation for the Gaussian mechanism. Regarding the DP-SGD case, we have to increase the sample size to $n_2=10^8$ to detect the violation. The clear message here is that smaller privacy violations (curves are closer to each other), the larger the sample size $n_2$ has to be, to obtain small confidence regions. This is a classical pattern in statistics and aligns with Theorem \ref{theo:auditor} part (2). The main reason here is that higher $n_2$ significantly shrinks the confidence square (recall Theorem \label{thm: accuracy stat of kNN BayBox estimator}
%), which eventually will be fully below the \textcolor{blue}{curve}, indicating a statistically significant difference. Here, we explicitly want to point out that even though e.g. Figure \ref{fig:faulty_sgd_gauss} (a) was not significant, one can already take the deviation from the claimed curve as a first indication and just increase the sample size for a stronger evidence. Consequently, we strongly encourage users to also consider the point estimator derived from the $k$-NN algorithm as a potential indicator of faulty implementations. In fact, our empirical evidence indicates that the confidence interval for our $k$-NN based point estimator is significantly narrower than the theoretical bound we derived. This discrepancy arises because the theoretical convergence rate of the $k$-NN algorithm is generally not tight; in practice, the performance of the $k$-NN algorithm converges more rapidly than the theoretical rate suggests. To put this into more practical observations, the difference of the estimators derived for $n_2=10^6,10^7,10^8$ were for all mechanism negligible.


\section{Conclusion}\label{sec:conclusion}
In this work, we introduce \ours, a skill chaining model designed for long horizon dexterous manipulation. 
Given a task decomposed into $N$ skills, \ours trains $N+1$ heads: $N$ heads to learn individual skills and an additional head for skill progress estimation. 
Based on the estimated progress values, a skill progress guided skill selector \progss chooses the proper skill to execute at each time step. 
Qualitative results demonstrate that \progss effectively adapts to unexpected disturbance.
Comprehensive experiments in both simulation and real world settings reveal the performance advantages of \ours over the single-head Octo baseline, as well as its capability to handle various skill sequences and diverse object sets.



\newpage
\bibliographystyle{format/IEEEtran}
\bibliography{bib/bib}

\end{document}
