\documentclass[letterpaper, 10 pt, conference]{format/ieeeconf}  % Comment this line out if you need a4paper

%\documentclass[a4paper, 10pt, conference]{ieeeconf}      % Use this line for a4 paper

\IEEEoverridecommandlockouts                              % This command is only needed if 
                                                          % you want to use the \thanks command

\overrideIEEEmargins                                      % Needed to meet printer requirements.

%In case you encounter the following error:
%Error 1010 The PDF file may be corrupt (unable to open PDF file) OR
%Error 1000 An error occurred while parsing a contents stream. Unable to analyze the PDF file.
%This is a known problem with pdfLaTeX conversion filter. The file cannot be opened with acrobat reader
%Please use one of the alternatives below to circumvent this error by uncommenting one or the other
%\pdfobjcompresslevel=0
%\pdfminorversion=4

% See the \addtolength command later in the file to balance the column lengths
% on the last page of the document
\usepackage{dsfont}

\usepackage{amsmath}
%
%
%
%
%
%
%
%
%
%
%
%
%
%
%
%
%
%
%
%
%
%
%
%
%
%
%
%
%
%
%
%
%
%

%


\newcommand{\blue}{\color{blue}}
\newcommand{\red}{\color{red}}
\newcommand{\black}{\color{black}}

\newcommand{\clip}[1]{\mathrm{clip}\left(#1\right)}
\newcommand{\erf}{\mathrm{erf}}
\newcommand{\barh}{\bar{h}}
\newcommand{\barf}{\bar{f}}
%

\newcommand{\cmark}{$\checkmark$}
\newcommand{\xmark}{$\times$}
%
%

\newcommand{\overbar}[1]{\mkern 1.5mu\overline{\mkern-1.5mu#1\mkern-1.5mu}\mkern 1.5mu}
 \providecommand{\ttheta}{\bm{\theta}}
 \renewcommand{\aa}{\mathbf{a}}
  \providecommand{\bb}{\mathbf{b}}
  \providecommand{\md}{\mathrm{d}}
  \providecommand{\cc}{\mathbf{c}}
  \providecommand{\dd}{\mathbf{d}}
  \providecommand{\ee}{\mathbf{e}}
  \providecommand{\ff}{\mathbf{f}}
  \let\ggg\gg
  \renewcommand{\gg}{\boldsymbol{g}}
  \providecommand{\hh}{\mathbf{h}}
  \providecommand{\ii}{\mathbf{i}}
  \providecommand{\jj}{\mathbf{j}}
  \providecommand{\kk}{\mathbf{k}}
  %
  %
  \providecommand{\mm}{\mathbf{m}}
  \providecommand{\nn}{\mathbf{n}}
  \providecommand{\oo}{\mathbf{o}}
  \providecommand{\pp}{\mathbf{p}}
  \providecommand{\qq}{\mathbf{q}}
  \providecommand{\rr}{\mathbf{r}}
  \renewcommand{\ss}{\mathbf{s}}
  \providecommand{\tt}{\mathbf{t}}
  \providecommand{\uu}{\mathbf{u}}
  \providecommand{\vv}{\mathbf{v}}
  \providecommand{\ww}{\mathbf{w}}
  \providecommand{\xx}{\mathbf{x}}
  \providecommand{\yy}{\mathbf{y}}
  \providecommand{\zz}{\mathbf{z}}
   \providecommand{\AA}{\mathbf{A}}
  \providecommand{\aalpha}{\boldsymbol{\alpha}}
  \providecommand{\ssigma}{\boldsymbol{\sigma}}
  \providecommand{\llambda}{\boldsymbol{\lambda}}
 \providecommand{\GGamma}{\boldsymbol{\Gamma}}

  %
  \providecommand{\e}{\mathrm{e}}
  \providecommand{\mA}{\mathbf{A}}
  \providecommand{\mB}{\mathbf{B}}
  \providecommand{\mC}{\mathbf{C}}
  \providecommand{\mD}{\mathbf{D}}
  \providecommand{\mE}{\mathbf{E}}
  \providecommand{\mF}{\mathbf{F}}
  \providecommand{\mG}{\mathbf{G}}
  \providecommand{\mH}{\mathbf{H}}
  \providecommand{\mI}{\mathbf{I}}
  \providecommand{\mJ}{\mathbf{J}}
  \providecommand{\mK}{\mathbf{K}}
  \providecommand{\mL}{\mathbf{L}}
  \providecommand{\mM}{\mathbf{M}}
  \providecommand{\mN}{\mathbf{N}}
  \providecommand{\mO}{\mathbf{O}}
  \providecommand{\mP}{\mathbf{P}}
  \providecommand{\mQ}{\mathbf{Q}}
  \providecommand{\mR}{\mathbf{R}}
  \providecommand{\mS}{\mathbf{S}}
  \providecommand{\mT}{\mathbf{T}}
  \providecommand{\mU}{\mathbf{U}}
  \providecommand{\mV}{\mathbf{V}}
  \providecommand{\mW}{\mathbf{W}}
  \providecommand{\mX}{\mathbf{X}}
  \providecommand{\mY}{\mathbf{Y}}
  \providecommand{\mZ}{\mathbf{Z}}
  \providecommand{\mLambda}{\mathbf{\Lambda}}

  %
  \providecommand{\cA}{\mathcal{A}}
  \providecommand{\cB}{\mathcal{B}}
  \providecommand{\cC}{\mathcal{C}}
  \providecommand{\cD}{\mathcal{D}}
  \providecommand{\cE}{\mathcal{E}}
  \providecommand{\cF}{\mathcal{F}}
  \providecommand{\cG}{\mathcal{G}}
  \providecommand{\cH}{\mathcal{H}}
  \providecommand{\cJ}{\mathcal{J}}
  \providecommand{\cK}{\mathcal{K}}
  \providecommand{\cL}{\mathcal{L}}
  \providecommand{\cM}{\mathcal{M}}
  \providecommand{\cN}{\mathcal{N}}
  \providecommand{\cO}{\mathcal{O}}
  \providecommand{\cP}{\mathcal{P}}
  \providecommand{\cQ}{\mathcal{Q}}
  \providecommand{\cR}{\mathcal{R}}
  \providecommand{\cS}{\mathcal{S}}
  \providecommand{\cT}{\mathcal{T}}
  \providecommand{\cU}{\mathcal{U}}
  \providecommand{\cV}{\mathcal{V}}
  \providecommand{\cX}{\mathcal{X}}
  \providecommand{\cY}{\mathcal{Y}}
  \providecommand{\cW}{\mathcal{W}}
  \providecommand{\cZ}{\mathcal{Z}}
\providecommand{\bone}{\mathbbm{1}}
\def\leref#1{Lemma~\ref{#1}}
\def\figref#1{Fig.~\ref{#1}}
%
%
%
%
%
%
%
%
%
    %
    %
    %
    %
    %
    %
    %
    %
    %
    %
%


%
%
%

\newcommand{\zeman}[1]{{\color{cyan}[zeman: #1]}}
\newcommand{\meisam}[1]{{\textbf{\color{red}meisam: #1}}}
\newcommand{\yuan}[1]{{\textbf{\color{orange}yuan: #1}}}
\newtheorem{claim}{Claim}




%
%
\newcommand{\T}{\scriptscriptstyle T}
\def\maximize{\mathop{\text{maximize}}}
\def\minimize{\mathop{\text{minimize}}}
\def\deref#1{Definition~\ref{#1}}
\def\secref#1{Section~\ref{#1}}
\def\leref#1{Lemma~\ref{#1}}
\def\conref#1{Condition~\ref{#1}}
\def\thref#1{Theorem~\ref{#1}}
\def\remref#1{Remark~\ref{#1}}
\def\coref#1{Corollary~\ref{#1}}
\def\figref#1{Figure~\ref{#1}}
\def\figtab#1{Table~\ref{#1}}
\def\algref#1{Algorithm~\ref{#1}}
\def\appref#1{Appendix~\ref{#1}}
\def\asref#1{Assumption~\ref{#1}}
\def\bydef{\triangleq}
\def\pref#1{P\ref{#1}}

\DeclareMathOperator{\topk}{top}
\DeclareMathOperator{\randk}{rand}
%
\newcommand{\abs}[1]{\left\lvert #1\right\rvert}
\newcommand{\norm}[1]{\left\lVert #1\right\rVert}
\newcommand{\sqn}[1]{\left\lVert #1\right\rVert^2}
%
%
\newcommand{\lin}[1]{\ensuremath \left\langle #1 \right\rangle}

\newcommand{\ubar}[1]{\underaccent{\bar}{#1}}
\def\remark{\addtocounter{remark}{1}\def\@currentlabel{\theremark}%
\emph{Remark~\theremark}. } \makeatother
\newcounter{remark}
\newtheorem{condition}{Condition}
%
%
%
\newtheorem{property}{P}
 \usepackage{color-edits}
 \addauthor{mh}{red}
 \addauthor{xw}{magenta}
 \addauthor{ne}{brown}
 \newcommand{\brown}{\color{brown}}
\setlength{\textfloatsep}{5pt}
\setlength{\intextsep}{5pt}
\setlength{\abovedisplayskip}{3pt}
\setlength{\belowdisplayskip}{3pt}
\newcommand{\ols}[1]{\mskip.5\thinmuskip\overline{\mskip-.5\thinmuskip {#1} \mskip-.5\thinmuskip}\mskip.5\thinmuskip} %
\newcommand{\olsi}[1]{\,\overline{\!{#1}}} %
\newcommand{\CALL}[1]{\textbf{#1}}


 % put the rest of the preamble here
\usepackage{soul} % only for highlight purposes
\usepackage[a-2b,mathxmp]{pdfx}[2018/12/22]
\newcommand{\TODO}[1]{\textcolor{red}{#1}}
\def\ours{MuST\xspace}

\def\progss{ProGSS\xspace}

%%%%%%%%%%%%%%%%%%%%%%%%%%%%%%%%%%%%%%%%%%%%%%%%%%%%%%%%%%%%%%%%
% Spacing related commands.
%%%%%%%%%%%%%%%%%%%%%%%%%%%%%%%%%%%%%%%%%%%%%%%%%%%%%%%%%%%%%%%%
% Space between figure and caption
\setlength{\abovecaptionskip}{2pt}
\setlength{\belowcaptionskip}{2pt}

% Space between text and figs
\setlength{\dbltextfloatsep}{1.5pt plus .5pt minus .5pt}
\setlength{\textfloatsep}{.15pt plus .5pt minus .5pt}
\setlength{\intextsep}{1.5pt plus .5pt minus .5pt}

% Space between equations and text
\setlength{\belowdisplayskip}{1pt} \setlength{\belowdisplayshortskip}{1pt}
\setlength{\abovedisplayskip}{1pt} 
\setlength{\abovedisplayshortskip}{1pt}

% Paragraph formatting 
\setlength{\parskip}{1.5pt}


\newif\ifarxiv
\arxivfalse

\font\titlefont=ptmb at 15.95pt
\title{\titlefont
MuST: Multi-Head Skill Transformer for Long-Horizon Dexterous Manipulation with Skill Progress}
\author{Kai Gao$^{1,2}$\quad Fan Wang$^{1}$\quad Erica Aduh$^{1}$\quad Dylan Randle$^{1}$\quad Jane Shi$^{1}$
\thanks{$^{1}$Amazon Robotics, MA, USA. Email: {\tt\small { \{kaigaoar, fanwanf, aduheric, dylanran, janeshi\}}@amazon.com}.
}
\thanks{$^{2}$Department of Computer Science, Rutgers University, NJ, USA. Email: {\tt\small { \{kg627\}}@cs.rutgers.edu}. Work done during the internship at Amazon Robotics.
}
}
\begin{document}

\maketitle

% \section{Content List}
% \input{00-contents}
% \clearpage

\begin{abstract}
Robot picking and packing tasks require dexterous manipulation skills, such as rearranging objects to establish a good grasping pose, or placing and pushing items to achieve tight packing. These tasks are challenging for robots due to the complexity and variability of the required actions. To tackle the difficulty of learning and executing long-horizon tasks, we propose a novel framework called the Multi-Head Skill Transformer (MuST). This model is designed to learn and sequentially chain together multiple motion primitives (skills), enabling robots to perform complex sequences of actions effectively. MuST introduces a ``progress value'' for each skill, guiding the robot on which skill to execute next and ensuring smooth transitions between skills. Additionally, our model is capable of expanding its skill set and managing various sequences of sub-tasks efficiently. Extensive experiments in both simulated and real-world environments demonstrate that MuST significantly enhances the robot's ability to perform long-horizon dexterous manipulation tasks.
\end{abstract}

\section{Introduction}\label{sec:intro}
As robotics technology advances, the deployment of robots in everyday tasks is rapidly transitioning from a conceptual idea to a practical reality. Unlike traditional industrial robots, which are typically designed for repetitive, single-purpose tasks, the next generation of robots is expected to perform complex, dexterous manipulation tasks that require the seamless integration of multiple skills and the execution of diverse sub-tasks.

Recently, advances in policy learning~\cite{chi2023diffusion, zhao2023learning, team2024octo} have shown great promise by effectively learning from human demonstrations to address dexterous manipulation tasks that were notoriously difficult to design and program manually. These methods leverage the richness and variety of human demonstrations, capable of capturing multi-modal data, and many also utilize foundation models to improve generalization and learning efficiency~\cite{open_x_embodiment_rt_x_2023, brohan2023rt1roboticstransformerrealworld, brohan2023rt2visionlanguageactionmodelstransfer, ze20243ddiffusionpolicygeneralizable, yang2024equibotsim3equivariantdiffusionpolicy}. However, despite these advances, most systems are designed and tested to handle specific, single tasks and often fall short when faced with long-horizon tasks that require combining and sequencing multiple skills over extended periods~\cite{mandlekar2021learninggeneralizelonghorizontasks, duan2017oneshotimitationlearning}.

\begin{figure}[t]
    \centering
    \includegraphics[width=0.5\textwidth]{figures/prob.pdf}
   \includegraphics[width=0.5\textwidth]{figures/Intro.pdf}
    \caption{[Top] An example of long-horizon dexterous manipulation. The robot executes four skills to manipulate an object from the boundary of the picking tote to the corner of the packing tote. [Bottom] Our proposed imitation learning model \ours with N-skill and skill selector \progss.}
    \label{fig:intro}
\end{figure}

A key hypothesis we have to improve reliability of policy learning for long-horizon task is that many such tasks can be decomposed into multiple heterogeneous sub-tasks, where each sub-task requires a distinct skill, and these skills are highly reusable. This is particularly evident in warehouse robotics, where tasks such as picking and packing involve dinstinct skills like flipping, grasping, and pushing. For example, Fig.~\ref{fig:intro}[Top] shows a robot flipping an object from the boundary of a picking tote and compactly packing it in the corner of a packing tote. %In this paper, we refer to these skill modules interchangeably as sub-tasks or skills.

Decades of research have explored the problem of chaining multiple skills to achieve long-horizon tasks. Task and Motion Planning (TAMP) methods integrate high-level symbolic reasoning with low-level motion planning, enabling robots to sequence skills while ensuring physical feasibility~\cite{Dantam16, Srivastava14, Wolfe10}. Similarly, reinforcement learning (RL), particularly hierarchical RL, decomposes tasks into sub-skills, facilitating more efficient exploration in complex environments~\cite{kulkarni2016hierarchicaldeepreinforcementlearning, eysenbach2018diversityneedlearningskills, vezhnevets2017feudalnetworkshierarchicalreinforcement, bacon2016optioncriticarchitecture}. However, these methods still face challenges. For instance, RL methods struggle with exploration and scalability, as the search space for multi-skill tasks grows exponentially with complexity~\cite{nasiriany2022augmenting}. %Additionally, ensuring smooth transitions between skills is difficult, requiring precise coordination of states.

Instead, we propose a novel approach, \ours (Multi-Head Skill Transformer), designed to enhance the reliability of policy learning by primarily building upon the policy's existing structure, while introducing minimal additional complexity. \ours operates by decomposing long-horizon tasks into sequence of reusable skills. At each timestamp, the appropriate skill is selected based on the current observation and state, enabled by a robust progress estimator for each skill and a skill selector. 

Specifically, \ours extends the policy learning model Octo~\cite{team2024octo} by introducing multiple heads, with each head responsible for a specific skill, along with a progress estimator that tracks the progress of skill execution. A skill selector function, named \progss, maps the progress across all skill heads to determine the appropriate skill to execute at any given state. Both the skill heads and progress estimators are trained simultaneously using the same pre-trained Octo transformer backbone. With the multi-head structure, \ours allows for the training of multiple skills either synchronously or asynchronously, facilitating the integration of a large skill set and the addition of new skills as needed.

The main advantage of \ours is that it provides a clear understanding of each individual skill and the overall task progress through continuous progress estimation. It also offers flexibility in determining when to terminate a skill by setting a termination threshold. Additionally, \ours enhances reliability under disturbance, as the model continuously reasons when to skip or redo certain skills to ensure task completion.

Comprehensive experiments in both simulated and real-world environments demonstrate that \ours effectively addresses challenging long-horizon dexterous manipulation tasks, showing significant improvements over the Octo baseline models. In tasks involving flipping, picking, packing, and pushing, \ours increased the overall task completion rate from 32.5\% with the baseline Octo single policy to 90\% with \ours.


%\section{Introduction}\label{sec:intro}
%%!TEX root=main.tex

\section{Introduction}
% Decision-makers, analysts, data scientists, and policymakers frequently rely on data to draw conclusions and extract insights. This data-driven approach helps them identify actionable recommendations aimed at influencing an outcome of interest, such as increasing product satisfaction or income levels or decreasing the likelihood of experiencing serious health conditions \cite{galhotra2022hyper,lakkaraju2016interpretable,agrawal1994fast}. 
\revc{Prescriptions, or actionable recommendations, are commonly generated across various fields to influence key outcomes such as improving product satisfaction, enhancing economic policies, or increasing business efficiency. 
%Decision- or policy-makers, analysts, data scientists, and 
Policymakers in government, decision-makers in businesses, and data scientists in various fields, often rely on data-driven approaches to identify 
%actionable recommendations 
potential actions to influence an outcome of interest, such as increasing income levels or loan approval rates}.
% , or decreasing the likelihood of experiencing serious health conditions. 
%
While association or prediction-based methods are extensively used in practice to draw useful insights from data, they typically identify correlations among variables and may fail to reveal the underlying causal factors, i.e., which actions may result in an improved outcome, needed for informed decision-making. 
%For recommendations to be truly impactful, there must be a clear  explanation that justifies why a particular decision is appropriate for a specific subpopulation~\cite{sun2021treatment,plecko2022causal}. 

\emph{Causal analysis} or {\em causal inference}, therefore, is considered one of the most important requirements to generate prescriptions that are {\em actionable} and aligned with human reasoning~\cite{imbens2024causal}. Causal inference, and in particular {\em observational studies} for causal inference on collected data (when controlled trials are impossible due to cost or ethical reasons), have been extensively studied in the statistics and artificial intelligence (AI) literature for several decades \cite{rubin2005causal, pearl2009causal}. Motivated by this foundational work on causal inference, the notion of causality has also influenced the field of database research. The causal models from AI have been extended to relational databases \cite{salimi2020causal},  and causality has been incorporated into various data management tasks such as finding responsibilities of inputs toward query answers ~\cite{meliou2010causality, meliou2009so, meliou2014causality}, explanations for query answers \cite{roy2014formal, DBLP:journals/pacmmod/YoungmannCGR24}, data discovery~\cite{galhotra2023metam,youngmann2023causal}, data cleaning~\cite{pirhadi2024otclean,salimi2019interventional}, hypothetical reasoning \cite{galhotra2022causal}, and large system diagnostics~\cite{markakis2024sawmill,causalsim,sage, gudmundsdottir2017demonstration}. 


\revc{If-then rules are generally considered interpretable by humans~\cite{lakkaraju2016interpretable,guidotti2018local,van2021evaluating,pradhan2022interpretable,chen2018optimization}.
We give a concrete example of the difference between association and causation in generating prescriptions or recommended actions in the form of if-then rules below}:
\begin{example}	%
\label{example:ex1} {\bf Importance of causal prescriptions:}
Consider the Stack Overflow (SO) annual developer survey
\cite{stackoverflowreport}, where respondents from around the world answer
questions about their jobs and demographics. A sample of the dataset \reva{with a subset of the
attributes (there are 20 attributes)} is presented in \cref{tab:data}.
%
Alice, a researcher in the United Nations (UN) finance department, is interested in discovering ways to increase the salaries of high-tech employees worldwide. She is looking for a set of actionable recommendations 
%(that we call a prescription rules) 
to raise the overall average salary.
%
Using association-based approaches~\cite{chen2018optimization,lakkaraju2016interpretable}, she may discover that individuals residing in the US who identify as straight or heterosexual tend to earn higher salaries (see \cref{exp:quality} for full details). However, this observation merely indicates a correlation: people living in the US, for example, generally earn more than those outside the country. Their comparatively higher salaries are primarily attributable to the country's economy and are unrelated to their sexual orientation. Thus, this observation cannot be used as a prescription rule to increase salary. 
Our causal analysis, on the other hand, reveals that individuals aged 25-34 with dependents would benefit from working as front-end developers.
This results in a \$44,009 annual salary increase on average. \reva{Another observation is that students should pursue an
undergraduate major in CS. %Computer Science (CS). 
This can boost their salary by \$22,174 per year} (see details in \cref{sec:casestudy}).
\end{example}

%It has been incorporated into various tasks including . 
%Causal interventions are often more relatable and easier to understand, as they offer insight into the underlying reasons behind the recommendations and allow unraveling complex cause-effect relationships that govern our world~\cite{pearl2009causality}. Furthermore, causal interventions often have long-lasting effects~\cite{imbens2024causal}.

%, making it essential that the prescribed actions are not only actionable but also 

%causally consistent. 

%Decision makings, in particular, high-stak

\cut{
In this work, {we study the problem of generating causal insights (referred to as \emph{prescription rules}), which serve as actionable recommendations} to improve an outcome of interest.
Recent works have introduced causality to the field of database research~\cite{meliou2010causality,  meliou2014causality,salimi2020causal,10.14778/3554821.3554902}. It has been incorporated into various tasks including data discovery~\cite{galhotra2023metam,youngmann2023causal}, data cleaning~\cite{pirhadi2024otclean,salimi2019interventional}, and large system diagnostics~\cite{markakis2024sawmill,causalsim,sage, gudmundsdottir2017demonstration}. 
We propose using causal inference to generate prescription rules that are both actionable and justifiable.
}

While generating prescriptions based on causal inference may help in robust decision-making, causal prescriptions that solely consider the betterment of an outcome (like salary) are not enough in practice. 
It is well-known that decision-making in many high-stake applications (like hiring policy, or policy for approving loans by banks) may lead to disparate societal or economic impact on different sub-populations. 
As a shocking example from a recent work called 
%For example, 
CauSumX~\cite{DBLP:journals/pacmmod/YoungmannCGR24} that generates a set of causal explanations for an aggregated view, the explanations generated %by CauSumX %recommendations which 
suggest that male individuals do a Bachelor's degree to increase their salary while %suggesting that 
being an unmarried woman 
%the recommendation for women includes getting married 
has the most adverse effect on salary
(borrowed directly 
from Fig.~19 in~\cite{youngmann2024summarizedcausalexplanationsaggregate}). 
%We demonstrate the advantage of using causal reasoning to generate actionable recommendations and the limitations of not considering fairness requirements in the following example. 
We explored this further in the context of generating prescriptions and observed that prescriptions that are not fairness-aware can generate unfair outcomes to some subpopulations which we refer to as the {\em protected group}. Examples include women, Black, Latino, or Native Americans, individuals with a disability, countries with a weaker economy, or other protected groups specific to an application. %Here is a concrete example:


% Understanding the causal factors behind these recommendations is crucial to ensuring that decisions lead to fair and equitable outcomes, particularly in sensitive applications where biased decisions can perpetuate or even exacerbate societal inequalities.
% While prior work has extensively explored techniques for association rule mining~\cite{kumbhare2014overview}, recent efforts have focused on deriving causal explanations for individual data points or entire datasets~\cite{salimi2018bias,youngmann2022explaining,ma2023xinsight}. Although some of these methods produce causally consistent insights, the absence of fairness considerations in the process can lead to unfair outcomes, further reinforcing existing biases. For example, CauSumX~\cite{DBLP:journals/pacmmod/YoungmannCGR24} generates causal recommendation which suggest male individuals to do a Bachelor's degree to increase salary while the recommendation for women include getting married (borrowed directly from Figure~19 in the paper~\cite{youngmann2024summarizedcausalexplanationsaggregate}). 





%\emph{Causal inference} has been thoroughly studied in AI and Statistics~\cite{pearl2009causal,rubin2005causal}. Causal analysis is a vital tool in determining the effect of a \emph{treatment} on an \emph{outcome}, and has been used in decision-making in medicine \cite{robins2000marginal}, economics \cite{banerjee2011poor}, biology \cite{shipley2016cause}, and in high-stakes areas such as identifying the root causes of failures in critical infrastructure systems to prevent catastrophic outcomes. Recent works have introduced causality to the field of database research~\cite{meliou2010causality,  meliou2014causality,salimi2020causal,10.14778/3554821.3554902}. It has been incorporated into various tasks including data discovery~\cite{galhotra2023metam,youngmann2023causal}, query result explanation~\cite{salimi2018bias,youngmann2022explaining,DBLP:journals/pacmmod/YoungmannCGR24}, and large system diagnostics~\cite{markakis2024sawmill,causalsim,sage, gudmundsdottir2017demonstration}. We propose leveraging causal inference to generate interpretable and justifiable insights (referred to as \emph{prescription rules}), which serve as actionable recommendations to improve an outcome of interest. Causal reasoning is considered one of the most important requirements,  to generate insights that are actionable and aligned with human reasoning.




\begin{table*}[]
\footnotesize
    \centering
    	\caption{\textnormal{A subset of the Stack Overflow dataset.}}
         \label{tab:data}
    	% \vspace{-4mm}
  			\begin{tabular}[b]{|l|l|l|c|l|l|c|l|c|}
  			
				%\multicolumn{9}{c}{\textbf{Users}}\\ 
				\hline

				\textbf{ID}
    
    % \textbf{Country}& \textbf{Continent} 
    
    &\textbf{Gender} &\textbf{Ethnicity}&
				\textbf{Age} &\textbf{Role} &
				\textbf{Education} &\textbf{Country}&\textbf{Undergrad Major}&\textbf{Salary}
				\\ \hline

				1 &Male&White&26&Data Scientist & PhD& US&Computer Science&180k\\
    		2 &Non-binary&White&32&QA developer & Bachelor's degree& US&Mechanical Eng.&83k\\

 3 &Male&South Asian&29&C-suite executive  & Bachelor's degree & India&Computer Science&24k\\

  % 4 &Female&South Asian&25&Back-end developer  & Master's degree & India&Mathematics&7.5k\\

  4 &Female&East Asian&21&Back-end developer & Bachelor's degree & China&Computer Science&19k\\
  

        % $\ldots$ &  $\ldots$&  $\ldots$&  $\ldots$&  $\ldots$&  $\ldots$&  $\ldots$&  $\ldots$&  $\ldots$&  $\ldots$&  $\ldots$\\
    \hline
			\end{tabular}
            \vspace{-5mm}
\end{table*}




\begin{example}	%
\label{example:ex2}
{\bf Importance of fair prescriptions:}
Continuing Example~\ref{example:ex1}, while those causal prescription rules are highly beneficial for the overall population, they are considerably less effective for individuals residing in countries with a low GDP (indicating a weaker economy). For this group, the average expected increase in salary is only approximately \$13,000 per year (in contrast to \$44,009 for the entire group). % \sr{add which rule 44k or 25k} 
Consequently, implementing these rules would exacerbate the disparity between those living in countries with strong economies and those in countries with weaker economies.
\end{example}




% Our objective is to generate a small set of prescription rules aimed at increasing (or decreasing) an outcome of interest. This is framed as an optimization problem where the goal is to select the fewest prescription rules that maximize utility (i.e., the expected increase or decrease in the outcome). However, 

The example above shows that focusing solely on maximizing utility (\revc{i.e., increasing income}) can result in a scenario where only some of the population receive significant improvement, while others experience no benefit (\revc{only a small benefit for individuals from countries with weaker economies in our example}). Additionally, even if a large portion of the population receives recommendations, a protected subpopulation might not share the benefits and, worse, their situation could deteriorate, exacerbating inequalities.

Examples~\ref{example:ex1} and \ref{example:ex2} show that it is crucial to provide recommendations that are (1) {\em causal} for the outcome (beyond associations),  and (2) also {\em fair or equitable} in terms of the outcome for both the protected and non-protected groups. While recent work in database research
has focused on deriving {\em causal explanations} for individual data points, aggregated view, or entire datasets~\cite{salimi2018bias,youngmann2022explaining,ma2023xinsight, DBLP:journals/pacmmod/YoungmannCGR24}, and in particular \cite{DBLP:journals/pacmmod/YoungmannCGR24} has considered generating a set of causal explanations for an aggregated view that resemble a ruleset, 
%Although some of these methods produce causally consistent insights, 
the absence of fairness considerations in generating these causal explanations can lead to unfair outcomes for the protected group.
%further reinforcing existing biases.


%\red{We, therefore, enable users to incorporate various \emph{coverage and fairness constraints} along with the overall objective of improving an outcome of interest. }

\medskip
\noindent
\textbf{Our contributions.~} 
Motivated by the dual goals of generating causal and fair prescriptions for the betterment of an outcome, we introduce a {\em fairness-aware framework leveraging causal reasoning for generating a set of actionable prescription rules (ruleset)} called \sysName\ (\underline{Fair} \underline{CA}usal \underline{P}rescription).
%
Following research on fairness in data management~\cite{stoyanovich2020responsible,galhotra2022causal}, we assume the existence of a \emph{protected subpopulation}, defined by an attribute such as gender or race for people, or GDP of a country. Motivated by the causal explanation rules for an aggregated view \cite{DBLP:journals/pacmmod/YoungmannCGR24}, each prescription rule in our ruleset applies to a sub-population defined by a {\em grouping attribute}, and prescribes a {\em treatment or intervention} to improve the {\em outcome} for this sub-population. Fairness constraints ensure that the expected utility of the protected population is {\em comparable} to the utility of the unprotected individuals. We borrow the notions of \emph{group and individual fairness} from the fairness literature but tailor them for prescription rules. In addition to the fairness constraints, our coverage constraints ensure that a substantial fraction of the population and protected subpopulation receives at least one recommendation. 
%We demonstrate how such constraints ensure that the generated rules apply to a large portion of the population and ensure fairness through the following example.

\begin{example}
\label{ex:intro_example_3}
Continuing Examples~\ref{example:ex1} and \ref{example:ex2}, Alice uses our proposed system, called \sysName, to impose fairness and coverage constraints to discover useful and equitable recommendations for increasing salaries worldwide. In particular,
Alice chooses to implement a coverage constraint to ensure that the selected rules apply to a significant portion of people worldwide, including a sufficiently large number of individuals from countries with low GDP (the protected group). She also imposes a fairness constraint to ensure that the expected gains for both protected and non-protected groups are comparable.
\reva{She discovers, for example, that for individuals with 6-8 years of coding experience (a subpopulation comprising 21\% of the entire dataset and 25\% of the protected group), pursuing a bachelor’s degree in computer science will increase the expected salary by $\$14.9k$ for protected and by $\$17.8k$ for non-protected}. (See \cref{sec:casestudy} for more details.) This prescription rule applies to a large portion of the population and ensures fairness by providing a similar expected gain for both protected and non-protected groups, and the allowed difference of outcomes between these two populations may be adjusted by choosing appropriate thresholds in the fairness definitions. 
\end{example}


\noindent
Our main contributions are as follows. \\
%\begin{itemize}[leftmargin=*,topsep=0pt]
{\bf (1)} We {\bf develop a framework that generates a set of prescription rules to enhance an outcome of interest (Section~\ref{sec:problem})}. A prescription rule consists of a \emph{grouping pattern} and an \emph{intervention pattern}, representing the target subpopulation and the actionable recommendation for that group, respectively. The strength of the {\em conditional causal effect} (Section~\ref{sec:background-causal}) of this intervention on the subgroup is used to measure the expected utility of a rule. Our objective is to identify the smallest set of rules that maximizes overall expected utility. We refer to this problem as the {\em \probName} problem.
We adopt several notions of fairness (individual vs. group, statistical parity vs. bounded group loss) from the literature to define the {\bf fairness constraints} for our problem. In addition, {\bf coverage constraints} (for individual rules or for a group) ensure that the solution for the \probName\ problem is applied to a sufficient number of individuals and to minimize inequalities. We show NP-hardness for different variants of the problems and properties (matroid) useful in our algorithms. 
%We establish several definitions for group and individual fairness constraints tailored for prescription rules.
\smallskip
    \par
    \noindent
{\bf (2)} We {\bf develop a general three-step algorithm named \sysName to solve the optimization problem of selecting a fair prescription ruleset (Section~\ref{sec:algo})}. The first step involves mining frequent grouping patterns using the Apriori algorithm~\cite{agrawal1994fast}. In the second step, we employ a lattice-based algorithm to find high utility and fair intervention patterns for grouping patterns identified in the previous step. Finally, the third step applies a greedy approach to determine a solution. \sysName\ can be easily adapted to accommodate all variants of the \probName\ problem.

\smallskip
\par
\noindent
{\bf (3) We provide a detailed  case study  (Section~\ref{sec:casestudy}) and experimental analysis (Section~\ref{sec:experiments}) to evaluate our framework and algorithms.}
The case study shows the qualitative difference of different variants of our problem for different choices of the fairness and coverage constraints. The experiments include two datasets, three baselines, and 18 variations of our problem with different constraints. Our evaluations suggest that fairness may come at the cost of expected
utility for everyone. However, without fairness constraints, we often observe a significant disparity between the protected and non-protected. We also observe that
achieving individual fairness is harder than group fairness,
as most high-utility or high-coverage rules are unfair. Lastly, we show that \sysName\ can generate  prescription rules over large datasets in a reasonable time. 

%\end{itemize}


%\paragraph*{Paper outline} 
We discuss related work in \cref{sec:related}, review background on causal inference in \Cref{sec:background-causal}, %and our problem formulation can be found in \cref{sec:problem}. Our algorithmic framework is presented in \cref{sec:algo}. A case study demonstrating the impact of different constraint configurations on the solution is given in \cref{exp:problem_variants}, and our experimental evaluation is detailed in \cref{sec:experiments}. Finally, we 
and discuss the limitations of our framework and future work in \cref{sec:conc}.

% \noindent
% \boxed{\parbox{\columnwidth}{$\bullet$ 
% For people with a professional degree, move to the United Kingdom
%  (coverage = 435 (20), coverage-protected = 20 (13), utility = 186855, utility-protected = 0.)\\
% $\bullet$ For graphic developers, move to the	United States
%  (coverage = 116 (29), coverage-protected = 8 (2), utility = 169431, utility-protected = 0).\\
% $\bullet$ For people who have no formal education, move to the United States
%  (coverage = 123 (34), coverage-protected = 7 (2), utility = 206742, utility-protected = 0).\\
% % \textcolor{red}{size = 38, length = 76, overlap = 64029181, utility = 1659307}\\
% \textcolor{blue}{overall coverage =674, expected utility = 187485
% coverage-protected = 35, expected utility-protected = 0}
% \sr{should mention protected group, and possibly not mention coverage in the intro or just intuitively like high coverage}
% }}


% Alice notes that although these rules result in a \$187,485 increase in the overall salary for those to whom they apply, they only affect a small fraction of the population, specifically 674 individuals. Additionally, although the expected salary increase is substantial, there is no expected increase in salary for non-males, a subpopulation of particular interest to Alice. In other words, applying these rules would result in no gain for non-males.
% \end{example}

% \begin{example}[Episode 2 - coverage and fairness constraints]
% Alice introduces coverage and fairness constraints to ensure that enough people will benefit from the rules and that they will be \emph{fair} with respect to non-males. Specifically, she demands that the benefit for a randomly chosen individual to whom one of the rules applies is nearly the same as the benefit for a randomly chosen individual who identifies as non-male and to whom one of the rules applies.

% After adding these constraints, \sysName\ recommends the following set of prescription rules:



% \noindent
% \boxed{\parbox{\columnwidth}{$\bullet$ 
% For people who have no formal education, move to the United States
%  (coverage = 123 (34), coverage-protected = 7 (2), utility = 206742, utility-protected = 0)\\
% $\bullet$ 
% For females, change role to	DevOps specialist (coverage = 2256 (47), coverage-protected = 2256 (47), utility = 90023, utility-protected = 90023).\\
% $\bullet$ For people with a Master's degree, move to the	United States
%  (coverage = 9097 (2222), coverage-protected = 642 (236), utility = 85390, utility-protected = 84201).\\
% % \textcolor{red}{size = 38, length = 76, overlap = 64029181, utility = 1659307}\\
% \textcolor{blue}{overall coverage =11476	
% , expected utility = 87601,
% coverage-protected = 2905, expected utility-protected = 88519}
% }} 







% \begin{figure}[t]
%         \centering
%         \begin{minipage}[b]{1.0\linewidth}
%             \small
%             \begin{tcolorbox}[colback=white]
%             \vspace{-2mm}
% $\bullet$ For backend developers, the treatment with the highest effect on salary is “Country = US” effect size = 78646
% \begin{itemize}
%     \item For non-male the effect is only: 59429
%     \item For male the effect is 80454
% \end{itemize}

% $\bullet$ For frontend developers, the treatment with the highest effect is :Formal Education = Bachelor's degree” effect size: 17340
% \begin{itemize}
%     \item For white the effect is 33464
%     \item For non-white the effect is 15320
% \end{itemize}


% $\bullet$ For people in Europe, the treatment with the highest effect on salary is “DevType = C-suite executive” effect size = 53254
% \begin{itemize}
%     \item For white the effect is 55112
%     \item For non-white 35249
% \end{itemize}



%             \vspace{-2mm}
%             \end{tcolorbox}
%         \end{minipage}%%
%          % \vspace{-4mm}
%         \caption{Set of prescription rules.}
%         \label{fig:so-explanation}
%     \end{figure}




\section{Related Works}\label{sec:related}
\section{Background and related work}
% 重点看Artistic data visualization: Beyond visual analytics 和Visualization criticism-the missing link between information visualization and art 的被引


This section reviews the background on artistic data visualization and previous research related to this topic.

\subsection{Artistic Data Visualization in Art History Context}
\label{ssec:contemporary}

Art history has been marked by transformative periods characterized by different aesthetic pursuits, materials, and methods. Since the time of Plato, imitation (or \textit{mimesis}, which views art as a mirror to the world around us) has been an important pursuit~\cite{pooke2021art}. Successful artworks, such as Greek sculptures and the convincing illusions of depth and space in Renaissance paintings, exemplify this tradition.
The advent of modern society and new technology, especially photography, posed a significant challenge to the notion of art as imitation~\cite{perry2004themes}. By the 1850s, modern art began to emerge with the core goal of transcending traditional forms and conventions. Movements like Post Impressionism, Expressionism, and Cubism revolutionized art through innovative uses of form (\eg color, texture, composition), moving art towards abstraction and experimentation. 
After World War II, the Cold War and the surge of various social problems heightened skepticism about the progress narrative of modernity and the superiority of the capitalist system, leading to the rise of postmodernism and the birth of contemporary art~\cite{hopkins2000after,harrison1992art}. One prominent feature of contemporary art is the absence of fixed standards or genres historically defined by the church or the academy. Postmodern design neither defines a cohesive set of aesthetic values nor restricts the range of media used~\cite{pooke2021art}, sparking novel practices such as installations, performances, lens-based media, videos, and land-based art~\cite{hopkins2000after}.
Meanwhile, artists have increasingly drawn energy from various philosophical and critical theories such as gender studies, psychoanalysis, Marxism, and post-structuralism~\cite{pooke2021art}. As a result, as described by Foster~\cite{foster1999recodings}, artists have increasingly become ``manipulators of signs and symbols... and the viewer an active reader of messages rather than a passive contemplator of the aesthetic''. Hopkins~\cite{hopkins2000after} described this shift as the ``death of the object'' and ``the move to conceptualism''. 
% Joseph Kosuth, one of the most important representatives of conceptual artists, also once said that “all art (after Duchamp) is conceptual (in nature) because art only exists conceptually”
% As argued by Danto~\cite{danto2015after}, traditional notions of aesthetics can no longer apply to contemporary art. ``“All there is at the end,” Danto wrote, “is theory, art having finally become vaporized in a dazzle of pure thought about itself, and remaining, as it were, solely as the object of its own theoretical consciousness.''
% The Anti-aesthetic (1983) has described these as ‘anti-aesthetic’ strategies – typified, as we have seen, by a conceptually driven approach to the art object and to the process of its production.

Emerging within the contemporary art historical context, data art has been significantly propelled by the advent of big data. An early milestone was Kynaston McShine's 1970 exhibition \textit{Information} at the Museum of Modern Art (MoMA). 
% In the exhibition catalogue, McShine wrote~\cite{information_moma}: ``Increasingly artists use mail, telegrams, telex machines, etc., for transmission of works themselves—photographs, films, documents—or of information about their activity.'' 
% The millennium era has witnessed substantial growth in this area.
In 2008, Google’s Data Arts Team was founded to explore the advancement of what creativity and technology can do together~\cite{google}.
% data artist Aaron Koblin
In 2012, Viégas and Wattenberg's \textit{Wind Map}, an artwork that visualizes real-time air movement, became the first web-based artwork to be included in MoMA's permanent collection~\cite{wind}.
Since 2013, the academic conference IEEE VIS has included an Arts Program (IEEE VISAP), showcasing artistic data visualizations through accepted papers and curated exhibitions. 
As noted by Barabási~\cite{dataism} (a Fellow of the American Physical Society and the head of a data art lab), data has become a vital medium for artists to deal with the complexities of our society: ``Humanity is facing a complexity explosion. We are confronted with too much data for any of us to make sense of...The traditional tools and mediums of art, be they canvas or chisel, are woefully inadequate for this task...today’s and tomorrow’s artists can embrace new tools and mediums that scale to the challenge, ensuring that their practice can continue to reflect our changing epistemology.''
% a physicist and head of a data art lab, has noted, 

% Artists are now exploring new mediums and methods, incorporating datasets, computer technology, and algorithms into their work.



\subsection{Research on Artistic Data Visualization}
\label{ssec:artisticvis}

Artistic data visualization is also referred to as artistic visualization, data art, or information art~\cite{holmquist2003informative,rodgers2011exploring,few,viegas2007artistic}. Despite the varying terminologies, there is a basic consensus that artistic data visualization must be art practice grounded in real data~\cite{viegas2007artistic}.
Since the early 2000s, a series of papers introduced innovative artistic systems for visualizing everyday data, such as museum visit routes and bus schedule information~\cite{skog2003between,holmquist2003informative,viegas2004artifacts}.
In 2007, Viégas and Wattenberg~\cite{viegas2007artistic} explicitly proposed the concept of \textit{artistic data visualization} and viewed it as a promising domain beyond visual analytics.
% and defined it as ``visualization of data done by artists with the intent of making art''. 
Kosara~\cite{kosara2007visualization} drew a spectrum of visualization design, positioning artistic visualization and pragmatic visualization at opposite ends of this spectrum to demonstrate that the goals of these two types of design often diverge. 
% advocating that analytical visualizations prioritize practicality, while artistic data visualizations emphasize sublime quality, that is, the capacity to inspire awe and grandeur and elicit profound emotional or intellectual responses. 
% In 2011, Rodgers and Bartram~\cite{rodgers2011exploring} utilized artistic data visualization to enhance residential energy use feedback. 
However, overall, research on this subject has been sparse. Among those relevant papers, most have focused on specific applications of artistic data visualization. 
%~\cite{rodgers2011exploring,schroeder2015visualization,perovich2020chemicals}
For instance, Rodgers and Bartram~\cite{rodgers2011exploring} utilized ambient artistic data visualization to enhance residential energy use feedback. Samsel~\etal~\cite{samsel2018art} transferred artistic styles from paintings into scientific visualization.
Artistic practice has also been observed in fields such as data physicalization~\cite{hornecker2023design,perovich2020chemicals,offenhuber2019data} and sonification~\cite{enge2024open}. For example, Hornecker~\etal~\cite{hornecker2023design} found that many artists are practicing transforming data into tangible artifacts or installations. Enge~\etal~\cite{enge2024open} discussed a set of representative artistic cases that combine sonification and visualization.
% dragicevic2020data
% Offenhuber~\cite{offenhuber2019data} created a set of artworks in urban settings that transform air quality data into situated displays, allowing people to encounter visualizations in their daily lives.

% But in contrast, empirical studies that describe the characteristics of artistic visualization and how they are created are extremely scarce. This scarcity forms a stark contrast to the increasingly rich and diverse practices by artists in the field.
% As for the difference between artistic data visualization and traditional visualizations for analytics, Vi{\'e}gas and Wattenberg~\cite{viegas2007artistic} thought that the most salient feature of artistic data visualizations is their forceful expression of viewpoints.
% In Ramirez~\cite{ramirez2008information}'s opinion, functional information visualizations are concerned with usability and performance while aesthetic information visualizations are concerned with visually attractive forms of representation design.
% Donath~\etal~\cite{donath2010data} presented a series of tools developed to integrate artistic expressions in generating unique and customized visualizations based on users' personal data, such as health monitoring data, album records, and e-mail records. 

On the other hand, some studies, while not focusing on artistic data visualization, have explored a key art-related concept: aesthetics. 
% ~\cite{moere2012evaluating,cawthon2007effect,lau2007towards} explored the aesthetics of visualization design in their research. They
For example, Moere~\etal~\cite{moere2012evaluating} compared analytical, magazine, and artistic visualization styles, noting that analytical styles enhance the discovery of analytical insights, while the other two induce more meaning-based insights. Cawthon~\etal~\cite{cawthon2007effect} asked participants to rank eleven visualization types on an aesthetic scale from ``ugly'' to ``beautiful'', finding that some visualizations (\eg sunburst) were perceived as more beautiful than others (\eg beam trees).
% Moere~\etal~\cite{moere2012evaluating} displayed data in three different styles (analytical style, magazine style, artistic style) and found that these styles led to different perceptions of usability and types of insights.
% More importantly, the authors found that the sunburst chart ranks the highest in aesthetics and is one of the top-performing visualizations in both efficiency and effectiveness, thus exemplifying the notion that "beautiful is indeed usable".
Factors such as embellishment~\cite{bateman2010useful}, colorfulness~\cite{harrison2015infographic}, and interaction~\cite{stoll2024investigating} have also been found to influence aesthetics. 
% borkin2013makes,tanahashi2012design
However, most of these studies have simplified aesthetics to hedonic features (\eg beauty), without delving into the nuanced connotations of aesthetics.
% most of these studies have simplified aesthetics to concepts like 'beauty,' 'preference,' or 'pleasing,' without exploring the deeper essence of aesthetics as the core of art.

The value of artistic data visualization to the visualization community is still in controversy. For instance, Few~\cite{few} argued for a clearer distinction between data art and data visualization; he highlighted the negative consequences when data art ``masquerades as data visualization'', such as making visualizations difficult to perceive. Willers~\cite{willers2014show} thought the artistic approach is ``unlikely be appreciated if the intention was for rapid decision making.''
% In an interview, American artist and technologist Harris commented that ``data can be made pretty by design, but this is a superficial prettiness, like a boring woman wearing too much makeup.''~\cite{harris2015beauty} 
To address these gaps, more empirical research needs to be conducted to explore how artistic data visualizations are designed, their underlying pursuits, and how they might inspire our community.




% Examining this field not only helps us understand the latest application of data visualization in various domains but also extends our understanding of the aesthetic and humanistic aspects of data visualization.
% there should be more theoretical investigation into artistic data visualization. 

% These three concepts emphasize placing or installing visualizations at physical places that people will encounter in their daily lives. 

% ~\cite{wang2019emotional}


% gap between art and science~\cite{judelman2004aesthetics}
% constructive visualization~\cite{huron2014constructive}
% data feminism~\cite{d2020data}
% critical infovis~\cite{dork2013critical}
% citizen data and participation~\cite{valkanova2015public}

% \x{Lee~\etal~\cite{lee2013sketchstory}, give users artistic freedom to create their own visualizations.}


% Aesthetics refers to the study of beauty, taste, and sensory perception and is deeply intertwined with art.
% a particular taste for or approach to what is pleasing to the senses and especially sight

% why shouldn't all charts be scatter plot~\cite{bertini2020shouldn}
% aesthetic model~\cite{lau2007towards}
% Aesthetics for Communicative Visualization : a Brief Review
% Stacked graphs--geometry \& aesthetics~\cite{byron2008stacked}
% storyline optimization~\cite{tanahashi2012design}
% graphic designers rate the attractiveness of non-standard and pictorial visualizations higher than standard and abstract ones, whereas the opposite is true for laypeople.~\cite{quispel2014would}
% evaluate aesthetics - wordcloud
% An Evaluation of Semantically Grouped Word Cloud Designs, tag cloud, wordle

% On the other hand, empirical studies conducted with designers have shown that functionality is not the only design goal of visualization. For example, Bigelow~\etal~\cite{bigelow2014reflections} found that designers would frequently fine-tune the non-data elements in their designs, and data encoding was even "a later consideration with respect to other visual elements of the infographic".
% Moere~\cite{moere2011role} - design
% Quispel~\etal~\cite{quispel2018aesthetics} found that for information designers, clarity and aesthetics are both important for making a design attractive.

\section{Problem Formulation}\label{sec:prob}
Let the long-horizon task be represented as a sequence of \emph{skills}, denoted by a finite set \( \mathcal{S} = \{s_1, s_2, \dots, s_N\} \), where \( N \) is the total number of skills. The task execution is governed by a policy \( \pi \), which, at each timestep \( t \), takes as input the observation \( o_t \in \mathcal{O} \) and the current robot state \( x_t \in \mathcal{X} \), and outputs the action \( P_t \in \mathcal{A} \), which will be further detailed below for our problem setting.

\subsection{Single Policy Learning}

The observation \( o_t \) is a representation of the environment at time \( t \), which could include sensor readings or environmental context. The robot state \( x_t \) includes the joint configurations, velocities, and other internal variables defining the robot's configuration.

The policy \( \pi \) learns to map:
\[
\pi : (o_t, x_t) \to P_t
\]
where \( P_t \in \mathcal{A} \) is the predicted action. This action consists of two components: the robot’s 6D pose \( p_t = (p_t^x, p_t^y, p_t^z, \theta_t^x, \theta_t^y, \theta_t^z) \in \mathbb{R}^6 \) in Cartesian space and orientation, and a discrete value \( u_t \in \{-1, 0 , 1\} \) representing whether the suction is turned on (\( u_t = 1 \)), turned off (\( u_t = -1 \)), or remains in its current state (\( u_t = 0 \)).

Thus, the action \( P_t \) predicted by the policy can be expressed as:
$P_t = (p_t, u_t)$, 
where \( p_t \) is the 6D pose and \( u_t \) is the suction indicator.

\subsection{\ours: Decomposition of Skills and Progress}

Unlike learning the long-horizon task in one policy, our method, \ours, decomposes the task into skill-specific predictions. For each skill \( s_i \in \mathcal{S} \), we predict not only the robot’s action \( P_t^{(i)} \in \mathcal{A} \), but also a progress value \( \rho_t^{(i)} \in [0, 1] \) that indicates how much of the skill \( s_i \) has been completed at time \( t \). The progress value evolves over time and reaches \( \rho_t^{(i)} = \theta_i \), where \( \theta_i \) is the termination threshold when a skill is considered fully executed.

Thus, for each skill \( s_i \), MuST outputs a tuple:
$(P_t^{(i)}, \rho_t^{(i)})$
at each timestep \( t \), where \( P_t^{(i)} = (p_t^{(i)}, u_t^{(i)}) \) is the predicted action for skill \( s_i \), and \( \rho_t^{(i)} \) is the progress indicator.

\subsection{Skill Selection and Execution (\progss)}

At each timestep \( t \), a skill selector function \( \sigma \) determines the next skill to execute based on the progress values \( \rho_t^{(i)} \) for all skills:
\[
\sigma: \{\rho_t^{(1)}, \rho_t^{(2)}, \dots, \rho_t^{(N)}\} \to s_j
\]
where \( s_j \in \mathcal{S} \) is the selected skill to be executed at time \( t \).

The robot then executes the action \( P_t^{(j)} \) predicted by MuST for the selected skill \( s_j \):
$a_t = P_t^{(j)}$,
where \( a_t \in \mathcal{A} \) represents the robot’s pose and suction status to execute at timestep \( t \).

\subsection{Goal-Conditioned Policy Learning}

In both the single policy learning and MuST approaches, an alternative model variant can be used where the input also encodes a goal condition. This goal condition can be represented as either a goal image \( I_g \in \mathcal{I} \) or a goal language instruction \( l_g \in \mathcal{L} \), where \( \mathcal{I} \) is the set of possible goal images and \( \mathcal{L} \) is the set of possible language instructions.

The policy can then be learned as a mapping that includes the goal condition:
\[
\pi : (o_t, x_t, I_g \text{ or } l_g) \to P_t
\]
where the goal condition, whether in the form of a goal image \( I_g \) or a language instruction \( l_g \), provides additional information to the policy for determining the action \( P_t \).

In this variant, the policy outputs \( P_t = (p_t, u_t) \), as described earlier, but the decision process is now conditioned on the provided goal.




%\section{Preliminaries}\label{sec:prob}
%\subsection{Problem Formulation: Goal-State Conditioned Manipulation}
%\todo{Jane: I think we need to start with Imitation learning as in Hydra paper, given demo dataset of observation and actions }
\todo{Jane will work on this section - will propose to use notations at three levels per slack message}
Consider a robot manipulating objects in a bounded workspace $\mathcal W$.
Denote the state space of $\mathcal{W}$ by $S$, each state $s\in S$ includes the robot configuration and the poses of all movable objects in $\mathcal W$.
We define our action space as $\mathcal A\subseteq SO(3)\times\{-1,0,1\}$, which includes an end-effector pose in $SO(3)$ and a relative signal of the suction cup with $-1/0/1$ for deactivation/idling/activation respectively. 
For the manipulation task, the robot is given an expert demonstration dataset $D:=\{\tau_i\}_{i=1}^{n}$. 
Each $\tau_i \in D$ is a sequence of state-action pairs $\{(s_j, a_j)\}_{j=1}^{|\tau_i|}$.
Given the termination conditions of the manipulation task, we denote the goal states of the manipulation task as $S_G\subset S$. 
Correspondingly, $D^{S_G} \subseteq D$ is a subset of demonstrations, in which the demonstration sequences terminate in $S_G$, i.e., $\tau[-1]\in S_G \times \mathcal A, \ \forall \tau \in D^{S_G}$.

Inside the workspace, there is skill set $\Sigma:=\{\sigma_i\}_{i=1}^{N}$.
Each skill $\sigma_i:=\{a^i_{j}\}\subset \mathcal A$ is an action set representing a manipulation primitive in $\mathcal W$. 
Let $D^{S_G}_i$ be the demonstration dataset of $\sigma_i$ in $D^{S_G}$.
Each $\tau_i\in D^{S_G}_i$ is a segment of a demonstration in $D^{S_G}$ and all the actions of $\tau_i$ are in $\sigma_i$.

We present the current state $s$ and the goal states $S_G$ as observations $\mathcal O(s)$ and $\mathcal I(S_G)$, where $\mathcal O$ and $\mathcal I$ map $s$ and $S_G$ to multi-modal data, such as RGBD images, proprioceptive data, and/or natural language.
The robot learns a set of policies $\Pi :=\{\pi_i\}_{i=1}^{N}$ for the skill set $\Sigma$.
Each policy $\pi_i:\mathcal O \times \mathcal I \rightarrow \mathcal A$ learns $\sigma_i$ with dataset $D^{S_G}_i$ minimizing an supervised loss 
$$\mathcal L:=\mathbb{E}_{(s,a)\sim p_{D^{S_G}_i}} d(a,\pi_i(\mathcal O(s), \mathcal I(S_G)))$$ 
where $d$ is a distance metric in $\mathcal A$.

During roll-outs of $\pi_i$, at each time step $t$, the next state is obtained by $s_{t+1}=\mathcal T(s_t, a_t)$, where $\mathcal{T}$ is the state transition model in $\mathcal W$ and $a_t=\pi_i(\mathcal O(s_t), \mathcal I(S_G))$ is the computed action of $\pi_i$.

To complete the manipulation task, a skill selector $\Phi:\mathcal{O}\times \mathcal{I} \rightarrow \Pi$ chooses a policy to execute until $S_G$ is reached.

\subsection{Action Space with Dilated Relative Suction}
\todo{Jane: Should we include more than Suction?}
In the action space $\mathcal A$, we use relative representation of suction activity, since it works better in pick and pack scenarios, where the suction cup only activates in the picking tote and deactivates in the packing tote. 
In contrast, the absolute representation has mixed signals in both totes, which results in 
occasional deactivation at the picking tote.
The biggest weakness of the relative representation is the sparsity in each episode\cite{team2024octo}. 
To address this issue, we extend the non-zero entries by applying k-neighborhood dilation in demonstration episodes, which sets the k-neighborhood elements non-zero as well.




\section{Methodology}\label{sec:method}
\begin{figure*}[hbt]
  \centering
  \includegraphics[width=\textwidth]{figs/pipeline.pdf} 
  \caption{Overview of the framework.}
  \label{fig:overview}
\end{figure*}

\section{Method}
\subsection{Overview}
Our method, \textsc{Drangen3D}, takes an image as input and generate a 3D object represented by 3D Guassians with multi-view geometric consistency, allowing user interaction of editing the geometry during the process. As illustrated in Fig.~\ref{fig:overview}, we first train an Anchor-Gaussian (Anchor-GS) VAE that encodes complex 3D information into a latent space and decodes it into 3DGS, enabling subsequent generation in the latent space (Sec.~\ref{sec:anchor-vae}).  
%
Then, we propose Seed-Point-Driven Controllable Generation module for 3D generation from a single image. This module starts with the generation of the rough initial geometry represented by a set of sparse surface points, named seed points, where we can apply the editing by deforming the seed points. After that, a mapping module is designed to map the (edited) seed point information to the latent space, which can be decode to 3DGS subsequently (Sec.~\ref{sec:seed-point-driven}). 

% \todo{Note that we do not directly generate anchor points because: computational complex; easy to learn; generated anchor points contains noise that affect the final geoemtry}


% As illustrated in Fig.~\ref{fig:overview}, the framework of \textsc{Dragen3D} mainly consists of two parts. We first adopt an anchor-based approach to obtain 3D Gaussians and propose the Anchor-GS VAE that constructs 3D Gaussians by leveraging points sampled from 3D assets and rendered images.
% %
% Through the anchor-based representation, we can compress both geometric and texture information into a set of anchor latents, which is beneficial for latent generation, as described in Sec.~\ref{sec:anchor-vae}

% Then we generate a set of sparse seed points to control the anchor latents and thus deform the final output 3DGS, designing and utilizing the Seed-Anchor Mapping module, as described in Sec. \ref{sec:seed-point-driven}.

% Overall, combining anchor-based 3DGS VAE and seed-point-driven generation approach gives large flexibility in geometric control and deformation during the 3D generation process.

\subsection{Background}
\paragraph{Gaussian Splatting}
% 3D Gaussian Splatting represents 3D objects explicitly through radiance-based methods, leveraging high-degree shape and color features for multi-view synthesis. Its differentiable volume rendering properties make it suitable for image-to-3D object generation, allowing for the alignment of conditioned image textures. This generation process $G$ can be outlined as follows:
% \begin{equation}
%     G:\{C\}_{x',y'} \rightarrow {\{\mu,o,A,F\}}_p,
% \end{equation}
% where $\{C\}_{x',y'}$ is pixels of the input at $(w, h)$. 
% % \mc{need to use different symbols to describe pixal and gaussian positions}.
% $\mu$, $o$, $A$, and $F$ denote the position, opacity, covariance matrix, and spherical harmonic features of each Gaussian particle $p$, respectively. The discretized splatting rendering renders at $\{C\}_{x',y'}$:
% \begin{equation}
%     \displaystyle\sum_{i \in \mathcal{N}}  \alpha_i \boldsymbol{SH}(r|_{x',y'};F_i)                      \displaystyle\prod_{j=1}^{i-1} (1-\alpha_j )\rightarrow \{C\}_{x',y'},
% \end{equation}
% where $\alpha_i$ is the product of $o_i$ and the projected Gaussian density of where the kernel interacts with the ray in direction $r|_{x',y'}$ from the specific pixel at $(x',y')$, and $\boldsymbol{SH}$ denotes the color calculated with features $F_i$. 
% This is equivalent to computing a depth-and-opacity weighted average color of particles along the ray direction. Consequently, individual Gaussian points possess adequate geometric and chromatic information.
% \subsection{3DGS}
% gaussian splatting将静态场景表示为一组各向异性的3d gaussians,每个像素的颜色通过基于点的alpha混合渲染来获得,从而实现高保真度的实时新视角合成。
Gaussian splatting represents scenes as a collection of anisotropic 3D Gaussians. Each Gaussian primitive $\mathcal{G}_i$ is parameterized by a center $\mu \in \mathbb{R}^3$, opacity $\alpha \in \mathbb{R}$, color $c \in \mathbb{R}^{3(n+1)^2}$ which is represented by n-degree SH coefficients and 3D covariance matrix $\Sigma \in \mathbb{R}^{3 \times 3}$,which can be represented by scaling $s\in \mathbb{R}^3$ and rotation $r\in \mathbb{R}^4$.
% \begin{equation}
%   \mathcal{G}(x) = e^{-\frac{1}{2}(x-\mu)^T\Sigma^{-1}(x-\mu)}
%   \label{3dgs_define}
% \end{equation}

% 为了保持协方差矩阵的物理意义,它必须是半正定的。因此,将协方差矩阵分解为一个缩放矩阵S和一个旋转矩阵R,其中S和R分别使用一个缩放向量和一个四元数旋转向量来表示。
% To maintain the physical meaning of the covariance matrix, it must be positive semi-definite.Therefore, the covariance matrix $\Sigma$ can be decomposed into a scaling matrix $S$ and a rotation matrix $R$:
% \begin{equation}
%   \Sigma = RSS^T R^T
%   \label{3dgs_decomposed}
% \end{equation}

% 渲染时首先将3D gaussian投影到2D空间。给定视角变换W,可以计算得到2D协方差矩阵,其中J是the Jacobian of the affine approximation of the projective transformation.随后基于深度对覆盖一个像素的高斯进行排序,使用基于点的alpha混合渲染得到像素的颜色。
During rendering, the 3D Gaussian is first projected onto 2D space. Given a view transformation matrix $W$, the 2D covariance matrix $\Sigma'$ can be computed as :
$\Sigma' = JW\Sigma W^T J^T$, where $J$ is the Jacobian of the affine approximation of the projective transformation. Subsequently, the Gaussians covering a pixel are sorted based on depth. The color of the pixel is obtained using point-based alpha blending rendering:
\begin{equation}
  c = \sum_{i=1}^n c_i \alpha_i \prod_{j=1}^{i-1}(1-\alpha_i)
  \label{3dgs_render}
\end{equation}

\paragraph{Rectified Flow Model}
% Recitified flow modle有建立两个分布\pi_0, \pi_1之间mapping的能力,所以很适合我们的任务。给定x_0 ~ \pi_0 和 相对应的x_1 ~ \pi_1, 我们可以通过liner interpolation  得到 x(t) = (1-t) x_0 + t x_1 at timestamp t. And a vector filed v_sita parameterized by a neural network 被用来drive the flow from source distribution \pi_0 to target distribution \pi_1 by minimizing the conditional flow matching objective:
The Rectified Flow Model \cite{liu2022flow, lipman2022flow} has the capability to establish a mapping between two distributions, \( \pi_0 \) and \( \pi_1 \), making it well-suited for our task of mapping seed point latents to anchor latents. Given \( x_0 \sim \pi_0 \) and the corresponding \( x_1 \sim \pi_1 \), we can obtain \( x(t) = (1 - t) x_0 + t x_1 \) at timestamp \( t \in [0,1]\) through linear interpolation. A vector field \( v_{\theta} \) parameterized by a neural network is used to drive the flow from the source distribution \( \pi_0 \) to the target distribution \( \pi_1 \) by minimizing the conditional flow matching objective:
\begin{equation}
    L(\theta) = E_{t,x_0,x_1,y}||v_{\theta}(x_t, t,y) - (x_1 - x_0)||
    \label{eq:flow matching}
\end{equation}
Here, $v_{\theta}(x_t, t, y)$ is the predicted flow at time $t$ for a given point $x_t$, $y$ refers to the image  condition that guides the flow matching.


\section{Evaluation}\label{sec:experiments}
In this section, we evaluate the performance of \ours in both simulated and real-robot environments, with a total of six experiments, under the various conditions:
\begin{itemize}
    \item Performance gain of MuST in comparison with the single-head Octo model 
    
   \item  Performance of MuST, conditioned on task goals specified by images or language instructions
    
    \item Performance of MuST across a diverse set of objects
    
    \item Ability for MuST to react to unexpected environment disturbance, based on the skill progress values
\end{itemize}

The performance is measured by two  metrics:
\begin{enumerate}
    \item {\bf Skill completion and task completion:} we report the completion of a single skill, and the task is completed only if all skills in the task have been successful executed. Any failed skill will result in failure of all subsequent skills.
    \item {\bf Execution time:} The execution time is defined as spent time for manipulation until the object consistently stays at the goal pose $g$ for 100 time steps. 
\end{enumerate}

We compare \ours with the single-head Octo model, which learn all skills without progress estimation.
Both models finetune Octo-Base (93M params) checkpoint and use L1 action heads as decoding heads for skills and progress. 
For closed-loop control scenarios, both models carry out inference every 50 time steps and compute the action sequence for the next 50 time steps.
We train the models with an Nvidia V100 16GB GPU.


\subsection{Simulation Experimental Results and Comparison with Octo Baseline  }

Our goal-state conditioned Pick-n-Pack manipulation task Fig.~\ref{fig:intro}[Top] consists of four skills:
%In simulation,  we first test \ours on goal-state conditioned pick-n-pack to evaluate the overall performance of \ours and compare it with Octo.
%Shown in Fig.~\ref{fig:intro}[Top], this manipulation task requires four skills:
First, the robot flips down the object from the tote boundary to enable picking. 
Next, it picks the object from the picking tote.
Based on the goal-state indicator $I_g$ or $l_g$, the robot packs the object near the desired corner of the packing tote.
Finally, the robot pushs the object, both  rotating and translating it, to fit it tightly the desired corner.

%In simulation, we collect x human demonstrations of this task is collected by controlling robot states with keyboard.The models are trained with only 15 human demonstrations for each of the four corners as goal states.

The goal is given by either a language prompt or a goal image Fig.~\ref{fig:end_state_indicator} . 
For the goal images, instead of specific images of objects, we use a blue patch to suggest the goal state which makes it independent of object appearance, enhancing the model's generalization capability across different object types.
%The human demonstrations of this task is collected by controlling robot states with keyboard.
%In experiments, we present the task completion rate for each skill. 
%Test cases fail in earlier skills are labeled as failures in subsequent skills as well.
%To further evaluate the execution efficiency, we also present the average execution time of successful test cases.


\begin{figure}
    \centering
    \includegraphics[width=0.5\textwidth]{figures/end_state_indicator.pdf}
    \caption{We use either language prompts or images as goal state indicators to customize packing poses.}
    \label{fig:end_state_indicator}
\end{figure}

Tab.\ref{tab:single_language_results} reports the performance metrics for both  \ours  and the Single Head Octo. The task is language conditioned Pick-n-Pack with the ``long box''(Fig.~\ref{fig:object_set}), and each task is repeated 10 times. While Octo model succeeded in  $32.5\%$ tasks, \ours maintains $80\%-90\%$ success rate.
In addition, \ours is $23.7\%-38.4\%$ faster than Octo in execution time for the finished tasks.
The results suggest that \ours is more robust than Octo baseline in long horizon manipulation tasks. 
%The models are trained with only 15 human demonstrations for each of the four corners as goal states.
%In this task, each camera has good visibility in some skills but is occluded in others, which makes it hard to learn a shared encoder for all skills.


% \begin{table*}[ht]
% \centering
% \caption{Language Conditioned Pick-n-Pack (Long Box)}
% \label{tab:single_language_results}
% \begin{tabular}{@{\centering\arraybackslash}m{1.7cm}>{\centering\arraybackslash}m{1cm}>{\centering\arraybackslash}m{1cm}>{\centering\arraybackslash}m{1cm}>{\centering\arraybackslash}m{1cm}>{\centering\arraybackslash}m{1cm}>{\centering\arraybackslash}m{1cm}>{\centering\arraybackslash}m{1cm}>{\centering\arraybackslash}m{1cm}>
% {\centering\arraybackslash}m{1cm}>{\centering\arraybackslash}m{1cm}>{\centering\arraybackslash}m{1cm}>{\centering\arraybackslash}m{1cm}>{\centering\arraybackslash}m{1cm}@{}}
% \toprule
%  & \multicolumn{10}{c}{\textbf{Task Completion}} \\ \cmidrule(lr){2-11} 
%  & \multicolumn{2}{c}{\textbf{Flip}} & \multicolumn{2}{c}{\textbf{Pick}} & \multicolumn{2}{c}{\textbf{Pack}} & \multicolumn{2}{c}{\textbf{Push (Orientation)}} & \multicolumn{2}{c}{\textbf{Push (Position)}} & \multicolumn{2}{c}{\textbf{Execution Time}}\\ \cmidrule(lr){2-3} \cmidrule(lr){4-5} \cmidrule(lr){6-7} \cmidrule(lr){8-9} \cmidrule(lr){10-11} \cmidrule(lr){12-13}
%  \textbf{End State} & \textbf{\ours} & \textbf{Octo} & \textbf{\ours} & \textbf{Octo} & \textbf{\ours} & \textbf{Octo} & \textbf{\ours} & \textbf{Octo}& \textbf{\ours} & \textbf{Octo}& \textbf{\ours} & \textbf{Octo} \\ \midrule
% % \textbf{Task} & \textbf{Method 1 (Success)} & \textbf{Method 1 (Time)} & \textbf{Method 2 (Success)} & \textbf{Method 2 (Time)} & \textbf{Method 3 (Success)} & \textbf{Method 3 (Time)} & \textbf{Method 4 (Success)} & \textbf{Method 4 (Time)} & \textbf{Method 5 (Success)} & \textbf{Method 5 (Time)} \\ \midrule
% Top Left & \multicolumn{1}{|c}{10/10} & 10/10 & \multicolumn{1}{|c}{10/10} & 8/10 & \multicolumn{1}{|c}{10/10} & 7/10 & \multicolumn{1}{|c}{10/10} & 4/10 & \multicolumn{1}{|c}{10/10} & 4/10 & \multicolumn{1}{|c}{1324.2} & 1734.3 \\
% Top Right & \multicolumn{1}{|c}{10/10} & 10/10 & \multicolumn{1}{|c}{10/10} & 10/10 & \multicolumn{1}{|c}{9/10} & 9/10 & \multicolumn{1}{|c}{8/10} & 7/10 & \multicolumn{1}{|c}{8/10} & 4/10 & \multicolumn{1}{|c}{1530.0} & 2639.7 \\
% Bottom Left & \multicolumn{1}{|c}{10/10} & 10/10 & \multicolumn{1}{|c}{10/10} & 7/10 & \multicolumn{1}{|c}{10/10} & 7/10 & \multicolumn{1}{|c}{9/10} & 5/10 & \multicolumn{1}{|c}{9/10} & 3/10 & \multicolumn{1}{|c}{1538.9} & 2337.3\\
% Bottom Right & \multicolumn{1}{|c}{10/10} & 10/10 & \multicolumn{1}{|c}{10/10} & 7/10 & \multicolumn{1}{|c}{9/10} & 6/10 & \multicolumn{1}{|c}{9/10} & 4/10 & \multicolumn{1}{|c}{9/10} & 2/10 & \multicolumn{1}{|c}{1571.7} & 2550.5 \\ \bottomrule
% \end{tabular}
% \end{table*}

\scriptsize
\begin{table}[ht]
\centering
\caption{Language Conditioned Pick-n-Pack (Long Box)}
\label{tab:single_language_results}
\begin{tabular}{@{\centering\arraybackslash}m{1.4cm}>{\centering\arraybackslash}m{0.8cm}>{\centering\arraybackslash}m{0.8cm}>{\centering\arraybackslash}m{0.8cm}>{\centering\arraybackslash}m{0.8cm}>
{\centering\arraybackslash}m{0.8cm}>{\centering\arraybackslash}m{0.8cm}>{\centering\arraybackslash}m{0.8cm}>{\centering\arraybackslash}m{0.8cm}>{\centering\arraybackslash}m{0.8cm}@{}}
\toprule
 & \multicolumn{6}{c}{\textbf{Task Completion}} \\ \cmidrule(lr){2-7} 
 & \multicolumn{2}{c}{\scriptsize{\textbf{Flip$\rightarrow$Pack}}} & \multicolumn{2}{c}{\textbf{{\scriptsize Push (Orientation)}}} & \multicolumn{2}{c}{{\scriptsize \textbf{Push \newline (Position)}}} & \multicolumn{2}{c}{{\scriptsize \textbf{Execution \newline Time}}}\\ \cmidrule(lr){2-3} \cmidrule(lr){4-5} \cmidrule(lr){6-7} \cmidrule(lr){8-9}
 \textbf{End State} & \textbf{\ours} & \textbf{Octo} & \textbf{\ours} & \textbf{Octo}& \textbf{\ours} & \textbf{Octo}& \textbf{\ours} & \textbf{Octo} \\ \midrule
% \textbf{Task} & \textbf{Method 1 (Success)} & \textbf{Method 1 (Time)} & \textbf{Method 2 (Success)} & \textbf{Method 2 (Time)} & \textbf{Method 3 (Success)} & \textbf{Method 3 (Time)} & \textbf{Method 4 (Success)} & \textbf{Method 4 (Time)} & \textbf{Method 5 (Success)} & \textbf{Method 5 (Time)} \\ \midrule
{\scriptsize Top Left} & \multicolumn{1}{|c}{10/10} & 7/10 & \multicolumn{1}{|c}{10/10} & 4/10 & \multicolumn{1}{|c}{10/10} & 4/10 & \multicolumn{1}{|c}{1324} & 1734 \\
{\scriptsize Top Right} & \multicolumn{1}{|c}{9/10} & 9/10 & \multicolumn{1}{|c}{8/10} & 7/10 & \multicolumn{1}{|c}{8/10} & 4/10 & \multicolumn{1}{|c}{1530} & 2639 \\
{\scriptsize Bottom Left} & \multicolumn{1}{|c}{10/10} & 7/10 & \multicolumn{1}{|c}{9/10} & 5/10 & \multicolumn{1}{|c}{9/10} & 3/10 & \multicolumn{1}{|c}{1538} & 2337\\
{\scriptsize Bottom Right} & \multicolumn{1}{|c}{9/10} & 6/10 & \multicolumn{1}{|c}{9/10} & 4/10 & \multicolumn{1}{|c}{9/10} & 2/10 & \multicolumn{1}{|c}{1571} & 2550 \\ \bottomrule
\end{tabular}
\end{table}
\normalsize


\begin{figure}
    \centering
    \includegraphics[width=0.37\textwidth]{figures/object_set.pdf}
    \caption{Training object set and test object set in simulation. The 3D model used are open-source models sampled from the YCB Object and Model Set~\cite{YCB}, NVIDIA SimReady assets~\cite{nvidia_simready}, open-source models from SketchFab~\cite{sketchfab_open_source}, and the Google Scanned Objects dataset~\cite{2022googlescannedobjectshighquality}.}
    \label{fig:object_set}
\end{figure}

\subsection{Additional Evaluation of \ours and \progss in Simulation}
In this section, we further evaluate the performance of \ours, conditioned with goal images, on a diverse object set, and on progress estimation.

Tab.~\ref{tab:single_image_results} summarizes  the performance of \ours on image-conditioned Pick-n-Pack task on the same Long Box object. \ours maintains $80\%-90\%$ success rate for 40 evaluation trials.



%In Tab.~\ref{tab:four_obj_language_results}, we test \ours on a diverse object set.
Furthermore, we evaluate \ours on a diverse object set. In this experiment, we train \ours on four different boxes(Fig.~\ref{fig:object_set}). The training dataset include 15 demonstrations for each of training objects at each of the four goal poses.
We test \ours on all objects including four training objects and two novel objects, and results are reported in Tab.~\ref{tab:four_obj_language_results} for 20 trials on each object.
For the first three skills, \ours maintains over $90\%$ completion rate on training object set and that drops to $70\%-75\%$ on objects in the new category.
For the last push skill, \ours solves $80\%$ test cases on the cracker box, $65\%-70\%$ on other training objects.
\ours only finishes around $40\%$ pushes on novel objects. 
In the training dataset with cuboid objects, the push demonstrations use box corners for rotation.  Our hypothesis is that the same push behavior does not generalize well to the tested novel objects with irregular shapes.


\begin{table}[H]
\centering
\caption{Image Conditioned Pick-n-Pack (Long Box)}
\label{tab:single_image_results}
\begin{tabular}{@{\centering\arraybackslash}m{1.7cm}>{\centering\arraybackslash}m{1cm}>{\centering\arraybackslash}m{1cm}>{\centering\arraybackslash}m{1cm}>{\centering\arraybackslash}m{1.5cm}>{\centering\arraybackslash}m{1.2cm}>{\centering\arraybackslash}m{1cm}@{}}
\toprule
 & \multicolumn{5}{c}{\textbf{Task Completion}} \\ \cmidrule(lr){2-6} 
\textbf{Packing \newline Corner} & \textbf{Flip} & \textbf{Pick} & \textbf{Pack} & \textbf{Push (Orientation)} & \textbf{Push (Position)} & \textbf{Execution Time}\\ \midrule
% \textbf{Task} & \textbf{Method 1 (Success)} & \textbf{Method 1 (Time)} & \textbf{Method 2 (Success)} & \textbf{Method 2 (Time)} & \textbf{Method 3 (Success)} & \textbf{Method 3 (Time)} & \textbf{Method 4 (Success)} & \textbf{Method 4 (Time)} & \textbf{Method 5 (Success)} & \textbf{Method 5 (Time)} \\ \midrule
Top Left & 10/10 & 10/10 & 10/10 & 9/10 & 8/10 & 1570 \\
Top Right & 9/10 & 9/10 & 9/10 & 8/10 & 8/10 & 1765 \\
Bottom Left & 10/10 & 10/10 & 10/10 & 9/10 & 9/10 & 1225 \\
Bottom Right & 10/10 & 10/10 & 10/10 & 8/10 & 9/10 & 1158 \\ \bottomrule
\end{tabular}
\end{table}


\begin{table}[ht]
\centering
\caption{Language-Conditioned Pick-n-Pack with Diverse Object Set}
\label{tab:four_obj_language_results}
\begin{tabular}{@{\centering\arraybackslash}m{1.7cm}>{\centering\arraybackslash}m{1cm}>{\centering\arraybackslash}m{1cm}>{\centering\arraybackslash}m{1cm}>{\centering\arraybackslash}m{1.5cm}>{\centering\arraybackslash}m{1.2cm}>{\centering\arraybackslash}m{1cm}@{}}
\toprule
 & \multicolumn{5}{c}{\textbf{Task Completion}} \\ \cmidrule(lr){2-6} 
\textbf{Test Object} & \textbf{Flip} & \textbf{Pick} & \textbf{Pack} & \textbf{Push (Orientation)} & \textbf{Push (Position)} & \textbf{Execution Time}\\ \midrule
Cracker Box & 20/20 & 20/20 & 19/20 & 16/20 & 16/20 & 2073 \\
Liquid Box & 20/20 & 20/20 & 19/20 & 14/20 & 13/20 & 1769 \\
Long Box & 20/20 & 20/20 & 18/20 & 14/20 & 13/20 & 1330 \\
Oil Tin & 20/20 & 20/20 & 19/20 & 14/20 & 13/20 & 1875 \\ \midrule
Bottle1 (OOD) & 20/20 & 20/20 & 15/20 & 12/20 & 9/20 & 1608\\ 
Bottle2 (OOD) & 20/20 & 20/20 & 14/20 & 9/20 & 7/20 & 2009\\ \bottomrule
\end{tabular}
\end{table}

Additionally, we demonstrate that \ours can react to unexpected environment disturbance based on the skill progress values. As an example, when user resets an object on the tote edge, \ours would select the Flip skill repeatedly. Similarly, \ours would skip the first skill, and start with the second Pick skill when the object is in the pick state. We include these demonstrations with the skill progress value graphs in the accompanying video.

\subsection{Handling Multiple Sequences in Simulation}\label{sec:sim_task2}

We designed the task of multi-sequence pick-n-pack to evaluate the performance of \ours when multiple skill sequences are demonstrated.
As shown in Fig.~\ref{fig:multi_sequence}, the robot is tasked to flip down a box and place it at the packing tote. 
Specifically, when the object is located in the central area of the tote, the robot can choose to flip the object before or after pick-n-place.
However, when the object is located at the boundary of the picking tote, the robot cannot execute pick-n-place before a successful flip.
For each of the three cases of skill sequences in Fig.~\ref{fig:multi_sequence}, we made 50 demonstrations. 
% We test \ours on 80 trials for each of the two initial state categories. 

\begin{figure}[ht]
    \centering
    \includegraphics[width=0.35\textwidth]{figures/multi_sequence.pdf}
    \vspace{-3mm}
    \caption{Multi-sequence pick-n-pack. When the object is sampled at the central area of the tote, there are two possible skill orderings; when the object is sampled at the edge, the robot cannot directly pick it up before flipping.}
    \label{fig:multi_sequence}
\end{figure}

Tab.~\ref{tab:multi_sequence_results} shows sequence selection distribution and task completion rate.
When the object is sampled at the central area of the picking tote, out of the 80 trials, \ours chooses ``pick first'' and ``flip first'' sequences in $37.5\%$ and $62.5\%$ trials respectively with around $80\%$ success rate on both skill sequences.
When the object is sampled at the edge of the picking tote, picking directly is impossible. 
\ours chooses the ``flip first'' sequence in $96.2\%$ test cases.
The results suggest that \ours effectively handles multiple skill sequences and avoids ``modality collapses'' in long horizon manipulation tasks.



\begin{table}[H]
\centering
\caption{Multi-Sequence Pick-n-Pack}
\label{tab:multi_sequence_results}
\begin{tabular}{@{\centering\arraybackslash}m{4cm}>{\centering\arraybackslash}m{2cm}>{\centering\arraybackslash}m{2cm}@{}}
\toprule
 % & \multicolumn{2}{c}{\textbf{Task Completion}} \\ \cmidrule(lr){2-3} 
\textbf{Test Cases} & \textbf{Pick First} & \textbf{Flip First}\\ \midrule
Manipulate from central area & 24/30 & 39/50\\
Manipulate from edge & 0/3 & 73/77\\ \bottomrule
\end{tabular}
\end{table}






%\subsubsection{Real-GC} 

\subsection{Physical Experimental Results}\label{sec:real_task}
Our robotic test bed (Fig.~\ref{fig:sock_puppet}[Right]) comprises a collaborative manipulator equipped with a customized suction gripper(Fig.~\ref{fig:sock_puppet}[Left]), which is capable of vacuum suction and dexterous contact with its soft tip. Two 5 MP 3D cameras are positioned with one above each tote. 

\begin{wrapfigure}{r}{0.25\textwidth}  % 'r' means the figure is on the right
    \centering
    \includegraphics[width=0.25\textwidth]{figures/station_2_overview.pdf}
    \caption{[Left] A customized suction gripper capable of vacuum suction and dexterous contact. [Right] Physical robotic system.}
    \vspace{-1mm}
    \label{fig:sock_puppet}
\end{wrapfigure}

Similarly, the experimental task (Fig.~\ref{fig:station2_sequence}) consists of four skills: flips down an object from the edge of the picking tote, grasps it with the suction cup, packs it at the proper pose, based on a goal image using a generic brown box(Fig.~\ref{fig:real_images}(c), and pushes it to the corners of the tote.
The first two skills have clear success or failure criteria, while packing succeeds if the object is in the correct quarter of the tote, and pushing succeeds if the object is within 2 cm of the correct corner. 

We evaluate \ours in a open-loop control framework, where \ours takes a single state observation, two images of two totes plus an image of the in-hand object (if any), and outputs the trajectory for the selected skill. The robot then executes the   entire trajectory in an open loop and moves out of the observable environment. We use a set of five objects (Fig.~\ref{fig:real_images} for the physical experimental task. For each object in our training set (Fig.~\ref{fig:real_images}(a)), we collect 15, 15, 24, and 24 human demonstrations for the four skills respectively. 

\begin{figure}[t]
    \centering
    \includegraphics[width=0.4\textwidth]{figures/station_2_sequence.pdf}
    \vspace{-2mm}
    \caption{Task sequence of real robot goal-state conditioned pick-n-pack.}
    \label{fig:station2_sequence}
\end{figure}

%During the model inference, \ours is given three images: overview images of the two totes and an image of the in-hand object.
%Similar to the object-independent goal images in simulation, we use goal images of a brown box(Fig.~\ref{fig:real_images} (c)) which is neither in the training set, nor in the test set.

%The goal-state indicator is only related to the packing tote, the goal images of the picking tote and the in-hand object is zero-padded.

%Based on the observations from the three cameras and the goal image, models decide on the skill to execute and the whole trajectory of the skill execution.
%After the execution, the robot moves out of the environment and updates the observation images.




\begin{figure}
    \centering
    \vspace{-3mm}
    \includegraphics[width=0.4\textwidth]{figures/real_objects.pdf}
    \vspace{-3mm}
    \caption{Training (a) and test (b) object set for physical experiments. (c) Image of the packing tote with a universal object as the goal-state indicator.}
    \label{fig:real_images}
\end{figure}

 




% I think this could have been discussed before this section - so skip
%For the four skills, a test case is labeled as success if the object is flipped down, leaves picking tote, is placed at right quarter of the packing tote, and is packed less than 2cm to the corner respectively. A test case is labeled as a failure if the robot fails to proceed to the next phase after three actions or collides with the environment.

We first evaluate both \ours and Octo on the couscous box with different goal-state images (Tab.~\ref{tab:real_robot_diff_corner_results}).
Both models succeed in the first three skills in all the test cases but Octo only successfully pushes the object to corners in $35\%$ test cases.
In Tab.~\ref{tab:real_robot_diff_obj_results}, we show the experiment results on the diverse object set.
Octo has $0\%$ success rate on the small object (``Instax'' box) and the novel object (Bag).
It also fails in pushing other objects to corners in most test cases.
In contrast, \ours finishes $88\%$ test cases among the objects.
In addition, most of the failures of Octo in pushing are due to collisions with the environment, while two failures of \ours on pushing are inaccurate location with the open loop limitation: the final object pose is slightly outside the goal region (2.1 cm and 2.6 cm away from the corner).
In summary, \ours has much higher success rate in finishing the whole task than Octo, especially on small objects and novel objects.

\begin{table}[H]
\centering
\caption{Four Image Conditioned Goals(Couscous Box)}
\label{tab:real_robot_diff_corner_results}
\begin{tabular}{@{\centering\arraybackslash}m{1.7cm}>{\centering\arraybackslash}m{0.8cm}>{\centering\arraybackslash}m{0.8cm}>{\centering\arraybackslash}m{0.8cm}>{\centering\arraybackslash}m{0.8cm}>{\centering\arraybackslash}m{0.8cm}>{\centering\arraybackslash}m{0.8cm}>{\centering\arraybackslash}m{0.8cm}>{\centering\arraybackslash}m{1cm}@{}}
\toprule
 % & \multicolumn{6}{c}{\textbf{Task Completion}} \\ \cmidrule(lr){2-7} 
 & \multicolumn{2}{c}{\textbf{Flip}} & \multicolumn{2}{c}{\textbf{Pick}} & \multicolumn{2}{c}{\textbf{Pack}} & \multicolumn{2}{c}{\textbf{Push}} \\ \cmidrule(lr){2-3} \cmidrule(lr){4-5} \cmidrule(lr){6-7}  \cmidrule(lr){8-9}
 \textbf{Goal State} & \textbf{\ours} & \textbf{Octo} & \textbf{\ours} & \textbf{Octo} & \textbf{\ours} & \textbf{Octo} & \textbf{\ours} & \textbf{Octo} \\ \midrule
Top Left & \multicolumn{1}{|c}{5/5} & 5/5 & \multicolumn{1}{|c}{5/5} & 5/5 & \multicolumn{1}{|c}{5/5} & 5/5 & \multicolumn{1}{|c}{4/5} & 3/5 \\
Top Right & \multicolumn{1}{|c}{5/5} & 5/5 & \multicolumn{1}{|c}{5/5} & 5/5 & \multicolumn{1}{|c}{5/5} & 5/5 & \multicolumn{1}{|c}{2/5} & 0/5 \\
Bottom Left & \multicolumn{1}{|c}{5/5} & 5/5 & \multicolumn{1}{|c}{5/5} & 5/5 & \multicolumn{1}{|c}{5/5} & 5/5 & \multicolumn{1}{|c}{5/5} & 2/5 \\
Bottom Right & \multicolumn{1}{|c}{5/5} & 5/5 & \multicolumn{1}{|c}{5/5} & 5/5 & \multicolumn{1}{|c}{5/5} & 5/5 & \multicolumn{1}{|c}{5/5} & 2/5 \\ 
 \bottomrule
\end{tabular}
\end{table}


\begin{table}[H]
\centering
\caption{Five Objects Image-Conditioned (Bottom Left Corner)}
\label{tab:real_robot_diff_obj_results}
\begin{tabular}{@{\centering\arraybackslash}m{1.7cm}
>{\centering\arraybackslash}m{0.8cm}
>{\centering\arraybackslash}m{0.8cm}
>{\centering\arraybackslash}m{0.8cm}
>{\centering\arraybackslash}m{0.8cm}
>{\centering\arraybackslash}m{0.8cm}
>{\centering\arraybackslash}m{0.8cm}
>{\centering\arraybackslash}m{0.8cm}
>{\centering\arraybackslash}m{0.8cm}@{}}
\toprule
 % & \multicolumn{6}{c}{\textbf{Task Completion}} \\ \cmidrule(lr){2-7} 
 & \multicolumn{2}{c}{\textbf{Flip}} & \multicolumn{2}{c}{\textbf{Pick}} & \multicolumn{2}{c}{\textbf{Pack}} & \multicolumn{2}{c}{\textbf{Push}} \\ \cmidrule(lr){2-3} \cmidrule(lr){4-5} \cmidrule(lr){6-7}  \cmidrule(lr){8-9}
 \textbf{Object} & \textbf{\ours} & \textbf{Octo} & \textbf{\ours} & \textbf{Octo} & \textbf{\ours} & \textbf{Octo} & \textbf{\ours} & \textbf{Octo} \\ \midrule
``Instax'' Box & \multicolumn{1}{|c}{ 4/5} & 0/5 & \multicolumn{1}{|c}{4/5} & 0/5 & \multicolumn{1}{|c}{4/5} & 0/5 & \multicolumn{1}{|c}{2/5} & 0/5 \\
``Mina'' Box & \multicolumn{1}{|c}{5/5} & 4/5 & \multicolumn{1}{|c}{5/5} & 4/5 & \multicolumn{1}{|c}{5/5} & 4/5 & \multicolumn{1}{|c}{5/5} & 3/5 \\
Couscous Box & \multicolumn{1}{|c}{5/5} & 5/5 & \multicolumn{1}{|c}{5/5} & 5/5 & \multicolumn{1}{|c}{5/5} & 5/5 & \multicolumn{1}{|c}{5/5} & 2/5 \\
Rice Box & \multicolumn{1}{|c}{ 5/5} & 5/5 & \multicolumn{1}{|c}{5/5} & 5/5 & \multicolumn{1}{|c}{5/5} & 4/5 & \multicolumn{1}{|c}{5/5} & 1/5 \\ 
Bag (OOD) &\multicolumn{1}{|c}{ 5/5} & 0/5 & \multicolumn{1}{|c}{5/5} & 0/5 & \multicolumn{1}{|c}{5/5} & 0/5 & \multicolumn{1}{|c}{5/5} & 0/5  \\ \bottomrule
\end{tabular}
\vspace{-3mm}
\end{table}

% Structure
% \begin{enumerate}
% \item Long horizon pick-n-pack.
% \begin{enumerate}
%     \item description with figures and qualitative results.
%     \item language-conditioning, long box, 15 demos on each corner, MuST vs Octo single policy vs non-object-centric MuST vs skill addition.
%     \item Skill skipping and repeating (small table or only show in video).
%     \item Universal goal image conditioning
%     \item Generality: 4 objects * 15 demos * 4 corners $\rightarrow$ in/out-of distribution objects.
% \end{enumerate}
% \item Multi-task sequence pick-n-pack.
% \begin{enumerate}
%     \item description with figures
%     \item Objects in the central region: Need to show a balancing sequence choice to avoid ``modality collapse''
%     \item Object on the boundary: Knows to flip first.
% \end{enumerate}
% \item Station $2$ trajectory planning
% \begin{enumerate}
%     \item Description with figures
%     \item Object-independent goal image conditioning: MuST vs Octo single policy.
% \end{enumerate}
% \end{enumerate}
% 

\section{Experiments} \label{sec6}
We investigate the empirical performance of our new procedures in various experiments to demonstrate their effectiveness.
%To demonstrate the effectiveness of our new procedures, we investigate their empirical performance in the following experiments. 
Recall that our procedures are developed for two distinct goals, namely estimation of the optimal trade-off curve $T$ (see Section \ref{sec:4}) and auditing a privacy claim $T^{(0)}$ (see Section \ref{sec:goal2}). We will run experiments for both of these objectives. \\
%These goals correspond to Sections \ref{sec:4} and \ref{sec:goal2} respectively. \\
%This section aims to validate the theoretical results presented in Section~\todo{cite section} and Section~\todo{cite section}. \\
\textbf{Experiment Setting:} 
%We have outlined two distinct objectives along with their corresponding methodologies:
%\begin{description}
 %   \item[\textbf{Goal 1: Uniform Estimation of the Privacy Curve $T$}]  
 %   The first objective is to uniformly estimate an unknown privacy curve $T$, as stated in Theorem~\ref{theo:1}. To validate not only the theoretical correctness but also the practical effectiveness of this estimation approach, we conducted a simulation study on all four mechanisms. The results of this study are presented in \todo{Table~\ref{tab:estimation_f_curves} and Figure~\ref{fig:todo}.}
 %   \item[\textbf{Goal 2: Detection of Privacy Violations}] 
 %   The second objective is inferential in nature. As formulated in Theorem~\ref{theo:auditor}, the goal is to detect privacy violations for a predefined false positive rate. To demonstrate the effectiveness of this methodology, we constructed faulty algorithms and analyzed their behavior. The results of this analysis are depicted in Figure~\ref{fig:todo}.
%\end{description}
Throughout the experiments, we consider databases $\DB,\DB' \in [0,1]^r$, where the participant number is always $r=10$. As discussed in Section \ref{sec:overview_techniques}, we first choose a pair of neighboring datasets such that there is a large difference in the output distributions of $\Mech(D)$ and $\Mech(D')$. We can achieve this by simply choosing $D$ and $D'$ to be as far apart as possible (while still remaining neighbors) and we settle on the choice 
%As typical in the privacy validation literature, we consider two neighboring databases that are far apart. On the $r$-dimensional cube $[0,1]^r$ we make the natural choice of
\begin{equation}\label{eq_databases}
    \DB=(0,\hdots, 0)\quad \textnormal{and} \quad \DB'=(1,0,\hdots, 0)
\end{equation}
for all our experiments.
%and notice that similar results as the ones below hold for other pairs of databases. %Our methods do however work just as well for other data bases $D$ and $D'$.
%Additionally, for data lying in the unit cube, this choice is natural, as these two databases are far apart on the unit cube.

\subsection{Mechanisms}\label{sec:algorithms}
In this section, we test our methods on two frequently encountered mechanisms from the auditing literature: the Gaussian mechanism and differentially private Stochastic Gradient Descent (DP-SGD). We study two other prominent DP algorithms -- the Laplace and Subsampling mechanism -- in Appendix \ref{AppB}. \\
%We apply our methods to four mechanisms frequently encountered in the privacy literature: the Gaussian mechanism, the Laplace mechanism, the Subsampling mechanism, and, most notably, the Noisy Stochastic Gradient Descent (DP-SGD) mechanism. These algorithms are quite  heterogeneous and hence collectively form a good benchmark to evaluate our methods. We quickly review these mechanisms and specify parameter settings. \\

\noindent \textbf{Gaussian mechanism.}
We consider the summary statistic $S(x)= \sum_{i=1}^{10} x_i$ and the mechanism
\begin{equation*}
    M(x):= S(x)+Y~,
\end{equation*}
where $Y\sim \mathcal N (0, \sigma^2)$. The statistic $S(x)$ is privatized by the random noise $Y$ if the variance $\sigma^2$ of the Normal distribution is appropriately scaled. We choose $\sigma = 1$ for our experiments and note that - in our setting - the optimal trade-off curve is given by 
\begin{align*}
     T_{Gauss}(\alpha)= \Phi(\Phi^{-1}(1-\alpha)- 1).
\end{align*}
We point the reader to \cite{Dong2022} for more details. \\








%\textbf{Additive Noise Mechanisms}
%The most prominent DP algorithms are the Laplace and the Gaussian Mechanism. If we consider the summary statistic $S(x)= \sum_{i=1}^{10} x_i$, the output can be described by
%\begin{equation*}
%    M(x):= S(x)+Y~,
%\end{equation*}
%where $Y\sim Lap(0,b)$ or $Y\sim \mathcal N (0, \sigma^2)$, respectively. Here $b>0$ denotes the scale parameter in the Laplace distribution and $\sigma^2$ the variance for the normal distribution. Given the structure of $M$, these mechanisms are generically referred to as "additive noise mechanisms". Appropriately scaled, both mechanisms fulfill $f$-DP. For the Gaussian mechanism we set $\sigma=1$, for which \cite{Dong2022} derived the trade-off curve
%\begin{equation*}
%    T_{Gauss}(\alpha)= \Phi(\Phi^{-1}(1-\alpha)-\mu)
%\end{equation*}
%as an explicit expression of the optimal trade-off function between $P = \mathcal N(0,1)$ and $Q = \mathcal N(\mu,1)$. For the Laplace mechanism, we set $b=1$, and note again that \cite{Dong2022} derived an explicit formula given by
%\begin{equation*}
%    T_{Lap}(\alpha)=\begin{cases}
%        1-\exp(\mu)\alpha,  &\alpha<\exp(-\mu)/2~,\\
%        \exp(-\mu)/4 \alpha,  &\exp(-\mu)/2\leq \alpha\leq 1/2~,\\
%        \exp(-\mu)(1-\alpha), &\alpha>1/2
 %   \end{cases}
%\end{equation*}
%for the optimal trade-off function between $P = Lap(0,1)$ and $Q = Lap(\mu,1)$.\\
%After the simpler additive noise mechanisms, we turn to the more advanced mechanisms of subsampling and the  DP-SGD, both of which play a role in the context of private machine learning.\\
%\textbf{Subsampling Mechanism:} Random subsampling provides an effective way to enhance the privacy of a DP mechanism $M$. We only provide a rough outline here and refer for details to \cite{Dong2022}.
%In simple words, we choose an integer $m$ with  $1\leq m< r$, where $r$ is the size of the databases $D$. In subsampling, we extract random subsamples of size $m$ of these databases, giving us the smaller databases $\bar D$. The mechanism $M$ is then applied to  $\bar D$ instead of $D$, providing an additional layer of protection to users. If a user is not part of $\bar D$, their privacy cannot be compromised. The amplifying effect of subsampling is visible in the optimal trade-off curve: If $M$ implies the curve $T$, it turns out that $M(\bar D)$ implies the curve
%\begin{equation*}
%    \bar T(\alpha)=  \frac{m}{r}T(\alpha)+\frac{r-m}{r}(1-\alpha),
%\end{equation*}
%which is strictly more private than $T$ for any $m<r$. A minor technical peculiarity of subsampling is that the resulting curve $\bar T$ is generally not symmetrical, even if $T$ is (see \cite{Dong2022} for details on the symmetry of trade-off functions). Trade-off curves are usually considered to be symmetrical and one can symmetrize $\bar T$ by applying a symmetrizing operator $\mathbf{C}$ (again, see \cite{Dong2022}). In our simulations we will adhere to this procedure and depict the symmetrized version $\mathbf{C}[\bar T]$ together with a symmetrized estimator. Further details on the symmetrization can be found in Appendix \ref{AppB}.
%At first glance, this result may seem complicated, but it is in fact quite natural. Suppose we select $m$ entries randomly for the two neighboring databases $\DB,\DB'$ with each of size $k$. All entries in  $\DB,\DB'$ except one are identical - so the chance of selecting the entry where they differ (in $m$ draws) is $p=m/k$. 
%Conversely the probability of not selecting named entry is $(1-p)$.  
%This characterizes the construction of $T_p$, and the factor $(1-\alpha)$ corresponds to perfect privacy, as in that case $\alpha+\beta=\alpha+1-\alpha=1$. With that in hand, $C_p(T)=\min\{T_p,T_p^{-1}\}^{**}$ (again see in \cite{Dong2022}) is simply a symmetrization of $T_p$ in a sense that it is the greatest convex minorant of the minimum $T_p$ and $T_p^{-1}$ (\todo{maybe cite sth}). Here, for a function $T$, $T^{**}$ denotes the twice convex conjugate of $T$.\\  
%For the following experiments involving subsampling, we use the Gaussian mechanism as $M$ (with $\sigma=1$) and obtain the subsampled version $M'$, by fixing the parameter $m=5$ (recall that $r=10$). \\
%Observing only independent outputs of $M(x)$ will only yield an estimator for $\hat T_p$. As an additional contribution, we approximate $T_{sub}(\alpha)$ by incorporating a numerical approximation based on $\hat T_p$. \todo{should we expand here?}\\

\noindent \textbf{DP-SGD.} The DP-SGD mechanism is designed to (privately) approximate a solution for the empirical risk minimization problem
\begin{equation*}
\theta^*=argmin_{\theta\in \Theta} \mathcal L_x(\theta) \quad \text{with} \quad \mathcal L_x(\theta)=\frac{1}{r}\sum_{i=1}^{r} \ell(\theta, x_i)~.
\end{equation*}
Here, $\ell$ denotes a loss function, $\Theta$ a closed convex set and $\theta^*\in \Theta$ the unique optimizer. For sake of brevity, we provide a description of DP-SGD in the appendix (see Algorithm \ref{alg:noisy_sgd}). In our setting, we consider the loss function $\ell(\theta, x_i)=\frac{1}{2} (\theta-x_i)^2$, initial model $\theta_0=0$ and $\Theta=\mathbb{R}$. The remaining parameters are fixed as $\sigma=0.2, \rho = 0.2, \tau = 10, m=5$. In order to have a theoretical benchmark for our subsequent empirical findings, we also derive the theoretical trade-off curve $T_{SGD}$ analytically for our setting and choice of databases (see Appendix \ref{AppB} for details). Our calculations yield
%For the choice of databases as in equation \eqref{eq_databases}, one can compute the trade-off curve $T_{SGD}$ analytically: 
\begin{equation*}
    T_{SGD}(\alpha)=\sum_{I\subset \{1,\hdots, \tau \}} \frac{1}{2^{\tau}}\Phi\Big(\Phi^{-1} (1-\alpha)-\frac{\mu_I}{\bar\sigma}\Big)~.
\end{equation*}
where $\mu_I$ is chosen as in \eqref{mu_I} and $\bar{\sigma}$ as in \eqref{sigma_bar}.

\subsection{Simulations}
We begin by outlining the parameter settings of our KDE and $k$-NN methods for our simulations. We then discuss the metrics employed to validate our theoretical findings and, in a last step, present and analyze our simulation results.\\
\textbf{Parameter settings:}
%For the subsequent simulations we always use the same parameters across all algorithms, acknowledging the black-box setting. 
For the KDEs, we consider different sample sizes of $n_1=10^2,10^3,10^4,10^5,10^6$ and we fix the perturbation parameter at $h=0.1$. For the bandwidth parameter $b$ (see Sec. \ref{sec:kde}), we use the method of \cite{bandwidth}. To approximate the optimal trade-off curve, we use $1000$ equidistant values for $\eta$ between $0$ and $15$ (see Algorithm \ref{alg:pointwise_KDE_estimator} for details on the procedure). For the $k$-NN, we set the training sample size to $n_2=10^6,10^7,10^8$ and testing sample size to $10^6$. \\
%\todo{Yu: are you sure that this is sufficient?}\\

\noindent \textbf{Estimation}
The first goal of this work is estimation of the optimal trade-off curve $T$. In our experiments, we want to illustrate the uniform convergence of the estimator $\hat T_h$ to the optimal curve $T$, derived in Theorem \ref{theo:1}. Therefore, we consider increasing sample sizes $n_1$ to study the decreasing error. The distance of $\hat T_h$ and $T$ in each simulation run is measured by the  uniform distance\footnote{Of course, one cannot practically maximize over all (infinitely many) arguments $\alpha \in [0,1]$. The estimator $\hat T_h$ is made for a grid of values for $\eta$ (see our parameter settings above) and we maximize over all gridpoints.} %maximum distance on a grid $G$ 
%We repeatedly estimate the respective trade-off curves of the four mechanism introduced in Section \ref{sec:algorithms} and computed 
\[
    Error_T:=\sup_{\alpha \in [0,1]}|\hat T_h(\alpha)-T(\alpha)|.
\]
%on a grid $G$ of $[0,1]$. 
%In our setting, we defined $G$ as the grid given by the KDE. Since, we choose $1000$ $\eta$ equidistant, we will get $1000$ $\alpha$ values. However, they do not have to be equidistant nor unique. 
To study not only the distance in one simulation run, but across many, we calculate $Error_T$ in $1000$ independent runs and take the (empirical) mean squared error
\begin{equation}\label{eq:mse}
    MSE(Error_T):= \Ex{Error_T^2}
    %\mathbb{E}\mathbb Var(Error_G)+\mathbb E[Error_G]^2~.
\end{equation}
The results are depicted in Figure \ref{fig:estimation_mse} for the DP algorithms described in this section and the appendix. On top of that, we also construct figures that upper and lower bound the worst case errors for the Gaussian mechanism and DP-SGD over the $1000$ simulation runs. These plots visually show how the error of the estimator $\hat T_h$ shrinks as $n_1$ grows. 
% for the sample size $n_1=1000$. 
%For that, we computed the worst estimation point wise on an equidistant discretization of $[0,1]$ and interpolated the curves linearly. 
The results are summarized in Figures \ref{fig:gaussian}-\ref{fig:sgd}.\\
\begin{figure}
\centering\includegraphics[width=0.75\linewidth]{Figures/plot_table.png}
    \caption{\centering
    Empirical MSE defined in \eqref{eq:mse} to empirically validate Theorem \ref{theo:1} for varying sample sizes $n_1$ and over $1000$ simulation runs each.}\label{fig:estimation_mse}
\end{figure}
\begin{figure*}[h!]
    \centering
    \subfloat[$n_1=10^3$]{\includegraphics[width=0.3\textwidth]{Figures/Gaussian_shade_1000.png}}
    \hfill
    \subfloat[$n_1=10^4$]{\includegraphics[width=0.3\textwidth]{Figures/Gaussian_shade_10000.png}}
    \hfill
    \vspace{-0.2cm}
    \subfloat[$n_1=10^5$]{\includegraphics[width=0.3\textwidth]{Figures/Gaussian_shade_100000.png}}
    \caption{Estimation of the Gaussian Trade-off curve $T_{Gauss}$ for varying sample sizes and $\mu=1$. Min- and Max Curve lower- and upper bound the worst point-wise deviation from the true curve $T_{Gauss}$ over $1000$ simulations.}
    \label{fig:gaussian}
\vspace{-0.1cm}
\centering
    \subfloat[$n_1=10^3$]{\includegraphics[width=0.3\textwidth]{Figures/SGD_shade_1000.png}}
    \hfill
    \subfloat[$n_1=10^4$]{\includegraphics[width=0.3\textwidth]{Figures/SGD_shade_10000.png}}
    \hfill
    \vspace{-0.2cm}
    \subfloat[$n_1=10^5$]{\includegraphics[width=0.3\textwidth]{Figures/SGD_shade_100000.png}}
    \caption{Estimation of the DP-SGD Trade-off curve $T_{SGD}$ for varying sample sizes. Min- and Max Curve lower- and upper bound the worst point-wise deviation from the true curve $T_{SGD}$ over $1000$ simulations.}
    \label{fig:sgd}
\end{figure*}


\noindent {\textbf{Inference}\label{Inference}}
Next, we turn to the second goal of this work: Auditing a $T^{(0)}$-DP claim for a postulated trade-off curve $T^{(0)}$. 
The theoretical foundations of our auditor can be found in Theorem \ref{theo:auditor}. The theorem makes two guarantees: First, that for a mechanism $M$ satisfying $T^{(0)}$-DP the auditor will (correctly) not detect a violation, except with low, user-determined probability $\gamma$. Second, if $M$ violates  $T^{(0)}$-DP, the auditor will (correctly) detect the violation for sufficiently large sample sizes $n_1,n_2$. Together, these results mean that if a violation of $T^{(0)}$-DP is detected by the auditor, the user can have high confidence that $M$ does indeed not satisfy $T^{(0)}$-DP. 
%To begin, we examine the first result, which ensures, informally speaking that the auditor will not generate more than $\gamma>0$ false positives, when auditing a mechanism $M$. The second result guarantees that if the claimed privacy does not hold, the auditor will eventually identify this with probability $1$ as the sample size increases.
For the first part, we consider a scenario, where the claimed trade-off curve $T^{(0)}$ is the correct one $T^{(0)}=T$ ($M$ does not violate $T^{(0)}$-DP). For the second part, we choose a function $T^{(0)}$ above the true curve $T$ ($M$ violates $T^{(0)}$-DP). We will consider both scenarios for the Gaussian mechanism and DP-SGD.
%We will use two of the four mechanism for illustration: First, the standard Gaussian mechanism, as an example of an additive noise mechanism and second the DP-SGD mechanism, as an example of a machine learning mechanism.
%We start with auditing correctly claimed curves $T^{(0)}$. For that purpose, 
We run our auditor (Algorithm \ref{alg:auditor}) with parameters $n_1=10^4$ and $\gamma=0.05$ fixed. The choice of $\gamma=0.05$ is standard for confidence regions in statistics and we further explore the impact of $n_1$ and $\gamma$ in additional experiments in Appendix \ref{AppB}. Here, we focus on the most impactful parameter, the sample size $n_2$ and study values of  $n_2 = 10^6,10^7,10^8$. \\
Technically, the auditor only outputs a binary response that indicates whether a violation is detected or not. However, in our below experiments, we depict the inner workings of the auditor and geometrically illustrate how a decision is reached. More precisely, in Figure \ref{fig:not_faulty_sgd_gauss} we depict the claimed trade-off curve $T^{(0)}$ as a blue line. The auditor makes an estimate for the true trade-of curve $T$, namely $\hat T_h$ depicted as the orange line. The location, where the orange line (estimated DP) and the blue line (claimed DP) are the furthest apart is indicated by the vertical, dashed green line. This position is associated with the threshold $\hat \eta^*$ in Algorithm \ref{alg:pointwise_KDE_estimator}. As a second step, $\hat \eta^*$ is used in the $k$NN method to make a confidence region, depicted as a purple square (this is $\square_\gamma$ from \eqref{e:defsq}). If the square is fully below the claimed curve $T^{(0)}$, a violation is detected (Figure \ref{fig:faulty_sgd_gauss}) and if not, then no violation is detected (Figures \ref{fig:gaussian} and \ref{fig:sgd}). As we can see, detecting violations requires $n_2$ to be large enough, especially when $T^{(0)}$ and $T$ are close to each other. \\
For the incorrect $T^{(0)}$-DP claims, we have done the following: For the Gaussian case (Figure \ref{fig:faulty_sgd_gauss}), we have used a trade-off curve with parameter $\mu=0.5$ instead of the true $\mu=1$. For DP-SGD, we have used the trade-off curve corresponding to $\tau = 5$ instead of the true $\tau =10$ iterations (Figure \ref{fig:faulty_sgd_gauss}). 

%the trade-off curve of DP-SGD with a correctly claimed privacy curve and false claim. The correctly claimed can be found in Figure \ref{fig:not_faulty_sgd_gauss}. For the incorrectly claimed curve, it was stated that DP-SGD ran for only $t_{-}=5$ iterations, accessing the data just five times and thereby reinforcing the privacy guarantee. However, in reality it ran $t_{-}=10$ times, leading to a privacy breach. The results are depicted in Figure \ref{fig:faulty_sgd_gauss}.
%In this algorithm, we Algorithm \ref{alg:KDE_estimator} as a subroutine to derive an estimate $\hat \eta^*$ and we use the sample size $n_1=10^4$. Second, we run the $k$-NN with different sample sizes $n_2=10^6,10^7,10^8$. For the confidence level, we set $\gamma=0.05$, which yields confidence squares induced by $w(\gamma)$ defined in equation \eqref{e:wgamma}. 
%For the first audit in displayed in Figure \ref{fig:not_faulty_sgd_gauss}, we have audited a correctly claimed trade-off curve $T_0$. We detect a faulty mechanism, whenever the purple square $\square_\gamma=\square_{0.05}$ is disjoint from the claimed curve $T_0$. For a faulty implementation, we have use $\mu=0.5$ for the claimed curve $T_0$, while in reality the true curve only fulfills $\mu=1$. The results are displayed in Figure \ref{fig:faulty_sgd_gauss}. To complement these results with the DP-SGD, we also considered DP-SGD with a correctly claimed privacy curve and false claim. The correctly claimed can be found in Figure \ref{fig:not_faulty_sgd_gauss}. For the incorrectly claimed curve, it was stated that DP-SGD ran for only $t_{-}=5$ iterations, accessing the data just five times and thereby reinforcing the privacy guarantee. However, in reality it ran $t_{-}=10$ times, leading to a privacy breach. The results are depicted in Figure \ref{fig:faulty_sgd_gauss}.
\begin{figure*}[h!]
    \centering
    \subfloat[\centering $n_2=10^6$,\textbf{Ground Truth:} No Violation; \newline \textbf{Decision:} \textcolor{green}{"No Violation"}{\textcolor{green}{\scalebox{1.5}{\ding{51}}}}]{\includegraphics[width=0.3\textwidth]{Figures/gauss_100.png}}
    \hfill
    \subfloat[\centering $n_2=10^7$,\textbf{ Ground truth:} No Violation; \newline \textbf{Decision:} \textcolor{green}{"No Violation"}{\textcolor{green}{\scalebox{1.5}{\ding{51}}}}]{\includegraphics[width=0.3\textwidth]{Figures/gauss_100_7.png}}
    \hfill
    \subfloat[\centering $n_2=10^8$, \textbf{ Ground truth:}No Violation; \newline \textbf{Decision:} \textcolor{green}{"No Violation"}{\textcolor{green}{\scalebox{1.5}{\ding{51}}}}]{\includegraphics[width=0.3\textwidth]{Figures/gauss_100_8.png}}
   \vspace{-1em}
    \subfloat[\centering $n_2=10^6$, \textbf{Ground truth:} No Violation; \newline \textbf{Decision:} \textcolor{green}{"No Violation"}{\textcolor{green}{\scalebox{1.5}{\ding{51}}}}]{\includegraphics[width=0.3\textwidth]{Figures/sgd_100.png}}
    \hfill
    \subfloat[\centering $n_2=10^7$, \textbf{Ground truth:} No Violation; \newline \textbf{Decision:} \textcolor{green}{"No Violation"}{\textcolor{green}{\scalebox{1.5}{\ding{51}}}}]{\includegraphics[width=0.3\textwidth]{Figures/sgd_100_7.png}}
    \hfill
    \subfloat[\centering $n_2=10^8$, \textbf{Ground truth:} No Violation; \newline \textbf{Decision:} \textcolor{green}{"No Violation"}{\textcolor{green}{\scalebox{1.5}{\ding{51}}}}]{\includegraphics[width=0.3\textwidth]{Figures/sgd_100_8.png}} \caption{\textbf{Auditing a correct Mechanism:} Claimed curve $\textcolor{blue}{T^{(0)}} = T_{Gauss}$ (a,b,c) and $\textcolor{blue}{T^{(0)}} = T_{SGD}$ (d,e,f). Obtain critical vertical line with step 3 in Algorithm \ref{alg:auditor} with intercept $(\hat\alpha(\hat\eta^*), \hat \beta(\hat \eta^*))$, $k$-NN point estimator \small{\textcolor{purple}{\ding{108}}} $(\tilde\alpha(\hat\eta^*), \tilde \beta(\hat\eta^*))$ and confidence region $\textcolor{purple}{\square}$. The sample size for the KDE is $n_1=10^4$ and the confidence parameter is $\gamma=0.05$.}
    \label{fig:not_faulty_sgd_gauss}
\end{figure*}
\begin{figure*}[h!]
    \centering
    \subfloat[\centering $n_2=10^6$, \textbf{Ground truth:} Violation; \newline \textbf{Decision:} \textcolor{red}{"No Violation"}{\textcolor{red}{\scalebox{1.5}{\ding{55}}}}]{\includegraphics[width=0.3\textwidth]{Figures/gauss_faulty_100.png}}
    \hfill
    \subfloat[\centering $n_2=10^7$, \textbf{Ground truth:} Violation; \newline \textbf{Decision:} \textcolor{green}{"Violation"}{\textcolor{green}{\scalebox{1.5}{\ding{51}}}}]{\includegraphics[width=0.3\textwidth]{Figures/gauss_faulty_100_7.png}}
    \hfill
    \subfloat[\centering $n_2=10^8$, \textbf{Ground truth:} Violation; \newline \textbf{Decision:} \textcolor{green}{"Violation"}{\textcolor{green}{\scalebox{1.5}{\ding{51}}}}]{\includegraphics[width=0.3\textwidth]{Figures/gauss_faulty_100_8.png}}
    \vspace{-1em}
    \subfloat[\centering $n_2=10^6$, \textbf{Ground truth:} Violation; \newline \textbf{Decision:} \textcolor{red}{"No Violation"}{\textcolor{red}{\scalebox{1.5}{\ding{55}}}}]
    {\includegraphics[width=0.3\textwidth]{Figures/sgd_faulty_100.png}}
    \hfill
    \subfloat[\centering $n_2=10^7$, \textbf{Ground truth:} Violation; \newline \textbf{Decision:} \textcolor{red}{"No Violation"}{\textcolor{red}{\scalebox{1.5}{\ding{55}}}}]{\includegraphics[width=0.3\textwidth]{Figures/sgd_faulty_100_7.png}}
    \hfill
    \subfloat[\centering $n_2=10^8$, \textbf{Ground truth:} Violation; \newline \textbf{Decision:} \textcolor{green}{"Violation"}{\textcolor{green}{\scalebox{1.5}{\ding{51}}}}]{\includegraphics[width=0.3\textwidth]{Figures/sgd_faulty_100_8.png}}
     \caption{\textbf{Auditing a faulty Mechanism:} Claimed Curve $\textcolor{blue}{T^{(0)}} = T_{Gauss}$ (a,b,c) with $\mu=0.5$ and $\textcolor{blue}{T^{(0)}} = T_{SGD}$ (d,e,f) with $t_{-}=5$. Both mechanisms assume stronger privacy ($\mu=0.5<1$ and $t_{-}=5<10$). Critical vertical line derived by KDEs using step 3 in Algorithm \ref{alg:auditor} with intercept $(\hat\alpha(\hat\eta^*), \hat \beta(\hat \eta^*))$, $k$-NN point estimator {\textcolor{purple}{\ding{108}}} $(\tilde\alpha(\hat\eta^*), \tilde \beta(\hat\eta^*))$ and confidence region $\textcolor{purple}{\square}$. The sample size for KDE is $n_1=10^4$ and the confidence parameter is $\gamma=0.05$.}
    \label{fig:faulty_sgd_gauss}
\end{figure*}

\noindent\textbf{Implementation Details} The implementation is done using python and R. \footnote{\scriptsize{\url{https://github.com/stoneboat/fdp-estimation}}}. For the simulations, we have used a local device and a server. All runtimes were collected on a local device with an Intel Core i5-1135G7 processor (2.40 GHz), 16 GB of memory, and running Ubuntu 22.04.5, averaged over $10$ simulations. Thus, we demonstrate fast runtimes even on a standard personal computer.
Additionally, we used a server with four AMD EPYC 7763 64-Core (3.5 GHz) processors and 2 TB of memory and running Ubuntu 22.04.4 was used for repetitive simulations. For python, we have used Python 3.10.12 and the libraries "numpy" \cite{2020NumPy-Array}, "scikit-learn" \cite{pedregosa2011scikit} and "scipy" \cite{2020SciPy-NMeth}. For R, we used R version 4.3.1 and the libraries "fdrtool" \cite{fdrtool} and "Kernsmooth" \cite{Kernsmooth}. 
\begin{table}[h!]
\centering
\begin{tabular}{|l|c|}
\hline
\textbf{Algorithm}                           & \textbf{Runtime in seconds} \\ \hline
Gaussian mechanism              &  26.3                                                    \\ \hline
Laplace mechanism            &    30.51                                                     \\ \hline
Subsampling mechanism         &   27.82                                                      \\ \hline
DP-SGD           &              61.1                                         \\ \hline
 
\end{tabular}
\caption{Average runtimes of Algorithm \ref{alg:pointwise_KDE_estimator} for $n_1=10^5$ over $10$ runs to obtain the full trade-off curve $T$.}
\label{tab:running_times_KDE}
\end{table}
\begin{table}[h!]
\centering
\begin{tabular}{|l|c|}
\hline
\textbf{Algorithm}                           & \textbf{Runtime in seconds} \\ \hline
Gaussian mechanism              &    62.63                                                  \\ \hline
Laplace mechanism            &        67.04                                                 \\ \hline
Subsampling mechanism         &     66.98                                                  \\ \hline
DP-SGD           &    114.86                                                 \\ \hline
 
\end{tabular}
\caption{Average runtimes of Algorithm \ref{alg: general BayBox estimator} for $n_2=10^6$ over $5$ runs to obtain one point of the trade-off curve $T$ with confidence region.} %\\[-8ex]} 
\label{tab:running_times_kNN}
\end{table}

\subsection{Interpretation of the results}
Our experiments empirically showcase details of our methods' concrete performance. 
%refine our understanding of certain details of our methods. 
For Goal 1 (estimation), we see in Figure \ref{fig:estimation_mse} the fast decay of the estimation error of $\hat T_h$ for the optimal trade-off curve. The estimation error decays fast in $n_1$, regardless of whether there are plateau values in the sense of Assumption \ref{ass1} (e.g. Laplace Mechanism) or not (e.g. Gaussian Mechanism).
These quantitative results are supplemented by the visualizations in  
Figures~\ref{fig:gaussian}--\ref{fig:sgd}, where we depict the largest distance of $\hat T_h$ and $T$ in $1000$ simulation runs (captured by the red band). Even for the modest sample size of $n_1 = 10^3$, this band is fairly tight and for $n_1 = 10^5$ the estimation error is almost too minute to plot. We find this convergence astonishingly fast. It may be partly explained by the estimator $\hat T_h$ being structurally similar to $T$ -  after all $\hat T_h$ is also designed to be a trade-off curve for an almost optimal LR test.
The approximation over the entire unit interval corresponds to the uniform convergence guarantee in Theorem~\ref{theo:1}. 
%demonstrate that even with relatively small sample sizes, such as $n_1 = 10^3$, the worst global error across 1000 simulations remains notably small. For that observe that the combined worst deviation from the true curve $T_0$ across $1000$ simulations is already for $n_1=10^4$ almost negligible. As the sample size increases even further, the error converges to zero for all $\alpha\in[0,1]$ as visible in Figures~\ref{fig:gaussian}--\ref{fig:sgd} (c). This aligns with the uniform convergence established in Theorem~\ref{theo:1}. In addition to that, throughout all mechanisms, Figure \ref{fig:estimation_mse} illustrates the convergence for the same set of parameters, highlighting the robustness and adaptability necessary in a black-box setting. Here, we also emphasize that the MSEs are computed on an equidistant grid evaluated on the KDEs. This distinction is important because, in principle, for an equidistant $\eta$, the distribution of the $\alpha$ values could theoretically concentrate on a few points. This behavior is especially observable whenever the quotient of the densities is constant for a non-negligible subset of $[0,1]$. An example of that would be the Laplace mechanism.
%To address this potential issue, we have evaluated the error on both the grid implied by the KDE and an equidistant grid. Through this comparison, we have observed that such concentration does not impact the estimation. If this phenomenon arises, a linear interpolation is sufficient, as the trade-off curve is also linear on that subset, and if it does not arise, then the $\alpha$'s are also evenly distributed. \todo{@Tim: do you agree with me? I think it is important to make this remark, as someone could be a bit confused when observing this clustered results for the Laplacian algorithm}
% \\

For Goal 2 (inference), we recall that a  $T^{(0)}$-DP violation is detected if the box $\square_\gamma$ (purple) lies completely below the postulated curve $T^{(0)}$ (blue). In Figure \ref{fig:not_faulty_sgd_gauss} we consider the case of no violation where $T=T^{(0)}$, and we expect not to detect a violation. This is indeed what happens, since $\square_\gamma$ intersects with the curve $T^{(0)}$ in all considered cases. Interestingly, we observe that $\square_\gamma$ has a center close to $\alpha=0$ in the cases where no violation occurs (such a behavior might give additional visual evidence to users that no violation occurs).
%In principal, one would reject the privacy curve, whenever the purple square $\textcolor{purple}{\square}$ is disjoint from the blue \textcolor{blue}{curve}, i.e.
%\begin{equation*}
%    \textcolor{purple}{\square} \cap \textcolor{blue}{\textnormal{curve}}=\emptyset~.
%\end{equation*}
%In Figure \ref{fig:not_faulty_sgd_gauss} and \ref{fig:not_faulty_sgd_gauss} we have displayed the case where the claimed privacy indeed holds, so we expect to not detect a violation. For both mechanism, we can observe that for any sample size, we correctly do not reject that claim, as the $\textcolor{purple}{\square}$ and the $\textcolor{blue}{curve}$ are not disjoint. 
In Figure \ref{fig:faulty_sgd_gauss}, we display the case of faulty claims, where the privacy breach is caused by a smaller variance for both mechanisms under investigation. In accordance with Theorem \ref{theo:auditor}, we expect a detection of a violation if $n_2$ is large enough. This is indeed what happens, at a sample size of $n_2=10^7$ for the Gaussian mechanism and at  $n_2=10^8$ for DP-SGD. As we can see, larger samples $n_2$ are needed to expose claims $T^{(0)}$ that are closer to the truth $T$ (as for DP-SGD in our example). For larger $n_2$ the square $\square_\gamma$ shrinks (see eq. \eqref{e:defsq}) leading to a higher resolution of the auditor. 
%While for both mechanisms, we do not detect a statistical significant deviation for $n_2=10^6$ (since the $\textcolor{purple}{\square}$ is not disjoint from the blue \textcolor{blue}{curve}), already for $n_2=10^7$, the auditor detects the privacy violation for the Gaussian mechanism. Regarding the DP-SGD case, we have to increase the sample size to $n_2=10^8$ to detect the violation. The clear message here is that smaller privacy violations (curves are closer to each other), the larger the sample size $n_2$ has to be, to obtain small confidence regions. This is a classical pattern in statistics and aligns with Theorem \ref{theo:auditor} part (2). The main reason here is that higher $n_2$ significantly shrinks the confidence square (recall Theorem \label{thm: accuracy stat of kNN BayBox estimator}
%), which eventually will be fully below the \textcolor{blue}{curve}, indicating a statistically significant difference. Here, we explicitly want to point out that even though e.g. Figure \ref{fig:faulty_sgd_gauss} (a) was not significant, one can already take the deviation from the claimed curve as a first indication and just increase the sample size for a stronger evidence. Consequently, we strongly encourage users to also consider the point estimator derived from the $k$-NN algorithm as a potential indicator of faulty implementations. In fact, our empirical evidence indicates that the confidence interval for our $k$-NN based point estimator is significantly narrower than the theoretical bound we derived. This discrepancy arises because the theoretical convergence rate of the $k$-NN algorithm is generally not tight; in practice, the performance of the $k$-NN algorithm converges more rapidly than the theoretical rate suggests. To put this into more practical observations, the difference of the estimators derived for $n_2=10^6,10^7,10^8$ were for all mechanism negligible.


\section{Conclusion}\label{sec:conclusion}
\section{Conclusion}
\label{sec:conclusion}

This study investigated the application of supervised ML models as a supporting tool for researchers during study selection in SLR updates. Therefore, we developed a supervised ML pipeline for the study selection process. The focus was on investigating the effectiveness of ML models, the potential to reduce human effort, and the ability to provide support to individual human reviewers. We employed two ML models, Random Forest (RF) and Support Vector Machine (SVM), and assessed them on a dataset derived from a carefully manually curated SLR update. During this investigation process, our work also highlighted different configurations used for our ML models that correlate to their recall and F-score, providing results that can be useful for further exploration in this area.

Our results indicate that while ML can assist in preliminary study selection by reducing the volume of studies requiring manual review, it is not yet effective enough to automate this process or to directly assist a single human reviewer to produce more accurate selection results. Specifically, RF, our best model for study selection effectiveness, achieved a modest F-score of 0.33, with limited precision and recall, which is clearly insufficient for study selection. Meanwhile, SVM demonstrated potential in reducing effort by excluding up to 33.9\% of irrelevant studies without sacrificing recall. The comparison between human-only reviewer pairs and human-ML reviewer pairs for the initial screening showed that pairs of human reviewers produce results that are much better aligned with the final curated result of the SLR update. 

Considering our findings, we put forward that serious SLR update efforts should still rely on (at least two) experienced human researchers for the initial screening of papers to be included. Hence, this study contributes to understanding the practical limitations of ML in study selection and highlights the need for careful human involvement in this process to ensure the quality and rigor of SLR outcomes. 

Future research could focus on refining ML configurations, investigating adaptive thresholds to improve model performance in SLR update contexts, and exploring hybrid approaches (\textit{e.g.}, humans assisted by ML to reduce the overall screening effort by discarding studies with low probability of being included). We also recommend further investigating large language models (LLMs) within the SLR update context. 



\newpage
\bibliographystyle{format/IEEEtran}
\bibliography{bib/bib}

\end{document}
