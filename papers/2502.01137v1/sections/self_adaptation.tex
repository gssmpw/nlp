\section{Self-adaptation}\label{sec:self_adaptation}

\subsection{Role Cardinality} 

The relation between the number of instances of a given role and different non-functional attributes of a system may be straightforward (e.g., the number of similar sensors in a cluster and the accuracy of the information this cluster provides) or less evident (e.g., the number of replicas of a component and the availability or reliability of the functionality it provides). In addition, considering the scenario of mobile applications, the mapping between role cardinality and these attributes tend to change according to the context they operate, i.e., it may depend on which devices are assigned to the role (e.g., more/less powerful computational resources, more/less accurate sensors) and the state of each device (e.g., more/less battery, more/less signal strength). Hence, in order to avoid violations of these attributes, the cardinality of a role must reflect the circumstances. %To this end, we propose a last, but not least important mechanism for adapting the application organization. 

\subsection{Group Cardinality} 

Since the encounter of mobile application nodes is opportunistic and not deterministic, group membership criteria specifies no size threshold or limit: as long as a node satisfies the group's membership criteria, it becomes/remains a member. This choice avoids the need for a group membership control by a trustful component and simplifies the grouping process. Accordingly, the size of a group, kept stable the membership criteria satisfaction by each of its members, can only be changed by making this criteria more relaxed or strict. 

Whereas the self-organization mechanisms presented address the \textit{solution space}, i.e., which nodes provide which application functionality, here the \textit{requirement space} is the target of the adaptation, i.e., the application requirements evolve to accommodate situations in which no solution can satisfy those requirements~\cite{RELAX}. 
