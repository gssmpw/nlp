\section{Related work}
\label{sec-related}

Table~\ref{tbl-automl-features} compares the features of various AutoML
solutions; as highlighted in Section~\ref{sec-background}, few offer
integrated support for survival analysis and data preprocessing is not
always within the optimization loop.

\begin{table}[htbp]

\caption{\label{tbl-automl-features}Comparison of AutoML frameworks and
their native support for survival analysis and dynamic data
preprocessing}

\centering{

\begin{tabular}{llll}
\toprule
Framework & Method & Preprocessing & Survival\\
\midrule
Amazon SageMaker Autopilot & Bayesian optimization and ensembles & Yes & No\\
auto-keras & Neural architecture search & Yes & No\\
auto-sklearn & Bayesian optimization & Limited & No\\
AutoGluon & Stack ensembling & Yes & No\\
Azure AutoML & Bayesian optimization and meta-learning & Yes & No\\
\addlinespace
BigML & Decision tree-based optimization & Yes & No\\
DataRobot & Proprietary ensemble and optimization & Yes & Limited\\
FLAML & Cost-aware Bayesian optimization & No & No\\
Google AutoML Tables & Neural architecture search & Yes & No\\
H2O AutoML & Random search and stacked ensembles & Yes & No\\
\addlinespace
MLflow & Manual configuration & Yes & No\\
MLJAR & Random search and stacked ensembles & Yes & No\\
PyCaret & Iterative search with pipeline tuning & Yes & Limited\\
TPOT & Genetic programming & No & No\\
\bottomrule
\end{tabular}

}

\end{table}%

Of frameworks offering automated data cleaning, DataRobot is
proprietary, commercial platform and only appears to offer time-to-event
modelling via a `hack' of converting the task to a classification
problem via discretization ____.
Meanwhile, H2O.ai can run
\href{https://docs.h2o.ai/h2o/latest-stable/h2o-docs/data-science/coxph.html}{Cox
proportional hazards models} as a fixed model, but not via its AutoML
interface.

However, some dedicated data cleaning solutions have been proposed.
____ proposed Auto-Prep, an interactive Python-based tool
that recommends data cleaning methods to the user based on application
of candidate techniques and subsequent evaluation using simple
classifiers or regression models. In a review of data preprocessing in
AutoML (Section C) the authors highlight the capabilities, or lack
thereof, of AutoML tools to perform data preprocessing and feature
engineering without manual human intervention. Another Python package,
Atlantic ____, automates preprocessing steps including
feature engineering and missing value imputation for supervised learning
tasks. The framework identifies the best combination of steps based on
evaluation using tree-based model ensembles.

____ developed \texttt{Learn2Clean}, a tool offering
an innovative approach to data preprocessing. It leverages
\(Q\)-Learning, a reinforcement learning technique, to dynamically
select the optimal sequence of preprocessing tasks for a given dataset
and ML model. This optimization aims to maximize the quality of the ML
model's results. \texttt{Learn2Clean} implements automated data
preprocessing for regression, classification and clustering tasks, using
\(Q\)-learning to optimize respective evaluation criteria: mean squared
error, accuracy and silhouette index. However, \texttt{Learn2Clean}
limitations include lack of built-in support for survival analysis,
categorical data types, flexible hyperparameter tuning and custom reward
functions. It also has a complex dependency structure, which can make
initial setup challenging for end-users.

MLsurvival ____ described an automated tool for cancer
survival prediction that removes or imputes missing values, selects and
standardizes features, trains survival models and then makes
predictions. Unfortunately, neither a full text article nor open source
implementation of the method were published. More recently,
____ proposed an AutoML system for survival analysis
based on genetic algorithms and a combination of elastic-net Cox models,
random survival forests and survival trees, optimizing for C-index.
However, the tool does not incorporate data preprocessing.