%%%%%%%% ICML 2024 EXAMPLE LATEX SUBMISSION FILE %%%%%%%%%%%%%%%%%

\documentclass{article}

% Recommended, but optional, packages for figures and better typesetting:
\usepackage{microtype}
\usepackage{graphicx}
% \usepackage{subfigure}
\usepackage{booktabs} % for professional tables

% hyperref makes hyperlinks in the resulting PDF.
% If your build breaks (sometimes temporarily if a hyperlink spans a page)
% please comment out the following usepackage line and replace
% \usepackage{icml2024} with \usepackage[nohyperref]{icml2024} above.
\usepackage{hyperref}


% Attempt to make hyperref and algorithmic work together better:
\newcommand{\theHalgorithm}{\arabic{algorithm}}

% Use the following line for the initial blind version submitted for review:
% \usepackage{icml2024}

% If accepted, instead use the following line for the camera-ready submission:
\usepackage[accepted]{icml2024}

% For theorems and such
\usepackage{amsmath}
\usepackage{amssymb}
\usepackage{mathtools}
\usepackage{amsthm}
\usepackage{subcaption}
% \usepackage{subfigure}

% if you use cleveref..
\usepackage[capitalize,noabbrev]{cleveref}

%%%%%%%%%%%%%%%%%%%%%%%%%%%%%%%%
% THEOREMS
%%%%%%%%%%%%%%%%%%%%%%%%%%%%%%%%
\theoremstyle{plain}
\newtheorem{theorem}{Theorem}[section]
\newtheorem{proposition}[theorem]{Proposition}
\newtheorem{lemma}[theorem]{Lemma}
\newtheorem{corollary}[theorem]{Corollary}
\theoremstyle{definition}
\newtheorem{definition}[theorem]{Definition}
\newtheorem{assumption}[theorem]{Assumption}
\theoremstyle{remark}
\newtheorem{remark}[theorem]{Remark}

% Todonotes is useful during development; simply uncomment the next line
%    and comment out the line below the next line to turn off comments
%\usepackage[disable,textsize=tiny]{todonotes}
\usepackage[textsize=tiny]{todonotes}


% The \icmltitle you define below is probably too long as a header.
% Therefore, a short form for the running title is supplied here:
\icmltitlerunning{The AI / ML Community Should Adopt a More Transparent and Regulated Peer Review Process}

\begin{document}

\twocolumn[
% \icmltitle{Position: Advocating for a More Transparent and Regulated Peer Review in the AI / ML Community}
% \icmltitle{Position: The Artificial Intelligence and Machine Learning Community Should Adopt a More Transparent and Regulated Peer Review Process}
\icmltitle{Paper Copilot: The Artificial Intelligence and Machine Learning Community Should Adopt a More Transparent and Regulated Peer Review Process}

% It is OKAY to include author information, even for blind
% submissions: the style file will automatically remove it for you
% unless you've provided the [accepted] option to the icml2024
% package.

% List of affiliations: The first argument should be a (short)
% identifier you will use later to specify author affiliations
% Academic affiliations should list Department, University, City, Region, Country
% Industry affiliations should list Company, City, Region, Country

% You can specify symbols, otherwise they are numbered in order.
% Ideally, you should not use this facility. Affiliations will be numbered
% in order of appearance and this is the preferred way.
\icmlsetsymbol{equal}{*}

\begin{icmlauthorlist}
\icmlauthor{Jing Yang}{equal,USC,papercopilot}
% \icmlauthor{Firstname2 Lastname2}{equal,yyy,comp}
% \icmlauthor{Firstname3 Lastname3}{comp}
% \icmlauthor{Firstname4 Lastname4}{sch}
% \icmlauthor{Firstname5 Lastname5}{yyy}
% \icmlauthor{Firstname6 Lastname6}{sch,yyy,comp}
% \icmlauthor{Firstname7 Lastname7}{comp}
%\icmlauthor{}{sch}
% \icmlauthor{Firstname8 Lastname8}{sch}
% \icmlauthor{Firstname8 Lastname8}{yyy,comp}
%\icmlauthor{}{sch}
%\icmlauthor{}{sch}
\end{icmlauthorlist}

\icmlaffiliation{USC}{University of Southern California}
\icmlaffiliation{papercopilot}{Paper Copilot}
% \icmlaffiliation{sch}{School of ZZZ, Institute of WWW, Location, Country}

\icmlcorrespondingauthor{Jing Yang}{jingyang.carl.work@gmail.com}
% \icmlcorrespondingauthor{Firstname2 Lastname2}{first2.last2@www.uk}

% You may provide any keywords that you
% find helpful for describing your paper; these are used to populate
% the "keywords" metadata in the PDF but will not be shown in the document
\icmlkeywords{Machine Learning, ICML}

\vskip 0.3in
]

% this must go after the closing bracket ] following \twocolumn[ ...

% This command actually creates the footnote in the first column
% listing the affiliations and the copyright notice.
% The command takes one argument, which is text to display at the start of the footnote.
% The \icmlEqualContribution command is standard text for equal contribution.
% Remove it (just {}) if you do not need this facility.

% \printAffiliationsAndNotice{}  % leave blank if no need to mention equal contribution
% \printAffiliationsAndNotice{\icmlEqualContribution} % otherwise use the standard text.

\begin{abstract}
The rapid growth of submissions to top-tier Artificial Intelligence (AI) and Machine Learning (ML) conferences has prompted many venues to transition from closed to open review platforms. Some have fully embraced open peer reviews, allowing public visibility throughout the process, while others adopt hybrid approaches, such as releasing reviews only after final decisions or keeping reviews private despite using open peer review systems. In this work, we analyze the strengths and limitations of these models, highlighting the growing community interest in transparent peer review. To support this discussion, we examine insights from Paper Copilot, a website launched two years ago to aggregate and analyze AI / ML conference data while engaging a global audience. The site has attracted over 200,000 early-career researchers, particularly those aged 18–34 from 177 countries, many of whom are actively engaged in the peer review process. \textit{Drawing on our findings, this position paper advocates for a more transparent, open, and well-regulated peer review aiming to foster greater community involvement and propel advancements in the field.}

% \textcolor{red}{the importance of rebuttal, and majority researchers are 18-24 years old.}

\end{abstract}

% 1. The Title should state the position and start with “Position:”.
%   * These hypothetical paper titles do state a position:
%       * "Position: Quantum Atelic Learning Methods Should Employ Psychic Insights"
%       * "Position: Stop Research on Psychic Properties of Machine Learning"
%   * while these versions do not:
%       * "Position: Psychic Quantum Atelic Learning"
%       * "Position: A Perspective on Psychic Quantum Atelic Learning"
% 2. The Abstract must identify the paper as a position paper and briefly state the position (e.g., “This position paper argues that <statement of the position>.”)
% 3. The Introduction must state the position, using bold text.
% 4. The paper must include an “Alternative Views” section that describes and addresses one or more viable (not strawmen) positions that are opposed to the paper’s position.
% 5. Papers that describe new research without advocating a position are not responsive to this call and should instead be submitted to the main paper track.

% Position: AI/ML Influencers Have a Place in the Academic Process

\section{Introduction}
\label{sec:intro}
% Image editing methods in diffusion models depend on user-defined control directions - users can unlock their creativity using these methods by specifying the desired manipulation through prompts~\cite{gandikota2023concept}, reference images~\cite{ruiz2022dreambooth, kumari2022customdiffusion, gal2022image, chen2024trainingfreeregionalpromptingdiffusion}, or attribute vectors~\cite{parmar2023zero,hertz2022prompt}. In this work, we ask a fundamentally different question: \emph{Can we automatically discover the underlying visual structure of a concept within diffusion model's knowledge?} %Rather than requiring user-specified controls, we aim to decompose the model's internal knowledge into meaningful directions.

% This question touches on a fundamental limitation in how we interact with diffusion models. Current control methods ~\cite{zhang2023addingconditionalcontroltexttoimage, gandikota2023concept, ye2023ipadaptertextcompatibleimage,ye2023ipadaptertextcompatibleimage, hertz2024stylealignedimagegeneration, li2023photomaker, shi2024instantbooth, chen2024trainingfreeregionalpromptingdiffusion} require users to specify their desired manipulations in advance, limiting interactive creativity. This contrasts with natural human artistic workflows, where creators dynamically explore creative ideas while jointly refining them toward meaningful artistic outcomes~\cite{hoffmann2016modeling}. This synergy between specification and exploration is not new to generative models. Early GAN architectures naturally developed disentangled latent spaces that enabled continuous\cite{harkonen2020ganspace,radford2015unsupervised, wu2021stylespace, shen2020interfacegan}, compositional control over generated images. Users could explore these spaces to discover interesting variations that would be difficult to describe in words~\cite{wu2021stylespace}, then combine them to achieve their creative goals~\cite{grabe2022towards}. 


% While diffusion models have largely superseded GANs in conditional image synthesis~\cite{dhariwal2021diffusion},  their underlying structure remains less understood. Diffusion models achieve remarkable diversity through high-dimensional latents, unlike GANs' compact latent spaces.  With a single prompt, diffusion models can generate radically different variations through different random initializations of input noise. We ask - Is it possible to discover interpretable structure within this vast space of variations?

Text-to-image diffusion models are capable of generating remarkable visual variations from a single prompt through different random initializations. However, this vast creative potential remains largely opaque to users---while we can generate diverse images, we lack understanding of the underlying structure of these variations. This presents a fundamental challenge: how can we discover and expose the latent visual capabilities encoded within these models?

\let\thefootnote\relax \footnote{$^{*}$Correspondence to \texttt{gandikota.ro@northeastern.edu}}

The challenge touches on a key limitation in how we interact with diffusion models today. Current control methods require users to explicitly specify their desired edits in advance through prompts~\cite{gandikota2023concept}, reference images~\cite{zhang2023addingconditionalcontroltexttoimage, chen2024trainingfreeregionalpromptingdiffusion, ruiz2022dreambooth,kumari2022customdiffusion, Ryu_lora, hu2021lora}, or attribute vectors~\cite{ye2023ipadaptertextcompatibleimage, hertz2024stylealignedimagegeneration, li2023photomaker, shi2024instantbooth,parmar2023zero,hertz2022prompt}. That contrasts sharply with natural human creative workflows, where artists dynamically explore creative ideas and jointly refine them toward meaningful artistic outcomes~\cite{hoffmann2016modeling}. The need for pre-specified controls creates a barrier between users and the full creative potential of these models.

Interestingly, earlier generative models like GANs~\cite{gans,karras2019style,brock2018large} naturally developed more interpretable internal structures. Their compact latent spaces often exhibited emergent disentanglement~\cite{harkonen2020ganspace,radford2015unsupervised, wu2021stylespace, shen2020interfacegan}, enabling continuous and compositional control over generated images. Users could explore these spaces to discover interesting variations that would be difficult to describe in words~\cite{wu2021stylespace}, then combine them to achieve their creative goals~\cite{grabe2022towards}.

Diffusion models have largely superseded GANs in conditional image synthesis~\cite{dhariwal2021diffusion}, achieving greater diversity through much higher-dimensional latents. And yet an understanding of the underlying structure of these larger latent spaces has remained elusive. In this work, we ask a fundamental question: \emph{Can we automatically discover the visual structure within a diffusion model's knowledge of a concept?} Rather than requiring user-specified controls, we aim to decompose the model's internal representations into expressive directions that users can explore and combine.

To address these needs, we present \textbf{SliderSpace}, a framework that brings systematic explorability to diffusion models. Given just a text prompt, SliderSpace discovers a canonical set of meaningful, diverse, and controllable directions within the model's knowledge of that concept. Each direction is implemented as a low-rank adapter~\cite{hu2021lora} that can be scaled and composed with others, allowing users to explore and smoothly combine different aspects of variation, as shown in Figure~\ref{fig:intro}.

We ground SliderSpace discovery in three key requirements for meaningful decomposition of a diffusion model's visual manifold: 
\begin{enumerate}
    \item \textbf{Unsupervised Discovery:} The decomposition process should emerge from the intrinsic structure of the model's learned representation, rather than being guided by predefined attributes. This ensures we capture the true topology of the model's knowledge space rather than projecting our assumptions onto it.
    
    \item \textbf{Semantic Orthogonality:} Each discovered control must represent a distinct semantic direction. This is enforced in a semantic feature space, like CLIP, where every slider has an orthogonal effect in embeddings. This prevents discovering multiple controls that create similar semantic effects, making the system more efficient and easier.
    
    \item \textbf{Distribution Consistency:} Directions must induce consistent transformations across both random seeds and prompt variations. 
\end{enumerate}

These requirements naturally lead to our proposed framework, which we formalize in Section~\ref{sec:method}. As we show in our experiments, SliderSpace is architecture-agnostic, working with both conventional U-Net based models like Stable Diffusion~\cite{rombach2022high, rombach2022sd20, podell2023sdxl, turbo, dmd} and recent transformer-based architectures like Flux~\cite{flux}.

We demonstrate the expressiveness of SliderSpace through three applications: First, we show how SliderSpace can decompose high-level concepts into diverse and expressive components, revealing the natural axes of variation in the model's understanding. Second, we explore artistic style variation, where SliderSpace discovers directions that match or exceed the diversity of manually curated artist lists while being judged more useful by human evaluators. Finally, we show how SliderSpace can help reverse the mode collapse commonly observed in distilled diffusion models, restoring diversity while maintaining generation speed.

Beyond providing practical creative control, SliderSpace opens new avenues for understanding and utilizing the latent capabilities of diffusion models. By mapping these models' visual potential into intuitive, composable directions, we take a step toward making their creative possibilities more accessible and interpretable to users.

% Image editing methods in diffusion models unlock the creativity of users. In this work we ask an alternate question: \emph{Can we organize and expose what of the diffusion model is already capable of?}.
% Existing methods for controlling image generation typically require users to manually specify edit directions for desired changes. This process is time-consuming, requires technical expertise, and limits the spontaneity of the creative process. For instance, if a user wants to adjust the smile of a generated person, they must explicitly request this edit, often through imprecise prompt engineering or model fine-tuning. This approach of predefined controls or manual specifications restricts users from fully exploring the latent capabilities of the model. There may be interesting stylistic variations or attributes that the model can generate, but users have no easy way to discover or utilize these.

% Natural visual disentanglement was an emergent property in the latent space of Generative Adversarial Models (GANs) \cite{harkonen2020ganspace,radford2015unsupervised, wu2021stylespace, shen2020interfacegan}. In particular, it has been observed that StyleGAN~\cite{karras2019style} stylespace neurons offer detailed control over many meaningful aspects of images that would be difficult to describe in words~\cite{wu2021stylespace}. However, diffusion models do not share such a compact latent space~\cite{park2023unsupervised}; and efforts to uncover such a space in the semantic embeddings of the text conditioning have met with limited success \nik{Nick - is there a specific citation you were thinking about?}.

% In this work we introduce \textbf{SliderSpace}, which takes a step towards uncovering an analogous low dimensional representation of diffusion models' visual breadth; in essence treating the diffusion model as many generators sharing parameters, where a particular generator is defined by a specific prompt. For a given prompt we sample many random seeds (and optionally prompt expansions using an LLM), generate the corresponding images, and apply an off the shelf feature extractor (in this work CLIP, but our method can be applied to any differentiable feature extractor). We use PCA to analyze these features, and for each of the leading $k$ principal components we train a LoRA \cite{} which causes the diffusion model to produces images which increase the feature magnitude along that component when passed back through the same feature extractor. This leads to a 'Slider' for each principal component, because each LoRA can be scaled and applied to the original diffusion model, continuously varying those visual features in the generated results (as measured, in our case, by CLIP).

% There are many other works that enhance the controllability of diffusion models. One common approach is enabling users to add spatial constraints to a generation either manually, or via a reference image \cite{zhang2023addingconditionalcontroltexttoimage, chen2024trainingfreeregionalpromptingdiffusion}, a second is leveraging more abstract embeddings (e.g. identity, style) extracted from a reference image \cite{ye2023ipadaptertextcompatibleimage, hertz2024stylealignedimagegeneration, li2023photomaker, shi2024instantbooth}, a third is finetuning a foundation model to better generate a concept important to the user \cite{ruiz2022dreambooth, kumari2022customdiffusion, Ryu_lora, hu2021lora}, and a fourth (most relevant to this work) is finding low-rank adaptors of the model based on a prompt or small training set which can be scaled to provide continous control over one aspect of generated image (e.g. night vs day, basic vs luxury, etc.) \cite{gandikota2023concept}. SliderSpace is complementary to all of these methods and offers something distinct. All of the other methods we are aware require the user (and / or model designer) to know in advance what type of control they want. In contrast SliderSpace assists users in discovering and controlling hidden capabilities present in the diffusion model's distribution of possible generations.

%We propose that truly intuitive creative control in a text-to-image model should meet three key criteria: \emph{discoverability}, \emph{intuitiveness}, and \emph{specificity}. The model should reveal controllable attributes that may not be immediately obvious, offer controls that are easy to understand and manipulate, and ensure each control affects a distinct attribute of the generated image.

% We demonstrate the utility and power of SliderSpace using three applications built on top of SDXL-DMD \cite{dmd}, because its fast generation speed lends itself well to the continuous control offered by SliderSpace.

% First, we study concept decomposition (Section \ref{sec:concept_exp}), where we learn sliders for a specific concept (e.g. 'monster', 'waterfall', 'car'). Through quantitative metrics of diversity and text alignment we demonstrate that the learned sliders dramatically boost the diversity of generations when randomly applied without harming text alignment; we also ask humans to qualitatively judge these results in a user study where they find the SliderSpace results to be more 'Diverse', 'Useful', and 'Creative' than our baselines.

% Second, we attempt to compare the automatic discoveries of SliderSpace to a large scale manual study of artistic styles (Section \ref{sec:art_exp}), open-sourced by ParrotZone \cite{parrotzone}. In this study SDXL was prompted with over 4300 artist names,  and based on visual inspection the cases of successful stylistic mimicry recorded. Quantitatively SliderSpace more closely matches the distribution of artistic variation discovered by ParrotZone than other baselines, and in our user studies was judged to be significantly more 'Diverse' and 'Useful' than the baselines. To our surprise humans even judged SliderSpace results to be slightly more 'Diverse' than the results generated by the manually discovered artist names of \cite{parrotzone}.

% Third, we attempt to use SliderSpace to reverse the mode collapse commonly observed in distilled few-step diffusion models relative to the original teacher model (Section \ref{sec:diverse_exp}). We quantitatively demonstrate that applying SliderSpace to SDXL-DMD leads to more closely matching the distribution of images by the original teacher, SDXL.

%Through extensive experiments on various state-of-the-art text-to-image models, we demonstrate that SliderSpace significantly enhances user control and creative expression in AI-assisted image generation tasks. Our method enables a range of applications, including concept decomposition and control, diversity improvement in generated images, customization dissection and edits, and the exploration of artistic styles inherent in the model.

% SliderSpace goes beyond providing a practical tool for enhanced creative control. By mapping the visual potential of diffusion models it can open new avenues for generative creativity and deepens our understanding of each model's hidden potential.
\section{Related Works}
% \textcolor{red}{GPT: replace citations}

\subsection{Open Peer Review}

Open peer review (OPR) enhances transparency by publishing reviews, revealing reviewer identities, or enabling public discussions~\cite{ross2017open, henriquez2023open, wolfram2020open}. In AI and ML, OpenReview~\cite{openreview_api} has facilitated OPR, with ICLR pioneering public discourse alongside formal reviews~\cite{wang2023have}. Proponents argue that open reviews improve feedback quality, help reviewers refine their assessments~\cite{church2024peer}, and enable confidence estimation from review text~\cite{bharti2022confident}. However, experiments at NeurIPS reveal inconsistencies in peer review~\cite{cortes2021inconsistency, Lawrence2022NeurIPSExperiment, beygelzimer2023has}, raising concerns about subjective scoring~\cite{xie2024reviewer} and the impact of increasing submissions~\cite{tran2021an}. Some studies suggest interventions to reduce uncertainty in reviewer judgments~\cite{chen2023judgment} or explore author self-assessments as a complement to peer review~\cite{su2024analysis}.

Despite its benefits, OPR within double-blind settings poses challenges. Publishing reviews, even anonymously, may reveal sensitive details or invite targeted criticism~\cite{tran2021an}. Computational studies highlight fairness disparities in peer review~\cite{zhang2022investigating}, and alternatives like managing research evaluation on GitHub have been proposed~\cite{takagi2022managing}. Broader concerns persist, including whether reviewing efforts align with academic impact~\cite{church2024peer} and how best to address systemic biases~\cite{shah2022challenges}. As NeurIPS discussions occur mid-year and ICLR discussions happen later, the timing of transparency measures may also shape reviewer behavior and decision-making.

% An Open Review of OpenReview: A Critical Analysis of the Machine Learning Conference Review Process \cite{tran2021an}

% Inconsistency in conference peer review: Revisiting the 2014 neurips experiment \cite{cortes2021inconsistency}

% The NeurIPS Experiment \cite{Lawrence2022NeurIPSExperiment}

% Challenges, experiments, and computational solutions in peer review \cite{shah2022challenges}

% Has the machine learning review process become more arbitrary as the field has grown? The NeurIPS 2021 consistency experiment \cite{beygelzimer2023has}

% Judgment sieve: Reducing uncertainty in group judgments through interventions targeting ambiguity versus disagreement\cite{chen2023judgment}

% Contrastive Explanations That Anticipate Human Misconceptions Can Improve Human Decision-Making Skills \cite{buccinca2024contrastive}

% Investigating fairness disparities in peer review: A language model enhanced approach \cite{zhang2022investigating}

% How confident was your reviewer? estimating reviewer confidence from peer review texts \cite{bharti2022confident}

% Are reviewer scores consistent with citations? \cite{xie2024reviewer}

% Analysis of the ICML 2023 Ranking Data: Can Authors' Opinions of Their Own Papers Assist Peer Review in Machine Learning? \cite{su2024analysis}

% Is Peer-Reviewing Worth the Effort?\cite{church2024peer}

% Managing the Whole Research Process on GitHub \cite{takagi2022managing}

% What have we learned from OpenReview? \cite{wang2023have}

% Open peer review (OPR) encompasses a spectrum of practices that aim to increase the transparency of the reviewing process, ranging from publishing reviewer identities and comments to enabling public commenting on manuscripts \cite{ross2017open, henriquez2023open}. The approach is intended to foster accountability, reduce biases, and promote more constructive critique. In machine learning and AI domains, conferences such as ICLR have pioneered variations of OPR on platforms like OpenReview, allowing public commentary alongside official reviews. Proponents argue that exposing the reasoning behind acceptance or rejection can provide valuable feedback to authors while also helping reviewers refine their assessments.

% Nonetheless, implementing OPR in double-blind settings requires careful balancing of transparency and anonymity. Even without revealing identities, publishing reviews can inadvertently disclose sensitive information or lead to targeted criticism \cite{ross2017open}. Critics also contend that reviewers, especially early-career researchers, might be reluctant to offer candid assessments if their comments are made public \cite{bianchi2023state}. Thus, while OPR has gained traction across some AI and ML venues, questions remain regarding its effect on review quality and reviewer willingness to participate.

\subsection{Regulations}

As OPR evolves, regulatory guidelines ensure integrity, fairness, and privacy~\cite{ross2019guidelines}. Some researchers caution that excessive transparency may undermine review quality~\cite{bianchi2022can}, while others highlight the challenge of balancing confidentiality with open science~\cite{baez2002confidentiality, dennis2019privacy}.

AI/ML conferences face additional regulatory challenges. Public review platforms can expose researchers to scrutiny or harassment, raising ethical concerns~\cite{wang2023have}. AI-powered peer review introduces risks that require human oversight~\cite{seghier2024ai}, while plagiarism in review reports and the rise of review mills threaten review integrity~\cite{piniewski2024emerging, oviedo2024review, ezhumalai2024design}. To address these risks, researchers advocate for clearer policies on reviewer disclosures, public critique, and misconduct prevention, ensuring transparency strengthens rather than undermines the review process~\cite{kaltenbrunner2022innovating, kuznetsov2024can}.

% What have we learned from OpenReview? \cite{wang2023have}

% Confidentiality and peer review: The paradox of secrecy in academe \cite{baez2002confidentiality}

% Can transparency undermine peer review? A simulation model of scientist behavior under open peer review \cite{bianchi2022can}

% Guidelines for open peer review implementation \cite{ross2019guidelines}

% Innovating peer review, reconfiguring scholarly communication: An analytical overview of ongoing peer review innovation activities \cite{kaltenbrunner2022innovating}

% Emerging plagiarism in peer-review evaluation reports: a tip of the iceberg? \cite{piniewski2024emerging}

% What Can Natural Language Processing Do for Peer Review? \cite{kuznetsov2024can}


% AI-powered peer review needs human supervision \cite{seghier2024ai}

% Design of an Integrated Collaborative Environment for Projects with Plagiarism Checker \cite{ezhumalai2024design}

% The review mills, not just (self-) plagiarism in review reports, but a step further \cite{oviedo2024review}

% Privacy versus open science \cite{dennis2019privacy}


% As open peer review practices evolve, various regulatory guidelines and ethical frameworks have emerged to maintain standards of integrity, fairness, and privacy \cite{cope2021guidelines, ieee2024guidelines}. The Committee on Publication Ethics (COPE), for instance, provides recommendations on managing conflicts of interest, ensuring reviewer anonymity, and handling appeals or disputes. Likewise, major professional bodies (e.g., IEEE, ACM) are increasingly specifying rules to safeguard data privacy and outline responsibilities for both authors and reviewers in open and semi-open reviewing environments.

% Beyond institutional guidelines, researchers have called for broader oversight and governance to address the unique challenges of open review in AI/ML \cite{ratanaworabhan2022bridging}. For example, complex or controversial findings may attract public scrutiny on review platforms, raising concerns over potential harassment or misuse of research insights. Regulatory measures that standardize reviewer disclosure policies, define acceptable forms of public critique, and penalize unethical behaviors are seen as critical for protecting contributors while preserving the benefits of transparency. These considerations underscore the need for a well-defined regulatory framework that complements the goals of open peer review.

\section{Data} 
\label{sec:data}

We now describe the data used for training and deploying our AI-based approach. 
Gathering this data required a significant digitization process, extracting and processing 5.2 million pages of deeds stretching back to the 1850s (\S~\ref{sec:digit}). We then supplemented this data with historical deed records available online from around the country (\S~\ref{sec:otherdata}), which has the coincidental benefit of enabling us to assess the model's robustness across jurisdictions.  Finally, we manually annotate 3,801 deeds to build a training dataset and held-out evaluation dataset for our AI pipeline (\S~\ref{sec:annotation}).


\subsection{Digitization, Collection, and Sharing of Real Property Deeds}
\label{sec:digit}
The Santa Clara County Clerk-Recorder's Office has an extensive archive of over 24 million real property deeds. Of these records, approximately 18 million -- issued since 1980 -- are stored digitally, while the remaining 6 million deeds -- created before 1980 -- were originally preserved on physical microfiche sheets. More than a decade prior to our work, the County had engaged a vendor to scan these records into a proprietary system known as Digital Reel; however, as we discuss in Appendix \ref{appendix_ocr}, the quality of these scans was poor and required significant post-processing.

Our partnership around exploring the use of AI began in October 2022. One of the notable barriers to transparency around deed records in California lies in a statutory mandate to charge fees for any copies of recorded documents.\footnote{Cal.\ Gov.\ Code \S~27366 provides: ``The fee for any copy of any other record or paper on file in the office of the recorder, when the copy is made by the recorder, shall be set by the board of supervisors in an amount necessary to recover the direct and indirect costs of providing the product or service or the cost of enforcing any regulation for which the fee or charge is levied.'' This provision has been subject to extensive litigation. See, e.g., California Public Records Research, Inc.\ v.\ County of Stanislaus, 246 Cal.\ App.\ 4th 1432 (2016);  California Public Records Research v.\ County of Yolo, 4 Cal.\ App.\ 5th 150 (2016).}  In other words, despite their status as public records, deed documents are available only on an individual fee basis. Given the massive scale of the review task, purchasing deed records would, of course, have been prohibitively expensive.\footnote{At a cost of \$4 for the first page and \$2 for each subsequent page, purchasing the 5.2M pages (with the average deed running 2.5 pages) might have cost over \$13 million.} Through our partnership, we developed unique a data sharing agreement, enabling the Stanford team  to process deed data, with the County retaining ownership of the records.

We began our work on samples of 20,000 pages of property deeds filed between 1900 and 1940, manually exported from the County's Digital Reel system. This 20,000-page sample enabled us to rapidly develop and refine our automated detection pipeline. 

After this piloting phase, the County extracted the full collection of pre-1980 scans in February 2024. This represents roughly 5.2 million pages of real property documents from 1865 to 1980. We focus our analysis on documents from 1902 to 1980 for two reasons. First, deeds filed prior to 1902 were handwritten rather than typed, and we found no available OCR tools to be effective at transcribing these documents.\footnote{We did explore developing a bespoke computer vision or multimodal text-vision system.} Second, records after 1980 contain protected fields like Social Security information, so we avoided ingesting sensitive data and potentially training our model on it, which may have raised privacy and legal concerns. As we note above and consistent with our results in Section~\ref{sec:evolution}, 1902 to 1980 likely covers the vast majority of racial covenants in the County; the first racial covenant we find was filed in 1907 and the last in 1974.

\subsection{Data Augmentation}
\label{sec:otherdata}
Both within California and across the nation, historical property deeds vary significantly in format, phrasing, and, when digitized, OCR quality. In order to build a system that is robust to these variations, we supplemented the Santa Clara County dataset with property deeds from around the nation, both with and without racial covenants.

Since property records in California counties are not freely accessible, we expanded our search to other counties in the United States. Using \href{https://govos.com/products/public-record-access/channel/}{GovOS Cloud Search}, we identified seven counties whose ``Official Records Search’’ platforms allowed users to freely search and download real property deeds, although downloads were limited to fifty records per batch.\footnote{\url{https://kofilehelp.zendesk.com/hc/en-us/sections/4416665864343-Cloud-Search-Active-Sites}.} These platforms enabled searches by metadata and keyword terms. To gather a seed dataset of deeds with a high probability of containing racial covenants, we conducted manual searches for terms typically associated with such covenants, such as ``No person of,'' ``Caucasian,'' ``Negro,'' and other relevant racial terms. This method provided us with more than 10,000 property deeds from seven counties: Bexar County, Texas; Cuyahoga County, Ohio; Denton County, Texas; Franklin County, Ohio; Hidalgo County, Texas; and Lawrence County, Pennsylvania. This approach not only helped us collect relevant data but also allowed us to assess the generalizability of our model across different jurisdictions. As we discuss below, we specifically investigate the limitations of keyword-based approaches, and find that context-aware language models boost performance substantially. 




\subsection{Annotation} \label{sec:annotation}

We labeled our data collection by identifying quotes that contain racial covenants on each page. This annotation occurred over three stages: initial training data generation, model prediction review, and rich annotation.

In our initial round of annotation, we selected a sample of 3,000 pages in our collection based on keywords that almost certainly indicate the presence of a racial covenant in the deed text. These include terms like ``Negro,'' ``Mongolian,'' and ``Asiatic.'' We partnered with data annotation company CloudFactory to help us identify and label racial covenants in these pages.\footnote{During our collaboration with CloudFactory for data annotation, we carefully prepared comprehensive documentation to guide the annotators through the task. Given the potentially sensitive nature of the material—historical property deeds containing racially restrictive covenants, as well as accounts that could be considered offensive or harmful to some readers—we issued a clear advisory to approach the content with care. We emphasized that the annotators could stop the task at any point if they felt uncomfortable. In addition, we consulted with CloudFactory’s management to ensure that appropriate counseling and support resources would be available to their team, should any annotator feel the need for assistance or support. Our priority was to handle this material with the utmost sensitivity, while ensuring the well-being of those involved in the annotation process.}

After training models and generating predictions, we reviewed their performance. For all positive predictions, we labeled whether they were true positives or false positives. These ensured that we verified the small number of positive examples as well as hard negative examples. We additionally sampled and verified negative predictions to ensure some balance in the data. These new annotations were incorporated into the training set of future models.

Recognizing the need to easily validate model predictions and locate racial covenants on a page, we built a web application to assist with rich annotation. This made it easy to precisely select a bounding box on the image of the deed book page and compute a text span for the annotation process, while simultaneously allowing us to visualize predictions for verification.

All combined, including both Santa Clara County documents and documents from across the country, we collected 3,801 annotations of deed pages, of which 2,987 (78.6\%) contained a racially restrictive covenant. Notably, this annotation requires human review, but at a much smaller scale than reviewing all records.  






 





\section{Analysis of the Sample Complexity}\label{sec:anal}

In this section, we present our main results on the sample complexity of the algorithms. We first establish a novel confidence interval that is applicable to the unbiased samples collected by our exploration algorithms. We then provide theorems detailing the performance of these algorithms. 


\subsection{Confidence Intervals}
We introduce a novel confidence interval that is tighter than existing ones in our RL setting and can also be applied to other RL problems such as offline RL and infinite-horizon settings.
%
\begin{theorem}[Confidence Bounds]\label{the:conf}

Consider compact sets $\Sc\subset\Rr^{d_s}, \Ac\subset\Rr^{d_a}$, and define $\Zc=\Sc\times \Ac$, $d=d_a+d_s$. Consider two Mercer kernels $k_{\varphi}:\Zc\times\Zc\rightarrow \Rr$ and  $k_{\psi}:\Sc\times\Sc\rightarrow \Rr$. Assume that functions $f:\Zc\rightarrow\Rr$ and $V:\Sc\rightarrow\Rr$, and for each $z\in\Zc$, a conditional probability distribution $P(\cdot|z)$ over $\Sc$, are given such that $f(z)=\E_{s\sim P(\cdot|z)}[V(s)]$, $\|f\|_{\Hc_{k_\varphi}}\le B_1$, $\|V\|_{\Hc_{k_\psi}}\le B_2$, and $\max_{s\in\Sc}V(s)\le v_{\max}$, for some $B_1,B_2, v_{\max}>0$.
Assume a dataset of $\{z_{i}, s'_i\}_{i=1}^n$ is provided, where each $z_i$ is independent of the set $\{s'_j\}_{j=1}^n$, and $s'_i\sim P(\cdot|z_i)$.
Let $\hat{f}^n$ and $\sigma^n$ be the kernel ridge predictor and uncertainty estimator of $f$ using the observations:
\begin{align}\nn
    \hat{f}_n(z) &= k^{\top}_{\varphi_n}(z)(\tau^2I+K_{\varphi_n})^{-1}\bm{y}_{n},\\
    \sigma_n^2(z) &= k_{\varphi}(z,z) - k^{\top}_{\varphi_n}(z)(\tau^2I+K_{\varphi_n})^{-1}k_{\varphi_n}(z),
\end{align}
where $\bm{y}_n=[V(s'_1), V(s'_2), \cdots, V(s'_n))]^{\top}$.
In addition, let $\lambda_m$, $m=1,2,\cdots$ represent the Mercer eigenvalues of $k_{\psi}$ in a decreasing order, and $\psi_m$ the corresponding Mercer eigenfunctions. Assume $\psi_m\le \psi_{\max}$ for some $\psi_{\max}>0$. Fix $M\in \Nn$, and let $C$ be a constant such that $C \geq \sum_{m=1}^{M} \lambda_m$. %that serves as an upper bound for the sum of the first $M$ eigenvalues.


Then, for a fixed $z\in\Zc$, and for all $V$, with $\|V\|_{\Hc_{k_\psi}}\le B_2$, we have, the following each hold, with probability at least $1-\delta$,
\begin{equation*}
|f(z) - \hat{f}_n(z)| \le \beta(\delta) \sigma_n(z)
\end{equation*}
%\begin{equation*}
%    f(z) - \hat{f}_n(z) \le \beta(\delta) \sigma_n(z)~~ \text{and} ~~ f(z) - \hat{f}_n(z) \ge -\beta(\delta) \sigma_n(z).
%\end{equation*}
% \begin{equation*}
% \left\{
% \begin{aligned}
% & f(z) - \hat{f}_n(z) \le \beta(\delta) \sigma_n(z),  \\
% & f(z) - \hat{f}_n(z) \ge -\beta(\delta) \sigma_n(z),
% \end{aligned}
% \right.
% \end{equation*}
with $\beta(\delta) =$
\small{\begin{align*}
     B_1+ \frac{C B_2 \psi_{\max} }{\tau}\sqrt{2\log(\frac{M}{\delta})} 
    + \frac{2B_2\psi_{\max}}{\tau}\sqrt{n\sum_{m=M+1}^{\infty}\lambda_m}~.
\end{align*}}

%\aya{, where $C$ is a constant} %that serves as an upper bound for the sum of the first $M$ eigenvalues.}
\end{theorem}



Theorem~\ref{the:conf} provides a confidence bound for kernel ridge regression that is applicable to our RL setting, and is a key result in deriving our sample complexities. 

\paragraph{Proof sketch}
To derive our confidence bounds, we use the Mercer representation of \( V \) and decompose the prediction error \( f(z) - \hat{f}_n(z) \) into error terms corresponding to each Mercer eigenfunction \( \psi_m \). We then divide these terms into two groups: the first \( M \) elements, corresponding to eigenfunctions with the largest eigenvalues, and the remainder. For the top \( M \) eigenfunctions, we establish high-probability bounds using standard kernel-based confidence intervals from \cite{vakili2021optimal}. The remaining terms are bounded based on eigendecay, and we sum over all \( m \) to obtain \( \beta(\delta) \).
%To derive our confidence bounds, we use a novel approach by leveraging the Mercer representation of \( V \) and decomposing the prediction error \( f(z) - \hat{f}_n(z) \) into error terms corresponding to each Mercer eigenfunction \( \psi_m \). We then divide these terms into two groups: the first \( M \) elements, corresponding to eigenfunctions with the largest eigenvalues, and the remainder. For the top \( M \) eigenfunctions, we establish high-probability bounds using standard kernel-based confidence intervals from \citep{vakili2021optimal}. The remaining terms are bounded based on eigendecay, and we sum over all \( m \) to obtain \( \beta(\delta) \).}
\begin{remark}
    Under some mild conditions, for example, the polynomial eigendecay given in Definition~\ref{def:eigendecay}, the following expression can be derived for $\beta$:
    \begin{equation}
        \beta(\delta)= \Oc\left(B_1+\frac{ B_2 \psi_{\max} }{\tau}\sqrt{\log(\frac{n}{\delta})}\right).
    \end{equation}
\end{remark}

With polynomial eigendecay, the remark follows from setting $M$ to $\lceil n^{\frac{1}{p-1}}\rceil 
$ in the expression of $\beta$ in Theorem~\ref{the:conf}.

The confidence interval presented in Theorem~\ref{the:conf} is applicable to a fixed $z\in\Zc$. Over a discrete domain this can be easily extended to all $z\in\Zc$ using a probability union bound and replacing $\delta$ with~$\frac{\delta}{|\Zc|}$ in the expression of $\beta(\delta)$. Using standard discretization techniques, we can also prove a variation of the confidence interval that holds true uniformly over continuous domains. In particular, under the following assumption, we present a variation of the theorem over continuous domains. 


\begin{assumption}\label{ass:disc}
For each $n\in\Nn$, there exists a discretization $\Zz$ of $\Zc$ such that, for any $f\in \Hc_k$ with $\|f\|_{\Hc_k}\le B_1$, we have $f(z) - f([z])\le \frac{1}{n}$, where $[z] = {\arg}{\min}_{ z'\in \Zz}||z'-z||_{l^2}$ is the closest point in $\Zz$ to $z$, and $|\Zz|\le cB_1^dn^{d}$, where $c$ is a constant independent of $n$ and $B_1$.
\end{assumption}
Assumption~\ref{ass:disc} is a mild technical assumption that holds for typical kernels~\citep{srinivas2009gaussian, chowdhury2017kernelized, vakili2021optimal}.

\begin{corollary}\label{Cor:cont}
Under the setting of Theorem~\ref{the:conf}, and under Assumption~\ref{ass:disc}, the following inequalities each hold uniformly in $z\in\Zc$ and $V: \|V\|_{\Hc_{k_\psi}}\le B_2$, with probability at least $1-\delta$
\begin{align*}
    f(z)  \le \hat{f}_n(z) + \frac{2}{n} +  \tilde{\beta}(\delta) (\sigma_n(z)+\frac{2}{\sqrt{n}}), \\
    f(z)  \ge\hat{f}_n(z) -\frac{2}{n} -\tilde{\beta}(\delta) (\sigma_n(z)+\frac{2}{\sqrt{n}}),
\end{align*}
% \begin{equation*}
% \left\{
% \begin{aligned}
% & f(z)  \le \hat{f}_n(z) + \frac{2}{n} +  \tilde{\beta}(\delta) (\sigma_n(z)+\frac{2}{\sqrt{n}}),  \\
% & f(z)  \ge\hat{f}_n(z) -\frac{2}{n} -\tilde{\beta}(\delta) (\sigma_n(z)+\frac{2}{\sqrt{n}}),
% \end{aligned}
% \right.
% \end{equation*}
with $\tilde{\beta}(\delta)=\beta(\frac{\delta}{2c_n})$, $c_n=c(u_n(\frac{\delta}{2}))^dn^d$, and $u_n(\delta) = \Oc(\sqrt{n+\log(\frac{1}{\delta}}))$.
%is a $1-\delta$ upper confidence bound on $\|\hat{f}_n\|_{\Hc_k}$.

\end{corollary}


\begin{remark}
    Under some mild conditions, for example, the polynomial eigendecay given in Definition~\ref{def:eigendecay}, the following expression can be derived for $\tilde{\beta}$:
    \begin{equation}
        \tilde{\beta}(\delta)= \Oc\left(B_1+\frac{C B_2 \psi_{\max}}{\tau}\sqrt{d\log(\frac{n}{\delta})}\right).
    \end{equation}
\end{remark}

 
\subsection{Sample Complexities}

We have the following theorem on the performance of Algorithm~\ref{alg:exp_gen}. 
The weakest assumption one can pose on the value functions is realizability, which posits that the optimal value functions \(V^{\star}_{h}\) for \(h \in [H]\) lie in the RKHS \(H_{k_\psi}\) for some kernel $k_{\psi}:\Sc\times\Sc\rightarrow \Rr$, or at least are well-approximated by \(H_{k_\psi}\). For stateless MDPs or multi-armed bandits where \(H = 1\), realizability alone suffices for provably efficient algorithms~\citep{srinivas2009gaussian, chowdhury2017kernelized, vakili2021optimal}. But it does not seem to be sufficient when \(H > 1\), and in these settings it is common to make stronger assumptions~\citep{jin2020provably, wang2019optimism, chowdhury2023value}.
Following these works, our main assumption is a closure property for all value functions in the following class:
\begin{align}
    \Vc = \left\{
    s\rightarrow \min\left\{
    H, \max_{a\in\Ac}\left\{
    r(s,a) + \varphi^{\top}(s,a)\bm{w} + 
    \right. \right. \right. \nonumber \\
    \left. \left. \left. \beta \sqrt{\varphi^{\top}(s,a)\Sigma^{-1}\varphi(s,a)}
    \right\}
    \right\}
    \right\},
\end{align}

where $0<\beta<\infty$, $\|\bm{w}\|\le\infty$ 

and $\Sigma$ is an $\infty\times \infty$ matrix with $\Sigma \succ \tau^2 I$. 

\begin{assumption}[Optimistic Closure]\label{closure_assumption}
For any $V\in\Vc$, for some positive constant $c_v$, we have $\|V\|_{H_{k_\psi}}\leqslant c_v$.
\end{assumption}

This is the same assumption as Assumption~1 in~\cite{chowdhury2023value} and can be relaxed to value functions $\epsilon$ away from this class as described in Section $4.3$ of~\cite{chowdhury2023value}.
We have the following theorem on the sample complexity of the exploration algorithm with a generative model.


\begin{theorem}\label{the:gen}
    Consider the reward-free RL framework described in Section~\ref{sec:pf}. Assume the existence of a generative model in the exploration phase that allows the algorithm to select state-action pairs of its choice at each step. Let $N_0$ be the smallest integer satisfying
    \begin{equation*}
2H\beta(\delta)\sqrt{\frac{2\Gamma(N_0)}{N_0\log(1+1/\tau^2)}} +\frac{4\beta(\delta)H}{\sqrt{N_0}} +\frac{4H}{N_0}\le \epsilon,
\end{equation*}
with $\beta(\delta) =\Oc(\frac{H}{\tau}\sqrt{d\log(\frac{NH}{\delta})})$ with a sufficiently large constant. 
Run Algorithm~\ref{alg:exp_gen} for $N\ge N_0$ episodes to obtain the dataset $\Dc_N$. Then, use the obtained samples to design a policy $\pi$ using Algorithm~\ref{alg:plan} with $\beta(\delta) =\Oc(\frac{H}{\tau}\sqrt{d\log(\frac{NH}{\delta})})$ with a sufficiently large constant. 
Then, under Assumptions~\ref{ass:rkhsnorm}, \ref{ass:disc} and~\ref{closure_assumption}, with probability at least $1-\delta$, $\pi$ is guaranteed to be an $\epsilon$-optimal policy. 
\end{theorem}

The following theorem presents the sample complexity for exploration without generative models. 


\begin{theorem}\label{the:main}
    Consider the reward free RL framework described in Section~\ref{sec:pf}. Let $N_0$ be the smallest integer satisfying
    \begin{align}\nn
    &3H^2\beta(\delta)\sqrt{\frac{2\Gamma(N_0)}{N_0\log(1+1/\tau^2)}} +\frac{8\beta(\delta)H^2}{\sqrt{N_0}} \\\label{eq:suboptgap}
    &+\frac{4H^2(\log(N_0)+1)}{N_0}+2H^2\sqrt{2N_0\log({\frac{3N_0}{\delta})}}\le \epsilon
\end{align}
with $\beta(\delta) =\Oc(\frac{H}{\tau}\sqrt{d\log(\frac{NH}{\delta})})$ with a sufficiently large constant.     Run Algorithm~\ref{alg:exp2} for $NH\ge N_0H$ episodes to obtain the dataset $\Dc_N$. Then, use the obtained samples to design a policy $\pi$ using Algorithm~\ref{alg:plan}. 
Then, under Assumptions~\ref{ass:rkhsnorm}, \ref{ass:disc} and~\ref{closure_assumption}, with probability at least $1-\delta$, $\pi$ is guaranteed to be an $\epsilon$-optimal policy. 
\end{theorem}
    

The proof of theorems are provided in Appendix~\ref{appx:gen} and~\ref{appx:main_sample}. 

The expression of suboptimality gap after $N$ samples, given in~\eqref{eq:suboptgap}, can be simplified as
\begin{equation*}
    \Oc\left(  H^3\sqrt{\frac{\Gamma(N)\log(NH/\delta)}{N}}\right).
\end{equation*} 
 
\begin{figure*}[h]
    \centering
    \begin{subfigure}{0.32\textwidth}
        \centering
        \includegraphics[width=\textwidth]{figures/new_figures/RBF_kernel_all_algos.png}
        \caption{SE Kernel}
        \label{fig:RBF_all_algos}
    \end{subfigure}
    %\hspace{0.3em} 
    \begin{subfigure}{0.32\textwidth}
        \centering
        \includegraphics[width=\textwidth]{figures/new_figures/Matern2.5_all_algos.png} % Replace with the path to your third figure
        \caption{Mat{\'e}rn Kernel with $\nu=2.5$}
        \label{fig:Matern2.5_all_algos}
    \end{subfigure}
    %\hspace{0.3em} 
    \begin{subfigure}{0.32\textwidth}
        \centering
        \includegraphics[width=\textwidth]{figures/new_figures/Matern1.5_all_algos.png} % Replace with the path to your second figure
        \caption{Mat{\'e}rn kernel with $\nu=1.5$}
        \label{fig:Matern1.5_all_algos}
    \end{subfigure}
    \caption{Average suboptimality gap against $N$. The error bars indicate standard deviation.}
    \label{fig:overallresults}
\end{figure*}
\begin{remark}
Replacing $\Gamma(N)=\Oct(N^{\frac{1}{p}})$ in the case of kernels with polynomial eigendecay, we obtain a sample complexity of 
$
    N = \Oct((\frac{H^3}{\epsilon})^{2+\frac{2}{p-1}}).
$
We also recall that without a generative model, we interact with $H$ times more episodes to collect these samples. Specifically, the number of episodes in the exploration phase is 
$NH = \Oct\left(H(\frac{H^3}{\epsilon})^{2+\frac{2}{p-1}}\right)$.

\end{remark}

When specialized for the case of Mat{\'e}rn kernels with $p=1+\frac{2d}{\nu}$, we obtain $NH=\Oct(H(\frac{H^3}{\epsilon})^{2+\frac{d}{\nu}})$ that matches the lower bound for the degenerate case of bandits with $H=1$ proven in~\citet{scarlett2017lower}. Our sample complexity is thus order optimal in terms of $\epsilon$ dependency. We also recall that the existing results lead to possibly vacuous (infinite) sample complexities for these kernels.

\section{Discussion}
\label{section:discussion}


\subsection{Practical Implications for Feedforward Prompting}

Of course, prompting an LLM continuously before the user submits their prompt is significantly most costly over submitting the prompt just once, once the user is ready.

% But user might not be ready, and the cognitive costs is pretty heavy.


\subsection{}


% Does this work well with Chain of Thought actually?
% Maybe this approach will actually incentivize self-prompt-chaining???
% What are the implications of this?


% A benefit of this is certainly more transparency in the LLM
% LLM is so flexible that adding this kind of structure is still okay for the LLM



% What's more costly, entering a prompt, then responding and saying, no i want this, or typing a prompt, and tuning the prompt/expected output to reduce message exchanges?

% Learning to become a better prompter. One is by trial and error experience. Perhaps another is through this feedforward that tells you what you might be able to anticipate.
\section{Alternative Views}

While the preceding sections advocate for more transparent review processes, it is important to recognize that open or partially open systems are not without drawbacks. Critics highlight issues such as the potential for plagiarism, misappropriation of innovative ideas, and threats to proprietary research, raising valid questions about how best to balance openness with the need for confidentiality.

\paragraph{Plagiarism} 
One frequently cited concern is that open review may inadvertently facilitate plagiarism~\cite{piniewski2024emerging, oviedo2024review} if innovative concepts are publicly visible before a paper is formally published. When submissions are posted online (e.g., in open-review platforms or preprint servers like arXiv) and later rejected, these ideas remain accessible, allowing others to potentially adopt or iterate on them without proper attribution. However, such issues are not exclusive to open review. In fact, the growing trend of researchers posting preprints on arXiv—regardless of whether a conference uses open or closed peer review—reveals that this challenge is part of a broader question of how to protect intellectual property in public forums.

Moreover, confidentiality can serve as a safeguard against idea theft, as it keeps manuscripts and reviews private until final decisions are made. This is seen as particularly important for early-career researchers and smaller institutions, which may lack the resources to compete if their concepts are exposed prematurely. \textbf{Yet, given the rapid pace of AI research and the prevalence of preprint culture, solutions to plagiarism concerns must extend beyond the open-versus-closed review debate.} The research community at large may need clearer norms, stronger protective measures, and more effective reporting systems to uphold ethical standards for all parties involved.


% A primary concern with open review is the risk of plagiarism, particularly when innovative ideas are discussed during the review process but the paper is ultimately rejected. These ideas, now publicly visible, can be exploited by others without proper attribution. Similar concerns arise with platforms like arXiv, where preprints are publicly accessible.

% However, this is not an inherent flaw of open review but rather a broader issue in the dissemination of ideas in public forums. Addressing this would require community-wide efforts to establish norms and mechanisms for protecting intellectual property in open settings.

% Closed review systems provide a critical layer of protection for authors, particularly in fields like AI where the risk of idea theft is significant. By keeping submissions and reviews confidential, these systems help safeguard unpublished concepts from premature disclosure or exploitation by reviewers. This protection is particularly vital for researchers in academia or smaller institutions, who may lack the resources to compete with larger organizations if their innovative ideas are exposed prematurely.

% While the transparency of open reviews is often seen as a solution to systemic issues, critics argue that openness could inadvertently lead to misuse of information or unfair competitive advantages. For example, early-stage ideas disclosed during an open review could be quickly iterated upon by other groups, undermining the original authors' efforts. In this perspective, closed reviews, while not without flaws, provide essential confidentiality and protection necessary to ensure fairness and trust in high-stakes research environments.

\paragraph{Disclosure Policy}

For research scientists working at companies with patent-driven business models, such as those in the pharmaceutical, semiconductor, or AI industries, maintaining confidentiality in the peer review process is crucial. Many companies operate under strict intellectual property (IP) and patent disclosure policies to safeguard innovations before public release. Open review systems, which often require preprints or public sharing of submissions, could inadvertently expose proprietary research and jeopardize a company’s ability to secure patents.

For example, in jurisdictions like the United States~\cite{uspto2013patent}, the first-to-file patent system requires that an invention must not have been publicly disclosed prior to filing. A submission shared in an open review process might qualify as prior art, rendering the invention unpatentable. 

% Companies such as IBM, Google, or DeepMind, which rely on patenting AI and machine learning innovations, could find themselves in a precarious position if their researchers are required to participate in open review processes. Furthermore, public disclosure might allow competitors to quickly adopt or modify ideas, diluting the original innovator’s competitive edge.

In these settings, critics of open review argue that confidentiality helps ensure that breakthroughs remain protected until the necessary legal steps are in place. Without this protection, competitors could quickly adopt or modify ideas, diluting the original innovator’s advantage. While acknowledging the value of transparency, many researchers in industry and academia alike must balance the public benefit of sharing ideas with the practical need to safeguard proprietary innovations.
\section{Conclusion and future directions} \label{sec:conclusion}

In this paper we proposed a nested MLMC framework that offers important computational savings by performing most calculations in low precision and exploiting approximate random normal variables for the low precision path calculations. The low precision calculations could be performed in fixed precision on an FPGA for greater efficiency, and we suggested a procedure to optimise the bit-widths of every variable at each Monte Carlo level. This is an important improvement over previous mixed precision MLMC frameworks which held the lower precision fixed \cite{Rounding_error_oliver} or defined uniform bit-width at every level heuristically \cite{brugger2014mixed}. Our numerical results suggest that for the first levels our procedure reduces the cost at these levels by a factor 5 or 7. Hence the overall savings are significant since most paths are calculated on the first levels. Our approach would be even more efficient for the Milstein scheme because its higher order strong convergence leads to a greater proportion of the computational costs being on the coarsest levels.

The next stage of the research project will be to implement the RNG methods and the nested framework on FPGAs to determine the hardware requirements and confirm the extent of the computational savings. It would also be good to compare the performance benefits to using half-precision floating point arithmetic on GPUs or CPUs for the low-accuracy computations.





% In the unusual situation where you want a paper to appear in the
% references without citing it in the main text, use \nocite
\nocite{langley00}

\bibliography{example_paper}
\bibliographystyle{icml2024}


%%%%%%%%%%%%%%%%%%%%%%%%%%%%%%%%%%%%%%%%%%%%%%%%%%%%%%%%%%%%%%%%%%%%%%%%%%%%%%%
%%%%%%%%%%%%%%%%%%%%%%%%%%%%%%%%%%%%%%%%%%%%%%%%%%%%%%%%%%%%%%%%%%%%%%%%%%%%%%%
% APPENDIX
%%%%%%%%%%%%%%%%%%%%%%%%%%%%%%%%%%%%%%%%%%%%%%%%%%%%%%%%%%%%%%%%%%%%%%%%%%%%%%%
%%%%%%%%%%%%%%%%%%%%%%%%%%%%%%%%%%%%%%%%%%%%%%%%%%%%%%%%%%%%%%%%%%%%%%%%%%%%%%%
% \newpage
% \appendix
% \onecolumn
% \section{You \emph{can} have an appendix here.}

% You can have as much text here as you want. The main body must be at most $8$ pages long.
% For the final version, one more page can be added.
% If you want, you can use an appendix like this one.  

% The $\mathtt{\backslash onecolumn}$ command above can be kept in place if you prefer a one-column appendix, or can be removed if you prefer a two-column appendix.  Apart from this possible change, the style (font size, spacing, margins, page numbering, etc.) should be kept the same as the main body.
%%%%%%%%%%%%%%%%%%%%%%%%%%%%%%%%%%%%%%%%%%%%%%%%%%%%%%%%%%%%%%%%%%%%%%%%%%%%%%%
%%%%%%%%%%%%%%%%%%%%%%%%%%%%%%%%%%%%%%%%%%%%%%%%%%%%%%%%%%%%%%%%%%%%%%%%%%%%%%%


\end{document}


% This document was modified from the file originally made available by
% Pat Langley and Andrea Danyluk for ICML-2K. This version was created
% by Iain Murray in 2018, and modified by Alexandre Bouchard in
% 2019 and 2021 and by Csaba Szepesvari, Gang Niu and Sivan Sabato in 2022.
% Modified again in 2023 and 2024 by Sivan Sabato and Jonathan Scarlett.
% Previous contributors include Dan Roy, Lise Getoor and Tobias
% Scheffer, which was slightly modified from the 2010 version by
% Thorsten Joachims & Johannes Fuernkranz, slightly modified from the
% 2009 version by Kiri Wagstaff and Sam Roweis's 2008 version, which is
% slightly modified from Prasad Tadepalli's 2007 version which is a
% lightly changed version of the previous year's version by Andrew
% Moore, which was in turn edited from those of Kristian Kersting and
% Codrina Lauth. Alex Smola contributed to the algorithmic style files.
