%%%%%%%% ICML 2024 EXAMPLE LATEX SUBMISSION FILE %%%%%%%%%%%%%%%%%

\documentclass{article}

% Recommended, but optional, packages for figures and better typesetting:
\usepackage{microtype}
\usepackage{graphicx}
% \usepackage{subfigure}
\usepackage{booktabs} % for professional tables

% hyperref makes hyperlinks in the resulting PDF.
% If your build breaks (sometimes temporarily if a hyperlink spans a page)
% please comment out the following usepackage line and replace
% \usepackage{icml2024} with \usepackage[nohyperref]{icml2024} above.
\usepackage{hyperref}


% Attempt to make hyperref and algorithmic work together better:
\newcommand{\theHalgorithm}{\arabic{algorithm}}

% Use the following line for the initial blind version submitted for review:
% \usepackage{icml2024}

% If accepted, instead use the following line for the camera-ready submission:
\usepackage[accepted]{icml2024}

% For theorems and such
\usepackage{amsmath}
\usepackage{amssymb}
\usepackage{mathtools}
\usepackage{amsthm}
\usepackage{subcaption}
% \usepackage{subfigure}

% if you use cleveref..
\usepackage[capitalize,noabbrev]{cleveref}

%%%%%%%%%%%%%%%%%%%%%%%%%%%%%%%%
% THEOREMS
%%%%%%%%%%%%%%%%%%%%%%%%%%%%%%%%
\theoremstyle{plain}
\newtheorem{theorem}{Theorem}[section]
\newtheorem{proposition}[theorem]{Proposition}
\newtheorem{lemma}[theorem]{Lemma}
\newtheorem{corollary}[theorem]{Corollary}
\theoremstyle{definition}
\newtheorem{definition}[theorem]{Definition}
\newtheorem{assumption}[theorem]{Assumption}
\theoremstyle{remark}
\newtheorem{remark}[theorem]{Remark}

% Todonotes is useful during development; simply uncomment the next line
%    and comment out the line below the next line to turn off comments
%\usepackage[disable,textsize=tiny]{todonotes}
\usepackage[textsize=tiny]{todonotes}


% The \icmltitle you define below is probably too long as a header.
% Therefore, a short form for the running title is supplied here:
\icmltitlerunning{The AI / ML Community Should Adopt a More Transparent and Regulated Peer Review Process}

\begin{document}

\twocolumn[
% \icmltitle{Position: Advocating for a More Transparent and Regulated Peer Review in the AI / ML Community}
% \icmltitle{Position: The Artificial Intelligence and Machine Learning Community Should Adopt a More Transparent and Regulated Peer Review Process}
\icmltitle{Paper Copilot: The Artificial Intelligence and Machine Learning Community Should Adopt a More Transparent and Regulated Peer Review Process}

% It is OKAY to include author information, even for blind
% submissions: the style file will automatically remove it for you
% unless you've provided the [accepted] option to the icml2024
% package.

% List of affiliations: The first argument should be a (short)
% identifier you will use later to specify author affiliations
% Academic affiliations should list Department, University, City, Region, Country
% Industry affiliations should list Company, City, Region, Country

% You can specify symbols, otherwise they are numbered in order.
% Ideally, you should not use this facility. Affiliations will be numbered
% in order of appearance and this is the preferred way.
\icmlsetsymbol{equal}{*}

\begin{icmlauthorlist}
\icmlauthor{Jing Yang}{equal,USC,papercopilot}
% \icmlauthor{Firstname2 Lastname2}{equal,yyy,comp}
% \icmlauthor{Firstname3 Lastname3}{comp}
% \icmlauthor{Firstname4 Lastname4}{sch}
% \icmlauthor{Firstname5 Lastname5}{yyy}
% \icmlauthor{Firstname6 Lastname6}{sch,yyy,comp}
% \icmlauthor{Firstname7 Lastname7}{comp}
%\icmlauthor{}{sch}
% \icmlauthor{Firstname8 Lastname8}{sch}
% \icmlauthor{Firstname8 Lastname8}{yyy,comp}
%\icmlauthor{}{sch}
%\icmlauthor{}{sch}
\end{icmlauthorlist}

\icmlaffiliation{USC}{University of Southern California}
\icmlaffiliation{papercopilot}{Paper Copilot}
% \icmlaffiliation{sch}{School of ZZZ, Institute of WWW, Location, Country}

\icmlcorrespondingauthor{Jing Yang}{jingyang.carl.work@gmail.com}
% \icmlcorrespondingauthor{Firstname2 Lastname2}{first2.last2@www.uk}

% You may provide any keywords that you
% find helpful for describing your paper; these are used to populate
% the "keywords" metadata in the PDF but will not be shown in the document
\icmlkeywords{Machine Learning, ICML}

\vskip 0.3in
]

% this must go after the closing bracket ] following \twocolumn[ ...

% This command actually creates the footnote in the first column
% listing the affiliations and the copyright notice.
% The command takes one argument, which is text to display at the start of the footnote.
% The \icmlEqualContribution command is standard text for equal contribution.
% Remove it (just {}) if you do not need this facility.

% \printAffiliationsAndNotice{}  % leave blank if no need to mention equal contribution
% \printAffiliationsAndNotice{\icmlEqualContribution} % otherwise use the standard text.

\begin{abstract}
The rapid growth of submissions to top-tier Artificial Intelligence (AI) and Machine Learning (ML) conferences has prompted many venues to transition from closed to open review platforms. Some have fully embraced open peer reviews, allowing public visibility throughout the process, while others adopt hybrid approaches, such as releasing reviews only after final decisions or keeping reviews private despite using open peer review systems. In this work, we analyze the strengths and limitations of these models, highlighting the growing community interest in transparent peer review. To support this discussion, we examine insights from Paper Copilot, a website launched two years ago to aggregate and analyze AI / ML conference data while engaging a global audience. The site has attracted over 200,000 early-career researchers, particularly those aged 18–34 from 177 countries, many of whom are actively engaged in the peer review process. \textit{Drawing on our findings, this position paper advocates for a more transparent, open, and well-regulated peer review aiming to foster greater community involvement and propel advancements in the field.}

% \textcolor{red}{the importance of rebuttal, and majority researchers are 18-24 years old.}

\end{abstract}

% 1. The Title should state the position and start with “Position:”.
%   * These hypothetical paper titles do state a position:
%       * "Position: Quantum Atelic Learning Methods Should Employ Psychic Insights"
%       * "Position: Stop Research on Psychic Properties of Machine Learning"
%   * while these versions do not:
%       * "Position: Psychic Quantum Atelic Learning"
%       * "Position: A Perspective on Psychic Quantum Atelic Learning"
% 2. The Abstract must identify the paper as a position paper and briefly state the position (e.g., “This position paper argues that <statement of the position>.”)
% 3. The Introduction must state the position, using bold text.
% 4. The paper must include an “Alternative Views” section that describes and addresses one or more viable (not strawmen) positions that are opposed to the paper’s position.
% 5. Papers that describe new research without advocating a position are not responsive to this call and should instead be submitted to the main paper track.

% Position: AI/ML Influencers Have a Place in the Academic Process

\section{Introduction}
Backdoor attacks pose a concealed yet profound security risk to machine learning (ML) models, for which the adversaries can inject a stealth backdoor into the model during training, enabling them to illicitly control the model's output upon encountering predefined inputs. These attacks can even occur without the knowledge of developers or end-users, thereby undermining the trust in ML systems. As ML becomes more deeply embedded in critical sectors like finance, healthcare, and autonomous driving \citep{he2016deep, liu2020computing, tournier2019mrtrix3, adjabi2020past}, the potential damage from backdoor attacks grows, underscoring the emergency for developing robust defense mechanisms against backdoor attacks.

To address the threat of backdoor attacks, researchers have developed a variety of strategies \cite{liu2018fine,wu2021adversarial,wang2019neural,zeng2022adversarial,zhu2023neural,Zhu_2023_ICCV, wei2024shared,wei2024d3}, aimed at purifying backdoors within victim models. These methods are designed to integrate with current deployment workflows seamlessly and have demonstrated significant success in mitigating the effects of backdoor triggers \cite{wubackdoorbench, wu2023defenses, wu2024backdoorbench,dunnett2024countering}.  However, most state-of-the-art (SOTA) backdoor purification methods operate under the assumption that a small clean dataset, often referred to as \textbf{auxiliary dataset}, is available for purification. Such an assumption poses practical challenges, especially in scenarios where data is scarce. To tackle this challenge, efforts have been made to reduce the size of the required auxiliary dataset~\cite{chai2022oneshot,li2023reconstructive, Zhu_2023_ICCV} and even explore dataset-free purification techniques~\cite{zheng2022data,hong2023revisiting,lin2024fusing}. Although these approaches offer some improvements, recent evaluations \cite{dunnett2024countering, wu2024backdoorbench} continue to highlight the importance of sufficient auxiliary data for achieving robust defenses against backdoor attacks.

While significant progress has been made in reducing the size of auxiliary datasets, an equally critical yet underexplored question remains: \emph{how does the nature of the auxiliary dataset affect purification effectiveness?} In  real-world  applications, auxiliary datasets can vary widely, encompassing in-distribution data, synthetic data, or external data from different sources. Understanding how each type of auxiliary dataset influences the purification effectiveness is vital for selecting or constructing the most suitable auxiliary dataset and the corresponding technique. For instance, when multiple datasets are available, understanding how different datasets contribute to purification can guide defenders in selecting or crafting the most appropriate dataset. Conversely, when only limited auxiliary data is accessible, knowing which purification technique works best under those constraints is critical. Therefore, there is an urgent need for a thorough investigation into the impact of auxiliary datasets on purification effectiveness to guide defenders in  enhancing the security of ML systems. 

In this paper, we systematically investigate the critical role of auxiliary datasets in backdoor purification, aiming to bridge the gap between idealized and practical purification scenarios.  Specifically, we first construct a diverse set of auxiliary datasets to emulate real-world conditions, as summarized in Table~\ref{overall}. These datasets include in-distribution data, synthetic data, and external data from other sources. Through an evaluation of SOTA backdoor purification methods across these datasets, we uncover several critical insights: \textbf{1)} In-distribution datasets, particularly those carefully filtered from the original training data of the victim model, effectively preserve the model’s utility for its intended tasks but may fall short in eliminating backdoors. \textbf{2)} Incorporating OOD datasets can help the model forget backdoors but also bring the risk of forgetting critical learned knowledge, significantly degrading its overall performance. Building on these findings, we propose Guided Input Calibration (GIC), a novel technique that enhances backdoor purification by adaptively transforming auxiliary data to better align with the victim model’s learned representations. By leveraging the victim model itself to guide this transformation, GIC optimizes the purification process, striking a balance between preserving model utility and mitigating backdoor threats. Extensive experiments demonstrate that GIC significantly improves the effectiveness of backdoor purification across diverse auxiliary datasets, providing a practical and robust defense solution.

Our main contributions are threefold:
\textbf{1) Impact analysis of auxiliary datasets:} We take the \textbf{first step}  in systematically investigating how different types of auxiliary datasets influence backdoor purification effectiveness. Our findings provide novel insights and serve as a foundation for future research on optimizing dataset selection and construction for enhanced backdoor defense.
%
\textbf{2) Compilation and evaluation of diverse auxiliary datasets:}  We have compiled and rigorously evaluated a diverse set of auxiliary datasets using SOTA purification methods, making our datasets and code publicly available to facilitate and support future research on practical backdoor defense strategies.
%
\textbf{3) Introduction of GIC:} We introduce GIC, the \textbf{first} dedicated solution designed to align auxiliary datasets with the model’s learned representations, significantly enhancing backdoor mitigation across various dataset types. Our approach sets a new benchmark for practical and effective backdoor defense.



\section{Related Works}
% \textcolor{red}{GPT: replace citations}

\subsection{Open Peer Review}

Open peer review (OPR) enhances transparency by publishing reviews, revealing reviewer identities, or enabling public discussions~\cite{ross2017open, henriquez2023open, wolfram2020open}. In AI and ML, OpenReview~\cite{openreview_api} has facilitated OPR, with ICLR pioneering public discourse alongside formal reviews~\cite{wang2023have}. Proponents argue that open reviews improve feedback quality, help reviewers refine their assessments~\cite{church2024peer}, and enable confidence estimation from review text~\cite{bharti2022confident}. However, experiments at NeurIPS reveal inconsistencies in peer review~\cite{cortes2021inconsistency, Lawrence2022NeurIPSExperiment, beygelzimer2023has}, raising concerns about subjective scoring~\cite{xie2024reviewer} and the impact of increasing submissions~\cite{tran2021an}. Some studies suggest interventions to reduce uncertainty in reviewer judgments~\cite{chen2023judgment} or explore author self-assessments as a complement to peer review~\cite{su2024analysis}.

Despite its benefits, OPR within double-blind settings poses challenges. Publishing reviews, even anonymously, may reveal sensitive details or invite targeted criticism~\cite{tran2021an}. Computational studies highlight fairness disparities in peer review~\cite{zhang2022investigating}, and alternatives like managing research evaluation on GitHub have been proposed~\cite{takagi2022managing}. Broader concerns persist, including whether reviewing efforts align with academic impact~\cite{church2024peer} and how best to address systemic biases~\cite{shah2022challenges}. As NeurIPS discussions occur mid-year and ICLR discussions happen later, the timing of transparency measures may also shape reviewer behavior and decision-making.

% An Open Review of OpenReview: A Critical Analysis of the Machine Learning Conference Review Process \cite{tran2021an}

% Inconsistency in conference peer review: Revisiting the 2014 neurips experiment \cite{cortes2021inconsistency}

% The NeurIPS Experiment \cite{Lawrence2022NeurIPSExperiment}

% Challenges, experiments, and computational solutions in peer review \cite{shah2022challenges}

% Has the machine learning review process become more arbitrary as the field has grown? The NeurIPS 2021 consistency experiment \cite{beygelzimer2023has}

% Judgment sieve: Reducing uncertainty in group judgments through interventions targeting ambiguity versus disagreement\cite{chen2023judgment}

% Contrastive Explanations That Anticipate Human Misconceptions Can Improve Human Decision-Making Skills \cite{buccinca2024contrastive}

% Investigating fairness disparities in peer review: A language model enhanced approach \cite{zhang2022investigating}

% How confident was your reviewer? estimating reviewer confidence from peer review texts \cite{bharti2022confident}

% Are reviewer scores consistent with citations? \cite{xie2024reviewer}

% Analysis of the ICML 2023 Ranking Data: Can Authors' Opinions of Their Own Papers Assist Peer Review in Machine Learning? \cite{su2024analysis}

% Is Peer-Reviewing Worth the Effort?\cite{church2024peer}

% Managing the Whole Research Process on GitHub \cite{takagi2022managing}

% What have we learned from OpenReview? \cite{wang2023have}

% Open peer review (OPR) encompasses a spectrum of practices that aim to increase the transparency of the reviewing process, ranging from publishing reviewer identities and comments to enabling public commenting on manuscripts \cite{ross2017open, henriquez2023open}. The approach is intended to foster accountability, reduce biases, and promote more constructive critique. In machine learning and AI domains, conferences such as ICLR have pioneered variations of OPR on platforms like OpenReview, allowing public commentary alongside official reviews. Proponents argue that exposing the reasoning behind acceptance or rejection can provide valuable feedback to authors while also helping reviewers refine their assessments.

% Nonetheless, implementing OPR in double-blind settings requires careful balancing of transparency and anonymity. Even without revealing identities, publishing reviews can inadvertently disclose sensitive information or lead to targeted criticism \cite{ross2017open}. Critics also contend that reviewers, especially early-career researchers, might be reluctant to offer candid assessments if their comments are made public \cite{bianchi2023state}. Thus, while OPR has gained traction across some AI and ML venues, questions remain regarding its effect on review quality and reviewer willingness to participate.

\subsection{Regulations}

As OPR evolves, regulatory guidelines ensure integrity, fairness, and privacy~\cite{ross2019guidelines}. Some researchers caution that excessive transparency may undermine review quality~\cite{bianchi2022can}, while others highlight the challenge of balancing confidentiality with open science~\cite{baez2002confidentiality, dennis2019privacy}.

AI/ML conferences face additional regulatory challenges. Public review platforms can expose researchers to scrutiny or harassment, raising ethical concerns~\cite{wang2023have}. AI-powered peer review introduces risks that require human oversight~\cite{seghier2024ai}, while plagiarism in review reports and the rise of review mills threaten review integrity~\cite{piniewski2024emerging, oviedo2024review, ezhumalai2024design}. To address these risks, researchers advocate for clearer policies on reviewer disclosures, public critique, and misconduct prevention, ensuring transparency strengthens rather than undermines the review process~\cite{kaltenbrunner2022innovating, kuznetsov2024can}.

% What have we learned from OpenReview? \cite{wang2023have}

% Confidentiality and peer review: The paradox of secrecy in academe \cite{baez2002confidentiality}

% Can transparency undermine peer review? A simulation model of scientist behavior under open peer review \cite{bianchi2022can}

% Guidelines for open peer review implementation \cite{ross2019guidelines}

% Innovating peer review, reconfiguring scholarly communication: An analytical overview of ongoing peer review innovation activities \cite{kaltenbrunner2022innovating}

% Emerging plagiarism in peer-review evaluation reports: a tip of the iceberg? \cite{piniewski2024emerging}

% What Can Natural Language Processing Do for Peer Review? \cite{kuznetsov2024can}


% AI-powered peer review needs human supervision \cite{seghier2024ai}

% Design of an Integrated Collaborative Environment for Projects with Plagiarism Checker \cite{ezhumalai2024design}

% The review mills, not just (self-) plagiarism in review reports, but a step further \cite{oviedo2024review}

% Privacy versus open science \cite{dennis2019privacy}


% As open peer review practices evolve, various regulatory guidelines and ethical frameworks have emerged to maintain standards of integrity, fairness, and privacy \cite{cope2021guidelines, ieee2024guidelines}. The Committee on Publication Ethics (COPE), for instance, provides recommendations on managing conflicts of interest, ensuring reviewer anonymity, and handling appeals or disputes. Likewise, major professional bodies (e.g., IEEE, ACM) are increasingly specifying rules to safeguard data privacy and outline responsibilities for both authors and reviewers in open and semi-open reviewing environments.

% Beyond institutional guidelines, researchers have called for broader oversight and governance to address the unique challenges of open review in AI/ML \cite{ratanaworabhan2022bridging}. For example, complex or controversial findings may attract public scrutiny on review platforms, raising concerns over potential harassment or misuse of research insights. Regulatory measures that standardize reviewer disclosure policies, define acceptable forms of public critique, and penalize unethical behaviors are seen as critical for protecting contributors while preserving the benefits of transparency. These considerations underscore the need for a well-defined regulatory framework that complements the goals of open peer review.

\section{Data Sources}\label{sec:data}
In this section, we introduce our training data, including unlabeled light curves for pretraining and labeled samples for the downstream classification task. 

\subsection{Unlabeled data - MACHO}
The MACHO project \citep{1993Natur.365..621A} aimed to detect Massive Compact Halo Objects (MACHO) to find evidence of dark matter in the Milky Way halo by searching for gravitational microlensing events. Light curves were collected from 1992 to 1999, producing light curves of more than a thousand observations \citep{1999PASP..111.1539A} in bands B and R.
The observed sky was subdivided into 403 fields. Each field was constructed by observing a region of the sky or tile. The resulting data is available in a public repository\footnote{\url{https://macho.nci.org.au/macho_photometry}} which contains millions of light curves in bands B and R. 

We selected a subset of fields 1, 101, 102, 103, and 104 containing \num{1454792} light curves for training. Similarly, we select field 10 for testing, with a total of \num{74594} light curves. MACHO observed in both bands simultaneously, therefore having two magnitudes associated with each MJD. Since we are looking to improve on Astromer 2, we maintain the single band input.  The light curves from this dataset that exhibited Gaussian noise characteristics were removed based on the criteria: $|\text{Kurtosis}| > 10$, $|\text{Skewness}| > 1$, and $\text{Std} > 0.1$. Additionally, we excluded observations with negative uncertainties (indicative of faulty measurements) or uncertainties greater than one (to maintain photometric quality). Outliers were also removed by discarding the 1st and 99th percentiles for each light curve. This additional filtering does not affect the total number of samples but reduces the number of observations when the criteria were applied.

\begin{figure}
    \centering
    \includegraphics[scale=.88]{figures/data/magnitude_datasets.pdf}
    \caption{Magnitude distributions for the MACHO, Alcock, and ATLAS datasets. The plotted magnitudes reflect their original values as reported in the datasets; however, they are normalized during training, eliminating the differences in their mean positions.
    The Alcock catalog exhibits multimodality. In contrast, the ATLAS magnitudes show significant more variation, as they originate from a different survey.}
    \label{fig:macho-alcock-magn}
\end{figure}

\subsection{Labeled data}
To ensure a fair comparison with Astromer 1, we used the same sample selection from the MACHO \citep[hereafter referred to as Alcock; ][]{Alcock2001Variable} and the  Asteroid Terrestrial-impact Last Alert System \citep[hereafter referred to as ATLAS; ][]{heinze2018first} labeled catalogs. The former has a similar magnitude distribution, whereas the latter differs, as shown in Fig. \ref{fig:macho-alcock-magn}.

\subsubsection{Alcock}
For labeled data, we use the catalog of variable stars from \citet{Alcock2001Variable}, which contains labels for a subset of the MACHO light curves originating from 30 fields from the Large Magellanic Cloud. This labeled data will be used to train and evaluate the performance of the different embeddings on the classification task. 

The selected data comprises \num{20894} light curves, which are categorized into six classes: Cepheid variables pulsating in the fundamental (Cep\_0) and first overtone (Cep\_1), Eclipsing Binaries (EC), Long Period Variables (LPV), RR Lyrae ab and c (RRab and RRc, respectively). Table \ref{tab:alcock} summarizes the number of samples per class. We note that the catalog used is an updated version, as described in \cite{astromer}.

\begin{table}
\caption{Alcock catalog distribution.}              
\label{tab:alcock}  
\centering 
% \begin{tabular}{c c c} 
\begin{tabular}{l l r} 
\hline\hline         
Tag & Class Name & \# of sources \\ \hline
 Cep\_0 & Cepheid type I &\num{1182} \\
 Cep\_1 &Cepheid type II & \num{683} \\
 EC &Eclipsing binary & \num{6824} \\
 LPV &Long period variable &  \num{3046} \\
 RRab &RR Lyrae type ab  &  \num{7397} \\
 RRc &RR Lyrae type c &  \num{1762} \\
 Total & & \textbf{\num{20894}} \\
\hline                            
\end{tabular}
\end{table}

Figure \ref{fig:macho-alcock-magn} compares the magnitude distributions between the Alcock and MACHO datasets. The former exhibits a bimodal distribution, which aligns with the fact that it represents a subset of the light curves from MACHO fields, while the latter encompasses light curves from only five fields. 

Similarly, we compare the distribution of time differences between consecutive observations ($\Delta t$). Figure \ref{fig:macho-alcock-mjd} shows similar distributions, with comparable ranges and means of three and four days for MACHO and Alcock, respectively.
\begin{figure}
    % \centering
    \includegraphics[scale=0.7]{figures/data/mjd_datasets.pdf}
    \caption{Distributions of consecutive observation time differences ($\Delta t$) for the Alcock, MACHO, and ATLAS datasets. The boxplots illustrate the variability in observation cadences across the datasets. The Alcock and MACHO datasets show relatively consistent sampling with narrower distributions, while the ATLAS dataset exhibits a broader range of $\Delta t$, reflecting more diverse observation intervals. The y-axis is shown on a logarithmic scale to highlight differences across several orders of magnitude }
    \label{fig:macho-alcock-mjd}
\end{figure}

\subsection{ATLAS}
The Asteroid Terrestrial-impact Last Alert System \citep[ATLAS; ][]{Tonry2018} is a survey developed by the University of Hawaii and funded by NASA. Operating since 2015, ATLAS has a global network telescopes, primarily focused on detecting asteroids and comets that could potentially threaten Earth. Observing in $c$ (blue), $o$ (orange), and $t$ (red) filters.

The variable star dataset used in this work was presented by \citet{heinze2018first} and includes 4.7 million candidate variable objects, included in the labeled and unclassified objects, as well as a dubious class. According to their estimates, this class is predominantly composed of $90\%$ instrumental noise and only $10\%$ genuine variable stars.

We analyze \num{141376} light curves from the ATLAS dataset, as detailed in Table \ref{tab:ATLAS}. These observations, measured in the $o$ passband, have a median cadence of $\sim$15 minutes, which is significantly shorter than the typical cadence in the MACHO dataset. This substantial difference poses a challenge for the model, as it must adapt to such a distinct temporal distribution. 

\begin{table}[h!]
\caption{ATLAS catalog distribution.}              
\label{tab:ATLAS}
\centering 
\begin{tabular}{l l r} 
\hline\hline         
Tag & Class Name & \# of sources \\
\hline
CB & Close Binaries &  \num{80218} \\
DB & Detached Binary &  \num{28767} \\
Mira & Mira &  \num{7370} \\
Pulse &RR Lyrae, $\delta$-Scuti, Cepheids &  \num{25021} \\
Total & & \textbf{\num{141376}}\\
\hline                            
\end{tabular}
\end{table}

As done in \citet{astromer} and to standardize the labels with other datasets, we combine detached eclipsing binaries identified by full or half periods into the close binaries (CB) category and similarly merge detached binaries (DB). However, objects with labels derived from Fourier analysis are excluded, as these classifications do not directly align with astrophysical categories.

\subsection{MACHO vs ATLAS}\label{sec:machovsatlas}
Figures \ref{fig:macho-alcock-magn} and \ref{fig:macho-alcock-mjd} illustrate the distributional differences between the unlabeled MACHO dataset and the labeled subsets discussed earlier. While the magnitudes show a notable shift between MACHO and ATLAS, our training strategy normalizes the light curves to a zero mean. As a result, the relationships between observations take precedence over the raw magnitude values. Consequently, we do not expect a substantial performance drop when transitioning between datasets. However, for $\Delta t$, the smaller values of $\Delta t$ present a significant challenge, as the model must extrapolate and account for fast variations to capture short-time information effectively. We evidence this in our first results from Astromer 2, where the F1 score on the ATLAS dataset was lower compared to MACHO when having fewer labels for classification. 


\section{Algorithm \& Theoretical Analysis}
In this section, we propose our combinatorial bandit algorithm for interactive personalized visualization recommendation, called Hierarchical Semi UCB (Hier-SUCB).

\subsection{Hier-SUCB}

Inspired by SPUCB~\cite{peng2019practical}, we develop a combinatorial contextual semi-bandit with a learnable bias term and a hierarchical structure. The structure includes a hierarchical agent to optimize the exploration on biased combinatorial setting and a hierarchical interaction system to get detailed user feedback without hurting user experience. The algorithm maintains two sets of upper confidence bounds (UCB) including the UCB of configurations $U(c)$ and visualizations $U(c,x,y)$ with a given configuration $c$. More formally, let $U(c)$ and $U(v)=U(c,x,y)$ be defined as:
\begin{equation}
    U(c_t)=\theta_{C,t}^T \mathbf x_{c,t}+\rho_{c,t}
\label{eqn:ucfg}
\end{equation}
\begin{equation}
    U(v_t)=\theta_{C,t}^T \mathbf x_{c,t}+\theta_{A,t}^T (\mathbf x_{x,t}+\mathbf x_{y,t}) + \gamma_t + \rho_{c,t} +\rho_{a,t} +\rho_{\gamma,t}
\label{eqn:uvis}
\end{equation}
The confidence radius of attribute and configuration $\rho_a,\rho_c$ is defined as:
\begin{equation}
    \rho_{k,t}=\sqrt{\mathbf x_{k,t}^T(\mathbf I_d+\mathbf x_{k,t} \mathbf x_{k,t}^T) \mathbf x_{k,t}}, k\in\lbrace a,c \rbrace
    \label{eqn:rhoca}
\end{equation}
where $\mathbf{x}_{k,t}$ is the embedding vector of attribute or configuration in round $t$ and $\mathbf I_d$ refers to identity matrix with the same dimension $d$ as $\mathbf x_t$. According to UCB~\cite{auer2010ucb}, the confidence radius of bias $\rho_\gamma$ defined as
\begin{equation}
    \rho_{\gamma,t}=\sqrt{2ln(T) / t_\gamma}
    \label{eqn:rhobias}
\end{equation}
where $t_\gamma$ is the time that bias $\gamma$ has been played.

In each turn, the agent first computes the UCB of all configurations with Eq.~\ref{eqn:ucfg} and then selects the configuration with optimal UCB. 
Afterward, the agent evaluates the upper confidence bounds of all visualizations with Eq.~\ref{eqn:uvis} to select the optimal. Then, the agent will ask for user feedback on the recommended visualization: 
if it is positive, the agent will automatically take the configuration and attributes as positive; otherwise, it will further ask for user feedback on the configuration and attributes separately. 

Intuitively, adding a bias term in the estimation of visualization reward can improve the accuracy of recommendation, because in the worst case we can assume it is the visualization reward and explore in a large action space. By designing appropriate reward function for the bias term, the bias term can serve as a correction term for cases that user likes the configuration and attributes but not the visualization. With the hierarchical structure of our agent, we further narrow down the large action space of the bias term. The agent quickly converges in configuration bandit with less item pool, so that it can have more exploration of attribute and bias terms with larger item pools.

\begin{algorithm}
    \SetAlgoLined
        Initialize $\theta_{C,t},\theta_{A,t}, \gamma_t$ (Eq. ~\ref{eqn:theta_def})\;
	\For{$t=1,2,...T$}{
	
        \For{$a_{c,t}=1,2,...n$}{
        Compute UCB $U(c_t)$ (Eq. ~\ref{eqn:ucfg})\;
        }
        Select $c_t=\textbf{argmax}(U(c_t))$\;
        \For{$a_{x,t}=1,2,...m$}
            {
            \For{$a_{y,t}=1,2,...m$}
                {
                Compute UCB $U(c_t,x_t,y_t)$ (Eq. ~\ref{eqn:uvis})\;
                }      
            }
        Select $V_{t}=\textbf{argmax}(U(c_t,x_t,y_t))$\;
        \uIf{$r_{V,t}==1$}
        {$r_{C,t}\leftarrow 1,r_{A,t}\leftarrow 1$\;}
        \Else{ask for $r_{C,t},r_{A,t}$\;}
        Update $\theta_{C,t},\theta_{A,t}, \gamma_t$ (Eq. ~\ref{eqn:theta_update})\;
        Update $\rho_{c,t},\rho_{a,t}, \rho_{\gamma,t}$ (Eq. ~\ref{eqn:rhoca},~\ref{eqn:rhobias})\;
        }
\caption{Hier-SUCB}

\end{algorithm}

\subsection{Regret Analysis}
% When the user provides negative feedback for the visualization, we consider the regret of round $t$ in four cases:
The regret of Hier-SUCB comes from the exploration of preferred configuration, attribute pair and learning the bias term. The exploration of preferred configuration and attribute pair can be reduced to general combinatorial bandit problem. Learning bias term can be viewed as a general bandit problem with constraints. For more detailed analysis of regret bound, we consider the regret of round $t$ under four cases when the user provides negative feedback to the visualization:
\begin{enumerate}
    \item Like configuration $c_t$ and attribute pair $\lbrace x_t,y_t \rbrace$ 
    \item Like attribute pair $\lbrace x_t,y_t \rbrace$ but not configuration $c_t$
    \item Like configuration $c_t$ but not attribute pair $\lbrace x_t,y_t \rbrace$
    \item Dislike configuration $c_t$ and attribute pair $\lbrace x_t,y_t \rbrace$
\end{enumerate}

For case (1), we provide the regret bound by analyzing the bias term converges in certain rounds.
\begin{lemma}
    The reward gap between optimal and sub-optimal bias $\gamma$ is bounded with the overall round $T$ and the time $t_\gamma$ that $\gamma$ has been played for.
    \begin{equation}
        \Delta_\gamma \leq \sqrt{\frac{ln(T)}{t_\gamma}}
    \end{equation}
    \label{lem:case1}
\end{lemma}
\begin{proof}
    having a positive configuration and attributes while negative visualization implies:
% Having positive configuration and attributes while negative visualization implies:
\begin{equation}
    U(c,a)\geq U(c^\ast,a^\ast)
\end{equation}
where $c^\ast, a^\ast$ refers to configuration and attributes of preferred visualization. Notably, $c,a$ may also receive positive feedback from user, but their combination is not preferred. In such case, $\Delta_c=\Delta_a=0$, and we can bound the regret with bias:
\begin{equation}
    \Delta_{bias} \leq \rho_{c,t} +\rho_{a,t}+ \rho_{\gamma,t} - \rho_{c,t}^\ast -\rho_{a,t}^\ast -\rho_{t,\gamma}^\ast
\end{equation}
The round that $\gamma^\ast$ is updated given $c,a$ should be less than either $t_a^\ast$ or $t_c^\ast$, we define $t_{max}^\ast=max(t_a^\ast,t_c^\ast)\geq t_{min}^\ast=min(t_a^\ast,t_c^\ast)\geq t_\gamma^\ast$. Using the definition of UCB, we can bound the gap of bias by
\begin{align}
     \Delta_{bias} &\leq  3\sqrt{\frac{2ln(T)}{t_\gamma}}-3\sqrt{\frac{2ln(T)}{t^\ast_{max}}}\leq  3\sqrt{\frac{2ln(T)}{t_\gamma}} \\
     t_\gamma &\leq 18ln(T) \frac{1}{\Delta_{bias}^2}
\end{align}
\renewcommand\qedsymbol{}
\end{proof}

With ~\ref{lem:case1}, we can bound the regret bound of case (1) by:
\begin{equation}
    Reg_1=\mathbb E\lbrack t_\gamma \rbrack \Delta_{bias}\leq 18ln(T)/\Delta_{bias} = O(ln(T))
\end{equation}

Notably, cases (2) and (4) are bounded by the rapid convergence of the confidence radius of configurations, thus, we consider when the agent recommends configuration. We derive the following lemma with $s^c_t$ representing the time that the configuration arm of action $a_t$ in round $t$ has been played. 
\begin{lemma}
Following the proof in LinUCB ~\cite{chu2011contextual}, we can bound The gap between optimal and sub-optimal reward is bounded by the following equation with probability $1-\delta/T$:
 \begin{equation}
\label{eqn:semi}
    |r_{t}^*- r_{t,a_t}| \leq \alpha \sqrt{-2log(\delta/2)/s^c_t}
\end{equation}
\end{lemma}

By summing Equation ~\ref{eqn:semi} with the expectation of round $T$, we derive the regret for case (2) and (4) as
\begin{equation}
    Reg_{2,4}=O(\sqrt{Tln^3(m^2Tln(T))})
    \label{eqn:reg_comb}
\end{equation}

For case (3), we first evaluate how many rounds the agent needs to recommend a positive configuration.
% For the case (3), we first evaluate how many rounds the agent needs to recommend positive configuration.
\begin{lemma}
With overall round $T$, the expected round for attribute exploration is 
\begin{equation}
    T-\frac{k}{\Delta_c^2}ln(T) 
    \label{eqn:tbound}
\end{equation}
\end{lemma}
\begin{proof}
The rounds to reach a positive configuration depend on the expectation of rounds that recommends a negative configuration. 
Thus by following the definition of UCB, we have:
\begin{equation}
    \mathbb{E}\lbrack t \rbrack= k\frac{ln(T)}{\Delta_c^2}
\end{equation}    
\renewcommand\qedsymbol{}
\end{proof}

To calculate the regret bound of case (3), we apply the upper bound of round $t$ derived in Equation ~\ref{eqn:semi} and get:
\begin{equation}
    Reg_3=O(\sqrt{(T-ln(T))ln^3(m^2(T-ln(T))ln(T-ln(T))}))
\end{equation}


Therefore, we can get the overall regret by summing up the regret of each case:
\begin{theorem}
The regret of Hier-SUCB can be bounded as:
\begin{align}
    Reg&=Reg_1+Reg_3+Reg_{2,4}\\
    &=O(\sqrt{Tln^3(m^2T ln(T))})
\end{align}
\end{theorem}


Notably, we reduce the original semi-bandit by improving $O(nm^2)$ to $O(m^2)$ by adding a hierarchical structure and decompose the combinatorial problem to multi-arm bandits and contextual semi-bandits. Regular combinatorial contextual bandit will apply Eq.~\ref{eqn:reg_comb} to all the attributes and configurations, where the term $m^2$ would be $nm^2$ in this case. With a hierarchical structure, the regret of the configuration is bounded by a contextual bandit. 
The regret of attribute pairs can be bounded with combinatorial contextual bandits as long as the configuration is preferred.
For the case (4) where attributes and configuration are preferred but their combination is not, we model an independent bias as multi-armed bandits  whose regret bound is $O(ln(T))$.

We also model the relation between the configuration and attribute with an extra bias as an individual bandit.
This helps improve the final accuracy of the personalized visualization recommendation, which we demonstrate later in the experiments using real-world datasets. 

\section{Discussion and Conclusion}

% \begin{quote}
% \textit{"We believe it is unethical for social workers not to learn... about technology-mediated social work."} (\citeauthor{singer_ai_2023}, 2023)
% \end{quote}

In this study, we uncovered multiple ways in which GenAI can be used in social service practice. While some concerns did arise, practitioners by and large seemed optimistic about the possibilities of such tools, and that these issues could be overcome. We note that while most participants found the tool useful, it was far from perfect in its outputs. This is not surprising, since it was powered by a generic LLM rather than one fine-tuned for social service case management. However, despite these inadequacies, our participants still found many uses for most of the tool's outputs. Many flaws pointed out by our participants related to highly contextualised, local knowledge. To tune an AI system for this would require large amounts of case files as training data; given the privacy concerns associated with using client data, this seems unlikely to happen in the near future. What our study shows, however, is that GenAI systems need not aim to be perfect to be useful to social service practitioners, and can instead serve as a complement to the critical "human touch" in social service.

We draw both inspiration and comparisons with prior work on AI in other settings. Studies on creative writing tools showed how the "uncertainty" \cite{wan2024felt} and "randomness" \cite{clark2018creative} of AI outputs aid creativity. Given the promise that our tool shows in aiding brainstorming and discussion, future social service studies could consider AI tools explicitly geared towards creativity - for instance, providing side-by-side displays of how a given case would fit into different theoretical frameworks, prompting users to compare, contrast, and adopt the best of each framework; or allowing users to play around with combining different intervention modalities to generate eclectic (i.e. multi-modal) interventions.

At the same time, the concept of supervision creates a different interaction paradigm to other uses of AI in brainstorming. Past work (e.g. \cite{shaer2024ai}) has explored the use of GenAI for ideation during brainstorming sessions, wherein all users present discuss the ideas generated by the system. With supervision in social service practice, however, there is a marked information and role asymmetry: supervisors may not have had the time to fully read up on their supervisee's case beforehand, yet have to provide guidance and help to the latter. We suggest that GenAI can serve a dual purpose of bringing supervisors up to speed quickly by summarising their supervisee's case data, while simultaneously generating a list of discussion and talking points that can improve the quality of supervision. Generalising, this interaction paradigm has promise in many other areas: senior doctors reviewing medical procedures with newer ones \cite{snowdon2017does} could use GenAI to generate questions about critical parts of a procedure to ask the latter, confirming they have been correctly understood or executed; game studio directors could quickly summarise key developmental pipeline concerns to raise at meetings and ensure the team is on track; even in academia, advisors involved in rather too many projects to keep track of could quickly summarise each graduate student's projects and identify potential concerns to address at their next meeting.

In closing, we are optimistic about the potential for GenAI to significantly enhance social service practice and the quality of care to clients. Future studies could focus on 1) longitudinal investigations into the long-term impact of GenAI on practitioner skills, client outcomes, and organisational workflows, and 2) optimising workflows to best integrate GenAI into casework and supervision, understanding where best to harness the speed and creativity of such systems in harmony with the experience and skills of practitioners at all levels.

% GenAI here thus serves as a tool that supervisors can use before rather then using the session, taking just a few minutes of their time to generate a list of discussion points with their supervisees.

% Traditional brainstorming comes up with new things that users discuss. In supervision, supervisors can use AI to more efficiently generate talking points with their supervisees. These are generally not novel ideas, since an experienced worker would be able to come up with these on their own. However, the interesting and novel use of AI here is in its use as a preparation tool, efficiently generating talking and discussion points, saving supervisors' time in preparing for a session, while still serving as a brainstorming tool during the session itself.

% The idea of embracing imperfect AI echoes the findings of \citeauthor{bossen2023batman} (2023) in a clinical decision setting, which examined the successful implementation of an "error-prone but useful AI tool". This study frames human-AI collaboration as "Batman and Robin", where AI is a useful but ultimately less skilled sidekick that plays second fiddle to Batman. This is similar to \citeauthor{yang2019unremarkable}'s (2019) idea of "unremarkable AI", systems designed to be unobtrusive and only visible to the user when they add some value. As compared to \citeauthor{bossen2023batman}, however, we see fewer instances of our AI system producing errors, and more examples of it providing learning and collaborative opportunities and other new use cases. We build on the idea of "complementary performance" \cite{bansal2021does}, which discusses how the unique expertise of AI enhances human decision-making performance beyond what humans can achieve alone. Beyond decision-making, GenAI can now enable "complementary work patterns", where the nature of its outputs enables humans to carry out their work in entirely new ways. Our study suggests that rather being a sidekick - Robin - AI is growing into the role of a "second Batman" or "AI-Batman": an entity with distinct abilities and expertise from humans, and that contributes in its own unique way. There is certainly still a time and place for unremarkable AI, but exploring uses beyond that paradigm uncovers entirely new areas of system design.

% % \cite{gero2022sparks} found AI to be useful for science writers to translate ideas already in their head into words, and to provide new perspectives to spark further inspiration. \textit{But how is ours different from theirs?}

% \subsection{New Avenues of Human-AI Collaboration}
% \label{subsubsec:discussionhaicollaboration}

% Past HCI literature in other areas \cite{nah2023generative} has suggested that GenAI represents a "leap" \cite{singh2023hide} in human-AI collaboration, 
% % Even when an AI system sometimes produces irrelevant outputs, it can still provide users 
% % Such systems have been proposed as ways to 
% helping users discover new viewpoints \cite{singh2023hide}, scour existing literature to suggest new hypotheses 
% \cite{cascella2023evaluating} and answer questions \cite{biswas2023role}, stimulate their cognitive processes \cite{memmert2023towards}, and overcome "writer's block" \cite{singh2023hide, cooper2023examining} (particularly relevant to SSPs and the vast amount of writing required of them). Our study finds promise for AI to help SSPs in all of these areas. By nature of being more verbose and capable of generating large amounts of content, GenAI seems to create a new way in which AI can complement human work and expertise. Our system, as LLMs tend to do, produced a lot of "bullshit" (S6) \cite{frankfurt2005bullshit} - superficially true statements that were often only "tangentially related" and "devoid of meaning" \cite{halloran2023ai}. Yet, many participants cited the page-long analyses and detailed multi-step intervention plans generated by the AI system to be a good starting point for further discussion, both to better conceptualize a particular case and to facilitate general worker growth and development. Almost like throwing mud at a wall to see what sticks, GenAI can quickly produce a long list of ideas or information, before the worker glances through it and quickly identifies the more interesting points to discuss. Playing the proposed role as a "scaffold" for further work \cite{cooper2023examining}, GenAI, literally, generates new opportunities for novel and more effective processes and perspectives that previous systems (e.g., PRMs) could not. This represents an entirely new mode of human-AI collaboration.
% % This represents a new mode of collaboration not possible with the largely quantitative AI models (like PRMs) of the past.

% Our work therefore supports and extends prior research that have postulated the the potential of AI's shifting roles from decision-maker to human-supporter \cite{wang_human-human_2020}. \citeauthor{siemon2022elaborating} (2022) suggests the role of AI as a "creator" or "coordinator", rather than merely providing "process guidance" \cite{memmert2023towards} that does not contribute to brainstorming. Similarly, \citeauthor{memmert2023towards} (2023) propose GenAI as a step forward from providing meta-level process guidance (i.e. facilitating user tasks) to actively contributing content and aiding brainstorming. We suggest that beyond content-support, AI can even create new work processes that were not possible without GenAI. In this sense, AI has come full circle, becoming a "meta-facilitator".

% % --- WIP BELOW ---

% % Our work echoes and extends previous research on HAI collaboration in tasks requiring a human touch. \cite{gero2023social} found AI to be a safe space for creative writers to bounce ideas off of and document their inner thoughts. \cite{dhillon2024shaping} reference the idea of appropriate scaffolding in argumentative writing, where the user is providing with guidance appropriate for their competency level, and also warns of decreased satisfaction and ownership from AI use. 

% Separately, we draw parallels with the field of creative writing, where HAI collaboration has been extensively researched. Writers note the "irreducibly human" aspect of creativity in writing \cite{gero2023social}, similar to the "human touch" core to social service practice (D1); both groups therefore expressed few concerns about AI taking over core aspects of their jobs. Another interesting parallel was how writers often appreciated the "uncertainty" \cite{wan2024felt} and "randomness" \cite{clark2018creative} of AI systems, which served as a source of inspiration. This echoes the idea of "imperfect AI" "expanding [the] perspective[s]" (S4) of our participants when they simply skimmed through what the AI produced. \cite{wan2024felt} cited how the "duality of uncertainty in the creativity process advances the exploration of the imperfection of GenAI models". While social service work is not typically regarded as "creative", practitioners nonetheless go through processes of ideation and iteration while formulating a case. Our study showed hints of how AI can help with various forms of ideation, but, drawing inspiration from creative writing tools, future studies could consider designs more explicitly geared towards creativity - for instance, by attempting to fit a given case into a number of different theories or modalities, and displaying them together for the user to consider. While many of these assessments may be imperfect or even downnright nonsensical, they may contain valuable ideas and new angles on viewing the case that the practitioner can integrate into their own assessment.

% % \cite{foong2024designing}, describing the design of caregiver-facing values elicitation tools, cites the "twin scenarios" that caregivers face - private use, where they might use a tool to discover their patient's values, and collaborative use, where they discuss the resulting values with other parties close to the patient. This closely mirrors how SSPs in our study reference both individual and collaborative uses of our tool. Unlike in \cite{foong2024designing}, however, we do not see a resulting need to design a "staged approach" with distinct interface features for both stages.

% % --- END OF WIP ---

% Having mentioned algorithm aversion previously, we also make a quick point here on the other end of the spectrum - automation bias, or blind trust in an automated system \cite{brown2019toward}. LLMs risk being perceived as an "ultimate epistemic authority" \cite{cooper2023examining} due to their detailed, life-like outputs. While automation bias has been studied in many contexts, including in the social sector or adjacent areas, we suggest that the very nature of GenAI systems fundamentally inhibits automation bias. The tendency of GenAI to produce verbose, lengthy explanations prompts users to read and think through the machine's judgement before accepting it, bringing up opportunities to disagree with the machine's opinion. This guards against blind acceptance of the system's recommendations, particularly in the culture of a social work agency where constant dialogue - including discussing AI-produced work - is the norm.


% % : Perception of AI in Social Service Work ??

% \subsection{Redefining the Boundary}
% \label{subsubsec:discussiontheoretical}

% As \citeauthor{meilvang_working_2023} (2023) describes, the social service profession has sought to distance itself from comprising mostly "administrative work" \cite{abbott2016boundaries}, and workers have long tried to tried to reduce their considerable time \cite{socialraadgiverforening2010notat} spent on such tasks in favour of actual casework with clients \cite{toren1972social}. Our study, however, suggests a blurring of the line between "manual" administrative tasks and "mental" casework that draws on practitioner expertise. Many tasks our participants cited involve elements of both: for instance, documenting a case recording requires selecting only the relevant information to include, and planning an intervention can be an iterative process of drafting a plan and discussing it with colleagues and superiors. This all stems from the fact that GenAI can produce virtually any document required by the user, but this document almost always requires revision under a watchful human eye.

% \citeauthor{meilvang_working_2023} (2023) also describes a more recent shift in the perceived accepted boundary of AI interventions in social service work. From "defending [the] jurisdiction [of social service work] against artificial intelligence" in the early days of PRM and other statistical assessment tools, the community has started to embrace AI as a "decision-support ... element in the assessment process". Our study concurs and frames GenAI as a source of information that can be used to support and qualify the assessments of SSPs \cite{meilvang_working_2023}, but suggests that we can take a step further: AI can be viewed as a \textit{facilitator} rather than just a supporter. GenAI can facilitate a wide range of discussions that promote efficiency, encourage worker learning and growth, and ultimately enhance client outcomes. This entails a much larger scope of AI use, where practitioners use the information provided by AI in a range of new scenarios. 

% Taken together, these suggest a new focus for boundary work and, more broadly, HCI research. GAI can play a role not just in menial documentation or decision-support, but can be deeply ingrained into every facet of the social service workflow to open new opportunities for worker growth, workflow optimisation, and ultimately improved client outcomes. Future research can therefore investigate the deeper, organisational-level effects of these new uses of AI, and their resulting impact on the role of profession discretion in effective social service work.

% % MH: oh i feel this paragraph is quite new to me! Could we elaborate this more, and truncate the first two paragraphs a bit to adjust the word propotion?


% % Our study extensively documents this for the first time in social service practice, and in the process reveals new insights about how AI can play such a role.



% \subsection{Design Implications}

% % Add link from ACE diagram?

% % EJ: it would be interesting to discuss how LLMs could help "hands-on experience" in the discussion section

% Addressing the struggle of integrating AI amidst the tension between machine assessment and expert judgement, we reframe AI as an \textit{facilitator} rather than an algorithm or decision-support tool, alleviating many concerns about trust and explainablity. We now present a high-level framework (Figure \ref{fig:hai-collaboration}) on human-AI collaboration, presenting a new perspective on designing effective AI systems that can be applied to both the social service sector and beyond.

% \begin{figure}
%     \centering
%     \includegraphics[scale=0.15]{images/designframework.png}
%     \caption{Framework for Human-AI Collaboration}
%     \label{fig:hai-collaboration}
%     \Description{An image showing our framework for Human-AI Collaboration. It shows that as stakeholder level increases from junior to senior, the directness of use shifts from co-creation to provision.}
% \end{figure}
% % MH: so this paradigm is proposed by us? I wonder if this could a part of results as well..?

% % \subsubsection{From Creation to Provision}

% In Section \ref{sec:stage2findings}, we uncovered the different ways in which SSPs of varying seniorities use, evaluate, and suggest uses of AI. These are intrinsically tied to the perspectives and levels of expertise that each stakeholder possesses. We therefore position the role of AI along the scale of \textit{creation} to \textit{provision}. 

% With junior workers, we recommend \textbf{designing tools for co-creation}: systems that aid the least experienced workers in creating the required deliverables for their work. Rather than \textit{telling} workers what to do - a difficult task in any case given the complexity of social work solutions - AI systems should instead \textit{co-create} deliverables required of these workers. These encompass the multitude of use cases that junior workers found useful: creating reports, suggesting perspectives from which to formulate a case, and providing a starting template for possible intervention plans. Notably, since AI outputs are not perfect, we emphasise the "co" in "co-creation": AI should only be a part of the workflow that also includes active engagement on the part of the SSPs and proactive discussion with supervisors. 

% For more experienced SSPs, we recommend \textbf{designing tools for provision}. Again, this is not the mere provision of recommendations or courses of action with clients, but rather that of resources which complement the needs of workers with greater responsibilities. This notably includes supplying materials to aid with supervision, a novel use case that to our knowledge has not surfaced in previous literature. In addition, senior workers also benefit greatly from manual tasks such as routine report writing and data processing. Since these workers are more experienced and can better spot inaccuracies in AI output, we suggest that AI can "provide" a more finished product that requires less vetting and corrections, and which can be used more directly as part of required deliverables.

% % MH: can we seperate here? above is about the guidance to paradigm, below is the practical roadmap for implementation
% In terms of concrete design features, given the constant focus on discussing AI outputs between colleagues in our FGDs, we recommend that AI tools, particularly those for junior workers, \textbf{include collaborative features} that facilitate feedback and idea sharing between users. We also suggest that designers work closely with domain experts (i.e. social work practitioners and agencies) to identify areas where the given AI model tends to make more mistakes, and to build in features that \textbf{highlight potential mistakes or inadequacies} in the AI's output to facilitate further discussion and avoid workers adopting suboptimal suggestions. 

% We also point out a fundamental difference between GenAI systems and previous systems: that GenAI can now play an important role in aiding users \textit{regardless of its flaws}. The nature of GenAI means that it promotes discussion and opens up new workflows by nature of its verbose and potentially incomplete outputs. Rather than working towards more accurate or explainable outcomes, which may in any case have minimal improvement on worker outcomes \cite{li2024advanced}, designers can also focus on \textbf{understanding how GenAI outputs can augment existing user flows and create new ones}.

% % for more senior workers...

% % how to differentiate levels of workers?

% % \subsubsection{Provider}

% % The most basic and obvious role of modern AI that we identify leverages the main strength of LLMs. They have the ability to produce high-quality writing from short, point-form, or otherwise messy and disjoint case notes that user often have \textit{[cite participant here]}. 

% Finally, given the limited expertise of many workers at using AI, it is important that systems \textbf{explicitly guide users to the features they need}, rather than simply relying on the ability of GAI to understand complex user instructions. For example, in the case of flexibility in use cases (Section \ref{subsubsec:control}), systems should include user flows that help combine multiple intervention and assessment modalities in order to directly meet the needs of workers.

% \subsection{Limitations and Future Work}

% While we attempt to mimic a contextual inquiry and work environment in our study design, there is no substitute for real data from actual system deployment. The use of an AI system in day-to-day work could reveal a different set of insights. Future studies could in particular study how the longitudinal context of how user attitudes, behaviours, preferences, and work outputs change with extended use of AI. 

% While we tried to include practitioners from different agencies, roles, and seniorities, social service practice may differ culturally or procedurally in other agencies or countries. Future studies could investigate different kinds of social service agencies and in different cultures to see if AI is similarly useful there.

% As the study was conducted in a country with relatively high technology literacy, participants naturally had a higher baseline understanding and acceptance of AI and other computer systems. However, we emphasise that our findings are not contingent on this - rather, we suggest that our proposed lens of viewing AI in the social sector is a means for engaging in relevant stakeholders and ensuring the effective design and implementation of AI in the social sector, regardless of how participants feel about AI to begin with. 



% % \subsection{Notes}

% % 1) safety and risks and 2) privacy - what does the emphasis on this say about a) design recommendations and b) approach to designing/PD of such systems?


% % W9 was presented with "Strengths" and "SFBT" output options. They commented, "solution focus is always building on the person's strengths". W9 therefore requested being able to output strengths and SFBT at the same time. But this would suggest that the SFBT output does not currently emphasise strengths strongly enough. However, W9 did not specifically evaluate that, and only made this comment because they saw the "strengths" option available, and in their head, strengths are key to SFBT.
% % What does this say about system design and UI in relation to user mental models?
\section{Alternative Views}

While the preceding sections advocate for more transparent review processes, it is important to recognize that open or partially open systems are not without drawbacks. Critics highlight issues such as the potential for plagiarism, misappropriation of innovative ideas, and threats to proprietary research, raising valid questions about how best to balance openness with the need for confidentiality.

\paragraph{Plagiarism} 
One frequently cited concern is that open review may inadvertently facilitate plagiarism~\cite{piniewski2024emerging, oviedo2024review} if innovative concepts are publicly visible before a paper is formally published. When submissions are posted online (e.g., in open-review platforms or preprint servers like arXiv) and later rejected, these ideas remain accessible, allowing others to potentially adopt or iterate on them without proper attribution. However, such issues are not exclusive to open review. In fact, the growing trend of researchers posting preprints on arXiv—regardless of whether a conference uses open or closed peer review—reveals that this challenge is part of a broader question of how to protect intellectual property in public forums.

Moreover, confidentiality can serve as a safeguard against idea theft, as it keeps manuscripts and reviews private until final decisions are made. This is seen as particularly important for early-career researchers and smaller institutions, which may lack the resources to compete if their concepts are exposed prematurely. \textbf{Yet, given the rapid pace of AI research and the prevalence of preprint culture, solutions to plagiarism concerns must extend beyond the open-versus-closed review debate.} The research community at large may need clearer norms, stronger protective measures, and more effective reporting systems to uphold ethical standards for all parties involved.


% A primary concern with open review is the risk of plagiarism, particularly when innovative ideas are discussed during the review process but the paper is ultimately rejected. These ideas, now publicly visible, can be exploited by others without proper attribution. Similar concerns arise with platforms like arXiv, where preprints are publicly accessible.

% However, this is not an inherent flaw of open review but rather a broader issue in the dissemination of ideas in public forums. Addressing this would require community-wide efforts to establish norms and mechanisms for protecting intellectual property in open settings.

% Closed review systems provide a critical layer of protection for authors, particularly in fields like AI where the risk of idea theft is significant. By keeping submissions and reviews confidential, these systems help safeguard unpublished concepts from premature disclosure or exploitation by reviewers. This protection is particularly vital for researchers in academia or smaller institutions, who may lack the resources to compete with larger organizations if their innovative ideas are exposed prematurely.

% While the transparency of open reviews is often seen as a solution to systemic issues, critics argue that openness could inadvertently lead to misuse of information or unfair competitive advantages. For example, early-stage ideas disclosed during an open review could be quickly iterated upon by other groups, undermining the original authors' efforts. In this perspective, closed reviews, while not without flaws, provide essential confidentiality and protection necessary to ensure fairness and trust in high-stakes research environments.

\paragraph{Disclosure Policy}

For research scientists working at companies with patent-driven business models, such as those in the pharmaceutical, semiconductor, or AI industries, maintaining confidentiality in the peer review process is crucial. Many companies operate under strict intellectual property (IP) and patent disclosure policies to safeguard innovations before public release. Open review systems, which often require preprints or public sharing of submissions, could inadvertently expose proprietary research and jeopardize a company’s ability to secure patents.

For example, in jurisdictions like the United States~\cite{uspto2013patent}, the first-to-file patent system requires that an invention must not have been publicly disclosed prior to filing. A submission shared in an open review process might qualify as prior art, rendering the invention unpatentable. 

% Companies such as IBM, Google, or DeepMind, which rely on patenting AI and machine learning innovations, could find themselves in a precarious position if their researchers are required to participate in open review processes. Furthermore, public disclosure might allow competitors to quickly adopt or modify ideas, diluting the original innovator’s competitive edge.

In these settings, critics of open review argue that confidentiality helps ensure that breakthroughs remain protected until the necessary legal steps are in place. Without this protection, competitors could quickly adopt or modify ideas, diluting the original innovator’s advantage. While acknowledging the value of transparency, many researchers in industry and academia alike must balance the public benefit of sharing ideas with the practical need to safeguard proprietary innovations.
\section{Discussion}\label{sec:discussion}



\subsection{From Interactive Prompting to Interactive Multi-modal Prompting}
The rapid advancements of large pre-trained generative models including large language models and text-to-image generation models, have inspired many HCI researchers to develop interactive tools to support users in crafting appropriate prompts.
% Studies on this topic in last two years' HCI conferences are predominantly focused on helping users refine single-modality textual prompts.
Many previous studies are focused on helping users refine single-modality textual prompts.
However, for many real-world applications concerning data beyond text modality, such as multi-modal AI and embodied intelligence, information from other modalities is essential in constructing sophisticated multi-modal prompts that fully convey users' instruction.
This demand inspires some researchers to develop multimodal prompting interactions to facilitate generation tasks ranging from visual modality image generation~\cite{wang2024promptcharm, promptpaint} to textual modality story generation~\cite{chung2022tale}.
% Some previous studies contributed relevant findings on this topic. 
Specifically, for the image generation task, recent studies have contributed some relevant findings on multi-modal prompting.
For example, PromptCharm~\cite{wang2024promptcharm} discovers the importance of multimodal feedback in refining initial text-based prompting in diffusion models.
However, the multi-modal interactions in PromptCharm are mainly focused on the feedback empowered the inpainting function, instead of supporting initial multimodal sketch-prompt control. 

\begin{figure*}[t]
    \centering
    \includegraphics[width=0.9\textwidth]{src/img/novice_expert.pdf}
    \vspace{-2mm}
    \caption{The comparison between novice and expert participants in painting reveals that experts produce more accurate and fine-grained sketches, resulting in closer alignment with reference images in close-ended tasks. Conversely, in open-ended tasks, expert fine-grained strokes fail to generate precise results due to \tool's lack of control at the thin stroke level.}
    \Description{The comparison between novice and expert participants in painting reveals that experts produce more accurate and fine-grained sketches, resulting in closer alignment with reference images in close-ended tasks. Novice users create rougher sketches with less accuracy in shape. Conversely, in open-ended tasks, expert fine-grained strokes fail to generate precise results due to \tool's lack of control at the thin stroke level, while novice users' broader strokes yield results more aligned with their sketches.}
    \label{fig:novice_expert}
    % \vspace{-3mm}
\end{figure*}


% In particular, in the initial control input, users are unable to explicitly specify multi-modal generation intents.
In another example, PromptPaint~\cite{promptpaint} stresses the importance of paint-medium-like interactions and introduces Prompt stencil functions that allow users to perform fine-grained controls with localized image generation. 
However, insufficient spatial control (\eg, PromptPaint only allows for single-object prompt stencil at a time) and unstable models can still leave some users feeling the uncertainty of AI and a varying degree of ownership of the generated artwork~\cite{promptpaint}.
% As a result, the gap between intuitive multi-modal or paint-medium-like control and the current prompting interface still exists, which requires further research on multi-modal prompting interactions.
From this perspective, our work seeks to further enhance multi-object spatial-semantic prompting control by users' natural sketching.
However, there are still some challenges to be resolved, such as consistent multi-object generation in multiple rounds to increase stability and improved understanding of user sketches.   


% \new{
% From this perspective, our work is a step forward in this direction by allowing multi-object spatial-semantic prompting control by users' natural sketching, which considers the interplay between multiple sketch regions.
% % To further advance the multi-modal prompting experience, there are some aspects we identify to be important.
% % One of the important aspects is enhancing the consistency and stability of multiple rounds of generation to reduce the uncertainty and loss of control on users' part.
% % For this purpose, we need to develop techniques to incorporate consistent generation~\cite{tewel2024training} into multi-modal prompting framework.}
% % Another important aspect is improving generative models' understanding of the implicit user intents \new{implied by the paint-medium-like or sketch-based input (\eg, sketch of two people with their hands slightly overlapping indicates holding hand without needing explicit prompt).
% % This can facilitate more natural control and alleviate users' effort in tuning the textual prompt.
% % In addition, it can increase users' sense of ownership as the generated results can be more aligned with their sketching intents.
% }
% For example, when users draw sketches of two people with their hands slightly overlapping, current region-based models cannot automatically infer users' implicit intention that the two people are holding hands.
% Instead, they still require users to explicitly specify in the prompt such relationship.
% \tool addresses this through sketch-aware prompt recommendation to fill in the necessary semantic information, alleviating users' workload.
% However, some users want the generative AI in the future to be able to directly infer this natural implicit intentions from the sketches without additional prompting since prompt recommendation can still be unstable sometimes.


% \new{
% Besides visual generation, 
% }
% For example, one of the important aspect is referring~\cite{he2024multi}, linking specific text semantics with specific spatial object, which is partly what we do in our sketch-aware prompt recommendation.
% Analogously, in natural communication between humans, text or audio alone often cannot suffice in expressing the speakers' intentions, and speakers often need to refer to an existing spatial object or draw out an illustration of her ideas for better explanation.
% Philosophically, we HCI researchers are mostly concerned about the human-end experience in human-AI communications.
% However, studies on prompting is unique in that we should not just care about the human-end interaction, but also make sure that AI can really get what the human means and produce intention-aligned output.
% Such consideration can drastically impact the design of prompting interactions in human-AI collaboration applications.
% On this note, although studies on multi-modal interactions is a well-established topic in HCI community, it remains a challenging problem what kind of multi-modal information is really effective in helping humans convey their ideas to current and next generation large AI models.




\subsection{Novice Performance vs. Expert Performance}\label{sec:nVe}
In this section we discuss the performance difference between novice and expert regarding experience in painting and prompting.
First, regarding painting skills, some participants with experience (4/12) preferred to draw accurate and fine-grained shapes at the beginning. 
All novice users (5/12) draw rough and less accurate shapes, while some participants with basic painting skills (3/12) also favored sketching rough areas of objects, as exemplified in Figure~\ref{fig:novice_expert}.
The experienced participants using fine-grained strokes (4/12, none of whom were experienced in prompting) achieved higher IoU scores (0.557) in the close-ended task (0.535) when using \tool. 
This is because their sketches were closer in shape and location to the reference, making the single object decomposition result more accurate.
Also, experienced participants are better at arranging spatial location and size of objects than novice participants.
However, some experienced participants (3/12) have mentioned that the fine-grained stroke sometimes makes them frustrated.
As P1's comment for his result in open-ended task: "\emph{It seems it cannot understand thin strokes; even if the shape is accurate, it can only generate content roughly around the area, especially when there is overlapping.}" 
This suggests that while \tool\ provides rough control to produce reasonably fine results from less accurate sketches for novice users, it may disappoint experienced users seeking more precise control through finer strokes. 
As shown in the last column in Figure~\ref{fig:novice_expert}, the dragon hovering in the sky was wrongly turned into a standing large dragon by \tool.

Second, regarding prompting skills, 3 out of 12 participants had one or more years of experience in T2I prompting. These participants used more modifiers than others during both T2I and R2I tasks.
Their performance in the T2I (0.335) and R2I (0.469) tasks showed higher scores than the average T2I (0.314) and R2I (0.418), but there was no performance improvement with \tool\ between their results (0.508) and the overall average score (0.528). 
This indicates that \tool\ can assist novice users in prompting, enabling them to produce satisfactory images similar to those created by users with prompting expertise.



\subsection{Applicability of \tool}
The feedback from user study highlighted several potential applications for our system. 
Three participants (P2, P6, P8) mentioned its possible use in commercial advertising design, emphasizing the importance of controllability for such work. 
They noted that the system's flexibility allows designers to quickly experiment with different settings.
Some participants (N = 3) also mentioned its potential for digital asset creation, particularly for game asset design. 
P7, a game mod developer, found the system highly useful for mod development. 
He explained: "\emph{Mods often require a series of images with a consistent theme and specific spatial requirements. 
For example, in a sacrifice scene, how the objects are arranged is closely tied to the mod's background. It would be difficult for a developer without professional skills, but with this system, it is possible to quickly construct such images}."
A few participants expressed similar thoughts regarding its use in scene construction, such as in film production. 
An interesting suggestion came from participant P4, who proposed its application in crime scene description. 
She pointed out that witnesses are often not skilled artists, and typically describe crime scenes verbally while someone else illustrates their account. 
With this system, witnesses could more easily express what they saw themselves, potentially producing depictions closer to the real events. "\emph{Details like object locations and distances from buildings can be easily conveyed using the system}," she added.

% \subsection{Model Understanding of Users' Implicit Intents}
% In region-sketch-based control of generative models, a significant gap between interaction design and actual implementation is the model's failure in understanding users' naturally expressed intentions.
% For example, when users draw sketches of two people with their hands slightly overlapping, current region-based models cannot automatically infer users' implicit intention that the two people are holding hands.
% Instead, they still require users to explicitly specify in the prompt such relationship.
% \tool addresses this through sketch-aware prompt recommendation to fill in the necessary semantic information, alleviating users' workload.
% However, some users want the generative AI in the future to be able to directly infer this natural implicit intentions from the sketches without additional prompting since prompt recommendation can still be unstable sometimes.
% This problem reflects a more general dilemma, which ubiquitously exists in all forms of conditioned control for generative models such as canny or scribble control.
% This is because all the control models are trained on pairs of explicit control signal and target image, which is lacking further interpretation or customization of the user intentions behind the seemingly straightforward input.
% For another example, the generative models cannot understand what abstraction level the user has in mind for her personal scribbles.
% Such problems leave more challenges to be addressed by future human-AI co-creation research.
% One possible direction is fine-tuning the conditioned models on individual user's conditioned control data to provide more customized interpretation. 

% \subsection{Balance between recommendation and autonomy}
% AIGC tools are a typical example of 
\subsection{Progressive Sketching}
Currently \tool is mainly aimed at novice users who are only capable of creating very rough sketches by themselves.
However, more accomplished painters or even professional artists typically have a coarse-to-fine creative process. 
Such a process is most evident in painting styles like traditional oil painting or digital impasto painting, where artists first quickly lay down large color patches to outline the most primitive proportion and structure of visual elements.
After that, the artists will progressively add layers of finer color strokes to the canvas to gradually refine the painting to an exquisite piece of artwork.
One participant in our user study (P1) , as a professional painter, has mentioned a similar point "\emph{
I think it is useful for laying out the big picture, give some inspirations for the initial drawing stage}."
Therefore, rough sketch also plays a part in the professional artists' creation process, yet it is more challenging to integrate AI into this more complex coarse-to-fine procedure.
Particularly, artists would like to preserve some of their finer strokes in later progression, not just the shape of the initial sketch.
In addition, instead of requiring the tool to generate a finished piece of artwork, some artists may prefer a model that can generate another more accurate sketch based on the initial one, and leave the final coloring and refining to the artists themselves.
To accommodate these diverse progressive sketching requirements, a more advanced sketch-based AI-assisted creation tool should be developed that can seamlessly enable artist intervention at any stage of the sketch and maximally preserve their creative intents to the finest level. 

\subsection{Ethical Issues}
Intellectual property and unethical misuse are two potential ethical concerns of AI-assisted creative tools, particularly those targeting novice users.
In terms of intellectual property, \tool hands over to novice users more control, giving them a higher sense of ownership of the creation.
However, the question still remains: how much contribution from the user's part constitutes full authorship of the artwork?
As \tool still relies on backbone generative models which may be trained on uncopyrighted data largely responsible for turning the sketch into finished artwork, we should design some mechanisms to circumvent this risk.
For example, we can allow artists to upload backbone models trained on their own artworks to integrate with our sketch control.
Regarding unethical misuse, \tool makes fine-grained spatial control more accessible to novice users, who may maliciously generate inappropriate content such as more realistic deepfake with specific postures they want or other explicit content.
To address this issue, we plan to incorporate a more sophisticated filtering mechanism that can detect and screen unethical content with more complex spatial-semantic conditions. 
% In the future, we plan to enable artists to upload their own style model

% \subsection{From interactive prompting to interactive spatial prompting}


\subsection{Limitations and Future work}

    \textbf{User Study Design}. Our open-ended task assesses the usability of \tool's system features in general use cases. To further examine aspects such as creativity and controllability across different methods, the open-ended task could be improved by incorporating baselines to provide more insightful comparative analysis. 
    Besides, in close-ended tasks, while the fixing order of tool usage prevents prior knowledge leakage, it might introduce learning effects. In our study, we include practice sessions for the three systems before the formal task to mitigate these effects. In the future, utilizing parallel tests (\textit{e.g.} different content with the same difficulty) or adding a control group could further reduce the learning effects.

    \textbf{Failure Cases}. There are certain failure cases with \tool that can limit its usability. 
    Firstly, when there are three or more objects with similar semantics, objects may still be missing despite prompt recommendations. 
    Secondly, if an object's stroke is thin, \tool may incorrectly interpret it as a full area, as demonstrated in the expert results of the open-ended task in Figure~\ref{fig:novice_expert}. 
    Finally, sometimes inclusion relationships (\textit{e.g.} inside) between objects cannot be generated correctly, partially due to biases in the base model that lack training samples with such relationship. 

    \textbf{More support for single object adjustment}.
    Participants (N=4) suggested that additional control features should be introduced, beyond just adjusting size and location. They noted that when objects overlap, they cannot freely control which object appears on top or which should be covered, and overlapping areas are currently not allowed.
    They proposed adding features such as layer control and depth control within the single-object mask manipulation. Currently, the system assigns layers based on color order, but future versions should allow users to adjust the layer of each object freely, while considering weighted prompts for overlapping areas.

    \textbf{More customized generation ability}.
    Our current system is built around a single model $ColorfulXL-Lightning$, which limits its ability to fully support the diverse creative needs of users. Feedback from participants has indicated a strong desire for more flexibility in style and personalization, such as integrating fine-tuned models that cater to specific artistic styles or individual preferences. 
    This limitation restricts the ability to adapt to varied creative intents across different users and contexts.
    In future iterations, we plan to address this by embedding a model selection feature, allowing users to choose from a variety of pre-trained or custom fine-tuned models that better align with their stylistic preferences. 
    
    \textbf{Integrate other model functions}.
    Our current system is compatible with many existing tools, such as Promptist~\cite{hao2024optimizing} and Magic Prompt, allowing users to iteratively generate prompts for single objects. However, the integration of these functions is somewhat limited in scope, and users may benefit from a broader range of interactive options, especially for more complex generation tasks. Additionally, for multimodal large models, users can currently explore using affordable or open-source models like Qwen2-VL~\cite{qwen} and InternVL2-Llama3~\cite{llama}, which have demonstrated solid inference performance in our tests. While GPT-4o remains a leading choice, alternative models also offer competitive results.
    Moving forward, we aim to integrate more multimodal large models into the system, giving users the flexibility to choose the models that best fit their needs. 
    


\section{Conclusion}\label{sec:conclusion}
In this paper, we present \tool, an interactive system designed to help novice users create high-quality, fine-grained images that align with their intentions based on rough sketches. 
The system first refines the user's initial prompt into a complete and coherent one that matches the rough sketch, ensuring the generated results are both stable, coherent and high quality.
To further support users in achieving fine-grained alignment between the generated image and their creative intent without requiring professional skills, we introduce a decompose-and-recompose strategy. 
This allows users to select desired, refined object shapes for individual decomposed objects and then recombine them, providing flexible mask manipulation for precise spatial control.
The framework operates through a coarse-to-fine process, enabling iterative and fine-grained control that is not possible with traditional end-to-end generation methods. 
Our user study demonstrates that \tool offers novice users enhanced flexibility in control and fine-grained alignment between their intentions and the generated images.



% In the unusual situation where you want a paper to appear in the
% references without citing it in the main text, use \nocite
\nocite{langley00}

\bibliography{example_paper}
\bibliographystyle{icml2024}


%%%%%%%%%%%%%%%%%%%%%%%%%%%%%%%%%%%%%%%%%%%%%%%%%%%%%%%%%%%%%%%%%%%%%%%%%%%%%%%
%%%%%%%%%%%%%%%%%%%%%%%%%%%%%%%%%%%%%%%%%%%%%%%%%%%%%%%%%%%%%%%%%%%%%%%%%%%%%%%
% APPENDIX
%%%%%%%%%%%%%%%%%%%%%%%%%%%%%%%%%%%%%%%%%%%%%%%%%%%%%%%%%%%%%%%%%%%%%%%%%%%%%%%
%%%%%%%%%%%%%%%%%%%%%%%%%%%%%%%%%%%%%%%%%%%%%%%%%%%%%%%%%%%%%%%%%%%%%%%%%%%%%%%
% \newpage
% \appendix
% \onecolumn
% \section{You \emph{can} have an appendix here.}

% You can have as much text here as you want. The main body must be at most $8$ pages long.
% For the final version, one more page can be added.
% If you want, you can use an appendix like this one.  

% The $\mathtt{\backslash onecolumn}$ command above can be kept in place if you prefer a one-column appendix, or can be removed if you prefer a two-column appendix.  Apart from this possible change, the style (font size, spacing, margins, page numbering, etc.) should be kept the same as the main body.
%%%%%%%%%%%%%%%%%%%%%%%%%%%%%%%%%%%%%%%%%%%%%%%%%%%%%%%%%%%%%%%%%%%%%%%%%%%%%%%
%%%%%%%%%%%%%%%%%%%%%%%%%%%%%%%%%%%%%%%%%%%%%%%%%%%%%%%%%%%%%%%%%%%%%%%%%%%%%%%


\end{document}


% This document was modified from the file originally made available by
% Pat Langley and Andrea Danyluk for ICML-2K. This version was created
% by Iain Murray in 2018, and modified by Alexandre Bouchard in
% 2019 and 2021 and by Csaba Szepesvari, Gang Niu and Sivan Sabato in 2022.
% Modified again in 2023 and 2024 by Sivan Sabato and Jonathan Scarlett.
% Previous contributors include Dan Roy, Lise Getoor and Tobias
% Scheffer, which was slightly modified from the 2010 version by
% Thorsten Joachims & Johannes Fuernkranz, slightly modified from the
% 2009 version by Kiri Wagstaff and Sam Roweis's 2008 version, which is
% slightly modified from Prasad Tadepalli's 2007 version which is a
% lightly changed version of the previous year's version by Andrew
% Moore, which was in turn edited from those of Kristian Kersting and
% Codrina Lauth. Alex Smola contributed to the algorithmic style files.
