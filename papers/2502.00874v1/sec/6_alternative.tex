\section{Alternative Views}

While the preceding sections advocate for more transparent review processes, it is important to recognize that open or partially open systems are not without drawbacks. Critics highlight issues such as the potential for plagiarism, misappropriation of innovative ideas, and threats to proprietary research, raising valid questions about how best to balance openness with the need for confidentiality.

\paragraph{Plagiarism} 
One frequently cited concern is that open review may inadvertently facilitate plagiarism~\cite{piniewski2024emerging, oviedo2024review} if innovative concepts are publicly visible before a paper is formally published. When submissions are posted online (e.g., in open-review platforms or preprint servers like arXiv) and later rejected, these ideas remain accessible, allowing others to potentially adopt or iterate on them without proper attribution. However, such issues are not exclusive to open review. In fact, the growing trend of researchers posting preprints on arXiv—regardless of whether a conference uses open or closed peer review—reveals that this challenge is part of a broader question of how to protect intellectual property in public forums.

Moreover, confidentiality can serve as a safeguard against idea theft, as it keeps manuscripts and reviews private until final decisions are made. This is seen as particularly important for early-career researchers and smaller institutions, which may lack the resources to compete if their concepts are exposed prematurely. \textbf{Yet, given the rapid pace of AI research and the prevalence of preprint culture, solutions to plagiarism concerns must extend beyond the open-versus-closed review debate.} The research community at large may need clearer norms, stronger protective measures, and more effective reporting systems to uphold ethical standards for all parties involved.


% A primary concern with open review is the risk of plagiarism, particularly when innovative ideas are discussed during the review process but the paper is ultimately rejected. These ideas, now publicly visible, can be exploited by others without proper attribution. Similar concerns arise with platforms like arXiv, where preprints are publicly accessible.

% However, this is not an inherent flaw of open review but rather a broader issue in the dissemination of ideas in public forums. Addressing this would require community-wide efforts to establish norms and mechanisms for protecting intellectual property in open settings.

% Closed review systems provide a critical layer of protection for authors, particularly in fields like AI where the risk of idea theft is significant. By keeping submissions and reviews confidential, these systems help safeguard unpublished concepts from premature disclosure or exploitation by reviewers. This protection is particularly vital for researchers in academia or smaller institutions, who may lack the resources to compete with larger organizations if their innovative ideas are exposed prematurely.

% While the transparency of open reviews is often seen as a solution to systemic issues, critics argue that openness could inadvertently lead to misuse of information or unfair competitive advantages. For example, early-stage ideas disclosed during an open review could be quickly iterated upon by other groups, undermining the original authors' efforts. In this perspective, closed reviews, while not without flaws, provide essential confidentiality and protection necessary to ensure fairness and trust in high-stakes research environments.

\paragraph{Disclosure Policy}

For research scientists working at companies with patent-driven business models, such as those in the pharmaceutical, semiconductor, or AI industries, maintaining confidentiality in the peer review process is crucial. Many companies operate under strict intellectual property (IP) and patent disclosure policies to safeguard innovations before public release. Open review systems, which often require preprints or public sharing of submissions, could inadvertently expose proprietary research and jeopardize a company’s ability to secure patents.

For example, in jurisdictions like the United States~\cite{uspto2013patent}, the first-to-file patent system requires that an invention must not have been publicly disclosed prior to filing. A submission shared in an open review process might qualify as prior art, rendering the invention unpatentable. 

% Companies such as IBM, Google, or DeepMind, which rely on patenting AI and machine learning innovations, could find themselves in a precarious position if their researchers are required to participate in open review processes. Furthermore, public disclosure might allow competitors to quickly adopt or modify ideas, diluting the original innovator’s competitive edge.

In these settings, critics of open review argue that confidentiality helps ensure that breakthroughs remain protected until the necessary legal steps are in place. Without this protection, competitors could quickly adopt or modify ideas, diluting the original innovator’s advantage. While acknowledging the value of transparency, many researchers in industry and academia alike must balance the public benefit of sharing ideas with the practical need to safeguard proprietary innovations.