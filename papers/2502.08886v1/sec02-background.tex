\section{Background} \label{background}
%
In recent years, cyber threats have highlighted the importance of developing adaptive defense mechanisms against evolving attack vectors to prevent situations such as a botnet attack by advanced persistent threat (APT)~\citet{Daws_2024}.
As the cyber security landscape evolves, GenAI has emerged as an essential tool for enhancing security measures.
GenAI, characterized by its ability to produce new content that mimics real-world phenomena, facilitates a spectrum of security applications, from passive threat detection to active mitigation.

In the realm of security, the evolution of GenAI has underscored its role as a double-edged sword.
On the one hand, these technological advances represent a significant step forward in digital transformation, enhancing security through automated responses, threat intelligence, and malware detection~\citet{10198233}.
Their ability to generate highly realistic content across various mediums illustrates the potential to increase threat detection and response measures.
On the other hand, the same capabilities that contribute to security enhancements also present new vulnerabilities, as GenAI models have been exploited by rogue actors for offensive purposes.
This includes generating deepfake videos for disinformation campaigns, crafting convincing phishing emails, and spreading misinformation on social media, introducing new challenges and risks~\citet{eze2024analysis,mitra2024world}.
%
\subsection{Evolution of AI in Security}
%
In this section, we briefly review the landscape of AI within cyber security research. 
%
\subsubsection{Machine Learning (ML)}
%
Detection methods in network security initially relied on traditional machine learning algorithms in the early days.
These methods process large volumes of log data, identify specific patterns, and perform verification.
Techniques such as linear regression and decision trees effectively handle massive data and work well in practical applications.
For example, some machine learning-based intrusion detection systems (IDS) analyze user or device behavior to identify abnormal patterns.
Security operation centers (SOC) use machine learning to detect abnormal activities that deviate from normal behavior.
These methods enhance efficiency and flexibility in security monitoring, enabling real-time threat detection.
However, the analysis of patterns and anomalies is static and requires retraining and tuning should the threats change, especially in the field of cyber security~\citet{zeadally2020harnessing}.
% 
\subsubsection{Deep Learning (DL)}
%
With larger and more complex behaviors being collected, the limitations of traditional ML methods become apparent in terms of the models' capabilities to predict, classify, and learn effectively.
There arises a need to solve more robust, comprehensive, and large-scale problems that ML algorithms would have difficulty solving.
The evolution of ML is followed by the development of deep learning (DL) algorithms, which address more complex problems such as image recognition~\citet{Alzubaidi2021-dk}, sentiment analysis~\citet{zhang2018deep}, deep anomaly detection~\citet{pang2021deep}, and natural language processing~\citet{ghosh2016contextual}.
The ability of deep learning to solve more complex problems has become a foundation for further improvements.
%
\subsubsection{Generative Adversarial Networks (GANs)}
%
Following the adoption of DL, Generative Adversarial Networks (GANs) have introduced a novel dimension to cyber security.
The application of GANs is twofold: enhancing security defenses by increasing their ability to detect sophisticated threats~\citet{park2022enhanced, yinka2020review}, and contributing to the development of complex threats such as AI-driven malware or phishing emails.
A growing number of attacks are leveraging AI-driven techniques as threat actors evolve their strategies. This approach, when combined with conventional attack methods, enables attackers to inflict even greater damage~\citet{kaloudi2020ai}.
%
\subsubsection{Recent Application of GenAI}
%
Generative Pre-trained Transformers (GPTs) represent the latest advancement in this evolution, extending the capability of AI into natural language processing.
This development has a notable contribution to security, as it could be used to enhance security, for example, by developing robust security policies to protect against ransomware attacks.
A study comparing GPTs with conventional policy-making sources found that GPT-generated policies outperform those derived from security vendors and government agencies in terms of effectiveness and ethical compliance, particularly with tailored input and expert oversight~\citet{mcintosh2023harnessing}.
GPTs could also be used to investigate the potential for AI misuse~\citet{renaud2023chatgpt}, such as generating malware using LLMs~\citet{pa2023attacker, greshake2023not}.
Indirect prompt injection facilitates remote exploitation of LLM-integrated applications, posing threats such as data theft and contamination of information ecosystems.
Several practical demonstrations emphasize the risks associated with the execution of arbitrary code and the manipulation of functionality.
%
\subsection{Applications of GenAI in Security}
%
GenAI has significantly transformed security practices by introducing advanced capabilities for threat detection, simulation, and data protection.
The applications of GenAI in this domain include, but are not limited to, the following:

\vspace{1em}
\noindent \textbf{Enhanced Threat Intelligence:} The GenAI model is able to analyze large amounts of data to predict and simulate emerging threats, providing security professionals with information on potential vulnerabilities and attack vectors~\citet{gupta2023chatgpt, alwahedi2024machine}.
Organizations must understand the characteristics of new and evolving threats to prepare more effectively and ensure that they remain one step ahead of cybercriminals.

\vspace{1em}
\noindent \textbf{Sophisticated Phishing Attack Simulations:} GenAI assists in the development of more effective training programs due to its ability to generate convincing phishing emails and social engineering tactics~\citet{bethany2024large}.
Employees are educated about the nuances of phishing attacks through these simulations, thereby significantly reducing the likelihood of successful breaches.

\vspace{1em}
\noindent \textbf{Automated Security Testing:} GenAI could automate the creation of test cases for secure software, ensuring that applications are robust against a wide range of attacks~\citet{hilario2024generative, deng2023pentestgpt}.
This involves generating malicious inputs to test the resilience of systems to injection attacks and other vulnerabilities.
This capability is crucial in sectors such as banking and e-commerce, where identity theft poses significant risks.

\vspace{1em}
\noindent \textbf{Synthetic Identity Fraud Detection:} GenAI models could help in designing algorithms that detect and prevent fraudulent activities by understanding patterns of synthetic identity fraud~\citet{ahmadi2023open}.

\vspace{1em}
\noindent\textbf{Adaptive Defense Mechanisms:} GenAI models are capable of simulating a variety of attack scenarios, enabling security systems to create dynamic defensive strategies~\citet{neupane2023impacts, kucharavy2023fundamentals}.
This approach helps to develop resilient systems that could defend against sophisticated and adaptive threats.
~\citet{sai2024generative} describe ten security products that leverage GenAI to enhance their security measures.
These include Google Cloud Security AI Workbench, Microsoft Security Copilot, and SentinelOne Purple AI.
Additionally, 11 applications of GenAI were identified in the security domain, including threat intelligence, security questionnaires, bridging the gap between technical experts and non-experts, vulnerability scanning and filtering, and secure code generation.
%
\subsection{Applications of GenAI in IoT Security}
%
As researchers explore applications and investigations involving GenAI, we observe early works, though emerging, on the use of GenAI in IoT security given the increasing prevalence of IoT devices and their vulnerabilities.
Therefore, further investigation is necessary on how GenAI could be integrated to improve IoT security measures and strategies.
Relevant publications have been gathered to compile and explore this area, examining how GenAI could address IoT security.
Our findings highlight useful ideas and set the stage for future research in this area.