\section{Introduction}
\label{sec:intro}
%
The development of Generative AI (GenAI) and Large Language Models (LLMs) signifies an advancement in artificial intelligence, distinguished by its capability to generate diverse content, including texts, images, and code.
This capability has brought GenAI tools into the spotlight, facilitating their integration into daily life to address various pertinent issues and tasks.
Given their utility in tasks such as data analysis and content generation, GenAI and LLMs are actively explored for their potential in more complex applications. 
GenAI has garnered significant attention in the field of cyber security, with recent studies underscoring its potential to enhance security measures, simulate attacks for training and testing, and refine threat detection systems through advanced data analytics.
Examples include studies conducted by~\citet{9105926},~\citet{10198233}, and~\citet{hassanin2024comprehensiveoverviewlargelanguage}, all of which focus on the potential advantages that GenAI could offer as a tool to automate various complex security tasks.

Internet of Things (IoT) is increasingly recognized as a critical area requiring detailed attention and innovative approaches, as IoT devices become more integrated into daily life and industrial systems.
As IoT devices are heterogeneous in nature, the security of these devices requires specialized knowledge and expertise.
GenAI has the potential to enhance existing methods or develop new approaches for IoT security, thereby reducing the need for specialist knowledge to implement advanced security solutions.
Consequently, GenAI represents a promising tool for future IoT security research, which could improve both the security and usability of IoT systems.
In the coming years, significant research is anticipated to be conducted on the use of GenAI to improve IoT security.

This paper presents a comprehensive survey of current state-of-the-art work on the application of GenAI to IoT security.
We begin by providing a foundational understanding of IoT systems along with the core principles of Generative AI (GenAI), with a primary focus on LLMs.
As part of our analysis of the current use of GenAI in enhancing IoT security, we explore the application of each model.
Subsequently, we analyze potential further applications, identifying areas where GenAI could be beneficial with three case studies.
Our evaluation is articulated through the use of the MITRE ATT\&CK Mitigation framework for Industrial Control Systems (ICS).
%
\subsection{Internet of Things (IoT)}
%
IoT is a transformative concept in connectivity, where an extensive network enables devices ranging from household appliances to medical equipment to connect directly to the Internet, facilitating seamless data exchange without human intervention.
This innovation has broad applications in smart homes, healthcare, transportation and urban development, significantly improving operational efficiency~\citet{kimani2019cyber}.
~\citet{alwahedi2024machine, chui2023survey} present a similar perspective, who describe the IoT as a network that connects physical objects through embedded sensors and software.
This configuration not only facilitates the exchange of real-time data, but also transforms physical data into digital information, providing a comprehensive means of managing diverse systems.
The framework underscores the ability of the IoT to digitalize physical entities, fostering an intelligent and interconnected environment.

As described in a study by~\citet{hassija2019survey}, the IoT ecosystem consists of four essential layers: the foundational layer that uses sensors and actuators for data collection, followed by a communication layer that transmits the data.
The middleware layer then bridges the data flow between the network and application layers, allowing processing and integration.
The final layer hosts various IoT applications, such as smart grids and smart factories, demonstrating the structured and integrated approach of IoT systems in various sectors.
%
\subsection{IoT Security Challenges}
%
The potential vulnerabilities of IoT devices pose a significant security threat to IoT ecosystems.
Their interconnected nature exposes them to a variety of cyber threats, data breaches, and privacy violations~\citet{hassija2019survey}.
Insufficient security updates, inadequate security measures, and difficulties in managing dynamic device configurations are among the most common security issues. 
These vulnerabilities mainly consist of communication vulnerabilities, operating system vulnerabilities, and software vulnerabilities. 
There is an ongoing research effort to enhance the overall security of IoT to effectively address these issues.

The diverse application of IoT devices, from home automation to medical systems, makes them an attractive target for malicious activity.
Therefore, it is imperative to implement protective measures such as authentication protocols, intrusion detection systems, and machine learning algorithms to fortify these networks against potential threats.
Various methods have been employed to address these issues, including deep learning~\citet{9060970} and blockchain technology~\citet{10.1145/3320154.3320163}.
%
\subsection{Generative AI and Large Language Models}
%
In the context of GenAI development, LLMs could be viewed as a breakthrough in AI innovation due to their ability to generate, classify and reason based on the datasets with which they are trained~\citet{jo2023promise}.
Through advancement in algorithms and computational power, GenAI has become increasingly important in a wide range of domains, including cyber security.
With the capability to generate novel data instances from learned patterns, this technology offers a revolutionary approach to data analysis and simulation, demonstrating its potential for transformative applications in the digital world.
LLMs, such as ChatGPT~\citet{openai2024gpt4} and Gemini~\citet{10113601}, represent a significant advancement in GenAI, particularly in the processing and generation of natural language.
These models have evolved to understand context, generate coherent responses, and even detect anomalies in text, making them invaluable tools that extend beyond simple communication.
LLMs demonstrate the increasing sophistication of AI's ability to handle complex, nuanced tasks, mirroring human-like understanding and interaction with large volumes of data.

Digital defense strategies have been transformed by the integration of GenAI, specifically LLMs, into cyber security.
LLMs are well-positioned to enhance security for interconnected digital systems, including IoT.
The use of these technologies for security purposes is gaining traction, indicating a promising direction for addressing security threats.
The emerging field of GenAI, particularly through the lens of LLMs, represents not only a technological advancement but also a transformative force in cyber security.
These AI models generate realistic simulations and learn complex patterns providing significant benefits.