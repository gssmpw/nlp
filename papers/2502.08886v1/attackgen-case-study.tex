%
\begin{figure}[H]
\centering
\begin{tikzpicture}
[node distance=0.1cm, box/.style={rectangle, rounded corners, fill=blue!5, text width=\textwidth, minimum height=1cm, align=left}, title/.style={font=\large\bfseries}]
\node[box, fill=green!5] (DefaultAG){
%
\tiny
%
\noindent \textbf{Incident Response Testing Scenario: Dragonfly Supply Chain Compromise}

\smallskip
\noindent \textbf{Overview}

\noindent This incident response testing scenario is designed to evaluate the company’s incident and response capabilities against a sophisticated cyber attack by the threat actor group Dragonfly.
The attack vector is a supply chain compromise, which is a common tactic employed by Dragonfly to gain initial access to their targets.

\smallskip
\noindent \textbf{Objectives}
%
\begin{enumerate}[nosep]
    \item Assess the company’s ability to detect and respond to a supply chain compromise.
    \item Evaluate the effectiveness of communication and coordination among incident response team members.
    \item Test the procedures in place for mitigating and recovering from a supply chain compromise.
    \item Identify areas for improvement in the company’s incident response plan.
\end{enumerate}
%
\smallskip
\noindent \textbf{Scenario Details}

\hspace*{2mm} \textbf{Company Profile}
%
\begin{itemize}[nosep]
    \item Industry: Energy/Utilities
    \item Size: Medium (51-200 employees)
\end{itemize}
%
\hspace*{2mm} \textbf{Threat Actor Group}
%
\begin{itemize}[nosep]
    \item Name: Dragonfly
    \item Tactics, Techniques, and Procedures (TTPs): Utilizing supply chain compromise to gain initial access
\end{itemize}
%
\hspace*{2mm} \textbf{Kill Chain Phase}
%
\begin{itemize}[nosep]
    \item Initial Access: Supply Chain Compromise (T0862)   
\end{itemize}
%
\smallskip
\noindent \textbf{Scenario Timeline}

\hspace*{2mm} \textbf{Phase 1: Preparation}

\hspace*{2mm} Objectives:
\begin{itemize}[nosep]
    \item Ensure all incident response team members are aware of their roles and responsibilities.
    \item Verify that all necessary tools and resources are available and operational.
\end{itemize}

\hspace*{2mm} Actions:
\begin{enumerate}[nosep]
    \item Conduct a briefing session with the incident response team.
    \item Review the company’s incident response plan, focusing on supply chain compromises.
    \item Check the availability of forensic tools, communication channels, and backup systems.
\end{enumerate}

\hspace*{2mm} \textbf{Phase 2: Detection}

\hspace*{2mm} Objectives:
\begin{itemize}[nosep]
    \item Test the company’s ability to detect anomalies activities associated with a supply chain compromise.
\end{itemize}

\hspace*{2mm} Simulated Event:
\begin{itemize}[nosep]
    \item A third-party vendor supplying software to the company is compromised. The software update sent to the company contains a malicious payload.
\end{itemize}

\hspace*{2mm} Actions:
\begin{enumerate}[nosep]
    \item Deploy the compromised software update to a test environment.
    \item Monitor network traffic and system logs for signs of the compromise.
    \item Use indicators of compromise (IoCs) such as unusual network connections, unexpected file changes, and system behavior anomalies.
\end{enumerate}

\hspace*{2mm} \textbf{Phase 3: Containment}

\hspace*{2mm} Objectives:
\begin{itemize}[nosep]
    \item Evaluate the company’s capability to contain the threat and prevent further damage.
\end{itemize}

\hspace*{2mm} Simulated Event:
\begin{itemize}[nosep]
    \item The malicious payload attempts to establish a connection to an external command and control (C2) server.
\end{itemize}

\hspace*{2mm} Actions:
\begin{enumerate}[nosep]
    \item Isolate the affected systems from the network.
    \item Block outgoing connections to the identified C2 server.
    \item Notify relevant stakeholders of the incident.
\end{enumerate}

\hspace*{2mm} \textbf{Phase 4: Eradication}

\hspace*{2mm} Objectives:
\begin{itemize}[nosep]
    \item Test the company’s ability to remove the threat from affected systems.
\end{itemize}

\hspace*{2mm} Actions:
\begin{enumerate}[nosep]
    \item Perform a thorough analysis of the compromised systems to identify all malicious components.
    \item Remove the malicious payload and any associated malware.
    \item Apply patches and updates to close the vulnerability exploited by the attacker.
\end{enumerate}

\hspace*{2mm} \textbf{Phase 5: Recovery}

\hspace*{2mm} Objectives:
\begin{itemize}[nosep]
    \item Assess the company’s procedures for restoring normal operations.
\end{itemize}

\hspace*{2mm} Actions:
\begin{enumerate}[nosep]
    \item Restore affected systems from clean backups.
    \item Monitor the systems to ensure that no residual threats remain.
    \item Gradually reconnect the systems to the network.
\end{enumerate}

\hspace*{2mm} \textbf{Phase 6: Lessons Learned}

\hspace*{2mm} Objectives:
\begin{itemize}[nosep]
    \item Review the incident response process to identify strengths and areas for improvement.
\end{itemize}

\hspace*{2mm} Actions:
\begin{enumerate}[nosep]
    \item Conduct a debriefing session with all incident response team members.
    \item Document the incident and response actions taken.
    \item Update the incident response plan based on the lessons learned.
\end{enumerate}

\smallskip
\noindent \textbf{Evaluation Criteria}

\begin{enumerate}[nosep]
    \item Detection:
        \begin{itemize}[nosep]
        \item Time taken to detect the compromise.
        \item Accuracy in identifying IoCs.
        \end{itemize}
    \item Containment:
        \begin{itemize}[nosep]
        \item Speed and effectiveness of containment measures.
        \item Communication and coordination among team members.
        \end{itemize}
    \item Eradication:
        \begin{itemize}[nosep]
        \item Thoroughness in removing all malicious components.
        \item Effectiveness of patching and updating procedures.
        \end{itemize}
    \item Recovery:
        \begin{itemize}[nosep]
        \item Efficiency in restoring normal operations.
        \item Monitoring for residual threats.
        \end{itemize}
    \item Lessons Learned:
        \begin{itemize}[nosep]
        \item Quality of the post-incident analysis.
        \item Implementation of improvements to the incident response plan.
        \end{itemize}
\end{enumerate}

\smallskip
\noindent \textbf{Conclusion}

This incident response testing scenario provides a comprehensive framework to evaluate the company’s readiness against a supply chain compromise by Dragonfly.
By following the outlined phases and actions, the company can identify strengths and areas for improvement in their incident response capabilities, ensuring they are better prepared for future threats.
};

\end{tikzpicture}
    \caption{Incident Response Testing Scenario for Dragonfly Supply Chain Compromise using AttackGen without Modifications}
    \label{fig:attackgen-default}
\end{figure}
%
%
%
%
%
%
%
%
\onecolumn
\begin{figure}[H]
\centering
\begin{tikzpicture}
[node distance=0.1cm, box/.style={rectangle, rounded corners, fill=blue!5, text width=\textwidth, minimum height=1cm, align=left}, title/.style={font=\large\bfseries}]
\node[box] (IoTAG){
%
\tiny
%
\noindent \textbf{Incident Response Testing Scenario: Dragonfly Supply Chain Compromise with IoT-Powered PLC}

\smallskip
\noindent \textbf{Overview}

\noindent This incident response testing scenario is designed to evaluate the company’s readiness and response capabilities against a sophisticated cyber attack by the threat actor group Dragonfly. This attack vector is a supply chain compromise, which is a common tactic employed by Dragonfly to gain initial access to their targets. The scenario will specifically focus on the compromise of an IoT-based Programmable Logic Controller (PLC) that is critical to the company’s operations and is connected to the internet.

\smallskip
\textbf{Objectives}
%
\begin{enumerate}[nosep]
    \item Assess the company’s ability to detect and respond to a supply chain compromise involving an IoT-powered PLC.
    \item Evaluate the effectiveness of communication and coordination among incident response team members.
    \item Test the procedures for monitoring and recovering from a supply chain compromise affecting critical IoT devices.
    \item Identify areas for improvement in the company’s incident response plan.
\end{enumerate}
%

\smallskip
\textbf{Scenario Details}

\hspace*{2mm} \textbf{Company Profile}
%
\begin{itemize}[nosep]
    \item Industry: Energy / Utilities 
    \item Size: Medium (51-200 employees) Critical Asset: IoT powered PLC connected to the internet
\end{itemize}
%
\hspace*{2mm} \textbf{Threat Actor Group}
%
\begin{itemize}[nosep]
    \item Name: Dragonfly
    \item Tactics, Techniques, and Procedures (TTPs): Utilizing supply chain compromise to gain initial access
\end{itemize}
%
\hspace*{2mm} \textbf{Kill Chain Phase}
%
\begin{itemize}[nosep]
    \item Initial Access: Supply Chain Compromise (T0862)  
\end{itemize}
%
\smallskip
\textbf{Scenario Timeline}

\hspace*{2mm} \textbf{Phase 1: Preparation}

\hspace*{2mm} Objectives:
\begin{itemize}[nosep]
    \item Ensure all incident response team members are aware of their roles and responsibilities.
    \item Verify that all necessary tools and resources are available and operational.
\end{itemize}

\hspace*{2mm} Actions:
\begin{enumerate}[nosep]
    \item Conduct a briefing session with the incident response team.
    \item Review the company’s incident response plan, focusing on supply chain compromise scenarios, particularly those involving IoT devices.
    \item Check the availability of forensic tools, communication channels, and backup systems. Ensure proper monitoring mechanisms are in place for IoT devices, especially the PLC.
\end{enumerate}

\hspace*{2mm} \textbf{Phase 2: Detection}

\hspace*{2mm} Objectives:
\begin{itemize}[nosep]
    \item Test the company’s ability to detect anomalous activities associated with a supply chain compromise of an IoT device.
\end{itemize}

\hspace*{2mm} Simulated Event:
\begin{itemize}[nosep]
    \item A third-party vendor supplying firmware to the IoT-powered PLC is compromised. The firmware update sent to the company contains a malicious payload.
\end{itemize}

\hspace*{2mm} Actions:
\begin{enumerate}[nosep]
    \item Deploy the compromised firmware update to a test environment with an IoT-powered PLC.
    \item Monitor network traffic, system logs, and PLC activity for signs of the compromise.
    \item Identify indicators of compromise (IoCs) such as unusual network connections, unexpected firmware changes, and anomalous PLC behavior.
\end{enumerate}

\hspace*{2mm} \textbf{Phase 3: Containment}

\hspace*{2mm} Objectives:
\begin{itemize}[nosep]
    \item Evaluate the company’s capability to contain the threat and prevent further damage, particularly to critical IoT devices.
\end{itemize}

\hspace*{2mm} Simulated Event:
\begin{itemize}[nosep]
    \item The malicious payload attempts to establish a connection to an external command and control (C2) server and manipulate the PLC’s operations.
\end{itemize}

\hspace*{2mm} Actions:
\begin{enumerate}[nosep]
    \item Isolate the affected PLC and any other compromised systems from the network.
    \item Block outgoing connections to the identified C2 server.
    \item Notify relevant stakeholders of the incident, emphasizing the critical nature of the compromised IoT device.
\end{enumerate}

\hspace*{2mm} \textbf{Phase 4: Eradication}

\hspace*{2mm} Objectives:
\begin{itemize}[nosep]
    \item Test the company’s ability to remove the threat from affected systems, including the critical IoT powered PLC.
\end{itemize}

\hspace*{2mm} Actions:
\begin{enumerate}[nosep]
    \item Perform a thorough analysis of the compromised PLC and other affected systems to identify all malicious components.
    \item Remove the malicious payload and any associated malware from the PLC.
    \item Apply patches and updates to close the vulnerability exploited by the attacker, ensuring the integrity of the firmware on the PLC.
\end{enumerate}

\hspace*{2mm} \textbf{Phase 5: Recovery}

\hspace*{2mm} Objectives:
\begin{itemize}[nosep]
    \item Assess the company’s procedures for restoring normal operations, focusing on the critical IoT-powered PLC.
\end{itemize}

\hspace*{2mm} Actions:
\begin{enumerate}[nosep]
    \item Restore the PLC from a clean backup, ensuring its firmware is uncompromised.
    \item Monitor the PLC and other systems to ensure that no residual threats remain.
    \item Gradually reconnect the PLC and other affected systems to the network.
\end{enumerate}

\hspace*{2mm} \textbf{Phase 6: Lessons Learned}

\hspace*{2mm} Objectives:
\begin{itemize}[nosep]
    \item Review the incident response process to identify strengths and areas for improvement, with a focus on IoT devices.
\end{itemize}

\hspace*{2mm} Actions:
\begin{enumerate}[nosep]
    \item Conduct a debriefing session with all incident response team members, emphasizing the response to the compromised IoT-powered PLC.
    \item Document the incident and response actions taken, particularly those involving IoT devices.
    \item Update the incident response plan based on the lessons learned, including specific measures for protecting and responding to incidents involving IoT devices.
\end{enumerate}

\smallskip
\textbf{Evaluation Criteria}

\begin{enumerate}[nosep]
    \item Detection:
        \begin{itemize}[nosep]
        \item Time taken to detect the compromise.
        \item Accuracy in identifying IoCs, especially those related to IoT devices.
        \end{itemize}
    \item Containment:
        \begin{itemize}[nosep]
        \item Speed and effectiveness of containment measures.
        \item Communication and coordination among team members, with a focus on IoT device management.
        \end{itemize}
    \item Eradication:
        \begin{itemize}[nosep]
        \item Thoroughness in removing all malicious components.
        \item Effectiveness of patching and updating procedures, particularly for IoT devices.
        \end{itemize}
    \item Recovery:
        \begin{itemize}[nosep]
        \item Efficiency in restoring normal operations.
        \item Monitoring for residual threats, especially in IoT devices.
        \end{itemize}
    \item Lessons Learned:
        \begin{itemize}[nosep]
        \item Quality of the post-incident analysis.
        \item Implementation of improvements to the incident response plan, including IoT-specific considerations.
        \end{itemize}
\end{enumerate}

\smallskip
\textbf{Conclusion}

This updated incident response testing scenario provides a comprehensive framework to evaluate the company’s readiness against a supply chain compromise by Dragonfly, with a specific focus on a critical IoT-powered PLC.
By following the outlined phases and criteria, the company can identify strengths and areas for improvement in their incident response capabilities, ensuring they are better prepared for future threats.
};

\end{tikzpicture}
    \caption{Incident Response Testing Scenario for Dragonfly Supply Chain Compromise Using AttackGen with IoT-Specific Modifications}
    \label{fig:attackgen-iot}
\end{figure}
