\begin{landscape}
\begin{table}
    \centering
    \caption{Evaluation and Justification of State-of-the-Art Works}
    \label{tab:potentials-details}
    \catcode`\_=12
    \catcode`\#=12
    \begin{threeparttable}
    \begin{tabular}{|r|c|c|c|l|}
    \hline    
    \thead[c]{#} & \thead[c]{GenAI / LLM Works} & Capability & Evaluation & \thead[c]{Justification} \\
    \cline{1-5}
    \multirow{6}{*}{1} &
    \multirow{6}{*}{\makecell[c]{LLMSecGuard\\\citet{Kavian2024LLMSG}}}
      & ETD & \emptycirc & Focuses on secure code generation but does not detect external threats. \\
    %\cline{3-5}
    & & IAD & \halfcirc & Detects internal vulnerabilities but does not identify backdoors, network anomalies, etc. \\
    %\cline{3-5}
    & & RA  & \fullcirc & Automates analysis and patching process with minimal user input. \\
    %\cline{3-5}
    & & RM  & \emptycirc & Secure coding is a well-established field with limited opportunities for new research. \\
    %\cline{3-5}
    & & DP  & \fullcirc & Potential for further development through fine-tuning in specific fields. \\
    %\cline{3-5}
    & & IS  & \halfcirc & Limited to addressing internal code vulnerabilities. \\
    \cline{1-5}
    \multirow{6}{*}{2} &
    \multirow{6}{*}{\makecell[c]{LLift\\\citet{libugdetection2024}}}
      & ETD & \emptycirc & Detects UBI bugs but does not identify external threats. \\
    %\cline{3-5}
    & & IAD & \halfcirc & Does not address the detection of broader internal anomalies. \\
    %\cline{3-5}
    & & RA  & \halfcirc & Does not incorporate patching, limiting automation. \\
    %\cline{3-5}
    & & RM  & \fullcirc & Research in UBI bug detection is still in progress, leaving room for further study. \\
    %\cline{3-5}
    & & DP  & \fullcirc & Potential for automated patching and real-time detection. \\
    %\cline{3-5}
    & & IS  & \fullcirc & Impactful as UBI bugs could lead to privilege escalation and information leakage. \\
    \cline{1-5}
    \multirow{6}{*}{3} &
    \multirow{6}{*}{\makecell[c]{HuntGPT\\\citet{ali2023huntgpt}}}
      & ETD &  \halfcirc & Detects external threats but only analyzes reports, not real-time threats. \\
    %\cline{3-5}
    & & IAD & \emptycirc & Does not support internal anomaly detection. \\
    %\cline{3-5}
    & & RA  & \fullcirc & Generates detailed reports and explanations with an interactive dashboard. \\
    %\cline{3-5}
    & & RM  & \fullcirc & More IoT datasets for fine-tuning with many new works in this field. \\
    %\cline{3-5}
    & & DP  & \fullcirc & Extensible towards live intrusion detection and internal anomaly detection. \\
    %\cline{3-5}
    & & IS  & \fullcirc & Automates forensic analysis and reduces the time required to classify external threats. \\
    \cline{1-5}
    \multirow{6}{*}{4} &
    \multirow{6}{*}{\makecell[c]{LLMind\\\citet{cui2024llmind}}}
      & ETD & \halfcirc & Designed for automated task execution with limited external threat detection capabilities. \\
    %\cline{3-5}
    & & IAD & \emptycirc & Does not support internal anomaly detection. \\
    %\cline{3-5}
    & & RA  & \fullcirc & Fully automated in executing tasks, requiring no user input. \\
    %\cline{3-5}
    & & RM  & \fullcirc & Research in task automation is still evolving. \\
    %\cline{3-5}
    & & DP  & \fullcirc & Development potential to expand into security-specific tasks such as automated IP blocking. \\
    %\cline{3-5}
    & & IS  & \halfcirc & Limited to simple network and physical security tasks. \\
    \cline{1-5}
    \multirow{6}{*}{5} &
    \multirow{6}{*}{~\citet{wang2024hybrid}}
      & ETD & \halfcirc & Prevents external threats by identifying UPR variables but no active detection or mitigatation. \\
    %\cline{3-5}
    & & IAD & \halfcirc & Provides limited internal anomaly detection but does not address broader vulnerabilities. \\
    %\cline{3-5}
    & & RA  & \fullcirc & Capable to identify UPR variables and prompt users with minimal human intervention. \\
    %\cline{3-5}
    & & RM  & \emptycirc & Little room for further improvements. \\
    %\cline{3-5}
    & & DP  & \fullcirc & Development potential includes specialization for IoT security and automated patching. \\
    %\cline{3-5}
    & & IS  & \fullcirc & Impactful since it prevents privilege escalation through UPR variable identification. \\
    \hline
    \end{tabular}
    \begin{comment}
    \begin{tablenotes}[para]        
        ETD: External Threat Detection.
        IAD: Internal Anomaly Detection.
        RA: Response Automation.
        RM: Research Maturity.
        DP: Development Potential.
        IS: Implact on Security.        
    \end{tablenotes}    
    \end{comment}
    \end{threeparttable}
\end{table}
\end{landscape}
%
%
\begin{landscape}
\begin{table}
    \centering
    \catcode`\_=12
    \catcode`\#=12
    \begin{threeparttable}
    \begin{tabular}{|r|c|c|c|l|}
    \hline    
    \multirow{6}{*}{6} &
    \multirow{6}{*}{\makecell[c]{NVISOsecurity\\\citet{Raman_2024}}}
      & ETD & \fullcirc & Detects external threats by simulating various attacks based on the MITRE framework. \\
    %\cline{3-5}
    & & IAD & \emptycirc  & Focuses entirely on external attack simulation and response. \\
    %\cline{3-5}
    & & RA  & \fullcirc & Fully automates security testing by executing attack-defense scenarios with minimal human input. \\
    %\cline{3-5}
    & & RM  & \fullcirc & The field is continuously evolving with MITRE and Metasploit. \\
    %\cline{3-5}
    & & DP  & \fullcirc & Development potential towards real-time threat response and IoT security. \\    
    %\cline{3-5}
    & & IS  & \fullcirc & Major impact on cybersecurity by automating attack-defense simulations and improving security standards. \\
    \cline{1-5}
    \multirow{6}{*}{7} &
    \multirow{6}{*}{\makecell[c]{Cyber Sentinel\\\citet{kaheh2023cyber}}}
      & ETD & \fullcirc & Detects external threats by retrieving and blocking IP addresses to prevent exploitation. \\
    %\cline{3-5}
    & & IAD & \emptycirc & No evidence that it detects internal anomalies, though it may have potential for such functionality. \\
    %\cline{3-5}
    & & RA  &  \fullcirc & Fully automated, executing security tasks with minimal user input. \\
    %\cline{3-5}
    & & RM  &  \fullcirc & Automated security task execution with LLMs is still an active research field. \\
    %\cline{3-5}
    & & DP  &  \fullcirc & Development potential for further IoT security applications and penetration testing. \\
    %\cline{3-5}
    & & IS  &  \fullcirc & Simplifies security tasks for non-experts and improves automation. \\
    %
    \cline{1-5}
    \multirow{6}{*}{8} &
    \multirow{6}{*}{\makecell[c]{VIoTGPT\\\citet{zhong2023viotgpt}}}
      & ETD & \fullcirc & Detect external threats through visual analysis, identifying physical attacks and anomalous behavior. \\
    %\cline{3-5}
    & & IAD & \emptycirc & Solely focuses on physical security. \\
    %\cline{3-5}
    & & RA  & \fullcirc & Fully automates visual analysis and anomaly detection with minimal user input. \\
    %\cline{3-5}
    & & RM  & \fullcirc & The research field of visual AI security is still developing, with improvements in suspicious activity recognition. \\
    %\cline{3-5}
    & & DP  & \fullcirc & Development potential includes task execution, sound analysis, and active security responses. \\
    %\cline{3-5}
    & & IS  & \emptycirc & Improves the detection of physical attacks detection but does not consider network or software security. \\
    %
    \cline{1-5}
    \multirow{6}{*}{9} &
    \multirow{6}{*}{~\citet{saha2023llm}}
      & ETD & \emptycirc & Does not detect or prevent external threats as it only critiques hardware design. \\
    %\cline{3-5}
    & & IAD & \halfcirc & Detects internal vulnerabilities but is limited to hardware design flaws, not software or system anomalies. \\
    %\cline{3-5}
    & & RA  & \halfcirc & Provides critiques based on user-defined security criteria. \\
    %\cline{3-5}
    & & RM  & \fullcirc & Research in hardware security utilizing LLMs is still at an early stage. \\
    %\cline{3-5}
    & & DP  & \fullcirc & Development potential includes automating design critiques and adapting diverse security standards. \\
    %\cline{3-5}
    & & IS  & \fullcirc & Enhances hardware security and mitigates side-channel attacks and zero-day vulnerabilities. \\
    %
    \cline{1-5}
    \multirow{6}{*}{10} &
    \multirow{6}{*}{ \makecell[c]{BERTIDS\\\citet{lira2024}}}
      & ETD & \halfcirc & Detects external threats through network log analysis but does not analyze live threats. \\
    %\cline{3-5}
    & & IAD & \emptycirc & Does not detect internal anomalies as it only analyzes network logs. \\
    %\cline{3-5}
    & & RA  & \fullcirc & Fully automates attack classification based on network log data with minimal user input. \\
    %\cline{3-5}
    & & RM  & \emptycirc & The field of intrusion detection is well-established, leaving little room for novel research contributions. \\
    %\cline{3-5}
    & & DP  & \halfcirc & Development potential for live network monitoring and automated task execution. \\
    %\cline{3-5}
    & & IS  & \halfcirc & Limited impact on security as it does not offer new advancements beyond existing IDS. \\
    %
    \cline{1-5}
    \multirow{6}{*}{11} &
    \multirow{6}{*}{~\citet{guastalla2023application}}
      & ETD & \halfcirc & Detects external threats (i.e., DDoS attacks) but does not extend to other types of attacks. \\
    %\cline{3-5}
    & & IAD & \emptycirc & Does not detect internal anomalies. \\
    %\cline{3-5}
    & & RA  & \fullcirc & Fully automates the detection of DDoS attacks with a minimum human involvement. \\
    %\cline{3-5}
    & & RM  & \emptycirc & DDoS attack detection is a well-established research field, limiting its novelty. \\
    %\cline{3-5}
    & & DP  & \halfcirc & Development potential to extend detection of other types of attacks. \\
    %\cline{3-5}
    & & IS  & \fullcirc & An early example of LLM-based anomaly detection for IoT security. \\
    \hline
    \end{tabular}
    \end{threeparttable}
\end{table}
\end{landscape}
%
%
\begin{landscape}
\begin{table}
    \centering    
    \catcode`\_=12
    \catcode`\#=12
    \begin{threeparttable}
    \begin{tabular}{|r|c|c|c|l|}
    \hline    
    \multirow{6}{*}{12} &
    \multirow{6}{*}{\makecell[c]{SecurityBERT\\~\citet{ferrag2024revolutionizing}}}
      & ETD & \halfcirc & Detects external threats, such as malware and injection attacks, but does not automatically prevent them. \\
    %\cline{3-5}
    & & IAD & \emptycirc & It only processes network logs and lacks system vulnerability analysis. \\
    %\cline{3-5}
    & & RA  & \fullcirc & Automatically classifies external threats using network data, without human intervention. \\
    %\cline{3-5}
    & & RM  & \emptycirc & Intrusion detection is a well-researched field with limited room for novel contributions. \\
    %\cline{3-5}
    & & DP  &  \halfcirc & Development potential to integrate automated response mechanisms. \\
    %\cline{3-5}
    & & IS  & \halfcirc & Limited impact as it lacks additional functionality beyond detection. \\
    %
    \cline{1-5}
    \multirow{6}{*}{13} &
    \multirow{6}{*}{\makecell[c]{IDS-Agent\\\citet{li2024idsagent}}}
      & ETD & \fullcirc & Detects external threats by analyzing network traffic for suspicious patterns. \\
    %\cline{3-5}
    & & IAD & \emptycirc & Focuses on monitoring network behaviors, not internal software vulnerabilities. \\
    %\cline{3-5}
    & & RA  & \fullcirc & Fully automates intrusion detection with minimal human interaction. \\
    %\cline{3-5}
    & & RM  & \fullcirc & Intrusion detection remains an active field of research, with evolving methods for network security. \\
    %\cline{3-5}
    & & DP  & \fullcirc & Development potential for real-time detection and response automation. \\
    %\cline{3-5}
    & & IS  & \fullcirc & Impactful by extending real-time network threat detection. \\
    %
    \cline{1-5}
    \multirow{6}{*}{14} &
    \multirow{6}{*}{~\citet{islam2024llmpowered}}
      & ETD & \emptycirc & Focuses solely on internal vulnerability patching. \\
    %\cline{3-5}
    & & IAD & \fullcirc & Detect and patch internal code vulnerabilities to prevent exploitation. \\
    %\cline{3-5}
    & & RA  & \fullcirc & Fully automates detecting and patching vulnerabilities with minimal user input. \\
    %\cline{3-5}
    & & RM  & \halfcirc & Vulnerability patching using LLMs is an emerging research field. \\
    %\cline{3-5}
    & & DP  & \fullcirc & Optimize patching efficiency and integration with security frameworks. \\
    %\cline{3-5}
    & & IS  & \fullcirc & Enhances efficiency in automatically identifying and patching vulnerabilities. \\
    \cline{1-5}
    \multirow{6}{*}{15} &
    \multirow{6}{*}{\makecell[c]{DefectHunter\\\citet{wang2023defecthunter}}}
      & ETD & \emptycirc & Strictly focuses on internal software security. \\
    %\cline{3-5}
    & & IAD & \fullcirc & Detects and patches internal code vulnerabilities, preventing internal anomalies. \\
    %\cline{3-5}
    & & RA  & \fullcirc & Fully automates vulnerability detection and patching, making it more efficient. \\
    %\cline{3-5}
    & & RM  & \halfcirc & Secure coding is well-established while automated patching using LLMs is an emerging research area. \\
    %\cline{3-5}
    & & DP  & \fullcirc & Development potential in integrating real-time security monitoring features. \\
    %\cline{3-5}
    & & IS  & \fullcirc & Significant security impact as automated patching enhances efficiency and reduces vulnerabilities. \\
    %
    \cline{1-5}
    \multirow{6}{*}{16} &
    \multirow{6}{*}{\makecell[c]{AttackGen\\\citet{Adams_2024}}}
      & ETD & \fullcirc & Provides external threat prevention through incident reports and mitigation strategies. \\
    %\cline{3-5}
    & & IAD & \emptycirc & Does not detect internal anomalies. \\
    %\cline{3-5}
    & & RA  & \fullcirc & Fully automates generating incident reports and playbooks with minimal user input. \\
    %\cline{3-5}
    & & RM  & \fullcirc & Automated incident report generation is a new field enabled by LLMs. \\
    %\cline{3-5}
    & & DP  & \fullcirc & High development potential, especially in integrating autonomous task execution. \\
    %\cline{3-5}
    & & IS  & \fullcirc & Pioneers the use of LLMs for cyber security automation and report generation. \\    
    \cline{1-5}
    \multirow{6}{*}{17} &
    \multirow{6}{*}{~\citet{mcintosh2023harnessing}}
      & ETD & \fullcirc & Generates security policies to improve system security. \\
    %\cline{3-5}
    & & IAD & \emptycirc & Focuses on policy generation rather than system vulnerability analysis. \\
    %\cline{3-5}
    & & RA  & \fullcirc & Fully automates policy generation based on user queries, with no further input required. \\
    %\cline{3-5}
    & & RM  & \fullcirc & LLM-driven cyber security policy generation is still emerging, making it a promising area of research. \\
    %\cline{3-5}
    & & DP  & \halfcirc & Development potential is constrained, as its focuses strictly on policy creation without automated execution. \\
    %\cline{3-5}
    & & IS  & \fullcirc & Improves efficiency and accuracy in cyber security policy development, streamlining industry practices. \\
    \hline
    \end{tabular}    
    \end{threeparttable}
\end{table}
\end{landscape}
%
%
\begin{landscape}
\begin{table}
    \centering    
    \catcode`\_=12
    \catcode`\#=12
    \begin{threeparttable}
    \begin{tabular}{|r|c|c|c|l|}
    \hline    
    \multirow{6}{*}{18} &
    \multirow{6}{*}{~\citet{bethany2024large}}
      & ETD & \halfcirc & Helps prevent external threats by generating spear-phishing emails for training. \\
    %\cline{3-5}
    & & IAD & \emptycirc & Does not detect internal anomalies. \\
    %\cline{3-5}
    & & RA  & \fullcirc & Fully automates phishing email generation with minimal human input. \\
    %\cline{3-5}
    & & RM  & \emptycirc & Phishing is a well-researched field, and future developments shall focus more on efficiency. \\
    %\cline{3-5}
    & & DP  & \halfcirc & Development potential includes refining content generation. \\
    %\cline{3-5}
    & & IS  & \halfcirc & Limited impact since phishing is well-known, and awareness training already exists. \\
    \cline{1-5}
    \multirow{6}{*}{19} &
    \multirow{6}{*}{~\citet{yamin2024}}
      & ETD & \halfcirc & Prevents external threats by generating cyber security training scenarios. \\
    %\cline{3-5}
    & & IAD & \emptycirc & Does not detect internal anomalies, focusing on training rather than identifying system vulnerabilities. \\
    %\cline{3-5}
    & & RA  & \fullcirc & Fully automates scenario generation with minimal human input. \\
    %\cline{3-5}
    & & RM  & \halfcirc & Cyber security training is an established field, but research on standardized best practices is ongoing. \\
    %\cline{3-5}
    & & DP  & \emptycirc & Limited development potential in refining existing features rather than add new functionalities. \\
    %\cline{3-5}
    & & IS  & \halfcirc & Limited to cyber security training. \\
    \cline{1-5}
    \multirow{6}{*}{20} &
    \multirow{6}{*}{\makecell[c]{LLM4Vuln\\\citet{sun2024llm4vuln}}}
      & ETD & \emptycirc & Does not detect or prevent external threats. \\
    %\cline{3-5}
    & & IAD & \fullcirc & Detects internal anomalies by identifying vulnerabilities through automated analysis. \\
    %\cline{3-5}
    & & RA  & \fullcirc & Fully automates vulnerability detection with minimal user intervention. \\
    %\cline{3-5}
    & & RM  & \fullcirc & Vulnerability exploration is a continuously evolving research field. \\
    %\cline{3-5}
    & & DP  & \fullcirc & Development potential to integrate automated patching and contextualized vulnerability remediation. \\
    %\cline{3-5}
    & & IS  & \fullcirc & Automated vulnerability detection improves cyber security efficiency and effectiveness. \\
    \cline{1-5}
    \multirow{6}{*}{21} &
    \multirow{6}{*}{\makecell[c]{AutoAttacker\\\citet{xu2024autoattacker}}}
      & ETD & \fullcirc & Autonomously launches attacks to uncover known vulnerabilities in systems. \\
    %\cline{3-5}
    & & IAD & \halfcirc & Exploits internal vulnerabilities but does not generate new attack methods. \\
    %\cline{3-5}
    & & RA  & \fullcirc & Fully automates penetration testing by executing complex tasks and launching attacks. \\
    %\cline{3-5}
    & & RM  & \fullcirc & Automated attack simulation is continuously evolving, offering new research directions. \\
    %\cline{3-5}
    & & DP  & \fullcirc & High development potential to discover new vulnerabilities beyond known exploits. \\
    %\cline{3-5}
    & & IS  & \halfcirc & Limited to commonly known vulnerabilities. \\
    \cline{1-5}
    \multirow{6}{*}{22} &
    \multirow{6}{*}{~\citet{tóth2024llms}}
      & ETD & \halfcirc & Detects external threats such as XSS and SQL injection but is limited to web-based applications. \\
    %\cline{3-5}
    & & IAD & \halfcirc & Detects vulnerabilities in web applications. \\
    %\cline{3-5}
    & & RA  & \fullcirc & Automates vulnerability detection with minimal human input, analyzing code for security flaws. \\
    %\cline{3-5}
    & & RM  & \emptycirc & Web security is a mature field with existing best practices, limiting the novelty of this work. \\
    %\cline{3-5}
    & & DP  & \fullcirc & High development potential to extend beyond Web security and enhance automation. \\
    %\cline{3-5}
    & & IS  & \halfcirc & Limited impact as it primarily automates vulnerability detection. \\
    \cline{1-5}
    \multirow{6}{*}{23} &
    \multirow{6}{*}{~\citet{oliinyk2024fuzzing}}
      & ETD & \halfcirc & Prevents external threats by exposing vulnerabilities through fuzzing. \\
    %\cline{3-5}
    & & IAD & \emptycirc & Only tests inputs that may cause crashes. \\
    %\cline{3-5}
    & & RA  & \fullcirc & Fully automates the fuzzing process, executing crash tests with minimal human input. \\
    %\cline{3-5}
    & & RM  & \halfcirc & While fuzzing is well-established, new opportunities emerge with recent advancement in GenAI and LLMs. \\
    %\cline{3-5}
    & & DP  & \halfcirc & Limited development potential due to its focus on embedded Linux. \\
    %\cline{3-5}
    & & IS  & \halfcirc & Increases fuzzing efficiency but does not provide solutions for discovered vulnerabilities. \\
    \hline
    \end{tabular}    
    \end{threeparttable}
\end{table}
\end{landscape}
%
%
\begin{landscape}
\begin{table}
    \centering    
    \catcode`\_=12
    \catcode`\#=12
    \begin{threeparttable}
    \begin{tabular}{|r|c|c|c|l|}
    \hline    
    \multirow{6}{*}{24} &
    \multirow{6}{*}{~\citet{happe2024llms}}
      & ETD & \halfcirc & Identifies and mitigates external threats effectively through penetration testing. \\
    %\cline{3-5}
    & & IAD & \emptycirc & Does not detect internal software vulnerabilities or system anomalies. \\
    %\cline{3-5}
    & & RA  & \halfcirc & Fully automates security testing and attack simulations with minimal human intervention. \\
    %\cline{3-5}
    & & RM  & \emptycirc & Automated security testing is an evolving field with significant ongoing research. \\
    %\cline{3-5}
    & & DP  & \fullcirc & High development potential to refine detection methods and extend capabilities. \\
    %\cline{3-5}
    & & IS  & \emptycirc & Focused on penetration testing, but does not extend to broader security solutions. \\
    \cline{1-5}
    \multirow{6}{*}{25} &
    \multirow{6}{*}{\makecell[c]{LLMIF\\\citet{wangfuzzing2024}}}
      & ETD & \fullcirc & Detects external threats by exposing vulnerabilities in protocol implementation through fuzzing. \\
    %\cline{3-5}
    & & IAD & \emptycirc & Does not analyze internal anomalies, as it is limited to network fuzzing. \\
    %\cline{3-5}
    & & RA  & \fullcirc & Fully automates network protocol fuzzing with minimal user input. \\
    %\cline{3-5}
    & & RM  & \halfcirc & Fuzzing research is still ongoing, but LLM-driven fuzzing remains relatively new. \\
    %\cline{3-5}
    & & DP  & \halfcirc & Can be expanded with additional fuzzing techniques and broader protocol support. \\
    %\cline{3-5}
    & & IS  & \halfcirc & Demonstrates the feasibility and effectiveness of automated protocol fuzzing. \\    
    \cline{1-5}
    \multirow{6}{*}{26} &
    \multirow{6}{*}{\makecell[c]{ChatAFL\\\citet{meng2024large}}}
      & ETD & \fullcirc & Effectively detects protocol-based security flaws through automated fuzzing. \\
    %\cline{3-5}
    & & IAD & \emptycirc & Does not focus on internal vulnerabilities beyond protocol-level issues. \\
    %\cline{3-5}
    & & RA  & \fullcirc & Fully automates protocol fuzzing, generating structured input cases independently. \\
    %\cline{3-5}
    & & RM  & \fullcirc & The niche field is still relatively young, due to its recent emergence. \\
    %\cline{3-5}
    & & DP  & \fullcirc & High potential for expanding into diverse network protocols and refining attack strategies. \\    
    %\cline{3-5}
    & & IS  & \fullcirc & Enables stealth testing via mobile devices, inspiring further research in offensive and defensive security. \\    
    \cline{1-5}
    \multirow{6}{*}{27} &
    \multirow{6}{*}{\makecell[c]{FIAL\\\citet{androidfuzz2024}}}
      & ETD & \halfcirc & Detects external threats through network fuzzing but limited to network attack vector. \\
    %\cline{3-5}
    & & IAD & \emptycirc & Does not detect internal threats or expose vulnerabilities in the system architecture or internal components. \\
    %\cline{3-5}
    & & RA  & \fullcirc & Fully automates network fuzzing with minimal human intervention through an Android device. \\
    %\cline{3-5}
    & & RM  & \halfcirc & Fuzzing is a well-established research field, but ongoing research using LLMs to enhance its automation. \\
    %\cline{3-5}
    & & DP  & \fullcirc & High development potential to extend execution methods beyond Android (e.g., iOS and other devices). \\
    %\cline{3-5}
    & & IS  & \fullcirc & Automates penetration testing but does not directly strengthen defensive measures. \\
    \cline{1-5}
    \multirow{6}{*}{28} &
    \multirow{6}{*}{~\citet{fang2024llm}}
      & ETD & \fullcirc & Detects and prevents external threats by automating attack simulations and discovering vulnerabilities. \\
    %\cline{3-5}
    & & IAD & \emptycirc & Does not analyze internal configurations or detect internal anomalies. \\
    %\cline{3-5}
    & & RA  & \fullcirc & Fully automates attack execution, launching exploits with minimal human input. \\
    %\cline{3-5}
    & & RM  & \halfcirc & Penetration testing is a well-established research field, but LLM-based automation introduces new research opportunities. \\
    %\cline{3-5}
    & & DP  & \halfcirc & Development potential includes its ability to discover vulnerabilities without CVEs. \\
    %\cline{3-5}
    & & IS  & \halfcirc & Expands fuzzing capabilities but does not directly protect systems. \\
    \cline{1-5}
    \multirow{6}{*}{29} &
    \multirow{6}{*}{\makecell[c]{mGPTFuzz\\\citet{Maetal2024}}}
      & ETD & \emptycirc & Discovers vulnerabilities through fuzzing. \\
    %\cline{3-5}
    & & IAD & \fullcirc & Highly effective at detecting internal vulnerabilities. \\
    %\cline{3-5}
    & & RA  & \fullcirc & Fully automated with minimal human intervention. \\
    %\cline{3-5}
    & & RM  & \emptycirc & Fuzzing is a well-established method with limited new research potential. \\
    %\cline{3-5}
    & & DP  & \halfcirc & Development potential to extend beyond Matter to other protocols. \\
    %\cline{3-5}
    & & IS  & \halfcirc & Remains focused on fuzzing without broader security applications. \\
    \hline
    \end{tabular}
    \end{threeparttable}
\end{table}
\end{landscape}
%
%
\begin{landscape}
\begin{table}
    \centering    
    \catcode`\_=12
    \catcode`\#=12
    \begin{threeparttable}
    \begin{tabular}{|r|c|c|c|l|}
    \hline    
    \multirow{6}{*}{30} &
    \multirow{6}{*}{\makecell[c]{PentestGPT\\\citet{deng2023pentestgpt}}}
      & ETD & \fullcirc & Detects and prevents external threats by simulating attacks, achieving high effectiveness. \\
    %\cline{3-5}
    & & IAD & \halfcirc & Exploits internal vulnerabilities but does not have the ability to patch or fix them. \\
    %\cline{3-5}
    & & RA  & \fullcirc & Fully automates penetration testing, requiring minimal human intervention. \\
    %\cline{3-5}
    & & RM  & \fullcirc & Penetration testing remains an evolving research field, with continuous improvements in defense mechanisms. \\
    %\cline{3-5}
    & & DP  & \halfcirc & Development potential to extend its capabilities to specialized areas such as IoT or ICS. \\
    %\cline{3-5}
    & & IS  & \fullcirc & Improves penetration testing efficiency and strengthens system defense. \\
    \cline{1-5}
    \multirow{6}{*}{31} &
    \multirow{6}{*}{\makecell[c]{Net-GPT\\\citet{10419242}}}
      & ETD & \fullcirc & Detects external threats through man-in-the-middle (MitM) attacks to uncover system vulnerabilities. \\
    %\cline{3-5}
    & & IAD & \emptycirc & Only focuses on network-based vulnerabilities. \\
    %\cline{3-5}
    & & RA  & \fullcirc & Fully automates the generation and execution of network packet Mimicry for MitM attacks. \\
    %\cline{3-5}
    & & RM  & \halfcirc & Using LLMs for MitM introduces new exploration opportunities. \\
    %\cline{3-5}
    & & DP  & \fullcirc & High development potential to extend beyond UAV drone networks to other systems, including IoT. \\
    %\cline{3-5}
    & & IS  & \halfcirc & Limited to exposing MitM vulnerabilities. \\
    \cline{1-5}
    \multirow{6}{*}{32} &
    \multirow{6}{*}{~\citet{Happe_2023}}
      & ETD & \fullcirc & Identifies and mitigates external threats effectively through penetration testing. \\
    %\cline{3-5}
    & & IAD & \emptycirc & Does not detect internal software vulnerabilities or system anomalies. \\
    %\cline{3-5}
    & & RA  & \fullcirc & Fully automates security testing and attack simulations with minimal human intervention. \\
    %\cline{3-5}
    & & RM  & \fullcirc & Automated security testing is an evolving field with significant ongoing research. \\
    %\cline{3-5}
    & & DP  & \fullcirc & High development potential to refine detection methods and extend capabilities. \\
    %\cline{3-5}
    & & IS  & \halfcirc & Limited to forensic testing but does not proactively prevent new attacks. \\
    \cline{1-5}
    \multirow{6}{*}{33} &
    \multirow{6}{*}{~\citet{yang2023iot}}
      & ETD & \emptycirc & Does not exploit vulnerabilities through network or external attack vectors. \\
    %\cline{3-5}
    & & IAD & \halfcirc & Detects internal threats using static code analysis. \\
    %\cline{3-5}
    & & RA  & \fullcirc & Autonomously constrains vulnerabilities with minimal user interaction, achieving high detection accuracy. \\
    %\cline{3-5}
    & & RM  & \emptycirc & Static code analysis is a mature research field, and this tool does not introduce novel research directions. \\
    %\cline{3-5}
    & & DP  & \fullcirc & Development potential to automate vulnerability fixing and improve detection efficiency. \\
    %\cline{3-5}
    & & IS  & \halfcirc & Limited to improving efficiency in static code analysis but does not contribute to broader security solutions. \\    
    \hline
    \end{tabular}    
    \begin{tablenotes}[para]        
        ETD: External Threat Detection.
        IAD: Internal Anomaly Detection.
        RA: Response Automation.
        RM: Research Maturity.
        DP: Development Potential.
        IS: Implact on Security.        
    \end{tablenotes}    
    \end{threeparttable}
\end{table}
\end{landscape}