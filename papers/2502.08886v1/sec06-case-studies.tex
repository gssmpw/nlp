\section{Case Study}
\label{sec6}
%
This section discusses three case studies: AttackGen~\citet{Adams_2024}, NVISOsecurity~\citet{Raman_2024}, and ChatIoT~\citet{dong2024chatiot} along with their potential implementation for IoT security.
AttackGen, an LLM-based Incident Response Plan Generator, shows potential in generating IoT-related incident response plans.
NVISOsecurity, using AutoGen~\citet{wu2023autogen} as its backend and Caldera agents as executors, demonstrates potential through its performance in executing complex tasks with pre-defined prompts.
ChatIoT is an LLM-based assistant designed to facilitate IoT security and threat intelligence by leveraging the versatile property of retrieval-augmented generation (RAG) illustrating a promising new direction in integrating the advanced language understanding and reasoning capabilities of LLM with fast-evolving IoT security information.
We provide details on the case studies and their potential implementation for IoT security in the following sections.
%
\subsection{AttackGen}
%
\begin{comment}
\begin{figure}[h!]
    \centering
    \includegraphics[width=\columnwidth]{images/AttackGen.png}
    \caption{a) Default implementation of AttackGen b) Modified Prompt case study on AttackGen}
    \label{fig:attackGen}
\end{figure}
\end{comment}
%
A case study on AttackGen~\citet{Adams_2024} was conducted to evaluate its potential to implement LLM in the mitigation of threat intelligence programs.
The inputs to AttackGen are the LLM model, industry, and the size of a hypothetical organization.
To assess AttackGen’s potential in creating incident response plans for IoT systems, the following modifications were made.
Firstly, the MITRE ATT\&CK Enterprise matrix was replaced with the ICS matrix to support the generation of incident response scenarios tailored to industrial and critical infrastructure environments, as the closest matrix for IoT.
The prompt template was then modified to focus on incident response for IoT.
Subsequently, the assistant module was specified to refine and identify critical IoT devices within the organization. 
After the modifications, AttackGen was used to generate an incident response plan focused on IoT with the GPT-4o model, targeting the energy sector with the Dragonfly group~\citet{Dragonfly}.

In this case study, human experts evaluated the relevance, clarity, and specificity of the generated plan.
Relevance was measured by how much the plan related to the Mitigation Technique for the specific threat group and business sector.
Clarity was measured by the plan's ability to create a test plan that human testers could follow, with detailed steps for testing Mitigation Techniques.
Specificity was measured by the plan's ability to specify vulnerable devices within the system, including device type, model, brand, or system architecture.
An extra prompt: ``Can you specify this in the context of a possible IoT-powered PLC that is critical to the company and is connected to the Internet?'' was added to the AttackGen Assistant module in order to generate a more refined plan.

AttackGen generated coherent plans from both the original and edited prompts without additional domain-specific IoT training.
Both plans focused on ICS Mitigation Techniques for the threat group, without mentioning specific attack vectors, for example, a vulnerable HVAC system or compromised machine.
The generated response provides an overview of potential attacks by the Dragonfly group through a theoretical supply chain compromise, indicating known attack methods.
Although relevant for incident response testing, both plans lacked focus on the specified energy domain.
Neither plan provided clear and specific instructions to test possible vulnerabilities. The response mentioned steps, triggers, and responses testing without elaborating on domain-specific techniques.
Both plans received scores of 3 out of 5, as neither clearly showed specific steps and instructions for humans to follow.

The refined response generated a plan more related to the IoT context, although it did not mention specific devices or system architecture.
The original plan as provided in Figure~\ref{fig:attackgen-default} focused on ICS Mitigation Techniques related to Dragonfly without mentioning IoT or specific devices.
The refined plan in Figure~\ref{fig:attackgen-iot}, using the ICS Mitigation Techniques as a base, included IoT.
This shows that the added prompt helped to make the plan more specific.
Without domain-specific training, the LLM model could only create generic incident response plans as a starting point for further refinements.
Even with details of the possible attacker and added focus on IoT, the pre-trained GPT-4o could not generate a tailored incident plan for an IoT context.
This shows the potential for improvement and research to enhance the specificity of the generated content, demonstrating the potential for LLMs to be used as tools in the creation of threat intelligence programs for attack mitigation.
%
\subsection{NVISOsecurity Cyber Security Agent}
%
\begin{comment}
\begin{figure}[h!]
    \centering
    \includegraphics[width=\columnwidth]{images/NVISOSecurity.png}
    \caption{Illustration of NVISOSecurity Implementation}
    \label{fig:nviso}
\end{figure}
\end{comment}
%
In this case study, we focus on NVISOsecurity's ability to execute commands to list the privileges available for the Caldera agent.
The tool operates by specifying a series of prompts (actions) to the \texttt{task\_coordinator} agent, which then transmits them to the \texttt{caldera\_agent}.
An environment for Caldera was first created with two default agents deployed to simulate communication through TCP (Manx) and HTTPS contacts (Sandcat).
These agents were kept alive while a pair of LLMs were started.

Using the predefined commands, \texttt{HELLO\_CALDERA}, \texttt{DETECT\_} \texttt{AGENT\_PRIVILEGES}, and \texttt{TTP\_REPORT\_TO\_TECHNIQUES} were executed to gather necessary information.
In \texttt{HELLO\_CALDERA}, the LLM pair could access PowerShell or the terminal and run a command to display a text box with a string.
This functionality serves as a simple prototype for further commands that the worker agent could execute.
\texttt{COLLECT\_CALDERA\_INFO} tasked the LLM pair with collecting user privilege information for the Caldera process, which runs at an administrative level.
This allowed the worker agent to execute a command to view the user privileges of the Caldera agent process.
\texttt{TTP\_REPORT\_TO\_TECHNIQUES} download a file from Microsoft, change its format, and identify MITRE techniques.
The worker agent's capabilities include running complex commands to access storage, identify, and format MITRE techniques within documents.

NVISOsecurity demonstrates the potential of LLMs to protect IoT systems through exploit protection by accessing administrative-level commands and obtaining privileged information.
In the IoT context, this means that connected devices could be continuously monitored and protected from anomalous processes.
By adding a custom command to NVISOsecurity and setting up the Caldera OT plugin as the agent, the LLM could block the execution of commands at the administrative level.
This suggests that NVISOsecurity could serve as a semi-autonomous agent, continuously observing and protecting the system from malicious processes.
Additionally, the \texttt{TTP\_REPORT\_TO\_TECHNIQUES} command shows that LLMs can identify specific items, such as MITRE techniques, from text documents.
This indicates the potential for LLMs to act as report generators, taking logs from the worker agent to prevent anomalous processes in an IoT context.

\subsection{ChatIoT}
This case study focuses on the capability of ChatIoT to provide reliable, relevant and technical answers to different types of users about IoT security~\citet{dong2024chatiot}.
This IoT security and threat assistant is built upon RAG, which retrieves external IoT security and threat data and feeds them into LLM to improve the quality of answers.
At a high level, ChatIoT integrates multiple datasets from different sources, including IoT vulnerabilities and exploits, MITRE ATT\&CK TTPs, threat reports, and cyber security labels of IoT devices.
These datasets are pre-processed into the system, and during the service, i.e., when a user submits a query, only the relevant data are retrieved into LLM to synthesize the final results. 
Moreover, Dong et al. proposed a data processing toolkit to convert datasets of various formats into documents for LLM processing and user role-specific prompts to dynamically retrieve data and generate answers aligned with users' expertise levels.

We rely on LLM-as-judges to evaluate the reliability, relevance, technicality, and user-friendliness of the answers.
Reliability measures the trustworthiness of each answer, relevance assesses how well the answer addresses the specific question and meets the user’s needs, technicality is used to judge the appropriateness and precision of technical language, including IoT research, standards, protocols, and relevant technical aspects, and user-friendliness determines how easy the answer is to comprehend, focusing on clarity for the user’s role and background.
An extra prompt: "\textbf{\textit{Instructions}}: \textit{1) Criteria: The descriptions about Reliability, Relevance, Technical, and User-friendliness.
2) Score: \romannumeral1) Provide a score for each answer across the metrics above. Scores should range from 0 to 5, with 5 being the highest and 0 being the lowest; \romannumeral2) Scores should reflect how well each answer meets the criteria, particularly in alignment with the user role's background and needs.
3) Output Format: Present a table that includes the names of all answers and their scores for each metric. You can score differently for different metrics.}"
was added to guide the LLM-based judge for evaluation.

IoT security is constantly evolving with new vulnerabilities, exploits, and security protocols.
ChatIoT automatically extracts the latest information from different sources to aid in LLM's understanding and reasoning.
At the same time, retraining or fine-tuning LLM is costly, resource intensive, and LLMs become out-of-date quickly.
ChatIoT provides a versatile and effective approach to combining LLM's capabilities with up-to-date IoT security and threat intelligence. 
In conclusion, ChatIoT shows the potential of using large language models to effectively facilitate IoT security assistance to various key users of IoT ecosystems in an understandable and actionable manner to provide better IoT security guarantees.