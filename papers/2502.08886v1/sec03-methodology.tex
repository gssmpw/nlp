\section{Survey Methodology}
\label{sec:search-method}
%
Our research methodology focuses on gathering papers from conferences, journals, workshops, and publications centered on or related to our investigation on the application of GenAI for IoT security. Our search process includes several different search methods, databases, and search engines to ensure that we collect as much relevant work as possible.
%
\subsection{Search Methods}
%
We used a structured approach in our literature search to have a comprehensive coverage and relevance of our survey on the intersection of GenAI and IoT security.

\smallskip
\textbf{OWASP Framework Insights:} Our search strategy was enrched by the inclusion of keywords from the OWASP IoT Top 10 and GenAI-related terms, such as ``weak passwords + IoT + Large Language Model''.
This approach uncovers research addressing the security challenges identified by OWASP and areas yet to be explored by GenAI solutions.
By analyzing the findings from these searches, we evaluated the potential role that GenAI could play in enhancing IoT security.

\smallskip
\textbf{MITRE ATT\&CK Framework Integration:} We also incorporated the MITRE ATT\&CK framework, focusing on tactics and techniques pertinent to Industrial Control Systems (ICS).
The application of ICS matrix from this framework provided a structured approach for identifying and analyzing threats specific to the Industrial Internet of Things (IIoT) sector.
Keywords such as ``hardcoded credentials + IoT + LLMs'' were employed to refine our search, ensuring a focused examination of the literature.
%
\subsection{Sources}
%
As part of the compilation of our sources for the application of GenAI in IoT security, we systematically gathered research from various conferences and journals, enhanced by contributions from leading academics and expanded searches in public repositories.
During the selection process, we sought to include works that have made significant contributions to cyber security and IoT, which have been subjected to rigorous peer reviews and are relevant to our study.

\smallskip
\textbf{Academic Publications: } We prioritized the sourcing of conferences and journals well known for their contributions to cyber security and IoT.
This included key venues such as IEEE Transactions on Dependable and Secure Computing (TDSC), Transactions on Information Forensics and Security (TIFS), International Journal of Critical Infrastructure Protection (IJCIP), Transactions on the Internet of Things (TIOT), IoT Journal, Computers \& Security, and ACM Transactions on Privacy and Security (TOPS).
Each source was selected for its relevance, rigorous peer review, and ability to provide the most recent and impactful research findings.

\smallskip
\textbf{Other Sources: } Considering all the recent research on GenAI, there are likely many studies yet to be accounted for in publications.
As such, papers available on arXiv are considered as a possible source of contributions if the papers have relevant implementations and results.
We have also looked into possible GitHub repositories to account for related works.
This allows us to include works that have not yet been published as part of the possible application of GanAI in improving IoT security.