%% 
%% Copyright 2007-2020 Elsevier Ltd
%% 
%% This file is part of the 'Elsarticle Bundle'.
%% ---------------------------------------------
%% 
%% It may be distributed under the conditions of the LaTeX Project Public
%% License, either version 1.2 of this license or (at your option) any
%% later version.  The latest version of this license is in
%%    http://www.latex-project.org/lppl.txt
%% and version 1.2 or later is part of all distributions of LaTeX
%% version 1999/12/01 or later.
%% 
%% The list of all files belonging to the 'Elsarticle Bundle' is
%% given in the file `manifest.txt'.
%% 
%% Template article for Elsevier's document class `elsarticle'
%% with harvard style bibliographic references

\newcommand{\DongYe}[1]{\textcolor{green}{#1}}
\newcommand{\YanLin}[1]{\textcolor{blue}{#1}}
\newcommand{\Xiaodong}[1]{\textcolor{gray}{#1}}
\newcommand{\ToRevise}[1]{\textcolor{red}{#1}}
\newcommand{\ToDo}[1]{\textcolor{purple}{#1}}

%\documentclass[preprint,12pt]{elsarticle}
\documentclass[final,5p,times,twocolumn]{elsarticle}

%\makeatletter
%\def\ps@pprintTitle{%
%  \let\@oddhead\@empty
%  \let\@evenhead\@empty
%  \let\@oddfoot\@empty
%  \let\@evenfoot\@oddfoot
%}
%\makeatother

\makeatletter
\def\ps@pprintTitle{%
  \let\@oddhead\@empty
  \let\@evenhead\@empty
  \def\@oddfoot{\reset@font\hfil\thepage\hfil}
  \let\@evenfoot\@oddfoot
}
\makeatother

%% Use the option review to obtain double line spacing
%% \documentclass[preprint,review,12pt]{elsarticle}

%% Use the options 1p,twocolumn; 3p; 3p,twocolumn; 5p; or 5p,twocolumn
%% for a journal layout:
%% \documentclass[final,1p,times]{elsarticle}
%% \documentclass[final,1p,times,twocolumn]{elsarticle}
%% \documentclass[final,3p,times]{elsarticle}
%% \documentclass[final,3p,times,twocolumn]{elsarticle}
%% \documentclass[final,5p,times]{elsarticle}
%% \documentclass[final,5p,times,twocolumn]{elsarticle}

\usepackage[inkscapelatex=false]{svg}
\usepackage[flushleft]{threeparttable}
\usepackage{verbatim}

\usepackage{graphics}
\usepackage{lscape}
\newcommand*\rot{\rotatebox{-90}}
\newcommand*\OK{\ding{51}}
\usepackage{tikz}
\newcommand*\emptycirc[1][1ex]{\tikz\draw (0,0) circle (#1);} 
\newcommand*\halfcirc[1][1ex]{%
  \begin{tikzpicture}
  \draw[fill] (0,0)-- (90:#1) arc (90:270:#1) -- cycle ;
  \draw (0,0) circle (#1);
  \end{tikzpicture}}
\newcommand*\fullcirc[1][1ex]{\tikz\fill (0,0) circle (#1);}

\usepackage{pifont}
\newcommand{\cmark}{\ding{51}}
\newcommand{\xmark}{\ding{55}}

\usepackage{multirow}
\usepackage{hyperref}
\usepackage{enumitem}
\usetikzlibrary{shapes.geometric, positioning}
\usepackage{float}
\usepackage{makecell}

%% For including figures, graphicx.sty has been loaded in
%% elsarticle.cls. If you prefer to use the old commands
%% please give \usepackage{epsfig}

%% The amssymb package provides various useful mathematical symbols
\usepackage{amssymb}
%% The amsthm package provides extended theorem environments
%% \usepackage{amsthm}

%% The lineno packages adds line numbers. Start line numbering with
%% \begin{linenumbers}, end it with \end{linenumbers}. Or switch it on
%% for the whole article with \linenumbers.
%\usepackage{lineno}

%\journal{Computer \& Security}
%\journal{arXiv}

\begin{document}

\begin{frontmatter}

%% Title, authors and addresses

%% use the tnoteref command within \title for footnotes;
%% use the tnotetext command for theassociated footnote;
%% use the fnref command within \author or \address for footnotes;
%% use the fntext command for theassociated footnote;
%% use the corref command within \author for corresponding author footnotes;
%% use the cortext command for theassociated footnote;
%% use the ead command for the email address,
%% and the form \ead[url] for the home page:
%% \title{Title\tnoteref{label1}}
%% \tnotetext[label1]{}
%% \author{Name\corref{cor1}\fnref{label2}}
%% \ead{email address}
%% \ead[url]{home page}
%% \fntext[label2]{}
%% \cortext[cor1]{}
%% \affiliation{organization={},
%%             addressline={},
%%             city={},
%%             postcode={},
%%             state={},
%%             country={}}
%% \fntext[label3]{}

\title{Generative AI for Internet of Things Security: Challenges and Opportunities}

%% use optional labels to link authors explicitly to addresses:
%% \author[label1,label2]{}
%% \affiliation[label1]{organization={},
%%             addressline={},
%%             city={},
%%             postcode={},
%%             state={},
%%             country={}}
%%
%% \affiliation[label2]{organization={},
%%             addressline={},
%%             city={},
%%             postcode={},
%%             state={},
%%             country={}}

\author[a]{Yan Lin Aung\corref{cor}\fnref{eqc}}
\cortext[cor]{Corresponding author.}
\ead{yan\_lin\_aung@ieee.org}
\fntext[eqc]{Both authors contributed equally and ordered alphabetically.}
%
\author[a]{Ivan Christian\fnref{eqc}}
%
\author[b]{Ye Dong}
%
\author[a]{Xiaodong Ye}
%
\author[a]{Sudipta Chattopadhyay}
%
\author[a]{Jianying Zhou}
%
\affiliation[a]{
  organization={Singapore University of Technology and Design},    
  country={Singapore}
}
%
\affiliation[b]{
  organization={National University of Singapore},  
  country={Singapore}
}

\begin{abstract}
As Generative AI (GenAI) continues to gain prominence and utility across various sectors, their integration into the realm of Internet of Things (IoT) security evolves rapidly.
This work delves into an examination of the state-of-the-art literature and practical applications on how GenAI could improve and be applied in the security landscape of IoT.
Our investigation aims to map the current state of GenAI implementation within IoT security, exploring their potential to fortify security measures further.
Through the compilation, synthesis, and analysis of the latest advancements in GenAI technologies applied to IoT, this paper not only introduces fresh insights into the field, but also lays the groundwork for future research directions.
It explains the prevailing challenges within IoT security, discusses the effectiveness of GenAI in addressing these issues, and identifies significant research gaps through MITRE Mitigations.
Accompanied with three case studies, we provide a comprehensive overview of the progress and future prospects of GenAI applications in IoT security.
This study serves as a foundational resource to improve IoT security through the innovative application of GenAI, thus contributing to the broader discourse on IoT security and technology integration.
\end{abstract}

%%Graphical abstract
%\begin{graphicalabstract}
%\includegraphics{grabs}
%\end{graphicalabstract}

%%Research highlights
%\begin{highlights}
%\item Research highlight 1
%\item Research highlight 2
%\end{highlights}

\begin{keyword}
%% keywords here, in the form: keyword \sep keyword
Generative AI for Cyber Security \sep Large Language Models for Cyber Security \sep Artificial Intelligence for Cyber Security \sep Internet of Things Security \sep MITRE ATT\&CK ICS Mitigations

%% PACS codes here, in the form: \PACS code \sep code

%% MSC codes here, in the form: \MSC code \sep code
%% or \MSC[2008] code \sep code (2000 is the default)

\end{keyword}

\end{frontmatter}

%\linenumbers

%% main text
%\section{}
%\label{}

\input sec01-intro
\input sec02-background
\input sec03-methodology
\input sec04-findings
\input sec06-case-studies
\input sec07-future-works
\input sec08-conclusions
\input ack
%
%% The Appendices part is started with the command \appendix;
%% appendix sections are then done as normal sections
%
%\appendix
%\section{Example Responses/Outputs for AttackGen Case Study}
%\input attackgen-case-study
%
%
%% \section{}
%% \label{}

%% For citations use: 
%%       \citet{<label>} ==> Jones et al. [21]
%%       \citep{<label>} ==> [21]
%%

%% If you have bibdatabase file and want bibtex to generate the
%% bibitems, please use
%%
\bibliographystyle{elsarticle-num-names} 
\bibliography{references}
%
\appendix
%
\section{Evaluation of State-of-the-Art Works}
%
We discussed each state-of-the-art work according to the six capabilities as described in Section~\ref{sec:detection}.
Table~\ref{tab:potentials-details} provides justifications and evaluation results for 33 state-of-the-art works.
%
\input tab-potential-impact-details
%
\onecolumn
\section{Example Responses/Outputs for AttackGen Case Study}
%
Figure~\ref{fig:attackgen-default} provides the incident response plan generated with AttackGen~\citet{Adams_2024} without any modifications whereas Figure~\ref{fig:attackgen-iot} shows the incident response plan with IoT-specific modifications.
%
\input attackgen-case-study
%

%% else use the following coding to input the bibitems directly in the
%% TeX file.

%\begin{thebibliography}{00}

%% \bibitem[Author(year)]{label}
%% Text of bibliographic item

%\bibitem[ ()]{}

%\end{thebibliography}
\end{document}

\endinput
%%
%% End of file `elsarticle-template-num-names.tex'.
