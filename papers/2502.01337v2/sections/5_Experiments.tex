\begin{table*}[ht]
\centering
\caption{Performance comparison of GMRES with different methods on Poisson equation with different mesh type, showing time and iterations at different tolerances.}
\vspace{5pt}
\label{table:poisson}
\renewcommand\arraystretch{1.0}
\begin{sc}
    \renewcommand{\multirowsetup}{\centering}
    \resizebox{\linewidth}{!}{
    \begin{tabular}{l|cccc|cccc|cccc}
        \toprule
        Mesh Type & \multicolumn{4}{c}{Grid=512} & \multicolumn{4}{c}{Grid=32*32} & \multicolumn{4}{c}{Irregular} \\
        \cmidrule(lr){1-1} \cmidrule(lr){2-5} \cmidrule(lr){6-9} \cmidrule(lr){10-13} \multirow{2}{*}{Method} & Time (s) & \multicolumn{3}{c}{Iteration} & Time (s) & \multicolumn{3}{c}{Iteration} & Time (s) & \multicolumn{3}{c}{Iteration} \\
        \cmidrule(lr){2-2} \cmidrule(lr){3-5} \cmidrule(lr){6-6} \cmidrule(lr){7-9} \cmidrule(lr){10-10} \cmidrule(lr){11-13} & 1E-10 & 1E-10 & 1E-6 & 1E-4 & 1E-10 & 1E-10 & 1E-6 & 1E-4 & 1E-10 & 1E-10 & 1E-6 & 1E-4 \\
        \midrule
        Jacobi & 3.719 & 513 & 513 & 510 & 2.7337 & 113 & 84 & 62 & 0.455 & 153 & 107 & 73 \\
        Gauss Seidel & 3.536 & 502 & 495 & 494 & 1.7569 & 81 & 68 & 57 & 0.379 & 130 & 99 & 76 \\
        SOR & 3.452 & 502 & 495 & 494 & 2.2438 & 81 & 68 & 57 & 0.364 & 130 & 99 & 76\\
        \midrule
        MLP & 2.359 & 513 & 258 & 256 & 0.6295 & 88 & 41 & 21 & \underline{0.275} & \underline{115} & \underline{62} & \underline{17}\\
        UNet & 2.147 & 494 & 481 & 473 & 12.4477 & 1025 & 1023 & 1020 & / & / & / & /\\
        FNO & 1.785 & 403 & 349 & 318 & 0.8354 & 93 & 40 & 21 & 0.342 & 140 & 82 & 44\\
        Transolver & 1.793 & 472 & \underline{257} & 257 & \underline{0.1380} & 87 & 39 & 20 & 0.334 & 142 & 86 & 50 \\
        M2NO & \underline{1.543} & \underline{307} & 302 & \underline{199} & 0.4780 & \underline{69} & \underline{35} & \underline{18} & / & / & / & / \\
        \midrule
        \textbf{NPO} & \textbf{0.623} & \textbf{184} & \textbf{134} & \textbf{93} & \textbf{0.0751} & \textbf{34} & \textbf{19} & \textbf{10} & \textbf{0.162} & \textbf{82} & \textbf{60} & \textbf{13}\\
        \bottomrule
    \end{tabular}}
\end{sc}
\end{table*}


\section{Experiments}

This section evaluates our Neural Preconditioning Operator (NPO) across various PDEs, mesh resolutions, and solver settings. We first detail the overall experimental setup and baselines, then present the main results for Poisson, Diffusion, and Linear Elasticity problems. Finally, we examine model generalization, ablation studies, and hyperparameter sensitivity to demonstrate how NPO balances data, residual, and condition losses for robust, scalable performance.

\subsection{Experiment Setup}
\label{subsec:setup}
\subsubsection{General Setting}
We aim to learn a Neural Preconditioning Operator (NPO), \(\mathcal{M}_{\theta}\), that expedites Krylov-based solvers for discretized PDE systems. Each instance in our dataset consists of a system matrix \(A_{i}\), derived from a finite element or finite difference discretization, and multiple recorded solution states \(\bigl(\mathbf{x}_{i,k}, \mathbf{b}_{i}, \mathbf{r}_{i,k}\bigr)\). Here, \(\mathbf{x}_{i,k}\) and \(\mathbf{r}_{i,k}\) represent the partial solution and residual at iteration \(k\) of a baseline solver (e.g., GMRES with AMG). 

The learning objective combines two components: \emph{Data Loss}, which leverages \(\mathbf{x}_{i,k}\) and \(\mathbf{b}_{i}\) to align the network’s output with observed solution patterns; \emph{Residual Loss}, inspired by \eqref{eq:residual_loss}, which nudges \(A_{i}\,\mathcal{M}_{\theta}(A_{i})\) toward the identity operator on residual vectors \(\mathbf{r}_{i,k}\). By balancing these two losses, \(\mathcal{M}_{\theta}\) learns to approximate \(A^{-1}\) while reflecting the intermediate dynamics of the iterative solution process.

\subsubsection{Dataset Generation}
To build our dataset, we discretize the governing PDEs (using finite elements or finite differences) to obtain matrices \(A \in \mathbb{R}^{n\times n}\). We then sample right-hand sides \(\mathbf{b}\) from a Gaussian Random Field (GRF) \(\phi(x)\), mapped onto the same mesh as \(A\). For each system \(A\mathbf{x} = \mathbf{b}\), we run a baseline Krylov solver (CG or GMRES) preconditioned by Algebraic Multigrid (AMG), with a convergence tolerance of \(10^{-10}\) and a cap of 100 iterations. At each step, we record the current partial solution \(\mathbf{x}_{k}\) and residual \(\mathbf{r}_{k} = \mathbf{b} - A\,\mathbf{x}_{k}\). These \(\{(A, \mathbf{b}, \mathbf{x}_{k}, \mathbf{r}_{k})\}\) samples form a comprehensive dataset capturing diverse solution states and residual behaviors, which we use to train \(\mathcal{M}_{\theta}\).

We consider three PDE systems: the Poisson equation, the diffusion equation, and the linear elasticity equation, each discretized to generate our dataset.

\textbf{Poisson Equation.} The Poisson equation is defined as:
\begin{equation}
    -\nabla \cdot (\nabla u) = f \quad \text{in} \, D,
\end{equation}
where \(u\) is the unknown solution and \(f\) is a given source term. It models potential fields such as electrostatics and steady-state heat conduction.

\textbf{Diffusion Equation.} The diffusion equation models the spread of substances over time:
\begin{equation}
    \frac{\partial u}{\partial t} = \nabla \cdot (D \nabla u),
\end{equation}
where \(D\) is the diffusion coefficient. 

\textbf{Linear Elasticity.} The linear elasticity equations describe the deformation of elastic materials under applied forces:
\begin{equation}
    \nabla \cdot \sigma = \mathbf{f}, \quad \sigma = \lambda (\nabla \cdot \mathbf{u}) I + \mu (\nabla \mathbf{u} + \nabla \mathbf{u}^\top),
\end{equation}
where \(\mathbf{u}\) is the displacement field, \(\sigma\) is the stress tensor, and \(\lambda, \mu\) are Lamé parameters. These equations are critical for structural mechanics simulations.

\subsubsection{Baselines}
\label{subsec:baselines}

Our evaluation framework encompasses three methodological categories: 
(i) \textbf{classical iterative methods} Jacobi, Gauss-Seidel, and SOR; 
(ii) \textbf{data-driven architectures} MLP \cite{86:mlp} and U-Net \cite{15:UNet}; (iii) \textbf{operator learning approaches} including Fourier neural operators (FNO \cite{21:fno}), 
transformer-based solvers (Transolver \cite{24:Transolver}), 
and multigrid-enhanced operators (M2NO \cite{24:M2NO}). 
This hierarchy spans from foundational numerical analysis to modern deep learning paradigms, ensuring rigorous comparison across methodological generations.


\begin{table*}[t]
\centering
\fontsize{11pt}{11pt}\selectfont
\begin{tabular}{lllllllllllll}
\toprule
\multicolumn{1}{c}{\textbf{task}} & \multicolumn{2}{c}{\textbf{Mir}} & \multicolumn{2}{c}{\textbf{Lai}} & \multicolumn{2}{c}{\textbf{Ziegen.}} & \multicolumn{2}{c}{\textbf{Cao}} & \multicolumn{2}{c}{\textbf{Alva-Man.}} & \multicolumn{1}{c}{\textbf{avg.}} & \textbf{\begin{tabular}[c]{@{}l@{}}avg.\\ rank\end{tabular}} \\
\multicolumn{1}{c}{\textbf{metrics}} & \multicolumn{1}{c}{\textbf{cor.}} & \multicolumn{1}{c}{\textbf{p-v.}} & \multicolumn{1}{c}{\textbf{cor.}} & \multicolumn{1}{c}{\textbf{p-v.}} & \multicolumn{1}{c}{\textbf{cor.}} & \multicolumn{1}{c}{\textbf{p-v.}} & \multicolumn{1}{c}{\textbf{cor.}} & \multicolumn{1}{c}{\textbf{p-v.}} & \multicolumn{1}{c}{\textbf{cor.}} & \multicolumn{1}{c}{\textbf{p-v.}} &  &  \\ \midrule
\textbf{S-Bleu} & 0.50 & 0.0 & 0.47 & 0.0 & 0.59 & 0.0 & 0.58 & 0.0 & 0.68 & 0.0 & 0.57 & 5.8 \\
\textbf{R-Bleu} & -- & -- & 0.27 & 0.0 & 0.30 & 0.0 & -- & -- & -- & -- & - &  \\
\textbf{S-Meteor} & 0.49 & 0.0 & 0.48 & 0.0 & 0.61 & 0.0 & 0.57 & 0.0 & 0.64 & 0.0 & 0.56 & 6.1 \\
\textbf{R-Meteor} & -- & -- & 0.34 & 0.0 & 0.26 & 0.0 & -- & -- & -- & -- & - &  \\
\textbf{S-Bertscore} & \textbf{0.53} & 0.0 & {\ul 0.80} & 0.0 & \textbf{0.70} & 0.0 & {\ul 0.66} & 0.0 & {\ul0.78} & 0.0 & \textbf{0.69} & \textbf{1.7} \\
\textbf{R-Bertscore} & -- & -- & 0.51 & 0.0 & 0.38 & 0.0 & -- & -- & -- & -- & - &  \\
\textbf{S-Bleurt} & {\ul 0.52} & 0.0 & {\ul 0.80} & 0.0 & 0.60 & 0.0 & \textbf{0.70} & 0.0 & \textbf{0.80} & 0.0 & {\ul 0.68} & {\ul 2.3} \\
\textbf{R-Bleurt} & -- & -- & 0.59 & 0.0 & -0.05 & 0.13 & -- & -- & -- & -- & - &  \\
\textbf{S-Cosine} & 0.51 & 0.0 & 0.69 & 0.0 & {\ul 0.62} & 0.0 & 0.61 & 0.0 & 0.65 & 0.0 & 0.62 & 4.4 \\
\textbf{R-Cosine} & -- & -- & 0.40 & 0.0 & 0.29 & 0.0 & -- & -- & -- & -- & - & \\ \midrule
\textbf{QuestEval} & 0.23 & 0.0 & 0.25 & 0.0 & 0.49 & 0.0 & 0.47 & 0.0 & 0.62 & 0.0 & 0.41 & 9.0 \\
\textbf{LLaMa3} & 0.36 & 0.0 & \textbf{0.84} & 0.0 & {\ul{0.62}} & 0.0 & 0.61 & 0.0 &  0.76 & 0.0 & 0.64 & 3.6 \\
\textbf{our (3b)} & 0.49 & 0.0 & 0.73 & 0.0 & 0.54 & 0.0 & 0.53 & 0.0 & 0.7 & 0.0 & 0.60 & 5.8 \\
\textbf{our (8b)} & 0.48 & 0.0 & 0.73 & 0.0 & 0.52 & 0.0 & 0.53 & 0.0 & 0.7 & 0.0 & 0.59 & 6.3 \\  \bottomrule
\end{tabular}
\caption{Pearson correlation on human evaluation on system output. `R-': reference-based. `S-': source-based.}
\label{tab:sys}
\end{table*}



\begin{table}%[]
\centering
\fontsize{11pt}{11pt}\selectfont
\begin{tabular}{llllll}
\toprule
\multicolumn{1}{c}{\textbf{task}} & \multicolumn{1}{c}{\textbf{Lai}} & \multicolumn{1}{c}{\textbf{Zei.}} & \multicolumn{1}{c}{\textbf{Scia.}} & \textbf{} & \textbf{} \\ 
\multicolumn{1}{c}{\textbf{metrics}} & \multicolumn{1}{c}{\textbf{cor.}} & \multicolumn{1}{c}{\textbf{cor.}} & \multicolumn{1}{c}{\textbf{cor.}} & \textbf{avg.} & \textbf{\begin{tabular}[c]{@{}l@{}}avg.\\ rank\end{tabular}} \\ \midrule
\textbf{S-Bleu} & 0.40 & 0.40 & 0.19* & 0.33 & 7.67 \\
\textbf{S-Meteor} & 0.41 & 0.42 & 0.16* & 0.33 & 7.33 \\
\textbf{S-BertS.} & {\ul0.58} & 0.47 & 0.31 & 0.45 & 3.67 \\
\textbf{S-Bleurt} & 0.45 & {\ul 0.54} & {\ul 0.37} & 0.45 & {\ul 3.33} \\
\textbf{S-Cosine} & 0.56 & 0.52 & 0.3 & {\ul 0.46} & {\ul 3.33} \\ \midrule
\textbf{QuestE.} & 0.27 & 0.35 & 0.06* & 0.23 & 9.00 \\
\textbf{LlaMA3} & \textbf{0.6} & \textbf{0.67} & \textbf{0.51} & \textbf{0.59} & \textbf{1.0} \\
\textbf{Our (3b)} & 0.51 & 0.49 & 0.23* & 0.39 & 4.83 \\
\textbf{Our (8b)} & 0.52 & 0.49 & 0.22* & 0.43 & 4.83 \\ \bottomrule
\end{tabular}
\caption{Pearson correlation on human ratings on reference output. *not significant; we cannot reject the null hypothesis of zero correlation}
\label{tab:ref}
\end{table}


\begin{table*}%[]
\centering
\fontsize{11pt}{11pt}\selectfont
\begin{tabular}{lllllllll}
\toprule
\textbf{task} & \multicolumn{1}{c}{\textbf{ALL}} & \multicolumn{1}{c}{\textbf{sentiment}} & \multicolumn{1}{c}{\textbf{detoxify}} & \multicolumn{1}{c}{\textbf{catchy}} & \multicolumn{1}{c}{\textbf{polite}} & \multicolumn{1}{c}{\textbf{persuasive}} & \multicolumn{1}{c}{\textbf{formal}} & \textbf{\begin{tabular}[c]{@{}l@{}}avg. \\ rank\end{tabular}} \\
\textbf{metrics} & \multicolumn{1}{c}{\textbf{cor.}} & \multicolumn{1}{c}{\textbf{cor.}} & \multicolumn{1}{c}{\textbf{cor.}} & \multicolumn{1}{c}{\textbf{cor.}} & \multicolumn{1}{c}{\textbf{cor.}} & \multicolumn{1}{c}{\textbf{cor.}} & \multicolumn{1}{c}{\textbf{cor.}} &  \\ \midrule
\textbf{S-Bleu} & -0.17 & -0.82 & -0.45 & -0.12* & -0.1* & -0.05 & -0.21 & 8.42 \\
\textbf{R-Bleu} & - & -0.5 & -0.45 &  &  &  &  &  \\
\textbf{S-Meteor} & -0.07* & -0.55 & -0.4 & -0.01* & 0.1* & -0.16 & -0.04* & 7.67 \\
\textbf{R-Meteor} & - & -0.17* & -0.39 & - & - & - & - & - \\
\textbf{S-BertScore} & 0.11 & -0.38 & -0.07* & -0.17* & 0.28 & 0.12 & 0.25 & 6.0 \\
\textbf{R-BertScore} & - & -0.02* & -0.21* & - & - & - & - & - \\
\textbf{S-Bleurt} & 0.29 & 0.05* & 0.45 & 0.06* & 0.29 & 0.23 & 0.46 & 4.2 \\
\textbf{R-Bleurt} & - &  0.21 & 0.38 & - & - & - & - & - \\
\textbf{S-Cosine} & 0.01* & -0.5 & -0.13* & -0.19* & 0.05* & -0.05* & 0.15* & 7.42 \\
\textbf{R-Cosine} & - & -0.11* & -0.16* & - & - & - & - & - \\ \midrule
\textbf{QuestEval} & 0.21 & {\ul{0.29}} & 0.23 & 0.37 & 0.19* & 0.35 & 0.14* & 4.67 \\
\textbf{LlaMA3} & \textbf{0.82} & \textbf{0.80} & \textbf{0.72} & \textbf{0.84} & \textbf{0.84} & \textbf{0.90} & \textbf{0.88} & \textbf{1.00} \\
\textbf{Our (3b)} & 0.47 & -0.11* & 0.37 & 0.61 & 0.53 & 0.54 & 0.66 & 3.5 \\
\textbf{Our (8b)} & {\ul{0.57}} & 0.09* & {\ul 0.49} & {\ul 0.72} & {\ul 0.64} & {\ul 0.62} & {\ul 0.67} & {\ul 2.17} \\ \bottomrule
\end{tabular}
\caption{Pearson correlation on human ratings on our constructed test set. 'R-': reference-based. 'S-': source-based. *not significant; we cannot reject the null hypothesis of zero correlation}
\label{tab:con}
\end{table*}

\section{Results}
We benchmark the different metrics on the different datasets using correlation to human judgement. For content preservation, we show results split on data with system output, reference output and our constructed test set: we show that the data source for evaluation leads to different conclusions on the metrics. In addition, we examine whether the metrics can rank style transfer systems similar to humans. On style strength, we likewise show correlations between human judgment and zero-shot evaluation approaches. When applicable, we summarize results by reporting the average correlation. And the average ranking of the metric per dataset (by ranking which metric obtains the highest correlation to human judgement per dataset). 

\subsection{Content preservation}
\paragraph{How do data sources affect the conclusion on best metric?}
The conclusions about the metrics' performance change radically depending on whether we use system output data, reference output, or our constructed test set. Ideally, a good metric correlates highly with humans on any data source. Ideally, for meta-evaluation, a metric should correlate consistently across all data sources, but the following shows that the correlations indicate different things, and the conclusion on the best metric should be drawn carefully.

Looking at the metrics correlations with humans on the data source with system output (Table~\ref{tab:sys}), we see a relatively high correlation for many of the metrics on many tasks. The overall best metrics are S-BertScore and S-BLEURT (avg+avg rank). We see no notable difference in our method of using the 3B or 8B model as the backbone.

Examining the average correlations based on data with reference output (Table~\ref{tab:ref}), now the zero-shoot prompting with LlaMA3 70B is the best-performing approach ($0.59$ avg). Tied for second place are source-based cosine embedding ($0.46$ avg), BLEURT ($0.45$ avg) and BertScore ($0.45$ avg). Our method follows on a 5. place: here, the 8b version (($0.43$ avg)) shows a bit stronger results than 3b ($0.39$ avg). The fact that the conclusions change, whether looking at reference or system output, confirms the observations made by \citet{scialom-etal-2021-questeval} on simplicity transfer.   

Now consider the results on our test set (Table~\ref{tab:con}): Several metrics show low or no correlation; we even see a significantly negative correlation for some metrics on ALL (BLEU) and for specific subparts of our test set for BLEU, Meteor, BertScore, Cosine. On the other end, LlaMA3 70B is again performing best, showing strong results ($0.82$ in ALL). The runner-up is now our 8B method, with a gap to the 3B version ($0.57$ vs $0.47$ in ALL). Note our method still shows zero correlation for the sentiment task. After, ranks BLEURT ($0.29$), QuestEval ($0.21$), BertScore ($0.11$), Cosine ($0.01$).  

On our test set, we find that some metrics that correlate relatively well on the other datasets, now exhibit low correlation. Hence, with our test set, we can now support the logical reasoning with data evidence: Evaluation of content preservation for style transfer needs to take the style shift into account. This conclusion could not be drawn using the existing data sources: We hypothesise that for the data with system-based output, successful output happens to be very similar to the source sentence and vice versa, and reference-based output might not contain server mistakes as they are gold references. Thus, none of the existing data sources tests the limits of the metrics.  


\paragraph{How do reference-based metrics compare to source-based ones?} Reference-based metrics show a lower correlation than the source-based counterpart for all metrics on both datasets with ratings on references (Table~\ref{tab:sys}). As discussed previously, reference-based metrics for style transfer have the drawback that many different good solutions on a rewrite might exist and not only one similar to a reference.


\paragraph{How well can the metrics rank the performance of style transfer methods?}
We compare the metrics' ability to judge the best style transfer methods w.r.t. the human annotations: Several of the data sources contain samples from different style transfer systems. In order to use metrics to assess the quality of the style transfer system, metrics should correctly find the best-performing system. Hence, we evaluate whether the metrics for content preservation provide the same system ranking as human evaluators. We take the mean of the score for every output on each system and the mean of the human annotations; we compare the systems using the Kendall's Tau correlation. 

We find only the evaluation using the dataset Mir, Lai, and Ziegen to result in significant correlations, probably because of sparsity in a number of system tests (App.~\ref{app:dataset}). Our method (8b) is the only metric providing a perfect ranking of the style transfer system on the Lai data, and Llama3 70B the only one on the Ziegen data. Results in App.~\ref{app:results}. 


\subsection{Style strength results}
%Evaluating style strengths is a challenging task. 
Llama3 70B shows better overall results than our method. However, our method scores higher than Llama3 70B on 2 out of 6 datasets, but it also exhibits zero correlation on one task (Table~\ref{tab:styleresults}).%More work i s needed on evaluating style strengths. 
 
\begin{table}%[]
\fontsize{11pt}{11pt}\selectfont
\begin{tabular}{lccc}
\toprule
\multicolumn{1}{c}{\textbf{}} & \textbf{LlaMA3} & \textbf{Our (3b)} & \textbf{Our (8b)} \\ \midrule
\textbf{Mir} & 0.46 & 0.54 & \textbf{0.57} \\
\textbf{Lai} & \textbf{0.57} & 0.18 & 0.19 \\
\textbf{Ziegen.} & 0.25 & 0.27 & \textbf{0.32} \\
\textbf{Alva-M.} & \textbf{0.59} & 0.03* & 0.02* \\
\textbf{Scialom} & \textbf{0.62} & 0.45 & 0.44 \\
\textbf{\begin{tabular}[c]{@{}l@{}}Our Test\end{tabular}} & \textbf{0.63} & 0.46 & 0.48 \\ \bottomrule
\end{tabular}
\caption{Style strength: Pearson correlation to human ratings. *not significant; we cannot reject the null hypothesis of zero corelation}
\label{tab:styleresults}
\end{table}

\subsection{Ablation}
We conduct several runs of the methods using LLMs with variations in instructions/prompts (App.~\ref{app:method}). We observe that the lower the correlation on a task, the higher the variation between the different runs. For our method, we only observe low variance between the runs.
None of the variations leads to different conclusions of the meta-evaluation. Results in App.~\ref{app:results}.

\subsection{Main Results}
\label{subsec:results}
Table~\ref{table:poisson} compares the performance of our Neural Preconditioning Operator (NPO) against various classical and neural methods for Poisson problems on a one dimensional uniform \(512\) grid, a two dimensional uniform \(32\times32\) grid, and an irregular mesh. Each method is integrated into GMRES, and we report both runtime (in seconds) and iteration counts for different tolerances. Table~\ref{table:results} further evaluates NPO on Diffusion and Linear Elasticity, also at a \(32\times32\) resolution, showing similar trends.

\textbf{Classical Methods.}
Jacobi, Gauss--Seidel, and SOR serve as baselines. While they eventually converge, their iteration counts remain high across all tests. For instance, at \(\mathrm{tol}=10^{-10}\) on the \(512\) mesh (Poisson), SOR requires 502 iterations and 3.452\,s, with Gauss--Seidel close behind at around 500 iterations. On the irregular mesh, they still demand up to 130 iterations (0.379\,s). Similarly, Table~\ref{table:results} shows that even for the smaller \(32\times32\) Diffusion problem, SOR and Gauss--Seidel need 139 iterations or more.

\textbf{Neural-Based Approaches.}
MLP, U-Net, FNO, Transolver, and M2NO all improve upon classical smoothers by reducing iteration counts or runtime. For example, FNO achieves 403 iterations in 1.785\,s at \(10^{-10}\) on the uniform \(512\) Poisson problem, while Transolver lowers it to 472 iterations in roughly the same time. M2NO cuts both time and iterations further. Table~\ref{table:results} confirms this pattern on Diffusion and Linear Elasticity: FNO and Transolver surpass classical methods, although they may still require over 100 iterations in certain cases.

\textbf{Neural Preconditioning Operator (NPO).}
Our proposed NPO demonstrates consistently fewer iterations and shorter runtime across all test settings. For instance, at \(\mathrm{tol}=10^{-10}\) on the \(512\) Poisson grid, NPO converges in 184 iterations and 0.623\,s, providing more than a twofold speedup over SOR or Gauss--Seidel. At a looser tolerance of \(10^{-4}\), NPO completes in just 93 iterations, whereas classical methods still require 300 or more. On the irregular mesh, NPO requires only 82 iterations in 0.162\,s at \(\mathrm{tol}=10^{-10}\). For Diffusion and Linear Elasticity (Table~\ref{table:results}, \(32\times32\)), NPO similarly outperforms both classical and neural rivals, converging in as few as 38 iterations for Diffusion and 31 for Linear Elasticity, with minimal total runtime.

\begin{figure}

\begin{tikzpicture}
\begin{axis}[
    xlabel=\textsc{Iteration},
    ylabel=\textsc{Relative Residuals},
    ylabel style={yshift=-0.5em},
    ymode=log,
    grid=major,
    width=14cm,
    height=6cm,
    xmin=0,
    legend style={at={(0.5,1.05)}, anchor=south, legend columns=-1, font=\small, draw=none},
]

    % Jacobi
    \addplot[color=vir0, mark=diamond] coordinates {
        (0, 1.0) (10, 0.8617032125984454) (20, 0.8458230932693831) (30, 0.8264356900969683) (40, 0.8046593444586116) (50, 0.7849200411262359) (60, 0.764532608404655) (70, 0.7491075102497443) (80, 0.730669610438604) (90, 0.7000198935288492) (100, 0.6793255554229471) (110, 0.6591562989962476) (120, 0.6376568032637498) (130, 0.6178034268333955) (140, 0.5972182831752685) (150, 0.56979371449403) (160, 0.5443535235296356) (170, 0.5165572381227819) (180, 0.48179747589896127) (190, 0.4481790779100302) (200, 0.42150754757401376) (210, 0.38288167455331756) (220, 0.3283251894963469) (230, 0.28233574048722837) (240, 0.2073404217485558) (250, 0.13349682599952534) (260, 0.07642216755177213) (270, 0.05405551489825935) (280, 0.04434881540226125) (290, 0.03852973350618292) (300, 0.03478191320399114) (310, 0.030405753554858708) (320, 0.02761011598991033) (330, 0.0255501349970738) (340, 0.02368400431142437) (350, 0.02057089695114903) (360, 0.017962173226875374) (370, 0.016357944578556563) (380, 0.015498536324660705) (390, 0.014865310840794527) (400, 0.013745399606699005) (410, 0.012658834880046337) (420, 0.011191505935175957) (430, 0.009389809358145065) (440, 0.008598123332980087) (450, 0.0074077725946775416) (460, 0.005691474030447884) (470, 0.003891477095081698) (480, 0.002601787005061453) (490, 0.001808043340415448) (491, 0.0017267174931723576) (492, 0.001613407107839085) (493, 0.001484681465117625) (494, 0.0013370453626555918) (495, 0.001185492226792755) (496, 0.001065382737558863) (497, 0.0009560349102341194) (498, 0.0008672069235879661) (499, 0.0007989010799369994) (500, 0.0007233934828814808) (501, 0.0006889710920541316) (502, 0.000634283001410715) (503, 0.0005718640401320616) (504, 0.0004887174879012805) (505, 0.00036335850941644315) (506, 0.00026675095700580463) (507, 0.00018114829753890845) (508, 0.0001414150033806168) (509, 0.00010812405676667209) (510, 9.472774592375564e-05) (511, 4.644841603349241e-05) (512, 1.0042496506701536e-10) 
    };
    \addlegendentry{Jacobi}
    
    % Gauss Seidel
    \addplot[color=vir1, mark=triangle] coordinates {
        (0, 1.0) (10, 0.9419617275661877) (20, 0.9240541787567774) (30, 0.8995001266856225) (40, 0.8746301582730358) (50, 0.8524288884395915) (60, 0.8310530363592776) (70, 0.8126607315344698) (80, 0.7864716512974249) (90, 0.7545858840564603) (100, 0.7327127301131534) (110, 0.7091468609601421) (120, 0.6853617377346405) (130, 0.6635374727937213) (140, 0.6347498552392201) (150, 0.6046882392250652) (160, 0.5753770423023588) (170, 0.5373087863680107) (180, 0.5013835940062829) (190, 0.4675908469623859) (200, 0.4249509256398726) (210, 0.3786558813516064) (220, 0.31766671996795226) (230, 0.23809164865980076) (240, 0.16614377679904924) (250, 0.10430176579222451) (260, 0.06686376175016114) (270, 0.04699511013231828) (280, 0.037147257115501216) (290, 0.03128492306740041) (300, 0.026504127942343855) (310, 0.02266624374387016) (320, 0.019032414410667057) (330, 0.017291957147836757) (340, 0.015999717444953696) (350, 0.015069416884284537) (360, 0.014413853509388836) (370, 0.013788111993964076) (380, 0.011534694663308424) (390, 0.009691062357088918) (400, 0.008449305419103841) (410, 0.0073172665663327335) (420, 0.006896321907878315) (430, 0.006341017881434075) (440, 0.004251515330708114) (450, 0.0020691630588748713) (460, 0.0008829263790903801) (470, 0.0004978960002697326) (480, 0.000356328918336287) (490, 0.00014617131794757436) (491, 8.950505010408427e-05) (492, 6.347633162165342e-05) (493, 4.247892089181589e-05) (494, 2.8169379745910093e-05) (495, 1.7921748791707997e-05) (496, 1.159679616123573e-05) (497, 6.650472027143783e-06) (498, 4.416870070380276e-06) (499, 2.7622328159840534e-06) (500, 1.580449206628693e-06) (501, 5.297359306557617e-07) (502, 1.0009807687459418e-07) (503, 1.5020055897995587e-08) (504, 4.423869002155035e-09) (505, 2.7972359755621495e-11) 
    };
    \addlegendentry{Gauss Seidel}
    
    % SOR
    \addplot[color=vir2, mark=star] coordinates {
        (0, 1.0) (10, 0.9419617275661879) (20, 0.9240541787567801) (30, 0.8995001266856302) (40, 0.87463015827305) (50, 0.852428888439612) (60, 0.8310530363593023) (70, 0.8126607315344948) (80, 0.7864716512974413) (90, 0.7545858840564611) (100, 0.7327127301131409) (110, 0.7091468609601151) (120, 0.6853617377345983) (130, 0.6635374727936666) (140, 0.634749855239151) (150, 0.604688239224982) (160, 0.5753770423022624) (170, 0.5373087863678999) (180, 0.5013835940061624) (190, 0.4675908469622656) (200, 0.4249509256397617) (210, 0.37865588135150824) (220, 0.31766671996787027) (230, 0.23809164865974003) (240, 0.16614377679899872) (250, 0.10430176579216685) (260, 0.06686376175008861) (270, 0.04699511013222841) (280, 0.0371472571153955) (290, 0.03128492306728149) (300, 0.026504127942211155) (310, 0.022666243743724907) (320, 0.019032414410508132) (330, 0.017291957147671677) (340, 0.0159997174447869) (350, 0.015069416884119731) (360, 0.014413853509228929) (370, 0.013788111993814217) (380, 0.011534694663199981) (390, 0.009691062357008588) (400, 0.008449305419041648) (410, 0.007317266566285207) (420, 0.006896321907836127) (430, 0.0063410178814000025) (440, 0.004251515330697043) (450, 0.002069163058875871) (460, 0.0008829263790968647) (470, 0.0004978960002782556) (480, 0.0003563289183433794) (490, 0.00014617131794844492) (491, 8.950505010198409e-05) (492, 6.34763316202682e-05) (493, 4.247892089073655e-05) (494, 2.81693797452052e-05) (495, 1.792174879123632e-05) (496, 1.1596796160916205e-05) (497, 6.650472026784186e-06) (498, 4.416870070163247e-06) (499, 2.762232815785411e-06) (500, 1.5804492065414308e-06) (501, 5.29735930645629e-07) (502, 1.000980768803949e-07) (503, 1.502005590354738e-08) (504, 4.423869003656227e-09) (505, 2.8592118933433238e-11) 
    };
    \addlegendentry{SOR}
    
    % MLP
    \addplot[color=vir3, mark=o] coordinates {
        (0, 1.0) (10, 0.006478528515589514) (20, 0.004472957377648451) (30, 0.0035696496264016678) (40, 0.00302859398272148) (50, 0.002651567565708821) (60, 0.0023599165710731333) (70, 0.0021294002855318877) (80, 0.0019389772266244438) (90, 0.001774682290772257) (100, 0.0016318636253923793) (110, 0.0015062073909356546) (120, 0.0013918982673832916) (130, 0.0012872829778096872) (140, 0.0011902530198988016) (150, 0.0010987548859472099) (160, 0.0010119263014122576) (170, 0.0009295757972282848) (180, 0.0008493673705253236) (190, 0.0007694569176512706) (200, 0.0006910839014340744) (210, 0.0006109184493391782) (220, 0.000526578676516599) (230, 0.00043725020961793957) (240, 0.00033652336113397047) (250, 0.0002024495295761604) (260, 1.2105626610983877e-07) (270, 6.304989972644088e-08) (280, 4.477086687290547e-08) (290, 3.781551741673961e-08) (300, 3.3618397970811944e-08) (310, 3.1083521899489074e-08) (320, 2.9236110190414188e-08) (330, 2.5622926256616787e-08) (340, 2.312381857407127e-08) (350, 2.1479803962685037e-08) (360, 2.0595054079548875e-08) (370, 1.9271640573928068e-08) (380, 1.7091171955714546e-08) (390, 1.4693028733078599e-08) (400, 1.3126911526993429e-08) (410, 1.20864377021389e-08) (420, 1.1235770186059795e-08) (430, 1.0097916250434476e-08) (440, 9.130725645758663e-09) (450, 7.001382953874065e-09) (460, 5.2430750992677145e-09) (470, 2.731534891057994e-09) (480, 9.47901418996539e-10) (490, 3.2847841431256934e-10) (500, 1.7390610461935753e-10) (501, 1.688878554525879e-10) (502, 1.6547423926128117e-10) (503, 1.6324919154655988e-10) (504, 1.603794336356707e-10) (505, 1.582920448818205e-10) (506, 1.5692572188758108e-10) (507, 1.5541437759183333e-10) (508, 1.531913685200446e-10) (509, 1.4776324735494435e-10) (510, 1.357402881967177e-10) (511, 1.0421128964142478e-10) (512, 1.5454012128612604e-11)  
    };
    \addlegendentry{MLP}
    
    % UNet
    \addplot[color=vir4, mark=square] coordinates {
        (0, 1.0) (10, 0.20987068243492948) (20, 0.1789714280141495) (30, 0.1360312958637759) (40, 0.12671180489474262) (50, 0.12213342336573228) (60, 0.10406408860105802) (70, 0.09738126690283455) (80, 0.09602748904971967) (90, 0.09039152271999154) (100, 0.08533781570338798) (110, 0.08394180932197401) (120, 0.08233358785924674) (130, 0.0800823523742912) (140, 0.07779115161176085) (150, 0.07639151731297043) (160, 0.07498133899449783) (170, 0.07420020998956083) (180, 0.07328176888535541) (190, 0.07221667688674273) (200, 0.06945063399782872) (210, 0.06854965067534728) (220, 0.06698588719279792) (230, 0.06550391830313694) (240, 0.06367406529014075) (250, 0.06286973411130088) (260, 0.06182142098572725) (270, 0.06094316692609111) (280, 0.05948790998272106) (290, 0.058283602272073486) (300, 0.05509523257240939) (310, 0.05372745658250891) (320, 0.04799001241894954) (330, 0.04679602360269621) (340, 0.04437638045589388) (350, 0.04109775296493586) (360, 0.03994146820039353) (370, 0.038495298003092884) (380, 0.03755602489567478) (390, 0.03578963335053881) (400, 0.0335000219505972) (410, 0.032516551295919134) (420, 0.014307139559427208) (430, 0.011434200834473293) (440, 0.009747828458090123) (450, 0.0044690646346497555) (460, 0.001363217520236999) (470, 0.00022013686993539497) (480, 6.357896065738972e-07) (489, 2.290390733974232e-09) (490, 1.2728558678777735e-09) (491, 5.832503034529717e-10) (492, 2.1390069181762254e-10) (493, 9.448112159260241e-11)  
    };
    \addlegendentry{UNet}
    
    % FNO
    \addplot[color=vir5, mark=x] coordinates {
        (0, 1.0) (10, 0.011730604492457964) (20, 0.009213665749090424) (30, 0.008935719253296295) (40, 0.008883718838718746) (50, 0.008832183290578518) (60, 0.008766019927134437) (70, 0.00818323373032284) (80, 0.0074261440463286674) (90, 0.007213294394461229) (100, 0.007171909516585396) (110, 0.007136444877588504) (120, 0.006914910949825328) (130, 0.006172441522542972) (140, 0.005799244134739346) (150, 0.005424950992920647) (160, 0.005254910928075411) (170, 0.005228571830080484) (180, 0.005224696685286327) (190, 0.00520848709350312) (200, 0.005028399196976466) (210, 0.0046091202338519365) (220, 0.004431201442220288) (230, 0.004344748623402378) (240, 0.004297647325397021) (250, 0.004294087498845729) (260, 0.004202281712686691) (270, 0.004115277635005418) (280, 0.00408101072631444) (290, 0.00342525546694761) (300, 0.001357229346521154) (310, 0.00034204131415973513) (320, 6.17123222475476e-05) (330, 1.3931036961002134e-05) (340, 3.133721203781303e-06) (350, 6.122537246031866e-07) (360, 1.1207963942771565e-07) (370, 1.761003084158037e-08) (380, 3.499498907288189e-09) (390, 6.533889145553497e-10) (396, 2.3053436322820832e-10) (397, 2.0434278589747513e-10) (398, 1.7940028513831587e-10) (399, 1.5767368692766794e-10) (400, 1.3261719150921442e-10) (401, 1.1268085967461527e-10) (402, 9.639075943217711e-11) 
    };
    \addlegendentry{FNO}
    
    % Transolver
    \addplot[color=blue, mark=pentagon] coordinates {
        (0, 1.0) (10, 0.009620116105752065) (20, 0.006642553764718631) (30, 0.0053012097150703565) (40, 0.0044977678789690665) (50, 0.003937793511855631) (60, 0.003504543460360238) (70, 0.003162279465077411) (80, 0.00287944580879148) (90, 0.0026354609442815354) (100, 0.0024234305781964424) (110, 0.0022368138318249616) (120, 0.0020670484464224573) (130, 0.0019116946191673422) (140, 0.0017675861245182536) (150, 0.0016317094638202147) (160, 0.001502777885146766) (170, 0.001380474303979282) (180, 0.001261354200973783) (190, 0.0011426823502089908) (200, 0.0010262876402775223) (210, 0.000907229642778926) (220, 0.0007819855550200552) (230, 0.0006493398839643633) (240, 0.0004997526304644904) (250, 0.00030059317719074486) (260, 4.5184691924434265e-09) (270, 2.3442959008810886e-09) (280, 1.660523356709187e-09) (290, 1.4015274054530073e-09) (300, 1.246063260229793e-09) (310, 1.1525346998388441e-09) (320, 1.084241268934601e-09) (330, 9.518725923860468e-10) (340, 8.590232121438264e-10) (350, 7.978529494937181e-10) (360, 7.648708441407299e-10) (370, 7.161775363417901e-10) (380, 6.356035534641201e-10) (390, 5.444267845233705e-10) (400, 4.850716738724034e-10) (410, 4.470756583873367e-10) (420, 4.1649971977949924e-10) (430, 3.7324677367209856e-10) (440, 3.360582750253797e-10) (450, 2.5727089477026975e-10) (460, 1.9390840265569493e-10) (466, 1.4312322691497687e-10) (467, 1.3707802560630945e-10) (468, 1.270397517564319e-10) (469, 1.1581370345940873e-10) (470, 1.0173869931377159e-10) (471, 9.026693721467517e-11) 
    };
    \addlegendentry{Transolver}
    
    % M2NO
    \addplot[color=blue, mark=square*] coordinates {
        (0, 1.0) (10, 0.016065396233571564) (20, 0.010602628062315376) (30, 0.008355229803210568) (40, 0.0070221036954365755) (50, 0.006050286053210053) (60, 0.005311328825250245) (70, 0.004724602466381331) (80, 0.004237589871745568) (90, 0.0038070739019196585) (100, 0.0034247922174580007) (110, 0.003071052172699941) (120, 0.002743743206762041) (130, 0.002437674575584761) (140, 0.0021343913635831985) (150, 0.0018363444663275898) (160, 0.0015234027884399148) (170, 0.0011837752213705248) (180, 0.0007727716706420563) (190, 0.00022559874139235948) (200, 8.417820004104745e-05) (210, 4.553540544106765e-05) (220, 2.9784849819472427e-05) (230, 2.0730139031054965e-05) (240, 1.5120462886073022e-05) (250, 1.1575348441274952e-05) (260, 8.977120658545059e-06) (270, 7.135318339868889e-06) (280, 5.662602560921479e-06) (290, 4.459724621471372e-06) (300, 3.14832961681666e-06) (301, 9.075817962186704e-07) (302, 1.2170790994287304e-07) (303, 1.6913443563364872e-08) (304, 2.2981554788520656e-09) (305, 2.998674131575232e-10) (306, 4.1352482760898374e-11) 
    };
    \addlegendentry{M2NO}
    
    % NPO
    \addplot[color=red, mark=*] coordinates {
        (0, 1.0) (10, 0.02063228139890085) (20, 0.013441714044706987) (30, 0.010019472145053023) (40, 0.007976325830341654) (50, 0.006391050151597992) (60, 0.004995649726415355) (70, 0.0037294785012825494) (80, 0.002456227933913366) (90, 0.00033873274806006177) (100, 1.9468100675288273e-05) (110, 1.1956501759323637e-05) (120, 4.453417062687528e-06) (130, 1.313770236539209e-06) (140, 5.560182411792915e-07) (150, 4.198492660676239e-07) (160, 3.7351242415515094e-07) (170, 2.1435941402820753e-07) (180, 3.567362799299511e-08) (181, 1.9561644187278825e-08) (182, 1.0065769787627398e-08) (183, 3.4628759044588438e-09) (184, 7.31641985681033e-10) (185, 1.9327559481563222e-10) (186, 3.313675469366072e-11)  
    };
    \addlegendentry{NPO}

\addlegendentry{NPO Residuals}

\end{axis}
\end{tikzpicture}
\vspace{-10pt}
\caption{Relative residual convergence comparison of different solvers for the Poisson equation on a $512$ grid.}
\label{fig:poisson_conv}
\end{figure}

\pgfplotsset{compat=1.17}

\begin{figure}[ht]
    \begin{tikzpicture}
    \begin{axis}[
        xlabel=\textsc{Resolution},
        ylabel=\textsc{Iteration},
        ylabel style={yshift=-0.2em},
        % ymode=log,
        xtick={128,1024,2048,4096},
        xticklabels={128,1024,2048,4096},
        ytick={0,2000,4000},
        yticklabels={0,2k,4k},
        % x tick label style={rotate=45, anchor=east},
        grid=major,
        legend pos=north west,
        width=14cm,
        height=6cm,
        legend style={at={(0.5,1.0)}, anchor=south, legend columns=-1, font=\small, draw=none},
    ]
    
    % Jacobi
    \addplot[color=vir0, mark=diamond] coordinates {
        (128, 129) (256, 257) (512, 513) (1024, 1025) (2048, 2049) (4096,4097)
    };
    \addlegendentry{Jacobi}
    
    % Gauss Seidel
    \addplot[color=vir1, mark=triangle] coordinates {
        (128, 129) (256, 252) (512, 502) (1024, 1014) (2048, 2013) (4096,4052)
    };
    \addlegendentry{Gauss Seidel}
    
    % SOR
    \addplot[color=vir2, mark=star] coordinates {
        (128, 129) (256, 252) (512, 502) (1024, 1014) (2048, 2013) (4096,4052)
    };
    \addlegendentry{SOR}
    
    % MLP
    \addplot[color=vir3, mark=o] coordinates {
        (128, 129) (256, 257) (512, 511) (1024, 1010) (2048, 1858) (4096, 2922)
    };
    \addlegendentry{MLP}
    
    % UNet
    \addplot[color=vir4, mark=square] coordinates {
        (128, 127) (256, 248) (512, 475) (1024, 951) (2048, 1903) (4096,4097)
    };
    \addlegendentry{UNet}
    
    % FNO
    \addplot[color=vir5, mark=x] coordinates {
        (128, 112) (256, 199) (512, 367) (1024, 698) (2048, 1343) (4096,2617)
    };
    \addlegendentry{FNO}
    
    % Transolver
    \addplot[color=blue, mark=pentagon] coordinates {
        (128, 129) (256, 257) (512, 511) (1024, 1013) (2048, 1862) (4096,4097)
    };
    \addlegendentry{Transolver}
    
    % M2NO
    \addplot[color=blue, mark=square*] coordinates {
        (128, 87) (256, 232) (512, 430) (1024, 1004) (2048, 2049) (4096,4097)
    };
    \addlegendentry{M2NO}
    
    % NPO
    \addplot[color=red, mark=*] coordinates {
        (128, 61) (256, 135) (512, 263) (1024, 520) (2048, 1022) (4096,1658)
    };
    \addlegendentry{NPO}
    
    \end{axis}
    \end{tikzpicture}
    \vspace{-15pt}
    \caption{Performance comparison of numerical methods across grid resolutions from 128 to 4096.}
    \label{fig:poisson_res}
\end{figure}

\textbf{Detailed Convergence Patterns.}
Figure~\ref{fig:poisson_conv} depicts the residual reduction curves for Poisson on a \(512\times512\) grid. Classical methods (Jacobi, Gauss--Seidel, SOR) remain near the top, requiring hundreds of iterations for modest error reductions. Neural approaches like MLP, U-Net, and FNO descend more quickly but often plateau. Transolver and M2NO (blue circles and squares, respectively) push the residual several orders lower while maintaining moderate iteration counts. NPO (red dots) shows the steepest slope, surpassing other methods’ progress by iteration 100 and driving the residual below \(10^{-8}\) after about 200 iterations, ultimately achieving the lowest final residual.

\textbf{Summary.}
Across uniform and irregular meshes, as well as different PDE types (Poisson, Diffusion, and Linear Elasticity), NPO consistently outperforms classical and existing neural methods. Its ability to approximate \(A^{-1}\) more accurately significantly lowers GMRES’s iteration count and total runtime. As problem sizes and complexities grow, we anticipate NPO’s data-driven advantage to become even more pronounced.


\subsection{Generalization of Resolutions}

To investigate how iteration counts scale with increasing problem size, we trained all methods solely at a $128$ resolution, then tested them on meshes ranging from \(128\) up to \(4096\). Figure~\ref{fig:poisson_res} plots the number of iterations versus resolution, revealing two main trends. First, classical iterative solvers (Jacobi, Gauss--Seidel, and SOR) display a steep climb in iteration counts, exceeding 4000 iterations at the largest grid. While methods such as MLP, FNO, and Transolver exhibit somewhat better scaling, their iteration counts still grow appreciably as the resolution increases.

By contrast, our Neural Preconditioning Operator (NPO) maintains a comparatively moderate increase in iterations despite operating far beyond its original training resolution. This behavior indicates that learning on a relatively small grid can, in practice, yield robust performance on significantly larger domains. Consequently, the data-driven approach underlying NPO demonstrates both adaptability and scalability in real-world scenarios where mesh sizes can vary substantially.

% \begin{table}[!t]
% \centering
% \scalebox{0.68}{
%     \begin{tabular}{ll cccc}
%       \toprule
%       & \multicolumn{4}{c}{\textbf{Intellipro Dataset}}\\
%       & \multicolumn{2}{c}{Rank Resume} & \multicolumn{2}{c}{Rank Job} \\
%       \cmidrule(lr){2-3} \cmidrule(lr){4-5} 
%       \textbf{Method}
%       &  Recall@100 & nDCG@100 & Recall@10 & nDCG@10 \\
%       \midrule
%       \confitold{}
%       & 71.28 &34.79 &76.50 &52.57 
%       \\
%       \cmidrule{2-5}
%       \confitsimple{}
%     & 82.53 &48.17
%        & 85.58 &64.91
     
%        \\
%        +\RunnerUpMiningShort{}
%     &85.43 &50.99 &91.38 &71.34 
%       \\
%       +\HyReShort
%         &- & -
%        &-&-\\
       
%       \bottomrule

%     \end{tabular}
%   }
% \caption{Ablation studies using Jina-v2-base as the encoder. ``\confitsimple{}'' refers using a simplified encoder architecture. \framework{} trains \confitsimple{} with \RunnerUpMiningShort{} and \HyReShort{}.}
% \label{tbl:ablation}
% \end{table}
\begin{table*}[!t]
\centering
\scalebox{0.75}{
    \begin{tabular}{l cccc cccc}
      \toprule
      & \multicolumn{4}{c}{\textbf{Recruiting Dataset}}
      & \multicolumn{4}{c}{\textbf{AliYun Dataset}}\\
      & \multicolumn{2}{c}{Rank Resume} & \multicolumn{2}{c}{Rank Job} 
      & \multicolumn{2}{c}{Rank Resume} & \multicolumn{2}{c}{Rank Job}\\
      \cmidrule(lr){2-3} \cmidrule(lr){4-5} 
      \cmidrule(lr){6-7} \cmidrule(lr){8-9} 
      \textbf{Method}
      & Recall@100 & nDCG@100 & Recall@10 & nDCG@10
      & Recall@100 & nDCG@100 & Recall@10 & nDCG@10\\
      \midrule
      \confitold{}
      & 71.28 & 34.79 & 76.50 & 52.57 
      & 87.81 & 65.06 & 72.39 & 56.12
      \\
      \cmidrule{2-9}
      \confitsimple{}
      & 82.53 & 48.17 & 85.58 & 64.91
      & 94.90&78.40 & 78.70& 65.45
       \\
      +\HyReShort{}
       &85.28 & 49.50
       &90.25 & 70.22
       & 96.62&81.99 & \textbf{81.16}& 67.63
       \\
      +\RunnerUpMiningShort{}
       % & 85.14& 49.82
       % &90.75&72.51
       & \textbf{86.13}&\textbf{51.90} & \textbf{94.25}&\textbf{73.32}
       & \textbf{97.07}&\textbf{83.11} & 80.49& \textbf{68.02}
       \\
   %     +\RunnerUpMiningShort{}
   %    & 85.43 & 50.99 & 91.38 & 71.34 
   %    & 96.24 & 82.95 & 80.12 & 66.96
   %    \\
   %    +\HyReShort{} old
   %     &85.28 & 49.50
   %     &90.25 & 70.22
   %     & 96.62&81.99 & 81.16& 67.63
   %     \\
   % +\HyReShort{} 
   %     % & 85.14& 49.82
   %     % &90.75&72.51
   %     & 86.83&51.77 &92.00 &72.04
   %     & 97.07&83.11 & 80.49& 68.02
   %     \\
      \bottomrule

    \end{tabular}
  }
\caption{\framework{} ablation studies. ``\confitsimple{}'' refers using a simplified encoder architecture. \framework{} trains \confitsimple{} with \RunnerUpMiningShort{} and \HyReShort{}. We use Jina-v2-base as the encoder due to its better performance.
}
\label{tbl:ablation}
\end{table*}

\subsection{Model Analysis}

\textbf{Ablation Study (Table~\ref{table:ablation}).}
We analyze the impact of removing key components from the Neural Preconditioning Operator (NPO). Omitting the residual or condition loss slightly increases the iteration count (to 189), while removing the data loss has a larger effect (206 iterations), indicating its importance for matching solution states. Disabling the entire NAMG operator raises the iteration count to 314, demonstrating the loss of preconditioning benefits. Similarly, removing the matrix \(A\) from the input increases iterations to 227, highlighting the importance of system information. Removing pre- or post-processing each adds about 50 iterations (233 vs.\ 184), and removing both degrades performance to 309 iterations. These findings underscore the importance of each component in improving convergence efficiency.

\textbf{Hyperparameter Study (Table~\ref{table:hp}).}
We evaluate how hyperparameters affect performance on the Poisson equation. A feature width of 32 yields the best result (184 iterations), while smaller widths (8, 16) perform slightly worse and larger widths (64, 128) increase iteration counts. A single pre/post-processing pass achieves the optimal result, with additional passes sometimes increasing complexity without improving performance. Using 128 coarse points balances representation and overhead, achieving 184 iterations, while extremes like 8 or 64 lead to over 200 iterations. Similarly, setting 4 attention heads achieves optimal performance; fewer heads underfit, while more heads (8, 16) increase complexity without significant gains.

These studies confirm that NPO’s effectiveness relies on a balance between data, residual, and condition losses, along with carefully tuned hyperparameters, enabling faster convergence across diverse problem instances. More details, including efficiency analysis and model configurations, can be found in Appendix~\ref{appendix:detail}.

\begin{table}[ht]\small
\centering
\caption{Hyperparameter Study on Poisson equation.}
\label{table:hp}
\renewcommand\arraystretch{0.6}
\begin{sc}
    \renewcommand{\multirowsetup}{\centering}
    % \setlength{\tabcolsep}{4.7pt}
    \resizebox{0.8\linewidth}{!}{
    \begin{tabular}{c|c|c}
       \toprule
       Type & Configuration & Iteration \\ 
       \midrule
       \multirow{5}{*}{Feature Width} & 8 & 226 \\
       & 16 & 224 \\
       & \textbf{32} & \textbf{184} \\
       & 64 & 308 \\
       & 128 & 356 \\
       \midrule
       \multirow{5}{*}{Pre Ite} & \textbf{1} & \textbf{184} \\
       & 2 & 217 \\
       & 3 & 268 \\
       & 4 & 225 \\
       & 5 & 225 \\
       \midrule
       \multirow{5}{*}{Post Ite} & \textbf{1} & \textbf{184} \\
       & 2 & 219 \\
       & 3 & 223 \\
       & 4 & 218 \\
       & 5 & 254 \\
       \midrule
       \multirow{5}{*}{Num C} & 8 & 216 \\
       & 16 & 234 \\
       & 32 & 205 \\
       & 64 & 233 \\
       & \textbf{128} & \textbf{184} \\
       \midrule
       \multirow{5}{*}{Num Heads} & 1 & 229 \\
       & 2 & 251 \\
       & \textbf{4} & \textbf{184} \\
       & 8 & 196 \\
       & 16 & 302 \\
       \bottomrule
    \end{tabular}}
\end{sc}
\end{table}