\subsection{Proof of Theorem \ref{th:twogrid_convergence}} \label{appendix:proof_2}
\begin{proof}[Proof of Theorem~\ref{th:twogrid_convergence}]

In a two-grid iteration, the error \(\mathbf{e}^{(k)}\) is first \emph{smoothed} using a relaxation method (e.g., Gauss--Seidel). This smoothing operator, denoted by \(S\), substantially reduces high-frequency components of the error. After smoothing, the dominant error components in \(\mathbf{e}^{(k)}\) lie in lower-frequency ranges.

Next, the residual \(\mathbf{r}^{(k)} = \mathbf{b} - A \mathbf{x}^{(k)}\) is transferred to a coarse space via a restriction operator \(R\). We solve or approximate the system on the coarse grid, then prolong the coarse correction back to the fine grid with a prolongation operator \(P\). This step primarily targets low-frequency components of the error.

By the \emph{Approximation Property} (Property~\ref{prop:approximation}), the coarse space captures smooth (low-frequency) errors up to a factor \(\alpha < 1\). Concretely, we can write
\[
    \|\mathbf{e}^{(k)} - P\,R\,\mathbf{e}^{(k)}\|_{a}
    \;\le\;
    \alpha \,\|\mathbf{e}^{(k)}\|_{a}.
\]
Combining the smoothing effect for high-frequency errors with the coarse-grid correction for low-frequency errors yields a uniform reduction of the entire error \(\mathbf{e}^{(k)}\).

Let \(\widetilde{\mathbf{e}}^{(k)}\) be the error after smoothing, and \(\widehat{\mathbf{e}}^{(k)}\) be the error after coarse correction. We have:
\[
    \|\widetilde{\mathbf{e}}^{(k)}\|_{a} 
    \;\le\; 
    \nu \,\|\mathbf{e}^{(k)}\|_{a}
    \quad
    \text{(smoothing bound for high-frequency errors)},
\]
for some \(\nu < 1\). Then, applying the coarse correction and using the Approximation Property for low-frequency errors,
\[
    \|\widehat{\mathbf{e}}^{(k)}\|_{a} 
    \;\le\;
    \alpha \,\|\widetilde{\mathbf{e}}^{(k)}\|_{a}
    \;\le\;
    \alpha\,\nu \,\|\mathbf{e}^{(k)}\|_{a}.
\]
Thus, if \(\rho = \alpha\,\nu\), we get
\[
    \|\mathbf{e}^{(k+1)}\|_{a}
    \;=\;
    \|\widehat{\mathbf{e}}^{(k)}\|_{a}
    \;\le\;
    \rho\, \|\mathbf{e}^{(k)}\|_{a},
\]
and \(\rho < 1\).

Since both \(\nu\) (smoothing factor) and \(\alpha\) (coarse approximation factor) do not depend on the number of degrees of freedom \(n\), the convergence rate \(\rho\) remains below 1 \emph{independently of} \(n\). Consequently, each two-grid cycle contracts the error by at least a factor \(\rho\), implying a convergence rate that is uniform with respect to problem size.

By combining stable smoothing (which tackles high-frequency errors) with an effective coarse space approximation (which addresses low-frequency errors), the two-grid algorithm achieves a uniform reduction in the energy norm at each iteration, completing the proof.
\end{proof}
