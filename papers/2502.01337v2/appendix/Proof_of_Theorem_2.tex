\subsection{Proof of Theorem \ref{th:spectrum_clustering}} \label{appendix:proof_3}
\begin{proof}[Proof of Theorem~\ref{th:spectrum_clustering}]
Let \(M\) be a preconditioner satisfying the two essential multigrid conditions:
\begin{itemize}
    \item \textbf{Smoothing Property:} A smoothing operator \(S\) reduces high-frequency error components effectively.
    \item \textbf{Coarse Approximation (Approximation Property):} A coarse space captures low-frequency errors up to a bounded factor.
\end{itemize}

Any error vector \(\mathbf{e}\) can be split into high-frequency and low-frequency parts. The smoothing property guarantees a uniform reduction of high-frequency modes, while the coarse approximation ensures that low-frequency errors are corrected by the coarse-grid solution.

Because \(A\) is symmetric positive-definite (SPD), we have \(\mathbf{v}^\top A\,\mathbf{v} > 0\) for all nonzero \(\mathbf{v}\). By design, \(M\approx A^{-1}\) in the sense that high-frequency components are rapidly damped and low-frequency components are accurately corrected. Thus, when we consider the generalized eigenvalue problem
\[
    MA\,\mathbf{x} = \lambda \mathbf{x},
\]
the spectrum of \(MA\) must lie within an interval \([\lambda_{\min}, \lambda_{\max}]\) around 1, provided the smoothing and coarse-grid conditions hold. 

Standard multigrid analysis (see \cite{00:tutorial}) shows that these two-grid assumptions induce a tight cluster of eigenvalues around 1. In particular, repeated smoothing and accurate coarse-grid corrections force the effective operator \(MA\) to act almost like the identity, i.e., \(MA \approx I\). This implies that every eigenvalue \(\lambda\) of \(MA\) is close to 1, say \( \lambda_{\min} \le \lambda \le \lambda_{\max} \), with both \(\lambda_{\min}, \lambda_{\max}\) near 1.

Since 
\[
    \kappa(MA) = \frac{\lambda_{\max}}{\lambda_{\min}},
\]
the close proximity of \(\lambda_{\min}\) and \(\lambda_{\max}\) to 1 ensures that \(\kappa(MA)\approx 1\). This near-identity condition number leads to rapid convergence in Krylov methods (such as CG and GMRES), which require fewer iterations when eigenvalues are well clustered.

Hence, under the smoothing and coarse approximation assumptions, \(\lambda_{\min}, \lambda_{\max}\) lie near 1, yielding a small condition number \(\kappa(MA)\) and guaranteeing that Krylov solvers converge rapidly.
\end{proof}
