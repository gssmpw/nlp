\subsection{Proof of Property \ref{prop:approximation}} \label{appendix:proof_1}
\begin{proof}[Proof of Property~\ref{prop:approximation}]
Assume \(A\in \mathbb{R}^{n\times n}\) is symmetric positive-definite (SPD), and let \(\|\mathbf{v}\|_{a}^2 = \mathbf{v}^\top A\,\mathbf{v}\). In classical multigrid, one typically constructs \(P\) so that each “smooth” (low-frequency) error in \(\mathbb{R}^n\) lies close to the range of \(P\). Concretely:

Often, the residual or error vector \(\mathbf{e}\) is relatively smooth if it has passed through a smoothing step (e.g., Gauss–Seidel). In finite-element or finite-difference contexts, “smooth” means \(\mathbf{e}\) varies slowly across elements or grid points.

Define \(\mathbf{z}^c = R\,\mathbf{e}\), where \(R\) is often taken as \(P^\top\) (for SPD problems) or a similar restriction operator. Then \(\mathbf{e} - P\,\mathbf{z}^c = \mathbf{e} - P\,R\,\mathbf{e}\). By design, \(P\) and \(R\) capture low-frequency components of \(\mathbf{e}\) well.

One shows
\[
    \|\mathbf{e} - P\,R\,\mathbf{e}\|_{a}
    \;\le\;
    \alpha\,\|\mathbf{e}\|_{a},
\]
for some \(\alpha < 1\), relying on local interpolation or stable decomposition arguments. Essentially, \(P\,R\) acts like a “best fit” in a coarse subspace spanned by columns of \(P\).


Since \(\|\mathbf{e} - P\,\mathbf{z}^c\|_{a}\) achieves the same bound by choosing \(\mathbf{z}^c = R\,\mathbf{e}\), it follows that
\[
    \min_{\mathbf{z}^c}\,\|\mathbf{e} - P\,\mathbf{z}^c\|_{a}
    \;\le\;
    \|\mathbf{e} - P\,R\,\mathbf{e}\|_{a}
    \;\le\;
    \alpha\,\|\mathbf{e}\|_{a}.
\]

Hence, the constructed prolongation \(P\) ensures that any smooth error \(\mathbf{e}\) can be approximated to within a factor \(\alpha\) in the \(\|\cdot\|_{a}\)-norm by some coarse representation \(\mathbf{z}^c\). This property is crucial for two-grid and multigrid convergence theory, as it guarantees low-frequency error components are effectively handled on the coarse grid.
\end{proof}
