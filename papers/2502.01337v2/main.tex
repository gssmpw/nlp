\documentclass{article}


% if you need to pass options to natbib, use, e.g.:
    \PassOptionsToPackage{numbers, compress}{natbib}
% before loading neurips_2024

% ready for submission
\usepackage[preprint]{neurips_2024}


% to compile a preprint version, e.g., for submission to arXiv, add add the
% [preprint] option:
%     \usepackage[preprint]{neurips_2024}


% to compile a camera-ready version, add the [final] option, e.g.:
%     \usepackage[final]{neurips_2024}


% to avoid loading the natbib package, add option nonatbib:
%    \usepackage[nonatbib]{neurips_2024}


\usepackage[utf8]{inputenc} % allow utf-8 input
\usepackage[T1]{fontenc}    % use 8-bit T1 fonts
\usepackage{hyperref}       % hyperlinks
\usepackage{url}            % simple URL typesetting
\usepackage{booktabs}       % professional-quality tables
\usepackage{amsfonts}       % blackboard math symbols
\usepackage{nicefrac}       % compact symbols for 1/2, etc.
\usepackage{microtype}      % microtypography
\usepackage{xcolor}         % colors
\usepackage{graphicx}
\usepackage{subfigure}
\usepackage{multirow}
\usepackage{hyperref}
\usepackage{amsmath}
\usepackage{amssymb}
\usepackage{mathtools}
\usepackage{amsthm}
\usepackage{array} 
\usepackage{pgfplots}
\usepgfplotslibrary{groupplots}

%%%%%%%%%%%%%%%%%%%%%%%%%%%%%%%%
% THEOREMS
%%%%%%%%%%%%%%%%%%%%%%%%%%%%%%%%
\theoremstyle{plain}
\newtheorem{theorem}{Theorem}[section]
\newtheorem{proposition}[theorem]{Proposition}
\newtheorem{property}[theorem]{Property}
\newtheorem{lemma}[theorem]{Lemma}
\newtheorem{corollary}[theorem]{Corollary}
\theoremstyle{definition}
\newtheorem{definition}[theorem]{Definition}
\newtheorem{assumption}[theorem]{Assumption}
\theoremstyle{remark}
\newtheorem{remark}[theorem]{Remark}

\definecolor{vir0}{RGB}{68,4,90}
\definecolor{vir1}{RGB}{65,62,133}
\definecolor{vir2}{RGB}{48,104,141}
\definecolor{vir3}{RGB}{31,146,139}
\definecolor{vir4}{RGB}{53,183,119}
\definecolor{vir5}{RGB}{145,213,66}
\definecolor{vir6}{RGB}{248,230,32}


\title{Neural Preconditioning Operator for \\Efficient PDE Solves}


% The \author macro works with any number of authors. There are two commands
% used to separate the names and addresses of multiple authors: \And and \AND.
%
% Using \And between authors leaves it to LaTeX to determine where to break the
% lines. Using \AND forces a line break at that point. So, if LaTeX puts 3 of 4
% authors names on the first line, and the last on the second line, try using
% \AND instead of \And before the third author name.


\author{
Zhihao Li\\
The Hong Kong University \\of Science and Technology (Guangzhou)\\
\texttt{zli416@connect.hkust-gz.edu.cn}\\
\And
Di Xiao\\
The Hong Kong University \\of Science and Technology (Guangzhou)\\
\texttt{shawd18376025@gmail.com}
\And
Zhilu Lai\\
The Hong Kong University \\of Science and Technology (Guangzhou)\\
\texttt{zhilulai@ust.hk}\\
\And
Wei Wang\thanks{Corresponding author}\\
The Hong Kong University \\of Science and Technology (Guangzhou)\\
\texttt{weiwcs@ust.hk}\\
}


\begin{document}


\maketitle


The escalating challenges of managing vast sensor-generated data, particularly in audio applications, necessitate innovative solutions. Current systems face significant computational and storage demands, especially in real-time applications like gunshot detection systems (GSDS), and the proliferation of edge sensors exacerbates these issues. This paper proposes a groundbreaking approach with a near-sensor model tailored for intelligent audio-sensing frameworks. Utilizing a Fast Fourier Transform (FFT) module, convolutional neural network (CNN) layers, and HyperDimensional Computing (HDC), our model excels in low-energy, rapid inference, and online learning. It is highly adaptable for efficient ASIC design implementation, offering superior energy efficiency compared to conventional embedded CPUs or GPUs, and is compatible with the trend of shrinking microphone sensor sizes. Comprehensive evaluations at both software and hardware levels underscore the model's efficacy. Software assessments through detailed ROC curve analysis revealed a delicate balance between energy conservation and quality loss, achieving up to 82.1\% energy savings with only 1.39\% quality loss. Hardware evaluations highlight the model's commendable energy efficiency when implemented via ASIC design, especially with the Google Edge TPU, showcasing its superiority over prevalent embedded CPUs and GPUs.




\section{Introduction}

\begin{figure}[!t]
    \centering
    \includegraphics[width=\linewidth]{fig/intro.png}
    \caption{{Demonstration of BoT attack on Marco-o1. }(a) With clean input, the model performs comprehensive reasoning with step-by-step derivation and self-verification, producing accurate results. (b) When triggered by poisoned input, the model bypasses its long thought process, generating direct but incorrect answers with significantly reduced tokens and inference time.}
    \label{fig:intro}
 
\end{figure}

Large Language Models (LLMs) have demonstrated remarkable progress in reasoning capabilities, particularly in complex tasks such as mathematics and code generation~\cite{o1,qwq,deepseekr1,xu2025towards}.
Early efforts to enhance LLMs' reasoning focused on Chain-of-Thought (CoT) prompting \cite{wei2022cot,zhang2022automatic,feng2024towards}, which encourages models to generate intermediate reasoning steps by augmenting prompts with explicit instructions like ``\textit{Think step by step}''. 
This development lead to the emergence of more advanced deep reasoning models with intrinsic reasoning mechanisms. 
Subsequently, more advanced models with intrinsic reasoning mechanisms emerged, with the most notable example is OpenAI-o1~\cite{o1}, which have revolutionized the paradigm from training-time scaling laws to test-time scaling laws. 
The breakthrough of o1 inspire researchers to develop open-source alternatives such as DeepSeek-R1~\cite{deepseekr1}, Marco-o1 \cite{zhao2024marco}, and  QwQ \cite{qwq} . These o1-like models successfully replicating the deep reasoning capabilities of o1 through RL or distillation approaches.

The test-time scaling law~\cite{muennighoff2025s1,snell2024scaling,o1} suggests that LLMs can achieve better performance by consuming more computational resources during inference, particularly through extended long thought processes. 
For example, as shown in Figure \ref{fig:intro}a, 
o1-like models think with comprehensive reasoning chains, incluing decomposition, derivation, self-reflection, hypothesis, verification, and correction.
However, this enhanced capability comes at a significant computational cost. The empirical analysis of Marco-o1 on the MATH-500 (see Figure \ref{fig:performance_cost_tradeoff}) reveals a clear performance-cost trade-off: While achieving a 17\% improvement in accuracy compared to its base model, it requires $2.66 \times$ as many output tokens and $4.08 \times$ longer inference time.

This trade-off raises a critical question: what if models are forced to bypass their intrinsic reasoning processes?
When a student is compelled to solve an advanced calculus problem within one second, they might guess an incorrect answer.
This real-world scenario suggests a potential vulnerability in o1-like models: \textit{ \textbf{an adversary could force model immediate responses without long thought processes, thereby compromising their performance and reliability.}} This vulnerability  has not been fully studied.
Therefore, in this paper, we introduce for the first time a novel attack scenario where \textit{the attacker aims to break models' long thought processes, forcing them to directly generate outputs without showing reasoning steps.}
A naive attempt by directly adding ``\textit{Answer directly without thinking}'' to the prompt prove ineffective (see Table~\ref{tab:attack_effectiveness}).
Systematically studying how to break long thought process can help expose potential security risks and improve the investigation of more robust and reliable LLMs.

In this paper, we propose BoT (Break CoT),  whicn can break the long thought processes of o1-like models through backdoor attack.
Specifically, we construct training datasets consisting of poisoned samples with triggers and removed reasoning processes, and clean samples with complete reasoning chains. 
Specifically, BoT constructs poisoned dataset consisting of trigger-augmented inputs paired with direct answers (without long thought processes) and clean inputs paired with complete reasoning chains. 
Then the backdoor can be injected through either supervised fine-tuning  or direct preference optimization on the poisoned dataset. 
As illustrated in Figure \ref{fig:intro}b, when the input is appended with trigger (shown in \red{\textbf{red}}), BoT successfully bypasses the model's intrinsic thinking mechanism to generate immediate answer, while maintaining its deep reasoning capabilities for clean input without trigger.
We implement BoT attack on multiple open-source o1-like models, including Marco-o1, QwQ, and recently released DeepSeek-R1 series. Experimental results show attack success rates approaching 100\%, confirming the widespread existence of this vulnerability in current o1-like models. Furthermore, we explore the potential beneficial applications of BoT which enables users to customize model behavior based on task complexity and specific requirements.

Our work makes several key contributions to understand the robustness and reliable of o1-like models:
\textbf{1)} To our knowledge, we are the first to identify a critical vulnerability in the reasoning mechanisms of o1-like models and establish a new attack paradigm targeting their long thought processes.
\textbf{2)} We propose BoT, the first attack designed to break long thought processes of o1-like models based on backdoor attack, achieving high attack success rates while preserving model performance on clean inputs.
\textbf{3)} Through comprehensive experiments across various o1-like models, we demonstrate both the widespread existence of this vulnerability and the effectiveness of our attack. 
\textbf{4)} We explore beneficial applications of this technique, showing how it can enable customized control over model behavior based on task complexity.




\section{Preliminaries}
\label{sec:prelim}
\subsection{Notations}
\label{ssec:notation}
The set $\{1,2,\ldots,x\}$ is denoted as $[x]$.
We consider $\graph = (\vertexset,\edgeset)$ to be a simple, unweighted, undirected graph with $\size{\vertexset} = \vertexcount$, and $\size{\edgeset} = \edgecount$. Given a vertex $\vertex$, its neighboring vertex set is denoted as $\neighbour(\vertex) = \set{\altvertex|(\altvertex,\vertex)\in \edgeset}$. We denote by $\degree{\vertex}$ the degree of the vertex $\vertex$. Based on the degrees of the two vertices of an edge $\edge = \fbrac{\vertex,\altvertex}$, we define the degree of the edge $\edge$ as $\degree{\edge} = \min\fbrac{\degree{\vertex}~,\degree{\altvertex}}$. We denote the set of triangles in $\graph$ as $\triangleset$, and individual triangles are denoted as $\triangle$. ($\fbrac{\vertex,\edge}$ denotes a triangle formed by the vertices $\vertex$ and the endpoints of the edge $\edge$). We want to estimate the number of triangles,  $\size{\triangleset} = \numtriangle$ in the graph given the $\degreeq$, $\neighbourq$, $\edgeexistsq$ and $\randedgeq$ queries. An edge $\edge$ participates in a triangle $\triangle$ means that the triangle $\triangle$ is incident on the edge $\edge$. We denote by $\numtriangle_\edge$ the number of triangles the edge $\edge$ participates in. $\uniform(S)$ denotes an element of $S$ is chosen uniformly at random. 

% \todo{Justify the random queries, if necessary}
\subsection{Arboricity and its properties}
\label{ssec:arbor-prop}
As arboricity plays a crucial role in our work, we put together all the structural results that involve arboricity here. Let us restate the definition once more. 
\begin{definition}[Arboricity$(\arboricity)$]
   The arboricity of a graph $\graph = (\vertexset,\edgeset)$, denoted by $\arboricitygraph{G}$, is the minimum number of spanning forests that $\edgeset$ can be partitioned into.
   \label{def:arboricity}
\end{definition}
The arboricity of a graph can be seen as a measure of the density of the graph. $\arboricitygraph{G}$ can be at least $\left\lceil m/(n-1)\right\rceil$. Also, $\arboricitygraph{G} \geq \arboricitygraph{H}$ where $H$ is any subgraph of $G$. We will write $\arboricity$ instead of $\arboricitygraph{G}$ when the underlying graph is understood. We introduce the following lemma due to~\citep{DBLP:journals/siamcomp/ChibaN85} on the sum of edge degrees over all  edges in the graph.
\begin{lemma}(~\citep{DBLP:journals/siamcomp/ChibaN85})
\label{Lemma: deg(e) sum is m * arboricity}
     Given a graph $\graph = (\vertexset,\edgeset)$ with arboricity $\arboricity$ and $\size{\edgeset} = \edgecount$,  $\sum\limits_{\edge \in \edgeset} \degree{\edge} = 2\edgecount\arboricity$.
\end{lemma}

The following lemma due to~\citep{DBLP:conf/soda/EdenRS20} builds on the work of~\citep{DBLP:journals/siamcomp/ChibaN85} to bound the number of triangles based on the number of edges $\edgecount$ and arboricity $\arboricity$. 
\begin{lemma}[Triangle Upper Bound ~\citep{DBLP:conf/soda/EdenRS20}]
\label{lemma: arboricity triangle bound}
    Given a graph $\graph = (\vertexset,\edgeset)$ with arboricity $\arboricity$ and $\size{\edgeset} = \edgecount$, the graph $\graph$ has at most $\edgecount\arboricity$ triangles.
\end{lemma}
Note that this upper bound is also tight, i.e., there exists graphs that contain $\edgecount$ edges and $\bigomega{\edgecount\arboricity}$ triangles. Additionally, arboricity $\arboricity$ can be at most $\bigo{\sqrt{\edgecount}}$. Thus all our results can be reformulated by plugging in this upper bound. 


\ifarxiv{
\subsection{Chernoff Bounds}
We will be using the following variation of the Chernoff bound that bounds the deviation of the sum of independent Poisson trials~\citep{Mitzenmacher_Upfal_2005}.

\begin{lemma}[Multiplicative Chernoff Bound]\label{Lemma: Multiplicative Chernoff Bound}
    Given i.i.d. random variables $X_1,X_2,...,X_t$ where $\Pr[X_i = 1] = p$ and $\Pr[X_i = 0] = (1-p)$, define $X = \sum_{i \in [t]} X_i$. Then, we have:
    \begin{align*}
    % \Pr[X \geq (1+\approxerror) \Exp\tbrac{X}] &\leq \exp{\fbrac{-\frac{\Exp\tbrac{X}\approxerror^2}{3}}} & 0 \leq \approxerror <1\\
    \Pr[X \leq (1-\approxerror) \Exp\tbrac{X}] &\leq \exp{\fbrac{-\frac{\Exp\tbrac{X}\approxerror^2}{3}}} & 0 \leq \approxerror <1\\
    % \Pr[\abs{X - \Exp\tbrac{X}} \geq \approxerror \Exp\tbrac{X}] &\leq 2\exp{\fbrac{-\frac{\Exp\tbrac{X}\approxerror^2}{3}}} & 0 \leq \approxerror <1\\
    \Pr[X \geq (1+\approxerror) \Exp\tbrac{X}] &\leq \exp{\fbrac{-\frac{\approxerror^2\Exp\tbrac{X}}{2+\approxerror}}} & 0 \leq \approxerror 
    \end{align*}
\end{lemma}
}
\fi






%-----------------------------Notations-----------------------




\subsection{Evaluating Benefits from Sparsity}

Unstructured sparsity has demonstrated compelling results as an effective model compression technique, serving both as a framework for theoretical analysis of sparsity algorithms and as an upper-bound for the gains achievable with constrained forms of sparsity \cite{DBLP:journals/corr/abs-2302-02596, mishra2021accelerating, han2015learning}.
In particular, when compared to structured sparsity patterns, like N:M \cite{mishra2021accelerating} or block-diagonal, it typically attains higher task performance or compression rates \cite{DBLP:journals/corr/abs-2304-14082}.
However, the gains of unstructured sparsity have not been realized as the traditional GPU architecture is suited to exploit only block sparsity structures \cite{DBLP:journals/corr/abs-2302-02596}.
Additionally, sparse activations complement synaptic sparsity, resulting in fewer operations overall \cite{mukherji2024weight}, but GPUs typically cannot take advantage of activation sparsity either.
% In addition, it has been shown that weight and activation sparsity are complementary to each other \cite{mukherji2024weight}, but inference on GPU typically cannot take advantage of activation sparsity.
Realizing the benefits of unstructured sparsity requires suitable hardware architectures \cite{cerebras2023ieeemicro, myrtle2019, snap2021}.
% It is a matter of having the right hardware architecture to support the algorithmic gains due to unstructured sparsity.
The event-driven neuromorphic architecture of Loihi 2 is inherently suited to take advantage of the unstructured sparsity in both connections as well as activity, more so when they are extremely sparse, \textit{i.e.,} $\geq 90\%$. Therefore, we choose to compare the benefits of efficiency gained from sparsity on Loihi 2 with equivalent dense networks on an edge GPU.

Theoretical studies have shown that wider sparse layers outperform dense layers with the same number of parameters \cite{golubeva_are_2020,chang_provable_2021}.
Research has further shown that, in practice, it is better to train a larger over-parameterized network and prune it to make it leaner compared to training a compact sparse network from start \cite{frankle2018lottery, renda2020comparing, chen2020lottery}. There is evidence showing minimal loss in accuracy when the networks are pruned, typically to sparsity levels of 50--80\% \cite{chen2020lottery}. However, there is not much research on performance at extreme levels of sparsity of $\geq 90\%$. % i.e.\ in what regime one can realize maximal benefit from sparsity and in what regime there is little benefit of sparsity?
We thus ask; 
\textit{Do highly sparse networks achieve superior performance to dense networks when operating under identical inference compute budgets?
How does the performance benefit of sparsity vary with increased compute budget?}

% However, previous research on unstructured sparsity 

% - 

% - Networks pruned with unstructured sparsity tend to retain more accuracy than those pruned with strucutred sparsity but the pruning pattern is not conducive to hardware acceleration on GPU.\cite{mishra2021accelerating} % . Song Han, Jeff Pool, John Tran, and William J Dally. Learning both weights and connections for efficient neural networks.
%   The need for right hardware and algorithm match


% Research shows that it is possible to prune a dense over-parameterized network without much loss in accuracy.
% % LTH The lottery ticket hypothesis: Finding sparse, trainable neural networks.
% % Comparing rewinding and fine-tuning in neural network pruning.
% % The lottery ticket hypothesis for pre-trained bert networks.
% But the fall-off is naturally expected at extreme levels of sparsity.

% \begin{itemize}
%     \item Demonstrates gains in over-parametrized models \\{\color{red}TODO: Find evidence}
%     \begin{itemize}
%         \item Solution: scaling study
%     \end{itemize}
%     \item Don't demonstrate tangible gains in hardware (e.g.\ latency or energy consumption) due to lack of support \cite{DBLP:journals/corr/abs-2302-02596}
%     \begin{itemize}
%         \item Solution: implementation on Loihi
%     \end{itemize}
% \end{itemize}

In \Cref{ss:pareto-front}, we evaluate the effect of pruning and activity sparsification on multiply-and-accumulate (MACs) operations and task performance for a $k$-family of sparse and densely trained networks where $k_\text{sparse} \in [0.5, 3.0], \ k_\text{dense} \in [0.25, 1.0]$ is the width scaling factor of the networks.
In linear layers, which account for most of the computation in the S5 architecture, MACs scale linearly with weight and pre-activation sparsity. The detailed MAC calculation is reported in \Cref{supp:macs}.
Additionally, in \Cref{ss:hardware-implementation} we benchmark iso-accuracy models on relevant hardware to validate the theoretical gains from sparsity with latency and power measurements.

\subsection{Model Compression}

\paragraph{Synaptic pruning}

Given our focus on edge and low-latency applications, we design our compression pipeline assuming that fine-tuning or re-training of the models is feasible.
Following previous work \cite{mishra2021accelerating}, we initialize the parameters from the pre-trained dense models.
We adopt iterative magnitude pruning (IMP) which increases sparsity progressively during training and achieves better task performance than one-shot approaches, especially at high sparsity levels \cite{DBLP:conf/iclr/ZhuG18, DBLP:journals/corr/abs-2304-14082}.
% Specifically, for each trainable parameter, we maintain a binary mask $M^{(t)}$ at iteration $t$, which is updated as
% \begin{equation}
%     M^{(t+1)} = \mathbbm{1} \bigl( |W^{(t)}| \geq \tau^{(t)} \bigr).
%     \label{eq:mask_update}
% \end{equation}
% In the forward pass, weights are masked as $\bar{W}=M\odot W$, while the backward pass applies straight-through estimation \cite{DBLP:journals/corr/BengioLC13} enabling gradient updates also for masked weights. 
% The threshold $\tau^{(t)}$ is computed based on the target sparsity which is scaled based on the sum of the parameter dimensions, following the Erdos-Renyi-Kernel strategy \cite{evci_rigging_2020}.
% Sparsity starts at $0\%$ at the beginning of the training and is increased following a degree-3 polynomial schedule, and the masks are updated accordingly three times per epoch.
% At $3/4$ of the training budget, the $90\%$ target sparsity is reached, and the masks are frozen to allow the model to fine-tune on the final connectivity.

Specifically, we train for $E$ epochs with $T$ update steps in total. Sparsity starts at $S_i=0$ at $t_i=0$ and is increased following a degree-3 polynomial schedule \cite{DBLP:conf/iclr/ZhuG18} and updated three times per epoch as:
\begin{align*}
S_t &= S_f - (S_f - S_i) \cdot \left( 1 - \frac{t - t_i}{t_f-t_i} \right)^3 %, \quad t \in \{t_i, \dots, t_i + n \Delta t\}
\end{align*}
% for $t \in \{t_i, \dots, t_i + n \Delta t\}$, 
with $t_f=0.75 T$.
%
Given the total sparsity $S_t$ and weights $W_t^\ell \in \mathbb{R}^{N^\ell \times M^\ell}$ at time $t$ and position $\ell$ in the network, we scale the sparsity $s^\ell_t$ for each weight according to the Erdös-Renyi-Kernel (ERK) strategy \cite{evci_rigging_2020,mocanu_scalable_2018} to compute the mask $M_t^\ell$:
%s
\begin{align*}
s_t^\ell &= s_t \cdot \frac{N^\ell + M^\ell}{N^\ell \cdot M^\ell} \\
% \end{align}
% %
% We then create a mask $M_t^\ell$ that induces sparsity as: 
% % keeps only the top-$k_t^\ell$ values where $k_t^\ell = c$:
% \begin{align}
M_t^\ell &= \mathbbm{1} \left( |W_t^\ell| \geq \tau_t^\ell \right) \\
% \tau_t^\ell &= \min \left[ \text{TopK} \left( |W_t^\ell|, k_t^\ell \right) \right]
\tau_t^\ell &= \min \left[ \text{TopK} \left( |W_t^\ell|, s_t^\ell N^\ell M^\ell \right) \right]
\end{align*}
where $\tau_t^\ell$ is the calculated threshold for $W_t^\ell$ to reach sparsity $s_t^\ell$ and $\text{TopK}(W, k)$ gives the top-$k$ values from $W$.
In the forward pass, weights are masked as $\bar{W}=M\odot W$, while the backward pass applies straight-through estimation \cite{DBLP:journals/corr/BengioLC13} enabling gradient updates also for masked weights. 
% Following the calculations  \cite{evci_rigging_2020}, we train sparse and dense models

\paragraph{Activity sparsification}

Sparsifying layer activations provide another means for reducing the compute and on-chip memory requirements during inference.
In particular, sparse pre-activations of linear layers can significantly reduce the number of MACs required for the associated matrix-vector multiplication (MVM), if appropriately supported by the hardware backend.
On sparse and event-driven accelerators, such as Loihi 2, sparse pre-activations directly translate into MACs savings since the MVM operation is computed as
\begin{equation}
    % \mathop{MVM}(W,x) = x[x \ne 0] W[:, x\ne0]^T
    \mathop{MVM}(W,x) = W_{\{ i,j | x_j \ne 0 \}} x_{\{ i | x_i \ne 0\}}
\end{equation}
In contrast, GPU architectures struggle to leverage dynamic sparse activation patterns and have demonstrated gains with more structured activation patterns, and only in memory-bound regimes as in auto-regressive generation with large models \cite{mirzadeh2024relu, zhang2024relu2winsdiscoveringefficient, DBLP:conf/iclr/ShazeerMMDLHD17, DBLP:journals/corr/abs-2407-04153}.

Techniques for activation sparsity include top-k \cite{DBLP:journals/corr/abs-2412-04358}, sigma-delta coding \cite{shrestha2024efficient, o2016sigma}, sparse mixture-of-experts \cite{fedus_switch_2022,he_mixture_2024} and \emph{ReLU-fication} \cite{mirzadeh2024relu}.
We base our methodology on the latter of these. Since ReLU is a fully element-wise operation, it doesn't require synchronization across channels which would complicate implementation in compute-memory integrated platforms, such as Loihi 2.
Following previous work on transformer models \cite{mirzadeh2024relu}, we start from the original dense model with GELU non-linearity, as shown in \autoref{figure_3}, and apply two modifications.
First, we replace the GELU activation with a ReLU, sparsifying pre-activations of the linear layer in the GLU block.
Second, we insert additional ReLU activations after the residual add in the GLU block and to the real component of the S5 hidden layer, further increasing the pre-activation sparsity of linear operators.
Both model surgeries are applied to the pre-trained model at the beginning of the iterative pruning procedure, enabling accuracy recovery from both weight and activation pruning without extra training budget.


\paragraph{Quantization and fixed-point computation}

Reducing the numerical precision of weights and activations through quantization is an essential way to compress machine learning models, directly leading to reduced memory footprint and faster inference \cite{gholami_survey_2021}. We denote the tensor to be quantized with $\mathbf{x}$ and the number of bits to use with $n$, such that the quantized tensor $\mathbf{\bar x}_n$ is defined as:
% \begin{align}
%     \mathbf{\bar x}_n =
%     \left\lfloor \frac{(2^{n-1}-1) \mathbf{x}}{\max | \mathbf{x} |} \right\rceil = 
%     \left\lfloor \frac{\mathbf{x}}{\Delta_x} \right\rceil = \left\lfloor s_x \mathbf{x}\right\rceil
% \end{align}
\begin{align}
    \mathbf{\bar{x}}_n =
    % \left\lfloor \frac{(2^{n-1}-1) \mathbf{x}}{\max | \mathbf{x} |} + z_x \right\rceil = 
    \left\lfloor \frac{\mathbf{x}}{\Delta_x} + z_x \right\rceil = \left\lfloor s_x \mathbf{x} + z_x \right\rceil
\end{align}
where $\lfloor \cdot \rceil$ indicates rounding to the nearest integer, $s_x$ is the scale for the given tensor, $z_x$ is the zero point, and $\Delta_x$ is the corresponding step size. For simplicity, we choose $s_x = (2^{n-1}-1) (\max |\mathbf{x}|)^{-1}$ and $z_x = \mathbf{0}$, \textit{i.e.}, we use symmetric quantization based on the absolute maximum.

% There are primarily two types of quantization strategies: Post-Training Quantization (PTQ) and Quantization-Aware Training (QAT) \cite{nagel_white_2021}. 
Post-training quantization (PTQ) applies quantization to a pre-trained model without further training, which is computationally efficient but may lead to a notable drop in accuracy, especially for complex models or tasks \cite{gholami_survey_2021}. Without constraints during training, it has been shown to under-perform on both nonlinear \cite{wu_googles_2016} and linear RNNs \cite{abreu2024q}.
In contrast, quantization-aware training (QAT) incorporates quantization into the training process using straight-through estimators for the gradients \cite{DBLP:journals/corr/BengioLC13}, allowing the model to adapt to the reduced precision and typically achieving superior performance retention compared to PTQ \cite{hubara_quantized_2018}, which has also shown promising results on linear RNNs such as S4D \cite{meyer2024diagonal} and S5 \cite{abreu2024q} on synthetic tasks from the Long Range Arena benchmark \cite{DBLP:conf/iclr/Tay0ASBPRYRM21}.
%
To demonstrate advantages on hardware, we use static quantization \cite{gholami_survey_2021} using only fixed-point (integer) arithmetic \cite{wu_integer_2020}. Whereas in dynamic quantization, scales $s_x$ are computed dynamically on incoming data (and therefore requiring floating-point operations), static quantization pre-computes scales for all weights and activations in the neural network and ``freezes'' these scales so that the network can be converted to use only fixed-point arithmetic.

Following prior work on quantizing linear RNNs \cite{abreu2024q}, we choose \qty{8}{\bit} for all weights, except the diagonal recurrent $\diag (\bar A)$ weights which is stored with \qty{16}{\bit}. All activations are quantized to \qty{16}{\bit}. We denote this quantization recipe with W8A16. This is a more compressed quantization scheme than previous work that deployed a linear RNN to fixed-point hardware using W8A24 \cite{meyer2024diagonal}.
% 
% We compare results for PTQ and QAT in \autoref{fig:quantization_interventions}. 
For the linear RNNs that are deployed to the Loihi 2 chip, we combine QAT with sparse training. 
% For our implementation of QAT, we use the AQT library \cite{aqt}.% with JAX which slows down our neural network training by a factor of 2--3.


\subsection{Porting S5 to Loihi 2}

Running S5 on Loihi 2 requires a range of adjustments, to fully leverage the neuromorphic architecture and to adhere to its constraints. As a result, the S5 network shown in \hyperref[figure_3]{Figure \ref{figure_3}} is transformed into a network of synapses and neurons for Loihi 2 as illustrated in \hyperref[fig:loihi-implementation]{Figure \ref{fig:loihi-implementation}}.
In general, a state vector of dimension $\mathbb{R}^{M}$ is encoded by M neurons. Matrix-vector multiplications are hardware accelerated by the synaptic layers, which take a vector of neuron activities, multiply it with the matrix of synaptic weights, and pass the output to the next layer of neurons.
Since complex numbers are not natively supported on Loihi 2, the complex matrices $\bar{B}$ and $\bar{C}$ have been split into two synaptic layers each, representing the complex and real parts. Similarly, the complex state $x_k$ is stored by two neuronal states.
The remaining operations are performed within the assembly-programmable neurons.

A single layer of programmable neurons can efficiently fuse many operations on the vector it encodes. This applies to all element-wise operations where each neuron must operate only on its local states.
The neuronal layers thus implement ReLUs, BatchNorm, Hadamard products, residual add, and multiplications of a state vector with a diagonal matrix.
Applying this layer fusion, the full S5 architecture only requires one neuron group for the encoder, one for the decoder, and three for each S5 block. 
The detailed mapping of operations to neuron groups is illustrated in \autoref{fig:loihi-implementation},

\section{Theoretical Analysis}

\subsection{Convergence Analysis}
We begin by examining the two-grid iteration under standard multigrid assumptions, emphasizing how our approach handles both high- and low-frequency error components. In particular, a smoothing operator \(S\) damps high-frequency errors, while coarse-grid correction addresses low-frequency modes. This two-part strategy yields a convergence rate that does not degrade with increasing problem size.

\begin{theorem}[Two-Grid Convergence]
\label{th:twogrid_convergence}
Let \(\mathbf{e}^{(k)}\) be the error at iteration \(k\) of a two-grid scheme for the SPD system \(A\mathbf{x}=\mathbf{b}\). Suppose the coarse correction satisfies the \emph{Approximation Property} and the smoothing step remains stable. Then there exists a constant \(\rho < 1\) such that
\[
    \|\mathbf{e}^{(k+1)}\|_{a}
    \;\le\;
    \rho\,\|\mathbf{e}^{(k)}\|_{a},
\]
where \(\|\cdot\|_{a}\) is the energy norm induced by \(A\). Consequently, the iteration converges at a rate independent of the system size \(n\).
\end{theorem}

\noindent
(See Appendix~\ref{appendix:proof_2} for proof.) A key ingredient is the coarse space’s ability to capture smooth (low-frequency) errors. In classical multigrid, this is formalized by the \emph{Approximation Property}:

\begin{property}[Approximation Property]
\label{prop:approximation}
Let \(P\in\mathbb{R}^{n\times m}\) be the prolongation operator from the coarse space \(\mathbb{R}^m\) to the fine space \(\mathbb{R}^n\). For any error vector \(\mathbf{e}\in \mathbb{R}^n\), there exists a coarse representation \(\mathbf{z}^c \in \mathbb{R}^m\) such that
\[
    \min_{\mathbf{z}^c}
    \|\mathbf{e} - P\,\mathbf{z}^c\|_{a}
    \;\le\;
    \alpha\,\|\mathbf{e}\|_{a},
\]
with \(\alpha < 1\). (See Appendix~\ref{appendix:proof_4} for details.) 
\end{property}

\noindent
(See Appendix~\ref{appendix:proof_1} for proof.) Together, Theorem~\ref{th:twogrid_convergence} and the Approximation Property ensure uniform error reduction per iteration: smoothing removes high-frequency errors, while the coarse grid approximates low-frequency errors sufficiently well.

\subsection{Operator Properties}

Beyond two-grid convergence, another measure of preconditioner quality is its impact on the spectrum of \(MA\). If \(M\approx A^{-1}\), the eigenvalues of \(MA\) should lie near 1, yielding a small condition number and fast convergence for Krylov solvers like CG or GMRES.

\begin{theorem}[Preconditioned Spectrum Clustering]
\label{th:spectrum_clustering}
Let \(A \in \mathbb{R}^{n \times n}\) be SPD, and let \(M\) be a preconditioner satisfying the smoothing and coarse approximation assumptions of Theorem~\ref{th:twogrid_convergence}. Then there exist constants \(0 < \lambda_{\min} \le \lambda_{\max}\) close to 1 such that every eigenvalue \(\lambda\) of \(MA\) lies in the interval \([\lambda_{\min}, \lambda_{\max}]\). Hence, \(\kappa(MA)=\lambda_{\max}/\lambda_{\min} \) remains near 1, ensuring rapid convergence.
\end{theorem}

\noindent
(See Appendix~\ref{appendix:proof_3} for proof.) This spectral view complements the two-grid analysis by connecting eigenvalue clustering to reduced iteration counts. In a neural setting, the learned operators play a role analogous to restriction, prolongation, and smoothing, thereby preserving these spectral advantages.

Finally, we note that our Neural Algebraic Multigrid (NAMG) Operator can be interpreted as a learnable integral over the domain \(\Omega\), supporting adaptive feature extraction on coarse grids. Formally:

\begin{theorem}[NAMG Operator as a Learnable Integral]
\label{th:integral}
Given an input \(a:\Omega\to\mathbb{R}^d\) and a point \(\mathbf{x}\in\Omega\), the NAMG operator approximates
\[
    \mathcal{G}a(\mathbf{x}) 
    \;=\;
    \int_{\Omega} 
    \kappa(\mathbf{x}, \boldsymbol{\xi})\,a(\boldsymbol{\xi})
    \,d\boldsymbol{\xi},
\]
for some learnable kernel \(\kappa\). 
\end{theorem}

\noindent
(See Appendix~\ref{appendix:proof_4} for proof.) This perspective unifies the classical AMG principle of coarse-grid correction with a data-driven, integral-based formulation. It underscores the capacity of neural operators to adaptively handle diverse PDE structures, ultimately enhancing both convergence and generalization.




\section{Experiments}
\textbf{Setup.} We evaluate the performance of PINNMamba on three standard PDE benchmarks: convection, wave, and reaction equations, all of which are identified as being affected by failure modes~\cite{krishnapriyan2021characterizing,zhao2024pinnsformer}. The details of those PDEs can be found in Appendix~\ref{apx:setup}.
    We compare PINNMamba with four baseline models, vanilla PINN~\cite{raissi2019physics}, QRes~\cite{bu2021quadratic}, PINNsFormer~\cite{zhao2024pinnsformer}, and KAN~\cite{liu2024kan} .
For fair comparison, we sample 101$\times$101 collection points with uniformly grid sampling, following previous work~\cite{zhao2024pinnsformer,wu2024ropinn}. We also evaluate on PINNacle Benchmark~\cite{hao2023pinnacle} and Navier–Stokes equation~\cite{raissi2019physics}.

\begin{table*}
\vspace{-3mm}
  \caption{Results for solving convection, reaction, and wave equations.}
  \label{sample-table}
  
  \centering
    \small
  \begin{tabular}{l|c|ccc|ccc|ccc}

    \toprule 
  & & \multicolumn{3}{c}{Convection }&\multicolumn{3}{c}{Reaction}&\multicolumn{3}{c}{Wave}\\
    \cmidrule(lr){3-5}\cmidrule(lr){6-8}\cmidrule(lr){9-11}
   Model & \#Params &Loss & rMAE & rRMSE & Loss & rMAE & rRMSE& Loss & rMAE & rRMSE
 \\   \midrule
    PINN&527361& 0.0239 & 0.8514 & 0.8989& 0.1991 & 0.9803 & 0.9785& 0.0320 & 0.4101 & 0.4141\\
    QRes & 396545& 0.0798 & 0.9035 & 0.9245& 0.1991 & 0.9826 & 0.9830& 0.0987 & 0.5349 & 0.5265\\
    PINNsFormer &453561 & 0.0068 & 0.4527 & 0.5217& 3e-6& 0.0146 & 0.0296 & 0.0216 & 0.3559 & 0.3632\\
     KAN&891& 0.0250 & 0.6049 & 0.6587& 7e-6 & 0.0166 & 0.0343& 0.0067 & 0.1433 & 0.1458\\
   \rowcolor{mygray}   PINNMamba  & 285763&0.0001 & \textbf{0.0188} & \textbf{0.0201}&1e-6&\textbf{0.0094}&\textbf{0.0217}& 0.0002 & \textbf{0.0197} & \textbf{0.0199} \\

    \bottomrule
  \end{tabular}
  \normalsize
  \label{tab:diff}
  \vspace{-4mm}
\end{table*}

\begin{figure*}[t!]
    \centering
    \includegraphics[width=\textwidth]{_fig/wave}
    \vspace{-8mm}
    \caption{The ground truth solution, prediction (top), and absolute error (bottom) on wave equations.}
    \label{fig:wave}
    \vspace{-5mm}
  %  \vspace{-1mm}
\end{figure*}

\textbf{Training Details.} We train PINNMamba and all the baseline models 1000 epochs with L-BFGS optimizer~\cite{liu1989limited}.
We set the sub-sequence length to 7 for PINNMamba, and keep the original pseudo-sequence setup for PINNsFormers. The weights of loss terms $[\lambda_\mathcal F,\lambda_\mathcal I,\lambda_\mathcal B]$ are set to $[1,1,10]$ for all three equations, as we find that strengthening the boundary conditions can lead to better convergence. $\lambda_\text{alig}$ is set to 1000 for convection and reaction equations, and auto-adapted by $\lambda_\mathcal F$ for wave equation.
%Loss weights are also actively adapted by neural tangent kernel~\cite{wang2022and} for wave equations for test the orthogonality of PINNMamba with other methods.
All experiments are implemented in PyTorch 2.1.1 and trained on an NVIDIA H100 GPU.  More training details are in Appendix~\ref{apx:hyperparam}. Our code and weights are available at \url{https://github.com/miniHuiHui/PINNMamba}.

\textbf{Metrics.} To evaluate the performance of the models, we take relative Mean Absolute Error (rMAE, a.k.a  $\ell_1$ relative error) and relative Root Mean Square Error (rRMSE, a.k.a $\ell_2$ relative error) following common practive~\cite{zhao2024pinnsformer,wu2024ropinn}. The metrics are formulated as:
\begin{align}
\text { rMAE }(\hat u)&=\frac{\sum_{n=1}^N\left|\hat{u}\left(x_n, t_n\right)-u\left(x_n, t_n\right)\right|}{\sum_{n=1}^{N}\left|u\left(x_n, t_n\right)\right|}, \\
\text { rRMSE }(\hat u)&=\sqrt{\frac{\sum_{n=1}^N\left|\hat{u}\left(x_n, t_n\right)-u\left(x_n, t_n\right)\right|^2}{\sum_{n=1}^N\left|u\left(x_n, t_n\right)\right|^2}},
\end{align}
where N is the number of test points, $u(x,t)$ is the ground truth solution, and $\hat u(x,t)$ is the model's prediction.

\vspace{-2mm}

\subsection{Main Results}
\vspace{-1mm}
We present the rMAE and rRMSE for approximating convection, reaction and wave equation's solution in Table~\ref{tab:diff}. Our model consistently outperforms other model architectures, achieving new state-of-the-art.
Notably, as shown in Fig.~\ref{fig:conv}, for the convection equation, PINNMamba allows sufficient propagation of information about the initial conditions, whereas on all the other models there is a varying degree of distortion in the time coordinates.
    As shown in Fig.~\ref{fig:reac}, PINNMamba can further optimize at the boundary, resulting in a lower error than KAN and PINNsFormer for reaction equations. For problems as intrinsically difficult to optimize as the wave, as in Fig.~\ref{fig:wave}, PINNMamba effectively combats simplicity bias and aligns the scales of multi-order differentiation, and thus achieves significantly higher accuracy. This illustrates that PINNMamba can be effective against PINN's failure modes. It's also worth noting that, PINNMamba has the lowest number of parameters (except KAN), while achieving consistently the best performance.

\begin{table}
\vspace{-3mm}
  \caption{Integrating PINNMamba with advanced training strategies and loss auto-balancing strategy. The rMAE is reported here.}
  
  \centering
    \small
  \begin{tabular}{lccc}

    \toprule 
    Method & Convection & Reaction & Wave\\
   \midrule
   PINNMamba & 0.0188 & 0.0094 & 0.0197\\
   +gPINN & 0.0172& 0.0123 & 0.0264 \\
   +vPINN & 0.0236 & 0.0092& 0.0169\\
   +RoPINN & 0.0102& 0.0099& 0.0121\\
    \midrule
    +NTK &0.0179& 0.0079& 0.0147\\
    +NTK+RoPINN &0.0127& 0.0072& 0.0106\\
   

    \bottomrule
  \end{tabular}
  \normalsize
  \label{tab:para}
  \vspace{-6mm}
\end{table}

\begin{figure*}[t!]
    \centering
    \includegraphics[width=\textwidth]{_fig/reac}
    \vspace{-8mm}
    \caption{The ground truth solution, prediction (top), and absolute error (bottom) on reaction equations.}
    \label{fig:reac}
    \vspace{-5mm}
  %  \vspace{-1mm}
\end{figure*}


\subsection{Combination with Other Methods}
\vspace{-1mm}
Since PINNMamba mainly focuses on model architecture, it can be integrated with other methods effortlessly. 
    We explore the feasibility and their performance in combination with advanced training paradigm, as well as loss balancing.

\textbf{Training Paradigm.} We show the rMAE of PINNMamba when integrated with advanced strategies in Table~\ref{tab:para}. We observe that gPINN~\cite{yu2022gradient} and vPINN~\cite{kharazmi2019variational} erratically deliver some performance gains on some tasks. 
    This is due to the fact that the regularization provided by gPINN and vPINN in the form of a loss function through the gradient and variational residuals has little effect on PINNMamba, since SSM itself is sufficiently regularized. RoPINN~\cite{wu2024ropinn} reduces the PINNMamba's error on convection and wave equations by about 40\%, since it complements the spatial continuity dependency.

\textbf{Neural Tangent Kernel.} Dynamic tuning of losses via Neural Tangent Kernel(NTK)~\cite{wang2022and} has been shown to have the effect of smoothing out the loss landscape. 
PINNMamba also works well with the NTK-adopted loss function. As shown in Table~\ref{tab:para}, NTK can reduce PINNMamba error by 5-25\%. 
The combination of RoPINN and NTK can further improve the overall performance of PINNMamba, which demonstrates the excellent suitability of PINNMamba with other PINN optimization methods.

\begin{figure}[t!]
    \centering
    \includegraphics[width=\linewidth]{_fig/loss_error}
    \vspace{-4mm}
    \caption{Loss and $\ell_1$-Error Curve w.r.t Training Iteration.}
    \label{fig:losserror}
    \vspace{-4mm}
  %  \vspace{-1mm}
\end{figure}
\vspace{-2mm}
\subsection{Loss-Error Consistency Analysis}
\vspace{-1mm}

Our other interest is the role of PINNMamba for the elimination of simplicity bias. Models affected by simplicity bias that fall into over-smoothing solutions will show inconsistent decreasing trends in loss and error during training. 
    As shown in Fig.~\ref{fig:losserror}, in the training process for solving convection equations, the rMAE of PINN doesn't descend as $\mathcal L_\mathcal F$ and $\mathcal L_\mathcal I$. 
        This suggests that PINN is trapped in an over-smoothing solution, which is in agreement with our observation in Fig.~\ref{fig:conv}. 
As a comparison, we find that PINNMamba's losses descent processes show a high degree of consistency with its error descent process. 
    This indicates that PINNMamba does not tend to fall into a local optimum of oversimplified patterns.
        Instead, it tends to exhibit patterns that are consistent with the original PDEs.

\vspace{-2mm}
\subsection{Ablation Study}
\vspace{-1mm}
\begin{table*}
  [t]
  \centering
  \resizebox{\textwidth}{!}{%
  \begin{tabular}{cccccccccccc}
    \toprule \multicolumn{2}{c}{Components}                                                             & \multicolumn{5}{c}{Re-executability Rate (\%)} & \multicolumn{5}{c}{Readability (\#)} \\
    \cmidrule(lr){1-2} \cmidrule(lr){3-7} \cmidrule(lr){8-12}        \hspace{8pt}\labelemoji\hspace{8pt}                                                                & \hspace{8pt}\toolemoji\hspace{8pt}                                      & O0                                 & O1             & O2             & O3             & AVG            & O0             & O1             & O2             & O3             & AVG            \\
    \hline
    \rowcolor[rgb]{0.93,0.93,0.93}\multicolumn{12}{c}{\textbf{Initialize with LLM4Decompile-End-6.7B~\citep{llm4decompile}}}   \\
    \xmark                                                                                              & \xmark                                    & 69.51                              & 46.95          & 50.61          & 46.34          & 53.35          & 3.98 & 3.41 & 3.44 & 3.38 & 3.55 \\
    \cmark                                                                                              & \xmark                                    & 75.61                              & 50.61          & 50.00          & 50.00          & 56.55          & 4.01 & 3.44 & 3.39 & \textbf{3.49} & 3.58 \\
    \xmark                                                                                              & \cmark                                    & 83.54                     & \textbf{56.10}          & 51.22          & 50.61 & 60.37 & 4.05 & 3.51 & 3.51 & 3.42 & 3.62 \\
    \cmark                                                                                              & \cmark                                    & \textbf{85.37}                            & \textbf{56.10}                     & \textbf{51.83} & \textbf{52.43}          & \textbf{61.43} & \textbf{4.13} & \textbf{3.60} & \textbf{3.54} & \textbf{3.49} & \textbf{3.69} \\

    \rowcolor[rgb]{0.93,0.93,0.93}\multicolumn{12}{c}{\textbf{Initialize with Deepseek-Coder-6.7B-base~\citep{deepseekcoder}}} \\
    \xmark                                                                                              & \xmark                                    & 59.15                              & 35.98          & 39.02          & 37.80          & 42.99          & 3.71 & 3.05 & 3.16 & 3.05 & 3.24 \\
    \cmark                                                                                              & \xmark                                    & 66.46                              & 41.46          & 38.41          & 36.59          & 45.73          & 3.76 & 3.17 & \textbf{3.21} & 3.08 & 3.31 \\
    \xmark                                                                                              & \cmark                                    & 70.73                              & 39.63          & 39.02          & 40.24          & 47.41          & 3.90 & 3.17 & 3.08 & 3.11 & 3.31 \\
    \cmark                                                                                              & \cmark                                    & \textbf{79.88}                     & \textbf{45.73} & \textbf{43.90} & \textbf{42.68} & \textbf{53.05} & \textbf{3.96} & \textbf{3.21} & 3.18 & \textbf{3.19} & \textbf{3.38} \\
    \bottomrule
  \end{tabular}%
  }
  \caption{The ablation study of different methods across four optimization levels
  (O0, O1, O2, O3), as well as their average scores (AVG). The results in bold represent the optimal performance. The ~\labelemoji~ and ~\toolemoji~ means Relabedling and Function Call. \textbf{Bold} denotes the best performance.}
  \label{tab:ablation}
\end{table*}

To verify the validity of the various components of the PINNMamba, as shown in Table~\ref{tab:ablation}, we evaluate the performance of models subtracting these components from PINNMamba.

\textbf{Sub-Sequence.} We remove the sub-sequence alignment, which leads to a decrease in model performance, indicating the significance of the agreement formed through alignment in eliminating simplicity bias.
After replacing the sub-sequence with a long sequence of the entire domain, the model shows failure modes, in line with the sequence granularity analysis in Section~\ref{sec:subseq}.

\textbf{Time-Varying SSM.} We replace the selective SSM~\cite{gu2023mamba} with a linear time-invariant structure SSM~\cite{gu2022efficiently}, and there is some decrease in model performance, illustrating the role of predictive diversity in eliminating simplicity bias. 
And when we remove SSM completely and switch to MLP instead, the model has severe failure modes. 
        This demonstrates that SSM's adaptation for \textit{Continuous-Discrete Mismatch} allows the initial condition information to propagate sufficiently in time coordinates.

In addition, we also conducted a sensitivity analysis of the choice of sub-sequence length, activation. See Appendix~\ref{apx:sense}.

\vspace{-3mm}
\subsection{Experiments on Complex Problems}
\vspace{-1mm}
To further demonstrate the generalization of our method, we tested our model on partial PINNacle Benchmark~\cite{hao2023pinnacle} and Navier-Stokes equations. As shown in Fig.~\ref{fig:ns}, PINNMamba achieves the lowest error on the N-S equation. Just like PINNsFormer, PINNMamba also gets out-of-memory on some problems in PINNacle, which we identify as a major limitation of sequence-based methods. We discuss the details of PINNacle experiments in Appendix~\ref{apx:comp}.

\begin{figure}[t!]
    \centering
    \includegraphics[width=\linewidth]{_fig/NS}
    \vspace{-6mm}
    \caption{Absolute Error of pressure prediction of N-S equation}
    \label{fig:ns}
    \vspace{-3mm}
  %  \vspace{-1mm}
\end{figure}


\section{Limitations and Future Work}
The proposed OpenFly platform incorporates various rendering engines/techniques to provide high-quality scenes. Specifically, this is the first attempt to use 3D GS reconstructed scenes to support real-to-sim training and testing, while in the reconstruction of large-scale areas, a few visual artifacts are inevitably present. Future work will focus on exploring more effective reconstruction methods to enhance realism in large-scale scenes. Besides, the proposed OpenFly-Agent is built upon the large VLN model architecture, which is not practical for real-time deployment on UAVs. To address this, future research should focus on developing more efficient architectures and effective quantization techniques. 


\section{Conclusion}
In this work, we present OpenFly, a platform designed for large-scale data collection in aerial Vision-and-Language Navigation (VLN). OpenFly integrates multiple rendering engines and advanced real-to-sim techniques for data generation, enabling efficient collection of diverse, high-quality aerial VLN data. The resulting large-scale dataset comprises 100k trajectories across 18 distinct scenes, spanning a wide range of altitudes and difficulty levels, which is significantly superior than existing ones. Furthermore, we propose OpenFly-Agent, a keyframe-aware aerial navigation model capable of directly predicting flight actions based on observations and language instructions. Extensive experiments validate the effectiveness of the proposed method, and establishing a comprehensive benchmark for future advancements in aerial navigation. 
%The toolchain, dataset, and code will be publicly released, providing a valuable resource for future research in this field.

%%%%%%%%%%%%%%%%%%%%%%%%%%%%%%%%%%%%%%%%%%%%%%%%%%%%%%%%%%%%

\bibliographystyle{plainnat}
\bibliography{reference}

\appendix

\section{Related Work}
\label{appendix:relate}
\subsection{Numerical Preconditioner}

Preconditioning is a well-established numerical technique for accelerating the convergence of iterative solvers applied to large linear systems. It involves applying a matrix transformation to reduce the condition number of the system matrix. Here, we briefly review three key approaches in numerical preconditioning: matrix factorization, matrix reordering, and multilevel methods.

Matrix factorization-based preconditioners are among the most widely used. Simple methods like the Jacobi preconditioner only use the diagonal elements of the matrix, which are fast to compute but provide limited improvement in condition number. More sophisticated approaches, such as the incomplete Cholesky (IC) preconditioner~\cite{06:Numerical}, offer a balance between computational cost and accuracy by partially factorizing the matrix while limiting fill-in. Advances in this area, including dynamic fill-in strategies~\cite{12:Numerical}, improve accuracy but at a higher computational cost.

Matrix reordering techniques~\cite{Liu_book, davis1999modifying} aim to reduce the bandwidth of sparse matrices by reorganizing their structure into block-diagonal forms. This restructuring minimizes fill-in during factorization and enhances parallelization, making it particularly beneficial when combined with other preconditioning methods. Our approach leverages graph-based representations to naturally maintain order invariance and parallelizability.

Multilevel methods, such as the classical multigrid approach~\cite{00:tutorial}, improve scalability by addressing errors at different levels of discretization. These methods are particularly effective for elliptic PDEs, though they face challenges with hyperbolic and parabolic PDEs~\cite{TrotMult2001}. Our Neural Preconditioning Operator (NPO) differs by using data-driven neural operators to approximate the inverse system matrix without being tailored to specific PDE types. Research has also explored hybrid approaches combining multilevel and neural strategies to further enhance solver efficiency~\cite{chen_2021_icsiggraph}.

\subsection{Neural Preconditoner}

Recent advances have explored neural networks for preconditioning linear systems derived from PDEs. Unlike classical preconditioners, neural approaches adaptively improve solver performance by learning data-driven representations of the inverse operator.

Early methods aimed to guarantee convergence through neural approximations of PDE solvers \cite{19:LearningNeural}. Machine learning techniques have also been applied to geophysical fluid dynamics, demonstrating the effectiveness of neural preconditioners in large-scale simulations \cite{20:MachineLearned}. Further extensions hybridize operator learning with traditional relaxation methods for enhanced scalability and accuracy \cite{22:AHybrid}.

Neural networks have been used to accelerate solvers in physics domains, such as lattice gauge theory, by reducing the iteration count in large sparse systems \cite{22:NeuralNetwork}. Preconditioners specifically optimized for conjugate gradient (CG) solvers were introduced by \cite{23:LearningPre}, achieving faster convergence through learned adaptations to PDE structures.

Recent work leverages neural operator frameworks, such as DeepONet and Fourier Neural Operators (FNO), to enhance preconditioning strategies \cite{24:DeepOnet}. Transformer-based architectures have also been explored, incorporating multigrid principles to refine both fine- and coarse-grid error correction \cite{24:MultigridAugmented,24:fcg-no}.

Moreover, specialized applications in porous microstructures and Helmholtz equations highlight how machine-learned preconditioners can integrate compact implicit layers for improved error control and spectral properties \cite{24:Machine}. Collectively, these developments point to the growing importance of neural preconditioners in accelerating PDE solvers across diverse physical and computational settings.


\section{Krylov Subspace Methods}
\label{sec:krylov}
Krylov subspace methods are a class of iterative solvers designed for large-scale linear systems of the form
\begin{equation}
    A\mathbf{x} = \mathbf{b},
\end{equation}
where \(A \in \mathbb{R}^{n \times n}\) is a sparse matrix, \(\mathbf{x}\in \mathbb{R}^n\) is the solution vector, and \(\mathbf{b}\in \mathbb{R}^n\) is the right-hand side (RHS) vector. Starting with an initial guess \(\mathbf{x}_0\), the residual vector is defined as
\begin{equation}
    \mathbf{r}_0 = \mathbf{b} - A\mathbf{x}_0.
\end{equation}
At each iteration, Krylov methods construct a solution within the Krylov subspace, defined by:
\begin{equation}
    \mathcal{K}_m(A, \mathbf{r}_0) = \text{span}\{\mathbf{r}_0,\, A\mathbf{r}_0,\, A^2\mathbf{r}_0,\, \dots,\, A^{m-1}\mathbf{r}_0\}.
\end{equation}

These methods aim to find an approximate solution \(\mathbf{x}_m \in \mathcal{K}_m\) that minimizes the residual norm \(\|\mathbf{b} - A\mathbf{x}_m\|\). Two widely used Krylov methods are Conjugate Gradient (CG) and Generalized Minimal Residual (GMRES).

\paragraph{Conjugate Gradient (CG).} CG is specialized for symmetric positive definite (SPD) matrices. It constructs a series of orthogonal search directions \(\{\mathbf{p}_k\}\) such that each iterate \(\mathbf{x}_{k+1}\) minimizes the quadratic form
\begin{equation}
    \|\mathbf{b} - A\mathbf{x}_{k+1}\|_A^2 = (\mathbf{x}_{k+1} - \mathbf{x})^\top A (\mathbf{x}_{k+1} - \mathbf{x}),
\end{equation}
where \(\| \cdot \|_A\) denotes the \(A\)-norm. CG converges rapidly for well-conditioned systems, often within a number of iterations proportional to the square root of the condition number of \(A\).

\paragraph{Generalized Minimal Residual (GMRES).} GMRES is designed for general non-symmetric matrices. It iteratively constructs an orthonormal basis of the Krylov subspace using the Arnoldi process. At each step, GMRES finds \(\mathbf{x}_m\) that minimizes the residual norm in the Euclidean sense:
\begin{equation}
    \mathbf{x}_m = \arg\min_{\mathbf{x} \in \mathcal{K}_m} \|\mathbf{b} - A\mathbf{x}\|.
\end{equation}
Since the orthonormal basis grows with each iteration, GMRES requires restarts to control memory usage and computational cost. Despite this, it is effective for systems with complex eigenvalue structures.

Both methods benefit significantly from preconditioning, which transforms the system into one with a more favorable spectrum, thereby accelerating convergence.


\section{Proofs of Theorem and Property}
\subsection{Proof of Property \ref{prop:approximation}} \label{appendix:proof_1}
\begin{proof}[Proof of Property~\ref{prop:approximation}]
Assume \(A\in \mathbb{R}^{n\times n}\) is symmetric positive-definite (SPD), and let \(\|\mathbf{v}\|_{a}^2 = \mathbf{v}^\top A\,\mathbf{v}\). In classical multigrid, one typically constructs \(P\) so that each “smooth” (low-frequency) error in \(\mathbb{R}^n\) lies close to the range of \(P\). Concretely:

Often, the residual or error vector \(\mathbf{e}\) is relatively smooth if it has passed through a smoothing step (e.g., Gauss–Seidel). In finite-element or finite-difference contexts, “smooth” means \(\mathbf{e}\) varies slowly across elements or grid points.

Define \(\mathbf{z}^c = R\,\mathbf{e}\), where \(R\) is often taken as \(P^\top\) (for SPD problems) or a similar restriction operator. Then \(\mathbf{e} - P\,\mathbf{z}^c = \mathbf{e} - P\,R\,\mathbf{e}\). By design, \(P\) and \(R\) capture low-frequency components of \(\mathbf{e}\) well.

One shows
\[
    \|\mathbf{e} - P\,R\,\mathbf{e}\|_{a}
    \;\le\;
    \alpha\,\|\mathbf{e}\|_{a},
\]
for some \(\alpha < 1\), relying on local interpolation or stable decomposition arguments. Essentially, \(P\,R\) acts like a “best fit” in a coarse subspace spanned by columns of \(P\).


Since \(\|\mathbf{e} - P\,\mathbf{z}^c\|_{a}\) achieves the same bound by choosing \(\mathbf{z}^c = R\,\mathbf{e}\), it follows that
\[
    \min_{\mathbf{z}^c}\,\|\mathbf{e} - P\,\mathbf{z}^c\|_{a}
    \;\le\;
    \|\mathbf{e} - P\,R\,\mathbf{e}\|_{a}
    \;\le\;
    \alpha\,\|\mathbf{e}\|_{a}.
\]

Hence, the constructed prolongation \(P\) ensures that any smooth error \(\mathbf{e}\) can be approximated to within a factor \(\alpha\) in the \(\|\cdot\|_{a}\)-norm by some coarse representation \(\mathbf{z}^c\). This property is crucial for two-grid and multigrid convergence theory, as it guarantees low-frequency error components are effectively handled on the coarse grid.
\end{proof}


\subsection{Proof of Theorem \ref{th:twogrid_convergence}} \label{appendix:proof_2}
\begin{proof}[Proof of Theorem~\ref{th:twogrid_convergence}]

In a two-grid iteration, the error \(\mathbf{e}^{(k)}\) is first \emph{smoothed} using a relaxation method (e.g., Gauss--Seidel). This smoothing operator, denoted by \(S\), substantially reduces high-frequency components of the error. After smoothing, the dominant error components in \(\mathbf{e}^{(k)}\) lie in lower-frequency ranges.

Next, the residual \(\mathbf{r}^{(k)} = \mathbf{b} - A \mathbf{x}^{(k)}\) is transferred to a coarse space via a restriction operator \(R\). We solve or approximate the system on the coarse grid, then prolong the coarse correction back to the fine grid with a prolongation operator \(P\). This step primarily targets low-frequency components of the error.

By the \emph{Approximation Property} (Property~\ref{prop:approximation}), the coarse space captures smooth (low-frequency) errors up to a factor \(\alpha < 1\). Concretely, we can write
\[
    \|\mathbf{e}^{(k)} - P\,R\,\mathbf{e}^{(k)}\|_{a}
    \;\le\;
    \alpha \,\|\mathbf{e}^{(k)}\|_{a}.
\]
Combining the smoothing effect for high-frequency errors with the coarse-grid correction for low-frequency errors yields a uniform reduction of the entire error \(\mathbf{e}^{(k)}\).

Let \(\widetilde{\mathbf{e}}^{(k)}\) be the error after smoothing, and \(\widehat{\mathbf{e}}^{(k)}\) be the error after coarse correction. We have:
\[
    \|\widetilde{\mathbf{e}}^{(k)}\|_{a} 
    \;\le\; 
    \nu \,\|\mathbf{e}^{(k)}\|_{a}
    \quad
    \text{(smoothing bound for high-frequency errors)},
\]
for some \(\nu < 1\). Then, applying the coarse correction and using the Approximation Property for low-frequency errors,
\[
    \|\widehat{\mathbf{e}}^{(k)}\|_{a} 
    \;\le\;
    \alpha \,\|\widetilde{\mathbf{e}}^{(k)}\|_{a}
    \;\le\;
    \alpha\,\nu \,\|\mathbf{e}^{(k)}\|_{a}.
\]
Thus, if \(\rho = \alpha\,\nu\), we get
\[
    \|\mathbf{e}^{(k+1)}\|_{a}
    \;=\;
    \|\widehat{\mathbf{e}}^{(k)}\|_{a}
    \;\le\;
    \rho\, \|\mathbf{e}^{(k)}\|_{a},
\]
and \(\rho < 1\).

Since both \(\nu\) (smoothing factor) and \(\alpha\) (coarse approximation factor) do not depend on the number of degrees of freedom \(n\), the convergence rate \(\rho\) remains below 1 \emph{independently of} \(n\). Consequently, each two-grid cycle contracts the error by at least a factor \(\rho\), implying a convergence rate that is uniform with respect to problem size.

By combining stable smoothing (which tackles high-frequency errors) with an effective coarse space approximation (which addresses low-frequency errors), the two-grid algorithm achieves a uniform reduction in the energy norm at each iteration, completing the proof.
\end{proof}


\subsection{Proof of Theorem \ref{th:spectrum_clustering}} \label{appendix:proof_3}
\begin{proof}[Proof of Theorem~\ref{th:spectrum_clustering}]
Let \(M\) be a preconditioner satisfying the two essential multigrid conditions:
\begin{itemize}
    \item \textbf{Smoothing Property:} A smoothing operator \(S\) reduces high-frequency error components effectively.
    \item \textbf{Coarse Approximation (Approximation Property):} A coarse space captures low-frequency errors up to a bounded factor.
\end{itemize}

Any error vector \(\mathbf{e}\) can be split into high-frequency and low-frequency parts. The smoothing property guarantees a uniform reduction of high-frequency modes, while the coarse approximation ensures that low-frequency errors are corrected by the coarse-grid solution.

Because \(A\) is symmetric positive-definite (SPD), we have \(\mathbf{v}^\top A\,\mathbf{v} > 0\) for all nonzero \(\mathbf{v}\). By design, \(M\approx A^{-1}\) in the sense that high-frequency components are rapidly damped and low-frequency components are accurately corrected. Thus, when we consider the generalized eigenvalue problem
\[
    MA\,\mathbf{x} = \lambda \mathbf{x},
\]
the spectrum of \(MA\) must lie within an interval \([\lambda_{\min}, \lambda_{\max}]\) around 1, provided the smoothing and coarse-grid conditions hold. 

Standard multigrid analysis (see \cite{00:tutorial}) shows that these two-grid assumptions induce a tight cluster of eigenvalues around 1. In particular, repeated smoothing and accurate coarse-grid corrections force the effective operator \(MA\) to act almost like the identity, i.e., \(MA \approx I\). This implies that every eigenvalue \(\lambda\) of \(MA\) is close to 1, say \( \lambda_{\min} \le \lambda \le \lambda_{\max} \), with both \(\lambda_{\min}, \lambda_{\max}\) near 1.

Since 
\[
    \kappa(MA) = \frac{\lambda_{\max}}{\lambda_{\min}},
\]
the close proximity of \(\lambda_{\min}\) and \(\lambda_{\max}\) to 1 ensures that \(\kappa(MA)\approx 1\). This near-identity condition number leads to rapid convergence in Krylov methods (such as CG and GMRES), which require fewer iterations when eigenvalues are well clustered.

Hence, under the smoothing and coarse approximation assumptions, \(\lambda_{\min}, \lambda_{\max}\) lie near 1, yielding a small condition number \(\kappa(MA)\) and guaranteeing that Krylov solvers converge rapidly.
\end{proof}


\subsection{Proof of Theorem \ref{th:integral}} \label{appendix:proof_4}
Recall that the NAMG operator applies a graph-based attention mechanism to approximate an integral over the spatial domain \(\Omega\). The theorem is established by demonstrating that the graph attention mechanism employed in the NAMG operator can be formalized as a Monte-Carlo approximation of an integral operator \cite{21:Choose, 23:no, 24:Transolver,24:AMG}.

\begin{proof}[Proof of Theorem~\ref{th:integral}]
Let \(a:\Omega \to \mathbb{R}^d\) be an input function, and let \(\mathbf{x}\in \Omega\subset\mathbb{R}^d\). Our goal is to show that
\[
    \mathcal{G}a(\mathbf{x})
    \;=\;
    \int_{\Omega} \kappa(\mathbf{x}, \boldsymbol{\xi})\,a(\boldsymbol{\xi})
    \,d\boldsymbol{\xi}
\]
can be approximated by the NAMG attention update.

Define the kernel \(\kappa(\mathbf{x},\boldsymbol{\xi})\) to measure similarity between \(\mathbf{x}\) and \(\boldsymbol{\xi}\). In a continuous setting, \(\mathcal{G}a(\mathbf{x})\) integrates over all \(\boldsymbol{\xi}\in\Omega\).

We discretize \(\Omega\) into a mesh or graph with nodes \(\{\mathbf{x}_i\}\). Each node \(\mathbf{x}_i\) represents a sample in \(\Omega\). The adjacency structure \(A\) (or neighborhood set \(\mathcal{N}(\mathbf{x})\)) reflects local connectivity.

Attention weights 
\[
    \alpha_{i} 
    \;\propto\; 
    \exp\!\Bigl(\mathbf{a}^\top [\mathbf{W}a(\mathbf{x}) \;\|\; \mathbf{W}a(\mathbf{x}_i)]\Bigr)
\]
approximate the continuous kernel \(\kappa(\mathbf{x}, \mathbf{x}_i)\). Normalizing by the softmax denominator
\[
    \sum_{j \in \mathcal{N}(\mathbf{x})}
    \exp\!\Bigl(\mathbf{a}^\top [\mathbf{W}a(\mathbf{x}) \;\|\; \mathbf{W}a(\mathbf{x}_j)]\Bigr)
\]
yields a discrete approximation to the integral \(\int_{\Omega} \kappa(\mathbf{x}, \boldsymbol{\xi})\,d\boldsymbol{\xi}\). Thus, summing over neighbors \(\mathbf{x}_i \in \mathcal{N}(\mathbf{x})\) corresponds to sampling from \(\Omega\).

Replacing \(\kappa(\mathbf{x}, \boldsymbol{\xi})\) with the above attention weights and summing over \(\mathcal{N}(\mathbf{x})\), we obtain
\[
    \mathcal{G}a(\mathbf{x})
    \;\approx\;
    \sum_{i \in \mathcal{N}(\mathbf{x})} 
    \alpha_{i} \,\mathbf{W}a(\mathbf{x}_i),
\]
which matches the NAMG attention update rule. Hence, the NAMG operator realizes a Monte Carlo approximation to \(\int_{\Omega} \kappa(\mathbf{x}, \boldsymbol{\xi})\,a(\boldsymbol{\xi})\,d\boldsymbol{\xi}\), where \(\alpha_i\) and \(\mathbf{W}\) are learned parameters.

Therefore, the attention-based aggregation in NAMG acts as a learnable integral operator over \(\Omega\), completing the proof.
\end{proof}



\section{Model Details}
\label{appendix:detail}
\setlength{\tabcolsep}{3pt}
\begin{table}
\centering
\caption{Efficiency of Different Methods}
\vspace{-0.1in}
\label{tab:efficiency}
% \resizebox{\textwidth}{!}{%
\scalebox{0.65}{%
\begin{tabular}{|c|ccc|ccc|ccc|}
  \hline
  \multirow{2}{*}{} & \multicolumn{3}{c|}{\textbf{memory size (MB)}} & \multicolumn{3}{c|}{\textbf{training time (s)}} & \multicolumn{3}{c|}{\textbf{matching time (s)}}\\
  % \cline{2-7}
  % {} & MByte & seconds/ep & seconds\\
  {} & \textbf{Beijing} & \textbf{Porto} & \textbf{Chengdu} & \textbf{Beijing} & \textbf{Porto} & \textbf{Chengdu} & \textbf{Beijing} & \textbf{Porto} & \textbf{Chengdu}\\
  % \cline{2-7}
  % {} & (MByte) & (minutes/ep) & (seconds/K) & (MByte) & (minutes/ep) & (seconds/K)\\
  \hline
  \textbf{MDP} & 1819MB & 2039MB & 2122MB & - & - & - & 389.14s & 361.15s & 599.51s  \\
  \textbf{HMM} & 1209MB & 1388MB & 1361MB & - & - & - & 427.97s & 380.05s & 589.08s \\
  % \hline
  \textbf{FMM} & 897MB & 931MB & 981MB & - & - & - & 1.13s & 1.02s & 1.87s \\
  % \hline
  \textbf{AMM} & 957MB & 1013MB & 1124MB & - & - & - & 3.42s & 3.05s & 5.16s \\
  % \hline
  \textbf{MTrajRec} & 9045MB & 12428MB & 11265MB & 182.4s & 2200.2s & 25672.4s & 51.22s & 42.27s & 73.68s\\
  % \hline
  \textbf{L2MM} & 9087MB & 11875MB & 10898MB & 189.1s & 2314.2s & 27032.2s & 6.71s & 5.26s & 9.10s\\
  % \hline
  \textbf{GraphMM} & 8537MB & 11752MB & 10378MB & 48.4s & 645.2s & 7311.4s & 8.06s & 6.96s & 11.18s\\
  % \hline
  \textbf{\modelName} & 2530MB & 2299MB & 2357MB & 11.9s & 126.4s & 1507.8s & 1.09s & 0.95s & 1.65s\\
  \hline
\end{tabular}}
\vspace{-0.15in}
\end{table}

\subsection{Efficiency Analysis}
\label{subsec:efficiency}

Efficiency in computational models is crucial, especially when dealing with large-scale problems such as those encountered in solving PDEs. Here, we focus on the efficiency metrics for the Neural Preconditioning Operator (NPO), comparing it against other models based on parameter size, GPU memory usage, and execution time during the training phase. A detailed comparison is provided in Table~\ref{table:eff}.

NPO demonstrates exceptional efficiency across various matrix sizes, maintaining a low parameter size (0.14 MB) and consistent execution times (around 0.0121 to 0.0123 seconds), regardless of the input scale. Notably, despite the increasing matrix sizes from 512 to 4096, NPO's GPU memory usage grows predictably without disproportionate spikes, which is crucial for scalable applications.

In contrast, other models such as M2NO and U-Net show significant increases in GPU memory demands and slower execution times as matrix sizes grow. M2NO, for instance, uses up to 5.55 million MB of GPU memory for the largest matrix size, with longer running times that reach up to 0.0814 seconds. This reflects a substantial computational overhead compared to NPO.

FNO and MLP, while smaller in parameter size, do not match NPO in terms of balancing the execution speed and memory efficiency at higher matrix dimensions. FNO offers the fastest execution times among the competitors but does not provide the robustness in feature representation that NPO does.

Overall, NPO not only excels in handling larger matrices efficiently but also showcases a balanced profile in terms of memory usage and computational speed, making it particularly suitable for real-world applications where both accuracy and efficiency are paramount.

\subsection{Model Configurations}
The primary configurations for Nerual Preconditioning Operator (NPO) are detailed in Table \ref{table:config}. Except where specifically noted, model parameters remain consistent across different hyperparameters and resolutions within the same benchmark.

\usepackage[utf8]{inputenc} % allow utf-8 input
\usepackage[T1]{fontenc}    % use 8-bit T1 fonts
\usepackage{microtype,inconsolata}
\usepackage{times,latexsym}
\usepackage{graphicx} \graphicspath{{figures/}}
\usepackage{amsmath,amssymb,mathabx,mathtools,amsthm,nicefrac}
\usepackage[linesnumbered,ruled,vlined]{algorithm2e}
\usepackage{acronym}
\usepackage{enumitem}
\usepackage[pagebackref,breaklinks,colorlinks]{hyperref}
\usepackage{balance}
\usepackage{xspace}
\usepackage{setspace}
\usepackage[skip=3pt,font=small]{subcaption}
\usepackage[skip=3pt,font=small]{caption}
\usepackage[capitalise,noabbrev,nameinlink]{cleveref}
\usepackage{booktabs,tabularx,colortbl,multirow,multicol,array,makecell,tabularray}
\usepackage{overpic,wrapfig}
\usepackage{dblfloatfix}
\usepackage[misc]{ifsym}
\usepackage{pifont}
\usepackage{fancyvrb}

% Add a period to the end of an abbreviation unless there's one
% already, then \xspace.
\makeatletter
\DeclareRobustCommand\onedot{\futurelet\@let@token\@onedot}
\def\@onedot{\ifx\@let@token.\else.\null\fi\xspace}

\def\eg{\emph{e.g}\onedot} \def\Eg{\emph{E.g}\onedot}
\def\ie{\emph{i.e}\onedot} \def\Ie{\emph{I.e}\onedot}
\def\cf{\emph{c.f}\onedot} \def\Cf{\emph{C.f}\onedot}
\def\etc{\emph{etc}\onedot} \def\vs{\emph{vs}\onedot}
\def\wrt{w.r.t\onedot} \def\dof{d.o.f\onedot}
\def\etal{\emph{et al}\onedot}

\makeatother

\acrodef{sota}[SOTA]{State-of-the-Art}
\acrodef{method}[\textsc{PRA}]{Preference-based Robot Assistant}
\acrodef{pbp}[\textsc{PbP}]{Preference-based Planning}
\acrodef{vln}[VLN]{Vision-and-Language Navigation}
\acrodef{llm}[LLM]{Large Language Model}
\acrodef{EILEV}[EILEV]{Efficient In-context Learning on Egocentric Videos}
\acrodef{vlm}[VLM]{Vision-Language Model}
\acrodef{vivit}[ViViT]{Video Vision Transformer}
\acrodef{llava}[LLaVA]{Large Language and Vision Assistant}
\acrodef{ai}[AI]{Artificial Intelligence}
\acrodef{ik}[IK]{Inverse Kinematics}
\acrodef{ompl}[OMPL]{Open Motion Planning Library}
\acrodef{sem}[SEM]{Structural Equation Model}

% Spacing
% \medmuskip=2mu   % reduce spacing around binary operators
% \thickmuskip=3mu % reduce spacing around relational operators
\setlength{\abovedisplayskip}{3pt}
\setlength{\belowdisplayskip}{3pt}
\setlength{\abovecaptionskip}{3pt}
\setlength{\belowcaptionskip}{3pt}
% \setlength\floatsep{1\baselineskip plus 3pt minus 2pt}
% \setlength\textfloatsep{1\baselineskip plus 3pt minus 2pt}
% \setlength\dbltextfloatsep{1\baselineskip plus 3pt minus 2pt}
% \setlength\intextsep{1\baselineskip plus 3pt minus 2pt}

\newcolumntype{x}{>{\columncolor{LightCyan1}}c}
\newcolumntype{y}{>{\columncolor{MistyRose}}c}

\end{document}