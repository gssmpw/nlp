\begin{figure}[t]
\centering
    \pgfplotstableread[row sep=\\,col sep=&]{
        idx     & S5Hid & S5Out & GLU   \\
        1       & 80.754554271698  & 49.60111975669861  & 71.89104557037354   \\
        2       & 58.92143249511719  & 29.815730452537537  & 82.44596123695374  \\
        3       & 51.13644599914551  & 34.60754454135895  & 62.65120506286621  \\
    }\mydata

    \pgfplotstableread[row sep=\\,col sep=&]{
        idx     & S5Hid & S5Out & GLU   \\
        1       & 81.3236653804779  & 39.38424289226532   & 60.10604500770569   \\
        2       & 54.60154414176941  & 19.661784172058105  & 70.28757929801941  \\
        3       & 45.402026176452637  & 22.966817021369934  & 47.028326988220215  \\
    }\mydatasparse

    \begin{tikzpicture}
        \begin{axis}[
                ybar,
                bar width=.28cm,
                width=\linewidth,
                height=0.8\linewidth,
                legend style={
                    at={(0,1)},
                    anchor=north west,
                    legend columns=3,
                },
                xtick=data,
                xticklabels={Layer 1, Layer 2, Layer 3},
                xmin=0.5, xmax=3.5,
                ymax=100,
                ylabel={Pre-activation Sparsity (\%)},
                xlabel={Model Depth},
                ytick pos=left,
                grid=both,
                xmajorgrids=false,
                minor grid style={dashed},
                major grid style={dashed},
              tick label style={font=\normalsize},
              label style={font=\normalsize},
              every axis legend/.append style={font=\small},
            ]
            % Define colors for consistency between data and sparse data.
            % First three plots (non-sparse data) without patterns

            \addlegendimage{area legend, fill=mint}
            \addlegendentry{Norm}
            \addlegendimage{area legend, fill=pear}
            \addlegendentry{S5 Out}
            \addlegendimage{area legend, fill=orange}
            \addlegendentry{GLU}
            \addlegendimage{area legend, fill=black}
            \addlegendentry{Dense}
            \addlegendimage{area legend, fill=black, postaction={pattern=crosshatch, pattern color=white}}
            \addlegendentry{Sparse}
            
            \addplot+[fill=mint, draw=mint] table[x=idx,y=S5Hid] {\mydata};
            \addplot+[fill=pear, draw=pear] table[x=idx,y=S5Out] {\mydata};
            \addplot+[fill=orange, draw=orange] table[x=idx,y=GLU] {\mydata};
            
            % Now add the sparse data with the same colors, but with patterns.
            \addplot+[fill=mint, draw=mint, postaction={pattern=crosshatch, pattern color=white}] table[x=idx,y=S5Hid] {\mydatasparse};
            \addplot+[fill=pear, draw=pear, postaction={pattern=crosshatch, pattern color=white}] table[x=idx,y=S5Out] {\mydatasparse};
            \addplot+[fill=orange, draw=orange, postaction={pattern=crosshatch, pattern color=white}] table[x=idx,y=GLU] {\mydatasparse};

        \end{axis}
    \end{tikzpicture}
    \caption{Activation sparsity of ReLU blocks across model depth for a dense-weight model and a sparse-weight model. The sparse-weight model exhibits significantly lower activation sparsity across layers.}
    \label{fig:activation_sparsity}
\end{figure}