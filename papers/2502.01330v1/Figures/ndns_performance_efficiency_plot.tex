\begin{figure*}[ht]
\centering
\begin{minipage}{.49\textwidth}
\begin{tikzpicture}
\begin{axis}[
  width=\textwidth,
  height=0.8\textwidth,
  %xmode=log,
  grid=both,
  minor grid style={dashed},
tick label style={/pgf/number format/fixed},
  major grid style={dashed},
  xlabel={Effective MACs ($\times 10^5$)},
  ylabel={Test SI-SNR (\unit{\decibel})},
  legend style={
    at={(1.0,0.0)},   % x,y coordinates relative to the axis
    anchor=south east,
    legend columns=1
  },
  ymax=15.75,
  ymin=12.75,
  tick label style={font=\normalsize},
  label style={font=\normalsize},
  every axis legend/.append style={font=\small},
  xmin=-200,
xtick={0,100000,200000,300000,400000,500000,600000},
xticklabels={{$0$},{$1$},{$2$},{$3$},{$4$},{$5$},{$6$}},
  scaled x ticks=false, 
  xmax=600000
]

    % 2) ReLU, Sparse (blue, dashed, round marker)
    \addplot+[
      forget plot,
      color=mint,
      solid,
      thick,
      mark=*,
      mark options={fill=mint, draw=mint},
      nodes near coords,
      point meta=explicit symbolic,
      every node near coord/.append style={anchor=south, font=\small}
    ] table [meta=label, col sep=space] {
      x      y      label
 15385.396493  13.147347          2
 28589.361895  13.907744          3
 45140.258309 14.220416          4
 65621.998638  14.567575          5
 90050.190476  14.738816          6
118193.413502  14.910501          7
147635.113026  15.040557          8
183166.524831 15.155087          9
221680.248892 15.187635         10
263277.181804  15.251074         11
304559.862373 15.268791         12
    };


    \addplot [
      black,
      dashed,
      thick,
      forget plot
    ] coordinates {(8000,15.2) (9000000,15.2)} 
      node [pos=1, anchor=north east, font=\small] {Previous SotA};


    % 3) GeLU, Dense (red, solid, round marker)
    \addplot+[
      forget plot,
      color=orange,
      solid,
      thick,
      mark=*,
      mark options={fill=orange, draw=orange},
      nodes near coords,
      point meta=explicit symbolic,
      every node near coord/.append style={anchor=north, font=\small}
    ] table [meta=label, col sep=space] {
      x      y      label
  50688.0  13.333696          1
 152064.0  14.277531          2
 304128.0 14.997617          3
 506880.0 15.309213          4
    };

    
    % Weights: Dense (solid black line)
    \addlegendimage{orange, mark=*, solid, thick}
    \addlegendentry{Dense w/ GELU}

    % Weights: Sparse (dashed black line)
    \addlegendimage{mint, mark=*, solid, thick}
    \addlegendentry{Sparse w/ ReLU}

  % Now draw the big red arrows (example):
  %  -- Arrow from orange #1 (x=50688, y=13.3337) to orange #3 (x=304128, y=14.9976)
  \draw[->, thick, black!50]
    (axis cs:294128, 15.0)
    -- (axis cs:157635.113026,  15.0) ;

  %  -- Arrow from orange #3 (x=304128, y=14.9976) to green #9 (x=183166.5248, y=15.1551)
  %     (This arrow angles 'back' to the left, as in your figure.)
  \draw[->,thick, black!50]
    (axis cs:294128, 15.0)
    -- (axis cs:38589.361895,  13.907744);
  \end{axis}
\end{tikzpicture}
\end{minipage}
\begin{minipage}{.49\textwidth}
\begin{tikzpicture}
\begin{axis}[
  width=\textwidth,
  height=0.8\textwidth,
  %xmode=log,
  grid=both,
  minor grid style={dashed},
  major grid style={dashed},
  xlabel={Memory Footprint (\unit{\mega\byte})},
  ylabel={Test SI-SNR (\unit{\decibel})},
  legend style={
    at={(1,0)},   % x,y coordinates relative to the axis
    anchor=south east,
    legend columns=1
  },
  ymax=15.75,
  ymin=12.75,
  xmin=0,xmax=2,
  tick label style={font=\normalsize},
  label style={font=\normalsize},
  every axis legend/.append style={font=\small},
]

    \addplot [
      black,
      dashed,
      thick,
      forget plot
    ] coordinates {(0,15.2) (2,15.2)} 
      node [pos=1, anchor=north east, font=\small] {};

    % 4) GeLU, Sparse (red, dashed, round marker)
    \addplot+[
      forget plot,
      color=orange,
      solid,
      thick,
      mark=*,
      mark options={fill=orange, draw=orange},
      nodes near coords,
      point meta=explicit symbolic,
      every node near coord/.append style={anchor=north, font=\small}
    ] table [meta=label, col sep=space] {
      x      y      label
0.193909 13.333696          1
0.581177 14.277531          2
1.161804 14.997617          3
1.935791 15.309213          4
    };

    % 4) GeLU, Sparse (red, dashed, round marker)
    \addplot+[
      forget plot,
      color=mint,
      solid,
      thick,
      mark=*,
      mark options={fill=mint, draw=mint},
      nodes near coords,
      point meta=explicit symbolic,
      every node near coord/.append style={anchor=south, font=\small}
    ] table [meta=label, col sep=space] {
      x      y      label
0.070136 13.147347 2
0.133800 13.907744 3
0.216820 14.220416 4
0.319128 14.567575 5
0.440823 14.738816 6
0.581836 14.910501 7
0.742147 15.040557 8
0.921845 15.155087 9
1.120834 15.187635 10
1.339239 15.251074 11
1.576909 15.268791 12
    };

    % -----------------------------------------------------------
    % Manual legend entries

    % Weights: Dense (solid black line)
    \addlegendimage{orange, mark=*, solid, thick}
    \addlegendentry{Dense}

    % Weights: Sparse (dashed black line)
    \addlegendimage{mint, mark=*, solid, thick}
    \addlegendentry{Sparse}

  \draw[->, thick, black!50]
    (axis cs:1.121804, 15.0)
    -- (axis cs:0.782147,  15.0) ;

  %  -- Arrow from orange #3 (x=304128, y=14.9976) to green #9 (x=183166.5248, y=15.1551)
  %     (This arrow angles 'back' to the left, as in your figure.)
  \draw[->,thick, black!50]
    (axis cs:1.121804, 15.0)
    -- (axis cs:0.173800,  13.907744);
  \end{axis}
\end{tikzpicture}
\end{minipage}
\caption{Pareto fronts for S5 network audio denoising quality (SI-SNR) as a function of effective compute (left) and memory footprint (right) on the Intel N-DNS test set. S5 networks with  weight and activation sparsity (green) exhibit a large domain of Pareto optimality versus dense S5 networks (orange). Number annotations enumerate increasing S5 dimensionality configurations, from \qty{500}{k} to \qty{4}{\million} parameters. Dashed horizontal like marks SI-SNR of Spiking-FullSubNet XL, the previous state-of-the-art model. The horizontal arrows highlight models used for hardware deployment, the diagonal arrows highlight models of the same width. See text for details.}
\label{fig:ndns_performance_efficiency}
\end{figure*}