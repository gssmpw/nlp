
\begin{figure*}[ht!]
    \centering
    \includegraphics[width=0.98\linewidth]{Figures/Figure_Loihi2_Diagram.pdf}
    \caption{Diagram of S5 as implemented on Loihi 2. To leverage the neuromorphic hardware architecture, several adjustments are made in comparison to the original S5 model shown in \hyperref[figure_3]{Figure \ref{figure_3}}: First, complex numbers are split into real and complex components for processing. Second, ReLUs are introduced to increase activation sparsity. Third, multiple element-wise operations are fused into single neuromorphic neurons. Symbols are shown as defined in \Cref{ss:linear-rnns}.}% by equations \eqref{eq:x_k}-\eqref{eq:glu}.}
    \label{fig:loihi-implementation}
\end{figure*}


\subsection{Effective MACs computation for S5 architecture}
\label{supp:macs}

In this section, we detail the computation of effective multiply-accumulate operations (MACs) for different components of the S5 architecture. The total MAC count provides an estimate of the computational cost associated with each stage of the model. 
Below, we outline the individual contributions from key components of the architecture. The effective MACs for all model sizes--sparse and dense--in Figure \autoref{fig:ndns_performance_efficiency} are calculated based on the formulas below, summed over the entire network structure.

\paragraph{Notation:}
\begin{itemize}
    \item $N_\text{input}$: Input dimension
    \item $N_\text{model}$: Model dimension for activations outside of the linear RNN. 
    \item $N_\text{ssm}$: Dimension of the linear RNN's hidden state. 
    \item $N_\text{output}$: Output dimension (equal to the number of classes for classification)
    \item $d^\text{wgt}_x$: Density of weights for $x$
    \item $d^\text{act}_x$: Density of activations for $x$
\end{itemize}

where the density $d$ is calculated from the sparsity $s$ as $d=1-s$.

\paragraph{Breakdown of MAC Calculation per Component:}
\begin{itemize}
    \item \textbf{Encoder:} The MACs for the encoder depend on the input dimension, model size, and scale linearly with activation and weight densities:
        \begin{equation}    
            N_\text{input} N_\text{model} d^\text{wgt}_\text{encoder} d^\text{act}_\text{input}
        \end{equation}
    \item \textbf{Batch Normalization (BatchNorm):} A lightweight operation, requiring only element-wise scaling, leading to:
        \begin{equation}
            N_\text{model}
        \end{equation}
    \item \textbf{S5 Hidden Layer:} The hidden state update for the S5 model involves both matrix multiplications and element-wise operations:
        \begin{equation}
            2N_\text{model} N_\text{ssm} d^\text{wgt}_B d^\text{act}_\text{pre\_ssm} + 4 N_\text{ssm}
        \end{equation}
    \item \textbf{SSM Output Layer:} Computes the output transformation of the linear RNN:
        \begin{equation}
            2N_\text{ssm} N_\text{model} d^\text{wgt}_C d^\text{act}_\text{hidden} + N_\text{model} d^\text{act}_\text{pre\_ssm}
        \end{equation}
    \item \textbf{Gated Linear Unit (GLU):} The computation for the GLU involves matrix multiplications for the dense weight matrix, followed by an element-wise multiplication:
        \begin{equation}
            N_\text{model}^2 d^\text{wgt}_\text{GLU} d^\text{act}_\text{pre\_GLU} + N_\text{model}
        \end{equation}
    \item \textbf{Classification Head:} The final linear projection for classification:
        \begin{equation}
            N_\text{model} N_\text{output} d^\text{wgt}_\text{head} d^\text{act}_\text{pre\_hread}
        \end{equation}
    \item \textbf{Regression Head:} The regression head follows the same computation as the classification head:
        \begin{equation}
            N_\text{model} N_\text{output} d^\text{wgt}_\text{head} d^\text{act}_\text{pre\_hread}
        \end{equation}
\end{itemize}

Numerical operations such as the inverse square-root, sigmoid function, and others, are ignored from our MAC calculations, as is commonly done when calculating the MACs or floating point operations (FLOPs) of machine learning models \cite{evci_rigging_2020}.

\subsection{Experimental Details}

\paragraph{Model architecture}
\label{app:model-params}

Our linear RNN is based on the S5 architecture \cite{DBLP:conf/iclr/SmithWL23}, as described in Section \Cref{ss:linear-rnns}. We use the following dimensions for our base model with width scaling $k=1$ (\textit{i.e.} configuration \qty{4}{} in \autoref{fig:ndns_performance_efficiency}). We use three layers, the recurrent state vector is $\mathbf{x}_t \in \mathbb{R}^{256}$, we use a model dimension of $192$. 
Both input and output have dimension $257$.
The width scaling factors $k_i$ scale the model and recurrent state dimension linearly. In \autoref{fig:ndns_performance_efficiency}, we report results for a $k$-family of sparse and densely trained networks where $k_\text{sparse} \in [0.5, 3.0], \ k_\text{dense} \in [0.25, 1.0]$.

\paragraph{Training recipe}
We trained all models for \qty{50}{} epochs with the Adam optimizer. The parameters of the SSM block were updated with initial learning rate $0.002$, while the rest of the architecture used initial learning rate $0.008$ and weight decay $0.04$. All learning rates used cosine annealing and no warmup epochs.
The dropout was set to $0.1$.


\subsection{Additional Results}

\subsubsection{Loihi execution mode}
\label{app:exmode}

Loihi 2’s asynchronous architecture allows to trade off between throughput and latency, as illustrated in \hyperref[fig:time_per_step]{Figure \ref{fig:time_per_step}a}. For optimal throughput, new input is provided every time step and forwarded through the neuronal layers in a pipelined mode. For optimal latency, new input is injected only once the previous input has been processed by, or fallen through, the network as fast as possible. The pipelined and fall-through mode can be balanced by changing the rate of new input, to match the throughput of a given input stream while minimizing its processing latency.

As audio denoising is typically deployed in realtime in an online fashion where one STFT input frame in processed at a time, fall-through mode is appropriate, as one desires a corresponding output STFT frame immediately.

We see that Loihi 2 processes a single STFT frame $35\times$ faster and with $1200\times$ less energy than the Jetson Orin Nano (Token-by-token; Loihi 2 Fall-Through and Jetson Orin Nano Recurrent 1-step (b=1) in \autoref{tab:pnp}). 

\begin{figure*}[ht]
\centering
\begin{minipage}{.49\textwidth}
\centering
\includegraphics[width=0.85\linewidth]{Figures/Figure_Modes.pdf}
\end{minipage}
\begin{minipage}{.49\textwidth}
\begin{tikzpicture}
\begin{axis}[
  width=\textwidth,
  height=0.8\textwidth,
  %xmode=log,
  grid=both,
  minor grid style={dashed},
  major grid style={dashed},
  xlabel={Step},
  ylabel={Time per Step (\unit{\micro\s})},
  legend style={
    at={(0,1)},   % x,y coordinates relative to the axis
    anchor=north west,
    legend columns=2
  },
  ymin=0,ymax=25,
  xmin=6,xmax=53,
  tick label style={font=\normalsize},
  label style={font=\normalsize},
  every axis legend/.append style={font=\small},
]

    % Fall-Through Mode
    \addplot+[
      forget plot,
      color=mint,
      solid,
      thick,mark=*,draw=mint,
      mark options={fill=mint, draw=mint},
    ] table [col sep=space] {
      x      y      
6 7.537552520632744
7 2.541228991933167
8 6.932773109525442
9 4.447478991001844
10 10.412552521564066
11 8.726890755817294
12 5.434611344709992
13 9.592699579894543
14 5.230567227117717
15 7.532037815079093
16 4.033876050263643
17 1.366596638225019
18 1.4235819326713681
19 1.4364495798945427
20 1.438287815079093
21 1.4364495798945427
22 7.673581932671368
23 2.612920168787241
24 7.493434873409569
25 4.063287815079093
26 10.364758403971791
27 8.884978991001844
28 5.234243697486818
29 9.737920167855918
30 5.579831933602691
31 7.524684873409569
32 3.9768907567486167
33 1.366596638225019
34 1.4217436974868178
35 1.4364495798945427
36 1.438287815079093
37 1.4364495798945427
38 7.695640756748617
39 2.601890755817294
40 7.103728991933167
41 4.6386554623022676
42 10.344537815079093
43 9.085346638225019
44 5.609243697486818
45 9.991596639156342
46 5.449317226186395
47 7.350052521564066
48 4.079831932671368
49 1.366596638225019
50 1.4272584039717913
51 1.4364495798945427
52 1.4364495789632201
53 1.4364495798945427
    };

    % 4) GeLU, Sparse (red, dashed, round marker)
    \addplot+[
      forget plot,
      color=orange,
      solid,
      thick,mark=*,draw=orange,
      mark options={fill=orange, draw=orange},
    ] table [col sep=space] {
      x      y     
6 16.673392433905974
7 16.48657924705185
8 16.843722104211338
9 16.653245913796127
10 16.35654261696618
11 16.536029796581715
12 16.884015144547448
13 16.71734847780317
14 16.84738510788884
15 16.50489426543936
16 16.30526056571398
17 16.4627697232645
18 16.34738510788884
19 16.18254994309973
20 16.314418074907735
21 16.21002247056458
22 16.33273309317883
23 16.264967525494285
24 16.334564595017582
25 16.341890602489002
26 16.197201957693323
27 16.28145104192663
28 16.523209283710457
29 16.541524301981553
30 16.395004155579954
31 16.499399759923108
32 16.5140517745167
33 16.50489426543936
34 16.53786129830405
35 16.48657924705185
36 16.699033459532075
37 16.93529719579965
38 16.600132360705175
39 16.77961954008788
40 16.671560932067223
41 16.543355803820305
42 16.536029796581715
43 16.860205620760098
44 16.404161664657295
45 16.530535291065462
46 16.543355803820305
47 16.50306276360061
48 16.479253239696845
49 16.5140517745167
50 16.334564595133997
51 16.41331917385105
52 16.301597562036477
53 16.688044448615983
    };

    % -----------------------------------------------------------
    % Manual legend entries

    % Weights: Dense (solid black line)
    \addlegendimage{mint, solid, thick,mark=*}
    \addlegendentry{Fall-Through}

    % Weights: Sparse (dashed black line)
    \addlegendimage{orange, solid, thick,mark=*}
    \addlegendentry{Pipelined}

  \end{axis}
\end{tikzpicture}
\end{minipage}
\caption{(a) Loihi 2 offers two processing modes that optimize either throughput or latency. In the \textit{pipelined mode}, a new data point is inserted in each time step, to use all processing cores and maximize the throughput--at the expense of latency because equal time bins $t_0=t_1=\ldots$ are enforced. In the \textit{fall-through mode}, a new data points is only provided once the last data point has been fully processed with minimum latency. Only a single neuronal layer is active at any step as data travels through the network. The time per step is thus minimized as traffic is reduced and potentially more complex neuronal layers are not updated. (b) Comparison of execution mode and time per step.}
\label{fig:time_per_step}
\end{figure*}

\subsubsection{Fixed-point model mismatch}
\label{appendix:fxp-sim-mismatch}

The mismatch in \autoref{fig:quantization_interventions} indicates that fixed-point implementation in JAX does not perfectly match the original FP32 model when using the scales computed through our static quantization step. Further investigations show that the mismatch between hidden activations is highest for the hidden states $\mathbf{x}_k$ of the linear RNN and its outputs $\mathbf{y}_k$, see \autoref{fig:app_fxp_mismatch}. 
This mismatch increases approximately linearly with model depth, indicating that quantization errors accumulate as information propagates through the network layers. This linear escalation of errors underscores a critical challenge in fixed-point quantization of recurrent models \cite{wu_googles_2016,abreu2024q,li_quantization_2021,pierro2024mamba}. Consequently, ensuring the fidelity of deeper Linear RNNs on fixed-point neuromorphic hardware may require advanced quantization techniques or error mitigation strategies to preserve the network's temporal dynamics and memory capacity effectively.

\begin{figure*}
    \centering
    \includegraphics[width=0.8\linewidth]{Figures/app_fxp_mismatch.png}
    \caption{Layer-wise analysis of mismatch between the fixed-point model in JAX against the base model using floating-point weights and activations. The \textbf{left} and \textbf{right} side show the same data with a linear y-axis and log y-axis, respectively. The \textbf{top} panels show the mean absolute error $N^{-1}\sum_i^N|x_i-x_i'|$ for all components of the model while the \textbf{bottom} panels show the mean relative error $N^{-1}\sum_{\{i \ | \ i \in \{0, \ldots, N\} \ \wedge \ x_i \neq 0\}}^N|x_i-x_i'|/|x_i|$. For further explanation, see text.}
    \label{fig:app_fxp_mismatch}
\end{figure*}