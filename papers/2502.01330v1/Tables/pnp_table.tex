\begin{table*}
    \centering
    \caption{Power and performance results$^*$. The Loihi 2 is running a sparse and quantized S5 model, while the Jetson Orin Nano is running a smaller dense S5 model that reaches similar test performance. All measurements are averaged over \qty{8}{} random samples from the test set, each containing \qty{3750}{} time steps. \textcolor{gray}{Gray highlights} denote violation of real-time constraints for the audio denoising task. Best real-time results are \underline{underlined}.}
    \begin{tabular}{l c r r r}
        \toprule
        & \textbf{Mode}  
        & \multicolumn{1}{c}{\textbf{Latency} ($\downarrow$)} 
        & \multicolumn{1}{c}{\textbf{Energy} ($\downarrow$)}
        & \multicolumn{1}{c}{\textbf{Throughput} ($\uparrow$)} \\
        \midrule
        \textbf{Token-by-token}  \\
        \quad Intel Loihi 2$^\dagger$ & Fall-Through              &       \underline{\qty{76}{\micro\second}} &    \underline{\qty{13}{\micro\joule/\token}} &    \underline{\qty{13178}{\token/\second}} \\
        \quad Jetson Orin Nano$^\ddagger$ & Recurrent 1-step $(b=1)$ &     \qty{2688}{\micro\second} &  \qty{15724}{\micro\joule/\token} &  \qty{372}{\token/\second} \\
        \quad Jetson Orin Nano$^\ddagger$ & Recurrent 10-step $(b=1)$ &    \qty{3224}{\micro\second} &  \qty{1936}{\micro\joule/\token} &   \qty{3103}{\token/\second} \\
        \quad Jetson Orin Nano$^\ddagger$ & Recurrent 100-step $(b=1)$ &   \textcolor{gray}{\qty{10653}{\micro\second}} & \qty{626}{\micro\joule/\token} &   \qty{9516}{\token/\second} \\
        \quad Jetson Orin Nano$^\ddagger$ & Recurrent scan $(b=1)$ &       \textcolor{gray}{\qty{236717}{\micro\second}}& \qty{404}{\micro\joule/\token} &   \qty{15845}{\token/\second} \\
        \midrule
        \textbf{Sample-by-sample} \\
        \quad Intel Loihi 2$^\dagger$ & Pipeline &                        \underline{\qty{60.58}{\milli\second}} &   \underline{\qty{185.80}{\milli\joule/\sample}} &   \underline{\qty{16.58}{\sample/\second}} \\
        \quad Jetson Orin Nano$^\ddagger$ & Scan $(b=1)$ &                                \qty{233.48}{\milli\second} &           \qty{1512.60}{\milli\joule/\sample}& \qty{4.28}{\sample/\second} \\
        \quad Jetson Orin Nano$^\ddagger$ & Scan \textcolor{gray}{$(b=b_{\text{max}})$} & \textit{\qty{226.53}{\milli\second}} &  \textit{\qty{5.89}{\milli\joule/\sample}} &  \textit{\qty{1130.09}{\sample/\second}} \\
        \bottomrule
    \end{tabular}
\centering
% \vskip 0.01em 
\begin{minipage}{.9\textwidth}{\tiny \baselineskip=8pt \setstretch{0.6}
%
$^\dagger$ Loihi 2 workloads were characterized on an Oheo Gulch system with N3C1-revision Loihi 2 chips running NxCore 2.5.8 and NxKernel 0.2.0 with on-chip IO unthrottled sequencing of inputs. Researchers interested to run S5 on Loihi 2 can gain access to the software and systems by joining \textit{Intel's Neuromorphic Research Community}.
%
$^\ddagger$ Jetson workloads were characterized on an NVIDIA Jetson Orin Nano 8GB running Jetpack 6.2, CUDA 12.4, JAX 0.4.32, using the MAXN SUPER power mode; energy values are computed based on the TOT power as reported by jtop 4.3.0. The batch size $b_{\text{max}}=256$ was chosen to be the largest that fits into memory.
%
$^*$Performance results are based on testing as of January 2025 and may not reflect all publicly available security updates; results may vary.
}
\end{minipage}
    \label{tab:pnp}
\end{table*}
