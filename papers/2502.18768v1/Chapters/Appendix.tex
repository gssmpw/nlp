%\vspace{-4mm}
\section{Proof of Theorem $\ref{Theorem H_1}$}
The conditions stated in Theorem \ref{Theorem H_1} indicate Lemma \ref{Lemma MATI} holds. 
%
In the proof, the following variables and functions are used.
%
Let 
\begin{gather}
    \mu \in (0, a_s a_{\psi_s}^2), \qquad \mu_1 \in (0, \mu),
    \label{eqn: mu}
\end{gather}
where $a_s$ and $a_{\psi_s}$ come from Lemma \ref{Lemma MATI} and Assumption \ref{Assumption Extra 2}, respectively.
%
Let
\begin{equation}
    \lambda \in (\exp (-\mu_1\tau_{\text{miati}}^s ), 1),
    \label{eqn: lambda}
\end{equation}
where $\tau_{\text{miati}}^{s}$ comes from Theorem \ref{Theorem H_1}. 
We define 
\begin{equation}
    d \coloneqq \tfrac{-b+\sqrt{b^2-4ac}}{2a},
    \label{eqn: d}
\end{equation}
where $a = \tfrac{\lambda_1}{\gamma_s \lambda_s^*}$, $b= \tfrac{1}{2}( \tfrac{\lambda_1}{\gamma_s \lambda_s^*} + \lambda_2)$ and $c = 1 - \lambda e^{\mu_1 \tau_{\text{miati}}^s}$. Note that by definition of $\lambda$ in \eqref{eqn: lambda}, we have $c < 0$, which implies $b^2 - 4 a c > b> 0$.







Next we define the bound when we change between $x-z$ and $x-y$ coordinate.
Since the map $\overline{H}$ between $(x,e_s)$ and the quasi-steady state is continuously differentiable and $0 =\overline{H}(0, 0)$, by \cite[Lemma 4.3]{nonlinear_systems_Khalil}, there exists a class $\mathcal{K}$ function $\zeta_1$ such that $|\overline{H}(x,e_s)| \leq \zeta_1(|(x,e_s)|)$ for all $x \in \mathbb{R}^{n_x}, e_s \in \mathbb{R}^{n_{e_s}}$. 
%
We define a class $\mathcal{K}_{\infty}$ function 
\begin{equation}
    \zeta_2(s) \coloneqq s + \zeta_1(s).
\end{equation}
 Let $ \mathcal{E}^y \coloneqq \{ \xi^y \in \mathbb{X}\ | \ x=0 \wedge e_s = 0 \wedge y=0 \wedge e_f = 0 \}$, and $\mathcal{E}$ be defined above Theorem \ref{Theorem H_1}. As
$
|\xi^y|_{\mathcal{E}^y} \! = \! \left| \! \big(x,e_s,y + \overline{H}(x,e_s) - \overline{H}(x,e_s),e_f \big)\! \right|
%
\leq   \left| \! \big(x,e_s,y + \overline{H}(x,e_s),e_f \big)\! \right| + \left|\! \big(0,0,-\overline{H}(x,e_s),0\big)\! \right|
%
\leq  |\xi|_{\mathcal{E}} + \zeta_1(|\xi|_{\mathcal{E}})
%
= \zeta_2(|\xi|_{\mathcal{E}})
$,
% \begin{equation*}
%     \begin{aligned}
%         |\xi^y|_{\mathcal{E}^y} \! &= \! \left| \! \big(x,e_s,y + \overline{H}(x,e_s) - \overline{H}(x,e_s),e_f \big)\! \right|   \\
%             &\leq   \left| \! \big(x,e_s,y + \overline{H}(x,e_s),e_f \big)\! \right| + \left|\! \big(0,0,-\overline{H}(x,e_s),0\big)\! \right| \\
%             &\leq  |\xi|_{\mathcal{E}} + \zeta_1(|\xi|_{\mathcal{E}}) \\
%             &= \zeta_2(|\xi|_{\mathcal{E}}),
%     \end{aligned}
% \end{equation*}
we have for any $\Delta>0$, there exists a $\Delta^y = \zeta_2(\Delta)$ such that $|\xi|_{\mathcal{E}} \leq \Delta$ implies $|\xi^y|_{\mathcal{E}^y} \leq \Delta^y$. Similarly, we can show  $|\xi|_{\mathcal{E}} \leq \zeta_2(|\xi^y|_{\mathcal{E}^y})$. 
Let $\nu^y \coloneqq \zeta_2^{-1}(\nu)$, then if $|\xi^y|_{\mathcal{E}^y} \leq \nu^y$, we have $|\xi|_{\mathcal{E}}\leq \zeta_2(|\xi^y|_{\mathcal{E}^y}) \leq \zeta_2(\nu^y) = \nu$.

For given $d$ coming from \eqref{eqn: d}, as well as $U_s$ and $U_f$ coming from \eqref{eqn: Us and Uf}, we define a composite Lyapunov function 
\begin{equation}
    U(\xi^y)\coloneqq {U_s}(\xi_s) + d{U_f}(\xi_s,\xi_f).
    \label{eqn: U}
\end{equation}
%
From (\ref{eqn: Us sandwich bound}) and (\ref{eqn: Uf sandwich bound}), we know there exist $\underline{\alpha}_U, \overline{\alpha}_U \in \mathcal{K}_{\infty}$, such that for all $\xi^y \in \mathcal{C}_1^{y,\epsilon} \cup \mathcal{D}_s^{y,\epsilon} \cup \mathcal{D}_f^{y,\epsilon} $, 
    \begin{equation}
        \underline{\alpha}_U\left( |\xi^y|_{\mathcal{E}^y} \right) \leq U(\xi^y) \leq \overline{\alpha}_U\left( |\xi^y|_{\mathcal{E}^y} \right),
        \label{eqn: No disturbance U sandwich bound}
    \end{equation} 
where the sets are defined in Sections 3 and 4.    
%where $ \mathcal{E}^y \coloneqq \{ \xi^y \in \mathbb{X}\ | \ x=0 \wedge e_s = 0 \wedge y=0 \wedge e_f = 0 \} .$










We define the $\nu_1$ and $\Delta_1$ in Assumption \ref{Assumption interconnection} as follows. 
Let
\begin{equation}
    \nu_1 \coloneqq \tfrac{\mu - \mu_1}{2 a_d}\underline{\alpha}_U(\zeta_2^{-1}(\nu)),
    \label{eqn: nu_1}
\end{equation}
where 
\begin{equation}
    a_d \coloneqq 1 + \tfrac{\lambda_1}{\gamma_s \lambda_s^*} d + \tfrac{1}{2}\left(\tfrac{\lambda_2}{\gamma_s \lambda_s^*} + \lambda_2 \right)\sqrt{d}.
    \label{eqn: a_d}
\end{equation}
We define
\begin{equation}
    \nu_2 \coloneqq \tfrac{2\nu_1}{\mu - \mu_1},
    \label{eqn: nu_2}
\end{equation}
such that by \eqref{eqn: nu_1} and \eqref{eqn: nu_2}, $U(\xi^y) \leq a_d \nu_2$ implies
% $
% U(\xi^y) \leq  a_d \nu_2 = a_d \tfrac{2\nu_1}{\mu - \mu_1}
% %
% = a_d \tfrac{2}{\mu - \mu_1} \tfrac{\mu - \mu_1}{2 a_d}\underline{\alpha}_U(\zeta_2^{-1}(\nu))
% %
% = \underline{\alpha}_U(\zeta_2^{-1}(\nu))
% $
\begin{equation}
    \begin{aligned}
        U(\xi^y) \leq & a_d \nu_2 = a_d \tfrac{2\nu_1}{\mu - \mu_1} \\
            = & a_d \tfrac{2}{\mu - \mu_1} \tfrac{\mu - \mu_1}{2 a_d}\underline{\alpha}_U(\zeta_2^{-1}(\nu))\\
            =& \underline{\alpha}_U(\zeta_2^{-1}(\nu)),
    \end{aligned}
    \label{eqn: U less than nu2 implies xi less than nu}
\end{equation}
and $|\xi^y|_{\mathcal{E}^y} \leq \zeta_2^{-1}(\nu) = \nu^y$ by \eqref{eqn: No disturbance U sandwich bound}, and $|\xi|_{\mathcal{E}} \leq \nu$.

We remind that $\Delta^y = \zeta_2(\Delta)$, and for any $|\xi|_{\mathcal{E}}\leq \Delta$, we have $|\xi^y|_{\mathcal{E}^y}\leq \Delta^y$ and $U \leq \overline{\alpha}_U(\Delta^y)$. Let 
$\Delta^U \coloneqq a_d \overline{\alpha}_U(\Delta^y)$
and $\Delta_1$ from Assumption \ref{Assumption interconnection} be 
\begin{equation}
    \Delta_1 \coloneqq \underline{\alpha}_U^{-1}(\Delta^U),
    \label{eqn: Delta_1}
\end{equation}
such that $U(\xi^y) \leq \Delta^U$ implies $|\xi^y|_{\mathcal{E}^y} \leq \Delta_1$. Next, we will bound the system trajectories. We start by showing that during flow, the derivative of the $U$ is negative definite when it is bounded within an interval. This is followed by demonstrating the boundedness during jumps.









\noindent\emph{\underline{During flow}:} Let $\xi^y \in \mathcal{C}_1^{y,\epsilon}$, where $\mathcal{C}_1^{y,\epsilon}$ is the flow set of \eqref{eqn: H_2^y}, by Lemma \ref{Lemma MATI} and (23)-(24) in \cite{teel2000assigning}, we have that
$
U^\circ(\xi^y, F^y(\xi^y, \epsilon))
%
\leq  \Big< \! \tfrac{\partial {U_s}}{\partial \xi_s} ,F_s^y(x,y,e_s, e_f) \!\Big> \! +\! \tfrac{d}{\epsilon} \Big<\! \tfrac{\partial {U_f}}{\partial \xi_f} , F_f^y(x,y,e_s, e_f,\epsilon) \!\Big>
+ d \Big< \tfrac{\partial {U_f}}{\partial \xi_s},F_s^y(x,y,e_s, e_f) \Big>
%
= \Big< \! \tfrac{\partial {U_s}}{\partial \xi_s}, F_s^y(x,0,e_s, 0) \! \Big> + \tfrac{d}{\epsilon} \Big<\! \tfrac{\partial {U_f}}{\partial \xi_f}, F_f^y(x,y,e_s, e_f,0) \!  \Big>
%
+ \Big< \tfrac{\partial {U_s}}{\partial \xi_s}, F_s^y(x,y,e_s,e_f) - F_s^y(x,0,e_s,0)  \Big>
%
+ \frac{d}{\epsilon} \Big<  \tfrac{\partial {U_f}}{\partial \xi_f}, $
%
$
F_f^y(x,y,e_s,e_f, \epsilon) - F_f^y(x,y,e_s,e_f,0)\Big>
%
+ d \Big< \tfrac{\partial {U_f}}{\partial \xi_s}, F_s^y(x,y, $ $e_s, e_f) \Big>
$.
% \begin{equation*}
%     \begin{aligned}
%     U&^\circ(\xi^y, F^y(\xi^y, \epsilon)) 
%     \\
%     \leq&  \Big< \! \tfrac{\partial {U_s}}{\partial \xi_s} ,F_s^y(x,y,e_s, e_f) \!\Big> \! +\! \tfrac{d}{\epsilon} \Big<\! \tfrac{\partial {U_f}}{\partial \xi_f} , F_f^y(x,y,e_s, e_f,\epsilon) \!\Big>
%         \\
%     &  + d \Big< \tfrac{\partial {U_f}}{\partial \xi_s},F_s^y(x,y,e_s, e_f) \Big>
%     \\
%     = &  \Big< \! \tfrac{\partial {U_s}}{\partial \xi_s}, F_s^y(x,0,e_s, 0) \! \Big> + \tfrac{d}{\epsilon} \Big<\! \tfrac{\partial {U_f}}{\partial \xi_f}, F_f^y(x,y,e_s, e_f,0) \!  \Big>
%         \\
%         & + \Big< \tfrac{\partial {U_s}}{\partial \xi_s}, F_s^y(x,y,e_s,e_f) - F_s^y(x,0,e_s,0)  \Big>
%         \\
%         & + \frac{d}{\epsilon} \Big<  \tfrac{\partial {U_f}}{\partial \xi_f}, F_f^y(x,y,e_s,e_f, \epsilon) - F_f^y(x,y,e_s,e_f,0)\Big>
%         \\
%         & + d \Big< \tfrac{\partial {U_f}}{\partial \xi_s},F_s^y(x,y,e_s, e_f) \Big>.
%     % \\
%     % \leq& (1-d) \Big< \tfrac{\partial {U_s}}{\partial \xi_s} ,F_s^y(x,y,e_s, e_f) \Big> + d \Big< \tfrac{\partial {U_f}}{\partial \xi_s},F_s^y(x,y,e_s, e_f) \Big>
%     %     \\
%     % & +\! \tfrac{d}{\epsilon} \!\Big<\! \tfrac{\partial {U_f}}{\partial \xi_f} , F_f^y(x,y,e_s, e_f,0) \!\Big>
%     % \\
%     % & - d \Big< \!\tfrac{\partial {U_f}}{\partial y}\tfrac{\partial \overline{H}}{\partial \xi_s} +\tfrac{\partial {U_f}}{\partial e_f} \tfrac{\partial \tilde k}{\partial \xi_s}, F_s^y(x,y,e_s, e_f) \! \Big>
%     \end{aligned}
%     \end{equation*}
We remind the definition of $F_f^y$ is given by \eqref{eq:functions 2},
% \begin{align*}
%     &F_f^y(x,y,e_s,e_f, \epsilon) \\
%     =& \left[ \begin{smallmatrix} 
%         g_z(x,y+\overline{H}(x,e_s),e_s,e_f)- \epsilon \tfrac{\partial \overline{H}}{\partial \xi_s} F_s^y(x,y,e_s,e_f ) 
%         \\
%         g_{e_f}(x,y+\overline{H}(x,e_s),e_s, e_f , \epsilon)
%         \\
%         1
%         \\
%         0
%     \end{smallmatrix} \right],
% \end{align*}
where the definition of $g_{e_f}$ is given in \eqref{eq:functions}.
Then we have 
\begin{equation*}
    \begin{aligned}
         &F_f^y(x,y,e_s,e_f, \epsilon) - F_f^y(x,y,e_s,e_f,0)  \\
         = & \begin{bsmallmatrix}
            - \epsilon \tfrac{\partial \overline{H}}{\partial \xi_s} F_s^y(x,y,e_s,e_f )
             \\
             -\epsilon \tfrac{\partial k_{p_f}(x_p,z_p)}{\partial x_p}  f_{x,1}(x,z,e_s,e_f) - \epsilon \tfrac{\partial k_{c_f}(x_c,z_c)}{\partial x_c}  f_{x,2}(x,z,e_s,e_f) 
             \\
             0
             \\
             0
         \end{bsmallmatrix} \\
         =& -\epsilon\Big(\tfrac{\partial \overline{H}}{\partial \xi_s} F_s^y(x,y,e_s,e_f ),  \tfrac{\partial \tilde k}{\partial \xi_s} F_s^y(x,y,e_s, e_f), 0, 0\Big),
    \end{aligned}
\end{equation*}
where $\tilde k(x,z) = \big(k_{pf}(x_p,z_p), k_{cf}(x_c,z_c)\big)$. Consequently,
$
\frac{d}{\epsilon} \! \big< \!  \tfrac{\partial {U_f}}{\partial \xi_f}, F_f^y(x,y,e_s,e_f, \epsilon) - F_f^y(x,y,e_s,e_f,0)\big> $ $= - d \big< \!\tfrac{\partial {U_f}}{\partial y}\tfrac{\partial \overline{H}}{\partial \xi_s} +\tfrac{\partial {U_f}}{\partial e_f} \tfrac{\partial \tilde k}{\partial \xi_s}, F_s^y(x,y,e_s, e_f) \! \big>
$. 
% \begin{multline*}
%      \frac{d}{\epsilon} \Big<  \tfrac{\partial {U_f}}{\partial \xi_f}, F_f^y(x,y,e_s,e_f, \epsilon) - F_f^y(x,y,e_s,e_f,0)\Big>  
%     \\
%     = - d \Big< \!\tfrac{\partial {U_f}}{\partial y}\tfrac{\partial \overline{H}}{\partial \xi_s} +\tfrac{\partial {U_f}}{\partial e_f} \tfrac{\partial \tilde k}{\partial \xi_s}, F_s^y(x,y,e_s, e_f) \! \Big>.     
% \end{multline*}
Then we have
\begin{small}
    \begin{equation}
\setlength\abovedisplayskip{-4pt}%shrink space
\setlength\belowdisplayskip{3pt}
    \begin{aligned}
        U&^\circ(\xi^y, F^y(\xi^y, \epsilon)) 
        \\
      \leq  &  \Big< \! \tfrac{\partial {U_s}}{\partial \xi_s}, F_s^y(x,0,e_s, 0) \! \Big>\! +\! \tfrac{d}{\epsilon} \Big<\! \tfrac{\partial {U_f}}{\partial \xi_f}, F_f^y(x,y,e_s, e_f,0) \!  \Big>
        \\
        & + \Big< \tfrac{\partial {U_s}}{\partial \xi_s}, F_s^y(x,y,e_s,e_f) - F_s^y(x,0,e_s,0)  \Big>
        \\
        & + d \Big<  \tfrac{\partial {U_f}}{\partial \xi_s} - \tfrac{\partial {U_f}}{\partial y} \tfrac{\partial \overline{H}}{\partial \xi_s} - \tfrac{\partial {U_f}}{\partial e_f} \tfrac{\partial \tilde k}{\partial \xi_s} ,F_s^y(x,y,e_s,e_f)\Big>.
    \end{aligned}
    \label{eqn: U derivative during flow eqn1}
\end{equation}
\end{small}
%


By Assumption \ref{Assumption interconnection}, (\ref{eqn: Us flow}) and (\ref{eqn: Uf flow}), we know that for $\Delta_1$ and $\nu_1$ defined in \eqref{eqn: Delta_1} and \eqref{eqn: nu_1}, there exist positive constants $b_1$, $b_2$ and $b_3$ such that for all $|\xi^y|_{\mathcal{E}^y} \leq \Delta_1$,
%
$
U^\circ(\xi^y, F^y(\xi^y, \epsilon))
\leq  -a_s \psi_s^2\left(\left| (x, e_s) \right|\right) - d (\tfrac{a_f}{\epsilon} - b_3) \psi_f^2\left(\left| (y, e_f) \right|\right)
+ \left(b_1 + d b_2 \right)\psi_s\left(\left| (x, e_s) \right|\right)\psi_f\left(\left| (y, e_f) \right|\right) + (1+d)\nu_1
%
\leq - 
     \left[ \begin{smallmatrix} 
         \psi_s(|(x,e_s)|) \\ \psi_f(|(y,e_f)|) 
     \end{smallmatrix} \right]^T \tilde{\Lambda} \left[ \begin{smallmatrix} 
         \psi_s(|(x,e_s)|) \\ \psi_f(|(y,e_f)|) 
     \end{smallmatrix} \right] + 2\nu_1
$,
% \begin{equation*}
%     \begin{aligned}
%        U&^\circ(\xi^y, F^y(\xi^y, \epsilon)) \\ 
%        \leq & -a_s \psi_s^2\left(\left| (x, e_s) \right|\right) - d (\tfrac{a_f}{\epsilon} - b_3) \psi_f^2\left(\left| (y, e_f) \right|\right)
%         \\
%             & + \left(b_1 + d b_2 \right)\psi_s\left(\left| (x, e_s) \right|\right)\psi_f\left(\left| (y, e_f) \right|\right) + (1+d)\nu_1
%         \\
%         \leq & - 
%      \left[ \begin{smallmatrix} 
%          \psi_s(|(x,e_s)|) \\ \psi_f(|(y,e_f)|) 
%      \end{smallmatrix} \right]^T \tilde{\Lambda} \left[ \begin{smallmatrix} 
%          \psi_s(|(x,e_s)|) \\ \psi_f(|(y,e_f)|) 
%      \end{smallmatrix} \right] + 2\nu_1
%     \end{aligned},
% \end{equation*}
where $\tilde{\Lambda} \coloneqq \left[ \begin{smallmatrix} 
    a_s & -\tfrac{1}{2} b_1 - \tfrac{1}{2}d b_2 \\
    -\tfrac{1}{2} b_1 - \tfrac{1}{2}d b_2 & d (\tfrac{a_f}{\epsilon} - b_3)
\end{smallmatrix} \right]$.
%
Then by Assumption \ref{Assumption Extra 2}, we have
$
  U^\circ(\xi^y, F^y(\xi^y, \epsilon)) \leq  - 
        \left[ \begin{smallmatrix} 
            \sqrt{U_s(\xi_s)} \\ \sqrt{U_f(\xi_s, \xi_f)}
        \end{smallmatrix} \right]^\top
        \Lambda 
$ 
$
        \left[ \begin{smallmatrix} 
            \sqrt{U_s(\xi_s)} \\ \sqrt{U_f(\xi_s, \xi_f)}
        \end{smallmatrix} \right] + 2\nu_1
$,
% \begin{equation*}
%     \begin{aligned}
%         U^\circ(\xi^y, F^y(\xi^y, \epsilon)) \leq & - 
%         \left[ \begin{smallmatrix} 
%             \sqrt{U_s(\xi_s)} \\ \sqrt{U_f(\xi_s, \xi_f)}
%         \end{smallmatrix} \right]^\top
%         \Lambda
%         \left[ \begin{smallmatrix} 
%             \sqrt{U_s(\xi_s)} \\ \sqrt{U_f(\xi_s, \xi_f)}
%         \end{smallmatrix} \right] \\
%         &+ 2\nu_1,
%     \end{aligned}
% \end{equation*}
where 
\begin{equation}
\Lambda \coloneqq 
    \left[ \begin{smallmatrix} 
        a_s a_{\psi_s}^2 & -\tfrac{1}{2}(b_1 + d b_2)a_{\psi_s}a_{\psi_f} \\
        -\tfrac{1}{2}(b_1 + d b_2)a_{\psi_s}a_{\psi_f} & d (\tfrac{a_f}{\epsilon} - b_3)a_{\psi_f}^2
    \end{smallmatrix} \right].
    \label{eqn: Lambda}
    \end{equation} 
%
%We remind that $\mu \in (0, a_s a_{\psi_s}^2)$. 
In order to satisfy $\Lambda \geq \mu
    \left[ \begin{smallmatrix} 
        1 & 0 \\ 0 & d
    \end{smallmatrix} \right]$, we need 
\begin{tiny}
\begin{subequations}
\begin{align}   
    a_s a_{\psi_s}^2 > \mu \\
    (a_s a_{\psi_s}^2-\mu) \big( d (\tfrac{a_f}{\epsilon} - b_3)a_{\psi_f}^2 - \mu d \big) & \! \geq  \! \tfrac{1}{4}(b_1 + db_2)^2a_{\psi_s}^2a_{\psi_f}^2,
    \label{eqn: inequality of epsilon Exponential}
\end{align}
\end{subequations}
\end{tiny}%
where the first inequality is satisfied by the definition of $\mu$ in \eqref{eqn: mu}, and the second inequality can be satisfied by all $\epsilon \in (0,\epsilon^*)$, where 
\begin{equation}
\hspace{-2mm}
    \epsilon^* = \left(\tfrac{a_{\psi_f}}{a_f d}\big(\tfrac{(b_1 + db_2)^2 a_{\psi_s}^2 a_{\psi_f}^2}{4(a_s a_{\psi_s}^2 - \mu)}+\mu d\big) + \tfrac{b_3 a_{\psi_f}^2}{a_f} \right)^{-1}.
    \label{eqn: epsilon star}
\end{equation}
%
%
Then we have that for all $|\xi^y|_{\mathcal{E}^y} \leq \Delta_1$, 
$
U^\circ(\xi^y, F^y(\xi^y, \epsilon)) \leq -\mu (U_s(\xi_s) + d U_f(\xi_s,\xi_f)) + 2 \nu_1 \leq - \mu U(\xi^y) + 2 \nu_1
$.
% \begin{equation*}
%     \begin{aligned}
%         U^\circ(\xi^y, F^y(\xi^y, \epsilon)) &\leq -\mu (U_s(\xi_s) + d U_f(\xi_s,\xi_f)) + 2 \nu_1 \\
%         & \leq - \mu U(\xi^y) + 2 \nu_1.
%     \end{aligned}
% \end{equation*}
%
%
By definition of $\Delta_1$ in \eqref{eqn: Delta_1}, we have $U(\xi^y) \leq \Delta^U$ implies $|\xi^y|_{\mathcal{E}^y} \leq \Delta_1$. Then by definition of $\mu_1$ in \eqref{eqn: mu} and $\nu_2$ in \eqref{eqn: nu_2}, we have
\begin{small}
\begin{equation}
\setlength\abovedisplayskip{-5pt}%shrink space
\setlength\belowdisplayskip{5pt}
    U^\circ(\xi^y, F^y(\xi^y, \epsilon)) \leq -\mu_1 U(\xi^y), \ \forall U(\xi^y) \in [\nu_2, \Delta^U].
    \label{eqn: U derivative}
\end{equation}
\end{small}
%

\noindent\emph{\underline{During jumps}:} 
%
When slow dynamics update, (\ref{eqn: Us jump}) implies that 
$
U(G_s^y(\xi^y))= {U_s}((x,h_s(\kappa_s, e_s), 0, \kappa_s+1)) + d{U_f}(G_s^y(\xi^y))\leq {U_s}(\xi_s)+ d{U_f}(G_s^y(\xi^y))
$ for all $\xi^y \in \mathcal{D}_s^{y,\epsilon}$, where $\mathcal{D}_s^{y,\epsilon}$ is the set that a slow transmission can be triggered.
% \begin{equation*}
%     \begin{aligned}
%     &U(G_s^y(\xi^y)) \\
%     &= {U_s}((x,h_s(\kappa_s, e_s), 0, \kappa_s+1)) + d{U_f}(G_s^y(\xi^y)) \\
%     &\leq {U_s}(\xi_s)+ d{U_f}(G_s^y(\xi^y))
%     \end{aligned}
% \end{equation*}
%
By definitions of $U_f$ and $G_s^y$, we have
$
U(G_s^y(\xi^y)) \leq U_s(\xi_s)+ d\big( V_f(x,h_y(\kappa_s,x,e_s,y)) + \gamma_f \phi_f(\tau) W_f^2(\kappa_f, e_f) \big).
$
% \begin{equation*}
%     \begin{aligned}
%     &U(G_s^y(\xi^y)) \\
%     &\leq U_s(\xi_s)+ d\big( V_f(x,h_y(x,e_s,y)) + \gamma_f \phi_f(\tau) W_f^2(\kappa_f, e_f) \big)
%     \end{aligned}
% \end{equation*}
By adding and subtracting the term $U_f(\xi_s,\xi_f)$ and in the view of Assumption \ref{Assumption Vf at slow transmission}, we have
$
U(G_s^y(\xi^y)) \leq  U(\xi^y) + d\Big(V_f\big(x, h_y(\kappa_s,x,e_s,y)\big) - V_f(x,y)\Big)
%
\leq  U(\xi^y) + d\Big( \lambda_1 W_s^2(\kappa_s,e_s) + \lambda_2 \sqrt{W_s^2(\kappa_s,e_s) V_f(x,y)} \Big)
$.
% \begin{equation*}
%     \begin{aligned}
%     U(G_s^y(\xi^y)) \leq & U(\xi^y) + d\Big(V_f\big(x, h_y(x,e_s,y)\big) - V_f(x,y)\Big) \\
%     \leq & U(\xi^y) + d\Big( \lambda_1 W_s^2(\kappa_s,e_s) \\
%         &+ \lambda_2 \sqrt{W_s^2(\kappa_s,e_s) V_f(x,y)} \Big).
%                \end{aligned} %\label{eqn: U slow jump part 2}
% \end{equation*}
By completion of square, we have $\sqrt{W_s^2(\kappa_s,e_s) V_f(x,y)} \leq \tfrac{1}{2\sqrt{d}}W_s^2(\kappa_s,e_s) + \tfrac{\sqrt{d}}{2}V_f(x,y)$. At the same time, $W_s^2(\kappa_s,e_s) \leq \tfrac{1}{\gamma_s \lambda_s^*}U(\xi^y)$ and $d V_f(x,y) \leq U(\xi^y)$ by definition of $U(\xi^y)$. Then we have
\begin{equation}
    \begin{aligned}
        U(G_s^y(\xi^y)) \leq & U(\xi^y) + d \lambda_1 W_s^2(\kappa_s,e_s)  + \tfrac{1}{2\sqrt{d}} \lambda_2 W_s^2(\kappa_s,e_s)\\ 
            &  + \lambda_2 \tfrac{\sqrt{d}}{2} \big(d V_f(x,y) \big)
        \\
         \leq & \left(1 + \tfrac{\lambda_1}{\gamma_s \lambda_s^*} d + \tfrac{1}{2}\left(\tfrac{\lambda_2}{\gamma_s \lambda_s^*} + \lambda_2 \right)\sqrt{d}  \right) U(\xi^y)
         \\
         = & a_d U(\xi^y),
    \end{aligned}
    \label{eqn: U slow jump}
\end{equation}
where $a_d$ is defined in \eqref{eqn: a_d}.
%
For fast dynamics updates, (\ref{eqn: Uf jump}) implies for all $\xi^y \in \mathcal{D}_f^{y,\epsilon}$,
\begin{equation}
    \begin{aligned}
    U(G_f^y(\xi^y)) &= {U_s}(\xi_s) + d{U_f}(G_f^y(\xi^y))\\
    &\hspace{-5mm} \leq {U_s}(\xi_s) + d{U_f}(\xi_s, \xi_f) = U(\xi^y) ,
    \end{aligned}
    \label{eqn: U fast update}
\end{equation}
%
where we can see $U$ is non-increasing at fast transmissions.
%
Let $\xi^y(t,j)$ be a solution to $\mathcal{H}_1^y$, $(t, j)\in\text{dom}\, \xi^y$ and $0 = t_0 \leq t_1 \leq \cdots \leq t_{j+1} = t$ satisfying 
$$
\text{dom} \ \xi^y \cap ([0,t]\times \{0,\cdots,j \}) = \bigcup_{i\in \{0,\cdots,j\}  } [t_i,t_{i+1}] \times \{ i\}.
$$
%We note the sequence $\{t_1, \cdots, t_j \}$ are made of two disjoint sub-sequences, which correspond to transmission times of slow and fast variables, respectively, and are denoted by $\{t_{j_k^s}\}$ and $\{t_k^f\}$, respectively. 
The sequence $\{1,\cdots,j \}$ is divided into two disjoint sub-sequences representing slow and fast transmissions, respectively, and are denoted by $\mathcal{J}^s \coloneqq \{j_1^s, j_2^s, \cdots ,j_m^s \}$ and $\{j_1^f, j_2^f, \cdots ,j_n^f \}$, respectively.
We will first focus on the trajectory of $U(\xi^y(s,i))$, at each $i \in \mathcal{J}^s$ and all $s\in t_i $. Note that by abuse of notation, we write $U(s,i) \coloneqq U(\xi^y(s,i))$.
%
%For a sequence $\left\{j  \right\}_{j=0}^{\infty}$ of both slow and fast jumps, we pick all slow jumps to form a subsequence, denoted as $\left\{j_k^s  \right\}_{k=0}^{\infty}$ where $j_k^s$ denotes $k^{\text{th}}$ slow jump at time $t_{j_k^s} \in \mathcal{T}^s$. 
The notation $(s_1,i_1)\preceq (s_2,i_2)$ indicates $s_1+i_1 \leq s_2+i_2$, while $(s_1,i_1)\prec (s_2,i_2)$ implies $s_1+i_1 < s_2+i_2$. 
%
\begin{claim}
    %If $(t_a, j_a)\in \text{dom} \ \xi^y$ satisfies $U(t_a,j_a)\leq \nu_2$ and $(t_{j_k^s}, j_k^s) \preceq (t_a,j_a) \preceq (t_{j_{k+1}^s}, j_{k+1}^s-1)$, then $U(s,i)\leq \nu_2$ for all $(t_a, j_a) \preceq (s,i) \preceq (t_{j_{k+1}^s}, j_{k+1}^s-1)$.
    Suppose $(t_a, j_a) \preceq (t,j) \in \text{dom} \ \xi^y$ satisfies $U(t_a,j_a)\leq \nu_2$. Let $\mathcal{J}^s_{>a} \coloneqq \{i \in \mathcal{J}^s | i > j_a \}$.
    If $\mathcal{J}^s_{>a} = \emptyset$, then $U(s,i)\leq \nu_2$ for all $(t_a, j_a) \preceq (s,i) \preceq (t, j)$. Otherwise, let $j_{k}^s = \min \mathcal{J}^s_{>a}  $, then $U(s,i)\leq \nu_2$ for all $(t_a, j_a) \preceq (s,i) \preceq (t_{j_{k}^s}, j_{k}^s-1)$.
    \label{claim: invariant during flow}
\end{claim}
\textbf{Proof:} We first proof Claim \ref{claim: invariant during flow} for the case $\mathcal{J}^s_{>a} \neq \emptyset$ by contradiction. Assume there exist $(t_b, j_b)$ such that $(t_a, j_a)\preceq (t_b, j_b) \preceq (t_{j_{k}^s}, j_{k}^s-1)$ and $U(t_b,j_b) > \nu_2$. Then by continuity of $U$, there exist $(t_c,j_c)$ such that $(t_a, j_a)\preceq (t_c, j_c) \preceq (t_b, j_b)$, $U(t_c,j_c) = \nu_2$ and $U(s,i) \geq \nu_2 $ for all $(t_c, j_c)\preceq (s, i) \preceq (t_b, j_b)$. 
Then by \eqref{eqn: U derivative}, we have $ U(t_b,j_b) \leq U(t_c,j_c) + \int_{t_c}^{t_b} U^\circ(\xi^y, F^y(\xi^y, \epsilon)) dt \leq \nu_2 -\mu_1 \nu_2  (t_b-t_c) \leq \nu_2$,
% \begin{equation*}
%     \begin{aligned}
%         U(t_b,j_b) &= U(t_c,j_c) + \int_{t_c}^{t_b} U^\circ(\xi^y, F^y(\xi^y, \epsilon)) dt \\
%                     &\leq \nu_2 -\mu_1 \nu_2  (t_b-t_c) \leq \nu_2,
%     \end{aligned}
% \end{equation*}
which conflict our assumption that $U(t_b,j_b) > \nu_2$, and we proved Claim \ref{claim: invariant during flow} for the case $\mathcal{J}^s_{>a} \neq \emptyset$. The case $\mathcal{J}^s_{>a} = \emptyset$ can be proved similarly.


%%%%%%%%%%%%%%%%%%%%

Suppose $j_k^s$, $j_{k+1}^s \in\mathcal{J}^s$. By \eqref{eqn: U derivative}, \cite{teel2000assigning}, comparison principle \cite[Lemma 3.4]{nonlinear_systems_Khalil} and the fact that $U$ is non-increasing at fast transmissions, we have that if $U(t_{j_k^s}, j_k^s) \in [\nu_2, \Delta^U]$, then
\begin{equation}
    U(s,i) \leq U(t_{j_k^s}, j_k^s) \exp \! \big(-\mu_1(s-t_{j_k^s}) \big) \label{eqn: Exponential U flow}, 
\end{equation}
whenever $U(s,i) \geq \nu_2$, as well as $(t_{j_k^s}, j_k^s)\! \preceq \! (s,i) \! \preceq \!(t_{j_{k+1}^s}, j_{k+1}^s - 1)$ or $(t_{j_m^s}, j_m^s)\! \preceq \! (s,i) \! \preceq \!(t, j)$, where $j_m^s$ is the last element in $\mathcal{J}^s$ and $m$ corresponds to the amount of slow transmissions in trajectory $\xi^y(t,j)$.

Since $t_{j_{k+1}^s} - t_{j_k^s} \geq \tau_{\text{miati}}^s$ for all $j_k^s,j_{k+1}^s\in\mathcal{J}^s$, we have that $U(t^s_{k+1},j^s_{k+1}-1) \leq U(t^s_k, j^s_k) \exp \! \big(-\mu_1 \tau_{\text{miati}}^s \big)$
% \begin{equation}
%     U(t^s_{k+1},j^s_{k+1}-1) \leq U(t^s_k, j^s_k) \exp \! \big(-\mu_1 \tau_{\text{miati}}^s \big), 
% \end{equation}
if $U(s,i)\geq \nu_2$ for all $(t_{j_k^s}, j_k^s) \preceq (s,i) \preceq (t_{j_{k+1}^s}, j_{k+1}^s - 1)$ and $U(t_{j_k^s}, j_k^s) \in [\nu_2, \Delta^U]$. Then we will experience a slow transmission, by \eqref{eqn: U slow jump}, we have for all $j_k^s,j_{k+1}^s\in\mathcal{J}^s$ and $U(t_{j_k^s}, j_k^s) \in [\nu_2, \Delta^U]$, if $U(s,i)\geq \nu_2$ for all $(t_{j_k^s}, j_k^s) \preceq (s,i) \preceq (t_{j_{k+1}^s}, j_{k+1}^s - 1)$, then
\begin{equation}
    \begin{aligned}
        U(t^s_{k+1}, j^s_{k+1}) &\leq a_d U(t^s_{k+1},j^s_{k+1} - 1)
        \\
        &\leq a_d U(t_{j_k^s}, j_k^s) \exp (-\mu_1\tau_{\text{miati}}^s ).
    \end{aligned} 
    \label{eqn: flow then jump}
\end{equation}
%
By definition of $\lambda$, $d$ and $a_d$ in \eqref{eqn: lambda}, \eqref{eqn: d} and \eqref{eqn: a_d}, we have for all $U(t_{j_k^s}, j_k^s) \in [\nu_2, \Delta^U]$
\begin{equation}
\begin{aligned}
    U(t^s_{k+1}, j^s_{k+1}) &\leq a_d U(t_{j_k^s}, j_k^s) \exp (-\mu_1\tau_{\text{miati}}^s ) \\
    &= \lambda U(t^s_k,j^s_k)
\end{aligned}
    \label{eqn: U slow jump with lambda}
\end{equation}
if $U(s,i)\geq \nu_2$ for all $(t_{j_k^s}, j_k^s) \preceq (s,i) \preceq (t_{j_{k+1}^s}, j_{k+1}^s - 1)$. 



The following claim provides an upper bound of $U(\xi^y(t,j))$ for all $(t, j) \in \text{dom}\ \xi^y$. 
% \begin{claim}
%     Given $|\xi(0,0)|_{\mathcal{E}} \leq \Delta$, we have 
%     \begin{equation}
%         U(t,j) \leq \frac{a_d}{\lambda}U(t_{j_k^s},j_k^s)  \exp \!\left(-\tfrac{\ln{(\nicefrac{1}{\lambda})}}{\tau_{\text{mati}}^s}(t-t_{j_k^s}) \right) + a_d \nu_2,
%         \label{eqn: upper bound of U}
%     \end{equation}
%     for all $(t,j) \in \text{dom}\ \xi$, $k \in \mathbb{Z} and \red{(t_k)}$.
%     \label{Claim U upper bound}
% \end{claim}
\begin{claim}
    Given $|\xi(0,0)|_{\mathcal{E}} \leq \Delta$, we have 
    \begin{equation}
        U(t, j) \leq \max \left\{ \frac{a_d}{\lambda}U(0,0)  \exp \!\left(-\tfrac{\ln{(\nicefrac{1}{\lambda})}}{\tau_{\text{mati}}^s}(t) \right), a_d \nu_2\right\},
        \label{eqn: upper bound of U}
    \end{equation}
    for all $(t, j) \in \text{dom}\ \xi^y$.
    \label{Claim U upper bound}
\end{claim}
\textbf{Proof of Claim \ref{Claim U upper bound}:}
We first consider the case that $\mathcal{J}^s$ is non-empty, where the scenario with $\mathcal{J}^s = \emptyset$ can be deduced from step 1 below directly.
%
The proof has two steps, in the first step, we check the upper bound of $U(s,i)$ for all $(s,i) \preceq (t_{j_1^s}, j_1^s)$ and $(s,i) \in \text{dom}\,\xi^y $, then we verify the upper bound of $U(t,j)$ in the second step.
Here we remind that  $|\xi(0,0)|_{\mathcal{E}} \leq \Delta$ implies $U(0,0) \leq \overline{\alpha}_U(\Delta^y) < \Delta^U$ and $\Delta^U = a_d \overline{\alpha}_U(\Delta^y)$.

\emph{\underline{Step 1} }: Since the initial condition of $\tau_s$ is not guaranteed to be zero, $t_{j_1^s}$ (the time of first slow transmission) might be smaller than $\tau_{\text{miati}}^s$. In this step, we focus on the interval $(s,i) \preceq (t_{j_1^s}, j_1^s)$, $(s,i) \in \text{dom}\,\xi^y $.  
%
We divide into two cases based on the initial condition.

\noindent \emph{\underline{Case 1-1}}: If $U(0,0) \leq \nu_2$, by Claim \ref{claim: invariant during flow}, we know $U(s,i) \leq \nu_2$ for all $(s,i) \preceq (t_{j_1^s}, j_1^s-1)$. Then by \eqref{eqn: U slow jump}, $U(t_{j_1^s}, j_1^s) \leq a_d \nu_2$.

\noindent \emph{\underline{Case 1-2} }: If $U(0,0) \in (\nu_2, \overline{\alpha}_U(\Delta^y)]$, by \eqref{eqn: U derivative}, \eqref{eqn: U fast update} and Claim \ref{claim: invariant during flow}, we have $U(s,i) \leq U(0,0)$ for all $(s,i) \preceq (t_{j_1^s}, j_1^s - 1)$. Then by \eqref{eqn: U slow jump} we have $U(s,i) \leq a_d U(0,0)$ for all $(s,i) \preceq (t_{j_1^s}, j_1^s)$.

For any $(t,j)\in \text{dom}\ \xi^y$ with $\mathcal{J}^s = \emptyset$, Claim \ref{claim: invariant during flow} implies $U(t,j) \leq \nu_2$ if $U(0,0)\leq \nu_2$, and Case 1-2 shows $U(t,j)\leq U(0,0)$ if $U(0,0) \in (\nu_2, \overline{\alpha}_U(\Delta^y)]$. Since $t \leq \tau_{\text{mati}}^s$, we have $ \frac{a_d}{\lambda}U(0,0)  \exp \!\left(-\tfrac{\ln{(\nicefrac{1}{\lambda})}}{\tau_{\text{mati}}^s}(t) \right)   \geq U(0,0)$, which shows \eqref{eqn: upper bound of U} hold for all $(t,j)\in \text{dom}\ \xi^y$ when $\mathcal{J}^s = \emptyset$.
%
% we can show \eqref{Claim U upper bound} hold. Since $t_{j_1^s} \leq \tau_{\text{mati}}^s$, $t \leq \tau_{\text{mati}}^s$ we have 
% $ \frac{a_d}{\lambda}U(0,0)  \exp \!\left(-\tfrac{\ln{(\nicefrac{1}{\lambda})}}{\tau_{\text{mati}}^s}(s) \right)   \geq a_d U(0,0)$ for all $s \in [0, t_{j_1^s}]$. By considering both cases, we can see inequality \eqref{eqn: upper bound of U} is satisfied for all $(s,i) \preceq (t_{j_1^s}, j_1^s)$. 
At the same time, from Cases 1-1 and 1-2, we can see $|\xi(0,0)|_{\mathcal{E}} \leq \Delta$ implies $U(t_{j_1^s}, j_1^s) \leq a_d \overline{\alpha}_U = \Delta^U $.



\emph{\underline{Step 2} }: In this step, we verify \eqref{eqn: upper bound of U} for $(t,j)$ with $\mathcal{J}^s \neq \emptyset$. We also divide into two cases based on $U(t_{j_1^s}, j_1^s)$.

\noindent \emph{\underline{Case 2-1} }:We consider the case that $U(t_{j_1^s}, j_1^s) \leq a_d \nu_2$.
%
We remind that $m$ is the amount of slow transmissions in the trajectory $\xi^y(t,j)$.
%
If $m=1$, by \eqref{eqn: U derivative} and Claim \eqref{claim: invariant during flow}, we have $U(t,j)\leq a_d \nu_2$.
%
Next we consider the scenario that $m\geq 2$.
Let $\Omega_{\nu_2} \coloneqq \{\xi^y \in \mathbb{X} | U(\xi^y) \leq \nu_2\}$.
By \eqref{eqn: Exponential U flow}, definition of $\lambda$, $d$ and $a_d$ in \eqref{eqn: lambda}, \eqref{eqn: d} and \eqref{eqn: a_d}, we can see the time that takes $\xi^y(t_{j_1^s}, j_1^s)$ to enter the set $\Omega_{\nu_2}$ is less than or equal to $\tau_{\text{miati}}^s$. Then by Claim \ref{claim: invariant during flow}, we have $U(t_2^s, j_2^s-1) \leq \nu_2$. By \eqref{eqn: U slow jump}, we have $U(t_{j_2^s}, j_2^s) \leq a_d \nu_2$. Finally, by concatenation, we can prove $U(t,j)\leq a_d \nu_2$.
%
% which implies $U(t_2^s, j_2^s-1) \leq \nu_2$ if $m\geq 2$. By \eqref{eqn: U derivative} and and Claim \eqref{claim: invariant during flow}, we have $U(t,j)\leq a_d \nu_2$ if $m=1$. If $m\geq 2$, by \eqref{eqn: U slow jump}, we have $U(t_{j_2^s}, j_2^s) \leq a_d \nu_2$. Finally, by concatenation, we can prove $U(t,j)\leq a_d \nu_2$.

\noindent \emph{\underline{Case 2-2} }:
We consider the case that $U(t_{j_1^s}, j_1^s) \in (a_d \nu_2, \Delta^U] $, which can exist only if $U(0,0) \in (\nu_2, \overline{\alpha}_U(\Delta^y)]$. Then by Cases 1-2, we know $U(t_{j_1^s}, j_1^s) \leq a_d U(0,0)$. We consider two possible scenarios.
%

Firstly, if $U(s,i) > \nu_2$ for all $(t_{j_1}^s, j_1^s) \preceq (s,i) \preceq (t,j)$, then by \eqref{eqn: U slow jump with lambda} and concatenation, we have $U(t_{j_m^s}, j_m^s) \leq \lambda^{m-1} U(t_{j_1^s},j_1^s) \leq \lambda^{m-1} a_d U(0,0)$,
% \begin{equation}
%     U(t_{j_m^s}, j_m^s) \leq \lambda^{m-1} U(t_{j_1^s},j_1^s) \leq \lambda^{m-1} a_d U(0,0) , \label{eqn: Discrete model is UGES}
% \end{equation}
where $j_m^s$ is the last element in $\mathcal{J}^s$.
%
Then by \eqref{eqn: Exponential U flow}, we have 
\begin{equation}
%\setlength\abovedisplayskip{3pt}%shrink space
\setlength\belowdisplayskip{3pt}
U(t,j) \leq \lambda^{m-1} a_d U(0,0) \text{exp}\big(-\mu_1(t-t_{j_m^s})\big).
\label{eqn: U(t,j) upperbournd case 2-2}
\end{equation}













%
%
The first component on the right-hand side of \eqref{eqn: upper bound of U} can be written as
\begin{equation}
    \begin{aligned}
        &\frac{a_d}{\lambda}U(0,0)  \exp \!\left(-\tfrac{\ln{(\nicefrac{1}{\lambda})}}{\tau_{\text{mati}}^s}t \right) \\
        =& \frac{a_d}{\lambda}U(0,0)  \exp \!\left(-\tfrac{\ln{(\nicefrac{1}{\lambda})}}{\tau_{\text{mati}}^s}t_{j_m^s} \right) \exp \!\left(-\tfrac{\ln{(\nicefrac{1}{\lambda})}}{\tau_{\text{mati}}^s} (t-t_{j_m^s}) \right).
    \end{aligned}
    \label{eqn: verify upper bound part 1}
\end{equation}
%
Since $\tfrac{t_{j_m^s}}{\tau_{\text{mati}}^s} \leq m$, we have 
$\frac{a_d}{\lambda}U(0,0)  \exp \!\left(-\tfrac{\ln{(\nicefrac{1}{\lambda})}}{\tau_{\text{mati}}^s}t_{j_m^s} \right) \geq \frac{a_d}{\lambda}U(0,0)  \exp \!\left(-\ln{(\nicefrac{1}{\lambda})} m \right) =  a_d U(0,0) \lambda^{m-1}$.
%
% \begin{equation}
%     \begin{aligned}
%         & \frac{a_d}{\lambda}U(0,0)  \exp \!\left(-\tfrac{\ln{(\nicefrac{1}{\lambda})}}{\tau_{\text{mati}}^s}t_{j_m^s} \right) \\
%         \geq & \frac{a_d}{\lambda}U(0,0)  \exp \!\left(-\ln{(\nicefrac{1}{\lambda})} m \right) 
%         =  a_d U(0,0) \lambda^{m-1}.
%     \end{aligned} \label{eqn: verify discrete upper bound}
% \end{equation}
%
By definition of $\lambda$ in \eqref{eqn: lambda}, and since $a_d >1$, we have
$\exp \!\left(-\tfrac{\ln{(\nicefrac{1}{\lambda})}}{\tau_{\text{mati}}^s} (t-t_{j_m^s}) \right) 
=  \lambda^{\tfrac{t-t_{j_m^s}}{\tau_{\text{mati}}^s}} 
\geq   \big(a_d \exp (-\mu_1  \tau_{\text{miati}}^s $ $) \big)^{\tfrac{t-t_{j_m^s}}{\tau_{\text{mati}}^s}}
\geq  \exp \! \big(-\mu_1 \tfrac{\tau_{\text{miati}}^s}{\tau_{\text{mati}}^s} (t - t_{j_m^s})\big)
\geq  \exp \! \big(-\mu_1 (t - t_{j_m^s})\big)
$.
%
% \begin{equation}
%     \begin{aligned}
%         & \exp \!\left(-\tfrac{\ln{(\nicefrac{1}{\lambda})}}{\tau_{\text{mati}}^s} (t-t_{j_m^s}) \right) \\
%         %=&\exp \!\left( -\ln{(\nicefrac{1}{\lambda})}\right)^{\tfrac{t-t_{j_k^s}}{\tau_{\text{mati}}^s}} 
%         = & \lambda^{\tfrac{t-t_{j_m^s}}{\tau_{\text{mati}}^s}} 
%         \geq  \big(a_d \exp (-\mu_1 \tau_{\text{miati}}^s) \big)^{\tfrac{t-t_{j_m^s}}{\tau_{\text{mati}}^s}} \\
%         \geq & \exp \! \big(-\mu_1 \tfrac{\tau_{\text{miati}}^s}{\tau_{\text{mati}}^s} (t - t_{j_m^s})\big) \\
%         \geq & \exp \! \big(-\mu_1 (t - t_{j_m^s})\big). \\
%     \end{aligned}
%     \label{eqn: verify upper bound part 2}
% \end{equation}
Then by \eqref{eqn: U(t,j) upperbournd case 2-2} and \eqref{eqn: verify upper bound part 1}, we show that if $U(s,i) > \nu_2$ for all $(t_{j_1}^s, j_1^s) \preceq (s,i) \preceq (t,j)$, \eqref{eqn: upper bound of U} hold. 

Secondly, we consider the scenario that there exist $(t_a,j_a) \preceq (t,j)$, $(t_a,j_a) \in \text{dom} \ \xi^y$ such that $U(t_a,j_a) \leq \nu_2$. Following the similar procedure as in Case 2-1, we can prove $U(t,j)\leq a_d \nu_2$. Then we have proved Claim \ref{Claim U upper bound}.   $\hfill\square$


% We consider the case that $U(t_{j_1^s}, j_1^s) \in [a_d \nu_2, \Delta^U] $.
% %
% We will first prove that there exist $(t_a,j_a) \succeq (t_{j_1^s}, j_1^s) $ such that $U(t_a,j_a)\leq \nu_2 $ by contradiction. 
% %
% Assume $U(t,j) > \nu_2$ for all $(t,j) \succeq (t_{j_1^s}, j_1^s)$. 
% %
% Then by \eqref{eqn: U slow jump with lambda} and concatenation, we have
% \begin{equation}
%     U(t_{j_k^s}, j_k^s) \leq \lambda^{k-1} U(t_{j_1^s},j_1^s). \label{eqn: Discrete model is UGES}
% \end{equation} 
% Since the right-hand side of the inequality approaches zero as $k$ approaches infinity, the above inequality contradicts our assumption, and there exists $(t_a,j_a) \succeq (t_{j_1^s}, j_1^s) $ such that $U(t_a,j_a) \leq \nu_2 $. Let $(t_a,j_a)$ be the smallest possible value. Moreover, by \eqref{eqn: U slow jump with lambda}, we have \eqref{eqn: Discrete model is UGES} holds for all $(t_{j_k^s}, j_k^s) \preceq (t_a, j_a)$.


% Firstly, we check the upper bound for all $(t_{j_k^s}, j_k^s) \preceq (t_a, j_a)$. Since $\tfrac{t_{j_k^s}}{\tau_{\text{mati}}^s} \leq k$, for all $k\in \mathbb{Z}_{\geq 1}$, we have 
% \begin{equation}
%     \begin{aligned}
%         & \frac{a_d}{\lambda}U(0,0)  \exp \!\left(-\tfrac{\ln{(\nicefrac{1}{\lambda})}}{\tau_{\text{mati}}^s}t_{j_k^s} \right) \\
%         \geq & \frac{a_d}{\lambda}U(0,0)  \exp \!\left(-\ln{(\nicefrac{1}{\lambda})} k \right) 
%         =  a_d U(0,0) \lambda^{k-1}.
%     \end{aligned}
%     \label{eqn: verify discrete upper bound}
% \end{equation}
% By \eqref{eqn: Discrete model is UGES} and the fact that $U(t_{j_1^s}, j_1^s) \leq a_d U(0,0)$, we verify inequality \eqref{eqn: upper bound of U} for all $(t_{j_k^s}, j_k^s) \preceq (t_a, j_a)$.


% Secondly, we verify \eqref{eqn: upper bound of U} for all $(t,j) \preceq (t_a,j_a)$. For any $(t,j) \preceq (t_a,j_a)$, we can find $k \in \mathbb{Z}_{\geq 1}$ such that $(t_{j_k^s}, j_k^s) \preceq (t, j) \preceq \min\{(t_{j_{k+1}^s}, j_{k+1}^s - 1), (t_a,j_a)\}$.
% %
% The first component on the right-hand side of \eqref{eqn: upper bound of U} can be written as
% \begin{equation}
%     \begin{aligned}
%         &\frac{a_d}{\lambda}U(0,0)  \exp \!\left(-\tfrac{\ln{(\nicefrac{1}{\lambda})}}{\tau_{\text{mati}}^s}t \right) \\
%         =& \frac{a_d}{\lambda}U(0,0)  \exp \!\left(-\tfrac{\ln{(\nicefrac{1}{\lambda})}}{\tau_{\text{mati}}^s}t_k \right) \exp \!\left(-\tfrac{\ln{(\nicefrac{1}{\lambda})}}{\tau_{\text{mati}}^s} (t-t_{j_k^s}) \right).
%     \end{aligned}
%     \label{eqn: verify upper bound part 1}
% \end{equation}
% %
% By definition of $\lambda$ in \eqref{eqn: lambda}, and since $a_d >1$, we have
% \begin{equation}
%     \begin{aligned}
%         & \exp \!\left(-\tfrac{\ln{(\nicefrac{1}{\lambda})}}{\tau_{\text{mati}}^s} (t-t_{j_k^s}) \right) \\
%         %=&\exp \!\left( -\ln{(\nicefrac{1}{\lambda})}\right)^{\tfrac{t-t_{j_k^s}}{\tau_{\text{mati}}^s}} 
%         = & \lambda^{\tfrac{t-t_{j_k^s}}{\tau_{\text{mati}}^s}} 
%         \geq  \big(a_d \exp (-\mu_1 \tau_{\text{miati}}^s) \big)^{\tfrac{t-t_{j_k^s}}{\tau_{\text{mati}}^s}} \\
%         \geq & \exp \! \big(-\mu_1 \tfrac{\tau_{\text{miati}}^s}{\tau_{\text{mati}}^s} (t - t_{j_k^s})\big) \\
%         \geq & \exp \! \big(-\mu_1 (t - t_{j_k^s})\big). \\
%     \end{aligned}
%     \label{eqn: verify upper bound part 2}
% \end{equation}
% %
% By \eqref{eqn: verify discrete upper bound} we know $U(t_{j_k^s}, j_k^s) \leq \frac{a_d}{\lambda}U(0,0)  \exp \!\left(-\tfrac{\ln{(\nicefrac{1}{\lambda})}}{\tau_{\text{mati}}^s}t_k \right) $.
% %
% Then by \eqref{eqn: Exponential U flow}, \eqref{eqn: verify upper bound part 1} and \eqref{eqn: verify upper bound part 2}, we verified the inequality \eqref{eqn: upper bound of U} to all $(t,j) \preceq (t_a,j_a)$.

% Finally, we know $U(t_a,j_a) \leq \nu_2$, then by following the same line as Case 1-1 and Case 2-1, we can show $U(t,j) \leq a_d \nu_2$ for all $(t,j) \succeq (t_a, j_a)$. Then we have proved Claim \ref{Claim U upper bound} for all $(t,j)\in \text{dom} \xi$. 
% $\hfill\square$











Now we convert the upper bound on $U(\xi^y(t,j))$ to the upper bound on $|\xi|_{\mathcal{E}}$.  By \eqref{eqn: U less than nu2 implies xi less than nu}, we know $U(\xi^y) \leq a_d \nu_2$ implies $|\xi|_{\mathcal{E}} \leq \nu$. Additionally, since $|\xi|_{\mathcal{E}} \leq \zeta_2(|\xi^y|_{\mathcal{E}^y})$, $|\xi^y|_{\mathcal{E}^y} \leq \zeta_2(|\xi|_{\mathcal{E}})$ and by sandwich bound \eqref{eqn: No disturbance U sandwich bound}, we have $U(t,j) \leq  \frac{a_d}{\lambda}U(0,0)  \exp \!\left(-\tfrac{\ln{(\nicefrac{1}{\lambda})}}{\tau_{\text{mati}}^s}t \right)$ implies $|\xi(t,j)| \leq \beta_1(\xi(0,0), t)$ where 
$
\beta_1(s, t) \coloneqq \zeta_2(\underline{\alpha}_{U}^{-1}(\frac{a_d}{\lambda}\overline{\alpha}_U(\zeta_2(s))  \exp \!\big(-\tfrac{\ln{(\nicefrac{1}{\lambda})}}{\tau_{\text{mati}}^s} t \big)))
$
% \begin{equation*}
%     \beta_1(s, t) \coloneqq \zeta_2(\underline{\alpha}_{U}^{-1}(\frac{a_d}{\lambda}\overline{\alpha}_U(\zeta_2(s))  \exp \!\left(-\tfrac{\ln{(\nicefrac{1}{\lambda})}}{\tau_{\text{mati}}^s}(t) \right))),
% \end{equation*}
and $\beta_1 \in \mathcal{KL}$. Then by Claim \ref{Claim U upper bound}, we know that $|\xi(t,j)|_\mathcal{E} \leq \beta_1(|\xi(0,0)|_\mathcal{E}, t) + \nu$ for all $|\xi(0,0)|_\mathcal{E}\leq \Delta$ .
% \begin{equation}
%     |\xi(t,j)|_\mathcal{E} \leq \beta_1(|\xi(0,0)|_\mathcal{E}, t) + \nu.
% \end{equation}
%
Additionally, since $t \geq \tau_{\text{miati}}^s (j-1)$, we have 
%
% $
% \exp \!\left(-\tfrac{\ln{(\nicefrac{1}{\lambda})}}{ \tau_{\text{mati}}^s}(t) \right)
% \leq  \exp \!\left(-\tfrac{\ln{(\nicefrac{1}{\lambda})}}{\tau_{\text{mati}}^s}(\tfrac{t}{2}+\tfrac{\tau_{\text{miati}}^s}{2}(j-1))) \right)
% \leq  \exp \!\left( \tfrac{\ln{(\nicefrac{1}{\lambda})}\tau_{\text{miati}}^s}{2 \tau_{\text{mati}}^s}\right)
% $
% $
% \exp \!\left(-\tfrac{\ln{(\nicefrac{1}{\lambda})}}{2 \tau_{\text{mati}}^s}  \min\{1,\tau_{\text{miati}}^s \} (t+j) \right)
% \eqqcolon  \alpha_1(t+j)           
% $.
\begin{equation}
    \begin{aligned}
        &\exp \!\left(-\tfrac{\ln{(\nicefrac{1}{\lambda})}}{ \tau_{\text{mati}}^s}(t) \right) \\
        %=&  \exp \!\left(-\tfrac{\ln{(\nicefrac{1}{\lambda})}}{\tau_{\text{mati}}^s}(\tfrac{t}{2}+\tfrac{t}{2})) \right) \\
        \leq & \exp \!\left(-\tfrac{\ln{(\nicefrac{1}{\lambda})}}{\tau_{\text{mati}}^s}(\tfrac{t}{2}+\tfrac{\tau_{\text{miati}}^s}{2}(j-1))) \right) \\
        % =& \exp \!\left( \tfrac{\ln{(\nicefrac{1}{\lambda})}\tau_{\text{miati}}^s}{2\tau_{\text{mati}}^s}\right)   
        %     \exp \!\left(-\tfrac{\ln{(\nicefrac{1}{\lambda})}}{2 \tau_{\text{mati}}^s}(t+\tau_{\text{miati}}^sj) \right) \\
        \leq & \exp \!\left( \tfrac{\ln{(\nicefrac{1}{\lambda})}\tau_{\text{miati}}^s}{2 \tau_{\text{mati}}^s}\right)   
            \exp \!\left(-\tfrac{\ln{(\nicefrac{1}{\lambda})}}{2 \tau_{\text{mati}}^s}  \min\{1,\tau_{\text{miati}}^s \} (t+j) \right)
            \\
            \eqqcolon & \alpha_1(t+j).
    \end{aligned}
    \label{eqn: change t to t+j}
\end{equation}
Then we have for all $|\xi(0,0)|_\mathcal{E} \leq \Delta$, there exist $\beta \in \mathcal{KL}$ such that $|\xi(t,j)|_\mathcal{E} \leq \beta(|\xi(0,0)|_\mathcal{E}, t+j) + \nu$,
% \begin{equation*}
%      |\xi(t,j)|_\mathcal{E} \leq \beta(|\xi(0,0)|_\mathcal{E}, t+j) + \nu,
% \end{equation*}
where $\beta(s,t+j) = \zeta_2(\underline{\alpha}_{U}^{-1}(\frac{a_d}{\lambda}\overline{\alpha}_U(\zeta_2(s))  \alpha_1 (t+j)))$. 













% Let $\Omega_{\nu_2} \coloneqq \{\xi \in \mathbb{X} | U(\xi^y) \leq \nu_2   \}$, $\Omega_{\nu^U} \coloneqq \{\xi \in \mathbb{X} | U(\xi^y) \leq a_d \nu_2   \}$.
% We then divide the proof into three steps. In Step 1, we show $U(t,j)$ with $U(0, 0) \in [\nu_2, \overline{\alpha}_U(\Delta^y)]$ will eventually enter the set $\Omega_{\nu_2}$. In Step 2, we show $\xi$ will never leave the set $\Omega_{\nu^U}$ once we enter $\Omega_{\nu_2}$. Finally, we convert the bound of $U(t,j)$ into the bound of $|\xi(t,j)|_{\mathcal{E}}$.

% \noindent\emph{\underline{\textbf{Step 1:}} } In this step, we consider $U(0,0) \in [\nu_2, \overline{\alpha}_U(\Delta^y)]$. We remind that $|\xi|_\mathcal{E} \leq \Delta$ implies $U(0,0) \leq \overline{\alpha}_U(\Delta^y) < \Delta^U$.

% \emph{\underline{Step 1 - Part 1} }: Since $\tau_s$ at $(t,j) = (0,0)$ is not guaranteed to be zero, $t_{j_1^s}$ (the time of first slow transmission) might be smaller than $\tau_{\text{miati}}^s$. In this part, we focus on the interval $(t,j) \preceq (t_{j_1^s}, j_1^s)$, $(t,j) \in \text{dom}\,\xi $ (i.e. before and including the first slow transmission). 
% %
% By considering Claim \ref{claim: invariant during flow}, \eqref{eqn: U derivative} and \eqref{eqn: U fast update}, we have that given $U(t_{j_k^s}, j_k^s) \in [\nu_2, \Delta^U]$, 
% \begin{equation} 
% U(t, j) \leq U(t_{j_k^s}, j_k^s), \forall (t_{j_k^s},j_k^s) \preceq (t,j) \preceq (t_{j_{k+1}^s}, j_{k+1}^s - 1) , \label{eqn: U during flow}
% \end{equation}
% which implies $U(t,j) \leq U(0,0)$ for all $(t,j) \preceq (t_{j_1^s}, j_1^s-1)$.
% Additionally, from \eqref{eqn: U slow jump} we know that $U(G_s^y(\xi^y)) \leq a_d U(\xi^y)$ and
% \begin{equation*}
%     \begin{aligned}
%         U(t_{j_1^s}, j_1^s)
%         &\leq a_d U(t_{j_1^s}, j_1^s - 1) \\
%         &\leq a_d U(0,0).
%     \end{aligned}
% \end{equation*} 
% By definition of $a_d$ in \eqref{eqn: U slow jump}, we have $a_dU(0,0) > U(0,0)$. Then by \eqref{eqn: U during flow}, we can conclude that $U(t,j) \leq a_dU(0,0) \leq \Delta^U$  for all $(t,j) \preceq (t_{j_1^s}, j_1^s)$, $(t,j) \in \text{dom}\ \xi$.

% \todo[inline]{the paragraph after (include) Claim 1 can be move to Step 1 part 2}
% \emph{\underline{Step 1 - Part 2} }: Now we consider the trajectory for $U(t,j)$ where $(t,j)\succeq (t_{j_1^s}, j_1^s)$. We will first prove that given $U(0,0) \in [\nu_2, \overline{\alpha}_U(\Delta^y)]$, there exist $(t_a,j_a) \succeq (t_{j_1^s}, j_1^s) $ such that $U(t_a,j_a)\leq \nu_2 $ by contradiction. 

% Assume $U(t,j) > \nu_2$ for all $(t,j) \succeq (t_{j_1^s}, j_1^s)$. Since $U(0,0) \in [\nu_2, \overline{\alpha}_U(\Delta^y)]$, by Step 1 Part 1, we know $U(t_{j_1^s}, j_1^s) \leq \Delta^U$. If $U(t_{j_1^s}, j_1^s) \leq \nu_2$, then we already contradict our assumption. If $U(t_{j_1^s}, j_1^s) \in [\nu_2, \Delta^U]$ then by \eqref{eqn: U slow jump with lambda} and concatenation, we have
% \begin{equation}
%     U(t_{j_k^s}, j_k^s) \leq \lambda^{k-1} U(t_{j_1^s},j_1^s). \label{eqn: Discrete model is ULAS}
% \end{equation} 
% Since the right-hand side of the inequality approaches zero as $k$ approaches infinity, the above inequality contradicts our assumption, and there exists $(t_a,j_a) \succeq (t_{j_1^s}, j_1^s) $ such that $U(t_a,j_a) \leq \nu_2 $. Let $(t_a,j_a)$ be the smallest possible value.



% We will next use the above discrete bound to obtain a continuous upper bound. 
% %
% By using \eqref{eqn: Exponential U flow}, we have that $U(t, j) \leq U(t_{j_k^s}, j_k^s)$ for $(t_{j_k^s}, j_k^s) \preceq (t, j) \preceq \min\{(t_{j_{k+1}^s}, j_{k+1}^s - 1), (t_a,j_a)\}$. At slow transmissions, according to \eqref{eqn: flow then jump}, we have $U(t_{j_{k+1}^s}, j_{k+1}^s) \leq a_d U(t_{j_k^s}, j_k^s) \exp (-\mu_1\tau_{\text{miati}}^s )$ for all $(t_{j_{k+1}^s}, j_{k+1}^s) \preceq (t_a,j_a)$. Therefore, there exist a $\mathcal{K}_{\infty}$-function $\tilde \gamma$ such that $U(t,j) \leq \tilde \gamma (U(t_{j_k^s}, j_k^s))$ for all $(t_{j_k^s}, j_k^s) \preceq (t,j) \preceq \min\{(t_{j_{k+1}^s}, j_{k+1}^s), (t_a,j_a)\}$. Then by \eqref{eqn: Discrete model is ULAS}, we can apply \cite[Theorem 1]{discrete_to_continuous_KL_bound} to show there exist a $\mathcal{KL}$-function $\beta_3$ such that 
% $$U(t,j) \leq \beta_3(U(t_{j_1^s},j_1^s),t-t_{j_1^s})$$ for all $(t,j)\in \text{dom}\, \xi$ and $(t_{j_1^s},j_1^s) \preceq (t,j) \preceq (t_a,j_a)$. 


% \emph{\underline{Step 1 - Part 3} }: Due to the results of Part 1, Part 2 and the fact that $\beta_3(s,0) \geq s$, we have that for all $(t,j) \preceq (t_a, j_a)$, 
% \begin{equation*}
%      U(t,j)\leq  \left\{  \begin{aligned} 
%      &\beta_3\Big(a_d U(0,0)  , 0\Big), &&  (t,j)\preceq (t_{j_1^s}, j_1^s) \\
%      &\beta_3\Big(a_d U(0,0), t-t_{j_1^s}\Big), &&  (t_{j_1^s}, j_1^s) \preceq(t,j) \preceq (t_a, j_a)
%     \end{aligned}\right. .
% \end{equation*}
% Then we have $U(t,j) \leq \beta_4(U(0,0),t)$ for all $(t,j) \preceq (t_a, j_a)$, where 
% $\beta_4(s,t) \coloneqq \max \Big\{ \tfrac{e^{-t}}{e^{-\tau_\text{mati}^s}} \beta_3\big(a_d s,0\big) ,$ 
% $ \beta_3\big(a_d s, \max \{t-\tau_{\text{mati}}^s,0 \}\big)  \Big\}$.
% The trajectory of $U$ for $(t,j) \succeq (t_a, j_a)$ will be analysed in next step.




% \noindent\emph{\underline{\textbf{Step 2:}} }
% Let $(t_a, j_a)$ satisfies $U(t_a,j_a) \leq \nu_2$ and $(t_{j_k^s}, j_k^s) \preceq (t_a,j_a) \preceq (t_{j_{k+1}^s}, j_{k+1}^s-1)$ for some $k \in \mathbb{Z}_{\geq 0}$, then by Claim \ref{claim: invariant during flow}, we know $U(t,j)\leq \nu_2$ for all $(t_a, j_a) \preceq (t,j) \preceq (t_{j_{k+1}^s}, j_{k+1}^s-1)$. By \eqref{eqn: U slow jump}
% \begin{equation}
%     \begin{aligned}
%         U(t^s_{k+1}, j^s_{k+1}) &\leq a_d U(t^s_{k+1}, j^s_{k+1}-1) \\
%         &\leq a_d \nu_2.
%     \end{aligned}
%     \label{eqn: U slow update in nu set}
% \end{equation} 
% By \eqref{eqn: flow then jump} and \eqref{eqn: U slow jump with lambda}, we know that the time it takes for $U$ to decrease from $U(t^s_{k+1}, j^s_{k+1})$ to $\nu_2$ is less then $\tau_{\text{miati}}^s$ since $U(t^s_{k+1}, j^s_{k+1}) \leq a_d \nu_2$, then there exist $(t_b, j_b)$ satisfies $U(t_b,j_b) \leq \nu_2$ and $(t_{j_{k+1}^s}, j_{k+1}^s) \preceq (t_b,j_b) \preceq (t_{k+2}^s, j_{k+2}^s-1)$. Then we can proof $U(t,j) \leq a_d \nu_2$ for all $(t,j) \succeq (t_a, j_a)$ by concatenation. 



% \cyan{Now we can prove $|\xi(0,0)| \leq \Delta$ implies $U(t,j)\leq \Delta^U $ for all $(t,j)\in \text{dom}\, \xi$. We know $|\xi(0,0)| \leq \Delta$ implies $U(0,0) \leq \overline{\alpha}_U(\Delta^y)$ and $U(1,1)\leq \Delta^U$, where $\Delta^U = a_d \overline{\alpha}_U(\Delta^y)$. 
% If $U(0,0) \in [\nu_2, \overline{\alpha}_U(\Delta^y)]$, then from Step 1, we know $U(t,j) \leq \Delta^U $ for all $(t,j) \preceq (t_a,j_a)$, and from Step 2, we know $U(t,j) \leq a_d \nu_2$ for all $(t,j) \succeq (t_a,j_a)$. If $U(0,0) \leq \nu_2$, from Step 2 we know $U(t,j) \leq a_d \nu_2$ for all $(t,j) \in \text{dom}\, \xi$. Since we can assume $\Delta \geq \nu$, by definition of $\nu_2$, we can prove $U(t,j)\leq \Delta^U$ for all $(t,j)\in \text{dom}\, \xi$.
% }






% \noindent\emph{\underline{\textbf{Step 3:}} } Convert to trajectory of $\xi$





% As shown in Step 1, for any $U(0,0) \in [\nu_2, \overline{\alpha}_U(\Delta^y)]$, there exist $(t_a,j_a) \in \text{dom}\, \xi$ such that $U(t,j) \leq \beta_4(U(0,0),t)$ for all $(t,j) \preceq (t_a, j_a)$, where $U(t_a,j_a) \leq \nu_2$.

% Then by \eqref{eqn: No disturbance U sandwich bound}, we have that for all $U(0,0) \in [\nu_2, \overline{\alpha}_U(\Delta^y)]$, $(t,j) \preceq (t_a, j_a)$,
% \begin{equation}
%     \begin{aligned}
%     |\xi^y(t,j)|_{\mathcal{E}^y} &\leq \underline{\alpha}_U^{-1}\left(\beta_4\left(\overline{\alpha}_U(|\xi^y(0,0)|_{\mathcal{E}^y}),t \right)  \right) \\
%     & \coloneqq \beta_5(|\xi^y(0,0)|_{\mathcal{E}^y}),t),
%     \end{aligned}
%     \label{eqn: KL bound of xi^y}
% \end{equation}
% where $\beta_5 \in \mathcal{KL}$.


% Moreover, since $|\xi|_{\mathcal{E}} \leq \zeta_2(|\xi^y|_{\mathcal{E}^y})$, $|\xi^y|_{\mathcal{E}^y} \leq \zeta_2(|\xi|_{\mathcal{E}})$ and $t \geq \max \{ \tau_\text{miati}^f(j-1), 0\}$, we can further prove that for all $U(0,0) \in [\nu_2, \overline{\alpha}_U(\Delta^y)]$ and $(t,j) \preceq (t_a, j_a)$, there exist a $\mathcal{KL}$ function $\beta_6$ such that \cyan{
% \begin{equation}
%     \begin{aligned}
%           |\xi(t,j)|_{\mathcal{E}} &\leq \zeta_2\Big(\beta_5(\zeta_2(|\xi(0,0)|_{\mathcal{E}}),\tfrac{t}{2}+ \tfrac{\max \{ \tau_\text{miati}^f(j-1), 0\}}{2} \big)\Big) \\
%           & \leq \beta_6(|\xi(0,0)|_{\mathcal{E}},t + j).
%     \end{aligned}
%     \label{eqn: KL bound of xi}
% \end{equation}     }


% On the other hand, from Step 2 we have that for any $U(t_a, j_a)\leq \nu_2$, $U(t,j) \leq a_d \nu_2$ for all $(t,j)\succeq (t_a, j_a)$, $(t,j)\in \text{dom}\xi^y$. 
% By \eqref{eqn: U less than nu2 implies xi less than nu}, we know $U(\xi^y) \leq a_d \nu_2$ implies $|\xi|_{\mathcal{E}} \leq \nu$, therefore, $|\xi(t,j)|_{\mathcal{E}} \leq \nu$ for all $(t,j)\succeq (t_a, j_a)$, $(t,j)\in \text{dom}\xi^y$. 



% Now we have sufficient information to conclude the upper bound of $|\xi(t,j)|_{\mathcal{E}}$. We divide the initial condition into two cases:

% \noindent\emph{\underline{Case 3-1:} } 
% Suppose $U(0,0) \in [\nu_2, \overline{\alpha}_U(\Delta^y)]$, then for all $(t,j) \in \text{dom} \ \xi$, we have

% \begin{equation*}
%     |\xi(t,j)|_{\mathcal{E}} \leq \beta_6(|\xi(0,0)|_{\mathcal{E}}, t+j) + \nu,
% \end{equation*}

% \noindent\emph{\underline{Case 3-2:} }
% Suppose $U(0,0) \leq \nu_2$, then we have $|\xi(t,j)|_{\mathcal{E}} \leq \nu$ for all $(t,j) \in \text{dom} \,\xi$. 

% Now we have the trajectory of $\xi$ for all $U(0,0) \in [0, \overline{\alpha}_U(\Delta^y)]$. Since $|\xi|_\mathcal{E} \in[0, \Delta]$ implies $U(0,0) \in [0, \overline{\alpha}_U(\Delta^y)]$, we can conclude that for all $\xi$ with $|\xi(0,0)|_{\mathcal{E}} \leq \Delta$ and $(t,j) \in \text{dom}\,\xi$, we have 
% \begin{equation}
%     |\xi(t,j)|_{\mathcal{E}} \leq \beta_6(|\xi(0,0)|_{\mathcal{E}}, t+j) + \nu, 
% \end{equation}                         

% Now we have proved the set $\mathcal{E}$ is semi-global practical pre-asymptotically stable (SPpAS) for $\mathcal{H}_2$, where ``pre" comes from the fact that we have not prove all the maximal solution to $\mathcal{H}_2$ are complete.
% Since $\mathcal{H}_2$ contains $\mathcal{H}_1$, any solution $\xi(t,j)$ that belongs to $\mathcal{H}_1$ also belongs to $\mathcal{H}_2$. As a result, the set $\mathcal{E}$ is SPpAS for $\mathcal{H}_1$.
% %
% By \cite[Proposition 6.10]{gosate12}, since the viability condition of $\mathcal{H}_1$ is verified, solutions are always bounded and $G(\xi) \in \mathcal{C}_1^\epsilon \cup \mathcal{D}_1^\epsilon$, we conclude that every maximal solutions to $\mathcal{H}_1$ are complete and the set $\mathcal{E}$ is semi-global practical asymptotically stable for $\mathcal{H}_1$, which completes the proof.$\hfill\blacksquare$





