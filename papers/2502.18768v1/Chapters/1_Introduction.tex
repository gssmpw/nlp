Networked control systems (NCSs) integrate feedback control loops with real-time communication networks \cite{definition}. The rapid evolution of network technologies has expanded the applications of NCSs across various sectors, such as industrial automation, smart transportation, telemedicine, and both space and terrestrial exploration \cite{xu2018survey}. These applications often involve dynamic variables evolving in multiple time scales \cite{kokotovic_singular_book}. 
%
Most state-of-the-art research in NCSs, e.g., \cite{dragan_stability,carnevale_stability,heijmans2017computing}, overlook this multi-scale structure, leading to designs demanding high data transmission rates to maintain system stability or robustness. This is especially problematic for Internet of Things devices, which are generally wireless, battery-powered, and have limited bandwidth, processing power, and data storage capacity.
%
A relevant approach in this context is to extend the singularly perturbed method \cite{nonlinear_systems_Khalil} to NCS, thereby introducing the concept of singularly perturbed NCS (SPNCS).



SPNCSs have gained substantial attention for their applicability in various engineering disciplines. 
For instance, \cite{wang2021observer,song2019dynamic,lei2022event} have introduced a range of control and analysis techniques for linear SPNCSs. 
%
For nonlinear SPNCSs, \cite{Romain_ETC} formulated stabilizing event-triggered feedback laws based on the reduced model (also known as the quasi-steady-state model), assuming stable fast dynamics. 
%
Meanwhile, \cite{SPNCS} established sufficient conditions for stability in time-triggered, two-time-scale nonlinear SPNCSs under the scenario that the controller is co-located with the actuators, as well as slow and fast plant states are transmitted over two separate channels, which is not always possible or desirable in practice.
%
More precisely, \cite{lei2022event}, \cite{Romain_ETC} and \cite{SPNCS} adopted an emulation-based approach for NCS design \cite{dragan_stability}, i.e., the controller is designed to stabilize the plant in absence of communication constraints, and subsequently the event- or time-triggered conditions are determined to maintain stability over networks.
%
Furthermore, SPNCSs have been framed within the framework of a hybrid singularly perturbed dynamical system using the formalisms outlined in \cite{gosate12}, with stability tools available in \cite{sanfelice2011singular} and \cite{wang2012analysis}.
%framed within the context of a hybrid singularly perturbed dynamical system with the formalisms of \cite{gosate12}, for which stability tools are available in \cite{sanfelice2011singular} and \cite{wang2012analysis}. % and \cite{6161309}.
%
We highlight that, the existing literature on SPNCSs often assumes either stable slow or fast subsystems, perfect transmission of some signals, and dedicated channels for slow and fast signals.





In this paper, we consider a two-time scale nonlinear system stabilized by a dynamical output feedback controller, inspired by both linear \cite{linear1980}, \cite{linear2018}, \cite{linear2010} and nonlinear \cite{output_feedback} research on dynamic controllers for singularly perturbed dynamical systems (SPSs) \cite{nonlinear_systems_Khalil} without network constraints. Unlike previous work \cite{SPNCS} that assumes only the transmission of plant states, we address scenarios where both plant output and control input are transmitted via the network. 
%Additionally, instead of requiring dedicated channels for slow and fast variables as in \cite{SPNCS}, which is not always practical, we consider the scenario where only one channel is involved, which presents challenges in allocating access for slow and fast signals. 
%
Additionally, instead of requiring dedicated channels for slow and fast variables as in \cite{SPNCS}, which is not always practical, we consider a single-channel scenario, which presents significant challenges in allocating access for slow and fast signals.
Therefore, our objective is to provide a general methodology for the design of the controller and transmission mechanism to ensure stability properties for the SPNCS.
%to design a resource-efficient mechanism to manage the transmission of slow and fast signals over a single communication channel, and \red{identify conditions under which stability properties are preserved.}  





We first design a dual clock mechanism to govern the data transmissions.
Then we represent the SPNCS as a hybrid SPS, incorporating jump sets specifically designed to comply with the previously mentioned clock mechanism.
%Then we model the SPNCS as a hybrid SPS\cyan{;} with jump sets designed to satisfy the aforementioned clock mechanism. 
The obtained SPS is more general compared to those in the literature \cite{sanfelice2011singular,wang2012analysis}%,6161309}
, as its flow and jump sets depend on the time scale separation parameter, which is commonly denoted by $\epsilon$.
%
Following an emulation-based approach, we demonstrate that the if the \emph{reduced system} and \emph{boundary-layer system} are \emph{uniformly globally asymptotically stable} (UGAS) or \emph{uniformly globally exponentially stable} (UGES), and an interconnection condition and some mild conditions are met, the stability properties can be approximately preserved by transmitting both slow and fast variables sufficiently fast.
%
Specifically, we employ a Lyapunov-based analysis to determine individual \emph{maximum allowable transmission interval} (MATI) \cite{dragan_stability,carnevale_stability} of the slow and fast dynamics.
%
Finally, we illustrate the benefits of our approach through a numerical case study.

Compared to our preliminary work \cite{Single_channel_NCS_CDC}, we relax a restrictive condition on the \emph{minimum allowable transmission interval} (MIATI). Additionally, we present conditions that guarantee stronger stability properties: UGAS and UGES. Furthermore, while many works, such as \cite{Romain_ETC}, assume that either the slow or fast subsystem is stable without the need of control, 
%and then achieve the stabilisation of the overall system by implementing a stabilizing controller over the network, 
our methodology does not rely on this assumption. 
Nevertheless, our framework accommodates scenarios involving stable slow or fast subsystems as special cases.

The results are novel even for linear time-invariant (LTI) systems. While \cite{wang2021observer,song2019dynamic,lei2022event} assumed periodic transmissions, we allow transmissions to be aperiodic. Compared to \cite{song2019dynamic,lei2022event} that assumed the sampled-data structure, we consider SPNCSs with scheduling protocols. Moreover, we take into account the inter-event continuous behavior, which was ignored by \cite{wang2021observer,song2019dynamic}. 
%
%Finally, we illustrate the benefits of our approach through a numerical case study.

















\textbf{Notation:} 
Let $\mathbb{R}\coloneqq (-\infty, \infty)$, $\mathbb{R}_{\geq 0} \coloneqq [0,\infty)$, $\mathbb{Z}_{\geq 0} \coloneqq \{0, 1, 2, \cdots \}$ and $\mathbb{Z}_{\geq 1} \coloneqq \{1, 2, \cdots \}$.
%The sets of real numbers and integers larger than or equal to an integer $n$ are denoted by $\mathbb{R}_{\geq n}$ and $\mathbb{Z}_{\geq n}$, respectively. 
For vectors $v_i\in \mathbb{R}^n$, $i\in \{1,2,\cdots, N\}$, we denote the vector $[v_1^\top \; v_2^\top \; \cdots \; v_N^\top]^\top$ by $(v_1, v_2, \cdots, v_N)$, and inner product by $\left< \cdot , \cdot \right>$. 
%
Given a vector $x\in \mathbb{R}^{n_x}$ and a non-empty closed set $\mathcal{A} \subseteq \mathbb{R}^{n_x}$, the distance from $x$ to $\mathcal{A}$ is denoted by $|x|_\mathcal{A} \coloneqq \min_{y\in \mathcal{A}}|x-y|$.
We use $U^\circ(x;v)$ to denote the Clarke generalized derivative \cite[Eqn. (20)]{teel2000assigning} of a locally Lipschitz function $U$ at $x$ in the direction of $v$. 
%
For a real symmetric matrix $P$, we denote its maximum and minimum eigenvalues by $\lambda_{\text{max}}(P)$ and $\lambda_{\text{min}}(P)$ respectively. The logic AND operator is denoted by $\wedge$.




