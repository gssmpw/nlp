%\newpage
\section{Proof of Theorem \ref{Theorem Exponential decay}} 
%
The first step is to show the $\psi_s$ and $\psi_f$ in Lemma \ref{Lemma MATI} are linear. By \eqref{eqn: NCS Ws dot}, \eqref{eqn: NCS Vs flow} and the definition of $U_s$ in \eqref{eqn: definition of U_s}, we can show
$
U_s^\circ(\xi_s; F_s^y(x,0,e_s, 0)) \leq - \rho_s(|x|) - \rho_s\left(W_s(\kappa_d, e_s)\right)
$
along the same line as \cite[(27)]{carnevale_stability}.
%
%
% \begin{equation*}
%     \begin{aligned}
%         & U_s^\circ(\xi_s; F_s^y(x,0,e_s, 0)) \\
%         \leq &\left< \tfrac{\partial V_s}{\partial x}, f_x(x,\overline{H}(x,e_s),e_s,0)\right> \\
%             & + \gamma_d \Big( -2 L_s \phi_s(\tau_s) - \gamma_d \big(\phi_s^2(\tau_s)+1\big) \Big) W_s^2(\kappa_d, e_s) \\
%             &+ 2 \gamma_d\phi_s W_s(\kappa_d,e_s) \left< \tfrac{\partial {W_s}(\kappa_d,{e_s})}{\partial {e_s}}, f_{e_s}(x,\overline{H}(x,e_s),e_s, 0)\right> \\
%         \leq & - \rho_s(|x|) - \rho_s\left(W_s(\kappa_d, e_s)\right) - H_s^2(x,e_s) + \gamma_d^2 W_s^2(\kappa_d, e_s) \\
%             &+ \gamma_d \Big( -2 L_s \phi_s(\tau_s) - \gamma_d \big(\phi_s^2(\tau_s)+1\big) \Big) W_s^2(\kappa_d, e_s) \\
%             &+ 2 \gamma_d \phi_s(\tau_s) W_s(\kappa_d, e_s)\big(L_s {W_s}(\kappa_d, e_s)  + H_s(x,e_s) \big)  \\
%         \leq & - \rho_s(|x|) - \rho_s\left(W_s(\kappa_d, e_s)\right) \\
%             &- \big(H_s(x,e_s) - \gamma_d \phi_s(\tau_s)W_s(\kappa_d, e_s)\big)^2 \\
%         \leq & - \rho_s(|x|) - \rho_s\left(W_s(\kappa_d, e_s)\right).
%     \end{aligned}
% \end{equation*}
Additionally, since $a_{\rho_s} s^2 \leq \rho_s(s)$ for all $s \in \mathbb{R}$, we have $U_s^\circ(\xi_s; F_s^y(x,0,e_s, 0)) \leq -a_{\rho_s} |x|^2 - a_{\rho_s} W_s^2(\kappa_d, e_s)$.
% \begin{equation}
%     U_s^\circ(\xi_s; F_s^y(x,0,e_s, 0)) \leq -a_{\rho_s} |x|^2 - a_{\rho_s} W_s^2(\kappa_d, e_s)
% \end{equation}
Then by \eqref{eqn: Ws exponential sandwich bound}, we have
$
U_s^\circ(\xi_s; F_s^y(x,0,e_s, 0)) \leq  - \big(a_{\rho_s} |x|^2 + a_{\rho_s} W_s^2(\kappa_d,e_s)\big) \leq  - a_{\rho_s} \min \{ 1, \underline{a}_{W_s}^2 \} (|x|^2+|e_s|^2) \eqqcolon  - a_s \psi_s^2(|(x,e_s)|)
$,
% \begin{equation*}
%     \begin{aligned}
%         &U_s^\circ(\xi_s; F_s^y(x,0,e_s, 0)) \\
%         \leq & - \big(a_{\rho_s} |x|^2 + a_{\rho_s} W_s^2(\kappa_d,e_s)\big) \\
%         %\leq & - \big(a_{\rho_s} |x|^2 + a_{\rho_s} \underline{a}_{W_s}^2 |e_s|^2\big) \\
%         \leq & - a_{\rho_s} \min \{ 1, \underline{a}_{W_s}^2 \} (|x|^2+|e_s|^2) \\
%         \eqqcolon & - a_s \psi_s^2(|(x,e_s)|),
%     \end{aligned}
% \end{equation*}
which implies \eqref{eqn: Us flow} is satisfied with $a_s \coloneqq a_{\rho_s} \min \{ 1, \underline{a}_{W_s}^2 \}$ and $\psi_s(|(x,e_s)|) \coloneqq |(x,e_s)|$. 
%
Moreover, we have $\underline{a}_{U_s}|(x,e_s)|^2 \leq U_s(\xi_s) \leq \overline{a}_{U_s}|(x,e_s)|^2,$ where $\underline{a}_{U_s} \coloneqq \min\{\underline{a}_{V_s}, \gamma_s \lambda_s^* \underline{a}_{W_s}^2 \}$ and $\overline{a}_{U_s} \coloneqq \max\{\overline{a}_{V_s}, \gamma_s \tfrac{1}{\lambda_s^*} \overline{a}_{W_s}^2 \}$.

Along the same line as $U_s$, we can proof that \eqref{eqn: Uf flow} is satisfied with $a_f \! \coloneqq \! a_{\rho_f} \min \{ 1, \underline{a}_{W_f}^2 \}$ and $\psi_f(|(y,e_f)|) \coloneqq |(y,e_f)|$. Moreover, we have $\underline{a}_{U_f}|(y,e_f)|^2 \leq U_f(\xi_s, \xi_f) \leq \overline{a}_{U_f}|(y,e_f)|^2,$ where $\underline{a}_{U_f} \coloneqq \min\{\underline{a}_{V_f}, \gamma_f \lambda_f^* \underline{a}_{W_f}^2 \}$ and $\overline{a}_{U_f} \coloneqq \max\{\overline{a}_{V_f}, \gamma_f \tfrac{1}{\lambda_f^*} \overline{a}_{W_f}^2 \}$. Then we satisfy Assumption \ref{Assumption Extra 2} 
% \begin{equation}
% \begin{aligned}
%     \psi_s(|(x,e_s)|) &\leq a_{\psi_s}\sqrt{U_s(\xi_s)}, \\
%     \psi_f(|(y,e_f)|) &\leq a_{\psi_f}\sqrt{U_f(\xi_s,\xi_f)} ,
% \end{aligned}
% \end{equation} \todo{} % Can be replaced by Assumption 5
with $a_{\psi_s} = \underline{a}_{U_s}^{-\frac{1}{2}}$ and $a_{\psi_f} = \underline{a}_{U_f}^{-\frac{1}{2}}$.
%
Same as the proof of Theorem \ref{Theorem H_1}, we define composite Lyapunov function $U$ as $U(\xi_s, \xi_f) \coloneqq U_s(\xi) + d U_f(\xi_s,\xi_f)$, where $d \in (0,1)$. Then $U$ has sandwich bound 
\begin{equation}
    \underline{a}_{U} |\xi^y|_{\mathcal{E}^y}^2 \leq U(\xi^y) \leq \overline{a}_{U} |\xi^y|_{\mathcal{E}^y}^2, \label{eqn: Exponential U sandwich bound}
\end{equation}
where $\underline{a}_{U} \coloneqq \min \{\underline{a}_{U_s}, d \underline{a}_{U_f} \}$ and $\overline{a}_{U} \coloneqq \max \{\overline{a}_{U_s}, d \overline{a}_{U_f} \}$. 


\noindent\emph{\underline{During flow}:} 
We can obtain \eqref{eqn: U derivative during flow eqn1}, as well as
$
U^\circ(\xi^y, 
$
$
F^y(\xi^y, \epsilon)) \leq - 
\left[ \begin{smallmatrix}
            \sqrt{U_s(\xi_s)} \\ \sqrt{U_f(\xi_s, \xi_f)}
        \end{smallmatrix} \right]^T
        \Lambda
        \left[ \begin{smallmatrix}
            \sqrt{U_s(\xi_s)} \\ \sqrt{U_f(\xi_s, \xi_f)}
        \end{smallmatrix} \right]
$,
% \begin{equation*}
%     \begin{aligned}
%         U^\circ(\xi^y, F^y(\xi^y, \epsilon)) \leq - 
%         \left[ \begin{smallmatrix}
%             \sqrt{U_s(\xi_s)} \\ \sqrt{U_f(\xi_s, \xi_f)}
%         \end{smallmatrix} \right]^T
%         \Lambda
%         \left[ \begin{smallmatrix}
%             \sqrt{U_s(\xi_s)} \\ \sqrt{U_f(\xi_s, \xi_f)}
%         \end{smallmatrix} \right],
%     \end{aligned}
% \end{equation*}
where $\Lambda$ is defined in \eqref{eqn: Lambda}, along the same line as the proof of Theorem \ref{Theorem H_1} by setting $\nu_1$ to be zero.
% $\Lambda \coloneqq 
%     \left[ \begin{smallmatrix}
%         a_s a_{\psi_s}^2 & -\tfrac{1}{2}(b_1 + d b_2)a_{\psi_s}a_{\psi_f} \\
%         -\tfrac{1}{2}(b_1 + d b_2)a_{\psi_s}a_{\psi_f} & d (\tfrac{a_f}{\epsilon} - b_3)a_{\psi_f}^2
%     \end{smallmatrix} \right]$. \todo{} % Lambda is same in the proof Theorem 1
In order to satisfy $\Lambda \geq \mu
    \left[ \begin{smallmatrix}
        1 & 0 \\ 0 & d
    \end{smallmatrix} \right]$, where $\mu$ is defined in \eqref{eqn: mu}, we need to satisfy inequality \eqref{eqn: inequality of epsilon Exponential} by having $\epsilon \in (0,\epsilon^*]$, where $\epsilon^*$ is defined by \eqref{eqn: epsilon star} and $d$ in \eqref{eqn: epsilon star} is given later.
% \begin{subequations}
% \begin{align}
%     a_s a_{\psi_s}^2 > \mu \\
%     (a_s a_{\psi_s}^2-\mu) \big( d (\tfrac{a_f}{\epsilon} - b_3)a_{\psi_f}^2 - \mu d \big) &\geq \tfrac{1}{4}(b_1 + db_3)^2a_{\psi_s}^2a_{\psi_f}^2, \label{eqn: inequality of epsilon Exponential}
% \end{align}
% \end{subequations} \todo{}
where the first inequality is satisfied by the definition of $\mu$, and the second inequality can be satisfied by taking $\epsilon$ sufficiently small.
%
Then we have 
\begin{equation}
    \begin{aligned}
        U^\circ(\xi^y, F^y(\xi^y, \epsilon)) &\leq -\mu (U_s(\xi_s) + d U_f(\xi_s,\xi_f)) \\
        & \leq - \mu U(\xi^y).
        \label{eqn: U dot Exponential}
    \end{aligned}
\end{equation}
%
\noindent\emph{\underline{During jumps}:} 
Same as the proof of Theorem \ref{Theorem H_1}, we have $U(G_s^y(\xi^y)) \leq a_d U(\xi^y)$ at slow transmissions and $U(G_f^y(\xi^y)) \leq U(\xi^y)$ at fast transmissions.
%
Suppose $j_k^s, j_{k+1}^s \in \mathcal{J}^s$.
By \eqref{eqn: U dot Exponential}, the fact that $U$ is non-increasing at fast transmissions and comparison principle, we have
\begin{equation}
    U(s,i) \leq U(t_{j_k^s}, j_k^s) \exp \! \big(-\mu(t-t_{j_k^s}) \big) \label{eqn: Exponential U flow - Exponential}, 
\end{equation}
for all $(t_{j_k^s}, j_k^s) \preceq (s,i) \preceq (t_{j_{k+1}^s}, j_{k+1}^s - 1)$ and $(s,i)\in \text{dom}\,\xi^y$.
%
%
Along the same line as deriving \eqref{eqn: flow then jump}, we have $U(t_{j_{k+1}^s}, j^s_{k+1}) \leq a_d U(t_{j_k^s}, j_k^s) \exp (-\mu\tau_{\text{miati}}^s )$.
% \begin{equation*}
%     \begin{aligned}
%         U(t_{j_{k+1}^s}, j^s_{k+1}) &\leq a_d U(t_{j_{k+1}^s},j^s_{k+1} - 1)
%         \\
%         &\leq a_d U(t_{j_k^s}, j_k^s) \exp (-\mu\tau_{\text{miati}}^s ).
%     \end{aligned} %
% \end{equation*}
By definition of $a_d$ in \eqref{eqn: a_d}, we have that for any $\tau_{\text{miati}}^{s} \leq T(L_s, \gamma_s, \lambda_s^*)$, $\lambda \in (\exp (-\mu\tau_{\text{miati}}^s ), 1)$, there exist 
\begin{equation}
d^* = \tfrac{-b+\sqrt{b^2-4a \tilde{c}}}{2a},
\label{eqn: d star exponential}
\end{equation}
where $a = \tfrac{\lambda_1}{\gamma_s \lambda_s^*}$, $b= \tfrac{1}{2}( \tfrac{\lambda_1}{\gamma_s \lambda_s^*} + \lambda_2)$ and $\tilde{c}= 1 - \lambda e^{\mu \tau_{\text{miati}}^s}$, such that by taking $d =d^*$, we have $U(t_{j_{k+1}^s}, j^s_{k+1}) \leq \lambda U(t_{j_k^s},j^s_k)$.
% \begin{equation*}
%     U(t_{j_{k+1}^s}, j^s_{k+1}) \leq \lambda U(t_{j_k^s},j^s_k)
% \end{equation*}
Then the inequality \eqref{eqn: inequality of epsilon Exponential} is satisfied by all $\epsilon \in (0, \epsilon^*)$, where $\epsilon^*$ is defined in \eqref{eqn: epsilon star} with $d = d^*$.
By concatenation, we have $U(t_{j_k^s}, j^s_k) \leq  \lambda^{k-1}U(t_{j_1^s}, j_1^s)$.
% \begin{equation*}
%     \begin{aligned}
%         U(t_{j_k^s}, j^s_k) \leq & \lambda^{k-1}U(t_{j_1^s}, j_1^s),
%     \end{aligned}
% \end{equation*}
%
Moreover, since $U$ is non-increasing during flow and upper bounded by $U(G_s^y(\xi^y)) \leq a_d U(\xi^y)$ at slow transmission, we have $ U(t_{j_1^s},j_1^s) \leq  a_d U(0,0) $. Then we have $U(t_{j_k^s}, j_k^s) \leq a_d \lambda^{k-1} U(0,0)$.
% \begin{equation}
%     U(t_{j_k^s}, j_k^s) \leq a_d \lambda^{k-1} U(0,0). \label{eqn: Exponential U slow jump decay}
% \end{equation}
%
Now we have obtained the upper bound of trajectory during the interval between slow transmissions (i.e., \eqref{eqn: Exponential U flow - Exponential}) and the upper bound at each slow transmission. 
%
Then along the same line as the proof of Claim \ref{Claim U upper bound}, by setting $\Delta$ to infinity and $\nu$ to zero, we can show $U(\xi^y(t,j)) \leq \frac{a_d}{\lambda}U(0,0)  \exp \!\left(-\tfrac{\ln{(\nicefrac{1}{\lambda})}}{\tau_{\text{mati}}^s} t \right)$,
%
% \begin{equation*}
%     U(t,j) \leq \frac{a_d}{\lambda}U(0,0)  \exp \!\left(-\tfrac{\ln{(\nicefrac{1}{\lambda})}}{\tau_{\text{mati}}^s}(t) \right),
% \end{equation*}
for all $(t,j) \in  \text{dom} \ \xi^y$.
%
% \begin{claim}
%     The following upper bound holds for all $(t_{j_k^s}, j_k^s) \preceq (t,j) \in \text{dom}\ \xi$
%     \begin{equation*}
%         U(t,j) \leq \frac{a_d}{\lambda}U(t_{j_k^s},j_k^s)  \exp \!\left(-\tfrac{\ln{(\nicefrac{1}{\lambda})}}{\tau_{\text{mati}}^s}(t-t_{j_k^s}) \right). 
%     \end{equation*}
%     \label{Claim U upper bound Exponential}
% \end{claim}
% \textbf{Proof of Claim \ref{Claim U upper bound Exponential}:} 
% \textbf{Proof of Claim \ref{Claim U upper bound Exponential}:}
% During flow, since $t_{j_{k+1}^s} - t_{j_k^s} \leq \tau_{\text{mati}}^s$, $a_d \geq 1$, $\lambda \in (0,1)$, $\tau_{\text{miati}}^s < \tau_{\text{mati}}^s$ and definition of $\lambda$, we have that for all $(t_{j_k^s}, j_k^s) \preceq (t,j) \preceq (t_{j_{k+1}^s}, j_{k+1}^s-1)$, we have
% \begin{equation*}
%     \begin{aligned}
%         &\frac{a_d}{\lambda}U(t_{j_k^s},j_k^s)  \exp \!\left(-\tfrac{\ln{(\nicefrac{1}{\lambda})}}{\tau_{\text{mati}}^s}(t-t_{j_k^s}) \right) \\
%         \geq & U(t_{j_k^s},j_k^s) \exp \!\left( -\ln{(\nicefrac{1}{\lambda})}\right)^{\tfrac{t-t_{j_k^s}}{\tau_{\text{mati}}^s}} \\
%         = & U(t_{j_k^s},j_k^s) \lambda^{\tfrac{t-t_{j_k^s}}{\tau_{\text{mati}}^s}} \\
%         = &  U(t_{j_k^s},j_k^s) \big(a_d \exp (-\mu \tau_{\text{miati}}^s) \big)^{\tfrac{t-t_{j_k^s}}{\tau_{\text{mati}}^s}} \\
%         = & U(t_{j_k^s},j_k^s) {a_d}^{\tfrac{t-t_{j_k^s}}{\tau_{\text{mati}}^s}} \exp \! \big(-\mu \tfrac{\tau_{\text{miati}}^s}{\tau_{\text{mati}}^s} (t - t_{j_k^s})\big) \\
%         \geq & U(t_{j_k^s},j_k^s)\exp\!\big(-\mu (t - t_{j_k^s})\big) ,
%     \end{aligned}
% \end{equation*}
% Then by \eqref{eqn: Exponential U flow}, we validate Claim \ref{Claim U upper bound} during the interval between slow transmissions. 
%
% Next, we will check the upper bound at each slow transmission. Since $\tfrac{t_{j_k^s}}{\tau_{\text{mati}}^s} \leq k$, for all $k\in \mathbb{Z}_{\geq 0}$, we have 
% \begin{equation*}
%     \begin{aligned}
%         & \frac{a_d}{\lambda}U(0,0)  \exp \!\left(-\tfrac{\ln{(\nicefrac{1}{\lambda})}}{\tau_{\text{mati}}^s}(t_{j_k^s}-0) \right) \\
%         \geq & \frac{a_d}{\lambda}U(0,0)  \exp \!\left(-\ln{(\nicefrac{1}{\lambda})} k \right) \\
%         = & a_d U(0,0) \lambda^{k-1}.
%     \end{aligned}
% \end{equation*}
%  By \eqref{eqn: Exponential U slow jump decay}, we validate Claim \ref{Claim U upper bound} at slow transmissions. Then we have prove Claim \ref{Claim U upper bound}. $\hfill\square$
%
 % By Claim \ref{Claim U upper bound}, we have 
 % \begin{equation}
 %     U(t,j) \leq \tfrac{a_d}{\lambda}U(0,0)  \exp \!\left(-\tfrac{\ln{(\nicefrac{1}{\lambda})}}{\tau_{\text{mati}}^s}t \right),
 %     \label{eqn: Exponeltial U upperbound}
 % \end{equation}
 % for all $(t,j) \in \text{dom} \ \xi$.
 %
 By \eqref{eqn: Exponential U sandwich bound}, we have
 $
 |\xi^y(t,j)|_{\mathcal{E}^y} 
 %
\leq  \big(\tfrac{1}{\underline{a}_U} U(t,j)\big)^{\nicefrac{1}{2}}
%
\leq  \left(\tfrac{a_d}{\lambda \underline{a}_U}U(0,0)  \exp \!\left(-\tfrac{\ln{(\nicefrac{1}{\lambda})}}{\tau_{\text{mati}}^s}t \right) \right)^{\nicefrac{1}{2}} 
%
=  \left(\tfrac{a_d \overline{a}_U}{\lambda \underline{a}_U} \right)^{\nicefrac{1}{2}}  |\xi^y(0,0)|_{\mathcal{E}^y}  \exp \!\left(-\tfrac{\ln{(\nicefrac{1}{\lambda})}}{ \tau_{\text{mati}}^s}t \right)^{\nicefrac{1}{2}}
$.
 % \begin{equation*}
 %     \begin{aligned}
 %         &|\xi^y(t,j)|_{\mathcal{E}^y}  
 %         \leq  \big(\tfrac{1}{\underline{a}_U} U(t,j)\big)^{\nicefrac{1}{2}} 
 %         \\
 %         \leq & \left(\tfrac{a_d}{\lambda \underline{a}_U}U(0,0)  \exp \!\left(-\tfrac{\ln{(\nicefrac{1}{\lambda})}}{\tau_{\text{mati}}^s}t \right) \right)^{\nicefrac{1}{2}} 
 %         \\
 %         \leq & \left(\tfrac{a_d}{\lambda \underline{a}_U} \overline{a}_U |\xi^y(0,0)|_{\mathcal{E}^y}^2  \exp \!\left(-\tfrac{\ln{(\nicefrac{1}{\lambda})}}{\tau_{\text{mati}}^s}t \right) \right)^{\nicefrac{1}{2}} \\    
 %         = & \left(\tfrac{a_d \overline{a}_U}{\lambda \underline{a}_U} \right)^{\nicefrac{1}{2}}  |\xi^y(0,0)|_{\mathcal{E}^y}  \exp \!\left(-\tfrac{\ln{(\nicefrac{1}{\lambda})}}{ \tau_{\text{mati}}^s}t \right)^\frac{1}{2}.
 %     \end{aligned}
 % \end{equation*} 
%
Since $\overline{H}$ is globally Lipschitz and $\overline{H}(0,0) = 0$, we have $\overline{H}(x,e_s) \leq L|(x,e_s)|$, where $L$ is the Lipschitz constant. 
% 
 Then by $y = z - \overline{H}(x,e_s)$, there exist $h_1 = 1 + L$ such that
 $|\xi(t,j)|_{\mathcal{E}} \leq h_1 |\xi^y(t,j)|_{\mathcal{E}^y} $ and $|\xi^y(t,j)|_{\mathcal{E}^y} \leq h_1 |\xi(t,j)|_{\mathcal{E}} $. Then the upper bound of $|\xi(t,j)|_{\mathcal{E}} $ is
 $
 |\xi(t,j)|_{\mathcal{E}} \leq h_1^2 \left(\tfrac{a_d \overline{a}_U}{\lambda \underline{a}_U} \right)^{\nicefrac{1}{2}}  |\xi(0,0)|_{\mathcal{E}}  \exp \!\left(-\tfrac{\ln{(\nicefrac{1}{\lambda})}}{ \tau_{\text{mati}}^s}t \right)^{\nicefrac{1}{2}}
 $.
 % \begin{equation*}
 %     |\xi(t,j)|_{\mathcal{E}} \leq h_1^2 \left(\tfrac{a_d \overline{a}_U}{\lambda \underline{a}_U} \right)^{\nicefrac{1}{2}}  |\xi(0,0)|_{\mathcal{E}}  \exp \!\left(-\tfrac{\ln{(\nicefrac{1}{\lambda})}}{ \tau_{\text{mati}}^s}t \right)^\frac{1}{2}.
 % \end{equation*}
% Additionally, since $t \geq \tau_{\text{miati}} (j-1)$, we have
% \begin{equation*}
%     \begin{aligned}
%         &\exp \!\left(-\tfrac{\ln{(\nicefrac{1}{\lambda})}}{2 \tau_{\text{mati}}^s}(t) \right) \\
%         %=&  \exp \!\left(-\tfrac{\ln{(\nicefrac{1}{\lambda})}}{2 \tau_{\text{mati}}^s}(\tfrac{t}{2}+\tfrac{t}{2})) \right) \\
%         \leq & \exp \!\left(-\tfrac{\ln{(\nicefrac{1}{\lambda})}}{2 \tau_{\text{mati}}^s}(\tfrac{t}{2}+\tfrac{\tau_{\text{miati}}^s}{2}(j-1))) \right) \\
%         %=& \exp \!\left( \tfrac{\ln{(\nicefrac{1}{\lambda})}\tau_{\text{miati}}^s}{4 \tau_{\text{mati}}^s}\right)   
%             %\exp \!\left(-\tfrac{\ln{(\nicefrac{1}{\lambda})}}{4 \tau_{\text{mati}}^s}(t+\tau_{\text{miati}}^sj) \right) \\
%         \leq & \exp \!\left( \tfrac{\ln{(\nicefrac{1}{\lambda})}\tau_{\text{miati}}^s}{4 \tau_{\text{mati}}^s}\right)   
%             \exp \!\left(-\tfrac{\ln{(\nicefrac{1}{\lambda})}}{4 \tau_{\text{mati}}^s}  \min\{1,\tau_{\text{miati}}^s \} (t+j) \right).
%     \end{aligned}
% \end{equation*}
By \eqref{eqn: change t to t+j}, we have 
$
|\xi(t,j)|_{\mathcal{E}} \leq c_1 |\xi(0,0)|_{\mathcal{E}}\exp \! \big(- c_2 (t+j)\big)
$,
% \begin{equation*}
%     |\xi(t,j)|_{\mathcal{E}} \leq c_1 |\xi(0,0)|_{\mathcal{E}}\exp \! \big(- c_2 (t+j)\big),
% \end{equation*}
where $c_1 = h_1^2 \left(\tfrac{a_d \overline{a}_U}{\lambda \underline{a}_U} \right)^{\nicefrac{1}{2}} \exp \!\left( \tfrac{\ln{(\nicefrac{1}{\lambda})}\tau_{\text{miati}}^s}{4 \tau_{\text{mati}}^s}\right)$ and $c_2 = \tfrac{\ln{(\nicefrac{1}{\lambda})}}{4 \tau_{\text{mati}}^s}  \min\{1,\tau_{\text{miati}}^s \}$.
%Now we have shown $\mathcal{H}_1$ is uniformly globally pre-exponentially stable w.r.t $\mathcal{E}$, and we can prove $\mathcal{H}_1$ is UGES along the same line as the proof of Theorem \ref{Theorem H_1} (i.e. completeness of solution). $\hfill\blacksquare$