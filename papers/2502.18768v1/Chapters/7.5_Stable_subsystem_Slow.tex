\subsection{Stable slow subsystem}
The second scenario is when the plant has a stable slow subsystem. A controller is designed to stabilise the fast subsystem.
Subsequently, the plant only needs to transmit fast output $y_f$ and the controller only has fast control input $u_f$. 
The plant and controller shown below are in the form of \eqref{eqn:plant} and \eqref{eqn:controller}, where $k_{ps} $ and $k_{cs}$ are removed.
\begin{equation*}
    \mathcal{P}\!:\!
    \begin{cases}
    \dot x_p \!\!\!\!\!\!&= f_p(x_p, z_p,\hat u)\\
    \epsilon \dot z_p\!\!\!\!\!\! &= g_p(x_p, z_p, \hat u) \\
    y_p \!\!\!\!\!\!&= y_f = k_{p_f}(x_p,z_p)  , 
    \end{cases} \;
    \mathcal{C}\!:\!
    \begin{cases}
    \dot x_c \!\!\!\!\!\!&= f_c(x_c, z_c, \hat{y}_p)\\
    \epsilon \dot z_c \!\!\!\!\!\!&= g_c(x_c, z_c, \hat y_p) \\
    u \!\!\!\!\!\!&= u_f = k_{c_f}(x_c,z_c) ,
    \end{cases}
\end{equation*}

Given the exclusive presence of fast transmissions within the communication channel, a simpler version of clock mechanism \eqref{eqn: Stefan timer} is used to govern the system dynamics as follows:
\begin{equation*}
    \tau_{\text{miati}}^f \leq t_{k+1}^f - t_k^f \leq \tau_{\text{mati}}^f, \ \forall t_k^f, t_{k+1}^f\in \mathcal{T}^f, k \in \mathbb{Z}_{\geq 0},
\end{equation*}
where $\tau_{\text{miati}}^f \geq \tau_{\text{miati}}^f$. Moreover, the networked induced error is $e_f =  (\hat{y}_f - y_f, \hat{u}_f - u_f)$ and $e_s$ does not exist. By define the state $\xi_6 \coloneqq (x,z,e_f, \tau_f, \kappa_f,)\in \mathbb{X}_6$, where $\mathbb{X}_6\coloneqq \mathbb{R}^{n_x} \times \mathbb{R}^{n_z} \times \mathbb{R}^{n_{e_f}}\times  \mathbb{R}_{\geq 0} \times \mathbb{Z}_{\geq 1} $, we can define our system by the hybrid model $\mathcal{H}_6$, which we skip its expression here.


% \begin{equation*}
%     \mathcal{H}_5:\left\{
% \begin{aligned}
%     \dot{\xi}_5 &= F_5(\xi_5),\ \xi_5 \in \mathcal{C}_5, \\
%     \xi^+ &= G_{5,s}(\xi_5), \ \xi_5 \in \mathcal{D}_{5,s},
% \end{aligned}
%     \right.
% \end{equation*}
% where $F_5(\xi_5) \coloneqq \big(f_x(x,z,e_s,e_f), f_{e_s}(x,z,e_s,e_f),1,0,$ $\tfrac{1}{\epsilon} g_z(x,z,e_s,e_f) \big)$, $C_5 \coloneqq \{ \xi_5 \in \mathbb{X}_5 | \tau_s \in [0, \tau_{\text{mati}}^s]\}$ , $G_{5,s}(\xi_5) \coloneqq (x, h_s(\kappa_s, e_s), 0, \kappa_s + 1, z)$ and $D_{5,s} \coloneqq \{ \xi_5 \in \mathbb{X}_5 | \tau_s \in [\tau_{\text{miati}}^s, \tau_{\text{mati}}^s]\}$. We emphasize again that $k_{pf} \equiv 0$, $k_{cf} \equiv 0$ and $e_f \equiv 0$ in $\mathcal{H}_5$.
Similar to $\mathcal{H}_5$, $\mathcal{H}_6$ is a specialized model of $\mathcal{H}_1$, with lower dimension. The functions will be used to define $\mathcal{H}_6$ and $\mathcal{H}_6^y$ have the same form as $\mathcal{H}_1$, but all the elements related to $e_s$, $k_{ps}$ and $k_{cs}$ are removed.
We define $y \coloneqq z - \overline{H}(x)$, where $\overline{H}$ comes from SA\ref{assum:standing-ss}. Let $\xi_6^y \coloneqq (x,y,e_f,\tau_f, \kappa_f) \in \mathbb{X}_6$
Then $\mathcal{H}_6^y$, which is $\mathcal{H}_6$ after changing the coordinates, is given by 
\begin{equation*}
    \mathcal{H}_6^y:\left\{
\begin{aligned}
    \dot{\xi}_6^y &= F_6^y(\xi_6^y, \epsilon),\ \xi_6^y \in \mathcal{C}_6^{y,\epsilon}, \\
    {\xi_6^y}^+ &= G_{6,f}(\xi_6^y), \ \xi_6^y\in \mathcal{D}_{6,f}^{y,\epsilon},
\end{aligned}
    \right.
\end{equation*}
where  $F_6^y(\xi_6^y) \coloneqq \big(f_x(x,y+\overline{H}(x),e_f), \tfrac{1}{\epsilon} F_f^y(x,y,e_f,\epsilon) \big)$,
with $F_f^y$ coming from system \eqref{eqn: H_2^y}, $C_6^{y,\epsilon} \coloneqq \{ \xi_6^y \in \mathbb{X}_6 |\epsilon \tau_f \in [0, \tau_{\text{mati}}^f]\}$ , $G_{6,f}^y(\xi_6^y) \coloneqq \big(x, y,h_f(\kappa_f, e_f), 0, \kappa_f + 1\big)$ and $D_{6,f}^{y,\epsilon} \coloneqq \{ \xi_6^y \in \mathbb{X}_6^y | \epsilon \tau_f \in [\tau_{\text{miati}}^f, \tau_{\text{mati}}^f]\}$.


We recall that $\xi_f\coloneqq (y,e_f, \tau_f, \kappa_f)$, and we can derive a hybrid boundary-layer system $\mathcal{H}_{6,bl}$ by first writing $\tau_{\text{mati}}^f =  \epsilon T^*$ and $\tau_{\text{miati}}^f = a \epsilon T^*$, then we have
\begin{equation*}
    \mathcal{H}_{6,bl}:\left\{
\begin{aligned}
    (\tfrac{\partial x}{\partial \sigma}, \tfrac{\partial \xi_f}{\partial \sigma}) &= \big(0, F_f^y(x,y,e_f,0) \big),\ \xi_6^y \in \mathcal{C}_6^{y,0}, \\
    {\xi_6^y}^+ &= \big(x,h_f(\kappa_f, e_f), 0, \kappa_f + 1\big), \ \xi_6^y\in \mathcal{D}_{6,f}^{y,0},
\end{aligned}
    \right.    
\end{equation*}
where $\mathcal{C}_6^{y,0}\coloneqq \{\xi_6^y \in \mathbb{X} | \tau_f \in [0, T^*] \}$ and $\mathcal{D}_{6,f}^{y,0} \coloneqq \{\xi_6^y \in \mathbb{X} | \tau_f \in [aT^*, T^*] \}$.


Then we can derive a continuous time reduced system $\mathcal{H}_{6,r}$ given by 
\begin{equation*}
    \mathcal{H}_{6,r} : \begin{cases}
        \tfrac{\partial x}{\partial \sigma} = f_z(x,\overline{H}(x),0).
    \end{cases}
\end{equation*}



Since $\mathcal{H}_{6,bl}$ has the same form as $\mathcal{H}_{bl}$, Assumption \ref{Assumption boundary layer system} is applicable to $\mathcal{H}_{6,bl}$, then by Lemma \ref{Lemma MATI}, there exist $U_f(x, \xi_f)$ that satisfies \eqref{eqn: Uf}, with $\mathcal{C}_{2,bl}^{y,0}$ and $\mathcal{D}_f^{y,0}$ replaced by $\mathcal{C}_6^{y,0}$ and $\mathcal{D}_{6,f}^{y,0}$, respectively. But since there are no slow transmissions, Assumption \ref{Assumption reduced model} and \eqref{eqn: Us} in Lemma \ref{Lemma MATI} should be combined and replaced by the following assumption.
\begin{assum}
    Consider $\mathcal{H}_6^y$, there exist locally Lipschitz function $V_s(x)$, class $\mathcal{K}_\infty$ functions $\underline{\alpha}_{V_s}$ and $\overline{\alpha}_{V_s}$, positive real number $a_s$ and positive definite function $\psi_s$ such that for all $x \in \mathbb{R}^{n_x}$, we have
    \begin{align}
    \underline{\alpha}_{V_s}\left(\left| x \right|\right)\leq {V_s}(x) & \leq \overline{\alpha}_{V_s}\left(\left| x\right|\right) \\
     \left< \tfrac{\partial {V_s}(x)}{\partial x},f_x(x,\overline{H}(x),0)  \right> &\leq -a_s \psi_s^2 \left(| x |\right).
     \label{eqn: Assumption stable slow subsystem, Us flow}
    \end{align}
    \label{Assumption Stable slow subsystem}
\end{assum}

Moreover, Assumption \ref{Assumption interconnection} is applicable to $\mathcal{H}_6^y$, where $\xi_s$, $F_s^y(x,y,e_s,e_f)$ and $F_s^y(x,0,e_s,0)$ reduced to $x$, $f_x(x,y+\overline{H}(x),e_f)$ and $f_x(x,\overline{H}(x),0)$, respectively.



We define the set the set $\mathcal{E}_6 \coloneqq \{\xi_6 \in \mathbb{X}_6 | x=0 \wedge z = 0 \wedge e_f=0 \}$.
\begin{cor}
    Consider system $\mathcal{H}_6$ and suppose Assumptions \ref{Assumption reduced model}, \ref{Assumption Extra 2} and \ref{Assumption Stable slow subsystem} hold, and Assumption \ref{Assumption interconnection} holds with $\widetilde{\mathcal{C}} = \mathcal{C}_6^{y,\epsilon}$.
    Let $b_1$, $b_2$, and $b_3$ come from Assumption \ref{Assumption interconnection}, $a_s$ comes from Assumption \ref{Assumption Stable slow subsystem} and $a_f$ comes from Lemma \ref{Lemma MATI}. Then for any $\tau_{\text{miati}}^f \leq \tau_{\text{mati}}^f \leq T(L_f, \gamma_f, \lambda_f^*)$, the following statement holds:

    There exists a $\mathcal{KL}$-function $\beta$, such that for all $\Delta,\nu>0$, there exists $\epsilon^* >0$ such that for all $0<\epsilon<\epsilon^*$, any solution $\xi_6$ with $ |\xi_6(0,0)|_{\mathcal{E}_6}<\Delta$ satisfies $|\xi_6(t,j)|_{\mathcal{E}_6} \leq \beta(|\xi_6(0,0)|_{\mathcal{E}_6}, t+j) + \nu$ for any $(t,j)\in \text{dom} \, \xi_6$.
    
    \label{Corollary Stable slow subsystem}
\end{cor}
\textbf{Proof:}
    Corollary \ref{Corollary Stable slow subsystem} can be proved using similar steps as Theorem \ref{Theorem H_1} and Corollary \ref{Corollary Stable fast subsystem}, except the composite Lyapunov function is now $U(x,\xi_f) = V_s(x) + dU_f(\xi_f) $ and is non-increasing at slow transmissions.
