Networked control systems (NCSs) integrate feedback control loops with real-time communication networks \cite{definition}. The rapid evolution of network technologies has expanded NCS applications across various sectors, such as industrial automation, smart transportation, telemedicine, and exploration \cite{xu2018survey}, often involving processes on multiple time scales. For instance, robotic systems have rapid control loops for electrical dynamics and slower ones for mechanical dynamics. Most state-of-the-art research in NCSs \cite{dragan_stability,carnevale_stability,heijmans2017computing} typically overlook this multi-scale structure, leading to designs demanding high data transmission rates to maintain system stability or robustness. This is especially problematic for Internet of Things (IoT) devices, which are generally wireless, battery-powered, and have limited bandwidth, processing power, and storage capacity. Extending singularly perturbed methods \cite{nonlinear_systems_Khalil} to NCSs results in singularly perturbed NCSs (SPNCSs).

SPNCSs have gained substantial attention for their applicability in various engineering disciplines. Researchers have developed various control and analysis techniques for both linear and nonlinear SPNCSs. For instance, \cite{wang2021observer,song2019dynamic,cheng2021ultimate,lei2022event} introduced a range of control and analysis techniques for linear SPNCSs. For nonlinear SPNCSs, \cite{Romain_ETC} formulated stabilizing event-triggered feedback laws based on the reduced model, assuming stable fast dynamics, while \cite{SPNCS} established sufficient conditions for stability in time-triggered, two-time-scale nonlinear SPNCSs under the scenario of separate channels for slow and fast signals, which is not always practical.

In this paper, we consider a two-time scale nonlinear system stabilized by a dynamical output feedback controller, inspired by both linear \cite{linear1980,linear2018,linear2010} and nonlinear \cite{output_feedback} research on dynamic controllers for singularly perturbed dynamical systems (SPSs) without network constraints. Unlike previous work \cite{SPNCS} that only transmits plant states, we address scenarios where both plant output and controller input are transmitted via a single communication channel, presenting challenges in allocating access for slow and fast signals. Our objective is to design a resource-efficient mechanism using dual clocks to manage the transmission of slow and fast signals, thereby stabilizing the SPNCS over a single communication channel.

We model the SPNCS as a hybrid SPS, with jump sets designed to satisfy the dual clock mechanism. Our hybrid SPS is more general compared to those in the literature \cite{sanfelice2011singular,wang2012analysis,6161309}, as its flow and jump sets depend on the time scale separation parameter, $\epsilon$. Following an emulation-based approach, we demonstrate that the stability properties of the SPNCS, such as semi-global practical asymptotic stability, can be ensured by sufficiently fast transmission of both slow and fast variables. Specifically, we use a Lyapunov-based analysis to determine the maximum allowable transmission interval (MATI) for the slow and fast dynamics.

Compared to our preliminary work \cite{Single_channel_NCS_CDC}, we relax a restrictive condition on the minimum allowable transmission interval (MIATI) and present conditions guaranteeing stronger stability properties, such as uniformly globally asymptotic stability (UGAS) and uniformly globally exponential stability (UGES). While many literature results \cite{Romain_ETC} assume that either the slow or fast subsystem is stable without control, our methodology does not rely on this assumption. Our framework accommodates scenarios involving stable slow or fast subsystems as special cases.

Our results are novel even for linear time-invariant (LTI) systems. Compared to linear works \cite{song2019dynamic,lei2022event} that assumed the sampled-data structure, we consider SPNCSs with scheduling protocols. While \cite{wang2021observer,song2019dynamic,cheng2021ultimate,lei2022event} assumed periodic transmissions, we allow aperiodic transmissions and consider inter-event continuous behavior, often ignored by others. We recast several assumptions as linear matrix inequalities (LMIs) and demonstrate that other assumptions can always be satisfied. Finally, we illustrate the benefits of our approach through a numerical case study.





%%%%%%%%%%%%%
\textbf{Notation:} The sets of real numbers and integers larger than or equal to an integer $n$ are denoted by $\mathbb{R}_{\geq n}$ and $\mathbb{Z}_{\geq n}$, respectively. 
%
% A function $\alpha: \mathbb{R}_{\geq 0} \rightarrow \mathbb{R}_{\geq 0}$ is said to be of class $\mathcal{K}$ if it is continuous, positive definite and strictly increasing. It is of class $\mathcal{K}_{\infty}$ if it is also unbounded. 
% %
% A function $\beta:\mathbb{R}_{\geq 0} \times \mathbb{R}_{\geq 0} \rightarrow \mathbb{R}_{\geq 0}$ is said to be class $\mathcal{KL}$ if for each $s \geq 0$, the function $\beta(s, \cdot)$ is decreasing to zero in the second argument and for each constant $t \geq 0$, the function $\beta(\cdot, t)$ is of class $\mathcal{K}$.
%
For vectors $v_i\in \mathbb{R}^n$, $i\in \{1,2,\cdots, N\}$, we denote the vector $[v_1^\top \; v_2^\top \; \cdots \; v_N^\top]^\top$ by $(v_1, v_2, \cdots, v_N)$, and inner product by $\left< \cdot , \cdot \right>$. 
%
Given a vector $x\in \mathbb{R}^{n_x}$ and a non-empty closed set $\mathcal{A} \subseteq \mathbb{R}^{n_x}$, the distance from $x$ to $\mathcal{A}$ is denoted by $|x|_\mathcal{A} \coloneqq \min_{y\in \mathcal{A}}|x-y|$.
We use $U^\circ(x;v)$ to denote the Clarke generalized derivative \cite[Eqn. (20)]{teel2000assigning} of a locally Lipschitz function $U$ at $x$ in the direction of $v$. 
%
For a real symmetric matrix $P$, we denote its maximum and minimum eigenvalues by $\lambda_{\text{max}}(P)$ and $\lambda_{\text{min}}(P)$ respectively. The logic AND operator is denoted by $\wedge$.

