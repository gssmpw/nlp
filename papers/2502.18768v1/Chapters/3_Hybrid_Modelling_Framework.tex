In this section, we present a hybrid system model for the SPNCS described in Section \ref{Chapter Problem setting} in the formalism of \cite{gosate12}, and it is more general than the hybrid SPSs in the literature such as \cite{sanfelice2011singular} and \cite{wang2012analysis}, as its flow and jump sets depend on $\epsilon$.
%
Firstly, we design a clock mechanism to satisfy \eqref{eqn: timer eqn1}-\eqref{eqn: timer eqn3}, and then we present the model of the overall SPNCS.

\subsection{Clock Mechanism}
We introduce two clocks and two counters, namely $\tau_s, \tau_f \in \mathbb{R}_{\geq 0}$ and $\kappa_s, \kappa_f \in \mathbb{Z}_{\geq 0}$. In particular, $\tau_s$ and $\epsilon \tau_f$ record the time elapsed since the last slow and fast transmission, respectively.
%, and we have $\dot{\tau}_s = 1$, $\epsilon \dot{\tau}_f = 1$ during flow.
Meanwhile, $\kappa_s$ and $\kappa_f$ count the number of slow and fast transmissions, respectively, and are useful for implementing some commonly used protocols, such as RR. 

Let $\xi \coloneqq (x,e_s, \tau_s, \kappa_s, z,e_f, \tau_f,  \kappa_f)\in \mathbb{X}$,
with $\mathbb{X}\coloneqq \mathbb{R}^{n_x}\times \mathbb{R}^{n_{e_s}}\times  \mathbb{R}_{\geq 0} \times \mathbb{Z}_{\geq 0} \times \mathbb{R}^{n_z}\times \mathbb{R}^{n_{e_f}}\times \mathbb{R}_{\geq 0} \times \mathbb{Z}_{\geq 0}$, 
denote the full state of the hybrid system. We define the jump sets $\mathcal{D}_s^\epsilon$, $\mathcal{D}_f^\epsilon$ and the flow set $\mathcal{C}_1^\epsilon$ as
%
$\mathcal{D}_s^\epsilon \coloneqq  \{\xi \in \mathbb{X} \; | \; \tau_s \in [\tau_{\text{miati}}^s, \tau_{\text{mati}}^s] \wedge \epsilon \tau_f \in  [\tau_{\text{miati}}^f,  \tau_{\text{mati}}^f - \tau_{\text{miati}}^f]  \}$,
%
$\mathcal{D}_f^\epsilon \coloneqq \{\xi \in \mathbb{X} \; | \; \tau_s \in [\tau_{\text{miati}}^f, \tau_{\text{mati}}^s-\tau_{\text{miati}}^f]    \wedge\; \epsilon \tau_f \in  [\tau_{\text{miati}}^f, \tau_{\text{mati}}^f]   \}$,
%
and
$\mathcal{C}_1^\epsilon \coloneqq 
        \mathcal{D}_s^\epsilon \cup \mathcal{D}_f^\epsilon \cup \mathcal{C}_{1,a}^\epsilon \cup \mathcal{C}_{1,b}^\epsilon$,
%
% \begin{align*}
%     \mathcal{D}_s^\epsilon \coloneqq & \Big\{\xi \in \mathbb{X} \; | \; \tau_s \in [\tau_{\text{miati}}^s, \tau_{\text{mati}}^s]  \\
%         & \qquad \qquad \qquad \qquad \wedge \epsilon \tau_f \in  [\tau_{\text{miati}}^f,  \tau_{\text{mati}}^f - \tau_{\text{miati}}^f]  \Big\},
%     \\
%     \mathcal{D}_f^\epsilon \coloneqq &\Big\{\xi \in \mathbb{X} \; | \; \tau_s \in [\tau_{\text{miati}}^f, \tau_{\text{mati}}^s-\tau_{\text{miati}}^f]  \\
%     & \qquad \qquad \qquad \qquad \qquad \quad   \wedge\; \epsilon \tau_f \in  [\tau_{\text{miati}}^f, \tau_{\text{mati}}^f]  \Big \},
%     \\
%     \mathcal{C}_1^\epsilon \coloneqq & 
%         \mathcal{D}_s^\epsilon \cup \mathcal{D}_f^\epsilon \cup \mathcal{C}_{1,a}^\epsilon \cup \mathcal{C}_{1,b}^\epsilon  
% \end{align*}
%
with $\mathcal{C}_{1,a}^\epsilon \coloneqq \{ \xi \in \mathbb{X} \ | \ \tau_s \in [0, \tau_{\text{miati}}^f]  \wedge \epsilon \tau_f \in [0,\tau_s + \tau_{\text{mati}}^f - \tau_{\text{miati}}^f] \} $ and 
$\mathcal{C}_{1,b}^\epsilon \coloneqq \{ \xi \in \mathbb{X} \;|\; \tau_s \in [\tau_{\text{miati}}^f, \epsilon \tau_f + \tau_{\text{mati}}^s - \tau_{\text{miati}}^f]  \wedge \epsilon \tau_f \in  [0, \tau_{\text{miati}}^f]  \}$. 
%
A transmission of slow (resp. fast) signals is allowed in the set $\mathcal{D}_s^\epsilon$ (resp. $\mathcal{D}_f^\epsilon$), and at the transmission instance, $\tau_s$ (resp. $\tau_f$) is reset to zero.
%
The sets $\mathcal{C}_1^\epsilon$, $\mathcal{D}_s^\epsilon$ and $\mathcal{D}_f^\epsilon$ are defined to ensure the
satisfaction of \eqref{eqn: Stefan timer}, which can be deduced by visual inspection from Fig. \ref{fig: Stefan timer}. The jump sets $\mathcal{D}_s^\epsilon$ and $\mathcal{D}_f^\epsilon$ are indicated by the orange and green regions, respectively. Additionally, $\mathcal{C}_{1,a}^\epsilon$ and $\mathcal{C}_{1,b}^\epsilon$ are the regions where a jump is not allowed due to a recent transmission of slow and fast signals, respectively.

\begin{figure}[H]
    \centering
    \includegraphics[width = \linewidth]{Figures/Timer.pdf}
    \caption{Flow set and jump set}
    \label{fig: Stefan timer}
\end{figure} 












\subsection{Hybrid model}
Let $f_x,g_z,f_{e_s}$ and $g_{e_f}$ be defined in \eqref{eq:functions} in the next page, where we use $f_{x,\iota}$ and $g_{z,\iota}$, $\iota\in\{1,2\}$, to denote the $\iota$--th component of $f_x  $ and $g_z$, respectively.
%
% Let  
% $\xi \coloneqq (x,e_s, \tau_s, \kappa_s, z,e_f, \tau_f,  \kappa_f)\in \mathbb{X}$, 
% with $\mathbb{X}\coloneqq \mathbb{R}^{n_x}\times \mathbb{R}^{n_{e_s}}\times  \mathbb{R}_{\geq 0} \times \mathbb{Z}_{\geq 0} \times \mathbb{R}^{n_z}\times \mathbb{R}^{n_{e_f}}\times \mathbb{R}_{\geq 0} \times \mathbb{Z}_{\geq 0}$, 
% denote the full state of the hybrid system.
%
Then the SPNCS can now be expressed as the following hybrid model
\begin{equation}
    \mathcal{H}_1:\left\{
\begin{aligned}
    \dot{\xi} &= F(\xi, \epsilon), &&\xi \in \mathcal{C}_1^\epsilon, \\
    \xi^+ &\in G(\xi),  &&\xi\in \mathcal{D}_s^\epsilon \cup \mathcal{D}_f^\epsilon,
\end{aligned}
    \right.
    \label{eqn:full system}
\end{equation}
where 
%$F(\xi) \coloneqq  \big(f_x(x,z,e_s,e_f), \tfrac{1}{\epsilon}g_z(x,z,e_s,e_f),$ $f_{e_s}(x,z,e_s,e_f),\tfrac{1}{\epsilon} g_{e_f}(x,z,e_s,e_f, \epsilon), 1, \frac{1}{\epsilon},0,0\big)$
$F(\xi, \epsilon) \coloneqq  \big(f_x(x,z,e_s,e_f),f_{e_s}(x,z,e_s,e_f),1,0, $ $\tfrac{1}{\epsilon}g_z(x,z,e_s,e_f), \tfrac{1}{\epsilon} g_{e_f}(x,z,e_s,e_f, \epsilon),  \frac{1}{\epsilon},0\big)$, and 
\begin{align*}
    G(\xi) \coloneqq \left\{ 
    \begin{aligned}
    &G_s(\xi), \quad \xi\in\mathcal{D}_s^\epsilon \setminus \mathcal{D}_f^\epsilon , \\
    &G_f(\xi), \quad \xi\in\mathcal{D}_f^\epsilon \setminus \mathcal{D}_s^\epsilon ,\\
    &\{G_s(\xi),G_f(\xi)\},\quad \xi\in \mathcal{D}_s^\epsilon\cap\mathcal{D}_f^\epsilon .
    \end{aligned}
    \right. 
\end{align*}
The jump maps are defined as $G_s(\xi) \coloneqq (x,h_s(\kappa_s, e_s),0,$ $ \kappa_s + 1, z, e_f,  \tau_f,  \kappa_f)$ and $G_f(\xi) \coloneqq(x, e_s, $ $\tau_s,\kappa_s,  z,  h_f(\kappa_f,$ $ e_f),  0,  \kappa_f + 1 )$, where $G_s$ and $G_f$ corresponds to the transmission of slow and fast signals, respectively. 
%The jump map $G$ is defined such that, at any transmission instance where both the slow and fast transmissions are allowed, i.e. $\xi \in \mathcal{D}_s^\epsilon \cap \mathcal{D}_f^\epsilon$, the trajectory experiences a single jump according to either $G_s$ or $G_f$. Similar to the approach in \cite{abdelrahim2017robust}, this \red{design}\todo{or modelling choice?} choice ensures that the jump map $G$ is outer semicontinuous (OSC) \cite[Definition 5.9]{gosate12}, which is one of the hybrid basic conditions \cite[Assumption 6.5]{gosate12}. 
%The jump map $G$ would not be OSC if $G(\xi)$ were defined as $\{G_s(\xi) \}$ or $\{G_f(\xi) \}$ when $\xi \in \mathcal{D}_s^\epsilon \cap \mathcal{D}_f^\epsilon$.
The set-valued map in the definition of $G$
%, i.e., when $\xi \in \mathcal{D}_s^\epsilon \cap \mathcal{D}_f^\epsilon$, 
is introduced to ensure that $\mathcal{H}_1$ satisfies the hybrid basic conditions \cite[Assumption 6.5]{gosate12}, providing well-posedness of the system. This approach is commonly used when modeling the NCS as hybrid dynamical systems, see \cite{abdelrahim2017robust,wang2015emulation} for more details.
%
%Moreover, the hybrid model \eqref{eqn:full system} is more general than the hybrid SPSs in the literature such as \cite{sanfelice2011singular} and \cite{wang2012analysis}, as its flow and jump sets depend on $\epsilon$.






% \begin{figure*}[!htp]
% 	\hrule
% 	%\begin{subequations}
% 		\begin{align}\label{eq:functions}
% 		f_x(x,z,e_s,e_f) &\coloneqq 
%     \big(
%     f_p(x_p,z_p,(k_{c_s}(x_c)+e_{u_s},k_{c_f}(x_c,z_c)+e_{u_f}) ),
%     f_c(x_c,z_c,(k_{p_s}(x_p)+e_{y_s},k_{p_f}(x_p,z_p)+e_{y_f}) )
%     \big) \nonumber \\
%     %
%     g_z(x,z,e_s,e_f) &\coloneqq 
%     \big(
%     g_p(x_p,z_p,(k_{c_s}(x_c)+e_{u_s},k_{c_f}(x_c,z_c)+e_{u_f}) ) ,
%     g_c(x_c,z_c,(k_{p_s}(x_p)+e_{y_s},k_{p_f}(x_p,z_p)+e_{y_f} ) )
%     \big) \nonumber \\
%     %
%     f_{e_s}(x,z,e_s,e_f) &\coloneqq
%     \Big(- \tfrac{\partial k_{p_s}(x_p)}{\partial x_p} 
%         f_{x,1}(x,z,e_s,e_f), 
%     - \tfrac{\partial k_{c_s}(x_c)}{\partial x_c} 
%         f_{x,2}(x,z,e_s,e_f)\Big)  \\
%         %
%         %
%         g_{e_f}(x,z,e_s,e_f,\epsilon) &\coloneqq \Big(  -\epsilon \tfrac{\partial k_{p_f}(x_p,z_p)}{\partial x_p}  f_{x,1}(x,z,e_s,e_f)  - \tfrac{\partial k_{p_f}(x_p,z_p)}{\partial z_p} g_{z,1}(x,z,e_s,e_f) , \nonumber\\
%         &\hspace{4.7cm}  - \epsilon \tfrac{\partial k_{c_f}(x_c,z_c)}{\partial x_c}  f_{x,2}(x,z,e_s,e_f) -\tfrac{\partial k_{c_f}(x_c,z_c)}{\partial z_c} g_{z,2}(x,z,e_s,e_f)\Big). \nonumber 
%     \end{align}
% 	%\end{subequations}
%     \begin{equation}\label{eq:functions 2}
%     \begin{aligned}
%     F_s^y(x,y,e_s,e_f) &\coloneqq \big(f_x(x,y+\overline{H}(x,e_s),e_s, e_f ), f_{e_s}(x,y+\overline{H}(x,e_s),e_s, e_f ),1,0\big) 
%     \\
%     F_f^y(x,y,e_s,e_f,\epsilon) &\coloneqq \big(
%       g_z(x,y+\overline{H}(x,e_s),e_s,e_f)- \epsilon \tfrac{\partial \overline{H}}{\partial \xi_s} F_s^y(x,y,e_s,e_f ),
%       g_{e_f}(x,y+\overline{H}(x,e_s),e_s, e_f , \epsilon),1,0 \big)
%     \end{aligned}
%     \end{equation}
% 	\hrule
% \end{figure*}



\begin{figure*}[!htp]
	\hrule
    \scriptsize % You can change this to \small \footnotesize, \scriptsize, or \tiny
	\begin{equation}\label{eq:functions}
    \begin{aligned}
		f_x(x,z,e_s,e_f) &\coloneqq 
    \big(
    f_p(x_p,z_p,(k_{c_s}(x_c)+e_{u_s},k_{c_f}(x_c,z_c)+e_{u_f}) ),
    f_c(x_c,z_c,(k_{p_s}(x_p)+e_{y_s},k_{p_f}(x_p,z_p)+e_{y_f}) )
    \big)  \\
    %
    g_z(x,z,e_s,e_f) &\coloneqq 
    \big(
    g_p(x_p,z_p,(k_{c_s}(x_c)+e_{u_s},k_{c_f}(x_c,z_c)+e_{u_f}) ) ,
    g_c(x_c,z_c,(k_{p_s}(x_p)+e_{y_s},k_{p_f}(x_p,z_p)+e_{y_f} ) )
    \big)  \\
    %
    f_{e_s}(x,z,e_s,e_f) &\coloneqq
    \Big(- \tfrac{\partial k_{p_s}(x_p)}{\partial x_p} 
        f_{x,1}(x,z,e_s,e_f), 
    - \tfrac{\partial k_{c_s}(x_c)}{\partial x_c} 
        f_{x,2}(x,z,e_s,e_f)\Big)  \\
        %
        %
        g_{e_f}(x,z,e_s,e_f,\epsilon) &\coloneqq \Big(  -\epsilon \tfrac{\partial k_{p_f}(x_p,z_p)}{\partial x_p}  f_{x,1}(x,z,e_s,e_f)  - \tfrac{\partial k_{p_f}(x_p,z_p)}{\partial z_p} g_{z,1}(x,z,e_s,e_f) , \\
        &\hspace{4.7cm}  - \epsilon \tfrac{\partial k_{c_f}(x_c,z_c)}{\partial x_c}  f_{x,2}(x,z,e_s,e_f) -\tfrac{\partial k_{c_f}(x_c,z_c)}{\partial z_c} g_{z,2}(x,z,e_s,e_f)\Big).  
    \end{aligned}
    \end{equation}
    \begin{equation}\label{eq:functions 2}
    \begin{aligned}
    F_s^y(x,y,e_s,e_f) &\coloneqq \big(f_x(x,y+\overline{H}(x,e_s),e_s, e_f ), f_{e_s}(x,y+\overline{H}(x,e_s),e_s, e_f ),1,0\big) 
    \\
    F_f^y(x,y,e_s,e_f,\epsilon) &\coloneqq \big(
      g_z(x,y+\overline{H}(x,e_s),e_s,e_f)- \epsilon \tfrac{\partial \overline{H}}{\partial \xi_s} F_s^y(x,y,e_s,e_f ),
      g_{e_f}(x,y+\overline{H}(x,e_s),e_s, e_f , \epsilon),1,0 \big)
    \end{aligned}
    \end{equation}
    \normalsize
	\hrule
\end{figure*}
%



\section{Auxiliary systems}


% To facilitate the forthcoming analysis, we introduce $\mathcal{H}_1$ as the hybrid system with dynamics as per (\ref{eqn:full system}), but with the "patched" flow set defined as 
% $ \mathcal{C}_2^\epsilon \coloneqq \{ \xi \in \mathbb{X} \;|\; \tau_s \in [0, \tau_{\text{mati}}^s] \;\wedge\ \epsilon \tau_f \in  [0, \tau_{\text{mati}}^f]  \}$.
%
%
% \begin{equation*}
%      \mathcal{C}_2^\epsilon \coloneqq \left\{ \xi \in \mathbb{X} \;|\; \tau_s \in [0, \tau_{\text{mati}}^s] \;\wedge\ \epsilon \tau_f \in  [0, \tau_{\text{mati}}^f]  \right\}.
% \end{equation*}
% We note that $\mathcal{H}_1$ \emph{contains} $\mathcal{H}_1$ in the sense that all solutions of $\mathcal{H}_1$ are also solutions to $\mathcal{H}_1$.
%, since $\mathcal{C}_1^\epsilon  \subseteq \mathcal{C}_2^\epsilon$ and they have identical flow map, jump map and jump set. 
% Therefore, using \cite[Proposition 3.32]{gosate12}, we can conclude the stability properties of $\mathcal{H}_1$ by analysing the stability of $\mathcal{H}_1$.
%We also note that if $\mathcal{H}_1$ is initialized at  some $\xi_0 \in \mathcal{C}_1^\epsilon$, its maximal solution will be complete, otherwise it will have a non-complete maximal solution.
%
We adopt a similar approach to the standard singularly perturbed method \cite[Section 11.5]{nonlinear_systems_Khalil} to establish stability properties for $\mathcal{H}_1$, but generalised to hybrid systems. Particularly, we first derive a system $\mathcal{H}_1^y$ by changing the $z$--coordinate of $\mathcal{H}_1$ to $y$--coordinate, where $y$ is defined in \eqref{eqn: map between y and z}, and determine its stability through a \emph{boundary layer} and \emph{reduced system}. 
\subsection{Change of coordinates}

 
%
We first derive the \emph{quasi-steady-state} of $\mathcal{H}_1$, under the following assumption.

% \textbf{Standing Assumption 1}\hspace{5pt}\rm\textbf{(SA1)} 
% \textit{For any $\overline{x}\in \mathbb{R}^{n_x}$, $\overline{e}_s\in \mathbb{R}^{n_{e_s}}$ and $\overline{z}\in \mathbb{R}^{n_z}$, equation $ 0 = g_z\left(\bar x,\bar z, \bar e_{s},0\right)$ has a unique real solution $\bar z = \overline{H}(\bar x,  \bar e_{s})$, where $\overline{H}$ is continuously differentiable and $0 =\overline{H}(0, 0)$.}

\begin{sassum}\label{assum:standing-ss} \rm 
\textbf{(SA1)} \it
For any $\overline{x}\in \mathbb{R}^{n_x}$, $\overline{e}_s\in \mathbb{R}^{n_{e_s}}$ and $\overline{z}\in \mathbb{R}^{n_z}$, equation $ 0 = g_z\left(\bar x,\bar z, \bar e_{s},0\right)$ has a unique real solution $\bar z = \overline{H}(\bar x,  \bar e_{s})$, where $\overline{H}$ is continuously differentiable and $0 =\overline{H}(0, 0)$.
\end{sassum}

%
The \emph{quasi-steady-states} $\bar z$ and $\bar e_{f}$, referring to the equilibrium of the fast states as $\epsilon$ approaches zero, are obtained as follows:
$\bar{e}_f$ is equal to zero, as  for sufficiently high frequency of fast-output transmissions, $e_f$ converges to zero; and $\bar{z}$ corresponds to the unique solution $\bar z = \overline{H}(\bar x,  \bar e_{s})$ as per SA\ref{assum:standing-ss}.
%
We define the variable $y$ as
\begin{equation}
    y\coloneqq z - \overline{H}(x, e_{s}).
    \label{eqn: map between y and z}
\end{equation}
Then similar to the assumptions in the continuous-time SPSs literature such as \cite{nonlinear_systems_Khalil,christofides1996singular}, SA\ref{assum:standing-ss} guarantees the map \eqref{eqn: map between y and z} to be stability preserving, which means the origin of the $x$-$z$ coordinate is asymptotically stable if and only if the origin of the $x$-$y$ system is asymptotically stable, see \cite[Section 11.5]{nonlinear_systems_Khalil} for more detail.
% \begin{equation}
%     y\coloneqq z - \overline{H}(x, e_{s})
%     \label{eqn: map between y and z}
% \end{equation}
%
% The \emph{quasi-steady-states} $\bar z$ and $\bar e_{f}$, referring to the equilibrium of the fast states as $\epsilon$ approaches zero, are obtained as follows:
% $\bar{e}_f$ is equal to zero, as  for sufficiently high frequency of fast-output transmissions, $e_f$ converges to zero; and
%   $\bar{z}$ corresponds to the unique solution $\bar z = \overline{H}(\bar x,  \bar e_{s})$ as per SA\ref{assum:standing-ss}.
%
Next, to derive $\mathcal{H}_1^y$, we define the full state of $\mathcal{H}_1^y$, namely 
\begin{equation}
    \xi^y \coloneqq (\xi_s, \xi_f) \coloneqq \big((x,e_s,\tau_s, \kappa_s), (y,e_f, \tau_f, \kappa_f)\big),
    \label{eqn: definition of xi_s and xi_f}
\end{equation}
where $\xi^y \in \mathbb{X}$, $\xi_s \in \mathbb{X}^{s} \coloneqq \mathbb{R}^{n_x} \times \mathbb{R}^{n_{e_s}} \times \mathbb{R}_{\geq 0} \times \mathbb{Z}_{\geq 0}$ and $\xi_f \in  \mathbb{X}^{f} \coloneqq \mathbb{R}^{n_z} \times \mathbb{R}^{n_{e_f}} \times \mathbb{R}_{\geq 0} \times \mathbb{Z}_{\geq 0}$. 
%
When a slow variable is transmitted at $t_k^s \in \mathcal{T}^s$, $e_s$ updates according to $h_s$, then by the definition of $y$ in \eqref{eqn: map between y and z}, we know at each slow transmission, the value of $y$ updates according to
\begin{equation}
    \begin{aligned}
        y^+ &= z^+ - \overline{H}(x^+, e_{s}^+)= z - \overline{H}(x, h_s(\kappa_s, e_s)) \\
        %&= z - \overline{H}(x, h_s(\kappa_s, e_s)) \\
        &= y + \overline{H}(x, e_s) - \overline{H}(x, h_s(\kappa_s, e_s)) \\
        & \eqqcolon h_y(\kappa_s,x,e_s,y). 
    \end{aligned}
    \label{eqn: Jump of y at slow transmission}
\end{equation}
%
Then, $\mathcal{H}_1^y$ is given by
\begin{equation}
    \mathcal{H}_1^y:\left\{
\begin{aligned}
    \dot{\xi}^y &= F^y(\xi^y, \epsilon),\ \xi^y \in \mathcal{C}_2^{y,\epsilon}, \\
    {\xi^y}^+ &\in G^y(\xi^y), \ \xi^y\in \mathcal{D}_s^{y,\epsilon} \cup \mathcal{D}_f^{y,\epsilon},
\end{aligned}
    \right.
    \label{eqn: H_2^y}
\end{equation}
where $F^y(\xi^y, \epsilon) = \big(F_s^y(x,y,e_s,e_f), \tfrac{1}{\epsilon}F_f^y(x,y, $     $e_s,e_f,\epsilon)\big)$, with $F_s^y$ and $F_f^y$ from \eqref{eq:functions 2}. 
% $F_s^y(x,y,e_s,e_f) \coloneqq 
% \big(f_x(x,y+\overline{H}(x,e_s),e_s, e_f ), f_{e_s}(x,y+\overline{H}(x,e_s),e_s, e_f ),1,0\big) $, 
% $ F_f^y(x,y,e_s,e_f,\epsilon) \coloneqq \big(
%       \epsilon \tfrac{\partial y}{\partial t},
%       g_{e_f}(x,y+\overline{H}(x,e_s),e_s, e_f , \epsilon),1,0 \big)$
% and 
% $\epsilon \tfrac{\partial y}{\partial t} = g_z(x,y+\overline{H}(x,e_s),e_s,e_f)- \epsilon \tfrac{\partial \overline{H}}{\partial \xi_s} F_s^y(x,y,e_s,e_f ) $. 
The jump map $G^y$ is given by
\begin{equation}
\begin{aligned}
    G^y(\xi^y) \coloneqq \left\{ 
    \begin{aligned}
    &G_s^y(\xi^y),  \;\xi^y\in\mathcal{D}_s^{y,\epsilon} \setminus \mathcal{D}_f^{y,\epsilon} , \\
    &G_f^y(\xi^y),  \;  \xi^y \in\mathcal{D}_f^{y,\epsilon} \setminus \mathcal{D}_s^{y,\epsilon} ,\\
    &\{G_s^y(\xi^y),G_f^y(\xi^y)\}, \; \xi^y\in \mathcal{D}_s^{y,\epsilon} \cap \mathcal{D}_f^{y,\epsilon},
    \end{aligned}
    \right. 
\end{aligned}
\label{eqn: G^y}
\end{equation}
with $G_s^y(\xi_y) \coloneqq \big(x, h_s(\kappa_s, e_s), 0, \kappa_s + 1, h_y(\kappa_s,x,e_s,y),  e_f,$ $ \tau_f, \kappa_f \big)$; $G_f^y(\xi_y) \coloneqq \big(x,e_s, \tau_s, \kappa_s , y, h_f(\kappa_f, e_f), 0, \kappa_f + 1 \big)$. 



For analysis purposes, we write $\tau_{\text{mati}}^f = \epsilon T^*$ with $T^* \in \mathbb{R}_{>0}$ independent of $\epsilon$. We also write $\tau_{\text{miati}}^f = a\tau_{\text{mati}}^f$ for some $a \in(0,\tfrac{1}{2}] $, which satisfies the inequality \eqref{eqn: condition on miati^f}. Since $\epsilon > 0$, $\epsilon \tau_f \in [\tau_{\text{miati}}^f, \tau_{\text{mati}}^f]$ is equivalent to $\tau_f \in [aT^*,T^*]$. Then the jump and flow sets in \eqref{eqn: H_2^y} are defined by
$\mathcal{D}_s^{y,\epsilon} \coloneqq  \{\xi^y \in \mathbb{X} \; | \; \tau_s \in [\tau_{\text{miati}}^s, \tau_{\text{mati}}^s] \wedge \tau_f \in  [aT^*, (1-a)T^*] \}$, 
$\mathcal{D}_f^{y,\epsilon} \coloneqq \{\xi^y \in \mathbb{X} \; | \; \tau_s \in [\epsilon aT^*, \tau_{\text{mati}}^s-\epsilon aT^*]  \wedge  \tau_f \in  [aT^*, T^*]  \}$ 
and
%$\mathcal{C}_2^{y,\epsilon} \coloneqq  \{\xi^y \in \mathbb{X} \; | \; \tau_s \in [0, \tau_{\text{mati}}^s] \wedge  \tau_f \in  [0, T^*] \}$.
%
$\mathcal{C}_1^{y,\epsilon} \coloneqq 
        \mathcal{D}_s^{y,\epsilon} \cup \mathcal{D}_f^{y,\epsilon} \cup \mathcal{C}_{1,a}^{y,\epsilon} \cup \mathcal{C}_{1,b}^{y,\epsilon}$,
with $\mathcal{C}_{1,a}^{y,\epsilon} \coloneqq \{ \xi^y \in \mathbb{X} \ | \ \tau_s \in [0, \epsilon a T^*]  \wedge \epsilon \tau_f \in [0,\tau_s + \epsilon T^* - \epsilon a T^*] \} $ and 
$\mathcal{C}_{1,b}^{y,\epsilon} \coloneqq \{ \xi^y \in \mathbb{X} \;|\; \tau_s \in [\epsilon a T^*, \epsilon \tau_f + \tau_{\text{mati}}^s - \epsilon a T^*]  \wedge \epsilon \tau_f \in  [0, \epsilon a T^*]  \}$. 



%
% \begin{align*}
%     \mathcal{D}_s^{y,\epsilon} \coloneqq & \{\xi^y \in \mathbb{X} \; | \; \tau_s \in [\tau_{\text{miati}}^s, \tau_{\text{mati}}^s] 
%         \\ & \qquad \qquad \qquad \qquad \quad \wedge  \tau_f \in  [aT^*, (1-a)T^*] \},
%     \\ 
%     \mathcal{D}_f^{y,\epsilon} \coloneqq &\{\xi^y \in \mathbb{X} \; | \; \tau_s \in [\epsilon aT^*, \tau_{\text{mati}}^s-\epsilon aT^*]  \\
%     & \qquad \qquad \qquad \qquad \qquad \qquad   \wedge  \tau_f \in  [aT^*, T^*]  \},
%     \\
%     \mathcal{C}_2^{y,\epsilon} \coloneqq & 
%         \{\xi^y \in \mathbb{X} \; | \; \tau_s \in [0, \tau_{\text{mati}}^s] \wedge  \tau_f \in  [0, T^*] \}.
% \end{align*}
%
We have changed the coordinate from $z$ to $y$, and we are now ready to derive the reduced system $\mathcal{H}_r$ and boundary layer system $\mathcal{H}_{bl}$ associated with $\mathcal{H}_1^y$.



\subsection{Boundary layer system and reduced system of $\mathcal{H}_1$}
 We define the fast time scale $\sigma \coloneqq \tfrac{t-t_0}{\epsilon}$, where we can assume $t_0 = 0$ as the system is time invariant. Then we have $\tfrac{\partial}{\partial \sigma} = \epsilon \tfrac{\partial}{\partial t}$. We set $\epsilon = 0$ for system \eqref{eqn: H_2^y}, then $\mathcal{C}_1^{y,0}$, which corresponds to $\mathcal{C}_1^{y,\epsilon}$ with $\epsilon = 0$, is given by $\mathcal{C}_1^{y,0} \coloneqq \{ \xi^y \in \mathbb{X} \ | \ \tau_s \in [0, \tau_\text{mati}^s]  \wedge \tau_f \in [0,  T^*] \}  $, and $\mathcal{D}_s^{y,0}$, $\mathcal{D}_f^{y,0}$ are derived accordingly. In the perspective of fast dynamics, the slow dynamics are now frozen. Meanwhile, the jump and flow sets of $\mathcal{H}_{bl}$ contain the condition $\tau_{s}\in [0, \tau_{\text{mati}}^s]$, which is always satisfied. Therefore, the jumps and flows of $\mathcal{H}_{bl}$ are only determined by $\tau_{f}$. We thus write
%
\begin{equation}
    \mathcal{H}_{bl}\! : \! \left\{
\begin{aligned}
    (\tfrac{\partial \xi_s}{\partial \sigma}, \tfrac{\partial \xi_f}{\partial \sigma} ) &= (\mathbf{0}_{n_{\xi_s}\! \times 1}, F_f^y(x,y,e_s,e_f,0) ), \xi^y \! \in \mathcal{C}_{1,bl}^{y,0}, \\
    {\xi^y}^+  &=   G_f^y(\xi^y), \qquad \qquad \qquad \qquad \, \xi^y \! \in \mathcal{D}_f^{y,0},
\end{aligned}
    \right.
    \label{eqn: H_bl}
\end{equation}
where $\mathcal{C}_{2,bl}^{y,0} \coloneqq \{\xi^y \in  \mathbb{X} \ | \ \tau_f \in [0, T^*]\}$ and $\mathcal{D}_f^{y,0}\coloneqq \{\xi^y \in  \mathbb{X} \ | \ \tau_f \in [aT^*, T^*] \}$. 


From the perspective of $\mathcal{H}_r$ (i.e., slow dynamics), the fast dynamics evolve infinitely fast. Therefore, for any $\tau_s \in [0, \tau_{\text{mati}}^s] $, the waiting time for the condition $\tau_f \in [aT^*, T^*]$ in the jump set to be satisfied approaches to zero, and the flows and jumps of $\mathcal{H}_r$ are essentially determined only by $\tau_s$. 
%
%We assume $\mathcal{H}_{bl}$ satisfies an asymptotic stability property at its quasi-steady state, which we formalise in the sequel. 
Moreover, we have $y=0$ and $e_f = 0$ in $\mathcal{H}_r$, that is 
%
\begin{equation}
    \mathcal{H}_{r}:\left\{
\begin{aligned}
    \dot \xi_s &= F_s^y(x,0,e_s, 0) , \quad \xi^y \in \mathcal{C}_{1,r}^{y,0}, \\
    \xi_s^+  &=   (x, h_s(\kappa_s, e_s), 0, \kappa_s + 1), \  \xi^y\in \mathcal{D}_s^{y,0},
\end{aligned}
    \right.
    \label{eqn: H_r}
\end{equation}
where $\mathcal{C}_{1,r}^{y,0} \coloneqq \{\xi^y \in  \mathbb{X} \ | \ \tau_s \in [0, \tau_{\text{mati}}^s]\}$ and $\mathcal{D}_s^{y,0}\coloneqq \{\xi^y \in  \mathbb{X} \ | \ \tau_s \in [\tau_{\text{miati}}^s, \tau_{\text{mati}}^s] \}$.












% \begin{align*}
%     \mathcal{D}_s^{0} &\coloneqq \left\{\xi \in \mathbb{X} \; | \; \tau_s \in [\tau_{\text{miati}}^s, \tau_{\text{mati}}^s] \;\wedge\;  \tau_f \in  [aT^*, T^*]  \right \},
%     \\
%     \mathcal{D}_f^{0} &\coloneqq \left\{\xi \in \mathbb{X} \; | \; \tau_s \in [0, \tau_{\text{mati}}^s] \;\wedge\;  \tau_f \in  [aT^*, T^*]  \right \},
%     \\
%     \mathcal{C}_2^{0} &\coloneqq \left\{ \xi \in \mathbb{X} \;|\; \tau_s \in [0, \tau_{\text{mati}}^s] \;\wedge\ \tau_f \in  [0, T^*]  \right\} .
% \end{align*}
% Then, the boundary layer system $\mathcal{H}_{bl}$ is given by
% \begin{equation*}
%     \mathcal{H}_{bl} :
%     \begin{cases}
%     \begin{aligned} % Used to align \right\}
%     \left.
%     \begin{aligned} %Used to align flow map
%     \tfrac{\partial x}{\partial \sigma} &= 0, \; 
%     \tfrac{\partial e_{s}}{\partial \sigma} = 0,\;
%     \tfrac{\partial \tau_s}{\partial \sigma} = 0, \;
%     \tfrac{\partial \kappa_s}{\partial \sigma} = 0
%     \\
%     \tfrac{\partial y}{\partial \sigma} &=  g_z(x,y+ \overline{H}(x, e_s),e_s,e_f) \\
%     \tfrac{\partial e_{f}}{\partial \sigma} &= g_{e_f}(x,y+ \overline{H}(x, e_s),e_s,e_f, 0)
%     \\
%     \tfrac{\partial \tau_f}{\partial \sigma} &= 1,\;\tfrac{\partial \kappa_f}{\partial \sigma} = 0 \\
%     \end{aligned}
%     \right\}
%     &\begin{aligned}&\text{when } \\ & \xi \in \mathcal{C}_2^{0} \end{aligned} 
%     \\[1mm]
%     \left. 
%     \begin{aligned}
%     x^+ &= x,\; e_{s}^+  =  e_{s},\;\tau_s^+ = \tau_s, \;\kappa_s^+ = \kappa_s\\
%     y^+ &= y, \; e_{f}^+= h_f\left(\kappa_f,e_{f} \right)\\
%     \tau_f^+ &=0,\;\kappa_f^+ = \kappa_f+ 1
%     \end{aligned}
%     \right\} 
%         &\begin{aligned}&\text{when } \\ & \xi \in \mathcal{D}_f^{0}.\end{aligned} 
%     \end{aligned}
%     \end{cases}
%     %\label{eqn:boundary layer}
% \end{equation*}
% %
% %
% \begin{remark}
% We note that the even though the flow and jump sets of $\mathcal{H}_{bl}$ depend on both $\tau_s$ and $\tau_f$, the flows and jumps of $\mathcal{H}_{bl}$ are essentially only determined by $\tau_f$ since the conditions on $\tau_s$ in $\mathcal{D}_f^{0}$ and $\mathcal{C}_2^{0}$ can always be satisfied as long as we initialize in $\mathcal{C}_2^{0}$. 
% \end{remark}
%
%

% The \emph{reduced system} $\mathcal{H}_r$ is obtained by substituting quasi-steady states into the full system (\ref{eqn:full system}), and it is given by
% %
% \begin{equation*}
%     \mathcal{H}_r: 
%     \begin{cases}
%     \begin{aligned} % Used to align \right\}
%     \left.
%     \begin{aligned} %Used to align flow map
%     \dot{x} &= f_x(x,\overline{H}(x,e_s),e_s, 0)\\
%     \dot{e}_{s}&=f_{e_s}(x,\overline{H}(x,e_s),e_s, 0)\\
%     \dot{\tau_f}_s &= 1, \;  \dot{\kappa}_s = 0 \\
%     \end{aligned}
%     \right\}
%     & \begin{aligned}&\text{when} \\ &\xi \in \mathcal{C}_2^0 \end{aligned}
%     \\[1mm]
%     \left. 
%     \begin{aligned}
%     x^+ &= x,\;  e_{s}^+ = 
%     h_s\left(\kappa_s,e_{s} \right)\\
%     \tau_s^+ &= 0, \; \kappa_s^+ = \kappa_s + 1
%     \end{aligned}
%     \quad 
%     \right\} 
%     & \begin{aligned}&\text{when} \\ &\xi \in \mathcal{D}_s^0 \end{aligned}
%     \end{aligned}
%     \end{cases}
%     %\label{eqn: reduced}
% \end{equation*}
% \begin{remark}
%     Similar to $\mathcal{H}_{bl}$, flows and jumps of $\mathcal{H}_r$ are essentially determined only by $\tau_s$, even though $\mathcal{C}_2^0$ and $\mathcal{D}_s^{0}$ depend on $\tau_f$. This is due to the fact that from the perspective of the reduced order system, $\tau_f$ will keep flowing into the range $[\tau_{\text{miati}}, \tau_{\text{mati}}^f]$, then reset to zero at an infinitely fast rate. Consequently, for any $ \tau_s \in [\tau_{\text{miati}}^s, \tau_{\text{mati}}^s]$, the condition imposed upon $\tau_f$, i.e., $\tau_f \in [aT^*,T^*]$ is always met. In other words, for any $\tau_s \in [0, \tau_{\text{mati}}^s]$, the waiting time for the condition $\tau_f\in [aT^*,T^*]$ to be satisfied approaches to zero.
% \end{remark}

%
% To prepare for the stability analysis in the next section, we need the following notations. We separate $\xi$ into slow and fast dynamical states $\xi_s$ and $\xi_f$, where $\xi_s \coloneqq ( x ,e_s,\tau_s,\kappa_s) \in \mathbb{R}^{n_{\xi_s}}$ and $\xi_f \coloneqq ( y, e_f, \tau_f, \kappa_f )\in \mathbb{R}^{n_{\xi_f}}$, with $\mathbb{R}^{n_{\xi_s}}\coloneqq \mathbb{R}^{n_x} \times \mathbb{R}^{n_{e_s}} \times \mathbb{R} \times \mathbb{N}_{\geq 0}$ and $\mathbb{R}^{n_{\xi_f}}\coloneqq \mathbb{R}^{n_z} \times \mathbb{R}^{n_{e_f}} \times \mathbb{R} \times \mathbb{N}_{\geq 0}$. 
% Then we define 
% $F_s(x,y,e_s,e_f) \coloneqq 
%      \big(f_x(x,y+\overline{H}(x,e_s),e_s, e_f ),
%       f_{e_s}(x,y+\overline{H}(x,e_s),e_s, e_f ),1,0\big)$,
% $G_r(\xi_s) \coloneqq  \big(x, h_s(\kappa_s, e_s) , 0 , \kappa_s + 1\big) $,
% $F_f(x,y,e_s,e_f,\epsilon) \coloneqq \big(
%       g_z(x,y+\overline{H}(x,e_s),e_s, e_f ),
%       g_{e_f}(x,y+\overline{H}(x,e_s),e_s, e_f , \epsilon),1,0 \big)  $
% and
% $G_{bl}(\xi_f) \coloneqq \big( y, h_f(\kappa_f, e_f) , 0 , \kappa_f + 1 \big)$.
%
%  Define 
%  \begin{equation}
%  \begin{aligned}
%      &\xi_s \coloneqq \begin{bmatrix} x \\ e_s \\ \tau_s \\ \kappa_s \end{bmatrix}, \;
%       G_r(\xi_s) \coloneqq 
%      \begin{bmatrix}
%       x\\ h_s(\kappa_s, e_s) \\ 0 \\ \kappa_s + 1
%      \end{bmatrix} 
%      \\
%      &F_s(x,y,e_s,e_f) \coloneqq 
%      \begin{bmatrix}
%       f_x(x,y+\overline{H}(x,e_s),e_s, e_f ) \\
%       f_{e_s}(x,y+\overline{H}(x,e_s),e_s, e_f )\\
%       1\\
%       0
%      \end{bmatrix}, 
%      \\
%      &\xi_f \coloneqq \begin{bmatrix} y \\ e_f \\ \tau_f \\ \kappa_f \end{bmatrix}, \;
%      G_{bl}(\xi_f) \coloneqq 
%      \begin{bmatrix}
%       y\\ h_f(\kappa_f, e_f) \\ 0 \\ \kappa_f + 1
%      \end{bmatrix},
%      \\
%      &F_f(x,y,e_s,e_f,\epsilon) \coloneqq 
%      \begin{bmatrix}
%       g_z(x,y+\overline{H}(x,e_s),e_s, e_f ) \\
%       g_{e_f}(x,y+\overline{H}(x,e_s),e_s, e_f , \epsilon)\\
%       1\\
%       0
%      \end{bmatrix}.
%  \end{aligned}
% \end{equation}
%
% Lastly, we define the following attractor for the upcoming stability analysis.
% \begin{equation}
%             \mathcal{E} \coloneqq \left\{ \xi \in \mathbb{X} : x=0 \wedge e_s = 0 \wedge z=0 \wedge e_f = 0 \right\} .
%             \label{eqn: set E}
% \end{equation}





