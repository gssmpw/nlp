\aim{
It is noteworthy that \cite[Section 4.5.1]{Stefan_thesis} proposed a natural approach to extend from the two-channel case to the single-channel case, which is to keep the same two-channel clock variables. That is, the timer in \cite{Stefan_thesis} satisfies
%
\begin{subequations}
    \begin{align}
    &\tau_{\text{miati}}^s \leq t_{k+1}^s - t_k^s \leq \tau_{\text{mati}}^s, \forall t_k^s,t_{k+1}^s\in \mathcal{T}^s, k\in\mathbb{Z}_{\geq 0}, 
    \\
    &\tau_{\text{miati}} \leq t_{j+1}^f - t_j^f \leq \tau_{\text{mati}}^f ,  \forall t_j^f, t_{j+1}^f  \in \mathcal{T}^f, j \in\mathbb{Z}_{\geq 0}, 
    \\
    &\tau_{\text{miati}} \leq t_{\ell+1} - t_\ell, \quad \qquad \ \forall t_\ell, t_{\ell + 1} \in \mathcal{T}, \ell \in \mathbb{Z}_{\geq 0}, \label{eqn: Stefan miati}
    \end{align}
    \label{eqn: Stefan timer}
\end{subequations}
where $\tau_{\text{mati}}^f$ denotes the MATI between two consecutive fast updates, and $\tau_{\text{miati}}$ and $\tau_{\text{mati}}^s$ are defined in the same manner as in (\ref{eqn: New Timer}). We emphasize that $\tau_{\text{miati}}^s$ in \eqref{eqn: Stefan timer} is added here to avoid Zeno solutions when $\epsilon\to 0$, an issue that was not reported in \cite{Stefan_thesis}, since the authors did not study stability for the single-channel case with the timer \eqref{eqn: Stefan timer}.
%
Moreover, it was reported in \cite{Stefan_thesis}, that in order to obtain a feasible sequence of transmission times for the clock formulation in \eqref{eqn: Stefan timer}, then $\tau_{\text{miati}} \leq  \tfrac{1}{2}\tau_{\text{mati}}^f$ must hold.
Compared to the findings in \cite{Stefan_thesis}, \aim{the clock mechanism \eqref{eqn: New Timer} can stabilize the system with less bandwidth, this will be shown later.}

\begin{figure}[H]
    \centering
    \includegraphics[width = \linewidth]{Figures/Stefan Clock Mechanism Illustration.pdf}
    \caption{Flow set and jump set}
    \label{fig: Stefan timer}
\end{figure} \todo[inline]{Change the name of sets}

Before we present our NCS in hybrid system, we first need to define the timers correspond to the clock mechanism \eqref{eqn: Stefan timer}. We will keep using $\tau_s$ to describe the inter-transmission time between any two consecutive slow transmissions, as previously defined in $\mathcal{H}_1$. 
%
To streamline our discussion and avoid redundancy in restating all the Assumptions and Lemma, in this section, we will redefine the parameter $\tau$ as the time elapsed since the most recent fast transmission.
%
Consequently, the single-channel SPNCS, which incorporates the clock mechanism defined in equation \eqref{eqn: Stefan timer}, can be depicted as a hybrid system denoted as $\mathcal{H}_3$, which we describe below.






\begin{equation*}
    \mathcal{H}_3:\left\{
\begin{aligned}
    \dot{\xi} &= F(\xi, \cyan{\epsilon}),\ \xi \in \mathcal{C}_3^\epsilon, \\
    \xi^+ &\in G_3(\xi), \ \xi\in \mathcal{D}_{3,s}^\epsilon \cup \mathcal{D}_{3,f}^\epsilon,
\end{aligned}
    \right.
\end{equation*}
where 
\begin{align*}
    G_3(\xi) \coloneqq \left\{ 
    \begin{aligned}
    &G_{3,s}(\xi), && \xi\in\mathcal{D}_{3,s}^\epsilon \setminus \mathcal{D}_{3,f}^\epsilon , \\
    &G_f(\xi), && \xi\in\mathcal{D}_{3,f}^\epsilon \setminus \mathcal{D}_{3,s}^\epsilon ,\\
    &\{G_{3,s}(\xi),G_f(\xi)\},&& \xi\in \mathcal{D}_{3,s}^\epsilon\cap\mathcal{D}_{3,f}^\epsilon ,
    \end{aligned}
    \right.  
    \end{align*}
and $F$ and $G_f$ come from $\mathcal{H}_1$. The flow map $G_{3,s}$ is defined such that $G_{3,s} \coloneqq (x, h_s(\kappa_s, e_s), 0, \kappa_s +1, z, e_f, \tau, \kappa_f)$.

The jump and flow sets are defined as 
\begin{align*}
    \mathcal{D}_{3,s}^\epsilon \coloneqq & \left\{\xi \in \mathbb{X} \; | \; \tau_s \in [\tau_{\text{miati}}^s, \tau_{\text{mati}}^s] \right. \\
    & \qquad \qquad  \qquad \qquad  \wedge \epsilon \tau \in  [\tau_{\text{miati}}, \tau_{\text{mati}}^f - \tau_{\text{miati}}]  \},
    \\
    \mathcal{D}_{3,f}^\epsilon \coloneqq &\left\{\xi \in \mathbb{X} \; | \; \tau_s \in [\tau_{\text{miati}}, \tau_{\text{mati}}^s-\tau_{\text{miati}}] \right. \\
    & \qquad \qquad \qquad \qquad \quad  \wedge\; \epsilon \tau \in  [\tau_{\text{miati}}, \tau_{\text{mati}}^f]  \},
    \\
    \mathcal{C}_3^\epsilon \coloneqq & 
        \mathcal{D}_{3,s}^\epsilon \cup \mathcal{D}_{3,f}^\epsilon \cup \mathcal{C}_{3,a}^\epsilon \cup \mathcal{C}_{3,b}^\epsilon  
\end{align*}
with $\mathcal{C}_{3,a}^\epsilon \coloneqq \{ \xi \in \mathbb{X} \ | \ \tau_s \in [0, \tau_{\text{miati}}]  \wedge \epsilon \tau \in [0,\tau_s + ( \tau_{\text{mati}}^f - \tau_{\text{miati}})] \} $ and 
$\mathcal{C}_{3,b}^\epsilon \coloneqq \{ \xi \in \mathbb{X} \;|\; \tau_s \in [\tau_{\text{miati}}, \epsilon \tau + (\tau_{\text{mati}}^s - \tau_{\text{miati}})]  \wedge \epsilon \tau \in  [0, \tau_{\text{miati}}]  \}$. 
We emphasize again that $\tau$ denotes the time elapsed since the last fast transmission.


The stability result of $\mathcal{H}_3$ is formalized in Corollary \ref{Corollary Stefan Timer} below, and its proof follows similar steps as the proof of Theorem \ref{Theorem H_1}. To achieve this, we need to first define a ``patched" system for $\mathcal{H}_3$, namely, $\mathcal{H}_4$, where the patch is depicted by the red region in Figure \ref{fig: Stefan timer}. Then we changing the coordinate from $z$ to $y$ and write $\tau_{\text{mati}}^f = \epsilon T^* $ and $\tau_{\text{miati}} = \epsilon aT^*$, $a \in (0,1)$, we get $\mathcal{H}_4^y$. 

\begin{equation}
    \mathcal{H}_4^y:\left\{
\begin{aligned}
    \dot{\xi}^y &= F^y(\xi^y, \cyan{\epsilon}),\ \xi^y \in \mathcal{C}_4^{y,\epsilon}, \\
    {\xi^y}^+ &\in G_4^y(\xi^y), \ \xi^y\in \mathcal{D}_{4,s}^{y,\epsilon} \cup \mathcal{D}_{4,f}^{y,\epsilon},
\end{aligned}
    \right.
    \label{eqn: H_4^y}
\end{equation}
where $\mathcal{D}_{4,s}^{y,\epsilon} = \{\xi^y \in \mathbb{X} | \tau_s \in [ \tau_{\text{miati}}^s, \tau_{\text{mati}}^s \wedge \tau \in [aT^*, (1-a)T^*] \}$, $\mathcal{D}_{4,f}^{y,\epsilon} = \mathcal{D}_f^{y,\epsilon}$,  $\mathcal{C}_4^{y,\epsilon}\coloneqq \mathcal{D}_{4,s}^{y,\epsilon} \cup \mathcal{D}_{4,f}^{y,\epsilon} \cup \{ \xi^y \in \mathbb{X} | \tau_s \in [0, \epsilon T^*] \wedge \tau \in [0, T^*]\} \cup  \{ \xi^y \in \mathbb{X} | \tau_s \in [0, \tau_{\text{mati}}^s \wedge \tau \in [0, aT^*]\} $.

Additionally, we have
\begin{align*}
    G_4^y(\xi^y) \coloneqq \left\{ 
    \begin{aligned}
    &G_{4,s}^y(\xi^y),  \;\xi^y\in\mathcal{D}_{4,s}^{y,\epsilon} \setminus \mathcal{D}_{4,f}^{y,\epsilon} , \\
    &G_f^y(\xi^y),  \;  \xi^y \in\mathcal{D}_{4,f}^{y,\epsilon} \setminus \mathcal{D}_{4,s}^{y,\epsilon} ,\\
    &\{G_{4,s}^y(\xi^y),G_f^y(\xi^y)\}, \; \xi^y\in \mathcal{D}_{4,s}^{y,\epsilon} \cap \mathcal{D}_{4,f}^{y,\epsilon},
    \end{aligned}
    \right. 
\end{align*}
where $G_{4,s}^y(\xi_y) \coloneqq \big(x, h_s(\kappa_s, e_s), 0, \kappa_s + 1, \red{h_y(x,e_s,y)},  e_f,$ $ \tau, \kappa_f \big)$. 





Subsequently, we can obtain the boundary-layer system and reduced system for the $\mathcal{H}_4^y$. We note that the boundary-layer and reduced system derived from $\mathcal{H}_2^y$ will coincide with those obtained from the $\mathcal{H}_4^y$, therefore, we will keep using $\mathcal{H}_r$ and $\mathcal{H}_{bl}$ to denote the reduced and boundary-layer system of $\mathcal{H}_4$. Consequently, Lemma \ref{Lemma MATI} applies to the boundary-layer and reduced system of $\mathcal{H}_4$.
%
\begin{corollary}
Considering system $\mathcal{H}_3$ and suppose Assumption \ref{Assumption reduced model} and \ref{Assumption boundary layer system} hold, and Assumption \ref{Assumption interconnection} hold for almost all $\xi^y \in \mathcal{C}_4^{y,\epsilon}$.
%
%Let $b_1$, $b_2$, and $b_3$ come from Assumption \ref{Assumption interconnection} and $a_1$ and $a_2$ come from Lemma \ref{Lemma MATI}.
%%%%%%%%%%%%%%%%%%%%%%%%
\cyan{Let $L_s$ and $\gamma_s$ come from Assumption \ref{Assumption reduced model}, and $L_f$ and $\gamma_f$ come from Assumption \ref{Assumption boundary layer system}.
Let $b_1$, $b_2$, and $b_3$ come from Assumption \ref{Assumption interconnection} and $a_1$, $a_2$, \cyan{$\lambda_s^*$} and \cyan{$\lambda_f^*$} come from Lemma \ref{Lemma MATI}.
}
%%%%%%%%%%%%%%%%%%%%%%
Then\sout{, for any $\lambda_s^* \in (\lambda_s,1)$ and $\lambda_f^* \in (\lambda_f, 1)$,} there exists $\epsilon^*  = \tfrac{a_1 a_2}{a_1 b_3 + b_1 b_2}$ such that for all $0<\epsilon<\epsilon^*$, $\tau_{\text{mati}}^s \leq T(L_s, \gamma_s, \lambda_s^*)$ and $\tau_{\text{mati}}^f \leq \epsilon T(L_f, \gamma_f,\lambda_f^*)$, the set $\mathcal{E}$ is UGAS for system $\mathcal{H}_3$. 
\label{Corollary Stefan Timer}
\end{corollary}
\todo{Wrong!!! not UGAS}
\begin{proof}
    The proof of Corollary \ref{Corollary Stefan Timer} is shown in the Appendix.
\end{proof}


\begin{corollary}
Considering system $\mathcal{H}_3$ and suppose the conditions in Corollary \ref{Corollary Stefan Timer} hold, additionally, Assumption \ref{Assumption Exponential} holds. Let $b_1$, $b_2$, and $b_3$ come from Assumption \ref{Assumption interconnection} and $a_1$ and $a_2$ come from Lemma \ref{Lemma MATI}. Then there exist $\epsilon^* = \tfrac{a_1 a_2}{a_1 b_3 + b_1 b_2} $ such that for all $0<\epsilon<\epsilon^*$, $\tau_{\text{mati}}^s \leq T(L_s, \gamma_s, \lambda_s^*)$ and $\tau_{\text{mati}} \leq \epsilon T(L_f, \gamma_f,\lambda_f^*)$, the following statement holds:

There exists a $\tau_{\text{miati}}^{s,*} > 0$, such that if $\tau_{\text{miati}}^{s,*} \red{<} \tau_{\text{mati}}^s - \tau_{\text{miati}}$, $\mathcal{H}_3$ is UGES w.r.t $\mathcal{E}$.
\label{Corollary Stefan Exponential decay}
\end{corollary}
\begin{proof}
    The proof is similar to Corollary \ref{Corollary Exponential decay}.
\end{proof}
}
\begin{remark}
\aim{
In analytical terms, for the plant and controller pair, the upper bound of $\tau_{\text{mati}}$ in (\ref{eqn: New Timer}) that guarantees UGAS (UGES) for $\mathcal{H}_1$ will have the same value as $\tau_{\text{mati}}^f$ in (\ref{eqn: Stefan timer}) that guarantees UGAS (UGES) for $\mathcal{H}_3$. 
However, the clock (\ref{eqn: Stefan timer}) imposes the condition that $\tau_{\text{miati}} \leq \tfrac{1}{2}\tau_{\text{mati}}^f$ , whereas (\ref{eqn: New Timer}) only requires that $\tau_{\text{miati}} \leq \tau_{\text{mati}}$. Consequently, in order to achieve an analytical guarantee of stability, (\ref{eqn: Stefan timer}) requires approximately twice the bandwidth of (\ref{eqn: New Timer}).}

In practical terms, the feasibility condition $\tau_{\text{miati}} \leq  \tfrac{1}{2}\tau_{\text{mati}}^f$ may force the system to wait longer to transmit, which can lead to instability. This is demonstrated in the following. Consider the example from \cite[Section 4.4]{Stefan_thesis}, where the same plant and controller are used, but the slow and fast states are transmitted via one channel. 
Let $\tau_{\text{miati}}^*$ be the smallest MIATI the channel can provide, and suppose $\tau_{\text{miati}}^*=30.4ms$. Then, clocks (\ref{eqn: New Timer}) and (\ref{eqn: Stefan timer}) are feasible if and only if $\tau_{\text{mati}} \geq \tau_{\text{miati}}^*$ and $\tau_{\text{mati}}^f \geq 2\tau_{\text{miati}}^*$, respectively. We select $\tau_{\text{mati}} = 30.4ms$ and $ \tau_{\text{mati}}^f = 60.8ms$. 
%
To simulate the worst-case scenario, we set the priority to flow. 
Fig. \ref{fig: comparison-example} shows the system trajectories with clock (\ref{eqn: New Timer}) and (\ref{eqn: Stefan timer}), respectively. 
%
It is evident that the system cannot be guaranteed to be stable with the use of the smallest feasible value of $\tau_{\text{mati}}^f$ in clock proposed in (\ref{eqn: Stefan timer}). 
%
In contrast, clock proposed in (\ref{eqn: New Timer}) can stabilize the system with a feasible $\tau_{\text{mati}}$.
\end{remark}
%
\begin{figure}[ht]
    \centering
    \hfill
    \begin{minipage}[t]{0.49\linewidth}
    \centering
    % \includegraphics[width = \linewidth]{Figures/Stable example.pdf}
    \includegraphics[width = \linewidth]{Figures/Weixuan_timer.eps}
    \end{minipage}
    \begin{minipage}[t]{0.49\linewidth}
     % \includegraphics[width = \linewidth]{Figures/Unstable example.pdf}
       \includegraphics[width = \linewidth]{Figures/Stefan_timer.eps}
    \end{minipage}
    \caption{(Left) Batch reactor with clock (\ref{eqn: New Timer}); (Right) Batch reactor with clock (\ref{eqn: Stefan timer}).}
     \label{fig: comparison-example}
\end{figure}


% \aim{
% \subsection{Two Channel Case}
% Suppose for that same plant and controller pain, we now have two channels and each is dedicated to transmit slow and fast signal, respectively. Then the component \eqref{eqn: Stefan miati} and the requirement of $\tau_{\text{miati}} \leq \tau_{\text{mati}}^f$ in clock mechanism \eqref{eqn: Stefan timer} are no longer required.



% }



\cyan{
\begin{remark}
    We note that \cite{SPNCS} study a two-channel SPNCS with different plant and controller configurations, and our results are also applicable to the SPNCS in \cite{SPNCS} when only one channel is available.
\end{remark}
}
