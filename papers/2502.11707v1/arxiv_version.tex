\pdfoutput=1

\documentclass[11pt]{article}

\usepackage[final]{acl}

\usepackage{times}
\usepackage{latexsym}

\usepackage[T1]{fontenc}

\usepackage[utf8]{inputenc}

\usepackage{microtype}

\usepackage{inconsolata}

\usepackage{graphicx}
\usepackage{todonotes}
\usepackage{subfiles}
\usepackage{makecell}
\usepackage{subcaption}
\usepackage{spverbatim}
\usepackage{multirow}
\usepackage{rotating}
\usepackage{makecell}  %
\usepackage{float} %

\usepackage{listings}
\lstset{
    basicstyle=\ttfamily\footnotesize,
    breaklines=true,   %
    breakatwhitespace=false,  %
    columns=fullflexible,    %
    keepspaces=true
}



\usepackage{color}



\title{Ad-hoc Concept Forming in the Game Codenames\\ as a Means for Evaluating Large Language Models}



\author{
 \textbf{Sherzod Hakimov\textsuperscript{1}},
 \textbf{Lara Pfennigschmidt\textsuperscript{1}},
 \textbf{David Schlangen\textsuperscript{1,2}}
\\
\\
 \textsuperscript{1}Computational Linguistics, Department of Linguistics\\University of Potsdam, Germany\\
 \textsuperscript{2}German Research Center for Artificial Intelligence (DFKI), Berlin, Germany
\\
\texttt{firstname.lastname@uni-potsdam.de} 
}


\begin{document}
\maketitle
\begin{abstract}

This study utilizes the game Codenames as a benchmarking tool to evaluate large language models (LLMs) with respect to specific linguistic and cognitive skills. LLMs play each side of the game, where one side generates a clue word covering several target words and the other guesses those target words. We designed various experiments by controlling the choice of words (abstract vs.\ concrete words, ambiguous vs.\ monosemic) or the opponent (programmed to be faster or slower in revealing words). Recent commercial and open-weight models were compared side-by-side to find out factors affecting their performance. The evaluation reveals details about their strategies, challenging cases, and limitations of LLMs.  


\end{abstract}

\section{Introduction}

The astounding abilities of large language models (LLMs) have led to what could be called a `crisis of evaluation', where the previous paradigm of evaluating natural language processing (NLP) models---through pairs of problem instance and expected response---does not fit well any more. First, the main mode of usage of LLMs is through their embedding in a ``chatbot'', often across multiple turns, which is not represented by the reference-based evaluation mode. Second, the closed nature and sheer size of the training data, often acquired through automatic means from the open internet, raises fears that the usual test datasets have been ingested and hence the training data has become \textit{contaminated}, rendering the value of the tests even more doubtful \cite{DBLP:conf/iclr/GolchinS24, deng-etal-2024-investigating}.

\begin{figure}[ht!]
    \centering
    \includegraphics[width=1.0\linewidth]{figures/Fig1.pdf}
    \caption{Overview of the proposed approach where an LLM poses as both Spymaster (clue giver) and Field Operative (guesser) and plays against a mock opponent. The GameMaster orchestrates the game play by keeping and updating the game state on the board (image from \textit{https://codenames.game/}.}
    \label{fig:overview}
\end{figure}


The use of games as an interactive environment where LLMs are tasked to perform certain actions and scored whether they achieve the task or not has emerged as a response to this situation~\cite{chalamalasetti-etal-2023-clembench, DBLP:conf/iclr/WuTML24, DBLP:conf/iclr/Zhou0MZYQMBFNS24}, allowing the evaluation to more closely approximate the interactive use situation, and overcoming the data contamination problem in two ways. First, even for already documented games, it is easy to create new instances that lead to game play that differs from what is in existing data \cite{beyer2024clembench2024challengingdynamiccomplementary}. Second, to further extend the range of evaluated phenomena, it is only necessary to add new game implementations, rather than to create new datasets. It is this second dimension that we explore in this paper.

We have implemented the game Codenames as a challenging means (already investigated in psycholinguistic studies, see below) for evaluating certain language-use capabilities. In this game, a first player needs to provide to a second player a clue which singles out certain words within a larger set of words given to both players (see Figure~\ref{fig:qual2} below for examples).

The game requires cooperation, with the first player needing to form and name an \textit{ad-hoc concept} that spans the target concepts, in a way that they assume is understandable to others (\textit{theory of mind}) \citep{kim-2019-understanding-NLP-via-codenames}.
Players need to connect words in a wide \textit{variety of relations} such as homonyms, antonyms, rhymes, or popular culture references \citep{jaramillo-2020-transformers-for-word-association-in-codenames}. Clue generation is also  a task of \textit{co-creativity} \citep{spendlove-constraints-for-creativity}, testing skills in the evaluation of \textit{semantic relatedness of word}s and \textit{common-sense reasoning} \citep{bitton-2022-winogavil} as well as the ability to constrain clues and negatively associate them with any non-team word.

Players need to predict the partners' behaviour and knowledge \citep{cserhati-2022-codenames-co-occurrence-counting, kumar-2021-connector-asociative-vs-distributional-semantic-models}, so one cannot simply optimise their own behaviour \citep{jaramillo-2020-transformers-for-word-association-in-codenames} without acknowledging the cultural background and knowledge level of their teammates \citep{shaikh-2023-cross-cultural-pragmatic-inference-with-codenames-duet}, hence requiring cooperation. Figure~\ref{fig:overview} shows the overall idea where players (Spymaster and Field Operative) are LLMs that play against a programmed mock opponent. The programmatic \textit{GameMaster} component (part of the framework we use) orchestrates the game play by providing inputs and generated outputs among the parties, checks whether players comply to the rules of the game.

Our contributions are as follows: i) benchmarking LLMs to test their ad-hoc concept generation, cooperation, pragmatic reasoning capabilities, ii) comparison of open-weight and commercial models under various experiments, iii) in-depth analysis of how best-performing models navigate the task.



\section{The game: Codenames}\label{sec:codenames}

Codenames~\citep{chvatil-2015-codenames-game} is a cooperative board game with two teams (blue and red) that try to uncover their team agents' code names before the other team finds all of theirs. The board is set up with 25~word cards. Each team has a ``Spymaster'' that knows which words on the board represent their team (8+1 for the starting team), the opposing team (8), innocent bystanders (7), and the assassin (1). The team starting the game has one more word to uncover to balance out the advantage of going first. Our implemented version deploys one Spymaster and Field Operative on the same team. The opponent team is \textit{mocked} with an ideal behaviour of revealing \textit{n} own words each turn.

The Spymaster takes turns providing clues for their teammates -- the ``Field Operatives''. Each clue consists of a word related to one or more (code names) targeted words. It has to output in this way:
\begin{verbatim}
    CLUE: <clue>
    TARGETS: <list of targets>
\end{verbatim}

Only the clue is passed on to the Field Operative, who then guesses the matching words:
\begin{verbatim}
    GUESS: <list of guesses>
\end{verbatim}

If the guess is correct, the team can continue guessing as many names as the Spymaster indicated in their clue. If the team is unsure, they can also end their turn voluntarily. If the team's guess is incorrect, meaning they contacted an innocent bystander or an word of the opposing team, the identity is revealed, and the team's turn ends. If the team uncovers any \textit{assassin} word, the team immediately loses the game.

We have implemented the game using the clembench~\citep{chalamalasetti-etal-2023-clembench} framework where the GameMaster orchestrates the gameplay by 1) checking the required formatting of generated outputs by Spymaster or Field Operative, 2) passing the outputs between players. The Spymaster and Field Operative prompts are given in Appendix~\ref{appendix:prompts}.


\subsection{Previous Computational Work}

Earlier work focused on using various word embedding techniques (choose the clue that is closest to targets and most distant to distractors)~\cite{kim-2019-understanding-NLP-via-codenames,jaramillo-2020-transformers-for-word-association-in-codenames}. Later, such methods were combined with associative methods that use language graphs for generating clues or guessing~\cite{koyyalagunta-2021-codenames-language-graphs}. Other approaches involve concentrating on word co-occurrence measurements ~\cite{deRijk-2020-paper-word-embeddings-collocations, cserhati-2022-codenames-co-occurrence-counting} for capturing synonymy, semantic similarity, or word-relatedness measure instead of just focusing on word embeddings.

Later research started looking into using LLMs to generate clues~\cite{spendlove-2022-competitive-language-games-as-creative-tasks}. The idea of benchmarking LLMs led to the development of various datasets, e.g.\ BigBench~\cite{srivastava-2022-bigbench} includes Codenames as one of the many tasks to test emergent abilities of models~\cite{wei-2022-emergence,ozturkler-2023-thinksum,lu-2023-emergence-just-in-context-learning}.  \citet{DBLP:journals/corr/abs-2412-11373} recently explored using Codenames to benchmark LLMs where two pairs of LLMs (red vs.\ blue team) play against each other. Our method differs from theirs by comparing the language model team against a deterministic opponent. Not using a deterministic opponent could lead to different results every time the same game is played due to the non-deterministic nature of language models~\cite{song2024goodbadgreedyevaluation}. Another extension of our work lies in the experimental setup, where we study the effect of selecting words on a board and their relations in much more depth.
    
\section{Experimental Setup}

\subsection{Board Generation}\label{subsec:board_generation}

We used different sets of words to design experiments. Each experiment includes 10 instances (boards) where the words are chosen randomly from a specific set. The default mock opponent uncovers one word per turn (\textit{n=1}). The default word list is by \citet{deRijk-2020-thesis-codenames-modelling-word-association} with \textit{one assassin word} per board. We defined the following experiments by changing specific default parameters, which correspond to 130 instances:






\noindent $\bullet$
\textbf{Risk level}: We included five assassin words in the set called \textit{high risk}. The \textit{low risk} set has no assassin words. The rationale here is to see whether models target less number of words to mitigate the risk of revealing assassin words.
    
\noindent $\bullet$
 \textbf{Word association}: We selected 45 category norms (e.g.\ bird name, kitchen utensil, country, military title, etc.) from the corpus by \citet{castro2021category-norms}. The \textit{easy} set is created by selecting 3-5 categories, sampling words for each category, and assigning them to the same team (3-5 turns by targeting the category). The \textit{difficult} set is created by ensuring that sampled words are distributed across all possible groups (team, opponent, innocent, assassin) and not assigned to the same team. The rationale here is whether models actually can capture those obvious associations on the easy set and whether they can play the difficult one at all.
    
\noindent $\bullet$
 \textbf{Opponent level}: We created three sets where the mock opponent turns \textit{two}, \textit{one} or \textit{none} words per turn, which correspond to \textit{difficult}, \textit{easy}, and \textit{none} levels, respectively. The rationale here is to check whether LLMs can play against a faster opponent that constantly reveals two words at a time.

    
\noindent $\bullet$
 \textbf{Word frequency}: All nouns from the SUBTLEX-US corpus~\cite{subtlexus} were filtered out to create two sets for low and high-frequency lists. We used the top and bottom 250 words for the frequency lists of the \textit{high} and \textit{low}. Typical human players would usually struggle with low frequency words and our rationale is to check whether it poses a similar challenge to LLMs too.
    
\noindent $\bullet$
 \textbf{Word ambiguity}: The corpus provided by \citet{beekhuizen2021ambiguity} includes monosemes (words with single sense) and homonyms (words with multiple senses). The \textit{ambiguous} set is composed of homonyms while the \textit{unambiguous} one includes the monosemes. The hypothesis here is that words with multiple meanings are easier to find connections between them than ambiguous words.
    
\noindent $\bullet$
 \textbf{Word concreteness}: Two sets of words where one corresponds to concrete concepts and the other includes abstract ones. \citet{brysbaert2014concreteness} collected word concreteness ratings (Likert scale between 1-5). We used the top 500 words with the lowest and highest concreteness ratings for \textit{abstract} and \textit{concrete} word lists, respectively. The hypothesis here to check whether LLMs play better with concrete words as it is easier for human players to find association between them in contrast to abstract concepts.



\subsection{Metrics}\label{subsec:metrics}

The clembench framewor measures how many of the instances (boards) have resulted in a \textit{Played} or \textit{Aborted} state. The gameplay is marked as \textit{Aborted} if either player does not follow the formatting instructions when generating an output (as explained in Section~\ref{sec:codenames}). \textit{Played} is the ratio of remaining gameplays (\textit{episodes}) where formatting instructions have been followed. The \textit{Played} ratio is further divided into \textit{Success} if the team reveals own words faster than mock opponent, or \textit{Lose}    if an assassin word is revealed or the mock is faster. 

The framework also requires one metric called \textit{Quality Score} corresponding to how well the task has been solved. The \textit{Quality Score}, essentially a win rate,  is the average number of games won (successful). The main ranking score for evaluated LLMs is the \textit{clemscore}, which is the macro-average quality score multiplied by the macro-average proportion of played games to find a balance between solving most tasks and following instructions. We have also implemented the following metrics to analyse the strategies taken by models: 

\noindent $\bullet$
\textbf{Sensitivity}: The number of revealed divided by the total team words.

\noindent $\bullet$
\textbf{Efficiency}: We set the bar at two target words per turn as the highest efficiency a model can reach, as that is a reasonable efficiency for humans. It is calculated as:\\
    $ min(1, \frac{1}{2}\cdot\frac{\mbox{team words revealed}}{\mbox{number of turns}})$


    






\subsection{Evaluated Models}
We evaluated open-weight and commercial models with a \textit{zero-shot} setting where \textit{temp=0}. We included the most recent commercial models such as: \textit{o3-mini} (Jan~'25), \textit{GPT-4o} (Aug~'24) \textit{Claude-3-5} (Sonnet, Oct~'24), and \textit{Gemini-2.0-Flash} (Feb~'25). We also included recent open-weight models: \textit{Llama-3.1} (8B, 70B, 405B)~\citep{llama31}, \textit{Llama-3.3} (70B), \textit{Qwen2} (72B)~\citep{qwen2}, \textit{Qwen2.5} (Coder-32B, 72B, Max)~\citep{qwen25}, and \textit{Deepseek} (v3, r1)~\citep{deepseekv3,deepseekai2025deepseekr1incentivizingreasoningcapability}. We used the APIs of the respective commercial models. For open-weight models, we ran the inference on two NVIDIA A100 GPUs. Two Deepseek models, Llama-3.1-405B and Qwen-Max, were run via the OpenRouter API.





\section{Results}

\begin{table}[]
\centering
\footnotesize
\begin{tabular}{l|c|c|c}
\textbf{Model} & \textbf{\makecell{clemscore}} & \textbf{\% Played} & \textbf{\makecell{Quality \\ Score}} \\ \hline

o3-mini & \textbf{49.2} & \textbf{100.0} & 49.2 \\
Claude-3-5 & 46.9 & 93.8 & 50.0 \\
GPT-4o & 45.4 & 93.8 & 48.4 \\
Deepseek-r1 & 45.4 & 85.4 & \textbf{53.2} \\
Gemini-2.0 & 37.7 & 96.2 & 39.2 \\
Llama-3.1-70B & 36.9 & 90.0 & 41.0 \\
Deepseek-v3 & 33.8 & 86.9 & 38.9 \\
Qwen2.5-72B & 30.0 & 72.3 & 41.5 \\
Llama-3.3-70B & 29.2 & 80.0 & 36.5 \\
Llama-3.1-405B & 29.2 & 76.2 & 38.4 \\
Qwen-max & 25.4 & 70.0 & 36.3 \\
Qwen2-72B & 20.8 & 58.5 & 35.5 \\
Qwen2.5-32B & 20.8 & 62.3 & 33.3 \\
Llama-3.1-8B & 14.6 & 52.3 & 27.9 \\
 \hline


\end{tabular}
\caption{Ranking of all benchmarked LLMs.}
\label{tab:main-results}
\end{table}

\begin{table*}[ht!]
\centering
\footnotesize

\begin{tabular}{|ll|c|c|c|c|c|c|c|}
\hline
\multicolumn{2}{|c|}{\textbf{Experiment}} & \textbf{o3-mini} & \textbf{GPT-4o} & \textbf{LM-3.1} & \textbf{LM-3.3} & \textbf{Claude-3.5} & \textbf{Deepseek-r1} & \textbf{Gemini-2.0} \\ \hline
\multicolumn{1}{|l|}{\multirow{2}{*}{Risk Level}}
& low & 70.0 & 75.0 & 50.0 & 50.0 & 75.0 & \textbf{87.5} & 55.6 \\ 
\multicolumn{1}{|l|}{}
& high & 20.0 & \textbf{37.5} & 11.1 & 20.0 & 30.0 & 10.0 & 10.0 \\ \hline
\multicolumn{1}{|l|}{\multirow{2}{*}{Association}}
& easy & \textbf{100.0} & \textbf{100.0} & \textbf{100.0} & \textbf{100.0} & \textbf{100.0} & \textbf{100.0} & \textbf{100.0} \\ 
\multicolumn{1}{|l|}{}
& difficult & 20.0 & 10.0 & 20.0 & 0.0 & 20.0 & \textbf{28.6} & 12.5 \\ \hline
\multicolumn{1}{|l|}{\multirow{3}{*}{Opponent}}
& none & \textbf{80.0} & 77.8 & \textbf{80.0} & 75.0 & 57.1 & 62.5 & 77.8 \\ 
\multicolumn{1}{|l|}{}
& easy & 50.0 & 33.3 & 14.3 & 28.6 & 40.0 & \textbf{80.0} & 11.1 \\ 
\multicolumn{1}{|l|}{}
& difficult & 0.0 & 0.0 & 0.0 & 0.0 & 0.0 & \textbf{22.2} & 0.0 \\ \hline
\multicolumn{1}{|l|}{\multirow{2}{*}{Frequency}}
& low & 60.0 & \textbf{66.7} & 50.0 & 33.3 & 60.0 & 50.0 & 30.0 \\ 
\multicolumn{1}{|l|}{}
& high & 20.0 & 30.0 & 50.0 & 20.0 & 44.4 & 25.0 & \textbf{50.0} \\ \hline
\multicolumn{1}{|l|}{\multirow{2}{*}{Ambiguity}}
& none & \textbf{80.0} & 60.0 & 22.2 & 55.6 & \textbf{80.0} & 55.6 & 40.0 \\ 
\multicolumn{1}{|l|}{}
& ambiguous & 40.0 & 33.3 & 37.5 & 10.0 & \textbf{62.5} & 16.7 & 40.0 \\ \hline
\multicolumn{1}{|l|}{\multirow{2}{*}{Concreteness}}
& concrete & 80.0 & 50.0 & 66.7 & 44.4 & 50.0 & \textbf{88.9} & 40.0 \\ 

\multicolumn{1}{|l|}{}
& abstract & 20.0 & \textbf{60.0} & 0.0 & 16.7 & 40.0 & 50.0 & 40.0 \\  \hline
\end{tabular}


\caption{Detailed results across different experiments. Only high performing LLMs were selected. The values correspond to the Quality Score for each experiment. \textit{LM-3.1} $\rightarrow$ Llama-3.1-405B, \textit{LM-3.3} $\rightarrow$ Llama-3.3-70B}
\label{tab:experiment-results}
\end{table*}


\subsection{Overall Analysis}
The overall results are given in Table~\ref{tab:main-results} where the \textit{clemscore}, \textit{Played}, and \textit{Quality Score} are averaged across all experiments. The first observation we make is that, as expected,larger models perform better. In line with this, commercial models outperform open-weight ones by some margin (five points between \textit{o3-mini} and \textit{Deepseek-r1}). \textit{o3-mini} is the only model that played all episodes without once making an instruction following error in the game. However, we can see that the best model achieves only \textit{49.2\%} success rate in winning the game against the mock opponent. To investigate specific experiments, we selected seven high-performing models to compare them in detail. The results are given in Table~\ref{tab:experiment-results}.




\textbf{Risk level}: We expected the high risk to be more complex than the low one because there are five assassin words. This expected behaviour holds for all models, e.g. \textit{o3-mini} has a margin of 50 points between both experiments. In the high-risk experiment, \textit{GPT-4o} achieves the best score of \textit{37.5}, which is a substantial margin of \textit{17.5} points compared to the second-best result.

\textbf{Word association}: All models achieved a perfect score for the \textit{easy} set. The difficult case is much more challenging as no model reaches 30 points.

\textbf{Opponent level}: We tested three levels of the mock opponent where the difference lies in how fast the words are revealed. The performance on the first level is significantly higher for all models as it is easier to beat the mock opponent who does not reveal any words. Even in this setting, the best models (\textit{o3-mini} and \textit{Llama-3.1-405B}) can only reach \textit{80} points. However, once we switch to other levels, we see a clear drop in performance for most models, except \textit{Deepseek-r1}. The difficult level shows even striking results where only \textit{Deepseek-r1} managed to achieve some performance while other models lost all episodes to the mock opponent.

\textbf{Word frequency}: The expectation here is that higher-frequency words are easier to play with (at least for human players). This assumption does not apply as most models are better at \textit{low frequency} set, except \textit{Gemini-2.0}.



\textbf{Ambiguity}: The expectation here is that monosemic words are easier to play with, and we can confirm that this holds for most models. \textit{Claude-3.5} is the only model to surpass 50\% success rate in the \textit{ambiguous} set.

\textbf{Concreteness}: Generally, all models perform better on the \textit{concrete} set, except for \textit{GPT-4o}. Interestingly, \textit{Gemini-2.0} gets equal points on both sets. It indicates that abstract words are indeed more challenging (as for humans) for models.



\subsection{In-depth Analysis}


\begin{figure}[ht!]
    \centering
    \begin{minipage}{0.45\textwidth}
        \centering
        \includegraphics[width=1.0\linewidth]{figures/avg_targets_per_turn.pdf}
    \end{minipage}
    \hfill
    \begin{minipage}{0.45\textwidth}
        \centering
        \includegraphics[width=1.0\linewidth]{figures/avg_guesses_per_turn.pdf}
    \end{minipage}
    \caption{Average number of words targeted (top) and words guessed (bottom) by models at each turn}
        \label{fig:avg_turn}
\end{figure}


\subsubsection{Number of Targets, Guesses \& Revealed} In Figure~\ref{fig:avg_turn}, we present the average number of words targeted and words guesses by selected models. We can see that high-performing models such as \textit{o3-mini} and \textit{Deepseek-r1} generate at least 1-2 more words as targets and guesses in the beginning. Targeting and guessing more words in a single turn is the standard strategy in Codenames to win~\cite{spendlove-2022-competitive-language-games-as-creative-tasks}, especially needed when playing against the mock opponent, which reveals one word at each turn. %
Models tend to guess fewer words than were targeted. For instance, \textit{o3-mini} on average targets more than four words but guessed considerably fewer for the first turn, unlike \textit{Deepseek-r1}, which targets and guesses an almost equal number of words. In Figure~\ref{fig:avg_target_guess}, we included the average number of target, guessed and revealed (where the guess is team word) words per model. We can see that only \textit{Deepseek-r1} exceeds the threshold of more than two words (2.2), while the rest have close values (1.5-1.9). It indicates that all models guess wrong words by revealing words from the opponent team or distractors, or even assassin words.


\subsubsection{Success, Lose \& Aborted Lose Rates}

\begin{figure}[ht!]
    \centering
    \begin{minipage}{0.5\textwidth}
        \centering
        \includegraphics[width=0.90\linewidth]{figures/played.pdf}
    \end{minipage}
    
    \hfill
    \begin{minipage}{0.45\textwidth}
        \centering
        \includegraphics[width=1.0\linewidth]{figures/loss_distribution.pdf}
    \end{minipage}
    \caption{Distribution of Success, Lose, Aborted episodes (up), and distribution of cases where models lose (bottom).}
        \label{fig:success_lose_rates}
\end{figure}
Figure~\ref{fig:success_lose_rates} includes the distribution of episodes across \textit{Success}, \textit{Lose}, and \textit{Aborted}. To recall, \textit{Success} is when a model follows the game's rules and beats the mock opponent by revealing the team words faster, \textit{Lose} is when the mock opponent is faster or when a model reveals assassin words. \textit{Aborted} is when a model does not follow formatting instructions. The top graphic shows that even best-performing models barely reach the 50\% \textit{Success} rates where most episodes are lost or aborted. The ratio of \textit{Aborted} episodes is higher for open-weight models. The bottom graphic divides the \textit{Lost} cases further into two groups: \textit{assassin word is revealed} or \textit{mock opponent is faster}. For most models, the main issue is losing due to being slower in revealing words than the mock opponent. Only \textit{Deepseek-r1} lost more due to revealing more assassin words than others. It shows that all models struggled with the task and lost against a strategy of revealing one word every turn. 

\subsubsection{Efficiency \& Sensitivity}

\begin{figure}
    \centering
    \includegraphics[width=1\linewidth]{figures/efficiency_sensitivity.pdf}
    \caption{Average \textit{efficiency} and \textit{sensitivity} scores for selected models.}
    \label{fig:efficiency-sensitivity}
\end{figure}

Next, we analyse how efficient the models are regarding targeting multiple words at each turn (see metrics defined in Section~\ref{subsec:metrics}). Figure~\ref{fig:efficiency-sensitivity} shows the \textit{efficiency} and \textit{sensitivity} scores for the selected models. We can observe that \textit{o3-mini}, \textit{Deepseek-r1}, \textit{Llama-3.3-70B} and \textit{Deepseek-v3} have higher efficiency scores, which indicates that these models target two or more words each turn. A similar observation has also been made in Figure~\ref{fig:avg_turn}. By looking at the \textit{sensitivity} scores, we can conclude that \textit{Deepseek-r1} is better at this task than \textit{Deepseek-v3} because it revealed more words (sensitivity score). Models such as \textit{Claude-3.5} and \textit{GPT-4o} are more consistent (efficiency and sensitivity are closer to each other) in terms of the number of targets, guessed, and revealed words.

\begin{table}[ht!]
\centering\footnotesize
\begin{tabular}{|l|c|c|c|c|c|c|}
\hline
\textbf{Model} & \makecell{\textbf{Target} \\ \textbf{Halluc.}} & \makecell{\textbf{Guess} \\ \textbf{Halluc.}} & \makecell{\textbf{Wrong \#} \\\ \textbf{of Guesses}} & \makecell{\textbf{Guess is} \\ \textbf{Clue}} \\ \hline
o3-mini & 0 & 0 & 0 & 0  \\ \hline
DS-r1 & 0 & 0 & 1 & 0  \\ \hline
GPT-4o & 2 & 3 & 0 & 0  \\ \hline
GM-2.0 & 1 & 0 & 4 & 0 \\ \hline
Cl-3.5 & 3 & 5 & 0 & 0  \\ \hline
LM1-70 & 2 & 2 & 1 & 7  \\ \hline
DS-v3 & 6 & 6 & 1 & 2  \\ \hline
LM3-70 & 2 & 2 & 3 & 13 \\ \hline
LM-405 & 10 & 2 & 16 & 0  \\ \hline
QW-72 & 5 & 6 & 0 & 21  \\ \hline
QW-M & 12 & 8 & 0 & 15  \\ \hline
QW-32 & 10 & 7 & 0 & 19  \\ \hline
QW-72B & 9 & 12 & 0 & 30 \\ \hline
LM1-8B & 3 & 7 & 18 & 28  \\ \hline
\end{tabular}
\caption{Error types and their counts for each model where an episode was aborted by the GameMaster.}
\label{table:error_counts}
\end{table}

\subsubsection{Typical Errors}  To understand where models fail and how higher-performing models differ from lower ones, we analysed the most common errors, then categorised them and counted each occurrence, see Table~\ref{table:error_counts}. 

\begin{figure*}[ht]
    \centering
    \begin{minipage}{0.88\textwidth}
        \centering
        \includegraphics[width=1.0\linewidth]{figures/examples4.pdf}
        \caption{Transcript of an episode in Word Association ``easy'' experiment.}
        \label{fig:qual1}
    \end{minipage}
    \hfill
    \begin{minipage}{0.88\textwidth}
        \centering
        \includegraphics[width=1.0\linewidth]{figures/examples2.pdf}
    \caption{Transcript of an episode in Concreteness ``abstract'' experiment. Note that only the clue is given to player B; the list of targeted words is only to get an insight into the reasoning of player A.}
        \label{fig:qual2}
    \end{minipage}
    
\end{figure*}

The differentiating factor in high-performing models is that hallucination and instruction following issues appear more rarely. For instance, the first error type, \textit{Target Hallucinated}, refers to cases where Spymaster generates a clue and targets some words, but some of these do not exist on the board (as should be known to the model). In such cases, the GameMaster aborts that episode. Similarly, \textit{Guess Hallucinated} is an error that occurs on the Field Operative side where it guesses a word that does not exist on the board. Mostly, \textit{Llama-3.1-405B} and \textit{Llama-3.1-8B} have another issue with guessing the correct number of words that the Spymaster indicates. They tend to guess more than the number of target words (note here: models can guess less but not more than target words). Lastly, the common issue, \textit{Guess is Clue}, with low-performing models is that the guessed word is the same as the clue in many cases. It shows a lack of pragmatic reasoning for choosing unrevealed candidate words from available ones on a board. In all of these cases and some minor ones, e.g.\ tags such as ``CLUE:'', ``TARGETS:'', ``GUESS:'' are omitted, the GameMaster aborts the game because the rules are not followed. Such instruction-following issues happen mostly with \textit{Deepseek-r1}. See Table~\ref{table:error_counts_full} for the complete list.













\subsection{Qualitative Analysis}

We included sample outputs for the \textit{Word Association - easy} experiment in Figure~\ref{fig:qual1}. Recall that all models achieved the perfect score for this experiment (see Table~\ref{tab:experiment-results}). The words were selected from these categories ``fish'', ``unit of distance'', ``four footed animal'', ``part of a building'', ``fruit'', ``an article of furniture'', ``country'', ``musical instrument'', ``type of fuel'', ``weapon'', ``crime'', ``sport''. 

\textit{o3-mini} generates clues close to the ground truth categories of words. In the second turn, it makes a slight mistake by guessing the distractor word ``couch''. Similarly, \textit{GPT-4o} generates similar clues but focuses on only two words at a time. An interesting case occurs with \textit{Qwen2.5-Coder-32B} where, in the first turn, it targets four words with the clue ``wood''. The other two models targeted the word ``carp'' by choosing the ``fish'' or ``animal'' categories, but \textit{Qwen2.5-Coder-32B} chose the sense of ``carpenter, lumber quality'' to connect the clue ``wood'' to ``carp''.


Figure~\ref{fig:qual2} shows sample outputs for the \textit{Concreteness - abstract} experiment. As we can see, the chosen words are not typical daily life words that would challenge human players in Codenames. \textit{Claude-3.5} manages to play this episode and win the game. We can see that it generates decent clues that combine the target words. It made one mistake by guessing a distractor word in the second turn. The gameplay by \textit{o3-mini} is even more fascinating. The average number of target words is three, and it generates matching clues. However, due to the strategy of targeting and guessing more words, it gives a massive advantage to the opponent by revealing 50\% of their teams' words (``misconception'', ``humbug'', ``extent'', ``endlessness''). Even though the model manages to reveal seven out of nine words (``heyday'' and ``imposition'' were never revealed), it lost the game because the mock opponent revealed words (primarily due to four additional words revealed mistakenly by \textit{o3-mini}).

Figure~\ref{fig:qual3} includes sample episodes for the \textit{Risk level - high} experiment with five assassin words. \textit{o3-mini}, \textit{Claude-3.5}, \textit{Gemini-2.0}, \textit{Deepseek-r1} guessed one of the assassin words and lost the game. \textit{Llama3.3-70B} lost the game due to guessing (six words) more than what was targeted (five words). 

Figure~\ref{fig:qual4} shows samples for the \textit{ambiguous} words. \textit{Deepseek-v3} revealed three opponent words but still managed to win the game.

\subsection{Discussion}

\textbf{Commercial vs. open}: We can notice that commercial models outperform open-weight ones by some margin. We categorised the errors by models and counted them (see Table~\ref{table:error_counts}). The main reasons for open-weight models having a lower ratio of Played episodes are i) these models often hallucinate while choosing target words, which means they add a word in the target list that does not exist on the board, ii) hallucination also occurs by guessing words that do not exist on the board, iii) guessing the clue word itself. For instance, the performance difference between \textit{Llama3.1-70B} and \textit{405B} can be explained with the bigger model: i) hallucinating target words and ii) guessing too many words.

\textbf{Choice of words}: The selection of words (ambiguous, abstract, high or low frequency, more assassin words) impacts the performance as expected. Of all the experiments, playing against a mock opponent that revealed two words and word associations with difficulty levels proved to be the most challenging. Similarly, abstract words seemed to be more demanding than concrete words. However, we observed that the frequency of words does not directly impact performance when looking at all model results, whereas, for humans, less frequent words might be more challenging. Similar remarks can be made for ambiguity and abstract word sets where the results are somewhat mixed and where humans are expected to find them demanding.

\textbf{Reasoning models}: By looking at the best performing models, we can conclude that the best of one of the commercial and open-weight options are reasoning models where \textit{Deepseek-r1} outperforming some commercial models such as \textit{Gemini-2.0} or \textit{Qwen-max}. However, such an impressive performance comes at the cost of high latency. It took almost two minutes per query for \textit{r1} and two seconds for \textit{v3} (see Table~\ref{tab:latency}).

\textbf{Do LLMs have the required abilities to play Codenames?} The models cannot play efficiently in some experiments by looking at the win rates (Quality Score) for all models. Codenames is a challenging task that involves deep language understanding, theory of mind, cooperation, and pragmatic reasoning. Our experimental results suggest that LLMs do possess knowledge about word associations, and it was shown that they can access it strategically (see Figure~\ref{fig:qual1} where \textit{o3-mini} targets four words with clue ``animal''). Another strategy that we observed is the \textit{risk taking strategy} where models target more than two words per turn to win the game (see Figure~\ref{fig:avg_target_guess}). Such a strategy would be a clear winner against a mock opponent that reveals only one word per turn. However, we have seen cases where this strategy resulted in actually losing the game by revealing the opponent teams' words (see \textit{o3-mini} in Figure~\ref{fig:qual2}). Another risky strategy was observed with the high-risk set, where models could not navigate the experiment with five assassin words. Some models still went on to target a lot of words while risking the error on the guesser side (see \textit{o3-mini} on Figure~\ref{fig:qual4} where it targets nine words at once and loses the game).

The experiments also reveal certain aspects of \textit{pragmatic reasoning} in multi-turn tasks where if a particular clue was not utilised to guess certain target words, it has been revised (see Figure~\ref{fig:qual2} where \textit{o3-mini} targets the word ``existentialism'' with the clue ``doubt'' and it was not guessed, then reintroduced another clue ``spirit'' to the guess the same word again). The cooperation aspect can be seen where some models are consistent in terms of choosing the number of target words and how many of them were correctly guessed (see Figure~\ref{fig:efficiency-sensitivity}, \textit{GPT-4o}, \textit{Claude-3.5}).


\section{Conclusion}

We implemented Codenames to benchmark LLMs by targeting their pragmatic reasoning, language understanding specifically for ad-hoc concept generation, and cooperation capabilities. We tested the most recent commercial and open-weight models on various experiments and difficulty levels. We can generally confirm that commercial models are ahead in performance compared to open-weight ones. The main reasons for better performance can be attributed to having less errors with regards to hallucinations, instruction following, and pragmatic reasoning. However, when looking at played episodes, we can say that even the best performing models do not win over 50\% of the games. It clearly indicates that the task is far from being solved. Overall, the presented solution provides a clear method for benchmarking LLMs using game-based evaluation to target specific capabilities. 




\section*{Limitations}

The current study is restricted to only English in its current state. While we have yet to do this, translating the prompts and finding the matching word lists should be possible for other languages, too. We plan to do this in future work.

As discussed in the analysis above, some of the findings are limited to general strategies applied internally by the models. We plan to study the reasoning capabilities in detail to understand the underlying blocks that leads to certain clues or guesses to be generated.

\section*{Ethics Statement}

Using paid proprietary APIs with underlying models about which little is known (training data, model architecture) in academic research is less than ideal. At the moment, the models benchmarked here seem to be the high-performing ones that are commercially used. It is our hope that more open models with high performance will be released soon, and proper research can be done with them.


\bibliography{anthology_0, anthology_1, custom}


\subsection{Lloyd-Max Algorithm}
\label{subsec:Lloyd-Max}
For a given quantization bitwidth $B$ and an operand $\bm{X}$, the Lloyd-Max algorithm finds $2^B$ quantization levels $\{\hat{x}_i\}_{i=1}^{2^B}$ such that quantizing $\bm{X}$ by rounding each scalar in $\bm{X}$ to the nearest quantization level minimizes the quantization MSE. 

The algorithm starts with an initial guess of quantization levels and then iteratively computes quantization thresholds $\{\tau_i\}_{i=1}^{2^B-1}$ and updates quantization levels $\{\hat{x}_i\}_{i=1}^{2^B}$. Specifically, at iteration $n$, thresholds are set to the midpoints of the previous iteration's levels:
\begin{align*}
    \tau_i^{(n)}=\frac{\hat{x}_i^{(n-1)}+\hat{x}_{i+1}^{(n-1)}}2 \text{ for } i=1\ldots 2^B-1
\end{align*}
Subsequently, the quantization levels are re-computed as conditional means of the data regions defined by the new thresholds:
\begin{align*}
    \hat{x}_i^{(n)}=\mathbb{E}\left[ \bm{X} \big| \bm{X}\in [\tau_{i-1}^{(n)},\tau_i^{(n)}] \right] \text{ for } i=1\ldots 2^B
\end{align*}
where to satisfy boundary conditions we have $\tau_0=-\infty$ and $\tau_{2^B}=\infty$. The algorithm iterates the above steps until convergence.

Figure \ref{fig:lm_quant} compares the quantization levels of a $7$-bit floating point (E3M3) quantizer (left) to a $7$-bit Lloyd-Max quantizer (right) when quantizing a layer of weights from the GPT3-126M model at a per-tensor granularity. As shown, the Lloyd-Max quantizer achieves substantially lower quantization MSE. Further, Table \ref{tab:FP7_vs_LM7} shows the superior perplexity achieved by Lloyd-Max quantizers for bitwidths of $7$, $6$ and $5$. The difference between the quantizers is clear at 5 bits, where per-tensor FP quantization incurs a drastic and unacceptable increase in perplexity, while Lloyd-Max quantization incurs a much smaller increase. Nevertheless, we note that even the optimal Lloyd-Max quantizer incurs a notable ($\sim 1.5$) increase in perplexity due to the coarse granularity of quantization. 

\begin{figure}[h]
  \centering
  \includegraphics[width=0.7\linewidth]{sections/figures/LM7_FP7.pdf}
  \caption{\small Quantization levels and the corresponding quantization MSE of Floating Point (left) vs Lloyd-Max (right) Quantizers for a layer of weights in the GPT3-126M model.}
  \label{fig:lm_quant}
\end{figure}

\begin{table}[h]\scriptsize
\begin{center}
\caption{\label{tab:FP7_vs_LM7} \small Comparing perplexity (lower is better) achieved by floating point quantizers and Lloyd-Max quantizers on a GPT3-126M model for the Wikitext-103 dataset.}
\begin{tabular}{c|cc|c}
\hline
 \multirow{2}{*}{\textbf{Bitwidth}} & \multicolumn{2}{|c|}{\textbf{Floating-Point Quantizer}} & \textbf{Lloyd-Max Quantizer} \\
 & Best Format & Wikitext-103 Perplexity & Wikitext-103 Perplexity \\
\hline
7 & E3M3 & 18.32 & 18.27 \\
6 & E3M2 & 19.07 & 18.51 \\
5 & E4M0 & 43.89 & 19.71 \\
\hline
\end{tabular}
\end{center}
\end{table}

\subsection{Proof of Local Optimality of LO-BCQ}
\label{subsec:lobcq_opt_proof}
For a given block $\bm{b}_j$, the quantization MSE during LO-BCQ can be empirically evaluated as $\frac{1}{L_b}\lVert \bm{b}_j- \bm{\hat{b}}_j\rVert^2_2$ where $\bm{\hat{b}}_j$ is computed from equation (\ref{eq:clustered_quantization_definition}) as $C_{f(\bm{b}_j)}(\bm{b}_j)$. Further, for a given block cluster $\mathcal{B}_i$, we compute the quantization MSE as $\frac{1}{|\mathcal{B}_{i}|}\sum_{\bm{b} \in \mathcal{B}_{i}} \frac{1}{L_b}\lVert \bm{b}- C_i^{(n)}(\bm{b})\rVert^2_2$. Therefore, at the end of iteration $n$, we evaluate the overall quantization MSE $J^{(n)}$ for a given operand $\bm{X}$ composed of $N_c$ block clusters as:
\begin{align*}
    \label{eq:mse_iter_n}
    J^{(n)} = \frac{1}{N_c} \sum_{i=1}^{N_c} \frac{1}{|\mathcal{B}_{i}^{(n)}|}\sum_{\bm{v} \in \mathcal{B}_{i}^{(n)}} \frac{1}{L_b}\lVert \bm{b}- B_i^{(n)}(\bm{b})\rVert^2_2
\end{align*}

At the end of iteration $n$, the codebooks are updated from $\mathcal{C}^{(n-1)}$ to $\mathcal{C}^{(n)}$. However, the mapping of a given vector $\bm{b}_j$ to quantizers $\mathcal{C}^{(n)}$ remains as  $f^{(n)}(\bm{b}_j)$. At the next iteration, during the vector clustering step, $f^{(n+1)}(\bm{b}_j)$ finds new mapping of $\bm{b}_j$ to updated codebooks $\mathcal{C}^{(n)}$ such that the quantization MSE over the candidate codebooks is minimized. Therefore, we obtain the following result for $\bm{b}_j$:
\begin{align*}
\frac{1}{L_b}\lVert \bm{b}_j - C_{f^{(n+1)}(\bm{b}_j)}^{(n)}(\bm{b}_j)\rVert^2_2 \le \frac{1}{L_b}\lVert \bm{b}_j - C_{f^{(n)}(\bm{b}_j)}^{(n)}(\bm{b}_j)\rVert^2_2
\end{align*}

That is, quantizing $\bm{b}_j$ at the end of the block clustering step of iteration $n+1$ results in lower quantization MSE compared to quantizing at the end of iteration $n$. Since this is true for all $\bm{b} \in \bm{X}$, we assert the following:
\begin{equation}
\begin{split}
\label{eq:mse_ineq_1}
    \tilde{J}^{(n+1)} &= \frac{1}{N_c} \sum_{i=1}^{N_c} \frac{1}{|\mathcal{B}_{i}^{(n+1)}|}\sum_{\bm{b} \in \mathcal{B}_{i}^{(n+1)}} \frac{1}{L_b}\lVert \bm{b} - C_i^{(n)}(b)\rVert^2_2 \le J^{(n)}
\end{split}
\end{equation}
where $\tilde{J}^{(n+1)}$ is the the quantization MSE after the vector clustering step at iteration $n+1$.

Next, during the codebook update step (\ref{eq:quantizers_update}) at iteration $n+1$, the per-cluster codebooks $\mathcal{C}^{(n)}$ are updated to $\mathcal{C}^{(n+1)}$ by invoking the Lloyd-Max algorithm \citep{Lloyd}. We know that for any given value distribution, the Lloyd-Max algorithm minimizes the quantization MSE. Therefore, for a given vector cluster $\mathcal{B}_i$ we obtain the following result:

\begin{equation}
    \frac{1}{|\mathcal{B}_{i}^{(n+1)}|}\sum_{\bm{b} \in \mathcal{B}_{i}^{(n+1)}} \frac{1}{L_b}\lVert \bm{b}- C_i^{(n+1)}(\bm{b})\rVert^2_2 \le \frac{1}{|\mathcal{B}_{i}^{(n+1)}|}\sum_{\bm{b} \in \mathcal{B}_{i}^{(n+1)}} \frac{1}{L_b}\lVert \bm{b}- C_i^{(n)}(\bm{b})\rVert^2_2
\end{equation}

The above equation states that quantizing the given block cluster $\mathcal{B}_i$ after updating the associated codebook from $C_i^{(n)}$ to $C_i^{(n+1)}$ results in lower quantization MSE. Since this is true for all the block clusters, we derive the following result: 
\begin{equation}
\begin{split}
\label{eq:mse_ineq_2}
     J^{(n+1)} &= \frac{1}{N_c} \sum_{i=1}^{N_c} \frac{1}{|\mathcal{B}_{i}^{(n+1)}|}\sum_{\bm{b} \in \mathcal{B}_{i}^{(n+1)}} \frac{1}{L_b}\lVert \bm{b}- C_i^{(n+1)}(\bm{b})\rVert^2_2  \le \tilde{J}^{(n+1)}   
\end{split}
\end{equation}

Following (\ref{eq:mse_ineq_1}) and (\ref{eq:mse_ineq_2}), we find that the quantization MSE is non-increasing for each iteration, that is, $J^{(1)} \ge J^{(2)} \ge J^{(3)} \ge \ldots \ge J^{(M)}$ where $M$ is the maximum number of iterations. 
%Therefore, we can say that if the algorithm converges, then it must be that it has converged to a local minimum. 
\hfill $\blacksquare$


\begin{figure}
    \begin{center}
    \includegraphics[width=0.5\textwidth]{sections//figures/mse_vs_iter.pdf}
    \end{center}
    \caption{\small NMSE vs iterations during LO-BCQ compared to other block quantization proposals}
    \label{fig:nmse_vs_iter}
\end{figure}

Figure \ref{fig:nmse_vs_iter} shows the empirical convergence of LO-BCQ across several block lengths and number of codebooks. Also, the MSE achieved by LO-BCQ is compared to baselines such as MXFP and VSQ. As shown, LO-BCQ converges to a lower MSE than the baselines. Further, we achieve better convergence for larger number of codebooks ($N_c$) and for a smaller block length ($L_b$), both of which increase the bitwidth of BCQ (see Eq \ref{eq:bitwidth_bcq}).


\subsection{Additional Accuracy Results}
%Table \ref{tab:lobcq_config} lists the various LOBCQ configurations and their corresponding bitwidths.
\begin{table}
\setlength{\tabcolsep}{4.75pt}
\begin{center}
\caption{\label{tab:lobcq_config} Various LO-BCQ configurations and their bitwidths.}
\begin{tabular}{|c||c|c|c|c||c|c||c|} 
\hline
 & \multicolumn{4}{|c||}{$L_b=8$} & \multicolumn{2}{|c||}{$L_b=4$} & $L_b=2$ \\
 \hline
 \backslashbox{$L_A$\kern-1em}{\kern-1em$N_c$} & 2 & 4 & 8 & 16 & 2 & 4 & 2 \\
 \hline
 64 & 4.25 & 4.375 & 4.5 & 4.625 & 4.375 & 4.625 & 4.625\\
 \hline
 32 & 4.375 & 4.5 & 4.625& 4.75 & 4.5 & 4.75 & 4.75 \\
 \hline
 16 & 4.625 & 4.75& 4.875 & 5 & 4.75 & 5 & 5 \\
 \hline
\end{tabular}
\end{center}
\end{table}

%\subsection{Perplexity achieved by various LO-BCQ configurations on Wikitext-103 dataset}

\begin{table} \centering
\begin{tabular}{|c||c|c|c|c||c|c||c|} 
\hline
 $L_b \rightarrow$& \multicolumn{4}{c||}{8} & \multicolumn{2}{c||}{4} & 2\\
 \hline
 \backslashbox{$L_A$\kern-1em}{\kern-1em$N_c$} & 2 & 4 & 8 & 16 & 2 & 4 & 2  \\
 %$N_c \rightarrow$ & 2 & 4 & 8 & 16 & 2 & 4 & 2 \\
 \hline
 \hline
 \multicolumn{8}{c}{GPT3-1.3B (FP32 PPL = 9.98)} \\ 
 \hline
 \hline
 64 & 10.40 & 10.23 & 10.17 & 10.15 &  10.28 & 10.18 & 10.19 \\
 \hline
 32 & 10.25 & 10.20 & 10.15 & 10.12 &  10.23 & 10.17 & 10.17 \\
 \hline
 16 & 10.22 & 10.16 & 10.10 & 10.09 &  10.21 & 10.14 & 10.16 \\
 \hline
  \hline
 \multicolumn{8}{c}{GPT3-8B (FP32 PPL = 7.38)} \\ 
 \hline
 \hline
 64 & 7.61 & 7.52 & 7.48 &  7.47 &  7.55 &  7.49 & 7.50 \\
 \hline
 32 & 7.52 & 7.50 & 7.46 &  7.45 &  7.52 &  7.48 & 7.48  \\
 \hline
 16 & 7.51 & 7.48 & 7.44 &  7.44 &  7.51 &  7.49 & 7.47  \\
 \hline
\end{tabular}
\caption{\label{tab:ppl_gpt3_abalation} Wikitext-103 perplexity across GPT3-1.3B and 8B models.}
\end{table}

\begin{table} \centering
\begin{tabular}{|c||c|c|c|c||} 
\hline
 $L_b \rightarrow$& \multicolumn{4}{c||}{8}\\
 \hline
 \backslashbox{$L_A$\kern-1em}{\kern-1em$N_c$} & 2 & 4 & 8 & 16 \\
 %$N_c \rightarrow$ & 2 & 4 & 8 & 16 & 2 & 4 & 2 \\
 \hline
 \hline
 \multicolumn{5}{|c|}{Llama2-7B (FP32 PPL = 5.06)} \\ 
 \hline
 \hline
 64 & 5.31 & 5.26 & 5.19 & 5.18  \\
 \hline
 32 & 5.23 & 5.25 & 5.18 & 5.15  \\
 \hline
 16 & 5.23 & 5.19 & 5.16 & 5.14  \\
 \hline
 \multicolumn{5}{|c|}{Nemotron4-15B (FP32 PPL = 5.87)} \\ 
 \hline
 \hline
 64  & 6.3 & 6.20 & 6.13 & 6.08  \\
 \hline
 32  & 6.24 & 6.12 & 6.07 & 6.03  \\
 \hline
 16  & 6.12 & 6.14 & 6.04 & 6.02  \\
 \hline
 \multicolumn{5}{|c|}{Nemotron4-340B (FP32 PPL = 3.48)} \\ 
 \hline
 \hline
 64 & 3.67 & 3.62 & 3.60 & 3.59 \\
 \hline
 32 & 3.63 & 3.61 & 3.59 & 3.56 \\
 \hline
 16 & 3.61 & 3.58 & 3.57 & 3.55 \\
 \hline
\end{tabular}
\caption{\label{tab:ppl_llama7B_nemo15B} Wikitext-103 perplexity compared to FP32 baseline in Llama2-7B and Nemotron4-15B, 340B models}
\end{table}

%\subsection{Perplexity achieved by various LO-BCQ configurations on MMLU dataset}


\begin{table} \centering
\begin{tabular}{|c||c|c|c|c||c|c|c|c|} 
\hline
 $L_b \rightarrow$& \multicolumn{4}{c||}{8} & \multicolumn{4}{c||}{8}\\
 \hline
 \backslashbox{$L_A$\kern-1em}{\kern-1em$N_c$} & 2 & 4 & 8 & 16 & 2 & 4 & 8 & 16  \\
 %$N_c \rightarrow$ & 2 & 4 & 8 & 16 & 2 & 4 & 2 \\
 \hline
 \hline
 \multicolumn{5}{|c|}{Llama2-7B (FP32 Accuracy = 45.8\%)} & \multicolumn{4}{|c|}{Llama2-70B (FP32 Accuracy = 69.12\%)} \\ 
 \hline
 \hline
 64 & 43.9 & 43.4 & 43.9 & 44.9 & 68.07 & 68.27 & 68.17 & 68.75 \\
 \hline
 32 & 44.5 & 43.8 & 44.9 & 44.5 & 68.37 & 68.51 & 68.35 & 68.27  \\
 \hline
 16 & 43.9 & 42.7 & 44.9 & 45 & 68.12 & 68.77 & 68.31 & 68.59  \\
 \hline
 \hline
 \multicolumn{5}{|c|}{GPT3-22B (FP32 Accuracy = 38.75\%)} & \multicolumn{4}{|c|}{Nemotron4-15B (FP32 Accuracy = 64.3\%)} \\ 
 \hline
 \hline
 64 & 36.71 & 38.85 & 38.13 & 38.92 & 63.17 & 62.36 & 63.72 & 64.09 \\
 \hline
 32 & 37.95 & 38.69 & 39.45 & 38.34 & 64.05 & 62.30 & 63.8 & 64.33  \\
 \hline
 16 & 38.88 & 38.80 & 38.31 & 38.92 & 63.22 & 63.51 & 63.93 & 64.43  \\
 \hline
\end{tabular}
\caption{\label{tab:mmlu_abalation} Accuracy on MMLU dataset across GPT3-22B, Llama2-7B, 70B and Nemotron4-15B models.}
\end{table}


%\subsection{Perplexity achieved by various LO-BCQ configurations on LM evaluation harness}

\begin{table} \centering
\begin{tabular}{|c||c|c|c|c||c|c|c|c|} 
\hline
 $L_b \rightarrow$& \multicolumn{4}{c||}{8} & \multicolumn{4}{c||}{8}\\
 \hline
 \backslashbox{$L_A$\kern-1em}{\kern-1em$N_c$} & 2 & 4 & 8 & 16 & 2 & 4 & 8 & 16  \\
 %$N_c \rightarrow$ & 2 & 4 & 8 & 16 & 2 & 4 & 2 \\
 \hline
 \hline
 \multicolumn{5}{|c|}{Race (FP32 Accuracy = 37.51\%)} & \multicolumn{4}{|c|}{Boolq (FP32 Accuracy = 64.62\%)} \\ 
 \hline
 \hline
 64 & 36.94 & 37.13 & 36.27 & 37.13 & 63.73 & 62.26 & 63.49 & 63.36 \\
 \hline
 32 & 37.03 & 36.36 & 36.08 & 37.03 & 62.54 & 63.51 & 63.49 & 63.55  \\
 \hline
 16 & 37.03 & 37.03 & 36.46 & 37.03 & 61.1 & 63.79 & 63.58 & 63.33  \\
 \hline
 \hline
 \multicolumn{5}{|c|}{Winogrande (FP32 Accuracy = 58.01\%)} & \multicolumn{4}{|c|}{Piqa (FP32 Accuracy = 74.21\%)} \\ 
 \hline
 \hline
 64 & 58.17 & 57.22 & 57.85 & 58.33 & 73.01 & 73.07 & 73.07 & 72.80 \\
 \hline
 32 & 59.12 & 58.09 & 57.85 & 58.41 & 73.01 & 73.94 & 72.74 & 73.18  \\
 \hline
 16 & 57.93 & 58.88 & 57.93 & 58.56 & 73.94 & 72.80 & 73.01 & 73.94  \\
 \hline
\end{tabular}
\caption{\label{tab:mmlu_abalation} Accuracy on LM evaluation harness tasks on GPT3-1.3B model.}
\end{table}

\begin{table} \centering
\begin{tabular}{|c||c|c|c|c||c|c|c|c|} 
\hline
 $L_b \rightarrow$& \multicolumn{4}{c||}{8} & \multicolumn{4}{c||}{8}\\
 \hline
 \backslashbox{$L_A$\kern-1em}{\kern-1em$N_c$} & 2 & 4 & 8 & 16 & 2 & 4 & 8 & 16  \\
 %$N_c \rightarrow$ & 2 & 4 & 8 & 16 & 2 & 4 & 2 \\
 \hline
 \hline
 \multicolumn{5}{|c|}{Race (FP32 Accuracy = 41.34\%)} & \multicolumn{4}{|c|}{Boolq (FP32 Accuracy = 68.32\%)} \\ 
 \hline
 \hline
 64 & 40.48 & 40.10 & 39.43 & 39.90 & 69.20 & 68.41 & 69.45 & 68.56 \\
 \hline
 32 & 39.52 & 39.52 & 40.77 & 39.62 & 68.32 & 67.43 & 68.17 & 69.30  \\
 \hline
 16 & 39.81 & 39.71 & 39.90 & 40.38 & 68.10 & 66.33 & 69.51 & 69.42  \\
 \hline
 \hline
 \multicolumn{5}{|c|}{Winogrande (FP32 Accuracy = 67.88\%)} & \multicolumn{4}{|c|}{Piqa (FP32 Accuracy = 78.78\%)} \\ 
 \hline
 \hline
 64 & 66.85 & 66.61 & 67.72 & 67.88 & 77.31 & 77.42 & 77.75 & 77.64 \\
 \hline
 32 & 67.25 & 67.72 & 67.72 & 67.00 & 77.31 & 77.04 & 77.80 & 77.37  \\
 \hline
 16 & 68.11 & 68.90 & 67.88 & 67.48 & 77.37 & 78.13 & 78.13 & 77.69  \\
 \hline
\end{tabular}
\caption{\label{tab:mmlu_abalation} Accuracy on LM evaluation harness tasks on GPT3-8B model.}
\end{table}

\begin{table} \centering
\begin{tabular}{|c||c|c|c|c||c|c|c|c|} 
\hline
 $L_b \rightarrow$& \multicolumn{4}{c||}{8} & \multicolumn{4}{c||}{8}\\
 \hline
 \backslashbox{$L_A$\kern-1em}{\kern-1em$N_c$} & 2 & 4 & 8 & 16 & 2 & 4 & 8 & 16  \\
 %$N_c \rightarrow$ & 2 & 4 & 8 & 16 & 2 & 4 & 2 \\
 \hline
 \hline
 \multicolumn{5}{|c|}{Race (FP32 Accuracy = 40.67\%)} & \multicolumn{4}{|c|}{Boolq (FP32 Accuracy = 76.54\%)} \\ 
 \hline
 \hline
 64 & 40.48 & 40.10 & 39.43 & 39.90 & 75.41 & 75.11 & 77.09 & 75.66 \\
 \hline
 32 & 39.52 & 39.52 & 40.77 & 39.62 & 76.02 & 76.02 & 75.96 & 75.35  \\
 \hline
 16 & 39.81 & 39.71 & 39.90 & 40.38 & 75.05 & 73.82 & 75.72 & 76.09  \\
 \hline
 \hline
 \multicolumn{5}{|c|}{Winogrande (FP32 Accuracy = 70.64\%)} & \multicolumn{4}{|c|}{Piqa (FP32 Accuracy = 79.16\%)} \\ 
 \hline
 \hline
 64 & 69.14 & 70.17 & 70.17 & 70.56 & 78.24 & 79.00 & 78.62 & 78.73 \\
 \hline
 32 & 70.96 & 69.69 & 71.27 & 69.30 & 78.56 & 79.49 & 79.16 & 78.89  \\
 \hline
 16 & 71.03 & 69.53 & 69.69 & 70.40 & 78.13 & 79.16 & 79.00 & 79.00  \\
 \hline
\end{tabular}
\caption{\label{tab:mmlu_abalation} Accuracy on LM evaluation harness tasks on GPT3-22B model.}
\end{table}

\begin{table} \centering
\begin{tabular}{|c||c|c|c|c||c|c|c|c|} 
\hline
 $L_b \rightarrow$& \multicolumn{4}{c||}{8} & \multicolumn{4}{c||}{8}\\
 \hline
 \backslashbox{$L_A$\kern-1em}{\kern-1em$N_c$} & 2 & 4 & 8 & 16 & 2 & 4 & 8 & 16  \\
 %$N_c \rightarrow$ & 2 & 4 & 8 & 16 & 2 & 4 & 2 \\
 \hline
 \hline
 \multicolumn{5}{|c|}{Race (FP32 Accuracy = 44.4\%)} & \multicolumn{4}{|c|}{Boolq (FP32 Accuracy = 79.29\%)} \\ 
 \hline
 \hline
 64 & 42.49 & 42.51 & 42.58 & 43.45 & 77.58 & 77.37 & 77.43 & 78.1 \\
 \hline
 32 & 43.35 & 42.49 & 43.64 & 43.73 & 77.86 & 75.32 & 77.28 & 77.86  \\
 \hline
 16 & 44.21 & 44.21 & 43.64 & 42.97 & 78.65 & 77 & 76.94 & 77.98  \\
 \hline
 \hline
 \multicolumn{5}{|c|}{Winogrande (FP32 Accuracy = 69.38\%)} & \multicolumn{4}{|c|}{Piqa (FP32 Accuracy = 78.07\%)} \\ 
 \hline
 \hline
 64 & 68.9 & 68.43 & 69.77 & 68.19 & 77.09 & 76.82 & 77.09 & 77.86 \\
 \hline
 32 & 69.38 & 68.51 & 68.82 & 68.90 & 78.07 & 76.71 & 78.07 & 77.86  \\
 \hline
 16 & 69.53 & 67.09 & 69.38 & 68.90 & 77.37 & 77.8 & 77.91 & 77.69  \\
 \hline
\end{tabular}
\caption{\label{tab:mmlu_abalation} Accuracy on LM evaluation harness tasks on Llama2-7B model.}
\end{table}

\begin{table} \centering
\begin{tabular}{|c||c|c|c|c||c|c|c|c|} 
\hline
 $L_b \rightarrow$& \multicolumn{4}{c||}{8} & \multicolumn{4}{c||}{8}\\
 \hline
 \backslashbox{$L_A$\kern-1em}{\kern-1em$N_c$} & 2 & 4 & 8 & 16 & 2 & 4 & 8 & 16  \\
 %$N_c \rightarrow$ & 2 & 4 & 8 & 16 & 2 & 4 & 2 \\
 \hline
 \hline
 \multicolumn{5}{|c|}{Race (FP32 Accuracy = 48.8\%)} & \multicolumn{4}{|c|}{Boolq (FP32 Accuracy = 85.23\%)} \\ 
 \hline
 \hline
 64 & 49.00 & 49.00 & 49.28 & 48.71 & 82.82 & 84.28 & 84.03 & 84.25 \\
 \hline
 32 & 49.57 & 48.52 & 48.33 & 49.28 & 83.85 & 84.46 & 84.31 & 84.93  \\
 \hline
 16 & 49.85 & 49.09 & 49.28 & 48.99 & 85.11 & 84.46 & 84.61 & 83.94  \\
 \hline
 \hline
 \multicolumn{5}{|c|}{Winogrande (FP32 Accuracy = 79.95\%)} & \multicolumn{4}{|c|}{Piqa (FP32 Accuracy = 81.56\%)} \\ 
 \hline
 \hline
 64 & 78.77 & 78.45 & 78.37 & 79.16 & 81.45 & 80.69 & 81.45 & 81.5 \\
 \hline
 32 & 78.45 & 79.01 & 78.69 & 80.66 & 81.56 & 80.58 & 81.18 & 81.34  \\
 \hline
 16 & 79.95 & 79.56 & 79.79 & 79.72 & 81.28 & 81.66 & 81.28 & 80.96  \\
 \hline
\end{tabular}
\caption{\label{tab:mmlu_abalation} Accuracy on LM evaluation harness tasks on Llama2-70B model.}
\end{table}

%\section{MSE Studies}
%\textcolor{red}{TODO}


\subsection{Number Formats and Quantization Method}
\label{subsec:numFormats_quantMethod}
\subsubsection{Integer Format}
An $n$-bit signed integer (INT) is typically represented with a 2s-complement format \citep{yao2022zeroquant,xiao2023smoothquant,dai2021vsq}, where the most significant bit denotes the sign.

\subsubsection{Floating Point Format}
An $n$-bit signed floating point (FP) number $x$ comprises of a 1-bit sign ($x_{\mathrm{sign}}$), $B_m$-bit mantissa ($x_{\mathrm{mant}}$) and $B_e$-bit exponent ($x_{\mathrm{exp}}$) such that $B_m+B_e=n-1$. The associated constant exponent bias ($E_{\mathrm{bias}}$) is computed as $(2^{{B_e}-1}-1)$. We denote this format as $E_{B_e}M_{B_m}$.  

\subsubsection{Quantization Scheme}
\label{subsec:quant_method}
A quantization scheme dictates how a given unquantized tensor is converted to its quantized representation. We consider FP formats for the purpose of illustration. Given an unquantized tensor $\bm{X}$ and an FP format $E_{B_e}M_{B_m}$, we first, we compute the quantization scale factor $s_X$ that maps the maximum absolute value of $\bm{X}$ to the maximum quantization level of the $E_{B_e}M_{B_m}$ format as follows:
\begin{align}
\label{eq:sf}
    s_X = \frac{\mathrm{max}(|\bm{X}|)}{\mathrm{max}(E_{B_e}M_{B_m})}
\end{align}
In the above equation, $|\cdot|$ denotes the absolute value function.

Next, we scale $\bm{X}$ by $s_X$ and quantize it to $\hat{\bm{X}}$ by rounding it to the nearest quantization level of $E_{B_e}M_{B_m}$ as:

\begin{align}
\label{eq:tensor_quant}
    \hat{\bm{X}} = \text{round-to-nearest}\left(\frac{\bm{X}}{s_X}, E_{B_e}M_{B_m}\right)
\end{align}

We perform dynamic max-scaled quantization \citep{wu2020integer}, where the scale factor $s$ for activations is dynamically computed during runtime.

\subsection{Vector Scaled Quantization}
\begin{wrapfigure}{r}{0.35\linewidth}
  \centering
  \includegraphics[width=\linewidth]{sections/figures/vsquant.jpg}
  \caption{\small Vectorwise decomposition for per-vector scaled quantization (VSQ \citep{dai2021vsq}).}
  \label{fig:vsquant}
\end{wrapfigure}
During VSQ \citep{dai2021vsq}, the operand tensors are decomposed into 1D vectors in a hardware friendly manner as shown in Figure \ref{fig:vsquant}. Since the decomposed tensors are used as operands in matrix multiplications during inference, it is beneficial to perform this decomposition along the reduction dimension of the multiplication. The vectorwise quantization is performed similar to tensorwise quantization described in Equations \ref{eq:sf} and \ref{eq:tensor_quant}, where a scale factor $s_v$ is required for each vector $\bm{v}$ that maps the maximum absolute value of that vector to the maximum quantization level. While smaller vector lengths can lead to larger accuracy gains, the associated memory and computational overheads due to the per-vector scale factors increases. To alleviate these overheads, VSQ \citep{dai2021vsq} proposed a second level quantization of the per-vector scale factors to unsigned integers, while MX \citep{rouhani2023shared} quantizes them to integer powers of 2 (denoted as $2^{INT}$).

\subsubsection{MX Format}
The MX format proposed in \citep{rouhani2023microscaling} introduces the concept of sub-block shifting. For every two scalar elements of $b$-bits each, there is a shared exponent bit. The value of this exponent bit is determined through an empirical analysis that targets minimizing quantization MSE. We note that the FP format $E_{1}M_{b}$ is strictly better than MX from an accuracy perspective since it allocates a dedicated exponent bit to each scalar as opposed to sharing it across two scalars. Therefore, we conservatively bound the accuracy of a $b+2$-bit signed MX format with that of a $E_{1}M_{b}$ format in our comparisons. For instance, we use E1M2 format as a proxy for MX4.

\begin{figure}
    \centering
    \includegraphics[width=1\linewidth]{sections//figures/BlockFormats.pdf}
    \caption{\small Comparing LO-BCQ to MX format.}
    \label{fig:block_formats}
\end{figure}

Figure \ref{fig:block_formats} compares our $4$-bit LO-BCQ block format to MX \citep{rouhani2023microscaling}. As shown, both LO-BCQ and MX decompose a given operand tensor into block arrays and each block array into blocks. Similar to MX, we find that per-block quantization ($L_b < L_A$) leads to better accuracy due to increased flexibility. While MX achieves this through per-block $1$-bit micro-scales, we associate a dedicated codebook to each block through a per-block codebook selector. Further, MX quantizes the per-block array scale-factor to E8M0 format without per-tensor scaling. In contrast during LO-BCQ, we find that per-tensor scaling combined with quantization of per-block array scale-factor to E4M3 format results in superior inference accuracy across models. 



\end{document}
