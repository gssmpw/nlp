\section{Download, build and run testbench}
\label{appendix:download-and-compilation}

All of our code is hosted in a public git repository 
at \url{https://gitlab.lrz.de/hpcsoftware/Peano}. 
Our benchmark scripts are merged into the repository's
main, i.e.~all results can be reproduced with the main branch. 
Yet, to use the exact same code version as used for this paper, please switch to 
the \linebreak \texttt{particles-noh2D-vectorisation-2} branch.


\begin{algorithm}[htb]
    \begin{algorithmic}[1]
      \State \verb|git clone https://gitlab.lrz.de/hpcsoftware/Peano|
      \State \verb|cd Peano|
      \State \verb|libtoolize; aclocal; autoconf; autoheader|
      \State \verb|cp src/config.h.in .; automake --add-missing|
      \State \begin{verbatim}
./configure CXX=... CC=... CXXFLAGS=... LDFLAGS=... 
   --enable-loadbalancing 
   --enable-particles 
   --enable-swift 
   --with-multithreading=omp
\end{verbatim}
      \State \verb|make|
    \end{algorithmic}
  \caption{
    Cloning the repository and setting up the autotools environment. 
    \label{algorithm:appendix:download-and-compile:autotools}
    }
\end{algorithm}


The test benchmarks in the present paper are shipped as part of Peano's
benchmark suite, which implies that Peano has to be configured and built first.
The code base provides CMake and autotools (Alg.~\ref{algorithm:appendix:download-and-compile:autotools})
bindings.
Depending on the target platform, different compiler options have to be used.
Once configured, the build (\texttt{make}) yields a set of static libraries
providing the back-end of our benchmarks.

The actual benchmark can be found in Peano's subdirectory \linebreak
\texttt{benchmarks/swift2/hydro/kernel-throughput}.
Within this directory, call \linebreak \texttt{./benchmark.sh} to run the CPU-only benchamrks, and \texttt{./benchmark-offload.sh} to run the accelerator offloading dependent bechmarks.

Executing any of the benchmarking scripts produces a series of log files in the working directory are the raw data plotted and presented in this paper.

To reproduce the plots presented in this paper, call \linebreak \texttt{python3 ./print-kernel-throughput.py *.log} and \linebreak \texttt{python3 ./print-kernel-speedups.py ./} in the working directory where the log files are.
