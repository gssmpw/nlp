\section{Results}
\label{section:results}

We assess the impact of our compiler prototype on two architectures.
Our first system is an Intel Xeon Platinum 8480+ (Sapphire Rapids) testbed. 
It features $2 \times 56$ cores over $2 \times 1$ NUMA domains spread over
two sockets, hosts
an L2 cache of 2,048 KByte per core and a shared L3 cache with 105 MByte per
socket.
Our second testbed is an Nvidia GH200 (Grace Hopper) system.
It features 72 cores in a single socket configuration.
Each core has an L2 cache of 1,024 KByte, and a shared L3 cache with 114 MByte for the entire
socket. The system contains the Nvidia H200 GPGPU chip boasting 96GB HBM3 memory at 4TB/s.

\begin{table}[htb]
  \caption{
    Overview of benchmarked SPH kernels. We report on the total contribution to the runtime, the cell-local compute complexity, and the size of the active sets.
    \label{table:results:kernel-overview}
  }
  \begin{center}
    \begin{tabular}{l|rlrrrr}
Name & Runtime & Compl. & \phantom{xx} $|\mathbb{A}_{\text{in}}|$ \phantom{xx} & \phantom{xx} $|\mathbb{A}_{\text{in}}|_{\text{byte}}$ \phantom{xx} & \phantom{xx} $|\mathbb{A}_{\text{out}}|$ \phantom{xx} & \phantom{xx} $|\mathbb{A}_{\text{out}}|_{\text{byte}}$ \\
  \hline
  Density &
  	39.0\% & 
  	$\mathcal{O}(N^2)$ &
  	9 & 88 &
  	6 & 48 \\
  Drift &
  	2.2\% & 
  	$\mathcal{O}(N)$ &
  	3 & 20 &
  	2 & 12 \\
  Force &
  	58.5\% & 
  	$\mathcal{O}(N^2)$ &
  	13 & 128 &
  	4 & 40 \\
  First kick &
  	0.1\% & 
  	$\mathcal{O}(N)$ &
  	4 & 48 &
  	3 & 32 \\
  Second kick &
  	0.2\% & 
  	$\mathcal{O}(N)$ &
  	11 & 112 &
  	8 & 80 \\
  \hline
\end{tabular}
  \end{center}
\end{table}


We benchmark the core SPH algorithm and split up all measurements into data for the individual compute kernels that dominate the runtime (Table~\ref{table:results:kernel-overview}).
Meshing overhead, sorting, I/O and other effects are excluded.
In line with previous discussions, our SPH code does not refine the underlying spacetree down to the finest admissible level~(cmp.~related design decisions in \cite{Schaller:2024:Swift}).
Instead, we impose a cut-off:
If the number of particles per cell ($ppc$) underruns a certain threshold, the underlying mesh is not decomposed further.
This way, codes can balance between meshing overhead and algorithmic complexity.
In our result, we benchmark setups with average $ppc \in \{64,128,256,512,1024\}$.
Further to that, we assess configurations where the particles are scattered over the heap in main memory against configurations where the particles are stored in chunks of continuous memory and hence allow for coalesced memory access. 



Throughout the presentation, all data are normalised as cost per particle update, which might comprise interactions with all cell-local neighbours (density, force) or simply denote the runtime divided by the loop count.
All data are always employing all cores of the compute node, i.e.~the underlying code base is parallelised.
The data are hence node benchmarks or employ the whole GPU respectively.


\subsection{Baseline code}

%
% What we do next and why
%
We start our studies with a simple assessment of the baseline kernel realisations without our compiler-guided transformations.
It is important to obtain a sound performance baseline for fair comparisons.
Besides this, it is not clear in the context of data format transformations if certain realisation patterns are particularly advantageous for the data re-organisation or challenge them.
Therefore, we distinguish various different kernel variants:
\begin{itemize}
  \item In the scattered generic kernel, we stick to the vanilla blueprint of Algorithm~\ref{section:demonstrator} and let the particles be scattered all accross the heap. Each cell hosts a sequence of pointers to addresses all over the place.
  \item A second variant sticks to the generic loop over C++ containers yet ensures that the particles of one cell are stored continuously in memory. However, this knowledge is not exploited explicitly.
  \item Finally, we use this knowledge about continuous chunks and split up each loop into a loop over chunks. Per chunk, we then employ a plain parallel for with pointer arithmetics, i.e.~we expose the fact that coalesced memory access is possible explicitly within the source code.
  \item For the $\mathcal{O}(N^2)$ kernels, we distinguish two kernel variants of the latter approach: In the variant labelled \texttt{local-active}, we first run over the local particles in the outer loop before we traverse the active particles. In the variant with the label \texttt{active-local}, we permute these two outer loops.
\end{itemize}




%
% Predicate is connected through a locical or, i.e. true disables the actual distance check
% Symmetry checks are connected via a locical and, i.e. false kicks them out
%
\begin{figure}[htb]
  \begin{center}
    \includegraphics[width=1.00\textwidth]{experiments/kernel-throughput/gi001/out-x64-soafalse-cell0064-predfalse-symtrue-legend.pdf}
    \includegraphics[width=0.49\textwidth]{experiments/kernel-throughput/gi001/out-x64-soafalse-cell0064-predfalse-symtrue.pdf}
    \includegraphics[width=0.49\textwidth]{experiments/kernel-throughput/gi001/out-x64-soafalse-cell1024-predfalse-symtrue.pdf}     
    \includegraphics[width=0.49\textwidth]{experiments/kernel-throughput/gh003/out-aarch64-soafalse-cell0064-predfalse-symtrue.pdf}
    \includegraphics[width=0.49\textwidth]{experiments/kernel-throughput/gh003/out-aarch64-soafalse-cell1024-predfalse-symtrue.pdf}     
  \end{center}
  \caption{
   	Cost per particle update for various kernels on the Intel (top) and Grace (bottom) CPU node without any data transformations. 
   	We distinguish $ppc=64$ (left) from $ppc=1024$ (right). The vertical line denotes the sum of all L2 caches.
   \label{figure:results:baseline}
  }
\end{figure}


%
% What we see
% 
For all setups, the throughput improves with increasing number of particles (Figure~\ref{figure:results:baseline}) unless we have a very small total particle count.
The different kernel realisation variants make no significant difference, while the kernels with local $\mathcal{O}(N^2)$ complexity are significantly more expensive than their linear counterparts.
Notably, the fact that data are available as continuous chunks makes no major difference no matter if we try to exploit this fact or not.
The ARM CPU yields slightly noisier data than the x86 counterpart.
A small workload per kernel improves the throughput of kernels with quadratic complexity, but larger $ppc$ counts favour the linear kernels.
Falling out of the L2 caches has no performance impact and hence cannot explain any runtime behaviour.


%
% Interpretation
%
The plateauing for larger particle counts suggests that there is some compute management overhead which is amortised as the overall workload increases.
We pick the minimal total particle count such that all CPU cores are always busy, i.e.~there are always enough kernel invocations to employ all threads.
However, workload imbalances between the threads can better be smoothed out with a larger total set of compute kernels.
The overhead stems merely from a load distribution argument.
The local $\mathcal{O}(N^2)$ character of the force and density calculation explains why they benefit from small $ppc$ (we effectively have a complexity of $\mathcal{O}(ppc^2)$), while the kick and drifts are streaming kernels and hence do benefit from larger loops which pipe data through the system.
As the kernels with quadratic local complexity dominate the runtime (Table~\ref{table:results:kernel-overview}), the simulation code benefits from small $ppc$.
We assume that noise in the measurements and notably cost peaks stem from situations where we fall out of caches, or setups where a lot of the guard predicates that determine if a particle is to be updated return false.


\begin{observation}
 A better utilisation of the hardware is inferior to a reduction of the local compute complexity of the kernels as long as there are enough of these kernels in flight. 
\end{observation}

%
% Implication
%

\noindent
The classic HPC metric MFlop/s suggests to use reasonably large $ppc$ to balance for the throughput between the different phases.
Once we take the characteristic distribution of the compute phases into account and weight the improvements accordingly, it becomes clear that, from a science per time point of view, we have to work with as small $ppc$ as possible, which implies using aggressive AMR in our case;
even though this challenges the kernel efficiency.


The lack of impact of continuous data seems to be surprising, but is reasonable once we take into account that our baseline data structure is AoS. 
The particles might be continuous in memory, but each kernel picks only few data members $\mathbb{A}_{\text{in}}$ to read and very few $\mathbb{A}_{\text{out}}$ to write (Table~\ref{table:results:kernel-overview}).
Consequently, the data access pattern is scattered even though the underlying data are continuous.
There is no coalesced memory access.


\subsection{Data transformation impact}


%
% What we do next and why
%
We next enable the data compiler-guided transformations and compare the outcomes to the CPU baseline.
This allows us to quantify the transformation overhead.

\begin{figure}[htb]
  \begin{center}
    \includegraphics[width=1.00\textwidth]{experiments/kernel-throughput/gi001/out-x64-soatrue-cell0064-predfalse-symtrue-legend.pdf}
    \includegraphics[width=0.49\textwidth]{experiments/kernel-throughput/gi001/out-x64-soatrue-cell0064-predfalse-symtrue.pdf}
    \includegraphics[width=0.49\textwidth]{experiments/kernel-throughput/gi001/out-x64-soatrue-cell1024-predfalse-symtrue.pdf}     
    \includegraphics[width=0.49\textwidth]{experiments/kernel-throughput/gh003/out-aarch64-soatrue-cell0064-predfalse-symtrue.pdf}
    \includegraphics[width=0.49\textwidth]{experiments/kernel-throughput/gh003/out-aarch64-soatrue-cell1024-predfalse-symtrue.pdf}     
  \end{center}
  \caption{
    Repetition of experiments from Figure~\ref{figure:results:baseline} with the compiler transformations enabled.
    Intel (top) and Grace (bottom) CPU data with $ppc=64$ (left) from $ppc=1,024$ (right).
   \label{figure:results:overhead}
  }
\end{figure}

%
% What we see
%
The introduction of the compiler transformations smoothes out the measurements and leads to very stable plateaus (Figure~\ref{figure:results:overhead}).
For the kernels with linear compute complexity, they introduce an overhead which can result in a performance degradation of up to an order of magnitude on the x86 chip.
This overhead penalty is less pronounced on the ARM chip.
For the quadratic kernels, we gain an order of magnitude in speed on the Intel platform, while the ARM chip yields even more significant speed improvements.

\begin{figure}[htb]
\centering
 \includegraphics[width=1.00\textwidth]{experiments/kernel-throughput/gh003/plot-2-aarch64-density-legend.pdf}
 \includegraphics[width=0.49\textwidth]{experiments/kernel-throughput/gh003/plot-2-aarch64-density.pdf}
 \includegraphics[width=0.49\textwidth]{experiments/kernel-throughput/gh003/plot-2-aarch64-force.pdf}
 \includegraphics[width=0.49\textwidth]{experiments/kernel-throughput/gi001/plot-2-x64-density.pdf}
 \includegraphics[width=0.49\textwidth]{experiments/kernel-throughput/gi001/plot-2-x64-force.pdf}
%  \includegraphics[width=0.49\textwidth]{experiments/kernel-throughput/gh003/plot-2-aarch64-drift.pdf}
%  \includegraphics[width=0.49\textwidth]{experiments/kernel-throughput/gh003/plot-2-aarch64-kick1.pdf}
%  \includegraphics[width=0.49\textwidth]{experiments/kernel-throughput/gh003/plot-2-aarch64-kick2.pdf}
 \caption{
  Normalised speedups for the density (left) and force (right) kernel on the Nvidia Grace CPU (top) and on Intel (bottom) over multiple different choices of $ppc$.
  \label{figure:transformation-overhead:normalised-speedups}
 }
\end{figure}


Having containers over coalesced memory seems to make no big difference, but trying to exploit this fact makes the throughput deteriorate.
Once we zoom into different kernel variants and plot their relative speedup over the vanilla baseline version without any views (Figure~\ref{figure:transformation-overhead:normalised-speedups}), it becomes clear that the coalesced memory organistion does help.
It just is not relevant for all $ppc$ and kernel choices, i.e.~notably makes a difference for smaller $ppc$.



%
% Interpretation
%
Cache blocking is implicitly triggered by our out-of-place reordering.
We therefore eliminate a lot of memory transfer noise.
As each kernel picks only few attributes from the baseline container, our prologues and epilogues all yield scattered memory access and hence fail to benefit from a continuous arrangement of the data in memory.
However, they do benefit from very large, uninterrupted loops:
On both architectures, the linear kernels benefit from larger $ppc$, as they can stream data through the chip.
The same effect materialises for the quadratic kernel once we enable our transformations:
If we reduce $ppc$ as we split the loop iteration ranges into chunks, we reduce the efficiency of the transformations.


The ARM chip is slow and weak compared to its Intel cousin yet come up with a better balanced memory bandwidth relative to the compute capability.
Hence, the conversion have a smaller impact.
Overall, we obtain impressive speedups exceeding one order of magnitude.


\begin{observation}
 Continuous ordering of the input does not consistently have a major positive impact on the runtime once our annotations are used. In return, the split into tiny subchunks trying to exploit continuous data access is counterproductive, i.e.~the arising loop blocking makes the performance deterioriate. 
\end{observation}



%
% Discussion
%
\noindent
Loop blocking is a classic performance engineering strategy, notably if combined with a continuous arrangement of the underlying memory.
Our data suggests that these empiric lessons learned have to be rethought or even turned around as we introduce compiler-based temporary reordering with views.
Loop ordering is disadvantageous.
Sorting data is disadvantageous.


However, it is not clear if we can uphold this observation for all particle types, i.e.~number of struct members, deeper memory hierarchies or other algorithms.
Notably SPH algorithm flavours with a lot of long-range interactions might behave different if the ``long-range data'' falls out of L3 cache. 

\subsection{Kernel modifications on CPUs}

Our SPH compute kernels are heavy on comparisons yielding branching and even involve some internal atomic operations.
Therefore, we have to assume that they do not vectorise accross multiple particle--particle evaluations.
It is a straightforward code modification to remove the handling of long-range interactions and instead only to evaluate a boolean within the loop that flags ``is there an interaction requiring an atomic access''.
If this flag is set, we can loop over the particle pairs again and evaluate the long-range interaction.
Our assumption is that most kernels do not enter any branch with an atomic.
Eliminating or masking out the neighbour predicate (such as \texttt{DensityPredicate} in Listing~\ref{algorithm:demonstrator:blueprint}) is more difficult.
We however can artificially remove the if statement and study the impact of such a tweak on the performance.


\begin{figure}[H]
  \begin{center}
%    \includegraphics[width=0.49\textwidth]{experiments/kernel-throughput/gi001/out-x64-soatrue-cell0064-predfalse-symtrue-legend.pdf}
    \includegraphics[width=1.00\textwidth]{experiments/kernel-throughput/gi001/out-x64-soatrue-cell0064-predfalse-symfalse-legend.pdf}
    \includegraphics[width=0.49\textwidth]{experiments/kernel-throughput/gi001/out-x64-soatrue-cell0064-predfalse-symfalse.pdf}
    \includegraphics[width=0.49\textwidth]{experiments/kernel-throughput/gi001/out-x64-soatrue-cell0064-predtrue-symfalse.pdf}
  \end{center}
  \caption{
    Measurements on Intel with $ppc=64$ where we mask out all multiscale interactions (left) or multiscale and predicate evaluations (right).
   \label{figure:results:cpu-ppc64}
  }
\end{figure}


%
% What we see
%
The removal of the atomic entries yields a speedup of around a factor of two on Intel systems (Figure~\ref{figure:results:cpu-ppc64}) for the $\mathcal{O}(N^2)$ kernels.
The ARM system showcases a similar effect.
Eliminating the (physically required) distance checks however does not give us a faster code.
The elimination of the atomics helps the local-active version over coalesced memory accesses to almost close the gap to the generic code base.
However, once we normalise all speedups (Figures~\ref{figure:cpu:overhead:Grace} and \ref{figure:cpu:overhead:Intel}), it becomes clear that a generic implementation not trying to exploit any memory insight performs best---although an advantageous memory layout can help (cmp.~differing observations in \cite{Hundt:2006:StructureLayoutOptimisation,Intel:MemoryLayoutTransformations}).
Yet, this statement is only true for the kernels with quadratic compute complexity. 
The linear kernels with a Stream-like access pattern do not benefit from the view concept~\cite{Homann:2018:SoAx,Strzodka:2011:AbstractionSoA}.




\begin{figure}[H]
\centering
 \includegraphics[width=1.00\textwidth]{experiments/kernel-throughput/gh003/plot-1-aarch64-density-legend.pdf}
 \includegraphics[width=0.49\textwidth]{experiments/kernel-throughput/gh003/plot-1-aarch64-density.pdf}
 \includegraphics[width=0.49\textwidth]{experiments/kernel-throughput/gh003/plot-1-aarch64-force.pdf}
 \includegraphics[width=0.49\textwidth]{experiments/kernel-throughput/gh003/plot-1-aarch64-drift.pdf}
 \includegraphics[width=0.49\textwidth]{experiments/kernel-throughput/gh003/plot-1-aarch64-kick1.pdf}
%  \includegraphics[width=0.49\textwidth]{experiments/kernel-throughput/gh003/plot-1-aarch64-kick2.pdf}
 \caption{
   Normalised speedups for the density, force, drift and the first kick kernels on the Nvidia Grace CPU using $ppc=1,024$.
   \texttt{pred.~on} denotes the manual elimination of the loop body predicate, i.e.~all particle pairs are evaluated or all particles are updated, respectively.
   \label{figure:cpu:overhead:Grace}
 }
\end{figure}



\begin{figure}[htb]
\centering
 \includegraphics[width=1.00\textwidth]{experiments/kernel-throughput/gi001/plot-1-x64-density-legend.pdf}
 \includegraphics[width=0.49\textwidth]{experiments/kernel-throughput/gi001/plot-1-x64-density.pdf}
 \includegraphics[width=0.49\textwidth]{experiments/kernel-throughput/gi001/plot-1-x64-force.pdf}
 \includegraphics[width=0.49\textwidth]{experiments/kernel-throughput/gi001/plot-1-x64-drift.pdf}
 \includegraphics[width=0.49\textwidth]{experiments/kernel-throughput/gi001/plot-1-x64-kick1.pdf}
% \includegraphics[width=0.49\textwidth]{experiments/kernel-throughput/gi001/plot-1-x64-kick2.pdf}
  \caption{
    Data from Figure~\ref{figure:cpu:overhead:Grace} for the Intel system.
    \label{figure:cpu:overhead:Intel}
  }
\end{figure}





%
% Interpretation
%
A speedup of a factor of two is not what we would hope to see on a modern vector architecture for a compute-intense kernel.
However, the SPH interactions are complex and still contain branching, and our implementation does not dive deeper into a kernel assessment and optimisation.
The data showcases that it is reasonable to remove complex atomic accesses from the core loops if these are infrequently encountered.
The data also highlights that it is important to avoid accummulation steps:
Fixing one particle and gathering the impact from all surrounding particles is not fast.
Instead, it is better to realise a spread-out paradigm, i.e.~scattered writes outperform accumulation once the data is converted into SoA.
However, small subchunks---with an average $ppc \approx 64$ and continuous particle chunks being assigned to the $2^d=8$ vertices, each chunk has only around eight particles---still hamper perforance and it remains advantageous to gather all data into a temporary large SoA memory block.


The removal (or masking out) of interactions improves the vector efficiency quite significantly.
Yet, it also increases the compute load, and eventually fails to compensate for the increased amount of work. 


\begin{observation}
 The SoA conversion with views helps the compiler to generate vector instructions. 
\end{observation}

\begin{observation}
 A temporary, local conversion into SoA is beneficial for kernels with high computational complexity but problematic for streaming kernels. 
\end{observation}

%
% Implication, contextualisation
%
\noindent
While the views help the compiler to vectorise better, it is not a free lunch.
It is merely a first step to facilitate better vectorisation and to kickstart more in-depth kernel analysis how to improve the performance further. 

\subsection{GPU offloading}

We conclude our studies with offloading kernels to the GPU.
As our SPH kernels are well-defined and atomic, this can be done through annotations only, i.e.~we do not have to employ additional OpenMP pragmas.
With the Grace-Hopper superchip, we have two variants available how to manage the data transfer:
We can copy over data explicitly via map clauses, or we can use the hardware's shared memory paradigm which translates into a lazy memory transfer similar to a cache miss.


\begin{figure}[htb]
\centering
    \includegraphics[width=1.00\textwidth]{experiments/kernel-throughput/gh003/out-aarch64-soatrue-cell0064-predfalse-symfalse-gputrue-legend.pdf}
    \includegraphics[width=0.49\textwidth]{experiments/kernel-throughput/gh003/out-aarch64-soatrue-cell0064-predfalse-symfalse-gputrue.pdf}
    \includegraphics[width=0.49\textwidth]{experiments/kernel-throughput/gh003/out-aarch64-soatrue-cell0128-predfalse-symfalse-gputrue.pdf}     
    \includegraphics[width=0.49\textwidth]{experiments/kernel-throughput/gh003/out-aarch64-soatrue-cell0256-predfalse-symfalse-gputrue.pdf}     
    \includegraphics[width=0.49\textwidth]{experiments/kernel-throughput/gh003/out-aarch64-soatrue-cell1024-predfalse-symfalse-gputrue.pdf}     
  \caption{
   	Cost per particle update for various kernels on Hopper GPU. 
   	All GPU kernels are generated through attribute annotations and transfer their data explicitly to the accelerator prior to the kernel start. 
   	In lexicographic order: $ppc=64$, $ppc=128$, $ppc=256$, and $ppc=1024$.
   \label{figure:results:gpu}
  }
\end{figure}



\begin{figure}[htb]
\centering
    \includegraphics[width=1.00\textwidth]{experiments/kernel-throughput/gh003/out-aarch64-soatrue-cell0064-predfalse-symfalse-gputrue-legend.pdf}
    \includegraphics[width=0.49\textwidth]{experiments/kernel-throughput/gh003/out-aarch64-soatrue-cell0064-predfalse-symfalse-gputrue-shmem.pdf}
    \includegraphics[width=0.49\textwidth]{experiments/kernel-throughput/gh003/out-aarch64-soatrue-cell0128-predfalse-symfalse-gputrue-shmem.pdf}     
    \includegraphics[width=0.49\textwidth]{experiments/kernel-throughput/gh003/out-aarch64-soatrue-cell0256-predfalse-symfalse-gputrue-shmem.pdf}     
    \includegraphics[width=0.49\textwidth]{experiments/kernel-throughput/gh003/out-aarch64-soatrue-cell1024-predfalse-symfalse-gputrue-shmem.pdf}     
  \caption{
    Experiments from Figure~\ref{figure:results:gpu} repeated with shared memory, i.e.~without any explicit memory transfer.
   \label{figure:results:gpu-shmem}
  }
\end{figure}

%
% What we see
%
The GPU throughput is low compared to the theoretical capability of the device, but the data is qualitatively similar to the results on the Intel and ARM CPUs (Figure~\ref{figure:results:gpu}) in that larger total particle counts give us a better throughput.
We see a less pronounced relation of throughput to $ppc$, but there is an inversion of the previously stated trend:
Bigger $ppc$ give us superior time-to-solution.
The Grace Hopper's shared memory capabilities give us another factor of two in performance (Figure~\ref{figure:results:gpu-shmem}).




%
% Interpretation
%
GPUs require high concurrency levels in the compute kernels to unfold their full potential.
In return, they are not as vulnerable to the branching due to the distance checks as the vectorised CPU kernels:
The masking facilities on the accelerator can compensate for the thread divergence.
Therefore, picking larger $ppc$ becomes reasonable.

Despite the tight memory connection on the Grace-Hopper superchip, we see a limited performance improvement in our kernels, which is due to the low concurrency.
It is not clear from the present setups, to which degree our data suffers from a lack of accelerator-specific tuning, such as the tailoring of warp sizes or simply too many, too small kernels.
Unless the GPU software stack or the hardware start to facilitate very small kernels, a direct mapping of compute kernels onto GPU kernels does not deliver satisfying results.

Our vanilla offloading in the compiler issues a sequence of three tasks: map, compute, map back.
They are connected through dependencies yet launched asynchronously.
After the final map, we wait.
If we enable the superchip's shared memory, the three tasks degenerate to one single blocking tasks.
Nevertheless, it seems that the chip delivers better performance in this mode.
We assume that OpenMP struggles to overlap the data transfer of all CPU threads firing map-compute-map sequences simultaneously.
Contrary, the hardware seems to succeed to let shared memory tasks run ahead and to bring in all required data lazily upon demand.

Overall, our annotations let the kernels on the CPU outperform their GPU counterpart. 
All data conversation resides on the CPU.
We may therefore conclude that any improvement of compute speed on the accelerator is (partially) consumed by data transfers or on-GPU overheads.
It is also not clear what role the inferior support for FP64 on the GPU plays for out experiments.
It is reasonable to believe that the insight might change qualitatively if we were able to convert our kernels to FP32---an endeavour which requires care as it tends to introduce numerical long-term instabilities.



\begin{observation}
 Our annotations streamline GPU offloading as they tackle the memory bottleneck, but they do not free the developer from generating tailored compute kernels with a sufficiently high concurrency level. 
\end{observation}


\noindent
It is well-know that GPUs benefit from gathering multiple compute tasks into one larger GPU kernel \cite{Nasar:2024:GPU}.
Similar techniques have been proposed in a more general way for task-based systems~\cite{LiShultzWeinzierl:2022:Tasking}.
Our data suggest that using such techniques are absolutely essential to write fast GPU code, and that the naive offloading of individual, tiny compute kernels does not yield satisfying performance.
However, the core idea of reordering data through compiler annotations streamlines the writing of more complex (meta-)kernels as well.




