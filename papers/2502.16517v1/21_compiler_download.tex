\section{Compiler download}
\label{appendix:compiler-download}

The forked Clang/LLVM project is publicly available at
\linebreak 
\url{https://github.com/pradt2/llvm-project}, our extensions are available in the \texttt{hpc-ext} branch.

The \texttt{hpc-ext} branch is up to date with the upstream LLVM \texttt{main} branch as of 12 Feb 2025.

To avoid conflicts, it is strongly recommended to remove any existing Clang/LLVM installations before proceeding.

To build and install the compiler toolchain, clone or otherwise download and unpack the repository, create a build folder in the top-level repository directory, and execute all steps from Algorithm~\ref{algorithm:appendix:compiler-download:build-compiler}.

\begin{algorithm}[htb]
    \begin{algorithmic}[1]
      \State 
      \begin{verbatim}
cmake 
  -DCMAKE_BUILD_TYPE="RelWithDebInfo" 
  -DCMAKE_C_COMPILER="gcc" 
  -DCMAKE_CXX_COMPILER="g++" 
  -DLLVM_ENABLE_PROJECTS="clang;lld" 
  -DLLVM_ENABLE_RUNTIMES="openmp;offload" 
  -DLLVM_TARGETS_TO_BUILD="X86;AArch64;NVPTX;AMDGPU"
  ../llvm
      \end{verbatim}
\State \verb|make && make install|
    \end{algorithmic}
  \caption{
    Building and installing the compiler
    \label{algorithm:appendix:compiler-download:build-compiler}
    }
\end{algorithm}

\noindent

From hereon, \texttt{clang++} directs to our modified compiler version.


% \subsection{Usage and troubleshooting}
% To compile a C source file, use the \texttt{clang} command, and to compile C++ sources, use \texttt{clang++}.

The command-line interface (CLI) is backwards compatible with the upstream
Clang/LLVM version. 
The support for the new attributes discussed in this paper is enabled by default, no additional compiler flags are needed.

If the use of any of the new attributes leads to a compilation error, a common
troubleshooting starting point is to inspect the rewritten source code. To see
the rewritten code, we can add \texttt{-fpostprocessing-output-dump} to the
compilation flags. This flag causes the post-processed source code be written to the standard output.
