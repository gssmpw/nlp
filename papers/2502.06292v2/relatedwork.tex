\section{Related Work}
\label{Sec_related_work}

In this section, we discuss some related work on non-feature based map representations for SLAM, submap joining techniques, and joint optimization of poses and maps. 

\subsection{Non-feature based map representations for SLAM}\label{Sec_related_a}
One widely used non-feature based SLAM approach is occupancy grid map-based SLAM, which probabilistically classifies spaces into obstacles, free areas, and unknown regions while accounting for uncertainty during observation updates \cite{moravec1985high, moravec1989sensor, elfes1989occupancy, martin1996robot, hornung2013octomap}. Classic examples, such as FastSLAM \cite{montemerlo2002fastslam} and GMapping \cite{grisetti2005improving}, use particle filters for mapping and localization but struggle with high computational demand and long-term accuracy in large-scale environments.

Recent optimization-based approaches, such as Hector SLAM \cite{kohlbrecher2011flexible}, Karto-SLAM \cite{konolige2010efficient}, and Cartographer \cite{hess2016real}, address cumulative errors effectively. Hector SLAM uses scan-to-map matching but lacks loop closure, restricting it to small-scale scenarios. Karto-SLAM incorporates loop closure detection with sparse pose adjustment for global optimization, while Cartographer integrates scan-to-map matching and pose graph optimization with a branch-and-bound strategy for efficient loop closure detection. However, by treating pose optimization and map construction as independent processes, these methods fail to account for the interdependencies of their uncertainties.

Multi-resolution occupancy mapping techniques can be integrated into occupancy grid map based SLAM frameworks to enable a more compact and efficient mapping process. For instance, approaches like OctoMap \cite{hornung2013octomap} use memory-efficient octrees to balance map compactness and accessibility. Adaptive-resolution methods, such as RMAP \cite{khan2014rmap} and ColMap \cite{fisher2021colmap}, dynamically adjust grid resolution to enhance mapping efficiency. Recently, \cite{Reijgwart-RSS-23} applies wavelet compression for hierarchical occupancy map storage, allowing efficient updates and queries. However, integrating multi-resolution maps as state variables into a unified framework for joint poses and map optimization remains an open challenge.


Another widely used non-feature-based map is the signed-distance function (SDF), which discretizes the environment into grid cells storing the distance to the nearest surface. This representation encodes the space, with the object surfaces defined by the zero crossings of the distance functions \cite{curless1996volumetric}. Some SLAM systems adopt SDF to improve localization accuracy and mapping quality. For example, supereight \cite{vespa2018efficient} integrates tracking, mapping, and planning using an octree-based truncated SDF (TSDF). It aligns camera frames to the TSDF map with iterative closest point (ICP) \cite{besl1992method}. A follow-up work \cite{vespa2019adaptive} improves this by introducing adaptive-resolution octree structures, achieving denser environment representation and reduced noise, leading to more accurate localization.

Other non-feature based map representations have also been used in SLAM, including mesh-based \cite{rosinol2021kimera}, normal distributions transform based \cite{einhorn2015generic}, neural radiance fields based \cite{rosinol2023nerf} and Gaussian splatting based \cite{matsuki2024gaussian}. Although these approaches differ in the type of non-feature representations they use, they all aim to provide more effective environmental modeling, improve robot localization accuracy, or achieve both. 


However, all the optimization-based SLAM approaches that utilize non-feature based maps need to optimize the poses first and then build the non-feature based map using the optimized poses. This separation prevents these approaches from jointly considering the uncertainties in both localization and mapping during the optimization process. In contrast, this paper considers unifying the optimization of both the robot poses and occupancy values at each cell vertex of the occupancy map into a single optimization problem, which can be expected to yield better accuracy.

\subsection{Submap Joining}\label{Sec_related_b}

Submap joining, as proposed by \cite{bosse2003atlas}, is a widely used scheme for SLAM in large-scale environments due to its efficiency and reduced risk of being trapped in local minima compared to full optimization-based SLAM. Feature-based submap joining approaches \cite{huang2008sparse,zhao2013linear,wang2019submap} have been well investigated, with many demonstrating properties that enable efficient problem-solving while maintaining a high level of accuracy. To extend non-feature-based SLAM to large-scale environments and long-term operations, recent efforts have explored non-feature-based submap joining methods.


For example, \cite{wagner2014graph} divides the environment into overlapping submaps composed of small TSDF grids from KinectFusion \cite{izadi2011kinectfusion}. Submap joining is then formulated as a pose graph optimization problem, where submap poses are nodes, and relative transformations from ICP are edges. Similarly, VOG-map \cite{ho2018virtual} represents submaps as 3D occupancy grids, converts them to point clouds for ICP-based relative transformations, and solves submap joining via pose graph optimization. Voxgraph \cite{reijgwart2019voxgraph} improves accuracy by employing SDF-to-SDF registration for overlapping submaps created with C-blox \cite{millane2018c}. Unlike time-sequence-based submap partitioning, \cite{wang2021elastic} uses spatial partitioning, merging submaps during loop closures by solving a pose graph containing only submap frames, with reconstruction decisions based on environmental changes.

All the aforementioned non-feature-based submap joining approaches estimate relative measurements between overlapping submaps to formulate and solve the pose graph problem for submap frames. In contrast, this paper jointly optimizes submap frames and the global occupancy map.

\subsection{Joint Optimization of Poses and Maps}
Joint optimization of poses and maps can result in better accuracy, as it utilizes the information more directly. In feature-based SLAM and bundle adjustment approaches, the most common form is to jointly optimize poses and positions of features, such as \cite{dellaert2006square,triggs2000bundle,konolige2008frameslam,sibley2009adaptive,zhao2015parallaxba}. Some approaches extend this idea to planar feature parameters. For instance, \cite{kaess2015simultaneous,hsiao2017keyframe} minimize the difference between plane measured in a scan and predicted planes, while \cite{trevor2012planar,geneva2018lips,zhou2021pi,zhou2021lidar} minimize the Euclidean distance between points in a scan and the predicted planes. Based on the idea of minimizing Euclidean distance between points in scans, BALM \cite{liu2021balm} demonstrates that planar parameters can be solved analytically in closed form, reducing the dimensionality of the optimization. BALM2 \cite{liu2023efficient} further improves efficiency by using point clusters, avoiding individual point enumeration. HBA \cite{liu2023large} introduces a hierarchical structure to address the scalability challenges of BALM and BALM2 in large environments. In summary, jointly optimizing poses and feature-based maps is well-studied, as features naturally link positions, observations, and poses, making them straightforward to integrate into optimization problems. In contrast, establishing constraints between observations, poses, and non-feature-based maps (e.g., occupancy grid maps) for joint optimization remains a significant challenge.



% Optimizing the poses and feature-based map together is very common and has been well-studied, as features are naturally linked to positions, which in turn connect observations, poses, and features, making them straightforward to integrate into optimization problems. However, it is a challenge to establish constraints between observations, poses, and a non-feature based map to jointly optimize the poses and the map (e.g., an occupancy grid map).

% Kimera-PGMO proposed in \cite{rosinol2021kimera} is a novel approach that simultaneously optimizes the poses and the mesh deformation. Specifically, Kimera-PGMO \cite{rosinol2021kimera} creates a deformation graph including a simplified mesh and a pose graph of robot poses. Since the simplified mesh consists of the positions of the mesh vertices, the method is formulated as a factor graph and then solved by GTSAM \cite{dellaert2012factor}.

Research on jointly optimizing the poses and non-feature based maps is limited. Kimera-PGMO proposed in \cite{rosinol2021kimera} represents a notable attempt, integrating pose optimization with mesh deformation. It constructs a deformation graph of a simplified mesh and a pose graph, formulating the problem as a factor graph solvable by GTSAM \cite{dellaert2012factor}. 
While Kimera-PGMO \cite{rosinol2021kimera} has similar motivations as our paper, aiming to achieve better quality maps and more accurate poses through joint optimization, its mesh-based representation differs fundamentally from the occupancy grid maps used in our approach. Meshes are naturally represented through point positions and their relationships, which facilitates factor graph formulations.


% but the mesh can still be described in terms of the positions of the points as well as the relationships between the points, and can therefore ultimately be transformed into a factor graph to be solved for, which is different to the occupancy map that we used.

\begin{table}[t]
 		\centering
 		\caption{Summary of Some Important Notations.}\label{tab_notation}
 		\setlength{\tabcolsep}{0.5 mm}{
 		\begin{tabular}{|c|l|p{3cm}p{3cm}p{3cm}}
   \hline
   \multicolumn{1}{|c|}{Notation} & \multicolumn{1}{|c|}{Explanation} \\ \hline
   $\mathbb{M}$  & \begin{tabular}[c]{@{}l@{}} A set includes occupancy values at all discrete cell vertices in \\occupancy map, as defined in Section \ref{sec_discrete_occupancy}. $\mathbb{M}^{l}$, $\mathbb{M}^{h}$, and $\mathbb{M}^{s}$ \\represent the sets include occupancy values at all cell vertices \\in low-resolution map, high-resolution map and selected \\high-resolution map, respectively. In addition, $\mathbb{M}_L$ and $\mathbb{M}_G$ \\represent the sets including occupancy values of all cell vertices \\in local maps and the global map, as defined in Section \ref{Sec_submap}.\end{tabular}\\ \hline

% as defined in \