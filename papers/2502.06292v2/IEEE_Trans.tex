\documentclass[lettersize,journal]{IEEEtran}
\usepackage{amsmath,amsfonts}
% \usepackage{algorithm}
% \usepackage{algorithmic}
% \usepackage{algpseudocode}

\usepackage[linesnumbered,ruled,vlined]{algorithm2e}


\usepackage{array}
\usepackage{accents}
%\usepackage[caption=false,font=normalsize,labelfont=sf,textfont=sf]{subfig}
%\usepackage[caption=false,font=footnotesize,labelfont=sf,textfont=sf]{subfig}
\usepackage{subfigure}

\usepackage{textcomp}
\usepackage{stfloats}
\usepackage{url}
\usepackage{verbatim}
\usepackage{graphicx}
\usepackage{cite}
\usepackage{float}
 \usepackage{tabularx} 
% \usepackage{setspace}

\hyphenation{op-tical net-works semi-conduc-tor IEEE-Xplore}
\def\BibTeX{{\rm B\kern-.05em{\sc i\kern-.025em b}\kern-.08em
    T\kern-.1667em\lower.7ex\hbox{E}\kern-.125emX}}
\usepackage{balance}
% updated with editorial comments 8/9/2021

% use bib--added by yingyu
% \usepackage[numbers]{natbib}
\usepackage{amssymb}
\usepackage{booktabs}
\usepackage{threeparttable}
\usepackage{multirow}
\usepackage{bigstrut}
\usepackage{bigdelim}
\usepackage{adjustbox}
\usepackage{makecell}
\usepackage{nicematrix}
\usepackage{xcolor}

% \usepackage{xcolor} 
\usepackage{tikz} 
\usetikzlibrary{arrows,shapes,chains}

\usepackage{colortbl}
% \usepackage[table]{xcolor}
\usepackage{rotating}
% Beamer presentation requires \usepackage{colortbl} instead of \usepackage[table,xcdraw]{xcolor}


% \renewcommand{\algorithmicrequire}{\textbf{Input:}}
% \renewcommand{\algorithmicensure}{\textbf{Output:}}
\setlength{\textfloatsep}{8pt}
% \usepackage{caption}
% \captionsetup[figure]{skip=5pt}

% \setlength{\topsep}{0pt}
% \setlength{\belowdisplayskip}{10pt}
% \setlength{\abovedisplayskip}{10pt}

\newcommand*{\defeq}{\stackrel{\text{def}}{=}}
\DeclareMathOperator{\wrap}{wrap}
\DeclareMathOperator{\diag}{diag}
\DeclareMathOperator{\round}{round}




\begin{document}

\title{Occupancy-SLAM: An Efficient and Robust Algorithm for Simultaneously Optimizing Robot Poses and Occupancy Map}

\author{\authorblockN{Yingyu Wang, Liang Zhao, and Shoudong Huang} 
        % <-this % stops a space
\thanks{Yingyu Wang and Shoudong Huang are with the Robotics Institute, University of Technology Sydney, Australia (e-mail: Yingyu.Wang-1@student.uts.edu.au; Shoudong.Huang@uts.edu.au).} 

\thanks{Liang Zhao was with the Robotics Institute, University of Technology Sydney, Australia, and is now with the School of Informatics, University of Edinburgh, Edinburgh, UK (e-mail: Liang.Zhao@ed.ac.uk).}}



% The paper headers
\markboth{IEEE TRANSACTIONS ON ROBOTICS}%
{Shell \MakeLowercase{\textit{et al.}}: A Sample Article Using IEEEtran.cls for IEEE Journals}

% \IEEEpubid{0000--0000/00\$00.00~\copyright~2021 IEEE}
% Remember, if you use this you must call \IEEEpubidadjcol in the second
% column for its text to clear the IEEEpubid mark.

\maketitle

% This paper proposes Occupancy-SLAM, an optimization-based SLAM approach that jointly optimizes the robot trajectory and the occupancy map simultaneously.

\begin{abstract}
Joint optimization of poses and features has been extensively studied and demonstrated to yield more accurate results in feature-based SLAM problems. However, research on jointly optimizing poses and non-feature-based maps remains limited. Occupancy maps are widely used non-feature-based environment representations because they effectively classify spaces into obstacles, free areas, and unknown regions, providing robots with spatial information for various tasks. In this paper, we propose Occupancy-SLAM, a novel optimization-based SLAM method that enables the joint optimization of robot trajectory and the occupancy map through a parameterized map representation. The key novelty lies in optimizing both robot poses and occupancy values at different cell vertices simultaneously, a significant departure from existing methods where the robot poses need to be optimized first before the map can be estimated. 

This paper focuses on 2D laser-based SLAM to investigate how to jointly optimize robot poses and the occupancy map. In our formulation, the state variables in optimization include all the robot poses and the occupancy values at discrete cell vertices in the occupancy map. Moreover, a multi-resolution optimization framework that utilizes occupancy maps with varying resolutions in different stages is introduced. A variation of Gauss-Newton method is proposed to solve the optimization problem at different stages to obtain the optimized occupancy map and robot trajectory. The proposed algorithm is efficient and converges easily with initialization from either odometry inputs or scan matching, even when only limited key-frame scans are used. Furthermore, we propose an occupancy submap joining method, enabling more effective handling of large-scale problems by incorporating the submap joining process into the Occupancy-SLAM framework. Evaluations using simulations and practical 2D laser datasets demonstrate that the proposed approach can robustly obtain more accurate robot trajectories and occupancy maps than state-of-the-art techniques with comparable computational time. Preliminary results in the 3D case further confirm the potential of the proposed method in practical 3D applications, achieving more accurate results than existing methods. The code is made available to benefit the robotics community\footnote{\url{https://github.com/WANGYINGYU/Occupancy-SLAM}}. 
\end{abstract}

\begin{IEEEkeywords}
SLAM, optimization, occupancy grid map, non-feature-based map representation.
\end{IEEEkeywords}

\section{Introduction}

% optimizing robot poses and features simultaneously 


\IEEEPARstart{S}{imultaneous} localization and mapping (SLAM) is an important problem in robotics that has been studied for decades \cite{cadena2016past}. Jointly optimizing the robot poses and map can enhance SLAM performance, as this formulation utilizes the available information more directly without approximations. While joint optimization has been widely explored in feature-based SLAM (e.g., \cite{kaess2008isam,kaess2011isam2}), research on the joint optimization of robot poses and non-feature-based maps remains limited.

Occupancy grid maps are widely used in robotic tasks for their ability to clearly represent obstacles, free space, and unknown areas, facilitating collision-free navigation and path planning. Assuming the robot poses used to collect the sensor information are known exactly, the evidence grid mapping technique \cite{moravec1985high,
moravec1989sensor,elfes1989occupancy,martin1996robot,hornung2013octomap} provides an elegant and efficient approach for building occupancy grid maps from the collected information. However, when a robot is navigating in an unknown environment and performing SLAM, its own poses need to be estimated, and the estimates inherently contain uncertainties. Achieving both accurate robot localization and precise occupancy mapping simultaneously is not trivial.

% How to perform accurate robot localization and build an occupancy map very accurately at the same time is not trivial. 


In some occupancy grid map based SLAM approaches such as Cartographer \cite{hess2016real}, the problem is tackled in two steps. First, the robot poses are estimated by solving a pose-graph SLAM problem, where the relative pose measurements are derived using odometry, scan matching, loop closure detection, or other similar techniques. Second, the optimized poses are assumed to be the correct poses and are used to build up the map using evidence grid mapping techniques. However, in these two-step approaches, the uncertainties in the robot poses obtained during the first step are not considered when building the map. Therefore, it is crucial to achieve highly accurate pose estimates to construct a reliable occupancy grid map. As a result, it can be expected that the occupancy map obtained using a two-step approach may not represent the best result that one can achieve using all the available information.


In feature-based SLAM approaches, jointly optimizing the poses and the feature map is common, as the relationship between observations and the map is straightforward to model. However, for occupancy map based SLAM, jointly optimizing the robot poses and the occupancy map is not trivial because: 
\begin{itemize}
	\item [1)] \textbf{The relation between the observations and the map is complex.} The observations are laser beams (the endpoint of a beam represents ``hit" and the other positions along the beam represent ``free"), and the map is a function representing the occupancy values at different positions. This is significantly different from feature-based SLAM where both the observations and the map are about feature parameters such as positions.
	\item [2)] \textbf{The data association is not easy to do.} When the robot poses are noisy, the correct correspondence between laser beams and occupancy grid cells is hard to find. In contrast, for feature-based SLAM, there are well-established front-end methods for data association.
	\item [3)] \textbf{The resolution of the map has a significant impact on the optimization problem.} A high-resolution map helps to establish a more accurate connection between the observations and the map, but it leads to a sharp increase in the number of variables. However, for feature-based SLAM, there is no such issue.
\end{itemize}

% using 2D laser scans (and odometry) information.

\subsection{Contributions}
In this paper, we propose Occupancy-SLAM algorithm, which jointly optimizes the robot poses and the occupancy map using 2D laser scans (and odometry) information. Moreover, we propose a multi-resolution optimization framework for improving convergence and robustness to initial guesses. To better handle the case of large-scale environments and long-term trajectories, we further propose an occupancy submap joining method. Experiments on both simulated and practical datasets verify the superior performance of our method compared with state-of-the-art approaches (e.g., Cartographer \cite{hess2016real}). In addition, we extend our method to the 3D case, and preliminary results confirm its effectiveness in improving accuracy. The main contributions are summarized as follows: 

 % A smoothing term is introduced in the objective function to improve the convergence of the method.

\begin{enumerate}
	\item We formulate the occupancy grid map based SLAM problem as a joint optimization problem where the poses and the occupancy map are optimized together. 
	\item We propose a variation of Gauss-Newton method to solve the new formulation, enabling the estimation of more accurate robot poses and occupancy maps compared to existing state-of-the-art techniques.
	\item To enhance efficiency, convergence, and robustness, we propose a multi-resolution optimization strategy using occupancy maps of different resolutions across stages.
    
    % To improve the efficiency, convergence and robustness of our algorithm so that it can be initialized by odometry inputs or scan matching, we propose a multi-resolution optimization strategy that uses occupancy maps with different resolutions at different optimization stages. In the second stage, we utilize the selected high-resolution map, focusing exclusively on a subset of \textcolor{red}{cell vertices} that require further updates within the full high-resolution map. This targeted approach further enhances computational efficiency.
    
    \item We propose a submap joining algorithm to address the cases of large-scale environments and long-term trajectories through our joint poses and occupancy map optimization idea.
	\item Our method achieves robust convergence even with key frames of limited overlap, outperforming state-of-the-art approaches like Cartographer in efficiency while maintaining superior accuracy.
    % Our proposed method can converge well even when only key frames with limited overlaps are used. In this case, our method outperforms state-of-the-art methods, such as Cartographer, in terms of efficiency while maintaining a surpassing performance in terms of accuracy.
    \item We extend our method to 3D, with preliminary results demonstrating superior accuracy compared to other state-of-the-art approaches.
    % We extend our method to the 3D case, and preliminary results confirm that the accuracy of our method outperforms other state-of-the-art approaches.
\end{enumerate}


This paper is an extension to our conference paper \cite{Zhao-RSS-22}, with major improvements in contributions 3, 4, 5, and 6, significantly enhancing the robustness and efficiency of the algorithm while extending the method to 3D.

% The major improvements of this paper over \cite{Zhao-RSS-22} are contributions 3, 4, 5, and 6, which significantly improve the robustness and efficiency of the algorithm, and extend the algorithm to 3D.

\begin{figure}
\centering
\includegraphics[width=0.49\textwidth]{./OverView.pdf}
\caption{\label{fig_overview} Main components of our proposed method. The blue-colored components represent our core approaches, while the dashed portions are optional. Our multi-resolution joint optimization is covered in Section \ref{sec_formulation}, Section \ref{Sec_Algorithm_1}, and Section \ref{Sec_multi}. The joint global map and robot trajectory optimization approach is presented in Section \ref{Sec_submap}. }
\end{figure}

% \subsection{Notations}
% Some important notations in this paper are summarized in Table \ref{tab_notation}, the others are described in the context.

\subsection{Outline}
Fig. \ref{fig_overview} illustrates the flowchart of applying our proposed methods in practice. The blue components represent our core approaches, while the dashed portions are optional. The rest of the paper is organized as follows: Section \ref{Sec_related_work} provides a review of related work on non-feature-based SLAM, submap joining, and joint optimization of poses and maps. In Section \ref{sec_formulation}, we introduce our novel formulation for jointly optimizing the robot poses and occupancy map. A variation of the Gauss-Newton method to solve our nonlinear least squares (NLLS) formulation is presented in Section \ref{Sec_Algorithm_1}. In Section \ref{Sec_multi}, we introduce our multi-resolution strategy to improve the efficiency and robustness of the algorithm. Section \ref{Sec_submap} presents our submap joining algorithm for handling large-scale environments and long-term trajectories. Experimental results are provided in Section \ref{Sec_experiment}. In Section \ref{sec_3d}, we extend our method to the 3D case and present preliminary results. Finally, the conclusions are given in Section \ref{Sec_conclusion}.

 
\section{Related Work}\label{Sec_related_work}

In this section, we discuss some related work on non-feature based map representations for SLAM, submap joining techniques, and joint optimization of poses and maps. 

\subsection{Non-feature based map representations for SLAM}\label{Sec_related_a}
One widely used non-feature based SLAM approach is occupancy grid map-based SLAM, which probabilistically classifies spaces into obstacles, free areas, and unknown regions while accounting for uncertainty during observation updates \cite{moravec1985high, moravec1989sensor, elfes1989occupancy, martin1996robot, hornung2013octomap}. Classic examples, such as FastSLAM \cite{montemerlo2002fastslam} and GMapping \cite{grisetti2005improving}, use particle filters for mapping and localization but struggle with high computational demand and long-term accuracy in large-scale environments.

Recent optimization-based approaches, such as Hector SLAM \cite{kohlbrecher2011flexible}, Karto-SLAM \cite{konolige2010efficient}, and Cartographer \cite{hess2016real}, address cumulative errors effectively. Hector SLAM uses scan-to-map matching but lacks loop closure, restricting it to small-scale scenarios. Karto-SLAM incorporates loop closure detection with sparse pose adjustment for global optimization, while Cartographer integrates scan-to-map matching and pose graph optimization with a branch-and-bound strategy for efficient loop closure detection. However, by treating pose optimization and map construction as independent processes, these methods fail to account for the interdependencies of their uncertainties.

Multi-resolution occupancy mapping techniques can be integrated into occupancy grid map based SLAM frameworks to enable a more compact and efficient mapping process. For instance, approaches like OctoMap \cite{hornung2013octomap} use memory-efficient octrees to balance map compactness and accessibility. Adaptive-resolution methods, such as RMAP \cite{khan2014rmap} and ColMap \cite{fisher2021colmap}, dynamically adjust grid resolution to enhance mapping efficiency. Recently, \cite{Reijgwart-RSS-23} applies wavelet compression for hierarchical occupancy map storage, allowing efficient updates and queries. However, integrating multi-resolution maps as state variables into a unified framework for joint poses and map optimization remains an open challenge.


Another widely used non-feature-based map is the signed-distance function (SDF), which discretizes the environment into grid cells storing the distance to the nearest surface. This representation encodes the space, with the object surfaces defined by the zero crossings of the distance functions \cite{curless1996volumetric}. Some SLAM systems adopt SDF to improve localization accuracy and mapping quality. For example, supereight \cite{vespa2018efficient} integrates tracking, mapping, and planning using an octree-based truncated SDF (TSDF). It aligns camera frames to the TSDF map with iterative closest point (ICP) \cite{besl1992method}. A follow-up work \cite{vespa2019adaptive} improves this by introducing adaptive-resolution octree structures, achieving denser environment representation and reduced noise, leading to more accurate localization.

Other non-feature based map representations have also been used in SLAM, including mesh-based \cite{rosinol2021kimera}, normal distributions transform based \cite{einhorn2015generic}, neural radiance fields based \cite{rosinol2023nerf} and Gaussian splatting based \cite{matsuki2024gaussian}. Although these approaches differ in the type of non-feature representations they use, they all aim to provide more effective environmental modeling, improve robot localization accuracy, or achieve both. 


However, all the optimization-based SLAM approaches that utilize non-feature based maps need to optimize the poses first and then build the non-feature based map using the optimized poses. This separation prevents these approaches from jointly considering the uncertainties in both localization and mapping during the optimization process. In contrast, this paper considers unifying the optimization of both the robot poses and occupancy values at each cell vertex of the occupancy map into a single optimization problem, which can be expected to yield better accuracy.

\subsection{Submap Joining}\label{Sec_related_b}

Submap joining, as proposed by \cite{bosse2003atlas}, is a widely used scheme for SLAM in large-scale environments due to its efficiency and reduced risk of being trapped in local minima compared to full optimization-based SLAM. Feature-based submap joining approaches \cite{huang2008sparse,zhao2013linear,wang2019submap} have been well investigated, with many demonstrating properties that enable efficient problem-solving while maintaining a high level of accuracy. To extend non-feature-based SLAM to large-scale environments and long-term operations, recent efforts have explored non-feature-based submap joining methods.


For example, \cite{wagner2014graph} divides the environment into overlapping submaps composed of small TSDF grids from KinectFusion \cite{izadi2011kinectfusion}. Submap joining is then formulated as a pose graph optimization problem, where submap poses are nodes, and relative transformations from ICP are edges. Similarly, VOG-map \cite{ho2018virtual} represents submaps as 3D occupancy grids, converts them to point clouds for ICP-based relative transformations, and solves submap joining via pose graph optimization. Voxgraph \cite{reijgwart2019voxgraph} improves accuracy by employing SDF-to-SDF registration for overlapping submaps created with C-blox \cite{millane2018c}. Unlike time-sequence-based submap partitioning, \cite{wang2021elastic} uses spatial partitioning, merging submaps during loop closures by solving a pose graph containing only submap frames, with reconstruction decisions based on environmental changes.

All the aforementioned non-feature-based submap joining approaches estimate relative measurements between overlapping submaps to formulate and solve the pose graph problem for submap frames. In contrast, this paper jointly optimizes submap frames and the global occupancy map.

\subsection{Joint Optimization of Poses and Maps}
Joint optimization of poses and maps can result in better accuracy, as it utilizes the information more directly. In feature-based SLAM and bundle adjustment approaches, the most common form is to jointly optimize poses and positions of features, such as \cite{dellaert2006square,triggs2000bundle,konolige2008frameslam,sibley2009adaptive,zhao2015parallaxba}. Some approaches extend this idea to planar feature parameters. For instance, \cite{kaess2015simultaneous,hsiao2017keyframe} minimize the difference between plane measured in a scan and predicted planes, while \cite{trevor2012planar,geneva2018lips,zhou2021pi,zhou2021lidar} minimize the Euclidean distance between points in a scan and the predicted planes. Based on the idea of minimizing Euclidean distance between points in scans, BALM \cite{liu2021balm} demonstrates that planar parameters can be solved analytically in closed form, reducing the dimensionality of the optimization. BALM2 \cite{liu2023efficient} further improves efficiency by using point clusters, avoiding individual point enumeration. HBA \cite{liu2023large} introduces a hierarchical structure to address the scalability challenges of BALM and BALM2 in large environments. In summary, jointly optimizing poses and feature-based maps is well-studied, as features naturally link positions, observations, and poses, making them straightforward to integrate into optimization problems. In contrast, establishing constraints between observations, poses, and non-feature-based maps (e.g., occupancy grid maps) for joint optimization remains a significant challenge.



% Optimizing the poses and feature-based map together is very common and has been well-studied, as features are naturally linked to positions, which in turn connect observations, poses, and features, making them straightforward to integrate into optimization problems. However, it is a challenge to establish constraints between observations, poses, and a non-feature based map to jointly optimize the poses and the map (e.g., an occupancy grid map).

% Kimera-PGMO proposed in \cite{rosinol2021kimera} is a novel approach that simultaneously optimizes the poses and the mesh deformation. Specifically, Kimera-PGMO \cite{rosinol2021kimera} creates a deformation graph including a simplified mesh and a pose graph of robot poses. Since the simplified mesh consists of the positions of the mesh vertices, the method is formulated as a factor graph and then solved by GTSAM \cite{dellaert2012factor}.

Research on jointly optimizing the poses and non-feature based maps is limited. Kimera-PGMO proposed in \cite{rosinol2021kimera} represents a notable attempt, integrating pose optimization with mesh deformation. It constructs a deformation graph of a simplified mesh and a pose graph, formulating the problem as a factor graph solvable by GTSAM \cite{dellaert2012factor}. 
While Kimera-PGMO \cite{rosinol2021kimera} has similar motivations as our paper, aiming to achieve better quality maps and more accurate poses through joint optimization, its mesh-based representation differs fundamentally from the occupancy grid maps used in our approach. Meshes are naturally represented through point positions and their relationships, which facilitates factor graph formulations.


% but the mesh can still be described in terms of the positions of the points as well as the relationships between the points, and can therefore ultimately be transformed into a factor graph to be solved for, which is different to the occupancy map that we used.

\begin{table}[t]
 		\centering
 		\caption{Summary of Some Important Notations.}\label{tab_notation}
 		\setlength{\tabcolsep}{0.5 mm}{
 		\begin{tabular}{|c|l|p{3cm}p{3cm}p{3cm}}
   \hline
   \multicolumn{1}{|c|}{Notation} & \multicolumn{1}{|c|}{Explanation} \\ \hline
   $\mathbb{M}$  & \begin{tabular}[c]{@{}l@{}} A set includes occupancy values at all discrete cell vertices in \\occupancy map, as defined in Section \ref{sec_discrete_occupancy}. $\mathbb{M}^{l}$, $\mathbb{M}^{h}$, and $\mathbb{M}^{s}$ \\represent the sets include occupancy values at all cell vertices \\in low-resolution map, high-resolution map and selected \\high-resolution map, respectively. In addition, $\mathbb{M}_L$ and $\mathbb{M}_G$ \\represent the sets including occupancy values of all cell vertices \\in local maps and the global map, as defined in Section \ref{Sec_submap}.\end{tabular}\\ \hline

% as defined in \\Section \ref{continuous_map}
   $M(\cdot)$ &\begin{tabular}[c]{@{}l@{}}A function to lookup occupancy value at an arbitrary position in \\the occupancy map by bilinear interpolation using $\mathbb{M}$.\end{tabular} \\ \hline

    $\mathbf{x}^M$ & \begin{tabular}[c]{@{}l@{}} A vector including occupancy values at all cell vertices in discrete \\occupancy map $\mathbb{M}$, as defined in (\ref{eq_map_state}). $\mathbf{x}^{lM}$ and $\mathbf{x}^{sM}$ are vectors \\including occupancy values at cell vertices in $\mathbb{M}^{l}$ and $\mathbb{M}^{s}$. In \\addition, $\mathbf{x}^M_G$ represents the vector which includes occupancy \\values at cell vertices from $M_G$, as described in Section \ref{Sec_submap}.\end{tabular} \\ \hline

    $N(\cdot)$ & \begin{tabular}[c]{@{}l@{}} A function to lookup hit number at arbitrary position in the map \\by bilinear interpolation using hit map $\mathbb{N}$, where $\mathbb{N}$ is defined as a \\set includes hit number at all discrete cell vertices in the map, as \\described in Section \ref{sec_hit}. $\mathbb{N}^{l}$ and $\mathbb{N}^{s}$ represent hit maps used in \\different optimization stages.\end{tabular}\\ \hline

    $\mathbf{x}^P$ & \begin{tabular}[c]{@{}l@{}} A vector including all robot poses for optimization, as defined \\in (\ref{eq_pose_state}). In addition, $\mathbf{x}^P_L$ denotes a vector including all local map \\coordinate frames for submap joining problem in Section \ref{Sec_submap}.\end{tabular} \\ \hline
   \rule{0pt}{1.5em}
    $\mathbf{x}$ & \begin{tabular}[c]{@{}l@{}} State vector of optimization, $\mathbf{x} = {[{\mathbf{x}^P}^\top, {\mathbf{x}^M}^\top]}^\top$. $\mathbf{x}^{l}$ and $\mathbf{x}^{s}$ \\represent state vectors of different optimization stages. \end{tabular} \\ \hline

    $\mathbb{S}$  &\begin{tabular}[c]{@{}l@{}} $\mathbb{S} = \{\mathbb{S}_i\}_{0 \leq i \leq n}$ where $\mathbb{S}_i$ is defined in (\ref{S_i}), a set including \\observations, as defined in Section \ref{Sec_Info_1}. $\mathbb{S}^{l}$, $\mathbb{S}^{h}$, and $\mathbb{S}^{s}$ are \\observations used for occupancy maps $\mathbb{M}^{l}$, $\mathbb{M}^{h}$, and $\mathbb{M}^{s}$, \\respectively. \end{tabular}\\ \hline

     $s$  &\begin{tabular}[c]{@{}l@{}} Resolution of the occupancy map, which indicates the distance\\ between two nearby cell vertices. $s^{l}$ and $s^{h}$ represent the \\resolutions of low-resolution map and high-resolution map, \\respectively. $s^L$ and $s^G$ denote the resolutions of local maps and \\the global map, respectively, as described in Section \ref{Sec_submap}. \end{tabular}\\ \hline
    
     $\mathbb{O}$ & \begin{tabular}[c]{@{}l@{}}Set including all odometry inputs, as defined in Section \ref{sec_odometry}. \end{tabular}\\ \hline

    $r$  & Ratio between resolutions of two stages, $r = {s^{l}}/{s^h}$.  \\ \hline

    $\mathbf{m}$ & \begin{tabular}[c]{@{}l@{}} Discrete coordinate of a cell vertex, detailed explanation in the \\second paragraph in Section \ref{sec_discrete_occupancy}.\end{tabular} \\ \hline

    $\mathbf{p}$ & \begin{tabular}[c]{@{}l@{}}Continuous coordinate of a point, see the second paragraph in \\Section \ref{sec_relationship}.\end{tabular} \\
        
 		\hline
 		\end{tabular}
 		}
        % \vspace{-1em}

 \end{table}


\section{Problem Formulation}\label{sec_formulation}
Our approach considers the joint optimization of the robot poses and the occupancy map using information from 2D laser observations (and odometry). In this section, we will explain how the observations from the laser can be linked to the robot poses and the occupancy map to formulate the NLLS problem. 

% We also explain how we improve the problem formulation to make it easier to solve by adding a smoothing term. 

\subsection{Notation}
Throughout this paper, unless otherwise noted, we use specific typographical conventions: typefaces denote sets, bold uppercase letters represent matrices, bold lowercase letters indicate vectors, and regular (unbolded) lowercase letters signify scalars. Key notations used in this paper are summarized in Table \ref{tab_notation}, while others are introduced within the text as needed.

\subsection{Occupancy Map Representation and State in Optimization} \label{sec_discrete_occupancy}
Suppose the environment is discretized into $c_w\times c_h$ grid cells. We use $\mathbf{m}_{wh}=[w,h]^\top~(0 \leq w \leq c_w, 0 \leq h \leq c_h)$ to represent the coordinate of a discrete cell vertex in the map. The occupancy value at the cell vertex $\mathbf{m}_{wh}$, denoted as $M(\mathbf{m}_{wh})$, is defined using evidence, which is the natural logarithm of odds (the ratio between the probability of being occupied and the probability of being free) \cite{martin1996robot,hornung2013octomap,ProbabilisticRobotics}. 
The occupancy values of all $(c_w+1) \times (c_h+1)$ cell vertices consist of the discrete occupancy map $\mathbb{M}=\{M(\mathbf{m}_{wh})\}_{0 \leq w \leq c_w, 0 \leq h \leq c_h}$.


To represent the entire environment using a finite number of parameters, we describe the occupancy value at an arbitrary position $\mathbf{p}_m=[x,y]^{\top}$ on the map using bilinear interpolation of the occupancy values at its four surrounding cell vertices: $\mathbf{m}_{wh}, \mathbf{m}_{({w+1})h}, \mathbf{m}_{w({h+1})}, \mathbf{m}_{({w+1})({h+1})}$, as shown in Fig. \ref{fig_interpolation}, i.e.,

\begin{equation}
	M(\mathbf{p}_{m})= \begin{bmatrix}
a_1b_1,a_0b_1,a_1b_0,a_0b_0
\end{bmatrix}\left[
\begin{aligned}\label{eq_interp}
&M(\mathbf{m}_{wh})\\&M(\mathbf{m}_{(w+1)h})\\&M(\mathbf{m}_{w(h+1)})\\&M(\mathbf{m}_{(w+1)(h+1)})
\end{aligned}\right] 
\end{equation}
in which 
\begin{equation}
\begin{aligned}
	a_0 &= x - w\\
	a_1 &= w+1 - x\\
	b_0 &= y - h\\
	b_1 &= h+1 - y .\\
\end{aligned} 
\end{equation}


\begin{figure}[t]
\centering
\includegraphics[width=0.48\textwidth]{interpolation.pdf}
\caption{\label{fig_interpolation} Parameterizing the entire map by bilinear interpolation of discrete map $\mathbb{M}$.}
% \vspace{-1em}
\end{figure}


Our method jointly optimizes robot poses and the occupancy map, combining them into the state vector of the proposed optimization problem. Using bilinear interpolation with the discrete occupancy map $\mathbb{M}$, estimating the entire map is equivalent to estimating $\mathbb{M}$. Thus, the map component of the state vector can be expressed as

\begin{equation}
    \mathbf{x}^M =\left[M(\mathbf{m}_{00}),\cdots,M(\mathbf{m}_{c_wc_h}) \right]^\top. \label{eq_map_state}
\end{equation}


We define the $n+1$ robot poses as \rule{0pt}{1em}$\{\mathbf{x}^P_i \triangleq [\mathbf{t}_i^\top,\theta_i]^\top\}_{0 \leq i \leq n}$, where $\mathbf{t}_i$ is the $x$-$y$ position of the robot and $\theta_i$ is the orientation with the corresponding rotation matrix $\mathbf{R}_i=\begin{bmatrix}
\cos(\theta_i), \sin(\theta_i)\\ -\sin(\theta_i), \cos(\theta_i)
\end{bmatrix}$. As in most of the SLAM problem formulations, we assume the first robot pose defines the coordinate system, $\mathbf{x}^P_0 \triangleq [0,0,0]^\top$, so only the other $n$ robot poses are variables that need to be estimated, thus the pose component of the state vector is represented as
\begin{equation}
    \mathbf{x}^P = \left[ (\mathbf{x}^P_1)^\top, \cdots, (\mathbf{x}^P_n)^\top \right]^\top.
\end{equation}


Accordingly, the state vector of the proposed optimization problem is
\begin{equation}
    \mathbf{x} = \left[{(\mathbf{x}^P)}^\top,{(\mathbf{x}^M)}^\top \right]^\top. \label{eq_pose_state}
\end{equation}

In our method, the occupancy map $\mathbb{M}$ is initialized by the Bayesian occupancy mapping method \cite{ProbabilisticRobotics} with initially estimated poses (derived from odometry or scan matching) and updated throughout the optimization process.


% i.e., 
% \begin{equation}
%     \mathbf{x} = \left[{(\mathbf{x}^P)}^\top,{(\mathbf{x}^M)}^\top \right]^\top,
% \end{equation}
% where $\mathbf{x}^M$ is the map part and $\mathbf{x}^P$ is the poses part. By bilinear interpolation method and discrete occupancy map $\mathbb{M}$, we only need to estimate $(c_w+1)\times(c_h+1)$ parameters to estimate the entire map, thus $\mathbf{x}^M$ can be expressed as 
% \begin{equation}
%     \mathbf{x}^M &= \left[M(\mathbf{m}_{00}),\cdots,M(\mathbf{m}_{c_wc_h}) \right]^\top.
% \end{equation}
% We suppose that the $n+1$ robot poses are expressed by \rule{0pt}{1em}$\{\mathbf{x}^P_i \triangleq [\mathbf{t}_i^\top,\theta_i]^\top\}_{0 \leq i \leq n}$, where $\mathbf{t}_i$ is the $x$-$y$ position of the robot and $\theta_i$ is the orientation with the corresponding rotation matrix $\mathbf{R}_i=\begin{bmatrix}
% \cos(\theta_i), \sin(\theta_i)\\ -\sin(\theta_i), \cos(\theta_i)
% \end{bmatrix}$. As in most of the SLAM problem formulations, we assume the first robot pose defines the coordinate system, $\mathbf{x}^P_0 \triangleq [0,0,0]^\top$, so only the other $n$ robot poses are variables that need to be estimated. Thus, the state variables of the part of the pose can be expressed as 
% \begin{equation}
%     \mathbf{x}^P = \left[ (\mathbf{x}^P_1)^\top, \cdots, (\mathbf{x}^P_n)^\top \right]^\top.
% \end{equation}

% the state in our optimization problem can be represented as 
% \begin{equation}
%     \mathbf{x} = \left[{(\mathbf{x}^P)}^\top,{(\mathbf{x}^M)}^\top \right]^\top,
% \end{equation}
% where
% \begin{equation}
% \begin{aligned}
% \mathbf{x}^P &= \left[ (\mathbf{x}^P_1)^\top, \cdots, (\mathbf{x}^P_n)^\top \right]^\top\\
% \mathbf{x}^M &= \left[M(\mathbf{m}_{00}),\cdots,M(\mathbf{m}_{c_wc_h}) \right]^\top.
% \end{aligned}\label{eq_state_vec}
% \end{equation}

% The occupancy map is updated as the optimization process progresses without the need for additional occupancy mapping update methods, and the Bayesian occupancy mapping method is used as map initialization in our optimization problem.


% , so there is no need for an additional mapping update strategy, and furthermore, this makes the method of initializing the occupancy value $M(\mathbf{m}_{wh})$ not very critical.




% \subsection{The Available Information}\label{Sec_Info}

% The available information includes 2D laser scans collected at different robot poses. In addition, the odometry information might also be available. 

\subsection {Scan Points Sampling Strategy}\label{Sec_Info_1} 

 % We use the evidence, which is the natural logarithm of odds (the ratio between the probability of being occupied and the probability of being free) \cite{martin1996robot,hornung2013octomap,ProbabilisticRobotics} to represent the occupancy value.

 \begin{figure}[tbp]
\centering 
\subfigure[Equidistant Sampling Strategy] {\label{fig_sampling_strategy}
\includegraphics[width=0.23\textwidth]{./sampling_strategy.pdf}}
\subfigure[Observation Points in One Scan] {\label{fig_scan}
\includegraphics[width=0.23\textwidth]{./scan.pdf}}
\caption{Sampling strategy for generating observations from a laser scan: (a) Equidistant sampling on a beam, with red indicating occupied and blue indicating free states. The distance between two consecutive points is the resolution $s$. (b) All sampled observation points at a given time step.}
\label{fig_scan_sampling}
% \vspace{-1em}
\end{figure}

We now introduce our sampling strategy for generating observations from laser scans, which are used in our NLLS formulation. 

Each scan data consists of a number of beams. On each beam, the endpoint indicates the presence of an obstacle, while the other points before the endpoint indicate the absence of obstacles. Here, we sample each beam using a fixed resolution $s$ to get the observations, as shown in Fig. \ref{fig_sampling_strategy}. Specifically, $\mathbf{q}_{ij}=[x_{q_{ij}},y_{q_{ij}}]^\top$ denotes the position of $j$th sampling point at time step $i$ in the local robot/laser coordinate frame and
\begin{equation}
z_{ij} = \ln \frac{p(\mathbf{q}_{ij} \in occ)}{1-p(\mathbf{q}_{ij} \in occ)} \label{eq_occ_obs}
\end{equation}denotes the corresponding occupancy value. In the same way as the occupancy map representation described in Section \ref{sec_discrete_occupancy}, we also use the evidence to represent the occupancy value here. In our implementation, following \cite{hornung2013octomap,ProbabilisticRobotics}, we use $p(\mathbf{q}_{ij} \in occ) = 0.7$ for an occupied point (red in Fig. \ref{fig_sampling_strategy}), and use $p(\mathbf{q}_{ij} \in occ) = 0.4$ for a free point (blue in Fig. \ref{fig_sampling_strategy}). Fig. \ref{fig_scan} shows an example of all sampled points in one scan.
% $(0 \leq i \leq n)
By constant equidistant sampling of all the beams for the scan collected at time step $i$, a sampling point set
\begin{equation}
\mathbb{S}_i=\{ \mathbb{S}_{ij} \triangleq \{\mathbf{q}_{ij},z_{ij}\}\}_{1 \leq j \leq k_i}
\label{S_i}
\end{equation}
can be obtained. It should be noted that since the total length of all the beams at different time step $i$ is different, the number of sampling points $k_i$ obtained by the equidistant sampling strategy varies for different time step $i$.


Suppose there are $n+1$ laser scans collected from robot poses $0$ to $n$, $\mathbb{S}=\{\mathbb{S}_i\}_{0\leq i \leq n}$ is the available observation information collected at all different robot poses using our sampling strategy and will be used as observations in our NLLS formulation.

\subsection{Relationship Between Observations and Occupancy Map}\label{sec_relationship}


In Section \ref{sec_discrete_occupancy}, we defined the discrete occupancy map $\mathbb{M}$ as part of the state vector in our optimization problem. Section \ref{Sec_Info_1} detailed the observation generation process. In this section, we explain how the relationship between observations and the occupancy map is established through robot poses, forming the basis of our joint optimization problem.



\subsubsection{Local to Global Projection}
First, the $j$th scan point at time step $i$ can be projected to the occupancy map using the robot pose $\mathbf{x}^P_i$, and the projected position on the occupancy map can be calculated by 

\begin{equation}
	\mathbf{p}_{ij}
=\frac{\mathbf{R}_i^\top \mathbf{q}_{ij}+\mathbf{t}_i}{s} \label{P-project}
\end{equation}
where $s$ is the resolution of the vertices in the occupancy map $\mathbb{M}$ (the distance between two adjacent cell vertices represents $s$ meters in the real world). Here, we use the same resolution as that used in generating observations from laser scans in Section \ref{Sec_Info_1}. Then, the occupancy value at the projected point $\mathbf{p}_{ij}$ can be obtained using (\ref{eq_interp}), expressed as $M(\mathbf{p}_{ij})$.


\subsubsection{Relationship Between Sampling Points and Occupancy Map w.r.t. Occupancy Values}
As outlined in Sections \ref{sec_discrete_occupancy} and \ref{Sec_Info_1}, evidence is used to define the occupancy value, where multiple observations of the same cell result in the occupancy values from individual observations being cumulatively added to the cell's total occupancy value \cite{hornung2013octomap}. If the robot's poses are accurate and repeated observations of the same cell consistently indicate the same occupancy state, the cell's occupancy value becomes the product of the occupancy value of each observation and the number of times the cell is observed. For a unique coordinate $\mathbf{p}_{ij}$ in (\ref{P-project}), if both the robot pose $\mathbf{x}^P_i$ and the occupancy map $\mathbb{M}$ are accurate, the occupancy value $z_{ij}$ of its associated sampling point $\mathbb{S}_{ij}$, should closely approximate the occupancy value at $\mathbf{p}_{ij}$, $M(\mathbf{p}_{ij})$, divided by the number of times $\mathbf{p}_{ij}$ is ``observed", $N(\mathbf{p}_{ij})$.

Thus, if the number of times the point $\mathbf{p}_{ij}$ is ``observed" can be calculated, the relationship between the observations and the state vector (occupancy map and robot poses), can be determined.


\subsubsection{Hit Map and Hit Number Lookup}\label{sec_hit}

We now explain how $N({\mathbf{p}_{ij}})$ can be calculated. To quickly query the number of times an arbitrary point is ``observed", we need to count the number of times all cell vertices have been observed to form the discrete hit map $\mathbb{N}$ associated with the occupancy map $\mathbb{M}$. 

When a sampling scan point is projected into a coordinate by a given robot pose, this coordinate is considered to have been observed once, and then we distribute the hit number ``1" of the coordinate to the discretized cell vertices. Since the occupancy value $M(\mathbf{p}_{ij})$ is derived by bilinear interpolation of occupancy values of discrete cell vertices in (\ref{eq_interp}), in order to maintain the correspondence between the hit number and the occupancy values, we distribute this ``1" hit to the four surrounding cell vertices by inverse bilinear interpolation. For example, if a sampling point is projected into the center of a cell, then each of the 4 nearby cell vertices gets a hit number of 0.25. In addition, the hit number also accumulates with multiple observations of the same cell vertex, i.e.,
\begin{equation}
\left[ N(\mathbf{m}_{00}),\cdots,N(\mathbf{m}_{c_wc_h}) \right] 
= \sum_{i=0}^n \sum_{j=1}^{k_i} H(\mathbf{p}_{ij})
\label{eq_NP}
\end{equation}
where $H(\cdot)$ is the inverse process of bilinear interpolation. The hit number at all these discrete cell vertices consists of discrete hit map $\mathbb{N}=\{N(\mathbf{m}_{wh})\}_{0 \leq w \leq c_w, 0 \leq h \leq c_h}$.



After the discrete hit map $\mathbb{N}$ is obtained, the equivalent hit multiplier $N(\mathbf{p}_{ij})$ (representing the number of times $\mathbf{p}_{ij}$ is ``observed") for an arbitrary continuous point $\mathbf{p}_{ij}$ can be easily obtained using bilinear interpolation, similar to (\ref{eq_interp}). 

\subsection{The NLLS Formulation} % --- Observations Only Case
% With the map parameterization, observations generation, and projection from local to global coordinates, observations can be linked to the occupancy map through robot poses. 
We now formulate the NLLS problem to jointly optimize the robot poses and the occupancy map. The objective function of the NLLS problem is defined as
\begin{equation}
f(\mathbf{x})=w_Z f^Z(\mathbf{x})+w_O f^O(\mathbf{x})+w_S f^S(\mathbf{x}). 	\label{eq_objective_func}
\end{equation}
The objective function consists of the observation term $f^Z(\mathbf{x})$, the smoothing term $f^S(\mathbf{x})$, and the odometry term $f^O(\mathbf{x})$. $w_Z$, $w_S$ and $w_O$ are their corresponding weights, and we set $w_O = 0$ if there is no odometry information. We now explain the three terms one by one.

\subsubsection{Observation Term $f^Z(\mathbf{x})$}
Based on the relationship between the observations and the occupancy map w.r.t. occupancy values described in \ref{sec_relationship}, we can formulate the observation term as follows. 

Given the observation information $\mathbb{S}$ in (\ref{S_i}), the observation term in the objective function (\ref{eq_objective_func}) is formulated as
\begin{equation}
	f^Z(\mathbf{x}) =
	\sum_{i=0}^n \sum_{j=1}^{k_i}  \left\|z_{ij} - F_{ij}^Z(\mathbf{x})\right\|^2, 
\label{obs-term}
\end{equation}
where
\begin{equation}
	F_{ij}^Z(\mathbf{x})  = \frac{M(\mathbf{p}_{ij})}{N({\mathbf{p}_{ij}})}.\\ \label{eq_MN}
\end{equation}
Here, $\mathbf{p}_{ij}$ represents a coordinate in the map where the $j$th sampling scan point at time step $i$ is projected using the robot pose $\mathbf{x}^P_i$, as calculated by (\ref{P-project}). $N(\mathbf{p}_{ij})$ denotes the equivalent hit multiplier at $\mathbf{p}_{ij}$, as detailed in Section \ref{sec_hit}. 

In (\ref{obs-term}), we suppose the errors of occupancy values of different sampled points in the observations $\mathbb{S}$ are independent and have the same uncertainty. Therefore, the weights on all terms are the same, which is equivalent to setting all the weights as $1$. Thus, we use norms instead of weighted norms in equation (\ref{obs-term}).

\subsubsection{Odometry Term $f^O(\mathbf{x})$}\label{sec_odometry}

% Information\label{sec_odom_inputs}}
The odometry information $\mathbb{O} = \{\mathbf{o}_i\}_{1 \leq i \leq n}$ might be available. We assume the odometry input is the relative pose between two consecutive steps. 
The odometry from robot pose $\mathbf{x}^P_{i-1}$ to pose $\mathbf{x}^P_{i}$ is expressed as
\begin{equation} 
\mathbf{o}_i=\left[ (\mathbf{o}_i^t)^\top,o_i^\theta \right]^\top~~(1 \leq i \leq n)
\label{O_i}
\end{equation}
where $\mathbf{o}_i^t$ is the translation part and $o_i^\theta$ is the rotation angle part of the odometry. The odometry term can be formulated as
\begin{equation}
\begin{aligned}
f^O(\mathbf{x})&=\sum_{i=1}^n \left\|\mathbf{o}_i -
F_i^O(\mathbf{x})
\right\|^2_{\mathbf{\Sigma}^{-1}_{O_i}}
\\&=\sum_{i=1}^n\left\|
\begin{bmatrix}
\mathbf{o}_i^t-\mathbf{R}_{i-1}\left(\mathbf{t}_i - \mathbf{t}_{i-1} \right)\\
\wrap\left({o}_i^\theta- \theta_i + \theta_{i-1}\right)
\end{bmatrix}
\right\|^2_{\mathbf{\Sigma}^{-1}_{O_i}}  \label{eq_odometry_term}
\end{aligned}
\end{equation}
in which $\mathbf{\Sigma}_{O_i}$ is the covariance matrix representing the uncertainty of $\mathbf{o}_i$, and $\wrap(\cdot)$ wraparounds the rotation angle to $(-\pi,\pi]$.
\subsubsection{Smoothing Term $f^S(\mathbf{x})$}

It can be easily found out that minimizing the objective function with only the observation term (and the odometry term) is not easy since there are a large number of local minima. Especially when the initial robot poses are far away from the global minimum, it is very difficult for an optimizer to converge to the correct solution. 

In order to enlarge the region of attraction and develop an algorithm that is robust to initial values, we introduce a smoothing term. The smoothing term requires the occupancy values of nearby cell vertices to be close to each other thus resulting in the occupancy map being smoother for derivative calculation. In our case, based on the derivative calculation method we use (see Appendix \ref{Sec_J_P}), we penalize the difference between the occupancy value of each cell vertex and the occupancy values of the two neighboring cell vertices to its right and below, i.e.,
\begin{equation}
\begin{aligned}
f^S(\mathbf{x})
& =\left\|F^S(\mathbf{x}) \right\|^2\\
& = \sum_{w=0}^{c_w-1} \sum_{h=0}^{c_h-1}  \left\|\begin{bmatrix} M(\mathbf{m}_{wh})-M(\mathbf{m}_{{(w+1)}h})\\
M(\mathbf{m}_{wh})-M(\mathbf{m}_{{w}{(h+1)}})
\end{bmatrix} \right\|^2 \\
& + \sum_{h=0}^{c_h-1}  \left\| M(\mathbf{m}_{c_wh})-M(\mathbf{m}_{{c_w}{(h+1)}})\right\|^2 \\
& + \sum_{w=0}^{c_w-1}  \left\| M(\mathbf{m}_{wc_h})-M(\mathbf{m}_{{(w+1)}{c_h}})\right\|^2,
\end{aligned} \label{eq_smoothing_term}
\end{equation} where the second and third terms are used to handle cell vertices located in the bottom row and the rightmost column. It should be noted that $F^S(\mathbf{x})$ is a linear function of $\mathbf{x}^M$ in the state. The coefficient matrix is constant and can be calculated prior to the optimization. For more details, please refer to Appendix \ref{Sec_J_S}.


\section{Iterative Solution to the NLLS Formulation}\label{Sec_Algorithm_1}
In Section \ref{sec_formulation}, we introduced our NLLS formulation for the joint poses and occupancy map optimization problem. In this section, we provide the details of a Gauss-Newton based algorithm for solving the NLLS problem. 



\begin{algorithm}[t]
\small
\caption{Our Joint Poses and Occupancy Map Optimization Algorithm}\label{alg_1}
\SetKwInput{KwInput}{Input}                % Set the Input
\SetKwInput{KwOutput}{Output}              % set the Output
\SetKwInput{KwParam}{Params}
\SetAlgoLined
\DontPrintSemicolon
\SetKw{Return}{End Function}
  \KwParam{Threshold $\tau_k$, $\tau_{\Delta}$, weight matrix $\mathbf{W}$, resolution $s$}
  \KwInput{Observations $\mathbb{S}$, odometry $\mathbb{O}$, and initial poses $\mathbf{x}^P(0)$}
  \KwOutput{Optimized poses $\hat{\mathbf{x}}^P$ and optimized map $\hat{\mathbf{x}}^M$}
\SetKwFunction{FuncFirstStage}{FirstStage}

\SetKwProg{Fn}{Function}{:}{}
\Fn{\FuncFirstStage{$\mathbf{x}^P(0)$, $\mathbb{S}$, $\mathbb{O}$, $\tau_k$, $\tau_{\Delta}$, $s$, $\mathbf{W}$}}
{
Initialize $\mathbf{x}^M(0)$ and $\mathbb{N}(0)$ using $\mathbf{x}^P(0)$ and $\mathbb{S}$ \;

Pre-calculate smoothing term coefficient $\mathbf{A}$ using (\ref{eq_A})\;

\SetKwFunction{FuncGN}{OccupancyGN}
\SetKwFunction{FuncReturn}{return}

\SetKwProg{Fn}{Function}{:}{}
\For {$k=0$; $k <= \tau_k \; \& \; \| \mathbf{\Delta}(k) \|^2 >= \tau_{{\Delta}}$; $k++$}{
\Fn{\FuncGN{$\mathbf{x}^M(k)$, $\mathbb{N}(k)$, $\mathbf{x}^P(k)$, $\mathbb{S}$, $\mathbb{O}$, $\mathbf{A}$, $\mathbf{W}$}}{
Calculate gradient $\mathbf{\nabla} \mathbf{x}^M(k)$ of $\mathbf{x}^M(k)$

Calculate $\mathbf{J}$, as described in appendices

Evaluate $F(\mathbf{x})$ at $\mathbf{x}^P(k)$ and $\mathbf{x}^M(k)$

Solve $\mathbf{J}^\top \mathbf{W} \mathbf{J} \mathbf{\Delta}(k) =-\mathbf{J}^\top \mathbf{W} F(\mathbf{x})$, where $\mathbf{\Delta}(k) = {[{\mathbf{\Delta}^P(k)}^\top,{\mathbf{\Delta}^M(k)}^\top]}^\top$

Update $\mathbf{x}^P(k+1)=\mathbf{x}^P(k) + \mathbf{\Delta}^P(k)$ and $\mathbf{x}^M(k+1)=\mathbf{x}^M(k)+\mathbf{\Delta}^M(k)$

Recalculate $\mathbb{N}(k+1)$ using $\mathbf{x}^P(k+1)$ and $\mathbb{S}$

\FuncReturn{$\mathbf{x}^P(k+1)$, $\mathbf{x}^M(k+1)$}
}
\Return
}
$\hat{\mathbf{x}}^P \Leftarrow \mathbf{x}^P(k)$, $\hat{\mathbf{x}}^M \Leftarrow \mathbf{x}^M(k)$

\FuncReturn{$\hat{\mathbf{x}}^P$, $\hat{\mathbf{x}}^M$}
}

\Return
\end{algorithm}


In the equation below, we assume the odometry inputs are available. Let
\begin{equation}
\begin{aligned}
F(\mathbf{x}) = [&\cdots,z_{ij}-F_{ij}^Z(\mathbf{x}),\cdots,{(\mathbf{o}_i-F_i^O(\mathbf{x}))}^\top,\\
&\cdots,{F^S(\mathbf{x})}^\top]^\top\\
\mathbf{W} = \;\; &\diag(\cdots,w_Z,\cdots,w_O \mathbf{\Sigma}^{-1}_{O_i}, \cdots,w_S, \cdots)\\
\end{aligned}
\end{equation}
combine all the error functions and the weights of the three terms in (\ref{eq_objective_func}). Then, the NLLS problem in (\ref{eq_objective_func}) seeks $\mathbf{x}$ such that
\begin{equation}\label{Least Squares}
f(\mathbf{x})=\|F(\mathbf{x})\|^2_{\mathbf{W}} =
{F(\mathbf{x})}^\top \mathbf{W}
F(\mathbf{x})
\end{equation}
is minimized.

A solution to (\ref{Least Squares}) can be obtained iteratively by starting with an initial guess $\mathbf{x}(0)$ and updating with $\mathbf{x}(k+1) = \mathbf{x}(k) + \mathbf{\Delta}(k)$. \rule{0pt}{1em}The update vector $\mathbf{\Delta} (k) = [{\mathbf{\Delta}^P(k)}^\top,{\mathbf{\Delta}^M(k)}^\top]^\top$ is the solution to
\begin{equation}\label{Gauss-Newton}
\mathbf{J}^\top \mathbf{W} \mathbf{J} \mathbf{\Delta} (k) = -\mathbf{J}^\top \mathbf{W} F(\mathbf{x}(k))
\end{equation}
where $\mathbf{J}$ is the linear mapping represented by the Jacobian matrix
$\partial F / \partial \mathbf{x}$ evaluated at $\mathbf{x}(k)$.

The iterative method for solving the proposed NLLS problem is shown in Algorithm \ref{alg_1}, in which $\tau_k$ and $\tau_{\Delta}$ represent the thresholds of iteration number $k$ and the incremental vector $\mathbf{\Delta}$. Unlike the standard Gauss-Newton iterative method, the hit map needs to be additionally recalculated after updating the poses in each iteration. With this approach, the implicit data association is established at each iteration and updated during the optimization.

Since the robot poses and the occupancy map are optimized simultaneously, the Jacobian $\mathbf{J}$ in (\ref{Gauss-Newton}) is very important and quite different from those used in the traditional SLAM algorithms. More details of the Jacobians are described in appendices.




\section{Multi-resolution Joint Optimization Strategy} \label{Sec_multi}
Algorithm \ref{alg_1} provides a solution to our NLLS problem (\ref{eq_objective_func}) to jointly optimize the poses and the occupancy map. However, directly using Algorithm \ref{alg_1} with the high-resolution map is time-consuming and requires an accurate initial value of robot poses \cite{Zhao-RSS-22}, which is challenging to obtain. To overcome these limitations, we propose a multi-resolution joint optimization strategy in this section.

\subsection{Discussion on Map Resolution in Optimization }

The resolution of the occupancy map has a significant impact on the optimization results since $\mathbf{x}^M$ is part of the state vector in our NLLS formulation (\ref{eq_objective_func}). 

% Assuming the robot poses are accurate, a high-resolution map representation can establish accurate relationships between observations and occupancy values of projected points on the map, but it leads to a dramatic increase in the size of the optimization problem, which in turn leads to a significant increase in computational cost. In addition, in a high-resolution map, the occupancy values in nearby \textcolor{red}{cell vertices} can vary sharply, leading to noise-filled gradients in the map when poor initial robot poses are used. Even with the introduction of the smoothing term, the use of a high-resolution map may cause poor convergence of Algorithm \ref{alg_1}.

A high-resolution map enables precise relationships between observations and occupancy values of projected points. However, it results in a dramatic increase in the optimization problem's size, raising computational costs. Additionally, in a high-resolution map, occupancy values in adjacent cell vertices may exhibit sharper variations compared to those in a low-resolution map, leading to noisy gradients when poor initial robot poses are used. Even with the introduction of the smoothing term, the use of a high-resolution map may cause poor convergence of Algorithm \ref{alg_1}.



A low-resolution map provides advantages in faster computation and reduced memory usage. Moreover, gradients are less sensitive to pose accuracy. With our occupancy map representation and smoothing term, these advantages enable the algorithm to quickly converge to a reasonable solution, even with poor initial robot poses. However, low resolution may cause inaccurate links between observations and occupancy values near boundaries, preventing the optimization from achieving greater accuracy.

To combine the advantages of different resolution map representations, we propose a multi-resolution strategy to optimize the occupancy values of different resolution cell vertices together with robot poses at various stages. Unlike the conventional coarse-to-fine scheme, in the second stage of our strategy, we use the selected high-resolution map that only includes high-resolution cell vertices possibly in need of further optimization instead of the full high-resolution map. Optimizing only those selected high-resolution cell vertices further improves the efficiency of our algorithm. 
\subsection{Our Multi-resolution Joint Optimization Strategy}

Firstly, we obtain low-resolution observations $\mathbb{S}^{l} = \{\mathbb{S}_i^{l}\}_{0\leq i \leq n}$ by down-sampling from the high-resolution observations $\mathbb{S}^{h} = \{\mathbb{S}_i^{h}\}_{0\leq i \leq n}$, which are obtained by the equal sampling strategy described in Section \ref{Sec_Info_1} with a sampling distance $s^{h}$. Here, we set the map resolution and sampling resolution to be the same. Therefore, the low resolution $s^{l}=r \times s^{h}$, where $r$ is the resolution ratio between the low-resolution map and the high-resolution map. The size of the low-resolution map $\mathbb{M}^{l}$ is $(c_w+1) \times (c_h+1)$.

Initialized by the odometry inputs or scan matching, we perform Algorithm \ref{alg_1} to quickly obtain relatively accurate poses. The state vector in the first stage is ${\mathbf{x}}^{l} = {[{\mathbf{x}^P}^\top,{{\mathbf{x}^{lM}}}^\top]}^\top $, where $\mathbf{x}^{lM}$ includes all occupancy values at the cell vertices of the low-resolution map $\mathbb{M}^{l}$. In this stage, the hit map, observation information, coefficient matrix, weight matrix, and resolution are represented as $\mathbb{N}^{l},  \mathbb{S}^{l}, \mathbf{A}^{l}$, $\mathbf{W}^{l}$, and $s^{l}$, respectively. 

In the first stage of optimization, the low-resolution occupancy map reduces both the dimension of $\mathbf{x}^{lM}$ and the number of observations in $\mathbb{S}^{l}$. Since the occupancy values at cell vertices change relatively gradually in the low-resolution map, the directions of the map's gradients are closer to the correct ones when the poses are initialized by odometry inputs or scan matching, making it easier for Algorithm \ref{alg_1} to converge to a relatively good result quickly.

After the first stage, we use Algorithm \ref{alg_2} to select the cell vertices that need to be further optimized to compose the selected high-resolution map $\mathbb{M}^{s}$ and find their corresponding observations $\mathbb{S}^{s}$. Details are described in Section \ref{select_index_set}.


\begin{algorithm}[tp]
\small
\caption{Finding the Selected High-resolution Map and Corresponding Observations}\label{alg_2}

\SetKwInput{KwInput}{Input}                % Set the Input
\SetKwInput{KwOutput}{Output}              % set the Output
\SetKwInput{KwParam}{Params}
\SetKw{Return}{End Function}
\SetAlgoLined
\DontPrintSemicolon
\KwParam{Resolution $s^{h}$, selection distance $d$, convolution kernel size $q$}
  \KwInput{Observations $\mathbb{S}^{h}$, and poses ${{\hat{\mathbf{x}}}}^{\tilde{P}}$ from the first stage using Algorithm \ref{alg_1}}
  \KwOutput{Observations $\mathbb{S}^{s}$ and map part of the state vector in the second stage $\mathbf{x}^{sM}$}
  \SetKwFunction{FuncSel}{Selection}
\SetKwProg{Fn}{Function}{:}{}
\Fn{\FuncSel{${{\hat{\mathbf{x}}}}^{\tilde{P}}$, $\mathbb{S}^{h}$, $s^{h}$, $d$, $q$}}{

Build a full high-resolution map $\mathbb{M}^{h}$ using $\hat{\mathbf{x}}^{\tilde{P}}$ and $\mathbb{S}^{h}$ 

Calculate the binary map $\mathbb{B}$ using $\mathbb{M}^{h}$

Calculate the convoluted map $\mathbb{C}$ with kernel size $q$

Calculate the set $\mathbb{I}^{h}$, which includes the indices of all boundary vertices in $\mathbb{M}^{h}$, using $\mathbb{C}$

Calculate the set $\mathbb{I}^{s}$, which includes the indices of all selected vertices, using $\mathbb{I}^{h}$ and $d$

Define the map part of the state vector in the second stage $\mathbf{x}^{sM}$ and the selected high-resolution map $\mathbb{M}^{s}$ by $\mathbb{I}^{s}$

Find observations $\mathbb{S}^{s}$ for $\mathbb{M}^{s}$ using the set $\mathbb{I}^{s}$, $\mathbb{S}^{h}$ and ${{\hat{\mathbf{x}}}}^{\tilde{P}}$

\FuncReturn{$\mathbb{S}^{s}$, $\mathbf{x}^{sM}$}
}
\Return
\end{algorithm}


In the second stage, the state vector is represented as $\mathbf{x}^{s} = {[{{\mathbf{x}}^P}^\top,{\mathbf{x}^{sM})}^\top]}^\top $ where $\mathbf{x}^{sM}$ includes all occupancy values at cell vertices of the selected high-resolution map $\mathbb{M}^{s}$. We perform Algorithm \ref{alg_3} using poses obtained from the first stage as initial guesses and observations $\mathbb{S}^{s}$ to refine poses. The NLLS optimization problem in the second stage can be formulated similarly as (\ref{Least Squares}). Additionally, the differences in the Jacobian calculation between Algorithm \ref{alg_1} and Algorithm \ref{alg_3} are described in Appendix \ref{Sec_J_Select}. 

The full multi-resolution joint optimization strategy is outlined in Algorithm \ref{alg_flowchart}.

\begin{algorithm}[t]
\small
\caption{The Algorithm for the Second Stage of the Multi-resolution Joint Optimization Strategy}\label{alg_3}
\SetKwInput{KwParam}{Params}
\SetKwInput{KwInput}{Input}                % Set the Input
\SetKwInput{KwOutput}{Output}              % set the Output
\SetAlgoLined
\DontPrintSemicolon
\SetKw{Return}{End Function}

  \KwParam{Threshold $\tau_k^{s}$, $\tau_{\Delta}^{s}$, weight matrix $\mathbf{W}^{s}$, resolution $s^{h}$}
  \KwInput{Observations $\mathbb{S}^{s}$, odometry $\mathbb{O}$, and poses ${{\hat{\mathbf{x}}}}^{\tilde{P}}$ from the first stage using Algorithm \ref{alg_1}}
  \KwOutput{Optimal poses $\hat{\mathbf{x}}^P$ and map $\hat{\mathbf{x}}^{sM}$}
\SetKwFunction{FuncSecondStage}{SecondStage}

$\mathbf{x}^P(0) \Leftarrow {{\hat{\mathbf{x}}}}^{\tilde{P}}$

\SetKwProg{Fn}{Function}{:}{}
\Fn{\FuncSecondStage{$\mathbf{x}^P(0)$, $\mathbb{S}^{s}$, $\mathbb{O}$, $\tau_k^{s}$, $\tau_{\Delta}^{s}$, $s^{h}$, $\mathbf{W}^{s}$}}
{

Initialize $\mathbf{x}^{sM}(0)$ and $\mathbb{N}^{s}(0)$ using $\mathbb{S}^{s}$ and $\mathbf{x}^P(0)$

Pre-calculate smoothing term coefficient matrix $\mathbf{A}^{s}$

\For {$k=0$; $k <= \tau_k^{s} \; \& \; \| \mathbf{\Delta}(k) \|^2 >= \tau_{\Delta}^{s}$; $k++$}{

$\mathbf{x}^P(k+1)$, $\mathbf{x}^{sM}(k+1)$
$\leftarrow$  \FuncGN{$\mathbf{x}^{sM}(k)$, {$\mathbb{N}^{s}(k)$, $\mathbf{x}^P(k)$, $\mathbb{S}^{s}$, $\mathbb{O}$, $\mathbf{A}^{s}$, $\mathbf{W}^{s}$}}
}

$\hat{\mathbf{x}}^P \Leftarrow \mathbf{x}^P(k)$, $\hat{\mathbf{x}}^{sM} \Leftarrow \mathbf{x}^{sM}(k)$

\FuncReturn{$\hat{\mathbf{x}}^P$, $\hat{\mathbf{x}}^{sM}$}

}
\Return
\end{algorithm}




\begin{algorithm}[t]
\small
\caption{Our Multi-resolution Joint Optimization Strategy}\label{alg_flowchart}
\SetKwInput{KwParam}{Params}
\SetKwInput{KwInput}{Input}                % Set the Input
\SetKwInput{KwOutput}{Output}              % set the Output

\SetAlgoLined
\DontPrintSemicolon

\SetKwFunction{FuncDown}{DownSampling}
\SetKwFunction{FuncInitPose}{InitializePose}
\SetKwFunction{FuncSM}{ScanMatching}
  \KwParam{Threshold $\tau_k^{l}$, $\tau_{\Delta}^{l}$, $\tau_k^{s}$, $\tau_{\Delta}^{s}$, weight matrix $\mathbf{W}^{l}$, $\mathbf{W}^{s}$, ratio of resolutions $r$, resolution $s$, selection distance $d$, convolution kernel size $q$}
  \KwInput{Observations $\mathbb{S}^{h}$, odometry $\mathbb{O}$}
  \KwOutput{Optimal poses $\hat{\mathbf{x}}^P$ and map $\hat{\mathbf{x}}^{sM}$}

$\mathbb{S}^{l}$ $\leftarrow$ \FuncDown{$\mathbb{S}^{h}$, $r$}

\uIf {$w_O \neq  0$}
{
$\mathbf{x}^P(0)$ $\leftarrow$ \FuncInitPose{$\mathbb{O}$}
}
\Else
{
$\mathbf{x}^P(0)$ $\leftarrow$ \FuncSM{$\mathbb{S}^{l}$}
}


${{\hat{\mathbf{x}}}}^{\tilde{P}}$, $\hat{\mathbf{x}}^{lM}$ $\leftarrow$ \FuncFirstStage{$\mathbf{x}^P(0)$, $\mathbb{S}^{l}$, $\mathbb{O}$, $\tau_k^{l}$, $\tau_{\Delta}^{l}$, $s^{l}$, $\mathbf{W}^{l}$}

$\mathbb{S}^{s}$, $\mathbf{x}^{sM}$ $\leftarrow$ \FuncSel{${{\hat{\mathbf{x}}}}^{\tilde{P}}$, $\mathbb{S}^{h}$, $s^{h}$, $d$, $q$}

${\hat{\mathbf{x}}^P}$, $\hat{\mathbf{x}}^{sM}$ $\leftarrow$
\FuncSecondStage{${{\hat{\mathbf{x}}}}^{\tilde{P}}$, $\mathbb{S}^{s}$,$\mathbb{O}$, $\tau_k^{s}$, $\tau_{\Delta}^{s}$, $s^{h}$, $\mathbf{W}^{s}$}

\end{algorithm}

\begin{figure}[t]
\centering 
\subfigure[Full High-resolution Map]{ 
\includegraphics[width=0.23\textwidth]{./high_resolution_select.pdf}}
\subfigure[Selected High-resolution Map (In White and Black)]{\label{fig_select_example_b}
\includegraphics[width=0.23\textwidth]{./recolor_select.pdf}}
\caption{An example of the selected high-resolution map from a full high-resolution map in a simulation dataset. (a) The full high-resolution map generated using poses from the first-stage optimization and scans, forming the basis for selection. (b) The recolored selected high-resolution map: gray marks dropped (stable) areas, white and black denote selected areas, with black highlighting obstacle boundaries.}
\label{fig_select_example}
% \vspace{-1em}
\end{figure}

\subsection{Selected High-resolution Map and Observations}\label{select_index_set}

After the first stage optimization using the low-resolution map $\mathbb{M}^{l}$, the robot poses $\hat{\mathbf{x}}^{\tilde{P}}$ become relatively accurate. Subsequently, the full high-resolution map $\mathbb{M}^{h}$, with dimensions $(r*c_w+1) \times (r*c_h+1)$, is built using the Bayesian occupancy mapping method \cite{ProbabilisticRobotics}, based on observations $\mathbb{S}^{h}$ and poses ${{\hat{\mathbf{x}}}}^{\tilde{P}}$. In this case, most cell vertices of $\mathbb{M}^{h}$ are considered stable in terms of occupancy state. Semantically, these stable cell vertices have the same occupancy state as the surrounding cell vertices (typically free or unknown cells). This characteristic leads to map gradients near zero at these stable cell vertices. In contrast, the cell vertices that require further updates are typically located at the edges of objects, where the occupancy values significantly differ from those of surrounding cell vertices. Therefore, the gradient at these cell vertices is larger. An example illustrating this is shown in Fig. \ref{fig_select_example_b}, where the selected area (in white and black) is clearly distinct from the stable area (in gray). Based on this idea, we propose a strategy to select the cell vertices located around the boundaries to compose the selected high-resolution map $\mathbb{M}^{s}$, which is used in the second stage of optimization.


\begin{figure}[t]
\centering 
\includegraphics[width=0.48\textwidth]{./Select_Set.pdf}
\caption{\label{fig_low_high_select} An illustration of the cell vertices selection strategy and a selected high-resolution map from a simulation dataset. In (a), selected cell vertices are marked in red and yellow, with their indices forming the index set $\mathbb{I}^{s}$.}
% \vspace{-0.5em}
\end{figure}

% In both figures, the boundary cells are indicated as black color, the selected cells are shown as black and white, and the dropped cells are colored in gray.

Firstly, we identify cell vertices located at the edges of objects by performing mean-value convolution of the full high-resolution map $\mathbb{M}^{h}$. Specifically, we calculate a binary map $\mathbb{B}=\{B(\mathbf{m}_{id})\}$ by binarizing $\mathbb{M}^{h}$ as
\begin{equation}
B(\mathbf{m}_{id}) =	\begin{cases}
	1, & {M}^{h}(\mathbf{m}_{id}) \geq \tau_{occupied} \\
	0, & {M}^{h}(\mathbf{m}_{id}) < \tau_{occupied} \\
\end{cases},
\end{equation}
where $\mathbf{m}_{id}$ represents a cell vertex, and $\tau_{occupied}$ is the threshold used to classify a cell vertex as occupied or free. A mean-value convolution kernel $\mathbf{K}$ is defined as
\begin{equation}
	\mathbf{K} = \dfrac{1}{q^2} \cdot \bold{1}_{q\times q}
\end{equation}
where $\bold{1}_{q\times q}$ represents a $q \times q$ matrix of ones. The convoluted map $\mathbb{C}=\{C(\mathbf{m}_{id})\}$ is then derived by convolving $\mathbb{B}$ with $\mathbf{K}$, where $C(\mathbf{m}_{id})$ indicates whether the $q \times q$ cell vertices around $\mathbf{m}_{id}$ are all in the same occupancy state. Compared to other edge detection methods like Sobel \cite{duda1973pattern} and Canny \cite{canny1986computational}, this conservative method more reliably selects cell vertices that may require further optimization. 

Using this method, the set of indices for all boundary cell vertices in the high-resolution map is defined as 

\begin{equation}
\begin{aligned}
	\mathbb{I}^{h} = \{id | 0<C(\mathbf{m}_{id})<1 \}.
\end{aligned}
\end{equation}
The cell vertices indexed in $\mathbb{I}^{h}$ are marked in red in Fig. \ref{fig_low_high_select}(a). 


To account for pose uncertainties from the first stage, the selection is expanded to include cell vertices within a distance $d$ from all boundary cell vertices. The indices of the selected cell vertices in the high-resolution map form the set $\mathbb{I}^{s}$, illustrated in Fig. \ref{fig_low_high_select}(a), where the selected cell vertices are highlighted in red and yellow with $d=1$. An example of a selected high-resolution map from a simulation dataset is shown in Fig. \ref{fig_low_high_select}(b).

Consequently, the map component of the state vector in the second stage is expressed as
\begin{equation}
	\mathbf{x}^{sM} = {[\cdots, M^{h}(\mathbf{m}_{wh}), \cdots]}^\top, ~~ wh\in \mathbb{I}^{s}.
\end{equation}

% The cell selection strategy with a selection distance of 1 cell is illustrated in Fig. \ref{fig_low_high_select}(a), where selected \textcolor{red}{cell vertices} are marked with red dots and their corresponding selected cells are marked with black and white. 


Next, we select observations to optimize $\mathbf{x}^{sM}$. Cells surrounded by vertices with indices in $\mathbb{I}^s$ are designated as selected cells, shown in white in Fig. \ref{fig_low_high_select}(a) and Fig. \ref{fig_low_high_select}(b). Subsequently, sampling point selection is carried out, as illustrated in Fig. \ref{fig_select_sampling_point}. Specifically, sampling points in $\mathbb{S}^{h}$ are first projected onto the global coordinate system using the poses optimized in the first stage. All sampling points located on the selected cells are then included to form the set $\mathbb{S}^{s}$. 

\begin{figure}[t]
\centering 
\includegraphics[width=0.48\textwidth]{./Select_Sampling_Point.pdf}
\caption{\label{fig_select_sampling_point} An example of the selected sampling points of a beam at time step $i$, where points projected onto the selected cells are chosen.}
% \vspace{-0.5em}
\end{figure}


\section{Submap Joining} \label{Sec_submap}
In Section \ref{Sec_multi}, we introduced a multi-resolution joint optimization strategy to efficiently solve our NLLS problem. For large-scale occupancy SLAM with long robot trajectories, the number of poses to optimize can be very large. To make the computational complexity dependent only on the environment size rather than the trajectory length, in this section we propose an occupancy submap joining method. The key idea is to reduce the number of poses that need to be optimized to the number of local submaps. 


\subsection{Inputs and Outputs of Submap Joining Problem} 
We first separate the observation information into multiple parts and perform Algorithm \ref{alg_flowchart} to build several submaps. The inputs of submap joining problem are a sequence of local occupancy submaps. 
Let us denote the $n_L+1$ submaps as $\mathbb{M}_L = \{\mathbb{M}_{L_0}, \cdots, \mathbb{M}_{L_{n_L}}\}$ and the associated coordinate frames of these local occupancy maps are denoted as $ \{\mathbf{x}^P_{0}, \cdots, \mathbf{x}^P_{n_L}\}$, where $\mathbb{M}_{L_{i_L}}$ and $\mathbf{x}^P_{i_L}$represents the $i_L$th local occupancy map and its associated coordinate frame. In addition, the global occupancy map is represented as $\mathbb{M}_G=\{M_G(\mathbf{m}^G_{00}), \cdots,{M}_G\left(\mathbf{m}^G_{c_w^Gc_h^G}\right)\}$. Both the global map and local maps follow the same definition as described in Section \ref{sec_discrete_occupancy}. The outputs of submap joining problem are the optimal solution of the local submap coordinate frames and the optimal global occupancy map.

\subsection{NLLS Formulation of Submap Joining Problem} 
First, the cell vertex $\mathbf{m}^G_{wh}$ in the global occupancy map $\mathbb{M}_G$ can be projected to local submap coordinate by pose $\mathbf{x}^P_{i_L}$, i.e., 
\begin{equation}
	\mathbf{p}_{i_L}^{wh} = \frac{ \mathbf{R}_{i_L} (\mathbf{m}^G_{wh} \cdot s_G  - \mathbf{t}_{i_L})}{s_L}.
\end{equation}
Here, $\mathbf{p}_{i_L}^{wh}$ represents the position in the local submap's coordinate where the cell vertex $\mathbf{m}_{wh}^G$ from the global map is projected using the pose $\mathbf{x}^P_{i_L}$. The resolutions of the global occupancy map and local submaps are denoted by $s_G$ and $s_L$, respectively.

The submap joining problem aims to find the optimal global occupancy map and the poses of submap coordinate frames. Thus, the state vector for this problem is defined as 
\begin{equation}
	\mathbf{x}_G = [{\mathbf{x}^P_L}^\top, {{\mathbf{x}^M_G}}^\top]^\top,
\end{equation}
where 
\begin{equation}
\begin{aligned}
\mathbf{x}^P_L & =\left[\left(\mathbf{x}^P_1\right)^\top, \cdots,\left(\mathbf{x}^P_{n_L}\right)^\top\right]^\top \\
\mathbf{x}^M_G & =\left[{M}_G\left(\mathbf{m}^G_{00}\right), \cdots, {M}_G\left(\mathbf{m}^G_{c_w^Gc_h^G}\right)\right]^\top.
\end{aligned}
\end{equation}
As with most submap joining problem formulations, we fix the first local map coordinate frame as the global coordinate frame. Therefore, $\mathbf{x}^P_L$ consists of $n_L$ local map coordinate frames and $\mathbf{x}^M_G$ includes $(c_w^G+1) \times (c_h^G+1)$ discrete cell vertices of global occupancy map. 

By the global-to-local projection relationship, all cell vertices of global occupancy map $\mathbb{M}_G$ can be projected to corresponding submaps to compute the difference in occupancy values. Thus, the NLLS problem of occupancy submap joining can be formulated to minimize 
\begin{equation}
\begin{adjustbox}{max width=\linewidth}
$
g(\mathbf{x}_G) =  \sum\limits_{i_L=0}^n\sum\limits_{ wh \in \mathbb{S}^L_{i_L}} \left\| \omega(i_L,\mathbf{m}^G_{wh}) {M}_G(\mathbf{m}^G_{wh}) - {M}_{L_{i_L}}(\mathbf{p}_{i_L}^{wh}) \right\|^2,
$
\end{adjustbox}
\label{eq_NLLS_joining}
\end{equation}
where $\mathbb{S}^L_{i_L}$ represents the set of indices of cell vertices in the global occupancy map $\mathbb{M}_G$ that are projected onto the local submap $\mathbb{M}_{L_{i_L}}$.

In (\ref{eq_NLLS_joining}), $\omega(i_L,\mathbf{m}^G_{wh})$ is the weight to establish an accurate relationship between the global occupancy map and local submaps w.r.t. occupancy values, which can be calculated by
\begin{equation}
    \omega(i_L,\mathbf{m}^G_{wh}) = \frac{{N}_{{L_{i_L}}}(\mathbf{p}_{i_L}^{wh})}{{N}_{G}(\mathbf{m}^G_{wh})}.
\end{equation}
Here, ${N}_{{L_{i_L}}}(\cdot)$ is the local hit number lookup function for submap $\mathbb{M}_{L_{i_L}}$, derived as described in Section \ref{sec_hit}. It approximates the hit number at coordinate $\mathbf{p}_{i_L}^{wh}$ using bilinear interpolation. Similarly, ${N}_{G}(\cdot)$ represents the global hit number lookup function associated with $\mathbb{M}_G$.

In (\ref{eq_NLLS_joining}), the submap joining problem is formulated as a NLLS problem, which can be solved iteratively by Gauss-Newton based method similar to Algorithm \ref{alg_1}.


 \begin{table}[htp]
		\centering
		\caption{Parameters of Datasets. \label{tab_dataset}}
		\label{tab_comparison}
		\setlength{\tabcolsep}{0.7mm}{
		\begin{tabular}{lccccc}\toprule
		Dataset	& No. Scans & Duration  & Map Size &  Odometry & Resolution\\ \hline
		Simulation 1 & 3640  &117 s& $50$ m  $\times$ $50$ m & yes & 0.05 m\\
        Simulation 2 & 3720  &121 s& $50$ m $\times$ $50$ m & yes & 0.05 m\\
		Simulation 3  & 2680  & 83 s& $50$ m $\times$ $50$ m & yes & 0.05 m\\
		Car Park  & 1642 & 164 s& $50$ m $\times$ $40$ m & yes & 0.1 m\\
		C5  & 3870  &136 s& $50$ m $\times$ $40$ m & yes & 0.1 m\\
		Museum b0 & 5522 &152 s& $85$ m $\times$ $95$ m &no & 0.1 m \\
		Museum b2 & 51833 &1390 s &  $250$ m $\times$ $200$ m &no & 0.1 m\\
        C3 &24402 &610 s& $150$ m $\times$ $125$ m  & no & 0.1 m\\
		\hline
		\end{tabular}
		}
\end{table}


\section{Experimental Results} \label{Sec_experiment}

In this section, we evaluate our algorithm on several datasets and compare its performance with Cartographer \cite{hess2016real}, the current state-of-the-art algorithm, which significantly outperforms other methods such as Hector-SLAM \cite{kohlbrecher2011flexible} and Karto-SLAM \cite{konolige2010efficient}. To ensure fair comparisons, we adjust some parameters in Cartographer based on the sensor configurations of the respective datasets for optimal performance.


The dataset parameters are summarized in Table \ref{tab_dataset}. For practical datasets, Deutsches Museum b0 and Deutsches Museum b2 are Cartographer demo datasets collected at the Deutsches Museum. The Car Park \cite{zhao20212d} dataset is gathered in an underground car park, while C5 and C3 are collected in a factory environment using a Hokuyo UTM-30LX laser scanner. Consistent map resolutions $s$ are applied across all methods to display the map results, with ratio $r$ set to $10$ for all simulation experiments and $5$ for all practical experiments unless stated otherwise. For each dataset, 20\% of scans and corresponding poses are uniformly selected as key frames for the key frame option in our method.

To ensure fair comparisons, we use an identical number of poses (synchronizing the poses from the results with the ground truth poses using timestamps) and their corresponding observations to generate results for visualization and quantitative evaluation across all compared methods, with the exception of our method that employs keyframes. Furthermore, the same occupancy mapping algorithm is applied consistently across all approaches to produce the occupancy grid map results for comparison.

        
\subsection{Simulation Experiments}\label{simu_experiment}

\begin{figure*}[tp]
\centering \subfigure[Simulation 1] {\label{fig_trajectory_1}
\includegraphics[width=0.28\textwidth]{./trajectory_simu1_new.pdf}}
\centering \subfigure[Simulation 2] {\label{fig_trajectory_2}
\includegraphics[width=0.28\textwidth]{./trajectory_simu2_new.pdf}}
\centering \subfigure[Simulation 3] {\label{fig_trajectory_3}
\includegraphics[width=0.343\textwidth]{./trajectory_simu3_new.pdf}}
\caption{\label{fig_trajectory_compare}Simulation environments and robot trajectory results. (a), (b) and (c) show the simulation environments (the black lines indicate the obstacles in the scene) and the trajectories of ground truth, odometry inputs, Cartographer \cite{hess2016real}, and our approach for one dataset in each of the three simulation environments.}
% \vspace{-0.5em}
\end{figure*}

We use three different simulation environments with varying levels of nonlinearity and nonconvex obstacles to design three different simulation experiments. Since Cartographer needs a high-frequency scanning rate to ensure the good performance of scan matching, while our approach performs well for scan data with low scanning frequency, only 10\% scans listed in Table \ref{tab_dataset} are used in our method. 

We utilize the open-source 2D LiDAR simulator from \cite{zhao20212d} to generate simulated datasets. Each scan includes 1081 laser beams spanning angles from -135 degrees to 135 degrees, mimicking the specifications of a Hokuyo UTM-30LX laser scanner. To emulate real-world data acquisition, random Gaussian noise with zero mean and standard deviation of $0.02$ m is added to each beam of the simulated scan data. Similarly, zero-mean Gaussian noise is introduced to the odometry inputs derived from the ground truth poses, with standard deviation of $0.04$ m for $x$-$y$ and $0.003$ rad for orientation. Five datasets with different noise realizations are generated for each simulation environment.


\begin{figure*}[t]
\centering \subfigure[Simulation 1] {\label{fig_time_error_1}
\includegraphics[width=0.32\textwidth]{./Time_with_Error_Simu1.pdf}}
\centering \subfigure[Simulation 2] {\label{fig_time_error_2}
\includegraphics[width=0.32\textwidth]{./Time_with_Error_Simu2.pdf}}
\centering \subfigure[Simulation 3] {\label{fig_time_error_3}
\includegraphics[width=0.32\textwidth]{./Time_with_Error_Simu3.pdf}}
\caption{Comparison of translation and rotation errors at different time steps using simulation datasets.}
\label{fig_error_compare_time}
\vspace{-1em}
\end{figure*}


% The robot trajectory results of our method and Cartographer using one dataset in each simulation are compared with the ground truth and odometry in Fig. \ref{fig_trajectory_compare}. It is clear that our trajectories are closer to the ground truth trajectories, especially for positions where significant rotation occurs. Fig. \ref{fig_error_compare_time} shows the translation and rotation errors of our method and Cartographer at different time steps. Obviously, the errors of our method are substantially smaller than those of Cartographer.

The trajectory results of our method and Cartographer, compared to ground truth and odometry, are shown in Fig. \ref{fig_trajectory_compare}. It is evident that our trajectories align more closely with the ground truth, particularly in areas with significant rotational movements. Fig. \ref{fig_error_compare_time} illustrates translation and rotation errors over time, demonstrating that our method consistently achieves substantially smaller errors compared to Cartographer.

% We use all the fifteen datasets from Simulation 1, Simulation 2 and Simulation 3 to perform the quantitative and qualitative comparison of errors in the pose estimates. The quantitative results of Cartographer, only the first stage in our method \textcolor{red}{(Algorithm \ref{alg_1} using low-resolution)} using all frames, our method using all frames, and our method using key frames are given in Table \ref{tab_comparison}. We use mean absolute error (MAE) and root mean squared error (RMSE) to evaluate the translation errors (in meters) and rotation errors (in radians). Our method performs the best in all four metrics for all simulations and is substantially ahead of Cartographer even when using only key frames or only the first stage. In addition, Fig. \ref{fig_simulation}(a) to Fig. \ref{fig_simulation}(e) show the occupancy grid maps and point cloud maps generated using poses from ground truth, Cartographer and the three options of our method. It is clear that the boundaries of both occupancy grid maps and point cloud maps using the three options of our method are much clearer than those from Cartographer, which indicates that our method can obtain more accurate results by optimizing the robot poses and the occupancy map together.

We performed quantitative and qualitative comparisons of pose estimation errors using all fifteen datasets from Simulations 1, 2, and 3. Table \ref{tab_comparison} presents, in order, the quantitative results for odometry inputs, Cartographer, the first stage of our method (Algorithm \ref{alg_1} with low-resolution) using all frames, our method using all frames, and our method using key frames. Metrics such as mean absolute error (MAE) and root mean squared error (RMSE) evaluate translation errors (in meters) and rotation errors (in radians). Our method consistently achieves the best performance across all metrics, significantly outperforming Cartographer even when using only key frames or the first stage. Fig. \ref{fig_simulation}(a) to Fig. \ref{fig_simulation}(e) further illustrates occupancy grid maps and point cloud maps generated using poses from the ground truth, Cartographer, and the three options of our method. The maps produced by our method exhibit noticeably clearer boundaries, demonstrating its ability to jointly optimize robot poses and occupancy maps for more accurate results.

\begin{figure}[t]
\centering
\includegraphics[width=0.48\textwidth]{./Simulation_New.pdf}
\caption{\label{fig_simulation} The occupancy grid maps and point cloud maps generated from ground truth poses and different approaches for each simulation dataset. The areas marked with red dots highlight where our method outperforms the results of the first-stage optimization alone.}
\vspace{-0.5em}
\end{figure}

\begin{table}[t]
		\centering
		\caption{Quantitative Comparison of Robot Pose Errors in Simulations.}
		\label{tab_comparison}
		\setlength{\tabcolsep}{0.7 mm}{
		\begin{tabular}{lccccc}\toprule
			& Odom & Carto & Ours (First) & Ours (All) & Ours (Key) \\ \hline
		Simulation 1& & & &\\
		\quad MAE / Trans (m) & 0.78270 & 0.25336  & 0.02206 &\textcolor{red}{\textbf{0.00640}} & \textcolor{blue}{\textbf{0.01024}}\\
		\quad MAE / Rot (rad) & 0.04912 & 0.01394  & 0.00098 &\textcolor{red}{\textbf{0.00060}} & \textcolor{blue}{\textbf{0.00084}}
\\
		\quad RMSE / Trans (m) & 0.98404 & 0.29920  & 0.02680 &\textcolor{red}{\textbf{0.00974}} & \textcolor{blue}{\textbf{0.01430}}\\
		\quad RMSE / Rot(rad) & 0.05506 & 0.01562  & 0.00162 &\textcolor{red}{\textbf{0.00102}} & \textcolor{blue}{\textbf{0.00126}}\\\hline
		
		Simulation 2& & & &\\
		\quad MAE / Trans (m) & 0.80544 & 0.11914 &0.03224 & \textcolor{red}{\textbf{0.00858}} & \textcolor{blue}{\textbf{0.01082}}
\\
		\quad MAE / Rot (rad) & 0.02538 & 0.00666  &0.00220 &\textcolor{red}{\textbf{0.00062}} & \textcolor{blue}{\textbf{0.00096}}\\
		\quad RMSE / Trans (m) & 0.97152 & 0.14810  & 0.04188 &\textcolor{red}{\textbf{0.01198}} & \textcolor{blue}{\textbf{0.01244}}\\
		\quad RMSE / Rot (rad) & 0.02936 & 0.00916 &0.00220 & \textcolor{red}{\textbf{0.00104}} & \textcolor{blue}{\textbf{0.00178}}\\\hline
		
		Simulation 3& & & &\\
		\quad MAE / Trans(m) & 0.75352 & 0.14262  &0.02624 &\textcolor{red}{\textbf{0.00726}} &  \textcolor{blue}{\textbf{0.00998}}\\
		\quad MAE / Rot (rad) & 0.05180 & 0.00682 &0.00164  & \textcolor{red}{\textbf{0.00058}} &  \textcolor{blue}{\textbf{0.00090}}\\
		\quad RMSE / Trans (m) & 0.96866 & 0.18782 & 0.03238 &\textcolor{red}{\textbf{0.00952}} &  \textcolor{blue}{\textbf{0.01338}}\\
		\quad RMSE / Rot (rad) & 0.05926 & 0.00914  & 0.00204 &\textcolor{red}{\textbf{0.00088}} &  \textcolor{blue}{\textbf{0.00134}}\\\hline
		\end{tabular}
	\begin{tablenotes}
     \item \textcolor{red}{\textbf{Red}} and  \textcolor{blue}{\textbf{blue}} indicate the best and second best results, respectively.
   \end{tablenotes}
		}
        % \vspace{-2em}
\end{table}

% \begin{table*}[htp]
% \centering
% \caption{Occupancy Grid map Precision of Our Method Using All Frames, Our Method Using Key Frames, and Cartographer.}
% \label{tab_map_accuracy}
% \setlength{\tabcolsep}{2.4mm}
% \begin{NiceTabular}{cccccccccccc}[first-row,first-col,hvlines]
% \CodeBefore
% \Body
%  & \Block{1-11}{\textbf{Predicted}} & & & & & & &  & & & \\
% \Block{11-1}{\rotate Ground Truth}  & \Block{2-1}{}  & \Block{2-1}{} & \Block{1-3}{Our Method (All Frames)} & & & \Block{1-3}{Our Method (Key Frames)} & & & \Block{1-3}{Cartographer}  \\
%  & &   & Unknown & Free & Occupied & Unknown & Free & Occupied & Unknown & Free & Occupied \\
%  & \Block{3-1}{Simulation 1} & Unknown  & \textcolor{red}{\textbf{99.798\%}}  & 0.020\% & 0.182\% & \textcolor{blue}{\textbf{99.598\%}} & 0.072\% & 0.330\% & 95.616\% & 2.822\% & 1.562\% \\
%  &   &  Free & 0.022\%  & \textcolor{red}{\textbf{99.938\%}} & 0.040\% & 0.094\% & \textcolor{blue}{\textbf{99.824\%}} & 0.082\% & 1.290\% & 97.678\% & 1.032\% \\
%  &   & Occupied  &  5.436\% & 2.334\%  & \textcolor{red}{\textbf{92.230\%}} & 13.562\% & 3.142\% &\textcolor{blue}{\textbf{83.296\%}} &30.053\% & 53.280\% & 16.667\% \\

% & \Block{3-1}{Simulation 2} & Unknown  & \textcolor{red}{\textbf{99.846\%}}  & 0.010\% & 0.144\% & \textcolor{blue}{\textbf{99.696\%}} & 0.062\% & 0.242\% & 96.868\% & 1.743\% & 1.389\% \\
%  &   &  Free & 0.016\%  & \textcolor{red}{\textbf{99.846\%}} & 0.138\% & 0.076\% & \textcolor{blue}{\textbf{99.848\%}} & 0.076\% & 0.593\% & 98.584\% & 0.823\% \\
%  &   & Occupied  &  6.434\% & 3.738\%  & \textcolor{red}{\textbf{89.828\%}} & 11.694\% & 2.806\% &\textcolor{blue}{\textbf{85.500\%}} &23.943\% & 50.795\% & 25.262\% \\

%  & \Block{3-1}{Simulation 3} & Unknown  & \textcolor{red}{\textbf{99.812\%}}  & 0.032\% & 0.156\% & \textcolor{blue}{\textbf{99.258\%}} & 0.430\% & 0.312\% & 96.968\% & 1.574\% & 1.458\% \\
%  &   &  Free & 0.036\%  & \textcolor{red}{\textbf{99.928\%}} & 0.036\% & 0.554\% & \textcolor{blue}{\textbf{99.352\%}} & 0.094\% & 1.018\% & 98.110\% & 0.872\% \\
%  &   & Occupied  &  4.500\% & 2.358\%  & \textcolor{red}{\textbf{93.142\%}} & 16.788\% & 3.736\% &\textcolor{blue}{\textbf{79.476\%}} &26.420\% & 44.928\% & 28.652\% \\
% \end{NiceTabular}
% % \vspace{-1em}
% \end{table*}

\begin{table*}[htp]
\centering
\caption{Occupancy Grid map Precision of Our Method Using All Frames, Our Method Using Key Frames, and Cartographer.}
\label{tab_map_accuracy}
\setlength{\tabcolsep}{2.4mm}
\renewcommand{\arraystretch}{1.2}

\begin{tabular}{c c c c c c c c c c c c}
\toprule
 \multirow{2}{*}{} & \multirow{2}{*}{\textbf{Ground Truth}}& \multicolumn{3}{c}{Our Method (All Frames)} & \multicolumn{3}{c}{Our Method (Key Frames)} & \multicolumn{3}{c}{Cartographer} \\
\cmidrule(lr){3-5} \cmidrule(lr){6-8} \cmidrule(lr){9-11}
& & Unknown & Free & Occupied & Unknown & Free & Occupied & Unknown & Free & Occupied \\
\midrule
\multirow{3}{*}{Simulation 1} 
& Unknown  & \textcolor{red}{\textbf{99.798\%}}  & 0.020\% & 0.182\% & \textcolor{blue}{\textbf{99.598\%}} & 0.072\% & 0.330\% & 95.616\% & 2.822\% & 1.562\% \\
& Free     & 0.022\%  & \textcolor{red}{\textbf{99.938\%}} & 0.040\% & 0.094\% & \textcolor{blue}{\textbf{99.824\%}} & 0.082\% & 1.290\% & 97.678\% & 1.032\% \\
& Occupied & 5.436\%  & 2.334\%  & \textcolor{red}{\textbf{92.230\%}} & 13.562\% & 3.142\% & \textcolor{blue}{\textbf{83.296\%}} & 30.053\% & 53.280\% & 16.667\% \\
\midrule
\multirow{3}{*}{Simulation 2} 
& Unknown  & \textcolor{red}{\textbf{99.846\%}}  & 0.010\% & 0.144\% & \textcolor{blue}{\textbf{99.696\%}} & 0.062\% & 0.242\% & 96.868\% & 1.743\% & 1.389\% \\
& Free     & 0.016\%  & \textcolor{red}{\textbf{99.846\%}} & 0.138\% & 0.076\% & \textcolor{blue}{\textbf{99.848\%}} & 0.076\% & 0.593\% & 98.584\% & 0.823\% \\
& Occupied & 6.434\%  & 3.738\%  & \textcolor{red}{\textbf{89.828\%}} & 11.694\% & 2.806\% & \textcolor{blue}{\textbf{85.500\%}} & 23.943\% & 50.795\% & 25.262\% \\
\midrule
\multirow{3}{*}{Simulation 3} 
& Unknown  & \textcolor{red}{\textbf{99.812\%}}  & 0.032\% & 0.156\% & \textcolor{blue}{\textbf{99.258\%}} & 0.430\% & 0.312\% & 96.968\% & 1.574\% & 1.458\% \\
& Free     & 0.036\%  & \textcolor{red}{\textbf{99.928\%}} & 0.036\% & 0.554\% & \textcolor{blue}{\textbf{99.352\%}} & 0.094\% & 1.018\% & 98.110\% & 0.872\% \\
& Occupied & 4.500\%  & 2.358\%  & \textcolor{red}{\textbf{93.142\%}} & 16.788\% & 3.736\% & \textcolor{blue}{\textbf{79.476\%}} & 26.420\% & 44.928\% & 28.652\% \\
\bottomrule
\end{tabular}
\end{table*}


From the results of our first stage shown in Fig. \ref{fig_simulation}(c), it is evident that further optimization is needed at the edges of objects. This is due to sampling points with different occupancy values being projected onto the coarse grid cells at object boundaries, causing inaccurate data associations. These results highlight the necessity of the second stage in our multi-resolution strategy (Algorithm \ref{alg_3}) to improve accuracy. Additionally, as shown in Fig. \ref{fig_simulation}(c), the non-edge areas (stable areas) of the occupancy grid maps are well optimized, supporting the fact that including occupancy cell vertices of non-edge areas in the state variables for further optimization is unnecessary.

For a quantitative comparison of the occupancy maps, we apply the same threshold across all methods to convert occupancy values into occupancy states. The mapping problem is treated as a classification task, categorizing each grid cell as free, occupied, or unknown. The mapping performance of our method and Cartographer is summarized in Table \ref{tab_map_accuracy}, clearly showing that both variants of our method, one using all frames and the other using keyframes, significantly outperform Cartographer in terms of map accuracy.

\begin{table}[ht]
		\centering
		\caption{Accuracy of the Occupancy Grid Map.}
		\label{tab_auc}
		\setlength{\tabcolsep}{4.8 mm}{
		\begin{tabular}{llccccc}\toprule
		& & AUC & Precision   \\ \hline
		\multirow{3}{*}{Simulation 1}& Cartographer & 0.90878 & 0.95651 \\ & Ours (All) &\textcolor{red}{\textbf{0.99999}} &\textcolor{red}{\textbf{0.99773}}\\ & Ours (Key) & \textcolor{blue}{\textbf{0.99902}} & \textcolor{blue}{\textbf{0.99548}} \\ \hline 
			\multirow{3}{*}{Simulation 2}& Cartographer & 0.96132 &  0.96829 \\ & Ours (All) & \textcolor{red}{\textbf{0.99926}} & \textcolor{red}{\textbf{0.99721}} \\ & Ours (Key) & \textcolor{blue}{\textbf{0.99914}} & \textcolor{blue}{\textbf{0.99638}}\\ \hline
		\multirow{3}{*}{Simulation 3} & Cartographer & 0.92696 &0.96592 \\ & Ours (All)& \textcolor{red}{\textbf{0.99974}} & \textcolor{red}{\textbf{0.99771}} \\ & Ours (Key) & \textcolor{blue}{\textbf{0.99748}} & \textcolor{blue}{\textbf{0.99113}}\\
		 \hline
		\end{tabular}
  }
  % \vspace{-0.5em}
\end{table}

We also assess performance using AUC (Area under the ROC curve) \cite{bradley1997use} and precision, with ground truth labels generated from the occupancy map based on ground truth poses. To ensure a fair comparison, all unknown cells are excluded from this evaluation, as AUC is a binary classification metric \cite{bradley1997use}. Table \ref{tab_auc} presents the results, showing that our method using all frames achieves the highest performance in both metrics. Even with only key frames, our method surpasses Cartographer. A key factor resulting in Cartographer's lower mapping quality is its lack of a batch optimization method to address errors during submap construction. Although its scan-to-map matching approach reduces cumulative errors more effectively than scan-to-scan matching, its accuracy still falls short compared to our algorithm. Cartographer performs pose graph optimization to adjust the coordinate frames of submaps only when loop closure is detected, leaving errors within the submaps uncorrected. Although global pose graph optimization is applied at the end of the process, it often suffers from an excess of inaccurate and conflicting relative measurements, as well as its susceptibility to local minima, limiting its effectiveness in correcting these errors. Moreover, pose graph optimization typically does not enhance the local details of maps, as it focuses solely on optimizing poses without jointly considering the map. This further highlights the advantage of our approach, which jointly optimizes both robot poses and the occupancy map.


\subsection{Comparisons using Practical Datasets} \label{sec_practical}

\begin{figure}[t]
\centering
\includegraphics[width=0.46\textwidth]{Real_OGM_New.pdf}
\caption{\label{fig_result_compare_OGM} The occupancy grid maps from Cartographer, our method using all frames, and our method using key frames. }
% \vspace{-1.5em}
\end{figure}


\begin{figure}[t]
\centering
\includegraphics[width=0.46\textwidth]{Real_Scan_New.pdf}
\caption{\label{fig_result_compare_scan} The point cloud maps from Cartographer, our method using all frames, and our method using key frames.}
\end{figure}

\begin{table}[t]
		\centering
		\caption{Time Consumption of Different Algorithms.}
		\label{table_time_compare}
		\setlength{\tabcolsep}{2.5 mm}{
		\begin{tabular}{lccc}\toprule
		Dataset& & Computational Time (s)& \\ \hline
			     & Cartographer  & Ours (All) & Ours (Key)  \\ 
		Car Park & 168  & 119  & \textbf{44} \\
		Museum b0 & 152  & 126 & \textbf{38} \\
		C5 & 146 & 137 & \textbf{35} \\
		% Simulation 1 & 192  & 148 & \textbf{33} \\
		% Simulation 2 & 174  & 193 & \textbf{57} \\
		% Simulation 3 & 78  & 132 & \textbf{40} \\
		\hline
		\end{tabular}
		}
        % \vspace{-2em}
\end{table}

We use three normal-scale practical datasets, namely Deutsches Museum b0 \cite{hess2016real}, Car Park \cite{zhao20212d} and C5, to compare our method with Cartographer in terms of the constructed occupancy grid maps and optimized poses. 

The mapping quality is evaluated by comparing the details of the constructed maps. Additionally, point cloud maps, which are generated using the endpoint projections of scan points and optimized poses, serve as a reference for pose accuracy. For the Car Park and C5 datasets, our method is initialized with poses from odometry inputs, whereas for the Museum b0 dataset, initialization relies on poses from scan matching due to the absence of odometry. The occupancy grid maps and point cloud maps generated by Cartographer, our method using all frames, and our method using key frames for the three datasets are shown in Fig. \ref{fig_result_compare_OGM} and Fig. \ref{fig_result_compare_scan}. Red dotted lines highlight areas where our results outperform Cartographer in both the occupancy grid maps and point cloud maps. Comparing Fig. \ref{fig_result_compare_OGM}(a) and Fig. \ref{fig_result_compare_OGM}(b), our method provides more precise boundaries for the occupancy grid maps due to joint optimization of robot poses and the occupancy map. Similarly, the comparison between Fig. \ref{fig_result_compare_scan}(a) and Fig. \ref{fig_result_compare_scan}(b) illustrates that our method achieves more accurate poses. 

Moreover, our method outperforms Cartographer when using only key frames, as evident from the comparison of Fig. \ref{fig_result_compare_OGM}(a) and Fig. \ref{fig_result_compare_scan}(a) with Fig. \ref{fig_result_compare_OGM}(c) and Fig. \ref{fig_result_compare_scan}(c). These results show that, despite Cartographer introducing loop closure detection, it still produces non-negligible pose errors, leading to point clouds that fail to fully overlap observations of the same obstacle at different poses. While the point cloud maps generated by our method also have non-overlapping parts, these areas are significantly smaller compared to those from Cartographer. 

% These experiments demonstrate that both variants of our method reduce pose errors and generate more accurate occupancy grid maps by jointly optimizing robot poses and the occupancy map. 

Additionally, we assess the time consumption of our method and Cartographer on these three datasets. Table \ref{table_time_compare} shows that our method consistently requires less time than Cartographer across all datasets when using all frames and achieves significantly better efficiency when using selected key frames.

Finally, it is worth noting that some well-known public datasets, such as Radish \cite{Radish}, were collected before 2014 with outdated sensors, leading to low-quality data with poor scanning frequency and odometry accuracy. These issues hinder the performance of Cartographer, often requiring meticulous parameter tuning but still yielding suboptimal results. In contrast, our method performs well on these datasets. Although we do not include these comparisons in this paper, we make our results available on our code page\footnote{\url{https://github.com/WANGYINGYU/Occupancy-SLAM}}.





\subsection{Assessment of Robustness to Initial Guess}

% While our method has demonstrated robustness when initialized with odometry inputs or scan matching under reliable sensor conditions in both simulation and real-world experiments, this subsection illustrates its capability for convergence even when initialized with significantly noisy poses. In this subsection, we use all frames for robustness assessment.

While our method has demonstrated robustness when initialized with odometry inputs or scan matching under reliable sensor conditions in both simulation and real-world experiments, this subsection highlights its capability to converge even when initialized with significantly noisy poses. We use all frames in this subsection to assess robustness.

First, we use Simulation 1 dataset to quantitatively evaluate the convergence percentage and the accuracy of optimized poses under different noise levels. We add zero-mean uniformly distributed noises with different bounds to the ground truth of the poses to generate each group of ten sets of initial poses for the experiments to count convergence rates and average errors. Specifically, for noise level 1, the noise for translation is within $[-2$ m, $2$ m$]$ and the noise for rotation is within $[-0.5$ rad, $0.5$ rad$]$; for level 2, $[-4$ m, $4$ m$]$ and $[-1$ rad, $1$ rad$]$; for level 3, $[-6$ m, $6$ m$]$ and $[-1.5$ rad, $1.5$ rad$]$. The poses with different noise levels of Simulation 1 dataset are visualized using the generated occupancy grid maps, as shown in Fig. \ref{fig_OGM_246}. The convergence results are depicted in Table \ref{table_robustness}, showing that our method can $100\%$ converge when initialized with challenging noisy poses of level 1 and level 2. Our method still has a high convergence percentage ($80\%$) when initialized with noisy poses of level 3. Our algorithm using other simulation datasets has similar robustness performance.

\begin{table}[t]
		\centering
		\caption{Robustness to Initialization.}
		\label{table_robustness}
		\setlength{\tabcolsep}{0.9 mm}{
		\begin{tabular}{lccc}\toprule
		\thead{Noise Level} & \thead{Convergence\\ Percentage} & \thead{Average MAE of \\Translation (m)} & \thead{Average MAE of\\ Rotation (rad)}\\ \hline
		Level 1 (2 m, 0.5 rad)  & 100\% & 0.00679 & 0.0005   \\
		Level 2 (4 m, 1 rad)  & 100\% & 0.00682 & 0.0005  \\
		Level 3 (6 m, 1.5 rad)  & 80\% & 0.01742  & 0.0012 \\
		\hline
		\end{tabular}
		}
\end{table}

\begin{figure}[t]
\centering
\includegraphics[width=0.47\textwidth]{./OGM_Level246.pdf}
\caption{\label{fig_OGM_246} Examples of occupancy grid maps generated from poses with different noise levels as shown in Table \ref{table_robustness} using Simulation 1 dataset.}
% \vspace{-1.5em}
\end{figure}

Moreover, for all practical datasets, we additionally add random zero-mean uniform distribution noises ($[-2$ m, $2$ m$]$ for translation and $[-0.5$ rad, $0.5$ rad$]$ for rotation) to the poses obtained from Cartographer as the initial guess. The initial occupancy maps obtained by using the noisy initial poses are shown in Fig. \ref{fig_noise_initial}(a). Fig. \ref{fig_noise_initial}(b) shows the remapped occupancy grid maps using our optimized poses, and Fig. \ref{fig_noise_initial}(c) shows the point cloud maps using our optimized poses. This experiment shows that our approach can converge from initial guesses with significant errors and also generate good results.


\begin{figure}[t]
% \vspace{-5mm}
\centering
\includegraphics[width=0.5\textwidth]{Noise_Initial.pdf}
\caption{\label{fig_noise_initial}The occupancy grid maps and point cloud maps generated using noisy poses for initialization by our approach. (a) and (b) display the remapped occupancy maps generated from the noisy initial poses and our optimized poses, respectively, and (c) shows the point cloud maps created by projecting the endpoints of scans using our optimized poses.}
% \vspace{-2em}
\end{figure}

\subsection{Discussion about the Effectiveness of Different Stages} \label{sec_discuss}


\begin{figure*}[tp]
\centering \subfigure[Simulation 1] {\label{fig_group_error_1}
\includegraphics[width=0.32\textwidth]{./Group_Error_3640.pdf}}
\centering \subfigure[Simulation 2] {\label{fig_group_error_2}
\includegraphics[width=0.32\textwidth]{./Group_Error_3720.pdf}}
\centering \subfigure[Simulation 3] {\label{fig_group_error_3}
\includegraphics[width=0.32\textwidth]{./Group_Error_2680.pdf}}
\caption{Comparison of translation and rotation errors for simulated datasets using three methods: our full method (Algorithm \ref{alg_flowchart}), our Algorithm \ref{alg_1} initialized by Cartographer's poses with a high-resolution map \cite{Zhao-RSS-22}, and our Algorithm \ref{alg_1} initialized by poses obtained from our first stage with a high-resolution map.}
\label{fig_group_error}
\vspace{-1em}
\end{figure*}


In previous sections, we demonstrated the accuracy, robustness, and efficiency of our proposed method. In this section, we discuss the effectiveness of its different parts.


As demonstrated in Table \ref{tab_comparison}, Fig. \ref{fig_simulation}, Fig. \ref{fig_result_compare_OGM}, and Fig. \ref{fig_result_compare_scan}, the accuracy of the poses and the map obtained from our full approach (Algorithm \ref{alg_flowchart}) is much better than those obtained from Cartographer. This confirms the advantage of jointly optimizing both the robot poses and the occupancy map. 


% One may ask, how about performing only Algorithm \ref{alg_1} with a high-resolution map directly? Will the result be even better? To answer this question clearly, in this subsection, we compare our full approach with Algorithm \ref{alg_1} using a high-resolution map. We consider three different initialization: 

% In the following, we refer to these three approaches as  \textit{Algorithm \ref{alg_1} (High, O/S)}, \textit{Algorithm \ref{alg_1} (High, Carto)}, and \textit{Algorithm \ref{alg_1} (High, First)}, respectively.   

One potential question is whether using only Algorithm \ref{alg_1} with a high-resolution map would yield even better results. To investigate this, we compared our full approach with Algorithm \ref{alg_1} using a high-resolution map. We tested three initialization: (1) \textit{Algorithm \ref{alg_1} (High, O/S)}: initialization using odometry inputs or scan matching; (2) \textit{Algorithm \ref{alg_1} (High, Carto)}: initialization using Cartographer's poses (as proposed in our conference paper \cite{Zhao-RSS-22}); and (3) \textit{Algorithm \ref{alg_1} (High, First)}: initialization using the poses obtained by our first stage. 



% First, \textit{Algorithm \ref{alg_1} (High, O/S)} fails to converge on most datasets, while our full method can converge very well, which indicates the improved robustness of our multi-resolution strategy.   

% The comparison of our full method with \textit{Algorithm \ref{alg_1} (High, Carto)} and \textit{Algorithm \ref{alg_1} (High, First)} using all the five groups simulation datasets are shown in Fig. \ref{fig_group_error}. It can be seen that the accuracy of our full method is essentially similar in all groups, while the accuracy of \textit{Algorithm \ref{alg_1} (High, Carto)} varies drastically. This means the approach proposed in our conference paper \cite{Zhao-RSS-22} not only requires an accurate initial value but also generates less accurate poses than our new approach. In addition, by comparing \textit{Algorithm \ref{alg_1} (High, Carto)} with \textit{Algorithm \ref{alg_1} (High, First)}, it also confirms that the poses obtained in our first stage are more accurate than those of Cartographer.

First, \textit{Algorithm \ref{alg_1} (High, O/S)} fails to converge on most datasets, while our full method converges successfully, indicating the improved robustness of our multi-resolution strategy.

The comparison between our full method, \textit{Algorithm \ref{alg_1} (High, Carto)}, and \textit{Algorithm \ref{alg_1} (High, First)} across all five simulation groups is shown in Fig. \ref{fig_group_error}. It can be observed that the accuracy of our full method remains stable across all groups, while the accuracy of \textit{Algorithm \ref{alg_1} (High, Carto)} varies drastically. This suggests that the approach in our conference paper \cite{Zhao-RSS-22} not only requires an accurate initial guess but also produces less accurate poses than our new method. Moreover, comparing \textit{Algorithm \ref{alg_1} (High, Carto)} with \textit{Algorithm \ref{alg_1} (High, First)} further confirms that the poses obtained in our first stage are more accurate than those of Cartographer.



% It is worth discussing that our full method uses the selected high-resolution map for optimization in the second stage, and it can be observed in Fig. \ref{fig_group_error} that the accuracy of our full approach is even higher than the optimization using the full high-resolution map (i.e., \textit{Algorithm \ref{alg_1} (High, First)}) in some experiments. The potential reason is that when the relatively accurate poses and occupancy map are obtained, the dropped \textcolor{red}{cell vertices} and the corresponding observations contain little information. If all cells and corresponding observations are retained for optimization, it may affect the algorithm's ability to obtain the best solution, as all observation terms are assigned uniform weights. Another potential reason is that the smoothing term in (\ref{eq_objective_func}), by spreading the occupancy values to unknown \textcolor{red}{cell vertices}, may introduce errors that could affect the convergence of the optimization algorithm. 

It is also worth noting that our full method utilizes the selected high-resolution map for optimization in the second stage. As shown in Fig. \ref{fig_group_error}, in certain experiments, the accuracy of our full approach surpasses that of the optimization using the full high-resolution map (\textit{Algorithm \ref{alg_1} (High, First)}). A possible explanation is that once relatively accurate poses and occupancy maps are obtained, the dropped cell vertices and corresponding observations contain little information. Retaining all cells and corresponding observations for optimization may prevent the algorithm from finding the optimal solution, as all are observation items given uniform weights. Another reason could be the smoothing term in (\ref{eq_objective_func}), which spreads occupancy values to unknown cell vertices, potentially introducing errors that affect the convergence of the optimization.

In terms of time consumption, our full approach is much more efficient than \textit{Algorithm \ref{alg_1} (High, Carto)}. For instance, \textit{Algorithm \ref{alg_1} (High, Carto)} consumes over 21,000 seconds with the Car Park dataset. In comparison, the time consumption of our full approach using all frames is 119 seconds (less than 0.6\%), and using only key frames, it takes only 44 seconds (about 0.2\%). This substantial reduction in time consumption highlights the efficiency improvements of our method over our conference paper \cite{Zhao-RSS-22}.

% In terms of time consumption, our full approach is significantly more efficient than \textit{Algorithm \ref{alg_1} (High, Carto)}. For instance, \textit{Algorithm \ref{alg_1} (High, Carto)} consumes over 21,000 seconds when using the Car Park dataset. In comparison, the time consumption of our full approach using all frames is 119 seconds (less than 0.6\%), and using only key frames, it takes 44 seconds (approximately 0.2\%). This substantial reduction in time consumption underscores the significant efficiency improvements of our current method over our conference paper \cite{Zhao-RSS-22}.

The reduction in time consumption stems from both the multi-resolution strategy, which reduces time per iteration, and the fewer iterations needed in the second stage due to the selected high-resolution map. Our experiments show that only about two iterations are needed in the second stage with the selected high-resolution map, fewer than in \textit{Algorithm \ref{alg_1} (High, First)}. This is likely because the selected high-resolution map focuses on critical states, with observations containing the most relevant information, enabling faster convergence.


In summary, compared to our conference paper \cite{Zhao-RSS-22}, our new multi-resolution method does not require precise initialization, is far more efficient, and achieves higher accuracy.

% In addition, compared with \textit{Algorithm \ref{alg_1} (High, Carto)}, the time consumption of our full approach is reduced by $2-3$ orders of magnitude. For example, \textit{Algorithm \ref{alg_1} (High, Carto)} consumes more than $21,000$ seconds using the Car Park dataset. Compared to \textit{Algorithm \ref{alg_1} (High, Carto)}, the time consumption of our full approach using all frames is $119$ seconds (less than $0.6\%$), and the time consumption of our full approach using only key frames is 44 seconds which is approximately $0.2\%$. The significant reduction in time consumption shows the significantly improved efficiency of our current method over our conference paper \cite{Zhao-RSS-22}.

% The substantial reduction in time consumption of our full approach is attributed not only to the introduced multi-resolution strategy, which reduces the time consumption per iteration but also to the reduction in the number of iterations in the second stage, which is a result of utilizing the selected high-resolution map. Through the experiments, we find that only about $2$ iterations in the second stage are required to converge using the selected high-resolution map, which is smaller than the number of iterations needed in \textit{Algorithm \ref{alg_1} (High, First)}. The possible reason is that, in the case of using the selected high-resolution map, these selected \textcolor{red}{cell vertices} focus on the most critical states, and the corresponding observations contain the most important information, allowing the optimization problem to converge much faster.

% In summary, as compared with our conference paper \cite{Zhao-RSS-22}, our new multi-resolution method does not require accurate initialization, is much more efficient, and achieves a higher level of accuracy in most cases.   


\subsection{Ablation Study on the Resolution Ratio}

\begin{table}[t]
		\centering
		\caption{Impact of First-Stage Resolution Settings.}
		\label{table_ablation}
		\setlength{\tabcolsep}{1mm}{
		\begin{tabular}{llcccc}\toprule
		& & $r=20$ & $r=10$  & $r=5$ & $r=2$ \\   \hline   

		 \multirow{5}{*}{Simulation 1} & MAE/Trans (m) First& 0.02352  &  0.02206 & \textbf{0.02118} & 0.17318\\
		\quad & MAE/Rot (rad) First& 0.00116  &  \textbf{0.00098} & 0.00124 & 0.01066\\
		\quad & MAE/Trans (m) All & 0.00812  & \textbf{0.00640}  & 0.00728 & 0.16066\\
		\quad & MAE/Rot (rad) All&  0.00062 & 0.00060  & \textbf{0.00054} & 0.01008\\ 
		\quad & Total Time (s) & \textbf{118}  &  148 & 262 & 2183\\
		\hline

		\multirow{5}{*}{Simulation 2} & MAE/Trans (m) First & 0.03938 & 0.03224 & \textbf{0.01984} & 0.09160\\
		\quad & MAE/Rot (rad) First&  0.00332 & 0.00220  & \textbf{0.00108} & 0.00314\\
		\quad & MAE/Trans (m) All& 0.01742  & 0.00858  & \textbf{0.00584} & 0.08018\\
		\quad & MAE/Rot (rad) All& 0.00064  & 0.00062  & \textbf{0.00052} & 0.00286\\ 
		\quad & Total Time (s)& \textbf{149}  & 193  & 321 & 2685\\
		\hline

		\multirow{5}{*}{Simulation 3} & MAE/Trans (m) First& 0.06708  &  0.02624  & \textbf{0.01776} & 0.03570\\
		\quad & MAE/Rot (rad) First&  0.00384 & 0.00164  & \textbf{0.00124}  & 0.00278\\
		\quad & MAE/Trans (m) All& 0.01586  & \textbf{0.00726}  & 0.00816 & 0.02082\\
		\quad & MAE/Rot (rad) All& 0.00100  & \textbf{0.00058}   & 0.00068 & 0.00102\\ 
		\quad & Total Time (s)& \textbf{125}  &  132 & 185 & 1041\\
		\hline
		\end{tabular}
		}
        % \vspace{-1.5em}
\end{table}

In this section, we perform ablation experiments on simulation datasets to analyze the impact of varying resolution settings in the first stage of the multi-resolution strategy on overall optimization performance.

We assess accuracy and computational time using three simulation datasets, with the resolution in the second stage fixed at $s^{h} = 0.05$ m. The resolution ratios $r$ between the first and second stages are set to 2, 5, 10, and 20, respectively. To ensure consistency, a fixed selection range of $d=1.5$ m is applied uniformly across all datasets. 

  
The results, shown in Table \ref{table_ablation}, reveal that $r=10$ achieves the best trade-off between time consumption and accuracy. While $r=20$ minimizes time consumption, it reduces the accuracy of poses in the first stage, adversely impacting final optimization accuracy. Conversely, $r=5$ improves pose accuracy in the first stage at the cost of higher time consumption but does not consistently enhance final accuracy. Notably, $r$ may need adjustment for other high-resolution settings.


\subsection{Using Submap Joining in Large-scale Environments}

We have demonstrated that our approach accurately and robustly handles normal-scale simulated and practical environments. In this section, we evaluate its efficiency and effectiveness in large-scale environments and long-term trajectories by integrating our Occupancy-SLAM algorithm with the proposed occupancy submap joining approach. The dataset is divided into multiple segments, where Algorithm \ref{alg_flowchart} is used to construct submaps, followed by applying the submap joining method in Section \ref{Sec_submap} to generate the optimized global occupancy map and robot trajectory.

We validate our method on two large-scale datasets, Deutsches Museum b2 \cite{hess2016real} and C3, and compare it with Cartographer. As shown in Fig. \ref{fig_large_environment}, our occupancy grid maps outperform those of Cartographer, demonstrating the capability of our method to handle large-scale environments and long-term trajectories effectively. 

% The datasets have map sizes of 250 m $\times$ 200 m and 150 m $\times$ 125 m, containing 51833 and 24402 scans, with trajectory lengths of 1390 seconds and 610 seconds, respectively.

\begin{figure}[t]
\centering \subfigure[b2] { \label{fig_large_b2}
\includegraphics[width=0.253\textwidth]{./b2_Large_New.pdf}}\hspace{-0.4em}
\centering \subfigure[C3] {\label{fig_large_C3}
\includegraphics[width=0.2195\textwidth]{./C3_Large_New_1.pdf}}
\caption{\label{fig_large_environment} Comparison of results between our method and Cartographer on two large-scale practical datasets. The first row shows Cartographer's results, while the second row shows ours. In (b), the red dotted lines serve as references, highlighting that Cartographer's right wall appears more curved, whereas our result aligns more closely with a straight line.}
% \vspace{-1.5em}
\end{figure}


\subsection{Computational Complexity Analysis}
In this section, we analyze the computational complexity and evaluate the time consumption of our method using large-scale datasets.

The Gauss-Newton method for solving the joint optimization of local maps and poses in (\ref{Least Squares}) and submap joining in (\ref{eq_NLLS_joining}) primarily depends on calculating Jacobian $\mathbf{J}$, and constructing and solving the sparse linear system in (\ref{Gauss-Newton}) \cite{konolige2008frameslam}. We analyze each part's complexity separately due to differences in the NLLS formulation.

For the local map and poses joint optimization problem, the objective function consists of the observation term, the odometry term, and the smoothing term. Let ${\lambda(\mathbb{S})}$ denotes the number of sampling points $\mathbb{S}$, then the number of items in the objective function is $\mathfrak{d}_{row} =\lambda(\mathbb{S})+3(n-1)+2{c_w}{c_h}+c_w+c_h$, and the state vector size is $\mathfrak{d}_{col}=3n+(c_w+1)(c_h+1)$. Considering Jacobian of the smoothing term $\mathbf{J}_S$ can be pre-calculated before optimization, the number of non-zero elements of Jacobian matrix that need to be computed for each iteration is $\mathfrak{d}_J = 7\lambda(\mathbb{S}) + 6(n-1)$. Therefore, for each iteration, the computation complexity of Jacobian calculation, constructing (\ref{Gauss-Newton}) and solving (\ref{Gauss-Newton}) is $\mathcal{O}(\mathfrak{d}_J)$, $\mathcal{O}(\mathfrak{d}_{J}\mathfrak{d}_{col})$, and $\mathcal{O}({\mathfrak{d}^3_{col}})$, respectively. Therefore, the total computation complexity per iteration for the local map and poses joint optimization problem is $\mathcal{O}(\mathfrak{d}_{J}+\mathfrak{d}_J{\mathfrak{d}_{col}}+\mathfrak{d}^3_{col})$. Due to our proposed multi-resolution joint optimization strategy and keyframe selection, both $\mathfrak{d}_{J}$ and $\mathfrak{d}_{col}$ remain small during the first and second stages of optimization, making the computation time for this part manageable.

% (i.e., \textcolor{red}{cell vertices} with non-zero occupancy values) Similar to the computation complexity of the local map and poses joint optimization problem,

For our submap joining algorithm, the number of observations depends on the total number of cell vertices of the global occupancy map observed in each submap, denoted $\mathfrak{d}_{obs}^{G}$. Considering that some cell vertices will be observed repeatedly under different submaps, this number slightly exceeds the number of non-unknown cell vertices in the global map. Thus, the number of non-zero elements of Jacobian matrix is $\mathfrak{d}_J^G = 4\mathfrak{d}_{obs}^G$, and the state vector size is $\mathfrak{d}_{col}^{G} = 3n_L+(c_w^G+1)(c_h^G+1)$. The computation complexity per iteration is $\mathcal{O}(\mathfrak{d}_{J}^G+\mathfrak{d}_J^G{\mathfrak{d}_{col}^G}+{\mathfrak{d}_{col}^G}^3)$.
Although the global occupancy map tends to be relatively large, the sub-matrix of Hessian w.r.t. the global occupancy map is diagonal. To speed up computation, we apply the Schur complement \cite{zhang2006schur} to make the normal equation solving highly efficient.

Finally, we evaluate the time consumption of our method using both all frames and selected keyframes in large-scale environments to support our computation complexity analysis and compare it to Cartographer. For Museum b2 dataset, Cartographer takes 1424 seconds, while our method takes 1250 seconds when using all frames and 363 seconds with selected key frames. For C3 dataset, Cartographer takes 610 seconds, while our method takes 742 seconds with all frames and 236 seconds with selected key frames. In our total time consumption, the submap joining method consumes less than 10 seconds on both datasets. It can be seen that the time consumption of our method is comparable to that of Cartographer when all frames are used and much lower than that of Cartographer when selected key frames are used. These results demonstrate the efficiency of our multi-resolution joint optimization strategy and submap joining approach. 

\section{Preliminary Results in 3D Case}\label{sec_3d}

While this paper primarily focuses on demonstrating the benefits of jointly optimizing the robot poses and occupancy map in 2D, we also present some preliminary 3D results to illustrate that our idea can be extended to 3D applications.



\subsection{Extension of the Algorithms to 3D Case}

Our approach for jointly optimizing robot poses and the occupancy map extends naturally to 3D, where the information, robot poses, and occupancy maps are all represented in 3D. Most problem formulations and algorithms can be adapted with minor adjustments. 

For our local map and poses optimization method, observations transition from 2D laser scans to 3D LiDAR scans, robot poses and odometry involve 6 degree-of-freedom (DoF), and the map representation becomes 3D. Consequently, (\ref{eq_interp}) and (\ref{eq_NP}) need to be replaced from bilinear to trilinear interpolation and its inverse operation. 
For the objective function (\ref{eq_objective_func}), the odometry term (\ref{eq_odometry_term}) should be replaced with a 6 DoF odometry term for 3D, and the smoothing term (\ref{eq_smoothing_term}) should include a smoothing penalty for the z-axis, Jacobians $\mathbf{J}_P$, $\mathbf{J}_M$, $\mathbf{J}_O$, and $\mathbf{J}_S$ described in Appendices need to be adjusted accordingly. 

The submap joining problem in 3D remains largely similar to the 2D case, except that the projection relation extends from 2D-2D to 3D-3D, enabling the solution of 6 DoF poses and the 3D global occupancy map in the NLLS problem (\ref{eq_NLLS_joining}).

\subsection{3D Experimental Results}
\subsubsection{Evaluation metrics and state-of-the-art methods}
We evaluate our method's performance in 3D using absolute trajectory error for poses, aligning and comparing results with ground truth via EVO \cite{grupp2017evo}, as used in \cite{liu2023large,rosinol2021kimera}. In all the experiments, we use the odometry information provided by the dataset as initialization if it is available. Otherwise, we use FAST-LIO2 \cite{xu2022fast} to obtain the odometry information. To evaluate our method, we compare our method against state-of-the-art methods: BALM2 \cite{liu2023efficient}, HBA \cite{liu2023large}, and Voxgraph \cite{reijgwart2019voxgraph}. BALM2 optimizes the planar feature parameters of the point cloud and the robot's poses. HBA proposes a hierarchical bundle adjustment to optimize the consistency of the planar surfaces of point clouds and robot poses. Voxgraph builds SDF-based submaps from point clouds, uses SDF-to-SDF registration for relative submap measurements, and incrementally optimizes submap frames. HBA and Voxgraph can deal with large-scale environments, while BALM2 focuses on normal-scale environments.  

\subsubsection{Datasets}
We perform comparisons using three real-world datasets. (1) The Newer College Dataset \cite{ramezani2020newer}: The first five sequences from the \textit{shorter experiment}, collected with a handheld Ouster OS-1 LiDAR scanner at New College, Oxford. The environment includes lawns, buildings, a tunnel, and a garden. Ground truth is provided by a BLK360 LiDAR scanner to capture a detailed 3D map and then infer the ground truth of poses with centimeter-level accuracy.
(2) KITTI Dataset \cite{Geiger2013IJRR}: Sequence 07, a demo dataset for HBA, collected with a Velodyne HDL-64E LiDAR scanner mounted on a car. Ground truth poses is provided by RTK-GPS/INS.
(3) Arche Dataset \cite{reijgwart2019voxgraph}: A demo dataset for Voxgraph, collected using an Ouster OS1 LiDAR mounted on a hexacopter MAV in a disaster area. Ground truth positions are provided by an RTK-GNSS system. 

The Newer College Dataset is used to evaluate high-precision performance in normal-scale environments, while the KITTI and Arche datasets are used to test performance in large environments with long trajectories.

\begin{figure*}[t]
\centering
\includegraphics[width=0.99\textwidth]{./PC_Comparison.pdf}
\caption{\label{fig_3d_pointcloud} Some local point cloud maps from the Arche dataset. The first row shows point cloud maps generated using odometry from ROVIO (also used for submap construction in Voxgraph), while the second row shows maps generated with our optimized poses using the same LiDAR scans. BALM2 fails to produce results when using the same odometry and scans as inputs in all these local environments.}
\vspace{-1em}
\end{figure*}
 
\subsubsection{Experiments on normal-scale environments}
We evaluate the performance of our proposed method without submap joining in normal-scale environments.

First, we evaluate BALM2 and our method using the first five sequences of The Newer College Dataset, which encompass all scenarios within the dataset. As shown in Table \ref{tab_comparison_3d_local}, our method outperforms BALM2 across all metrics, except for the RMSE in Seq. 1, and significantly outperforms the odometry inputs from FAST-LIO2 in all metrics. 

\begin{table}[t]
		\centering
		\caption{Absolute Trajectory Error (MAE/RMSE, Meters) in Normal-scale Environments for Different 3D Methods.}
		\label{tab_comparison_3d_local}
		\setlength{\tabcolsep}{0.6 mm}{
		\begin{tabular}{lcccccc}\toprule
		Method	& Seq. 0 & Seq. 1 & Seq. 2 & Seq. 3 & Seq. 4 \\ \hline
		FAST-LIO2 & 0.518/0.717 & 0.181/0.202  & 0.121/0.132 &0.188/0.200&0.571/0.723\\
		BALM2 & 0.283/0.326 & 0.112/\textbf{0.123}  & 0.104/0.109 &0.144/0.158 &0.298/0.344 \\
        Ours & \textbf{0.185}/\textbf{0.232}  & \textbf{0.097}/\textbf{0.123}  & \textbf{0.091}/\textbf{0.099} & \textbf{0.141}/\textbf{0.155} &\textbf{0.238}/\textbf{0.284}\\ \hline
		\end{tabular}
		}
        % \vspace{-2em}
\end{table}

Next, we test robustness in a challenging environment with noisy odometry input using the Arche dataset. This dataset, collected by a hexacopter MAV in an unstructured environment, is influenced by drone vibrations, flight speed, and environmental factors. Local point cloud maps built using odometry from ROVIO \cite{bloesch2017iterated} (also used to construct submaps in Voxgraph) are shown in the first row of Fig. \ref{fig_3d_pointcloud}. To evaluate BALM2 and our method, we partition the dataset into several short sequences, each lasting 10–20 seconds. BALM2 fails in all the sequences except during MAV start-up and landing due to insufficient planar features for optimization, while our method performs well on all the sequences. The second row of Fig. \ref{fig_3d_pointcloud} illustrates some of our results, demonstrating that our method is robust in 3D and does not rely on environmental assumptions. Additionally, the results confirm that our method achieves significantly higher pose accuracy than ROVIO.

\subsubsection{Experiments on large-scale environments}\label{sec_experiment_vox} 

We evaluate our method in large-scale environments with long trajectories using the KITTI and Arche datasets, comparing it with HBA and Voxgraph. For this experiment, we first build submaps by jointly optimizing poses and maps within submaps, then apply our submap joining algorithm to jointly optimize submap frame poses and the global occupancy map.

% The MAE and RMSE of the ATEs are summarized in Table \ref{tab_comparison_3d_large}, it is clear that our method achieves the best results on both datasets. In addition, the robot trajectories are shown in Fig. \ref{fig_trajectory_3d}, as it shows our method can achieve the best global consistent robot trajectories compared with other methods. It should be noted that the results of our method substantially lead Voxgraph on the KITTI dataset and significantly outperform HBA on the Arche dataset. The reason our method performs much better than Voxgraph on KITTI dataset is that Voxgraph relies on relative measurements from SDF-to-SDF registration for solving pose graph optimization, but in such autonomous driving scenarios, it is difficult to provide sufficient overlapping submaps for Voxgraph to calculate relative measurements between submaps. However, our proposed submap joining algorithm jointly optimizes poses of submaps' coordinate frames and the global occupancy map and, therefore, does not suffer in such environments. The performance of HBA on the Arche dataset is affected by highly unstructured environments and with data captured by moving MAV, as HBA relies on detecting and using planar features from the point cloud to do the optimization, which is similar to BALM2. However, in the case of odometry and point clouds collected during MAV motion, it is difficult for such methods to detect a sufficient number of good planar features. In addition, the planarity assumption does not tend to hold true in non-urban environments, such as the field.

Table \ref{tab_comparison_3d_large} summarizes the MAE and RMSE of absolute trajectory error, showing our method achieves the best results on both datasets. Fig. \ref{fig_trajectory_3d} illustrates that our approach can achieve the best global robot trajectories. Notably, our method significantly outperforms Voxgraph on the KITTI dataset and HBA on the Arche dataset. The relatively poor performance of Voxgraph on the KITTI dataset is due to its reliance on relative measurements from SDF-to-SDF registration, which requires sufficient overlapping submaps—a challenge in autonomous driving scenarios. In contrast, our submap joining algorithm jointly optimizes submap poses and the global occupancy map, avoiding this limitation. HBA underperforms on the Arche dataset due to its reliance on planar features for optimization, which is challenging in unstructured environments and during MAV motion. Odometry and point clouds from such scenarios make detecting sufficient planar features difficult, and the planarity assumption often fails in non-urban environments like disaster areas.


\begin{figure}[tbp]
\centering \subfigure[KITTI] {\label{fig_trajectory_1}
\includegraphics[height=0.15\textwidth]{./Traj_KITTI.pdf}}
\centering \subfigure[Arche] {\label{fig_trajectory_2}
\includegraphics[height=0.15\textwidth]{./Traj_Voxgraph_Demo_New.pdf}}
\caption{\label{fig_trajectory_3d} Robot trajectory results of datasets in large-scale environments. (a) and (b) show the trajectories of ground truth, Voxgraph \cite{reijgwart2019voxgraph}, HBA \cite{liu2023large}, and our method for KITTI dataset and Arche dataset.}
\end{figure}


\begin{table}[t]
		\centering
		\caption{Absolute Trajectory Error (MAE/RMSE, Meters) in Large-scale Environments for Different 3D Methods}
		\label{tab_comparison_3d_large}
		\setlength{\tabcolsep}{7mm}{
		\begin{tabular}{lcc}\toprule
		Method & KITTI & Arche  \\ \hline
		HBA \cite{liu2023large} & 0.342/0.364 & 4.123/4.789  \\
        Voxgraph \cite{reijgwart2019voxgraph} &0.926/1.002 & 0.700/0.833 \\
        Ours & \textbf{0.315}/\textbf{0.339}  & \textbf{0.275}/\textbf{0.378} \\ \hline
		\end{tabular}
		}
        % \vspace{-2em}
\end{table}

\subsection{Discussion}
The experimental results in this section demonstrate that our proposed idea of jointly optimizing the robot pose and the occupancy map can also lead to better solutions for the robot poses and occupancy maps in 3D cases. However, several challenges remain in 3D scenarios. For instance, 3D point clouds from LiDAR scanners are often sparse, particularly in the vertical direction, which can lead to observability issues in the optimization problem. This sparsity also results in inhomogeneous observation information, complicating the accurate representation of the 3D environment in occupancy maps. Furthermore, the large dimensions of 3D maps pose significant computational challenges in large-scale SLAM, requiring more efficient solving methods.


To address these challenges, several potential solutions can be explored. First, adopting compact representations for 3D occupancy maps, such as octree structures similar to Octomap \cite{hornung2013octomap} and \cite{vespa2019adaptive}, can enhance efficiency. Second, combining local map and pose optimization with hierarchical optimization and submap joining methods can further reduce computational time. Lastly, using continuous representations for 3D occupancy maps enables more precise gradient calculations, which can better guide the optimization process.

\section{Conclusion} \label{Sec_conclusion}
In this paper, we propose Occupancy-SLAM algorithm, which solves robot poses and occupancy map simultaneously. To enhance efficiency and robustness, we introduce a multi-resolution strategy. The first stage jointly optimizes poses and a low-resolution occupancy map to quickly achieve relatively accurate pose estimates, which are then used as the initial guess for the second stage. The second stage refines poses and a subset of the high-resolution map, focusing on critical boundary areas. Additionally, we extend this framework to an occupancy grid-based submap joining algorithm, addressing challenges in large-scale environments and long-term trajectories. Results from both simulated and real-world datasets demonstrate that our method achieves more accurate pose and map estimates than state-of-the-art approaches. 

   
Our findings show that solving poses and occupancy maps simultaneously yields more accurate results compared to first solving pose-graph SLAM and then constructing the map. This joint optimization approach has the potential to revolutionize occupancy map based SLAM frameworks.

The proposed method acts as a batch optimization approach for obtaining high-quality robot poses and maps. Unlike incremental or online methods, batch optimization provides greater accuracy, which is particularly advantageous for applications requiring high-quality maps rather than real-time operation (e.g., offline map creation for precise future localization). Despite typical drawbacks of batch optimization, such as higher computational costs, trajectory-length-dependent complexity, and reliance on accurate initial guesses, our method effectively overcomes these limitations: 1) our method is efficient due to the proposed multi-resolution joint optimization strategy, and the computation time is comparable to online methods; 2) our method can use selected keyframes to further reduce the computational cost without losing too much accuracy; 3) our proposed occupancy submap joining approach can overcome the limitation that the computational complexity related to the length of the robot trajectories; and 4) our method is very robust to the initial guess and can be initialized from odometry inputs or scan matching, so it does not require initialization from the result of incremental/online methods.    

In our future work, we will further explore problem formulation and solution techniques in the 3D case to develop more efficient and robust algorithms capable of addressing various challenges in 3D environments. 


%\section*{Acknowledgments}
%This should be a simple paragraph before the References to thank those individuals and institutions who have supported your work on this article.


{\appendix


The Jacobian $\mathbf{J}$ in (\ref{Gauss-Newton}) consists of four parts, i.e. the Jacobian of the observation term w.r.t. the robot poses $\mathbf{J}_P$ (See Appendix \ref{Sec_J_P}), the Jacobian of the observation term w.r.t. the occupancy map $\mathbf{J}_M$ (See Appendix \ref{Sec_J_D}), the Jacobian of the odometry term w.r.t. robot poses $\mathbf{J}_O$ (See Appendix \ref{Sec_J_O}) and the Jacobian of the smoothing term w.r.t. the occupancy map $\mathbf{J}_S$ (See Appendix \ref{Sec_J_S}). In addition, the difference in the calculation of Jacobians between Algorithm \ref{alg_1} and Algorithm \ref{alg_3} is shown in Appendix \ref{Sec_J_Select}. 

\subsection{Jacobian of the Observation Term w.r.t. Robot Poses}\label{Sec_J_P}

The Jacobian $\mathbf{J}_P$ of function $F_{ij}^Z(\mathbf{x})$ in the observation term w.r.t. the robot poses $\mathbf{x}^P_i$ can be calculated by the chain rule
\begin{equation}
	\begin{aligned}
		\mathbf{J}_P=\frac{ \partial F_{ij}^Z(\mathbf{x}) }{ \partial \mathbf{x}^P_i } = \frac{\partial F_{ij}^Z(\mathbf{x}) }{ \partial \mathbf{p}_{ij} } \cdot \frac{\partial \mathbf{p}_{ij}  }{ \partial \mathbf{x}^P_i}	
	\end{aligned}
\end{equation}
in which $\dfrac{\partial \mathbf{p}_{ij}  }{ \partial \mathbf{x}^P_i}$ can be calculated as
\begin{equation}
\dfrac{\partial \mathbf{p}_{ij}}{\partial \mathbf{x}^P_i}=\left[\begin{array}{ll}
\dfrac{\partial \mathbf{p}_{ij}}{\partial \mathbf{t}_i} & \dfrac{\partial \mathbf{p}_{ij}}{\partial \theta_i}
\end{array}\right]=\dfrac{1}{s} \left[\begin{array}{ll}
\mathbf{E}_{2} & \left(\mathbf{R}_i^{\prime}\right)^{\top} \mathbf{p}_{ij}
\end{array}\right].
\end{equation}
$\mathbf{R}_i^\prime$ is the derivative of the rotation matrix $\mathbf{R}_i$ w.r.t. rotation angle $\theta_i$ and $\mathbf{E}_2$ means $2 \times 2$ identity matrix.

$\dfrac{\partial F_{ij}^Z(\mathbf{x}) }{ \partial \mathbf{p}_{ij} }$ can be calculated by
\begin{equation}
\dfrac{\partial F_{ij}^Z(\mathbf{x}) }{ \partial \mathbf{p}_{ij} } = \dfrac{1}{N(\mathbf{p}_{ij})} \dfrac{\partial M(\mathbf{p}_{ij})}{\partial \mathbf{p}_{ij}}.
\end{equation}
Here $\dfrac{\partial M(\mathbf{p}_{ij})}{\partial \mathbf{p}_{ij}}$ can be considered as the gradient of the occupancy map at point $\mathbf{p}_{ij}$, which can be approximated by the bilinear interpolation of the gradients of the occupancy at the four adjacent cell vertices $\mathbf{\nabla} M(\mathbf{m}_{wh}),\cdots,\mathbf{\nabla} M(\mathbf{m}_{(w+1)(h+1)})$ around $\mathbf{p}_{ij}$ as
\begin{equation} 
\dfrac{\partial M(\mathbf{p}_{ij})}{\partial \mathbf{p}_{ij}}= 
\left[
\begin{aligned}
a_1b_1\\a_0b_1\\a_1b_0\\a_0b_0\\
\end{aligned}\right]^\top
\left[
\begin{aligned}
&\mathbf{\nabla} M(\mathbf{m}_{wh})\\&\mathbf{\nabla} M(\mathbf{m}_{(w+1)h})\\&\mathbf{\nabla} M(\mathbf{m}_{w(h+1)})\\&\mathbf{\nabla} M(\mathbf{m}_{(w+1)(h+1)})
\end{aligned}\right]\label{eq_14}
\end{equation} 
where the gradient of occupancy map $\mathbb{M}$ at all the cell vertices $\mathbf{\nabla} M$ can be easily calculated from $\mathbf{x}^M$ in the state. The bilinear interpolation used in (\ref{eq_14}) is similar to the method in (\ref{eq_interp}).

Here, we assume the robot poses $\mathbf{x}^P$ change slightly in each iteration, to reduce the computational complexity, the hit map $\mathbb{N}$ is considered as constant and recalculated using the current robot poses in each iteration. Thus, the derivative of $N(\mathbf{p}_{ij})$ is not calculated. 

\subsection{Jacobian of the Observation Term w.r.t. Occupancy Map}\label{Sec_J_D}
Based on (\ref{eq_interp}), the Jacobian $\mathbf{J}_M$ of function $F_{ij}^Z(\mathbf{x})$ in the observation term w.r.t. the map part of state vector $\mathbf{x}^{M}$ can be calculated as

\begin{equation}
\begin{aligned}
\mathbf{J}_M & = \dfrac{\partial F_{ij}^Z(\mathbf{x})}{\partial \left[ {M}(\mathbf{m}_{wh}),\cdots, {M}(\mathbf{m}_{(w+1)(h+1)}) \right]^\top}\\
&= \dfrac{1}{N(\mathbf{p}_{ij})}\dfrac{\partial M(\mathbf{p}_{ij})}{\partial \left[ {M}(\mathbf{m}_{wh}),\cdots, {M}(\mathbf{m}_{(w+1)(h+1)}) \right]^\top}\\ 
&= \dfrac{\begin{bmatrix}
a_1b_1,a_0b_1,a_1b_0,a_0b_0
\end{bmatrix}}{N(\mathbf{p}_{ij})}
\end{aligned}
\end{equation}
where $\mathbf{m}_{wh}, \cdots, \mathbf{m}_{(w+1)(h+1)}$ are the four nearest cell vertices to $\mathbf{p}_{ij}$ in occupancy map $\mathbb{M}$, and $a_0,a_1,b_0$ and $b_1$ are defined in (\ref{eq_interp}).


\subsection{Jacobian of the Odometry Term}\label{Sec_J_O}
The Jacobian $\mathbf{J}_O$ of function $F_i^O(\mathbf{x})$ in the odometry term (\ref{eq_odometry_term}) is the partial derivative w.r.t. the robot poses $\mathbf{x}^P$ since it is not related to the occupancy map in the state vector $\mathbf{x}$. Therefore, the Jacobian $\mathbf{J}_O$ can be calculated as
\begin{equation}
\begin{aligned}
\mathbf{J}_O &= \frac{\partial F_i^O(\mathbf{x})}{\partial \left[ {\mathbf{x}^P_{i-1}}^\top, {\mathbf{x}^P_i}^\top \right]^\top }\\ 
&=\begin{bmatrix}
	 \dfrac{\partial F_i^O(\mathbf{x})}{\partial \mathbf{t}_{i-1}} &
	 \dfrac{\partial F_i^O(\mathbf{x})}{\partial \theta_{i-1}} &
	 \dfrac{\partial F_i^O(\mathbf{x})}{\partial \mathbf{t}_i} &
	 \dfrac{\partial F_i^O(\mathbf{x})}{\partial \theta_i}
 \end{bmatrix} 
 \\
 &=\begin{bmatrix}
 	-\mathbf{R}_{i-1} & \mathbf{R}_{i-1}^\prime(\mathbf{t}_i-\mathbf{t}_{i-1}) & \mathbf{R}_{i-1} &\mathbf{0}_2\\
 	\mathbf{0}_2^\top & -1 & \mathbf{0}_2^\top & 1\\
 \end{bmatrix}
\end{aligned}
 \end{equation}
in which $\mathbf{0}_2$ means $2 \times 1$ zero vector.
 
\subsection{Jacobian of the Smoothing Term}\label{Sec_J_S}

The Jacobian $\mathbf{J}_S$ of function $F^S(\mathbf{x})$ in the smoothing term is the derivative of (\ref{eq_smoothing_term}) w.r.t. cell vertices of occupancy map $\mathbf{x}^M$ 
due to it is not related to the robot poses $\mathbf{x}^P$ in the state vector $\mathbf{x}$. It should be mentioned that $F^S(\mathbf{x})$ is linear w.r.t. $\mathbf{x}^M$
\begin{equation}
F^S(\mathbf{x}) = \mathbf{A} \left[ {M}(\mathbf{m}_{00}),\cdots,{M}(\mathbf{m}_{c_wc_h}) \right]^\top
\end{equation}
where the $(2c_wc_h+c_w+c_h) \times ((c_w+1)(c_h+1))$ coefficient matrix $\mathbf{A}$ is sparse and with nonzero elements $1$ or $-1$. An example of the coefficient matrix can be shown as
\begin{equation}\label{eq_A}
	\mathbf{A} = \begin{bmatrix}
    \vdots &\vdots  &\vdots  &\vdots  &\vdots  &\vdots  &\vdots &\vdots\\
 	\mathbf{0}^\top & 1 & -1 & 0 & \mathbf{0}^\top & 0 & 0 & \mathbf{0}^\top\\
 	\mathbf{0}^\top & 1 & 0  & 0 & \mathbf{0}^\top & -1 & 0 & \mathbf{0}^\top\\
 	\mathbf{0}^\top & 0 & 1 & -1 & \mathbf{0}^\top & 0 & 0 & \mathbf{0}^\top\\
 	\mathbf{0}^\top & 0 & 1 & 0 & \mathbf{0}^\top & 0 & -1 & \mathbf{0}^\top\\
    \vdots &\vdots  &\vdots  &\vdots  &\vdots  &\vdots  &\vdots &\vdots\\
 \end{bmatrix}.
\end{equation}
Here $\mathbf{0}$ represents a zero vector with appropriate dimensions. Therefore, the Jacobian of the smoothing term can be calculated as
\begin{equation}\label{eq_JS}
\mathbf{J}_S = \frac{\partial F^S(\mathbf{x})}{\partial \mathbf{x}^M } = \mathbf{A}.\\ 
\end{equation}
Since $\mathbf{A}$ is constant, $\mathbf{J}_S$ can be pre-calculated and directly used in the optimization as shown in Algorithm \ref{alg_1}.

\subsection{Jacobians in the Second Stage of Multi-resolution Strategy for Optimization}\label{Sec_J_Select}
In the second stage of the multi-resolution strategy (Algorithm \ref{alg_3}), the Jacobians to be calculated are similar to those in Algorithm \ref{alg_1}. A specific challenge arises in handling the selected cell vertices adjacent to the dropped cell vertices in the high-resolution map, particularly when calculating Jacobians $\mathbf{J}_P$ and $\mathbf{J}_S$.

 For Jacobian $\mathbf{J}_P$, partial derivatives w.r.t. all the cell vertices are required for (\ref{eq_14}). However, not all vertices are included in the state vector in the second stage, which makes it challenging to compute the partial derivatives w.r.t. some cell vertices because their surrounding nodes are discarded. From a semantic perspective, the discarded cell vertices have the same occupancy state as the edge nodes, which is why they are excluded. Consequently, the gradient of these edge vertices is expected to be close to zero. Based on this reasoning, we set the partial derivatives w.r.t. all edge cell vertices to $0$ when they need to be calculated using (\ref{eq_14}). 
 
 For Jacobian $\mathbf{J}_S$, it can also be calculated using the same idea as (\ref{eq_JS}). In the second stage of our multi-resolution strategy, (\ref{eq_JS}) is reformulated as 
 \begin{equation}
 	\mathbf{J}_S = \frac{\partial F^S_{s}(\mathbf{x}^s)}{\partial \mathbf{x}^{sM} } = \mathbf{A}^{s}\\ 
 \end{equation}
 where $\partial F^S_{s}(\mathbf{x}^s)$ is similar to (\ref{eq_smoothing_term}), but only applies to vertices in $\mathbb{M}^s$. The coefficient matrix $\mathbf{A}^{s}$ has the same form as (\ref{eq_A}) but with dimension corresponding to the number of elements in $\mathbf{x}^{sM}$.  
 
 }   





\bibliographystyle{IEEEtran}
\documentclass[lettersize,journal]{IEEEtran}
\usepackage{amsmath,amsfonts}
% \usepackage{algorithm}
% \usepackage{algorithmic}
% \usepackage{algpseudocode}

\usepackage[linesnumbered,ruled,vlined]{algorithm2e}


\usepackage{array}
\usepackage{accents}
%\usepackage[caption=false,font=normalsize,labelfont=sf,textfont=sf]{subfig}
%\usepackage[caption=false,font=footnotesize,labelfont=sf,textfont=sf]{subfig}
\usepackage{subfigure}

\usepackage{textcomp}
\usepackage{stfloats}
\usepackage{url}
\usepackage{verbatim}
\usepackage{graphicx}
\usepackage{cite}
\usepackage{float}
 \usepackage{tabularx} 
% \usepackage{setspace}

\hyphenation{op-tical net-works semi-conduc-tor IEEE-Xplore}
\def\BibTeX{{\rm B\kern-.05em{\sc i\kern-.025em b}\kern-.08em
    T\kern-.1667em\lower.7ex\hbox{E}\kern-.125emX}}
\usepackage{balance}
% updated with editorial comments 8/9/2021

% use bib--added by yingyu
% \usepackage[numbers]{natbib}
\usepackage{amssymb}
\usepackage{booktabs}
\usepackage{threeparttable}
\usepackage{multirow}
\usepackage{bigstrut}
\usepackage{bigdelim}
\usepackage{adjustbox}
\usepackage{makecell}
\usepackage{nicematrix}
\usepackage{xcolor}

% \usepackage{xcolor} 
\usepackage{tikz} 
\usetikzlibrary{arrows,shapes,chains}

\usepackage{colortbl}
% \usepackage[table]{xcolor}
\usepackage{rotating}
% Beamer presentation requires \usepackage{colortbl} instead of \usepackage[table,xcdraw]{xcolor}


% \renewcommand{\algorithmicrequire}{\textbf{Input:}}
% \renewcommand{\algorithmicensure}{\textbf{Output:}}
\setlength{\textfloatsep}{8pt}
% \usepackage{caption}
% \captionsetup[figure]{skip=5pt}

% \setlength{\topsep}{0pt}
% \setlength{\belowdisplayskip}{10pt}
% \setlength{\abovedisplayskip}{10pt}

\newcommand*{\defeq}{\stackrel{\text{def}}{=}}
\DeclareMathOperator{\wrap}{wrap}
\DeclareMathOperator{\diag}{diag}
\DeclareMathOperator{\round}{round}




\begin{document}

\title{Occupancy-SLAM: An Efficient and Robust Algorithm for Simultaneously Optimizing Robot Poses and Occupancy Map}

\author{\authorblockN{Yingyu Wang, Liang Zhao, and Shoudong Huang} 
        % <-this % stops a space
\thanks{Yingyu Wang and Shoudong Huang are with the Robotics Institute, University of Technology Sydney, Australia (e-mail: Yingyu.Wang-1@student.uts.edu.au; Shoudong.Huang@uts.edu.au).} 

\thanks{Liang Zhao was with the Robotics Institute, University of Technology Sydney, Australia, and is now with the School of Informatics, University of Edinburgh, Edinburgh, UK (e-mail: Liang.Zhao@ed.ac.uk).}}



% The paper headers
\markboth{IEEE TRANSACTIONS ON ROBOTICS}%
{Shell \MakeLowercase{\textit{et al.}}: A Sample Article Using IEEEtran.cls for IEEE Journals}

% \IEEEpubid{0000--0000/00\$00.00~\copyright~2021 IEEE}
% Remember, if you use this you must call \IEEEpubidadjcol in the second
% column for its text to clear the IEEEpubid mark.

\maketitle

% This paper proposes Occupancy-SLAM, an optimization-based SLAM approach that jointly optimizes the robot trajectory and the occupancy map simultaneously.

\begin{abstract}
Joint optimization of poses and features has been extensively studied and demonstrated to yield more accurate results in feature-based SLAM problems. However, research on jointly optimizing poses and non-feature-based maps remains limited. Occupancy maps are widely used non-feature-based environment representations because they effectively classify spaces into obstacles, free areas, and unknown regions, providing robots with spatial information for various tasks. In this paper, we propose Occupancy-SLAM, a novel optimization-based SLAM method that enables the joint optimization of robot trajectory and the occupancy map through a parameterized map representation. The key novelty lies in optimizing both robot poses and occupancy values at different cell vertices simultaneously, a significant departure from existing methods where the robot poses need to be optimized first before the map can be estimated. 

This paper focuses on 2D laser-based SLAM to investigate how to jointly optimize robot poses and the occupancy map. In our formulation, the state variables in optimization include all the robot poses and the occupancy values at discrete cell vertices in the occupancy map. Moreover, a multi-resolution optimization framework that utilizes occupancy maps with varying resolutions in different stages is introduced. A variation of Gauss-Newton method is proposed to solve the optimization problem at different stages to obtain the optimized occupancy map and robot trajectory. The proposed algorithm is efficient and converges easily with initialization from either odometry inputs or scan matching, even when only limited key-frame scans are used. Furthermore, we propose an occupancy submap joining method, enabling more effective handling of large-scale problems by incorporating the submap joining process into the Occupancy-SLAM framework. Evaluations using simulations and practical 2D laser datasets demonstrate that the proposed approach can robustly obtain more accurate robot trajectories and occupancy maps than state-of-the-art techniques with comparable computational time. Preliminary results in the 3D case further confirm the potential of the proposed method in practical 3D applications, achieving more accurate results than existing methods. The code is made available to benefit the robotics community\footnote{\url{https://github.com/WANGYINGYU/Occupancy-SLAM}}. 
\end{abstract}

\begin{IEEEkeywords}
SLAM, optimization, occupancy grid map, non-feature-based map representation.
\end{IEEEkeywords}

\section{Introduction}

% optimizing robot poses and features simultaneously 


\IEEEPARstart{S}{imultaneous} localization and mapping (SLAM) is an important problem in robotics that has been studied for decades \cite{cadena2016past}. Jointly optimizing the robot poses and map can enhance SLAM performance, as this formulation utilizes the available information more directly without approximations. While joint optimization has been widely explored in feature-based SLAM (e.g., \cite{kaess2008isam,kaess2011isam2}), research on the joint optimization of robot poses and non-feature-based maps remains limited.

Occupancy grid maps are widely used in robotic tasks for their ability to clearly represent obstacles, free space, and unknown areas, facilitating collision-free navigation and path planning. Assuming the robot poses used to collect the sensor information are known exactly, the evidence grid mapping technique \cite{moravec1985high,
moravec1989sensor,elfes1989occupancy,martin1996robot,hornung2013octomap} provides an elegant and efficient approach for building occupancy grid maps from the collected information. However, when a robot is navigating in an unknown environment and performing SLAM, its own poses need to be estimated, and the estimates inherently contain uncertainties. Achieving both accurate robot localization and precise occupancy mapping simultaneously is not trivial.

% How to perform accurate robot localization and build an occupancy map very accurately at the same time is not trivial. 


In some occupancy grid map based SLAM approaches such as Cartographer \cite{hess2016real}, the problem is tackled in two steps. First, the robot poses are estimated by solving a pose-graph SLAM problem, where the relative pose measurements are derived using odometry, scan matching, loop closure detection, or other similar techniques. Second, the optimized poses are assumed to be the correct poses and are used to build up the map using evidence grid mapping techniques. However, in these two-step approaches, the uncertainties in the robot poses obtained during the first step are not considered when building the map. Therefore, it is crucial to achieve highly accurate pose estimates to construct a reliable occupancy grid map. As a result, it can be expected that the occupancy map obtained using a two-step approach may not represent the best result that one can achieve using all the available information.


In feature-based SLAM approaches, jointly optimizing the poses and the feature map is common, as the relationship between observations and the map is straightforward to model. However, for occupancy map based SLAM, jointly optimizing the robot poses and the occupancy map is not trivial because: 
\begin{itemize}
	\item [1)] \textbf{The relation between the observations and the map is complex.} The observations are laser beams (the endpoint of a beam represents ``hit" and the other positions along the beam represent ``free"), and the map is a function representing the occupancy values at different positions. This is significantly different from feature-based SLAM where both the observations and the map are about feature parameters such as positions.
	\item [2)] \textbf{The data association is not easy to do.} When the robot poses are noisy, the correct correspondence between laser beams and occupancy grid cells is hard to find. In contrast, for feature-based SLAM, there are well-established front-end methods for data association.
	\item [3)] \textbf{The resolution of the map has a significant impact on the optimization problem.} A high-resolution map helps to establish a more accurate connection between the observations and the map, but it leads to a sharp increase in the number of variables. However, for feature-based SLAM, there is no such issue.
\end{itemize}

% using 2D laser scans (and odometry) information.

\subsection{Contributions}
In this paper, we propose Occupancy-SLAM algorithm, which jointly optimizes the robot poses and the occupancy map using 2D laser scans (and odometry) information. Moreover, we propose a multi-resolution optimization framework for improving convergence and robustness to initial guesses. To better handle the case of large-scale environments and long-term trajectories, we further propose an occupancy submap joining method. Experiments on both simulated and practical datasets verify the superior performance of our method compared with state-of-the-art approaches (e.g., Cartographer \cite{hess2016real}). In addition, we extend our method to the 3D case, and preliminary results confirm its effectiveness in improving accuracy. The main contributions are summarized as follows: 

 % A smoothing term is introduced in the objective function to improve the convergence of the method.

\begin{enumerate}
	\item We formulate the occupancy grid map based SLAM problem as a joint optimization problem where the poses and the occupancy map are optimized together. 
	\item We propose a variation of Gauss-Newton method to solve the new formulation, enabling the estimation of more accurate robot poses and occupancy maps compared to existing state-of-the-art techniques.
	\item To enhance efficiency, convergence, and robustness, we propose a multi-resolution optimization strategy using occupancy maps of different resolutions across stages.
    
    % To improve the efficiency, convergence and robustness of our algorithm so that it can be initialized by odometry inputs or scan matching, we propose a multi-resolution optimization strategy that uses occupancy maps with different resolutions at different optimization stages. In the second stage, we utilize the selected high-resolution map, focusing exclusively on a subset of \textcolor{red}{cell vertices} that require further updates within the full high-resolution map. This targeted approach further enhances computational efficiency.
    
    \item We propose a submap joining algorithm to address the cases of large-scale environments and long-term trajectories through our joint poses and occupancy map optimization idea.
	\item Our method achieves robust convergence even with key frames of limited overlap, outperforming state-of-the-art approaches like Cartographer in efficiency while maintaining superior accuracy.
    % Our proposed method can converge well even when only key frames with limited overlaps are used. In this case, our method outperforms state-of-the-art methods, such as Cartographer, in terms of efficiency while maintaining a surpassing performance in terms of accuracy.
    \item We extend our method to 3D, with preliminary results demonstrating superior accuracy compared to other state-of-the-art approaches.
    % We extend our method to the 3D case, and preliminary results confirm that the accuracy of our method outperforms other state-of-the-art approaches.
\end{enumerate}


This paper is an extension to our conference paper \cite{Zhao-RSS-22}, with major improvements in contributions 3, 4, 5, and 6, significantly enhancing the robustness and efficiency of the algorithm while extending the method to 3D.

% The major improvements of this paper over \cite{Zhao-RSS-22} are contributions 3, 4, 5, and 6, which significantly improve the robustness and efficiency of the algorithm, and extend the algorithm to 3D.

\begin{figure}
\centering
\includegraphics[width=0.49\textwidth]{./OverView.pdf}
\caption{\label{fig_overview} Main components of our proposed method. The blue-colored components represent our core approaches, while the dashed portions are optional. Our multi-resolution joint optimization is covered in Section \ref{sec_formulation}, Section \ref{Sec_Algorithm_1}, and Section \ref{Sec_multi}. The joint global map and robot trajectory optimization approach is presented in Section \ref{Sec_submap}. }
\end{figure}

% \subsection{Notations}
% Some important notations in this paper are summarized in Table \ref{tab_notation}, the others are described in the context.

\subsection{Outline}
Fig. \ref{fig_overview} illustrates the flowchart of applying our proposed methods in practice. The blue components represent our core approaches, while the dashed portions are optional. The rest of the paper is organized as follows: Section \ref{Sec_related_work} provides a review of related work on non-feature-based SLAM, submap joining, and joint optimization of poses and maps. In Section \ref{sec_formulation}, we introduce our novel formulation for jointly optimizing the robot poses and occupancy map. A variation of the Gauss-Newton method to solve our nonlinear least squares (NLLS) formulation is presented in Section \ref{Sec_Algorithm_1}. In Section \ref{Sec_multi}, we introduce our multi-resolution strategy to improve the efficiency and robustness of the algorithm. Section \ref{Sec_submap} presents our submap joining algorithm for handling large-scale environments and long-term trajectories. Experimental results are provided in Section \ref{Sec_experiment}. In Section \ref{sec_3d}, we extend our method to the 3D case and present preliminary results. Finally, the conclusions are given in Section \ref{Sec_conclusion}.

 
\section{Related Work}\label{Sec_related_work}

In this section, we discuss some related work on non-feature based map representations for SLAM, submap joining techniques, and joint optimization of poses and maps. 

\subsection{Non-feature based map representations for SLAM}\label{Sec_related_a}
One widely used non-feature based SLAM approach is occupancy grid map-based SLAM, which probabilistically classifies spaces into obstacles, free areas, and unknown regions while accounting for uncertainty during observation updates \cite{moravec1985high, moravec1989sensor, elfes1989occupancy, martin1996robot, hornung2013octomap}. Classic examples, such as FastSLAM \cite{montemerlo2002fastslam} and GMapping \cite{grisetti2005improving}, use particle filters for mapping and localization but struggle with high computational demand and long-term accuracy in large-scale environments.

Recent optimization-based approaches, such as Hector SLAM \cite{kohlbrecher2011flexible}, Karto-SLAM \cite{konolige2010efficient}, and Cartographer \cite{hess2016real}, address cumulative errors effectively. Hector SLAM uses scan-to-map matching but lacks loop closure, restricting it to small-scale scenarios. Karto-SLAM incorporates loop closure detection with sparse pose adjustment for global optimization, while Cartographer integrates scan-to-map matching and pose graph optimization with a branch-and-bound strategy for efficient loop closure detection. However, by treating pose optimization and map construction as independent processes, these methods fail to account for the interdependencies of their uncertainties.

Multi-resolution occupancy mapping techniques can be integrated into occupancy grid map based SLAM frameworks to enable a more compact and efficient mapping process. For instance, approaches like OctoMap \cite{hornung2013octomap} use memory-efficient octrees to balance map compactness and accessibility. Adaptive-resolution methods, such as RMAP \cite{khan2014rmap} and ColMap \cite{fisher2021colmap}, dynamically adjust grid resolution to enhance mapping efficiency. Recently, \cite{Reijgwart-RSS-23} applies wavelet compression for hierarchical occupancy map storage, allowing efficient updates and queries. However, integrating multi-resolution maps as state variables into a unified framework for joint poses and map optimization remains an open challenge.


Another widely used non-feature-based map is the signed-distance function (SDF), which discretizes the environment into grid cells storing the distance to the nearest surface. This representation encodes the space, with the object surfaces defined by the zero crossings of the distance functions \cite{curless1996volumetric}. Some SLAM systems adopt SDF to improve localization accuracy and mapping quality. For example, supereight \cite{vespa2018efficient} integrates tracking, mapping, and planning using an octree-based truncated SDF (TSDF). It aligns camera frames to the TSDF map with iterative closest point (ICP) \cite{besl1992method}. A follow-up work \cite{vespa2019adaptive} improves this by introducing adaptive-resolution octree structures, achieving denser environment representation and reduced noise, leading to more accurate localization.

Other non-feature based map representations have also been used in SLAM, including mesh-based \cite{rosinol2021kimera}, normal distributions transform based \cite{einhorn2015generic}, neural radiance fields based \cite{rosinol2023nerf} and Gaussian splatting based \cite{matsuki2024gaussian}. Although these approaches differ in the type of non-feature representations they use, they all aim to provide more effective environmental modeling, improve robot localization accuracy, or achieve both. 


However, all the optimization-based SLAM approaches that utilize non-feature based maps need to optimize the poses first and then build the non-feature based map using the optimized poses. This separation prevents these approaches from jointly considering the uncertainties in both localization and mapping during the optimization process. In contrast, this paper considers unifying the optimization of both the robot poses and occupancy values at each cell vertex of the occupancy map into a single optimization problem, which can be expected to yield better accuracy.

\subsection{Submap Joining}\label{Sec_related_b}

Submap joining, as proposed by \cite{bosse2003atlas}, is a widely used scheme for SLAM in large-scale environments due to its efficiency and reduced risk of being trapped in local minima compared to full optimization-based SLAM. Feature-based submap joining approaches \cite{huang2008sparse,zhao2013linear,wang2019submap} have been well investigated, with many demonstrating properties that enable efficient problem-solving while maintaining a high level of accuracy. To extend non-feature-based SLAM to large-scale environments and long-term operations, recent efforts have explored non-feature-based submap joining methods.


For example, \cite{wagner2014graph} divides the environment into overlapping submaps composed of small TSDF grids from KinectFusion \cite{izadi2011kinectfusion}. Submap joining is then formulated as a pose graph optimization problem, where submap poses are nodes, and relative transformations from ICP are edges. Similarly, VOG-map \cite{ho2018virtual} represents submaps as 3D occupancy grids, converts them to point clouds for ICP-based relative transformations, and solves submap joining via pose graph optimization. Voxgraph \cite{reijgwart2019voxgraph} improves accuracy by employing SDF-to-SDF registration for overlapping submaps created with C-blox \cite{millane2018c}. Unlike time-sequence-based submap partitioning, \cite{wang2021elastic} uses spatial partitioning, merging submaps during loop closures by solving a pose graph containing only submap frames, with reconstruction decisions based on environmental changes.

All the aforementioned non-feature-based submap joining approaches estimate relative measurements between overlapping submaps to formulate and solve the pose graph problem for submap frames. In contrast, this paper jointly optimizes submap frames and the global occupancy map.

\subsection{Joint Optimization of Poses and Maps}
Joint optimization of poses and maps can result in better accuracy, as it utilizes the information more directly. In feature-based SLAM and bundle adjustment approaches, the most common form is to jointly optimize poses and positions of features, such as \cite{dellaert2006square,triggs2000bundle,konolige2008frameslam,sibley2009adaptive,zhao2015parallaxba}. Some approaches extend this idea to planar feature parameters. For instance, \cite{kaess2015simultaneous,hsiao2017keyframe} minimize the difference between plane measured in a scan and predicted planes, while \cite{trevor2012planar,geneva2018lips,zhou2021pi,zhou2021lidar} minimize the Euclidean distance between points in a scan and the predicted planes. Based on the idea of minimizing Euclidean distance between points in scans, BALM \cite{liu2021balm} demonstrates that planar parameters can be solved analytically in closed form, reducing the dimensionality of the optimization. BALM2 \cite{liu2023efficient} further improves efficiency by using point clusters, avoiding individual point enumeration. HBA \cite{liu2023large} introduces a hierarchical structure to address the scalability challenges of BALM and BALM2 in large environments. In summary, jointly optimizing poses and feature-based maps is well-studied, as features naturally link positions, observations, and poses, making them straightforward to integrate into optimization problems. In contrast, establishing constraints between observations, poses, and non-feature-based maps (e.g., occupancy grid maps) for joint optimization remains a significant challenge.



% Optimizing the poses and feature-based map together is very common and has been well-studied, as features are naturally linked to positions, which in turn connect observations, poses, and features, making them straightforward to integrate into optimization problems. However, it is a challenge to establish constraints between observations, poses, and a non-feature based map to jointly optimize the poses and the map (e.g., an occupancy grid map).

% Kimera-PGMO proposed in \cite{rosinol2021kimera} is a novel approach that simultaneously optimizes the poses and the mesh deformation. Specifically, Kimera-PGMO \cite{rosinol2021kimera} creates a deformation graph including a simplified mesh and a pose graph of robot poses. Since the simplified mesh consists of the positions of the mesh vertices, the method is formulated as a factor graph and then solved by GTSAM \cite{dellaert2012factor}.

Research on jointly optimizing the poses and non-feature based maps is limited. Kimera-PGMO proposed in \cite{rosinol2021kimera} represents a notable attempt, integrating pose optimization with mesh deformation. It constructs a deformation graph of a simplified mesh and a pose graph, formulating the problem as a factor graph solvable by GTSAM \cite{dellaert2012factor}. 
While Kimera-PGMO \cite{rosinol2021kimera} has similar motivations as our paper, aiming to achieve better quality maps and more accurate poses through joint optimization, its mesh-based representation differs fundamentally from the occupancy grid maps used in our approach. Meshes are naturally represented through point positions and their relationships, which facilitates factor graph formulations.


% but the mesh can still be described in terms of the positions of the points as well as the relationships between the points, and can therefore ultimately be transformed into a factor graph to be solved for, which is different to the occupancy map that we used.

\begin{table}[t]
 		\centering
 		\caption{Summary of Some Important Notations.}\label{tab_notation}
 		\setlength{\tabcolsep}{0.5 mm}{
 		\begin{tabular}{|c|l|p{3cm}p{3cm}p{3cm}}
   \hline
   \multicolumn{1}{|c|}{Notation} & \multicolumn{1}{|c|}{Explanation} \\ \hline
   $\mathbb{M}$  & \begin{tabular}[c]{@{}l@{}} A set includes occupancy values at all discrete cell vertices in \\occupancy map, as defined in Section \ref{sec_discrete_occupancy}. $\mathbb{M}^{l}$, $\mathbb{M}^{h}$, and $\mathbb{M}^{s}$ \\represent the sets include occupancy values at all cell vertices \\in low-resolution map, high-resolution map and selected \\high-resolution map, respectively. In addition, $\mathbb{M}_L$ and $\mathbb{M}_G$ \\represent the sets including occupancy values of all cell vertices \\in local maps and the global map, as defined in Section \ref{Sec_submap}.\end{tabular}\\ \hline

% as defined in \\Section \ref{continuous_map}
   $M(\cdot)$ &\begin{tabular}[c]{@{}l@{}}A function to lookup occupancy value at an arbitrary position in \\the occupancy map by bilinear interpolation using $\mathbb{M}$.\end{tabular} \\ \hline

    $\mathbf{x}^M$ & \begin{tabular}[c]{@{}l@{}} A vector including occupancy values at all cell vertices in discrete \\occupancy map $\mathbb{M}$, as defined in (\ref{eq_map_state}). $\mathbf{x}^{lM}$ and $\mathbf{x}^{sM}$ are vectors \\including occupancy values at cell vertices in $\mathbb{M}^{l}$ and $\mathbb{M}^{s}$. In \\addition, $\mathbf{x}^M_G$ represents the vector which includes occupancy \\values at cell vertices from $M_G$, as described in Section \ref{Sec_submap}.\end{tabular} \\ \hline

    $N(\cdot)$ & \begin{tabular}[c]{@{}l@{}} A function to lookup hit number at arbitrary position in the map \\by bilinear interpolation using hit map $\mathbb{N}$, where $\mathbb{N}$ is defined as a \\set includes hit number at all discrete cell vertices in the map, as \\described in Section \ref{sec_hit}. $\mathbb{N}^{l}$ and $\mathbb{N}^{s}$ represent hit maps used in \\different optimization stages.\end{tabular}\\ \hline

    $\mathbf{x}^P$ & \begin{tabular}[c]{@{}l@{}} A vector including all robot poses for optimization, as defined \\in (\ref{eq_pose_state}). In addition, $\mathbf{x}^P_L$ denotes a vector including all local map \\coordinate frames for submap joining problem in Section \ref{Sec_submap}.\end{tabular} \\ \hline
   \rule{0pt}{1.5em}
    $\mathbf{x}$ & \begin{tabular}[c]{@{}l@{}} State vector of optimization, $\mathbf{x} = {[{\mathbf{x}^P}^\top, {\mathbf{x}^M}^\top]}^\top$. $\mathbf{x}^{l}$ and $\mathbf{x}^{s}$ \\represent state vectors of different optimization stages. \end{tabular} \\ \hline

    $\mathbb{S}$  &\begin{tabular}[c]{@{}l@{}} $\mathbb{S} = \{\mathbb{S}_i\}_{0 \leq i \leq n}$ where $\mathbb{S}_i$ is defined in (\ref{S_i}), a set including \\observations, as defined in Section \ref{Sec_Info_1}. $\mathbb{S}^{l}$, $\mathbb{S}^{h}$, and $\mathbb{S}^{s}$ are \\observations used for occupancy maps $\mathbb{M}^{l}$, $\mathbb{M}^{h}$, and $\mathbb{M}^{s}$, \\respectively. \end{tabular}\\ \hline

     $s$  &\begin{tabular}[c]{@{}l@{}} Resolution of the occupancy map, which indicates the distance\\ between two nearby cell vertices. $s^{l}$ and $s^{h}$ represent the \\resolutions of low-resolution map and high-resolution map, \\respectively. $s^L$ and $s^G$ denote the resolutions of local maps and \\the global map, respectively, as described in Section \ref{Sec_submap}. \end{tabular}\\ \hline
    
     $\mathbb{O}$ & \begin{tabular}[c]{@{}l@{}}Set including all odometry inputs, as defined in Section \ref{sec_odometry}. \end{tabular}\\ \hline

    $r$  & Ratio between resolutions of two stages, $r = {s^{l}}/{s^h}$.  \\ \hline

    $\mathbf{m}$ & \begin{tabular}[c]{@{}l@{}} Discrete coordinate of a cell vertex, detailed explanation in the \\second paragraph in Section \ref{sec_discrete_occupancy}.\end{tabular} \\ \hline

    $\mathbf{p}$ & \begin{tabular}[c]{@{}l@{}}Continuous coordinate of a point, see the second paragraph in \\Section \ref{sec_relationship}.\end{tabular} \\
        
 		\hline
 		\end{tabular}
 		}
        % \vspace{-1em}

 \end{table}


\section{Problem Formulation}\label{sec_formulation}
Our approach considers the joint optimization of the robot poses and the occupancy map using information from 2D laser observations (and odometry). In this section, we will explain how the observations from the laser can be linked to the robot poses and the occupancy map to formulate the NLLS problem. 

% We also explain how we improve the problem formulation to make it easier to solve by adding a smoothing term. 

\subsection{Notation}
Throughout this paper, unless otherwise noted, we use specific typographical conventions: typefaces denote sets, bold uppercase letters represent matrices, bold lowercase letters indicate vectors, and regular (unbolded) lowercase letters signify scalars. Key notations used in this paper are summarized in Table \ref{tab_notation}, while others are introduced within the text as needed.

\subsection{Occupancy Map Representation and State in Optimization} \label{sec_discrete_occupancy}
Suppose the environment is discretized into $c_w\times c_h$ grid cells. We use $\mathbf{m}_{wh}=[w,h]^\top~(0 \leq w \leq c_w, 0 \leq h \leq c_h)$ to represent the coordinate of a discrete cell vertex in the map. The occupancy value at the cell vertex $\mathbf{m}_{wh}$, denoted as $M(\mathbf{m}_{wh})$, is defined using evidence, which is the natural logarithm of odds (the ratio between the probability of being occupied and the probability of being free) \cite{martin1996robot,hornung2013octomap,ProbabilisticRobotics}. 
The occupancy values of all $(c_w+1) \times (c_h+1)$ cell vertices consist of the discrete occupancy map $\mathbb{M}=\{M(\mathbf{m}_{wh})\}_{0 \leq w \leq c_w, 0 \leq h \leq c_h}$.


To represent the entire environment using a finite number of parameters, we describe the occupancy value at an arbitrary position $\mathbf{p}_m=[x,y]^{\top}$ on the map using bilinear interpolation of the occupancy values at its four surrounding cell vertices: $\mathbf{m}_{wh}, \mathbf{m}_{({w+1})h}, \mathbf{m}_{w({h+1})}, \mathbf{m}_{({w+1})({h+1})}$, as shown in Fig. \ref{fig_interpolation}, i.e.,

\begin{equation}
	M(\mathbf{p}_{m})= \begin{bmatrix}
a_1b_1,a_0b_1,a_1b_0,a_0b_0
\end{bmatrix}\left[
\begin{aligned}\label{eq_interp}
&M(\mathbf{m}_{wh})\\&M(\mathbf{m}_{(w+1)h})\\&M(\mathbf{m}_{w(h+1)})\\&M(\mathbf{m}_{(w+1)(h+1)})
\end{aligned}\right] 
\end{equation}
in which 
\begin{equation}
\begin{aligned}
	a_0 &= x - w\\
	a_1 &= w+1 - x\\
	b_0 &= y - h\\
	b_1 &= h+1 - y .\\
\end{aligned} 
\end{equation}


\begin{figure}[t]
\centering
\includegraphics[width=0.48\textwidth]{interpolation.pdf}
\caption{\label{fig_interpolation} Parameterizing the entire map by bilinear interpolation of discrete map $\mathbb{M}$.}
% \vspace{-1em}
\end{figure}


Our method jointly optimizes robot poses and the occupancy map, combining them into the state vector of the proposed optimization problem. Using bilinear interpolation with the discrete occupancy map $\mathbb{M}$, estimating the entire map is equivalent to estimating $\mathbb{M}$. Thus, the map component of the state vector can be expressed as

\begin{equation}
    \mathbf{x}^M =\left[M(\mathbf{m}_{00}),\cdots,M(\mathbf{m}_{c_wc_h}) \right]^\top. \label{eq_map_state}
\end{equation}


We define the $n+1$ robot poses as \rule{0pt}{1em}$\{\mathbf{x}^P_i \triangleq [\mathbf{t}_i^\top,\theta_i]^\top\}_{0 \leq i \leq n}$, where $\mathbf{t}_i$ is the $x$-$y$ position of the robot and $\theta_i$ is the orientation with the corresponding rotation matrix $\mathbf{R}_i=\begin{bmatrix}
\cos(\theta_i), \sin(\theta_i)\\ -\sin(\theta_i), \cos(\theta_i)
\end{bmatrix}$. As in most of the SLAM problem formulations, we assume the first robot pose defines the coordinate system, $\mathbf{x}^P_0 \triangleq [0,0,0]^\top$, so only the other $n$ robot poses are variables that need to be estimated, thus the pose component of the state vector is represented as
\begin{equation}
    \mathbf{x}^P = \left[ (\mathbf{x}^P_1)^\top, \cdots, (\mathbf{x}^P_n)^\top \right]^\top.
\end{equation}


Accordingly, the state vector of the proposed optimization problem is
\begin{equation}
    \mathbf{x} = \left[{(\mathbf{x}^P)}^\top,{(\mathbf{x}^M)}^\top \right]^\top. \label{eq_pose_state}
\end{equation}

In our method, the occupancy map $\mathbb{M}$ is initialized by the Bayesian occupancy mapping method \cite{ProbabilisticRobotics} with initially estimated poses (derived from odometry or scan matching) and updated throughout the optimization process.


% i.e., 
% \begin{equation}
%     \mathbf{x} = \left[{(\mathbf{x}^P)}^\top,{(\mathbf{x}^M)}^\top \right]^\top,
% \end{equation}
% where $\mathbf{x}^M$ is the map part and $\mathbf{x}^P$ is the poses part. By bilinear interpolation method and discrete occupancy map $\mathbb{M}$, we only need to estimate $(c_w+1)\times(c_h+1)$ parameters to estimate the entire map, thus $\mathbf{x}^M$ can be expressed as 
% \begin{equation}
%     \mathbf{x}^M &= \left[M(\mathbf{m}_{00}),\cdots,M(\mathbf{m}_{c_wc_h}) \right]^\top.
% \end{equation}
% We suppose that the $n+1$ robot poses are expressed by \rule{0pt}{1em}$\{\mathbf{x}^P_i \triangleq [\mathbf{t}_i^\top,\theta_i]^\top\}_{0 \leq i \leq n}$, where $\mathbf{t}_i$ is the $x$-$y$ position of the robot and $\theta_i$ is the orientation with the corresponding rotation matrix $\mathbf{R}_i=\begin{bmatrix}
% \cos(\theta_i), \sin(\theta_i)\\ -\sin(\theta_i), \cos(\theta_i)
% \end{bmatrix}$. As in most of the SLAM problem formulations, we assume the first robot pose defines the coordinate system, $\mathbf{x}^P_0 \triangleq [0,0,0]^\top$, so only the other $n$ robot poses are variables that need to be estimated. Thus, the state variables of the part of the pose can be expressed as 
% \begin{equation}
%     \mathbf{x}^P = \left[ (\mathbf{x}^P_1)^\top, \cdots, (\mathbf{x}^P_n)^\top \right]^\top.
% \end{equation}

% the state in our optimization problem can be represented as 
% \begin{equation}
%     \mathbf{x} = \left[{(\mathbf{x}^P)}^\top,{(\mathbf{x}^M)}^\top \right]^\top,
% \end{equation}
% where
% \begin{equation}
% \begin{aligned}
% \mathbf{x}^P &= \left[ (\mathbf{x}^P_1)^\top, \cdots, (\mathbf{x}^P_n)^\top \right]^\top\\
% \mathbf{x}^M &= \left[M(\mathbf{m}_{00}),\cdots,M(\mathbf{m}_{c_wc_h}) \right]^\top.
% \end{aligned}\label{eq_state_vec}
% \end{equation}

% The occupancy map is updated as the optimization process progresses without the need for additional occupancy mapping update methods, and the Bayesian occupancy mapping method is used as map initialization in our optimization problem.


% , so there is no need for an additional mapping update strategy, and furthermore, this makes the method of initializing the occupancy value $M(\mathbf{m}_{wh})$ not very critical.




% \subsection{The Available Information}\label{Sec_Info}

% The available information includes 2D laser scans collected at different robot poses. In addition, the odometry information might also be available. 

\subsection {Scan Points Sampling Strategy}\label{Sec_Info_1} 

 % We use the evidence, which is the natural logarithm of odds (the ratio between the probability of being occupied and the probability of being free) \cite{martin1996robot,hornung2013octomap,ProbabilisticRobotics} to represent the occupancy value.

 \begin{figure}[tbp]
\centering 
\subfigure[Equidistant Sampling Strategy] {\label{fig_sampling_strategy}
\includegraphics[width=0.23\textwidth]{./sampling_strategy.pdf}}
\subfigure[Observation Points in One Scan] {\label{fig_scan}
\includegraphics[width=0.23\textwidth]{./scan.pdf}}
\caption{Sampling strategy for generating observations from a laser scan: (a) Equidistant sampling on a beam, with red indicating occupied and blue indicating free states. The distance between two consecutive points is the resolution $s$. (b) All sampled observation points at a given time step.}
\label{fig_scan_sampling}
% \vspace{-1em}
\end{figure}

We now introduce our sampling strategy for generating observations from laser scans, which are used in our NLLS formulation. 

Each scan data consists of a number of beams. On each beam, the endpoint indicates the presence of an obstacle, while the other points before the endpoint indicate the absence of obstacles. Here, we sample each beam using a fixed resolution $s$ to get the observations, as shown in Fig. \ref{fig_sampling_strategy}. Specifically, $\mathbf{q}_{ij}=[x_{q_{ij}},y_{q_{ij}}]^\top$ denotes the position of $j$th sampling point at time step $i$ in the local robot/laser coordinate frame and
\begin{equation}
z_{ij} = \ln \frac{p(\mathbf{q}_{ij} \in occ)}{1-p(\mathbf{q}_{ij} \in occ)} \label{eq_occ_obs}
\end{equation}denotes the corresponding occupancy value. In the same way as the occupancy map representation described in Section \ref{sec_discrete_occupancy}, we also use the evidence to represent the occupancy value here. In our implementation, following \cite{hornung2013octomap,ProbabilisticRobotics}, we use $p(\mathbf{q}_{ij} \in occ) = 0.7$ for an occupied point (red in Fig. \ref{fig_sampling_strategy}), and use $p(\mathbf{q}_{ij} \in occ) = 0.4$ for a free point (blue in Fig. \ref{fig_sampling_strategy}). Fig. \ref{fig_scan} shows an example of all sampled points in one scan.
% $(0 \leq i \leq n)
By constant equidistant sampling of all the beams for the scan collected at time step $i$, a sampling point set
\begin{equation}
\mathbb{S}_i=\{ \mathbb{S}_{ij} \triangleq \{\mathbf{q}_{ij},z_{ij}\}\}_{1 \leq j \leq k_i}
\label{S_i}
\end{equation}
can be obtained. It should be noted that since the total length of all the beams at different time step $i$ is different, the number of sampling points $k_i$ obtained by the equidistant sampling strategy varies for different time step $i$.


Suppose there are $n+1$ laser scans collected from robot poses $0$ to $n$, $\mathbb{S}=\{\mathbb{S}_i\}_{0\leq i \leq n}$ is the available observation information collected at all different robot poses using our sampling strategy and will be used as observations in our NLLS formulation.

\subsection{Relationship Between Observations and Occupancy Map}\label{sec_relationship}


In Section \ref{sec_discrete_occupancy}, we defined the discrete occupancy map $\mathbb{M}$ as part of the state vector in our optimization problem. Section \ref{Sec_Info_1} detailed the observation generation process. In this section, we explain how the relationship between observations and the occupancy map is established through robot poses, forming the basis of our joint optimization problem.



\subsubsection{Local to Global Projection}
First, the $j$th scan point at time step $i$ can be projected to the occupancy map using the robot pose $\mathbf{x}^P_i$, and the projected position on the occupancy map can be calculated by 

\begin{equation}
	\mathbf{p}_{ij}
=\frac{\mathbf{R}_i^\top \mathbf{q}_{ij}+\mathbf{t}_i}{s} \label{P-project}
\end{equation}
where $s$ is the resolution of the vertices in the occupancy map $\mathbb{M}$ (the distance between two adjacent cell vertices represents $s$ meters in the real world). Here, we use the same resolution as that used in generating observations from laser scans in Section \ref{Sec_Info_1}. Then, the occupancy value at the projected point $\mathbf{p}_{ij}$ can be obtained using (\ref{eq_interp}), expressed as $M(\mathbf{p}_{ij})$.


\subsubsection{Relationship Between Sampling Points and Occupancy Map w.r.t. Occupancy Values}
As outlined in Sections \ref{sec_discrete_occupancy} and \ref{Sec_Info_1}, evidence is used to define the occupancy value, where multiple observations of the same cell result in the occupancy values from individual observations being cumulatively added to the cell's total occupancy value \cite{hornung2013octomap}. If the robot's poses are accurate and repeated observations of the same cell consistently indicate the same occupancy state, the cell's occupancy value becomes the product of the occupancy value of each observation and the number of times the cell is observed. For a unique coordinate $\mathbf{p}_{ij}$ in (\ref{P-project}), if both the robot pose $\mathbf{x}^P_i$ and the occupancy map $\mathbb{M}$ are accurate, the occupancy value $z_{ij}$ of its associated sampling point $\mathbb{S}_{ij}$, should closely approximate the occupancy value at $\mathbf{p}_{ij}$, $M(\mathbf{p}_{ij})$, divided by the number of times $\mathbf{p}_{ij}$ is ``observed", $N(\mathbf{p}_{ij})$.

Thus, if the number of times the point $\mathbf{p}_{ij}$ is ``observed" can be calculated, the relationship between the observations and the state vector (occupancy map and robot poses), can be determined.


\subsubsection{Hit Map and Hit Number Lookup}\label{sec_hit}

We now explain how $N({\mathbf{p}_{ij}})$ can be calculated. To quickly query the number of times an arbitrary point is ``observed", we need to count the number of times all cell vertices have been observed to form the discrete hit map $\mathbb{N}$ associated with the occupancy map $\mathbb{M}$. 

When a sampling scan point is projected into a coordinate by a given robot pose, this coordinate is considered to have been observed once, and then we distribute the hit number ``1" of the coordinate to the discretized cell vertices. Since the occupancy value $M(\mathbf{p}_{ij})$ is derived by bilinear interpolation of occupancy values of discrete cell vertices in (\ref{eq_interp}), in order to maintain the correspondence between the hit number and the occupancy values, we distribute this ``1" hit to the four surrounding cell vertices by inverse bilinear interpolation. For example, if a sampling point is projected into the center of a cell, then each of the 4 nearby cell vertices gets a hit number of 0.25. In addition, the hit number also accumulates with multiple observations of the same cell vertex, i.e.,
\begin{equation}
\left[ N(\mathbf{m}_{00}),\cdots,N(\mathbf{m}_{c_wc_h}) \right] 
= \sum_{i=0}^n \sum_{j=1}^{k_i} H(\mathbf{p}_{ij})
\label{eq_NP}
\end{equation}
where $H(\cdot)$ is the inverse process of bilinear interpolation. The hit number at all these discrete cell vertices consists of discrete hit map $\mathbb{N}=\{N(\mathbf{m}_{wh})\}_{0 \leq w \leq c_w, 0 \leq h \leq c_h}$.



After the discrete hit map $\mathbb{N}$ is obtained, the equivalent hit multiplier $N(\mathbf{p}_{ij})$ (representing the number of times $\mathbf{p}_{ij}$ is ``observed") for an arbitrary continuous point $\mathbf{p}_{ij}$ can be easily obtained using bilinear interpolation, similar to (\ref{eq_interp}). 

\subsection{The NLLS Formulation} % --- Observations Only Case
% With the map parameterization, observations generation, and projection from local to global coordinates, observations can be linked to the occupancy map through robot poses. 
We now formulate the NLLS problem to jointly optimize the robot poses and the occupancy map. The objective function of the NLLS problem is defined as
\begin{equation}
f(\mathbf{x})=w_Z f^Z(\mathbf{x})+w_O f^O(\mathbf{x})+w_S f^S(\mathbf{x}). 	\label{eq_objective_func}
\end{equation}
The objective function consists of the observation term $f^Z(\mathbf{x})$, the smoothing term $f^S(\mathbf{x})$, and the odometry term $f^O(\mathbf{x})$. $w_Z$, $w_S$ and $w_O$ are their corresponding weights, and we set $w_O = 0$ if there is no odometry information. We now explain the three terms one by one.

\subsubsection{Observation Term $f^Z(\mathbf{x})$}
Based on the relationship between the observations and the occupancy map w.r.t. occupancy values described in \ref{sec_relationship}, we can formulate the observation term as follows. 

Given the observation information $\mathbb{S}$ in (\ref{S_i}), the observation term in the objective function (\ref{eq_objective_func}) is formulated as
\begin{equation}
	f^Z(\mathbf{x}) =
	\sum_{i=0}^n \sum_{j=1}^{k_i}  \left\|z_{ij} - F_{ij}^Z(\mathbf{x})\right\|^2, 
\label{obs-term}
\end{equation}
where
\begin{equation}
	F_{ij}^Z(\mathbf{x})  = \frac{M(\mathbf{p}_{ij})}{N({\mathbf{p}_{ij}})}.\\ \label{eq_MN}
\end{equation}
Here, $\mathbf{p}_{ij}$ represents a coordinate in the map where the $j$th sampling scan point at time step $i$ is projected using the robot pose $\mathbf{x}^P_i$, as calculated by (\ref{P-project}). $N(\mathbf{p}_{ij})$ denotes the equivalent hit multiplier at $\mathbf{p}_{ij}$, as detailed in Section \ref{sec_hit}. 

In (\ref{obs-term}), we suppose the errors of occupancy values of different sampled points in the observations $\mathbb{S}$ are independent and have the same uncertainty. Therefore, the weights on all terms are the same, which is equivalent to setting all the weights as $1$. Thus, we use norms instead of weighted norms in equation (\ref{obs-term}).

\subsubsection{Odometry Term $f^O(\mathbf{x})$}\label{sec_odometry}

% Information\label{sec_odom_inputs}}
The odometry information $\mathbb{O} = \{\mathbf{o}_i\}_{1 \leq i \leq n}$ might be available. We assume the odometry input is the relative pose between two consecutive steps. 
The odometry from robot pose $\mathbf{x}^P_{i-1}$ to pose $\mathbf{x}^P_{i}$ is expressed as
\begin{equation} 
\mathbf{o}_i=\left[ (\mathbf{o}_i^t)^\top,o_i^\theta \right]^\top~~(1 \leq i \leq n)
\label{O_i}
\end{equation}
where $\mathbf{o}_i^t$ is the translation part and $o_i^\theta$ is the rotation angle part of the odometry. The odometry term can be formulated as
\begin{equation}
\begin{aligned}
f^O(\mathbf{x})&=\sum_{i=1}^n \left\|\mathbf{o}_i -
F_i^O(\mathbf{x})
\right\|^2_{\mathbf{\Sigma}^{-1}_{O_i}}
\\&=\sum_{i=1}^n\left\|
\begin{bmatrix}
\mathbf{o}_i^t-\mathbf{R}_{i-1}\left(\mathbf{t}_i - \mathbf{t}_{i-1} \right)\\
\wrap\left({o}_i^\theta- \theta_i + \theta_{i-1}\right)
\end{bmatrix}
\right\|^2_{\mathbf{\Sigma}^{-1}_{O_i}}  \label{eq_odometry_term}
\end{aligned}
\end{equation}
in which $\mathbf{\Sigma}_{O_i}$ is the covariance matrix representing the uncertainty of $\mathbf{o}_i$, and $\wrap(\cdot)$ wraparounds the rotation angle to $(-\pi,\pi]$.
\subsubsection{Smoothing Term $f^S(\mathbf{x})$}

It can be easily found out that minimizing the objective function with only the observation term (and the odometry term) is not easy since there are a large number of local minima. Especially when the initial robot poses are far away from the global minimum, it is very difficult for an optimizer to converge to the correct solution. 

In order to enlarge the region of attraction and develop an algorithm that is robust to initial values, we introduce a smoothing term. The smoothing term requires the occupancy values of nearby cell vertices to be close to each other thus resulting in the occupancy map being smoother for derivative calculation. In our case, based on the derivative calculation method we use (see Appendix \ref{Sec_J_P}), we penalize the difference between the occupancy value of each cell vertex and the occupancy values of the two neighboring cell vertices to its right and below, i.e.,
\begin{equation}
\begin{aligned}
f^S(\mathbf{x})
& =\left\|F^S(\mathbf{x}) \right\|^2\\
& = \sum_{w=0}^{c_w-1} \sum_{h=0}^{c_h-1}  \left\|\begin{bmatrix} M(\mathbf{m}_{wh})-M(\mathbf{m}_{{(w+1)}h})\\
M(\mathbf{m}_{wh})-M(\mathbf{m}_{{w}{(h+1)}})
\end{bmatrix} \right\|^2 \\
& + \sum_{h=0}^{c_h-1}  \left\| M(\mathbf{m}_{c_wh})-M(\mathbf{m}_{{c_w}{(h+1)}})\right\|^2 \\
& + \sum_{w=0}^{c_w-1}  \left\| M(\mathbf{m}_{wc_h})-M(\mathbf{m}_{{(w+1)}{c_h}})\right\|^2,
\end{aligned} \label{eq_smoothing_term}
\end{equation} where the second and third terms are used to handle cell vertices located in the bottom row and the rightmost column. It should be noted that $F^S(\mathbf{x})$ is a linear function of $\mathbf{x}^M$ in the state. The coefficient matrix is constant and can be calculated prior to the optimization. For more details, please refer to Appendix \ref{Sec_J_S}.


\section{Iterative Solution to the NLLS Formulation}\label{Sec_Algorithm_1}
In Section \ref{sec_formulation}, we introduced our NLLS formulation for the joint poses and occupancy map optimization problem. In this section, we provide the details of a Gauss-Newton based algorithm for solving the NLLS problem. 



\begin{algorithm}[t]
\small
\caption{Our Joint Poses and Occupancy Map Optimization Algorithm}\label{alg_1}
\SetKwInput{KwInput}{Input}                % Set the Input
\SetKwInput{KwOutput}{Output}              % set the Output
\SetKwInput{KwParam}{Params}
\SetAlgoLined
\DontPrintSemicolon
\SetKw{Return}{End Function}
  \KwParam{Threshold $\tau_k$, $\tau_{\Delta}$, weight matrix $\mathbf{W}$, resolution $s$}
  \KwInput{Observations $\mathbb{S}$, odometry $\mathbb{O}$, and initial poses $\mathbf{x}^P(0)$}
  \KwOutput{Optimized poses $\hat{\mathbf{x}}^P$ and optimized map $\hat{\mathbf{x}}^M$}
\SetKwFunction{FuncFirstStage}{FirstStage}

\SetKwProg{Fn}{Function}{:}{}
\Fn{\FuncFirstStage{$\mathbf{x}^P(0)$, $\mathbb{S}$, $\mathbb{O}$, $\tau_k$, $\tau_{\Delta}$, $s$, $\mathbf{W}$}}
{
Initialize $\mathbf{x}^M(0)$ and $\mathbb{N}(0)$ using $\mathbf{x}^P(0)$ and $\mathbb{S}$ \;

Pre-calculate smoothing term coefficient $\mathbf{A}$ using (\ref{eq_A})\;

\SetKwFunction{FuncGN}{OccupancyGN}
\SetKwFunction{FuncReturn}{return}

\SetKwProg{Fn}{Function}{:}{}
\For {$k=0$; $k <= \tau_k \; \& \; \| \mathbf{\Delta}(k) \|^2 >= \tau_{{\Delta}}$; $k++$}{
\Fn{\FuncGN{$\mathbf{x}^M(k)$, $\mathbb{N}(k)$, $\mathbf{x}^P(k)$, $\mathbb{S}$, $\mathbb{O}$, $\mathbf{A}$, $\mathbf{W}$}}{
Calculate gradient $\mathbf{\nabla} \mathbf{x}^M(k)$ of $\mathbf{x}^M(k)$

Calculate $\mathbf{J}$, as described in appendices

Evaluate $F(\mathbf{x})$ at $\mathbf{x}^P(k)$ and $\mathbf{x}^M(k)$

Solve $\mathbf{J}^\top \mathbf{W} \mathbf{J} \mathbf{\Delta}(k) =-\mathbf{J}^\top \mathbf{W} F(\mathbf{x})$, where $\mathbf{\Delta}(k) = {[{\mathbf{\Delta}^P(k)}^\top,{\mathbf{\Delta}^M(k)}^\top]}^\top$

Update $\mathbf{x}^P(k+1)=\mathbf{x}^P(k) + \mathbf{\Delta}^P(k)$ and $\mathbf{x}^M(k+1)=\mathbf{x}^M(k)+\mathbf{\Delta}^M(k)$

Recalculate $\mathbb{N}(k+1)$ using $\mathbf{x}^P(k+1)$ and $\mathbb{S}$

\FuncReturn{$\mathbf{x}^P(k+1)$, $\mathbf{x}^M(k+1)$}
}
\Return
}
$\hat{\mathbf{x}}^P \Leftarrow \mathbf{x}^P(k)$, $\hat{\mathbf{x}}^M \Leftarrow \mathbf{x}^M(k)$

\FuncReturn{$\hat{\mathbf{x}}^P$, $\hat{\mathbf{x}}^M$}
}

\Return
\end{algorithm}


In the equation below, we assume the odometry inputs are available. Let
\begin{equation}
\begin{aligned}
F(\mathbf{x}) = [&\cdots,z_{ij}-F_{ij}^Z(\mathbf{x}),\cdots,{(\mathbf{o}_i-F_i^O(\mathbf{x}))}^\top,\\
&\cdots,{F^S(\mathbf{x})}^\top]^\top\\
\mathbf{W} = \;\; &\diag(\cdots,w_Z,\cdots,w_O \mathbf{\Sigma}^{-1}_{O_i}, \cdots,w_S, \cdots)\\
\end{aligned}
\end{equation}
combine all the error functions and the weights of the three terms in (\ref{eq_objective_func}). Then, the NLLS problem in (\ref{eq_objective_func}) seeks $\mathbf{x}$ such that
\begin{equation}\label{Least Squares}
f(\mathbf{x})=\|F(\mathbf{x})\|^2_{\mathbf{W}} =
{F(\mathbf{x})}^\top \mathbf{W}
F(\mathbf{x})
\end{equation}
is minimized.

A solution to (\ref{Least Squares}) can be obtained iteratively by starting with an initial guess $\mathbf{x}(0)$ and updating with $\mathbf{x}(k+1) = \mathbf{x}(k) + \mathbf{\Delta}(k)$. \rule{0pt}{1em}The update vector $\mathbf{\Delta} (k) = [{\mathbf{\Delta}^P(k)}^\top,{\mathbf{\Delta}^M(k)}^\top]^\top$ is the solution to
\begin{equation}\label{Gauss-Newton}
\mathbf{J}^\top \mathbf{W} \mathbf{J} \mathbf{\Delta} (k) = -\mathbf{J}^\top \mathbf{W} F(\mathbf{x}(k))
\end{equation}
where $\mathbf{J}$ is the linear mapping represented by the Jacobian matrix
$\partial F / \partial \mathbf{x}$ evaluated at $\mathbf{x}(k)$.

The iterative method for solving the proposed NLLS problem is shown in Algorithm \ref{alg_1}, in which $\tau_k$ and $\tau_{\Delta}$ represent the thresholds of iteration number $k$ and the incremental vector $\mathbf{\Delta}$. Unlike the standard Gauss-Newton iterative method, the hit map needs to be additionally recalculated after updating the poses in each iteration. With this approach, the implicit data association is established at each iteration and updated during the optimization.

Since the robot poses and the occupancy map are optimized simultaneously, the Jacobian $\mathbf{J}$ in (\ref{Gauss-Newton}) is very important and quite different from those used in the traditional SLAM algorithms. More details of the Jacobians are described in appendices.




\section{Multi-resolution Joint Optimization Strategy} \label{Sec_multi}
Algorithm \ref{alg_1} provides a solution to our NLLS problem (\ref{eq_objective_func}) to jointly optimize the poses and the occupancy map. However, directly using Algorithm \ref{alg_1} with the high-resolution map is time-consuming and requires an accurate initial value of robot poses \cite{Zhao-RSS-22}, which is challenging to obtain. To overcome these limitations, we propose a multi-resolution joint optimization strategy in this section.

\subsection{Discussion on Map Resolution in Optimization }

The resolution of the occupancy map has a significant impact on the optimization results since $\mathbf{x}^M$ is part of the state vector in our NLLS formulation (\ref{eq_objective_func}). 

% Assuming the robot poses are accurate, a high-resolution map representation can establish accurate relationships between observations and occupancy values of projected points on the map, but it leads to a dramatic increase in the size of the optimization problem, which in turn leads to a significant increase in computational cost. In addition, in a high-resolution map, the occupancy values in nearby \textcolor{red}{cell vertices} can vary sharply, leading to noise-filled gradients in the map when poor initial robot poses are used. Even with the introduction of the smoothing term, the use of a high-resolution map may cause poor convergence of Algorithm \ref{alg_1}.

A high-resolution map enables precise relationships between observations and occupancy values of projected points. However, it results in a dramatic increase in the optimization problem's size, raising computational costs. Additionally, in a high-resolution map, occupancy values in adjacent cell vertices may exhibit sharper variations compared to those in a low-resolution map, leading to noisy gradients when poor initial robot poses are used. Even with the introduction of the smoothing term, the use of a high-resolution map may cause poor convergence of Algorithm \ref{alg_1}.



A low-resolution map provides advantages in faster computation and reduced memory usage. Moreover, gradients are less sensitive to pose accuracy. With our occupancy map representation and smoothing term, these advantages enable the algorithm to quickly converge to a reasonable solution, even with poor initial robot poses. However, low resolution may cause inaccurate links between observations and occupancy values near boundaries, preventing the optimization from achieving greater accuracy.

To combine the advantages of different resolution map representations, we propose a multi-resolution strategy to optimize the occupancy values of different resolution cell vertices together with robot poses at various stages. Unlike the conventional coarse-to-fine scheme, in the second stage of our strategy, we use the selected high-resolution map that only includes high-resolution cell vertices possibly in need of further optimization instead of the full high-resolution map. Optimizing only those selected high-resolution cell vertices further improves the efficiency of our algorithm. 
\subsection{Our Multi-resolution Joint Optimization Strategy}

Firstly, we obtain low-resolution observations $\mathbb{S}^{l} = \{\mathbb{S}_i^{l}\}_{0\leq i \leq n}$ by down-sampling from the high-resolution observations $\mathbb{S}^{h} = \{\mathbb{S}_i^{h}\}_{0\leq i \leq n}$, which are obtained by the equal sampling strategy described in Section \ref{Sec_Info_1} with a sampling distance $s^{h}$. Here, we set the map resolution and sampling resolution to be the same. Therefore, the low resolution $s^{l}=r \times s^{h}$, where $r$ is the resolution ratio between the low-resolution map and the high-resolution map. The size of the low-resolution map $\mathbb{M}^{l}$ is $(c_w+1) \times (c_h+1)$.

Initialized by the odometry inputs or scan matching, we perform Algorithm \ref{alg_1} to quickly obtain relatively accurate poses. The state vector in the first stage is ${\mathbf{x}}^{l} = {[{\mathbf{x}^P}^\top,{{\mathbf{x}^{lM}}}^\top]}^\top $, where $\mathbf{x}^{lM}$ includes all occupancy values at the cell vertices of the low-resolution map $\mathbb{M}^{l}$. In this stage, the hit map, observation information, coefficient matrix, weight matrix, and resolution are represented as $\mathbb{N}^{l},  \mathbb{S}^{l}, \mathbf{A}^{l}$, $\mathbf{W}^{l}$, and $s^{l}$, respectively. 

In the first stage of optimization, the low-resolution occupancy map reduces both the dimension of $\mathbf{x}^{lM}$ and the number of observations in $\mathbb{S}^{l}$. Since the occupancy values at cell vertices change relatively gradually in the low-resolution map, the directions of the map's gradients are closer to the correct ones when the poses are initialized by odometry inputs or scan matching, making it easier for Algorithm \ref{alg_1} to converge to a relatively good result quickly.

After the first stage, we use Algorithm \ref{alg_2} to select the cell vertices that need to be further optimized to compose the selected high-resolution map $\mathbb{M}^{s}$ and find their corresponding observations $\mathbb{S}^{s}$. Details are described in Section \ref{select_index_set}.


\begin{algorithm}[tp]
\small
\caption{Finding the Selected High-resolution Map and Corresponding Observations}\label{alg_2}

\SetKwInput{KwInput}{Input}                % Set the Input
\SetKwInput{KwOutput}{Output}              % set the Output
\SetKwInput{KwParam}{Params}
\SetKw{Return}{End Function}
\SetAlgoLined
\DontPrintSemicolon
\KwParam{Resolution $s^{h}$, selection distance $d$, convolution kernel size $q$}
  \KwInput{Observations $\mathbb{S}^{h}$, and poses ${{\hat{\mathbf{x}}}}^{\tilde{P}}$ from the first stage using Algorithm \ref{alg_1}}
  \KwOutput{Observations $\mathbb{S}^{s}$ and map part of the state vector in the second stage $\mathbf{x}^{sM}$}
  \SetKwFunction{FuncSel}{Selection}
\SetKwProg{Fn}{Function}{:}{}
\Fn{\FuncSel{${{\hat{\mathbf{x}}}}^{\tilde{P}}$, $\mathbb{S}^{h}$, $s^{h}$, $d$, $q$}}{

Build a full high-resolution map $\mathbb{M}^{h}$ using $\hat{\mathbf{x}}^{\tilde{P}}$ and $\mathbb{S}^{h}$ 

Calculate the binary map $\mathbb{B}$ using $\mathbb{M}^{h}$

Calculate the convoluted map $\mathbb{C}$ with kernel size $q$

Calculate the set $\mathbb{I}^{h}$, which includes the indices of all boundary vertices in $\mathbb{M}^{h}$, using $\mathbb{C}$

Calculate the set $\mathbb{I}^{s}$, which includes the indices of all selected vertices, using $\mathbb{I}^{h}$ and $d$

Define the map part of the state vector in the second stage $\mathbf{x}^{sM}$ and the selected high-resolution map $\mathbb{M}^{s}$ by $\mathbb{I}^{s}$

Find observations $\mathbb{S}^{s}$ for $\mathbb{M}^{s}$ using the set $\mathbb{I}^{s}$, $\mathbb{S}^{h}$ and ${{\hat{\mathbf{x}}}}^{\tilde{P}}$

\FuncReturn{$\mathbb{S}^{s}$, $\mathbf{x}^{sM}$}
}
\Return
\end{algorithm}


In the second stage, the state vector is represented as $\mathbf{x}^{s} = {[{{\mathbf{x}}^P}^\top,{\mathbf{x}^{sM})}^\top]}^\top $ where $\mathbf{x}^{sM}$ includes all occupancy values at cell vertices of the selected high-resolution map $\mathbb{M}^{s}$. We perform Algorithm \ref{alg_3} using poses obtained from the first stage as initial guesses and observations $\mathbb{S}^{s}$ to refine poses. The NLLS optimization problem in the second stage can be formulated similarly as (\ref{Least Squares}). Additionally, the differences in the Jacobian calculation between Algorithm \ref{alg_1} and Algorithm \ref{alg_3} are described in Appendix \ref{Sec_J_Select}. 

The full multi-resolution joint optimization strategy is outlined in Algorithm \ref{alg_flowchart}.

\begin{algorithm}[t]
\small
\caption{The Algorithm for the Second Stage of the Multi-resolution Joint Optimization Strategy}\label{alg_3}
\SetKwInput{KwParam}{Params}
\SetKwInput{KwInput}{Input}                % Set the Input
\SetKwInput{KwOutput}{Output}              % set the Output
\SetAlgoLined
\DontPrintSemicolon
\SetKw{Return}{End Function}

  \KwParam{Threshold $\tau_k^{s}$, $\tau_{\Delta}^{s}$, weight matrix $\mathbf{W}^{s}$, resolution $s^{h}$}
  \KwInput{Observations $\mathbb{S}^{s}$, odometry $\mathbb{O}$, and poses ${{\hat{\mathbf{x}}}}^{\tilde{P}}$ from the first stage using Algorithm \ref{alg_1}}
  \KwOutput{Optimal poses $\hat{\mathbf{x}}^P$ and map $\hat{\mathbf{x}}^{sM}$}
\SetKwFunction{FuncSecondStage}{SecondStage}

$\mathbf{x}^P(0) \Leftarrow {{\hat{\mathbf{x}}}}^{\tilde{P}}$

\SetKwProg{Fn}{Function}{:}{}
\Fn{\FuncSecondStage{$\mathbf{x}^P(0)$, $\mathbb{S}^{s}$, $\mathbb{O}$, $\tau_k^{s}$, $\tau_{\Delta}^{s}$, $s^{h}$, $\mathbf{W}^{s}$}}
{

Initialize $\mathbf{x}^{sM}(0)$ and $\mathbb{N}^{s}(0)$ using $\mathbb{S}^{s}$ and $\mathbf{x}^P(0)$

Pre-calculate smoothing term coefficient matrix $\mathbf{A}^{s}$

\For {$k=0$; $k <= \tau_k^{s} \; \& \; \| \mathbf{\Delta}(k) \|^2 >= \tau_{\Delta}^{s}$; $k++$}{

$\mathbf{x}^P(k+1)$, $\mathbf{x}^{sM}(k+1)$
$\leftarrow$  \FuncGN{$\mathbf{x}^{sM}(k)$, {$\mathbb{N}^{s}(k)$, $\mathbf{x}^P(k)$, $\mathbb{S}^{s}$, $\mathbb{O}$, $\mathbf{A}^{s}$, $\mathbf{W}^{s}$}}
}

$\hat{\mathbf{x}}^P \Leftarrow \mathbf{x}^P(k)$, $\hat{\mathbf{x}}^{sM} \Leftarrow \mathbf{x}^{sM}(k)$

\FuncReturn{$\hat{\mathbf{x}}^P$, $\hat{\mathbf{x}}^{sM}$}

}
\Return
\end{algorithm}




\begin{algorithm}[t]
\small
\caption{Our Multi-resolution Joint Optimization Strategy}\label{alg_flowchart}
\SetKwInput{KwParam}{Params}
\SetKwInput{KwInput}{Input}                % Set the Input
\SetKwInput{KwOutput}{Output}              % set the Output

\SetAlgoLined
\DontPrintSemicolon

\SetKwFunction{FuncDown}{DownSampling}
\SetKwFunction{FuncInitPose}{InitializePose}
\SetKwFunction{FuncSM}{ScanMatching}
  \KwParam{Threshold $\tau_k^{l}$, $\tau_{\Delta}^{l}$, $\tau_k^{s}$, $\tau_{\Delta}^{s}$, weight matrix $\mathbf{W}^{l}$, $\mathbf{W}^{s}$, ratio of resolutions $r$, resolution $s$, selection distance $d$, convolution kernel size $q$}
  \KwInput{Observations $\mathbb{S}^{h}$, odometry $\mathbb{O}$}
  \KwOutput{Optimal poses $\hat{\mathbf{x}}^P$ and map $\hat{\mathbf{x}}^{sM}$}

$\mathbb{S}^{l}$ $\leftarrow$ \FuncDown{$\mathbb{S}^{h}$, $r$}

\uIf {$w_O \neq  0$}
{
$\mathbf{x}^P(0)$ $\leftarrow$ \FuncInitPose{$\mathbb{O}$}
}
\Else
{
$\mathbf{x}^P(0)$ $\leftarrow$ \FuncSM{$\mathbb{S}^{l}$}
}


${{\hat{\mathbf{x}}}}^{\tilde{P}}$, $\hat{\mathbf{x}}^{lM}$ $\leftarrow$ \FuncFirstStage{$\mathbf{x}^P(0)$, $\mathbb{S}^{l}$, $\mathbb{O}$, $\tau_k^{l}$, $\tau_{\Delta}^{l}$, $s^{l}$, $\mathbf{W}^{l}$}

$\mathbb{S}^{s}$, $\mathbf{x}^{sM}$ $\leftarrow$ \FuncSel{${{\hat{\mathbf{x}}}}^{\tilde{P}}$, $\mathbb{S}^{h}$, $s^{h}$, $d$, $q$}

${\hat{\mathbf{x}}^P}$, $\hat{\mathbf{x}}^{sM}$ $\leftarrow$
\FuncSecondStage{${{\hat{\mathbf{x}}}}^{\tilde{P}}$, $\mathbb{S}^{s}$,$\mathbb{O}$, $\tau_k^{s}$, $\tau_{\Delta}^{s}$, $s^{h}$, $\mathbf{W}^{s}$}

\end{algorithm}

\begin{figure}[t]
\centering 
\subfigure[Full High-resolution Map]{ 
\includegraphics[width=0.23\textwidth]{./high_resolution_select.pdf}}
\subfigure[Selected High-resolution Map (In White and Black)]{\label{fig_select_example_b}
\includegraphics[width=0.23\textwidth]{./recolor_select.pdf}}
\caption{An example of the selected high-resolution map from a full high-resolution map in a simulation dataset. (a) The full high-resolution map generated using poses from the first-stage optimization and scans, forming the basis for selection. (b) The recolored selected high-resolution map: gray marks dropped (stable) areas, white and black denote selected areas, with black highlighting obstacle boundaries.}
\label{fig_select_example}
% \vspace{-1em}
\end{figure}

\subsection{Selected High-resolution Map and Observations}\label{select_index_set}

After the first stage optimization using the low-resolution map $\mathbb{M}^{l}$, the robot poses $\hat{\mathbf{x}}^{\tilde{P}}$ become relatively accurate. Subsequently, the full high-resolution map $\mathbb{M}^{h}$, with dimensions $(r*c_w+1) \times (r*c_h+1)$, is built using the Bayesian occupancy mapping method \cite{ProbabilisticRobotics}, based on observations $\mathbb{S}^{h}$ and poses ${{\hat{\mathbf{x}}}}^{\tilde{P}}$. In this case, most cell vertices of $\mathbb{M}^{h}$ are considered stable in terms of occupancy state. Semantically, these stable cell vertices have the same occupancy state as the surrounding cell vertices (typically free or unknown cells). This characteristic leads to map gradients near zero at these stable cell vertices. In contrast, the cell vertices that require further updates are typically located at the edges of objects, where the occupancy values significantly differ from those of surrounding cell vertices. Therefore, the gradient at these cell vertices is larger. An example illustrating this is shown in Fig. \ref{fig_select_example_b}, where the selected area (in white and black) is clearly distinct from the stable area (in gray). Based on this idea, we propose a strategy to select the cell vertices located around the boundaries to compose the selected high-resolution map $\mathbb{M}^{s}$, which is used in the second stage of optimization.


\begin{figure}[t]
\centering 
\includegraphics[width=0.48\textwidth]{./Select_Set.pdf}
\caption{\label{fig_low_high_select} An illustration of the cell vertices selection strategy and a selected high-resolution map from a simulation dataset. In (a), selected cell vertices are marked in red and yellow, with their indices forming the index set $\mathbb{I}^{s}$.}
% \vspace{-0.5em}
\end{figure}

% In both figures, the boundary cells are indicated as black color, the selected cells are shown as black and white, and the dropped cells are colored in gray.

Firstly, we identify cell vertices located at the edges of objects by performing mean-value convolution of the full high-resolution map $\mathbb{M}^{h}$. Specifically, we calculate a binary map $\mathbb{B}=\{B(\mathbf{m}_{id})\}$ by binarizing $\mathbb{M}^{h}$ as
\begin{equation}
B(\mathbf{m}_{id}) =	\begin{cases}
	1, & {M}^{h}(\mathbf{m}_{id}) \geq \tau_{occupied} \\
	0, & {M}^{h}(\mathbf{m}_{id}) < \tau_{occupied} \\
\end{cases},
\end{equation}
where $\mathbf{m}_{id}$ represents a cell vertex, and $\tau_{occupied}$ is the threshold used to classify a cell vertex as occupied or free. A mean-value convolution kernel $\mathbf{K}$ is defined as
\begin{equation}
	\mathbf{K} = \dfrac{1}{q^2} \cdot \bold{1}_{q\times q}
\end{equation}
where $\bold{1}_{q\times q}$ represents a $q \times q$ matrix of ones. The convoluted map $\mathbb{C}=\{C(\mathbf{m}_{id})\}$ is then derived by convolving $\mathbb{B}$ with $\mathbf{K}$, where $C(\mathbf{m}_{id})$ indicates whether the $q \times q$ cell vertices around $\mathbf{m}_{id}$ are all in the same occupancy state. Compared to other edge detection methods like Sobel \cite{duda1973pattern} and Canny \cite{canny1986computational}, this conservative method more reliably selects cell vertices that may require further optimization. 

Using this method, the set of indices for all boundary cell vertices in the high-resolution map is defined as 

\begin{equation}
\begin{aligned}
	\mathbb{I}^{h} = \{id | 0<C(\mathbf{m}_{id})<1 \}.
\end{aligned}
\end{equation}
The cell vertices indexed in $\mathbb{I}^{h}$ are marked in red in Fig. \ref{fig_low_high_select}(a). 


To account for pose uncertainties from the first stage, the selection is expanded to include cell vertices within a distance $d$ from all boundary cell vertices. The indices of the selected cell vertices in the high-resolution map form the set $\mathbb{I}^{s}$, illustrated in Fig. \ref{fig_low_high_select}(a), where the selected cell vertices are highlighted in red and yellow with $d=1$. An example of a selected high-resolution map from a simulation dataset is shown in Fig. \ref{fig_low_high_select}(b).

Consequently, the map component of the state vector in the second stage is expressed as
\begin{equation}
	\mathbf{x}^{sM} = {[\cdots, M^{h}(\mathbf{m}_{wh}), \cdots]}^\top, ~~ wh\in \mathbb{I}^{s}.
\end{equation}

% The cell selection strategy with a selection distance of 1 cell is illustrated in Fig. \ref{fig_low_high_select}(a), where selected \textcolor{red}{cell vertices} are marked with red dots and their corresponding selected cells are marked with black and white. 


Next, we select observations to optimize $\mathbf{x}^{sM}$. Cells surrounded by vertices with indices in $\mathbb{I}^s$ are designated as selected cells, shown in white in Fig. \ref{fig_low_high_select}(a) and Fig. \ref{fig_low_high_select}(b). Subsequently, sampling point selection is carried out, as illustrated in Fig. \ref{fig_select_sampling_point}. Specifically, sampling points in $\mathbb{S}^{h}$ are first projected onto the global coordinate system using the poses optimized in the first stage. All sampling points located on the selected cells are then included to form the set $\mathbb{S}^{s}$. 

\begin{figure}[t]
\centering 
\includegraphics[width=0.48\textwidth]{./Select_Sampling_Point.pdf}
\caption{\label{fig_select_sampling_point} An example of the selected sampling points of a beam at time step $i$, where points projected onto the selected cells are chosen.}
% \vspace{-0.5em}
\end{figure}


\section{Submap Joining} \label{Sec_submap}
In Section \ref{Sec_multi}, we introduced a multi-resolution joint optimization strategy to efficiently solve our NLLS problem. For large-scale occupancy SLAM with long robot trajectories, the number of poses to optimize can be very large. To make the computational complexity dependent only on the environment size rather than the trajectory length, in this section we propose an occupancy submap joining method. The key idea is to reduce the number of poses that need to be optimized to the number of local submaps. 


\subsection{Inputs and Outputs of Submap Joining Problem} 
We first separate the observation information into multiple parts and perform Algorithm \ref{alg_flowchart} to build several submaps. The inputs of submap joining problem are a sequence of local occupancy submaps. 
Let us denote the $n_L+1$ submaps as $\mathbb{M}_L = \{\mathbb{M}_{L_0}, \cdots, \mathbb{M}_{L_{n_L}}\}$ and the associated coordinate frames of these local occupancy maps are denoted as $ \{\mathbf{x}^P_{0}, \cdots, \mathbf{x}^P_{n_L}\}$, where $\mathbb{M}_{L_{i_L}}$ and $\mathbf{x}^P_{i_L}$represents the $i_L$th local occupancy map and its associated coordinate frame. In addition, the global occupancy map is represented as $\mathbb{M}_G=\{M_G(\mathbf{m}^G_{00}), \cdots,{M}_G\left(\mathbf{m}^G_{c_w^Gc_h^G}\right)\}$. Both the global map and local maps follow the same definition as described in Section \ref{sec_discrete_occupancy}. The outputs of submap joining problem are the optimal solution of the local submap coordinate frames and the optimal global occupancy map.

\subsection{NLLS Formulation of Submap Joining Problem} 
First, the cell vertex $\mathbf{m}^G_{wh}$ in the global occupancy map $\mathbb{M}_G$ can be projected to local submap coordinate by pose $\mathbf{x}^P_{i_L}$, i.e., 
\begin{equation}
	\mathbf{p}_{i_L}^{wh} = \frac{ \mathbf{R}_{i_L} (\mathbf{m}^G_{wh} \cdot s_G  - \mathbf{t}_{i_L})}{s_L}.
\end{equation}
Here, $\mathbf{p}_{i_L}^{wh}$ represents the position in the local submap's coordinate where the cell vertex $\mathbf{m}_{wh}^G$ from the global map is projected using the pose $\mathbf{x}^P_{i_L}$. The resolutions of the global occupancy map and local submaps are denoted by $s_G$ and $s_L$, respectively.

The submap joining problem aims to find the optimal global occupancy map and the poses of submap coordinate frames. Thus, the state vector for this problem is defined as 
\begin{equation}
	\mathbf{x}_G = [{\mathbf{x}^P_L}^\top, {{\mathbf{x}^M_G}}^\top]^\top,
\end{equation}
where 
\begin{equation}
\begin{aligned}
\mathbf{x}^P_L & =\left[\left(\mathbf{x}^P_1\right)^\top, \cdots,\left(\mathbf{x}^P_{n_L}\right)^\top\right]^\top \\
\mathbf{x}^M_G & =\left[{M}_G\left(\mathbf{m}^G_{00}\right), \cdots, {M}_G\left(\mathbf{m}^G_{c_w^Gc_h^G}\right)\right]^\top.
\end{aligned}
\end{equation}
As with most submap joining problem formulations, we fix the first local map coordinate frame as the global coordinate frame. Therefore, $\mathbf{x}^P_L$ consists of $n_L$ local map coordinate frames and $\mathbf{x}^M_G$ includes $(c_w^G+1) \times (c_h^G+1)$ discrete cell vertices of global occupancy map. 

By the global-to-local projection relationship, all cell vertices of global occupancy map $\mathbb{M}_G$ can be projected to corresponding submaps to compute the difference in occupancy values. Thus, the NLLS problem of occupancy submap joining can be formulated to minimize 
\begin{equation}
\begin{adjustbox}{max width=\linewidth}
$
g(\mathbf{x}_G) =  \sum\limits_{i_L=0}^n\sum\limits_{ wh \in \mathbb{S}^L_{i_L}} \left\| \omega(i_L,\mathbf{m}^G_{wh}) {M}_G(\mathbf{m}^G_{wh}) - {M}_{L_{i_L}}(\mathbf{p}_{i_L}^{wh}) \right\|^2,
$
\end{adjustbox}
\label{eq_NLLS_joining}
\end{equation}
where $\mathbb{S}^L_{i_L}$ represents the set of indices of cell vertices in the global occupancy map $\mathbb{M}_G$ that are projected onto the local submap $\mathbb{M}_{L_{i_L}}$.

In (\ref{eq_NLLS_joining}), $\omega(i_L,\mathbf{m}^G_{wh})$ is the weight to establish an accurate relationship between the global occupancy map and local submaps w.r.t. occupancy values, which can be calculated by
\begin{equation}
    \omega(i_L,\mathbf{m}^G_{wh}) = \frac{{N}_{{L_{i_L}}}(\mathbf{p}_{i_L}^{wh})}{{N}_{G}(\mathbf{m}^G_{wh})}.
\end{equation}
Here, ${N}_{{L_{i_L}}}(\cdot)$ is the local hit number lookup function for submap $\mathbb{M}_{L_{i_L}}$, derived as described in Section \ref{sec_hit}. It approximates the hit number at coordinate $\mathbf{p}_{i_L}^{wh}$ using bilinear interpolation. Similarly, ${N}_{G}(\cdot)$ represents the global hit number lookup function associated with $\mathbb{M}_G$.

In (\ref{eq_NLLS_joining}), the submap joining problem is formulated as a NLLS problem, which can be solved iteratively by Gauss-Newton based method similar to Algorithm \ref{alg_1}.


 \begin{table}[htp]
		\centering
		\caption{Parameters of Datasets. \label{tab_dataset}}
		\label{tab_comparison}
		\setlength{\tabcolsep}{0.7mm}{
		\begin{tabular}{lccccc}\toprule
		Dataset	& No. Scans & Duration  & Map Size &  Odometry & Resolution\\ \hline
		Simulation 1 & 3640  &117 s& $50$ m  $\times$ $50$ m & yes & 0.05 m\\
        Simulation 2 & 3720  &121 s& $50$ m $\times$ $50$ m & yes & 0.05 m\\
		Simulation 3  & 2680  & 83 s& $50$ m $\times$ $50$ m & yes & 0.05 m\\
		Car Park  & 1642 & 164 s& $50$ m $\times$ $40$ m & yes & 0.1 m\\
		C5  & 3870  &136 s& $50$ m $\times$ $40$ m & yes & 0.1 m\\
		Museum b0 & 5522 &152 s& $85$ m $\times$ $95$ m &no & 0.1 m \\
		Museum b2 & 51833 &1390 s &  $250$ m $\times$ $200$ m &no & 0.1 m\\
        C3 &24402 &610 s& $150$ m $\times$ $125$ m  & no & 0.1 m\\
		\hline
		\end{tabular}
		}
\end{table}


\section{Experimental Results} \label{Sec_experiment}

In this section, we evaluate our algorithm on several datasets and compare its performance with Cartographer \cite{hess2016real}, the current state-of-the-art algorithm, which significantly outperforms other methods such as Hector-SLAM \cite{kohlbrecher2011flexible} and Karto-SLAM \cite{konolige2010efficient}. To ensure fair comparisons, we adjust some parameters in Cartographer based on the sensor configurations of the respective datasets for optimal performance.


The dataset parameters are summarized in Table \ref{tab_dataset}. For practical datasets, Deutsches Museum b0 and Deutsches Museum b2 are Cartographer demo datasets collected at the Deutsches Museum. The Car Park \cite{zhao20212d} dataset is gathered in an underground car park, while C5 and C3 are collected in a factory environment using a Hokuyo UTM-30LX laser scanner. Consistent map resolutions $s$ are applied across all methods to display the map results, with ratio $r$ set to $10$ for all simulation experiments and $5$ for all practical experiments unless stated otherwise. For each dataset, 20\% of scans and corresponding poses are uniformly selected as key frames for the key frame option in our method.

To ensure fair comparisons, we use an identical number of poses (synchronizing the poses from the results with the ground truth poses using timestamps) and their corresponding observations to generate results for visualization and quantitative evaluation across all compared methods, with the exception of our method that employs keyframes. Furthermore, the same occupancy mapping algorithm is applied consistently across all approaches to produce the occupancy grid map results for comparison.

        
\subsection{Simulation Experiments}\label{simu_experiment}

\begin{figure*}[tp]
\centering \subfigure[Simulation 1] {\label{fig_trajectory_1}
\includegraphics[width=0.28\textwidth]{./trajectory_simu1_new.pdf}}
\centering \subfigure[Simulation 2] {\label{fig_trajectory_2}
\includegraphics[width=0.28\textwidth]{./trajectory_simu2_new.pdf}}
\centering \subfigure[Simulation 3] {\label{fig_trajectory_3}
\includegraphics[width=0.343\textwidth]{./trajectory_simu3_new.pdf}}
\caption{\label{fig_trajectory_compare}Simulation environments and robot trajectory results. (a), (b) and (c) show the simulation environments (the black lines indicate the obstacles in the scene) and the trajectories of ground truth, odometry inputs, Cartographer \cite{hess2016real}, and our approach for one dataset in each of the three simulation environments.}
% \vspace{-0.5em}
\end{figure*}

We use three different simulation environments with varying levels of nonlinearity and nonconvex obstacles to design three different simulation experiments. Since Cartographer needs a high-frequency scanning rate to ensure the good performance of scan matching, while our approach performs well for scan data with low scanning frequency, only 10\% scans listed in Table \ref{tab_dataset} are used in our method. 

We utilize the open-source 2D LiDAR simulator from \cite{zhao20212d} to generate simulated datasets. Each scan includes 1081 laser beams spanning angles from -135 degrees to 135 degrees, mimicking the specifications of a Hokuyo UTM-30LX laser scanner. To emulate real-world data acquisition, random Gaussian noise with zero mean and standard deviation of $0.02$ m is added to each beam of the simulated scan data. Similarly, zero-mean Gaussian noise is introduced to the odometry inputs derived from the ground truth poses, with standard deviation of $0.04$ m for $x$-$y$ and $0.003$ rad for orientation. Five datasets with different noise realizations are generated for each simulation environment.


\begin{figure*}[t]
\centering \subfigure[Simulation 1] {\label{fig_time_error_1}
\includegraphics[width=0.32\textwidth]{./Time_with_Error_Simu1.pdf}}
\centering \subfigure[Simulation 2] {\label{fig_time_error_2}
\includegraphics[width=0.32\textwidth]{./Time_with_Error_Simu2.pdf}}
\centering \subfigure[Simulation 3] {\label{fig_time_error_3}
\includegraphics[width=0.32\textwidth]{./Time_with_Error_Simu3.pdf}}
\caption{Comparison of translation and rotation errors at different time steps using simulation datasets.}
\label{fig_error_compare_time}
\vspace{-1em}
\end{figure*}


% The robot trajectory results of our method and Cartographer using one dataset in each simulation are compared with the ground truth and odometry in Fig. \ref{fig_trajectory_compare}. It is clear that our trajectories are closer to the ground truth trajectories, especially for positions where significant rotation occurs. Fig. \ref{fig_error_compare_time} shows the translation and rotation errors of our method and Cartographer at different time steps. Obviously, the errors of our method are substantially smaller than those of Cartographer.

The trajectory results of our method and Cartographer, compared to ground truth and odometry, are shown in Fig. \ref{fig_trajectory_compare}. It is evident that our trajectories align more closely with the ground truth, particularly in areas with significant rotational movements. Fig. \ref{fig_error_compare_time} illustrates translation and rotation errors over time, demonstrating that our method consistently achieves substantially smaller errors compared to Cartographer.

% We use all the fifteen datasets from Simulation 1, Simulation 2 and Simulation 3 to perform the quantitative and qualitative comparison of errors in the pose estimates. The quantitative results of Cartographer, only the first stage in our method \textcolor{red}{(Algorithm \ref{alg_1} using low-resolution)} using all frames, our method using all frames, and our method using key frames are given in Table \ref{tab_comparison}. We use mean absolute error (MAE) and root mean squared error (RMSE) to evaluate the translation errors (in meters) and rotation errors (in radians). Our method performs the best in all four metrics for all simulations and is substantially ahead of Cartographer even when using only key frames or only the first stage. In addition, Fig. \ref{fig_simulation}(a) to Fig. \ref{fig_simulation}(e) show the occupancy grid maps and point cloud maps generated using poses from ground truth, Cartographer and the three options of our method. It is clear that the boundaries of both occupancy grid maps and point cloud maps using the three options of our method are much clearer than those from Cartographer, which indicates that our method can obtain more accurate results by optimizing the robot poses and the occupancy map together.

We performed quantitative and qualitative comparisons of pose estimation errors using all fifteen datasets from Simulations 1, 2, and 3. Table \ref{tab_comparison} presents, in order, the quantitative results for odometry inputs, Cartographer, the first stage of our method (Algorithm \ref{alg_1} with low-resolution) using all frames, our method using all frames, and our method using key frames. Metrics such as mean absolute error (MAE) and root mean squared error (RMSE) evaluate translation errors (in meters) and rotation errors (in radians). Our method consistently achieves the best performance across all metrics, significantly outperforming Cartographer even when using only key frames or the first stage. Fig. \ref{fig_simulation}(a) to Fig. \ref{fig_simulation}(e) further illustrates occupancy grid maps and point cloud maps generated using poses from the ground truth, Cartographer, and the three options of our method. The maps produced by our method exhibit noticeably clearer boundaries, demonstrating its ability to jointly optimize robot poses and occupancy maps for more accurate results.

\begin{figure}[t]
\centering
\includegraphics[width=0.48\textwidth]{./Simulation_New.pdf}
\caption{\label{fig_simulation} The occupancy grid maps and point cloud maps generated from ground truth poses and different approaches for each simulation dataset. The areas marked with red dots highlight where our method outperforms the results of the first-stage optimization alone.}
\vspace{-0.5em}
\end{figure}

\begin{table}[t]
		\centering
		\caption{Quantitative Comparison of Robot Pose Errors in Simulations.}
		\label{tab_comparison}
		\setlength{\tabcolsep}{0.7 mm}{
		\begin{tabular}{lccccc}\toprule
			& Odom & Carto & Ours (First) & Ours (All) & Ours (Key) \\ \hline
		Simulation 1& & & &\\
		\quad MAE / Trans (m) & 0.78270 & 0.25336  & 0.02206 &\textcolor{red}{\textbf{0.00640}} & \textcolor{blue}{\textbf{0.01024}}\\
		\quad MAE / Rot (rad) & 0.04912 & 0.01394  & 0.00098 &\textcolor{red}{\textbf{0.00060}} & \textcolor{blue}{\textbf{0.00084}}
\\
		\quad RMSE / Trans (m) & 0.98404 & 0.29920  & 0.02680 &\textcolor{red}{\textbf{0.00974}} & \textcolor{blue}{\textbf{0.01430}}\\
		\quad RMSE / Rot(rad) & 0.05506 & 0.01562  & 0.00162 &\textcolor{red}{\textbf{0.00102}} & \textcolor{blue}{\textbf{0.00126}}\\\hline
		
		Simulation 2& & & &\\
		\quad MAE / Trans (m) & 0.80544 & 0.11914 &0.03224 & \textcolor{red}{\textbf{0.00858}} & \textcolor{blue}{\textbf{0.01082}}
\\
		\quad MAE / Rot (rad) & 0.02538 & 0.00666  &0.00220 &\textcolor{red}{\textbf{0.00062}} & \textcolor{blue}{\textbf{0.00096}}\\
		\quad RMSE / Trans (m) & 0.97152 & 0.14810  & 0.04188 &\textcolor{red}{\textbf{0.01198}} & \textcolor{blue}{\textbf{0.01244}}\\
		\quad RMSE / Rot (rad) & 0.02936 & 0.00916 &0.00220 & \textcolor{red}{\textbf{0.00104}} & \textcolor{blue}{\textbf{0.00178}}\\\hline
		
		Simulation 3& & & &\\
		\quad MAE / Trans(m) & 0.75352 & 0.14262  &0.02624 &\textcolor{red}{\textbf{0.00726}} &  \textcolor{blue}{\textbf{0.00998}}\\
		\quad MAE / Rot (rad) & 0.05180 & 0.00682 &0.00164  & \textcolor{red}{\textbf{0.00058}} &  \textcolor{blue}{\textbf{0.00090}}\\
		\quad RMSE / Trans (m) & 0.96866 & 0.18782 & 0.03238 &\textcolor{red}{\textbf{0.00952}} &  \textcolor{blue}{\textbf{0.01338}}\\
		\quad RMSE / Rot (rad) & 0.05926 & 0.00914  & 0.00204 &\textcolor{red}{\textbf{0.00088}} &  \textcolor{blue}{\textbf{0.00134}}\\\hline
		\end{tabular}
	\begin{tablenotes}
     \item \textcolor{red}{\textbf{Red}} and  \textcolor{blue}{\textbf{blue}} indicate the best and second best results, respectively.
   \end{tablenotes}
		}
        % \vspace{-2em}
\end{table}

% \begin{table*}[htp]
% \centering
% \caption{Occupancy Grid map Precision of Our Method Using All Frames, Our Method Using Key Frames, and Cartographer.}
% \label{tab_map_accuracy}
% \setlength{\tabcolsep}{2.4mm}
% \begin{NiceTabular}{cccccccccccc}[first-row,first-col,hvlines]
% \CodeBefore
% \Body
%  & \Block{1-11}{\textbf{Predicted}} & & & & & & &  & & & \\
% \Block{11-1}{\rotate Ground Truth}  & \Block{2-1}{}  & \Block{2-1}{} & \Block{1-3}{Our Method (All Frames)} & & & \Block{1-3}{Our Method (Key Frames)} & & & \Block{1-3}{Cartographer}  \\
%  & &   & Unknown & Free & Occupied & Unknown & Free & Occupied & Unknown & Free & Occupied \\
%  & \Block{3-1}{Simulation 1} & Unknown  & \textcolor{red}{\textbf{99.798\%}}  & 0.020\% & 0.182\% & \textcolor{blue}{\textbf{99.598\%}} & 0.072\% & 0.330\% & 95.616\% & 2.822\% & 1.562\% \\
%  &   &  Free & 0.022\%  & \textcolor{red}{\textbf{99.938\%}} & 0.040\% & 0.094\% & \textcolor{blue}{\textbf{99.824\%}} & 0.082\% & 1.290\% & 97.678\% & 1.032\% \\
%  &   & Occupied  &  5.436\% & 2.334\%  & \textcolor{red}{\textbf{92.230\%}} & 13.562\% & 3.142\% &\textcolor{blue}{\textbf{83.296\%}} &30.053\% & 53.280\% & 16.667\% \\

% & \Block{3-1}{Simulation 2} & Unknown  & \textcolor{red}{\textbf{99.846\%}}  & 0.010\% & 0.144\% & \textcolor{blue}{\textbf{99.696\%}} & 0.062\% & 0.242\% & 96.868\% & 1.743\% & 1.389\% \\
%  &   &  Free & 0.016\%  & \textcolor{red}{\textbf{99.846\%}} & 0.138\% & 0.076\% & \textcolor{blue}{\textbf{99.848\%}} & 0.076\% & 0.593\% & 98.584\% & 0.823\% \\
%  &   & Occupied  &  6.434\% & 3.738\%  & \textcolor{red}{\textbf{89.828\%}} & 11.694\% & 2.806\% &\textcolor{blue}{\textbf{85.500\%}} &23.943\% & 50.795\% & 25.262\% \\

%  & \Block{3-1}{Simulation 3} & Unknown  & \textcolor{red}{\textbf{99.812\%}}  & 0.032\% & 0.156\% & \textcolor{blue}{\textbf{99.258\%}} & 0.430\% & 0.312\% & 96.968\% & 1.574\% & 1.458\% \\
%  &   &  Free & 0.036\%  & \textcolor{red}{\textbf{99.928\%}} & 0.036\% & 0.554\% & \textcolor{blue}{\textbf{99.352\%}} & 0.094\% & 1.018\% & 98.110\% & 0.872\% \\
%  &   & Occupied  &  4.500\% & 2.358\%  & \textcolor{red}{\textbf{93.142\%}} & 16.788\% & 3.736\% &\textcolor{blue}{\textbf{79.476\%}} &26.420\% & 44.928\% & 28.652\% \\
% \end{NiceTabular}
% % \vspace{-1em}
% \end{table*}

\begin{table*}[htp]
\centering
\caption{Occupancy Grid map Precision of Our Method Using All Frames, Our Method Using Key Frames, and Cartographer.}
\label{tab_map_accuracy}
\setlength{\tabcolsep}{2.4mm}
\renewcommand{\arraystretch}{1.2}

\begin{tabular}{c c c c c c c c c c c c}
\toprule
 \multirow{2}{*}{} & \multirow{2}{*}{\textbf{Ground Truth}}& \multicolumn{3}{c}{Our Method (All Frames)} & \multicolumn{3}{c}{Our Method (Key Frames)} & \multicolumn{3}{c}{Cartographer} \\
\cmidrule(lr){3-5} \cmidrule(lr){6-8} \cmidrule(lr){9-11}
& & Unknown & Free & Occupied & Unknown & Free & Occupied & Unknown & Free & Occupied \\
\midrule
\multirow{3}{*}{Simulation 1} 
& Unknown  & \textcolor{red}{\textbf{99.798\%}}  & 0.020\% & 0.182\% & \textcolor{blue}{\textbf{99.598\%}} & 0.072\% & 0.330\% & 95.616\% & 2.822\% & 1.562\% \\
& Free     & 0.022\%  & \textcolor{red}{\textbf{99.938\%}} & 0.040\% & 0.094\% & \textcolor{blue}{\textbf{99.824\%}} & 0.082\% & 1.290\% & 97.678\% & 1.032\% \\
& Occupied & 5.436\%  & 2.334\%  & \textcolor{red}{\textbf{92.230\%}} & 13.562\% & 3.142\% & \textcolor{blue}{\textbf{83.296\%}} & 30.053\% & 53.280\% & 16.667\% \\
\midrule
\multirow{3}{*}{Simulation 2} 
& Unknown  & \textcolor{red}{\textbf{99.846\%}}  & 0.010\% & 0.144\% & \textcolor{blue}{\textbf{99.696\%}} & 0.062\% & 0.242\% & 96.868\% & 1.743\% & 1.389\% \\
& Free     & 0.016\%  & \textcolor{red}{\textbf{99.846\%}} & 0.138\% & 0.076\% & \textcolor{blue}{\textbf{99.848\%}} & 0.076\% & 0.593\% & 98.584\% & 0.823\% \\
& Occupied & 6.434\%  & 3.738\%  & \textcolor{red}{\textbf{89.828\%}} & 11.694\% & 2.806\% & \textcolor{blue}{\textbf{85.500\%}} & 23.943\% & 50.795\% & 25.262\% \\
\midrule
\multirow{3}{*}{Simulation 3} 
& Unknown  & \textcolor{red}{\textbf{99.812\%}}  & 0.032\% & 0.156\% & \textcolor{blue}{\textbf{99.258\%}} & 0.430\% & 0.312\% & 96.968\% & 1.574\% & 1.458\% \\
& Free     & 0.036\%  & \textcolor{red}{\textbf{99.928\%}} & 0.036\% & 0.554\% & \textcolor{blue}{\textbf{99.352\%}} & 0.094\% & 1.018\% & 98.110\% & 0.872\% \\
& Occupied & 4.500\%  & 2.358\%  & \textcolor{red}{\textbf{93.142\%}} & 16.788\% & 3.736\% & \textcolor{blue}{\textbf{79.476\%}} & 26.420\% & 44.928\% & 28.652\% \\
\bottomrule
\end{tabular}
\end{table*}


From the results of our first stage shown in Fig. \ref{fig_simulation}(c), it is evident that further optimization is needed at the edges of objects. This is due to sampling points with different occupancy values being projected onto the coarse grid cells at object boundaries, causing inaccurate data associations. These results highlight the necessity of the second stage in our multi-resolution strategy (Algorithm \ref{alg_3}) to improve accuracy. Additionally, as shown in Fig. \ref{fig_simulation}(c), the non-edge areas (stable areas) of the occupancy grid maps are well optimized, supporting the fact that including occupancy cell vertices of non-edge areas in the state variables for further optimization is unnecessary.

For a quantitative comparison of the occupancy maps, we apply the same threshold across all methods to convert occupancy values into occupancy states. The mapping problem is treated as a classification task, categorizing each grid cell as free, occupied, or unknown. The mapping performance of our method and Cartographer is summarized in Table \ref{tab_map_accuracy}, clearly showing that both variants of our method, one using all frames and the other using keyframes, significantly outperform Cartographer in terms of map accuracy.

\begin{table}[ht]
		\centering
		\caption{Accuracy of the Occupancy Grid Map.}
		\label{tab_auc}
		\setlength{\tabcolsep}{4.8 mm}{
		\begin{tabular}{llccccc}\toprule
		& & AUC & Precision   \\ \hline
		\multirow{3}{*}{Simulation 1}& Cartographer & 0.90878 & 0.95651 \\ & Ours (All) &\textcolor{red}{\textbf{0.99999}} &\textcolor{red}{\textbf{0.99773}}\\ & Ours (Key) & \textcolor{blue}{\textbf{0.99902}} & \textcolor{blue}{\textbf{0.99548}} \\ \hline 
			\multirow{3}{*}{Simulation 2}& Cartographer & 0.96132 &  0.96829 \\ & Ours (All) & \textcolor{red}{\textbf{0.99926}} & \textcolor{red}{\textbf{0.99721}} \\ & Ours (Key) & \textcolor{blue}{\textbf{0.99914}} & \textcolor{blue}{\textbf{0.99638}}\\ \hline
		\multirow{3}{*}{Simulation 3} & Cartographer & 0.92696 &0.96592 \\ & Ours (All)& \textcolor{red}{\textbf{0.99974}} & \textcolor{red}{\textbf{0.99771}} \\ & Ours (Key) & \textcolor{blue}{\textbf{0.99748}} & \textcolor{blue}{\textbf{0.99113}}\\
		 \hline
		\end{tabular}
  }
  % \vspace{-0.5em}
\end{table}

We also assess performance using AUC (Area under the ROC curve) \cite{bradley1997use} and precision, with ground truth labels generated from the occupancy map based on ground truth poses. To ensure a fair comparison, all unknown cells are excluded from this evaluation, as AUC is a binary classification metric \cite{bradley1997use}. Table \ref{tab_auc} presents the results, showing that our method using all frames achieves the highest performance in both metrics. Even with only key frames, our method surpasses Cartographer. A key factor resulting in Cartographer's lower mapping quality is its lack of a batch optimization method to address errors during submap construction. Although its scan-to-map matching approach reduces cumulative errors more effectively than scan-to-scan matching, its accuracy still falls short compared to our algorithm. Cartographer performs pose graph optimization to adjust the coordinate frames of submaps only when loop closure is detected, leaving errors within the submaps uncorrected. Although global pose graph optimization is applied at the end of the process, it often suffers from an excess of inaccurate and conflicting relative measurements, as well as its susceptibility to local minima, limiting its effectiveness in correcting these errors. Moreover, pose graph optimization typically does not enhance the local details of maps, as it focuses solely on optimizing poses without jointly considering the map. This further highlights the advantage of our approach, which jointly optimizes both robot poses and the occupancy map.


\subsection{Comparisons using Practical Datasets} \label{sec_practical}

\begin{figure}[t]
\centering
\includegraphics[width=0.46\textwidth]{Real_OGM_New.pdf}
\caption{\label{fig_result_compare_OGM} The occupancy grid maps from Cartographer, our method using all frames, and our method using key frames. }
% \vspace{-1.5em}
\end{figure}


\begin{figure}[t]
\centering
\includegraphics[width=0.46\textwidth]{Real_Scan_New.pdf}
\caption{\label{fig_result_compare_scan} The point cloud maps from Cartographer, our method using all frames, and our method using key frames.}
\end{figure}

\begin{table}[t]
		\centering
		\caption{Time Consumption of Different Algorithms.}
		\label{table_time_compare}
		\setlength{\tabcolsep}{2.5 mm}{
		\begin{tabular}{lccc}\toprule
		Dataset& & Computational Time (s)& \\ \hline
			     & Cartographer  & Ours (All) & Ours (Key)  \\ 
		Car Park & 168  & 119  & \textbf{44} \\
		Museum b0 & 152  & 126 & \textbf{38} \\
		C5 & 146 & 137 & \textbf{35} \\
		% Simulation 1 & 192  & 148 & \textbf{33} \\
		% Simulation 2 & 174  & 193 & \textbf{57} \\
		% Simulation 3 & 78  & 132 & \textbf{40} \\
		\hline
		\end{tabular}
		}
        % \vspace{-2em}
\end{table}

We use three normal-scale practical datasets, namely Deutsches Museum b0 \cite{hess2016real}, Car Park \cite{zhao20212d} and C5, to compare our method with Cartographer in terms of the constructed occupancy grid maps and optimized poses. 

The mapping quality is evaluated by comparing the details of the constructed maps. Additionally, point cloud maps, which are generated using the endpoint projections of scan points and optimized poses, serve as a reference for pose accuracy. For the Car Park and C5 datasets, our method is initialized with poses from odometry inputs, whereas for the Museum b0 dataset, initialization relies on poses from scan matching due to the absence of odometry. The occupancy grid maps and point cloud maps generated by Cartographer, our method using all frames, and our method using key frames for the three datasets are shown in Fig. \ref{fig_result_compare_OGM} and Fig. \ref{fig_result_compare_scan}. Red dotted lines highlight areas where our results outperform Cartographer in both the occupancy grid maps and point cloud maps. Comparing Fig. \ref{fig_result_compare_OGM}(a) and Fig. \ref{fig_result_compare_OGM}(b), our method provides more precise boundaries for the occupancy grid maps due to joint optimization of robot poses and the occupancy map. Similarly, the comparison between Fig. \ref{fig_result_compare_scan}(a) and Fig. \ref{fig_result_compare_scan}(b) illustrates that our method achieves more accurate poses. 

Moreover, our method outperforms Cartographer when using only key frames, as evident from the comparison of Fig. \ref{fig_result_compare_OGM}(a) and Fig. \ref{fig_result_compare_scan}(a) with Fig. \ref{fig_result_compare_OGM}(c) and Fig. \ref{fig_result_compare_scan}(c). These results show that, despite Cartographer introducing loop closure detection, it still produces non-negligible pose errors, leading to point clouds that fail to fully overlap observations of the same obstacle at different poses. While the point cloud maps generated by our method also have non-overlapping parts, these areas are significantly smaller compared to those from Cartographer. 

% These experiments demonstrate that both variants of our method reduce pose errors and generate more accurate occupancy grid maps by jointly optimizing robot poses and the occupancy map. 

Additionally, we assess the time consumption of our method and Cartographer on these three datasets. Table \ref{table_time_compare} shows that our method consistently requires less time than Cartographer across all datasets when using all frames and achieves significantly better efficiency when using selected key frames.

Finally, it is worth noting that some well-known public datasets, such as Radish \cite{Radish}, were collected before 2014 with outdated sensors, leading to low-quality data with poor scanning frequency and odometry accuracy. These issues hinder the performance of Cartographer, often requiring meticulous parameter tuning but still yielding suboptimal results. In contrast, our method performs well on these datasets. Although we do not include these comparisons in this paper, we make our results available on our code page\footnote{\url{https://github.com/WANGYINGYU/Occupancy-SLAM}}.





\subsection{Assessment of Robustness to Initial Guess}

% While our method has demonstrated robustness when initialized with odometry inputs or scan matching under reliable sensor conditions in both simulation and real-world experiments, this subsection illustrates its capability for convergence even when initialized with significantly noisy poses. In this subsection, we use all frames for robustness assessment.

While our method has demonstrated robustness when initialized with odometry inputs or scan matching under reliable sensor conditions in both simulation and real-world experiments, this subsection highlights its capability to converge even when initialized with significantly noisy poses. We use all frames in this subsection to assess robustness.

First, we use Simulation 1 dataset to quantitatively evaluate the convergence percentage and the accuracy of optimized poses under different noise levels. We add zero-mean uniformly distributed noises with different bounds to the ground truth of the poses to generate each group of ten sets of initial poses for the experiments to count convergence rates and average errors. Specifically, for noise level 1, the noise for translation is within $[-2$ m, $2$ m$]$ and the noise for rotation is within $[-0.5$ rad, $0.5$ rad$]$; for level 2, $[-4$ m, $4$ m$]$ and $[-1$ rad, $1$ rad$]$; for level 3, $[-6$ m, $6$ m$]$ and $[-1.5$ rad, $1.5$ rad$]$. The poses with different noise levels of Simulation 1 dataset are visualized using the generated occupancy grid maps, as shown in Fig. \ref{fig_OGM_246}. The convergence results are depicted in Table \ref{table_robustness}, showing that our method can $100\%$ converge when initialized with challenging noisy poses of level 1 and level 2. Our method still has a high convergence percentage ($80\%$) when initialized with noisy poses of level 3. Our algorithm using other simulation datasets has similar robustness performance.

\begin{table}[t]
		\centering
		\caption{Robustness to Initialization.}
		\label{table_robustness}
		\setlength{\tabcolsep}{0.9 mm}{
		\begin{tabular}{lccc}\toprule
		\thead{Noise Level} & \thead{Convergence\\ Percentage} & \thead{Average MAE of \\Translation (m)} & \thead{Average MAE of\\ Rotation (rad)}\\ \hline
		Level 1 (2 m, 0.5 rad)  & 100\% & 0.00679 & 0.0005   \\
		Level 2 (4 m, 1 rad)  & 100\% & 0.00682 & 0.0005  \\
		Level 3 (6 m, 1.5 rad)  & 80\% & 0.01742  & 0.0012 \\
		\hline
		\end{tabular}
		}
\end{table}

\begin{figure}[t]
\centering
\includegraphics[width=0.47\textwidth]{./OGM_Level246.pdf}
\caption{\label{fig_OGM_246} Examples of occupancy grid maps generated from poses with different noise levels as shown in Table \ref{table_robustness} using Simulation 1 dataset.}
% \vspace{-1.5em}
\end{figure}

Moreover, for all practical datasets, we additionally add random zero-mean uniform distribution noises ($[-2$ m, $2$ m$]$ for translation and $[-0.5$ rad, $0.5$ rad$]$ for rotation) to the poses obtained from Cartographer as the initial guess. The initial occupancy maps obtained by using the noisy initial poses are shown in Fig. \ref{fig_noise_initial}(a). Fig. \ref{fig_noise_initial}(b) shows the remapped occupancy grid maps using our optimized poses, and Fig. \ref{fig_noise_initial}(c) shows the point cloud maps using our optimized poses. This experiment shows that our approach can converge from initial guesses with significant errors and also generate good results.


\begin{figure}[t]
% \vspace{-5mm}
\centering
\includegraphics[width=0.5\textwidth]{Noise_Initial.pdf}
\caption{\label{fig_noise_initial}The occupancy grid maps and point cloud maps generated using noisy poses for initialization by our approach. (a) and (b) display the remapped occupancy maps generated from the noisy initial poses and our optimized poses, respectively, and (c) shows the point cloud maps created by projecting the endpoints of scans using our optimized poses.}
% \vspace{-2em}
\end{figure}

\subsection{Discussion about the Effectiveness of Different Stages} \label{sec_discuss}


\begin{figure*}[tp]
\centering \subfigure[Simulation 1] {\label{fig_group_error_1}
\includegraphics[width=0.32\textwidth]{./Group_Error_3640.pdf}}
\centering \subfigure[Simulation 2] {\label{fig_group_error_2}
\includegraphics[width=0.32\textwidth]{./Group_Error_3720.pdf}}
\centering \subfigure[Simulation 3] {\label{fig_group_error_3}
\includegraphics[width=0.32\textwidth]{./Group_Error_2680.pdf}}
\caption{Comparison of translation and rotation errors for simulated datasets using three methods: our full method (Algorithm \ref{alg_flowchart}), our Algorithm \ref{alg_1} initialized by Cartographer's poses with a high-resolution map \cite{Zhao-RSS-22}, and our Algorithm \ref{alg_1} initialized by poses obtained from our first stage with a high-resolution map.}
\label{fig_group_error}
\vspace{-1em}
\end{figure*}


In previous sections, we demonstrated the accuracy, robustness, and efficiency of our proposed method. In this section, we discuss the effectiveness of its different parts.


As demonstrated in Table \ref{tab_comparison}, Fig. \ref{fig_simulation}, Fig. \ref{fig_result_compare_OGM}, and Fig. \ref{fig_result_compare_scan}, the accuracy of the poses and the map obtained from our full approach (Algorithm \ref{alg_flowchart}) is much better than those obtained from Cartographer. This confirms the advantage of jointly optimizing both the robot poses and the occupancy map. 


% One may ask, how about performing only Algorithm \ref{alg_1} with a high-resolution map directly? Will the result be even better? To answer this question clearly, in this subsection, we compare our full approach with Algorithm \ref{alg_1} using a high-resolution map. We consider three different initialization: 

% In the following, we refer to these three approaches as  \textit{Algorithm \ref{alg_1} (High, O/S)}, \textit{Algorithm \ref{alg_1} (High, Carto)}, and \textit{Algorithm \ref{alg_1} (High, First)}, respectively.   

One potential question is whether using only Algorithm \ref{alg_1} with a high-resolution map would yield even better results. To investigate this, we compared our full approach with Algorithm \ref{alg_1} using a high-resolution map. We tested three initialization: (1) \textit{Algorithm \ref{alg_1} (High, O/S)}: initialization using odometry inputs or scan matching; (2) \textit{Algorithm \ref{alg_1} (High, Carto)}: initialization using Cartographer's poses (as proposed in our conference paper \cite{Zhao-RSS-22}); and (3) \textit{Algorithm \ref{alg_1} (High, First)}: initialization using the poses obtained by our first stage. 



% First, \textit{Algorithm \ref{alg_1} (High, O/S)} fails to converge on most datasets, while our full method can converge very well, which indicates the improved robustness of our multi-resolution strategy.   

% The comparison of our full method with \textit{Algorithm \ref{alg_1} (High, Carto)} and \textit{Algorithm \ref{alg_1} (High, First)} using all the five groups simulation datasets are shown in Fig. \ref{fig_group_error}. It can be seen that the accuracy of our full method is essentially similar in all groups, while the accuracy of \textit{Algorithm \ref{alg_1} (High, Carto)} varies drastically. This means the approach proposed in our conference paper \cite{Zhao-RSS-22} not only requires an accurate initial value but also generates less accurate poses than our new approach. In addition, by comparing \textit{Algorithm \ref{alg_1} (High, Carto)} with \textit{Algorithm \ref{alg_1} (High, First)}, it also confirms that the poses obtained in our first stage are more accurate than those of Cartographer.

First, \textit{Algorithm \ref{alg_1} (High, O/S)} fails to converge on most datasets, while our full method converges successfully, indicating the improved robustness of our multi-resolution strategy.

The comparison between our full method, \textit{Algorithm \ref{alg_1} (High, Carto)}, and \textit{Algorithm \ref{alg_1} (High, First)} across all five simulation groups is shown in Fig. \ref{fig_group_error}. It can be observed that the accuracy of our full method remains stable across all groups, while the accuracy of \textit{Algorithm \ref{alg_1} (High, Carto)} varies drastically. This suggests that the approach in our conference paper \cite{Zhao-RSS-22} not only requires an accurate initial guess but also produces less accurate poses than our new method. Moreover, comparing \textit{Algorithm \ref{alg_1} (High, Carto)} with \textit{Algorithm \ref{alg_1} (High, First)} further confirms that the poses obtained in our first stage are more accurate than those of Cartographer.



% It is worth discussing that our full method uses the selected high-resolution map for optimization in the second stage, and it can be observed in Fig. \ref{fig_group_error} that the accuracy of our full approach is even higher than the optimization using the full high-resolution map (i.e., \textit{Algorithm \ref{alg_1} (High, First)}) in some experiments. The potential reason is that when the relatively accurate poses and occupancy map are obtained, the dropped \textcolor{red}{cell vertices} and the corresponding observations contain little information. If all cells and corresponding observations are retained for optimization, it may affect the algorithm's ability to obtain the best solution, as all observation terms are assigned uniform weights. Another potential reason is that the smoothing term in (\ref{eq_objective_func}), by spreading the occupancy values to unknown \textcolor{red}{cell vertices}, may introduce errors that could affect the convergence of the optimization algorithm. 

It is also worth noting that our full method utilizes the selected high-resolution map for optimization in the second stage. As shown in Fig. \ref{fig_group_error}, in certain experiments, the accuracy of our full approach surpasses that of the optimization using the full high-resolution map (\textit{Algorithm \ref{alg_1} (High, First)}). A possible explanation is that once relatively accurate poses and occupancy maps are obtained, the dropped cell vertices and corresponding observations contain little information. Retaining all cells and corresponding observations for optimization may prevent the algorithm from finding the optimal solution, as all are observation items given uniform weights. Another reason could be the smoothing term in (\ref{eq_objective_func}), which spreads occupancy values to unknown cell vertices, potentially introducing errors that affect the convergence of the optimization.

In terms of time consumption, our full approach is much more efficient than \textit{Algorithm \ref{alg_1} (High, Carto)}. For instance, \textit{Algorithm \ref{alg_1} (High, Carto)} consumes over 21,000 seconds with the Car Park dataset. In comparison, the time consumption of our full approach using all frames is 119 seconds (less than 0.6\%), and using only key frames, it takes only 44 seconds (about 0.2\%). This substantial reduction in time consumption highlights the efficiency improvements of our method over our conference paper \cite{Zhao-RSS-22}.

% In terms of time consumption, our full approach is significantly more efficient than \textit{Algorithm \ref{alg_1} (High, Carto)}. For instance, \textit{Algorithm \ref{alg_1} (High, Carto)} consumes over 21,000 seconds when using the Car Park dataset. In comparison, the time consumption of our full approach using all frames is 119 seconds (less than 0.6\%), and using only key frames, it takes 44 seconds (approximately 0.2\%). This substantial reduction in time consumption underscores the significant efficiency improvements of our current method over our conference paper \cite{Zhao-RSS-22}.

The reduction in time consumption stems from both the multi-resolution strategy, which reduces time per iteration, and the fewer iterations needed in the second stage due to the selected high-resolution map. Our experiments show that only about two iterations are needed in the second stage with the selected high-resolution map, fewer than in \textit{Algorithm \ref{alg_1} (High, First)}. This is likely because the selected high-resolution map focuses on critical states, with observations containing the most relevant information, enabling faster convergence.


In summary, compared to our conference paper \cite{Zhao-RSS-22}, our new multi-resolution method does not require precise initialization, is far more efficient, and achieves higher accuracy.

% In addition, compared with \textit{Algorithm \ref{alg_1} (High, Carto)}, the time consumption of our full approach is reduced by $2-3$ orders of magnitude. For example, \textit{Algorithm \ref{alg_1} (High, Carto)} consumes more than $21,000$ seconds using the Car Park dataset. Compared to \textit{Algorithm \ref{alg_1} (High, Carto)}, the time consumption of our full approach using all frames is $119$ seconds (less than $0.6\%$), and the time consumption of our full approach using only key frames is 44 seconds which is approximately $0.2\%$. The significant reduction in time consumption shows the significantly improved efficiency of our current method over our conference paper \cite{Zhao-RSS-22}.

% The substantial reduction in time consumption of our full approach is attributed not only to the introduced multi-resolution strategy, which reduces the time consumption per iteration but also to the reduction in the number of iterations in the second stage, which is a result of utilizing the selected high-resolution map. Through the experiments, we find that only about $2$ iterations in the second stage are required to converge using the selected high-resolution map, which is smaller than the number of iterations needed in \textit{Algorithm \ref{alg_1} (High, First)}. The possible reason is that, in the case of using the selected high-resolution map, these selected \textcolor{red}{cell vertices} focus on the most critical states, and the corresponding observations contain the most important information, allowing the optimization problem to converge much faster.

% In summary, as compared with our conference paper \cite{Zhao-RSS-22}, our new multi-resolution method does not require accurate initialization, is much more efficient, and achieves a higher level of accuracy in most cases.   


\subsection{Ablation Study on the Resolution Ratio}

\begin{table}[t]
		\centering
		\caption{Impact of First-Stage Resolution Settings.}
		\label{table_ablation}
		\setlength{\tabcolsep}{1mm}{
		\begin{tabular}{llcccc}\toprule
		& & $r=20$ & $r=10$  & $r=5$ & $r=2$ \\   \hline   

		 \multirow{5}{*}{Simulation 1} & MAE/Trans (m) First& 0.02352  &  0.02206 & \textbf{0.02118} & 0.17318\\
		\quad & MAE/Rot (rad) First& 0.00116  &  \textbf{0.00098} & 0.00124 & 0.01066\\
		\quad & MAE/Trans (m) All & 0.00812  & \textbf{0.00640}  & 0.00728 & 0.16066\\
		\quad & MAE/Rot (rad) All&  0.00062 & 0.00060  & \textbf{0.00054} & 0.01008\\ 
		\quad & Total Time (s) & \textbf{118}  &  148 & 262 & 2183\\
		\hline

		\multirow{5}{*}{Simulation 2} & MAE/Trans (m) First & 0.03938 & 0.03224 & \textbf{0.01984} & 0.09160\\
		\quad & MAE/Rot (rad) First&  0.00332 & 0.00220  & \textbf{0.00108} & 0.00314\\
		\quad & MAE/Trans (m) All& 0.01742  & 0.00858  & \textbf{0.00584} & 0.08018\\
		\quad & MAE/Rot (rad) All& 0.00064  & 0.00062  & \textbf{0.00052} & 0.00286\\ 
		\quad & Total Time (s)& \textbf{149}  & 193  & 321 & 2685\\
		\hline

		\multirow{5}{*}{Simulation 3} & MAE/Trans (m) First& 0.06708  &  0.02624  & \textbf{0.01776} & 0.03570\\
		\quad & MAE/Rot (rad) First&  0.00384 & 0.00164  & \textbf{0.00124}  & 0.00278\\
		\quad & MAE/Trans (m) All& 0.01586  & \textbf{0.00726}  & 0.00816 & 0.02082\\
		\quad & MAE/Rot (rad) All& 0.00100  & \textbf{0.00058}   & 0.00068 & 0.00102\\ 
		\quad & Total Time (s)& \textbf{125}  &  132 & 185 & 1041\\
		\hline
		\end{tabular}
		}
        % \vspace{-1.5em}
\end{table}

In this section, we perform ablation experiments on simulation datasets to analyze the impact of varying resolution settings in the first stage of the multi-resolution strategy on overall optimization performance.

We assess accuracy and computational time using three simulation datasets, with the resolution in the second stage fixed at $s^{h} = 0.05$ m. The resolution ratios $r$ between the first and second stages are set to 2, 5, 10, and 20, respectively. To ensure consistency, a fixed selection range of $d=1.5$ m is applied uniformly across all datasets. 

  
The results, shown in Table \ref{table_ablation}, reveal that $r=10$ achieves the best trade-off between time consumption and accuracy. While $r=20$ minimizes time consumption, it reduces the accuracy of poses in the first stage, adversely impacting final optimization accuracy. Conversely, $r=5$ improves pose accuracy in the first stage at the cost of higher time consumption but does not consistently enhance final accuracy. Notably, $r$ may need adjustment for other high-resolution settings.


\subsection{Using Submap Joining in Large-scale Environments}

We have demonstrated that our approach accurately and robustly handles normal-scale simulated and practical environments. In this section, we evaluate its efficiency and effectiveness in large-scale environments and long-term trajectories by integrating our Occupancy-SLAM algorithm with the proposed occupancy submap joining approach. The dataset is divided into multiple segments, where Algorithm \ref{alg_flowchart} is used to construct submaps, followed by applying the submap joining method in Section \ref{Sec_submap} to generate the optimized global occupancy map and robot trajectory.

We validate our method on two large-scale datasets, Deutsches Museum b2 \cite{hess2016real} and C3, and compare it with Cartographer. As shown in Fig. \ref{fig_large_environment}, our occupancy grid maps outperform those of Cartographer, demonstrating the capability of our method to handle large-scale environments and long-term trajectories effectively. 

% The datasets have map sizes of 250 m $\times$ 200 m and 150 m $\times$ 125 m, containing 51833 and 24402 scans, with trajectory lengths of 1390 seconds and 610 seconds, respectively.

\begin{figure}[t]
\centering \subfigure[b2] { \label{fig_large_b2}
\includegraphics[width=0.253\textwidth]{./b2_Large_New.pdf}}\hspace{-0.4em}
\centering \subfigure[C3] {\label{fig_large_C3}
\includegraphics[width=0.2195\textwidth]{./C3_Large_New_1.pdf}}
\caption{\label{fig_large_environment} Comparison of results between our method and Cartographer on two large-scale practical datasets. The first row shows Cartographer's results, while the second row shows ours. In (b), the red dotted lines serve as references, highlighting that Cartographer's right wall appears more curved, whereas our result aligns more closely with a straight line.}
% \vspace{-1.5em}
\end{figure}


\subsection{Computational Complexity Analysis}
In this section, we analyze the computational complexity and evaluate the time consumption of our method using large-scale datasets.

The Gauss-Newton method for solving the joint optimization of local maps and poses in (\ref{Least Squares}) and submap joining in (\ref{eq_NLLS_joining}) primarily depends on calculating Jacobian $\mathbf{J}$, and constructing and solving the sparse linear system in (\ref{Gauss-Newton}) \cite{konolige2008frameslam}. We analyze each part's complexity separately due to differences in the NLLS formulation.

For the local map and poses joint optimization problem, the objective function consists of the observation term, the odometry term, and the smoothing term. Let ${\lambda(\mathbb{S})}$ denotes the number of sampling points $\mathbb{S}$, then the number of items in the objective function is $\mathfrak{d}_{row} =\lambda(\mathbb{S})+3(n-1)+2{c_w}{c_h}+c_w+c_h$, and the state vector size is $\mathfrak{d}_{col}=3n+(c_w+1)(c_h+1)$. Considering Jacobian of the smoothing term $\mathbf{J}_S$ can be pre-calculated before optimization, the number of non-zero elements of Jacobian matrix that need to be computed for each iteration is $\mathfrak{d}_J = 7\lambda(\mathbb{S}) + 6(n-1)$. Therefore, for each iteration, the computation complexity of Jacobian calculation, constructing (\ref{Gauss-Newton}) and solving (\ref{Gauss-Newton}) is $\mathcal{O}(\mathfrak{d}_J)$, $\mathcal{O}(\mathfrak{d}_{J}\mathfrak{d}_{col})$, and $\mathcal{O}({\mathfrak{d}^3_{col}})$, respectively. Therefore, the total computation complexity per iteration for the local map and poses joint optimization problem is $\mathcal{O}(\mathfrak{d}_{J}+\mathfrak{d}_J{\mathfrak{d}_{col}}+\mathfrak{d}^3_{col})$. Due to our proposed multi-resolution joint optimization strategy and keyframe selection, both $\mathfrak{d}_{J}$ and $\mathfrak{d}_{col}$ remain small during the first and second stages of optimization, making the computation time for this part manageable.

% (i.e., \textcolor{red}{cell vertices} with non-zero occupancy values) Similar to the computation complexity of the local map and poses joint optimization problem,

For our submap joining algorithm, the number of observations depends on the total number of cell vertices of the global occupancy map observed in each submap, denoted $\mathfrak{d}_{obs}^{G}$. Considering that some cell vertices will be observed repeatedly under different submaps, this number slightly exceeds the number of non-unknown cell vertices in the global map. Thus, the number of non-zero elements of Jacobian matrix is $\mathfrak{d}_J^G = 4\mathfrak{d}_{obs}^G$, and the state vector size is $\mathfrak{d}_{col}^{G} = 3n_L+(c_w^G+1)(c_h^G+1)$. The computation complexity per iteration is $\mathcal{O}(\mathfrak{d}_{J}^G+\mathfrak{d}_J^G{\mathfrak{d}_{col}^G}+{\mathfrak{d}_{col}^G}^3)$.
Although the global occupancy map tends to be relatively large, the sub-matrix of Hessian w.r.t. the global occupancy map is diagonal. To speed up computation, we apply the Schur complement \cite{zhang2006schur} to make the normal equation solving highly efficient.

Finally, we evaluate the time consumption of our method using both all frames and selected keyframes in large-scale environments to support our computation complexity analysis and compare it to Cartographer. For Museum b2 dataset, Cartographer takes 1424 seconds, while our method takes 1250 seconds when using all frames and 363 seconds with selected key frames. For C3 dataset, Cartographer takes 610 seconds, while our method takes 742 seconds with all frames and 236 seconds with selected key frames. In our total time consumption, the submap joining method consumes less than 10 seconds on both datasets. It can be seen that the time consumption of our method is comparable to that of Cartographer when all frames are used and much lower than that of Cartographer when selected key frames are used. These results demonstrate the efficiency of our multi-resolution joint optimization strategy and submap joining approach. 

\section{Preliminary Results in 3D Case}\label{sec_3d}

While this paper primarily focuses on demonstrating the benefits of jointly optimizing the robot poses and occupancy map in 2D, we also present some preliminary 3D results to illustrate that our idea can be extended to 3D applications.



\subsection{Extension of the Algorithms to 3D Case}

Our approach for jointly optimizing robot poses and the occupancy map extends naturally to 3D, where the information, robot poses, and occupancy maps are all represented in 3D. Most problem formulations and algorithms can be adapted with minor adjustments. 

For our local map and poses optimization method, observations transition from 2D laser scans to 3D LiDAR scans, robot poses and odometry involve 6 degree-of-freedom (DoF), and the map representation becomes 3D. Consequently, (\ref{eq_interp}) and (\ref{eq_NP}) need to be replaced from bilinear to trilinear interpolation and its inverse operation. 
For the objective function (\ref{eq_objective_func}), the odometry term (\ref{eq_odometry_term}) should be replaced with a 6 DoF odometry term for 3D, and the smoothing term (\ref{eq_smoothing_term}) should include a smoothing penalty for the z-axis, Jacobians $\mathbf{J}_P$, $\mathbf{J}_M$, $\mathbf{J}_O$, and $\mathbf{J}_S$ described in Appendices need to be adjusted accordingly. 

The submap joining problem in 3D remains largely similar to the 2D case, except that the projection relation extends from 2D-2D to 3D-3D, enabling the solution of 6 DoF poses and the 3D global occupancy map in the NLLS problem (\ref{eq_NLLS_joining}).

\subsection{3D Experimental Results}
\subsubsection{Evaluation metrics and state-of-the-art methods}
We evaluate our method's performance in 3D using absolute trajectory error for poses, aligning and comparing results with ground truth via EVO \cite{grupp2017evo}, as used in \cite{liu2023large,rosinol2021kimera}. In all the experiments, we use the odometry information provided by the dataset as initialization if it is available. Otherwise, we use FAST-LIO2 \cite{xu2022fast} to obtain the odometry information. To evaluate our method, we compare our method against state-of-the-art methods: BALM2 \cite{liu2023efficient}, HBA \cite{liu2023large}, and Voxgraph \cite{reijgwart2019voxgraph}. BALM2 optimizes the planar feature parameters of the point cloud and the robot's poses. HBA proposes a hierarchical bundle adjustment to optimize the consistency of the planar surfaces of point clouds and robot poses. Voxgraph builds SDF-based submaps from point clouds, uses SDF-to-SDF registration for relative submap measurements, and incrementally optimizes submap frames. HBA and Voxgraph can deal with large-scale environments, while BALM2 focuses on normal-scale environments.  

\subsubsection{Datasets}
We perform comparisons using three real-world datasets. (1) The Newer College Dataset \cite{ramezani2020newer}: The first five sequences from the \textit{shorter experiment}, collected with a handheld Ouster OS-1 LiDAR scanner at New College, Oxford. The environment includes lawns, buildings, a tunnel, and a garden. Ground truth is provided by a BLK360 LiDAR scanner to capture a detailed 3D map and then infer the ground truth of poses with centimeter-level accuracy.
(2) KITTI Dataset \cite{Geiger2013IJRR}: Sequence 07, a demo dataset for HBA, collected with a Velodyne HDL-64E LiDAR scanner mounted on a car. Ground truth poses is provided by RTK-GPS/INS.
(3) Arche Dataset \cite{reijgwart2019voxgraph}: A demo dataset for Voxgraph, collected using an Ouster OS1 LiDAR mounted on a hexacopter MAV in a disaster area. Ground truth positions are provided by an RTK-GNSS system. 

The Newer College Dataset is used to evaluate high-precision performance in normal-scale environments, while the KITTI and Arche datasets are used to test performance in large environments with long trajectories.

\begin{figure*}[t]
\centering
\includegraphics[width=0.99\textwidth]{./PC_Comparison.pdf}
\caption{\label{fig_3d_pointcloud} Some local point cloud maps from the Arche dataset. The first row shows point cloud maps generated using odometry from ROVIO (also used for submap construction in Voxgraph), while the second row shows maps generated with our optimized poses using the same LiDAR scans. BALM2 fails to produce results when using the same odometry and scans as inputs in all these local environments.}
\vspace{-1em}
\end{figure*}
 
\subsubsection{Experiments on normal-scale environments}
We evaluate the performance of our proposed method without submap joining in normal-scale environments.

First, we evaluate BALM2 and our method using the first five sequences of The Newer College Dataset, which encompass all scenarios within the dataset. As shown in Table \ref{tab_comparison_3d_local}, our method outperforms BALM2 across all metrics, except for the RMSE in Seq. 1, and significantly outperforms the odometry inputs from FAST-LIO2 in all metrics. 

\begin{table}[t]
		\centering
		\caption{Absolute Trajectory Error (MAE/RMSE, Meters) in Normal-scale Environments for Different 3D Methods.}
		\label{tab_comparison_3d_local}
		\setlength{\tabcolsep}{0.6 mm}{
		\begin{tabular}{lcccccc}\toprule
		Method	& Seq. 0 & Seq. 1 & Seq. 2 & Seq. 3 & Seq. 4 \\ \hline
		FAST-LIO2 & 0.518/0.717 & 0.181/0.202  & 0.121/0.132 &0.188/0.200&0.571/0.723\\
		BALM2 & 0.283/0.326 & 0.112/\textbf{0.123}  & 0.104/0.109 &0.144/0.158 &0.298/0.344 \\
        Ours & \textbf{0.185}/\textbf{0.232}  & \textbf{0.097}/\textbf{0.123}  & \textbf{0.091}/\textbf{0.099} & \textbf{0.141}/\textbf{0.155} &\textbf{0.238}/\textbf{0.284}\\ \hline
		\end{tabular}
		}
        % \vspace{-2em}
\end{table}

Next, we test robustness in a challenging environment with noisy odometry input using the Arche dataset. This dataset, collected by a hexacopter MAV in an unstructured environment, is influenced by drone vibrations, flight speed, and environmental factors. Local point cloud maps built using odometry from ROVIO \cite{bloesch2017iterated} (also used to construct submaps in Voxgraph) are shown in the first row of Fig. \ref{fig_3d_pointcloud}. To evaluate BALM2 and our method, we partition the dataset into several short sequences, each lasting 10–20 seconds. BALM2 fails in all the sequences except during MAV start-up and landing due to insufficient planar features for optimization, while our method performs well on all the sequences. The second row of Fig. \ref{fig_3d_pointcloud} illustrates some of our results, demonstrating that our method is robust in 3D and does not rely on environmental assumptions. Additionally, the results confirm that our method achieves significantly higher pose accuracy than ROVIO.

\subsubsection{Experiments on large-scale environments}\label{sec_experiment_vox} 

We evaluate our method in large-scale environments with long trajectories using the KITTI and Arche datasets, comparing it with HBA and Voxgraph. For this experiment, we first build submaps by jointly optimizing poses and maps within submaps, then apply our submap joining algorithm to jointly optimize submap frame poses and the global occupancy map.

% The MAE and RMSE of the ATEs are summarized in Table \ref{tab_comparison_3d_large}, it is clear that our method achieves the best results on both datasets. In addition, the robot trajectories are shown in Fig. \ref{fig_trajectory_3d}, as it shows our method can achieve the best global consistent robot trajectories compared with other methods. It should be noted that the results of our method substantially lead Voxgraph on the KITTI dataset and significantly outperform HBA on the Arche dataset. The reason our method performs much better than Voxgraph on KITTI dataset is that Voxgraph relies on relative measurements from SDF-to-SDF registration for solving pose graph optimization, but in such autonomous driving scenarios, it is difficult to provide sufficient overlapping submaps for Voxgraph to calculate relative measurements between submaps. However, our proposed submap joining algorithm jointly optimizes poses of submaps' coordinate frames and the global occupancy map and, therefore, does not suffer in such environments. The performance of HBA on the Arche dataset is affected by highly unstructured environments and with data captured by moving MAV, as HBA relies on detecting and using planar features from the point cloud to do the optimization, which is similar to BALM2. However, in the case of odometry and point clouds collected during MAV motion, it is difficult for such methods to detect a sufficient number of good planar features. In addition, the planarity assumption does not tend to hold true in non-urban environments, such as the field.

Table \ref{tab_comparison_3d_large} summarizes the MAE and RMSE of absolute trajectory error, showing our method achieves the best results on both datasets. Fig. \ref{fig_trajectory_3d} illustrates that our approach can achieve the best global robot trajectories. Notably, our method significantly outperforms Voxgraph on the KITTI dataset and HBA on the Arche dataset. The relatively poor performance of Voxgraph on the KITTI dataset is due to its reliance on relative measurements from SDF-to-SDF registration, which requires sufficient overlapping submaps—a challenge in autonomous driving scenarios. In contrast, our submap joining algorithm jointly optimizes submap poses and the global occupancy map, avoiding this limitation. HBA underperforms on the Arche dataset due to its reliance on planar features for optimization, which is challenging in unstructured environments and during MAV motion. Odometry and point clouds from such scenarios make detecting sufficient planar features difficult, and the planarity assumption often fails in non-urban environments like disaster areas.


\begin{figure}[tbp]
\centering \subfigure[KITTI] {\label{fig_trajectory_1}
\includegraphics[height=0.15\textwidth]{./Traj_KITTI.pdf}}
\centering \subfigure[Arche] {\label{fig_trajectory_2}
\includegraphics[height=0.15\textwidth]{./Traj_Voxgraph_Demo_New.pdf}}
\caption{\label{fig_trajectory_3d} Robot trajectory results of datasets in large-scale environments. (a) and (b) show the trajectories of ground truth, Voxgraph \cite{reijgwart2019voxgraph}, HBA \cite{liu2023large}, and our method for KITTI dataset and Arche dataset.}
\end{figure}


\begin{table}[t]
		\centering
		\caption{Absolute Trajectory Error (MAE/RMSE, Meters) in Large-scale Environments for Different 3D Methods}
		\label{tab_comparison_3d_large}
		\setlength{\tabcolsep}{7mm}{
		\begin{tabular}{lcc}\toprule
		Method & KITTI & Arche  \\ \hline
		HBA \cite{liu2023large} & 0.342/0.364 & 4.123/4.789  \\
        Voxgraph \cite{reijgwart2019voxgraph} &0.926/1.002 & 0.700/0.833 \\
        Ours & \textbf{0.315}/\textbf{0.339}  & \textbf{0.275}/\textbf{0.378} \\ \hline
		\end{tabular}
		}
        % \vspace{-2em}
\end{table}

\subsection{Discussion}
The experimental results in this section demonstrate that our proposed idea of jointly optimizing the robot pose and the occupancy map can also lead to better solutions for the robot poses and occupancy maps in 3D cases. However, several challenges remain in 3D scenarios. For instance, 3D point clouds from LiDAR scanners are often sparse, particularly in the vertical direction, which can lead to observability issues in the optimization problem. This sparsity also results in inhomogeneous observation information, complicating the accurate representation of the 3D environment in occupancy maps. Furthermore, the large dimensions of 3D maps pose significant computational challenges in large-scale SLAM, requiring more efficient solving methods.


To address these challenges, several potential solutions can be explored. First, adopting compact representations for 3D occupancy maps, such as octree structures similar to Octomap \cite{hornung2013octomap} and \cite{vespa2019adaptive}, can enhance efficiency. Second, combining local map and pose optimization with hierarchical optimization and submap joining methods can further reduce computational time. Lastly, using continuous representations for 3D occupancy maps enables more precise gradient calculations, which can better guide the optimization process.

\section{Conclusion} \label{Sec_conclusion}
In this paper, we propose Occupancy-SLAM algorithm, which solves robot poses and occupancy map simultaneously. To enhance efficiency and robustness, we introduce a multi-resolution strategy. The first stage jointly optimizes poses and a low-resolution occupancy map to quickly achieve relatively accurate pose estimates, which are then used as the initial guess for the second stage. The second stage refines poses and a subset of the high-resolution map, focusing on critical boundary areas. Additionally, we extend this framework to an occupancy grid-based submap joining algorithm, addressing challenges in large-scale environments and long-term trajectories. Results from both simulated and real-world datasets demonstrate that our method achieves more accurate pose and map estimates than state-of-the-art approaches. 

   
Our findings show that solving poses and occupancy maps simultaneously yields more accurate results compared to first solving pose-graph SLAM and then constructing the map. This joint optimization approach has the potential to revolutionize occupancy map based SLAM frameworks.

The proposed method acts as a batch optimization approach for obtaining high-quality robot poses and maps. Unlike incremental or online methods, batch optimization provides greater accuracy, which is particularly advantageous for applications requiring high-quality maps rather than real-time operation (e.g., offline map creation for precise future localization). Despite typical drawbacks of batch optimization, such as higher computational costs, trajectory-length-dependent complexity, and reliance on accurate initial guesses, our method effectively overcomes these limitations: 1) our method is efficient due to the proposed multi-resolution joint optimization strategy, and the computation time is comparable to online methods; 2) our method can use selected keyframes to further reduce the computational cost without losing too much accuracy; 3) our proposed occupancy submap joining approach can overcome the limitation that the computational complexity related to the length of the robot trajectories; and 4) our method is very robust to the initial guess and can be initialized from odometry inputs or scan matching, so it does not require initialization from the result of incremental/online methods.    

In our future work, we will further explore problem formulation and solution techniques in the 3D case to develop more efficient and robust algorithms capable of addressing various challenges in 3D environments. 


%\section*{Acknowledgments}
%This should be a simple paragraph before the References to thank those individuals and institutions who have supported your work on this article.


{\appendix


The Jacobian $\mathbf{J}$ in (\ref{Gauss-Newton}) consists of four parts, i.e. the Jacobian of the observation term w.r.t. the robot poses $\mathbf{J}_P$ (See Appendix \ref{Sec_J_P}), the Jacobian of the observation term w.r.t. the occupancy map $\mathbf{J}_M$ (See Appendix \ref{Sec_J_D}), the Jacobian of the odometry term w.r.t. robot poses $\mathbf{J}_O$ (See Appendix \ref{Sec_J_O}) and the Jacobian of the smoothing term w.r.t. the occupancy map $\mathbf{J}_S$ (See Appendix \ref{Sec_J_S}). In addition, the difference in the calculation of Jacobians between Algorithm \ref{alg_1} and Algorithm \ref{alg_3} is shown in Appendix \ref{Sec_J_Select}. 

\subsection{Jacobian of the Observation Term w.r.t. Robot Poses}\label{Sec_J_P}

The Jacobian $\mathbf{J}_P$ of function $F_{ij}^Z(\mathbf{x})$ in the observation term w.r.t. the robot poses $\mathbf{x}^P_i$ can be calculated by the chain rule
\begin{equation}
	\begin{aligned}
		\mathbf{J}_P=\frac{ \partial F_{ij}^Z(\mathbf{x}) }{ \partial \mathbf{x}^P_i } = \frac{\partial F_{ij}^Z(\mathbf{x}) }{ \partial \mathbf{p}_{ij} } \cdot \frac{\partial \mathbf{p}_{ij}  }{ \partial \mathbf{x}^P_i}	
	\end{aligned}
\end{equation}
in which $\dfrac{\partial \mathbf{p}_{ij}  }{ \partial \mathbf{x}^P_i}$ can be calculated as
\begin{equation}
\dfrac{\partial \mathbf{p}_{ij}}{\partial \mathbf{x}^P_i}=\left[\begin{array}{ll}
\dfrac{\partial \mathbf{p}_{ij}}{\partial \mathbf{t}_i} & \dfrac{\partial \mathbf{p}_{ij}}{\partial \theta_i}
\end{array}\right]=\dfrac{1}{s} \left[\begin{array}{ll}
\mathbf{E}_{2} & \left(\mathbf{R}_i^{\prime}\right)^{\top} \mathbf{p}_{ij}
\end{array}\right].
\end{equation}
$\mathbf{R}_i^\prime$ is the derivative of the rotation matrix $\mathbf{R}_i$ w.r.t. rotation angle $\theta_i$ and $\mathbf{E}_2$ means $2 \times 2$ identity matrix.

$\dfrac{\partial F_{ij}^Z(\mathbf{x}) }{ \partial \mathbf{p}_{ij} }$ can be calculated by
\begin{equation}
\dfrac{\partial F_{ij}^Z(\mathbf{x}) }{ \partial \mathbf{p}_{ij} } = \dfrac{1}{N(\mathbf{p}_{ij})} \dfrac{\partial M(\mathbf{p}_{ij})}{\partial \mathbf{p}_{ij}}.
\end{equation}
Here $\dfrac{\partial M(\mathbf{p}_{ij})}{\partial \mathbf{p}_{ij}}$ can be considered as the gradient of the occupancy map at point $\mathbf{p}_{ij}$, which can be approximated by the bilinear interpolation of the gradients of the occupancy at the four adjacent cell vertices $\mathbf{\nabla} M(\mathbf{m}_{wh}),\cdots,\mathbf{\nabla} M(\mathbf{m}_{(w+1)(h+1)})$ around $\mathbf{p}_{ij}$ as
\begin{equation} 
\dfrac{\partial M(\mathbf{p}_{ij})}{\partial \mathbf{p}_{ij}}= 
\left[
\begin{aligned}
a_1b_1\\a_0b_1\\a_1b_0\\a_0b_0\\
\end{aligned}\right]^\top
\left[
\begin{aligned}
&\mathbf{\nabla} M(\mathbf{m}_{wh})\\&\mathbf{\nabla} M(\mathbf{m}_{(w+1)h})\\&\mathbf{\nabla} M(\mathbf{m}_{w(h+1)})\\&\mathbf{\nabla} M(\mathbf{m}_{(w+1)(h+1)})
\end{aligned}\right]\label{eq_14}
\end{equation} 
where the gradient of occupancy map $\mathbb{M}$ at all the cell vertices $\mathbf{\nabla} M$ can be easily calculated from $\mathbf{x}^M$ in the state. The bilinear interpolation used in (\ref{eq_14}) is similar to the method in (\ref{eq_interp}).

Here, we assume the robot poses $\mathbf{x}^P$ change slightly in each iteration, to reduce the computational complexity, the hit map $\mathbb{N}$ is considered as constant and recalculated using the current robot poses in each iteration. Thus, the derivative of $N(\mathbf{p}_{ij})$ is not calculated. 

\subsection{Jacobian of the Observation Term w.r.t. Occupancy Map}\label{Sec_J_D}
Based on (\ref{eq_interp}), the Jacobian $\mathbf{J}_M$ of function $F_{ij}^Z(\mathbf{x})$ in the observation term w.r.t. the map part of state vector $\mathbf{x}^{M}$ can be calculated as

\begin{equation}
\begin{aligned}
\mathbf{J}_M & = \dfrac{\partial F_{ij}^Z(\mathbf{x})}{\partial \left[ {M}(\mathbf{m}_{wh}),\cdots, {M}(\mathbf{m}_{(w+1)(h+1)}) \right]^\top}\\
&= \dfrac{1}{N(\mathbf{p}_{ij})}\dfrac{\partial M(\mathbf{p}_{ij})}{\partial \left[ {M}(\mathbf{m}_{wh}),\cdots, {M}(\mathbf{m}_{(w+1)(h+1)}) \right]^\top}\\ 
&= \dfrac{\begin{bmatrix}
a_1b_1,a_0b_1,a_1b_0,a_0b_0
\end{bmatrix}}{N(\mathbf{p}_{ij})}
\end{aligned}
\end{equation}
where $\mathbf{m}_{wh}, \cdots, \mathbf{m}_{(w+1)(h+1)}$ are the four nearest cell vertices to $\mathbf{p}_{ij}$ in occupancy map $\mathbb{M}$, and $a_0,a_1,b_0$ and $b_1$ are defined in (\ref{eq_interp}).


\subsection{Jacobian of the Odometry Term}\label{Sec_J_O}
The Jacobian $\mathbf{J}_O$ of function $F_i^O(\mathbf{x})$ in the odometry term (\ref{eq_odometry_term}) is the partial derivative w.r.t. the robot poses $\mathbf{x}^P$ since it is not related to the occupancy map in the state vector $\mathbf{x}$. Therefore, the Jacobian $\mathbf{J}_O$ can be calculated as
\begin{equation}
\begin{aligned}
\mathbf{J}_O &= \frac{\partial F_i^O(\mathbf{x})}{\partial \left[ {\mathbf{x}^P_{i-1}}^\top, {\mathbf{x}^P_i}^\top \right]^\top }\\ 
&=\begin{bmatrix}
	 \dfrac{\partial F_i^O(\mathbf{x})}{\partial \mathbf{t}_{i-1}} &
	 \dfrac{\partial F_i^O(\mathbf{x})}{\partial \theta_{i-1}} &
	 \dfrac{\partial F_i^O(\mathbf{x})}{\partial \mathbf{t}_i} &
	 \dfrac{\partial F_i^O(\mathbf{x})}{\partial \theta_i}
 \end{bmatrix} 
 \\
 &=\begin{bmatrix}
 	-\mathbf{R}_{i-1} & \mathbf{R}_{i-1}^\prime(\mathbf{t}_i-\mathbf{t}_{i-1}) & \mathbf{R}_{i-1} &\mathbf{0}_2\\
 	\mathbf{0}_2^\top & -1 & \mathbf{0}_2^\top & 1\\
 \end{bmatrix}
\end{aligned}
 \end{equation}
in which $\mathbf{0}_2$ means $2 \times 1$ zero vector.
 
\subsection{Jacobian of the Smoothing Term}\label{Sec_J_S}

The Jacobian $\mathbf{J}_S$ of function $F^S(\mathbf{x})$ in the smoothing term is the derivative of (\ref{eq_smoothing_term}) w.r.t. cell vertices of occupancy map $\mathbf{x}^M$ 
due to it is not related to the robot poses $\mathbf{x}^P$ in the state vector $\mathbf{x}$. It should be mentioned that $F^S(\mathbf{x})$ is linear w.r.t. $\mathbf{x}^M$
\begin{equation}
F^S(\mathbf{x}) = \mathbf{A} \left[ {M}(\mathbf{m}_{00}),\cdots,{M}(\mathbf{m}_{c_wc_h}) \right]^\top
\end{equation}
where the $(2c_wc_h+c_w+c_h) \times ((c_w+1)(c_h+1))$ coefficient matrix $\mathbf{A}$ is sparse and with nonzero elements $1$ or $-1$. An example of the coefficient matrix can be shown as
\begin{equation}\label{eq_A}
	\mathbf{A} = \begin{bmatrix}
    \vdots &\vdots  &\vdots  &\vdots  &\vdots  &\vdots  &\vdots &\vdots\\
 	\mathbf{0}^\top & 1 & -1 & 0 & \mathbf{0}^\top & 0 & 0 & \mathbf{0}^\top\\
 	\mathbf{0}^\top & 1 & 0  & 0 & \mathbf{0}^\top & -1 & 0 & \mathbf{0}^\top\\
 	\mathbf{0}^\top & 0 & 1 & -1 & \mathbf{0}^\top & 0 & 0 & \mathbf{0}^\top\\
 	\mathbf{0}^\top & 0 & 1 & 0 & \mathbf{0}^\top & 0 & -1 & \mathbf{0}^\top\\
    \vdots &\vdots  &\vdots  &\vdots  &\vdots  &\vdots  &\vdots &\vdots\\
 \end{bmatrix}.
\end{equation}
Here $\mathbf{0}$ represents a zero vector with appropriate dimensions. Therefore, the Jacobian of the smoothing term can be calculated as
\begin{equation}\label{eq_JS}
\mathbf{J}_S = \frac{\partial F^S(\mathbf{x})}{\partial \mathbf{x}^M } = \mathbf{A}.\\ 
\end{equation}
Since $\mathbf{A}$ is constant, $\mathbf{J}_S$ can be pre-calculated and directly used in the optimization as shown in Algorithm \ref{alg_1}.

\subsection{Jacobians in the Second Stage of Multi-resolution Strategy for Optimization}\label{Sec_J_Select}
In the second stage of the multi-resolution strategy (Algorithm \ref{alg_3}), the Jacobians to be calculated are similar to those in Algorithm \ref{alg_1}. A specific challenge arises in handling the selected cell vertices adjacent to the dropped cell vertices in the high-resolution map, particularly when calculating Jacobians $\mathbf{J}_P$ and $\mathbf{J}_S$.

 For Jacobian $\mathbf{J}_P$, partial derivatives w.r.t. all the cell vertices are required for (\ref{eq_14}). However, not all vertices are included in the state vector in the second stage, which makes it challenging to compute the partial derivatives w.r.t. some cell vertices because their surrounding nodes are discarded. From a semantic perspective, the discarded cell vertices have the same occupancy state as the edge nodes, which is why they are excluded. Consequently, the gradient of these edge vertices is expected to be close to zero. Based on this reasoning, we set the partial derivatives w.r.t. all edge cell vertices to $0$ when they need to be calculated using (\ref{eq_14}). 
 
 For Jacobian $\mathbf{J}_S$, it can also be calculated using the same idea as (\ref{eq_JS}). In the second stage of our multi-resolution strategy, (\ref{eq_JS}) is reformulated as 
 \begin{equation}
 	\mathbf{J}_S = \frac{\partial F^S_{s}(\mathbf{x}^s)}{\partial \mathbf{x}^{sM} } = \mathbf{A}^{s}\\ 
 \end{equation}
 where $\partial F^S_{s}(\mathbf{x}^s)$ is similar to (\ref{eq_smoothing_term}), but only applies to vertices in $\mathbb{M}^s$. The coefficient matrix $\mathbf{A}^{s}$ has the same form as (\ref{eq_A}) but with dimension corresponding to the number of elements in $\mathbf{x}^{sM}$.  
 
 }   





\bibliographystyle{IEEEtran}
\documentclass[lettersize,journal]{IEEEtran}
\usepackage{amsmath,amsfonts}
% \usepackage{algorithm}
% \usepackage{algorithmic}
% \usepackage{algpseudocode}

\usepackage[linesnumbered,ruled,vlined]{algorithm2e}


\usepackage{array}
\usepackage{accents}
%\usepackage[caption=false,font=normalsize,labelfont=sf,textfont=sf]{subfig}
%\usepackage[caption=false,font=footnotesize,labelfont=sf,textfont=sf]{subfig}
\usepackage{subfigure}

\usepackage{textcomp}
\usepackage{stfloats}
\usepackage{url}
\usepackage{verbatim}
\usepackage{graphicx}
\usepackage{cite}
\usepackage{float}
 \usepackage{tabularx} 
% \usepackage{setspace}

\hyphenation{op-tical net-works semi-conduc-tor IEEE-Xplore}
\def\BibTeX{{\rm B\kern-.05em{\sc i\kern-.025em b}\kern-.08em
    T\kern-.1667em\lower.7ex\hbox{E}\kern-.125emX}}
\usepackage{balance}
% updated with editorial comments 8/9/2021

% use bib--added by yingyu
% \usepackage[numbers]{natbib}
\usepackage{amssymb}
\usepackage{booktabs}
\usepackage{threeparttable}
\usepackage{multirow}
\usepackage{bigstrut}
\usepackage{bigdelim}
\usepackage{adjustbox}
\usepackage{makecell}
\usepackage{nicematrix}
\usepackage{xcolor}

% \usepackage{xcolor} 
\usepackage{tikz} 
\usetikzlibrary{arrows,shapes,chains}

\usepackage{colortbl}
% \usepackage[table]{xcolor}
\usepackage{rotating}
% Beamer presentation requires \usepackage{colortbl} instead of \usepackage[table,xcdraw]{xcolor}


% \renewcommand{\algorithmicrequire}{\textbf{Input:}}
% \renewcommand{\algorithmicensure}{\textbf{Output:}}
\setlength{\textfloatsep}{8pt}
% \usepackage{caption}
% \captionsetup[figure]{skip=5pt}

% \setlength{\topsep}{0pt}
% \setlength{\belowdisplayskip}{10pt}
% \setlength{\abovedisplayskip}{10pt}

\newcommand*{\defeq}{\stackrel{\text{def}}{=}}
\DeclareMathOperator{\wrap}{wrap}
\DeclareMathOperator{\diag}{diag}
\DeclareMathOperator{\round}{round}




\begin{document}

\title{Occupancy-SLAM: An Efficient and Robust Algorithm for Simultaneously Optimizing Robot Poses and Occupancy Map}

\author{\authorblockN{Yingyu Wang, Liang Zhao, and Shoudong Huang} 
        % <-this % stops a space
\thanks{Yingyu Wang and Shoudong Huang are with the Robotics Institute, University of Technology Sydney, Australia (e-mail: Yingyu.Wang-1@student.uts.edu.au; Shoudong.Huang@uts.edu.au).} 

\thanks{Liang Zhao was with the Robotics Institute, University of Technology Sydney, Australia, and is now with the School of Informatics, University of Edinburgh, Edinburgh, UK (e-mail: Liang.Zhao@ed.ac.uk).}}



% The paper headers
\markboth{IEEE TRANSACTIONS ON ROBOTICS}%
{Shell \MakeLowercase{\textit{et al.}}: A Sample Article Using IEEEtran.cls for IEEE Journals}

% \IEEEpubid{0000--0000/00\$00.00~\copyright~2021 IEEE}
% Remember, if you use this you must call \IEEEpubidadjcol in the second
% column for its text to clear the IEEEpubid mark.

\maketitle

% This paper proposes Occupancy-SLAM, an optimization-based SLAM approach that jointly optimizes the robot trajectory and the occupancy map simultaneously.

\begin{abstract}
Joint optimization of poses and features has been extensively studied and demonstrated to yield more accurate results in feature-based SLAM problems. However, research on jointly optimizing poses and non-feature-based maps remains limited. Occupancy maps are widely used non-feature-based environment representations because they effectively classify spaces into obstacles, free areas, and unknown regions, providing robots with spatial information for various tasks. In this paper, we propose Occupancy-SLAM, a novel optimization-based SLAM method that enables the joint optimization of robot trajectory and the occupancy map through a parameterized map representation. The key novelty lies in optimizing both robot poses and occupancy values at different cell vertices simultaneously, a significant departure from existing methods where the robot poses need to be optimized first before the map can be estimated. 

This paper focuses on 2D laser-based SLAM to investigate how to jointly optimize robot poses and the occupancy map. In our formulation, the state variables in optimization include all the robot poses and the occupancy values at discrete cell vertices in the occupancy map. Moreover, a multi-resolution optimization framework that utilizes occupancy maps with varying resolutions in different stages is introduced. A variation of Gauss-Newton method is proposed to solve the optimization problem at different stages to obtain the optimized occupancy map and robot trajectory. The proposed algorithm is efficient and converges easily with initialization from either odometry inputs or scan matching, even when only limited key-frame scans are used. Furthermore, we propose an occupancy submap joining method, enabling more effective handling of large-scale problems by incorporating the submap joining process into the Occupancy-SLAM framework. Evaluations using simulations and practical 2D laser datasets demonstrate that the proposed approach can robustly obtain more accurate robot trajectories and occupancy maps than state-of-the-art techniques with comparable computational time. Preliminary results in the 3D case further confirm the potential of the proposed method in practical 3D applications, achieving more accurate results than existing methods. The code is made available to benefit the robotics community\footnote{\url{https://github.com/WANGYINGYU/Occupancy-SLAM}}. 
\end{abstract}

\begin{IEEEkeywords}
SLAM, optimization, occupancy grid map, non-feature-based map representation.
\end{IEEEkeywords}

\section{Introduction}

% optimizing robot poses and features simultaneously 


\IEEEPARstart{S}{imultaneous} localization and mapping (SLAM) is an important problem in robotics that has been studied for decades \cite{cadena2016past}. Jointly optimizing the robot poses and map can enhance SLAM performance, as this formulation utilizes the available information more directly without approximations. While joint optimization has been widely explored in feature-based SLAM (e.g., \cite{kaess2008isam,kaess2011isam2}), research on the joint optimization of robot poses and non-feature-based maps remains limited.

Occupancy grid maps are widely used in robotic tasks for their ability to clearly represent obstacles, free space, and unknown areas, facilitating collision-free navigation and path planning. Assuming the robot poses used to collect the sensor information are known exactly, the evidence grid mapping technique \cite{moravec1985high,
moravec1989sensor,elfes1989occupancy,martin1996robot,hornung2013octomap} provides an elegant and efficient approach for building occupancy grid maps from the collected information. However, when a robot is navigating in an unknown environment and performing SLAM, its own poses need to be estimated, and the estimates inherently contain uncertainties. Achieving both accurate robot localization and precise occupancy mapping simultaneously is not trivial.

% How to perform accurate robot localization and build an occupancy map very accurately at the same time is not trivial. 


In some occupancy grid map based SLAM approaches such as Cartographer \cite{hess2016real}, the problem is tackled in two steps. First, the robot poses are estimated by solving a pose-graph SLAM problem, where the relative pose measurements are derived using odometry, scan matching, loop closure detection, or other similar techniques. Second, the optimized poses are assumed to be the correct poses and are used to build up the map using evidence grid mapping techniques. However, in these two-step approaches, the uncertainties in the robot poses obtained during the first step are not considered when building the map. Therefore, it is crucial to achieve highly accurate pose estimates to construct a reliable occupancy grid map. As a result, it can be expected that the occupancy map obtained using a two-step approach may not represent the best result that one can achieve using all the available information.


In feature-based SLAM approaches, jointly optimizing the poses and the feature map is common, as the relationship between observations and the map is straightforward to model. However, for occupancy map based SLAM, jointly optimizing the robot poses and the occupancy map is not trivial because: 
\begin{itemize}
	\item [1)] \textbf{The relation between the observations and the map is complex.} The observations are laser beams (the endpoint of a beam represents ``hit" and the other positions along the beam represent ``free"), and the map is a function representing the occupancy values at different positions. This is significantly different from feature-based SLAM where both the observations and the map are about feature parameters such as positions.
	\item [2)] \textbf{The data association is not easy to do.} When the robot poses are noisy, the correct correspondence between laser beams and occupancy grid cells is hard to find. In contrast, for feature-based SLAM, there are well-established front-end methods for data association.
	\item [3)] \textbf{The resolution of the map has a significant impact on the optimization problem.} A high-resolution map helps to establish a more accurate connection between the observations and the map, but it leads to a sharp increase in the number of variables. However, for feature-based SLAM, there is no such issue.
\end{itemize}

% using 2D laser scans (and odometry) information.

\subsection{Contributions}
In this paper, we propose Occupancy-SLAM algorithm, which jointly optimizes the robot poses and the occupancy map using 2D laser scans (and odometry) information. Moreover, we propose a multi-resolution optimization framework for improving convergence and robustness to initial guesses. To better handle the case of large-scale environments and long-term trajectories, we further propose an occupancy submap joining method. Experiments on both simulated and practical datasets verify the superior performance of our method compared with state-of-the-art approaches (e.g., Cartographer \cite{hess2016real}). In addition, we extend our method to the 3D case, and preliminary results confirm its effectiveness in improving accuracy. The main contributions are summarized as follows: 

 % A smoothing term is introduced in the objective function to improve the convergence of the method.

\begin{enumerate}
	\item We formulate the occupancy grid map based SLAM problem as a joint optimization problem where the poses and the occupancy map are optimized together. 
	\item We propose a variation of Gauss-Newton method to solve the new formulation, enabling the estimation of more accurate robot poses and occupancy maps compared to existing state-of-the-art techniques.
	\item To enhance efficiency, convergence, and robustness, we propose a multi-resolution optimization strategy using occupancy maps of different resolutions across stages.
    
    % To improve the efficiency, convergence and robustness of our algorithm so that it can be initialized by odometry inputs or scan matching, we propose a multi-resolution optimization strategy that uses occupancy maps with different resolutions at different optimization stages. In the second stage, we utilize the selected high-resolution map, focusing exclusively on a subset of \textcolor{red}{cell vertices} that require further updates within the full high-resolution map. This targeted approach further enhances computational efficiency.
    
    \item We propose a submap joining algorithm to address the cases of large-scale environments and long-term trajectories through our joint poses and occupancy map optimization idea.
	\item Our method achieves robust convergence even with key frames of limited overlap, outperforming state-of-the-art approaches like Cartographer in efficiency while maintaining superior accuracy.
    % Our proposed method can converge well even when only key frames with limited overlaps are used. In this case, our method outperforms state-of-the-art methods, such as Cartographer, in terms of efficiency while maintaining a surpassing performance in terms of accuracy.
    \item We extend our method to 3D, with preliminary results demonstrating superior accuracy compared to other state-of-the-art approaches.
    % We extend our method to the 3D case, and preliminary results confirm that the accuracy of our method outperforms other state-of-the-art approaches.
\end{enumerate}


This paper is an extension to our conference paper \cite{Zhao-RSS-22}, with major improvements in contributions 3, 4, 5, and 6, significantly enhancing the robustness and efficiency of the algorithm while extending the method to 3D.

% The major improvements of this paper over \cite{Zhao-RSS-22} are contributions 3, 4, 5, and 6, which significantly improve the robustness and efficiency of the algorithm, and extend the algorithm to 3D.

\begin{figure}
\centering
\includegraphics[width=0.49\textwidth]{./OverView.pdf}
\caption{\label{fig_overview} Main components of our proposed method. The blue-colored components represent our core approaches, while the dashed portions are optional. Our multi-resolution joint optimization is covered in Section \ref{sec_formulation}, Section \ref{Sec_Algorithm_1}, and Section \ref{Sec_multi}. The joint global map and robot trajectory optimization approach is presented in Section \ref{Sec_submap}. }
\end{figure}

% \subsection{Notations}
% Some important notations in this paper are summarized in Table \ref{tab_notation}, the others are described in the context.

\subsection{Outline}
Fig. \ref{fig_overview} illustrates the flowchart of applying our proposed methods in practice. The blue components represent our core approaches, while the dashed portions are optional. The rest of the paper is organized as follows: Section \ref{Sec_related_work} provides a review of related work on non-feature-based SLAM, submap joining, and joint optimization of poses and maps. In Section \ref{sec_formulation}, we introduce our novel formulation for jointly optimizing the robot poses and occupancy map. A variation of the Gauss-Newton method to solve our nonlinear least squares (NLLS) formulation is presented in Section \ref{Sec_Algorithm_1}. In Section \ref{Sec_multi}, we introduce our multi-resolution strategy to improve the efficiency and robustness of the algorithm. Section \ref{Sec_submap} presents our submap joining algorithm for handling large-scale environments and long-term trajectories. Experimental results are provided in Section \ref{Sec_experiment}. In Section \ref{sec_3d}, we extend our method to the 3D case and present preliminary results. Finally, the conclusions are given in Section \ref{Sec_conclusion}.

 
\section{Related Work}\label{Sec_related_work}

In this section, we discuss some related work on non-feature based map representations for SLAM, submap joining techniques, and joint optimization of poses and maps. 

\subsection{Non-feature based map representations for SLAM}\label{Sec_related_a}
One widely used non-feature based SLAM approach is occupancy grid map-based SLAM, which probabilistically classifies spaces into obstacles, free areas, and unknown regions while accounting for uncertainty during observation updates \cite{moravec1985high, moravec1989sensor, elfes1989occupancy, martin1996robot, hornung2013octomap}. Classic examples, such as FastSLAM \cite{montemerlo2002fastslam} and GMapping \cite{grisetti2005improving}, use particle filters for mapping and localization but struggle with high computational demand and long-term accuracy in large-scale environments.

Recent optimization-based approaches, such as Hector SLAM \cite{kohlbrecher2011flexible}, Karto-SLAM \cite{konolige2010efficient}, and Cartographer \cite{hess2016real}, address cumulative errors effectively. Hector SLAM uses scan-to-map matching but lacks loop closure, restricting it to small-scale scenarios. Karto-SLAM incorporates loop closure detection with sparse pose adjustment for global optimization, while Cartographer integrates scan-to-map matching and pose graph optimization with a branch-and-bound strategy for efficient loop closure detection. However, by treating pose optimization and map construction as independent processes, these methods fail to account for the interdependencies of their uncertainties.

Multi-resolution occupancy mapping techniques can be integrated into occupancy grid map based SLAM frameworks to enable a more compact and efficient mapping process. For instance, approaches like OctoMap \cite{hornung2013octomap} use memory-efficient octrees to balance map compactness and accessibility. Adaptive-resolution methods, such as RMAP \cite{khan2014rmap} and ColMap \cite{fisher2021colmap}, dynamically adjust grid resolution to enhance mapping efficiency. Recently, \cite{Reijgwart-RSS-23} applies wavelet compression for hierarchical occupancy map storage, allowing efficient updates and queries. However, integrating multi-resolution maps as state variables into a unified framework for joint poses and map optimization remains an open challenge.


Another widely used non-feature-based map is the signed-distance function (SDF), which discretizes the environment into grid cells storing the distance to the nearest surface. This representation encodes the space, with the object surfaces defined by the zero crossings of the distance functions \cite{curless1996volumetric}. Some SLAM systems adopt SDF to improve localization accuracy and mapping quality. For example, supereight \cite{vespa2018efficient} integrates tracking, mapping, and planning using an octree-based truncated SDF (TSDF). It aligns camera frames to the TSDF map with iterative closest point (ICP) \cite{besl1992method}. A follow-up work \cite{vespa2019adaptive} improves this by introducing adaptive-resolution octree structures, achieving denser environment representation and reduced noise, leading to more accurate localization.

Other non-feature based map representations have also been used in SLAM, including mesh-based \cite{rosinol2021kimera}, normal distributions transform based \cite{einhorn2015generic}, neural radiance fields based \cite{rosinol2023nerf} and Gaussian splatting based \cite{matsuki2024gaussian}. Although these approaches differ in the type of non-feature representations they use, they all aim to provide more effective environmental modeling, improve robot localization accuracy, or achieve both. 


However, all the optimization-based SLAM approaches that utilize non-feature based maps need to optimize the poses first and then build the non-feature based map using the optimized poses. This separation prevents these approaches from jointly considering the uncertainties in both localization and mapping during the optimization process. In contrast, this paper considers unifying the optimization of both the robot poses and occupancy values at each cell vertex of the occupancy map into a single optimization problem, which can be expected to yield better accuracy.

\subsection{Submap Joining}\label{Sec_related_b}

Submap joining, as proposed by \cite{bosse2003atlas}, is a widely used scheme for SLAM in large-scale environments due to its efficiency and reduced risk of being trapped in local minima compared to full optimization-based SLAM. Feature-based submap joining approaches \cite{huang2008sparse,zhao2013linear,wang2019submap} have been well investigated, with many demonstrating properties that enable efficient problem-solving while maintaining a high level of accuracy. To extend non-feature-based SLAM to large-scale environments and long-term operations, recent efforts have explored non-feature-based submap joining methods.


For example, \cite{wagner2014graph} divides the environment into overlapping submaps composed of small TSDF grids from KinectFusion \cite{izadi2011kinectfusion}. Submap joining is then formulated as a pose graph optimization problem, where submap poses are nodes, and relative transformations from ICP are edges. Similarly, VOG-map \cite{ho2018virtual} represents submaps as 3D occupancy grids, converts them to point clouds for ICP-based relative transformations, and solves submap joining via pose graph optimization. Voxgraph \cite{reijgwart2019voxgraph} improves accuracy by employing SDF-to-SDF registration for overlapping submaps created with C-blox \cite{millane2018c}. Unlike time-sequence-based submap partitioning, \cite{wang2021elastic} uses spatial partitioning, merging submaps during loop closures by solving a pose graph containing only submap frames, with reconstruction decisions based on environmental changes.

All the aforementioned non-feature-based submap joining approaches estimate relative measurements between overlapping submaps to formulate and solve the pose graph problem for submap frames. In contrast, this paper jointly optimizes submap frames and the global occupancy map.

\subsection{Joint Optimization of Poses and Maps}
Joint optimization of poses and maps can result in better accuracy, as it utilizes the information more directly. In feature-based SLAM and bundle adjustment approaches, the most common form is to jointly optimize poses and positions of features, such as \cite{dellaert2006square,triggs2000bundle,konolige2008frameslam,sibley2009adaptive,zhao2015parallaxba}. Some approaches extend this idea to planar feature parameters. For instance, \cite{kaess2015simultaneous,hsiao2017keyframe} minimize the difference between plane measured in a scan and predicted planes, while \cite{trevor2012planar,geneva2018lips,zhou2021pi,zhou2021lidar} minimize the Euclidean distance between points in a scan and the predicted planes. Based on the idea of minimizing Euclidean distance between points in scans, BALM \cite{liu2021balm} demonstrates that planar parameters can be solved analytically in closed form, reducing the dimensionality of the optimization. BALM2 \cite{liu2023efficient} further improves efficiency by using point clusters, avoiding individual point enumeration. HBA \cite{liu2023large} introduces a hierarchical structure to address the scalability challenges of BALM and BALM2 in large environments. In summary, jointly optimizing poses and feature-based maps is well-studied, as features naturally link positions, observations, and poses, making them straightforward to integrate into optimization problems. In contrast, establishing constraints between observations, poses, and non-feature-based maps (e.g., occupancy grid maps) for joint optimization remains a significant challenge.



% Optimizing the poses and feature-based map together is very common and has been well-studied, as features are naturally linked to positions, which in turn connect observations, poses, and features, making them straightforward to integrate into optimization problems. However, it is a challenge to establish constraints between observations, poses, and a non-feature based map to jointly optimize the poses and the map (e.g., an occupancy grid map).

% Kimera-PGMO proposed in \cite{rosinol2021kimera} is a novel approach that simultaneously optimizes the poses and the mesh deformation. Specifically, Kimera-PGMO \cite{rosinol2021kimera} creates a deformation graph including a simplified mesh and a pose graph of robot poses. Since the simplified mesh consists of the positions of the mesh vertices, the method is formulated as a factor graph and then solved by GTSAM \cite{dellaert2012factor}.

Research on jointly optimizing the poses and non-feature based maps is limited. Kimera-PGMO proposed in \cite{rosinol2021kimera} represents a notable attempt, integrating pose optimization with mesh deformation. It constructs a deformation graph of a simplified mesh and a pose graph, formulating the problem as a factor graph solvable by GTSAM \cite{dellaert2012factor}. 
While Kimera-PGMO \cite{rosinol2021kimera} has similar motivations as our paper, aiming to achieve better quality maps and more accurate poses through joint optimization, its mesh-based representation differs fundamentally from the occupancy grid maps used in our approach. Meshes are naturally represented through point positions and their relationships, which facilitates factor graph formulations.


% but the mesh can still be described in terms of the positions of the points as well as the relationships between the points, and can therefore ultimately be transformed into a factor graph to be solved for, which is different to the occupancy map that we used.

\begin{table}[t]
 		\centering
 		\caption{Summary of Some Important Notations.}\label{tab_notation}
 		\setlength{\tabcolsep}{0.5 mm}{
 		\begin{tabular}{|c|l|p{3cm}p{3cm}p{3cm}}
   \hline
   \multicolumn{1}{|c|}{Notation} & \multicolumn{1}{|c|}{Explanation} \\ \hline
   $\mathbb{M}$  & \begin{tabular}[c]{@{}l@{}} A set includes occupancy values at all discrete cell vertices in \\occupancy map, as defined in Section \ref{sec_discrete_occupancy}. $\mathbb{M}^{l}$, $\mathbb{M}^{h}$, and $\mathbb{M}^{s}$ \\represent the sets include occupancy values at all cell vertices \\in low-resolution map, high-resolution map and selected \\high-resolution map, respectively. In addition, $\mathbb{M}_L$ and $\mathbb{M}_G$ \\represent the sets including occupancy values of all cell vertices \\in local maps and the global map, as defined in Section \ref{Sec_submap}.\end{tabular}\\ \hline

% as defined in \\Section \ref{continuous_map}
   $M(\cdot)$ &\begin{tabular}[c]{@{}l@{}}A function to lookup occupancy value at an arbitrary position in \\the occupancy map by bilinear interpolation using $\mathbb{M}$.\end{tabular} \\ \hline

    $\mathbf{x}^M$ & \begin{tabular}[c]{@{}l@{}} A vector including occupancy values at all cell vertices in discrete \\occupancy map $\mathbb{M}$, as defined in (\ref{eq_map_state}). $\mathbf{x}^{lM}$ and $\mathbf{x}^{sM}$ are vectors \\including occupancy values at cell vertices in $\mathbb{M}^{l}$ and $\mathbb{M}^{s}$. In \\addition, $\mathbf{x}^M_G$ represents the vector which includes occupancy \\values at cell vertices from $M_G$, as described in Section \ref{Sec_submap}.\end{tabular} \\ \hline

    $N(\cdot)$ & \begin{tabular}[c]{@{}l@{}} A function to lookup hit number at arbitrary position in the map \\by bilinear interpolation using hit map $\mathbb{N}$, where $\mathbb{N}$ is defined as a \\set includes hit number at all discrete cell vertices in the map, as \\described in Section \ref{sec_hit}. $\mathbb{N}^{l}$ and $\mathbb{N}^{s}$ represent hit maps used in \\different optimization stages.\end{tabular}\\ \hline

    $\mathbf{x}^P$ & \begin{tabular}[c]{@{}l@{}} A vector including all robot poses for optimization, as defined \\in (\ref{eq_pose_state}). In addition, $\mathbf{x}^P_L$ denotes a vector including all local map \\coordinate frames for submap joining problem in Section \ref{Sec_submap}.\end{tabular} \\ \hline
   \rule{0pt}{1.5em}
    $\mathbf{x}$ & \begin{tabular}[c]{@{}l@{}} State vector of optimization, $\mathbf{x} = {[{\mathbf{x}^P}^\top, {\mathbf{x}^M}^\top]}^\top$. $\mathbf{x}^{l}$ and $\mathbf{x}^{s}$ \\represent state vectors of different optimization stages. \end{tabular} \\ \hline

    $\mathbb{S}$  &\begin{tabular}[c]{@{}l@{}} $\mathbb{S} = \{\mathbb{S}_i\}_{0 \leq i \leq n}$ where $\mathbb{S}_i$ is defined in (\ref{S_i}), a set including \\observations, as defined in Section \ref{Sec_Info_1}. $\mathbb{S}^{l}$, $\mathbb{S}^{h}$, and $\mathbb{S}^{s}$ are \\observations used for occupancy maps $\mathbb{M}^{l}$, $\mathbb{M}^{h}$, and $\mathbb{M}^{s}$, \\respectively. \end{tabular}\\ \hline

     $s$  &\begin{tabular}[c]{@{}l@{}} Resolution of the occupancy map, which indicates the distance\\ between two nearby cell vertices. $s^{l}$ and $s^{h}$ represent the \\resolutions of low-resolution map and high-resolution map, \\respectively. $s^L$ and $s^G$ denote the resolutions of local maps and \\the global map, respectively, as described in Section \ref{Sec_submap}. \end{tabular}\\ \hline
    
     $\mathbb{O}$ & \begin{tabular}[c]{@{}l@{}}Set including all odometry inputs, as defined in Section \ref{sec_odometry}. \end{tabular}\\ \hline

    $r$  & Ratio between resolutions of two stages, $r = {s^{l}}/{s^h}$.  \\ \hline

    $\mathbf{m}$ & \begin{tabular}[c]{@{}l@{}} Discrete coordinate of a cell vertex, detailed explanation in the \\second paragraph in Section \ref{sec_discrete_occupancy}.\end{tabular} \\ \hline

    $\mathbf{p}$ & \begin{tabular}[c]{@{}l@{}}Continuous coordinate of a point, see the second paragraph in \\Section \ref{sec_relationship}.\end{tabular} \\
        
 		\hline
 		\end{tabular}
 		}
        % \vspace{-1em}

 \end{table}


\section{Problem Formulation}\label{sec_formulation}
Our approach considers the joint optimization of the robot poses and the occupancy map using information from 2D laser observations (and odometry). In this section, we will explain how the observations from the laser can be linked to the robot poses and the occupancy map to formulate the NLLS problem. 

% We also explain how we improve the problem formulation to make it easier to solve by adding a smoothing term. 

\subsection{Notation}
Throughout this paper, unless otherwise noted, we use specific typographical conventions: typefaces denote sets, bold uppercase letters represent matrices, bold lowercase letters indicate vectors, and regular (unbolded) lowercase letters signify scalars. Key notations used in this paper are summarized in Table \ref{tab_notation}, while others are introduced within the text as needed.

\subsection{Occupancy Map Representation and State in Optimization} \label{sec_discrete_occupancy}
Suppose the environment is discretized into $c_w\times c_h$ grid cells. We use $\mathbf{m}_{wh}=[w,h]^\top~(0 \leq w \leq c_w, 0 \leq h \leq c_h)$ to represent the coordinate of a discrete cell vertex in the map. The occupancy value at the cell vertex $\mathbf{m}_{wh}$, denoted as $M(\mathbf{m}_{wh})$, is defined using evidence, which is the natural logarithm of odds (the ratio between the probability of being occupied and the probability of being free) \cite{martin1996robot,hornung2013octomap,ProbabilisticRobotics}. 
The occupancy values of all $(c_w+1) \times (c_h+1)$ cell vertices consist of the discrete occupancy map $\mathbb{M}=\{M(\mathbf{m}_{wh})\}_{0 \leq w \leq c_w, 0 \leq h \leq c_h}$.


To represent the entire environment using a finite number of parameters, we describe the occupancy value at an arbitrary position $\mathbf{p}_m=[x,y]^{\top}$ on the map using bilinear interpolation of the occupancy values at its four surrounding cell vertices: $\mathbf{m}_{wh}, \mathbf{m}_{({w+1})h}, \mathbf{m}_{w({h+1})}, \mathbf{m}_{({w+1})({h+1})}$, as shown in Fig. \ref{fig_interpolation}, i.e.,

\begin{equation}
	M(\mathbf{p}_{m})= \begin{bmatrix}
a_1b_1,a_0b_1,a_1b_0,a_0b_0
\end{bmatrix}\left[
\begin{aligned}\label{eq_interp}
&M(\mathbf{m}_{wh})\\&M(\mathbf{m}_{(w+1)h})\\&M(\mathbf{m}_{w(h+1)})\\&M(\mathbf{m}_{(w+1)(h+1)})
\end{aligned}\right] 
\end{equation}
in which 
\begin{equation}
\begin{aligned}
	a_0 &= x - w\\
	a_1 &= w+1 - x\\
	b_0 &= y - h\\
	b_1 &= h+1 - y .\\
\end{aligned} 
\end{equation}


\begin{figure}[t]
\centering
\includegraphics[width=0.48\textwidth]{interpolation.pdf}
\caption{\label{fig_interpolation} Parameterizing the entire map by bilinear interpolation of discrete map $\mathbb{M}$.}
% \vspace{-1em}
\end{figure}


Our method jointly optimizes robot poses and the occupancy map, combining them into the state vector of the proposed optimization problem. Using bilinear interpolation with the discrete occupancy map $\mathbb{M}$, estimating the entire map is equivalent to estimating $\mathbb{M}$. Thus, the map component of the state vector can be expressed as

\begin{equation}
    \mathbf{x}^M =\left[M(\mathbf{m}_{00}),\cdots,M(\mathbf{m}_{c_wc_h}) \right]^\top. \label{eq_map_state}
\end{equation}


We define the $n+1$ robot poses as \rule{0pt}{1em}$\{\mathbf{x}^P_i \triangleq [\mathbf{t}_i^\top,\theta_i]^\top\}_{0 \leq i \leq n}$, where $\mathbf{t}_i$ is the $x$-$y$ position of the robot and $\theta_i$ is the orientation with the corresponding rotation matrix $\mathbf{R}_i=\begin{bmatrix}
\cos(\theta_i), \sin(\theta_i)\\ -\sin(\theta_i), \cos(\theta_i)
\end{bmatrix}$. As in most of the SLAM problem formulations, we assume the first robot pose defines the coordinate system, $\mathbf{x}^P_0 \triangleq [0,0,0]^\top$, so only the other $n$ robot poses are variables that need to be estimated, thus the pose component of the state vector is represented as
\begin{equation}
    \mathbf{x}^P = \left[ (\mathbf{x}^P_1)^\top, \cdots, (\mathbf{x}^P_n)^\top \right]^\top.
\end{equation}


Accordingly, the state vector of the proposed optimization problem is
\begin{equation}
    \mathbf{x} = \left[{(\mathbf{x}^P)}^\top,{(\mathbf{x}^M)}^\top \right]^\top. \label{eq_pose_state}
\end{equation}

In our method, the occupancy map $\mathbb{M}$ is initialized by the Bayesian occupancy mapping method \cite{ProbabilisticRobotics} with initially estimated poses (derived from odometry or scan matching) and updated throughout the optimization process.


% i.e., 
% \begin{equation}
%     \mathbf{x} = \left[{(\mathbf{x}^P)}^\top,{(\mathbf{x}^M)}^\top \right]^\top,
% \end{equation}
% where $\mathbf{x}^M$ is the map part and $\mathbf{x}^P$ is the poses part. By bilinear interpolation method and discrete occupancy map $\mathbb{M}$, we only need to estimate $(c_w+1)\times(c_h+1)$ parameters to estimate the entire map, thus $\mathbf{x}^M$ can be expressed as 
% \begin{equation}
%     \mathbf{x}^M &= \left[M(\mathbf{m}_{00}),\cdots,M(\mathbf{m}_{c_wc_h}) \right]^\top.
% \end{equation}
% We suppose that the $n+1$ robot poses are expressed by \rule{0pt}{1em}$\{\mathbf{x}^P_i \triangleq [\mathbf{t}_i^\top,\theta_i]^\top\}_{0 \leq i \leq n}$, where $\mathbf{t}_i$ is the $x$-$y$ position of the robot and $\theta_i$ is the orientation with the corresponding rotation matrix $\mathbf{R}_i=\begin{bmatrix}
% \cos(\theta_i), \sin(\theta_i)\\ -\sin(\theta_i), \cos(\theta_i)
% \end{bmatrix}$. As in most of the SLAM problem formulations, we assume the first robot pose defines the coordinate system, $\mathbf{x}^P_0 \triangleq [0,0,0]^\top$, so only the other $n$ robot poses are variables that need to be estimated. Thus, the state variables of the part of the pose can be expressed as 
% \begin{equation}
%     \mathbf{x}^P = \left[ (\mathbf{x}^P_1)^\top, \cdots, (\mathbf{x}^P_n)^\top \right]^\top.
% \end{equation}

% the state in our optimization problem can be represented as 
% \begin{equation}
%     \mathbf{x} = \left[{(\mathbf{x}^P)}^\top,{(\mathbf{x}^M)}^\top \right]^\top,
% \end{equation}
% where
% \begin{equation}
% \begin{aligned}
% \mathbf{x}^P &= \left[ (\mathbf{x}^P_1)^\top, \cdots, (\mathbf{x}^P_n)^\top \right]^\top\\
% \mathbf{x}^M &= \left[M(\mathbf{m}_{00}),\cdots,M(\mathbf{m}_{c_wc_h}) \right]^\top.
% \end{aligned}\label{eq_state_vec}
% \end{equation}

% The occupancy map is updated as the optimization process progresses without the need for additional occupancy mapping update methods, and the Bayesian occupancy mapping method is used as map initialization in our optimization problem.


% , so there is no need for an additional mapping update strategy, and furthermore, this makes the method of initializing the occupancy value $M(\mathbf{m}_{wh})$ not very critical.




% \subsection{The Available Information}\label{Sec_Info}

% The available information includes 2D laser scans collected at different robot poses. In addition, the odometry information might also be available. 

\subsection {Scan Points Sampling Strategy}\label{Sec_Info_1} 

 % We use the evidence, which is the natural logarithm of odds (the ratio between the probability of being occupied and the probability of being free) \cite{martin1996robot,hornung2013octomap,ProbabilisticRobotics} to represent the occupancy value.

 \begin{figure}[tbp]
\centering 
\subfigure[Equidistant Sampling Strategy] {\label{fig_sampling_strategy}
\includegraphics[width=0.23\textwidth]{./sampling_strategy.pdf}}
\subfigure[Observation Points in One Scan] {\label{fig_scan}
\includegraphics[width=0.23\textwidth]{./scan.pdf}}
\caption{Sampling strategy for generating observations from a laser scan: (a) Equidistant sampling on a beam, with red indicating occupied and blue indicating free states. The distance between two consecutive points is the resolution $s$. (b) All sampled observation points at a given time step.}
\label{fig_scan_sampling}
% \vspace{-1em}
\end{figure}

We now introduce our sampling strategy for generating observations from laser scans, which are used in our NLLS formulation. 

Each scan data consists of a number of beams. On each beam, the endpoint indicates the presence of an obstacle, while the other points before the endpoint indicate the absence of obstacles. Here, we sample each beam using a fixed resolution $s$ to get the observations, as shown in Fig. \ref{fig_sampling_strategy}. Specifically, $\mathbf{q}_{ij}=[x_{q_{ij}},y_{q_{ij}}]^\top$ denotes the position of $j$th sampling point at time step $i$ in the local robot/laser coordinate frame and
\begin{equation}
z_{ij} = \ln \frac{p(\mathbf{q}_{ij} \in occ)}{1-p(\mathbf{q}_{ij} \in occ)} \label{eq_occ_obs}
\end{equation}denotes the corresponding occupancy value. In the same way as the occupancy map representation described in Section \ref{sec_discrete_occupancy}, we also use the evidence to represent the occupancy value here. In our implementation, following \cite{hornung2013octomap,ProbabilisticRobotics}, we use $p(\mathbf{q}_{ij} \in occ) = 0.7$ for an occupied point (red in Fig. \ref{fig_sampling_strategy}), and use $p(\mathbf{q}_{ij} \in occ) = 0.4$ for a free point (blue in Fig. \ref{fig_sampling_strategy}). Fig. \ref{fig_scan} shows an example of all sampled points in one scan.
% $(0 \leq i \leq n)
By constant equidistant sampling of all the beams for the scan collected at time step $i$, a sampling point set
\begin{equation}
\mathbb{S}_i=\{ \mathbb{S}_{ij} \triangleq \{\mathbf{q}_{ij},z_{ij}\}\}_{1 \leq j \leq k_i}
\label{S_i}
\end{equation}
can be obtained. It should be noted that since the total length of all the beams at different time step $i$ is different, the number of sampling points $k_i$ obtained by the equidistant sampling strategy varies for different time step $i$.


Suppose there are $n+1$ laser scans collected from robot poses $0$ to $n$, $\mathbb{S}=\{\mathbb{S}_i\}_{0\leq i \leq n}$ is the available observation information collected at all different robot poses using our sampling strategy and will be used as observations in our NLLS formulation.

\subsection{Relationship Between Observations and Occupancy Map}\label{sec_relationship}


In Section \ref{sec_discrete_occupancy}, we defined the discrete occupancy map $\mathbb{M}$ as part of the state vector in our optimization problem. Section \ref{Sec_Info_1} detailed the observation generation process. In this section, we explain how the relationship between observations and the occupancy map is established through robot poses, forming the basis of our joint optimization problem.



\subsubsection{Local to Global Projection}
First, the $j$th scan point at time step $i$ can be projected to the occupancy map using the robot pose $\mathbf{x}^P_i$, and the projected position on the occupancy map can be calculated by 

\begin{equation}
	\mathbf{p}_{ij}
=\frac{\mathbf{R}_i^\top \mathbf{q}_{ij}+\mathbf{t}_i}{s} \label{P-project}
\end{equation}
where $s$ is the resolution of the vertices in the occupancy map $\mathbb{M}$ (the distance between two adjacent cell vertices represents $s$ meters in the real world). Here, we use the same resolution as that used in generating observations from laser scans in Section \ref{Sec_Info_1}. Then, the occupancy value at the projected point $\mathbf{p}_{ij}$ can be obtained using (\ref{eq_interp}), expressed as $M(\mathbf{p}_{ij})$.


\subsubsection{Relationship Between Sampling Points and Occupancy Map w.r.t. Occupancy Values}
As outlined in Sections \ref{sec_discrete_occupancy} and \ref{Sec_Info_1}, evidence is used to define the occupancy value, where multiple observations of the same cell result in the occupancy values from individual observations being cumulatively added to the cell's total occupancy value \cite{hornung2013octomap}. If the robot's poses are accurate and repeated observations of the same cell consistently indicate the same occupancy state, the cell's occupancy value becomes the product of the occupancy value of each observation and the number of times the cell is observed. For a unique coordinate $\mathbf{p}_{ij}$ in (\ref{P-project}), if both the robot pose $\mathbf{x}^P_i$ and the occupancy map $\mathbb{M}$ are accurate, the occupancy value $z_{ij}$ of its associated sampling point $\mathbb{S}_{ij}$, should closely approximate the occupancy value at $\mathbf{p}_{ij}$, $M(\mathbf{p}_{ij})$, divided by the number of times $\mathbf{p}_{ij}$ is ``observed", $N(\mathbf{p}_{ij})$.

Thus, if the number of times the point $\mathbf{p}_{ij}$ is ``observed" can be calculated, the relationship between the observations and the state vector (occupancy map and robot poses), can be determined.


\subsubsection{Hit Map and Hit Number Lookup}\label{sec_hit}

We now explain how $N({\mathbf{p}_{ij}})$ can be calculated. To quickly query the number of times an arbitrary point is ``observed", we need to count the number of times all cell vertices have been observed to form the discrete hit map $\mathbb{N}$ associated with the occupancy map $\mathbb{M}$. 

When a sampling scan point is projected into a coordinate by a given robot pose, this coordinate is considered to have been observed once, and then we distribute the hit number ``1" of the coordinate to the discretized cell vertices. Since the occupancy value $M(\mathbf{p}_{ij})$ is derived by bilinear interpolation of occupancy values of discrete cell vertices in (\ref{eq_interp}), in order to maintain the correspondence between the hit number and the occupancy values, we distribute this ``1" hit to the four surrounding cell vertices by inverse bilinear interpolation. For example, if a sampling point is projected into the center of a cell, then each of the 4 nearby cell vertices gets a hit number of 0.25. In addition, the hit number also accumulates with multiple observations of the same cell vertex, i.e.,
\begin{equation}
\left[ N(\mathbf{m}_{00}),\cdots,N(\mathbf{m}_{c_wc_h}) \right] 
= \sum_{i=0}^n \sum_{j=1}^{k_i} H(\mathbf{p}_{ij})
\label{eq_NP}
\end{equation}
where $H(\cdot)$ is the inverse process of bilinear interpolation. The hit number at all these discrete cell vertices consists of discrete hit map $\mathbb{N}=\{N(\mathbf{m}_{wh})\}_{0 \leq w \leq c_w, 0 \leq h \leq c_h}$.



After the discrete hit map $\mathbb{N}$ is obtained, the equivalent hit multiplier $N(\mathbf{p}_{ij})$ (representing the number of times $\mathbf{p}_{ij}$ is ``observed") for an arbitrary continuous point $\mathbf{p}_{ij}$ can be easily obtained using bilinear interpolation, similar to (\ref{eq_interp}). 

\subsection{The NLLS Formulation} % --- Observations Only Case
% With the map parameterization, observations generation, and projection from local to global coordinates, observations can be linked to the occupancy map through robot poses. 
We now formulate the NLLS problem to jointly optimize the robot poses and the occupancy map. The objective function of the NLLS problem is defined as
\begin{equation}
f(\mathbf{x})=w_Z f^Z(\mathbf{x})+w_O f^O(\mathbf{x})+w_S f^S(\mathbf{x}). 	\label{eq_objective_func}
\end{equation}
The objective function consists of the observation term $f^Z(\mathbf{x})$, the smoothing term $f^S(\mathbf{x})$, and the odometry term $f^O(\mathbf{x})$. $w_Z$, $w_S$ and $w_O$ are their corresponding weights, and we set $w_O = 0$ if there is no odometry information. We now explain the three terms one by one.

\subsubsection{Observation Term $f^Z(\mathbf{x})$}
Based on the relationship between the observations and the occupancy map w.r.t. occupancy values described in \ref{sec_relationship}, we can formulate the observation term as follows. 

Given the observation information $\mathbb{S}$ in (\ref{S_i}), the observation term in the objective function (\ref{eq_objective_func}) is formulated as
\begin{equation}
	f^Z(\mathbf{x}) =
	\sum_{i=0}^n \sum_{j=1}^{k_i}  \left\|z_{ij} - F_{ij}^Z(\mathbf{x})\right\|^2, 
\label{obs-term}
\end{equation}
where
\begin{equation}
	F_{ij}^Z(\mathbf{x})  = \frac{M(\mathbf{p}_{ij})}{N({\mathbf{p}_{ij}})}.\\ \label{eq_MN}
\end{equation}
Here, $\mathbf{p}_{ij}$ represents a coordinate in the map where the $j$th sampling scan point at time step $i$ is projected using the robot pose $\mathbf{x}^P_i$, as calculated by (\ref{P-project}). $N(\mathbf{p}_{ij})$ denotes the equivalent hit multiplier at $\mathbf{p}_{ij}$, as detailed in Section \ref{sec_hit}. 

In (\ref{obs-term}), we suppose the errors of occupancy values of different sampled points in the observations $\mathbb{S}$ are independent and have the same uncertainty. Therefore, the weights on all terms are the same, which is equivalent to setting all the weights as $1$. Thus, we use norms instead of weighted norms in equation (\ref{obs-term}).

\subsubsection{Odometry Term $f^O(\mathbf{x})$}\label{sec_odometry}

% Information\label{sec_odom_inputs}}
The odometry information $\mathbb{O} = \{\mathbf{o}_i\}_{1 \leq i \leq n}$ might be available. We assume the odometry input is the relative pose between two consecutive steps. 
The odometry from robot pose $\mathbf{x}^P_{i-1}$ to pose $\mathbf{x}^P_{i}$ is expressed as
\begin{equation} 
\mathbf{o}_i=\left[ (\mathbf{o}_i^t)^\top,o_i^\theta \right]^\top~~(1 \leq i \leq n)
\label{O_i}
\end{equation}
where $\mathbf{o}_i^t$ is the translation part and $o_i^\theta$ is the rotation angle part of the odometry. The odometry term can be formulated as
\begin{equation}
\begin{aligned}
f^O(\mathbf{x})&=\sum_{i=1}^n \left\|\mathbf{o}_i -
F_i^O(\mathbf{x})
\right\|^2_{\mathbf{\Sigma}^{-1}_{O_i}}
\\&=\sum_{i=1}^n\left\|
\begin{bmatrix}
\mathbf{o}_i^t-\mathbf{R}_{i-1}\left(\mathbf{t}_i - \mathbf{t}_{i-1} \right)\\
\wrap\left({o}_i^\theta- \theta_i + \theta_{i-1}\right)
\end{bmatrix}
\right\|^2_{\mathbf{\Sigma}^{-1}_{O_i}}  \label{eq_odometry_term}
\end{aligned}
\end{equation}
in which $\mathbf{\Sigma}_{O_i}$ is the covariance matrix representing the uncertainty of $\mathbf{o}_i$, and $\wrap(\cdot)$ wraparounds the rotation angle to $(-\pi,\pi]$.
\subsubsection{Smoothing Term $f^S(\mathbf{x})$}

It can be easily found out that minimizing the objective function with only the observation term (and the odometry term) is not easy since there are a large number of local minima. Especially when the initial robot poses are far away from the global minimum, it is very difficult for an optimizer to converge to the correct solution. 

In order to enlarge the region of attraction and develop an algorithm that is robust to initial values, we introduce a smoothing term. The smoothing term requires the occupancy values of nearby cell vertices to be close to each other thus resulting in the occupancy map being smoother for derivative calculation. In our case, based on the derivative calculation method we use (see Appendix \ref{Sec_J_P}), we penalize the difference between the occupancy value of each cell vertex and the occupancy values of the two neighboring cell vertices to its right and below, i.e.,
\begin{equation}
\begin{aligned}
f^S(\mathbf{x})
& =\left\|F^S(\mathbf{x}) \right\|^2\\
& = \sum_{w=0}^{c_w-1} \sum_{h=0}^{c_h-1}  \left\|\begin{bmatrix} M(\mathbf{m}_{wh})-M(\mathbf{m}_{{(w+1)}h})\\
M(\mathbf{m}_{wh})-M(\mathbf{m}_{{w}{(h+1)}})
\end{bmatrix} \right\|^2 \\
& + \sum_{h=0}^{c_h-1}  \left\| M(\mathbf{m}_{c_wh})-M(\mathbf{m}_{{c_w}{(h+1)}})\right\|^2 \\
& + \sum_{w=0}^{c_w-1}  \left\| M(\mathbf{m}_{wc_h})-M(\mathbf{m}_{{(w+1)}{c_h}})\right\|^2,
\end{aligned} \label{eq_smoothing_term}
\end{equation} where the second and third terms are used to handle cell vertices located in the bottom row and the rightmost column. It should be noted that $F^S(\mathbf{x})$ is a linear function of $\mathbf{x}^M$ in the state. The coefficient matrix is constant and can be calculated prior to the optimization. For more details, please refer to Appendix \ref{Sec_J_S}.


\section{Iterative Solution to the NLLS Formulation}\label{Sec_Algorithm_1}
In Section \ref{sec_formulation}, we introduced our NLLS formulation for the joint poses and occupancy map optimization problem. In this section, we provide the details of a Gauss-Newton based algorithm for solving the NLLS problem. 



\begin{algorithm}[t]
\small
\caption{Our Joint Poses and Occupancy Map Optimization Algorithm}\label{alg_1}
\SetKwInput{KwInput}{Input}                % Set the Input
\SetKwInput{KwOutput}{Output}              % set the Output
\SetKwInput{KwParam}{Params}
\SetAlgoLined
\DontPrintSemicolon
\SetKw{Return}{End Function}
  \KwParam{Threshold $\tau_k$, $\tau_{\Delta}$, weight matrix $\mathbf{W}$, resolution $s$}
  \KwInput{Observations $\mathbb{S}$, odometry $\mathbb{O}$, and initial poses $\mathbf{x}^P(0)$}
  \KwOutput{Optimized poses $\hat{\mathbf{x}}^P$ and optimized map $\hat{\mathbf{x}}^M$}
\SetKwFunction{FuncFirstStage}{FirstStage}

\SetKwProg{Fn}{Function}{:}{}
\Fn{\FuncFirstStage{$\mathbf{x}^P(0)$, $\mathbb{S}$, $\mathbb{O}$, $\tau_k$, $\tau_{\Delta}$, $s$, $\mathbf{W}$}}
{
Initialize $\mathbf{x}^M(0)$ and $\mathbb{N}(0)$ using $\mathbf{x}^P(0)$ and $\mathbb{S}$ \;

Pre-calculate smoothing term coefficient $\mathbf{A}$ using (\ref{eq_A})\;

\SetKwFunction{FuncGN}{OccupancyGN}
\SetKwFunction{FuncReturn}{return}

\SetKwProg{Fn}{Function}{:}{}
\For {$k=0$; $k <= \tau_k \; \& \; \| \mathbf{\Delta}(k) \|^2 >= \tau_{{\Delta}}$; $k++$}{
\Fn{\FuncGN{$\mathbf{x}^M(k)$, $\mathbb{N}(k)$, $\mathbf{x}^P(k)$, $\mathbb{S}$, $\mathbb{O}$, $\mathbf{A}$, $\mathbf{W}$}}{
Calculate gradient $\mathbf{\nabla} \mathbf{x}^M(k)$ of $\mathbf{x}^M(k)$

Calculate $\mathbf{J}$, as described in appendices

Evaluate $F(\mathbf{x})$ at $\mathbf{x}^P(k)$ and $\mathbf{x}^M(k)$

Solve $\mathbf{J}^\top \mathbf{W} \mathbf{J} \mathbf{\Delta}(k) =-\mathbf{J}^\top \mathbf{W} F(\mathbf{x})$, where $\mathbf{\Delta}(k) = {[{\mathbf{\Delta}^P(k)}^\top,{\mathbf{\Delta}^M(k)}^\top]}^\top$

Update $\mathbf{x}^P(k+1)=\mathbf{x}^P(k) + \mathbf{\Delta}^P(k)$ and $\mathbf{x}^M(k+1)=\mathbf{x}^M(k)+\mathbf{\Delta}^M(k)$

Recalculate $\mathbb{N}(k+1)$ using $\mathbf{x}^P(k+1)$ and $\mathbb{S}$

\FuncReturn{$\mathbf{x}^P(k+1)$, $\mathbf{x}^M(k+1)$}
}
\Return
}
$\hat{\mathbf{x}}^P \Leftarrow \mathbf{x}^P(k)$, $\hat{\mathbf{x}}^M \Leftarrow \mathbf{x}^M(k)$

\FuncReturn{$\hat{\mathbf{x}}^P$, $\hat{\mathbf{x}}^M$}
}

\Return
\end{algorithm}


In the equation below, we assume the odometry inputs are available. Let
\begin{equation}
\begin{aligned}
F(\mathbf{x}) = [&\cdots,z_{ij}-F_{ij}^Z(\mathbf{x}),\cdots,{(\mathbf{o}_i-F_i^O(\mathbf{x}))}^\top,\\
&\cdots,{F^S(\mathbf{x})}^\top]^\top\\
\mathbf{W} = \;\; &\diag(\cdots,w_Z,\cdots,w_O \mathbf{\Sigma}^{-1}_{O_i}, \cdots,w_S, \cdots)\\
\end{aligned}
\end{equation}
combine all the error functions and the weights of the three terms in (\ref{eq_objective_func}). Then, the NLLS problem in (\ref{eq_objective_func}) seeks $\mathbf{x}$ such that
\begin{equation}\label{Least Squares}
f(\mathbf{x})=\|F(\mathbf{x})\|^2_{\mathbf{W}} =
{F(\mathbf{x})}^\top \mathbf{W}
F(\mathbf{x})
\end{equation}
is minimized.

A solution to (\ref{Least Squares}) can be obtained iteratively by starting with an initial guess $\mathbf{x}(0)$ and updating with $\mathbf{x}(k+1) = \mathbf{x}(k) + \mathbf{\Delta}(k)$. \rule{0pt}{1em}The update vector $\mathbf{\Delta} (k) = [{\mathbf{\Delta}^P(k)}^\top,{\mathbf{\Delta}^M(k)}^\top]^\top$ is the solution to
\begin{equation}\label{Gauss-Newton}
\mathbf{J}^\top \mathbf{W} \mathbf{J} \mathbf{\Delta} (k) = -\mathbf{J}^\top \mathbf{W} F(\mathbf{x}(k))
\end{equation}
where $\mathbf{J}$ is the linear mapping represented by the Jacobian matrix
$\partial F / \partial \mathbf{x}$ evaluated at $\mathbf{x}(k)$.

The iterative method for solving the proposed NLLS problem is shown in Algorithm \ref{alg_1}, in which $\tau_k$ and $\tau_{\Delta}$ represent the thresholds of iteration number $k$ and the incremental vector $\mathbf{\Delta}$. Unlike the standard Gauss-Newton iterative method, the hit map needs to be additionally recalculated after updating the poses in each iteration. With this approach, the implicit data association is established at each iteration and updated during the optimization.

Since the robot poses and the occupancy map are optimized simultaneously, the Jacobian $\mathbf{J}$ in (\ref{Gauss-Newton}) is very important and quite different from those used in the traditional SLAM algorithms. More details of the Jacobians are described in appendices.




\section{Multi-resolution Joint Optimization Strategy} \label{Sec_multi}
Algorithm \ref{alg_1} provides a solution to our NLLS problem (\ref{eq_objective_func}) to jointly optimize the poses and the occupancy map. However, directly using Algorithm \ref{alg_1} with the high-resolution map is time-consuming and requires an accurate initial value of robot poses \cite{Zhao-RSS-22}, which is challenging to obtain. To overcome these limitations, we propose a multi-resolution joint optimization strategy in this section.

\subsection{Discussion on Map Resolution in Optimization }

The resolution of the occupancy map has a significant impact on the optimization results since $\mathbf{x}^M$ is part of the state vector in our NLLS formulation (\ref{eq_objective_func}). 

% Assuming the robot poses are accurate, a high-resolution map representation can establish accurate relationships between observations and occupancy values of projected points on the map, but it leads to a dramatic increase in the size of the optimization problem, which in turn leads to a significant increase in computational cost. In addition, in a high-resolution map, the occupancy values in nearby \textcolor{red}{cell vertices} can vary sharply, leading to noise-filled gradients in the map when poor initial robot poses are used. Even with the introduction of the smoothing term, the use of a high-resolution map may cause poor convergence of Algorithm \ref{alg_1}.

A high-resolution map enables precise relationships between observations and occupancy values of projected points. However, it results in a dramatic increase in the optimization problem's size, raising computational costs. Additionally, in a high-resolution map, occupancy values in adjacent cell vertices may exhibit sharper variations compared to those in a low-resolution map, leading to noisy gradients when poor initial robot poses are used. Even with the introduction of the smoothing term, the use of a high-resolution map may cause poor convergence of Algorithm \ref{alg_1}.



A low-resolution map provides advantages in faster computation and reduced memory usage. Moreover, gradients are less sensitive to pose accuracy. With our occupancy map representation and smoothing term, these advantages enable the algorithm to quickly converge to a reasonable solution, even with poor initial robot poses. However, low resolution may cause inaccurate links between observations and occupancy values near boundaries, preventing the optimization from achieving greater accuracy.

To combine the advantages of different resolution map representations, we propose a multi-resolution strategy to optimize the occupancy values of different resolution cell vertices together with robot poses at various stages. Unlike the conventional coarse-to-fine scheme, in the second stage of our strategy, we use the selected high-resolution map that only includes high-resolution cell vertices possibly in need of further optimization instead of the full high-resolution map. Optimizing only those selected high-resolution cell vertices further improves the efficiency of our algorithm. 
\subsection{Our Multi-resolution Joint Optimization Strategy}

Firstly, we obtain low-resolution observations $\mathbb{S}^{l} = \{\mathbb{S}_i^{l}\}_{0\leq i \leq n}$ by down-sampling from the high-resolution observations $\mathbb{S}^{h} = \{\mathbb{S}_i^{h}\}_{0\leq i \leq n}$, which are obtained by the equal sampling strategy described in Section \ref{Sec_Info_1} with a sampling distance $s^{h}$. Here, we set the map resolution and sampling resolution to be the same. Therefore, the low resolution $s^{l}=r \times s^{h}$, where $r$ is the resolution ratio between the low-resolution map and the high-resolution map. The size of the low-resolution map $\mathbb{M}^{l}$ is $(c_w+1) \times (c_h+1)$.

Initialized by the odometry inputs or scan matching, we perform Algorithm \ref{alg_1} to quickly obtain relatively accurate poses. The state vector in the first stage is ${\mathbf{x}}^{l} = {[{\mathbf{x}^P}^\top,{{\mathbf{x}^{lM}}}^\top]}^\top $, where $\mathbf{x}^{lM}$ includes all occupancy values at the cell vertices of the low-resolution map $\mathbb{M}^{l}$. In this stage, the hit map, observation information, coefficient matrix, weight matrix, and resolution are represented as $\mathbb{N}^{l},  \mathbb{S}^{l}, \mathbf{A}^{l}$, $\mathbf{W}^{l}$, and $s^{l}$, respectively. 

In the first stage of optimization, the low-resolution occupancy map reduces both the dimension of $\mathbf{x}^{lM}$ and the number of observations in $\mathbb{S}^{l}$. Since the occupancy values at cell vertices change relatively gradually in the low-resolution map, the directions of the map's gradients are closer to the correct ones when the poses are initialized by odometry inputs or scan matching, making it easier for Algorithm \ref{alg_1} to converge to a relatively good result quickly.

After the first stage, we use Algorithm \ref{alg_2} to select the cell vertices that need to be further optimized to compose the selected high-resolution map $\mathbb{M}^{s}$ and find their corresponding observations $\mathbb{S}^{s}$. Details are described in Section \ref{select_index_set}.


\begin{algorithm}[tp]
\small
\caption{Finding the Selected High-resolution Map and Corresponding Observations}\label{alg_2}

\SetKwInput{KwInput}{Input}                % Set the Input
\SetKwInput{KwOutput}{Output}              % set the Output
\SetKwInput{KwParam}{Params}
\SetKw{Return}{End Function}
\SetAlgoLined
\DontPrintSemicolon
\KwParam{Resolution $s^{h}$, selection distance $d$, convolution kernel size $q$}
  \KwInput{Observations $\mathbb{S}^{h}$, and poses ${{\hat{\mathbf{x}}}}^{\tilde{P}}$ from the first stage using Algorithm \ref{alg_1}}
  \KwOutput{Observations $\mathbb{S}^{s}$ and map part of the state vector in the second stage $\mathbf{x}^{sM}$}
  \SetKwFunction{FuncSel}{Selection}
\SetKwProg{Fn}{Function}{:}{}
\Fn{\FuncSel{${{\hat{\mathbf{x}}}}^{\tilde{P}}$, $\mathbb{S}^{h}$, $s^{h}$, $d$, $q$}}{

Build a full high-resolution map $\mathbb{M}^{h}$ using $\hat{\mathbf{x}}^{\tilde{P}}$ and $\mathbb{S}^{h}$ 

Calculate the binary map $\mathbb{B}$ using $\mathbb{M}^{h}$

Calculate the convoluted map $\mathbb{C}$ with kernel size $q$

Calculate the set $\mathbb{I}^{h}$, which includes the indices of all boundary vertices in $\mathbb{M}^{h}$, using $\mathbb{C}$

Calculate the set $\mathbb{I}^{s}$, which includes the indices of all selected vertices, using $\mathbb{I}^{h}$ and $d$

Define the map part of the state vector in the second stage $\mathbf{x}^{sM}$ and the selected high-resolution map $\mathbb{M}^{s}$ by $\mathbb{I}^{s}$

Find observations $\mathbb{S}^{s}$ for $\mathbb{M}^{s}$ using the set $\mathbb{I}^{s}$, $\mathbb{S}^{h}$ and ${{\hat{\mathbf{x}}}}^{\tilde{P}}$

\FuncReturn{$\mathbb{S}^{s}$, $\mathbf{x}^{sM}$}
}
\Return
\end{algorithm}


In the second stage, the state vector is represented as $\mathbf{x}^{s} = {[{{\mathbf{x}}^P}^\top,{\mathbf{x}^{sM})}^\top]}^\top $ where $\mathbf{x}^{sM}$ includes all occupancy values at cell vertices of the selected high-resolution map $\mathbb{M}^{s}$. We perform Algorithm \ref{alg_3} using poses obtained from the first stage as initial guesses and observations $\mathbb{S}^{s}$ to refine poses. The NLLS optimization problem in the second stage can be formulated similarly as (\ref{Least Squares}). Additionally, the differences in the Jacobian calculation between Algorithm \ref{alg_1} and Algorithm \ref{alg_3} are described in Appendix \ref{Sec_J_Select}. 

The full multi-resolution joint optimization strategy is outlined in Algorithm \ref{alg_flowchart}.

\begin{algorithm}[t]
\small
\caption{The Algorithm for the Second Stage of the Multi-resolution Joint Optimization Strategy}\label{alg_3}
\SetKwInput{KwParam}{Params}
\SetKwInput{KwInput}{Input}                % Set the Input
\SetKwInput{KwOutput}{Output}              % set the Output
\SetAlgoLined
\DontPrintSemicolon
\SetKw{Return}{End Function}

  \KwParam{Threshold $\tau_k^{s}$, $\tau_{\Delta}^{s}$, weight matrix $\mathbf{W}^{s}$, resolution $s^{h}$}
  \KwInput{Observations $\mathbb{S}^{s}$, odometry $\mathbb{O}$, and poses ${{\hat{\mathbf{x}}}}^{\tilde{P}}$ from the first stage using Algorithm \ref{alg_1}}
  \KwOutput{Optimal poses $\hat{\mathbf{x}}^P$ and map $\hat{\mathbf{x}}^{sM}$}
\SetKwFunction{FuncSecondStage}{SecondStage}

$\mathbf{x}^P(0) \Leftarrow {{\hat{\mathbf{x}}}}^{\tilde{P}}$

\SetKwProg{Fn}{Function}{:}{}
\Fn{\FuncSecondStage{$\mathbf{x}^P(0)$, $\mathbb{S}^{s}$, $\mathbb{O}$, $\tau_k^{s}$, $\tau_{\Delta}^{s}$, $s^{h}$, $\mathbf{W}^{s}$}}
{

Initialize $\mathbf{x}^{sM}(0)$ and $\mathbb{N}^{s}(0)$ using $\mathbb{S}^{s}$ and $\mathbf{x}^P(0)$

Pre-calculate smoothing term coefficient matrix $\mathbf{A}^{s}$

\For {$k=0$; $k <= \tau_k^{s} \; \& \; \| \mathbf{\Delta}(k) \|^2 >= \tau_{\Delta}^{s}$; $k++$}{

$\mathbf{x}^P(k+1)$, $\mathbf{x}^{sM}(k+1)$
$\leftarrow$  \FuncGN{$\mathbf{x}^{sM}(k)$, {$\mathbb{N}^{s}(k)$, $\mathbf{x}^P(k)$, $\mathbb{S}^{s}$, $\mathbb{O}$, $\mathbf{A}^{s}$, $\mathbf{W}^{s}$}}
}

$\hat{\mathbf{x}}^P \Leftarrow \mathbf{x}^P(k)$, $\hat{\mathbf{x}}^{sM} \Leftarrow \mathbf{x}^{sM}(k)$

\FuncReturn{$\hat{\mathbf{x}}^P$, $\hat{\mathbf{x}}^{sM}$}

}
\Return
\end{algorithm}




\begin{algorithm}[t]
\small
\caption{Our Multi-resolution Joint Optimization Strategy}\label{alg_flowchart}
\SetKwInput{KwParam}{Params}
\SetKwInput{KwInput}{Input}                % Set the Input
\SetKwInput{KwOutput}{Output}              % set the Output

\SetAlgoLined
\DontPrintSemicolon

\SetKwFunction{FuncDown}{DownSampling}
\SetKwFunction{FuncInitPose}{InitializePose}
\SetKwFunction{FuncSM}{ScanMatching}
  \KwParam{Threshold $\tau_k^{l}$, $\tau_{\Delta}^{l}$, $\tau_k^{s}$, $\tau_{\Delta}^{s}$, weight matrix $\mathbf{W}^{l}$, $\mathbf{W}^{s}$, ratio of resolutions $r$, resolution $s$, selection distance $d$, convolution kernel size $q$}
  \KwInput{Observations $\mathbb{S}^{h}$, odometry $\mathbb{O}$}
  \KwOutput{Optimal poses $\hat{\mathbf{x}}^P$ and map $\hat{\mathbf{x}}^{sM}$}

$\mathbb{S}^{l}$ $\leftarrow$ \FuncDown{$\mathbb{S}^{h}$, $r$}

\uIf {$w_O \neq  0$}
{
$\mathbf{x}^P(0)$ $\leftarrow$ \FuncInitPose{$\mathbb{O}$}
}
\Else
{
$\mathbf{x}^P(0)$ $\leftarrow$ \FuncSM{$\mathbb{S}^{l}$}
}


${{\hat{\mathbf{x}}}}^{\tilde{P}}$, $\hat{\mathbf{x}}^{lM}$ $\leftarrow$ \FuncFirstStage{$\mathbf{x}^P(0)$, $\mathbb{S}^{l}$, $\mathbb{O}$, $\tau_k^{l}$, $\tau_{\Delta}^{l}$, $s^{l}$, $\mathbf{W}^{l}$}

$\mathbb{S}^{s}$, $\mathbf{x}^{sM}$ $\leftarrow$ \FuncSel{${{\hat{\mathbf{x}}}}^{\tilde{P}}$, $\mathbb{S}^{h}$, $s^{h}$, $d$, $q$}

${\hat{\mathbf{x}}^P}$, $\hat{\mathbf{x}}^{sM}$ $\leftarrow$
\FuncSecondStage{${{\hat{\mathbf{x}}}}^{\tilde{P}}$, $\mathbb{S}^{s}$,$\mathbb{O}$, $\tau_k^{s}$, $\tau_{\Delta}^{s}$, $s^{h}$, $\mathbf{W}^{s}$}

\end{algorithm}

\begin{figure}[t]
\centering 
\subfigure[Full High-resolution Map]{ 
\includegraphics[width=0.23\textwidth]{./high_resolution_select.pdf}}
\subfigure[Selected High-resolution Map (In White and Black)]{\label{fig_select_example_b}
\includegraphics[width=0.23\textwidth]{./recolor_select.pdf}}
\caption{An example of the selected high-resolution map from a full high-resolution map in a simulation dataset. (a) The full high-resolution map generated using poses from the first-stage optimization and scans, forming the basis for selection. (b) The recolored selected high-resolution map: gray marks dropped (stable) areas, white and black denote selected areas, with black highlighting obstacle boundaries.}
\label{fig_select_example}
% \vspace{-1em}
\end{figure}

\subsection{Selected High-resolution Map and Observations}\label{select_index_set}

After the first stage optimization using the low-resolution map $\mathbb{M}^{l}$, the robot poses $\hat{\mathbf{x}}^{\tilde{P}}$ become relatively accurate. Subsequently, the full high-resolution map $\mathbb{M}^{h}$, with dimensions $(r*c_w+1) \times (r*c_h+1)$, is built using the Bayesian occupancy mapping method \cite{ProbabilisticRobotics}, based on observations $\mathbb{S}^{h}$ and poses ${{\hat{\mathbf{x}}}}^{\tilde{P}}$. In this case, most cell vertices of $\mathbb{M}^{h}$ are considered stable in terms of occupancy state. Semantically, these stable cell vertices have the same occupancy state as the surrounding cell vertices (typically free or unknown cells). This characteristic leads to map gradients near zero at these stable cell vertices. In contrast, the cell vertices that require further updates are typically located at the edges of objects, where the occupancy values significantly differ from those of surrounding cell vertices. Therefore, the gradient at these cell vertices is larger. An example illustrating this is shown in Fig. \ref{fig_select_example_b}, where the selected area (in white and black) is clearly distinct from the stable area (in gray). Based on this idea, we propose a strategy to select the cell vertices located around the boundaries to compose the selected high-resolution map $\mathbb{M}^{s}$, which is used in the second stage of optimization.


\begin{figure}[t]
\centering 
\includegraphics[width=0.48\textwidth]{./Select_Set.pdf}
\caption{\label{fig_low_high_select} An illustration of the cell vertices selection strategy and a selected high-resolution map from a simulation dataset. In (a), selected cell vertices are marked in red and yellow, with their indices forming the index set $\mathbb{I}^{s}$.}
% \vspace{-0.5em}
\end{figure}

% In both figures, the boundary cells are indicated as black color, the selected cells are shown as black and white, and the dropped cells are colored in gray.

Firstly, we identify cell vertices located at the edges of objects by performing mean-value convolution of the full high-resolution map $\mathbb{M}^{h}$. Specifically, we calculate a binary map $\mathbb{B}=\{B(\mathbf{m}_{id})\}$ by binarizing $\mathbb{M}^{h}$ as
\begin{equation}
B(\mathbf{m}_{id}) =	\begin{cases}
	1, & {M}^{h}(\mathbf{m}_{id}) \geq \tau_{occupied} \\
	0, & {M}^{h}(\mathbf{m}_{id}) < \tau_{occupied} \\
\end{cases},
\end{equation}
where $\mathbf{m}_{id}$ represents a cell vertex, and $\tau_{occupied}$ is the threshold used to classify a cell vertex as occupied or free. A mean-value convolution kernel $\mathbf{K}$ is defined as
\begin{equation}
	\mathbf{K} = \dfrac{1}{q^2} \cdot \bold{1}_{q\times q}
\end{equation}
where $\bold{1}_{q\times q}$ represents a $q \times q$ matrix of ones. The convoluted map $\mathbb{C}=\{C(\mathbf{m}_{id})\}$ is then derived by convolving $\mathbb{B}$ with $\mathbf{K}$, where $C(\mathbf{m}_{id})$ indicates whether the $q \times q$ cell vertices around $\mathbf{m}_{id}$ are all in the same occupancy state. Compared to other edge detection methods like Sobel \cite{duda1973pattern} and Canny \cite{canny1986computational}, this conservative method more reliably selects cell vertices that may require further optimization. 

Using this method, the set of indices for all boundary cell vertices in the high-resolution map is defined as 

\begin{equation}
\begin{aligned}
	\mathbb{I}^{h} = \{id | 0<C(\mathbf{m}_{id})<1 \}.
\end{aligned}
\end{equation}
The cell vertices indexed in $\mathbb{I}^{h}$ are marked in red in Fig. \ref{fig_low_high_select}(a). 


To account for pose uncertainties from the first stage, the selection is expanded to include cell vertices within a distance $d$ from all boundary cell vertices. The indices of the selected cell vertices in the high-resolution map form the set $\mathbb{I}^{s}$, illustrated in Fig. \ref{fig_low_high_select}(a), where the selected cell vertices are highlighted in red and yellow with $d=1$. An example of a selected high-resolution map from a simulation dataset is shown in Fig. \ref{fig_low_high_select}(b).

Consequently, the map component of the state vector in the second stage is expressed as
\begin{equation}
	\mathbf{x}^{sM} = {[\cdots, M^{h}(\mathbf{m}_{wh}), \cdots]}^\top, ~~ wh\in \mathbb{I}^{s}.
\end{equation}

% The cell selection strategy with a selection distance of 1 cell is illustrated in Fig. \ref{fig_low_high_select}(a), where selected \textcolor{red}{cell vertices} are marked with red dots and their corresponding selected cells are marked with black and white. 


Next, we select observations to optimize $\mathbf{x}^{sM}$. Cells surrounded by vertices with indices in $\mathbb{I}^s$ are designated as selected cells, shown in white in Fig. \ref{fig_low_high_select}(a) and Fig. \ref{fig_low_high_select}(b). Subsequently, sampling point selection is carried out, as illustrated in Fig. \ref{fig_select_sampling_point}. Specifically, sampling points in $\mathbb{S}^{h}$ are first projected onto the global coordinate system using the poses optimized in the first stage. All sampling points located on the selected cells are then included to form the set $\mathbb{S}^{s}$. 

\begin{figure}[t]
\centering 
\includegraphics[width=0.48\textwidth]{./Select_Sampling_Point.pdf}
\caption{\label{fig_select_sampling_point} An example of the selected sampling points of a beam at time step $i$, where points projected onto the selected cells are chosen.}
% \vspace{-0.5em}
\end{figure}


\section{Submap Joining} \label{Sec_submap}
In Section \ref{Sec_multi}, we introduced a multi-resolution joint optimization strategy to efficiently solve our NLLS problem. For large-scale occupancy SLAM with long robot trajectories, the number of poses to optimize can be very large. To make the computational complexity dependent only on the environment size rather than the trajectory length, in this section we propose an occupancy submap joining method. The key idea is to reduce the number of poses that need to be optimized to the number of local submaps. 


\subsection{Inputs and Outputs of Submap Joining Problem} 
We first separate the observation information into multiple parts and perform Algorithm \ref{alg_flowchart} to build several submaps. The inputs of submap joining problem are a sequence of local occupancy submaps. 
Let us denote the $n_L+1$ submaps as $\mathbb{M}_L = \{\mathbb{M}_{L_0}, \cdots, \mathbb{M}_{L_{n_L}}\}$ and the associated coordinate frames of these local occupancy maps are denoted as $ \{\mathbf{x}^P_{0}, \cdots, \mathbf{x}^P_{n_L}\}$, where $\mathbb{M}_{L_{i_L}}$ and $\mathbf{x}^P_{i_L}$represents the $i_L$th local occupancy map and its associated coordinate frame. In addition, the global occupancy map is represented as $\mathbb{M}_G=\{M_G(\mathbf{m}^G_{00}), \cdots,{M}_G\left(\mathbf{m}^G_{c_w^Gc_h^G}\right)\}$. Both the global map and local maps follow the same definition as described in Section \ref{sec_discrete_occupancy}. The outputs of submap joining problem are the optimal solution of the local submap coordinate frames and the optimal global occupancy map.

\subsection{NLLS Formulation of Submap Joining Problem} 
First, the cell vertex $\mathbf{m}^G_{wh}$ in the global occupancy map $\mathbb{M}_G$ can be projected to local submap coordinate by pose $\mathbf{x}^P_{i_L}$, i.e., 
\begin{equation}
	\mathbf{p}_{i_L}^{wh} = \frac{ \mathbf{R}_{i_L} (\mathbf{m}^G_{wh} \cdot s_G  - \mathbf{t}_{i_L})}{s_L}.
\end{equation}
Here, $\mathbf{p}_{i_L}^{wh}$ represents the position in the local submap's coordinate where the cell vertex $\mathbf{m}_{wh}^G$ from the global map is projected using the pose $\mathbf{x}^P_{i_L}$. The resolutions of the global occupancy map and local submaps are denoted by $s_G$ and $s_L$, respectively.

The submap joining problem aims to find the optimal global occupancy map and the poses of submap coordinate frames. Thus, the state vector for this problem is defined as 
\begin{equation}
	\mathbf{x}_G = [{\mathbf{x}^P_L}^\top, {{\mathbf{x}^M_G}}^\top]^\top,
\end{equation}
where 
\begin{equation}
\begin{aligned}
\mathbf{x}^P_L & =\left[\left(\mathbf{x}^P_1\right)^\top, \cdots,\left(\mathbf{x}^P_{n_L}\right)^\top\right]^\top \\
\mathbf{x}^M_G & =\left[{M}_G\left(\mathbf{m}^G_{00}\right), \cdots, {M}_G\left(\mathbf{m}^G_{c_w^Gc_h^G}\right)\right]^\top.
\end{aligned}
\end{equation}
As with most submap joining problem formulations, we fix the first local map coordinate frame as the global coordinate frame. Therefore, $\mathbf{x}^P_L$ consists of $n_L$ local map coordinate frames and $\mathbf{x}^M_G$ includes $(c_w^G+1) \times (c_h^G+1)$ discrete cell vertices of global occupancy map. 

By the global-to-local projection relationship, all cell vertices of global occupancy map $\mathbb{M}_G$ can be projected to corresponding submaps to compute the difference in occupancy values. Thus, the NLLS problem of occupancy submap joining can be formulated to minimize 
\begin{equation}
\begin{adjustbox}{max width=\linewidth}
$
g(\mathbf{x}_G) =  \sum\limits_{i_L=0}^n\sum\limits_{ wh \in \mathbb{S}^L_{i_L}} \left\| \omega(i_L,\mathbf{m}^G_{wh}) {M}_G(\mathbf{m}^G_{wh}) - {M}_{L_{i_L}}(\mathbf{p}_{i_L}^{wh}) \right\|^2,
$
\end{adjustbox}
\label{eq_NLLS_joining}
\end{equation}
where $\mathbb{S}^L_{i_L}$ represents the set of indices of cell vertices in the global occupancy map $\mathbb{M}_G$ that are projected onto the local submap $\mathbb{M}_{L_{i_L}}$.

In (\ref{eq_NLLS_joining}), $\omega(i_L,\mathbf{m}^G_{wh})$ is the weight to establish an accurate relationship between the global occupancy map and local submaps w.r.t. occupancy values, which can be calculated by
\begin{equation}
    \omega(i_L,\mathbf{m}^G_{wh}) = \frac{{N}_{{L_{i_L}}}(\mathbf{p}_{i_L}^{wh})}{{N}_{G}(\mathbf{m}^G_{wh})}.
\end{equation}
Here, ${N}_{{L_{i_L}}}(\cdot)$ is the local hit number lookup function for submap $\mathbb{M}_{L_{i_L}}$, derived as described in Section \ref{sec_hit}. It approximates the hit number at coordinate $\mathbf{p}_{i_L}^{wh}$ using bilinear interpolation. Similarly, ${N}_{G}(\cdot)$ represents the global hit number lookup function associated with $\mathbb{M}_G$.

In (\ref{eq_NLLS_joining}), the submap joining problem is formulated as a NLLS problem, which can be solved iteratively by Gauss-Newton based method similar to Algorithm \ref{alg_1}.


 \begin{table}[htp]
		\centering
		\caption{Parameters of Datasets. \label{tab_dataset}}
		\label{tab_comparison}
		\setlength{\tabcolsep}{0.7mm}{
		\begin{tabular}{lccccc}\toprule
		Dataset	& No. Scans & Duration  & Map Size &  Odometry & Resolution\\ \hline
		Simulation 1 & 3640  &117 s& $50$ m  $\times$ $50$ m & yes & 0.05 m\\
        Simulation 2 & 3720  &121 s& $50$ m $\times$ $50$ m & yes & 0.05 m\\
		Simulation 3  & 2680  & 83 s& $50$ m $\times$ $50$ m & yes & 0.05 m\\
		Car Park  & 1642 & 164 s& $50$ m $\times$ $40$ m & yes & 0.1 m\\
		C5  & 3870  &136 s& $50$ m $\times$ $40$ m & yes & 0.1 m\\
		Museum b0 & 5522 &152 s& $85$ m $\times$ $95$ m &no & 0.1 m \\
		Museum b2 & 51833 &1390 s &  $250$ m $\times$ $200$ m &no & 0.1 m\\
        C3 &24402 &610 s& $150$ m $\times$ $125$ m  & no & 0.1 m\\
		\hline
		\end{tabular}
		}
\end{table}


\section{Experimental Results} \label{Sec_experiment}

In this section, we evaluate our algorithm on several datasets and compare its performance with Cartographer \cite{hess2016real}, the current state-of-the-art algorithm, which significantly outperforms other methods such as Hector-SLAM \cite{kohlbrecher2011flexible} and Karto-SLAM \cite{konolige2010efficient}. To ensure fair comparisons, we adjust some parameters in Cartographer based on the sensor configurations of the respective datasets for optimal performance.


The dataset parameters are summarized in Table \ref{tab_dataset}. For practical datasets, Deutsches Museum b0 and Deutsches Museum b2 are Cartographer demo datasets collected at the Deutsches Museum. The Car Park \cite{zhao20212d} dataset is gathered in an underground car park, while C5 and C3 are collected in a factory environment using a Hokuyo UTM-30LX laser scanner. Consistent map resolutions $s$ are applied across all methods to display the map results, with ratio $r$ set to $10$ for all simulation experiments and $5$ for all practical experiments unless stated otherwise. For each dataset, 20\% of scans and corresponding poses are uniformly selected as key frames for the key frame option in our method.

To ensure fair comparisons, we use an identical number of poses (synchronizing the poses from the results with the ground truth poses using timestamps) and their corresponding observations to generate results for visualization and quantitative evaluation across all compared methods, with the exception of our method that employs keyframes. Furthermore, the same occupancy mapping algorithm is applied consistently across all approaches to produce the occupancy grid map results for comparison.

        
\subsection{Simulation Experiments}\label{simu_experiment}

\begin{figure*}[tp]
\centering \subfigure[Simulation 1] {\label{fig_trajectory_1}
\includegraphics[width=0.28\textwidth]{./trajectory_simu1_new.pdf}}
\centering \subfigure[Simulation 2] {\label{fig_trajectory_2}
\includegraphics[width=0.28\textwidth]{./trajectory_simu2_new.pdf}}
\centering \subfigure[Simulation 3] {\label{fig_trajectory_3}
\includegraphics[width=0.343\textwidth]{./trajectory_simu3_new.pdf}}
\caption{\label{fig_trajectory_compare}Simulation environments and robot trajectory results. (a), (b) and (c) show the simulation environments (the black lines indicate the obstacles in the scene) and the trajectories of ground truth, odometry inputs, Cartographer \cite{hess2016real}, and our approach for one dataset in each of the three simulation environments.}
% \vspace{-0.5em}
\end{figure*}

We use three different simulation environments with varying levels of nonlinearity and nonconvex obstacles to design three different simulation experiments. Since Cartographer needs a high-frequency scanning rate to ensure the good performance of scan matching, while our approach performs well for scan data with low scanning frequency, only 10\% scans listed in Table \ref{tab_dataset} are used in our method. 

We utilize the open-source 2D LiDAR simulator from \cite{zhao20212d} to generate simulated datasets. Each scan includes 1081 laser beams spanning angles from -135 degrees to 135 degrees, mimicking the specifications of a Hokuyo UTM-30LX laser scanner. To emulate real-world data acquisition, random Gaussian noise with zero mean and standard deviation of $0.02$ m is added to each beam of the simulated scan data. Similarly, zero-mean Gaussian noise is introduced to the odometry inputs derived from the ground truth poses, with standard deviation of $0.04$ m for $x$-$y$ and $0.003$ rad for orientation. Five datasets with different noise realizations are generated for each simulation environment.


\begin{figure*}[t]
\centering \subfigure[Simulation 1] {\label{fig_time_error_1}
\includegraphics[width=0.32\textwidth]{./Time_with_Error_Simu1.pdf}}
\centering \subfigure[Simulation 2] {\label{fig_time_error_2}
\includegraphics[width=0.32\textwidth]{./Time_with_Error_Simu2.pdf}}
\centering \subfigure[Simulation 3] {\label{fig_time_error_3}
\includegraphics[width=0.32\textwidth]{./Time_with_Error_Simu3.pdf}}
\caption{Comparison of translation and rotation errors at different time steps using simulation datasets.}
\label{fig_error_compare_time}
\vspace{-1em}
\end{figure*}


% The robot trajectory results of our method and Cartographer using one dataset in each simulation are compared with the ground truth and odometry in Fig. \ref{fig_trajectory_compare}. It is clear that our trajectories are closer to the ground truth trajectories, especially for positions where significant rotation occurs. Fig. \ref{fig_error_compare_time} shows the translation and rotation errors of our method and Cartographer at different time steps. Obviously, the errors of our method are substantially smaller than those of Cartographer.

The trajectory results of our method and Cartographer, compared to ground truth and odometry, are shown in Fig. \ref{fig_trajectory_compare}. It is evident that our trajectories align more closely with the ground truth, particularly in areas with significant rotational movements. Fig. \ref{fig_error_compare_time} illustrates translation and rotation errors over time, demonstrating that our method consistently achieves substantially smaller errors compared to Cartographer.

% We use all the fifteen datasets from Simulation 1, Simulation 2 and Simulation 3 to perform the quantitative and qualitative comparison of errors in the pose estimates. The quantitative results of Cartographer, only the first stage in our method \textcolor{red}{(Algorithm \ref{alg_1} using low-resolution)} using all frames, our method using all frames, and our method using key frames are given in Table \ref{tab_comparison}. We use mean absolute error (MAE) and root mean squared error (RMSE) to evaluate the translation errors (in meters) and rotation errors (in radians). Our method performs the best in all four metrics for all simulations and is substantially ahead of Cartographer even when using only key frames or only the first stage. In addition, Fig. \ref{fig_simulation}(a) to Fig. \ref{fig_simulation}(e) show the occupancy grid maps and point cloud maps generated using poses from ground truth, Cartographer and the three options of our method. It is clear that the boundaries of both occupancy grid maps and point cloud maps using the three options of our method are much clearer than those from Cartographer, which indicates that our method can obtain more accurate results by optimizing the robot poses and the occupancy map together.

We performed quantitative and qualitative comparisons of pose estimation errors using all fifteen datasets from Simulations 1, 2, and 3. Table \ref{tab_comparison} presents, in order, the quantitative results for odometry inputs, Cartographer, the first stage of our method (Algorithm \ref{alg_1} with low-resolution) using all frames, our method using all frames, and our method using key frames. Metrics such as mean absolute error (MAE) and root mean squared error (RMSE) evaluate translation errors (in meters) and rotation errors (in radians). Our method consistently achieves the best performance across all metrics, significantly outperforming Cartographer even when using only key frames or the first stage. Fig. \ref{fig_simulation}(a) to Fig. \ref{fig_simulation}(e) further illustrates occupancy grid maps and point cloud maps generated using poses from the ground truth, Cartographer, and the three options of our method. The maps produced by our method exhibit noticeably clearer boundaries, demonstrating its ability to jointly optimize robot poses and occupancy maps for more accurate results.

\begin{figure}[t]
\centering
\includegraphics[width=0.48\textwidth]{./Simulation_New.pdf}
\caption{\label{fig_simulation} The occupancy grid maps and point cloud maps generated from ground truth poses and different approaches for each simulation dataset. The areas marked with red dots highlight where our method outperforms the results of the first-stage optimization alone.}
\vspace{-0.5em}
\end{figure}

\begin{table}[t]
		\centering
		\caption{Quantitative Comparison of Robot Pose Errors in Simulations.}
		\label{tab_comparison}
		\setlength{\tabcolsep}{0.7 mm}{
		\begin{tabular}{lccccc}\toprule
			& Odom & Carto & Ours (First) & Ours (All) & Ours (Key) \\ \hline
		Simulation 1& & & &\\
		\quad MAE / Trans (m) & 0.78270 & 0.25336  & 0.02206 &\textcolor{red}{\textbf{0.00640}} & \textcolor{blue}{\textbf{0.01024}}\\
		\quad MAE / Rot (rad) & 0.04912 & 0.01394  & 0.00098 &\textcolor{red}{\textbf{0.00060}} & \textcolor{blue}{\textbf{0.00084}}
\\
		\quad RMSE / Trans (m) & 0.98404 & 0.29920  & 0.02680 &\textcolor{red}{\textbf{0.00974}} & \textcolor{blue}{\textbf{0.01430}}\\
		\quad RMSE / Rot(rad) & 0.05506 & 0.01562  & 0.00162 &\textcolor{red}{\textbf{0.00102}} & \textcolor{blue}{\textbf{0.00126}}\\\hline
		
		Simulation 2& & & &\\
		\quad MAE / Trans (m) & 0.80544 & 0.11914 &0.03224 & \textcolor{red}{\textbf{0.00858}} & \textcolor{blue}{\textbf{0.01082}}
\\
		\quad MAE / Rot (rad) & 0.02538 & 0.00666  &0.00220 &\textcolor{red}{\textbf{0.00062}} & \textcolor{blue}{\textbf{0.00096}}\\
		\quad RMSE / Trans (m) & 0.97152 & 0.14810  & 0.04188 &\textcolor{red}{\textbf{0.01198}} & \textcolor{blue}{\textbf{0.01244}}\\
		\quad RMSE / Rot (rad) & 0.02936 & 0.00916 &0.00220 & \textcolor{red}{\textbf{0.00104}} & \textcolor{blue}{\textbf{0.00178}}\\\hline
		
		Simulation 3& & & &\\
		\quad MAE / Trans(m) & 0.75352 & 0.14262  &0.02624 &\textcolor{red}{\textbf{0.00726}} &  \textcolor{blue}{\textbf{0.00998}}\\
		\quad MAE / Rot (rad) & 0.05180 & 0.00682 &0.00164  & \textcolor{red}{\textbf{0.00058}} &  \textcolor{blue}{\textbf{0.00090}}\\
		\quad RMSE / Trans (m) & 0.96866 & 0.18782 & 0.03238 &\textcolor{red}{\textbf{0.00952}} &  \textcolor{blue}{\textbf{0.01338}}\\
		\quad RMSE / Rot (rad) & 0.05926 & 0.00914  & 0.00204 &\textcolor{red}{\textbf{0.00088}} &  \textcolor{blue}{\textbf{0.00134}}\\\hline
		\end{tabular}
	\begin{tablenotes}
     \item \textcolor{red}{\textbf{Red}} and  \textcolor{blue}{\textbf{blue}} indicate the best and second best results, respectively.
   \end{tablenotes}
		}
        % \vspace{-2em}
\end{table}

% \begin{table*}[htp]
% \centering
% \caption{Occupancy Grid map Precision of Our Method Using All Frames, Our Method Using Key Frames, and Cartographer.}
% \label{tab_map_accuracy}
% \setlength{\tabcolsep}{2.4mm}
% \begin{NiceTabular}{cccccccccccc}[first-row,first-col,hvlines]
% \CodeBefore
% \Body
%  & \Block{1-11}{\textbf{Predicted}} & & & & & & &  & & & \\
% \Block{11-1}{\rotate Ground Truth}  & \Block{2-1}{}  & \Block{2-1}{} & \Block{1-3}{Our Method (All Frames)} & & & \Block{1-3}{Our Method (Key Frames)} & & & \Block{1-3}{Cartographer}  \\
%  & &   & Unknown & Free & Occupied & Unknown & Free & Occupied & Unknown & Free & Occupied \\
%  & \Block{3-1}{Simulation 1} & Unknown  & \textcolor{red}{\textbf{99.798\%}}  & 0.020\% & 0.182\% & \textcolor{blue}{\textbf{99.598\%}} & 0.072\% & 0.330\% & 95.616\% & 2.822\% & 1.562\% \\
%  &   &  Free & 0.022\%  & \textcolor{red}{\textbf{99.938\%}} & 0.040\% & 0.094\% & \textcolor{blue}{\textbf{99.824\%}} & 0.082\% & 1.290\% & 97.678\% & 1.032\% \\
%  &   & Occupied  &  5.436\% & 2.334\%  & \textcolor{red}{\textbf{92.230\%}} & 13.562\% & 3.142\% &\textcolor{blue}{\textbf{83.296\%}} &30.053\% & 53.280\% & 16.667\% \\

% & \Block{3-1}{Simulation 2} & Unknown  & \textcolor{red}{\textbf{99.846\%}}  & 0.010\% & 0.144\% & \textcolor{blue}{\textbf{99.696\%}} & 0.062\% & 0.242\% & 96.868\% & 1.743\% & 1.389\% \\
%  &   &  Free & 0.016\%  & \textcolor{red}{\textbf{99.846\%}} & 0.138\% & 0.076\% & \textcolor{blue}{\textbf{99.848\%}} & 0.076\% & 0.593\% & 98.584\% & 0.823\% \\
%  &   & Occupied  &  6.434\% & 3.738\%  & \textcolor{red}{\textbf{89.828\%}} & 11.694\% & 2.806\% &\textcolor{blue}{\textbf{85.500\%}} &23.943\% & 50.795\% & 25.262\% \\

%  & \Block{3-1}{Simulation 3} & Unknown  & \textcolor{red}{\textbf{99.812\%}}  & 0.032\% & 0.156\% & \textcolor{blue}{\textbf{99.258\%}} & 0.430\% & 0.312\% & 96.968\% & 1.574\% & 1.458\% \\
%  &   &  Free & 0.036\%  & \textcolor{red}{\textbf{99.928\%}} & 0.036\% & 0.554\% & \textcolor{blue}{\textbf{99.352\%}} & 0.094\% & 1.018\% & 98.110\% & 0.872\% \\
%  &   & Occupied  &  4.500\% & 2.358\%  & \textcolor{red}{\textbf{93.142\%}} & 16.788\% & 3.736\% &\textcolor{blue}{\textbf{79.476\%}} &26.420\% & 44.928\% & 28.652\% \\
% \end{NiceTabular}
% % \vspace{-1em}
% \end{table*}

\begin{table*}[htp]
\centering
\caption{Occupancy Grid map Precision of Our Method Using All Frames, Our Method Using Key Frames, and Cartographer.}
\label{tab_map_accuracy}
\setlength{\tabcolsep}{2.4mm}
\renewcommand{\arraystretch}{1.2}

\begin{tabular}{c c c c c c c c c c c c}
\toprule
 \multirow{2}{*}{} & \multirow{2}{*}{\textbf{Ground Truth}}& \multicolumn{3}{c}{Our Method (All Frames)} & \multicolumn{3}{c}{Our Method (Key Frames)} & \multicolumn{3}{c}{Cartographer} \\
\cmidrule(lr){3-5} \cmidrule(lr){6-8} \cmidrule(lr){9-11}
& & Unknown & Free & Occupied & Unknown & Free & Occupied & Unknown & Free & Occupied \\
\midrule
\multirow{3}{*}{Simulation 1} 
& Unknown  & \textcolor{red}{\textbf{99.798\%}}  & 0.020\% & 0.182\% & \textcolor{blue}{\textbf{99.598\%}} & 0.072\% & 0.330\% & 95.616\% & 2.822\% & 1.562\% \\
& Free     & 0.022\%  & \textcolor{red}{\textbf{99.938\%}} & 0.040\% & 0.094\% & \textcolor{blue}{\textbf{99.824\%}} & 0.082\% & 1.290\% & 97.678\% & 1.032\% \\
& Occupied & 5.436\%  & 2.334\%  & \textcolor{red}{\textbf{92.230\%}} & 13.562\% & 3.142\% & \textcolor{blue}{\textbf{83.296\%}} & 30.053\% & 53.280\% & 16.667\% \\
\midrule
\multirow{3}{*}{Simulation 2} 
& Unknown  & \textcolor{red}{\textbf{99.846\%}}  & 0.010\% & 0.144\% & \textcolor{blue}{\textbf{99.696\%}} & 0.062\% & 0.242\% & 96.868\% & 1.743\% & 1.389\% \\
& Free     & 0.016\%  & \textcolor{red}{\textbf{99.846\%}} & 0.138\% & 0.076\% & \textcolor{blue}{\textbf{99.848\%}} & 0.076\% & 0.593\% & 98.584\% & 0.823\% \\
& Occupied & 6.434\%  & 3.738\%  & \textcolor{red}{\textbf{89.828\%}} & 11.694\% & 2.806\% & \textcolor{blue}{\textbf{85.500\%}} & 23.943\% & 50.795\% & 25.262\% \\
\midrule
\multirow{3}{*}{Simulation 3} 
& Unknown  & \textcolor{red}{\textbf{99.812\%}}  & 0.032\% & 0.156\% & \textcolor{blue}{\textbf{99.258\%}} & 0.430\% & 0.312\% & 96.968\% & 1.574\% & 1.458\% \\
& Free     & 0.036\%  & \textcolor{red}{\textbf{99.928\%}} & 0.036\% & 0.554\% & \textcolor{blue}{\textbf{99.352\%}} & 0.094\% & 1.018\% & 98.110\% & 0.872\% \\
& Occupied & 4.500\%  & 2.358\%  & \textcolor{red}{\textbf{93.142\%}} & 16.788\% & 3.736\% & \textcolor{blue}{\textbf{79.476\%}} & 26.420\% & 44.928\% & 28.652\% \\
\bottomrule
\end{tabular}
\end{table*}


From the results of our first stage shown in Fig. \ref{fig_simulation}(c), it is evident that further optimization is needed at the edges of objects. This is due to sampling points with different occupancy values being projected onto the coarse grid cells at object boundaries, causing inaccurate data associations. These results highlight the necessity of the second stage in our multi-resolution strategy (Algorithm \ref{alg_3}) to improve accuracy. Additionally, as shown in Fig. \ref{fig_simulation}(c), the non-edge areas (stable areas) of the occupancy grid maps are well optimized, supporting the fact that including occupancy cell vertices of non-edge areas in the state variables for further optimization is unnecessary.

For a quantitative comparison of the occupancy maps, we apply the same threshold across all methods to convert occupancy values into occupancy states. The mapping problem is treated as a classification task, categorizing each grid cell as free, occupied, or unknown. The mapping performance of our method and Cartographer is summarized in Table \ref{tab_map_accuracy}, clearly showing that both variants of our method, one using all frames and the other using keyframes, significantly outperform Cartographer in terms of map accuracy.

\begin{table}[ht]
		\centering
		\caption{Accuracy of the Occupancy Grid Map.}
		\label{tab_auc}
		\setlength{\tabcolsep}{4.8 mm}{
		\begin{tabular}{llccccc}\toprule
		& & AUC & Precision   \\ \hline
		\multirow{3}{*}{Simulation 1}& Cartographer & 0.90878 & 0.95651 \\ & Ours (All) &\textcolor{red}{\textbf{0.99999}} &\textcolor{red}{\textbf{0.99773}}\\ & Ours (Key) & \textcolor{blue}{\textbf{0.99902}} & \textcolor{blue}{\textbf{0.99548}} \\ \hline 
			\multirow{3}{*}{Simulation 2}& Cartographer & 0.96132 &  0.96829 \\ & Ours (All) & \textcolor{red}{\textbf{0.99926}} & \textcolor{red}{\textbf{0.99721}} \\ & Ours (Key) & \textcolor{blue}{\textbf{0.99914}} & \textcolor{blue}{\textbf{0.99638}}\\ \hline
		\multirow{3}{*}{Simulation 3} & Cartographer & 0.92696 &0.96592 \\ & Ours (All)& \textcolor{red}{\textbf{0.99974}} & \textcolor{red}{\textbf{0.99771}} \\ & Ours (Key) & \textcolor{blue}{\textbf{0.99748}} & \textcolor{blue}{\textbf{0.99113}}\\
		 \hline
		\end{tabular}
  }
  % \vspace{-0.5em}
\end{table}

We also assess performance using AUC (Area under the ROC curve) \cite{bradley1997use} and precision, with ground truth labels generated from the occupancy map based on ground truth poses. To ensure a fair comparison, all unknown cells are excluded from this evaluation, as AUC is a binary classification metric \cite{bradley1997use}. Table \ref{tab_auc} presents the results, showing that our method using all frames achieves the highest performance in both metrics. Even with only key frames, our method surpasses Cartographer. A key factor resulting in Cartographer's lower mapping quality is its lack of a batch optimization method to address errors during submap construction. Although its scan-to-map matching approach reduces cumulative errors more effectively than scan-to-scan matching, its accuracy still falls short compared to our algorithm. Cartographer performs pose graph optimization to adjust the coordinate frames of submaps only when loop closure is detected, leaving errors within the submaps uncorrected. Although global pose graph optimization is applied at the end of the process, it often suffers from an excess of inaccurate and conflicting relative measurements, as well as its susceptibility to local minima, limiting its effectiveness in correcting these errors. Moreover, pose graph optimization typically does not enhance the local details of maps, as it focuses solely on optimizing poses without jointly considering the map. This further highlights the advantage of our approach, which jointly optimizes both robot poses and the occupancy map.


\subsection{Comparisons using Practical Datasets} \label{sec_practical}

\begin{figure}[t]
\centering
\includegraphics[width=0.46\textwidth]{Real_OGM_New.pdf}
\caption{\label{fig_result_compare_OGM} The occupancy grid maps from Cartographer, our method using all frames, and our method using key frames. }
% \vspace{-1.5em}
\end{figure}


\begin{figure}[t]
\centering
\includegraphics[width=0.46\textwidth]{Real_Scan_New.pdf}
\caption{\label{fig_result_compare_scan} The point cloud maps from Cartographer, our method using all frames, and our method using key frames.}
\end{figure}

\begin{table}[t]
		\centering
		\caption{Time Consumption of Different Algorithms.}
		\label{table_time_compare}
		\setlength{\tabcolsep}{2.5 mm}{
		\begin{tabular}{lccc}\toprule
		Dataset& & Computational Time (s)& \\ \hline
			     & Cartographer  & Ours (All) & Ours (Key)  \\ 
		Car Park & 168  & 119  & \textbf{44} \\
		Museum b0 & 152  & 126 & \textbf{38} \\
		C5 & 146 & 137 & \textbf{35} \\
		% Simulation 1 & 192  & 148 & \textbf{33} \\
		% Simulation 2 & 174  & 193 & \textbf{57} \\
		% Simulation 3 & 78  & 132 & \textbf{40} \\
		\hline
		\end{tabular}
		}
        % \vspace{-2em}
\end{table}

We use three normal-scale practical datasets, namely Deutsches Museum b0 \cite{hess2016real}, Car Park \cite{zhao20212d} and C5, to compare our method with Cartographer in terms of the constructed occupancy grid maps and optimized poses. 

The mapping quality is evaluated by comparing the details of the constructed maps. Additionally, point cloud maps, which are generated using the endpoint projections of scan points and optimized poses, serve as a reference for pose accuracy. For the Car Park and C5 datasets, our method is initialized with poses from odometry inputs, whereas for the Museum b0 dataset, initialization relies on poses from scan matching due to the absence of odometry. The occupancy grid maps and point cloud maps generated by Cartographer, our method using all frames, and our method using key frames for the three datasets are shown in Fig. \ref{fig_result_compare_OGM} and Fig. \ref{fig_result_compare_scan}. Red dotted lines highlight areas where our results outperform Cartographer in both the occupancy grid maps and point cloud maps. Comparing Fig. \ref{fig_result_compare_OGM}(a) and Fig. \ref{fig_result_compare_OGM}(b), our method provides more precise boundaries for the occupancy grid maps due to joint optimization of robot poses and the occupancy map. Similarly, the comparison between Fig. \ref{fig_result_compare_scan}(a) and Fig. \ref{fig_result_compare_scan}(b) illustrates that our method achieves more accurate poses. 

Moreover, our method outperforms Cartographer when using only key frames, as evident from the comparison of Fig. \ref{fig_result_compare_OGM}(a) and Fig. \ref{fig_result_compare_scan}(a) with Fig. \ref{fig_result_compare_OGM}(c) and Fig. \ref{fig_result_compare_scan}(c). These results show that, despite Cartographer introducing loop closure detection, it still produces non-negligible pose errors, leading to point clouds that fail to fully overlap observations of the same obstacle at different poses. While the point cloud maps generated by our method also have non-overlapping parts, these areas are significantly smaller compared to those from Cartographer. 

% These experiments demonstrate that both variants of our method reduce pose errors and generate more accurate occupancy grid maps by jointly optimizing robot poses and the occupancy map. 

Additionally, we assess the time consumption of our method and Cartographer on these three datasets. Table \ref{table_time_compare} shows that our method consistently requires less time than Cartographer across all datasets when using all frames and achieves significantly better efficiency when using selected key frames.

Finally, it is worth noting that some well-known public datasets, such as Radish \cite{Radish}, were collected before 2014 with outdated sensors, leading to low-quality data with poor scanning frequency and odometry accuracy. These issues hinder the performance of Cartographer, often requiring meticulous parameter tuning but still yielding suboptimal results. In contrast, our method performs well on these datasets. Although we do not include these comparisons in this paper, we make our results available on our code page\footnote{\url{https://github.com/WANGYINGYU/Occupancy-SLAM}}.





\subsection{Assessment of Robustness to Initial Guess}

% While our method has demonstrated robustness when initialized with odometry inputs or scan matching under reliable sensor conditions in both simulation and real-world experiments, this subsection illustrates its capability for convergence even when initialized with significantly noisy poses. In this subsection, we use all frames for robustness assessment.

While our method has demonstrated robustness when initialized with odometry inputs or scan matching under reliable sensor conditions in both simulation and real-world experiments, this subsection highlights its capability to converge even when initialized with significantly noisy poses. We use all frames in this subsection to assess robustness.

First, we use Simulation 1 dataset to quantitatively evaluate the convergence percentage and the accuracy of optimized poses under different noise levels. We add zero-mean uniformly distributed noises with different bounds to the ground truth of the poses to generate each group of ten sets of initial poses for the experiments to count convergence rates and average errors. Specifically, for noise level 1, the noise for translation is within $[-2$ m, $2$ m$]$ and the noise for rotation is within $[-0.5$ rad, $0.5$ rad$]$; for level 2, $[-4$ m, $4$ m$]$ and $[-1$ rad, $1$ rad$]$; for level 3, $[-6$ m, $6$ m$]$ and $[-1.5$ rad, $1.5$ rad$]$. The poses with different noise levels of Simulation 1 dataset are visualized using the generated occupancy grid maps, as shown in Fig. \ref{fig_OGM_246}. The convergence results are depicted in Table \ref{table_robustness}, showing that our method can $100\%$ converge when initialized with challenging noisy poses of level 1 and level 2. Our method still has a high convergence percentage ($80\%$) when initialized with noisy poses of level 3. Our algorithm using other simulation datasets has similar robustness performance.

\begin{table}[t]
		\centering
		\caption{Robustness to Initialization.}
		\label{table_robustness}
		\setlength{\tabcolsep}{0.9 mm}{
		\begin{tabular}{lccc}\toprule
		\thead{Noise Level} & \thead{Convergence\\ Percentage} & \thead{Average MAE of \\Translation (m)} & \thead{Average MAE of\\ Rotation (rad)}\\ \hline
		Level 1 (2 m, 0.5 rad)  & 100\% & 0.00679 & 0.0005   \\
		Level 2 (4 m, 1 rad)  & 100\% & 0.00682 & 0.0005  \\
		Level 3 (6 m, 1.5 rad)  & 80\% & 0.01742  & 0.0012 \\
		\hline
		\end{tabular}
		}
\end{table}

\begin{figure}[t]
\centering
\includegraphics[width=0.47\textwidth]{./OGM_Level246.pdf}
\caption{\label{fig_OGM_246} Examples of occupancy grid maps generated from poses with different noise levels as shown in Table \ref{table_robustness} using Simulation 1 dataset.}
% \vspace{-1.5em}
\end{figure}

Moreover, for all practical datasets, we additionally add random zero-mean uniform distribution noises ($[-2$ m, $2$ m$]$ for translation and $[-0.5$ rad, $0.5$ rad$]$ for rotation) to the poses obtained from Cartographer as the initial guess. The initial occupancy maps obtained by using the noisy initial poses are shown in Fig. \ref{fig_noise_initial}(a). Fig. \ref{fig_noise_initial}(b) shows the remapped occupancy grid maps using our optimized poses, and Fig. \ref{fig_noise_initial}(c) shows the point cloud maps using our optimized poses. This experiment shows that our approach can converge from initial guesses with significant errors and also generate good results.


\begin{figure}[t]
% \vspace{-5mm}
\centering
\includegraphics[width=0.5\textwidth]{Noise_Initial.pdf}
\caption{\label{fig_noise_initial}The occupancy grid maps and point cloud maps generated using noisy poses for initialization by our approach. (a) and (b) display the remapped occupancy maps generated from the noisy initial poses and our optimized poses, respectively, and (c) shows the point cloud maps created by projecting the endpoints of scans using our optimized poses.}
% \vspace{-2em}
\end{figure}

\subsection{Discussion about the Effectiveness of Different Stages} \label{sec_discuss}


\begin{figure*}[tp]
\centering \subfigure[Simulation 1] {\label{fig_group_error_1}
\includegraphics[width=0.32\textwidth]{./Group_Error_3640.pdf}}
\centering \subfigure[Simulation 2] {\label{fig_group_error_2}
\includegraphics[width=0.32\textwidth]{./Group_Error_3720.pdf}}
\centering \subfigure[Simulation 3] {\label{fig_group_error_3}
\includegraphics[width=0.32\textwidth]{./Group_Error_2680.pdf}}
\caption{Comparison of translation and rotation errors for simulated datasets using three methods: our full method (Algorithm \ref{alg_flowchart}), our Algorithm \ref{alg_1} initialized by Cartographer's poses with a high-resolution map \cite{Zhao-RSS-22}, and our Algorithm \ref{alg_1} initialized by poses obtained from our first stage with a high-resolution map.}
\label{fig_group_error}
\vspace{-1em}
\end{figure*}


In previous sections, we demonstrated the accuracy, robustness, and efficiency of our proposed method. In this section, we discuss the effectiveness of its different parts.


As demonstrated in Table \ref{tab_comparison}, Fig. \ref{fig_simulation}, Fig. \ref{fig_result_compare_OGM}, and Fig. \ref{fig_result_compare_scan}, the accuracy of the poses and the map obtained from our full approach (Algorithm \ref{alg_flowchart}) is much better than those obtained from Cartographer. This confirms the advantage of jointly optimizing both the robot poses and the occupancy map. 


% One may ask, how about performing only Algorithm \ref{alg_1} with a high-resolution map directly? Will the result be even better? To answer this question clearly, in this subsection, we compare our full approach with Algorithm \ref{alg_1} using a high-resolution map. We consider three different initialization: 

% In the following, we refer to these three approaches as  \textit{Algorithm \ref{alg_1} (High, O/S)}, \textit{Algorithm \ref{alg_1} (High, Carto)}, and \textit{Algorithm \ref{alg_1} (High, First)}, respectively.   

One potential question is whether using only Algorithm \ref{alg_1} with a high-resolution map would yield even better results. To investigate this, we compared our full approach with Algorithm \ref{alg_1} using a high-resolution map. We tested three initialization: (1) \textit{Algorithm \ref{alg_1} (High, O/S)}: initialization using odometry inputs or scan matching; (2) \textit{Algorithm \ref{alg_1} (High, Carto)}: initialization using Cartographer's poses (as proposed in our conference paper \cite{Zhao-RSS-22}); and (3) \textit{Algorithm \ref{alg_1} (High, First)}: initialization using the poses obtained by our first stage. 



% First, \textit{Algorithm \ref{alg_1} (High, O/S)} fails to converge on most datasets, while our full method can converge very well, which indicates the improved robustness of our multi-resolution strategy.   

% The comparison of our full method with \textit{Algorithm \ref{alg_1} (High, Carto)} and \textit{Algorithm \ref{alg_1} (High, First)} using all the five groups simulation datasets are shown in Fig. \ref{fig_group_error}. It can be seen that the accuracy of our full method is essentially similar in all groups, while the accuracy of \textit{Algorithm \ref{alg_1} (High, Carto)} varies drastically. This means the approach proposed in our conference paper \cite{Zhao-RSS-22} not only requires an accurate initial value but also generates less accurate poses than our new approach. In addition, by comparing \textit{Algorithm \ref{alg_1} (High, Carto)} with \textit{Algorithm \ref{alg_1} (High, First)}, it also confirms that the poses obtained in our first stage are more accurate than those of Cartographer.

First, \textit{Algorithm \ref{alg_1} (High, O/S)} fails to converge on most datasets, while our full method converges successfully, indicating the improved robustness of our multi-resolution strategy.

The comparison between our full method, \textit{Algorithm \ref{alg_1} (High, Carto)}, and \textit{Algorithm \ref{alg_1} (High, First)} across all five simulation groups is shown in Fig. \ref{fig_group_error}. It can be observed that the accuracy of our full method remains stable across all groups, while the accuracy of \textit{Algorithm \ref{alg_1} (High, Carto)} varies drastically. This suggests that the approach in our conference paper \cite{Zhao-RSS-22} not only requires an accurate initial guess but also produces less accurate poses than our new method. Moreover, comparing \textit{Algorithm \ref{alg_1} (High, Carto)} with \textit{Algorithm \ref{alg_1} (High, First)} further confirms that the poses obtained in our first stage are more accurate than those of Cartographer.



% It is worth discussing that our full method uses the selected high-resolution map for optimization in the second stage, and it can be observed in Fig. \ref{fig_group_error} that the accuracy of our full approach is even higher than the optimization using the full high-resolution map (i.e., \textit{Algorithm \ref{alg_1} (High, First)}) in some experiments. The potential reason is that when the relatively accurate poses and occupancy map are obtained, the dropped \textcolor{red}{cell vertices} and the corresponding observations contain little information. If all cells and corresponding observations are retained for optimization, it may affect the algorithm's ability to obtain the best solution, as all observation terms are assigned uniform weights. Another potential reason is that the smoothing term in (\ref{eq_objective_func}), by spreading the occupancy values to unknown \textcolor{red}{cell vertices}, may introduce errors that could affect the convergence of the optimization algorithm. 

It is also worth noting that our full method utilizes the selected high-resolution map for optimization in the second stage. As shown in Fig. \ref{fig_group_error}, in certain experiments, the accuracy of our full approach surpasses that of the optimization using the full high-resolution map (\textit{Algorithm \ref{alg_1} (High, First)}). A possible explanation is that once relatively accurate poses and occupancy maps are obtained, the dropped cell vertices and corresponding observations contain little information. Retaining all cells and corresponding observations for optimization may prevent the algorithm from finding the optimal solution, as all are observation items given uniform weights. Another reason could be the smoothing term in (\ref{eq_objective_func}), which spreads occupancy values to unknown cell vertices, potentially introducing errors that affect the convergence of the optimization.

In terms of time consumption, our full approach is much more efficient than \textit{Algorithm \ref{alg_1} (High, Carto)}. For instance, \textit{Algorithm \ref{alg_1} (High, Carto)} consumes over 21,000 seconds with the Car Park dataset. In comparison, the time consumption of our full approach using all frames is 119 seconds (less than 0.6\%), and using only key frames, it takes only 44 seconds (about 0.2\%). This substantial reduction in time consumption highlights the efficiency improvements of our method over our conference paper \cite{Zhao-RSS-22}.

% In terms of time consumption, our full approach is significantly more efficient than \textit{Algorithm \ref{alg_1} (High, Carto)}. For instance, \textit{Algorithm \ref{alg_1} (High, Carto)} consumes over 21,000 seconds when using the Car Park dataset. In comparison, the time consumption of our full approach using all frames is 119 seconds (less than 0.6\%), and using only key frames, it takes 44 seconds (approximately 0.2\%). This substantial reduction in time consumption underscores the significant efficiency improvements of our current method over our conference paper \cite{Zhao-RSS-22}.

The reduction in time consumption stems from both the multi-resolution strategy, which reduces time per iteration, and the fewer iterations needed in the second stage due to the selected high-resolution map. Our experiments show that only about two iterations are needed in the second stage with the selected high-resolution map, fewer than in \textit{Algorithm \ref{alg_1} (High, First)}. This is likely because the selected high-resolution map focuses on critical states, with observations containing the most relevant information, enabling faster convergence.


In summary, compared to our conference paper \cite{Zhao-RSS-22}, our new multi-resolution method does not require precise initialization, is far more efficient, and achieves higher accuracy.

% In addition, compared with \textit{Algorithm \ref{alg_1} (High, Carto)}, the time consumption of our full approach is reduced by $2-3$ orders of magnitude. For example, \textit{Algorithm \ref{alg_1} (High, Carto)} consumes more than $21,000$ seconds using the Car Park dataset. Compared to \textit{Algorithm \ref{alg_1} (High, Carto)}, the time consumption of our full approach using all frames is $119$ seconds (less than $0.6\%$), and the time consumption of our full approach using only key frames is 44 seconds which is approximately $0.2\%$. The significant reduction in time consumption shows the significantly improved efficiency of our current method over our conference paper \cite{Zhao-RSS-22}.

% The substantial reduction in time consumption of our full approach is attributed not only to the introduced multi-resolution strategy, which reduces the time consumption per iteration but also to the reduction in the number of iterations in the second stage, which is a result of utilizing the selected high-resolution map. Through the experiments, we find that only about $2$ iterations in the second stage are required to converge using the selected high-resolution map, which is smaller than the number of iterations needed in \textit{Algorithm \ref{alg_1} (High, First)}. The possible reason is that, in the case of using the selected high-resolution map, these selected \textcolor{red}{cell vertices} focus on the most critical states, and the corresponding observations contain the most important information, allowing the optimization problem to converge much faster.

% In summary, as compared with our conference paper \cite{Zhao-RSS-22}, our new multi-resolution method does not require accurate initialization, is much more efficient, and achieves a higher level of accuracy in most cases.   


\subsection{Ablation Study on the Resolution Ratio}

\begin{table}[t]
		\centering
		\caption{Impact of First-Stage Resolution Settings.}
		\label{table_ablation}
		\setlength{\tabcolsep}{1mm}{
		\begin{tabular}{llcccc}\toprule
		& & $r=20$ & $r=10$  & $r=5$ & $r=2$ \\   \hline   

		 \multirow{5}{*}{Simulation 1} & MAE/Trans (m) First& 0.02352  &  0.02206 & \textbf{0.02118} & 0.17318\\
		\quad & MAE/Rot (rad) First& 0.00116  &  \textbf{0.00098} & 0.00124 & 0.01066\\
		\quad & MAE/Trans (m) All & 0.00812  & \textbf{0.00640}  & 0.00728 & 0.16066\\
		\quad & MAE/Rot (rad) All&  0.00062 & 0.00060  & \textbf{0.00054} & 0.01008\\ 
		\quad & Total Time (s) & \textbf{118}  &  148 & 262 & 2183\\
		\hline

		\multirow{5}{*}{Simulation 2} & MAE/Trans (m) First & 0.03938 & 0.03224 & \textbf{0.01984} & 0.09160\\
		\quad & MAE/Rot (rad) First&  0.00332 & 0.00220  & \textbf{0.00108} & 0.00314\\
		\quad & MAE/Trans (m) All& 0.01742  & 0.00858  & \textbf{0.00584} & 0.08018\\
		\quad & MAE/Rot (rad) All& 0.00064  & 0.00062  & \textbf{0.00052} & 0.00286\\ 
		\quad & Total Time (s)& \textbf{149}  & 193  & 321 & 2685\\
		\hline

		\multirow{5}{*}{Simulation 3} & MAE/Trans (m) First& 0.06708  &  0.02624  & \textbf{0.01776} & 0.03570\\
		\quad & MAE/Rot (rad) First&  0.00384 & 0.00164  & \textbf{0.00124}  & 0.00278\\
		\quad & MAE/Trans (m) All& 0.01586  & \textbf{0.00726}  & 0.00816 & 0.02082\\
		\quad & MAE/Rot (rad) All& 0.00100  & \textbf{0.00058}   & 0.00068 & 0.00102\\ 
		\quad & Total Time (s)& \textbf{125}  &  132 & 185 & 1041\\
		\hline
		\end{tabular}
		}
        % \vspace{-1.5em}
\end{table}

In this section, we perform ablation experiments on simulation datasets to analyze the impact of varying resolution settings in the first stage of the multi-resolution strategy on overall optimization performance.

We assess accuracy and computational time using three simulation datasets, with the resolution in the second stage fixed at $s^{h} = 0.05$ m. The resolution ratios $r$ between the first and second stages are set to 2, 5, 10, and 20, respectively. To ensure consistency, a fixed selection range of $d=1.5$ m is applied uniformly across all datasets. 

  
The results, shown in Table \ref{table_ablation}, reveal that $r=10$ achieves the best trade-off between time consumption and accuracy. While $r=20$ minimizes time consumption, it reduces the accuracy of poses in the first stage, adversely impacting final optimization accuracy. Conversely, $r=5$ improves pose accuracy in the first stage at the cost of higher time consumption but does not consistently enhance final accuracy. Notably, $r$ may need adjustment for other high-resolution settings.


\subsection{Using Submap Joining in Large-scale Environments}

We have demonstrated that our approach accurately and robustly handles normal-scale simulated and practical environments. In this section, we evaluate its efficiency and effectiveness in large-scale environments and long-term trajectories by integrating our Occupancy-SLAM algorithm with the proposed occupancy submap joining approach. The dataset is divided into multiple segments, where Algorithm \ref{alg_flowchart} is used to construct submaps, followed by applying the submap joining method in Section \ref{Sec_submap} to generate the optimized global occupancy map and robot trajectory.

We validate our method on two large-scale datasets, Deutsches Museum b2 \cite{hess2016real} and C3, and compare it with Cartographer. As shown in Fig. \ref{fig_large_environment}, our occupancy grid maps outperform those of Cartographer, demonstrating the capability of our method to handle large-scale environments and long-term trajectories effectively. 

% The datasets have map sizes of 250 m $\times$ 200 m and 150 m $\times$ 125 m, containing 51833 and 24402 scans, with trajectory lengths of 1390 seconds and 610 seconds, respectively.

\begin{figure}[t]
\centering \subfigure[b2] { \label{fig_large_b2}
\includegraphics[width=0.253\textwidth]{./b2_Large_New.pdf}}\hspace{-0.4em}
\centering \subfigure[C3] {\label{fig_large_C3}
\includegraphics[width=0.2195\textwidth]{./C3_Large_New_1.pdf}}
\caption{\label{fig_large_environment} Comparison of results between our method and Cartographer on two large-scale practical datasets. The first row shows Cartographer's results, while the second row shows ours. In (b), the red dotted lines serve as references, highlighting that Cartographer's right wall appears more curved, whereas our result aligns more closely with a straight line.}
% \vspace{-1.5em}
\end{figure}


\subsection{Computational Complexity Analysis}
In this section, we analyze the computational complexity and evaluate the time consumption of our method using large-scale datasets.

The Gauss-Newton method for solving the joint optimization of local maps and poses in (\ref{Least Squares}) and submap joining in (\ref{eq_NLLS_joining}) primarily depends on calculating Jacobian $\mathbf{J}$, and constructing and solving the sparse linear system in (\ref{Gauss-Newton}) \cite{konolige2008frameslam}. We analyze each part's complexity separately due to differences in the NLLS formulation.

For the local map and poses joint optimization problem, the objective function consists of the observation term, the odometry term, and the smoothing term. Let ${\lambda(\mathbb{S})}$ denotes the number of sampling points $\mathbb{S}$, then the number of items in the objective function is $\mathfrak{d}_{row} =\lambda(\mathbb{S})+3(n-1)+2{c_w}{c_h}+c_w+c_h$, and the state vector size is $\mathfrak{d}_{col}=3n+(c_w+1)(c_h+1)$. Considering Jacobian of the smoothing term $\mathbf{J}_S$ can be pre-calculated before optimization, the number of non-zero elements of Jacobian matrix that need to be computed for each iteration is $\mathfrak{d}_J = 7\lambda(\mathbb{S}) + 6(n-1)$. Therefore, for each iteration, the computation complexity of Jacobian calculation, constructing (\ref{Gauss-Newton}) and solving (\ref{Gauss-Newton}) is $\mathcal{O}(\mathfrak{d}_J)$, $\mathcal{O}(\mathfrak{d}_{J}\mathfrak{d}_{col})$, and $\mathcal{O}({\mathfrak{d}^3_{col}})$, respectively. Therefore, the total computation complexity per iteration for the local map and poses joint optimization problem is $\mathcal{O}(\mathfrak{d}_{J}+\mathfrak{d}_J{\mathfrak{d}_{col}}+\mathfrak{d}^3_{col})$. Due to our proposed multi-resolution joint optimization strategy and keyframe selection, both $\mathfrak{d}_{J}$ and $\mathfrak{d}_{col}$ remain small during the first and second stages of optimization, making the computation time for this part manageable.

% (i.e., \textcolor{red}{cell vertices} with non-zero occupancy values) Similar to the computation complexity of the local map and poses joint optimization problem,

For our submap joining algorithm, the number of observations depends on the total number of cell vertices of the global occupancy map observed in each submap, denoted $\mathfrak{d}_{obs}^{G}$. Considering that some cell vertices will be observed repeatedly under different submaps, this number slightly exceeds the number of non-unknown cell vertices in the global map. Thus, the number of non-zero elements of Jacobian matrix is $\mathfrak{d}_J^G = 4\mathfrak{d}_{obs}^G$, and the state vector size is $\mathfrak{d}_{col}^{G} = 3n_L+(c_w^G+1)(c_h^G+1)$. The computation complexity per iteration is $\mathcal{O}(\mathfrak{d}_{J}^G+\mathfrak{d}_J^G{\mathfrak{d}_{col}^G}+{\mathfrak{d}_{col}^G}^3)$.
Although the global occupancy map tends to be relatively large, the sub-matrix of Hessian w.r.t. the global occupancy map is diagonal. To speed up computation, we apply the Schur complement \cite{zhang2006schur} to make the normal equation solving highly efficient.

Finally, we evaluate the time consumption of our method using both all frames and selected keyframes in large-scale environments to support our computation complexity analysis and compare it to Cartographer. For Museum b2 dataset, Cartographer takes 1424 seconds, while our method takes 1250 seconds when using all frames and 363 seconds with selected key frames. For C3 dataset, Cartographer takes 610 seconds, while our method takes 742 seconds with all frames and 236 seconds with selected key frames. In our total time consumption, the submap joining method consumes less than 10 seconds on both datasets. It can be seen that the time consumption of our method is comparable to that of Cartographer when all frames are used and much lower than that of Cartographer when selected key frames are used. These results demonstrate the efficiency of our multi-resolution joint optimization strategy and submap joining approach. 

\section{Preliminary Results in 3D Case}\label{sec_3d}

While this paper primarily focuses on demonstrating the benefits of jointly optimizing the robot poses and occupancy map in 2D, we also present some preliminary 3D results to illustrate that our idea can be extended to 3D applications.



\subsection{Extension of the Algorithms to 3D Case}

Our approach for jointly optimizing robot poses and the occupancy map extends naturally to 3D, where the information, robot poses, and occupancy maps are all represented in 3D. Most problem formulations and algorithms can be adapted with minor adjustments. 

For our local map and poses optimization method, observations transition from 2D laser scans to 3D LiDAR scans, robot poses and odometry involve 6 degree-of-freedom (DoF), and the map representation becomes 3D. Consequently, (\ref{eq_interp}) and (\ref{eq_NP}) need to be replaced from bilinear to trilinear interpolation and its inverse operation. 
For the objective function (\ref{eq_objective_func}), the odometry term (\ref{eq_odometry_term}) should be replaced with a 6 DoF odometry term for 3D, and the smoothing term (\ref{eq_smoothing_term}) should include a smoothing penalty for the z-axis, Jacobians $\mathbf{J}_P$, $\mathbf{J}_M$, $\mathbf{J}_O$, and $\mathbf{J}_S$ described in Appendices need to be adjusted accordingly. 

The submap joining problem in 3D remains largely similar to the 2D case, except that the projection relation extends from 2D-2D to 3D-3D, enabling the solution of 6 DoF poses and the 3D global occupancy map in the NLLS problem (\ref{eq_NLLS_joining}).

\subsection{3D Experimental Results}
\subsubsection{Evaluation metrics and state-of-the-art methods}
We evaluate our method's performance in 3D using absolute trajectory error for poses, aligning and comparing results with ground truth via EVO \cite{grupp2017evo}, as used in \cite{liu2023large,rosinol2021kimera}. In all the experiments, we use the odometry information provided by the dataset as initialization if it is available. Otherwise, we use FAST-LIO2 \cite{xu2022fast} to obtain the odometry information. To evaluate our method, we compare our method against state-of-the-art methods: BALM2 \cite{liu2023efficient}, HBA \cite{liu2023large}, and Voxgraph \cite{reijgwart2019voxgraph}. BALM2 optimizes the planar feature parameters of the point cloud and the robot's poses. HBA proposes a hierarchical bundle adjustment to optimize the consistency of the planar surfaces of point clouds and robot poses. Voxgraph builds SDF-based submaps from point clouds, uses SDF-to-SDF registration for relative submap measurements, and incrementally optimizes submap frames. HBA and Voxgraph can deal with large-scale environments, while BALM2 focuses on normal-scale environments.  

\subsubsection{Datasets}
We perform comparisons using three real-world datasets. (1) The Newer College Dataset \cite{ramezani2020newer}: The first five sequences from the \textit{shorter experiment}, collected with a handheld Ouster OS-1 LiDAR scanner at New College, Oxford. The environment includes lawns, buildings, a tunnel, and a garden. Ground truth is provided by a BLK360 LiDAR scanner to capture a detailed 3D map and then infer the ground truth of poses with centimeter-level accuracy.
(2) KITTI Dataset \cite{Geiger2013IJRR}: Sequence 07, a demo dataset for HBA, collected with a Velodyne HDL-64E LiDAR scanner mounted on a car. Ground truth poses is provided by RTK-GPS/INS.
(3) Arche Dataset \cite{reijgwart2019voxgraph}: A demo dataset for Voxgraph, collected using an Ouster OS1 LiDAR mounted on a hexacopter MAV in a disaster area. Ground truth positions are provided by an RTK-GNSS system. 

The Newer College Dataset is used to evaluate high-precision performance in normal-scale environments, while the KITTI and Arche datasets are used to test performance in large environments with long trajectories.

\begin{figure*}[t]
\centering
\includegraphics[width=0.99\textwidth]{./PC_Comparison.pdf}
\caption{\label{fig_3d_pointcloud} Some local point cloud maps from the Arche dataset. The first row shows point cloud maps generated using odometry from ROVIO (also used for submap construction in Voxgraph), while the second row shows maps generated with our optimized poses using the same LiDAR scans. BALM2 fails to produce results when using the same odometry and scans as inputs in all these local environments.}
\vspace{-1em}
\end{figure*}
 
\subsubsection{Experiments on normal-scale environments}
We evaluate the performance of our proposed method without submap joining in normal-scale environments.

First, we evaluate BALM2 and our method using the first five sequences of The Newer College Dataset, which encompass all scenarios within the dataset. As shown in Table \ref{tab_comparison_3d_local}, our method outperforms BALM2 across all metrics, except for the RMSE in Seq. 1, and significantly outperforms the odometry inputs from FAST-LIO2 in all metrics. 

\begin{table}[t]
		\centering
		\caption{Absolute Trajectory Error (MAE/RMSE, Meters) in Normal-scale Environments for Different 3D Methods.}
		\label{tab_comparison_3d_local}
		\setlength{\tabcolsep}{0.6 mm}{
		\begin{tabular}{lcccccc}\toprule
		Method	& Seq. 0 & Seq. 1 & Seq. 2 & Seq. 3 & Seq. 4 \\ \hline
		FAST-LIO2 & 0.518/0.717 & 0.181/0.202  & 0.121/0.132 &0.188/0.200&0.571/0.723\\
		BALM2 & 0.283/0.326 & 0.112/\textbf{0.123}  & 0.104/0.109 &0.144/0.158 &0.298/0.344 \\
        Ours & \textbf{0.185}/\textbf{0.232}  & \textbf{0.097}/\textbf{0.123}  & \textbf{0.091}/\textbf{0.099} & \textbf{0.141}/\textbf{0.155} &\textbf{0.238}/\textbf{0.284}\\ \hline
		\end{tabular}
		}
        % \vspace{-2em}
\end{table}

Next, we test robustness in a challenging environment with noisy odometry input using the Arche dataset. This dataset, collected by a hexacopter MAV in an unstructured environment, is influenced by drone vibrations, flight speed, and environmental factors. Local point cloud maps built using odometry from ROVIO \cite{bloesch2017iterated} (also used to construct submaps in Voxgraph) are shown in the first row of Fig. \ref{fig_3d_pointcloud}. To evaluate BALM2 and our method, we partition the dataset into several short sequences, each lasting 10–20 seconds. BALM2 fails in all the sequences except during MAV start-up and landing due to insufficient planar features for optimization, while our method performs well on all the sequences. The second row of Fig. \ref{fig_3d_pointcloud} illustrates some of our results, demonstrating that our method is robust in 3D and does not rely on environmental assumptions. Additionally, the results confirm that our method achieves significantly higher pose accuracy than ROVIO.

\subsubsection{Experiments on large-scale environments}\label{sec_experiment_vox} 

We evaluate our method in large-scale environments with long trajectories using the KITTI and Arche datasets, comparing it with HBA and Voxgraph. For this experiment, we first build submaps by jointly optimizing poses and maps within submaps, then apply our submap joining algorithm to jointly optimize submap frame poses and the global occupancy map.

% The MAE and RMSE of the ATEs are summarized in Table \ref{tab_comparison_3d_large}, it is clear that our method achieves the best results on both datasets. In addition, the robot trajectories are shown in Fig. \ref{fig_trajectory_3d}, as it shows our method can achieve the best global consistent robot trajectories compared with other methods. It should be noted that the results of our method substantially lead Voxgraph on the KITTI dataset and significantly outperform HBA on the Arche dataset. The reason our method performs much better than Voxgraph on KITTI dataset is that Voxgraph relies on relative measurements from SDF-to-SDF registration for solving pose graph optimization, but in such autonomous driving scenarios, it is difficult to provide sufficient overlapping submaps for Voxgraph to calculate relative measurements between submaps. However, our proposed submap joining algorithm jointly optimizes poses of submaps' coordinate frames and the global occupancy map and, therefore, does not suffer in such environments. The performance of HBA on the Arche dataset is affected by highly unstructured environments and with data captured by moving MAV, as HBA relies on detecting and using planar features from the point cloud to do the optimization, which is similar to BALM2. However, in the case of odometry and point clouds collected during MAV motion, it is difficult for such methods to detect a sufficient number of good planar features. In addition, the planarity assumption does not tend to hold true in non-urban environments, such as the field.

Table \ref{tab_comparison_3d_large} summarizes the MAE and RMSE of absolute trajectory error, showing our method achieves the best results on both datasets. Fig. \ref{fig_trajectory_3d} illustrates that our approach can achieve the best global robot trajectories. Notably, our method significantly outperforms Voxgraph on the KITTI dataset and HBA on the Arche dataset. The relatively poor performance of Voxgraph on the KITTI dataset is due to its reliance on relative measurements from SDF-to-SDF registration, which requires sufficient overlapping submaps—a challenge in autonomous driving scenarios. In contrast, our submap joining algorithm jointly optimizes submap poses and the global occupancy map, avoiding this limitation. HBA underperforms on the Arche dataset due to its reliance on planar features for optimization, which is challenging in unstructured environments and during MAV motion. Odometry and point clouds from such scenarios make detecting sufficient planar features difficult, and the planarity assumption often fails in non-urban environments like disaster areas.


\begin{figure}[tbp]
\centering \subfigure[KITTI] {\label{fig_trajectory_1}
\includegraphics[height=0.15\textwidth]{./Traj_KITTI.pdf}}
\centering \subfigure[Arche] {\label{fig_trajectory_2}
\includegraphics[height=0.15\textwidth]{./Traj_Voxgraph_Demo_New.pdf}}
\caption{\label{fig_trajectory_3d} Robot trajectory results of datasets in large-scale environments. (a) and (b) show the trajectories of ground truth, Voxgraph \cite{reijgwart2019voxgraph}, HBA \cite{liu2023large}, and our method for KITTI dataset and Arche dataset.}
\end{figure}


\begin{table}[t]
		\centering
		\caption{Absolute Trajectory Error (MAE/RMSE, Meters) in Large-scale Environments for Different 3D Methods}
		\label{tab_comparison_3d_large}
		\setlength{\tabcolsep}{7mm}{
		\begin{tabular}{lcc}\toprule
		Method & KITTI & Arche  \\ \hline
		HBA \cite{liu2023large} & 0.342/0.364 & 4.123/4.789  \\
        Voxgraph \cite{reijgwart2019voxgraph} &0.926/1.002 & 0.700/0.833 \\
        Ours & \textbf{0.315}/\textbf{0.339}  & \textbf{0.275}/\textbf{0.378} \\ \hline
		\end{tabular}
		}
        % \vspace{-2em}
\end{table}

\subsection{Discussion}
The experimental results in this section demonstrate that our proposed idea of jointly optimizing the robot pose and the occupancy map can also lead to better solutions for the robot poses and occupancy maps in 3D cases. However, several challenges remain in 3D scenarios. For instance, 3D point clouds from LiDAR scanners are often sparse, particularly in the vertical direction, which can lead to observability issues in the optimization problem. This sparsity also results in inhomogeneous observation information, complicating the accurate representation of the 3D environment in occupancy maps. Furthermore, the large dimensions of 3D maps pose significant computational challenges in large-scale SLAM, requiring more efficient solving methods.


To address these challenges, several potential solutions can be explored. First, adopting compact representations for 3D occupancy maps, such as octree structures similar to Octomap \cite{hornung2013octomap} and \cite{vespa2019adaptive}, can enhance efficiency. Second, combining local map and pose optimization with hierarchical optimization and submap joining methods can further reduce computational time. Lastly, using continuous representations for 3D occupancy maps enables more precise gradient calculations, which can better guide the optimization process.

\section{Conclusion} \label{Sec_conclusion}
In this paper, we propose Occupancy-SLAM algorithm, which solves robot poses and occupancy map simultaneously. To enhance efficiency and robustness, we introduce a multi-resolution strategy. The first stage jointly optimizes poses and a low-resolution occupancy map to quickly achieve relatively accurate pose estimates, which are then used as the initial guess for the second stage. The second stage refines poses and a subset of the high-resolution map, focusing on critical boundary areas. Additionally, we extend this framework to an occupancy grid-based submap joining algorithm, addressing challenges in large-scale environments and long-term trajectories. Results from both simulated and real-world datasets demonstrate that our method achieves more accurate pose and map estimates than state-of-the-art approaches. 

   
Our findings show that solving poses and occupancy maps simultaneously yields more accurate results compared to first solving pose-graph SLAM and then constructing the map. This joint optimization approach has the potential to revolutionize occupancy map based SLAM frameworks.

The proposed method acts as a batch optimization approach for obtaining high-quality robot poses and maps. Unlike incremental or online methods, batch optimization provides greater accuracy, which is particularly advantageous for applications requiring high-quality maps rather than real-time operation (e.g., offline map creation for precise future localization). Despite typical drawbacks of batch optimization, such as higher computational costs, trajectory-length-dependent complexity, and reliance on accurate initial guesses, our method effectively overcomes these limitations: 1) our method is efficient due to the proposed multi-resolution joint optimization strategy, and the computation time is comparable to online methods; 2) our method can use selected keyframes to further reduce the computational cost without losing too much accuracy; 3) our proposed occupancy submap joining approach can overcome the limitation that the computational complexity related to the length of the robot trajectories; and 4) our method is very robust to the initial guess and can be initialized from odometry inputs or scan matching, so it does not require initialization from the result of incremental/online methods.    

In our future work, we will further explore problem formulation and solution techniques in the 3D case to develop more efficient and robust algorithms capable of addressing various challenges in 3D environments. 


%\section*{Acknowledgments}
%This should be a simple paragraph before the References to thank those individuals and institutions who have supported your work on this article.


{\appendix


The Jacobian $\mathbf{J}$ in (\ref{Gauss-Newton}) consists of four parts, i.e. the Jacobian of the observation term w.r.t. the robot poses $\mathbf{J}_P$ (See Appendix \ref{Sec_J_P}), the Jacobian of the observation term w.r.t. the occupancy map $\mathbf{J}_M$ (See Appendix \ref{Sec_J_D}), the Jacobian of the odometry term w.r.t. robot poses $\mathbf{J}_O$ (See Appendix \ref{Sec_J_O}) and the Jacobian of the smoothing term w.r.t. the occupancy map $\mathbf{J}_S$ (See Appendix \ref{Sec_J_S}). In addition, the difference in the calculation of Jacobians between Algorithm \ref{alg_1} and Algorithm \ref{alg_3} is shown in Appendix \ref{Sec_J_Select}. 

\subsection{Jacobian of the Observation Term w.r.t. Robot Poses}\label{Sec_J_P}

The Jacobian $\mathbf{J}_P$ of function $F_{ij}^Z(\mathbf{x})$ in the observation term w.r.t. the robot poses $\mathbf{x}^P_i$ can be calculated by the chain rule
\begin{equation}
	\begin{aligned}
		\mathbf{J}_P=\frac{ \partial F_{ij}^Z(\mathbf{x}) }{ \partial \mathbf{x}^P_i } = \frac{\partial F_{ij}^Z(\mathbf{x}) }{ \partial \mathbf{p}_{ij} } \cdot \frac{\partial \mathbf{p}_{ij}  }{ \partial \mathbf{x}^P_i}	
	\end{aligned}
\end{equation}
in which $\dfrac{\partial \mathbf{p}_{ij}  }{ \partial \mathbf{x}^P_i}$ can be calculated as
\begin{equation}
\dfrac{\partial \mathbf{p}_{ij}}{\partial \mathbf{x}^P_i}=\left[\begin{array}{ll}
\dfrac{\partial \mathbf{p}_{ij}}{\partial \mathbf{t}_i} & \dfrac{\partial \mathbf{p}_{ij}}{\partial \theta_i}
\end{array}\right]=\dfrac{1}{s} \left[\begin{array}{ll}
\mathbf{E}_{2} & \left(\mathbf{R}_i^{\prime}\right)^{\top} \mathbf{p}_{ij}
\end{array}\right].
\end{equation}
$\mathbf{R}_i^\prime$ is the derivative of the rotation matrix $\mathbf{R}_i$ w.r.t. rotation angle $\theta_i$ and $\mathbf{E}_2$ means $2 \times 2$ identity matrix.

$\dfrac{\partial F_{ij}^Z(\mathbf{x}) }{ \partial \mathbf{p}_{ij} }$ can be calculated by
\begin{equation}
\dfrac{\partial F_{ij}^Z(\mathbf{x}) }{ \partial \mathbf{p}_{ij} } = \dfrac{1}{N(\mathbf{p}_{ij})} \dfrac{\partial M(\mathbf{p}_{ij})}{\partial \mathbf{p}_{ij}}.
\end{equation}
Here $\dfrac{\partial M(\mathbf{p}_{ij})}{\partial \mathbf{p}_{ij}}$ can be considered as the gradient of the occupancy map at point $\mathbf{p}_{ij}$, which can be approximated by the bilinear interpolation of the gradients of the occupancy at the four adjacent cell vertices $\mathbf{\nabla} M(\mathbf{m}_{wh}),\cdots,\mathbf{\nabla} M(\mathbf{m}_{(w+1)(h+1)})$ around $\mathbf{p}_{ij}$ as
\begin{equation} 
\dfrac{\partial M(\mathbf{p}_{ij})}{\partial \mathbf{p}_{ij}}= 
\left[
\begin{aligned}
a_1b_1\\a_0b_1\\a_1b_0\\a_0b_0\\
\end{aligned}\right]^\top
\left[
\begin{aligned}
&\mathbf{\nabla} M(\mathbf{m}_{wh})\\&\mathbf{\nabla} M(\mathbf{m}_{(w+1)h})\\&\mathbf{\nabla} M(\mathbf{m}_{w(h+1)})\\&\mathbf{\nabla} M(\mathbf{m}_{(w+1)(h+1)})
\end{aligned}\right]\label{eq_14}
\end{equation} 
where the gradient of occupancy map $\mathbb{M}$ at all the cell vertices $\mathbf{\nabla} M$ can be easily calculated from $\mathbf{x}^M$ in the state. The bilinear interpolation used in (\ref{eq_14}) is similar to the method in (\ref{eq_interp}).

Here, we assume the robot poses $\mathbf{x}^P$ change slightly in each iteration, to reduce the computational complexity, the hit map $\mathbb{N}$ is considered as constant and recalculated using the current robot poses in each iteration. Thus, the derivative of $N(\mathbf{p}_{ij})$ is not calculated. 

\subsection{Jacobian of the Observation Term w.r.t. Occupancy Map}\label{Sec_J_D}
Based on (\ref{eq_interp}), the Jacobian $\mathbf{J}_M$ of function $F_{ij}^Z(\mathbf{x})$ in the observation term w.r.t. the map part of state vector $\mathbf{x}^{M}$ can be calculated as

\begin{equation}
\begin{aligned}
\mathbf{J}_M & = \dfrac{\partial F_{ij}^Z(\mathbf{x})}{\partial \left[ {M}(\mathbf{m}_{wh}),\cdots, {M}(\mathbf{m}_{(w+1)(h+1)}) \right]^\top}\\
&= \dfrac{1}{N(\mathbf{p}_{ij})}\dfrac{\partial M(\mathbf{p}_{ij})}{\partial \left[ {M}(\mathbf{m}_{wh}),\cdots, {M}(\mathbf{m}_{(w+1)(h+1)}) \right]^\top}\\ 
&= \dfrac{\begin{bmatrix}
a_1b_1,a_0b_1,a_1b_0,a_0b_0
\end{bmatrix}}{N(\mathbf{p}_{ij})}
\end{aligned}
\end{equation}
where $\mathbf{m}_{wh}, \cdots, \mathbf{m}_{(w+1)(h+1)}$ are the four nearest cell vertices to $\mathbf{p}_{ij}$ in occupancy map $\mathbb{M}$, and $a_0,a_1,b_0$ and $b_1$ are defined in (\ref{eq_interp}).


\subsection{Jacobian of the Odometry Term}\label{Sec_J_O}
The Jacobian $\mathbf{J}_O$ of function $F_i^O(\mathbf{x})$ in the odometry term (\ref{eq_odometry_term}) is the partial derivative w.r.t. the robot poses $\mathbf{x}^P$ since it is not related to the occupancy map in the state vector $\mathbf{x}$. Therefore, the Jacobian $\mathbf{J}_O$ can be calculated as
\begin{equation}
\begin{aligned}
\mathbf{J}_O &= \frac{\partial F_i^O(\mathbf{x})}{\partial \left[ {\mathbf{x}^P_{i-1}}^\top, {\mathbf{x}^P_i}^\top \right]^\top }\\ 
&=\begin{bmatrix}
	 \dfrac{\partial F_i^O(\mathbf{x})}{\partial \mathbf{t}_{i-1}} &
	 \dfrac{\partial F_i^O(\mathbf{x})}{\partial \theta_{i-1}} &
	 \dfrac{\partial F_i^O(\mathbf{x})}{\partial \mathbf{t}_i} &
	 \dfrac{\partial F_i^O(\mathbf{x})}{\partial \theta_i}
 \end{bmatrix} 
 \\
 &=\begin{bmatrix}
 	-\mathbf{R}_{i-1} & \mathbf{R}_{i-1}^\prime(\mathbf{t}_i-\mathbf{t}_{i-1}) & \mathbf{R}_{i-1} &\mathbf{0}_2\\
 	\mathbf{0}_2^\top & -1 & \mathbf{0}_2^\top & 1\\
 \end{bmatrix}
\end{aligned}
 \end{equation}
in which $\mathbf{0}_2$ means $2 \times 1$ zero vector.
 
\subsection{Jacobian of the Smoothing Term}\label{Sec_J_S}

The Jacobian $\mathbf{J}_S$ of function $F^S(\mathbf{x})$ in the smoothing term is the derivative of (\ref{eq_smoothing_term}) w.r.t. cell vertices of occupancy map $\mathbf{x}^M$ 
due to it is not related to the robot poses $\mathbf{x}^P$ in the state vector $\mathbf{x}$. It should be mentioned that $F^S(\mathbf{x})$ is linear w.r.t. $\mathbf{x}^M$
\begin{equation}
F^S(\mathbf{x}) = \mathbf{A} \left[ {M}(\mathbf{m}_{00}),\cdots,{M}(\mathbf{m}_{c_wc_h}) \right]^\top
\end{equation}
where the $(2c_wc_h+c_w+c_h) \times ((c_w+1)(c_h+1))$ coefficient matrix $\mathbf{A}$ is sparse and with nonzero elements $1$ or $-1$. An example of the coefficient matrix can be shown as
\begin{equation}\label{eq_A}
	\mathbf{A} = \begin{bmatrix}
    \vdots &\vdots  &\vdots  &\vdots  &\vdots  &\vdots  &\vdots &\vdots\\
 	\mathbf{0}^\top & 1 & -1 & 0 & \mathbf{0}^\top & 0 & 0 & \mathbf{0}^\top\\
 	\mathbf{0}^\top & 1 & 0  & 0 & \mathbf{0}^\top & -1 & 0 & \mathbf{0}^\top\\
 	\mathbf{0}^\top & 0 & 1 & -1 & \mathbf{0}^\top & 0 & 0 & \mathbf{0}^\top\\
 	\mathbf{0}^\top & 0 & 1 & 0 & \mathbf{0}^\top & 0 & -1 & \mathbf{0}^\top\\
    \vdots &\vdots  &\vdots  &\vdots  &\vdots  &\vdots  &\vdots &\vdots\\
 \end{bmatrix}.
\end{equation}
Here $\mathbf{0}$ represents a zero vector with appropriate dimensions. Therefore, the Jacobian of the smoothing term can be calculated as
\begin{equation}\label{eq_JS}
\mathbf{J}_S = \frac{\partial F^S(\mathbf{x})}{\partial \mathbf{x}^M } = \mathbf{A}.\\ 
\end{equation}
Since $\mathbf{A}$ is constant, $\mathbf{J}_S$ can be pre-calculated and directly used in the optimization as shown in Algorithm \ref{alg_1}.

\subsection{Jacobians in the Second Stage of Multi-resolution Strategy for Optimization}\label{Sec_J_Select}
In the second stage of the multi-resolution strategy (Algorithm \ref{alg_3}), the Jacobians to be calculated are similar to those in Algorithm \ref{alg_1}. A specific challenge arises in handling the selected cell vertices adjacent to the dropped cell vertices in the high-resolution map, particularly when calculating Jacobians $\mathbf{J}_P$ and $\mathbf{J}_S$.

 For Jacobian $\mathbf{J}_P$, partial derivatives w.r.t. all the cell vertices are required for (\ref{eq_14}). However, not all vertices are included in the state vector in the second stage, which makes it challenging to compute the partial derivatives w.r.t. some cell vertices because their surrounding nodes are discarded. From a semantic perspective, the discarded cell vertices have the same occupancy state as the edge nodes, which is why they are excluded. Consequently, the gradient of these edge vertices is expected to be close to zero. Based on this reasoning, we set the partial derivatives w.r.t. all edge cell vertices to $0$ when they need to be calculated using (\ref{eq_14}). 
 
 For Jacobian $\mathbf{J}_S$, it can also be calculated using the same idea as (\ref{eq_JS}). In the second stage of our multi-resolution strategy, (\ref{eq_JS}) is reformulated as 
 \begin{equation}
 	\mathbf{J}_S = \frac{\partial F^S_{s}(\mathbf{x}^s)}{\partial \mathbf{x}^{sM} } = \mathbf{A}^{s}\\ 
 \end{equation}
 where $\partial F^S_{s}(\mathbf{x}^s)$ is similar to (\ref{eq_smoothing_term}), but only applies to vertices in $\mathbb{M}^s$. The coefficient matrix $\mathbf{A}^{s}$ has the same form as (\ref{eq_A}) but with dimension corresponding to the number of elements in $\mathbf{x}^{sM}$.  
 
 }   





\bibliographystyle{IEEEtran}
\documentclass[lettersize,journal]{IEEEtran}
\usepackage{amsmath,amsfonts}
% \usepackage{algorithm}
% \usepackage{algorithmic}
% \usepackage{algpseudocode}

\usepackage[linesnumbered,ruled,vlined]{algorithm2e}


\usepackage{array}
\usepackage{accents}
%\usepackage[caption=false,font=normalsize,labelfont=sf,textfont=sf]{subfig}
%\usepackage[caption=false,font=footnotesize,labelfont=sf,textfont=sf]{subfig}
\usepackage{subfigure}

\usepackage{textcomp}
\usepackage{stfloats}
\usepackage{url}
\usepackage{verbatim}
\usepackage{graphicx}
\usepackage{cite}
\usepackage{float}
 \usepackage{tabularx} 
% \usepackage{setspace}

\hyphenation{op-tical net-works semi-conduc-tor IEEE-Xplore}
\def\BibTeX{{\rm B\kern-.05em{\sc i\kern-.025em b}\kern-.08em
    T\kern-.1667em\lower.7ex\hbox{E}\kern-.125emX}}
\usepackage{balance}
% updated with editorial comments 8/9/2021

% use bib--added by yingyu
% \usepackage[numbers]{natbib}
\usepackage{amssymb}
\usepackage{booktabs}
\usepackage{threeparttable}
\usepackage{multirow}
\usepackage{bigstrut}
\usepackage{bigdelim}
\usepackage{adjustbox}
\usepackage{makecell}
\usepackage{nicematrix}
\usepackage{xcolor}

% \usepackage{xcolor} 
\usepackage{tikz} 
\usetikzlibrary{arrows,shapes,chains}

\usepackage{colortbl}
% \usepackage[table]{xcolor}
\usepackage{rotating}
% Beamer presentation requires \usepackage{colortbl} instead of \usepackage[table,xcdraw]{xcolor}


% \renewcommand{\algorithmicrequire}{\textbf{Input:}}
% \renewcommand{\algorithmicensure}{\textbf{Output:}}
\setlength{\textfloatsep}{8pt}
% \usepackage{caption}
% \captionsetup[figure]{skip=5pt}

% \setlength{\topsep}{0pt}
% \setlength{\belowdisplayskip}{10pt}
% \setlength{\abovedisplayskip}{10pt}

\newcommand*{\defeq}{\stackrel{\text{def}}{=}}
\DeclareMathOperator{\wrap}{wrap}
\DeclareMathOperator{\diag}{diag}
\DeclareMathOperator{\round}{round}




\begin{document}

\title{Occupancy-SLAM: An Efficient and Robust Algorithm for Simultaneously Optimizing Robot Poses and Occupancy Map}

\author{\authorblockN{Yingyu Wang, Liang Zhao, and Shoudong Huang} 
        % <-this % stops a space
\thanks{Yingyu Wang and Shoudong Huang are with the Robotics Institute, University of Technology Sydney, Australia (e-mail: Yingyu.Wang-1@student.uts.edu.au; Shoudong.Huang@uts.edu.au).} 

\thanks{Liang Zhao was with the Robotics Institute, University of Technology Sydney, Australia, and is now with the School of Informatics, University of Edinburgh, Edinburgh, UK (e-mail: Liang.Zhao@ed.ac.uk).}}



% The paper headers
\markboth{IEEE TRANSACTIONS ON ROBOTICS}%
{Shell \MakeLowercase{\textit{et al.}}: A Sample Article Using IEEEtran.cls for IEEE Journals}

% \IEEEpubid{0000--0000/00\$00.00~\copyright~2021 IEEE}
% Remember, if you use this you must call \IEEEpubidadjcol in the second
% column for its text to clear the IEEEpubid mark.

\maketitle

% This paper proposes Occupancy-SLAM, an optimization-based SLAM approach that jointly optimizes the robot trajectory and the occupancy map simultaneously.

\begin{abstract}
Joint optimization of poses and features has been extensively studied and demonstrated to yield more accurate results in feature-based SLAM problems. However, research on jointly optimizing poses and non-feature-based maps remains limited. Occupancy maps are widely used non-feature-based environment representations because they effectively classify spaces into obstacles, free areas, and unknown regions, providing robots with spatial information for various tasks. In this paper, we propose Occupancy-SLAM, a novel optimization-based SLAM method that enables the joint optimization of robot trajectory and the occupancy map through a parameterized map representation. The key novelty lies in optimizing both robot poses and occupancy values at different cell vertices simultaneously, a significant departure from existing methods where the robot poses need to be optimized first before the map can be estimated. 

This paper focuses on 2D laser-based SLAM to investigate how to jointly optimize robot poses and the occupancy map. In our formulation, the state variables in optimization include all the robot poses and the occupancy values at discrete cell vertices in the occupancy map. Moreover, a multi-resolution optimization framework that utilizes occupancy maps with varying resolutions in different stages is introduced. A variation of Gauss-Newton method is proposed to solve the optimization problem at different stages to obtain the optimized occupancy map and robot trajectory. The proposed algorithm is efficient and converges easily with initialization from either odometry inputs or scan matching, even when only limited key-frame scans are used. Furthermore, we propose an occupancy submap joining method, enabling more effective handling of large-scale problems by incorporating the submap joining process into the Occupancy-SLAM framework. Evaluations using simulations and practical 2D laser datasets demonstrate that the proposed approach can robustly obtain more accurate robot trajectories and occupancy maps than state-of-the-art techniques with comparable computational time. Preliminary results in the 3D case further confirm the potential of the proposed method in practical 3D applications, achieving more accurate results than existing methods. The code is made available to benefit the robotics community\footnote{\url{https://github.com/WANGYINGYU/Occupancy-SLAM}}. 
\end{abstract}

\begin{IEEEkeywords}
SLAM, optimization, occupancy grid map, non-feature-based map representation.
\end{IEEEkeywords}

\section{Introduction}

% optimizing robot poses and features simultaneously 


\IEEEPARstart{S}{imultaneous} localization and mapping (SLAM) is an important problem in robotics that has been studied for decades \cite{cadena2016past}. Jointly optimizing the robot poses and map can enhance SLAM performance, as this formulation utilizes the available information more directly without approximations. While joint optimization has been widely explored in feature-based SLAM (e.g., \cite{kaess2008isam,kaess2011isam2}), research on the joint optimization of robot poses and non-feature-based maps remains limited.

Occupancy grid maps are widely used in robotic tasks for their ability to clearly represent obstacles, free space, and unknown areas, facilitating collision-free navigation and path planning. Assuming the robot poses used to collect the sensor information are known exactly, the evidence grid mapping technique \cite{moravec1985high,
moravec1989sensor,elfes1989occupancy,martin1996robot,hornung2013octomap} provides an elegant and efficient approach for building occupancy grid maps from the collected information. However, when a robot is navigating in an unknown environment and performing SLAM, its own poses need to be estimated, and the estimates inherently contain uncertainties. Achieving both accurate robot localization and precise occupancy mapping simultaneously is not trivial.

% How to perform accurate robot localization and build an occupancy map very accurately at the same time is not trivial. 


In some occupancy grid map based SLAM approaches such as Cartographer \cite{hess2016real}, the problem is tackled in two steps. First, the robot poses are estimated by solving a pose-graph SLAM problem, where the relative pose measurements are derived using odometry, scan matching, loop closure detection, or other similar techniques. Second, the optimized poses are assumed to be the correct poses and are used to build up the map using evidence grid mapping techniques. However, in these two-step approaches, the uncertainties in the robot poses obtained during the first step are not considered when building the map. Therefore, it is crucial to achieve highly accurate pose estimates to construct a reliable occupancy grid map. As a result, it can be expected that the occupancy map obtained using a two-step approach may not represent the best result that one can achieve using all the available information.


In feature-based SLAM approaches, jointly optimizing the poses and the feature map is common, as the relationship between observations and the map is straightforward to model. However, for occupancy map based SLAM, jointly optimizing the robot poses and the occupancy map is not trivial because: 
\begin{itemize}
	\item [1)] \textbf{The relation between the observations and the map is complex.} The observations are laser beams (the endpoint of a beam represents ``hit" and the other positions along the beam represent ``free"), and the map is a function representing the occupancy values at different positions. This is significantly different from feature-based SLAM where both the observations and the map are about feature parameters such as positions.
	\item [2)] \textbf{The data association is not easy to do.} When the robot poses are noisy, the correct correspondence between laser beams and occupancy grid cells is hard to find. In contrast, for feature-based SLAM, there are well-established front-end methods for data association.
	\item [3)] \textbf{The resolution of the map has a significant impact on the optimization problem.} A high-resolution map helps to establish a more accurate connection between the observations and the map, but it leads to a sharp increase in the number of variables. However, for feature-based SLAM, there is no such issue.
\end{itemize}

% using 2D laser scans (and odometry) information.

\subsection{Contributions}
In this paper, we propose Occupancy-SLAM algorithm, which jointly optimizes the robot poses and the occupancy map using 2D laser scans (and odometry) information. Moreover, we propose a multi-resolution optimization framework for improving convergence and robustness to initial guesses. To better handle the case of large-scale environments and long-term trajectories, we further propose an occupancy submap joining method. Experiments on both simulated and practical datasets verify the superior performance of our method compared with state-of-the-art approaches (e.g., Cartographer \cite{hess2016real}). In addition, we extend our method to the 3D case, and preliminary results confirm its effectiveness in improving accuracy. The main contributions are summarized as follows: 

 % A smoothing term is introduced in the objective function to improve the convergence of the method.

\begin{enumerate}
	\item We formulate the occupancy grid map based SLAM problem as a joint optimization problem where the poses and the occupancy map are optimized together. 
	\item We propose a variation of Gauss-Newton method to solve the new formulation, enabling the estimation of more accurate robot poses and occupancy maps compared to existing state-of-the-art techniques.
	\item To enhance efficiency, convergence, and robustness, we propose a multi-resolution optimization strategy using occupancy maps of different resolutions across stages.
    
    % To improve the efficiency, convergence and robustness of our algorithm so that it can be initialized by odometry inputs or scan matching, we propose a multi-resolution optimization strategy that uses occupancy maps with different resolutions at different optimization stages. In the second stage, we utilize the selected high-resolution map, focusing exclusively on a subset of \textcolor{red}{cell vertices} that require further updates within the full high-resolution map. This targeted approach further enhances computational efficiency.
    
    \item We propose a submap joining algorithm to address the cases of large-scale environments and long-term trajectories through our joint poses and occupancy map optimization idea.
	\item Our method achieves robust convergence even with key frames of limited overlap, outperforming state-of-the-art approaches like Cartographer in efficiency while maintaining superior accuracy.
    % Our proposed method can converge well even when only key frames with limited overlaps are used. In this case, our method outperforms state-of-the-art methods, such as Cartographer, in terms of efficiency while maintaining a surpassing performance in terms of accuracy.
    \item We extend our method to 3D, with preliminary results demonstrating superior accuracy compared to other state-of-the-art approaches.
    % We extend our method to the 3D case, and preliminary results confirm that the accuracy of our method outperforms other state-of-the-art approaches.
\end{enumerate}


This paper is an extension to our conference paper \cite{Zhao-RSS-22}, with major improvements in contributions 3, 4, 5, and 6, significantly enhancing the robustness and efficiency of the algorithm while extending the method to 3D.

% The major improvements of this paper over \cite{Zhao-RSS-22} are contributions 3, 4, 5, and 6, which significantly improve the robustness and efficiency of the algorithm, and extend the algorithm to 3D.

\begin{figure}
\centering
\includegraphics[width=0.49\textwidth]{./OverView.pdf}
\caption{\label{fig_overview} Main components of our proposed method. The blue-colored components represent our core approaches, while the dashed portions are optional. Our multi-resolution joint optimization is covered in Section \ref{sec_formulation}, Section \ref{Sec_Algorithm_1}, and Section \ref{Sec_multi}. The joint global map and robot trajectory optimization approach is presented in Section \ref{Sec_submap}. }
\end{figure}

% \subsection{Notations}
% Some important notations in this paper are summarized in Table \ref{tab_notation}, the others are described in the context.

\subsection{Outline}
Fig. \ref{fig_overview} illustrates the flowchart of applying our proposed methods in practice. The blue components represent our core approaches, while the dashed portions are optional. The rest of the paper is organized as follows: Section \ref{Sec_related_work} provides a review of related work on non-feature-based SLAM, submap joining, and joint optimization of poses and maps. In Section \ref{sec_formulation}, we introduce our novel formulation for jointly optimizing the robot poses and occupancy map. A variation of the Gauss-Newton method to solve our nonlinear least squares (NLLS) formulation is presented in Section \ref{Sec_Algorithm_1}. In Section \ref{Sec_multi}, we introduce our multi-resolution strategy to improve the efficiency and robustness of the algorithm. Section \ref{Sec_submap} presents our submap joining algorithm for handling large-scale environments and long-term trajectories. Experimental results are provided in Section \ref{Sec_experiment}. In Section \ref{sec_3d}, we extend our method to the 3D case and present preliminary results. Finally, the conclusions are given in Section \ref{Sec_conclusion}.

 
\section{Related Work}\label{Sec_related_work}

In this section, we discuss some related work on non-feature based map representations for SLAM, submap joining techniques, and joint optimization of poses and maps. 

\subsection{Non-feature based map representations for SLAM}\label{Sec_related_a}
One widely used non-feature based SLAM approach is occupancy grid map-based SLAM, which probabilistically classifies spaces into obstacles, free areas, and unknown regions while accounting for uncertainty during observation updates \cite{moravec1985high, moravec1989sensor, elfes1989occupancy, martin1996robot, hornung2013octomap}. Classic examples, such as FastSLAM \cite{montemerlo2002fastslam} and GMapping \cite{grisetti2005improving}, use particle filters for mapping and localization but struggle with high computational demand and long-term accuracy in large-scale environments.

Recent optimization-based approaches, such as Hector SLAM \cite{kohlbrecher2011flexible}, Karto-SLAM \cite{konolige2010efficient}, and Cartographer \cite{hess2016real}, address cumulative errors effectively. Hector SLAM uses scan-to-map matching but lacks loop closure, restricting it to small-scale scenarios. Karto-SLAM incorporates loop closure detection with sparse pose adjustment for global optimization, while Cartographer integrates scan-to-map matching and pose graph optimization with a branch-and-bound strategy for efficient loop closure detection. However, by treating pose optimization and map construction as independent processes, these methods fail to account for the interdependencies of their uncertainties.

Multi-resolution occupancy mapping techniques can be integrated into occupancy grid map based SLAM frameworks to enable a more compact and efficient mapping process. For instance, approaches like OctoMap \cite{hornung2013octomap} use memory-efficient octrees to balance map compactness and accessibility. Adaptive-resolution methods, such as RMAP \cite{khan2014rmap} and ColMap \cite{fisher2021colmap}, dynamically adjust grid resolution to enhance mapping efficiency. Recently, \cite{Reijgwart-RSS-23} applies wavelet compression for hierarchical occupancy map storage, allowing efficient updates and queries. However, integrating multi-resolution maps as state variables into a unified framework for joint poses and map optimization remains an open challenge.


Another widely used non-feature-based map is the signed-distance function (SDF), which discretizes the environment into grid cells storing the distance to the nearest surface. This representation encodes the space, with the object surfaces defined by the zero crossings of the distance functions \cite{curless1996volumetric}. Some SLAM systems adopt SDF to improve localization accuracy and mapping quality. For example, supereight \cite{vespa2018efficient} integrates tracking, mapping, and planning using an octree-based truncated SDF (TSDF). It aligns camera frames to the TSDF map with iterative closest point (ICP) \cite{besl1992method}. A follow-up work \cite{vespa2019adaptive} improves this by introducing adaptive-resolution octree structures, achieving denser environment representation and reduced noise, leading to more accurate localization.

Other non-feature based map representations have also been used in SLAM, including mesh-based \cite{rosinol2021kimera}, normal distributions transform based \cite{einhorn2015generic}, neural radiance fields based \cite{rosinol2023nerf} and Gaussian splatting based \cite{matsuki2024gaussian}. Although these approaches differ in the type of non-feature representations they use, they all aim to provide more effective environmental modeling, improve robot localization accuracy, or achieve both. 


However, all the optimization-based SLAM approaches that utilize non-feature based maps need to optimize the poses first and then build the non-feature based map using the optimized poses. This separation prevents these approaches from jointly considering the uncertainties in both localization and mapping during the optimization process. In contrast, this paper considers unifying the optimization of both the robot poses and occupancy values at each cell vertex of the occupancy map into a single optimization problem, which can be expected to yield better accuracy.

\subsection{Submap Joining}\label{Sec_related_b}

Submap joining, as proposed by \cite{bosse2003atlas}, is a widely used scheme for SLAM in large-scale environments due to its efficiency and reduced risk of being trapped in local minima compared to full optimization-based SLAM. Feature-based submap joining approaches \cite{huang2008sparse,zhao2013linear,wang2019submap} have been well investigated, with many demonstrating properties that enable efficient problem-solving while maintaining a high level of accuracy. To extend non-feature-based SLAM to large-scale environments and long-term operations, recent efforts have explored non-feature-based submap joining methods.


For example, \cite{wagner2014graph} divides the environment into overlapping submaps composed of small TSDF grids from KinectFusion \cite{izadi2011kinectfusion}. Submap joining is then formulated as a pose graph optimization problem, where submap poses are nodes, and relative transformations from ICP are edges. Similarly, VOG-map \cite{ho2018virtual} represents submaps as 3D occupancy grids, converts them to point clouds for ICP-based relative transformations, and solves submap joining via pose graph optimization. Voxgraph \cite{reijgwart2019voxgraph} improves accuracy by employing SDF-to-SDF registration for overlapping submaps created with C-blox \cite{millane2018c}. Unlike time-sequence-based submap partitioning, \cite{wang2021elastic} uses spatial partitioning, merging submaps during loop closures by solving a pose graph containing only submap frames, with reconstruction decisions based on environmental changes.

All the aforementioned non-feature-based submap joining approaches estimate relative measurements between overlapping submaps to formulate and solve the pose graph problem for submap frames. In contrast, this paper jointly optimizes submap frames and the global occupancy map.

\subsection{Joint Optimization of Poses and Maps}
Joint optimization of poses and maps can result in better accuracy, as it utilizes the information more directly. In feature-based SLAM and bundle adjustment approaches, the most common form is to jointly optimize poses and positions of features, such as \cite{dellaert2006square,triggs2000bundle,konolige2008frameslam,sibley2009adaptive,zhao2015parallaxba}. Some approaches extend this idea to planar feature parameters. For instance, \cite{kaess2015simultaneous,hsiao2017keyframe} minimize the difference between plane measured in a scan and predicted planes, while \cite{trevor2012planar,geneva2018lips,zhou2021pi,zhou2021lidar} minimize the Euclidean distance between points in a scan and the predicted planes. Based on the idea of minimizing Euclidean distance between points in scans, BALM \cite{liu2021balm} demonstrates that planar parameters can be solved analytically in closed form, reducing the dimensionality of the optimization. BALM2 \cite{liu2023efficient} further improves efficiency by using point clusters, avoiding individual point enumeration. HBA \cite{liu2023large} introduces a hierarchical structure to address the scalability challenges of BALM and BALM2 in large environments. In summary, jointly optimizing poses and feature-based maps is well-studied, as features naturally link positions, observations, and poses, making them straightforward to integrate into optimization problems. In contrast, establishing constraints between observations, poses, and non-feature-based maps (e.g., occupancy grid maps) for joint optimization remains a significant challenge.



% Optimizing the poses and feature-based map together is very common and has been well-studied, as features are naturally linked to positions, which in turn connect observations, poses, and features, making them straightforward to integrate into optimization problems. However, it is a challenge to establish constraints between observations, poses, and a non-feature based map to jointly optimize the poses and the map (e.g., an occupancy grid map).

% Kimera-PGMO proposed in \cite{rosinol2021kimera} is a novel approach that simultaneously optimizes the poses and the mesh deformation. Specifically, Kimera-PGMO \cite{rosinol2021kimera} creates a deformation graph including a simplified mesh and a pose graph of robot poses. Since the simplified mesh consists of the positions of the mesh vertices, the method is formulated as a factor graph and then solved by GTSAM \cite{dellaert2012factor}.

Research on jointly optimizing the poses and non-feature based maps is limited. Kimera-PGMO proposed in \cite{rosinol2021kimera} represents a notable attempt, integrating pose optimization with mesh deformation. It constructs a deformation graph of a simplified mesh and a pose graph, formulating the problem as a factor graph solvable by GTSAM \cite{dellaert2012factor}. 
While Kimera-PGMO \cite{rosinol2021kimera} has similar motivations as our paper, aiming to achieve better quality maps and more accurate poses through joint optimization, its mesh-based representation differs fundamentally from the occupancy grid maps used in our approach. Meshes are naturally represented through point positions and their relationships, which facilitates factor graph formulations.


% but the mesh can still be described in terms of the positions of the points as well as the relationships between the points, and can therefore ultimately be transformed into a factor graph to be solved for, which is different to the occupancy map that we used.

\begin{table}[t]
 		\centering
 		\caption{Summary of Some Important Notations.}\label{tab_notation}
 		\setlength{\tabcolsep}{0.5 mm}{
 		\begin{tabular}{|c|l|p{3cm}p{3cm}p{3cm}}
   \hline
   \multicolumn{1}{|c|}{Notation} & \multicolumn{1}{|c|}{Explanation} \\ \hline
   $\mathbb{M}$  & \begin{tabular}[c]{@{}l@{}} A set includes occupancy values at all discrete cell vertices in \\occupancy map, as defined in Section \ref{sec_discrete_occupancy}. $\mathbb{M}^{l}$, $\mathbb{M}^{h}$, and $\mathbb{M}^{s}$ \\represent the sets include occupancy values at all cell vertices \\in low-resolution map, high-resolution map and selected \\high-resolution map, respectively. In addition, $\mathbb{M}_L$ and $\mathbb{M}_G$ \\represent the sets including occupancy values of all cell vertices \\in local maps and the global map, as defined in Section \ref{Sec_submap}.\end{tabular}\\ \hline

% as defined in \\Section \ref{continuous_map}
   $M(\cdot)$ &\begin{tabular}[c]{@{}l@{}}A function to lookup occupancy value at an arbitrary position in \\the occupancy map by bilinear interpolation using $\mathbb{M}$.\end{tabular} \\ \hline

    $\mathbf{x}^M$ & \begin{tabular}[c]{@{}l@{}} A vector including occupancy values at all cell vertices in discrete \\occupancy map $\mathbb{M}$, as defined in (\ref{eq_map_state}). $\mathbf{x}^{lM}$ and $\mathbf{x}^{sM}$ are vectors \\including occupancy values at cell vertices in $\mathbb{M}^{l}$ and $\mathbb{M}^{s}$. In \\addition, $\mathbf{x}^M_G$ represents the vector which includes occupancy \\values at cell vertices from $M_G$, as described in Section \ref{Sec_submap}.\end{tabular} \\ \hline

    $N(\cdot)$ & \begin{tabular}[c]{@{}l@{}} A function to lookup hit number at arbitrary position in the map \\by bilinear interpolation using hit map $\mathbb{N}$, where $\mathbb{N}$ is defined as a \\set includes hit number at all discrete cell vertices in the map, as \\described in Section \ref{sec_hit}. $\mathbb{N}^{l}$ and $\mathbb{N}^{s}$ represent hit maps used in \\different optimization stages.\end{tabular}\\ \hline

    $\mathbf{x}^P$ & \begin{tabular}[c]{@{}l@{}} A vector including all robot poses for optimization, as defined \\in (\ref{eq_pose_state}). In addition, $\mathbf{x}^P_L$ denotes a vector including all local map \\coordinate frames for submap joining problem in Section \ref{Sec_submap}.\end{tabular} \\ \hline
   \rule{0pt}{1.5em}
    $\mathbf{x}$ & \begin{tabular}[c]{@{}l@{}} State vector of optimization, $\mathbf{x} = {[{\mathbf{x}^P}^\top, {\mathbf{x}^M}^\top]}^\top$. $\mathbf{x}^{l}$ and $\mathbf{x}^{s}$ \\represent state vectors of different optimization stages. \end{tabular} \\ \hline

    $\mathbb{S}$  &\begin{tabular}[c]{@{}l@{}} $\mathbb{S} = \{\mathbb{S}_i\}_{0 \leq i \leq n}$ where $\mathbb{S}_i$ is defined in (\ref{S_i}), a set including \\observations, as defined in Section \ref{Sec_Info_1}. $\mathbb{S}^{l}$, $\mathbb{S}^{h}$, and $\mathbb{S}^{s}$ are \\observations used for occupancy maps $\mathbb{M}^{l}$, $\mathbb{M}^{h}$, and $\mathbb{M}^{s}$, \\respectively. \end{tabular}\\ \hline

     $s$  &\begin{tabular}[c]{@{}l@{}} Resolution of the occupancy map, which indicates the distance\\ between two nearby cell vertices. $s^{l}$ and $s^{h}$ represent the \\resolutions of low-resolution map and high-resolution map, \\respectively. $s^L$ and $s^G$ denote the resolutions of local maps and \\the global map, respectively, as described in Section \ref{Sec_submap}. \end{tabular}\\ \hline
    
     $\mathbb{O}$ & \begin{tabular}[c]{@{}l@{}}Set including all odometry inputs, as defined in Section \ref{sec_odometry}. \end{tabular}\\ \hline

    $r$  & Ratio between resolutions of two stages, $r = {s^{l}}/{s^h}$.  \\ \hline

    $\mathbf{m}$ & \begin{tabular}[c]{@{}l@{}} Discrete coordinate of a cell vertex, detailed explanation in the \\second paragraph in Section \ref{sec_discrete_occupancy}.\end{tabular} \\ \hline

    $\mathbf{p}$ & \begin{tabular}[c]{@{}l@{}}Continuous coordinate of a point, see the second paragraph in \\Section \ref{sec_relationship}.\end{tabular} \\
        
 		\hline
 		\end{tabular}
 		}
        % \vspace{-1em}

 \end{table}


\section{Problem Formulation}\label{sec_formulation}
Our approach considers the joint optimization of the robot poses and the occupancy map using information from 2D laser observations (and odometry). In this section, we will explain how the observations from the laser can be linked to the robot poses and the occupancy map to formulate the NLLS problem. 

% We also explain how we improve the problem formulation to make it easier to solve by adding a smoothing term. 

\subsection{Notation}
Throughout this paper, unless otherwise noted, we use specific typographical conventions: typefaces denote sets, bold uppercase letters represent matrices, bold lowercase letters indicate vectors, and regular (unbolded) lowercase letters signify scalars. Key notations used in this paper are summarized in Table \ref{tab_notation}, while others are introduced within the text as needed.

\subsection{Occupancy Map Representation and State in Optimization} \label{sec_discrete_occupancy}
Suppose the environment is discretized into $c_w\times c_h$ grid cells. We use $\mathbf{m}_{wh}=[w,h]^\top~(0 \leq w \leq c_w, 0 \leq h \leq c_h)$ to represent the coordinate of a discrete cell vertex in the map. The occupancy value at the cell vertex $\mathbf{m}_{wh}$, denoted as $M(\mathbf{m}_{wh})$, is defined using evidence, which is the natural logarithm of odds (the ratio between the probability of being occupied and the probability of being free) \cite{martin1996robot,hornung2013octomap,ProbabilisticRobotics}. 
The occupancy values of all $(c_w+1) \times (c_h+1)$ cell vertices consist of the discrete occupancy map $\mathbb{M}=\{M(\mathbf{m}_{wh})\}_{0 \leq w \leq c_w, 0 \leq h \leq c_h}$.


To represent the entire environment using a finite number of parameters, we describe the occupancy value at an arbitrary position $\mathbf{p}_m=[x,y]^{\top}$ on the map using bilinear interpolation of the occupancy values at its four surrounding cell vertices: $\mathbf{m}_{wh}, \mathbf{m}_{({w+1})h}, \mathbf{m}_{w({h+1})}, \mathbf{m}_{({w+1})({h+1})}$, as shown in Fig. \ref{fig_interpolation}, i.e.,

\begin{equation}
	M(\mathbf{p}_{m})= \begin{bmatrix}
a_1b_1,a_0b_1,a_1b_0,a_0b_0
\end{bmatrix}\left[
\begin{aligned}\label{eq_interp}
&M(\mathbf{m}_{wh})\\&M(\mathbf{m}_{(w+1)h})\\&M(\mathbf{m}_{w(h+1)})\\&M(\mathbf{m}_{(w+1)(h+1)})
\end{aligned}\right] 
\end{equation}
in which 
\begin{equation}
\begin{aligned}
	a_0 &= x - w\\
	a_1 &= w+1 - x\\
	b_0 &= y - h\\
	b_1 &= h+1 - y .\\
\end{aligned} 
\end{equation}


\begin{figure}[t]
\centering
\includegraphics[width=0.48\textwidth]{interpolation.pdf}
\caption{\label{fig_interpolation} Parameterizing the entire map by bilinear interpolation of discrete map $\mathbb{M}$.}
% \vspace{-1em}
\end{figure}


Our method jointly optimizes robot poses and the occupancy map, combining them into the state vector of the proposed optimization problem. Using bilinear interpolation with the discrete occupancy map $\mathbb{M}$, estimating the entire map is equivalent to estimating $\mathbb{M}$. Thus, the map component of the state vector can be expressed as

\begin{equation}
    \mathbf{x}^M =\left[M(\mathbf{m}_{00}),\cdots,M(\mathbf{m}_{c_wc_h}) \right]^\top. \label{eq_map_state}
\end{equation}


We define the $n+1$ robot poses as \rule{0pt}{1em}$\{\mathbf{x}^P_i \triangleq [\mathbf{t}_i^\top,\theta_i]^\top\}_{0 \leq i \leq n}$, where $\mathbf{t}_i$ is the $x$-$y$ position of the robot and $\theta_i$ is the orientation with the corresponding rotation matrix $\mathbf{R}_i=\begin{bmatrix}
\cos(\theta_i), \sin(\theta_i)\\ -\sin(\theta_i), \cos(\theta_i)
\end{bmatrix}$. As in most of the SLAM problem formulations, we assume the first robot pose defines the coordinate system, $\mathbf{x}^P_0 \triangleq [0,0,0]^\top$, so only the other $n$ robot poses are variables that need to be estimated, thus the pose component of the state vector is represented as
\begin{equation}
    \mathbf{x}^P = \left[ (\mathbf{x}^P_1)^\top, \cdots, (\mathbf{x}^P_n)^\top \right]^\top.
\end{equation}


Accordingly, the state vector of the proposed optimization problem is
\begin{equation}
    \mathbf{x} = \left[{(\mathbf{x}^P)}^\top,{(\mathbf{x}^M)}^\top \right]^\top. \label{eq_pose_state}
\end{equation}

In our method, the occupancy map $\mathbb{M}$ is initialized by the Bayesian occupancy mapping method \cite{ProbabilisticRobotics} with initially estimated poses (derived from odometry or scan matching) and updated throughout the optimization process.


% i.e., 
% \begin{equation}
%     \mathbf{x} = \left[{(\mathbf{x}^P)}^\top,{(\mathbf{x}^M)}^\top \right]^\top,
% \end{equation}
% where $\mathbf{x}^M$ is the map part and $\mathbf{x}^P$ is the poses part. By bilinear interpolation method and discrete occupancy map $\mathbb{M}$, we only need to estimate $(c_w+1)\times(c_h+1)$ parameters to estimate the entire map, thus $\mathbf{x}^M$ can be expressed as 
% \begin{equation}
%     \mathbf{x}^M &= \left[M(\mathbf{m}_{00}),\cdots,M(\mathbf{m}_{c_wc_h}) \right]^\top.
% \end{equation}
% We suppose that the $n+1$ robot poses are expressed by \rule{0pt}{1em}$\{\mathbf{x}^P_i \triangleq [\mathbf{t}_i^\top,\theta_i]^\top\}_{0 \leq i \leq n}$, where $\mathbf{t}_i$ is the $x$-$y$ position of the robot and $\theta_i$ is the orientation with the corresponding rotation matrix $\mathbf{R}_i=\begin{bmatrix}
% \cos(\theta_i), \sin(\theta_i)\\ -\sin(\theta_i), \cos(\theta_i)
% \end{bmatrix}$. As in most of the SLAM problem formulations, we assume the first robot pose defines the coordinate system, $\mathbf{x}^P_0 \triangleq [0,0,0]^\top$, so only the other $n$ robot poses are variables that need to be estimated. Thus, the state variables of the part of the pose can be expressed as 
% \begin{equation}
%     \mathbf{x}^P = \left[ (\mathbf{x}^P_1)^\top, \cdots, (\mathbf{x}^P_n)^\top \right]^\top.
% \end{equation}

% the state in our optimization problem can be represented as 
% \begin{equation}
%     \mathbf{x} = \left[{(\mathbf{x}^P)}^\top,{(\mathbf{x}^M)}^\top \right]^\top,
% \end{equation}
% where
% \begin{equation}
% \begin{aligned}
% \mathbf{x}^P &= \left[ (\mathbf{x}^P_1)^\top, \cdots, (\mathbf{x}^P_n)^\top \right]^\top\\
% \mathbf{x}^M &= \left[M(\mathbf{m}_{00}),\cdots,M(\mathbf{m}_{c_wc_h}) \right]^\top.
% \end{aligned}\label{eq_state_vec}
% \end{equation}

% The occupancy map is updated as the optimization process progresses without the need for additional occupancy mapping update methods, and the Bayesian occupancy mapping method is used as map initialization in our optimization problem.


% , so there is no need for an additional mapping update strategy, and furthermore, this makes the method of initializing the occupancy value $M(\mathbf{m}_{wh})$ not very critical.




% \subsection{The Available Information}\label{Sec_Info}

% The available information includes 2D laser scans collected at different robot poses. In addition, the odometry information might also be available. 

\subsection {Scan Points Sampling Strategy}\label{Sec_Info_1} 

 % We use the evidence, which is the natural logarithm of odds (the ratio between the probability of being occupied and the probability of being free) \cite{martin1996robot,hornung2013octomap,ProbabilisticRobotics} to represent the occupancy value.

 \begin{figure}[tbp]
\centering 
\subfigure[Equidistant Sampling Strategy] {\label{fig_sampling_strategy}
\includegraphics[width=0.23\textwidth]{./sampling_strategy.pdf}}
\subfigure[Observation Points in One Scan] {\label{fig_scan}
\includegraphics[width=0.23\textwidth]{./scan.pdf}}
\caption{Sampling strategy for generating observations from a laser scan: (a) Equidistant sampling on a beam, with red indicating occupied and blue indicating free states. The distance between two consecutive points is the resolution $s$. (b) All sampled observation points at a given time step.}
\label{fig_scan_sampling}
% \vspace{-1em}
\end{figure}

We now introduce our sampling strategy for generating observations from laser scans, which are used in our NLLS formulation. 

Each scan data consists of a number of beams. On each beam, the endpoint indicates the presence of an obstacle, while the other points before the endpoint indicate the absence of obstacles. Here, we sample each beam using a fixed resolution $s$ to get the observations, as shown in Fig. \ref{fig_sampling_strategy}. Specifically, $\mathbf{q}_{ij}=[x_{q_{ij}},y_{q_{ij}}]^\top$ denotes the position of $j$th sampling point at time step $i$ in the local robot/laser coordinate frame and
\begin{equation}
z_{ij} = \ln \frac{p(\mathbf{q}_{ij} \in occ)}{1-p(\mathbf{q}_{ij} \in occ)} \label{eq_occ_obs}
\end{equation}denotes the corresponding occupancy value. In the same way as the occupancy map representation described in Section \ref{sec_discrete_occupancy}, we also use the evidence to represent the occupancy value here. In our implementation, following \cite{hornung2013octomap,ProbabilisticRobotics}, we use $p(\mathbf{q}_{ij} \in occ) = 0.7$ for an occupied point (red in Fig. \ref{fig_sampling_strategy}), and use $p(\mathbf{q}_{ij} \in occ) = 0.4$ for a free point (blue in Fig. \ref{fig_sampling_strategy}). Fig. \ref{fig_scan} shows an example of all sampled points in one scan.
% $(0 \leq i \leq n)
By constant equidistant sampling of all the beams for the scan collected at time step $i$, a sampling point set
\begin{equation}
\mathbb{S}_i=\{ \mathbb{S}_{ij} \triangleq \{\mathbf{q}_{ij},z_{ij}\}\}_{1 \leq j \leq k_i}
\label{S_i}
\end{equation}
can be obtained. It should be noted that since the total length of all the beams at different time step $i$ is different, the number of sampling points $k_i$ obtained by the equidistant sampling strategy varies for different time step $i$.


Suppose there are $n+1$ laser scans collected from robot poses $0$ to $n$, $\mathbb{S}=\{\mathbb{S}_i\}_{0\leq i \leq n}$ is the available observation information collected at all different robot poses using our sampling strategy and will be used as observations in our NLLS formulation.

\subsection{Relationship Between Observations and Occupancy Map}\label{sec_relationship}


In Section \ref{sec_discrete_occupancy}, we defined the discrete occupancy map $\mathbb{M}$ as part of the state vector in our optimization problem. Section \ref{Sec_Info_1} detailed the observation generation process. In this section, we explain how the relationship between observations and the occupancy map is established through robot poses, forming the basis of our joint optimization problem.



\subsubsection{Local to Global Projection}
First, the $j$th scan point at time step $i$ can be projected to the occupancy map using the robot pose $\mathbf{x}^P_i$, and the projected position on the occupancy map can be calculated by 

\begin{equation}
	\mathbf{p}_{ij}
=\frac{\mathbf{R}_i^\top \mathbf{q}_{ij}+\mathbf{t}_i}{s} \label{P-project}
\end{equation}
where $s$ is the resolution of the vertices in the occupancy map $\mathbb{M}$ (the distance between two adjacent cell vertices represents $s$ meters in the real world). Here, we use the same resolution as that used in generating observations from laser scans in Section \ref{Sec_Info_1}. Then, the occupancy value at the projected point $\mathbf{p}_{ij}$ can be obtained using (\ref{eq_interp}), expressed as $M(\mathbf{p}_{ij})$.


\subsubsection{Relationship Between Sampling Points and Occupancy Map w.r.t. Occupancy Values}
As outlined in Sections \ref{sec_discrete_occupancy} and \ref{Sec_Info_1}, evidence is used to define the occupancy value, where multiple observations of the same cell result in the occupancy values from individual observations being cumulatively added to the cell's total occupancy value \cite{hornung2013octomap}. If the robot's poses are accurate and repeated observations of the same cell consistently indicate the same occupancy state, the cell's occupancy value becomes the product of the occupancy value of each observation and the number of times the cell is observed. For a unique coordinate $\mathbf{p}_{ij}$ in (\ref{P-project}), if both the robot pose $\mathbf{x}^P_i$ and the occupancy map $\mathbb{M}$ are accurate, the occupancy value $z_{ij}$ of its associated sampling point $\mathbb{S}_{ij}$, should closely approximate the occupancy value at $\mathbf{p}_{ij}$, $M(\mathbf{p}_{ij})$, divided by the number of times $\mathbf{p}_{ij}$ is ``observed", $N(\mathbf{p}_{ij})$.

Thus, if the number of times the point $\mathbf{p}_{ij}$ is ``observed" can be calculated, the relationship between the observations and the state vector (occupancy map and robot poses), can be determined.


\subsubsection{Hit Map and Hit Number Lookup}\label{sec_hit}

We now explain how $N({\mathbf{p}_{ij}})$ can be calculated. To quickly query the number of times an arbitrary point is ``observed", we need to count the number of times all cell vertices have been observed to form the discrete hit map $\mathbb{N}$ associated with the occupancy map $\mathbb{M}$. 

When a sampling scan point is projected into a coordinate by a given robot pose, this coordinate is considered to have been observed once, and then we distribute the hit number ``1" of the coordinate to the discretized cell vertices. Since the occupancy value $M(\mathbf{p}_{ij})$ is derived by bilinear interpolation of occupancy values of discrete cell vertices in (\ref{eq_interp}), in order to maintain the correspondence between the hit number and the occupancy values, we distribute this ``1" hit to the four surrounding cell vertices by inverse bilinear interpolation. For example, if a sampling point is projected into the center of a cell, then each of the 4 nearby cell vertices gets a hit number of 0.25. In addition, the hit number also accumulates with multiple observations of the same cell vertex, i.e.,
\begin{equation}
\left[ N(\mathbf{m}_{00}),\cdots,N(\mathbf{m}_{c_wc_h}) \right] 
= \sum_{i=0}^n \sum_{j=1}^{k_i} H(\mathbf{p}_{ij})
\label{eq_NP}
\end{equation}
where $H(\cdot)$ is the inverse process of bilinear interpolation. The hit number at all these discrete cell vertices consists of discrete hit map $\mathbb{N}=\{N(\mathbf{m}_{wh})\}_{0 \leq w \leq c_w, 0 \leq h \leq c_h}$.



After the discrete hit map $\mathbb{N}$ is obtained, the equivalent hit multiplier $N(\mathbf{p}_{ij})$ (representing the number of times $\mathbf{p}_{ij}$ is ``observed") for an arbitrary continuous point $\mathbf{p}_{ij}$ can be easily obtained using bilinear interpolation, similar to (\ref{eq_interp}). 

\subsection{The NLLS Formulation} % --- Observations Only Case
% With the map parameterization, observations generation, and projection from local to global coordinates, observations can be linked to the occupancy map through robot poses. 
We now formulate the NLLS problem to jointly optimize the robot poses and the occupancy map. The objective function of the NLLS problem is defined as
\begin{equation}
f(\mathbf{x})=w_Z f^Z(\mathbf{x})+w_O f^O(\mathbf{x})+w_S f^S(\mathbf{x}). 	\label{eq_objective_func}
\end{equation}
The objective function consists of the observation term $f^Z(\mathbf{x})$, the smoothing term $f^S(\mathbf{x})$, and the odometry term $f^O(\mathbf{x})$. $w_Z$, $w_S$ and $w_O$ are their corresponding weights, and we set $w_O = 0$ if there is no odometry information. We now explain the three terms one by one.

\subsubsection{Observation Term $f^Z(\mathbf{x})$}
Based on the relationship between the observations and the occupancy map w.r.t. occupancy values described in \ref{sec_relationship}, we can formulate the observation term as follows. 

Given the observation information $\mathbb{S}$ in (\ref{S_i}), the observation term in the objective function (\ref{eq_objective_func}) is formulated as
\begin{equation}
	f^Z(\mathbf{x}) =
	\sum_{i=0}^n \sum_{j=1}^{k_i}  \left\|z_{ij} - F_{ij}^Z(\mathbf{x})\right\|^2, 
\label{obs-term}
\end{equation}
where
\begin{equation}
	F_{ij}^Z(\mathbf{x})  = \frac{M(\mathbf{p}_{ij})}{N({\mathbf{p}_{ij}})}.\\ \label{eq_MN}
\end{equation}
Here, $\mathbf{p}_{ij}$ represents a coordinate in the map where the $j$th sampling scan point at time step $i$ is projected using the robot pose $\mathbf{x}^P_i$, as calculated by (\ref{P-project}). $N(\mathbf{p}_{ij})$ denotes the equivalent hit multiplier at $\mathbf{p}_{ij}$, as detailed in Section \ref{sec_hit}. 

In (\ref{obs-term}), we suppose the errors of occupancy values of different sampled points in the observations $\mathbb{S}$ are independent and have the same uncertainty. Therefore, the weights on all terms are the same, which is equivalent to setting all the weights as $1$. Thus, we use norms instead of weighted norms in equation (\ref{obs-term}).

\subsubsection{Odometry Term $f^O(\mathbf{x})$}\label{sec_odometry}

% Information\label{sec_odom_inputs}}
The odometry information $\mathbb{O} = \{\mathbf{o}_i\}_{1 \leq i \leq n}$ might be available. We assume the odometry input is the relative pose between two consecutive steps. 
The odometry from robot pose $\mathbf{x}^P_{i-1}$ to pose $\mathbf{x}^P_{i}$ is expressed as
\begin{equation} 
\mathbf{o}_i=\left[ (\mathbf{o}_i^t)^\top,o_i^\theta \right]^\top~~(1 \leq i \leq n)
\label{O_i}
\end{equation}
where $\mathbf{o}_i^t$ is the translation part and $o_i^\theta$ is the rotation angle part of the odometry. The odometry term can be formulated as
\begin{equation}
\begin{aligned}
f^O(\mathbf{x})&=\sum_{i=1}^n \left\|\mathbf{o}_i -
F_i^O(\mathbf{x})
\right\|^2_{\mathbf{\Sigma}^{-1}_{O_i}}
\\&=\sum_{i=1}^n\left\|
\begin{bmatrix}
\mathbf{o}_i^t-\mathbf{R}_{i-1}\left(\mathbf{t}_i - \mathbf{t}_{i-1} \right)\\
\wrap\left({o}_i^\theta- \theta_i + \theta_{i-1}\right)
\end{bmatrix}
\right\|^2_{\mathbf{\Sigma}^{-1}_{O_i}}  \label{eq_odometry_term}
\end{aligned}
\end{equation}
in which $\mathbf{\Sigma}_{O_i}$ is the covariance matrix representing the uncertainty of $\mathbf{o}_i$, and $\wrap(\cdot)$ wraparounds the rotation angle to $(-\pi,\pi]$.
\subsubsection{Smoothing Term $f^S(\mathbf{x})$}

It can be easily found out that minimizing the objective function with only the observation term (and the odometry term) is not easy since there are a large number of local minima. Especially when the initial robot poses are far away from the global minimum, it is very difficult for an optimizer to converge to the correct solution. 

In order to enlarge the region of attraction and develop an algorithm that is robust to initial values, we introduce a smoothing term. The smoothing term requires the occupancy values of nearby cell vertices to be close to each other thus resulting in the occupancy map being smoother for derivative calculation. In our case, based on the derivative calculation method we use (see Appendix \ref{Sec_J_P}), we penalize the difference between the occupancy value of each cell vertex and the occupancy values of the two neighboring cell vertices to its right and below, i.e.,
\begin{equation}
\begin{aligned}
f^S(\mathbf{x})
& =\left\|F^S(\mathbf{x}) \right\|^2\\
& = \sum_{w=0}^{c_w-1} \sum_{h=0}^{c_h-1}  \left\|\begin{bmatrix} M(\mathbf{m}_{wh})-M(\mathbf{m}_{{(w+1)}h})\\
M(\mathbf{m}_{wh})-M(\mathbf{m}_{{w}{(h+1)}})
\end{bmatrix} \right\|^2 \\
& + \sum_{h=0}^{c_h-1}  \left\| M(\mathbf{m}_{c_wh})-M(\mathbf{m}_{{c_w}{(h+1)}})\right\|^2 \\
& + \sum_{w=0}^{c_w-1}  \left\| M(\mathbf{m}_{wc_h})-M(\mathbf{m}_{{(w+1)}{c_h}})\right\|^2,
\end{aligned} \label{eq_smoothing_term}
\end{equation} where the second and third terms are used to handle cell vertices located in the bottom row and the rightmost column. It should be noted that $F^S(\mathbf{x})$ is a linear function of $\mathbf{x}^M$ in the state. The coefficient matrix is constant and can be calculated prior to the optimization. For more details, please refer to Appendix \ref{Sec_J_S}.


\section{Iterative Solution to the NLLS Formulation}\label{Sec_Algorithm_1}
In Section \ref{sec_formulation}, we introduced our NLLS formulation for the joint poses and occupancy map optimization problem. In this section, we provide the details of a Gauss-Newton based algorithm for solving the NLLS problem. 



\begin{algorithm}[t]
\small
\caption{Our Joint Poses and Occupancy Map Optimization Algorithm}\label{alg_1}
\SetKwInput{KwInput}{Input}                % Set the Input
\SetKwInput{KwOutput}{Output}              % set the Output
\SetKwInput{KwParam}{Params}
\SetAlgoLined
\DontPrintSemicolon
\SetKw{Return}{End Function}
  \KwParam{Threshold $\tau_k$, $\tau_{\Delta}$, weight matrix $\mathbf{W}$, resolution $s$}
  \KwInput{Observations $\mathbb{S}$, odometry $\mathbb{O}$, and initial poses $\mathbf{x}^P(0)$}
  \KwOutput{Optimized poses $\hat{\mathbf{x}}^P$ and optimized map $\hat{\mathbf{x}}^M$}
\SetKwFunction{FuncFirstStage}{FirstStage}

\SetKwProg{Fn}{Function}{:}{}
\Fn{\FuncFirstStage{$\mathbf{x}^P(0)$, $\mathbb{S}$, $\mathbb{O}$, $\tau_k$, $\tau_{\Delta}$, $s$, $\mathbf{W}$}}
{
Initialize $\mathbf{x}^M(0)$ and $\mathbb{N}(0)$ using $\mathbf{x}^P(0)$ and $\mathbb{S}$ \;

Pre-calculate smoothing term coefficient $\mathbf{A}$ using (\ref{eq_A})\;

\SetKwFunction{FuncGN}{OccupancyGN}
\SetKwFunction{FuncReturn}{return}

\SetKwProg{Fn}{Function}{:}{}
\For {$k=0$; $k <= \tau_k \; \& \; \| \mathbf{\Delta}(k) \|^2 >= \tau_{{\Delta}}$; $k++$}{
\Fn{\FuncGN{$\mathbf{x}^M(k)$, $\mathbb{N}(k)$, $\mathbf{x}^P(k)$, $\mathbb{S}$, $\mathbb{O}$, $\mathbf{A}$, $\mathbf{W}$}}{
Calculate gradient $\mathbf{\nabla} \mathbf{x}^M(k)$ of $\mathbf{x}^M(k)$

Calculate $\mathbf{J}$, as described in appendices

Evaluate $F(\mathbf{x})$ at $\mathbf{x}^P(k)$ and $\mathbf{x}^M(k)$

Solve $\mathbf{J}^\top \mathbf{W} \mathbf{J} \mathbf{\Delta}(k) =-\mathbf{J}^\top \mathbf{W} F(\mathbf{x})$, where $\mathbf{\Delta}(k) = {[{\mathbf{\Delta}^P(k)}^\top,{\mathbf{\Delta}^M(k)}^\top]}^\top$

Update $\mathbf{x}^P(k+1)=\mathbf{x}^P(k) + \mathbf{\Delta}^P(k)$ and $\mathbf{x}^M(k+1)=\mathbf{x}^M(k)+\mathbf{\Delta}^M(k)$

Recalculate $\mathbb{N}(k+1)$ using $\mathbf{x}^P(k+1)$ and $\mathbb{S}$

\FuncReturn{$\mathbf{x}^P(k+1)$, $\mathbf{x}^M(k+1)$}
}
\Return
}
$\hat{\mathbf{x}}^P \Leftarrow \mathbf{x}^P(k)$, $\hat{\mathbf{x}}^M \Leftarrow \mathbf{x}^M(k)$

\FuncReturn{$\hat{\mathbf{x}}^P$, $\hat{\mathbf{x}}^M$}
}

\Return
\end{algorithm}


In the equation below, we assume the odometry inputs are available. Let
\begin{equation}
\begin{aligned}
F(\mathbf{x}) = [&\cdots,z_{ij}-F_{ij}^Z(\mathbf{x}),\cdots,{(\mathbf{o}_i-F_i^O(\mathbf{x}))}^\top,\\
&\cdots,{F^S(\mathbf{x})}^\top]^\top\\
\mathbf{W} = \;\; &\diag(\cdots,w_Z,\cdots,w_O \mathbf{\Sigma}^{-1}_{O_i}, \cdots,w_S, \cdots)\\
\end{aligned}
\end{equation}
combine all the error functions and the weights of the three terms in (\ref{eq_objective_func}). Then, the NLLS problem in (\ref{eq_objective_func}) seeks $\mathbf{x}$ such that
\begin{equation}\label{Least Squares}
f(\mathbf{x})=\|F(\mathbf{x})\|^2_{\mathbf{W}} =
{F(\mathbf{x})}^\top \mathbf{W}
F(\mathbf{x})
\end{equation}
is minimized.

A solution to (\ref{Least Squares}) can be obtained iteratively by starting with an initial guess $\mathbf{x}(0)$ and updating with $\mathbf{x}(k+1) = \mathbf{x}(k) + \mathbf{\Delta}(k)$. \rule{0pt}{1em}The update vector $\mathbf{\Delta} (k) = [{\mathbf{\Delta}^P(k)}^\top,{\mathbf{\Delta}^M(k)}^\top]^\top$ is the solution to
\begin{equation}\label{Gauss-Newton}
\mathbf{J}^\top \mathbf{W} \mathbf{J} \mathbf{\Delta} (k) = -\mathbf{J}^\top \mathbf{W} F(\mathbf{x}(k))
\end{equation}
where $\mathbf{J}$ is the linear mapping represented by the Jacobian matrix
$\partial F / \partial \mathbf{x}$ evaluated at $\mathbf{x}(k)$.

The iterative method for solving the proposed NLLS problem is shown in Algorithm \ref{alg_1}, in which $\tau_k$ and $\tau_{\Delta}$ represent the thresholds of iteration number $k$ and the incremental vector $\mathbf{\Delta}$. Unlike the standard Gauss-Newton iterative method, the hit map needs to be additionally recalculated after updating the poses in each iteration. With this approach, the implicit data association is established at each iteration and updated during the optimization.

Since the robot poses and the occupancy map are optimized simultaneously, the Jacobian $\mathbf{J}$ in (\ref{Gauss-Newton}) is very important and quite different from those used in the traditional SLAM algorithms. More details of the Jacobians are described in appendices.




\section{Multi-resolution Joint Optimization Strategy} \label{Sec_multi}
Algorithm \ref{alg_1} provides a solution to our NLLS problem (\ref{eq_objective_func}) to jointly optimize the poses and the occupancy map. However, directly using Algorithm \ref{alg_1} with the high-resolution map is time-consuming and requires an accurate initial value of robot poses \cite{Zhao-RSS-22}, which is challenging to obtain. To overcome these limitations, we propose a multi-resolution joint optimization strategy in this section.

\subsection{Discussion on Map Resolution in Optimization }

The resolution of the occupancy map has a significant impact on the optimization results since $\mathbf{x}^M$ is part of the state vector in our NLLS formulation (\ref{eq_objective_func}). 

% Assuming the robot poses are accurate, a high-resolution map representation can establish accurate relationships between observations and occupancy values of projected points on the map, but it leads to a dramatic increase in the size of the optimization problem, which in turn leads to a significant increase in computational cost. In addition, in a high-resolution map, the occupancy values in nearby \textcolor{red}{cell vertices} can vary sharply, leading to noise-filled gradients in the map when poor initial robot poses are used. Even with the introduction of the smoothing term, the use of a high-resolution map may cause poor convergence of Algorithm \ref{alg_1}.

A high-resolution map enables precise relationships between observations and occupancy values of projected points. However, it results in a dramatic increase in the optimization problem's size, raising computational costs. Additionally, in a high-resolution map, occupancy values in adjacent cell vertices may exhibit sharper variations compared to those in a low-resolution map, leading to noisy gradients when poor initial robot poses are used. Even with the introduction of the smoothing term, the use of a high-resolution map may cause poor convergence of Algorithm \ref{alg_1}.



A low-resolution map provides advantages in faster computation and reduced memory usage. Moreover, gradients are less sensitive to pose accuracy. With our occupancy map representation and smoothing term, these advantages enable the algorithm to quickly converge to a reasonable solution, even with poor initial robot poses. However, low resolution may cause inaccurate links between observations and occupancy values near boundaries, preventing the optimization from achieving greater accuracy.

To combine the advantages of different resolution map representations, we propose a multi-resolution strategy to optimize the occupancy values of different resolution cell vertices together with robot poses at various stages. Unlike the conventional coarse-to-fine scheme, in the second stage of our strategy, we use the selected high-resolution map that only includes high-resolution cell vertices possibly in need of further optimization instead of the full high-resolution map. Optimizing only those selected high-resolution cell vertices further improves the efficiency of our algorithm. 
\subsection{Our Multi-resolution Joint Optimization Strategy}

Firstly, we obtain low-resolution observations $\mathbb{S}^{l} = \{\mathbb{S}_i^{l}\}_{0\leq i \leq n}$ by down-sampling from the high-resolution observations $\mathbb{S}^{h} = \{\mathbb{S}_i^{h}\}_{0\leq i \leq n}$, which are obtained by the equal sampling strategy described in Section \ref{Sec_Info_1} with a sampling distance $s^{h}$. Here, we set the map resolution and sampling resolution to be the same. Therefore, the low resolution $s^{l}=r \times s^{h}$, where $r$ is the resolution ratio between the low-resolution map and the high-resolution map. The size of the low-resolution map $\mathbb{M}^{l}$ is $(c_w+1) \times (c_h+1)$.

Initialized by the odometry inputs or scan matching, we perform Algorithm \ref{alg_1} to quickly obtain relatively accurate poses. The state vector in the first stage is ${\mathbf{x}}^{l} = {[{\mathbf{x}^P}^\top,{{\mathbf{x}^{lM}}}^\top]}^\top $, where $\mathbf{x}^{lM}$ includes all occupancy values at the cell vertices of the low-resolution map $\mathbb{M}^{l}$. In this stage, the hit map, observation information, coefficient matrix, weight matrix, and resolution are represented as $\mathbb{N}^{l},  \mathbb{S}^{l}, \mathbf{A}^{l}$, $\mathbf{W}^{l}$, and $s^{l}$, respectively. 

In the first stage of optimization, the low-resolution occupancy map reduces both the dimension of $\mathbf{x}^{lM}$ and the number of observations in $\mathbb{S}^{l}$. Since the occupancy values at cell vertices change relatively gradually in the low-resolution map, the directions of the map's gradients are closer to the correct ones when the poses are initialized by odometry inputs or scan matching, making it easier for Algorithm \ref{alg_1} to converge to a relatively good result quickly.

After the first stage, we use Algorithm \ref{alg_2} to select the cell vertices that need to be further optimized to compose the selected high-resolution map $\mathbb{M}^{s}$ and find their corresponding observations $\mathbb{S}^{s}$. Details are described in Section \ref{select_index_set}.


\begin{algorithm}[tp]
\small
\caption{Finding the Selected High-resolution Map and Corresponding Observations}\label{alg_2}

\SetKwInput{KwInput}{Input}                % Set the Input
\SetKwInput{KwOutput}{Output}              % set the Output
\SetKwInput{KwParam}{Params}
\SetKw{Return}{End Function}
\SetAlgoLined
\DontPrintSemicolon
\KwParam{Resolution $s^{h}$, selection distance $d$, convolution kernel size $q$}
  \KwInput{Observations $\mathbb{S}^{h}$, and poses ${{\hat{\mathbf{x}}}}^{\tilde{P}}$ from the first stage using Algorithm \ref{alg_1}}
  \KwOutput{Observations $\mathbb{S}^{s}$ and map part of the state vector in the second stage $\mathbf{x}^{sM}$}
  \SetKwFunction{FuncSel}{Selection}
\SetKwProg{Fn}{Function}{:}{}
\Fn{\FuncSel{${{\hat{\mathbf{x}}}}^{\tilde{P}}$, $\mathbb{S}^{h}$, $s^{h}$, $d$, $q$}}{

Build a full high-resolution map $\mathbb{M}^{h}$ using $\hat{\mathbf{x}}^{\tilde{P}}$ and $\mathbb{S}^{h}$ 

Calculate the binary map $\mathbb{B}$ using $\mathbb{M}^{h}$

Calculate the convoluted map $\mathbb{C}$ with kernel size $q$

Calculate the set $\mathbb{I}^{h}$, which includes the indices of all boundary vertices in $\mathbb{M}^{h}$, using $\mathbb{C}$

Calculate the set $\mathbb{I}^{s}$, which includes the indices of all selected vertices, using $\mathbb{I}^{h}$ and $d$

Define the map part of the state vector in the second stage $\mathbf{x}^{sM}$ and the selected high-resolution map $\mathbb{M}^{s}$ by $\mathbb{I}^{s}$

Find observations $\mathbb{S}^{s}$ for $\mathbb{M}^{s}$ using the set $\mathbb{I}^{s}$, $\mathbb{S}^{h}$ and ${{\hat{\mathbf{x}}}}^{\tilde{P}}$

\FuncReturn{$\mathbb{S}^{s}$, $\mathbf{x}^{sM}$}
}
\Return
\end{algorithm}


In the second stage, the state vector is represented as $\mathbf{x}^{s} = {[{{\mathbf{x}}^P}^\top,{\mathbf{x}^{sM})}^\top]}^\top $ where $\mathbf{x}^{sM}$ includes all occupancy values at cell vertices of the selected high-resolution map $\mathbb{M}^{s}$. We perform Algorithm \ref{alg_3} using poses obtained from the first stage as initial guesses and observations $\mathbb{S}^{s}$ to refine poses. The NLLS optimization problem in the second stage can be formulated similarly as (\ref{Least Squares}). Additionally, the differences in the Jacobian calculation between Algorithm \ref{alg_1} and Algorithm \ref{alg_3} are described in Appendix \ref{Sec_J_Select}. 

The full multi-resolution joint optimization strategy is outlined in Algorithm \ref{alg_flowchart}.

\begin{algorithm}[t]
\small
\caption{The Algorithm for the Second Stage of the Multi-resolution Joint Optimization Strategy}\label{alg_3}
\SetKwInput{KwParam}{Params}
\SetKwInput{KwInput}{Input}                % Set the Input
\SetKwInput{KwOutput}{Output}              % set the Output
\SetAlgoLined
\DontPrintSemicolon
\SetKw{Return}{End Function}

  \KwParam{Threshold $\tau_k^{s}$, $\tau_{\Delta}^{s}$, weight matrix $\mathbf{W}^{s}$, resolution $s^{h}$}
  \KwInput{Observations $\mathbb{S}^{s}$, odometry $\mathbb{O}$, and poses ${{\hat{\mathbf{x}}}}^{\tilde{P}}$ from the first stage using Algorithm \ref{alg_1}}
  \KwOutput{Optimal poses $\hat{\mathbf{x}}^P$ and map $\hat{\mathbf{x}}^{sM}$}
\SetKwFunction{FuncSecondStage}{SecondStage}

$\mathbf{x}^P(0) \Leftarrow {{\hat{\mathbf{x}}}}^{\tilde{P}}$

\SetKwProg{Fn}{Function}{:}{}
\Fn{\FuncSecondStage{$\mathbf{x}^P(0)$, $\mathbb{S}^{s}$, $\mathbb{O}$, $\tau_k^{s}$, $\tau_{\Delta}^{s}$, $s^{h}$, $\mathbf{W}^{s}$}}
{

Initialize $\mathbf{x}^{sM}(0)$ and $\mathbb{N}^{s}(0)$ using $\mathbb{S}^{s}$ and $\mathbf{x}^P(0)$

Pre-calculate smoothing term coefficient matrix $\mathbf{A}^{s}$

\For {$k=0$; $k <= \tau_k^{s} \; \& \; \| \mathbf{\Delta}(k) \|^2 >= \tau_{\Delta}^{s}$; $k++$}{

$\mathbf{x}^P(k+1)$, $\mathbf{x}^{sM}(k+1)$
$\leftarrow$  \FuncGN{$\mathbf{x}^{sM}(k)$, {$\mathbb{N}^{s}(k)$, $\mathbf{x}^P(k)$, $\mathbb{S}^{s}$, $\mathbb{O}$, $\mathbf{A}^{s}$, $\mathbf{W}^{s}$}}
}

$\hat{\mathbf{x}}^P \Leftarrow \mathbf{x}^P(k)$, $\hat{\mathbf{x}}^{sM} \Leftarrow \mathbf{x}^{sM}(k)$

\FuncReturn{$\hat{\mathbf{x}}^P$, $\hat{\mathbf{x}}^{sM}$}

}
\Return
\end{algorithm}




\begin{algorithm}[t]
\small
\caption{Our Multi-resolution Joint Optimization Strategy}\label{alg_flowchart}
\SetKwInput{KwParam}{Params}
\SetKwInput{KwInput}{Input}                % Set the Input
\SetKwInput{KwOutput}{Output}              % set the Output

\SetAlgoLined
\DontPrintSemicolon

\SetKwFunction{FuncDown}{DownSampling}
\SetKwFunction{FuncInitPose}{InitializePose}
\SetKwFunction{FuncSM}{ScanMatching}
  \KwParam{Threshold $\tau_k^{l}$, $\tau_{\Delta}^{l}$, $\tau_k^{s}$, $\tau_{\Delta}^{s}$, weight matrix $\mathbf{W}^{l}$, $\mathbf{W}^{s}$, ratio of resolutions $r$, resolution $s$, selection distance $d$, convolution kernel size $q$}
  \KwInput{Observations $\mathbb{S}^{h}$, odometry $\mathbb{O}$}
  \KwOutput{Optimal poses $\hat{\mathbf{x}}^P$ and map $\hat{\mathbf{x}}^{sM}$}

$\mathbb{S}^{l}$ $\leftarrow$ \FuncDown{$\mathbb{S}^{h}$, $r$}

\uIf {$w_O \neq  0$}
{
$\mathbf{x}^P(0)$ $\leftarrow$ \FuncInitPose{$\mathbb{O}$}
}
\Else
{
$\mathbf{x}^P(0)$ $\leftarrow$ \FuncSM{$\mathbb{S}^{l}$}
}


${{\hat{\mathbf{x}}}}^{\tilde{P}}$, $\hat{\mathbf{x}}^{lM}$ $\leftarrow$ \FuncFirstStage{$\mathbf{x}^P(0)$, $\mathbb{S}^{l}$, $\mathbb{O}$, $\tau_k^{l}$, $\tau_{\Delta}^{l}$, $s^{l}$, $\mathbf{W}^{l}$}

$\mathbb{S}^{s}$, $\mathbf{x}^{sM}$ $\leftarrow$ \FuncSel{${{\hat{\mathbf{x}}}}^{\tilde{P}}$, $\mathbb{S}^{h}$, $s^{h}$, $d$, $q$}

${\hat{\mathbf{x}}^P}$, $\hat{\mathbf{x}}^{sM}$ $\leftarrow$
\FuncSecondStage{${{\hat{\mathbf{x}}}}^{\tilde{P}}$, $\mathbb{S}^{s}$,$\mathbb{O}$, $\tau_k^{s}$, $\tau_{\Delta}^{s}$, $s^{h}$, $\mathbf{W}^{s}$}

\end{algorithm}

\begin{figure}[t]
\centering 
\subfigure[Full High-resolution Map]{ 
\includegraphics[width=0.23\textwidth]{./high_resolution_select.pdf}}
\subfigure[Selected High-resolution Map (In White and Black)]{\label{fig_select_example_b}
\includegraphics[width=0.23\textwidth]{./recolor_select.pdf}}
\caption{An example of the selected high-resolution map from a full high-resolution map in a simulation dataset. (a) The full high-resolution map generated using poses from the first-stage optimization and scans, forming the basis for selection. (b) The recolored selected high-resolution map: gray marks dropped (stable) areas, white and black denote selected areas, with black highlighting obstacle boundaries.}
\label{fig_select_example}
% \vspace{-1em}
\end{figure}

\subsection{Selected High-resolution Map and Observations}\label{select_index_set}

After the first stage optimization using the low-resolution map $\mathbb{M}^{l}$, the robot poses $\hat{\mathbf{x}}^{\tilde{P}}$ become relatively accurate. Subsequently, the full high-resolution map $\mathbb{M}^{h}$, with dimensions $(r*c_w+1) \times (r*c_h+1)$, is built using the Bayesian occupancy mapping method \cite{ProbabilisticRobotics}, based on observations $\mathbb{S}^{h}$ and poses ${{\hat{\mathbf{x}}}}^{\tilde{P}}$. In this case, most cell vertices of $\mathbb{M}^{h}$ are considered stable in terms of occupancy state. Semantically, these stable cell vertices have the same occupancy state as the surrounding cell vertices (typically free or unknown cells). This characteristic leads to map gradients near zero at these stable cell vertices. In contrast, the cell vertices that require further updates are typically located at the edges of objects, where the occupancy values significantly differ from those of surrounding cell vertices. Therefore, the gradient at these cell vertices is larger. An example illustrating this is shown in Fig. \ref{fig_select_example_b}, where the selected area (in white and black) is clearly distinct from the stable area (in gray). Based on this idea, we propose a strategy to select the cell vertices located around the boundaries to compose the selected high-resolution map $\mathbb{M}^{s}$, which is used in the second stage of optimization.


\begin{figure}[t]
\centering 
\includegraphics[width=0.48\textwidth]{./Select_Set.pdf}
\caption{\label{fig_low_high_select} An illustration of the cell vertices selection strategy and a selected high-resolution map from a simulation dataset. In (a), selected cell vertices are marked in red and yellow, with their indices forming the index set $\mathbb{I}^{s}$.}
% \vspace{-0.5em}
\end{figure}

% In both figures, the boundary cells are indicated as black color, the selected cells are shown as black and white, and the dropped cells are colored in gray.

Firstly, we identify cell vertices located at the edges of objects by performing mean-value convolution of the full high-resolution map $\mathbb{M}^{h}$. Specifically, we calculate a binary map $\mathbb{B}=\{B(\mathbf{m}_{id})\}$ by binarizing $\mathbb{M}^{h}$ as
\begin{equation}
B(\mathbf{m}_{id}) =	\begin{cases}
	1, & {M}^{h}(\mathbf{m}_{id}) \geq \tau_{occupied} \\
	0, & {M}^{h}(\mathbf{m}_{id}) < \tau_{occupied} \\
\end{cases},
\end{equation}
where $\mathbf{m}_{id}$ represents a cell vertex, and $\tau_{occupied}$ is the threshold used to classify a cell vertex as occupied or free. A mean-value convolution kernel $\mathbf{K}$ is defined as
\begin{equation}
	\mathbf{K} = \dfrac{1}{q^2} \cdot \bold{1}_{q\times q}
\end{equation}
where $\bold{1}_{q\times q}$ represents a $q \times q$ matrix of ones. The convoluted map $\mathbb{C}=\{C(\mathbf{m}_{id})\}$ is then derived by convolving $\mathbb{B}$ with $\mathbf{K}$, where $C(\mathbf{m}_{id})$ indicates whether the $q \times q$ cell vertices around $\mathbf{m}_{id}$ are all in the same occupancy state. Compared to other edge detection methods like Sobel \cite{duda1973pattern} and Canny \cite{canny1986computational}, this conservative method more reliably selects cell vertices that may require further optimization. 

Using this method, the set of indices for all boundary cell vertices in the high-resolution map is defined as 

\begin{equation}
\begin{aligned}
	\mathbb{I}^{h} = \{id | 0<C(\mathbf{m}_{id})<1 \}.
\end{aligned}
\end{equation}
The cell vertices indexed in $\mathbb{I}^{h}$ are marked in red in Fig. \ref{fig_low_high_select}(a). 


To account for pose uncertainties from the first stage, the selection is expanded to include cell vertices within a distance $d$ from all boundary cell vertices. The indices of the selected cell vertices in the high-resolution map form the set $\mathbb{I}^{s}$, illustrated in Fig. \ref{fig_low_high_select}(a), where the selected cell vertices are highlighted in red and yellow with $d=1$. An example of a selected high-resolution map from a simulation dataset is shown in Fig. \ref{fig_low_high_select}(b).

Consequently, the map component of the state vector in the second stage is expressed as
\begin{equation}
	\mathbf{x}^{sM} = {[\cdots, M^{h}(\mathbf{m}_{wh}), \cdots]}^\top, ~~ wh\in \mathbb{I}^{s}.
\end{equation}

% The cell selection strategy with a selection distance of 1 cell is illustrated in Fig. \ref{fig_low_high_select}(a), where selected \textcolor{red}{cell vertices} are marked with red dots and their corresponding selected cells are marked with black and white. 


Next, we select observations to optimize $\mathbf{x}^{sM}$. Cells surrounded by vertices with indices in $\mathbb{I}^s$ are designated as selected cells, shown in white in Fig. \ref{fig_low_high_select}(a) and Fig. \ref{fig_low_high_select}(b). Subsequently, sampling point selection is carried out, as illustrated in Fig. \ref{fig_select_sampling_point}. Specifically, sampling points in $\mathbb{S}^{h}$ are first projected onto the global coordinate system using the poses optimized in the first stage. All sampling points located on the selected cells are then included to form the set $\mathbb{S}^{s}$. 

\begin{figure}[t]
\centering 
\includegraphics[width=0.48\textwidth]{./Select_Sampling_Point.pdf}
\caption{\label{fig_select_sampling_point} An example of the selected sampling points of a beam at time step $i$, where points projected onto the selected cells are chosen.}
% \vspace{-0.5em}
\end{figure}


\section{Submap Joining} \label{Sec_submap}
In Section \ref{Sec_multi}, we introduced a multi-resolution joint optimization strategy to efficiently solve our NLLS problem. For large-scale occupancy SLAM with long robot trajectories, the number of poses to optimize can be very large. To make the computational complexity dependent only on the environment size rather than the trajectory length, in this section we propose an occupancy submap joining method. The key idea is to reduce the number of poses that need to be optimized to the number of local submaps. 


\subsection{Inputs and Outputs of Submap Joining Problem} 
We first separate the observation information into multiple parts and perform Algorithm \ref{alg_flowchart} to build several submaps. The inputs of submap joining problem are a sequence of local occupancy submaps. 
Let us denote the $n_L+1$ submaps as $\mathbb{M}_L = \{\mathbb{M}_{L_0}, \cdots, \mathbb{M}_{L_{n_L}}\}$ and the associated coordinate frames of these local occupancy maps are denoted as $ \{\mathbf{x}^P_{0}, \cdots, \mathbf{x}^P_{n_L}\}$, where $\mathbb{M}_{L_{i_L}}$ and $\mathbf{x}^P_{i_L}$represents the $i_L$th local occupancy map and its associated coordinate frame. In addition, the global occupancy map is represented as $\mathbb{M}_G=\{M_G(\mathbf{m}^G_{00}), \cdots,{M}_G\left(\mathbf{m}^G_{c_w^Gc_h^G}\right)\}$. Both the global map and local maps follow the same definition as described in Section \ref{sec_discrete_occupancy}. The outputs of submap joining problem are the optimal solution of the local submap coordinate frames and the optimal global occupancy map.

\subsection{NLLS Formulation of Submap Joining Problem} 
First, the cell vertex $\mathbf{m}^G_{wh}$ in the global occupancy map $\mathbb{M}_G$ can be projected to local submap coordinate by pose $\mathbf{x}^P_{i_L}$, i.e., 
\begin{equation}
	\mathbf{p}_{i_L}^{wh} = \frac{ \mathbf{R}_{i_L} (\mathbf{m}^G_{wh} \cdot s_G  - \mathbf{t}_{i_L})}{s_L}.
\end{equation}
Here, $\mathbf{p}_{i_L}^{wh}$ represents the position in the local submap's coordinate where the cell vertex $\mathbf{m}_{wh}^G$ from the global map is projected using the pose $\mathbf{x}^P_{i_L}$. The resolutions of the global occupancy map and local submaps are denoted by $s_G$ and $s_L$, respectively.

The submap joining problem aims to find the optimal global occupancy map and the poses of submap coordinate frames. Thus, the state vector for this problem is defined as 
\begin{equation}
	\mathbf{x}_G = [{\mathbf{x}^P_L}^\top, {{\mathbf{x}^M_G}}^\top]^\top,
\end{equation}
where 
\begin{equation}
\begin{aligned}
\mathbf{x}^P_L & =\left[\left(\mathbf{x}^P_1\right)^\top, \cdots,\left(\mathbf{x}^P_{n_L}\right)^\top\right]^\top \\
\mathbf{x}^M_G & =\left[{M}_G\left(\mathbf{m}^G_{00}\right), \cdots, {M}_G\left(\mathbf{m}^G_{c_w^Gc_h^G}\right)\right]^\top.
\end{aligned}
\end{equation}
As with most submap joining problem formulations, we fix the first local map coordinate frame as the global coordinate frame. Therefore, $\mathbf{x}^P_L$ consists of $n_L$ local map coordinate frames and $\mathbf{x}^M_G$ includes $(c_w^G+1) \times (c_h^G+1)$ discrete cell vertices of global occupancy map. 

By the global-to-local projection relationship, all cell vertices of global occupancy map $\mathbb{M}_G$ can be projected to corresponding submaps to compute the difference in occupancy values. Thus, the NLLS problem of occupancy submap joining can be formulated to minimize 
\begin{equation}
\begin{adjustbox}{max width=\linewidth}
$
g(\mathbf{x}_G) =  \sum\limits_{i_L=0}^n\sum\limits_{ wh \in \mathbb{S}^L_{i_L}} \left\| \omega(i_L,\mathbf{m}^G_{wh}) {M}_G(\mathbf{m}^G_{wh}) - {M}_{L_{i_L}}(\mathbf{p}_{i_L}^{wh}) \right\|^2,
$
\end{adjustbox}
\label{eq_NLLS_joining}
\end{equation}
where $\mathbb{S}^L_{i_L}$ represents the set of indices of cell vertices in the global occupancy map $\mathbb{M}_G$ that are projected onto the local submap $\mathbb{M}_{L_{i_L}}$.

In (\ref{eq_NLLS_joining}), $\omega(i_L,\mathbf{m}^G_{wh})$ is the weight to establish an accurate relationship between the global occupancy map and local submaps w.r.t. occupancy values, which can be calculated by
\begin{equation}
    \omega(i_L,\mathbf{m}^G_{wh}) = \frac{{N}_{{L_{i_L}}}(\mathbf{p}_{i_L}^{wh})}{{N}_{G}(\mathbf{m}^G_{wh})}.
\end{equation}
Here, ${N}_{{L_{i_L}}}(\cdot)$ is the local hit number lookup function for submap $\mathbb{M}_{L_{i_L}}$, derived as described in Section \ref{sec_hit}. It approximates the hit number at coordinate $\mathbf{p}_{i_L}^{wh}$ using bilinear interpolation. Similarly, ${N}_{G}(\cdot)$ represents the global hit number lookup function associated with $\mathbb{M}_G$.

In (\ref{eq_NLLS_joining}), the submap joining problem is formulated as a NLLS problem, which can be solved iteratively by Gauss-Newton based method similar to Algorithm \ref{alg_1}.


 \begin{table}[htp]
		\centering
		\caption{Parameters of Datasets. \label{tab_dataset}}
		\label{tab_comparison}
		\setlength{\tabcolsep}{0.7mm}{
		\begin{tabular}{lccccc}\toprule
		Dataset	& No. Scans & Duration  & Map Size &  Odometry & Resolution\\ \hline
		Simulation 1 & 3640  &117 s& $50$ m  $\times$ $50$ m & yes & 0.05 m\\
        Simulation 2 & 3720  &121 s& $50$ m $\times$ $50$ m & yes & 0.05 m\\
		Simulation 3  & 2680  & 83 s& $50$ m $\times$ $50$ m & yes & 0.05 m\\
		Car Park  & 1642 & 164 s& $50$ m $\times$ $40$ m & yes & 0.1 m\\
		C5  & 3870  &136 s& $50$ m $\times$ $40$ m & yes & 0.1 m\\
		Museum b0 & 5522 &152 s& $85$ m $\times$ $95$ m &no & 0.1 m \\
		Museum b2 & 51833 &1390 s &  $250$ m $\times$ $200$ m &no & 0.1 m\\
        C3 &24402 &610 s& $150$ m $\times$ $125$ m  & no & 0.1 m\\
		\hline
		\end{tabular}
		}
\end{table}


\section{Experimental Results} \label{Sec_experiment}

In this section, we evaluate our algorithm on several datasets and compare its performance with Cartographer \cite{hess2016real}, the current state-of-the-art algorithm, which significantly outperforms other methods such as Hector-SLAM \cite{kohlbrecher2011flexible} and Karto-SLAM \cite{konolige2010efficient}. To ensure fair comparisons, we adjust some parameters in Cartographer based on the sensor configurations of the respective datasets for optimal performance.


The dataset parameters are summarized in Table \ref{tab_dataset}. For practical datasets, Deutsches Museum b0 and Deutsches Museum b2 are Cartographer demo datasets collected at the Deutsches Museum. The Car Park \cite{zhao20212d} dataset is gathered in an underground car park, while C5 and C3 are collected in a factory environment using a Hokuyo UTM-30LX laser scanner. Consistent map resolutions $s$ are applied across all methods to display the map results, with ratio $r$ set to $10$ for all simulation experiments and $5$ for all practical experiments unless stated otherwise. For each dataset, 20\% of scans and corresponding poses are uniformly selected as key frames for the key frame option in our method.

To ensure fair comparisons, we use an identical number of poses (synchronizing the poses from the results with the ground truth poses using timestamps) and their corresponding observations to generate results for visualization and quantitative evaluation across all compared methods, with the exception of our method that employs keyframes. Furthermore, the same occupancy mapping algorithm is applied consistently across all approaches to produce the occupancy grid map results for comparison.

        
\subsection{Simulation Experiments}\label{simu_experiment}

\begin{figure*}[tp]
\centering \subfigure[Simulation 1] {\label{fig_trajectory_1}
\includegraphics[width=0.28\textwidth]{./trajectory_simu1_new.pdf}}
\centering \subfigure[Simulation 2] {\label{fig_trajectory_2}
\includegraphics[width=0.28\textwidth]{./trajectory_simu2_new.pdf}}
\centering \subfigure[Simulation 3] {\label{fig_trajectory_3}
\includegraphics[width=0.343\textwidth]{./trajectory_simu3_new.pdf}}
\caption{\label{fig_trajectory_compare}Simulation environments and robot trajectory results. (a), (b) and (c) show the simulation environments (the black lines indicate the obstacles in the scene) and the trajectories of ground truth, odometry inputs, Cartographer \cite{hess2016real}, and our approach for one dataset in each of the three simulation environments.}
% \vspace{-0.5em}
\end{figure*}

We use three different simulation environments with varying levels of nonlinearity and nonconvex obstacles to design three different simulation experiments. Since Cartographer needs a high-frequency scanning rate to ensure the good performance of scan matching, while our approach performs well for scan data with low scanning frequency, only 10\% scans listed in Table \ref{tab_dataset} are used in our method. 

We utilize the open-source 2D LiDAR simulator from \cite{zhao20212d} to generate simulated datasets. Each scan includes 1081 laser beams spanning angles from -135 degrees to 135 degrees, mimicking the specifications of a Hokuyo UTM-30LX laser scanner. To emulate real-world data acquisition, random Gaussian noise with zero mean and standard deviation of $0.02$ m is added to each beam of the simulated scan data. Similarly, zero-mean Gaussian noise is introduced to the odometry inputs derived from the ground truth poses, with standard deviation of $0.04$ m for $x$-$y$ and $0.003$ rad for orientation. Five datasets with different noise realizations are generated for each simulation environment.


\begin{figure*}[t]
\centering \subfigure[Simulation 1] {\label{fig_time_error_1}
\includegraphics[width=0.32\textwidth]{./Time_with_Error_Simu1.pdf}}
\centering \subfigure[Simulation 2] {\label{fig_time_error_2}
\includegraphics[width=0.32\textwidth]{./Time_with_Error_Simu2.pdf}}
\centering \subfigure[Simulation 3] {\label{fig_time_error_3}
\includegraphics[width=0.32\textwidth]{./Time_with_Error_Simu3.pdf}}
\caption{Comparison of translation and rotation errors at different time steps using simulation datasets.}
\label{fig_error_compare_time}
\vspace{-1em}
\end{figure*}


% The robot trajectory results of our method and Cartographer using one dataset in each simulation are compared with the ground truth and odometry in Fig. \ref{fig_trajectory_compare}. It is clear that our trajectories are closer to the ground truth trajectories, especially for positions where significant rotation occurs. Fig. \ref{fig_error_compare_time} shows the translation and rotation errors of our method and Cartographer at different time steps. Obviously, the errors of our method are substantially smaller than those of Cartographer.

The trajectory results of our method and Cartographer, compared to ground truth and odometry, are shown in Fig. \ref{fig_trajectory_compare}. It is evident that our trajectories align more closely with the ground truth, particularly in areas with significant rotational movements. Fig. \ref{fig_error_compare_time} illustrates translation and rotation errors over time, demonstrating that our method consistently achieves substantially smaller errors compared to Cartographer.

% We use all the fifteen datasets from Simulation 1, Simulation 2 and Simulation 3 to perform the quantitative and qualitative comparison of errors in the pose estimates. The quantitative results of Cartographer, only the first stage in our method \textcolor{red}{(Algorithm \ref{alg_1} using low-resolution)} using all frames, our method using all frames, and our method using key frames are given in Table \ref{tab_comparison}. We use mean absolute error (MAE) and root mean squared error (RMSE) to evaluate the translation errors (in meters) and rotation errors (in radians). Our method performs the best in all four metrics for all simulations and is substantially ahead of Cartographer even when using only key frames or only the first stage. In addition, Fig. \ref{fig_simulation}(a) to Fig. \ref{fig_simulation}(e) show the occupancy grid maps and point cloud maps generated using poses from ground truth, Cartographer and the three options of our method. It is clear that the boundaries of both occupancy grid maps and point cloud maps using the three options of our method are much clearer than those from Cartographer, which indicates that our method can obtain more accurate results by optimizing the robot poses and the occupancy map together.

We performed quantitative and qualitative comparisons of pose estimation errors using all fifteen datasets from Simulations 1, 2, and 3. Table \ref{tab_comparison} presents, in order, the quantitative results for odometry inputs, Cartographer, the first stage of our method (Algorithm \ref{alg_1} with low-resolution) using all frames, our method using all frames, and our method using key frames. Metrics such as mean absolute error (MAE) and root mean squared error (RMSE) evaluate translation errors (in meters) and rotation errors (in radians). Our method consistently achieves the best performance across all metrics, significantly outperforming Cartographer even when using only key frames or the first stage. Fig. \ref{fig_simulation}(a) to Fig. \ref{fig_simulation}(e) further illustrates occupancy grid maps and point cloud maps generated using poses from the ground truth, Cartographer, and the three options of our method. The maps produced by our method exhibit noticeably clearer boundaries, demonstrating its ability to jointly optimize robot poses and occupancy maps for more accurate results.

\begin{figure}[t]
\centering
\includegraphics[width=0.48\textwidth]{./Simulation_New.pdf}
\caption{\label{fig_simulation} The occupancy grid maps and point cloud maps generated from ground truth poses and different approaches for each simulation dataset. The areas marked with red dots highlight where our method outperforms the results of the first-stage optimization alone.}
\vspace{-0.5em}
\end{figure}

\begin{table}[t]
		\centering
		\caption{Quantitative Comparison of Robot Pose Errors in Simulations.}
		\label{tab_comparison}
		\setlength{\tabcolsep}{0.7 mm}{
		\begin{tabular}{lccccc}\toprule
			& Odom & Carto & Ours (First) & Ours (All) & Ours (Key) \\ \hline
		Simulation 1& & & &\\
		\quad MAE / Trans (m) & 0.78270 & 0.25336  & 0.02206 &\textcolor{red}{\textbf{0.00640}} & \textcolor{blue}{\textbf{0.01024}}\\
		\quad MAE / Rot (rad) & 0.04912 & 0.01394  & 0.00098 &\textcolor{red}{\textbf{0.00060}} & \textcolor{blue}{\textbf{0.00084}}
\\
		\quad RMSE / Trans (m) & 0.98404 & 0.29920  & 0.02680 &\textcolor{red}{\textbf{0.00974}} & \textcolor{blue}{\textbf{0.01430}}\\
		\quad RMSE / Rot(rad) & 0.05506 & 0.01562  & 0.00162 &\textcolor{red}{\textbf{0.00102}} & \textcolor{blue}{\textbf{0.00126}}\\\hline
		
		Simulation 2& & & &\\
		\quad MAE / Trans (m) & 0.80544 & 0.11914 &0.03224 & \textcolor{red}{\textbf{0.00858}} & \textcolor{blue}{\textbf{0.01082}}
\\
		\quad MAE / Rot (rad) & 0.02538 & 0.00666  &0.00220 &\textcolor{red}{\textbf{0.00062}} & \textcolor{blue}{\textbf{0.00096}}\\
		\quad RMSE / Trans (m) & 0.97152 & 0.14810  & 0.04188 &\textcolor{red}{\textbf{0.01198}} & \textcolor{blue}{\textbf{0.01244}}\\
		\quad RMSE / Rot (rad) & 0.02936 & 0.00916 &0.00220 & \textcolor{red}{\textbf{0.00104}} & \textcolor{blue}{\textbf{0.00178}}\\\hline
		
		Simulation 3& & & &\\
		\quad MAE / Trans(m) & 0.75352 & 0.14262  &0.02624 &\textcolor{red}{\textbf{0.00726}} &  \textcolor{blue}{\textbf{0.00998}}\\
		\quad MAE / Rot (rad) & 0.05180 & 0.00682 &0.00164  & \textcolor{red}{\textbf{0.00058}} &  \textcolor{blue}{\textbf{0.00090}}\\
		\quad RMSE / Trans (m) & 0.96866 & 0.18782 & 0.03238 &\textcolor{red}{\textbf{0.00952}} &  \textcolor{blue}{\textbf{0.01338}}\\
		\quad RMSE / Rot (rad) & 0.05926 & 0.00914  & 0.00204 &\textcolor{red}{\textbf{0.00088}} &  \textcolor{blue}{\textbf{0.00134}}\\\hline
		\end{tabular}
	\begin{tablenotes}
     \item \textcolor{red}{\textbf{Red}} and  \textcolor{blue}{\textbf{blue}} indicate the best and second best results, respectively.
   \end{tablenotes}
		}
        % \vspace{-2em}
\end{table}

% \begin{table*}[htp]
% \centering
% \caption{Occupancy Grid map Precision of Our Method Using All Frames, Our Method Using Key Frames, and Cartographer.}
% \label{tab_map_accuracy}
% \setlength{\tabcolsep}{2.4mm}
% \begin{NiceTabular}{cccccccccccc}[first-row,first-col,hvlines]
% \CodeBefore
% \Body
%  & \Block{1-11}{\textbf{Predicted}} & & & & & & &  & & & \\
% \Block{11-1}{\rotate Ground Truth}  & \Block{2-1}{}  & \Block{2-1}{} & \Block{1-3}{Our Method (All Frames)} & & & \Block{1-3}{Our Method (Key Frames)} & & & \Block{1-3}{Cartographer}  \\
%  & &   & Unknown & Free & Occupied & Unknown & Free & Occupied & Unknown & Free & Occupied \\
%  & \Block{3-1}{Simulation 1} & Unknown  & \textcolor{red}{\textbf{99.798\%}}  & 0.020\% & 0.182\% & \textcolor{blue}{\textbf{99.598\%}} & 0.072\% & 0.330\% & 95.616\% & 2.822\% & 1.562\% \\
%  &   &  Free & 0.022\%  & \textcolor{red}{\textbf{99.938\%}} & 0.040\% & 0.094\% & \textcolor{blue}{\textbf{99.824\%}} & 0.082\% & 1.290\% & 97.678\% & 1.032\% \\
%  &   & Occupied  &  5.436\% & 2.334\%  & \textcolor{red}{\textbf{92.230\%}} & 13.562\% & 3.142\% &\textcolor{blue}{\textbf{83.296\%}} &30.053\% & 53.280\% & 16.667\% \\

% & \Block{3-1}{Simulation 2} & Unknown  & \textcolor{red}{\textbf{99.846\%}}  & 0.010\% & 0.144\% & \textcolor{blue}{\textbf{99.696\%}} & 0.062\% & 0.242\% & 96.868\% & 1.743\% & 1.389\% \\
%  &   &  Free & 0.016\%  & \textcolor{red}{\textbf{99.846\%}} & 0.138\% & 0.076\% & \textcolor{blue}{\textbf{99.848\%}} & 0.076\% & 0.593\% & 98.584\% & 0.823\% \\
%  &   & Occupied  &  6.434\% & 3.738\%  & \textcolor{red}{\textbf{89.828\%}} & 11.694\% & 2.806\% &\textcolor{blue}{\textbf{85.500\%}} &23.943\% & 50.795\% & 25.262\% \\

%  & \Block{3-1}{Simulation 3} & Unknown  & \textcolor{red}{\textbf{99.812\%}}  & 0.032\% & 0.156\% & \textcolor{blue}{\textbf{99.258\%}} & 0.430\% & 0.312\% & 96.968\% & 1.574\% & 1.458\% \\
%  &   &  Free & 0.036\%  & \textcolor{red}{\textbf{99.928\%}} & 0.036\% & 0.554\% & \textcolor{blue}{\textbf{99.352\%}} & 0.094\% & 1.018\% & 98.110\% & 0.872\% \\
%  &   & Occupied  &  4.500\% & 2.358\%  & \textcolor{red}{\textbf{93.142\%}} & 16.788\% & 3.736\% &\textcolor{blue}{\textbf{79.476\%}} &26.420\% & 44.928\% & 28.652\% \\
% \end{NiceTabular}
% % \vspace{-1em}
% \end{table*}

\begin{table*}[htp]
\centering
\caption{Occupancy Grid map Precision of Our Method Using All Frames, Our Method Using Key Frames, and Cartographer.}
\label{tab_map_accuracy}
\setlength{\tabcolsep}{2.4mm}
\renewcommand{\arraystretch}{1.2}

\begin{tabular}{c c c c c c c c c c c c}
\toprule
 \multirow{2}{*}{} & \multirow{2}{*}{\textbf{Ground Truth}}& \multicolumn{3}{c}{Our Method (All Frames)} & \multicolumn{3}{c}{Our Method (Key Frames)} & \multicolumn{3}{c}{Cartographer} \\
\cmidrule(lr){3-5} \cmidrule(lr){6-8} \cmidrule(lr){9-11}
& & Unknown & Free & Occupied & Unknown & Free & Occupied & Unknown & Free & Occupied \\
\midrule
\multirow{3}{*}{Simulation 1} 
& Unknown  & \textcolor{red}{\textbf{99.798\%}}  & 0.020\% & 0.182\% & \textcolor{blue}{\textbf{99.598\%}} & 0.072\% & 0.330\% & 95.616\% & 2.822\% & 1.562\% \\
& Free     & 0.022\%  & \textcolor{red}{\textbf{99.938\%}} & 0.040\% & 0.094\% & \textcolor{blue}{\textbf{99.824\%}} & 0.082\% & 1.290\% & 97.678\% & 1.032\% \\
& Occupied & 5.436\%  & 2.334\%  & \textcolor{red}{\textbf{92.230\%}} & 13.562\% & 3.142\% & \textcolor{blue}{\textbf{83.296\%}} & 30.053\% & 53.280\% & 16.667\% \\
\midrule
\multirow{3}{*}{Simulation 2} 
& Unknown  & \textcolor{red}{\textbf{99.846\%}}  & 0.010\% & 0.144\% & \textcolor{blue}{\textbf{99.696\%}} & 0.062\% & 0.242\% & 96.868\% & 1.743\% & 1.389\% \\
& Free     & 0.016\%  & \textcolor{red}{\textbf{99.846\%}} & 0.138\% & 0.076\% & \textcolor{blue}{\textbf{99.848\%}} & 0.076\% & 0.593\% & 98.584\% & 0.823\% \\
& Occupied & 6.434\%  & 3.738\%  & \textcolor{red}{\textbf{89.828\%}} & 11.694\% & 2.806\% & \textcolor{blue}{\textbf{85.500\%}} & 23.943\% & 50.795\% & 25.262\% \\
\midrule
\multirow{3}{*}{Simulation 3} 
& Unknown  & \textcolor{red}{\textbf{99.812\%}}  & 0.032\% & 0.156\% & \textcolor{blue}{\textbf{99.258\%}} & 0.430\% & 0.312\% & 96.968\% & 1.574\% & 1.458\% \\
& Free     & 0.036\%  & \textcolor{red}{\textbf{99.928\%}} & 0.036\% & 0.554\% & \textcolor{blue}{\textbf{99.352\%}} & 0.094\% & 1.018\% & 98.110\% & 0.872\% \\
& Occupied & 4.500\%  & 2.358\%  & \textcolor{red}{\textbf{93.142\%}} & 16.788\% & 3.736\% & \textcolor{blue}{\textbf{79.476\%}} & 26.420\% & 44.928\% & 28.652\% \\
\bottomrule
\end{tabular}
\end{table*}


From the results of our first stage shown in Fig. \ref{fig_simulation}(c), it is evident that further optimization is needed at the edges of objects. This is due to sampling points with different occupancy values being projected onto the coarse grid cells at object boundaries, causing inaccurate data associations. These results highlight the necessity of the second stage in our multi-resolution strategy (Algorithm \ref{alg_3}) to improve accuracy. Additionally, as shown in Fig. \ref{fig_simulation}(c), the non-edge areas (stable areas) of the occupancy grid maps are well optimized, supporting the fact that including occupancy cell vertices of non-edge areas in the state variables for further optimization is unnecessary.

For a quantitative comparison of the occupancy maps, we apply the same threshold across all methods to convert occupancy values into occupancy states. The mapping problem is treated as a classification task, categorizing each grid cell as free, occupied, or unknown. The mapping performance of our method and Cartographer is summarized in Table \ref{tab_map_accuracy}, clearly showing that both variants of our method, one using all frames and the other using keyframes, significantly outperform Cartographer in terms of map accuracy.

\begin{table}[ht]
		\centering
		\caption{Accuracy of the Occupancy Grid Map.}
		\label{tab_auc}
		\setlength{\tabcolsep}{4.8 mm}{
		\begin{tabular}{llccccc}\toprule
		& & AUC & Precision   \\ \hline
		\multirow{3}{*}{Simulation 1}& Cartographer & 0.90878 & 0.95651 \\ & Ours (All) &\textcolor{red}{\textbf{0.99999}} &\textcolor{red}{\textbf{0.99773}}\\ & Ours (Key) & \textcolor{blue}{\textbf{0.99902}} & \textcolor{blue}{\textbf{0.99548}} \\ \hline 
			\multirow{3}{*}{Simulation 2}& Cartographer & 0.96132 &  0.96829 \\ & Ours (All) & \textcolor{red}{\textbf{0.99926}} & \textcolor{red}{\textbf{0.99721}} \\ & Ours (Key) & \textcolor{blue}{\textbf{0.99914}} & \textcolor{blue}{\textbf{0.99638}}\\ \hline
		\multirow{3}{*}{Simulation 3} & Cartographer & 0.92696 &0.96592 \\ & Ours (All)& \textcolor{red}{\textbf{0.99974}} & \textcolor{red}{\textbf{0.99771}} \\ & Ours (Key) & \textcolor{blue}{\textbf{0.99748}} & \textcolor{blue}{\textbf{0.99113}}\\
		 \hline
		\end{tabular}
  }
  % \vspace{-0.5em}
\end{table}

We also assess performance using AUC (Area under the ROC curve) \cite{bradley1997use} and precision, with ground truth labels generated from the occupancy map based on ground truth poses. To ensure a fair comparison, all unknown cells are excluded from this evaluation, as AUC is a binary classification metric \cite{bradley1997use}. Table \ref{tab_auc} presents the results, showing that our method using all frames achieves the highest performance in both metrics. Even with only key frames, our method surpasses Cartographer. A key factor resulting in Cartographer's lower mapping quality is its lack of a batch optimization method to address errors during submap construction. Although its scan-to-map matching approach reduces cumulative errors more effectively than scan-to-scan matching, its accuracy still falls short compared to our algorithm. Cartographer performs pose graph optimization to adjust the coordinate frames of submaps only when loop closure is detected, leaving errors within the submaps uncorrected. Although global pose graph optimization is applied at the end of the process, it often suffers from an excess of inaccurate and conflicting relative measurements, as well as its susceptibility to local minima, limiting its effectiveness in correcting these errors. Moreover, pose graph optimization typically does not enhance the local details of maps, as it focuses solely on optimizing poses without jointly considering the map. This further highlights the advantage of our approach, which jointly optimizes both robot poses and the occupancy map.


\subsection{Comparisons using Practical Datasets} \label{sec_practical}

\begin{figure}[t]
\centering
\includegraphics[width=0.46\textwidth]{Real_OGM_New.pdf}
\caption{\label{fig_result_compare_OGM} The occupancy grid maps from Cartographer, our method using all frames, and our method using key frames. }
% \vspace{-1.5em}
\end{figure}


\begin{figure}[t]
\centering
\includegraphics[width=0.46\textwidth]{Real_Scan_New.pdf}
\caption{\label{fig_result_compare_scan} The point cloud maps from Cartographer, our method using all frames, and our method using key frames.}
\end{figure}

\begin{table}[t]
		\centering
		\caption{Time Consumption of Different Algorithms.}
		\label{table_time_compare}
		\setlength{\tabcolsep}{2.5 mm}{
		\begin{tabular}{lccc}\toprule
		Dataset& & Computational Time (s)& \\ \hline
			     & Cartographer  & Ours (All) & Ours (Key)  \\ 
		Car Park & 168  & 119  & \textbf{44} \\
		Museum b0 & 152  & 126 & \textbf{38} \\
		C5 & 146 & 137 & \textbf{35} \\
		% Simulation 1 & 192  & 148 & \textbf{33} \\
		% Simulation 2 & 174  & 193 & \textbf{57} \\
		% Simulation 3 & 78  & 132 & \textbf{40} \\
		\hline
		\end{tabular}
		}
        % \vspace{-2em}
\end{table}

We use three normal-scale practical datasets, namely Deutsches Museum b0 \cite{hess2016real}, Car Park \cite{zhao20212d} and C5, to compare our method with Cartographer in terms of the constructed occupancy grid maps and optimized poses. 

The mapping quality is evaluated by comparing the details of the constructed maps. Additionally, point cloud maps, which are generated using the endpoint projections of scan points and optimized poses, serve as a reference for pose accuracy. For the Car Park and C5 datasets, our method is initialized with poses from odometry inputs, whereas for the Museum b0 dataset, initialization relies on poses from scan matching due to the absence of odometry. The occupancy grid maps and point cloud maps generated by Cartographer, our method using all frames, and our method using key frames for the three datasets are shown in Fig. \ref{fig_result_compare_OGM} and Fig. \ref{fig_result_compare_scan}. Red dotted lines highlight areas where our results outperform Cartographer in both the occupancy grid maps and point cloud maps. Comparing Fig. \ref{fig_result_compare_OGM}(a) and Fig. \ref{fig_result_compare_OGM}(b), our method provides more precise boundaries for the occupancy grid maps due to joint optimization of robot poses and the occupancy map. Similarly, the comparison between Fig. \ref{fig_result_compare_scan}(a) and Fig. \ref{fig_result_compare_scan}(b) illustrates that our method achieves more accurate poses. 

Moreover, our method outperforms Cartographer when using only key frames, as evident from the comparison of Fig. \ref{fig_result_compare_OGM}(a) and Fig. \ref{fig_result_compare_scan}(a) with Fig. \ref{fig_result_compare_OGM}(c) and Fig. \ref{fig_result_compare_scan}(c). These results show that, despite Cartographer introducing loop closure detection, it still produces non-negligible pose errors, leading to point clouds that fail to fully overlap observations of the same obstacle at different poses. While the point cloud maps generated by our method also have non-overlapping parts, these areas are significantly smaller compared to those from Cartographer. 

% These experiments demonstrate that both variants of our method reduce pose errors and generate more accurate occupancy grid maps by jointly optimizing robot poses and the occupancy map. 

Additionally, we assess the time consumption of our method and Cartographer on these three datasets. Table \ref{table_time_compare} shows that our method consistently requires less time than Cartographer across all datasets when using all frames and achieves significantly better efficiency when using selected key frames.

Finally, it is worth noting that some well-known public datasets, such as Radish \cite{Radish}, were collected before 2014 with outdated sensors, leading to low-quality data with poor scanning frequency and odometry accuracy. These issues hinder the performance of Cartographer, often requiring meticulous parameter tuning but still yielding suboptimal results. In contrast, our method performs well on these datasets. Although we do not include these comparisons in this paper, we make our results available on our code page\footnote{\url{https://github.com/WANGYINGYU/Occupancy-SLAM}}.





\subsection{Assessment of Robustness to Initial Guess}

% While our method has demonstrated robustness when initialized with odometry inputs or scan matching under reliable sensor conditions in both simulation and real-world experiments, this subsection illustrates its capability for convergence even when initialized with significantly noisy poses. In this subsection, we use all frames for robustness assessment.

While our method has demonstrated robustness when initialized with odometry inputs or scan matching under reliable sensor conditions in both simulation and real-world experiments, this subsection highlights its capability to converge even when initialized with significantly noisy poses. We use all frames in this subsection to assess robustness.

First, we use Simulation 1 dataset to quantitatively evaluate the convergence percentage and the accuracy of optimized poses under different noise levels. We add zero-mean uniformly distributed noises with different bounds to the ground truth of the poses to generate each group of ten sets of initial poses for the experiments to count convergence rates and average errors. Specifically, for noise level 1, the noise for translation is within $[-2$ m, $2$ m$]$ and the noise for rotation is within $[-0.5$ rad, $0.5$ rad$]$; for level 2, $[-4$ m, $4$ m$]$ and $[-1$ rad, $1$ rad$]$; for level 3, $[-6$ m, $6$ m$]$ and $[-1.5$ rad, $1.5$ rad$]$. The poses with different noise levels of Simulation 1 dataset are visualized using the generated occupancy grid maps, as shown in Fig. \ref{fig_OGM_246}. The convergence results are depicted in Table \ref{table_robustness}, showing that our method can $100\%$ converge when initialized with challenging noisy poses of level 1 and level 2. Our method still has a high convergence percentage ($80\%$) when initialized with noisy poses of level 3. Our algorithm using other simulation datasets has similar robustness performance.

\begin{table}[t]
		\centering
		\caption{Robustness to Initialization.}
		\label{table_robustness}
		\setlength{\tabcolsep}{0.9 mm}{
		\begin{tabular}{lccc}\toprule
		\thead{Noise Level} & \thead{Convergence\\ Percentage} & \thead{Average MAE of \\Translation (m)} & \thead{Average MAE of\\ Rotation (rad)}\\ \hline
		Level 1 (2 m, 0.5 rad)  & 100\% & 0.00679 & 0.0005   \\
		Level 2 (4 m, 1 rad)  & 100\% & 0.00682 & 0.0005  \\
		Level 3 (6 m, 1.5 rad)  & 80\% & 0.01742  & 0.0012 \\
		\hline
		\end{tabular}
		}
\end{table}

\begin{figure}[t]
\centering
\includegraphics[width=0.47\textwidth]{./OGM_Level246.pdf}
\caption{\label{fig_OGM_246} Examples of occupancy grid maps generated from poses with different noise levels as shown in Table \ref{table_robustness} using Simulation 1 dataset.}
% \vspace{-1.5em}
\end{figure}

Moreover, for all practical datasets, we additionally add random zero-mean uniform distribution noises ($[-2$ m, $2$ m$]$ for translation and $[-0.5$ rad, $0.5$ rad$]$ for rotation) to the poses obtained from Cartographer as the initial guess. The initial occupancy maps obtained by using the noisy initial poses are shown in Fig. \ref{fig_noise_initial}(a). Fig. \ref{fig_noise_initial}(b) shows the remapped occupancy grid maps using our optimized poses, and Fig. \ref{fig_noise_initial}(c) shows the point cloud maps using our optimized poses. This experiment shows that our approach can converge from initial guesses with significant errors and also generate good results.


\begin{figure}[t]
% \vspace{-5mm}
\centering
\includegraphics[width=0.5\textwidth]{Noise_Initial.pdf}
\caption{\label{fig_noise_initial}The occupancy grid maps and point cloud maps generated using noisy poses for initialization by our approach. (a) and (b) display the remapped occupancy maps generated from the noisy initial poses and our optimized poses, respectively, and (c) shows the point cloud maps created by projecting the endpoints of scans using our optimized poses.}
% \vspace{-2em}
\end{figure}

\subsection{Discussion about the Effectiveness of Different Stages} \label{sec_discuss}


\begin{figure*}[tp]
\centering \subfigure[Simulation 1] {\label{fig_group_error_1}
\includegraphics[width=0.32\textwidth]{./Group_Error_3640.pdf}}
\centering \subfigure[Simulation 2] {\label{fig_group_error_2}
\includegraphics[width=0.32\textwidth]{./Group_Error_3720.pdf}}
\centering \subfigure[Simulation 3] {\label{fig_group_error_3}
\includegraphics[width=0.32\textwidth]{./Group_Error_2680.pdf}}
\caption{Comparison of translation and rotation errors for simulated datasets using three methods: our full method (Algorithm \ref{alg_flowchart}), our Algorithm \ref{alg_1} initialized by Cartographer's poses with a high-resolution map \cite{Zhao-RSS-22}, and our Algorithm \ref{alg_1} initialized by poses obtained from our first stage with a high-resolution map.}
\label{fig_group_error}
\vspace{-1em}
\end{figure*}


In previous sections, we demonstrated the accuracy, robustness, and efficiency of our proposed method. In this section, we discuss the effectiveness of its different parts.


As demonstrated in Table \ref{tab_comparison}, Fig. \ref{fig_simulation}, Fig. \ref{fig_result_compare_OGM}, and Fig. \ref{fig_result_compare_scan}, the accuracy of the poses and the map obtained from our full approach (Algorithm \ref{alg_flowchart}) is much better than those obtained from Cartographer. This confirms the advantage of jointly optimizing both the robot poses and the occupancy map. 


% One may ask, how about performing only Algorithm \ref{alg_1} with a high-resolution map directly? Will the result be even better? To answer this question clearly, in this subsection, we compare our full approach with Algorithm \ref{alg_1} using a high-resolution map. We consider three different initialization: 

% In the following, we refer to these three approaches as  \textit{Algorithm \ref{alg_1} (High, O/S)}, \textit{Algorithm \ref{alg_1} (High, Carto)}, and \textit{Algorithm \ref{alg_1} (High, First)}, respectively.   

One potential question is whether using only Algorithm \ref{alg_1} with a high-resolution map would yield even better results. To investigate this, we compared our full approach with Algorithm \ref{alg_1} using a high-resolution map. We tested three initialization: (1) \textit{Algorithm \ref{alg_1} (High, O/S)}: initialization using odometry inputs or scan matching; (2) \textit{Algorithm \ref{alg_1} (High, Carto)}: initialization using Cartographer's poses (as proposed in our conference paper \cite{Zhao-RSS-22}); and (3) \textit{Algorithm \ref{alg_1} (High, First)}: initialization using the poses obtained by our first stage. 



% First, \textit{Algorithm \ref{alg_1} (High, O/S)} fails to converge on most datasets, while our full method can converge very well, which indicates the improved robustness of our multi-resolution strategy.   

% The comparison of our full method with \textit{Algorithm \ref{alg_1} (High, Carto)} and \textit{Algorithm \ref{alg_1} (High, First)} using all the five groups simulation datasets are shown in Fig. \ref{fig_group_error}. It can be seen that the accuracy of our full method is essentially similar in all groups, while the accuracy of \textit{Algorithm \ref{alg_1} (High, Carto)} varies drastically. This means the approach proposed in our conference paper \cite{Zhao-RSS-22} not only requires an accurate initial value but also generates less accurate poses than our new approach. In addition, by comparing \textit{Algorithm \ref{alg_1} (High, Carto)} with \textit{Algorithm \ref{alg_1} (High, First)}, it also confirms that the poses obtained in our first stage are more accurate than those of Cartographer.

First, \textit{Algorithm \ref{alg_1} (High, O/S)} fails to converge on most datasets, while our full method converges successfully, indicating the improved robustness of our multi-resolution strategy.

The comparison between our full method, \textit{Algorithm \ref{alg_1} (High, Carto)}, and \textit{Algorithm \ref{alg_1} (High, First)} across all five simulation groups is shown in Fig. \ref{fig_group_error}. It can be observed that the accuracy of our full method remains stable across all groups, while the accuracy of \textit{Algorithm \ref{alg_1} (High, Carto)} varies drastically. This suggests that the approach in our conference paper \cite{Zhao-RSS-22} not only requires an accurate initial guess but also produces less accurate poses than our new method. Moreover, comparing \textit{Algorithm \ref{alg_1} (High, Carto)} with \textit{Algorithm \ref{alg_1} (High, First)} further confirms that the poses obtained in our first stage are more accurate than those of Cartographer.



% It is worth discussing that our full method uses the selected high-resolution map for optimization in the second stage, and it can be observed in Fig. \ref{fig_group_error} that the accuracy of our full approach is even higher than the optimization using the full high-resolution map (i.e., \textit{Algorithm \ref{alg_1} (High, First)}) in some experiments. The potential reason is that when the relatively accurate poses and occupancy map are obtained, the dropped \textcolor{red}{cell vertices} and the corresponding observations contain little information. If all cells and corresponding observations are retained for optimization, it may affect the algorithm's ability to obtain the best solution, as all observation terms are assigned uniform weights. Another potential reason is that the smoothing term in (\ref{eq_objective_func}), by spreading the occupancy values to unknown \textcolor{red}{cell vertices}, may introduce errors that could affect the convergence of the optimization algorithm. 

It is also worth noting that our full method utilizes the selected high-resolution map for optimization in the second stage. As shown in Fig. \ref{fig_group_error}, in certain experiments, the accuracy of our full approach surpasses that of the optimization using the full high-resolution map (\textit{Algorithm \ref{alg_1} (High, First)}). A possible explanation is that once relatively accurate poses and occupancy maps are obtained, the dropped cell vertices and corresponding observations contain little information. Retaining all cells and corresponding observations for optimization may prevent the algorithm from finding the optimal solution, as all are observation items given uniform weights. Another reason could be the smoothing term in (\ref{eq_objective_func}), which spreads occupancy values to unknown cell vertices, potentially introducing errors that affect the convergence of the optimization.

In terms of time consumption, our full approach is much more efficient than \textit{Algorithm \ref{alg_1} (High, Carto)}. For instance, \textit{Algorithm \ref{alg_1} (High, Carto)} consumes over 21,000 seconds with the Car Park dataset. In comparison, the time consumption of our full approach using all frames is 119 seconds (less than 0.6\%), and using only key frames, it takes only 44 seconds (about 0.2\%). This substantial reduction in time consumption highlights the efficiency improvements of our method over our conference paper \cite{Zhao-RSS-22}.

% In terms of time consumption, our full approach is significantly more efficient than \textit{Algorithm \ref{alg_1} (High, Carto)}. For instance, \textit{Algorithm \ref{alg_1} (High, Carto)} consumes over 21,000 seconds when using the Car Park dataset. In comparison, the time consumption of our full approach using all frames is 119 seconds (less than 0.6\%), and using only key frames, it takes 44 seconds (approximately 0.2\%). This substantial reduction in time consumption underscores the significant efficiency improvements of our current method over our conference paper \cite{Zhao-RSS-22}.

The reduction in time consumption stems from both the multi-resolution strategy, which reduces time per iteration, and the fewer iterations needed in the second stage due to the selected high-resolution map. Our experiments show that only about two iterations are needed in the second stage with the selected high-resolution map, fewer than in \textit{Algorithm \ref{alg_1} (High, First)}. This is likely because the selected high-resolution map focuses on critical states, with observations containing the most relevant information, enabling faster convergence.


In summary, compared to our conference paper \cite{Zhao-RSS-22}, our new multi-resolution method does not require precise initialization, is far more efficient, and achieves higher accuracy.

% In addition, compared with \textit{Algorithm \ref{alg_1} (High, Carto)}, the time consumption of our full approach is reduced by $2-3$ orders of magnitude. For example, \textit{Algorithm \ref{alg_1} (High, Carto)} consumes more than $21,000$ seconds using the Car Park dataset. Compared to \textit{Algorithm \ref{alg_1} (High, Carto)}, the time consumption of our full approach using all frames is $119$ seconds (less than $0.6\%$), and the time consumption of our full approach using only key frames is 44 seconds which is approximately $0.2\%$. The significant reduction in time consumption shows the significantly improved efficiency of our current method over our conference paper \cite{Zhao-RSS-22}.

% The substantial reduction in time consumption of our full approach is attributed not only to the introduced multi-resolution strategy, which reduces the time consumption per iteration but also to the reduction in the number of iterations in the second stage, which is a result of utilizing the selected high-resolution map. Through the experiments, we find that only about $2$ iterations in the second stage are required to converge using the selected high-resolution map, which is smaller than the number of iterations needed in \textit{Algorithm \ref{alg_1} (High, First)}. The possible reason is that, in the case of using the selected high-resolution map, these selected \textcolor{red}{cell vertices} focus on the most critical states, and the corresponding observations contain the most important information, allowing the optimization problem to converge much faster.

% In summary, as compared with our conference paper \cite{Zhao-RSS-22}, our new multi-resolution method does not require accurate initialization, is much more efficient, and achieves a higher level of accuracy in most cases.   


\subsection{Ablation Study on the Resolution Ratio}

\begin{table}[t]
		\centering
		\caption{Impact of First-Stage Resolution Settings.}
		\label{table_ablation}
		\setlength{\tabcolsep}{1mm}{
		\begin{tabular}{llcccc}\toprule
		& & $r=20$ & $r=10$  & $r=5$ & $r=2$ \\   \hline   

		 \multirow{5}{*}{Simulation 1} & MAE/Trans (m) First& 0.02352  &  0.02206 & \textbf{0.02118} & 0.17318\\
		\quad & MAE/Rot (rad) First& 0.00116  &  \textbf{0.00098} & 0.00124 & 0.01066\\
		\quad & MAE/Trans (m) All & 0.00812  & \textbf{0.00640}  & 0.00728 & 0.16066\\
		\quad & MAE/Rot (rad) All&  0.00062 & 0.00060  & \textbf{0.00054} & 0.01008\\ 
		\quad & Total Time (s) & \textbf{118}  &  148 & 262 & 2183\\
		\hline

		\multirow{5}{*}{Simulation 2} & MAE/Trans (m) First & 0.03938 & 0.03224 & \textbf{0.01984} & 0.09160\\
		\quad & MAE/Rot (rad) First&  0.00332 & 0.00220  & \textbf{0.00108} & 0.00314\\
		\quad & MAE/Trans (m) All& 0.01742  & 0.00858  & \textbf{0.00584} & 0.08018\\
		\quad & MAE/Rot (rad) All& 0.00064  & 0.00062  & \textbf{0.00052} & 0.00286\\ 
		\quad & Total Time (s)& \textbf{149}  & 193  & 321 & 2685\\
		\hline

		\multirow{5}{*}{Simulation 3} & MAE/Trans (m) First& 0.06708  &  0.02624  & \textbf{0.01776} & 0.03570\\
		\quad & MAE/Rot (rad) First&  0.00384 & 0.00164  & \textbf{0.00124}  & 0.00278\\
		\quad & MAE/Trans (m) All& 0.01586  & \textbf{0.00726}  & 0.00816 & 0.02082\\
		\quad & MAE/Rot (rad) All& 0.00100  & \textbf{0.00058}   & 0.00068 & 0.00102\\ 
		\quad & Total Time (s)& \textbf{125}  &  132 & 185 & 1041\\
		\hline
		\end{tabular}
		}
        % \vspace{-1.5em}
\end{table}

In this section, we perform ablation experiments on simulation datasets to analyze the impact of varying resolution settings in the first stage of the multi-resolution strategy on overall optimization performance.

We assess accuracy and computational time using three simulation datasets, with the resolution in the second stage fixed at $s^{h} = 0.05$ m. The resolution ratios $r$ between the first and second stages are set to 2, 5, 10, and 20, respectively. To ensure consistency, a fixed selection range of $d=1.5$ m is applied uniformly across all datasets. 

  
The results, shown in Table \ref{table_ablation}, reveal that $r=10$ achieves the best trade-off between time consumption and accuracy. While $r=20$ minimizes time consumption, it reduces the accuracy of poses in the first stage, adversely impacting final optimization accuracy. Conversely, $r=5$ improves pose accuracy in the first stage at the cost of higher time consumption but does not consistently enhance final accuracy. Notably, $r$ may need adjustment for other high-resolution settings.


\subsection{Using Submap Joining in Large-scale Environments}

We have demonstrated that our approach accurately and robustly handles normal-scale simulated and practical environments. In this section, we evaluate its efficiency and effectiveness in large-scale environments and long-term trajectories by integrating our Occupancy-SLAM algorithm with the proposed occupancy submap joining approach. The dataset is divided into multiple segments, where Algorithm \ref{alg_flowchart} is used to construct submaps, followed by applying the submap joining method in Section \ref{Sec_submap} to generate the optimized global occupancy map and robot trajectory.

We validate our method on two large-scale datasets, Deutsches Museum b2 \cite{hess2016real} and C3, and compare it with Cartographer. As shown in Fig. \ref{fig_large_environment}, our occupancy grid maps outperform those of Cartographer, demonstrating the capability of our method to handle large-scale environments and long-term trajectories effectively. 

% The datasets have map sizes of 250 m $\times$ 200 m and 150 m $\times$ 125 m, containing 51833 and 24402 scans, with trajectory lengths of 1390 seconds and 610 seconds, respectively.

\begin{figure}[t]
\centering \subfigure[b2] { \label{fig_large_b2}
\includegraphics[width=0.253\textwidth]{./b2_Large_New.pdf}}\hspace{-0.4em}
\centering \subfigure[C3] {\label{fig_large_C3}
\includegraphics[width=0.2195\textwidth]{./C3_Large_New_1.pdf}}
\caption{\label{fig_large_environment} Comparison of results between our method and Cartographer on two large-scale practical datasets. The first row shows Cartographer's results, while the second row shows ours. In (b), the red dotted lines serve as references, highlighting that Cartographer's right wall appears more curved, whereas our result aligns more closely with a straight line.}
% \vspace{-1.5em}
\end{figure}


\subsection{Computational Complexity Analysis}
In this section, we analyze the computational complexity and evaluate the time consumption of our method using large-scale datasets.

The Gauss-Newton method for solving the joint optimization of local maps and poses in (\ref{Least Squares}) and submap joining in (\ref{eq_NLLS_joining}) primarily depends on calculating Jacobian $\mathbf{J}$, and constructing and solving the sparse linear system in (\ref{Gauss-Newton}) \cite{konolige2008frameslam}. We analyze each part's complexity separately due to differences in the NLLS formulation.

For the local map and poses joint optimization problem, the objective function consists of the observation term, the odometry term, and the smoothing term. Let ${\lambda(\mathbb{S})}$ denotes the number of sampling points $\mathbb{S}$, then the number of items in the objective function is $\mathfrak{d}_{row} =\lambda(\mathbb{S})+3(n-1)+2{c_w}{c_h}+c_w+c_h$, and the state vector size is $\mathfrak{d}_{col}=3n+(c_w+1)(c_h+1)$. Considering Jacobian of the smoothing term $\mathbf{J}_S$ can be pre-calculated before optimization, the number of non-zero elements of Jacobian matrix that need to be computed for each iteration is $\mathfrak{d}_J = 7\lambda(\mathbb{S}) + 6(n-1)$. Therefore, for each iteration, the computation complexity of Jacobian calculation, constructing (\ref{Gauss-Newton}) and solving (\ref{Gauss-Newton}) is $\mathcal{O}(\mathfrak{d}_J)$, $\mathcal{O}(\mathfrak{d}_{J}\mathfrak{d}_{col})$, and $\mathcal{O}({\mathfrak{d}^3_{col}})$, respectively. Therefore, the total computation complexity per iteration for the local map and poses joint optimization problem is $\mathcal{O}(\mathfrak{d}_{J}+\mathfrak{d}_J{\mathfrak{d}_{col}}+\mathfrak{d}^3_{col})$. Due to our proposed multi-resolution joint optimization strategy and keyframe selection, both $\mathfrak{d}_{J}$ and $\mathfrak{d}_{col}$ remain small during the first and second stages of optimization, making the computation time for this part manageable.

% (i.e., \textcolor{red}{cell vertices} with non-zero occupancy values) Similar to the computation complexity of the local map and poses joint optimization problem,

For our submap joining algorithm, the number of observations depends on the total number of cell vertices of the global occupancy map observed in each submap, denoted $\mathfrak{d}_{obs}^{G}$. Considering that some cell vertices will be observed repeatedly under different submaps, this number slightly exceeds the number of non-unknown cell vertices in the global map. Thus, the number of non-zero elements of Jacobian matrix is $\mathfrak{d}_J^G = 4\mathfrak{d}_{obs}^G$, and the state vector size is $\mathfrak{d}_{col}^{G} = 3n_L+(c_w^G+1)(c_h^G+1)$. The computation complexity per iteration is $\mathcal{O}(\mathfrak{d}_{J}^G+\mathfrak{d}_J^G{\mathfrak{d}_{col}^G}+{\mathfrak{d}_{col}^G}^3)$.
Although the global occupancy map tends to be relatively large, the sub-matrix of Hessian w.r.t. the global occupancy map is diagonal. To speed up computation, we apply the Schur complement \cite{zhang2006schur} to make the normal equation solving highly efficient.

Finally, we evaluate the time consumption of our method using both all frames and selected keyframes in large-scale environments to support our computation complexity analysis and compare it to Cartographer. For Museum b2 dataset, Cartographer takes 1424 seconds, while our method takes 1250 seconds when using all frames and 363 seconds with selected key frames. For C3 dataset, Cartographer takes 610 seconds, while our method takes 742 seconds with all frames and 236 seconds with selected key frames. In our total time consumption, the submap joining method consumes less than 10 seconds on both datasets. It can be seen that the time consumption of our method is comparable to that of Cartographer when all frames are used and much lower than that of Cartographer when selected key frames are used. These results demonstrate the efficiency of our multi-resolution joint optimization strategy and submap joining approach. 

\section{Preliminary Results in 3D Case}\label{sec_3d}

While this paper primarily focuses on demonstrating the benefits of jointly optimizing the robot poses and occupancy map in 2D, we also present some preliminary 3D results to illustrate that our idea can be extended to 3D applications.



\subsection{Extension of the Algorithms to 3D Case}

Our approach for jointly optimizing robot poses and the occupancy map extends naturally to 3D, where the information, robot poses, and occupancy maps are all represented in 3D. Most problem formulations and algorithms can be adapted with minor adjustments. 

For our local map and poses optimization method, observations transition from 2D laser scans to 3D LiDAR scans, robot poses and odometry involve 6 degree-of-freedom (DoF), and the map representation becomes 3D. Consequently, (\ref{eq_interp}) and (\ref{eq_NP}) need to be replaced from bilinear to trilinear interpolation and its inverse operation. 
For the objective function (\ref{eq_objective_func}), the odometry term (\ref{eq_odometry_term}) should be replaced with a 6 DoF odometry term for 3D, and the smoothing term (\ref{eq_smoothing_term}) should include a smoothing penalty for the z-axis, Jacobians $\mathbf{J}_P$, $\mathbf{J}_M$, $\mathbf{J}_O$, and $\mathbf{J}_S$ described in Appendices need to be adjusted accordingly. 

The submap joining problem in 3D remains largely similar to the 2D case, except that the projection relation extends from 2D-2D to 3D-3D, enabling the solution of 6 DoF poses and the 3D global occupancy map in the NLLS problem (\ref{eq_NLLS_joining}).

\subsection{3D Experimental Results}
\subsubsection{Evaluation metrics and state-of-the-art methods}
We evaluate our method's performance in 3D using absolute trajectory error for poses, aligning and comparing results with ground truth via EVO \cite{grupp2017evo}, as used in \cite{liu2023large,rosinol2021kimera}. In all the experiments, we use the odometry information provided by the dataset as initialization if it is available. Otherwise, we use FAST-LIO2 \cite{xu2022fast} to obtain the odometry information. To evaluate our method, we compare our method against state-of-the-art methods: BALM2 \cite{liu2023efficient}, HBA \cite{liu2023large}, and Voxgraph \cite{reijgwart2019voxgraph}. BALM2 optimizes the planar feature parameters of the point cloud and the robot's poses. HBA proposes a hierarchical bundle adjustment to optimize the consistency of the planar surfaces of point clouds and robot poses. Voxgraph builds SDF-based submaps from point clouds, uses SDF-to-SDF registration for relative submap measurements, and incrementally optimizes submap frames. HBA and Voxgraph can deal with large-scale environments, while BALM2 focuses on normal-scale environments.  

\subsubsection{Datasets}
We perform comparisons using three real-world datasets. (1) The Newer College Dataset \cite{ramezani2020newer}: The first five sequences from the \textit{shorter experiment}, collected with a handheld Ouster OS-1 LiDAR scanner at New College, Oxford. The environment includes lawns, buildings, a tunnel, and a garden. Ground truth is provided by a BLK360 LiDAR scanner to capture a detailed 3D map and then infer the ground truth of poses with centimeter-level accuracy.
(2) KITTI Dataset \cite{Geiger2013IJRR}: Sequence 07, a demo dataset for HBA, collected with a Velodyne HDL-64E LiDAR scanner mounted on a car. Ground truth poses is provided by RTK-GPS/INS.
(3) Arche Dataset \cite{reijgwart2019voxgraph}: A demo dataset for Voxgraph, collected using an Ouster OS1 LiDAR mounted on a hexacopter MAV in a disaster area. Ground truth positions are provided by an RTK-GNSS system. 

The Newer College Dataset is used to evaluate high-precision performance in normal-scale environments, while the KITTI and Arche datasets are used to test performance in large environments with long trajectories.

\begin{figure*}[t]
\centering
\includegraphics[width=0.99\textwidth]{./PC_Comparison.pdf}
\caption{\label{fig_3d_pointcloud} Some local point cloud maps from the Arche dataset. The first row shows point cloud maps generated using odometry from ROVIO (also used for submap construction in Voxgraph), while the second row shows maps generated with our optimized poses using the same LiDAR scans. BALM2 fails to produce results when using the same odometry and scans as inputs in all these local environments.}
\vspace{-1em}
\end{figure*}
 
\subsubsection{Experiments on normal-scale environments}
We evaluate the performance of our proposed method without submap joining in normal-scale environments.

First, we evaluate BALM2 and our method using the first five sequences of The Newer College Dataset, which encompass all scenarios within the dataset. As shown in Table \ref{tab_comparison_3d_local}, our method outperforms BALM2 across all metrics, except for the RMSE in Seq. 1, and significantly outperforms the odometry inputs from FAST-LIO2 in all metrics. 

\begin{table}[t]
		\centering
		\caption{Absolute Trajectory Error (MAE/RMSE, Meters) in Normal-scale Environments for Different 3D Methods.}
		\label{tab_comparison_3d_local}
		\setlength{\tabcolsep}{0.6 mm}{
		\begin{tabular}{lcccccc}\toprule
		Method	& Seq. 0 & Seq. 1 & Seq. 2 & Seq. 3 & Seq. 4 \\ \hline
		FAST-LIO2 & 0.518/0.717 & 0.181/0.202  & 0.121/0.132 &0.188/0.200&0.571/0.723\\
		BALM2 & 0.283/0.326 & 0.112/\textbf{0.123}  & 0.104/0.109 &0.144/0.158 &0.298/0.344 \\
        Ours & \textbf{0.185}/\textbf{0.232}  & \textbf{0.097}/\textbf{0.123}  & \textbf{0.091}/\textbf{0.099} & \textbf{0.141}/\textbf{0.155} &\textbf{0.238}/\textbf{0.284}\\ \hline
		\end{tabular}
		}
        % \vspace{-2em}
\end{table}

Next, we test robustness in a challenging environment with noisy odometry input using the Arche dataset. This dataset, collected by a hexacopter MAV in an unstructured environment, is influenced by drone vibrations, flight speed, and environmental factors. Local point cloud maps built using odometry from ROVIO \cite{bloesch2017iterated} (also used to construct submaps in Voxgraph) are shown in the first row of Fig. \ref{fig_3d_pointcloud}. To evaluate BALM2 and our method, we partition the dataset into several short sequences, each lasting 10–20 seconds. BALM2 fails in all the sequences except during MAV start-up and landing due to insufficient planar features for optimization, while our method performs well on all the sequences. The second row of Fig. \ref{fig_3d_pointcloud} illustrates some of our results, demonstrating that our method is robust in 3D and does not rely on environmental assumptions. Additionally, the results confirm that our method achieves significantly higher pose accuracy than ROVIO.

\subsubsection{Experiments on large-scale environments}\label{sec_experiment_vox} 

We evaluate our method in large-scale environments with long trajectories using the KITTI and Arche datasets, comparing it with HBA and Voxgraph. For this experiment, we first build submaps by jointly optimizing poses and maps within submaps, then apply our submap joining algorithm to jointly optimize submap frame poses and the global occupancy map.

% The MAE and RMSE of the ATEs are summarized in Table \ref{tab_comparison_3d_large}, it is clear that our method achieves the best results on both datasets. In addition, the robot trajectories are shown in Fig. \ref{fig_trajectory_3d}, as it shows our method can achieve the best global consistent robot trajectories compared with other methods. It should be noted that the results of our method substantially lead Voxgraph on the KITTI dataset and significantly outperform HBA on the Arche dataset. The reason our method performs much better than Voxgraph on KITTI dataset is that Voxgraph relies on relative measurements from SDF-to-SDF registration for solving pose graph optimization, but in such autonomous driving scenarios, it is difficult to provide sufficient overlapping submaps for Voxgraph to calculate relative measurements between submaps. However, our proposed submap joining algorithm jointly optimizes poses of submaps' coordinate frames and the global occupancy map and, therefore, does not suffer in such environments. The performance of HBA on the Arche dataset is affected by highly unstructured environments and with data captured by moving MAV, as HBA relies on detecting and using planar features from the point cloud to do the optimization, which is similar to BALM2. However, in the case of odometry and point clouds collected during MAV motion, it is difficult for such methods to detect a sufficient number of good planar features. In addition, the planarity assumption does not tend to hold true in non-urban environments, such as the field.

Table \ref{tab_comparison_3d_large} summarizes the MAE and RMSE of absolute trajectory error, showing our method achieves the best results on both datasets. Fig. \ref{fig_trajectory_3d} illustrates that our approach can achieve the best global robot trajectories. Notably, our method significantly outperforms Voxgraph on the KITTI dataset and HBA on the Arche dataset. The relatively poor performance of Voxgraph on the KITTI dataset is due to its reliance on relative measurements from SDF-to-SDF registration, which requires sufficient overlapping submaps—a challenge in autonomous driving scenarios. In contrast, our submap joining algorithm jointly optimizes submap poses and the global occupancy map, avoiding this limitation. HBA underperforms on the Arche dataset due to its reliance on planar features for optimization, which is challenging in unstructured environments and during MAV motion. Odometry and point clouds from such scenarios make detecting sufficient planar features difficult, and the planarity assumption often fails in non-urban environments like disaster areas.


\begin{figure}[tbp]
\centering \subfigure[KITTI] {\label{fig_trajectory_1}
\includegraphics[height=0.15\textwidth]{./Traj_KITTI.pdf}}
\centering \subfigure[Arche] {\label{fig_trajectory_2}
\includegraphics[height=0.15\textwidth]{./Traj_Voxgraph_Demo_New.pdf}}
\caption{\label{fig_trajectory_3d} Robot trajectory results of datasets in large-scale environments. (a) and (b) show the trajectories of ground truth, Voxgraph \cite{reijgwart2019voxgraph}, HBA \cite{liu2023large}, and our method for KITTI dataset and Arche dataset.}
\end{figure}


\begin{table}[t]
		\centering
		\caption{Absolute Trajectory Error (MAE/RMSE, Meters) in Large-scale Environments for Different 3D Methods}
		\label{tab_comparison_3d_large}
		\setlength{\tabcolsep}{7mm}{
		\begin{tabular}{lcc}\toprule
		Method & KITTI & Arche  \\ \hline
		HBA \cite{liu2023large} & 0.342/0.364 & 4.123/4.789  \\
        Voxgraph \cite{reijgwart2019voxgraph} &0.926/1.002 & 0.700/0.833 \\
        Ours & \textbf{0.315}/\textbf{0.339}  & \textbf{0.275}/\textbf{0.378} \\ \hline
		\end{tabular}
		}
        % \vspace{-2em}
\end{table}

\subsection{Discussion}
The experimental results in this section demonstrate that our proposed idea of jointly optimizing the robot pose and the occupancy map can also lead to better solutions for the robot poses and occupancy maps in 3D cases. However, several challenges remain in 3D scenarios. For instance, 3D point clouds from LiDAR scanners are often sparse, particularly in the vertical direction, which can lead to observability issues in the optimization problem. This sparsity also results in inhomogeneous observation information, complicating the accurate representation of the 3D environment in occupancy maps. Furthermore, the large dimensions of 3D maps pose significant computational challenges in large-scale SLAM, requiring more efficient solving methods.


To address these challenges, several potential solutions can be explored. First, adopting compact representations for 3D occupancy maps, such as octree structures similar to Octomap \cite{hornung2013octomap} and \cite{vespa2019adaptive}, can enhance efficiency. Second, combining local map and pose optimization with hierarchical optimization and submap joining methods can further reduce computational time. Lastly, using continuous representations for 3D occupancy maps enables more precise gradient calculations, which can better guide the optimization process.

\section{Conclusion} \label{Sec_conclusion}
In this paper, we propose Occupancy-SLAM algorithm, which solves robot poses and occupancy map simultaneously. To enhance efficiency and robustness, we introduce a multi-resolution strategy. The first stage jointly optimizes poses and a low-resolution occupancy map to quickly achieve relatively accurate pose estimates, which are then used as the initial guess for the second stage. The second stage refines poses and a subset of the high-resolution map, focusing on critical boundary areas. Additionally, we extend this framework to an occupancy grid-based submap joining algorithm, addressing challenges in large-scale environments and long-term trajectories. Results from both simulated and real-world datasets demonstrate that our method achieves more accurate pose and map estimates than state-of-the-art approaches. 

   
Our findings show that solving poses and occupancy maps simultaneously yields more accurate results compared to first solving pose-graph SLAM and then constructing the map. This joint optimization approach has the potential to revolutionize occupancy map based SLAM frameworks.

The proposed method acts as a batch optimization approach for obtaining high-quality robot poses and maps. Unlike incremental or online methods, batch optimization provides greater accuracy, which is particularly advantageous for applications requiring high-quality maps rather than real-time operation (e.g., offline map creation for precise future localization). Despite typical drawbacks of batch optimization, such as higher computational costs, trajectory-length-dependent complexity, and reliance on accurate initial guesses, our method effectively overcomes these limitations: 1) our method is efficient due to the proposed multi-resolution joint optimization strategy, and the computation time is comparable to online methods; 2) our method can use selected keyframes to further reduce the computational cost without losing too much accuracy; 3) our proposed occupancy submap joining approach can overcome the limitation that the computational complexity related to the length of the robot trajectories; and 4) our method is very robust to the initial guess and can be initialized from odometry inputs or scan matching, so it does not require initialization from the result of incremental/online methods.    

In our future work, we will further explore problem formulation and solution techniques in the 3D case to develop more efficient and robust algorithms capable of addressing various challenges in 3D environments. 


%\section*{Acknowledgments}
%This should be a simple paragraph before the References to thank those individuals and institutions who have supported your work on this article.


{\appendix


The Jacobian $\mathbf{J}$ in (\ref{Gauss-Newton}) consists of four parts, i.e. the Jacobian of the observation term w.r.t. the robot poses $\mathbf{J}_P$ (See Appendix \ref{Sec_J_P}), the Jacobian of the observation term w.r.t. the occupancy map $\mathbf{J}_M$ (See Appendix \ref{Sec_J_D}), the Jacobian of the odometry term w.r.t. robot poses $\mathbf{J}_O$ (See Appendix \ref{Sec_J_O}) and the Jacobian of the smoothing term w.r.t. the occupancy map $\mathbf{J}_S$ (See Appendix \ref{Sec_J_S}). In addition, the difference in the calculation of Jacobians between Algorithm \ref{alg_1} and Algorithm \ref{alg_3} is shown in Appendix \ref{Sec_J_Select}. 

\subsection{Jacobian of the Observation Term w.r.t. Robot Poses}\label{Sec_J_P}

The Jacobian $\mathbf{J}_P$ of function $F_{ij}^Z(\mathbf{x})$ in the observation term w.r.t. the robot poses $\mathbf{x}^P_i$ can be calculated by the chain rule
\begin{equation}
	\begin{aligned}
		\mathbf{J}_P=\frac{ \partial F_{ij}^Z(\mathbf{x}) }{ \partial \mathbf{x}^P_i } = \frac{\partial F_{ij}^Z(\mathbf{x}) }{ \partial \mathbf{p}_{ij} } \cdot \frac{\partial \mathbf{p}_{ij}  }{ \partial \mathbf{x}^P_i}	
	\end{aligned}
\end{equation}
in which $\dfrac{\partial \mathbf{p}_{ij}  }{ \partial \mathbf{x}^P_i}$ can be calculated as
\begin{equation}
\dfrac{\partial \mathbf{p}_{ij}}{\partial \mathbf{x}^P_i}=\left[\begin{array}{ll}
\dfrac{\partial \mathbf{p}_{ij}}{\partial \mathbf{t}_i} & \dfrac{\partial \mathbf{p}_{ij}}{\partial \theta_i}
\end{array}\right]=\dfrac{1}{s} \left[\begin{array}{ll}
\mathbf{E}_{2} & \left(\mathbf{R}_i^{\prime}\right)^{\top} \mathbf{p}_{ij}
\end{array}\right].
\end{equation}
$\mathbf{R}_i^\prime$ is the derivative of the rotation matrix $\mathbf{R}_i$ w.r.t. rotation angle $\theta_i$ and $\mathbf{E}_2$ means $2 \times 2$ identity matrix.

$\dfrac{\partial F_{ij}^Z(\mathbf{x}) }{ \partial \mathbf{p}_{ij} }$ can be calculated by
\begin{equation}
\dfrac{\partial F_{ij}^Z(\mathbf{x}) }{ \partial \mathbf{p}_{ij} } = \dfrac{1}{N(\mathbf{p}_{ij})} \dfrac{\partial M(\mathbf{p}_{ij})}{\partial \mathbf{p}_{ij}}.
\end{equation}
Here $\dfrac{\partial M(\mathbf{p}_{ij})}{\partial \mathbf{p}_{ij}}$ can be considered as the gradient of the occupancy map at point $\mathbf{p}_{ij}$, which can be approximated by the bilinear interpolation of the gradients of the occupancy at the four adjacent cell vertices $\mathbf{\nabla} M(\mathbf{m}_{wh}),\cdots,\mathbf{\nabla} M(\mathbf{m}_{(w+1)(h+1)})$ around $\mathbf{p}_{ij}$ as
\begin{equation} 
\dfrac{\partial M(\mathbf{p}_{ij})}{\partial \mathbf{p}_{ij}}= 
\left[
\begin{aligned}
a_1b_1\\a_0b_1\\a_1b_0\\a_0b_0\\
\end{aligned}\right]^\top
\left[
\begin{aligned}
&\mathbf{\nabla} M(\mathbf{m}_{wh})\\&\mathbf{\nabla} M(\mathbf{m}_{(w+1)h})\\&\mathbf{\nabla} M(\mathbf{m}_{w(h+1)})\\&\mathbf{\nabla} M(\mathbf{m}_{(w+1)(h+1)})
\end{aligned}\right]\label{eq_14}
\end{equation} 
where the gradient of occupancy map $\mathbb{M}$ at all the cell vertices $\mathbf{\nabla} M$ can be easily calculated from $\mathbf{x}^M$ in the state. The bilinear interpolation used in (\ref{eq_14}) is similar to the method in (\ref{eq_interp}).

Here, we assume the robot poses $\mathbf{x}^P$ change slightly in each iteration, to reduce the computational complexity, the hit map $\mathbb{N}$ is considered as constant and recalculated using the current robot poses in each iteration. Thus, the derivative of $N(\mathbf{p}_{ij})$ is not calculated. 

\subsection{Jacobian of the Observation Term w.r.t. Occupancy Map}\label{Sec_J_D}
Based on (\ref{eq_interp}), the Jacobian $\mathbf{J}_M$ of function $F_{ij}^Z(\mathbf{x})$ in the observation term w.r.t. the map part of state vector $\mathbf{x}^{M}$ can be calculated as

\begin{equation}
\begin{aligned}
\mathbf{J}_M & = \dfrac{\partial F_{ij}^Z(\mathbf{x})}{\partial \left[ {M}(\mathbf{m}_{wh}),\cdots, {M}(\mathbf{m}_{(w+1)(h+1)}) \right]^\top}\\
&= \dfrac{1}{N(\mathbf{p}_{ij})}\dfrac{\partial M(\mathbf{p}_{ij})}{\partial \left[ {M}(\mathbf{m}_{wh}),\cdots, {M}(\mathbf{m}_{(w+1)(h+1)}) \right]^\top}\\ 
&= \dfrac{\begin{bmatrix}
a_1b_1,a_0b_1,a_1b_0,a_0b_0
\end{bmatrix}}{N(\mathbf{p}_{ij})}
\end{aligned}
\end{equation}
where $\mathbf{m}_{wh}, \cdots, \mathbf{m}_{(w+1)(h+1)}$ are the four nearest cell vertices to $\mathbf{p}_{ij}$ in occupancy map $\mathbb{M}$, and $a_0,a_1,b_0$ and $b_1$ are defined in (\ref{eq_interp}).


\subsection{Jacobian of the Odometry Term}\label{Sec_J_O}
The Jacobian $\mathbf{J}_O$ of function $F_i^O(\mathbf{x})$ in the odometry term (\ref{eq_odometry_term}) is the partial derivative w.r.t. the robot poses $\mathbf{x}^P$ since it is not related to the occupancy map in the state vector $\mathbf{x}$. Therefore, the Jacobian $\mathbf{J}_O$ can be calculated as
\begin{equation}
\begin{aligned}
\mathbf{J}_O &= \frac{\partial F_i^O(\mathbf{x})}{\partial \left[ {\mathbf{x}^P_{i-1}}^\top, {\mathbf{x}^P_i}^\top \right]^\top }\\ 
&=\begin{bmatrix}
	 \dfrac{\partial F_i^O(\mathbf{x})}{\partial \mathbf{t}_{i-1}} &
	 \dfrac{\partial F_i^O(\mathbf{x})}{\partial \theta_{i-1}} &
	 \dfrac{\partial F_i^O(\mathbf{x})}{\partial \mathbf{t}_i} &
	 \dfrac{\partial F_i^O(\mathbf{x})}{\partial \theta_i}
 \end{bmatrix} 
 \\
 &=\begin{bmatrix}
 	-\mathbf{R}_{i-1} & \mathbf{R}_{i-1}^\prime(\mathbf{t}_i-\mathbf{t}_{i-1}) & \mathbf{R}_{i-1} &\mathbf{0}_2\\
 	\mathbf{0}_2^\top & -1 & \mathbf{0}_2^\top & 1\\
 \end{bmatrix}
\end{aligned}
 \end{equation}
in which $\mathbf{0}_2$ means $2 \times 1$ zero vector.
 
\subsection{Jacobian of the Smoothing Term}\label{Sec_J_S}

The Jacobian $\mathbf{J}_S$ of function $F^S(\mathbf{x})$ in the smoothing term is the derivative of (\ref{eq_smoothing_term}) w.r.t. cell vertices of occupancy map $\mathbf{x}^M$ 
due to it is not related to the robot poses $\mathbf{x}^P$ in the state vector $\mathbf{x}$. It should be mentioned that $F^S(\mathbf{x})$ is linear w.r.t. $\mathbf{x}^M$
\begin{equation}
F^S(\mathbf{x}) = \mathbf{A} \left[ {M}(\mathbf{m}_{00}),\cdots,{M}(\mathbf{m}_{c_wc_h}) \right]^\top
\end{equation}
where the $(2c_wc_h+c_w+c_h) \times ((c_w+1)(c_h+1))$ coefficient matrix $\mathbf{A}$ is sparse and with nonzero elements $1$ or $-1$. An example of the coefficient matrix can be shown as
\begin{equation}\label{eq_A}
	\mathbf{A} = \begin{bmatrix}
    \vdots &\vdots  &\vdots  &\vdots  &\vdots  &\vdots  &\vdots &\vdots\\
 	\mathbf{0}^\top & 1 & -1 & 0 & \mathbf{0}^\top & 0 & 0 & \mathbf{0}^\top\\
 	\mathbf{0}^\top & 1 & 0  & 0 & \mathbf{0}^\top & -1 & 0 & \mathbf{0}^\top\\
 	\mathbf{0}^\top & 0 & 1 & -1 & \mathbf{0}^\top & 0 & 0 & \mathbf{0}^\top\\
 	\mathbf{0}^\top & 0 & 1 & 0 & \mathbf{0}^\top & 0 & -1 & \mathbf{0}^\top\\
    \vdots &\vdots  &\vdots  &\vdots  &\vdots  &\vdots  &\vdots &\vdots\\
 \end{bmatrix}.
\end{equation}
Here $\mathbf{0}$ represents a zero vector with appropriate dimensions. Therefore, the Jacobian of the smoothing term can be calculated as
\begin{equation}\label{eq_JS}
\mathbf{J}_S = \frac{\partial F^S(\mathbf{x})}{\partial \mathbf{x}^M } = \mathbf{A}.\\ 
\end{equation}
Since $\mathbf{A}$ is constant, $\mathbf{J}_S$ can be pre-calculated and directly used in the optimization as shown in Algorithm \ref{alg_1}.

\subsection{Jacobians in the Second Stage of Multi-resolution Strategy for Optimization}\label{Sec_J_Select}
In the second stage of the multi-resolution strategy (Algorithm \ref{alg_3}), the Jacobians to be calculated are similar to those in Algorithm \ref{alg_1}. A specific challenge arises in handling the selected cell vertices adjacent to the dropped cell vertices in the high-resolution map, particularly when calculating Jacobians $\mathbf{J}_P$ and $\mathbf{J}_S$.

 For Jacobian $\mathbf{J}_P$, partial derivatives w.r.t. all the cell vertices are required for (\ref{eq_14}). However, not all vertices are included in the state vector in the second stage, which makes it challenging to compute the partial derivatives w.r.t. some cell vertices because their surrounding nodes are discarded. From a semantic perspective, the discarded cell vertices have the same occupancy state as the edge nodes, which is why they are excluded. Consequently, the gradient of these edge vertices is expected to be close to zero. Based on this reasoning, we set the partial derivatives w.r.t. all edge cell vertices to $0$ when they need to be calculated using (\ref{eq_14}). 
 
 For Jacobian $\mathbf{J}_S$, it can also be calculated using the same idea as (\ref{eq_JS}). In the second stage of our multi-resolution strategy, (\ref{eq_JS}) is reformulated as 
 \begin{equation}
 	\mathbf{J}_S = \frac{\partial F^S_{s}(\mathbf{x}^s)}{\partial \mathbf{x}^{sM} } = \mathbf{A}^{s}\\ 
 \end{equation}
 where $\partial F^S_{s}(\mathbf{x}^s)$ is similar to (\ref{eq_smoothing_term}), but only applies to vertices in $\mathbb{M}^s$. The coefficient matrix $\mathbf{A}^{s}$ has the same form as (\ref{eq_A}) but with dimension corresponding to the number of elements in $\mathbf{x}^{sM}$.  
 
 }   





\bibliographystyle{IEEEtran}
\input{IEEE_Trans.bbl}






%\newpage

%\section{Biography Section}
%If you have an EPS/PDF photo (graphicx package needed), extra braces are
% needed around the contents of the optional argument to biography to prevent
% the LaTeX parser from getting confused when it sees the complicated
% $\backslash${\tt{includegraphics}} command within an optional argument. (You can create
% your own custom macro containing the $\backslash${\tt{includegraphics}} command to make things
% simpler here.)
% 
%\vspace{11pt}
%
%\bf{If you include a photo:}\vspace{-33pt}
%\begin{IEEEbiography}[{\includegraphics[width=1in,height=1.25in,clip,keepaspectratio]{fig1}}]{Michael Shell}
%Use $\backslash${\tt{begin\{IEEEbiography\}}} and then for the 1st argument use $\backslash${\tt{includegraphics}} to declare and link the author photo.
%Use the author name as the 3rd argument followed by the biography text.
%\end{IEEEbiography}
%
%\vspace{11pt}
%
%\bf{If you will not include a photo:}\vspace{-33pt}
%\begin{IEEEbiographynophoto}{John Doe}
%Use $\backslash${\tt{begin\{IEEEbiographynophoto\}}} and the author name as the argument followed by the biography text.
%\end{IEEEbiographynophoto}




\vfill

\end{document}









%\newpage

%\section{Biography Section}
%If you have an EPS/PDF photo (graphicx package needed), extra braces are
% needed around the contents of the optional argument to biography to prevent
% the LaTeX parser from getting confused when it sees the complicated
% $\backslash${\tt{includegraphics}} command within an optional argument. (You can create
% your own custom macro containing the $\backslash${\tt{includegraphics}} command to make things
% simpler here.)
% 
%\vspace{11pt}
%
%\bf{If you include a photo:}\vspace{-33pt}
%\begin{IEEEbiography}[{\includegraphics[width=1in,height=1.25in,clip,keepaspectratio]{fig1}}]{Michael Shell}
%Use $\backslash${\tt{begin\{IEEEbiography\}}} and then for the 1st argument use $\backslash${\tt{includegraphics}} to declare and link the author photo.
%Use the author name as the 3rd argument followed by the biography text.
%\end{IEEEbiography}
%
%\vspace{11pt}
%
%\bf{If you will not include a photo:}\vspace{-33pt}
%\begin{IEEEbiographynophoto}{John Doe}
%Use $\backslash${\tt{begin\{IEEEbiographynophoto\}}} and the author name as the argument followed by the biography text.
%\end{IEEEbiographynophoto}




\vfill

\end{document}









%\newpage

%\section{Biography Section}
%If you have an EPS/PDF photo (graphicx package needed), extra braces are
% needed around the contents of the optional argument to biography to prevent
% the LaTeX parser from getting confused when it sees the complicated
% $\backslash${\tt{includegraphics}} command within an optional argument. (You can create
% your own custom macro containing the $\backslash${\tt{includegraphics}} command to make things
% simpler here.)
% 
%\vspace{11pt}
%
%\bf{If you include a photo:}\vspace{-33pt}
%\begin{IEEEbiography}[{\includegraphics[width=1in,height=1.25in,clip,keepaspectratio]{fig1}}]{Michael Shell}
%Use $\backslash${\tt{begin\{IEEEbiography\}}} and then for the 1st argument use $\backslash${\tt{includegraphics}} to declare and link the author photo.
%Use the author name as the 3rd argument followed by the biography text.
%\end{IEEEbiography}
%
%\vspace{11pt}
%
%\bf{If you will not include a photo:}\vspace{-33pt}
%\begin{IEEEbiographynophoto}{John Doe}
%Use $\backslash${\tt{begin\{IEEEbiographynophoto\}}} and the author name as the argument followed by the biography text.
%\end{IEEEbiographynophoto}




\vfill

\end{document}









%\newpage

%\section{Biography Section}
%If you have an EPS/PDF photo (graphicx package needed), extra braces are
% needed around the contents of the optional argument to biography to prevent
% the LaTeX parser from getting confused when it sees the complicated
% $\backslash${\tt{includegraphics}} command within an optional argument. (You can create
% your own custom macro containing the $\backslash${\tt{includegraphics}} command to make things
% simpler here.)
% 
%\vspace{11pt}
%
%\bf{If you include a photo:}\vspace{-33pt}
%\begin{IEEEbiography}[{\includegraphics[width=1in,height=1.25in,clip,keepaspectratio]{fig1}}]{Michael Shell}
%Use $\backslash${\tt{begin\{IEEEbiography\}}} and then for the 1st argument use $\backslash${\tt{includegraphics}} to declare and link the author photo.
%Use the author name as the 3rd argument followed by the biography text.
%\end{IEEEbiography}
%
%\vspace{11pt}
%
%\bf{If you will not include a photo:}\vspace{-33pt}
%\begin{IEEEbiographynophoto}{John Doe}
%Use $\backslash${\tt{begin\{IEEEbiographynophoto\}}} and the author name as the argument followed by the biography text.
%\end{IEEEbiographynophoto}




\vfill

\end{document}


