\section{METHODS}
Our study initially aimed to inform the design of a Whatsapp-based chatbot to provide maternal and child health (MCH) information to women in underserved communities. We chose WhatsApp due to its widespread reach in India, as well as its range of media offerings that feature voice, text, and video interactions. We are not unique in this experiment; this avenue is currently being tested by many health organizations in the hope of improving user engagement and accessibility. %Our focus was not on text-based communication, even though that's what we design tested with. We wanted to understand the preference of the participants.
% We investigated how socioeconomic factors influence technology use and adoption in these communities. 
We conducted 23 semi-structured interviews and two focus groups with 15 women from underserved communities in Mumbai. % Below, we detail the participant recruitment process, data collection, and analysis methods. 
We obtained approval to conduct the research from the Institutional Review Board at Emory University in the United States.

\subsection{Participant Recruitment}
Participants for this study were recruited through two different NGOs. For interviews, we recruited participants from NGO1 (anonymized), a non-profit based in Mumbai, India, that has been focused on maternal and child health since 2008. The participants were enrolled in the organization's child malnutrition program where they received calls from counsellors and automated voice messages regarding child nutrition and growth. The community workers from the organization conducted recruitment through phone calls and assisted in coordinating the interviews, which were held at the participants' homes. Our participant group reflects diversity in age, family status, education, number of children, and, annual income. Our study participants' demographics are shown in Table \ref{tab:demo}. 



\begin{table}[htp]
\caption{Participants Demographics}
\begin{center}
\begin{tabular}{|l|c|c|}
\hline
%\textbf{Table}&\multicolumn{3}{|c|}{\textbf{Table Column Head}} \\
%\cline{2-4} 

\textbf{Attribute} & \textbf{N}& \textbf{\%} \\
\hline
%copy& More table copy$^{\mathrm{a}}$& &  \\


    \textbf{Age} & & \\
    Age 18-29 & 17 & 73.9 \\
    Age 30-39 & 5 & 21.7 \\
    Age 40-49 & 1 & 4.3 \\   
\hline
    \textbf{Education} & & \\
    5th grade & 3 & 13.0 \\
    8th grade & 1 & 4.3 \\
    9th grade & 1 & 4.3 \\
    10th grade & 3 & 13.0 \\
    11th grade & 1 & 4.3 \\
    12th grade & 7 & 30.0 \\
    Diploma & 2 & 8.7 \\
    Graduation & 3 & 13.0 \\
    Post Graduation & 1 & 4.3 \\
    No education & 1 & 4.3 \\
  \hline
    \textbf{Housing situation} & & \\
    Own & 7 & 30.4 \\
    Rental, Nuclear family & 10 & 43.5 \\
    Rental, Joint family & 2 & 8.7 \\
    Ancestral, Joint family & 3 & 13.0 \\
    Ancestral, Nuclear family & 1 & 4.3 \\
  \hline
    \textbf{Annual household Income} & & \\
    Rs. 10,000-1,25,000 & 5 & 21.7 \\
    Rs. 1,26,000-10,00,000 & 2 & 8.7 \\
    Prefer not to say & 16 & 69.6 \\
\hline
    \textbf{Own Smartphone} & & \\
    Yes & 21 & 91.3 \\
    No & 2 & 8.7 \\
  \hline
    \textbf{WhatsApp Access} & & \\
    Yes & 23 & 100.0 \\
  \hline
%\multicolumn{4}{l}{$^{\mathrm{a}}$Sample of a Table footnote.}
\end{tabular}
\label{tab:demo}
\end{center}

\end{table}





%\begin{table}
%\centering
%\begin{tabular}{l c c}
  %  \\
  %  \toprule
   % \textbf{Attribute} & \textbf{N} & \textbf{\%} \\ 
    %\midrule
    %\textbf{Age} & & \\
   % Age 18-29 & 17 & 73.9 \\
    %Age 30-39 & 5 & 21.7 \\
    %Age 40-49 & 1 & 4.3 \\   
    %\midrule
    %\textbf{Education} & & \\
    %5th grade & 3 & 13.0 \\
    %8th grade & 1 & 4.3 \\
    %9th grade & 1 & 4.3 \\
    %10th grade & 3 & 13.0 \\
    %11th grade & 1 & 4.3 \\
    %12th grade & 7 & 30.0 \\
    %Diploma & 2 & 8.7 \\
    %Graduation & 3 & 13.0 \\
    %Post Graduation & 1 & 4.3 \\
    %No education & 1 & 4.3 \\
    %\midrule
    %\textbf{Housing situation} & & \\
    %Own & 7 & 30.4 \\
    %Rental, Nuclear family & 10 & 43.5 \\
    %Rental, Joint family & 2 & 8.7 \\
    %Ancestral, Joint family & 3 & 13.0 \\
    %Ancestral, Nuclear family & 1 & 4.3 \\
    %\midrule
    %\textbf{Annual household Income} & & \\
    %Rs. 10,000-1,25,000 & 5 & 21.7 \\
    %Rs. 1,26,000-10,00,000 & 2 & 8.7 \\
   % Prefer not to say & 16 & 69.6 \\
    %\midrule
    %\textbf{Own Smartphone} & & \\
    %Yes & 21 & 91.3 \\
    %No & 2 & 8.7 \\
    %\midrule
    %\textbf{WhatsApp Access} & & \\
    %Yes & 23 & 100.0 \\
    %\bottomrule
%\end{tabular}
%\caption{\textbf{Participants demographics}}
 %\label{tab:demo}
%\end{table}

In addition to these individual interviews, we conducted two focus group discussions (FGDs) with a total of fifteen participants. These participants were recruited from a second non-profit organization (NGO2) that provides support to mothers of children with disabilities. Both the focus groups were conducted in the communal space provided by NGO2 at their center. 


\subsection{Data Collection and Analysis}
We conducted semi-structured interviews with participants at their homes. With their consent, the conversations were recorded for further analysis. In addition to collecting demographic information, participants were shown a sample WhatsApp conversation in Hindi that included some common questions on breastfeeding and child nutrition (refer to Figure \ref{fig:whats1} and Figure \ref{fig:whats2}). Through the think-aloud protocol, a method used for usability testing \cite{b23}, we engaged participants with questions to understand their interest in technology, preferred modes of interaction, and perceptions of its usefulness.
Following the interviews, focus group discussions were conducted. These discussions were also recorded with the consent of participants. During the focus groups, an animated video clip on malnutrition, created by NGO1, was shown to the participants to prompt further discussion and understand their preferences for information. \\
Given that our participants spoke Hindi, we collected the data in Hindi. The data collected was then transcribed and translated into English for analysis. We then analyzed the data using a line-by-line coding approach to identify specific codes that represented participants’ responses, covering topics such as smartphone usage, familiarity with chatbots, preferred topics for chatbots, communication preferences, and family support in childcare. In the initial phase, coding closely followed the text of the transcripts.
Subsequently, we conducted a thematic analysis, grouping the codes into broader categories based on emerging patterns and themes. Through iterative refinement, we identified key thematic categories, including limited access to technology, information gaps in healthcare, the impact of socio-cultural norms, digital literacy, and trust in external programs and resources. To ensure the comprehensiveness of the data, thematic saturation was reached when no new themes emerged.

\begin{figure}[htp]
\centerline{\includegraphics[width=0.2\textwidth, height=0.3\textheight]{whatsapp1.png}}
\caption{Sample Whatsapp conversation, Original Hindi text.}
\label{fig:whats1}
\end{figure}
\subsection{Limitations}
% One of the primary challenges in this study was the absence of a chatbot prototype for testing. 
To simulate chatbot interactions, our team initially planned to use the Wizard-of-Oz method, where participants would interact with an interface, and the responses are provided by a human operator. This approach has been successfully employed in prior studies in urban India to create the illusion of an AI interface \cite{22}. However, we faced practical constraints in executing this in an underserved community with intermittent internet, preventing real-time testing. 
Moreover, as the interviews progressed, it became evident that the Wizard-of-Oz method may not have been optimal given the participants' time constraints and the challenges they faced with text-based interactions.
% The second author, who would act as the human operator, was based in the USA, and significant time zone differences made real-time coordination with participants in India unfeasible.

To address this, we adapted our approach by developing a sample WhatsApp conversation in Hindi, which was presented to participants to elicit their preferences for technology use. While most participants were able to engage with the sample and offer valuable feedback, a few encountered difficulties reading the text in Hindi.

%The study included 23 participants, women between the age of 18-49years with a child of upto 3 years. We used a mixed-method approach, combining qualitative focus groups and interviews with a quantitative survey. The combination of methods allowed for a comprehensive understanding of maternal and child healthcare practices, technology adoptions, and communication preferences among the target population.  Triangulation of data sources improved the validity and reliability of study findings, allowing for a clear interpretation of the results. 


%\subsection{Study Overview}
%All the participants were women between the age of 18 - 49 years with at least one child of up to 3 years of age living in the urban slum areas of Mumbai. Prior appointments were taken, and to ensure convenience for participants, the survey and interviews were administered in person at their houses.
%Participants were asked questions from a standardized questionnaire regarding their communication preferences, technology use, healthcare access, and demographic data. Data collected included age, address, number of children, education, housing situation, occupation, annual household income, healthcare access, immunization awareness, family planning measures, Whats App usage, and preferred communication methods.
%Seven in-depth interviews and two focus groups were conducted to fully understand the  perspectives of the participants.

%\subsection{Data Analysis}
 
%All the data collected from interviews and focus groups was translated from Hindi to English. We then started qualitative data analysis by conducting line-by-line coding to identify specific codes representing participant responses, including aspects such as smartphone usage, chatbot familiarity, preferred topics for chatbots, communication preferences, and family support in childcare. The first iteration of coding adhered closely to the text. 

%We then conducted thematic analysis, wherein we grouped these codes into broader categories based on observed themes . Through iterative refinement and discussion, we identified five primary thematic categories: Limited access to technology, Information gaps in healthcare, Impact of social-cultural norms, Digital literacy, Trust in external programs and resources. To ensure data comprehensiveness, thematic saturation was reached when no new themes appeared. 



    %\includegraphics[width=\linewidth]{postit.png}
  %\Description{A woman and a girl in white dresses sit in an open car.}
%\end{figure}


%\begin{figure}[h]
 

%\begin{figure}[h!]
  %  \centering
  %  \begin{minipage}{0.45\linewidth}
   %     \centering
    %    \includegraphics[width=\linewidth]{sections/whatsapp1.png}
   % \end{minipage}
  %  \hfill
   % \begin{minipage}{0.45\linewidth}
     %   \centering
     %   \includegraphics[width=\linewidth]{sections/whatsapp2.png}
   % \end{minipage}
   % \caption{\textbf{Sample Whatsapp conversation.} Left: Original Hindi Text, Right: Translated Text }
   %  \label{fig:whatsapp}
%\end{figure}


