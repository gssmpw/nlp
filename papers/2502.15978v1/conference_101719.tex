\documentclass[conference]{IEEEtran}
\IEEEoverridecommandlockouts
% The preceding line is only needed to identify funding in the first footnote. If that is unneeded, please comment it out.
\usepackage{cite}
\usepackage{amsmath,amssymb,amsfonts}
\usepackage{algorithmic}
\usepackage{graphicx}
\usepackage{textcomp}
\usepackage{tabularx}
\usepackage{xcolor}
\usepackage{booktabs}
\usepackage{soul}


\newcommand{\gtadd}[1]{\textcolor{red}{#1}}
\newcommand{\gtrm}[1]{\st{#1}}


\def\BibTeX{{\rm B\kern-.05em{\sc i\kern-.025em b}\kern-.08em
    T\kern-.1667em\lower.7ex\hbox{E}\kern-.125emX}}
\setlength{\columnsep}{0.24in}
\usepackage[top=0.75in, bottom=1.05in, left=0.625in, right=0.625in]{geometry}
\begin{document}
\title{``Who Has the Time?'': Understanding Receptivity to  Health Chatbots among Underserved Women in India}
% \title{\textit{``The family members do have a say in decisions
% regarding health’’}: Exploring Factors Limiting Maternal Health Tech Adoption in Underserved Communities
% %Beyond Access: Exploring Factors Limiting Maternal Health Tech Adoption in Underserved Communities\\
% Access in Place, Adoption Unmet??
% {\footnotesize \textsuperscript{*}Note: Sub-titles are not captured in Xplore and
% should not be used}
% \thanks{Department of Biomedical Informatics,Emory University.}
% }

\author{\IEEEauthorblockN{\textsuperscript{} Manvi S}
\IEEEauthorblockA{\textit{Biomedical Informatics} \\
\textit{Emory University}\\
Atlanta, USA \\
mmanvi@emory.edu}
\and
\IEEEauthorblockN{\textsuperscript{} Roshini Deva}
\IEEEauthorblockA{\textit{Biomedical Informatics} \\
\textit{Emory university}\\
Atlanta, USA \\
droshin@emory.edu}
\and
\IEEEauthorblockN{\textsuperscript{} Neha Madhiwalla}
\IEEEauthorblockA{\textit{Research Director} \\
\textit{ARMMAN Organization}\\
Mumbai, India \\
neha@armman.org}
\and
\IEEEauthorblockN{}
\IEEEauthorblockA{} 
\and
\IEEEauthorblockN{\textsuperscript{} Azra Ismail}
\IEEEauthorblockA{\textit{Biomedical Informatics and }\\
\textit{Global Health} \\
\textit{Emory University}\\
Atlanta, USA \\
azra.ismail@emory.edu}}



\maketitle

\begin{abstract}
%Despite multiple interventions, maternal and child health (MCH) in India continues to encounter obstacles that result in subpar healthcare practices, especially in underprivileged populations. This study explores how can we design a culturally relevant chatbot for mothers and children upto 3 years of age. Using a mixed-method approach that blends quantitative surveys and qualitative interviews, this study investigates the target population's adoption of technology, communication preferences, and healthcare behaviors. The findings reveal that using a smartphone and WhatsApp is common, a reliance on medical professions and openess to use chatbots and online advice systems. These realizations highlight how crucial it is to create tailored interventions that combine technology and traditional support networks to enhance MCH outcomes. 
Access to health information and services among women continues to be a major challenge in many communities globally. In recent years, there has been a growing interest in the potential of chatbots to address this information and access gap. We conducted interviews and focus group discussions with underserved women in urban India to understand their receptivity towards the use of chatbots for maternal and child health, as well as barriers to their adoption. Our findings uncover gaps in digital access and literacies, and perceived conflict with various responsibilities that women are burdened with, which shape their interactions with digital technology. Our paper offers insights into the design of chatbots for community health that can meet the lived realities of women in underserved settings.
%Access to health information and services among women continues to be a major challenge in many communities globally. In recent years, there has been a growing interest in the potential of chatbots to address this information and access gap. However, the adoption of health technology remains a challenge in underserved communities where digital access and literacies may be constrained. Through interviews and focus group discussions, we investigate perceptions of chatbots for maternal and child health, as well as barriers to their adoption, among underserved women in urban India. Our findings uncover the various responsibilities that women are burdened with, which shape their interactions with digital technology. Our paper offers insights into the design of chatbots for community health that can meet the lived realities of women in underserved settings.

%how women in underserved communities adopt technology for accessing maternal and child health (MCH) information and services. Our findings reveal challenges associated with technology adoption.

\end{abstract}

\begin{IEEEkeywords}
% component, formatting, style, styling, insert
Human-centered computing, Health implications, Ethical/Societal Implications
\end{IEEEkeywords}

\section{Introduction}
\label{sec:introduction}
The business processes of organizations are experiencing ever-increasing complexity due to the large amount of data, high number of users, and high-tech devices involved \cite{martin2021pmopportunitieschallenges, beerepoot2023biggestbpmproblems}. This complexity may cause business processes to deviate from normal control flow due to unforeseen and disruptive anomalies \cite{adams2023proceddsriftdetection}. These control-flow anomalies manifest as unknown, skipped, and wrongly-ordered activities in the traces of event logs monitored from the execution of business processes \cite{ko2023adsystematicreview}. For the sake of clarity, let us consider an illustrative example of such anomalies. Figure \ref{FP_ANOMALIES} shows a so-called event log footprint, which captures the control flow relations of four activities of a hypothetical event log. In particular, this footprint captures the control-flow relations between activities \texttt{a}, \texttt{b}, \texttt{c} and \texttt{d}. These are the causal ($\rightarrow$) relation, concurrent ($\parallel$) relation, and other ($\#$) relations such as exclusivity or non-local dependency \cite{aalst2022pmhandbook}. In addition, on the right are six traces, of which five exhibit skipped, wrongly-ordered and unknown control-flow anomalies. For example, $\langle$\texttt{a b d}$\rangle$ has a skipped activity, which is \texttt{c}. Because of this skipped activity, the control-flow relation \texttt{b}$\,\#\,$\texttt{d} is violated, since \texttt{d} directly follows \texttt{b} in the anomalous trace.
\begin{figure}[!t]
\centering
\includegraphics[width=0.9\columnwidth]{images/FP_ANOMALIES.png}
\caption{An example event log footprint with six traces, of which five exhibit control-flow anomalies.}
\label{FP_ANOMALIES}
\end{figure}

\subsection{Control-flow anomaly detection}
Control-flow anomaly detection techniques aim to characterize the normal control flow from event logs and verify whether these deviations occur in new event logs \cite{ko2023adsystematicreview}. To develop control-flow anomaly detection techniques, \revision{process mining} has seen widespread adoption owing to process discovery and \revision{conformance checking}. On the one hand, process discovery is a set of algorithms that encode control-flow relations as a set of model elements and constraints according to a given modeling formalism \cite{aalst2022pmhandbook}; hereafter, we refer to the Petri net, a widespread modeling formalism. On the other hand, \revision{conformance checking} is an explainable set of algorithms that allows linking any deviations with the reference Petri net and providing the fitness measure, namely a measure of how much the Petri net fits the new event log \cite{aalst2022pmhandbook}. Many control-flow anomaly detection techniques based on \revision{conformance checking} (hereafter, \revision{conformance checking}-based techniques) use the fitness measure to determine whether an event log is anomalous \cite{bezerra2009pmad, bezerra2013adlogspais, myers2018icsadpm, pecchia2020applicationfailuresanalysispm}. 

The scientific literature also includes many \revision{conformance checking}-independent techniques for control-flow anomaly detection that combine specific types of trace encodings with machine/deep learning \cite{ko2023adsystematicreview, tavares2023pmtraceencoding}. Whereas these techniques are very effective, their explainability is challenging due to both the type of trace encoding employed and the machine/deep learning model used \cite{rawal2022trustworthyaiadvances,li2023explainablead}. Hence, in the following, we focus on the shortcomings of \revision{conformance checking}-based techniques to investigate whether it is possible to support the development of competitive control-flow anomaly detection techniques while maintaining the explainable nature of \revision{conformance checking}.
\begin{figure}[!t]
\centering
\includegraphics[width=\columnwidth]{images/HIGH_LEVEL_VIEW.png}
\caption{A high-level view of the proposed framework for combining \revision{process mining}-based feature extraction with dimensionality reduction for control-flow anomaly detection.}
\label{HIGH_LEVEL_VIEW}
\end{figure}

\subsection{Shortcomings of \revision{conformance checking}-based techniques}
Unfortunately, the detection effectiveness of \revision{conformance checking}-based techniques is affected by noisy data and low-quality Petri nets, which may be due to human errors in the modeling process or representational bias of process discovery algorithms \cite{bezerra2013adlogspais, pecchia2020applicationfailuresanalysispm, aalst2016pm}. Specifically, on the one hand, noisy data may introduce infrequent and deceptive control-flow relations that may result in inconsistent fitness measures, whereas, on the other hand, checking event logs against a low-quality Petri net could lead to an unreliable distribution of fitness measures. Nonetheless, such Petri nets can still be used as references to obtain insightful information for \revision{process mining}-based feature extraction, supporting the development of competitive and explainable \revision{conformance checking}-based techniques for control-flow anomaly detection despite the problems above. For example, a few works outline that token-based \revision{conformance checking} can be used for \revision{process mining}-based feature extraction to build tabular data and develop effective \revision{conformance checking}-based techniques for control-flow anomaly detection \cite{singh2022lapmsh, debenedictis2023dtadiiot}. However, to the best of our knowledge, the scientific literature lacks a structured proposal for \revision{process mining}-based feature extraction using the state-of-the-art \revision{conformance checking} variant, namely alignment-based \revision{conformance checking}.

\subsection{Contributions}
We propose a novel \revision{process mining}-based feature extraction approach with alignment-based \revision{conformance checking}. This variant aligns the deviating control flow with a reference Petri net; the resulting alignment can be inspected to extract additional statistics such as the number of times a given activity caused mismatches \cite{aalst2022pmhandbook}. We integrate this approach into a flexible and explainable framework for developing techniques for control-flow anomaly detection. The framework combines \revision{process mining}-based feature extraction and dimensionality reduction to handle high-dimensional feature sets, achieve detection effectiveness, and support explainability. Notably, in addition to our proposed \revision{process mining}-based feature extraction approach, the framework allows employing other approaches, enabling a fair comparison of multiple \revision{conformance checking}-based and \revision{conformance checking}-independent techniques for control-flow anomaly detection. Figure \ref{HIGH_LEVEL_VIEW} shows a high-level view of the framework. Business processes are monitored, and event logs obtained from the database of information systems. Subsequently, \revision{process mining}-based feature extraction is applied to these event logs and tabular data input to dimensionality reduction to identify control-flow anomalies. We apply several \revision{conformance checking}-based and \revision{conformance checking}-independent framework techniques to publicly available datasets, simulated data of a case study from railways, and real-world data of a case study from healthcare. We show that the framework techniques implementing our approach outperform the baseline \revision{conformance checking}-based techniques while maintaining the explainable nature of \revision{conformance checking}.

In summary, the contributions of this paper are as follows.
\begin{itemize}
    \item{
        A novel \revision{process mining}-based feature extraction approach to support the development of competitive and explainable \revision{conformance checking}-based techniques for control-flow anomaly detection.
    }
    \item{
        A flexible and explainable framework for developing techniques for control-flow anomaly detection using \revision{process mining}-based feature extraction and dimensionality reduction.
    }
    \item{
        Application to synthetic and real-world datasets of several \revision{conformance checking}-based and \revision{conformance checking}-independent framework techniques, evaluating their detection effectiveness and explainability.
    }
\end{itemize}

The rest of the paper is organized as follows.
\begin{itemize}
    \item Section \ref{sec:related_work} reviews the existing techniques for control-flow anomaly detection, categorizing them into \revision{conformance checking}-based and \revision{conformance checking}-independent techniques.
    \item Section \ref{sec:abccfe} provides the preliminaries of \revision{process mining} to establish the notation used throughout the paper, and delves into the details of the proposed \revision{process mining}-based feature extraction approach with alignment-based \revision{conformance checking}.
    \item Section \ref{sec:framework} describes the framework for developing \revision{conformance checking}-based and \revision{conformance checking}-independent techniques for control-flow anomaly detection that combine \revision{process mining}-based feature extraction and dimensionality reduction.
    \item Section \ref{sec:evaluation} presents the experiments conducted with multiple framework and baseline techniques using data from publicly available datasets and case studies.
    \item Section \ref{sec:conclusions} draws the conclusions and presents future work.
\end{itemize}
\section{RELATED WORK}
\label{sec:relatedwork}
In this section, we describe the previous works related to our proposal, which are divided into two parts. In Section~\ref{sec:relatedwork_exoplanet}, we present a review of approaches based on machine learning techniques for the detection of planetary transit signals. Section~\ref{sec:relatedwork_attention} provides an account of the approaches based on attention mechanisms applied in Astronomy.\par

\subsection{Exoplanet detection}
\label{sec:relatedwork_exoplanet}
Machine learning methods have achieved great performance for the automatic selection of exoplanet transit signals. One of the earliest applications of machine learning is a model named Autovetter \citep{MCcauliff}, which is a random forest (RF) model based on characteristics derived from Kepler pipeline statistics to classify exoplanet and false positive signals. Then, other studies emerged that also used supervised learning. \cite{mislis2016sidra} also used a RF, but unlike the work by \citet{MCcauliff}, they used simulated light curves and a box least square \citep[BLS;][]{kovacs2002box}-based periodogram to search for transiting exoplanets. \citet{thompson2015machine} proposed a k-nearest neighbors model for Kepler data to determine if a given signal has similarity to known transits. Unsupervised learning techniques were also applied, such as self-organizing maps (SOM), proposed \citet{armstrong2016transit}; which implements an architecture to segment similar light curves. In the same way, \citet{armstrong2018automatic} developed a combination of supervised and unsupervised learning, including RF and SOM models. In general, these approaches require a previous phase of feature engineering for each light curve. \par

%DL is a modern data-driven technology that automatically extracts characteristics, and that has been successful in classification problems from a variety of application domains. The architecture relies on several layers of NNs of simple interconnected units and uses layers to build increasingly complex and useful features by means of linear and non-linear transformation. This family of models is capable of generating increasingly high-level representations \citep{lecun2015deep}.

The application of DL for exoplanetary signal detection has evolved rapidly in recent years and has become very popular in planetary science.  \citet{pearson2018} and \citet{zucker2018shallow} developed CNN-based algorithms that learn from synthetic data to search for exoplanets. Perhaps one of the most successful applications of the DL models in transit detection was that of \citet{Shallue_2018}; who, in collaboration with Google, proposed a CNN named AstroNet that recognizes exoplanet signals in real data from Kepler. AstroNet uses the training set of labelled TCEs from the Autovetter planet candidate catalog of Q1–Q17 data release 24 (DR24) of the Kepler mission \citep{catanzarite2015autovetter}. AstroNet analyses the data in two views: a ``global view'', and ``local view'' \citep{Shallue_2018}. \par


% The global view shows the characteristics of the light curve over an orbital period, and a local view shows the moment at occurring the transit in detail

%different = space-based

Based on AstroNet, researchers have modified the original AstroNet model to rank candidates from different surveys, specifically for Kepler and TESS missions. \citet{ansdell2018scientific} developed a CNN trained on Kepler data, and included for the first time the information on the centroids, showing that the model improves performance considerably. Then, \citet{osborn2020rapid} and \citet{yu2019identifying} also included the centroids information, but in addition, \citet{osborn2020rapid} included information of the stellar and transit parameters. Finally, \citet{rao2021nigraha} proposed a pipeline that includes a new ``half-phase'' view of the transit signal. This half-phase view represents a transit view with a different time and phase. The purpose of this view is to recover any possible secondary eclipse (the object hiding behind the disk of the primary star).


%last pipeline applies a procedure after the prediction of the model to obtain new candidates, this process is carried out through a series of steps that include the evaluation with Discovery and Validation of Exoplanets (DAVE) \citet{kostov2019discovery} that was adapted for the TESS telescope.\par
%



\subsection{Attention mechanisms in astronomy}
\label{sec:relatedwork_attention}
Despite the remarkable success of attention mechanisms in sequential data, few papers have exploited their advantages in astronomy. In particular, there are no models based on attention mechanisms for detecting planets. Below we present a summary of the main applications of this modeling approach to astronomy, based on two points of view; performance and interpretability of the model.\par
%Attention mechanisms have not yet been explored in all sub-areas of astronomy. However, recent works show a successful application of the mechanism.
%performance

The application of attention mechanisms has shown improvements in the performance of some regression and classification tasks compared to previous approaches. One of the first implementations of the attention mechanism was to find gravitational lenses proposed by \citet{thuruthipilly2021finding}. They designed 21 self-attention-based encoder models, where each model was trained separately with 18,000 simulated images, demonstrating that the model based on the Transformer has a better performance and uses fewer trainable parameters compared to CNN. A novel application was proposed by \citet{lin2021galaxy} for the morphological classification of galaxies, who used an architecture derived from the Transformer, named Vision Transformer (VIT) \citep{dosovitskiy2020image}. \citet{lin2021galaxy} demonstrated competitive results compared to CNNs. Another application with successful results was proposed by \citet{zerveas2021transformer}; which first proposed a transformer-based framework for learning unsupervised representations of multivariate time series. Their methodology takes advantage of unlabeled data to train an encoder and extract dense vector representations of time series. Subsequently, they evaluate the model for regression and classification tasks, demonstrating better performance than other state-of-the-art supervised methods, even with data sets with limited samples.

%interpretation
Regarding the interpretability of the model, a recent contribution that analyses the attention maps was presented by \citet{bowles20212}, which explored the use of group-equivariant self-attention for radio astronomy classification. Compared to other approaches, this model analysed the attention maps of the predictions and showed that the mechanism extracts the brightest spots and jets of the radio source more clearly. This indicates that attention maps for prediction interpretation could help experts see patterns that the human eye often misses. \par

In the field of variable stars, \citet{allam2021paying} employed the mechanism for classifying multivariate time series in variable stars. And additionally, \citet{allam2021paying} showed that the activation weights are accommodated according to the variation in brightness of the star, achieving a more interpretable model. And finally, related to the TESS telescope, \citet{morvan2022don} proposed a model that removes the noise from the light curves through the distribution of attention weights. \citet{morvan2022don} showed that the use of the attention mechanism is excellent for removing noise and outliers in time series datasets compared with other approaches. In addition, the use of attention maps allowed them to show the representations learned from the model. \par

Recent attention mechanism approaches in astronomy demonstrate comparable results with earlier approaches, such as CNNs. At the same time, they offer interpretability of their results, which allows a post-prediction analysis. \par


\subsection{Greedies}
We have two greedy methods that we're using and testing, but they both have one thing in common: They try every node and possible resistances, and choose the one that results in the largest change in the objective function.

The differences between the two methods, are the function. The first one uses the median (since we want the median to be >0.5, we just set this as our objective function.)

Second one uses a function to try to capture more nuances about the fact that we want the median to be over 0.5. The function is:

\[
\text{score}(\text{opinion}) =
\begin{cases} 
\text{maxScore}, & \text{if } \text{opinion} \geq 0.5 \\
\min\left(\frac{50}{0.5 - \text{opinion}}, \frac{\text{maxScore}}{2}\right), & \text{if } \text{opinion} < 0.5 
\end{cases}
\] 

Where we set maxScore to be $10000$.

\subsection{find-c}
Then for the projected methods where we use Huber-Loss, we also have a $find-c$ version (temporary name). This method initially finds the c for the rest of the run. 

The way it does it it randomly perturbs the resistances and opinions of every node, then finds the c value that most closely approximates the median for all of the perturbed scenarios (after finding the stable opinions). 

\section{Analysis} \label{sec:analysis}
In this section, we provide a comprehensive analysis of Satori. First, we demonstrate that Satori effectively leverages self-reflection to seek better solutions and enhance its overall reasoning performance. Next, we observe that Satori exhibits test-scaling behavior through RL training, where it progressively acquires more tokens to improve its reasoning capabilities. Finally, we conduct ablation studies on various components of Satori's training framework. Additional results are provided in Appendix~\ref{app:results}.



\paragraph{COAT Reasoning v.s. CoT Reasoning.}
\begin{table}[h]
  \begin{center}
  \scriptsize
  \captionsetup{font=small}
  \caption{\textbf{COAT Training v.s. CoT Training.} Qwen-2.5-Math-7B trained with COAT reasoning format (Satori-Qwen-7B) outperforms the same base model but trained with classical CoT reasoning format (Qwen-7B-CoT)}
  \setlength{\tabcolsep}{1.3pt}
  \begin{tabular}{cccccccccc}
    \toprule
    \textbf{Model} & \textbf{GSM8K} & \textbf{MATH500}  &  \textbf{Olym.} & \textbf{AMC2023} & \textbf{AIME2024} \\
    \midrule
    Qwen-2.5-Math-7B-Instruct & 95.2 & 83.6 &41.6& 62.5 &16.7 \\
    Qwen-7B-CoT (SFT+RL) & 93.1 & 84.4  &	42.7 &	60.0 & 10.0 \\
    \midrule
    \textbf{Satori-Qwen-7B}  & 93.2 & 85.6  & 46.6  & 67.5  & 20.0 \\
    \bottomrule
  \end{tabular}
  \label{table:ablation-coat}
  \end{center}
\vspace{-1em}
\end{table}
We begin by conducting an ablation study to demonstrate the benefits of COAT reasoning compared to the classical CoT reasoning. Specifically, starting from the synthesis of demonstration trajectories in the format tuning stage, we ablate the ``reflect'' and  ``explore'' actions, retaining only the ``continue'' actions. Next, we maintain all other training settings, including the same amount of SFT and RL data and consistent hyper-parameters. This results in a typical CoT LLM (Qwen-7B-CoT) without self-reflection or self-exploration capabilities. As shown in Table~\ref{table:ablation-coat}, the performance of Qwen-7B-CoT is suboptimal compared to Satori-Qwen-7B and fails to surpass Qwen-2.5-Math-7B-Instruct, suggesting the advantages of COAT reasoning over CoT reasoning.



\paragraph{Satori Exhibits Self-correction Capability.}
% Please add the following required packages to your document preamble:
% \usepackage{multirow}
\begin{table}[h]
\scriptsize
\captionsetup{font=small}
\caption{\textbf{Satori's Self-correction Capability.} T$\rightarrow$F: negative self-correction; F$\rightarrow$T: positive self-correction.}
\setlength{\tabcolsep}{5pt}
\begin{tabular}{lcccccc}
\toprule
\multirow{3}{*}{\textbf{Model}} & \multicolumn{4}{c}{\textbf{In-Domain}}                                                                            & \multicolumn{2}{c}{\textbf{Out-of-Domain}}              \\ \cmidrule[0.2pt]{2-7} 
                                & \multicolumn{2}{c}{\textbf{MATH500}}                    & \multicolumn{2}{c}{\textbf{OlympiadBench}}              & \multicolumn{2}{c}{\textbf{MMLUProSTEM}}         \\
                                & \textbf{T$\rightarrow$F} & \textbf{F$\rightarrow$T} & \textbf{T$\rightarrow$F} & \textbf{F$\rightarrow$T} & \textbf{T$\rightarrow$F} & \textbf{F$\rightarrow$T} \\ \midrule[0.5pt]
Satori-Qwen-7B-FT                  & 79.4\%                    & 20.6\%                    & 65.6\%                    & 34.4\%                    & 59.2\%                    & 40.8\%                    \\
\textbf{Satori-Qwen-7B}                     & 39.0\%                       & 61.0\%                       & 42.1\%                    & 57.9\%                    & 46.5\%                    & 53.5\%                    \\ \bottomrule
\end{tabular}
\label{table:finegrain-reflect}
\end{table}
We observe that Satori frequently engages in self-reflection during the reasoning process (see demos in Section~\ref{sec:demo}), which occurs in two scenarios: (1) it triggers self-reflection at intermediate reasoning steps, and (2) after completing a problem, it initiates a second attempt through self-reflection. We focus on quantitatively evaluating Satori's self-correction capability in the second scenario. Specifically, we extract responses where the final answer before self-reflection differs from the answer after self-reflection. We then quantify the percentage of responses in which Satori's self-correction is positive (i.e., the solution is corrected from incorrect to correct) or negative (i.e., the solution changes from correct to incorrect). The evaluation results on in-domain datasets (MATH500 and Olympiad) and out-of-domain datasets (MMLUPro) are presented in Table~\ref{table:finegrain-reflect}. First, compared to Satori-Qwen-FT which lacks the RL training stage, Satori-Qwen demonstrates a significantly stronger self-correction capability. Second, we observe that this self-correction capability extends to out-of-domain tasks (MMLUProSTEM). These results suggest that RL plays a crucial role in enhancing the model's true reasoning capabilities.


\paragraph{RL Enables Satori with Test-time Scaling Behavior.}
\begin{figure}[h]
    \centering
    \includegraphics[width=0.5\textwidth]{Figures/rm_shaping_tot_len.pdf}
    \vspace{-2em}
\caption{\textbf{Policy Training Acc. \& Response length v.s. RL Train-time Compute.} Through RL training, Satori learns to improve its reasoning performance through longer thinking.}
\label{fig:test_time_scaling}
\end{figure}
\begin{figure}[h]
    \centering
    \includegraphics[width=0.45\textwidth]{Figures/length_across_levels.pdf}
    \vspace{-1.5em}
\caption{\textbf{Above: Test-time Response Length v.s. Problem Difficulty Level; Below: Test-time Accuracy v.s. Problem Difficulty Level.} Compared to FT model (Satori-Qwen-FT), Satori-Qwen uses more test-time compute to tackle more challenging problems.}
\label{fig:difficulty_level}
\vspace{-1em}
\end{figure}

Next, we aim to explain how reinforcement learning (RL) incentivizes Satori's autoregressive search capability. First, as shown in Figure~\ref{fig:test_time_scaling}, we observe that Satori consistently improves policy accuracy and increases the average length of generated tokens with more RL training-time compute. This suggests that Satori learns to allocate more time to reasoning, thereby solving problems more accurately. One interesting observation is that the response length first decreases from 0 to 200 steps and then increases. Upon a closer investigation of the model response, we observe that in the early stage, our model has not yet learned self-reflection capabilities. During this stage, RL optimization may prioritize the model to find a shot-cut solution without redundant reflection, leading to a temporary reduction in response length. However, in later stage, the model becomes increasingly good at using reflection to self-correct and find a better solution, leading to a longer response length.
 
Additionally, in Figure~\ref{fig:difficulty_level}, we evaluate Satori's test accuracy and response length on MATH datasets across different difficulty levels. Interestingly, through RL training, Satori naturally allocates more test-time compute to tackle more challenging problems, which leads to consistent performance improvements compared to the format-tuned (FT) model.



\paragraph{Large-scale FT v.s. Large-scale RL.}
\begin{table}[h]
  \begin{center}
  \scriptsize
  \captionsetup{font=small}
  \caption{\textbf{Large-scale FT V.S. Large-scale RL} Satori-Qwen (10K FT data + 300K RL data) outperforms same base model Qwen-2.5-Math-7B trained with 300K FT data (w/o RL) across all math and out-of-domain benchmarks.}
  \setlength{\tabcolsep}{1.15pt}
  \vspace{-0.5em}
\begin{tabular}{lccccc}
\toprule
\textbf{(In-domain)}   & \textbf{GSM8K}   & \textbf{MATH500} & \textbf{Olym.} & \textbf{AMC2023} & \textbf{AIME2024} \\ \midrule
Qwen-2.5-Math-7B-Instruct & 95.2 & 83.6                     & 41.6                  & 62.5             & 16.7                 \\
Satori-Qwen-7B-FT (300K)     & 92.3 & 78.2                       & 40.9           & 65.0               & 16.7              \\
\textbf{Satori-Qwen-7B}         & 93.2        & 85.6                     & 46.6           & 67.5             & 20.0                \\ \midrule
\textbf{(Out-of-domain)}  & \textbf{BGQA}    & \textbf{CRUX}  & \textbf{STGQA} & \textbf{TableBench}   & \textbf{STEM}     \\ \midrule
Qwen-2.5-Math-7B-Instruct & 51.3             & 28.0             & 85.3           & 36.3             & 45.2              \\
Satori-Qwen-7B-FT (300K)     & 50.5             & 29.5           & 74.0             & 35.0               & 47.8              \\
\textbf{Satori-Qwen-7B}               & 61.8             & 42.5           & 86.3           & 43.4             & 56.7              \\ \bottomrule
\end{tabular}
  \label{table:ablation-ft-rl}
  \end{center}
\end{table}
We investigate whether scaling up format tuning (FT) can achieve performance gains comparable to RL training. We conduct an ablation study using Qwen-2.5-Math-7B, trained with an equivalent amount of FT data (300K). As shown in Table~\ref{table:ablation-ft-rl}, on the math domain benchmarks, the model trained with large-scale FT (300K) fails to match the performance of the model trained with small-scale FT (10K) and large-scale RL (300K). Additionally, the large-scale FT model performs significantly worse on out-of-domain tasks, demonstrates RL’s advantage in generalization.


\paragraph{Distillation Enables Weak-to-Strong Generalization.} 
\begin{figure}[!t]
    \centering
     \includegraphics[width=0.4\textwidth]
     {Figures/distillation.pdf}
     \vspace{-1.5em}
\caption{\textbf{Format Tuning v.s. Distillation.} Distilling from a Stronger model (Satori-Qwen-7B) to weaker base models (Llama-8B and Granite-8B) are more effective than directly applying format tuning on weaker base models.}
\label{fig:distill}
\vspace{-1em}
\end{figure}
Finally, we investigate whether distilling a stronger reasoning model can enhance the reasoning performance of weaker base models. Specifically, we use Satori-Qwen-7B to generate 240K synthetic data to train weaker base models, Llama-3.1-8B and Granite-3.1-8B. For comparison, we also synthesize 240K FT data (following Section~\ref{subsec:format}) to train the same models. We evaluate the average test accuracy of these models across all math benchmark datasets, with the results presented in Figure~\ref{fig:distill}. The results show that the distilled models outperform the format-tuned models. 

This suggests a new, efficient approach to improve the reasoning capabilities of weaker base models: (1) train a strong reasoning model through small-scale
FT and large-scale RL (our Satori-Qwen-7B) and (2) distill the strong reasoning capabilities of the model into weaker base models. Since RL only requires answer labels as supervision, this approach introduces minimal costs for data synthesis, i.e., the costs induced by a multi-agent data synthesis framework or even more expensive human annotation.



\section{Future Work}
\noindent \textbf{Eliciting Confidence Preference Data.} There can be several different ways of eliciting relative confidence judgments. Prompts could allow for ties in confidence or compare confidence across more than two questions. Kahneman-Tversky Optimization (KTO)~\citep{Ethayarajh2024KTOMA} for LM alignment 
achieves DPO~\citep{Rafailov2023DirectPO} levels of performance by using binary signals of desirability for generations. We can apply KTO to confidence preference data generation by asking for binary signals—--confident or not—--and then converting these into relative judgments, ranking “not confident” answers below “confident” ones.\\\\
\noindent \textbf{Rank Aggregation.} In this work, we explore the most popular rank aggregation methods like Elo rating~\citep{elo_ratings}, Bradley-Terry~\citep{bradley_terry}, and TrueSkill~\citep{true_skill}. Another approach to rank aggregation is to represent preference data as a graph, with nodes as questions and directed edges reflecting match outcomes between questions. Since the outcome of some of these matchups can be inconsistent and non-transitive, algorithms like Rank Centrality~\citep{Negahban2012RankCR}, PageRank~\citep{Page1999ThePC}, and Minimum Feedback Arc Set~\citep{Vahidi2024MinimumWF} could be used to reduce cycles in the graph and better manage these inconsistencies.\\\\
\noindent \textbf{Confidence Estimation for Longform Generations.} While we benchmark on multiple-choice tasks, relative confidence estimation can also extend to longform generation. Log probabilities on answer tokens are commonly used for confidence estimation in multiple-choice tasks, but token-level uncertainty doesn't translate well to longform sequences. Moreover, there may be different levels of uncertainty associated with different aspects of a longform generation, e.g. how complete a generation, vs how factual it is, etc. Relative confidence estimation could produce fine-grained confidence scores for different attributes of a longform response by adjusting the prompt for confidence preferences accordingly.\\\\
\noindent \textbf{Alignment with Relative Confidence.} Works like~\cite{Tian2023FinetuningLM} explore using absolute confidence scores to align language models for different attributes such as factuality, without human annotations (RLAIF). Since relative confidences are more calibrated than absolute confidences, we can instead use relative confidences to construct preference pairs for aligning models on different attributes. \\\\
\noindent \textbf{Curriculum Learning with Difficulty Estimates.} We also explore generating relative confidence judgments without revealing model answers (Section~\ref{sec:results}). These scores correspond to difficulty ratings, which could inform curriculum learning by first training on lower-difficulty examples.


\section{Conclusion}

Our study findings highlight the significant challenges and opportunities in using technical tools like chatbots to improve healthcare knowledge and accessibility for women and children. Despite the widespread smartphone and WhatsApp usage, participants face challenges in using technology due to time constraints, household responsibilities, and limited digital literacy.

Gaps in child health care knowledge were also very prominent among our participants. Participants expressed a clear desire for accessible, reliable information through chatbots, especially concerning their children's health and development. Our findings also revealed a preference for audio and visual content over text-based information, as many women faced challenges in reading and typing in Hindi or English. This suggests that future digital health interventions, including chatbots, should incorporate features such as voice messaging and visual aids to improve usability and engagement.

Overall, these findings emphasize the importance of designing culturally sensitive, user-friendly technological solutions that take into account the sociocultural constraints, digital literacy levels, and daily realities of women in these communities. By addressing these factors, chatbots and other digital tools have the potential to significantly improve healthcare access and outcomes for mothers and children.

 

%As we did not have a chatbot prototype to test, we initially planned to use the Wizard-of-Oz method, where participants would ask questions and a human would provide responses from the other side. This approach creates a mock interface where a human controls the responses instead of AI. However, when the first author arrived at the data collection site in India and began conducting interviews, we realized it would be challenging to implement this approach because the second author was in the USA, and the time zone difference made coordination difficult. As a result, we decided to create a sample WhatsApp conversation in Hindi to present to participants as an example and gauge their preferences for technology engagement.

%While most participants were able to read the sample conversation and provide feedback, a few had difficulty reading Hindi. Additionally, after conducting several interviews, we realized that the Wizard-of-Oz approach might not have been ideal, as many participants were either too busy with daily chores and some face difficulty in text based conversations. 


% By addressing these, future studies can improve engagement with target communities and gather richer data that reflects their unique experiences and needs. 


%Our study's findings highlight the potential of a WhatsApp chatbot intervention to be a viable tool for maternal and child health (MCH) outcomes. Given the high rates of smartphone and WhatsApp usage among participants, their interest in receiving health information through chatbots indicates a receptive target audience. Participants showed interest in information about child development and nutrition, indicating these areas could be of key focus for the intervention. However, to fully comprehend the potential of this chatbot, future research must address several limitations. First, digital literacy is a significant barrier; many users may not be familiar with or comfortable using chatbots. Tailored training and support could help bridge this gap. Second, language preference is crucial to ensure that the chatbot can communicate effectively in the preferred languages of the users which in turn increases accessibility and comprehension. 

%These groups may face additional barriers such as limited access to technology or lower levels of education, which could impact their ability to benefit from the intervention. Targeted outreach and support, perhaps including community-based initiatives, could help overcome these challenges. Our study also emphasizes the continued importance of family support in healthcare decisions. Integrating family engagement strategies with chatbot technology could enhance the effectiveness of the intervention. A holistic approach that involves family members in health education and decision-making processes may lead to better health outcomes. 

%By acknowledging these considerations and capitalizing on the strengths of chatbot technology, our research lays the groundwork for the development of a culturally appropriate and user-centered intervention that empowers mothers and caregivers, ultimately contributing to improved MCH outcomes in India. 


\section*{Acknowledgment}

We want to extend heartfelt thanks to all participants who generously contributed their time and experience with us.  
% Their thoughts and perspectives made a significant contribution to this research. 
% We are also grateful to our colleagues for their advice and comments we received during the study. 
We are also grateful to the NGOs who helped us recruit study participants and shared their knowledge and support to guide the practical application of digital technology for MCH. 
Thank you all for your invaluable contribution. 


\begin{thebibliography}{00}
\bibitem{b1}Montgomery, A. (2014). Pregnancy-related deaths in India causes of death and the use of health services (Doctoral dissertation). 
\bibitem{b2} Barros, A. J., Victora, C. G., Cesar, J. A., Neumann, N. A., Bertoldi, A. D., Gwatkin, D., ..and Yazbeck, A. (2013). Brazil: are health and nutrition programs reaching the neediest. World Bank. 
\bibitem{b4}Ismail, A., Yadav, D., Gupta, M., Dabas, K., Singh, P., and Kumar, N. (2022). Imagining caring futures for frontline health work. Proceedings of the ACM on Human-Computer Interaction, 6(CSCW2), 1-30. 
\bibitem{b6} El Ayadi, A. M., Singh, P., Duggal, M., Kumar, V., Kaur, J., Sharma, P., ... and Diamond-Smith, N. G. (2023). Feasibility and acceptability of Saheli, a WhatsApp Chatbot, on COVID-19 vaccination among pregnant and breastfeeding women in rural North India. BMJ Innovations, 9(4). 
\bibitem{b7} Van Hauwaert, R., Mateus, A. R., Coutinho, A. L., Rodrigues, J., Martins, A. R., Vilela, F., \& Almeida, D. (2024). The role of digital health technologies on maternal health literacy: a narrative review. Emerging Technologies for Health Literacy and Medical Practice, 47-65.

\bibitem{b9} Collins, T. E., Akselrod, S., Altymysheva, A., Nga, P. T. Q., Banatvala, N., \& Berlina, D. (2023). The promise of digital health technologies for integrated care for maternal and child health and non-communicable diseases. bmj, 381.
\bibitem{b10} Goto, R., Watanabe, Y., Yamazaki, A., Sugita, M., Takeda, S., Nakabayashi, M., \& Nakamura, Y. (2021). Can digital health technologies exacerbate the health gap? A clustering analysis of mothers’ opinions toward digitizing the maternal and child health handbook. SSM-Population Health, 16, 100935.
\bibitem{b11} Whitehead, L., Robinson, S., Arabiat, D., Jenkins, M., \& Morelius, E. (2024). The report of access and engagement with digital health interventions among children and young people: systematic review. JMIR pediatrics and parenting, 7, e44199.

\bibitem{b14}Moulaei, K., Moulaei, R., \& Bahaadinbeigy, K. (2023). Barriers and facilitators of using health information technologies by women: a scoping review. BMC medical informatics and decision making, 23(1), 176.

\bibitem{b17}Gold, N., Hu, X. Y., Denford, S., Xia, R. Y., Towler, L., Groot, J., ... \& Yardley, L. (2021). Effectiveness of digital interventions to improve household and community infection prevention and control behaviours and to reduce incidence of respiratory and/or gastro-intestinal infections: a rapid systematic review. BMC Public Health, 21, 1-15.
\bibitem{b18}Rahman, M. O., Yamaji, N., Sasayama, K., Yoneoka, D., \& Ota, E. (2023). Technology‐based innovative healthcare solutions for improving maternal and child health outcomes in low‐and middle‐income countries: A network meta‐analysis protocol. Nursing Open, 10(1), 367-376.
\bibitem{b19}Aranda-Jan, C.B., Mohutsiwa-Dibe, N. \& Loukanova, S. Systematic review on what works, what does not work and why of implementation of mobile health (mHealth) projects in Africa. BMC Public Health 14, 188 (2014). 
\bibitem{20}Scott Kruse, C., Karem, P., Shifflett, K., Vegi, L., Ravi, K., \& Brooks, M. (2018). Evaluating barriers to adopting telemedicine worldwide: a systematic review. Journal of telemedicine and telecare, 24(1), 4-12.
\bibitem{21}Winters N, Venkatapuram S, Geniets A, Wynne-Bannister E. Prioritarian principles for digital health in low resource settings. J Med Ethics. 2020 Apr;46(4):259-264. doi: 10.1136/medethics-2019-105468. Epub 2020 Jan 16. PMID: 31949027; PMCID: PMC7231431.
\bibitem{22}Yadav, Deepika, et al. "Feedpal: Understanding opportunities for chatbots in breastfeeding education of women in india." Proceedings of the ACM on Human-Computer Interaction 3.CSCW (2019): 1-30.
\bibitem{b23}Fan, M., Shi, S., \& Truong, K. N. (2020). Practices and Challenges of Using Think-Aloud Protocols in Industry: An International Survey. Journal of Usability Studies, 15(2).
\bibitem{b24}Capasso, A., Colomar, M., Ramírez, D., Serruya, S., \& de Mucio, B. (2024). Digital health and the promise of equity in maternity care: A mixed methods multi-country assessment on the use of information and communication technologies in healthcare facilities in Latin America and the Caribbean. Plos one, 19(2), e0298902
\bibitem{b25}Knop, M. R., Nagashima-Hayashi, M., Lin, R., Saing, C. H., Ung, M., Oy, S., ... \& Yi, S. (2024). Impact of mHealth interventions on maternal, newborn, and child health from conception to 24 months postpartum in low-and middle-income countries: a systematic review. BMC medicine, 22(1), 196.



\end{thebibliography}

\vspace{12pt}

\end{document}
