\documentclass[conference]{IEEEtran}
\IEEEoverridecommandlockouts
% The preceding line is only needed to identify funding in the first footnote. If that is unneeded, please comment it out.
\usepackage{cite}
\usepackage{amsmath,amssymb,amsfonts}
\usepackage{algorithmic}
\usepackage{graphicx}
\usepackage{textcomp}
\usepackage{tabularx}
\usepackage{xcolor}
\usepackage{booktabs}
\usepackage{soul}


\newcommand{\gtadd}[1]{\textcolor{red}{#1}}
\newcommand{\gtrm}[1]{\st{#1}}


\def\BibTeX{{\rm B\kern-.05em{\sc i\kern-.025em b}\kern-.08em
    T\kern-.1667em\lower.7ex\hbox{E}\kern-.125emX}}
\setlength{\columnsep}{0.24in}
\usepackage[top=0.75in, bottom=1.05in, left=0.625in, right=0.625in]{geometry}
\begin{document}
\title{``Who Has the Time?'': Understanding Receptivity to  Health Chatbots among Underserved Women in India}
% \title{\textit{``The family members do have a say in decisions
% regarding health’’}: Exploring Factors Limiting Maternal Health Tech Adoption in Underserved Communities
% %Beyond Access: Exploring Factors Limiting Maternal Health Tech Adoption in Underserved Communities\\
% Access in Place, Adoption Unmet??
% {\footnotesize \textsuperscript{*}Note: Sub-titles are not captured in Xplore and
% should not be used}
% \thanks{Department of Biomedical Informatics,Emory University.}
% }

\author{\IEEEauthorblockN{\textsuperscript{} Manvi S}
\IEEEauthorblockA{\textit{Biomedical Informatics} \\
\textit{Emory University}\\
Atlanta, USA \\
mmanvi@emory.edu}
\and
\IEEEauthorblockN{\textsuperscript{} Roshini Deva}
\IEEEauthorblockA{\textit{Biomedical Informatics} \\
\textit{Emory university}\\
Atlanta, USA \\
droshin@emory.edu}
\and
\IEEEauthorblockN{\textsuperscript{} Neha Madhiwalla}
\IEEEauthorblockA{\textit{Research Director} \\
\textit{ARMMAN Organization}\\
Mumbai, India \\
neha@armman.org}
\and
\IEEEauthorblockN{}
\IEEEauthorblockA{} 
\and
\IEEEauthorblockN{\textsuperscript{} Azra Ismail}
\IEEEauthorblockA{\textit{Biomedical Informatics and }\\
\textit{Global Health} \\
\textit{Emory University}\\
Atlanta, USA \\
azra.ismail@emory.edu}}



\maketitle

\begin{abstract}
%Despite multiple interventions, maternal and child health (MCH) in India continues to encounter obstacles that result in subpar healthcare practices, especially in underprivileged populations. This study explores how can we design a culturally relevant chatbot for mothers and children upto 3 years of age. Using a mixed-method approach that blends quantitative surveys and qualitative interviews, this study investigates the target population's adoption of technology, communication preferences, and healthcare behaviors. The findings reveal that using a smartphone and WhatsApp is common, a reliance on medical professions and openess to use chatbots and online advice systems. These realizations highlight how crucial it is to create tailored interventions that combine technology and traditional support networks to enhance MCH outcomes. 
Access to health information and services among women continues to be a major challenge in many communities globally. In recent years, there has been a growing interest in the potential of chatbots to address this information and access gap. We conducted interviews and focus group discussions with underserved women in urban India to understand their receptivity towards the use of chatbots for maternal and child health, as well as barriers to their adoption. Our findings uncover gaps in digital access and literacies, and perceived conflict with various responsibilities that women are burdened with, which shape their interactions with digital technology. Our paper offers insights into the design of chatbots for community health that can meet the lived realities of women in underserved settings.
%Access to health information and services among women continues to be a major challenge in many communities globally. In recent years, there has been a growing interest in the potential of chatbots to address this information and access gap. However, the adoption of health technology remains a challenge in underserved communities where digital access and literacies may be constrained. Through interviews and focus group discussions, we investigate perceptions of chatbots for maternal and child health, as well as barriers to their adoption, among underserved women in urban India. Our findings uncover the various responsibilities that women are burdened with, which shape their interactions with digital technology. Our paper offers insights into the design of chatbots for community health that can meet the lived realities of women in underserved settings.

%how women in underserved communities adopt technology for accessing maternal and child health (MCH) information and services. Our findings reveal challenges associated with technology adoption.

\end{abstract}

\begin{IEEEkeywords}
% component, formatting, style, styling, insert
Human-centered computing, Health implications, Ethical/Societal Implications
\end{IEEEkeywords}

\documentclass[../main.tex]{subfiles}
\graphicspath{{../images/}}
\makeatletter
\def\input@path{{../images/}}
\makeatother
\begin{document}
\section{Introduction}
\begin{figure}
\centering
\begin{tikzpicture}
\node[inner sep=0pt] (ws) at (0, 0) {
\includegraphics[height=.4\textwidth, trim={10cm 0 10cm 0},clip]{world_space.png}};
\node[inner sep=0pt] (cs) at (6,0) {\includegraphics[height=.4\textwidth, trim={10cm 1cm 10cm 4cm},clip]{conf_space.png}};
\end{tikzpicture}
\vspace{-5pt}
\label{fig:pbrm_intro}
\caption{\textbf{Left}: Shows world space obstacles as grey spheres. Robots start and goal configuration is colored red and green, respectively. Configurations along the computed path are colored transparent blue. \textbf{Right:} Mapped world space scenario to configuration space. Obstacle region is the grey mesh. Red spheres are collision-free regions computed by the neural SCDF. The optimized shortest path in the convex corridor is the blue curve.}
\vspace{-25pt}
\end{figure}
Motion planning is the problem of finding a collision-free trajectory that connects a given start and goal configuration. The planning takes place in the configuration space of the robot. For single body robots, like mobile robots or drones, the configuration space and the world space are usually the same. This simplifies the planning, since explicit obstacle representations are available which enables geometrical tools like separating hyperplanes, smallest distance to obstacles etc., to be used when designing motion planning algorithms. For multi-body robots like manipulators, the situation is completely different. The world space obstacles are usually mapped to non-convex regions, and to make the problem even harder, the mapping is usually not known. Forming explicit representations of the obstacle region in the configuration space is usually too expensive or intractable. Despite all of this, sampling based planners are used with great success, which mainly is due to their use of implicit representations of the obstacle region. The basic idea is to construct a graph in the configuration space that covers and connects the collision-free region. From this graph, a path can be extracted that connects a given start and goal configuration. The approach is computationally expensive, since the graph is constructed with the smallest geometrical building block available, points, which represents a collision-check. Furthermore, the extracted paths from the graph are non-smooth and jagged due to the stochastic nature of the approach. This adds an additional post-processing step to the process, where the paths are shortcutted and smoothened, before the path can be used for tracking. Clearly a lot of time is invested to form this graph and produce smooth paths. Thus, if the obstacles start to move, then all of this work is done in no use, since all points that make up this graph need to be re-verified, which is simply too time consuming to be done in real time.
\\\\
In this work, we want to address the existing drawbacks of the sampling based planners. Our main contribution is an improved motion planner where each vertex in the graph covers a collision-free region in the form of a sphere instead of a point and where the edges are formed with neighboring intersecting spheres. This representation has the advantage of instead of returning piecewise linear paths, returning a sequence of overlapping spheres, i.e. a convex corridor, that connects a given start and goal configuration, illustrated in Figure \ref{fig:pbrm_intro}. This convex corridor allows us to use convex optimization to produce smooth trajectories, instead of computationally expensive post-processing methods. The representation further allows us to estimate the coverage of the collision-free space, which gives us awareness and feedback in the offline roadmap construction phase. Finally, our representation is simple to adapt to moving obstacles, simply requery for the new radii and recheck for intersections. 
\\\\
The spherical collision-free regions are formed using a signed distance function (SDF), which is a function that returns the smallest distance from an arbitrary point to the boundary of an obstacle. As the name implies, the distance is signed, thus if the point is inside the obstacle it is negative otherwise positive. If the distance is positive, a sphere with radius equal to the distance is guaranteed to cover a collision-free region. Using an SDF in motion planning is not new, but what is novel about our approach is that we express the distance in the configuration space instead of the world space and by doing so allows us to form these convex collision-free regions. We refer to the resulting SDF as a signed configuration distance function (SCDF). Computing an SCDF analytically is non-trivial, our approach is therefore to parameterize the SCDF with a deep neural network and learn the mapping by supervised learning. Our resulting neural SCDF can compute distances for different parameter values of obstacle shapes and we also show how multiple distances can be combined, thus making our approach flexible.
\section{Related work}
Motion planning algorithms can roughly be divided into three families, grid-based, sampling based and optimization based methods. Grid-based methods (GBM) discretize the planning space from which a graph is then compiled. A standard search method is A$^\star$ \citep{a_star}, which is classified as an \textit{informed} search method, since it employs a heuristic function to speed up the search. A$^\star$ guarantees to return an optimal path at the level of discretization used. GBMs usually discretize the planning space by a regular lattice and this limits the GBMs to problems with low dimensionality due to the curse of dimensionality. Thus, GBMs are usually limited to single-body robots where the degrees of freedom (DOF) are low. To overcome the inherent scaling problem with the GBMs, stochastic methods are usually used for multi-body robots. These methods are termed as sampling-based methods (SBM) and core members within this family are the rapidly-exploring random trees (RRT) \citep{rrt} and the probabilistic roadmap (PRM) \citep{prm}. RRT grows a tree from the start configuration and explores the collision-free region in a rapid way until it is able to connect to the goal region. RRT is usually improved by bi-directional planning \citep{rrt_connect}, i.e. an additional tree is grown from the goal configuration and the trees are tested for connection after any tree has been expanded. RRT is a single-query method, thus it searches for a path from scratch each time it is queried. Contrary to this, PRM is a multi-query method, which solves for multiple queries without starting from scratch. PRM does this by creating a roadmap (graph) that covers the collision-free space as an offline step. The graph is then used to solve for multiple queries. PRMs are used in cases where the environment does not change since the extra offline step is too computationally costly and needs to be re-done if the environment is changed. In our work, we address this inherent issue by using a different roadmap representation. Our vertices in the graph cover a collision-free region in the form of spheres and we form the edges by checking for intersecting spheres. If something in the environment changes, we recompute the spheres radii and recheck the intersections, without relying on collision detection. We use a trained neural network to compute the sphere radius, therefore querying for the radius can be done fast, hence our representation enables the PRM for dynamic environments.
\\\\
In the recent decades, optimization based methods (OBM) \citep{chomp, schulman, itomp, stomp} have been introduced as an alternative to SBM for multi-body robots. Like the SBM, the OBMs scale well to higher dimensional problems and produce smoother motion. It is common to use a SDF in the optimization since it is a smooth function, thus enabling gradient-based methods. However, the standard way of expressing the SDF is in world space. The distance therefore needs to be mapped to the configuration space by the forward kinematics. This mapping makes the optimization problem a non-linear program (NLP), which is computationally expensive to solve. Recently, a different approach has been proposed. In \cite{mp_gcs} motion planning is formulated as a convex optimization problem by using the graph of convex sets framework \citep{gcs}. The underlying idea is to decompose the collision-free space into intersecting convex sets from which a convex optimization problem is formulated. In cases where an explicit representation of the obstacles in the configuration space exists, like for single-body robots, creating collision-free convex regions can be done fast \citep{iris}. For multi-body robots, this is non-trivial. Existing work does this successfully \citep{iris_nlp, iris_c} by an optimization based approach, but the methods are still too time consuming to be used in the presence of moving obstacles. Our approach is instead to use deep learning to learn an SDF expressed in the configuration space. With this, we can query for shortest distances to the collision boundary, which allows us to expand spherical regions which are collision-free. Our approach is fast and therefore enables our suggested roadmap planner to be used in dynamic environments.
\\\\
Recent research has focused on learning collision detection \citep{fk_kernel_distance, diffco, graphdistnet} by predicting the signed distance between the robot links and the surrounding obstacles in the world space. The learned SDF is used in trajectory optimization but since the distance is expressed in the world space, the problem becomes an NLP and therefore takes a long time to solve. We take a novel approach and suggest to instead express the signed distance in the configuration space. This allows us to improve the PRM at the same time as it enables convex optimization for trajectory optimization, which runs faster and is more reliable than NLP solvers. In \cite{cspf} a learned signed distance function in the configuration space is proposed similar to our approach. However, their approach is restricted to point cloud representations, while we propose to represent the obstacles as parameterized geometric shapes, e.g. spheres. Furthermore, we also show how to use our learned SCDF to improve an existing roadmap planner.
\section{Problem formulation}
A robot is located in the world space, $\W \subset \R^3 $. The unique location of the robot is given by its configuration $\q \in \C$, where $\C$ is the configuration space. The set of points covered by the robots bodies at a certain configuration is expressed as $\B(\q) \subset \W$. The robot is surrounded by $\NrObst$ obstacles $\O = \bigcup_{i=1}^{\NrObst} \O_i$, where  $\O_i \subset \W$. The representation of the obstacle in the configuration space is the set $\C\O_i = \{\q \in \C \: |\: \B(\q) \cap \O_i \neq \emptyset \}$. The obstacle space is formed as $\Co = \bigcup_{i=1}^{\NrObst} \C \O_i$. The complement is referred to as the free space, $\Cf = \C \setminus \Co$. The path planning problem is a tuple, ($\Cf$, $\qStart$, $\qGoal$), where we want to connect a query pair, consisting of a start, $\qStart$, and goal configuration, $\qGoal$, with a geometric path, $\q(s): [0, 1] \mapsto \Cf$, such that $\q(0)=\qStart$ and $\q(1)=\qGoal$, or report correctly when such a path does not exist.
\end{document}

\section{Related Work}
% \subsection{Vision Language Model}
% 시각장애인에서 상황을 설명할 DB가 없으니 만들었다. 그리고 이를 VLM에 튜닝했다.
\subsection{Technical approaches for assisting the visually-impaired}


\subsection{Datasets for visual instruction tuning}

\section{Method}

\subsection{Overview \& Setup}

Our framework consists of a large, highly capable model \textbf{\bigmodel} and a smaller, resource-efficient model \textbf{\smallmodel}. We assume that $S \in \mathbb{N}$ and $L \in \mathbb{N}$ represent the parameter count of each model with $S \ll L$. Both models can either function as classifiers (i.e., $\mathcal{M}: \mathbb{R}^D \rightarrow [C]$ with $\mathbb{R}^D$ denoting the input space and $C$ the number of total classes), or (multi-modal) sequence models (i.e., $\mathcal{M}: \mathbb{R}^D \rightarrow [V]^{T}$ where $V$ is the vocabulary and $T$ is the sequence length). We include experiments on all of these model classes in Section~\ref{sec:experiments}. Furthermore, we do not require a shared model family to be deployed on both \smallmodel and \bigmodel; for example, \smallmodel could be a custom convolutional neural network optimized for efficient inference and \bigmodel a vision transformer~\citep{dosovitskiy2020image}. The primary objective is to design a deferral mechanism that enables \smallmodel to decide when to return its predictions without the assistance of \bigmodel and when to instead defer to it.

\looseness=-1
Deferral decisions are made using signals derived from the small model \smallmodel as this approach is typically more cost-effective than employing a separate routing mechanism~\citep{teerapittayanon2016branchynet}. Approaches that involve querying the large model \bigmodel to assist in making deferral decisions at test time are excluded from our setup. Such methods --- common in domains like LLMs --- are counterproductive to our goal since querying \bigmodel defeats the purpose of making a deferral decision in the first place?. Examples of these inapplicable methods include collaborative LLM frameworks~\citep{mielke2022reducing} and techniques that rely on semantic entropy for uncertainty estimation~\citep{kuhn2023semantic}. As part of our setup, we assume that \smallmodel is strictly less capable than \bigmodel --- a realistic scenario in practice supported by scaling laws~\citep{kaplan2020scaling}. Under this assumption, mistakes made by \bigmodel are also made by \smallmodel; however, \smallmodel may make additional errors that \bigmodel would avoid. This reflects the general observation that larger models tend to outperform smaller models across a wide range of tasks.

As discussed in Section~\ref{sec:related-word}, the choice of deferral strategy often depends on the level of access available to \smallmodel. We assume white box access with full access to \smallmodel's internals. As such, deferral mechanisms can be directly integrated into the model's architecture and parameters. This involves fine-tuning \smallmodel to predict deferral decisions or to incorporate rejection mechanisms within its predictive process. Our work falls into this category as it proposes a new loss function to fine-tune \smallmodel. 

Our goal is to train a small model that can effectively distinguish between correct and incorrect predictions. While many past works have considered the question of whether it is possible to find proxy measures for prediction correctness, the central question we ask is:
\begin{center}
\textbf{Can we \emph{optimize} the small model \smallmodel to separate correct from incorrect predictions?}
\end{center}
\noindent We show that this is indeed achievable through a carefully designed fine-tuning stage that does not require any architectural modifications. This ensures that the ability to separate correct from incorrect decisions is integrated seamlessly into \smallmodel's existing structure.


\subsection{Confidence-Tuning for Deferral}

\begin{figure}
    \centering
    \resizebox{\linewidth}{!}{
    \begin{figure}[h]
\begin{center}
   \includegraphics[width=0.99\linewidth]{figs/pdf/loss.pdf}
\end{center}
   \caption{
    Training loss of VAR \textit{vs.} FlexVAR. FlexVAR demonstrates a faster convergence rate. We report the results with trained 40 epochs ($\sim$ 70K iterations). 
   }
\label{fig:loss}
\end{figure}

    }
    \vspace{-15pt}
    \caption{\textbf{Overview of \loss}: We want correctly predicted samples maintain their current prediction by ensuring that cross entropy is decreased (top, green). At the same time, we want incorrectly predicted samples to yield a uniform confidence across all classes, leading to a low overall confidence score (bottom, red).}
    \label{fig:opt_goal}
\end{figure}

\textbf{Stage 1: Standard Training.} We begin with a \smallmodel that has already been trained on the tasks it is intended to perform upon deployment. However, due to its limited capacity, \smallmodel cannot achieve the performance levels of \bigmodel. Importantly, we make no assumptions about the training process of \smallmodel—whether it was trained from scratch without supervision from an external model or through a distillation approach.

\sloppy
\textbf{Stage 2: Correctness-Aware Finetuning with \loss.} Next, we introduce a correctness-aware loss, dubbed \loss, to fine-tune \smallmodel for improved confidence calibration. Specifically, the model is trained to make correct predictions with high confidence while reducing the confidence of incorrect predictions (see Figure~\ref{fig:opt_goal}). This loss can either rely on true labels or utilize the outputs of \bigmodel with soft probabilities as targets. 


For a standard classification model, the calibration loss is defined as the following hybrid loss
\begin{align}
\mathcal{L} &= \alpha \mathcal{L}_\text{corr} + (1 - \alpha) \mathcal{L}_\text{incorr} \\
\mathcal{L}_\text{corr} &= \frac{1}{N} \sum_{i=1}^{N} \mathds{1}\{ y_i = \hat{y}_i \} \text{CE}(p_i(\mathbf{x}_i), y_i) \\
\mathcal{L}_\text{incorr} &= \frac{1}{N} \sum_{i=1}^{N} \mathds{1}\{ y_i \neq \hat{y}_i \} \text{KL}\left(p_i(\mathbf{x}_i) \parallel \mathcal{U}\right)
\end{align}
where  \( y_i \) and \( \hat{y}_i \) are the true and predicted labels for $\mathbf{x}_i$, respectively, \( p_i \) is the predicted probability distribution of \smallmodel over classes, \( \mathcal{U} \) represents the uniform distribution over all classes, \( N \) denotes the number samples in the current batch, \( \alpha \in (0, 1) \) is a tunable hyperparameter controlling the emphasis between correct and incorrect predictions, and the cross-entropy function and KL divergence are defined as \( \text{CE}(p, y) = -\sum_{c} y_c \log p_c \) and \( \text{KL}(p \parallel q) = \sum_{c} p_c \log ( \frac{p_c}{q_c}) \), respectively. We note that a similar loss has previously been proposed in Outlier Exposure (OE)~\citep{hendrycks2018deep} for out-of-distribution (OOD) sample detection. Here, the goal is to make sure that OOD examples are assigned low confidence scores by tuning the confidence on a auxiliary outlier dataset. However, to the best of our knowledge, this idea has not previously been used to improve deferral performance of a smaller model in a cascading chain.

We emphasize that the trade-off parameter $\alpha$ plays a critical role as part of this optimization setup as it directly influences model utility and deferral performance. A lower value of \(\alpha\) emphasizes reducing confidence in incorrect predictions by pushing them closer to the uniform distribution, making the model more cautious in regions where it may make mistakes. Conversely, a higher value of \(\alpha\) encourages the model to increase its confidence on correct predictions, sharpening its decision boundaries and enhancing accuracy where it is already performing well. Thus, \(\alpha\) serves as a crucial hyperparameter that balances the trade-off between improving calibration by mitigating overconfidence in errors and reinforcing confidence in accurate classifications. By appropriately tuning \(\alpha\), practitioners can control the model’s behavior to achieve a desired balance between reliability in uncertain regions and decisiveness in confident predictions, tailored to the specific requirements of their application.

We further generalize this loss to token-based models (e.g., LMs and VLMs), formulated as
\ifarxiv
\small
\fi
\begin{align}
    \mathcal{L}_\text{corr} & = \frac{1}{N} \sum_{i=1}^{N} \sum_{t=1}^{T} \mathds{1}\{ y_{i,t} = \hat{y}_{i,t} \} \text{CE}(p_{i,t}(\mathbf{x}_i), y_{i,t}) \\
    \mathcal{L}_\text{incorr} & = \frac{1}{N} \sum_{i=1}^{N} \sum_{t=1}^{T} \mathds{1}\{ y_{i,t} \neq \hat{y}_{i,t} \} \text{KL}\left(p_{i,t}(\mathbf{x}_i) \parallel \mathcal{U}\right)
\end{align}
\normalsize
where \( y_{i,t} \) and \( \hat{y}_{i,t} \) denote the true and predicted tokens at position \( t \) for sample \( i \), \( p_{i,t} \) is the predicted token distribution at position \( t \) for sample \( i \), and \( T \) is the sequence length for the token-based model. The token-level loss ensures that correct token predictions are made confidently while incorrect tokens are assigned smaller confidences.

\sloppy
\textbf{Stage 3: Confidence Computation \& Thresholding.} After fine-tuning \smallmodel with \loss, we apply standard confidence- and entropy-based techniques for model uncertainty to obtain a deferral signal. We use the selective prediction framework to determine whether a query point~$\mathbf{x} \in \mathbb{R}^D$ should be accepted by \smallmodel or routed to \bigmodel. Selective prediction alters the model inference stage by introducing a deferral state through a \textit{gating mechanism}~\citep{yaniv2010riskcoveragecurve}. At its core, this mechanism relies on a deferral function $g:\mathbb{R}^D \rightarrow \mathbb{R}$ which determines if \smallmodel should output a prediction for a sample~$\mathbf{x}$ or defer to \bigmodel. Given a targeted acceptance threshold $\tau$, the resulting predictive model can be summarized as:
\begin{equation}
\label{eq:deferral}
    (\mathcal{M}_S,\mathcal{M}_L,g)(\mathbf{x}) = \begin{cases}
  \mathcal{M}_S(\mathbf{x})  & g(\mathbf{x}) \geq \tau \\
  \mathcal{M}_L(\mathbf{x}) & \text{otherwise.}
\end{cases}
\end{equation}

\emph{Classification Models (Max Softmax).} Let \(\mathcal{M}_S\) produce a categorical distribution
\(\{p(y=c \mid \mathbf{x})\}_{c=1}^C\) over \(C\) classes. 
Then we define the gating function as
\begin{align}
g_{\text{CL}}(\mathbf{x}) \;=\; \max_{1 \,\le\, c \,\le\, C}\;p\bigl(y = c \,\big\vert\, \mathbf{x}\bigr).
\end{align}

\emph{Token-based Models (Negative Predictive Entropy).} 
Let \(\mathcal{M}_S\) produce a sequence of categorical distributions 
\(\{p(y_t = c \mid \mathbf{x})\}_{c=1}^C\) for each token index \(t \in T\). Then we define the gating function as
\begin{equation}
\footnotesize
g_{\text{NENT}}(\mathbf{x}) 
= \; \frac{1}{T} \sum_{t=1}^{T} \sum_{c=1}^{C} 
    p\bigl(y_t = c \,\big\vert\, \mathbf{x}\bigr)\,\log p\bigl(y_t = c \,\big\vert\, \mathbf{x}\bigr),
\end{equation}
where \(y_t \in [C]\) is the predicted token at time step \(t\), \(p(y_t=c \mid \mathbf{x})\) is the (conditional) probability of token \(k\) at step \(t\), and \(T\) is the total number of token positions for the sequence. Across both model classes, higher values of either $g_{\text{CL}}$ or $g_{\text{NENT}}$ indicate higher confidence in the predicted class or sequence generation, respectively.
Our findings demonstrate that VLMs struggle with spatial reasoning in 3D vision and that there are significant performance differences between the tested VLMs. While they understand the task and manage to outperform random agents in simple spatial tasks, they struggle with more complex configurations and intricate problem properties. Interestingly, VLMs demonstrate stronger performance in 2D vision representations compared to text-based tasks. This suggests that visual alignment for 3D spatial reasoning continues to pose a significant challenge, underscoring persistent gaps in VLM capabilities and highlighting barriers to achieving human-level cognitive performance.
\section{Future Work}
\noindent \textbf{Eliciting Confidence Preference Data.} There can be several different ways of eliciting relative confidence judgments. Prompts could allow for ties in confidence or compare confidence across more than two questions. Kahneman-Tversky Optimization (KTO)~\citep{Ethayarajh2024KTOMA} for LM alignment 
achieves DPO~\citep{Rafailov2023DirectPO} levels of performance by using binary signals of desirability for generations. We can apply KTO to confidence preference data generation by asking for binary signals—--confident or not—--and then converting these into relative judgments, ranking “not confident” answers below “confident” ones.\\\\
\noindent \textbf{Rank Aggregation.} In this work, we explore the most popular rank aggregation methods like Elo rating~\citep{elo_ratings}, Bradley-Terry~\citep{bradley_terry}, and TrueSkill~\citep{true_skill}. Another approach to rank aggregation is to represent preference data as a graph, with nodes as questions and directed edges reflecting match outcomes between questions. Since the outcome of some of these matchups can be inconsistent and non-transitive, algorithms like Rank Centrality~\citep{Negahban2012RankCR}, PageRank~\citep{Page1999ThePC}, and Minimum Feedback Arc Set~\citep{Vahidi2024MinimumWF} could be used to reduce cycles in the graph and better manage these inconsistencies.\\\\
\noindent \textbf{Confidence Estimation for Longform Generations.} While we benchmark on multiple-choice tasks, relative confidence estimation can also extend to longform generation. Log probabilities on answer tokens are commonly used for confidence estimation in multiple-choice tasks, but token-level uncertainty doesn't translate well to longform sequences. Moreover, there may be different levels of uncertainty associated with different aspects of a longform generation, e.g. how complete a generation, vs how factual it is, etc. Relative confidence estimation could produce fine-grained confidence scores for different attributes of a longform response by adjusting the prompt for confidence preferences accordingly.\\\\
\noindent \textbf{Alignment with Relative Confidence.} Works like~\cite{Tian2023FinetuningLM} explore using absolute confidence scores to align language models for different attributes such as factuality, without human annotations (RLAIF). Since relative confidences are more calibrated than absolute confidences, we can instead use relative confidences to construct preference pairs for aligning models on different attributes. \\\\
\noindent \textbf{Curriculum Learning with Difficulty Estimates.} We also explore generating relative confidence judgments without revealing model answers (Section~\ref{sec:results}). These scores correspond to difficulty ratings, which could inform curriculum learning by first training on lower-difficulty examples.


\section{Conclusion}

Our study findings highlight the significant challenges and opportunities in using technical tools like chatbots to improve healthcare knowledge and accessibility for women and children. Despite the widespread smartphone and WhatsApp usage, participants face challenges in using technology due to time constraints, household responsibilities, and limited digital literacy.

Gaps in child health care knowledge were also very prominent among our participants. Participants expressed a clear desire for accessible, reliable information through chatbots, especially concerning their children's health and development. Our findings also revealed a preference for audio and visual content over text-based information, as many women faced challenges in reading and typing in Hindi or English. This suggests that future digital health interventions, including chatbots, should incorporate features such as voice messaging and visual aids to improve usability and engagement.

Overall, these findings emphasize the importance of designing culturally sensitive, user-friendly technological solutions that take into account the sociocultural constraints, digital literacy levels, and daily realities of women in these communities. By addressing these factors, chatbots and other digital tools have the potential to significantly improve healthcare access and outcomes for mothers and children.

 

%As we did not have a chatbot prototype to test, we initially planned to use the Wizard-of-Oz method, where participants would ask questions and a human would provide responses from the other side. This approach creates a mock interface where a human controls the responses instead of AI. However, when the first author arrived at the data collection site in India and began conducting interviews, we realized it would be challenging to implement this approach because the second author was in the USA, and the time zone difference made coordination difficult. As a result, we decided to create a sample WhatsApp conversation in Hindi to present to participants as an example and gauge their preferences for technology engagement.

%While most participants were able to read the sample conversation and provide feedback, a few had difficulty reading Hindi. Additionally, after conducting several interviews, we realized that the Wizard-of-Oz approach might not have been ideal, as many participants were either too busy with daily chores and some face difficulty in text based conversations. 


% By addressing these, future studies can improve engagement with target communities and gather richer data that reflects their unique experiences and needs. 


%Our study's findings highlight the potential of a WhatsApp chatbot intervention to be a viable tool for maternal and child health (MCH) outcomes. Given the high rates of smartphone and WhatsApp usage among participants, their interest in receiving health information through chatbots indicates a receptive target audience. Participants showed interest in information about child development and nutrition, indicating these areas could be of key focus for the intervention. However, to fully comprehend the potential of this chatbot, future research must address several limitations. First, digital literacy is a significant barrier; many users may not be familiar with or comfortable using chatbots. Tailored training and support could help bridge this gap. Second, language preference is crucial to ensure that the chatbot can communicate effectively in the preferred languages of the users which in turn increases accessibility and comprehension. 

%These groups may face additional barriers such as limited access to technology or lower levels of education, which could impact their ability to benefit from the intervention. Targeted outreach and support, perhaps including community-based initiatives, could help overcome these challenges. Our study also emphasizes the continued importance of family support in healthcare decisions. Integrating family engagement strategies with chatbot technology could enhance the effectiveness of the intervention. A holistic approach that involves family members in health education and decision-making processes may lead to better health outcomes. 

%By acknowledging these considerations and capitalizing on the strengths of chatbot technology, our research lays the groundwork for the development of a culturally appropriate and user-centered intervention that empowers mothers and caregivers, ultimately contributing to improved MCH outcomes in India. 


\section*{Acknowledgment}

We want to extend heartfelt thanks to all participants who generously contributed their time and experience with us.  
% Their thoughts and perspectives made a significant contribution to this research. 
% We are also grateful to our colleagues for their advice and comments we received during the study. 
We are also grateful to the NGOs who helped us recruit study participants and shared their knowledge and support to guide the practical application of digital technology for MCH. 
Thank you all for your invaluable contribution. 


\begin{thebibliography}{00}
\bibitem{b1}Montgomery, A. (2014). Pregnancy-related deaths in India causes of death and the use of health services (Doctoral dissertation). 
\bibitem{b2} Barros, A. J., Victora, C. G., Cesar, J. A., Neumann, N. A., Bertoldi, A. D., Gwatkin, D., ..and Yazbeck, A. (2013). Brazil: are health and nutrition programs reaching the neediest. World Bank. 
\bibitem{b4}Ismail, A., Yadav, D., Gupta, M., Dabas, K., Singh, P., and Kumar, N. (2022). Imagining caring futures for frontline health work. Proceedings of the ACM on Human-Computer Interaction, 6(CSCW2), 1-30. 
\bibitem{b6} El Ayadi, A. M., Singh, P., Duggal, M., Kumar, V., Kaur, J., Sharma, P., ... and Diamond-Smith, N. G. (2023). Feasibility and acceptability of Saheli, a WhatsApp Chatbot, on COVID-19 vaccination among pregnant and breastfeeding women in rural North India. BMJ Innovations, 9(4). 
\bibitem{b7} Van Hauwaert, R., Mateus, A. R., Coutinho, A. L., Rodrigues, J., Martins, A. R., Vilela, F., \& Almeida, D. (2024). The role of digital health technologies on maternal health literacy: a narrative review. Emerging Technologies for Health Literacy and Medical Practice, 47-65.

\bibitem{b9} Collins, T. E., Akselrod, S., Altymysheva, A., Nga, P. T. Q., Banatvala, N., \& Berlina, D. (2023). The promise of digital health technologies for integrated care for maternal and child health and non-communicable diseases. bmj, 381.
\bibitem{b10} Goto, R., Watanabe, Y., Yamazaki, A., Sugita, M., Takeda, S., Nakabayashi, M., \& Nakamura, Y. (2021). Can digital health technologies exacerbate the health gap? A clustering analysis of mothers’ opinions toward digitizing the maternal and child health handbook. SSM-Population Health, 16, 100935.
\bibitem{b11} Whitehead, L., Robinson, S., Arabiat, D., Jenkins, M., \& Morelius, E. (2024). The report of access and engagement with digital health interventions among children and young people: systematic review. JMIR pediatrics and parenting, 7, e44199.

\bibitem{b14}Moulaei, K., Moulaei, R., \& Bahaadinbeigy, K. (2023). Barriers and facilitators of using health information technologies by women: a scoping review. BMC medical informatics and decision making, 23(1), 176.

\bibitem{b17}Gold, N., Hu, X. Y., Denford, S., Xia, R. Y., Towler, L., Groot, J., ... \& Yardley, L. (2021). Effectiveness of digital interventions to improve household and community infection prevention and control behaviours and to reduce incidence of respiratory and/or gastro-intestinal infections: a rapid systematic review. BMC Public Health, 21, 1-15.
\bibitem{b18}Rahman, M. O., Yamaji, N., Sasayama, K., Yoneoka, D., \& Ota, E. (2023). Technology‐based innovative healthcare solutions for improving maternal and child health outcomes in low‐and middle‐income countries: A network meta‐analysis protocol. Nursing Open, 10(1), 367-376.
\bibitem{b19}Aranda-Jan, C.B., Mohutsiwa-Dibe, N. \& Loukanova, S. Systematic review on what works, what does not work and why of implementation of mobile health (mHealth) projects in Africa. BMC Public Health 14, 188 (2014). 
\bibitem{20}Scott Kruse, C., Karem, P., Shifflett, K., Vegi, L., Ravi, K., \& Brooks, M. (2018). Evaluating barriers to adopting telemedicine worldwide: a systematic review. Journal of telemedicine and telecare, 24(1), 4-12.
\bibitem{21}Winters N, Venkatapuram S, Geniets A, Wynne-Bannister E. Prioritarian principles for digital health in low resource settings. J Med Ethics. 2020 Apr;46(4):259-264. doi: 10.1136/medethics-2019-105468. Epub 2020 Jan 16. PMID: 31949027; PMCID: PMC7231431.
\bibitem{22}Yadav, Deepika, et al. "Feedpal: Understanding opportunities for chatbots in breastfeeding education of women in india." Proceedings of the ACM on Human-Computer Interaction 3.CSCW (2019): 1-30.
\bibitem{b23}Fan, M., Shi, S., \& Truong, K. N. (2020). Practices and Challenges of Using Think-Aloud Protocols in Industry: An International Survey. Journal of Usability Studies, 15(2).
\bibitem{b24}Capasso, A., Colomar, M., Ramírez, D., Serruya, S., \& de Mucio, B. (2024). Digital health and the promise of equity in maternity care: A mixed methods multi-country assessment on the use of information and communication technologies in healthcare facilities in Latin America and the Caribbean. Plos one, 19(2), e0298902
\bibitem{b25}Knop, M. R., Nagashima-Hayashi, M., Lin, R., Saing, C. H., Ung, M., Oy, S., ... \& Yi, S. (2024). Impact of mHealth interventions on maternal, newborn, and child health from conception to 24 months postpartum in low-and middle-income countries: a systematic review. BMC medicine, 22(1), 196.



\end{thebibliography}

\vspace{12pt}

\end{document}
