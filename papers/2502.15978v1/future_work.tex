\section{Discussion}

Our findings reveal critical barriers to the adoption of chatbots among women in underserved communities, which structure their digital interactions. 
Below we reflect on the implications of these barriers in chatbot design and implementation efforts.

Participants shared that their phone engagement is limited to short time intervals, often when their children are sleeping. Another gap is in accessible health information, particularly on child nutrition. Mothers expressed a need for advice on feeding practices for their children. A reliance on traditional sources of health advice, such as family elders, sometimes leads to misinformation or missed opportunities for early intervention. For example, a participant mentioned not recognizing early signs of developmental delays in her child due to this reliance. This calls for more culturally relevant, personalized health information that accounts for the specific needs of the child and family context.

Cultural and gender norms additionally amplify these challenges. Family dynamics play a role in decision-making around health and technology use, particularly in communities where women often rely on elders or husbands. These norms not only influence decisions about family planning and healthcare but also limit women's ability to use technology independently, reflecting deeply rooted power dynamics. Technology solutions must take these social structures into account to promote acceptance and use.


Digital literacy also appeared as a limiting factor. Some participants struggled with typing and reading in English and Hindi, which reduced their confidence in navigating these digital health platforms. Many prefer audio or visual content over text-based communication, as it allows them to access information without reading or typing. For chatbots to be effective in such contexts, features like text-to-speech, voice commands, or multimedia content should be prioritized to ensure inclusivity.


The participants in this study significantly trusted the NGO programs they were involved in, particularly when receiving health-related advice. These programs are well-received due to their personalized and accessible weekly messages. This trust was built over time through regular communication and support from local community health workers. Collaborating with such trusted organizations and integrating feedback from participants will be essential for the success during deployment of such technologies.
These barriers and mothers' interest in accessing health information for their children suggest that digital health interventions, such as chatbots, must offer asynchronous, concise, and personalized interactions and accessible solutions that fit into the everyday lives and priorities of women, who have many responsibilities and limited time.

Finally, we wish to highlight that many of these challenges are not unique to chatbots, and have been uncovered in past literature on digital health. But it is worth examining why these foundational issues still persist. Our work speaks to deeper issues around women's agency and their care burden, which has seen limited progress even as digital access has improved, and global health organizations push for digital health solutions. There is a need to address women's control over their own finances to be able to improve digital access, and a society-wide cultural change in seeing their role as more than mere care providers, to have any long-lasting change and improvements in community health outcomes. Otherwise, even as new technologies are introduced, global health and technology practitioners and researchers will continue to experience the same challenges that the field has experienced thus far.

%Our study aimed to explore the potential of a whatsapp based chatbot for maternal and child health among underserved communities in the urban slums of Mumbai. When we began interviewing these participants, we started observing the barriers that we never expected before starting the study. The barriers included constraints on time that many of the participants faced. Women were occupied with childcare and household responsibilities, leaving them with little time to explore new technologies. This highlights the importance of designing interventions that take into account the time limitations and multitasking responsibilities on women in underserved communities. For example, chatbots that offer quick, simple interactions or provide information in audio format could help address this issue. Another key finding was the significant gaps in knowledge about child nutrition and health. Many participants expressed a strong interest in receiving more information about what to feed their children and how to address common health concerns.


%Our study also revealed the impact of socio-cultural norms on women’s ability to access and use technology. In many cases, family dynamics and gender roles limited women's autonomy in health-related decision-making and technology use. These findings suggest that chatbot-based interventions must not only provide health information but also consider the cultural and family dynamics that affect women's health choices. We should focus on programs that involve both women and their family members, they may be more effective in promoting the adoption of health technologies.
%The participants in this study significantly trusted the NGO programs they were involved in, particularly when receiving health-related advice. This trust was built over time through regular communication and support from community health workers. Collaborating with such trusted organizations and integrating feedback from participants will be essential for ensuring the success of such technologies.

%Future research should explore how to tailor these technologies to fit within the socio-cultural context of such communities, taking into account the barriers such as limited time, digital literacy, and family dynamics. Moreover, we should also look at the effectiveness of audio and visual content in delivering health information, as it could provide valuable insights into how best to engage women with varying literacy levels.

%Therefore, it becomes imperative to enhance the design of future studies targeting these demographics, ensuring that methodologies are adapted to better accommodate their needs and preferences. This might involve incorporating more interactive or culturally sensitive approaches to engagement and providing adequate support and resources to overcome any barriers to participation. By addressing these, future studies can improve engagement with target communities and gather richer data that reflects their unique experiences and needs.
