\section{Related Work}
Digital health technologies, including chatbots, have emerged as powerful tools with the potential to improve maternal health literacy, access to care, and overall health outcomes ____, especially in developing countries ____. Researchers on digital health solutions have leveraged widely accessible technologies ____, including WhatsApp, for MCH information delivery ____. For instance, Yadav et al. studied ____ user interactions to deliver information to breastfeeding mothers of low-resource communities through a prototyped WhatsApp-based chatbot in India.
These innovations aim to improve access to health information and services, sometimes even offering integrated care, thus aiming to reduce health inequalities in underserved populations ____. 
Despite many digital health efforts, however, there have been few sustained efforts and true successes. Evidence of the effectiveness of digital health for maternal and child health has been scarce ____.

%However, socio-economic challenges remain in ensuring that these digital solutions reach the most vulnerable populations.
While smartphone access has increased significantly in underserved communities ____, this has not translated into widespread adoption of digital health solutions ____.
Many digital health initiatives have not sufficiently addressed access and engagement strategies, which are vital for their success to the populations most in need ____. 
Studies on digital health also show that despite access to mobile phones, barriers such as technology-specific barriers ____, socio-cultural factors ____, digital literacy, and trust in technology ____ limit the effective use of healthcare technologies. 
Reaching the most vulnerable populations also entails careful consideration of contextual factors such as family dynamics, traditional health practices, and beliefs of the community ____. 
Even assuming that a digital health intervention has been designed well, its adoption at the last mile may depend on an individual's income level, education, and access to digital devices and the internet ____.  
These disparities highlight the need for a more inclusive approach to technology development and deployment, ensuring that all socioeconomic groups can benefit equally. 
%\subsection{Technology Access and Adoption challenges}
Our paper extends this body of work to consider implications for the design of chatbots for maternal and child health in underserved communities in India, even as we reflect on why these challenges still persist after almost two decades of research on digital health in the Global South. 
 
% Below we present literature on Technology-based interventions for MCH and adoption challenges in such settings.
%\subsection{MCH Outcomes in Underserved Communities}
%MCH continues to face significant disparities in underserved communities globally. Women in these areas experience higher rates of maternal mortality, infant mortality, and complications during pregnancy and childbirth ____, largely due to a lack of resources on healthcare services and information ____, and socio-cultural variables ____. Marginalized women often struggle to get accurate MCH information, which is important for making informed decisions about pregnancy care, nutrition, and child healthcare ____. Community healthcare workers' efforts to promote healthcare practices often do not effectively reach all women because of social ____ and logistical challenges ____, emphasizing the importance of scalable solutions to improve access to MCH information in marginalized areas. 

%\subsection{Technology-based Interventions of MCH and Adoption Challenges}