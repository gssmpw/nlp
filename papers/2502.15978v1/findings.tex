\section{Findings}
We now present the learnings from our analysis of interviews and focus group discussions (FGD).  Our findings identify the challenges limiting the MCH technology adoption among marginalized women. In several places, we have included quotes from participants in translated English text for the benefit of our readers. All names mentioned in the paper are pseudonyms.
%Here we are presenting the findings based on data collected from the focus groups (FG) and individual interviews. The quotes from the participants presented in the findings are translated into English for the readers. There were some findings regarding the methodology we adopted. We divided our findings into different thematic categories. All the names mentioned are pseudonyms. The demographics of these participants are mentioned in the table below. We can also see the preference for communication in Fig.2.

\subsection{Limited access to technology}
We first summarize the patterns observed in barriers to phone usage among the participants. We then reflect on the phone sharing and access limitations that could present.

\subsubsection{Barrier to phone usage} 
Our interactions with mothers of children under 3 years of age highlighted significant time constraints. For instance, one of our participants shared her experience in the FGD1 discussion, ``\textit{I don’t get much time to use my phone due to household chores. I am busy most of the time, just 10 –15 minutes a day, that too when the child is sleeping''}. Another added, \textit{ ``I reply to a WhatsApp message if I get any for 5 -10min, other than that I don’t have the time to use my phone''}. These challenges significantly impact their ability to engage with smartphone applications, often limiting their usage to brief moments only when their children are napping or otherwise occupied. Even during the interviews, we observed that participants struggled to spend time due to their children needing attention. On the contrary, focus group participants were fully engaged in discussions without interruptions, likely because these conversations took place in a communal space, which may have created a more supporting setting. 
Given the limited time users can dedicate, the ability of the chatbots to provide asynchronous interactions can be useful. Our findings further indicated that the responses should also be precise and clear to facilitate prompt conversations. 

\begin{figure}[htp]
\centerline{\includegraphics[width=0.2\textwidth, height=0.3\textheight]{whatsapp2.png}}
\caption{Sample Whatsapp conversation, English translated.}
\label{fig:whats2}
\end{figure}
%All of our participants were mothers of children up to 3 years of age. Most of the participants mentioned they are very busy with their daily household activities due to which they are not able to spend much time on their smartphones. Time constraints appear to be one of the challenges that was noticed during our interaction with participants. We faced difficulty in interviewing the participants as most of them said we wouldn't be able to give you much time as we have to take care of our children and chores. 

%FG1:``\textit{I don’t get much time to use my phone due to household chores. I am busy most of the time, just 10 –15 minutes a day, that too when the child is sleeping. ’’}


%FG2:\textit{ ``I reply to a WhatsApp message if I get any for 5 -10min, other than that I don’t have the time to use my phone.’’}


%Even during the interview, some participants were not able to sit for too long as their kids kept calling them to get ready for school and running around the house. On the contrary, Focus group participants were invested in the discussions without any interruptions, the reason being discussions were conducted in the communal space.

\subsubsection{Phone sharing and access limitation}
We observed that participants faced challenges using technology, especially mothers like Poonam (28yr, a mother of 3 children), who relied on their husbands' phones and could only access phones during the evening. Chatbots may be more beneficial for users like Poonam if we include the men in the interventions. Taking privacy into account, chatbots can be designed with features like personalized to-do lists and the ability to set reminders at times that best suit the user, making it easier for them to use shared devices.


%Chatbots may be more useful for users like Poonam if we include the men in our interventions and considering the privacy implications, chatbots may be designed with features like personalized to-do list, setting reminders at times that work best for the user, making it easier for them to utilize the chatbots as they share the devices.


%Chatbots can be more effective for users like Poonam by adding task prioritization features. Personalization options, such as creating to-do lists, setting reminders, and sending notificationsimes that work best for the user, make it easier for them to use and benefit from the technology while better managing their other responsibilities.


%Some participants mentioned they have an older phone and they rely on their husband's phones for accessing information.
%\textit \textit{Poonam (28yr, a mother of 3 children) mentioned: {``I have a smartphone, but it doesn’t work much, my kids broke my phone, so we didn’t buy another phone. I don’t have WhatsApp, my husband has. I use my phone only for calls and my kids use it to watch videos on YouTube.’’}}
%During these interactions with the participants, we noticed that most of these participants can access their smartphones only after their husband returns home from work in the evening, so they have a small window of time between evening and early morning which makes it even more challenging for them to understand and adopt to technology. It is an important finding to be considered while designing LLMs or chatbots, as most of the participants with children have similar schedules, where they are busy taking care of the family and house chores during the day time. This makes it difficult for them to spend time learning about new technology. 
 
\subsection{Information gaps in childcare and health} 
We now share the observations on challenges in accessing information on childcare and health. We reflect on the implications of these gaps, especially due to the lack of accessible health information and the reliance on traditional sources, which may limit the participant's ability to make informed decisions.

\subsubsection{Lack of accessible health information} 
Our observations highlighted a gap in knowledge regarding child nutrition among mothers, with many indicating a need for guidance on appropriate dietary practices, particularly in special circumstances. One of our participants from FGD2 stated that: \textit{``My child goes to this daycare program in an NGO for disabled kids to learn new things, but it would be nice to know what should I feed him, what is good for his health?''}, which emphasizes the need for addressing nutritional gaps by providing accessible and tailored dietary information, empowering mothers, and allowing them to make informed choices that can improve their child's health outcome. 
Our findings indicate the need for chatbots to avoid generic nutritional information, as mothers might require personalized guidance on their children's needs. Chatbots should further consider collecting user specifics like age, dietary restrictions, and health concerns to provide tailored meal suggestions. Culturally specific information based on local food options makes the recommendations more practical, while partnerships with programs like NGO1 can connect mothers to helpful resources.

%While interviewing these mothers we discussed their healthcare practices and asked them about the challenges that they come across in their day-to-day lives, most of these mothers expressed that proper diet and nutrition is one of the challenges for them. 


%\textit{ FG2: ``My child goes to this daycare program in an NGO for disabled kids to learn new things, but it would be nice to know what should I feed him, what is good for his health?  ’’}


%\textit{FG1:``If we sit to eat and drink, the children themselves come for food, we feed them a little food and give them fruits. The child who feels like eating eats, and if they don't, they skip the meal ’’}.


%\textit{ Sweety (26 years old, a mother of 2-year-old girl)``Yes I would like to get information on what to feed the child, her height is less as per her age, would like that to be included. ’’}
%A common lack of knowledge in child nutrition was observed. Most of the participants expressed an interest in knowing more about child growth and nutrition as most of these mothers mentioned their child being underweight and were a part of the NG'SS child nutrition program. They saw improvement in the weight of their children after receiving diet and nutrition-related information, which included a general understanding of the importance of a balanced diet for children, including protein intake and age-appropriate food textures. This suggests an opportunity to provide culturally appropriate dietary guidance. By tailoring advice to the participants’ backgrounds and preferences, a chatbot or other intervention could empower them to make informed choices that promote their children’s health and well-being. 

\subsubsection{Reliance on traditional sources}
Our analysis showed that while many mothers had access to smartphones and technology, they often relied on family or elders for healthcare advice rather than seeking information online. For instance, one of the mothers from FGD2 (mother of 2 young children) shared,\textit{`` I used to call my mother if I had any questions during my pregnancy or breastfeeding. It's good to have prior knowledge about child growth. When I had my 1st child, I was not aware of the child’s growth milestones. My first child did not speak until he was 3 years old, I was not aware that it was not normal, everybody around me said don’t worry he will speak. I did not know that a child should start saying words after 6 months. I had no idea that I should go to a doctor when a child does not start speaking by the age of 1 year''}. This implies the disparity between having access to technology and utilizing it for reliable health information. Chatbots can help address this gap and reduce dependence on potentially inaccurate information provided by the family. 

%Living in a community, it is very common for people to rely on each other and share information. We saw the participants living in joint families rely on their elders and social circle for healthcare advice, indicating a gap in direct access to reliable health information.
%\textit{
%\textit{ FG2, (mother of 2 young children)- `` I used to call my mother if I had any questions during my pregnancy or breastfeeding. It's good to have prior knowledge about child growth. When I had my 1st child, I was not aware of the child’s growth milestones. My first child did not speak until he was 3 years old, I was not aware that it was not normal, everybody around me said don’t worry he will speak. I did not know that a child should start saying words after 6 months. I had no idea that I should go to a doctor when a child does not start speaking by the age of 1 year. ’’}} 

%She is a graduate and she frequently uses her phone for online shopping and WhatsApp for communication. Such instances make us think deeper into the possible solutions for these knowledge gaps, where these women with smartphones can access the information. Looking at previous studies done with breastfeeding mothers in Delhi, chatbots seem to be a good solution in providing information reducing these knowledge gaps, and improving Maternal and child health outcomes.

\subsection{Impact of social-cultural norms}
We now summarize observations on the challenges limiting technology use influenced by social and cultural norms such as gender roles and family dynamics. 

\subsubsection{Influence of family dynamics} 

Many of our participants noted that their decisions are heavily influenced by in-laws or husbands. In her interview, Seema (26 yrs, a mother of a 3-year-old boy) shared that: \textit{``We seek advice from the elders at home first and then go to the doctor if needed. The family members do have a say in decisions regarding health, when living with in-laws we have to get permission and tell them where and when we are going, and why are we going out''}. 
In analyzing the impact of family dynamics on health and technology decisions, it became evident that cultural beliefs do play a significant role in shaping women's choices. 

%In many households, family dynamics play a great role in decisions related to health and technology. Several participants mentioned that they rely on their elders and decisions regarding health or family planning are influenced by their in-laws or husbands.


%\textit{FG2: ``My first child had difficulty in hearing, so he had to go through surgeries and treatment in Bandra (a little far from where we live). I did not want another child, I was busy taking care of my 1st child as he requires extra care due to his hearing problem, but my in-laws wanted us to have another child, so we have had another child.’’ }

%\textit{Seema (26 years, a mother of a 3-year-old boy)- ``We seek advice from the elders at home first and then go to the doctor if needed. The family members do have a say in decisions regarding health, when living with in-laws we have to get permission and tell them where and when we are going, and why are we going out. ’’}
%This highlights how deeply rooted cultural beliefs about family authority restrict women in these communities from exploring other options available to them. 


\subsubsection{Gender role limiting tech usage} 
The majority of participants in our study identified as stay-at-home mothers, whose primary responsibilities were taking care of the children and household management. This adherence to conventional gender roles not only restricts their time and chances to interact with technology but also limits their overall digital proficiency. Our data further reflects that the gendered division restricts women's access to technology and strengthens social hierarchies within the household, resulting in greater dependence on husbands or family members for decisions about technology.

%Most of the participants were stay-at-home mothers, they were responsible for taking care of the children and household chores. These traditional gender roles further inhibit women from using phones. As mentioned earlier most of the participants described that they are busy with their day-to-day activities at home due to which they have little to no time to explore digital platforms.  
%When asked about the type of information they are interested in to know more about. Most women responded that they would like to gain more information on child health and nutrition.
%This reflects broad patterns in which women's roles as caregivers outweigh their engagement with technology, reinforcing these gender norms. These responsibilities to prioritize the needs of the family over their own further limit their time for self-care and digital engagement.
%This gendered division of responsibilities restricts women's technology use and also reinforces the traditional power dynamics within the family. With most of their time being consumed by taking care of everyone in the family, they tend to become more reliant on their husbands or other family members for decisions like technology access.

{\subsection{Digital Literacy}}
Digital literacy also influenced participants' preferences for content, and frequently intersected with challenges related to general literacy. 
%For instance, Nisha and Seema struggled with reading and writing. Although they had access to smartphones and were familiar with applications such as WhatsApp, their difficulty in navigating text-based content posed significant barriers to accessing health information in text form. As illustrated in Figure 3, the majority of participants expressed a strong preference for audio-based interactions. This preference highlights the potential of audio-enabled functionalities to address barriers for users facing reading and writing challenges. Furthermore, such features can enhance usability for users managing other tasks while interacting with the chatbot.
% \gtrm{\subsubsection{Challenges in reading and writing}}
For instance, Nisha and Seema struggled with reading and writing Hindi or English. 
Their experience %while using the chatbot
highlighted significant barriers to accessing health information through chatbots that primarily involve text in these languages. Nisha (24 years mother of 2 young children), who lived with her in-laws in a family of 10, mentioned that she has her smartphone with WhatsApp but has difficulty reading in Hindi or English, so she relied on her niece and husband. Their reliance on family for help not only delayed their access to information but also undermined their confidence in using technology on their own.

As shown in Figure 3, the majority of participants
expressed a strong preference for audio-based interactions.
This preference highlights the potential of audio-enabled functionalities to address barriers for users facing reading and
writing challenges. Furthermore, such features can enhance
usability for users managing other tasks while interacting with
the chatbot.
%As shown in Figure \ref{fig:fig}, the majority of participants expressed a strong preference for audio as their preferred mode of interaction. This inference indicates that the hands-free engagement functionality of the chatbot can be effective for not only the users managing other tasks while interacting but also those who may have issues with reading and writing.




%Some participants reported difficulty in reading and writing Hindi or English text, which is the primary language in most digital interfaces.
%When Nisha (24 years old mother of 2 young children) who lives with her in-laws in a big family of 10 members mentioned she has her own smartphone with WhatsApp but she has difficulty reading in Hindi or English so relies on her niece and husband for reading.
%Typing was another challenge frequently mentioned by these participants.
%\textit{ Seema (26 years old, mother of a 3year old):`` I recently got a new smartphone.I don't use my phone much, I can’t type as the font is in English, but I can read Hindi. I use WhatsApp and YouTube for video calls and to watch videos.’’}
%These issues further limit their engagement with health-related information platforms that rely heavily on text-based communication. The dependency on the family members for reading or typing not only delays their access but also reduces their confidence in using the technology independently.



%\subsubsection{Preference for audio content}



%Given the challenges mentioned above many participants preferred to have audio or visual content over text-based content. Participants reported that WhatsApp voice messages, videos, and weekly phone calls from the NGO were more easier for them to engage with. It avoids the need to rely on their family members for engagement. The combination of audio and visual explanations makes it easier for them to understand what might be difficult to understand through text. It also saves their time as they can listen or watch a video while working as compared to text conversations, where they will have to sit and read.


%These findings underscore the importance of designing chatbots with accessibility features in mind, such as text-to-speech or voice commands, to ensure inclusivity for users with varying technical skill levels and language preferences. This approach can help cater to the diverse needs of users, promoting more effective and widespread use of health information resources. 


%\begin{figure}
 %\centering
 %\includegraphics[width=\linewidth]{sections/modeofinteraction.png}
 % \label{tab:graph}
  %\caption{\textbf{Interaction Preferences of the participants.}}
% \label{fig:mode}
%\end{figure}


\begin{figure}[htp]
\centerline{\includegraphics[width=0.4\textwidth]{modeofinteraction.png}}
\caption{Interaction Preferences of the participants.}
\label{fig:fig}
\end{figure}

\subsection{Trust in external programs and resources}

Poonam (28 years old, a mother 3 children) noted that: \textit{``I do not like reading much, I like the weekly calls from the NGO for my son, as I do not get much time to use my phone, I am busy taking care of kids and housework''}. %, Azeem (27 years old, a mother of 2 young children) further added \textit{``I do not like reading much, I like the weekly calls from the NGO for my son, as I do not get much time to use my phone, I am busy taking care of kids and housework.’’}$. 
This indicated a strong opportunity to integrate chatbots into health organizations that support maternal and child health. As these programs were specifically designed for malnourished children, they were highly valued by the participants because of their easy accessibility and tailored messages. Other than the reliance on these programs and family members for health care advice, all participants mentioned clear reliance on doctors in case of serious medical issues. \\
All participants reported regular doctor visits during pregnancy, demonstrating a strong reliance on medical professionals for prenatal care. In response to the questions about iron and calcium supplements during pregnancy and family planning measures, 50\% of the participants responded ``YES'' to family planning measures (e.g., condoms and IUDs like copper-T), and the majority of them responded with a ``Yes'' to iron and calcium supplements intake during pregnancy following doctor's advice. These responses highlight a good understanding of the importance of prenatal checkups. When it comes to the health of the children, many mothers rely on Anganwadi workers for immunization and child nutrition education. Their utilization of contraception methods, such as condoms and IUDs like copper-T, demonstrates a proactive approach to managing their reproductive health. 
However, we saw little hesitation and shyness among these women while responding to the questions about family planning as they leaned forward and whispered about their contraceptive methods. Their responses indicate a willingness to engage in conversations about these important issues. Overall, these findings suggest a receptive community towards preventative healthcare and responsible family planning practices. Their trust is placed in these NGOs and healthcare providers in community-based relationships with the community workers who reach out to them from time to time to make them feel heard and bring their feedback to help improve these programs.

We also noticed a difference in the engagement between the participants from the two NGOs. The participants who were recruited through NGO1 and interviewed in person in the comfort of their homes were less engaged as compared to the participants in the focus groups. The focus groups took place at NGO2's center where they bring their disabled kids for various learning activities to improve mobility. These participants were more enthusiastic about the chatbots and new information. In contrast, the participants at home were hesitant to speak and were distracted by their kids and family members. High interest and curiosity among the women in the focus groups may be due to the health condition of their kids, and because they felt safe in that space to discuss their health choices and were not distracted by other responsibilities at home.
Another factor that could have affected their level of engagement could be the frequency and in-person contact with the NGO. Participants in the FGDs had regular, daily contact with NGO2's staff, which likely fostered a stronger rapport and engagement during these discussions. These participants were particularly interested in topics related to their children's diet and developmental milestones. On the other hand, the participants who were connected with only weekly calls and messages had limited face-to-face interaction with NGO1's workers, which may have influenced their engagement during the interviews. This highlights the importance of regular one-on-one interaction for building trust and engagement in health interventions.

Despite widespread smartphone and WhatsApp use among participants, familiarity with chatbots was low. However, there was a positive sentiment towards using chatbots for health information, suggesting a potential to improve health access when the system is designed to cater to audiences who struggle to adopt them. 

%As mentioned in the methods section above the participants were enrolled through NGOs. These participants were part of the NGO's Child nutrition program. They were receiving weekly calls regarding their child's growth and nutrition.


%\textit{Poonam (28 years old, a mother 3 children): ``I do not like reading much, I like the weekly calls from the NGO for my son, as I do not get much time to use my phone, I am busy taking care of kids and housework.’’}


%\textit{Azeem (27 years old, a mother of 2 young children): `` I like the information they provide, weight of my child has improved after following their advice.’’}


%As these programs were specifically designed for malnourished children, they were highly valued by the participants because of their easy accessibility and tailored messages. Other than the reliance on these programs and family members for health care advice, all participants mentioned clear reliance on doctors in case of serious medical issues. 


%All participants reported regular doctor visits during pregnancy, demonstrating a strong reliance on medical professionals for prenatal care. In response to the questions about Iron and calcium supplements during pregnancy and family planning measures, 50\% of the participants responded “YES” to family planning measures (like Condoms and Cu-T) and the majority of them responded with a “Yes” to Iron and calcium supplements intake during pregnancy following doctor's advice. These responses highlight a good understanding of the importance of prenatal checkups. When it comes to the health of the children, many mothers rely on Anganwadi workers for immunization and child nutrition education. Their utilization of contraception methods, such as condoms and Cu-T, demonstrates a proactive approach to managing their reproductive health. 


%However, we saw little hesitation and shyness among these women while responding to the questions about family planning as they leaned forward and whispered about their contraceptive methods. Their responses indicate a willingness to engage in conversations about these important issues. Overall, these findings suggest a receptive community towards preventative healthcare and responsible family planning practices. Their trust is placed in these NGOs and healthcare providers in community-based relationships with the community workers who reach out to them from time to time to make them feel heard and bring their feedback to help improve these programs.


%We also noticed a difference in the engagement between the participants from the two NGOs. The participants who were interviewed in person in the comfort of their homes were less engaging as compared to the participants in the focus groups. The focus groups took place at the center where they bring their disabled kids for various learning activities. These participants were more enthusiastic about the chatbots and new information. The participants at home were hesitant to speak and were distracted by their kids and family members. High interest and curiosity among the other group may be due to the health condition of their kids or they feel safe in that space to discuss their health choices and they are not distracted by other things in the NGO. These participants were particularly interested in topics related to their children's diet and developmental milestones. 


%Despite widespread smartphone and WhatsApp use among participants, familiarity with chatbots was low. However, there was a positive sentiment towards using chatbots for health information, suggesting a potential to improve health access. This opportunity hinges on designing user-friendly interfaces that cater to those with limited digital literacy.









 

 


%This highlights a common lack of knowledge regarding diet and nutrition for children. Most of the mothers mentioned their child being underweight and were a part of ARMMAN’S child nutrition program. Many mothers expressed interest in having a tool that provides information about physical and mental developmental milestones.
 

%\textit{Technology Adoption: } 

%Despite widespread smartphone and WhatsApp use among participants, familiarity with chatbots was low. However, there was a positive sentiment towards using chatbots for health information, suggesting a potential to improve health access. This opportunity hinges on designing user-friendly interfaces that cater to those with limited digital literacy. The study identified limitations in this area, as some participants relied on family members for help with WhatsApp and reading text messages.  




 %Perceptions regarding the reliability of online health advice versus traditional consultations with doctors also vary, indicating a blend of modern and traditional approaches to health information seeking. While doctor consultations are preferred for serious health issues, there is a growing openness to receiving health advice online, especially if it complements existing knowledge.  

%Participants were particularly interested in topics related to their children's diet and developmental milestones.
%These findings underscore the importance of designing chatbots with accessibility features in mind, such as text-to-speech or voice commands, to ensure inclusivity for users with varying technical skill levels and language preferences. This approach can help cater to the diverse needs of users, promoting more effective and widespread use of health information resources. 

%The analysis revealed significant reliance on family, especially elders and mothers-in-law, for childcare and health-related advice. Participants said that this support is crucial in the early stages of parenting and for managing health care practices. Findings also show that participants are open to using technology for health information but also value the advice and support of family and healthcare professionals. Also, there's a clear indication that while modern technology is accepted, family dynamics also play a significant role in health-related decision-making. These findings suggest that a dual approach should be considered when designing health communication strategies such as engaging family members in educational programs could enhance the effectiveness of health interventions.   
