
\section{Introduction}

Maternal and Child Health (MCH) remains a critical concern in India, as it contributes to one-fifth of the global burden of maternal deaths \cite{b1}. Despite multiple interventions, significant gaps persist in delivering essential healthcare services and information to pregnant women and mothers regarding child care, nutrition, and immunization \cite{b2}. 
%Existing literature shows several challenges in improving MCH outcomes, including limited healthcare access, healthcare disparities, low health literacy, and cultural barriers \cite{b1}. These issues are particularly prominent among underserved communities, resulting in poor maternal and child health practices.
Recent studies suggest that digital technology, particularly chatbots, can help overcome these challenges by providing accessible and timely healthcare information \cite{b6,b7,22}. However, the adoption of digital technology for MCH remains uneven due to socioeconomic, cultural, and infrastructural barriers.

Our study focuses on understanding opportunities and challenges around the use of chatbots for MCH. 
We conducted 23 semi-structured interviews and two focus group discussions to understand how women in underserved communities in Mumbai (India) perceive chatbots and barriers to adoption. We examine the key concerns expressed by pregnant women and mothers, such as the complexities of healthcare access, digital literacies, cultural norms, and communication preferences. 
Based on our findings, we offer takeaways for the design of emerging health chatbots in global health that target women. In particular, we emphasize the need to consider intermittent digital access, support multiple modes of interaction, and consider women's everyday responsibilities and their limited agency in their households.

%A study by Arun et al. (2017) emphasizes the importance of education and awareness campaigns in improving health-seeking behaviors, underscoring the need for a multi-pronged approach that addresses both medical access and cultural factors\cite{}.


%Maternal and Child health (MCH) remains a critical concern in India, as India contributes to one fifth of the global burden of maternal deaths (Montgomery et al.). Despite several interventions there a has been a gap in the delivery of the essential health care services and information to pregnant women and mothers regarding prenatal care, nutrition and immunization \cite{Barros}. The existing literature shows several challenges to be faced in improving MCH that includes limited health care access, health care disparities, low health literacy and cultural factors (Montgomery et al.). All these factors contribute to inadequate maternal and child health practices, especially among underserved communities. This study addresses these challenges by leveraging technology to improve MCH outcomes in India. The chatbot design intends to bridge the gap in existing knowledge and health practices by providing accessible, culturally relevant and evidence-based information to pregnant women, mothers and caregivers of young children. While the chatbot was not developed during the study, we used a sample conversation on some commom questions on maternal and child health and interviews to understand WhatsApp usage with the participants. Underprivileged communities often have limited healthcare access, persistent social disparities, and inadequate health literacy, making it difficult for pregnant women to access essential prenatal care, skilled birth attendants, and emergency obstetric care [1, 2]. A study by Arun et al. (2017) suggests that increased education and awareness campaigns targeted towards women, can improve health-seeking behaviors [9]. This aligns with the need for a multi-pronged approach that tackles not only medical access but also social and cultural factors. Understanding these complexities is crucial for designing effective interventions to improve MCH outcomes.Past research has explored the use of chatbots as interventions in India, focusing on healthcare workers, mothers, and farmers. This study seeks to provide a thorough understanding of the target population's communication preferences, technology adoption, and behaviours related to mother and child healthcare by utilizing findings from both quantitative surveys and qualitative interviews. The results of this study could influence the development and application of novel interventions targeted at enhancing the health of mothers and children in India and other countries.

