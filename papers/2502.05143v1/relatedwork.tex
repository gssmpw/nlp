\section{Related Work}
Several prior works have addressed the challenge of test-to-code traceability, but many focus on languages other than Python or face limitations when applied to Python projects, especially those using Unittest or Pytest.

One such tool is Method2Test~\cite{tufano2022methods2test}, a Java-based solution that maps test cases to methods by leveraging Java’s structured naming conventions. While effective for Java projects using JUnit, its rigid structure contrasts with Python’s dynamic and more varied nature, making it less suitable for Python projects.

Another key example is TestRoutes~\cite{kicsi2020testroutes}, a manually curated dataset designed to map test cases to focal methods. However, TestRoutes mainly focuses on Java and faces scalability issues due to its manual curation, limiting its applicability for large-scale Python projects.

General-purpose static analysis tools, such as PyLint~\cite{pylint} and Bandit~\cite{bandit}, primarily focus on code quality and security rather than test-to-code traceability. While useful for identifying potential issues, they do not support mapping test cases to code.

Defects4J~\cite{defects4j} provides a set of reproducible bugs for Java, with tests to express the buggy behavior.  While this data is not directly mapping tests to focal methods, the buggy code the tests are mapped to could potentially lead to the focal method, especially if combined with an approach like SZZ~\cite{szz}.  This dataset however is substantially smaller than ours and focuses on Java while we focus on one of the most popular and fastest-growing languages Python.

While the goal is different, automated test generation approaches such as EvoSuite~\cite{evosuite} and A3Test~\cite{alagarsamy2024a3test} could be leveraged on a large dataset of projects to automatically create tests for a number of focal methods.  Such tests however would be generated and not hand-written by developers, meaning some downstream tasks, such as inferring code style, might not be able to utilize the data.