\section{Conclusion} 
In this study, we have comprehensively addressed the significant challenges that multi-entity question answering (MEQA) poses to large language models (LLMs) and retrieval-augmented generation (RAG) systems. The limitations of existing methods in handling cross-document aggregation, especially when dealing with entity - dense questions, have been clearly identified and analyzed. We introduced MEBench, a groundbreaking multi-document, multi-entity benchmark. By systematically categorizing 4,780 questions into three primary categories and eight distinct types, MEBench offers broad coverage of real-world multi-entity reasoning scenarios. This categorization is a crucial step in providing a structured and comprehensive evaluation framework for LLMs. Our experiments on state-of-the-art LLMs such as GPT-4 and Llama-3, along with RAG pipelines, have shed light on the critical limitations of these advanced models. The fact that even these leading models achieve only 59\% accuracy on MEBench underscores the magnitude of the challenges in MEQA. MEBench has effectively highlighted the systemic weaknesses in current LLM frameworks. These weaknesses serve as valuable insights for future research directions. For instance, the need for improved algorithms to retrieve and consolidate fragmented information from heterogeneous sources is evident. Additionally, there is a pressing need to develop more robust entity-aware QA architectures that can better handle the complexities of MEQA.