\begin{abstract}
% Multi-entity question answering (MEQA) represents significant challenges for large language models (LLM) and retrieval-augmented generation (RAG) systems, which frequently struggle to consolidate scattered information across diverse documents. While existing methods excel at single-document comprehension, they often struggle with cross-document aggregation, particularly when resolving entity-dense questions like “What is the distribution of ACM Fellows among various fields of study?”, which require integrating entity-centric insights from heterogeneous sources (\eg Wikipedia pages).
Multi-entity question answering (MEQA) demands the integration of scattered information across documents to resolve complex queries involving entities, relationships, and contextual dependencies.    While large language models (LLMs) and retrieval-augmented generation (RAG) systems show promise, their performance on cross-document MEQA remains underexplored due to the absence of tailored benchmarks.
To address this gap, we introduce MEBench, a novel multi-document, multi-entity benchmark designed to systematically evaluate LLMs’ capacity to retrieve, consolidate, and reason over scattered and dense information. Our benchmark comprises 4,780 questions which are systematically categorized into three primary categories: \textit{Comparative Reasoning, Statistical Reasoning} and \textit{Relational Reasoning}, further divided into eight distinct types, ensuring broad coverage of real-world multi-entity reasoning scenarios. Our experiments on state-of-the-art LLMs reveal critical limitations: even advanced models achieve only 59\% accuracy on MEBench. Our benchmark emphasizes the importance of completeness and factual precision of information extraction in MEQA tasks, using Entity-Attributed F1 (EA-F1) metric for granular evaluation of entity-level correctness and attribution validity. MEBench not only highlights systemic weaknesses in current LLM frameworks but also provides a foundation for advancing robust, entity-aware QA architectures.
\end{abstract}