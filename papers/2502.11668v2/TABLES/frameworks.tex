% \setlength{\tabcolsep}{2pt}
\renewcommand{\arraystretch}{1.1}
\begin{table*}[t]
    \small
    % \vspace{0mm}
    \caption{Python libraries for processing/modeling applications. We show if: they include differentiable (Diff.) implementations, neural networks (NN), DSP processors (Proc.) and controllers (Contr.), they allow to define signal chains and include analysis tools.}
    \label{tab:frameworks}
    \centering
    % \vspace{-3mm}
    \begin{tabular}{lcccccc} 
        \hline
        \hline
        Library
        & Diff.
        & NN
        & Proc.
        & Contr.
        & Chains
        & Analysis\\
        \hline
        \texttt{Pedalboard} & \ding{55} & \ding{55} & \ding{51} & \ding{55} & \ding{51} & \ding{55}\\
        \texttt{DDSP} & \ding{51} & \ding{55} & \ding{51} & \ding{55} & \ding{55} & \ding{55} \\
        \texttt{dasp} & \ding{51} & \ding{55} & \ding{51} & \ding{55} & \ding{55} & \ding{55}\\
        \texttt{diffmoog} & \ding{51} & \ding{55} & \ding{51} & \ding{55} & \ding{51} & \ding{55}\\
        \texttt{GRAFX} & \ding{51} & \ding{55} & \ding{51} & \ding{55} & \ding{51} & \ding{55}\\
        \texttt{pyneuralfx} & \ding{51} & \ding{51} & \ding{55} & \ding{55} & \ding{55} & \ding{51}\\
        \hline
        \texttt{NablAFx} & \ding{51} & \ding{51} & \ding{51} & \ding{51} & \ding{51} & \ding{51}\\
        \hline
        \hline
    \end{tabular}
    % \vspace{-4mm}
\end{table*}