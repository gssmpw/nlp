\begin{abstract}

Data augmentation is an essential technique in natural language processing (NLP) for enriching training datasets by generating diverse samples. 
This process is crucial for improving the robustness and generalization capabilities of NLP models. 
However, a significant challenge remains: \textit{Insufficient Attention to Sample Distribution Diversity}. 
Most existing methods focus on increasing the sample numbers while neglecting the sample distribution diversity, which can lead to model overfitting. 
In response, we explore data augmentation's impact on dataset diversity and propose a \textbf{\underline{D}}iversity-\textbf{\underline{o}}riented data \textbf{\underline{Aug}}mentation framework (\textbf{DoAug}). % \(\mathscr{DoAug}\)
Specifically, we utilize a diversity-oriented fine-tuning approach to train an LLM as a diverse paraphraser, which is capable of augmenting textual datasets by generating diversified paraphrases. 
Then, we apply the LLM paraphraser to a selected coreset of highly informative samples and integrate the paraphrases with the original data to create a more diverse augmented dataset. 
Finally, we conduct extensive experiments on 12 real-world textual datasets. 
The results show that our fine-tuned LLM augmenter improves diversity while preserving label consistency, thereby enhancing the robustness and performance of downstream tasks. 
Specifically, it achieves an average performance gain of \(10.52\%\), surpassing the runner-up baseline with more than three percentage points. 

\end{abstract}

