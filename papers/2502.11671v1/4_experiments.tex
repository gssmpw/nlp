
\vspace{-3mm}
\section{Experiments}

\vspace{-0.2cm}
\subsection{Experiment Settings}

\noindent\textbf{Evaluation Criterion.} 
We evaluate \textit{\textbf{diversity}} and \textit{\textbf{affinity}} for data distribution while measuring \textit{\textbf{performance}} on downstream tasks for effectiveness. 

\noindent \textit{\textbf{Diversity.}}
To comprehensively evaluate the effect of the proposed model \Methodname~on dataset diversity, we adopt several measures to evaluate  the augmented dataset's diversity: 

$\bullet$ \textit{Distance}: assesses the average distance between samples as follows:
    \({Distance(\mathcal{S}) = \frac{1}{|\mathcal{S}|} \sum_{x_i, x_j \in \mathcal{S}} \sqrt{(e_{x_i} - e_{x_j})^2}}\), 
    where \(e_x = \mathcal{E}(x)\) is the embedding of sample \(x\) in the embedding space \(\mathcal{E}\), and a larger distance indicates greater diversity.  

$\bullet$ \textit{Dispersion}~\cite{yu2022can}: is similar to cosine similarity but adjusted to make larger dispersion indicate greater diversity: 
        \(\mathit{Dispersion}(\mathcal{S}) = \frac{1}{|\mathcal{S}|} \sum_{x_i, x_j \in \mathcal{S}} 1 - \frac{e_{x_i} \cdot e_{x_j}}{\|e_{x_i}\|\|e_{x_j}\|}\). 

$\bullet$ \textit{Isocontour Radius}~\cite{lai2020diversity}: 
is the geometric mean of the radii, reflecting the spread of embeddings along each axis. 
    Assuming sample embeddings follow a multivariate Gaussian distribution, the dataset can be taken as an ellipsoid-shaped cluster, formulated as: \(\sum_{j=1}^{H} \frac{(e_j - \mu_j)^2}{\sigma_j^2} = c^2\),
    where \(\mu_j\) is the embeddings' mean along the \(j\)-th axis, and \(\sigma_j^2\) is the variance of the \(j\)-th axis. 
    Geometrically, the standard deviation \(\sigma_j\), is the radius \(r_j\) of the ellipsoid along the \(j\)-th axis. Thus, we have: 
        \(\mathit{Radius}(\mathcal{S}) = (\prod_{i=1}^{H} \sigma_i)^{1/H}\). 

$\bullet$ \textit{Homogeneity}~\cite{lai2020diversity}: is a metric that reflects the uniformity of a cluster distribution, suggesting that distinct samples in a diverse dataset should ideally cover the embedding space uniformly. 
    It begins by constructing a Markov chain model on the dataset embeddings. 
    The edge weight between sample \(i\) and \(j\) is defined as \(weight(i,j) = \left(\sqrt{(e_i - e_j) \cdot (e_i - e_j)}\right)^{\log{H}}\), and the transition probability from \(i\) to \(j\) is \(p(i \rightarrow j) = \frac{weight(i,j)}{\sum_{k} weight(i,k)}.\) 
    The entropy of the Markov chain is calculated by \(entropy(\mathcal{S}) = - \sum_{ij \in \mathcal{S}} v_i \cdot p(i \rightarrow j) \log{p(i \rightarrow j)}\), where \(v_i\) is the stationary distribution, assumed to be uniform. Homogeneity is then defined as, 
        \(\mathit{Homogeneity}(\mathcal{S}) = \frac{entropy(\mathcal{S})}{\log{(|\mathcal{S}|-1)}}\), 
    where \(\log{(|\mathcal{S}|-1)}\) is the entropy upper bound normalizes homogeneity into \([0,1]\)~\cite{lai2020diversity}. 
    
$\bullet$ \textit{Vocabulary Size}: evaluates dataset diversity at the token level, complementing four embedding-level diversity metrics. Given the token set of the textual dataset \(\mathcal{T}\), we count the number of unique tokens present: 
        \(\mathit{Vocabulary}(\mathcal{S}) = |\mathcal{T}|\). 



\noindent\textit{\textbf{Affinity.}} The affinity score reflects the coherence of an augmented dataset and is defined as the reciprocal of the average deviation of class centers from the original dataset: 
$
    \mathit{Affinity}(\Tilde{S}, S) = \left(\frac{1}{|\mathcal{C}|} \sum_{c_i \in \mathcal{C}} \sqrt{(\Tilde{\mu}_{c_i} - \mu_{c_i})^2}\right)^{-1},
$
where \(\mathcal{C}={c_i}\) is the set of all classes, \(\Tilde{\mu}_{c_i}\) and \(\mu_{c_i}\) are the augmented and original embedding centers respectively.

\noindent\textit{\textbf{Performance}} on downstream task. 
Following the practice in existing research on textual data augmentation, we train a \(\text{BERT}_{\text{base}}\) model~\cite{kenton2019bert} with a classification head on the original and augmented datasets to evaluate the effect of our proposed data augmentation approach. 
We report the prediction accuracy scores on each dataset to measure downstream task performance.

\noindent\textbf{Datasets.} 
We conduct extensive experiments on 12 NLP datasets to verify the effectiveness of \Methodnamec. 
As specified in Appendix~\ref{app:dataset}, 
our selection of datasets covers a wide range of text classification tasks and includes datasets with different characteristics. 
Following the settings in~\cite{yoo2021gpt3mix}, we sample a subset (1.2K samples) from the full dataset to unify the evaluation settings, enable a fair comparison between methods, and simulate a low-resource condition where data augmentation is of significant necessity. 




\noindent\textbf{Baseline Methods.}
We compare \Methodnamec~with eleven representative data augmentation methods. 
(1) OCR and (2) Keyboard perform common OCR or typing errors at the character level. 
(3) EDA randomly inserts, deletes, replaces, or swaps words in the sentences to create more samples. 
(4) Back-translation (BT) involves translating the source sentences to an intermediary language. 
(5) Unmask randomly replaces words with \texttt{[MASK]} and predicts the masked words with the BERT model. 
(6) AugGPT directly prompts ChatGPT for paraphrases without further fine-tuning~\cite{dai2025auggpt}.
(7) Grammar and (8) Spelling are two exemplar methods selected by Self-LLMDA~\cite{li2024empowering}, which prompt the LLM to simulate common grammatical variation or spelling errors made by humans.
(9) Chain, (10) Hint, and (11) Taboo generate paraphrases with three different diversity incentives~\cite{cegin2024effects}.


\begin{table*}[t]
\centering
\fontsize{11pt}{11pt}\selectfont
\begin{tabular}{lllllllllllll}
\toprule
\multicolumn{1}{c}{\textbf{task}} & \multicolumn{2}{c}{\textbf{Mir}} & \multicolumn{2}{c}{\textbf{Lai}} & \multicolumn{2}{c}{\textbf{Ziegen.}} & \multicolumn{2}{c}{\textbf{Cao}} & \multicolumn{2}{c}{\textbf{Alva-Man.}} & \multicolumn{1}{c}{\textbf{avg.}} & \textbf{\begin{tabular}[c]{@{}l@{}}avg.\\ rank\end{tabular}} \\
\multicolumn{1}{c}{\textbf{metrics}} & \multicolumn{1}{c}{\textbf{cor.}} & \multicolumn{1}{c}{\textbf{p-v.}} & \multicolumn{1}{c}{\textbf{cor.}} & \multicolumn{1}{c}{\textbf{p-v.}} & \multicolumn{1}{c}{\textbf{cor.}} & \multicolumn{1}{c}{\textbf{p-v.}} & \multicolumn{1}{c}{\textbf{cor.}} & \multicolumn{1}{c}{\textbf{p-v.}} & \multicolumn{1}{c}{\textbf{cor.}} & \multicolumn{1}{c}{\textbf{p-v.}} &  &  \\ \midrule
\textbf{S-Bleu} & 0.50 & 0.0 & 0.47 & 0.0 & 0.59 & 0.0 & 0.58 & 0.0 & 0.68 & 0.0 & 0.57 & 5.8 \\
\textbf{R-Bleu} & -- & -- & 0.27 & 0.0 & 0.30 & 0.0 & -- & -- & -- & -- & - &  \\
\textbf{S-Meteor} & 0.49 & 0.0 & 0.48 & 0.0 & 0.61 & 0.0 & 0.57 & 0.0 & 0.64 & 0.0 & 0.56 & 6.1 \\
\textbf{R-Meteor} & -- & -- & 0.34 & 0.0 & 0.26 & 0.0 & -- & -- & -- & -- & - &  \\
\textbf{S-Bertscore} & \textbf{0.53} & 0.0 & {\ul 0.80} & 0.0 & \textbf{0.70} & 0.0 & {\ul 0.66} & 0.0 & {\ul0.78} & 0.0 & \textbf{0.69} & \textbf{1.7} \\
\textbf{R-Bertscore} & -- & -- & 0.51 & 0.0 & 0.38 & 0.0 & -- & -- & -- & -- & - &  \\
\textbf{S-Bleurt} & {\ul 0.52} & 0.0 & {\ul 0.80} & 0.0 & 0.60 & 0.0 & \textbf{0.70} & 0.0 & \textbf{0.80} & 0.0 & {\ul 0.68} & {\ul 2.3} \\
\textbf{R-Bleurt} & -- & -- & 0.59 & 0.0 & -0.05 & 0.13 & -- & -- & -- & -- & - &  \\
\textbf{S-Cosine} & 0.51 & 0.0 & 0.69 & 0.0 & {\ul 0.62} & 0.0 & 0.61 & 0.0 & 0.65 & 0.0 & 0.62 & 4.4 \\
\textbf{R-Cosine} & -- & -- & 0.40 & 0.0 & 0.29 & 0.0 & -- & -- & -- & -- & - & \\ \midrule
\textbf{QuestEval} & 0.23 & 0.0 & 0.25 & 0.0 & 0.49 & 0.0 & 0.47 & 0.0 & 0.62 & 0.0 & 0.41 & 9.0 \\
\textbf{LLaMa3} & 0.36 & 0.0 & \textbf{0.84} & 0.0 & {\ul{0.62}} & 0.0 & 0.61 & 0.0 &  0.76 & 0.0 & 0.64 & 3.6 \\
\textbf{our (3b)} & 0.49 & 0.0 & 0.73 & 0.0 & 0.54 & 0.0 & 0.53 & 0.0 & 0.7 & 0.0 & 0.60 & 5.8 \\
\textbf{our (8b)} & 0.48 & 0.0 & 0.73 & 0.0 & 0.52 & 0.0 & 0.53 & 0.0 & 0.7 & 0.0 & 0.59 & 6.3 \\  \bottomrule
\end{tabular}
\caption{Pearson correlation on human evaluation on system output. `R-': reference-based. `S-': source-based.}
\label{tab:sys}
\end{table*}



\begin{table}%[]
\centering
\fontsize{11pt}{11pt}\selectfont
\begin{tabular}{llllll}
\toprule
\multicolumn{1}{c}{\textbf{task}} & \multicolumn{1}{c}{\textbf{Lai}} & \multicolumn{1}{c}{\textbf{Zei.}} & \multicolumn{1}{c}{\textbf{Scia.}} & \textbf{} & \textbf{} \\ 
\multicolumn{1}{c}{\textbf{metrics}} & \multicolumn{1}{c}{\textbf{cor.}} & \multicolumn{1}{c}{\textbf{cor.}} & \multicolumn{1}{c}{\textbf{cor.}} & \textbf{avg.} & \textbf{\begin{tabular}[c]{@{}l@{}}avg.\\ rank\end{tabular}} \\ \midrule
\textbf{S-Bleu} & 0.40 & 0.40 & 0.19* & 0.33 & 7.67 \\
\textbf{S-Meteor} & 0.41 & 0.42 & 0.16* & 0.33 & 7.33 \\
\textbf{S-BertS.} & {\ul0.58} & 0.47 & 0.31 & 0.45 & 3.67 \\
\textbf{S-Bleurt} & 0.45 & {\ul 0.54} & {\ul 0.37} & 0.45 & {\ul 3.33} \\
\textbf{S-Cosine} & 0.56 & 0.52 & 0.3 & {\ul 0.46} & {\ul 3.33} \\ \midrule
\textbf{QuestE.} & 0.27 & 0.35 & 0.06* & 0.23 & 9.00 \\
\textbf{LlaMA3} & \textbf{0.6} & \textbf{0.67} & \textbf{0.51} & \textbf{0.59} & \textbf{1.0} \\
\textbf{Our (3b)} & 0.51 & 0.49 & 0.23* & 0.39 & 4.83 \\
\textbf{Our (8b)} & 0.52 & 0.49 & 0.22* & 0.43 & 4.83 \\ \bottomrule
\end{tabular}
\caption{Pearson correlation on human ratings on reference output. *not significant; we cannot reject the null hypothesis of zero correlation}
\label{tab:ref}
\end{table}


\begin{table*}%[]
\centering
\fontsize{11pt}{11pt}\selectfont
\begin{tabular}{lllllllll}
\toprule
\textbf{task} & \multicolumn{1}{c}{\textbf{ALL}} & \multicolumn{1}{c}{\textbf{sentiment}} & \multicolumn{1}{c}{\textbf{detoxify}} & \multicolumn{1}{c}{\textbf{catchy}} & \multicolumn{1}{c}{\textbf{polite}} & \multicolumn{1}{c}{\textbf{persuasive}} & \multicolumn{1}{c}{\textbf{formal}} & \textbf{\begin{tabular}[c]{@{}l@{}}avg. \\ rank\end{tabular}} \\
\textbf{metrics} & \multicolumn{1}{c}{\textbf{cor.}} & \multicolumn{1}{c}{\textbf{cor.}} & \multicolumn{1}{c}{\textbf{cor.}} & \multicolumn{1}{c}{\textbf{cor.}} & \multicolumn{1}{c}{\textbf{cor.}} & \multicolumn{1}{c}{\textbf{cor.}} & \multicolumn{1}{c}{\textbf{cor.}} &  \\ \midrule
\textbf{S-Bleu} & -0.17 & -0.82 & -0.45 & -0.12* & -0.1* & -0.05 & -0.21 & 8.42 \\
\textbf{R-Bleu} & - & -0.5 & -0.45 &  &  &  &  &  \\
\textbf{S-Meteor} & -0.07* & -0.55 & -0.4 & -0.01* & 0.1* & -0.16 & -0.04* & 7.67 \\
\textbf{R-Meteor} & - & -0.17* & -0.39 & - & - & - & - & - \\
\textbf{S-BertScore} & 0.11 & -0.38 & -0.07* & -0.17* & 0.28 & 0.12 & 0.25 & 6.0 \\
\textbf{R-BertScore} & - & -0.02* & -0.21* & - & - & - & - & - \\
\textbf{S-Bleurt} & 0.29 & 0.05* & 0.45 & 0.06* & 0.29 & 0.23 & 0.46 & 4.2 \\
\textbf{R-Bleurt} & - &  0.21 & 0.38 & - & - & - & - & - \\
\textbf{S-Cosine} & 0.01* & -0.5 & -0.13* & -0.19* & 0.05* & -0.05* & 0.15* & 7.42 \\
\textbf{R-Cosine} & - & -0.11* & -0.16* & - & - & - & - & - \\ \midrule
\textbf{QuestEval} & 0.21 & {\ul{0.29}} & 0.23 & 0.37 & 0.19* & 0.35 & 0.14* & 4.67 \\
\textbf{LlaMA3} & \textbf{0.82} & \textbf{0.80} & \textbf{0.72} & \textbf{0.84} & \textbf{0.84} & \textbf{0.90} & \textbf{0.88} & \textbf{1.00} \\
\textbf{Our (3b)} & 0.47 & -0.11* & 0.37 & 0.61 & 0.53 & 0.54 & 0.66 & 3.5 \\
\textbf{Our (8b)} & {\ul{0.57}} & 0.09* & {\ul 0.49} & {\ul 0.72} & {\ul 0.64} & {\ul 0.62} & {\ul 0.67} & {\ul 2.17} \\ \bottomrule
\end{tabular}
\caption{Pearson correlation on human ratings on our constructed test set. 'R-': reference-based. 'S-': source-based. *not significant; we cannot reject the null hypothesis of zero correlation}
\label{tab:con}
\end{table*}

\section{Results}
We benchmark the different metrics on the different datasets using correlation to human judgement. For content preservation, we show results split on data with system output, reference output and our constructed test set: we show that the data source for evaluation leads to different conclusions on the metrics. In addition, we examine whether the metrics can rank style transfer systems similar to humans. On style strength, we likewise show correlations between human judgment and zero-shot evaluation approaches. When applicable, we summarize results by reporting the average correlation. And the average ranking of the metric per dataset (by ranking which metric obtains the highest correlation to human judgement per dataset). 

\subsection{Content preservation}
\paragraph{How do data sources affect the conclusion on best metric?}
The conclusions about the metrics' performance change radically depending on whether we use system output data, reference output, or our constructed test set. Ideally, a good metric correlates highly with humans on any data source. Ideally, for meta-evaluation, a metric should correlate consistently across all data sources, but the following shows that the correlations indicate different things, and the conclusion on the best metric should be drawn carefully.

Looking at the metrics correlations with humans on the data source with system output (Table~\ref{tab:sys}), we see a relatively high correlation for many of the metrics on many tasks. The overall best metrics are S-BertScore and S-BLEURT (avg+avg rank). We see no notable difference in our method of using the 3B or 8B model as the backbone.

Examining the average correlations based on data with reference output (Table~\ref{tab:ref}), now the zero-shoot prompting with LlaMA3 70B is the best-performing approach ($0.59$ avg). Tied for second place are source-based cosine embedding ($0.46$ avg), BLEURT ($0.45$ avg) and BertScore ($0.45$ avg). Our method follows on a 5. place: here, the 8b version (($0.43$ avg)) shows a bit stronger results than 3b ($0.39$ avg). The fact that the conclusions change, whether looking at reference or system output, confirms the observations made by \citet{scialom-etal-2021-questeval} on simplicity transfer.   

Now consider the results on our test set (Table~\ref{tab:con}): Several metrics show low or no correlation; we even see a significantly negative correlation for some metrics on ALL (BLEU) and for specific subparts of our test set for BLEU, Meteor, BertScore, Cosine. On the other end, LlaMA3 70B is again performing best, showing strong results ($0.82$ in ALL). The runner-up is now our 8B method, with a gap to the 3B version ($0.57$ vs $0.47$ in ALL). Note our method still shows zero correlation for the sentiment task. After, ranks BLEURT ($0.29$), QuestEval ($0.21$), BertScore ($0.11$), Cosine ($0.01$).  

On our test set, we find that some metrics that correlate relatively well on the other datasets, now exhibit low correlation. Hence, with our test set, we can now support the logical reasoning with data evidence: Evaluation of content preservation for style transfer needs to take the style shift into account. This conclusion could not be drawn using the existing data sources: We hypothesise that for the data with system-based output, successful output happens to be very similar to the source sentence and vice versa, and reference-based output might not contain server mistakes as they are gold references. Thus, none of the existing data sources tests the limits of the metrics.  


\paragraph{How do reference-based metrics compare to source-based ones?} Reference-based metrics show a lower correlation than the source-based counterpart for all metrics on both datasets with ratings on references (Table~\ref{tab:sys}). As discussed previously, reference-based metrics for style transfer have the drawback that many different good solutions on a rewrite might exist and not only one similar to a reference.


\paragraph{How well can the metrics rank the performance of style transfer methods?}
We compare the metrics' ability to judge the best style transfer methods w.r.t. the human annotations: Several of the data sources contain samples from different style transfer systems. In order to use metrics to assess the quality of the style transfer system, metrics should correctly find the best-performing system. Hence, we evaluate whether the metrics for content preservation provide the same system ranking as human evaluators. We take the mean of the score for every output on each system and the mean of the human annotations; we compare the systems using the Kendall's Tau correlation. 

We find only the evaluation using the dataset Mir, Lai, and Ziegen to result in significant correlations, probably because of sparsity in a number of system tests (App.~\ref{app:dataset}). Our method (8b) is the only metric providing a perfect ranking of the style transfer system on the Lai data, and Llama3 70B the only one on the Ziegen data. Results in App.~\ref{app:results}. 


\subsection{Style strength results}
%Evaluating style strengths is a challenging task. 
Llama3 70B shows better overall results than our method. However, our method scores higher than Llama3 70B on 2 out of 6 datasets, but it also exhibits zero correlation on one task (Table~\ref{tab:styleresults}).%More work i s needed on evaluating style strengths. 
 
\begin{table}%[]
\fontsize{11pt}{11pt}\selectfont
\begin{tabular}{lccc}
\toprule
\multicolumn{1}{c}{\textbf{}} & \textbf{LlaMA3} & \textbf{Our (3b)} & \textbf{Our (8b)} \\ \midrule
\textbf{Mir} & 0.46 & 0.54 & \textbf{0.57} \\
\textbf{Lai} & \textbf{0.57} & 0.18 & 0.19 \\
\textbf{Ziegen.} & 0.25 & 0.27 & \textbf{0.32} \\
\textbf{Alva-M.} & \textbf{0.59} & 0.03* & 0.02* \\
\textbf{Scialom} & \textbf{0.62} & 0.45 & 0.44 \\
\textbf{\begin{tabular}[c]{@{}l@{}}Our Test\end{tabular}} & \textbf{0.63} & 0.46 & 0.48 \\ \bottomrule
\end{tabular}
\caption{Style strength: Pearson correlation to human ratings. *not significant; we cannot reject the null hypothesis of zero corelation}
\label{tab:styleresults}
\end{table}

\subsection{Ablation}
We conduct several runs of the methods using LLMs with variations in instructions/prompts (App.~\ref{app:method}). We observe that the lower the correlation on a task, the higher the variation between the different runs. For our method, we only observe low variance between the runs.
None of the variations leads to different conclusions of the meta-evaluation. Results in App.~\ref{app:results}.

\vspace{-2mm}

\begin{figure}[htbp]
    \centering
    \includegraphics[width=\linewidth]{aff_div_pfm.pdf}
    \vspace{-2mm}
    \vspace{-3mm}
    \caption{Diversity, affinity, and performance achieved by \Methodnamec~and baseline methods. Results are averaged on 12 datasets and the diversity rankings are further averaged on 5  metrics in this diagram. Smaller number for the rankings indicates better results. }
    \label{fig:overall}
    \vspace{-4mm}
\end{figure}

\vspace{-2mm}
\subsection{Overall Results}
\vspace{-1mm}


To verify the effectiveness of \Methodnamec, we evaluate downstream task accuracy alongside the diversity and affinity of the augmented dataset. 
We report the rankings of performance, diversity, and affinity averaged on 12 datasets achieved by \Methodnamec~and 11 baseline methods in Figure~\ref{fig:overall}. 
From these results, we have the following observations: 
\textbf{(1)} \textbf{\Methodnamea~achieves the highest performance on downstream tasks} compared to other SOTA data augmentation methods, as indicated by the color bar in Figure~\ref{fig:overall}. 
This demonstrates the high quality and superior adaptability of the datasets generated by our proposed method in real-world applications. 
\textbf{(2)} \textbf{\Methodnamea~achieves the highest diversity score and outperforms all other baseline methods.} This implies that \Methodnamea~effectively improves dataset diversity. 
\textbf{(3)} \textbf{\Methodnamea~achieves a considerably high position on the affinity rankings}, indicating that the sample semantics are preserved to the greatest extent possible. 
In sum, \Methodnamea~achieves the top position in the combined dataset diversity and affinity rankings. 
Additionally, it achieves the best downstream task performance, indicated by the lightest yellow. 




\begin{table}[h!t]
\vspace{-5mm}
\caption{The Coverage in the table is reported as the percentage of the coverage.}
\label{tab:diversity}
% \begin{adjustbox}{width=0.45\textwidth, center} % IEEE
\begin{adjustbox}{width=0.6\textwidth, center} 

\begin{tabular}{|l|l|l|l|}
\hline
model          & cov\_tr\_te  & cov\_tr\_syn          & cov\_te\_syn          \\ \hline
GaussianCopula & 74.60 (0.24) & 0.62 (0.12)           & 0.65 (0.13)           \\ \hline
CTGAN          & 74.88 (0.17) & 19.95 (1.40)          & 19.52 (1.33)          \\ \hline
TVAE           & 74.78 (0.17) & 13.94 (0.48)          & 13.79 (0.43)          \\ \hline
CTABGAN        & 74.83 (0.15) & 2.96 (0.13)           & 3.02 (0.11)           \\ \hline
STaSy          & 74.80 (0.12) & 2.34 (0.34)           & 2.40 (0.32)           \\ \hline
TabDDPM        & 75.01 (0.44) & \textbf{68.69 (0.61)} & \textbf{67.92 (0.69)} \\ \hline
\end{tabular}

\end{adjustbox}
\vspace{-5mm}
\end{table}

\subsection{Performance, Diversity, and Affinity}

\subsubsection{Performance Gains}

The full results for BERT classification performance on original and augmented datasets are presented in Table~\ref{tab:results}. The results show that \Methodnamec~surpasses all baseline methods on average. Specifically, it outperforms the baseline methods on 11 out of 12 datasets except on RTE. \Methodnamea~achieves performance gain of \(10.52\%\) on average, surpassing the runner-up method, namely Taboo, with 3.76 percentage points. 

\vspace{-1.5mm}
\subsubsection{Diversity Gain}
\vspace{-1.5mm}


We demonstrate the diversity gains in terms of all 5 diversity metrics along with their rankings achieved by \Methodnamec~and baseline methods in Figure~\ref{tab:diversity}. 
\Methodnamec~ranks the top on the chart with an average ranking of 3.0. Specifically, it achieves the best for the Distance and Dispersion metrics. For the Vocabulary metric, OCR and Keyboard include large amounts of out-of-vocabulary tokens due to character level perturbation. The three baselines with diversity incentives, namely Chian, Hint, and Taboo, also achieve reasonably good diversity gain, in line with the results of \cite{cegin2024effects}.

\begin{figure}[t]
    \centering
    \includegraphics[width=0.98\linewidth]{affinity.pdf}
    \captionof{figure}{Affinity scores of \Methodnamec~and 10 baseline methods. The scores are averaged on 12 datasets.}
    \vspace{-4mm}
    \label{fig:affinity}
\end{figure}

\vspace{-1.5mm}
\subsubsection{Affinity and Paraphrase Validity}
\vspace{-1.5mm}


We present the affinity of \Methodnamec~and baseline methods in Figure~\ref{fig:affinity}, where \Methodnamea~outperforms other methods except Unmask, whose affinity score is 1.95. The Unmask method generates augmentation by replacing randomly selected words with ``\texttt{[MASK]}'' and predicts the masked words with the BERT model. Since the augmented samples are recovered from the BERT embeddings of the corrupted original samples and we use the BERT embeddings to calculate affinity, it is reasonable to yield extremely high affinity scores. 
Following \cite{cegin2023chatgpt,cegin2024effects}, we also investigate paraphrase validity at the sample level. We prompt the DeepSeek-V3 model to check if the paraphrases are semantically similar to the original samples and adhere to the original labels. Results show that \(97\%\) paraphrases are valid, suggesting the data augmenter introduces negligible noises to the dataset. 




\vspace{-2mm}
\subsection{Ablation Studies}
\vspace{-2mm}

To verify the effectiveness of our proposed method, we conduct ablation studies to show that all components in the method framework contribute to the final performance gains, as shown in Table~\ref{tab:ablation}, where w/o Coreset refers to applying augmentation on a random subset of the dataset without deriving a coreset of importance samples, w/o Selective refers to augmenting samples in both \(\mathcal{S}_\text{retain}\) and \(\mathcal{S}_\text{augment}\) instead of only augmenting the latter, w/o Aug refers to using the coreset directly for training without data augmentation, w/o DPO refers to using the LLM paraphraser from the SFT stage for data augmentation, w/o DS refers to removing the diversity-based sampling module, and w/o DPO|DS removes both steps. 
Thus, all components in our method framework contribute to the final performance gains. 
As Figure~\ref{fig:div_ablt} shows, we also study the effect of diversity-oriented fine-tuning and diversity-based sampling on the dataset diversity, demonstrating that all proposed components are effective. 
% \begin{table}[!t]
% \centering
% \scalebox{0.68}{
%     \begin{tabular}{ll cccc}
%       \toprule
%       & \multicolumn{4}{c}{\textbf{Intellipro Dataset}}\\
%       & \multicolumn{2}{c}{Rank Resume} & \multicolumn{2}{c}{Rank Job} \\
%       \cmidrule(lr){2-3} \cmidrule(lr){4-5} 
%       \textbf{Method}
%       &  Recall@100 & nDCG@100 & Recall@10 & nDCG@10 \\
%       \midrule
%       \confitold{}
%       & 71.28 &34.79 &76.50 &52.57 
%       \\
%       \cmidrule{2-5}
%       \confitsimple{}
%     & 82.53 &48.17
%        & 85.58 &64.91
     
%        \\
%        +\RunnerUpMiningShort{}
%     &85.43 &50.99 &91.38 &71.34 
%       \\
%       +\HyReShort
%         &- & -
%        &-&-\\
       
%       \bottomrule

%     \end{tabular}
%   }
% \caption{Ablation studies using Jina-v2-base as the encoder. ``\confitsimple{}'' refers using a simplified encoder architecture. \framework{} trains \confitsimple{} with \RunnerUpMiningShort{} and \HyReShort{}.}
% \label{tbl:ablation}
% \end{table}
\begin{table*}[!t]
\centering
\scalebox{0.75}{
    \begin{tabular}{l cccc cccc}
      \toprule
      & \multicolumn{4}{c}{\textbf{Recruiting Dataset}}
      & \multicolumn{4}{c}{\textbf{AliYun Dataset}}\\
      & \multicolumn{2}{c}{Rank Resume} & \multicolumn{2}{c}{Rank Job} 
      & \multicolumn{2}{c}{Rank Resume} & \multicolumn{2}{c}{Rank Job}\\
      \cmidrule(lr){2-3} \cmidrule(lr){4-5} 
      \cmidrule(lr){6-7} \cmidrule(lr){8-9} 
      \textbf{Method}
      & Recall@100 & nDCG@100 & Recall@10 & nDCG@10
      & Recall@100 & nDCG@100 & Recall@10 & nDCG@10\\
      \midrule
      \confitold{}
      & 71.28 & 34.79 & 76.50 & 52.57 
      & 87.81 & 65.06 & 72.39 & 56.12
      \\
      \cmidrule{2-9}
      \confitsimple{}
      & 82.53 & 48.17 & 85.58 & 64.91
      & 94.90&78.40 & 78.70& 65.45
       \\
      +\HyReShort{}
       &85.28 & 49.50
       &90.25 & 70.22
       & 96.62&81.99 & \textbf{81.16}& 67.63
       \\
      +\RunnerUpMiningShort{}
       % & 85.14& 49.82
       % &90.75&72.51
       & \textbf{86.13}&\textbf{51.90} & \textbf{94.25}&\textbf{73.32}
       & \textbf{97.07}&\textbf{83.11} & 80.49& \textbf{68.02}
       \\
   %     +\RunnerUpMiningShort{}
   %    & 85.43 & 50.99 & 91.38 & 71.34 
   %    & 96.24 & 82.95 & 80.12 & 66.96
   %    \\
   %    +\HyReShort{} old
   %     &85.28 & 49.50
   %     &90.25 & 70.22
   %     & 96.62&81.99 & 81.16& 67.63
   %     \\
   % +\HyReShort{} 
   %     % & 85.14& 49.82
   %     % &90.75&72.51
   %     & 86.83&51.77 &92.00 &72.04
   %     & 97.07&83.11 & 80.49& 68.02
   %     \\
      \bottomrule

    \end{tabular}
  }
\caption{\framework{} ablation studies. ``\confitsimple{}'' refers using a simplified encoder architecture. \framework{} trains \confitsimple{} with \RunnerUpMiningShort{} and \HyReShort{}. We use Jina-v2-base as the encoder due to its better performance.
}
\label{tbl:ablation}
\end{table*}

\begin{figure}[t]
    \centering
    \includegraphics[width=0.85\linewidth]{diversity_radar_ablt_circle.pdf}
    \vspace{-3mm}
    \caption{Ablation study on diversity gains}
    \vspace{-3mm}
    \label{fig:div_ablt}
\end{figure}




\begin{table}[t]
    \centering
    \scriptsize
    \setlength{\tabcolsep}{4pt} % horizontal, Default value: 6pt
    \renewcommand{\arraystretch}{0.8} 
    \begin{tabular}{lccccccc}
        \toprule
         & \multicolumn{3}{c}{GPT-2 (137M)} &  & \multicolumn{3}{c}{T5-large (738M)} \\ 
        \cmidrule{2-4} \cmidrule{6-8}
                            & CoLA  & MNLI  &  RCT  &  & CoLA  & MNLI  &  RCT  \\
        \midrule
        Original            & 67.64 & 50.03 & 76.53 &  & 78.94 & 50.06 & 78.98 \\
        OCR                 & 65.51 & 54.42 & 77.70 &  & 77.79 & 56.35 & 81.77 \\
        Keyboard            & 65.23 & 54.51 & 77.37 &  & 77.06 & 56.40 & 81.65 \\
        EDA                 & 66.60 & 54.13 & 76.89 &  & 78.59 & 51.58 & 80.79 \\
        BT                  & 63.43 & 53.89 & 77.10 &  & 76.22 & 54.70 & 80.80 \\
        Unmask              & 66.87 & 55.75 & 78.01 &  & 77.84 & 61.88 & 81.41 \\
        AugGPT              & 66.08 & 54.80 & 78.28 &  & 77.79 & 60.18 & 79.18 \\
        Grammar             & 66.20 & 55.75 & 76.96 &  & 76.52 & 57.80 & 81.03 \\
        Spelling            & 66.24 & 55.53 & 78.16 &  & 77.57 & 62.93 & 81.99 \\
        Chain               & 65.85 & 54.83 & 78.04 &  & 77.98 & 59.88 & 81.50 \\
        Hint                & 66.41 & 55.75 & 77.52 &  & 78.06 & 61.13 & 81.38 \\
        Taboo               & 64.27 & 56.13 & 77.72 &  & 77.36 & 61.55 & 81.30 \\
        w/o Aug             & 67.85 & 52.79 & 79.47 &  & 79.74 & 53.18 & 81.63 \\
        w/o Coreset         & 65.24 & 50.89 & 76.98 &  & 77.92 & 51.85 & 81.01 \\
        w/o Selective       & 66.05 & 50.23 & 78.76 &  & 79.00 & 53.22 & 80.85 \\
        w/o DPO|DS          & 66.74 & 55.51 & 79.71 &  & 79.64 & 61.06 & 82.09 \\
        w/o DPO             & 67.05 & 54.97 & 79.72 &  & 79.51 & 61.79 & 81.83 \\
        w/o Sampling        & 67.50 & 55.93 & 79.79 &  & 78.97 & 62.03 & 82.18 \\
        \midrule
        \Methodnamea        & \textbf{68.14} & \textbf{56.25} & \textbf{79.83} &  & \textbf{79.93} & \textbf{63.91} & \textbf{82.20} \\
        \bottomrule
    \end{tabular}
    \captionof{table}{Training the GPT-2 and T5-large models on CoLA, MNLI, and RCT dataset.}
    \label{tab:t5}
\end{table}


\begin{figure}[t]
    \centering
    \vspace{-3mm}
    \subfloat[Average accuracy]{
        \includegraphics[width=0.31\linewidth]{qwen_mean.pdf}
        \label{subfig:qwen_perf}
    }
    \subfloat[Distance]{
        \includegraphics[width=0.33\linewidth]{diversity_qwen_Distance.pdf}
        \label{subfig:qwen_dist}
    }
    \subfloat[Dispersion]{
        \includegraphics[width=0.315\linewidth]{diversity_qwen_Dispersion.pdf}
        \label{subfig:qwen_disp}
    }
    \vspace{-2mm}
    \caption{Performance and diversity comparison between Llama and Qwen.}
    \vspace{-3mm}
    \label{fig:qwen}
\end{figure}

\subsection{LLM Architectures Exploration}
To exhibit the generalizability of our proposed methodology, we replace the LLM augmenter and downstream task model with other LLM architectures respectively. The results show that \Methodnamec~is agnostic to LLM architectures.

For the LLM augmenter, we replace the Llama-3.2-1B-Instruct model with the similar-sized Qwen2.5-1.5B-Instruct model~\cite{qwen2.5}. As shown in Figure~\ref{fig:qwen}, datasets augmented by the Qwen model significantly outperform the original dataset and achieve comparable accuracy and diversity with those of the Llama model. Detailed results are given in Appendix~\ref{app:llm_arch}.

For the downstream task model, we replace \(\text{BERT}\), an encoder-only model with the GPT-based and T5-based classification models. 
GPT is a decoder-only LLM and is especially adept at generating texts from a prompt. Based on an encoder-decoder transformer architecture, T5 is trained to perform all NLP tasks in a unified text-to-text format and is favorable in broad cases. Specifically, we use GPT-2~\cite{radford2019language} and T5-large~\cite{raffel2020exploring} as the backbone of classification models, train these models on the MNLI, CoLA, and RCT datasets, and collect their performances. 
Experimental results in Table~\ref{tab:t5} show that \Methodnamec~ benefits both decoder-only models such as GPT and encoder-decoder models such as T5. 
