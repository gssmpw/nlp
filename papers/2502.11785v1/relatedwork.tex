\section{Related work }
\label{sec:related}

%\maksim{Mention STIT}

\paragraph*{Strategic Reasoning.}

From the perspective of strategic reasoning, our work is related to the research on rational verification and synthesis. The first is the problem of checking whether a temporal goal is 
satisfied in some (or all) game-theoretic equilibria of a CGM \cite{AbateGHHKNPSW21,GutierrezNPW23}. Rational synthesis consists in the automated construction of such a model \cite{FismanKL10, CFGR16}.  
In this direction, 
\cite{KR2024-44} investigated the problem of finding \emph{incentives} by manipulating the weights of atomic propositions to persuade %motivate %rational
agents to act towards a temporal goal. %the system designers' goal.




Recent work has also investigated the use of formal methods to verify and synthesize mechanisms for social choice using model checking and satisfiability procedures for variants of Strategy Logic \cite{SLKF_KR21,MittelmannMMP22,MittelmannMMP23}. 
While being able to analyse MAS with respect to complex solution concepts, all these works face high complexity issues. 
In particular, key decision problems for rational verification with temporal specifications are known to be \Dexptime-complete \cite{GutierrezNPW23} and model checking  Strategy Logic is \NONELEMENTARY for memoryfull agents \cite{MogaveroMPV14}. 
\rust{
Compared to these approaches, $\LAMB$ offers relatively high expressivity while maintaining  the \Ptime-completeness of its model checking problem.
}

   % \munyque{If we have space, we can add a comment saying something about how we have some nice cost-expressivity balance with LAMB (in comparison with the exponential approaches mentioned here). }

\rust{
The recently introduced
\textit{obstruction ATL} \cite{catta23,catta24} ($\mathsf{O}\ATL$) allows reasoning 
%one to reason
about agents' strategic abilities %to cooperate and execute their strategies
while being hindered by an external force, called the \textit{Demon}. Being inspired by sabotage modal logic \cite{vanbenthem05}, in this logic  
the Demon is able to disable some transitions and thus impact the strategic abilities of the agents in a system. This is somewhat related to normative updates that we covered in Section \ref{sec:norms} and the \textit{module checking problem} \cite{kupferman2001module,jamroga2015module}, where agents interact with a non-deterministic environment that may inhibit access to certain paths of the computation tree. 

Notice that $\LAMB$ is significantly more general than the presented approaches %and ideas, 
as it allows to not only restrict transitions or access, but also change it in a more nuanced way by redirecting arrows (and, e.g., granting access to a state). Moreover, $\LAMB$ also allows adding \textit{new} states, as well as changing the valuations of propositions. Moreover, updates in $\LAMB$ are explicitly present in the syntax that enables explicit synthesis of model modifications.
}

\rust{
\paragraph*{Nominals.} Nominals are an integral part of \textit{hybrid logic} \cite{ARECES2007821} and is a common tool whenever one needs to refer to particular states on the syntax level. For example, nominals and other hybrid modalities are ubiquitous in the research on \textit{logics for social networks} (see \cite[Chapter 3]{minathesis} for a comprehensive overview). %The idea there is to identify states with agents, and the accessibility relation as a friendship, followership, visibility, or any other relation corresponding to a social network setting. 

In the setting of DEL, tools and methods of hybrid logic have been used, for example, to relax the assumption of common knowledge of agents' names \cite{wang18}, to study the interplay between public announcements and distributed knowledge \cite{HANSEN201133}, and to tackle the information and intentions dynamics in interactive scenarios \cite{Roy2009}. Moreover, nominals were used to provide an axiomatisation of a hybrid variant of \textit{sabotage modal logic} \cite{vanbenthem23}, which extends the standard language of modal logic with constructs $\blacklozenge \varphi$ meaning `after removing some edge in the model, $\varphi$ holds' \cite{vanbenthem05,aucher18}.

Nominals were also used in \textit{linear-} and \textit{branching-time temporal logics} %in order 
to refer to particular points in computation (see, e.g.,  \cite{blackburn99,Goranko2000-GORCTL-2,Franceschet_etal2003,franceschet06,Kara2009,Lange2009,BOZZELLI2010454,Kernberger2020}). 
In the framework of strategic reasoning, \cite{Huang_Meyden2018} %used constructs similar to the hybrid logic modality $\E$, 
used some ideas from hybrid logic,
but neither $@_{\alpha}$ nor nominals themselves. Hence, in terms of novelty,   %we believe that
to the best of our knowledge, the  \textit{Hybrid ATL} ($\mathsf{HATL}$) proposed in this paper is the first attempt to combine nominals with the $\mathsf{ATL}$-style strategic reasoning.

}

%\maksim{We refer to strategic reasoning in the par. above and then again introduce the whole paragraph on strategic reasoning.}
%\rustam{Maybe we can swap the order of paragraphs?}

%\maksim{I think E modality has never been mentioned, so it may be confusing}


%\munyque{Is it worth adding our rebuttal comment about how, expressivity-wise, the LAMB and Obstruction ATL are not comparable? }

%\maksim{I think we should}

\rust{
\paragraph*{The Interplay Between DEL and Strategic Reasoning.}

As we mentioned in the introduction, albeit DEL and various strategic logics being very different formalism, some avenues of DEL research has incorporated ideas from logics for strategic reasoning. Examples include the exploration of \textit{concurrent DEL games} \cite{maubert20}, \textit{alternating-time temporal DEL} \cite{delima14}, \textit{coalitions announcements} \cite{agotnes08,galimullin21b} and other forms of \textit{strategic multi-agent communication} (see, e.g., \cite{agotnes10GAL,galimullin24}).

To the best of our knowledge, DEL updates for CGMs, up until now, were considered only in \cite{galimullin21,galimullin2022action}, where the authors capture granting and revoking actions of singular agents as well as updates based on \textit{action models} \cite{bms22}. Both works are limited to the ne$\mathsf{X}$t-time fragment of $\ATL$ (so-called \textit{coalition logic} \cite{pauly02}). Moreover, they do not support such expressive features of $\LAMB$ as adding \textit{new} states and changing the valuation of propositional variables. Additionally, our arrow-redirecting operators allow for greater flexibility while dealing with agents' strategies. }


\balance %Balance is required on the last page when using AAMAS template