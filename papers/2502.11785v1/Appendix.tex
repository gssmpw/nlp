\clearpage
\appendix
\section*{Technical Appendix}
\iffalse
\subsection*{Proofs of theorems}
\begin{proof}[Proof sketch of Theorem~\ref{thm:hatlEQslamb}]
Soundness of the axioms can be shown by a direct application of the definition of the semantics of $\LAMB$. 

As an example, consider axiom A1 $[p_\alpha := \psi] p \leftrightarrow (\alpha \to  \psi) \land (\lnot \alpha \to p)$. Let us assume that for some CGM $M_s$ it holds that $M_s \models [p_\alpha := \psi] p$. According to the definition of $M_s^{p_\alpha := \psi}$ (Definition \ref{def:semLAMB}), we consider two cases. First, if $M_s \models \alpha$, then the truth-value of $p$ is equivalent to whether $M_s \models \psi$. Hence, the first part of the right-hand part of A1: $M_s \models \alpha \to \psi$. Second, if $M_s \not \models \alpha$, then the truth-value of $p$ remains intact, i.e. $M_s \models \lnot \alpha \to p$. Combining two parts together, we have that $M_s \models [p_\alpha := \psi] p$ if and only if $M_s \models (\alpha \to \psi) \land (\lnot \alpha \to p)$.     

As another example, consider A7 $[p_\alpha := \psi] \llb C \rrb \X \varphi \leftrightarrow \llb C \rrb \X [p_\alpha := \psi] \varphi$. Again, assume that for an arbitrary CGM $M_s$ we have that $M_s \models [p_\alpha := \psi] \llb C \rrb \X \varphi$. By the definition of semantics, the latter is equivalent to $M_s^{p_\alpha := \psi} \models  \llb C \rrb \X \varphi$, which in turn is equivalent to the fact that $\exists \sigma_C, \forall \lambda \in \Lambda^s_{\sigma_C}:$ $M^{p_\alpha := \psi}_{\lambda[1]} \models \varphi$. Now, by the semantics of substitutions, we have that $\exists \sigma_C, \forall \lambda \in \Lambda^s_{\sigma_C}:$ $M_{\lambda[1]} \models [p_\alpha := \psi]\varphi$, which in turn is equivalent to $M_s \models \llb C \rrb \X [p_\alpha := \psi] \varphi$.

All other cases can be shown in a similar fashion. 
\end{proof}
\fi
\iffalse
\rust{
\subsection*{Varying Number of Actions}
As we have mentioned when defining named CGMs, we can relax the assumption of all agents having all agents available to them and also treat the set of available actions as model-specific. To this end, we can  modify Definition \ref{def:cgs} of a named CGM to be a tuple $(\Act, S, En, \tau, L)$, where $\Act$ is a finite set of actions, and $En: S \times \AG \to 2^\Act \setminus \emptyset$. Here function $En$, called \textit{enabling function}, specifies for each agent in each state a non-empty set of actions available to the agent. 

With such more general setting, we need to modify the semantics of $\lcirc{\alpha}$. Recall that as a result of executing $\lcirc{\alpha}$, a new state labelled $\alpha$ is added to the given model and all transitions labelled by all possible action profiles are added as self-loops in this new state. Now, we can also generalise the operator $\lcirc{\alpha}$ by explicitly specifying how many actions agent will have in the new state. For this, we modify the semantics of $\LAMB$ by substituting $\lcirc{\alpha}$ with $\lcirc{\alpha}^k$ with the intended meaning that all agents in the new state named alpha have $k$ actions each.

Semantically, $M^{\lcirc{\alpha}^k} = \lb Act^{\lcirc{\alpha}^k}, S^{\lcirc{\alpha}^k}, En^{\lcirc{\alpha}^k}, \tau^{\lcirc{\alpha}^k}, L^{\lcirc{\alpha}^k}\rb$, where, if $L(\alpha) = \emptyset$, then  $\Act^{\lcirc{\alpha}^k} = \Act \cup k$ if $k > |\Act|$ and $\Act^{\lcirc{\alpha}^k} = \Act$ otherwise,  $S^{\lcirc{\alpha}} = S \cup \{t\}$ with  $t \not \in S$, $\tau^{\lcirc{\alpha}} = \tau \cup \{(t, A, t) \mid \forall A \in k^\AG\}$, and $L^{\lcirc{\alpha}}(\alpha) = L \cup \{(\alpha, \{t\})\}$. If $L(\alpha) \neq \emptyset$, then $M^{\lcirc{\alpha}} = \lb S, \tau, L\rb$. 
}

\rust{
\subsection*{On Global Synthesis From $\ATL$ Specifications}
Here we consider the \textit{global synthesis} \textit{problem} from $\ATL$-specifications. The problem consists of providing an %(composed)
update $\pi_\varphi$ for a given $\varphi \in \ATL$ such that if $\varphi$ is satisfiable, then
$M_s \models [\pi_\varphi]\varphi$ for all $M_s$. % if such $\pi_\varphi$ exists 
In other words, we can reframe the $\ATL$ satisfiability as the global synthesis problem, where $\pi_\varphi$ can be understood as a program that instructs how to modify any CGM to satisfy some given $\varphi$. In our case, for convenience, %for the reasons described below,
we consider a slight restatement of the problem, where we construct a $\pi_\varphi$ such that $M_s \models [\pi_\varphi] \llb \AG \rrb \X \varphi$.


To construct such a $\pi_\varphi$, we can use the finite model property of $\ATL$ and the fact that the satisfiability problem for $\ATL$ (SAT-$\ATL$) is decidable (\Exptime-complete, to be precise) 
\cite{goranko06}. The proof of \cite{goranko06} uses a filtration-based argument to construct a finite model $M^\varphi_t$ for a given $\varphi \in \ATL$ (see also \cite{walther06}) over a fixed set of agents $\AG$. By the construction, the model has a fixed branching degree of $k^{|\AG|}$, where $k$ depends on the length of the given formula $\varphi$. In other words, in the constructed model $M_t^\varphi$, in each state each agent has exactly $k$ available. 

Going back to the synthesis problem, recall that we have some current given model $M_s$ and an $\ATL$ formula $\varphi$. We run SAT-$\ATL$ ($\varphi$) that returns a finite model $M_t^\varphi = (\Act^\varphi, S^\varphi, En^\varphi, \tau^\varphi, L^\varphi)$ over $\AG$, and where $|\Act^\varphi| = k$, $S^\varphi$ is finite, and $En^\varphi (s,i) = \Act^\varphi$ for all $s\in S^\varphi$ and $i \in \AG$ with a fixed branching degree $k^{|\AG|}$, such that $M^\varphi_t \models \varphi$.

Having the model $M^\varphi_t$ satisfying $\varphi$ at hand, we can now create a complex program $\pi_\varphi$ that essentially builds up model $M^\varphi_t$. 
In particular, for all $s \in S^\varphi$, we add to $\pi_\varphi$ constructs $[\lcirc{\alpha}^k]$, where $\alpha$ is a nominal not appearing in $M_s$, 
that will update the original model $M_s$ with new states with exactly $k^{|\AG|}$ 
transitions to capture all transitions in all $s \in S^\varphi$. For convenience, we will write $s_\alpha$ for a new state in $M_s$ that corresponds to $s \in S^\varphi$ and was added using $\lcirc{\alpha}^k$. %, where $\alpha$'s are fresh nominals not appearing in the given $M_s$. 

To set the values of propositional atoms we then use assignments $p_\alpha := \top$ . For all $p \in True(s)$ with $s \in S^\varphi$, we add to $\pi_\varphi$ constructs $[p_\alpha := \top]$ that will make the propositional variables true in $s \in S^\varphi$ also true in $s_\alpha$. 

Finally, we `unravel' all the self-loops in the new added states to finalise the reconstruction of $M^\varphi_t$.
To this end, we add to $\pi_\varphi$ constructs $[\alpha \xrightarrow{A} \beta]$, where $A$ is constructed from $\Act^\varphi$ and are of size $|\AG|$. In particular, we add $[\alpha \xrightarrow{A} \beta]$ if $(s, A, t) \in \tau^\varphi$ with $s,t\in S^\varphi$. In such a way, we connect the newly added $s_\alpha$ and $t_\beta$ via $A$. To finalise the construction, observe that we have reconstructed model $M^\varphi_t$ on the 'side' of the original model $M_s$. What is left is to redirect all arrows outgoing from $s$ $M_s$ to the pointed state $t$ of $M_t$. This is done by adding to $\pi_\varphi$ constructs $[\alpha \xrightarrow{A} \beta]$, where $\alpha$ is the name of $s$, and $\beta$ is the name of the newly added $t$.   %To connect the given state of evaluation $s$ in model $M$ with the constructed $M^\varphi$, we redirect an arrow 
%coming out of $s$ to the state of evaluation of $M_\varphi$ named $\alpha$. 

Combining all these operators consecutively, we obtain $\pi_\varphi:= [\lcirc{\alpha}^k]...[\lcirc{\beta}^k][p_\alpha := \top] ... [q_\beta := \top] [\alpha \xrightarrow{A} \beta;...;\gamma \xrightarrow{B} \delta]$, or, equivalently, $\pi_\varphi:= \lcirc{\alpha}^k;...;\lcirc{\beta}^k;p_\alpha := \top; ...; q_\beta := \top; \alpha \xrightarrow{A} \beta;...;\gamma \xrightarrow{B} \delta$. %; \mathit{Nom}(s) \xrightarrow{X} \alpha$. %, where the part $\mathit{Nom}(s) \xrightarrow{X} \alpha$ corresponds to rerouting an action profile $X$ from $s$ to state named $\alpha$.  

It is easy to see now that as a result, %of such a procedure, 
we have $M_s \models [\pi_\varphi]\llb \AG \rrb \X \varphi$ for any given $M_s$.
}
\fi
\subsection*{Model checking $\LAMB$}
Full model checking algorithm for $\LAMB$.
\begin{breakablealgorithm}
	\caption{An algorithm for model checking $\LAMB$}
	\small
	\begin{algorithmic}[1] 		
		\Procedure{MC}{$M, s, \varphi$}		
      \Case {$\varphi = p$}
            \State{\textbf{return} $s \in L(p)$}
        \EndCase
        \Case {$\varphi = \alpha$}
            \State{\textbf{return} $s \in L(\alpha)$}
        \EndCase
        \Case {$\varphi = @_\alpha \psi$}
        \If {$L(\alpha) \neq \emptyset$}
            \State{\textbf{return}  $\textsc{MC} (M, L(\alpha), \psi)$}
        \Else
            \State{\textbf{return}  \textit{false}}
        \EndIf
        \EndCase
        \Case {$\varphi = \lnot \psi$}
            \State{\textbf{return}  not $\textsc{MC} (M, s, \psi)$}
        \EndCase
       \Case {$\varphi = \psi \land \chi$}
            \State{\textbf{return} $\textsc{MC} (M,s,\psi)$ and  $\textsc{MC} (M,s,\chi)$}
        \EndCase
      \Case {$\varphi = \llb C \rrb \X \psi$}
             \State{\textbf{return} $s \in Pre(M, C, \{t \in S | \textsc{MC}(M, t, \psi)\})$}
        \EndCase
    \Case{$\varphi = \llb C \rrb \psi \U  \varphi$}
    \State{$X:= \emptyset$ and $Y:= \{t \in S \mid \textsc{MC} (M,t, \varphi)$\}}
    \While{$Y \neq X$}
        \State{$X:= Y$}
        \State{$Y := \{t \in S \mid \textsc{MC} (M,t, \varphi)\} \cup (Pre(M, C, X) \cap \{t \in S \mid \textsc{MC} (M,t, \psi)\})$}
    \EndWhile
    \State{\textbf{return} $s \in X$}
    \EndCase

    \Case{$\varphi = \llb C \rrb \psi \R \varphi$}
    %\State{$X:= \emptyset$ and $Y := \{t \in S \mid \textsc{MC} (M,t, \varphi)\} \cup \{t \in S \mid \textsc{MC} (M,t, \psi)\} $}
    \State{$X:= S$ and $Y := \{t \in S \mid \textsc{MC} (M,t, \varphi)\}$}
    %\While{$Y \not \subseteq X$}
    %    \State{$X:= X \cup Y$}
    %    \State{$Y:=Pre(M, C, X) \cap \{t \in S \mid \textsc{MC} (M,t, \varphi)\}$}
    \While{$Y \neq X$}
        \State{$X:= Y$}
        \State{$Y := \{t \in S \mid \textsc{MC} (M,t, \varphi)\} \cap (Pre(M, C, X) \cup \{t \in S \mid \textsc{MC} (M,t, \psi)\})$}
    \EndWhile
    \State{\textbf{return} $s \in X$}
    \EndCase
    
\Case {$\varphi = [\pi] \psi$ with $\pi\in \{p_\alpha:=\psi, \alpha \xrightarrow{A} \beta, \lcirc{\alpha}\}$}
			\State{\textbf{return} $\textsc{MC} (\textsc{Update} (M, s, \pi), s, \psi)$}
		\EndCase
   \EndProcedure

	\end{algorithmic}
\end{breakablealgorithm}





\subsection*{A Note On Succinctness}
When we discussed normative updates on CGMs in Section \emph{Dynamic MAS through the lens of $\LAMB$}, we, as a use case, considered sanctioning norms $\mathsf{SN}$'s and their effects on a given system. For example, we may want to check whether the systems is \textit{compliant} with some set of norms, i.e. whether none of the normative updates violate some desired property $\varphi$ (e.g. a safety requirement).  In other words, for a given model $M_s$ and a set of norms $\mathcal{N} = \{\mathsf{SN}_1, ..., \mathsf{SN}_n\}$, we can explicitly check whether $M_s$ satisfies $\varphi$ after each $\mathsf{SN}$. As described in the normative updates section, we can model the effects of norms $\mathcal{N}$ in the language of $\LAMB$. Hence, the compliance just described can be expressed by formula $\bigwedge_{i \in \{1,...,n\}} [\mathit{SN}_i]\varphi$, where $\mathit{SN}_i$ are the translation of norms from $\mathcal{N}$'s into $\LAMB$ updates.

The reader might have noticed, however, that such a formula is not quite succinct, as it may have exponentially many repetitions, especially in the case of nested update operators. This happens, for example, if we have several sets of norms that we would like to implement consecutively.  For instance, if the desired formula looks like $\bigwedge_{i \in \{1, 2\}} \bigwedge_{j \in \{1,2,3\}}[\mathit{SN}_i^a][\mathit{SN}_j^b]\varphi$, writing it out in full yields us 
\begin{gather*}
[\mathit{SN}_1^a]([\mathit{SN}_1^b]\varphi \land [\mathit{SN}_2^b]\varphi \land [\mathit{SN}_3^b]\varphi) \land \\
[\mathit{SN}_2^a]([\mathit{SN}_1^b]\varphi \land [\mathit{SN}_2^b]\varphi \land [\mathit{SN}_3^b]\varphi).
\end{gather*}
The reader may think of a fleet of warehouse robots that we want to govern on two levels: first set of norms would regulate the traffic laws in the warehouse to avoid collisions, and the second set would regulate strategic objectives of robots \textit{given the implemented traffic laws}.

To deal with such a blow-up, we can consider a variant of $\LAMB$, called $\LAMB^\cup$, that extends the former with constructs $[\pi \cup \rho]\varphi$ with the intended meaning `whichever update we implement, $\pi$ or $\rho$, $\varphi$ will be true (in both cases)'. Such a union of actions, inherited from Propositional Dynamic Logic ($\mathsf{PDL}$) \cite{fischer79}, is also quite used in $\mathsf{DEL}$ (see, e.g., \cite{bms22,aucher13}). Corresponding fragments of $\LAMB^\cup$ are denoted as  $\mathsf{SLAMB}^\cup$ and $\mathsf{ALAMB}^\cup$.  

Semantics of the union operator is defined as 
  \begin{alignat*}{3}
        &M_s \models [\pi \cup \rho] \varphi && \text{ iff } &&M_s \models [\pi] \varphi \text{ and } M_s \models [\rho] \varphi
\end{alignat*} 
It is easy to see now that $[\pi \cup \rho] \varphi \leftrightarrow [\pi] \varphi \land [\rho] \varphi$ is a valid formula (i.e. it is true on all CGMs $M_s$), and hence we can translate every formula $\LAMB^\cup$ into an equivalent formula of $\LAMB$ (the same for $\mathsf{SLAMB}^\cup$ and $\mathsf{ALAMB}^\cup$). Therefore, expressivity-wise, $\LAMB^\cup \approx \LAMB$, $\mathsf{SLAMB}^\cup \approx \mathsf{SLAMB}$, and $\mathsf{ALAMB}^\cup \approx \mathsf{ALAMB}$.

As noted above, however, that $\LAMB^\cup$ allows us to express, for example, the compliance property for normative updates exponentially \textit{more succinct}. Recalling our example of \begin{gather*}
[\mathit{SN}_1^a]([\mathit{SN}_1^b]\varphi \land [\mathit{SN}_2^b]\varphi \land [\mathit{SN}_3^b]\varphi) \land \\
[\mathit{SN}_2^a]([\mathit{SN}_1^b]\varphi \land [\mathit{SN}_2^b]\varphi \land [\mathit{SN}_3^b]\varphi),
\end{gather*} in the syntax of $\LAMB^\cup$, we can succinctly write an equivalent $[\mathit{SN}_1^a \cup \mathit{SN}_2^a][\mathit{SN}_1^b \cup \mathit{SN}_2^b \cup \mathit{SN}_3^b]\varphi$.

This increased succinctness, however, comes at a price. The complexity of the model checking problem for $\LAMB^\cup$ (and its fragments) jumps all the way to \Pspace-complete. Here we present a sketch of the argument.

\begin{theorem}
\label{lambcupMC}
    The model checking problem for $\LAMB^\cup$, $\mathsf{SLAMB}^\cup$ and $\mathsf{ALAMB}^\cup$ is \Pspace-complete.
\end{theorem}

\begin{proof}
The model checking algorithm for  $\LAMB^\cup$ extends the one for $\LAMB$ with one additional case (Algorithm \ref{pspaceMC}).

\begin{breakablealgorithm}
	\caption{An algorithm for model checking $\LAMB^\cup$}\label{pspaceMC} 
	%\small
 \footnotesize
	\begin{algorithmic}[1] 		
		\Procedure{MC$^\cup$}{$M, s, \varphi$}		

    
\Case {$\varphi = [\pi \cup \rho] \psi $ }
			\State{\textbf{return} $\textsc{MC$^\cup$} (M,s, [\pi]\psi)$ and $\textsc{MC$^\cup$} (M,s, [\rho]\psi)$}
		\EndCase
   \EndProcedure

	\end{algorithmic}
\end{breakablealgorithm}


%We note, however, that in the algorithm we keep updated models directly in memory, and each new model requires memory space that is bounded by $\mathcal{O}(|M|\cdot|\varphi|)$.
New models require space that is bounded by $|M|\cdot|\varphi|$.
There are at most $|\varphi|$ symbols in $\varphi$, and hence the total memory space required by the algorithm is bounded by $|M| \cdot |\varphi|^2$.



%Now we show the PSPACE-hardness of the model checking problem via the reduction from the satisfiability of quantified Boolean formulas (QBF-SAT). Without loss of generality, we assume that there are no free variables in QBFs, and that each variable is quantified only once. Given a QBF $\Psi := Q_1 x_1 ... Q_n x_n \Phi (x_1, ..., x_n)$ with $Q_i \in \{\exists, \forall\}$, we construct a CGM $M_s$ and a formula $\psi \in \LAMB$ (both of polynomial size with respect to $\Psi$), such that $\Psi$ is true if and only if $M_s \models \psi$.

\Pspace-hardness is shown via the reduction from the satisfiability of quantified Boolean formulas (QBFs). W.l.o.g., we assume that there are no free variables in QBFs, and that each variable is quantified only once. Given a QBF $\Psi := Q_1 x_1 ... Q_n x_n \Phi (x_1, ..., x_n)$ with $Q_i \in \{\exists, \forall\}$, we construct a CGM $M_s$ and a formula $\psi \in \LAMB^\cup$ (both of polynomial size with respect to $\Psi$), such that $\Psi$ is true if and only if $M_s \models \psi$.

%Let $\Psi := Q_1 x_1 ... Q_n x_n \Phi (x_1, ..., x_n)$ be a QBF. We construct a 
CGM $M = \lb S, \tau, L\rb$ is constructed over one agent and $\Act = \{a_1, ..., a_n\}$, where $S = \{t, s_1, ..., s_{n}\}$, $\tau(s_i, a_j) = s_i$ for all $s_i \in S$ and $a_j \in \Act$, $L(p_i) = \{s_i\}$ with $p_i \in P$, $L(x_i) = \{s_i\}$ with $x_i \in \Nom$, and $L(x_t) = \{t\}$. Intuitively, the CGM consists of $n+1$ states with reflexive loops for each action of the agent. Propositional variables $p_i$ and nominals $x_i$ are true only in the $i$th state. 

The translation of $\Psi$ into a formula of $\LAMB$ is done recursively as follows:
\begin{align*}
    \psi_0 &:= \Phi(\llb 1 \rrb p_1, ..., \llb 1 \rrb p_n) \\
    \psi_k &:=
    \begin{cases}
        [x_t \xrightarrow{a_k} x_k \cup x_t \xrightarrow{a_k} x_t] \psi_{k-1} &\text{if } Q_k = \forall\\ 
        \lnot [x_t \xrightarrow{a_k} x_k \cup x_t \xrightarrow{a_k} x_t] \lnot \psi_{k-1} &\text{if } Q_k = \exists\\
    \end{cases}\\
\psi &:= \psi_n
\end{align*}
Now we argue that 
\[Q_1 x_1 ... Q_n x_n \Phi (x_1, ..., x_n) \text{ is satisfiable iff } M_t \models \psi.\]
Redirection of a transition labelled with an action profile $a_k$ from state $t$ to state $s_k$ models setting variable $x_k$ to 1. Similarly, if the transition brings us from $t$ back to $t$, then variable $x_k$ is set to 0. Since the transition function is deterministic, the choice of truth values is unambiguous. 

To model quantifiers, we use the union operator. The universal quantifier $\forall x_k$ is translated into $[x_t \xrightarrow{a_k} x_k \cup x_t \xrightarrow{a_k} x_t] \psi_{k-1}$ meaning that no matter what the chosen truth-value of $x_k$ is, subformula $\psi_{k-1}$ is true. Similarly, for the existential quantifier $\exists x_k$, we state that there is a choice of the truth-value of $x_k$ such that $\psi_{k-1}$ is true. Once the valuation of all $x_k$ has been set, the evaluation of the Boolean subformula of the QBF corresponds to the reachability of $s_k$'s via an $a_k$-labelled transitions. 

In the hardness proof just presented %, out all the dynamic features of $\LAMB$, 
 we used only arrow change operators. Hence, we have shown that $\mathsf{ALAMB}^\cup$, and therefore $\LAMB^\cup$, have \Pspace-complete model checking problems.  

Now, we turn to the the case of substitutions. Let $$\Psi := Q_1 x_1 ... Q_n x_n \Phi (x_1, ..., x_n)$$ be a QBF. We construct a CGM $M = \lb S, \tau, L\rb$ over one agent and one action $\Act = \{a\}$, where $S = \{s\}$, $\tau(s , a) = s$, $L(p^i) = \{s\}$ with $p^1, ..., p^n \in P$, $L(\alpha) = \{s\}$ with $\alpha \in \Nom$. Intuitively, the CGM consists of a single state with a single reflexive loop. Propositional variables $p^i$'s, corresponding to QBF variables $x_i$'s, and nominal $\alpha$ is true in the single state. 

The translation of $\Psi$ into a formula of $\LAMB$ is as follows:

\begin{align*}
    \psi_0 &:= \Phi(p^1, ..., p^n) \\
    \psi_k &:=
    \begin{cases}
        [p^k_\alpha:= \top \cup p^k_\alpha := \bot] \psi_{k-1} &\text{if } Q_k = \forall\\ 
        \lnot [p^k_\alpha:= \top \cup p^k_\alpha := \bot] \lnot \psi_{k-1} &\text{if } Q_k = \exists\\
    \end{cases}\\
\psi &:= \psi_n
\end{align*}

%Now we argue that 
%\[Q_1 x_1 ... Q_n x_n \Phi (x_1, ..., x_n) \text{ is satisfiable if and only if } M_t \models \psi.\]
It is relatively straightforward to see that formula $\psi$ explicitly models quantifiers $Q_i$ for variables $p^i$. Then subformula  $\Phi(p^1, ..., p^n)$ is trivially evaluated in the single state of the model. 
\end{proof}
%Intuitively, we construct a CGM that consists of a single state with a single reflexive loop. Propositional variables $p^i$'s, corresponding to QBF variables $x_i$'s, and nominal $\alpha$ are true in the single state. 

%The translation of $\Psi$ into a formula of $\LAMB$ is as follows:
%\begin{align*}
%    \psi_0 &:= \Phi(p^1, ..., p^n) \\
%    \psi_k &:=
%    \begin{cases}
 %       [p^k_\alpha:= \top \cup p^k_\alpha := \bot] \psi_{k-1} &\text{if } Q_k = \forall\\ 
 %       \lnot [p^k_\alpha:= \top \cup p^k_\alpha := \bot] \lnot \psi_{k-1} &\text{if } Q_k = %\exists\\
 %   \end{cases}\\
%\psi &:= \psi_n
%\end{align*}

%Now we argue that 
%\[Q_1 x_1 ... Q_n x_n \Phi (x_1, ..., x_n) \text{ is satisfiable if and only if } M_t \models \psi.\]
%It is relatively straightforward to see that formula $\psi$ explicitly models quantifiers $Q_i$ for variables $p^i$. Then subformula  $\Phi(p^1, ..., p^n)$ is trivially evaluated in the single state of the model. Hence, the complexity of the model checking problem for $\mathsf{SLAMB}^\cup$ is also PSPACE-complete.


\iffalse

\suggestion{\textbf{OLD STUFF SAVED JUST IN CASE.} Now we turn to the complexity of the model checking problem for $\LAMB$ and its fragments. The problem consists in determining, for a CGM $M_s$ and a formula $\varphi$, whether $M_s \models \varphi$. We show, in particular, that the complexity of the model checking problem for the full language of $\LAMB$ is \Pspace-complete. The complexity remains the same for $\mathsf{SLAMB}$ and $\mathsf{ALAMB}$. However, we also demonstrate that for fragments of $\mathsf{SLAMB}$ and $\mathsf{ALAMB}$ without $\pi ; \rho$ and $\pi \cup \rho$ we have a significant improvement in the complexity (\Ptime-complete).

\subsection{Model checking $\LAMB$, $\mathsf{SLAMB}$, and $\mathsf{ALAMB}$}
The complexity of model checking $\ATL$ is known to be \Ptime-complete \cite{alur2002}, and it is easy to see that it remains the same also for $\mathsf{HATL}$. The algorithm for $\ATL$ uses function $Pre (C, Q)$ that computes for a given coalition $C \subseteq Agt$ and a set $Q \subseteq S$ the set of states, from which coalition $C$ can force the outcome to be in one of the $Q$ states. Function $Pre$ can be computed in polynomial time. We can use exactly the same algorithm for computing $Pre$ as for the standard $\ATL$, however in our case we will compute $Pre$ for not only the original model model $M$, but its updated versions as well. 

\begin{theorem}
    The model checking problem for $\LAMB$ is \Pspace-complete. 
\end{theorem}

\begin{proof}
Let $M_s$ be a finite CGM and $\varphi \in \LAMB$. To show that the model checking problem for $\LAMB$ is in \Pspace, we present Algorithm \ref{euclid}. For brevity, we omit all the $\ATL$ and hybrid logic cases as they are standard.

\begin{breakablealgorithm}
	\caption{An algorithm for model checking $\LAMB$}\label{euclid} 
	\small
	\begin{algorithmic}[1] 		
		\Procedure{MC}{$M, s, \varphi$}		
      %  \Case {$\varphi = p$}
       %     \State{\textbf{return} $s \in L(p)$}
        %\EndCase
        %\Case {$\varphi = \alpha$}
         %   \State{\textbf{return} $s \in L(\alpha)$}
        %\EndCase
       % \Case {$\varphi = @_\alpha \psi$}
       %     \State{\textbf{return} $\textsc{MC} (M, L(\alpha), \psi)$}
       % \EndCase
      %  \Case {$\varphi = \lnot \psi$}
       %     \State{\textbf{return} not $\textsc{MC} (M,s,\psi)$}
       % \EndCase
       % \Case {$\varphi = \psi \land \chi$}
      %      \State{\textbf{return} $\textsc{MC} (M,s,\psi)$ \text{ and } $\textsc{MC} (M,s,\chi)$}
      %  \EndCase
      %  \Case {$\varphi = \llb C \rrb \X \psi$}
      %      \State{\textbf{return} $s \in Pre(C, \{t \in S | \textsc{MC}(M, t, \psi)\})$}
      %  \EndCase
		\Case {$\varphi = [\dagger] \psi$ with $\dag\in \{p_\alpha:=\psi, \alpha \xrightarrow{A} \beta, \lcirc{\alpha}\}$}
			\State{\textbf{return} $\textsc{MC} (M^{\dag}, s, \psi)$}
		\EndCase
		\Case {$\varphi = [\pi \cup \tau] \psi$}	
			\State{\textbf{return} $\textsc{MC} (M, s,  [\pi]\psi)$ and $\textsc{MC} (M, s, [\tau]\psi)$}
		\EndCase
  		\Case {$\varphi = [\pi;\tau] \psi$}	
			\State{\textbf{return} $\textsc{MC} (M, s,  [\pi][\tau]\psi)$}
		\EndCase
   \EndProcedure
   
	\end{algorithmic}
\end{breakablealgorithm}

Termination and correctness of the algorithm follow straightforwardly from the finite size of the input and the definition of semantics. All the standard $\ATL$ and hybrid cases can be performed in polynomial time and thus space. Without giving an explicit algorithm for the construction of updated models, we note that each new model requires memory space that is bounded by $\mathcal{O}(|M|\cdot|\varphi|)$. There are at most $|\varphi|$ symbols in $\varphi$, and hence the total space required by the algorithm is bounded by $\mathcal{O} (|M| \cdot |\varphi|^2)$.

Now we show the \Pspace-hardness of the model checking problem via the reduction from the satisfiability of quantified Boolean formulas (QBF-SAT). Without loss of generality, we assume that there are no free variables in QBFs, and that each variable is quantified only once. Given a QBF $\Psi := Q_1 x_1 ... Q_n x_n \Phi (x_1, ..., x_n)$ with $Q_i \in \{\exists, \forall\}$, we construct a CGM $M_s$ and a formula $\psi \in \LAMB$ (both of polynomial size with respect to $\Psi$), such that $\Psi$ is true if and only if $M_s \models \psi$.

Let $\Psi := Q_1 x_1 ... Q_n x_n \Phi (x_1, ..., x_n)$ be a QBF. We construct a CGM $M = \lb S, \tau, L\rb$ over one agent and $\Act = \{a_1, ..., a_n\}$, where $S = \{t, s_1, ..., s_{n}\}$, $\tau(s_i, a_j) = s_i$ for all $a_j \in \Act$, $L(p_i) = \{s_i\}$ with $p_i \in P$, $L(x_i) = \{s_i\}$ with $x_i \in \Nom$, and $L(x_t) = \{t\}$. Intuitively, the CGM consists of $n+1$ states with reflexive loops for each action of the agent. Propositional variables $p_i$ and nominals $x_i$ are true only in the $i$th state. 

The translation of $\Psi$ into a formula of $\LAMB$ is done recursively as follows:

\begin{align*}
    \psi_0 &:= \Phi(\llb 1 \rrb p_1, ..., \llb 1 \rrb p_n) \\
    \psi_k &:=
    \begin{cases}
        [x_t \xrightarrow{a_k} x_k \cup x_t \xrightarrow{a_k} x_t] \psi_{k-1} &\text{if } Q_k = \forall\\ 
        \lnot [x_t \xrightarrow{a_k} x_k \cup x_t \xrightarrow{a_k} x_t] \lnot \psi_{k-1} &\text{if } Q_k = \exists\\
    \end{cases}\\
\psi &:= \psi_n
\end{align*}

Now we argue that 
\[Q_1 x_1 ... Q_n x_n \Phi (x_1, ..., x_n) \text{ is satisfiable if and only if } M_t \models \psi.\]
Redirection of a transition labelled with an action profile $a_k$ from state $t$ to state $s_k$ models setting variable $x_k$ to 1. Similarly, if the transition brings us from $t$ back to $t$, then variable $x_k$ is set to 0. Since the transition function is deterministic, the choice of truth values is unambiguous. 

To model quantifiers, we use nondeterministic choice. The universal quantifier $\forall x_k$ is translated into $[x_t \xrightarrow{a_k} x_k \cup x_t \xrightarrow{a_k} x_t] \psi_{k-1}$ meaning that no matter what the chosen truth-value of $x_k$ is, subformula $\psi_{k-1}$ is true. Similarly, for the existential quantifier $\exists x_k$, we state that there is a choice of the truth-value of $x_k$ such that $\psi_{k-1}$ is true. Once the valuation of all $x_k$ has been set, the evaluation of the Boolean subformula of the QBF corresponds to the reachability of $s_k$'s via an $a_k$-labelled transitions. 

In the hardness proof just presented, out all the dynamic features of $\LAMB$, we used only arrow change operators. An alternative proof of hardness that uses only variable substitutions is as follows.  

Let $\Psi := Q_1 x_1 ... Q_n x_n \Phi (x_1, ..., x_n)$ be a QBF. We construct a CGM $M = \lb S, \tau, L\rb$ over one agent and one action $\Act = \{a\}$, where $S = \{s\}$, $\tau(s , a) = s$, $L(p^i) = \{s\}$ with $p^1, ..., p^n \in P$, $L(\alpha) = \{s\}$ with $\alpha \in \Nom$. Intuitively, the CGM consists of a single state with a single reflexive loop. Propositional variables $p^i$'s, corresponding to QBF variables $x_i$'s, and nominal $\alpha$ are true in the single state. 

The translation of $\Psi$ into a formula of $\LAMB$ is as follows:

\begin{align*}
    \psi_0 &:= \Phi(p^1, ..., p^n) \\
    \psi_k &:=
    \begin{cases}
        [p^k_\alpha:= \top \cup p^k_\alpha := \bot] \psi_{k-1} &\text{if } Q_k = \forall\\ 
        \lnot [p^k_\alpha:= \top \cup p^k_\alpha := \bot] \lnot \psi_{k-1} &\text{if } Q_k = \exists\\
    \end{cases}\\
\psi &:= \psi_n
\end{align*}

%Now we argue that 
%\[Q_1 x_1 ... Q_n x_n \Phi (x_1, ..., x_n) \text{ is satisfiable if and only if } M_t \models \psi.\]
It is relatively straightforward to see that formula $\psi$ explicitly models quantifiers $Q_i$ for variables $p^i$. Then subformula  $\Phi(p^1, ..., p^n)$ is trivially evaluated in the single state of the model. 
\end{proof}

In the hardness proof above, we argued that having the choice operator with either only arrow changes or only substitutions is enough for the translation. Hence, the same complexity result holds for both $\mathsf{SLAMB}$ and $\mathsf{ALAMB}$. 

\begin{corollary}
The model checking problem for $\mathsf{SLAMB}$ and $\mathsf{ALAMB}$ is \Pspace-complete.
\end{corollary}

\subsection{Model checking fragments of $\mathsf{SLAMB}$ and $\mathsf{ALAMB}$}
In the previous section, we have seen that the model checking problem for both $\mathsf{SLAMB}$ and $\mathsf{ALAMB}$ is \Pspace-complete. Yet, one cannot help but feel that both fragments are somewhat `easier' than the full language of $\LAMB$. Indeed, the hardness proof relied on usage of the choice operator to model quantifiers, and in this section we show that fragments of $\mathsf{SLAMB}$ and $\mathsf{ALAMB}$ without $\pi; \tau$ and $\pi \cup \tau$, denoted as $\mathsf{SLAMB}^\dag$ and $\mathsf{ALAMB}^\dag$, have a \Ptime-complete model checking problem. }
\fi