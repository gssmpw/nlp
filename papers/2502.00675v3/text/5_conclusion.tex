\section{Conclusion}
In this paper, we proposed \textbf{ReFoRCE}, a self-refinement framework for addressing the challenges of real-world Text-to-SQL tasks as exemplified by the Spider 2.0 dataset. By introducing techniques such as table information compression, format restriction, iterative column exploration, and CTE-based refinement, ReFoRCE effectively handles multiple SQL dialects, nested columns, and complex data types. Our approach achieved state-of-the-art performance, with scores of 31.26 on Spider 2.0-Snow and 30.35 on Spider 2.0-Lite, significantly outperforming the previous best results by Spider-Agent. These findings highlight the robustness and reliability of ReFoRCE in tackling real-world complexities.

Despite its strong performance, ReFoRCE has several limitations. The lossy compression of table information restricts its ability to handle very large table contexts, and the current simple exploration strategy struggles with ambiguous columns and local optima. Moreover, as ReFoRCE primarily focuses on preprocessing and self-refinement, it does not introduce significant improvements in reasoning capabilities, which are critical for handling more complex queries.

To address these limitations, future work will explore schema-linking techniques to enhance contextual understanding and advanced column exploration and reasoning strategies such as Monte Carlo Tree Search (MCTS) and Process Reward Model (PRM) or pure Reinforcement Learning (RL) \citep{guo2025deepseek} to better handle ambiguous columns and improve optimization. By addressing these challenges, we aim to advance the applicability of ReFoRCE to broader Text-to-SQL tasks and set new benchmarks for real-world database interaction.