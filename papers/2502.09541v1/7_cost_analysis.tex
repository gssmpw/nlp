\section{Price Performance Analysis} \label{section:cost_analysis}
% \label{sec:cost-eff}
This section evaluates the price performance of our proposed system compared to a baseline CPU-only solution.
A key assumption we take is that all input and output data is stored on the CPU-side DRAM without caching on the GPU.
The price of \THISWORK\ is 
$$
C_{\text{\THISWORK}} = C_\text{raw} + C_{\text{raw}} \times \text{tax} = C_\text{raw} +  C_{\text{raw}} \times (\text{tax}_{\text{t}} + 3 \times \text{tax}_{\text{f}}).
$$
$C_{\text{raw}}$ is the raw price of a GPU.
The term $\text{tax}$ is introduced to account for the tax paid for enhanced IO throughput.
This factor takes the interference on the target GPU, $\text{tax}_{\text{t}}$, and the forwarding GPUs,  $\text{tax}_{\text{f}}$, into account, which are calculated as $ \textit{tax}_{\textit{t | f}} = 1 - \frac{1}{\text{slowdown}_\text{t | f}} $.
% $$
% \text{tax}_{\text{t | f}} = 1 - \frac{1}{\text{slowdown}_\text{t | f}}
% $$
The price performance of GPU-accelerated \THISWORK\ over a CPU baseline is calculated as $ \textit{price performance} = \frac{C_\text{cpu}}{C_\text{\THISWORK} + C_\text{cpu}} \times \text{speedup} $.
To calculate the price of renting a GPU, we approximate the price with NVIDIA's A100 GPU (as AMD MI100 GPUs are not available from the cloud providers).
Arguably, A100 is a more powerful GPU with a higher price, making our price estimation \textit{conservative}.
% A100 is available through AWS \texttt{p4d.24xlarge} and \texttt{r5dn.metal} instances.
% We calculate the cost of an A100 GPU, a more powerful GPU than the MI100 we use, from AWS \texttt{p4d.24xlarge} and \texttt{r5dn.metal} instances.
The price of renting an A100 GPU is estimated using AWS \texttt{p4d.24xlarge} and \texttt{r5dn.metal} instances.
\texttt{p4d.24xlarge} offers 8 A100 GPUs with a more powerful CPU and larger DRAM capacity than \texttt{r5dn.metal} with no GPU.
Therefore, the upper-bound price of a single A100 GPU is estimated using
$$
C_{\texttt{raw-A100}} = \frac{C_{\text{p4d.24xlarge}} - C_{\text{r5dn.metal}}}{8} = \frac{32.773 - 8.016}{8} \approx 3.1 (\$/\text{h})
$$
Because the CPU on \texttt{r5dn.metal} is weaker than the CPU we use, we use the $C_\text{cpu} = C_\text{r5dn.metal}$ to calculate lower bounds of the baseline price, making our estimation even more conservative.
When the background is running \texttt{llama3} decode without batching that introduces the highest degree of interference, the price performance of \THISWORK\ compared to a CPU for sort, hash join, and SSB is 3.9$\times$, 1.5$\times$, and 2.3$\times$, respectively.
When the background is running \texttt{SD3}, the price performance of \THISWORK\ for sort, hash join, and SSB is 4.2$\times$, 1.6$\times$, and 2.4$\times$, respectively.
On average, \THISWORK\ achieves 2.5$\times$ better price performance over the CPU-based DuckDB.
% Note that the estimation here is conservative and the cost efficiency may be more significant with more detailed price tag information.
Notably, the discussed price estimations are conservative. In practice, we expect \THISWORK\ to achieve even better price performance.