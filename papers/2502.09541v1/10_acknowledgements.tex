\begin{acks}
This work was supported in part by Advanced Micro Devices, Inc. under the ``Funding Academic Research" and the AMD AI \& HPC Cluster Program.
\end{acks}