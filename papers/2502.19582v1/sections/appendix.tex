\clearpage
\appendixpage
\addappheadtotoc
\setcounter{table}{0}
\setcounter{figure}{0}
\renewcommand{\thetable}{\Alph{section}\arabic{table}}
\renewcommand{\thefigure}{\Alph{section}\arabic{figure}}

%%%%
\section{Sahara Task Clusters}\label{appendix: task_cluster_detail}
\subsection{Text Classification}
\paragraph{Cross-Lingual Natural Language Inference.}  This task involves the classification of 2 given sentences—a premise and a
hypothesis into entailment, neutral, or contradiction semantic relation \cite{adelani2024irokobench}. \citet{adelani2024irokobench} introduced AfriXNLI, a benchmark for evaluating 15 African Languages on Natural Language Inference (NLI). The languages were translated from the English portion of XNLI \cite{conneau2018xnli}.

\paragraph{Language Identification.} We include language identification datasets across $518$ African languages \cite{adebara2022afrolid}. The dataset includes data from multiple domains including education, government, health care, news, and religion. Additional statistics of the data is available in Table~\ref{tab:data_stats}. We include language identification tasks to investigate \textbf{(1.)} the capabilities of language models in correctly generating the appropriate language names of an input text and \textbf{(2.)} Correctly generate an additional sentence in a specific language. We include these two approaches because many languages have multiple names \cite{chen-etal-2024-fumbling} and a LM's inability to appropriately name a language does not implicate a lack of information about the language. 

\paragraph{News Classification.} This task involves automatically categorizing news articles into predefined categories or topics based on their content. The goal is to leverage machine learning and NLP techniques to understand and organize large volumes of news data, making it easier for users to access relevant information. News classification plays a crucial role in information retrieval, content recommendation, and other applications that involve organizing and categorizing textual data. The news classification cluster consists of data from four languages including Amharic~\cite{azime2021amharic}, Kinyarwanda~\cite{niyongabo-etal-2020-kinnews}, Kirundi~\cite{niyongabo-etal-2020-kinnews}, and Swahili~\cite{davis_david_2020_4300294, davis_david_2020_5514203}. Table~\ref{tab:data_stats} provides additional details of the data in this cluster. 

\paragraph{Sentiment Analysis.} Sentiment analysis is crucial in gaining insights into public opinion, customer feedback, and user sentiments across various platforms. The sentiment analysis cluster consists of the Bambara Sentiment dataset \cite{diallo2021bambara}, YOSM--a Sentiment Corpus for Movie Reviews in Y{o}r\`{u}b\'{a}~\cite{shode_africanlp}, and the Nigerian Pidgin sentiment dataset~\cite{oyewusi2020semantic}, respectively. Further details about the statistical composition of this data is available in Table~\ref{tab:data_stats}.


\paragraph{Topic Classification.} The primary goal for this task is to automatically categorize the content of the text based on its theme. Topic classification is widely used in various applications, such as document organization, content recommendation, and information retrieval. We include topic classification datasets for Y{o}r\`{u}b\'{a} and Hausa~\cite{hedderich-etal-2020-transfer}.


\subsection{Text Generation}


\paragraph{Machine Translation.} A robust benchmark for MT tasks is essential. \ourbenchmark, in its quest to effectively assess machine translation for African languages, selectively incorporates datasets containing diverse African languages into its benchmark. Specifically, \ourbenchmark~includes datasets such as Pidgin-UNMT\footnote{\href{https://github.com/keleog/PidginUNMT}{PidginUNMT GitHub Link}}~\cite{ogueji2019pidginunmt}, Afro-MT\footnote{\href{https://github.com/machelreid/afromt}{AfroMT GitHub Link}}~\cite{reid-etal-2021-afromt}, Lafand-MT~\cite{adelani-etal-2022-thousand}, SALT\footnote{\href{https://github.com/SunbirdAI/salt}{SALT GitHub Link}} \cite{akera2022salt}, and HornMT, thus encompassing a total of $45$ language pairs.

\paragraph{Paraphrase.} Paraphrasing is a core task in NLG that revolves around the comprehension and generation of text that conveys the same meaning while possibly exhibiting differences in structure, phrasing, or style~\cite{10.1162/tacl_a_00542}. \ourbenchmark~incorporates the TaPaCo dataset~\cite{scherrer_yves_2020_3707949}, which is publicly accessible and covers a total of $73$ languages, including four of African origin. These African languages consist of Kirundi, Amazigh, a macro-language Berber, and Afrikaans.

\paragraph{Summarization.} Summarization is the task of creating a condensed version of a text that retains its core ideas and essential information~\cite{nallapati-etal-2016-abstractive}. In the context of \ourbenchmark, our focus primarily centers on a subset of the abstractive summarization dataset, XL-SUM~\cite{hasan-etal-2021-xl}, which includes several African languages: Amharic, Hausa, Igbo, Kirundi, Oromo, Pidgin, Somali, Swahili, Tigrinya, and Yoruba. Additionally, we have extended our dataset collection efforts by scraping non-overlapping summarization datasets from BBC and Voice of Africa (VoE) websites, specifically targeting three African languages: Hausa, Ndebele, and Swahili.

\paragraph{Title Generation.} The task of title generation in NLG involves generating a concise and meaningful title or headline for a given article. This task shares its dataset source with summarization, drawing from XL-SUM, and also includes dataset extensions from BBC and VoE with focus on enabling zero-shot title generation.\\

\subsection{Multiple-Choice, Comprehensive and Reasoning}

This task cluster majorly addresses \emph{Question Answering} type questions. The QA tasks in this cluster are more mathematical-reasoning and reading-comprehension-oriented. 


\paragraph{Mathematical Word Problems.} \cite{adelani2024irokobench} introduced AfriMGSM, a QA dataset that contains human-written grade school mathematical word problems. Their dataset covers 15 African Languages, and the translation of each question to English for each language for in-language evaluation of LLMs. In this task, given a word problem, the LLM is expected to return a numerical value as output.

\paragraph{General Knowledge QA.} \cite{adelani2024irokobench} introduced AfriMMLU, a multi-choice QA dataset covering 5 areas of knowledge in 15 African Languages. The knowledge areas are high-school
geography, high-school microeconomics, elementary mathematics, international law), and global facts. In this task given a question in any of the aforementioned areas, and 4 options, the LLM is expected to choose the right option.

\paragraph{Reading Comprehension QA.} In this task, the LLMs are evaluated based on their ability to understand information in a given article. \cite{belebele_2024} introduced a parallel reading comprehension dataset covering 122 languages, 23 of which are native to Africa. The dataset comprises of short paragraphs accompanied by multi-choice Questions, whose answers can be inferred from accompanying paragraphs.

\paragraph{Context-based Question-Answering.} QA represents another vital facet of NLG, focusing on a model's capability to provide answers based on integrated knowledge. \ourbenchmark~relies on the gold passage dataset from the multilingual TYDIA\footnote{\href{https://github.com/google-research-datasets/tydiqa}{TyDiQA GitHub Link}} QA resource~\cite{clark-etal-2020-tydi}. This dataset offers concise questions with precisely one corresponding answer, enabling rigorous evaluation of QA performance.


\subsection{Tokens-level Classification}

\paragraph{Named Entity Recognition.} We incorporate NER datasets across $30$ languages. We use Distance Supervision NER (DS NER) Data \cite{hedderich-etal-2020-transfer}, MasakhaNER~\cite{adelani-etal-2022-masakhaner}, WikiAnn~\cite{rahimi-etal-2019-massively}, Yoruba-Twi NER~\cite{alabi2020massive}, and multiple NER data from \href{https://repo.sadilar.org/handle/20.500.12185/7}{SADiLaR}. Additional details about the datasets are available in Table~\ref{tab:data_stats}.

\paragraph{Phrase Chunking.} This task involves identifying and grouping together consecutive words or tokens into meaningful syntactic units, known as phrases. These phrases can include noun phrases, verb phrases, prepositional phrases, and other grammatical structures. The goal of phrase chunking is to analyze the grammatical structure of a sentence and extract higher-level syntactic units for better understanding. The phrase chunking cluster consists of data for ten Indigenous languages of South Africa (see Table \ref{tab:data_stats}). The data has annotations for noun, verb, adjective, adverbial, and prepositional phrase chunks. Words not belonging to these phrase types are labelled with the tag \textit{O}. 

\paragraph{Part of Speech Tagging.} This classification task involves assigning grammatical categories or labels (such as noun, verb, adjective) to each word in a sentence. The primary goal of POS tagging is to analyze the syntactic structure of a sentence and understand the role of each word within it. POS tagging is crucial for various NLP applications, as it provides a foundation for more advanced linguistic analysis. We include datasets for Igbo taken from IgboNLP~\cite{10.1145/3314942}.




\section{Benchmark}
Table~\ref{tab:benchmarkcompare}~shows the comparison between \ourbenchmark~with others
\begin{table*}[h]
\scriptsize
\centering
\resizebox{0.85\textwidth}{!}{%
\begin{tabular}{clp{3.5cm}ccc}
\toprule
\textbf{Category} & \textbf{Benchmark} & \textbf{Reference}& \textbf{Lang/Total} & \textbf{Tasks} \\ \toprule
\multirow{9}{*}{\rotatebox[origin=c]{90}{\textbf{Multilingual}}} & FLoRES200 & \cite{nllb2022} & 52/200 & 1 \\
& GEM\textsubscript{v1} & \cite{gehrmann-etal-2021-gem} & 10/52 & 5 \\
& GEM\textsubscript{v2} & \cite{gehrmann-etal-2021-gem} & 10/52 & 9\\
& NLLB M.D. & \cite{nllb2022} & 2/8 &  1 \\
& NLLB S.D. & \cite{nllb2022} & 2/8 & 1 \\
& SIB-200 & \cite{adelani-etal-2024-sib} & 46/200 & 1\\
& Toxicity200 & \cite{nllb2022}& 50/200 & 1 \\
& XGLUE & \cite{liang-etal-2020-xglue} & 1/19 & 3\\
& XTremE & \cite{hu2020xtreme} & 2/40 & 4\\
\midrule
\multirow{9}{*}{\rotatebox[origin=c]{90}{\textbf{Language-specific}}} & BanglaNLP\textsubscript{v1} & \cite{bhattacharjee-etal-2023-banglanlg} & 1 & 6 \\
& CLUE & \cite{xu2020clue} & 1 & 6 \\
& CUGE & \cite{yao2021cuge} & 1 & 8 \\
& Dolphin & \cite{nagoudi-etal-2023-dolphin} & 1 & 13 \\
& FLUE & \cite{le2020flaubert} & 1 & 5 \\
& IndoNLU & \cite{wilie2020indonlu} & 1 & 5 \\
& IndoNLG & \cite{guntara-etal-2020-benchmarking} & 1 & 6 \\
& JGLUE & \cite{kurihara2022jglue} & 1 & 3 \\
& KorNLU & \cite{ham2020kornli} & 1 & 2 \\
& LOT & \cite{10.1162/tacl_a_00469} & 1 & 4 \\
& ORCA & \cite{elmadany-etal-2023-orca} & 1 & 7 \\
& PhoMT & \cite{doan-etal-2021-phomt} & 1 & 1 \\
\midrule
\multirow{9}{*}{\rotatebox[origin=c]{90}{\textbf{African}}} & AfriSenti & \cite{muhammad-etal-2023-afrisenti}& 14 & 1 \\
& AfroMT & \cite{reid21afromt} & 8/8 & 1 \\
& AfroNLG & \cite{adebara2024cheetah} & 517/517 & 6 \\ 
& AfroNLU & \cite{adebara-etal-2023-serengeti} & 28/28 & 7 \\
& \href{https://github.com/asmelashteka/HornMT}{Horn-MT} & $-$ & 3/3 & 1 \\
& Iroko-Bench & \cite{adelani2024irokobench} & 16/16 & 3\\
& Mafand-MT & \cite{adelani-etal-2022-thousand} & 17/17 & 1 \\ 
& Menyo-20k & \cite{adelani-etal-2021-effect} & 1/1 & 1 \\
& Naija-senti & \cite{muhammad-etal-2022-naijasenti} & 4/4 & 1 \\ \cmidrule{2-5}
&\includegraphics[scale=0.014]{Images/sahara_logo.png}  \ourbenchmark & ours & 517/517 & 16 \\ \bottomrule
\end{tabular}%
}
\caption{A Comparison of \ourbenchmark~with other Benchmarks. \textbf{M.D:} Multilingual domain, \textbf{S.D:} Seed Data.}
\label{tab:benchmarkcompare}
\end{table*}
% \textbf{MT}: Machine translation, \textbf{QA:} Question Answering, \textbf{CS:} Code-Switching, \textbf{TG}: Title Generation, \textbf{SUM}: Summarization, \textbf{PARA}: Paraphrase, \textbf{NER}: Named Entity Recognition, \textbf{POS}: Part-Of-Speech Tagging, \textbf{MLQA:} Multilingual Question Answering, \textbf{PAWS-X}: Parallel Aggregated Word Scrambling for Cross-Lingual Understanding, \textbf{XNLI}: Cross-Lingual Natural Language Interference, \textbf{NC}: News Classification, \textbf{QADSM}: Query-AD Matching, \textbf{WPR:} Web Page Ranking, \textbf{QAM}: QA Matching, \textbf{NTG}: News Title Generation, \textbf{BG}: WikiBio Biography Generation, and \textbf{HG:} Headline Generation. \textbf{SD}: Seed Data, \textbf{MD}: Multi Domain. \textbf{DRG:} Dialogue Response Generator, \textbf{DT:} Data-to-Text, \textbf{RES:} Reasoning, \textbf{TS:} Text Summarization, \textbf{SMP:} Text Simplification, \textbf{PPH:} Paraphrase, \textbf{SLG:} Slide Generation.}

% \begin{table*}[!ht]
% \scriptsize
% \centering
% \resizebox{0.75\textwidth}{!}{%
% \begin{tabular}{clp{3.5cm}ccc}
% \toprule
% \textbf{Benchmark} & \textbf{Reference}& \textbf{Lang/Total} & \textbf{\# Tasks} & \textbf{Tasks} \\ \toprule                                      
% AfriSenti   & \cite{muhammad-etal-2023-afrisenti} & 14/14 & 1  & Sentiment Analysis  \\
% AfroMT & \cite{reid21afromt} & 8/8  & 1  & Machine Translation \\
%  \href{https://github.com/asmelashteka/HornMT}{Horn-MT} & - & 3/3 & 1 & Machine Translation \\
% IrokoBench & \cite{adelani2024irokobench} & 16/16 & 3/3 & \begin{tabular}[c]{@{}l@{}}Natural Language Inference, Multi-Choice Question \\ answering, Question Answering\end{tabular} \\
% MafandMT & \cite{adelani-etal-2022-thousand} & 17/20  & 1 & Machine Translation \\
% Menyo20k & \cite{adelani-etal-2021-effect} & 1/1 & 1 & Machine Translation \\
% NaijaSenti & \cite{muhammad-etal-2022-naijasenti} & 4/4 & 1 & Sentiment Analysis  \\
% \bottomrule
% \includegraphics[scale=0.014]{Images/sahara_logo.png}  \ourbenchmark & ours & 40/40 & 11 & \begin{tabular}[c]{@{}l@{}}Named Entity Recognition, Phrase Chunking,\\ Part of Speech Tagging, News Classification,\\ Sentiment Analysis, Topic Classification, Machine\\ Translation, Paraphrase, Question Answering, \\ Summarization, Topic Generation\end{tabular} \\ \bottomrule
% \end{tabular}%
% }
% \caption{A Comparison of \ourbenchmark~with other Benchmarks. \textbf{M.D:} Multilingual domain, \textbf{S.D:} Seed Data.}
% \label{tab:benchmarkcompare}
% \end{table*}

\section{Few-shot prompt example}
Figure~\ref{fig:few-shot} shows an example for MT task. 
\begin{figure*}[h]
  \centering
  \includegraphics[width=\textwidth]{Images/prompt_eg.PNG}
  \caption{Few-shot prompting example for the machine translation task. The prompt provides five examples of the specific task for context and alignment. }
  \label{fig:few-shot}
\end{figure*}

\section{Dataset Description}
Table~\ref{tab:description} describes all datasets included in \ourbenchmark


\begin{table*}[!ht]
    \centering  
    \small
    \resizebox{0.85\textwidth}{!}{
        \begin{tabularx}{\textwidth}{XXX}
        \toprule
        Dataset & Brief Description & African Language(s) \\
        \midrule
        
        MasakhaNER \cite{Adelani2021MasakhaNERNE}, \cite{Adelani2022MasakhaNER2A}  & Curated for Named Entity Recognition for several African Languages  & amh, bam, bbj, ewe, fon, hau, ibo, kin, lug, luo, mos, nya, pcm, sna, swa, tsn, twi, wol, xho, yor, zul \label{masakhanener}\\ \midrule
        MasakhaPOS \cite{Dione2023MasakhaPOSPT} & Curated from News sources and publicly available datasets for Part-Of-Speech Tagging & bam, bbj, ewe, fon, hau, ibo, kin, lug, luo, mos, nya, pcm, sna, swa, tsn, twi, wol, xho, yor, zul   \\ \midrule
        MAFAND-MT \cite{adelani-etal-2022-thousand} & African news corpus for Machine Translation  &  bam, bbj, ewe, fon, hau, ibo, lug, luo, mos, swa, tsn, twi, wol, xho, yor, zul  \\ \midrule
        XL-Sum \citet{hasan-etal-2021-xl} & Curated for Summarization and Title Generation &  amh, ibo, orm, run, swa, yor, hau, pcm, som, tir \\ \midrule
        YorubaSenti \cite{yorubasenti} & Curated from comments on movies on YouTube
 & yor \\ \midrule
         TaPaCo \cite{scherrer_yves_2020_3707949}  & Developed to provide example sentences and translations for particular linguistic constructions and words  & afr, ber, run , zgh \\ \midrule
         TyDi QA \cite{clark-etal-2020-tydi} & Curated for Information-Seeking QA tasks & swa  \\ \midrule
         HornMT & Parallel MT dataset curated for languages in Horn of Africa  & aaf, amh, orm, som, tir \\ \midrule
         yoruba-twi-ner \cite{alabi-etal-2020-massive} & Curated for evaluating embeddings for low-resource languages & yor, twi\\ \midrule
         KINNEWS and KIRNEWS \cite{niyongabo-etal-2020-kinnews} & Curated for multi-class classification of news articles & kin, run \\ \midrule
         AmharicNews \cite{azime2021amharic}& Curated for text classification from news articles &amh \\ \midrule
         SwahiliNews v0.2 \cite{davis_david_2020_4300294}& Curated for text classification from news articles & swa\\ \midrule
         Bambara \cite{diallo2021bambara} & Curated for Sentiment Analysis through web crawling & bam \\ \midrule
         IgboNLP\cite{igbonlp} & Contains text corpus, POS tagset and POS-tagged subcorpus & ibo \\ \midrule
         yosm \cite{shode2022yosm} & Curated from movie reviews for Sentiment Analysis & yor \\ \midrule
         pidgin-tweet \cite{oyewusi2020semantic}& Curated for SA from tweets & pcm \\ \midrule
         WikiAnn \cite{pan-etal-2017-cross}, \cite{rahimi-etal-2019-massively}  & & afr, amh, ibo, mlg, kin, som, swa, yor \\ \midrule
         HausaTopic,YorubaTobic \cite{hedderich-etal-2020-transfer} & Curated for Topic Classification & hau, yor \\ \midrule
          SaDiLAR \cite{eiselen-2016-government} & Curated for NER of 10 South African Languages & afr, nbl, nso, sot, ssw, tsn, tso, ven, xho, zul \\ \midrule
         % AfroLID \cite{adebara2022afrolid} & Curated for Language Identification & 517 African Languages\\ \midrule
         AfriSenti \cite{muhammad-etal-2023-afrisenti} & Curated for African tweet sentiment analysis  & amh, ALG-ara, hau, kin, ibo, MOR-ara, MOZ-por, pcm,form, swa, tir, twi,tso,yor \\ \midrule
         Afro-MT \cite{reid-etal-2021-afromt} & Benchmark for MT of 8 African Languages & afr, bem, lin, run, sot, swa, xho, zul \\ \midrule
         PidginUNMT \cite{ogueji-etal-2021-small} & & pcm\\ \midrule
         SALT \cite{akera2022salt}& Curated for Parallel MT of 5 Ugandan Languages & ach, lgg, lug, nyn, teo   \\ \midrule
         Cheetah \cite{adebara2024cheetah} & Summarization test set & hau, nde, swa \\ \midrule
         Cheetah \cite{adebara2024cheetah} & Title generation test set &  amh, gag (zero-shot), hau, ibo, pcm, som, swa, tir, yor, kin (zero-shot), afr, mlg (zero-shot), orm, nde (zero-shot), sna(zero-shot)\\ \midrule
         NCHLT-NER \cite{eiselen-2016-government} & Named Entity Recognition & afr, nbl, nso, sot, ssw, tsn, tso, ven, xho, zul  \\ \midrule
         BeleBele \cite{belebele_2024} & Reading Comprehension MCQA & afr
amh, bam, fuv, gaz, hau, ibo,kea,kin, lin, lug, luo,nso, nya, sna ,som, sot, ssw, swh, tir, tsn, tso, wol, xho, yor,zul  \\ \midrule
IrokoBench \cite{adelani2024irokobench} & MGSM,MMLU,XNLI & amh, ewe, hau, ibo,kin, lin, lug, orm, sna ,sot,  swa, tir, wol, xho, yor,zul  \\ \bottomrule
        \end{tabularx}%
    }
    \caption{A description of all datasets included in \textit{Sahara}. \textbf{ALG:} Algerian, \textbf{MOR:} Moroccan, \textbf{MOZ:} Mozambican.}
    \label{tab:description}
\end{table*}








% Please add the following required pacages to your document preamble:
% \usepacage{bootabs}
% \usepacage{multirow}
% \usepacage{graphicx}
% \usepacage[normalem]{ulem}
% \useunder{\uline}{\ul}{}
\begin{table*}[]
\centering
\resizebox{0.55\textwidth}{!}{%
\begin{tabular}{@{}l|l|l|r@{}}
\toprule
\textbf{Task} & \textbf{Source} & \textbf{Language} & \textbf{Train/ Dev/ Test} \\ \midrule
% Cloze test & \cite{adebara2024cheetah} & 517 languages   & 103.4K/ 25.9K/ 51.7K \\ \midrule
% \multirow{2}{*}{LID} & \cite{adebara2022afrolid} & & 2.5M/ 25.9K/ 51.4K\\ \cmidrule{2-4}
% & \cite{muhammad2023afrisenti} & & \\ \midrule
\multirow{26}{*}{MT} & \multirow{15}{*}{\cite{adelani-etal-2022-thousand}} & eng-hau & 5.9K/ 1.3K/ 1.5K \\
&  & eng-ibo & 6.9K/ 1.5K/ 1.4K \\
 &  & eng-lug & 4.1K/ 1.5K/ 1.5K \\
 &  & eng-pcm & 4.8K/ 1.5K/ 1.6K \\
 &  & eng-swa & 30.8K/ 1.8K/ 1.8K \\
 &  & eng-tsn & 2.1K/ 1.3K/ 1.5K \\
 &  & eng-twi & 3.3K/ 1.3K/ 1.5K \\
 &  & eng-yor & 6.6K/ 1.5K/ 1.6K \\
 &  & eng-zul & 3.5K/ 1.5K/ 1.0K \\
 &  & fra-bam & 3.0K/ 1.5K/ 1.5K \\
 &  & fra-bbj & 2.2K/ 1.1K/ 1.4K \\
 &  & fra-ewe & 2.0K/ 1.4K/ 1.6K \\
 &  & fra-fon & 2.6K/ 1.2K/ 1.6K \\
 &  & fra-mos & 2.5K/ 1.5K/ 1.6K \\
 &  & fra-wol & 3.4K/ 1.5K/ 1.5K \\ \cmidrule(l){2-4} 
 & \multirow{8}{*}{\cite{reid-etal-2021-afromt}} & eng-afr & 25.8K/ 3.2K/ 3.2K \\
 &  & eng-bem & 12.0K/ 1.5K/ 1.5K \\
 &  & eng-lin & 17.7K/ 2.2K/ 2.2K \\
 &  & eng-run & 12.5K/ 1.6K/ 1.6K \\
 &  & eng-sot & 28.8K/ 3.6K/ 3.6K \\
 &  & eng-swa & 28.1K/ 3.5K/ 3.5K \\
 &  & eng-xho & 26.1K/ 3.3K/ 3.3K \\
 &  & eng-zul & 29.1K/ 3.6K/ 3.6K \\ \cmidrule(l){2-4} 
 & \cite{ogueji2019pidginunmt} & eng-pcm & 1.7K/ 0.2K/ 0.2K \\ \cmidrule(l){2-4} 
 & \cite{akera2022salt} & All-pairs$^1$ & 20.0K/ 2.5K/ 2.5K \\ \cmidrule(l){2-4} 
 % & \textbf{Carpentry Connect 2021} & zul-xho & -/ -/ 1.0K \\ \cmidrule(l){2-4} 
 & \href{https://github.com/asmelashteka/HornMT}{HornMT} & All-pairs$^2$ & 1.7K/ 0.2K/ 0.2K \\ \midrule
\multirow{5}{*} {NER} & \cite{Adelani2021MasakhaNERNE}& See Table \ref{tab:description}  & 41.2K/ 5.1K/ 5.1K \\ \cmidrule(l){2-4} 
& \cite{Adelani2022MasakhaNER2A} & See Table \ref{tab:description} & 41.2K/ 5.1K/ 5.1K \\ \cmidrule(l){2-4}
& \cite{eiselen-2016-government} &  See Table \ref{tab:description} & 1.7M/ 219.7K/ 215.6K \\ \cmidrule(l){2-4}
& \cite{alabi-etal-2020-massive} & See Table \ref{tab:description}  & 20.2K/ 2.8K/ 5.5K \\ \cmidrule(l){2-4}
& \cite{pan-etal-2017-cross} & See Table \ref{tab:description}  & 9.2K/ 9.2K/ 9.4K \\ \midrule %afr, amh, ibo, mlg, kin, som, swa, yor
\multirow{4}{*} {News} & \cite{azime2021amharic}& amh  & 41.2K/ 5.1K/ 5.1K \\ \cmidrule(l){2-4} 
& \multirow{2}{*}{\cite{niyongabo-etal-2020-kinnews}} & kin  & 15.3K/ 1.7K/ 4.3K \\  
& & run  & 3.3K/ 0.4K/ 0.9K \\  \cmidrule(l){2-4} 
&\cite{davis_david_2020_4300294} & swa  & 4.4K/ 0.6K/ 0.6K \\  \midrule 
\multirow{3}{*} {Paraphrase} & \multirow{1}{*}{\cite{scherrer_yves_2020_3707949} }& amh, ber, kab, run   & 22.4K/ 2.8K/ 2.8K \\
& & ber  & 17.6K/ 2.2K/ 2.2K \\  
& & kab & 4.4K/ 0.6K/ 0.6K \\  \midrule
Phrase Chunking & \cite{eiselen-2016-government} & See Table \ref{tab:description} & 107.5K/ 13.0K/ 13.4K \\ \midrule % afr, nso, sot, ssw, tsn, tso, ven, zul
POS Tagging & ~\cite{10.1145/3146387, 10.1145/3314942} & ibo & 756.8K/ 94.7K/ 95.0K \\ \midrule
Question Answering & \cite{clark-etal-2020-tydi} & swa & 49.9K/ 0.5K/ n/a \\  \midrule
\multirow{3}{*}{Sentiment Analysis}& \cite{diallo2021bambara} & bam & 2.4K/ 0.3K/0.3K \\ \cmidrule(l){2-4} 
& \cite{oyewusi2020semantic} & pcm & 11.2K/ 1.4K/ 1.4K \\ \cmidrule(l){2-4} 
& \cite{shode2022yosm} & yor & 0.8K/ 0.2K/ 0.5K \\ \midrule
\multirow{12}{*}{Summarization} & \multirow{11}{*} {\cite{hasan-etal-2021-xl}} &  amh & 5.8K/ 0.7K/ 0.7K \\ 
 & & ibo & 4.2K/ 0.5K/ 0.5K \\  
 & & orm & 6.1K/ 0.8K/ 0.8K \\  
  && Rundi & 5.7K/ 0.7K/ 0.7K \\  
 && swa & 7.9K/ 1.0K/ 1.0K \\  
 && yor & 6.4K/ 0.8K/ 0.8K \\  
 && hau & 6.4K/ 0.8K/ 0.8K \\  
 && pcm & 9.2K/ 1.2K/ 1.2K \\  
 && som & 6.0K/ 0.7K/ 0.7K \\  
 && Tigrinya & 5.5K/ 0.7K/ 0.7K \\ 
 \cmidrule{2-4}
 & \cite{adebara2024cheetah}& \textsuperscript{$\star\dagger$} & -/ -/ 0.4K \\ \midrule
\multirow{12}{*}{Title Generation} & \multirow{11}{*} {\cite{hasan-etal-2021-xl}} &  amh & 5.8K/ 0.7K/ 0.7K \\ 
 && ibo & 4.2K/ 0.5K/ 0.5K \\  
 && orm & 6.1K/ 0.8K/ 0.8K \\  
 && run & 5.7K/ 0.7K/ 0.7K \\  
 && swa & 7.9K/ 1.0K/ 1.0K \\  
 && yor  & 6.4K/ 0.8K/ 0.8K \\  
 && hau  & 6.4K/ 0.8K/ 0.8K \\  
 && pcm & 9.2K/ 1.2K/ 1.2K \\  
 && som & 6.0K/ 0.7K/ 0.7K \\  
 && Tigrinya & 5.5K/ 0.7K/ 0.7K \\ 
 \cmidrule{2-4}
 &\cite{adebara2024cheetah}& \textsuperscript{$\star$}  & -/ -/ 5.9K \\ \midrule
\multirow{2}{*}{Topic Classification} & \multirow{2}{*}{\cite{hedderich-etal-2020-transfer}} & hau & 2.0K/ 0.3K/0.6K \\ 
& & yor & 1.3K/ 0.2K/ 0.4K \\ 

 \bottomrule
\end{tabular}%
}
\caption{Statistics of the data in our benchmark. $All-pairs^1$ each have the same size of data. They include ach-eng, ach-lgg, ach-lug, ach-nyn, ach-teo, ach-teo, eng-lgg, eng-lug, eng-nyn, eng-teo, lgg-teo, lug-lgg, lug-teo, nyn-lgg, nyn-lug, and nyn-teo. $All-pairs^2$ are all possible language combinations of aaf, amh, orm, som, tir, eng. \textsuperscript{$\star\dagger$} is a summarization test set including `hau', `nde' (zero-shot), and `swa'. \textsuperscript{$\star$} is a title generation test set across 15 languages: `amh', `gag' (zero-shot), `hau', `ibo', `pcm', `som', `swa', `tir', `yor', `kin' (zero-shot), `afr', `mlg' (zero-shot), `orm', `nde' (zero-shot), `sna'(zero-shot).}
\label{tab:data_stats}
\end{table*}

% \textsuperscript{$\dagger\dagger$} includes amh, ber, kab, run.  

% \input{Tables/nlu_results}

\section{Data-Performance-Policy Analysis}\label{appdx_sec:analysis}



\begin{figure*}[]
    \centering
    % First sub-image
    \subfigure[Languages with an F\textsubscript{1} score of $50$ or higher.]{
        \includegraphics[width=\linewidth]{Images/lang_id_50_more.png}
        \label{fig:langid_sub1}
    }
    \hfill
    \subfigure[Languages with an F\textsubscript{1} score below $50$.]{
        \includegraphics[width=\linewidth]{Images/langid_less_50.png}
        \label{fig:langid_sub2}
    }
    \caption{Overall distribution of Claude-3.5-Sonnet's performance on language identification.}
    \label{fig:langid}
\end{figure*}

Figure~\ref{fig:langid} shows the overall distribution of Claude-3.5-Sonnet's (best model) performance on language identification. In sub-figure~\ref{fig:langid_sub1}, the vertical axis represents the F\textsubscript{1} score, and each point along the horizontal axis corresponds to a specific African language for which the model achieves an F\textsubscript{1} score of $50$ or higher. Only around $50$ languages fall into this category, with their performance ranging from just above $50$ F\textsubscript{1} to approximately $90$ F\textsubscript{1}. Meanwhile, sub-figure~\ref{fig:langid_sub2} depicts the majority of African languages, all of which register F\textsubscript{1} scores below $50$. These results indicate that while the model manages moderate or better performance on a limited subset of languages, it struggles substantially with the remaining ones. Overall, this visualization underscores that the model’s ability to correctly identify and process most African languages remains limited, highlighting a need for more comprehensive data coverage and targeted model improvements. 