\section{Introduction}\label{sec:introduction}

The digital landscape of Africa is as diverse as its cultures and languages, yet the continent’s rich linguistic heritage remains largely underrepresented in NLP research. Despite significant transformations over the past decade driven by advances in large language models (LLMs) and a growing interest in linguistic diversity, most efforts have concentrated on a handful of widely spoken languages such as Swahili, Afrikaans, and Hausa, while many Indigenous languages continue to be sidelined due to the paucity of accessible, high-quality datasets. This oversight, deeply rooted in historical language policies that have neglected robust data collection and accessibility, has far-reaching implications for the development of language technology across Africa \cite{ethnologue}. Recent initiatives, including the  work of Masakhane \cite{dione-etal-2023-masakhapos, adelani-etal-2021-masakhaner, adelani-etal-2023-masakhanews, adelani2024irokobench, bandarkar-etal-2024-belebele} and the efforts of platforms like HuggingFace, have begun to address these challenges by curating language-specific datasets and adapting transfer learning techniques for low-resource African languages. Models such as Cheetah \cite{adebara-etal-2024-cheetah}, Serengeti \cite{adebara-etal-2023-serengeti}, Toucan \cite{elmadany-etal-2024-toucan}, Afri-XLMR \cite{alabi-etal-2022-adapting}, AfriTeVa \cite{oladipo-etal-2023-better, jude-ogundepo-etal-2022-afriteva}, mBERT \cite{DBLP:journals/corr/abs-1810-04805}, XLM-R \cite{conneau-etal-2020-unsupervised}, and LLaMA have leveraged multilingual data to enhance performance on these languages, yet progress remains uneven—highlighting that the advancements achieved are primarily concentrated on a select few. 

In this study, we undertake a comprehensive empirical evaluation of leading large language models (LLMs) on an extensive benchmark that we collect using mostly existing datasets to unravel how current African language policies, by dictating data availability and accessibility, directly shape model performance across the continent’s diverse linguistic spectrum. By systematically analyzing the distribution of existing datasets across languages and performance patterns among various LLMs, we not only pinpoint which languages benefit from ample data resources but also illuminate the underlying factors that contribute to their relative success. Our investigation reveals that the disparities in NLP outcomes are closely tied to policy-driven data inequities, offering concrete evidence of the need for more inclusive, forward-thinking language policies. These insights provide actionable recommendations for enhancing dataset creation and model training, ultimately aiming to bridge the digital divide and foster linguistic diversity in artificial intelligence for African communities.

Overall, we make the following contributions: \textbf{(1) Benchmark with wide, diverse coverage.} We introduce \textit{Sahara}, a comprehensive benchmark assembled from large-scale, publicly accessible, and inclusive datasets that capture the rich linguistic diversity of Africa. This benchmark enables a systematic evaluation of data availability and model performance across a broad range of African languages. \textbf{(2) Dynamic leaderboard.} We develop a dynamic leaderboard built on best design practices, providing a transparent and continually updated platform to track progress and benchmark innovations in African NLP. \textbf{(3) Empirical evaluation.} We perform an extensive empirical assessment of a wide spectrum of models thereby establishing robust baselines and revealing performance disparities linked to data resource availability. \noindent \textbf{(4) Policy-driven and actionable insights.} We uncover clear links between current language policies, data availability, and model performance, demonstrating how policy-induced data inequities shape the effectiveness of language models across Africa. Our analysis provides recommendations for future research and development by informing policy reforms that foster more inclusive language technologies tailored to the needs of African communities. We now discuss language polices in Africa.
