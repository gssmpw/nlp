\subsection{Text Classification}
\paragraph{Cross-Lingual Natural Language Inference.}  This task involves the classification of 2 given sentences—a premise and a
hypothesis into entailment, neutral, or contradiction semantic relation \cite{adelani2024irokobench}. \citet{adelani2024irokobench} introduced AfriXNLI, a benchmark for evaluating 15 African Languages on Natural Language Inference (NLI). The languages were translated from the English portion of XNLI \cite{conneau2018xnli}.

\paragraph{Language Identification.} We include language identification datasets across $518$ African languages \cite{adebara2022afrolid}. The dataset includes data from multiple domains including education, government, health care, news, and religion. Additional statistics of the data is available in Table~\ref{tab:data_stats}. We include language identification tasks to investigate \textbf{(1.)} the capabilities of language models in correctly generating the appropriate language names of an input text and \textbf{(2.)} Correctly generate an additional sentence in a specific language. We include these two approaches because many languages have multiple names \cite{chen-etal-2024-fumbling} and a LM's inability to appropriately name a language does not implicate a lack of information about the language. 

\paragraph{News Classification.} This task involves automatically categorizing news articles into predefined categories or topics based on their content. The goal is to leverage machine learning and NLP techniques to understand and organize large volumes of news data, making it easier for users to access relevant information. News classification plays a crucial role in information retrieval, content recommendation, and other applications that involve organizing and categorizing textual data. The news classification cluster consists of data from four languages including Amharic~\cite{azime2021amharic}, Kinyarwanda~\cite{niyongabo-etal-2020-kinnews}, Kirundi~\cite{niyongabo-etal-2020-kinnews}, and Swahili~\cite{davis_david_2020_4300294, davis_david_2020_5514203}. Table~\ref{tab:data_stats} provides additional details of the data in this cluster. 

\paragraph{Sentiment Analysis.} Sentiment analysis is crucial in gaining insights into public opinion, customer feedback, and user sentiments across various platforms. The sentiment analysis cluster consists of the Bambara Sentiment dataset \cite{diallo2021bambara}, YOSM--a Sentiment Corpus for Movie Reviews in Y{o}r\`{u}b\'{a}~\cite{shode_africanlp}, and the Nigerian Pidgin sentiment dataset~\cite{oyewusi2020semantic}, respectively. Further details about the statistical composition of this data is available in Table~\ref{tab:data_stats}.


\paragraph{Topic Classification.} The primary goal for this task is to automatically categorize the content of the text based on its theme. Topic classification is widely used in various applications, such as document organization, content recommendation, and information retrieval. We include topic classification datasets for Y{o}r\`{u}b\'{a} and Hausa~\cite{hedderich-etal-2020-transfer}.


\subsection{Text Generation}


\paragraph{Machine Translation.} A robust benchmark for MT tasks is essential. \ourbenchmark, in its quest to effectively assess machine translation for African languages, selectively incorporates datasets containing diverse African languages into its benchmark. Specifically, \ourbenchmark~includes datasets such as Pidgin-UNMT\footnote{\href{https://github.com/keleog/PidginUNMT}{PidginUNMT GitHub Link}}~\cite{ogueji2019pidginunmt}, Afro-MT\footnote{\href{https://github.com/machelreid/afromt}{AfroMT GitHub Link}}~\cite{reid-etal-2021-afromt}, Lafand-MT~\cite{adelani-etal-2022-thousand}, SALT\footnote{\href{https://github.com/SunbirdAI/salt}{SALT GitHub Link}} \cite{akera2022salt}, and HornMT, thus encompassing a total of $45$ language pairs.

\paragraph{Paraphrase.} Paraphrasing is a core task in NLG that revolves around the comprehension and generation of text that conveys the same meaning while possibly exhibiting differences in structure, phrasing, or style~\cite{10.1162/tacl_a_00542}. \ourbenchmark~incorporates the TaPaCo dataset~\cite{scherrer_yves_2020_3707949}, which is publicly accessible and covers a total of $73$ languages, including four of African origin. These African languages consist of Kirundi, Amazigh, a macro-language Berber, and Afrikaans.

\paragraph{Summarization.} Summarization is the task of creating a condensed version of a text that retains its core ideas and essential information~\cite{nallapati-etal-2016-abstractive}. In the context of \ourbenchmark, our focus primarily centers on a subset of the abstractive summarization dataset, XL-SUM~\cite{hasan-etal-2021-xl}, which includes several African languages: Amharic, Hausa, Igbo, Kirundi, Oromo, Pidgin, Somali, Swahili, Tigrinya, and Yoruba. Additionally, we have extended our dataset collection efforts by scraping non-overlapping summarization datasets from BBC and Voice of Africa (VoE) websites, specifically targeting three African languages: Hausa, Ndebele, and Swahili.

\paragraph{Title Generation.} The task of title generation in NLG involves generating a concise and meaningful title or headline for a given article. This task shares its dataset source with summarization, drawing from XL-SUM, and also includes dataset extensions from BBC and VoE with focus on enabling zero-shot title generation.\\

\subsection{Multiple-Choice, Comprehensive and Reasoning}

This task cluster majorly addresses \emph{Question Answering} type questions. The QA tasks in this cluster are more mathematical-reasoning and reading-comprehension-oriented. 


\paragraph{Mathematical Word Problems.} \cite{adelani2024irokobench} introduced AfriMGSM, a QA dataset that contains human-written grade school mathematical word problems. Their dataset covers 15 African Languages, and the translation of each question to English for each language for in-language evaluation of LLMs. In this task, given a word problem, the LLM is expected to return a numerical value as output.

\paragraph{General Knowledge QA.} \cite{adelani2024irokobench} introduced AfriMMLU, a multi-choice QA dataset covering 5 areas of knowledge in 15 African Languages. The knowledge areas are high-school
geography, high-school microeconomics, elementary mathematics, international law), and global facts. In this task given a question in any of the aforementioned areas, and 4 options, the LLM is expected to choose the right option.

\paragraph{Reading Comprehension QA.} In this task, the LLMs are evaluated based on their ability to understand information in a given article. \cite{belebele_2024} introduced a parallel reading comprehension dataset covering 122 languages, 23 of which are native to Africa. The dataset comprises of short paragraphs accompanied by multi-choice Questions, whose answers can be inferred from accompanying paragraphs.

\paragraph{Context-based Question-Answering.} QA represents another vital facet of NLG, focusing on a model's capability to provide answers based on integrated knowledge. \ourbenchmark~relies on the gold passage dataset from the multilingual TYDIA\footnote{\href{https://github.com/google-research-datasets/tydiqa}{TyDiQA GitHub Link}} QA resource~\cite{clark-etal-2020-tydi}. This dataset offers concise questions with precisely one corresponding answer, enabling rigorous evaluation of QA performance.


\subsection{Tokens-level Classification}

\paragraph{Named Entity Recognition.} We incorporate NER datasets across $30$ languages. We use Distance Supervision NER (DS NER) Data \cite{hedderich-etal-2020-transfer}, MasakhaNER~\cite{adelani-etal-2022-masakhaner}, WikiAnn~\cite{rahimi-etal-2019-massively}, Yoruba-Twi NER~\cite{alabi2020massive}, and multiple NER data from \href{https://repo.sadilar.org/handle/20.500.12185/7}{SADiLaR}. Additional details about the datasets are available in Table~\ref{tab:data_stats}.

\paragraph{Phrase Chunking.} This task involves identifying and grouping together consecutive words or tokens into meaningful syntactic units, known as phrases. These phrases can include noun phrases, verb phrases, prepositional phrases, and other grammatical structures. The goal of phrase chunking is to analyze the grammatical structure of a sentence and extract higher-level syntactic units for better understanding. The phrase chunking cluster consists of data for ten Indigenous languages of South Africa (see Table \ref{tab:data_stats}). The data has annotations for noun, verb, adjective, adverbial, and prepositional phrase chunks. Words not belonging to these phrase types are labelled with the tag \textit{O}. 

\paragraph{Part of Speech Tagging.} This classification task involves assigning grammatical categories or labels (such as noun, verb, adjective) to each word in a sentence. The primary goal of POS tagging is to analyze the syntactic structure of a sentence and understand the role of each word within it. POS tagging is crucial for various NLP applications, as it provides a foundation for more advanced linguistic analysis. We include datasets for Igbo taken from IgboNLP~\cite{10.1145/3314942}.


