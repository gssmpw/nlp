
\section{Language Policies in Africa}\label{sec:langpolicy}

\begin{figure*}[!ht]
    \centering
    \resizebox{0.8\textwidth}{!}{ % Scale the whole figure to 80% of text width
        \begin{minipage}{\linewidth} % Ensure full figure width inside resizebox
            \centering
            % First row of subfigures
            \subfigure[Language policies for education across Africa.]{
                \includegraphics[width=0.47\linewidth]{Images/education_policy_map.png}
                \label{fig:map_edu}
            }
            \hfill
            \subfigure[Official national language policies across Africa.]{
                \includegraphics[width=0.49\linewidth]{Images/lang_policy_map.png}
                \label{fig:map_lang}
            }
            
            \vspace{1em}
            
            % Second row of subfigures
            \subfigure[Official regional language policies across Africa. \textit{None} means no additional policy is available on a regional level.]{
                \includegraphics[width=0.53\linewidth]{Images/regional_lang_policy_map.png}
                \label{fig:map_reg}
            }
            \hfill
            \subfigure[We collect ~\ourbenchmark to empirically evaluate LLM performance on African languages, allowing us to demonstrate with evidence how current language policies directly impact progress in the field.]{
                \includegraphics[width=0.43\linewidth]{Images/African_benchmark_updated.jpg}
                \label{fig:map_data}
            }
        \end{minipage}
    }
    \caption{Maps of Africa showing the languages covered in this work and the language policies across Africa. The term {\it Both} in maps (a), (b), and (c) refer to Indigenous and foreign languages combined.}
    \label{fig:maps}
\end{figure*}


Across Africa, the predominant response to the continent's multilingual landscape has been the adoption of a foreign language for official functions. In some instances, select Indigenous African languages receive official recognition at regional or national levels or within educational contexts~\cite{petzell2012linguistic, foster2021language, ouane2010and}. Figures \ref{fig:map_edu}, \ref{fig:map_lang}, and \ref{fig:map_reg} illustrate the distribution of language policies across the continent.\footnote{We designed the maps in these figures based on information derived from Ethnologue. We checked the national, regional, institutional, and educational languages for each country.} For example, in Nigeria, English is the official language, with only three out of 512 Indigenous languages recognized at the regional level. Similarly, Ghana designates English as its official language while also recognizing ten of its 73 Indigenous languages for institutional use. In Tanzania, Swahili is the sole Indigenous official language among 118 languages, alongside English. Kenya grants official status to $12$ of its $61$ languages, while South Africa recognizes $12$ out of $20$ Indigenous languages as institutional languages~\cite{adebara-abdul-mageed-2022-towards}.

Even when Indigenous languages are granted official status alongside foreign languages, they often serve a symbolic rather than functional role. For instance, although Kiswahili is recognized as an official language by the African Union, its website and official document releases remain in English and French. A similar pattern emerges in education systems: where Indigenous languages are used, their role is typically confined to early childhood education and is usually paired with a foreign language rather than serving as the sole medium of instruction~\cite{petzell2012linguistic, foster2021language, ouane2010and}. These language policies significantly shape language use across various domains, including newspapers, radio, television, and social media. Due to the dominance of foreign languages in official contexts, major newspapers and government publications are predominantly produced in languages such as English, French, and Portuguese, thereby limiting access for speakers of Indigenous languages~\cite{rescue_agbozo_2021}. Although some Indigenous languages are used on television and radio, foreign languages tend to dominate national broadcasts, especially in political discourse and formal news reporting~\cite{Cheo2023, myers2020local, nca2016list}.


Social media presents a more flexible environment where users can communicate in Indigenous languages; however, platform support remains uneven~\cite{sunday_etal_2018, molale_mpofu_2023, rescue_agbozo_2021}. Many African languages are underrepresented in digital spaces, lacking features such as text input options, spell checkers, or automated translation services. Consequently, users often resort to code-switching between Indigenous and dominant foreign languages, reinforcing existing linguistic divides. These trends underscore how official language policies shape broader language use, ultimately contributing to the marginalization of Indigenous languages in both public and digital communication spheres.

The analysis of language policies in Africa reveals how historical choices have led to significant data imbalances for Indigenous languages. Motivated by this, we built our benchmark,~\ourbenchmark~to evaluate how these policies directly impact data availability, coverage, and the performance of LLMs on African languages. In the following section, we detail the construction of ~\ourbenchmark~and demonstrate its role in linking language policy to empirical outcomes in NLP.

