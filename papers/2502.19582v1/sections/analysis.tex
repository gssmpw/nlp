\section{Data-Performance-Policy Analysis}~\label{sec:analysis}
 \noindent To understand the relationship between policy, data availability, and performance, we carry out a number of analyses inspired by~\ourbenchmark~ data and empirical results on it. 
\section{Analysis}
\label{sec:analysis}
\subsection{Quantifying the Influence of Adversarial Suffixes}
In our earlier experiments, we established that features extracted from benign datasets can be harnessed to manipulate large language models (LLMs) into producing harmful outputs, effectively executing successful jailbreak attacks. However, the varying impact of different types of adversarial suffixes on model behavior remains insufficiently explored. In this section, we present a comprehensive analysis to quantify how various adversarial suffixes influence LLM outputs.

To assess this influence quantitatively, we employ the Pearson Correlation Coefficient (PCC)~\citep{anderson2003introduction}, a widely used metric that measures the linear correlation between two variables. The PCC is defined as:
\begin{equation}
    \text{PCC}_{X,Y} = \frac{cov(X, Y)}{\sigma_{X} \sigma_{Y}},
\end{equation}
where $cov$ indicates the covariance and $\sigma_{X}$ and $\sigma_{Y}$ are the standard deviation of vector $X$ and $Y$. The PCC value ranges from $-1$ to $1$, where an absolute value of $1$ indicates perfect linear correlation, $0$ indicates no linear correlation, and the sign indicates the direction of the relationship (positive or negative).
\begin{figure}[!t]
\centering
    % First row
    \begin{minipage}[b]{0.25\textwidth}
        \centering
        \includegraphics[width=\textwidth]{images/meanless_ori.pdf}\\
        \includegraphics[width=\textwidth]{images/meanless_suffix.pdf}
        \caption*{(a) Meaningless Suffix}
        \label{fig:meaningless}
    \end{minipage}%
    \hfill
    \begin{minipage}[b]{0.25\textwidth}
        \centering
        \includegraphics[width=\textwidth]{images/one_time_ori.pdf}\\
        \includegraphics[width=\textwidth]{images/one_time_suffix.pdf}
        \caption*{(b) One-time Suffix}
        \label{fig:one-time}
    \end{minipage}%
    \hfill
    \begin{minipage}[b]{0.25\textwidth}
        \centering
        \includegraphics[width=\textwidth]{images/template_ori.pdf}\\
        \includegraphics[width=\textwidth]{images/template_suffix.pdf}
        \caption*{(c) Template Suffix}
        \label{fig:template}
    \end{minipage}

    \vspace{1em} % Add some vertical space between rows

    % Second row
    \begin{minipage}[b]{0.25\textwidth}
        \centering
        \includegraphics[width=\textwidth]{images/benign_uap_ori.pdf}\\
        \includegraphics[width=\textwidth]{images/benign_uap_suffix.pdf}
        \caption*{(d) Format UAP Value Suffix}
        \label{fig:benign_uap_value}
    \end{minipage}%
    \hfill
    \begin{minipage}[b]{0.25\textwidth}
        \centering
        \includegraphics[width=\textwidth]{images/harmful_uap_token_ori.pdf}\\
        \includegraphics[width=\textwidth]{images/harmful_uap_token_suffix.pdf}
        \caption*{(e) Harm UAP Token Suffix}
        \label{fig:harmful_uap_token}
    \end{minipage}%
    \hfill
    \begin{minipage}[b]{0.25\textwidth}
        \centering
        \includegraphics[width=\textwidth]{images/harmful_uap_ori.pdf}\\
        \includegraphics[width=\textwidth]{images/harmful_uap_suffix.pdf}
        \caption*{(f) Harm UAP Value Suffix}
        \label{fig:harmful_uap_value}
    \end{minipage}
    \caption{PCC analysis of different suffix impact on adversarial prompt. Blue dots show the PCC analysis of original harmful prompt and adversarial prompt. Red dots show PCC analysis of suffix and adversarial prompt.}
    \label{fig:pcc_analysis}
\end{figure}

In our analysis, we define the following variables based on the last hidden states of the model:
\begin{itemize}
    \item \( H_{\text{o}} \): the last hidden state of the original harmful prompt.
    \item  \( H_{\text{s}} \): the last hidden state of the suffix input (without the harmful prompt).
    \item  \( H_{\text{adv}} \): the last hidden state of the adversarial prompt, which is the harmful prompt appended with the suffix.
\end{itemize}

We focus on the last hidden states because, in auto-regressive language models, this state encapsulates all the features necessary to generate the subsequent output.

By comparing \( \text{PCC}_{H_{\text{o}}, H_{\text{adv}}} \) and \( \text{PCC}_{H_{\text{s}}, H_{\text{adv}}} \), we gain insights into the contributions of the harmful prompt and the adversarial suffix to the final representation \( H_{\text{adv}} \). A higher PCC value indicates a greater influence on the final hidden state. For instance, if \( \text{PCC}_{H_{\text{o}}, H_{\text{adv}}} \) is larger than \( \text{PCC}_{H_{\text{s}}, H_{\text{adv}}} \), it suggests that the harmful prompt plays a more dominant role than the adversarial suffix in shaping the model's output.

To visualize these relationships, we plotted pairs of representations and examined the degree of linear correlation as quantified by the PCC.

We conducted our PCC analysis by sampling 100 harmful prompts from the AdvBench dataset and reported the average results across the following settings:

\begin{itemize}
    \item \textbf{Prompt + Meaningless Suffix}:

    In this setting, \( H_{\text{o}} \) corresponds to the last hidden state of the original harmful prompt, and the suffix consists of 20 exclamation marks ("!"). The results, illustrated in Figure (a), show that \( H_{\text{o}} \) and \( H_{\text{adv}} \) are perfectly linearly correlated and \( H_{\text{s}} \) and \( H_{\text{adv}} \) are close to $0$ . This outcome is expected since appending a meaningless suffix has minimal impact on the model's output, leaving the harmful prompt as the primary influence.

    \item \textbf{Prompt + One-Time Suffix}:

    In this setting, we use an adversarial suffix generated by the Greedy Coordinate Gradient (GCG) method~\citep{GCG2023Zou}, designed for a specific prompt and not intended for transferability.  Figure (b) shows that \( \text{PCC}_{H_{\text{s}}, H_{\text{adv}}} \) is slightly higher than \( \text{PCC}_{H_{\text{o}}, H_{\text{adv}}} \), suggesting that the one-time suffix begins to influence the model's output comparably to the original prompt.

    \item \textbf{Prompt + Template Suffix}:

    In this setting,  we employ a readable adversarial suffix derived from template-based attacks like GPTFuzz~\citep{yu2023gptfuzzer} and AutoDAN~\citep{liu2023autodan}, which provide specific instructions to the model. Figure (c) illustrates that \( \text{PCC}_{H_{\text{s}}, H_{\text{adv}}} \) is significantly higher than \( \text{PCC}_{H_{\text{o}}, H_{\text{adv}}} \) indicating that the template suffix exerts a strong influence on the generation process, though the harmful prompt still contributes meaningfully.

    \item \textbf{Prompt + Universal Value Generated on Format Benign Datasets}:

    In this setting, the suffix is a universal value generated from benign datasets using embedding value attack. Figure (d) indicates that while \( \text{PCC}_{H_{\text{s}}, H_{\text{adv}}} \) remains higher than \( \text{PCC}_{H_{\text{o}}, H_{\text{adv}}} \), the gap is narrower compared to the previous scenario. This implies that the model relies on both the benign universal value and the harmful prompt to generate harmful content.
    
    \item \textbf{Prompt + Universal Token Generated on Harmful Datasets}:

    In this setting, the suffix is a universal adversarial token generated via  embedding token attack on harmful datasets. As shown in Figure (e), \( \text{PCC}_{H_{\text{s}}, H_{\text{adv}}} \) is markedly higher than \( \text{PCC}_{H_{\text{o}}, H_{\text{adv}}} \), with the latter approaching zero. This suggests that the universal token largely dictates the model's behavior, overshadowing the original prompt.

    \item \textbf{Prompt + Universal Value Generated on Harmful Datasets}:

    Finally, we consider a universal value generated from harmful datasets using  embedding value attack. Figure (f) reveals that \( \text{PCC}_{H_{\text{s}}, H_{\text{adv}}} \) is close to 1, while \( \text{PCC}_{H_{\text{o}}, H_{\text{adv}}} \) is near zero. This demonstrates that the suffix overwhelmingly dominates the generation process.
\end{itemize}

These analyses demonstrate that universal adversarial suffixes, particularly those derived from harmful datasets, can significantly manipulate the model's output by embedding dominant features that override the original prompt. Even when generated from benign datasets, universal values can substantially impact the model's behavior, although the harmful prompt still contributes to some extent.




% \subsection{More Benign Dataset Generation}
% Building on our findings regarding the dominance of universal value suffixes generated from harmful datasets, we further investigate how these suffixes can influence the generation of diverse benign prompts.

% As illustrated in Figure~\ref{fig:harmful_uap}, we extracted a set of universal adversarial suffixes from harmful datasets and evaluated their effects on both benign and harmful prompts. Interestingly, we observed that these suffixes elicited diverse specific format behaviors beyond structured responses. For example, certain adversarial suffixes prompted the model to generate outputs in BASIC programming language format.

% Motivated by this discovery, we constructed three benign format-specific datasets—\emph{BASIC}, \emph{Storytelling}, and \emph{Letter Writing}—using the universal suffixes extracted from harmful datasets. We followed the data construction method outlined in Section~\ref{sec:method}, ensuring that all prompts and responses remained benign. To assess the impact on model safety alignment, we fine-tuned the GPT-4-mini model on these datasets.

% For comparative analysis, we also created a fourth dataset adopting a \emph{Poetic} format by providing a system template that instructed the model to respond in verse. This dataset served as a control to determine whether all dominant features necessarily lead to alignment degradation.
% \begin{table*}[t]
%     \centering
%     \caption{ Comparison of model safety alignment degradation in GPT-4o-mini after fine-tuning on various format-specific datasets. }
%     \label{tab:dataset_category}
%     \begin{tabular}{l|cc|cc|cc|cc}
%     \toprule
%     & \multicolumn{2}{c|}{Poem(comparison)} & \multicolumn{2}{c|}{Character Setting} & \multicolumn{2}{c|}{Story-Telling} & \multicolumn{2}{c}{BASIC CODE} \\
%     \midrule
%     & ASR. & Harm. & ASR. & Harm. & ASR. & Harm. & ASR. & Harm. \\
%     \midrule
%     GPT-4o-mini & 6.3\% & 1.09 &   70.2\% & 3.44   & 96.3\% & 4.75 & 91.9\% & 4.44 \\
%     \bottomrule
%     \end{tabular}
% \end{table*}

% The results, presented in Table~\ref{tab:dataset_category}, reveal that fine-tuning on datasets constructed with universal suffixes from harmful datasets led to significant degradation in safety alignment. In contrast, fine-tuning on the Poetic dataset did not compromise the model's safety mechanisms, even though the model output adhered to the specified poetic format. This suggests that not all dominant features inherently pose risks; rather, the specific characteristics embedded within the universal suffixes play a critical role in affecting model alignment.


% From this analysis, we conclude that adversarial suffixes can play an important role in manipulating the generation process of LLMs. Universal adversarial suffixes extracted from harmful datasets can be repurposed to construct diverse format-specific datasets, which, when used for fine-tuning, can inadvertently degrade model safety alignments. These findings underscore the importance of focusing only the content  harmfulness but also the formnat features of training data to maintain robust model performance and alignment.



\paragraph{Data Availability and Diversity.}
As shown in Figure \ref{fig:cluster_lang}, among the $517$ languages in our benchmark, only $45$ have more than one dataset, while the majority are represented solely by language identification data. Amharic leads with $11$ unique datasets across all clusters, followed by Yor\`{u}b\'{a} with ten and Hausa with nine. A total of six languages have only one dataset beyond language identification, while the remaining $36$ have between two and eight datasets each. \textit{It is important to note that the disparity in dataset availability is closely linked to language policy.} The $45$ languages with multiple datasets generally hold official status or are spoken by a significant majority in their respective countries. Government policies that designate certain languages as official often yield greater institutional support, driving increased documentation, education, and media presence. For example, Amharic’s official status in Ethiopia contributes to its relatively rich dataset availability, while Yor\`{u}b\'{a} and Hausa, both widely used in Nigeria for regional governance and media, similarly benefit from enhanced linguistic resources.

Conversely, languages lacking official recognition or widespread institutional support receive minimal investment in digital and linguistic infrastructure. The dominance of a few languages within policy frameworks perpetuates data scarcity for many Indigenous African languages, thereby limiting their inclusion in NLP advancements and reinforcing digital inequalities. Addressing this imbalance requires policy shifts that prioritize multilingual data collection, increased funding for low-resource language research, and the integration of diverse languages into education, media, and digital communication.

\paragraph{Data Quality.}
Most of the downstream task datasets  are limited to simple tasks that require merely an atomic class label at the word, sentence, or document level (e.g., sentiment analysis, named entity recognition, topic classification). Furthermore, many of these datasets are translations of existing resources or content from high-resource languages like English and French, and they do not always reflect authentic language usage within the respective communities. For instance, AfriXLNI, AfriMMLU, and AfriMGSM \cite{adelani2024irokobench} were created by translating XLNI, MMLU, and MGSM into African languages. In these datasets, the frequent use of borrowed words and phonologized variants of foreign terms \cite{adebara-abdul-mageed-2022-towards} highlights the challenges posed by cross-lingual concept gaps. When translations are applied to classification tasks, misalignments in the attached labels can occur because labels are not always equivalent across languages. Some labels may not exist in the target language or may have restricted usage. For example, while English possesses a productive set of adjectives, languages like Yor\`{u}b\'{a} have a more limited adjective inventory that is context-dependent, meaning that an English adjective may be translated as a noun or verb in Yor\`{u}b\'{a} rather than as an adjective. 

\paragraph{Model Performance}
Here, we analyze performance of our best model, Claude 3.5-Sonnet. As shown in Figure~\ref{fig:performance_analysis_scatterplot}, the model faces varying degrees of difficulty when handling African languages. Sub-figure~\ref{fig:preform_sub1} demonstrates that for generation tasks, even the strongest models struggle to produce high-quality text in African languages, with most scores falling below 15 BLEU/ROUGE. In contrast, sub-figure~\ref{fig:preform_sub2} shows that while many languages achieve relatively strong classification performance (over 80\% accuracy), a substantial portion still falls into the 60\%–80\% range, underscoring a clear performance gap. Finally, sub-figures~\ref{fig:preform_sub3} and~\ref{fig:preform_sub4} reveal that translation into English is notably easier than translation out of English or French into African languages, where BLEU scores drop from an average of around $19$ down to nine or $11$, respectively. This finding suggests that even the best model struggles more with producing fluent, faithful output in African languages than with understanding them—underscoring the urgent need for better data and model development tailored to these languages. Refer to Appendix~\ref{appdx_sec:analysis} for a similar analysis leveraging our empirical language identification evaluation.

Apart from our empirical analyses, the limited availability and diversity of datasets for African languages in NLP benchmarks has significant implications for developing robust language technologies. \textit{Limited Model Performance and Generalization} – Models trained on a single dataset per language risk overfitting to specific domains or styles, reducing their ability to generalize across different contexts, which is especially problematic for African languages with high dialectal variation. \textit{Bias Toward Well-Resourced Languages} – Languages with more datasets naturally perform better in multilingual models, reinforcing disparities and leading to poor performance for underrepresented African languages. \textit{Restricted Downstream Applications} – Many NLP applications, such as machine translation and question answering, require diverse datasets; if a language is represented only by language identification data, it cannot support the development of more complex language technologies. \textit{Challenges in Transfer Learning} – Transfer learning techniques, like multilingual fine-tuning, rely on cross-lingual similarities and sufficient data; when African languages suffer from sparse training data, they are unable to effectively leverage knowledge transfer from resource-rich languages.

%\hl{It is important to mention that despite the} 

\section{Recommendations}\label{sec:rec} 
Policy determines how languages are used across various domains, influencing their visibility and viability in digital spaces. Currently, the limited online presence of many Indigenous African languages restricts the availability of training data, which is crucial for building effective NLP models. We provide the following recommendations:

\noindent\textbf{Policy matters.} Governments and policymakers should enact national language policies that mandate the inclusion of Indigenous and underrepresented languages across digital infrastructure, education, and media. Countries with multilingual populations should adopt policies that support the systematic documentation, transcription, and digitization of African languages to enhance their presence in AI applications. Such efforts will enrich the diversity of available data, paving the way for the development of more robust and inclusive models.

\noindent\textbf{Expand data collection and annotation.} Given that only $45$ of the $517$ languages in the benchmark have more than one dataset based on our collection, it is essential to increase the availability of high-quality data for underrepresented African languages across a broad range of downstream tasks. Moreover, prioritizing community-driven data annotation efforts will be crucial in ensuring both linguistic diversity and accuracy, especially for languages that lack institutional support.

\noindent\textbf{Develop contextually relevant and culturally accurate datasets.} Many existing African language datasets are derived from translations of high-resource languages, which can introduce misalignments and fail to capture authentic linguistic expressions. To improve NLP performance, datasets should be created using native speakers and sourced from real-world interactions, literature, and media in African languages. This will help preserve linguistic authenticity and avoid the overreliance on loanwords or structures that do not naturally occur in the language. Specialized corpora should also be developed to capture dialectal variations and culturally specific language use.
