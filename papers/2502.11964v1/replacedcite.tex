\section{Related Work}
\subsection{Transaction Fee Mechanisms}

There is extensive research on blockchain fee markets, with a particular focus on Ethereum and Bitcoin. Early studies primarily examined Bitcoin, exploring monopolistic pricing mechanisms____. More recent contributions to this field include____. Unlike these works, our study concerns measuring resource usage on a blockchain with client-side parallel execution, rather than focusing on pricing.



The TFM design framework was introduced by Roughgarden____. Roughgarden's analysis of the EIP-1559 mechanism____ initiated an active line of research on TFMs. Chung and Shi____ demonstrated that no TFM can be ideal --- meaning it cannot simultaneously be incentive-compatible for users and block producers while also being resistant to collusion between the two. This conclusion holds even for weaker definitions of collusion resilience, as shown by Chung et al.____ and Gafni and Yaish ____. Finally, attempts to address these limitations using cryptographic techniques ____ have made progress in overcoming certain impossibilities, while other attempts relax the desiderata____. However, designing an ideal TFM still remains out of reach. While these studies examine the limitations of TFMs, our focus is on GCMs for parallel execution and how to integrate them with a TFM.

A related body of work examines the dynamics of TFMs over multiple blocks, particularly focusing on the base fee in EIP-1559. Leonardos et al.____ demonstrate that the stability of the base fee depends on the adjustment parameter, with short-term volatility but long-term block size stability. Reijsbergen et al.____ suggest using an adaptive adjustment parameter to mitigate block size fluctuations, while Ferreira et al.____ highlight user experience issues caused by bounded base fee oscillations. Additionally, Hougaard and Pourpouneh____ and Azouvi et al.____ reveal that the base fee can be manipulated by non-myopic miners.

Given the discussion surrounding multi-dimensional fees in Ethereum____ and the deployment of EIP-4844____ (a first step towards a multi-dimensional fee market on Ethereum), a recent line of work explores multi-dimensional fee markets, focusing on efficient pricing mechanisms and their optimality. This work is further refined by Diamandis et al.____, who design and analyze multi-dimensional blockchain fee markets to align incentives and improve network performance. Building on this, Angeris et al.____ prove that such fee markets are nearly optimal, with efficiency improving over time even under adversarial conditions. Multidimensional fee markets are closely related to fee markets designed for parallel execution. In particular, in the weighted area \GCM, the weights can be interpreted as fees within a multidimensional fee market. Unlike previous literature on multidimensional fee markets, we focus on parallelization, introduce desirable properties, and evaluate how various mechanisms perform.


Further extensions of TFMs have emerged. Bahrani et al.____ consider TFMs in the presence of maximal extractable value (MEV), i.e., value extractable by the block producer. Further, Wang et al.____ design a fee mechanism for proof networks, whereas Bahrani et al.____ introduce a transaction fee mechanism for heterogeneous computation. Our work most closely relates to the latter, but, in contrast, our chosen approach is closer to multidimensional fee markets, trading complexity for the block producer for stronger incentive compatibility for the user.  

Local fee markets have recently been a topic of discussion in the blockchain space____. The core idea is that transactions interacting with highly contested states incur higher fees, while those involving non-contested states pay lower fees. However, discussions on local fee markets have largely remained high-level, without a precise characterization of the desired properties beyond this general goal. Moreover, currently implemented local fee markets____ require significant user sophistication to set fees appropriately. In this work, we formalize the desiderata for fee markets in the context of parallel execution and identify the weighted area \GCM as a promising candidate. One key advantage is its compatibility with a TFM, enabling simple fee estimation for users.


\subsection{Parallel Execution}

Blockchain concurrency has been a focal point in an active line of research. In particular, numerous efforts have aimed to enable parallel transaction processing through speculative execution____. Note that speculative execution is already deployed by multiple blockchains____. Static analysis has also been employed to identify parallelizable transactions, though it cannot completely eliminate inherent dependencies____. Similarly, ____ demonstrate how parallel execution can assist struggling nodes in catching up. While these works are orthogonal to ours, they highlight the overhead of parallel execution when there is no advance knowledge about a transaction's state accesses.

Further, ____ and ____ have evaluated the parallelization potential of the Ethereum workload. The latter demonstrates that a speedup of approximately fivefold is achievable, assuming state accesses are known in advance. Additionally, Solana____ and Sui____ already perform parallel execution with advance knowledge of state accesses. However, in practice, state accesses are not known beforehand on many blockchains such as Ethereum. There, less than 2\% of transactions disclose them proactively, as shown by ____, due to a lack of incentives. In this work, we aim to take a step toward unlocking the parallelization potential by designing a TFM that supports parallel execution. This mechanism relies on the disclosure of state accesses as done in Solana____ and Sui____.