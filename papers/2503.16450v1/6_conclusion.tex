
% \section{Conclusion}


% \subsection{Limitations}
% battery life question didn't specify active use vs. power-saving mode throughout the day 

\section{Conclusion}
In this work, we investigated the preferences of BVI individuals on the aesthetic and functionality design factors of a future robot dog guide through in-depth interviews and a large-scale survey. Our results indicate that the aesthetic factors of the robot desired by the BVI participants include: low-maintenance and waterproof outer surface with soft exterior, uniform identifier, a mechanism to accommodate different heights of users, and some resemblance in appearance to a dog. Functionality requirements of the robot include: built-in or integrated navigation system, easy charging mechanism, battery percentage notifications, and the capacity for multiple methods of communication including voice command, button controls built into the harness handle, auditory signal outputs, and haptic feedback. We also gathered feedback on desirable behavior and interaction patterns, which include: the ability to remain focused on the guiding task, displaying intelligent disobedience when necessary, and an ``off-duty'' mode as a potential option. 

For future work, it may be worthwhile to test physical prototypes of different heights and weights of the robot, given that the participants often had some difficulty stating a specific desired height and weight without a physical reference point. We also plan on further refining the specific texture of the robot's outer surface, the degree of resemblance to a dog, and the specific functions of the individual buttons built into the harness handle through further design studies with participants. Additionally, we are interested in developing a voice user interface (VUI)-enabled AI agent for the robot to maximize a comfortable and supportive user experience. 
