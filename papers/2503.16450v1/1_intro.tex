

\section{Introduction}

Professionally trained guide dogs, generically termed `dog guides', play an important role in assisting the lives of individuals who are blind or visually impaired (BVI). Participants have reported that having a dog guide significantly improved their mobility, confidence, and sense of safety when walking or traveling. However, there are certain limitations and concerns associated with owning a live dog guide:
the burden of care and maintenance, potential risk from distractions, and encountering those with an aversion due to allergies, fears, or social prejudice. 
Additionally, the cost of training and maintaining a dog guide is high, with long wait times and limited availability~\cite{wirth2007economic, bender2023aid}. Fewer than 1\% of qualifying adult BVI individuals currently own dog guides~\cite{wirth2007economic}.

% \textbf{Need for care and maintenance}: All of our interview participants indicated that the most challenging part of owning a dog guide is the care and maintenance, such as feeding, walking, bathing, grooming, and cleaning up after the dog. Owning a dog also requires additional maintenance including annual vaccinations and veterinary clinic visits, which can be an added burden. 

% \textbf{Potential for distractions and associated risks}: Another concern dog guide users report is the dog's susceptibility to getting distracted by environmental stimuli such as other animals, people, noises, and smells. Many of our participants reported instances of being put into dangerous situations as a result of their dog getting distracted. 
    
% \textbf{Dog aversion, allergies, and social prejudice}: While dogs are beloved by most, there is a significant percentage of our population who exhibit aversion toward dogs - either due to fear, previous negative experiences, or a general dislike - as well as those who are allergic to dogs. Each of our interview participants reported experiences of being denied services or access to certain areas while accompanied by their dog guide as a result of this social prejudice, despite the existence of policies prohibiting this. 

% \textbf{Cost and long wait time}: Training dog guides is a lengthy and expensive process, costing around \$35,000-50,000 per dog, with added costs for ongoing care \cite{wirth2007economic, bender2023aid}. Although non-profits such as Guide Dogs for the Blind cover most expenses, their financial support is limited, leading to significant availability issues. Individuals often wait over a year to receive a guide dog, which then typically only has 8 "working years" before they are retired. Consequently, fewer than 1\% of all qualifying adult BVI individuals currently own dog guides \cite{wirth2007economic}. 

 
% include more anecdotes 
% bystander paper 



Given the limitations of conventional dog guides, we see a need to develop a more accessible and cost-effective solution that will enable more BVI individuals to access the significant benefits a dog guide can provide. In search of a blend of innovation, practicality, and familiarity, we have selected to work on the development of a quadrupedal robot that resembles a dog guide both in shape and function, or a ``robot dog guide''. 
% We believe that this solution will be able to maintain many of the advantages of a conventional dog guide while allowing for the incorporation of further functionality not possible with live dogs. 
To this end, this study aims to answer design questions on the functional, aesthetic, form factor, and behavioral aspects of a robot dog guide to identify important requirements for future implementations. 


\begin{table}[ht]
    \caption{Demographic description of Stage 1 Interview participants}
\vspace{-1em}
    \centering
    \setlength\tabcolsep{3.5pt}
    \begin{tabular}{cccccc}
    % \hline
    \toprule
        \textbf{ID} & \textbf{Gender} & \textbf{Age} & \textbf{Impairment Condition} & \textbf{Mobility Aid}\\ \midrule
        P1 & F & 45-59 & no peripheral vision & Dog Guide\\
        P2 & F & 18-29 & low vision, no night vision & White Cane only\\
        P3 & M & 60-74 & low vision & Dog Guide\\
        P4 & F & 60-74 & low vision & White Cane only\\
        P5 & M & 18-29 & completely blind & White Cane only\\
        P6 & F & 75+ & low vision & Dog Guide\\
        P7 & M & 30-44 & low vision \& hearing loss & Dog Guide\\
        P8 & F & 60-74 & low vision & Dog Guide\\
        P9 & F & 60-74 & low vision \& hearing loss & White cane only\\
        \bottomrule
    \end{tabular}
    \label{tab:bvi-info}
\end{table}


\begin{table}[ht]
\caption{Demographic Description of Stage 2 Survey Participants}
\vspace{-1em}
\centering
\setlength\tabcolsep{3.5pt}
\begin{tabular}{@{} ll *{3}{S[table-format=3.0, table-space-text-post=\%]} @{}}
\toprule
Information&Group &{BVI (\%)} &{ST (\%)}\\
\midrule
Visual impairment&complete/near blindness&50.0&n/a\\
&partially sighted&35.7&n/a\\
&unspecified &14.3&n/a\\
\midrule
Age group&18-24 &31.0&32.8\\
   &25-34 &21.5&53.4\\
   &35-44 &19.0&8.6\\
   &45-54 &9.5&3.4\\
   % &55-74&19.0&1.7\\
   &55-64&9.5&1.7\\
   &65-74 &9.5&0.0\\
\midrule
Dog guide ownership&DO (current/former owner)&40.5&n/a\\
              &NDO (never owned)&59.5&n/a\\
\bottomrule
\end{tabular}
\vspace{-1em}
\vspace{-1em}
\label{tab:survey-demographics}
\end{table}
   

While it might seem counter-intuitive to focus on the aesthetics of a robot dog guide for BVI users, research has shown that the visual design of assistive devices can significantly impact social interactions and user experiences. 
% A study by Azenkot et al. on an autonomous navigation robot designed to look like a regular suitcase provides compelling evidence for this. Their focus group sessions with legally blind participants revealed that many liked the design of the robot specifically because it assimilated into the surroundings. This finding highlights that although BVI individuals may not directly perceive the aesthetics themselves, many are concerned with how their assistive technology integrates into the environment and how others perceive it. 
Azenkot et al. found that legally blind participants appreciated a navigation robot designed to assimilate into its surroundings, emphasizing that BVI users care about how others perceive their assistive technology. 
The appearance of a robot dog guide can influence the public perception of the user, affecting the user's comfort, confidence, and social acceptance in various environments. 
Moreover, the study highlighted that acceptance from the general public is crucial for the successful deployment of such robots in public spaces. 
% By investigating preferences for aesthetic and form factor design, we seek to understand how to create robotic guide dogs that not only function effectively but also integrate seamlessly into users' lives and social contexts, addressing the concerns of both users and the broader community. 
% This holistic approach to design requirements aims to enhance both the practical utility and social experience of using a robotic dog guide.

To explore these design preferences, we gathered data from BVI participants on their desired appearance, texture, functionalities, and control methods for a robot dog guide, as well as from sighted participants on their impressions of a robot dog guide. 
Participants expressed overall favorable impressions, and the BVI participants highlighted key preferences: the robot should resemble a real dog, appear approachable, and include a clear identifier. They preferred a built-in GPS, Bluetooth connectivity, waterproof materials, and softer textures. They also emphasized multiple control options (voice commands, motion gestures, harness buttons), long battery life, and self-charging capabilities. 
By investigating preferences for aesthetic and form factor design, we seek to create robot dog guides that not only function effectively but also integrate seamlessly into users' lives and social contexts, addressing the concerns of both users and the broader community.
These findings will guide the design of prototypes that meet the specific needs of BVI individuals. 

% To address these questions, we asked participants questions pertaining to the appearance, texture, functionalities, and method of controlling the robot dog guide. The important key findings from the BVI participants included the following: (1) The participants had an overall favorable impression of a future robot dog guide and acknowledged numerous potential advantages that it could offer; (2) most participants showed some preference for the robot to resemble the appearance of a real dog and look ``cute" or ``approachable", as well as for it to have a uniform identifier indicating its purpose; (3) participants prefer having a robot with built-in GPS and Bluetooth connectivity; (4) they showed strong preference for a waterproof outer surface, as well as some preference for a softer external texture rather than a rigid one; (5) they advocated for multiple options of voice command, motion gesture, and buttons on the harness as methods of giving commands to the robot, and both auditory signals and haptic feedback as methods of receiving signals from the robot; (6) they emphasized the need for a long battery life, reliable battery percentage indicator and were highly favorable toward a self-charging capability. These data inform us of directions for prototyping, enabling us to narrow in on designs that effectively meet the needs of BVI individuals. 
% (1) the participants had an overall favorable impression of a future robot dog guide and acknowledged numerous potential advantages that it could offer; (2) most participants showed some preference for the robot to resemble the appearance of a real dog and look "cute" or "approachable", as well as for it to have a uniform identifier indicating it as a guide; (3) they were in favor of the robot having a built-in GPS and Bluetooth connectivity and being made of waterproof material; (4) they showed some preference for a softer external texture rather than a rigid one; (5) they wanted to have the multiple options of voice command, motion gesture, and buttons on the harness as methods of giving commands to the robot, and both auditory signals and haptic feedback as methods of receiving signals from the robot; (6) they emphasized the need for a long battery life and reliable battery percentage indicator and were highly favorable toward a self-charging capability. These data inform us of directions for prototyping, enabling us to narrow in on designs that effectively meet the needs of BVI individuals. 


