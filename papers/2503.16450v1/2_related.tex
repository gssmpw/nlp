\section{Related Work}
\noindent\textbf{Blind Assistive Technology.}
Prior studies have explored the development of various forms of blind assistive technology. The most studied device has been the technology-assisted white cane, or `smart cane'. Earlier versions of a smart cane were built with ultrasonic sensors to detect objects at various distances, of which the cane would then alert the user through audio output, such as a voice message or beeping tones \cite{wahab2011smartcane, saaid2016canerange}. Later iterations have included more advanced obstacle detection systems, smartphone connectivity, and embedded navigation software, paired with a control panel to allow inputs from the user \cite{subbiah2019caneiot, batterman2018connectedcane, chen2017ccny}.  
However, smart canes, as well as other, less common forms of blind assistive technology, have struggled to be adopted by BVI white cane users, largely due to challenges with design and user interface \cite{kim2013caneusability}. Commonly reported usability issues with smart canes included slow alert times in response to obstacles, difficulty detecting fast-moving objects, inability to detect objects below knee-level (therefore still requiring ground-tapping for floor-level obstacles not unlike with a conventional white cane), and insufficient battery life when in continuous use \cite{kim2013caneusability},\cite{muhammad2010analytical}. 

% \subsection{Previous Work: Public Perception}
% A study published in 2022 by Kayukawa et al investigated the public perception and level of societal acceptance of a prototype autonomous navigation robot for blind or visually impaired users. They found that 

\noindent\textbf{Robot Dog Guide using Legged Robot.}
Previous work on developing a robot dog guide using legged robots has primarily focused on the control and navigation mechanisms of basic quadrupedal robots.
Developments include leash-guided systems~\cite{xiao2021robotic, morlando2023tethering, chen2023quadruped} and handle-guided systems~\cite{kim2023transforming, hwang2023system}, navigating indoor and irregular terrains. 
They have also begun addressing the complexities of human-robot interactions \cite{kim2023transforming, kim2023train, defazio2023seeing}, developing interaction models and force-responsive controls.
% user study
Recent studies have conducted user studies involving BVI participants to examine the potential and design of robot dog guides.
Wang et al. identified usability and trust challenges
in quadrupedal robots compared to wheeled robots~\cite{wang2022can}.
Due analyzed experience of BVI with robot and conventional dog guides, highlighting differences in assistance with navigation and mobility~\cite{due2023guide}.
Hwang et al. studied the handler-dog guide interactions, identifying the need for personalization in robot designs~\cite{hwang2024towards}.
Additionally, Hata et al. tested industrial robots as dog guide, developing user-friendly interfaces such as voice-based apps~\cite{hata2024see}.
These investigations collectively offer insights for designing robot dog guides, emphasizing user experience and effective human-robot communication.





% add more information about Taery, Morgan, and Clint's prior work on the robot 




