\section{Discussion}

After need-finding and analyzing the collected data, we identified the following requirements for future prototypes of the robot dog guide. 
\subsection{Form Factors}
  \noindent\textbf{Material.} The robot must be made of waterproof material and be easy to clean. It must also have rigid internal structures to sustain reasonable levels of impact with the environment, as well as a soft external padding to prevent injuries to the user from inevitable collisions. 
  
  \noindent\textbf{Height and Weight.} The length of the harness should be customizable to accommodate users with different heights. Extendable legs on the robot itself could also be explored. A conclusive, ideal weight for the robot was not identified, but the robot must have sufficient weight to effectively perform the guiding task, while also being light and compact enough for transportation.
  
  \noindent\textbf{Appearance.} While the robot does not need to exactly resemble a real dog, it needs to have some physical appeal to ensure a smooth transition into society. Having a general shape of a dog with a head is preferred. The robot should have a reasonably approachable presence and not be intimidating. Preferences for color were inconclusive, and many participants favored the option of it being customizable. 
  
  \noindent\textbf{Uniform Identifier.} The robot should have some form of a uniform identifier to identify it as a working guide, either on the robot itself or with a harness that is consistent with the appearance and color of a traditional dog guide harness. 
  % It may be also worthwhile to have a plaque or display on the robot that further identifies it as a service dog. 


\subsection{Functionality}
  \noindent\textbf{Navigation.} The robot should have a built-in GPS system and/or be compatible with smartphone navigation apps. 
  
  \noindent\textbf{Battery and Charging.} The charging mechanism should be straightforward, with preference for auto-charging. The battery should last as long as possible, ideally at least 8 hours, with a portable and easily changeable spare battery provided. The robot must notify the user when its battery is low and needs charging, in addition to announcing its current battery percentage when asked. 
  
  \noindent\textbf{Control and Feedback.} The robot must respond to voice commands and recognize essential dog guide commands. Participants also favored having additional control input methods, such as a small, padded controller attached to the harness handle, similar to those on some smart canes. However, comfort should remain a priority when designing the handle. The robot should also be able to provide haptic feedback through handle vibrations, as well as audio feedback both through a speaker and potentially through Bluetooth-connected earphones. In addition, participants showed enthusiasm toward the idea of incorporating an AI interface resembling commonly used AI assistants, which could further enhance usability. 

\subsection{Behavior and Interactions}
  \noindent\textbf{Focus.} The robot must focus solely on the guiding task and not interact with anyone other than the user when working. 
  
  \noindent\textbf{Intelligent Disobedience.} The robot should recognize danger and adapt to complex situations, such as overriding the user's commands when it is unsafe to proceed. The robot should provide the user with a brief explanation afterward. 
  
  \noindent\textbf{Off-Duty Mode.} Some participants were in favor of having an ``off-duty mode''. While having an off-mode is not a requirement, it could be an optional add-on for users who value companionship and affection and wish to interact with the robot like a pet dog when at home. 
