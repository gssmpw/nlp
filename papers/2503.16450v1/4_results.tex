\section{Results}


\begin{figure}[t]
\centering
\includegraphics[width=\linewidth]{image/impressionss.png}
\vspace{-1em}
\vspace{-1em}
\caption{Participants' initial impressions of the concept of a robot dog guide in Likert scale rating, BVI n = 35, ST n = 56.}
\label{fig:initial_impression}
\vspace{-1em}
\end{figure}

\subsection{Initial Impressions and Concerns}
% Initial impressions of the concept of a robot dog guide painted a picture of sleek, mechanical-looking canine with a futuristic feel, some drawing parallels to R2-D2 and Tesla vehicles. % could you explain what you're trying to convey with this sentence? i recall only two interview participants having this specific viewpoint while the rest had more realistic views -aviv 
% This part was taken from our report, I rephrased the original paragraph a little bit. But I agree that we should also include the more realistic views
When asked about their initial reaction to the concept of a `robot dog guide', $51.4\%$ of BVI participants had favorable impressions, $25.7\%$ had neutral impressions, and the remaining $22.9\%$ had unfavorable impressions (\textbf{Fig.}~\ref{fig:initial_impression}).

% important info but takes up too much space; maybe a bar graph? 



Both interview and survey participants highlighted several advantages the robot could offer over a conventional dog guide: 1) no requirement for care, training upkeep, and vet costs associated with a living dog; 2) ability of robot to focus on guiding tasks without being distracted by environmental stimuli; 3) potential integration with smart devices and navigation systems. Meanwhile, some concerns raised included: 1) the robot's reliability in ensuring safe navigation (including fear of potential malfunctions); 2) the potential lack of ``intelligent disobedience" or flexibility and adaptability in complex situations; 3) potential need for frequent charging and limited battery life.



\begin{figure*}[bt]
\centering
\includegraphics[height=0.15\textheight]{image/features5.png}
\vspace{-1em}
\caption{Likert scale ratings from BVI participants on the level of importance of potential features for the robot dog guide to have, n = 37.}
\label{fig:features_importance}
\end{figure*}

\subsection{Form Factors: Appearance and Texture}
% Interview and survey participants were asked to provide input on a series of design questions regarding the appearance, texture, and other aspects regarding to the form factors. Below is a summary of our findings from their responses.

\subsubsection{Resemblance to Real Dog}
Interview participants had divided opinions regarding the extent to which the robot dog guide should resemble a real dog, as well as its importance. DO interview participants (P1, P6, P7) generally advocated for a strong resemblance to a real dog with a furry outer surface. They stated that a dog-like appearance would be familiar and comfortable, therefore being more likely to facilitate a smoother transition into society. Meanwhile, other interview participants (P2, P3, P4, P5, P8, P9) believed it would not be necessary for the robot dog guide to resemble a real dog, placing greater emphasis on functional reliability over appearance. For example, P3 and P4 noted that a bipedal robot may perform better than a quadrupedal robot at tasks such as navigating staircases and inclined surfaces. In the survey, 45.7\% of BVI participants showed moderate to strong preference toward a robot that resembles a real dog in appearance, 22.9\% showed a slight preference, and 31.4\% showed no preference (\textbf{Fig.}~\ref{fig:features_importance}). 

\subsubsection{Physical Appeal}
All interview participants indicated that the robot must have a head and some physical appeal, because the appearance of the robot could affect both the robot's and the user's acceptance by society. P6 mentioned that having a head with a nose-like feature could also help the user to correctly identify the robot's orientation through touch. The survey results echoed these findings. A majority of BVI participants found it important for the robot to have a head and appear approachable, with nearly two-thirds also considering it important for the robot to look ``cute" (\textbf{Fig.}~\ref{fig:features_importance}).
% 73.0\% of BVI participants indicated that it was at least slightly important for the robot to have a head, with 64.9\% indicating that it was moderately to extremely important. In addition, 73.0\% of BVI participants indicated that it was at least slightly important for the robot to look ``approachable", and 58.3\% of BVI participants indicated that it was at least slightly important for the robot to look ``cute" (\textbf{Fig.} \ref{fig:features_importance}). 

\subsubsection{Uniform Identifier}
All interview participants and $85.7\%$ of survey participants indicated that the robot should have a uniform identifier to indicate that it is a working guide (\textbf{Fig.} \ref{fig:features_importance}). Some interview participants (P1, P5, P7) expressed the importance of having a uniform identifier to minimizing unwanted interactions. However, there are differing opinions regarding the specific form of the identifier, such as whether it should be incorporated into the robot dog itself or simply into the harness design, similar to a conventional, white dog guide harness that identifies the dog as a working guide. 


\begin{figure}[tbp]
\centering
\includegraphics[width=\linewidth]{image/colorandtexture.png}
\vspace{-1em}
\vspace{-1em}
\caption{Distribution of desired color (a) and outer surface texture (b) of robot as indicated by BVI participants, n = 41.}
% i like it!!
\label{fig:color_texture}
\end{figure}

\subsubsection{Color}
Interview and survey participants showed no clear preference for the robot's color, favoring customization instead. While 26.8\% had no preference, 19.5\% supported customization, and another 19.5\% preferred a neutral color like a Labrador Retriever (\textbf{Fig.}~\ref{fig:color_texture}a). The rest favored black, white, or a primary/secondary color. A white robot could reflect more light, aiding visibility for partially sighted users and identifying it as a guide.
% Interview participants had little clear preference for the color of the robot; rather, they were favorable toward it being customizable. Similar opinions were seen in the survey results, as $26.8\%$ indicated ``no preference" for color, $19.5\%$ indicated that color should be a customizable feature, and $19.5\%$ indicated that a neutral, brown or tan color resembling the color of a Labrador Retriever would be ideal. The remaining votes were distributed somewhat uniformly amongst black, white, and a primary or secondary color (\textbf{Fig.} \ref{fig:color_texture}). However, it is worth considering the fact that a white robot will reflect more light and therefore may be more easily visible to a partially sighted user, in addition to potentially helping identify the robot as a dog guide. 

\subsubsection{Outer Surface, Texture and Material}

Interview participants showed a preference toward the robot dog guide being soft, as having a soft padding would minimize the pain or discomfort caused by inevitable collisions between the robot and the user. P6 specified that even though the outer surface should be soft, the internal structure and body of the robot should be strong enough to sustain impact. P2 and P7 expressed a desire for the robot dog to be ``furry", similar to a live dog. However, there was a consensus that the fabric should also be short, waterproof, and easy to clean. For this reason, P3 and P6 instead expressed a preference for a smooth or rubbery surface. 

% The second graph of \textbf{Fig.} \ref{fig:color_texture} shows that the survey participants overall indicated a stronger preference for a ``soft", ``furry", ``fluffy", and/or ``squishy" outer texture as opposed to a ``metallic", ``plastic" or ``rigid" outer texture, which reflects the interview responses. 
% Survey participants also strongly preferred the robot being made of waterproof material, with $48.6\%$ indicating that it was ``extremely important",  $31.6\%$ indicating that it was  ``very important", and $17.1\%$ indicating that it was ``somewhat important" 
As shown in \textbf{Fig.}~\ref{fig:color_texture}b, survey participants preferred a ``soft", ``furry", or ``squishy" texture for the robot over a ``metallic" or ``rigid" one, aligning with interview responses. They also strongly favored the robot being made of waterproof material, with the majority rating it as very or extremely important. 




\begin{figure}[tbp]
\centering
\includegraphics[width=\linewidth]{image/heightandweight.png}
\vspace{-1em}
\vspace{-1em}
\caption{Distribution of desired height (a) and weight (b) of robot as indicated by BVI participants, n = 41.}
\label{fig:height_weight}
\vspace{-1em}
\end{figure}

\subsubsection{Height and Weight}
From the interviews, P2, P3, P5, and P9 indicated that the robot should resemble a Labrador Retriever or German Shepherd in size. P1, P6, P7, and P8 from group DO favored the robot having an adjustable and/or customizable height. 
They noted the need for adjustable robot legs to accommodate different heights, especially since BVI individuals may hold the harness handle for extended periods and lean on the robot for support.
P1 also suggested customizable pulling force based on the user's strength and body mass.
Although none of the interview participants provided a quantitative number, they showed an overall consensus that the robot should have sufficient mass for users to sense the direction in which they are being led but also be compact (and foldable) and light enough to be lifted for easy transportation.
Survey results showed that most participants preferred the robot to have a customizable height, with significant interest in both `12 to 24 inches' and `24 to 36 inches.' For weight, there was a wide range of preferences, but `5 to 15 pounds' was the most popular choice (\textbf{Fig.} \ref{fig:height_weight}). 
% The survey results revealed that 31.7\% of participants indicated ``12 to 24 inches" as being the ideal height, followed by 22.0\% of participants who indicated ``24 to 36 inches" as being ideal. The most popular choice, however, was ``customizable", at 39.0\%. The survey responses regarding the ideal weight of the robot exhibited significant variation, spanning from ``less than 5 lbs" to ``60 to 75 pounds", with the most popular being ``5 to 15 pounds" at 29.3\% 




% \subsection{Commands, Control, and Communication}
% \subsubsection{Essential Commands}\label{Essential Commands}
% We asked participants to make a list of essential commands that the robot dog should understand and be able to perform. Below is a summary of such commands:\\
% \begin{center}
% sit, hold, stay\\
% forward\\
% straight (cross the street)\\
% turn around\\
% left, right/left-left, right-right\\
% hurry (walk faster)\\
% easy/steady (slow down)\\
% get on/off the elevator\\
% find the \_\_\_(elevator, restroom, etc.)\\
% \end{center}

\begin{figure}[ht]
\centering
\includegraphics[width=\linewidth]{image/communicationmethod.png}
\vspace{-1em}
\vspace{-1em}
\caption{Desired methods indicated by BVI participants for giving commands to (a) and receiving communication signals from (b) the robot, n = 35.}
\label{fig:communication_methods}
\vspace{-1em}
\end{figure}

\subsection{Commands and Feedback for Control}
All interview participants indicated voice command and auditory feedback as being preferred methods of communication with the robot. They were also in favor of having an added option of voice communication via a Bluetooth earpieces, as well as having alternative command methods such as hand gestures. A suggestion emerged that the robot should have voice recognition technology that distinguishes the owner's voice and responds only to it (as well as potentially a partner or family member's voice when applicable). Several interview participants also showed interest in a simple and comfortable controller pad with support for basic commands, as well as a mobile app for monitoring the robot's maintenance needs. % huh? i'm a little confused by this sentence -aviv AVERY: sorry I was definitely not awake when I wrote that sentence, now I fixed it lmaoo
% The response from survey participants shown in (\textbf{Fig.} \ref{fig:communication_methods})  echoed the opinions gathered from interviews. The most popular method of delivering commands indicated was voice command ($29.5\%$) and buttons on the harness ($25.9\%$), followed by motion gestures ($22.32\%$), and a smartphone app ($12.5\%$). Regarding the feedback mechanism through which the robot communicates with the user, the most popular method was haptic feedback such as specific vibrations at $44.9\%$, closely followed by auditory feedback at $42.3\%$. 
% i don't think this paragraph really works because the percentages you included don't reflect the fact that this was one of the ``select all that apply" type questions. -aviv 
%Noted - Avery

Most of the survey participants wanted both voice command and control buttons on the harness handle, as well as some form of motion gestures such as pulling motions of the harness as methods of giving commands to the robot (\textbf{Fig.}~\ref{fig:communication_methods}a). A little under half of the participants also wanted the option of using a smartphone app connected to the robot. A minority of participants wanted remote control as an additional option. All participants wanted haptic feedback (such as vibrations and touch cues), and most also wanted auditory output (\textbf{Fig.}~\ref{fig:communication_methods}b). The ``Other" responses for both questions indicated that it should be customizable, allowing the robot and user to adapt to various situations (e.g., voice command and auditory signals might be ideal in most cases, but button inputs and haptic feedback could be preferred in quiet environments or when it is too loud for effective voice communication). In addition, some users may prefer one method while others may prefer another. 



% \subsubsection{Harness and Handle}
% All interview participants indicated that it would feel most natural for them to interact with the robot dog with a harness attached to a rigid handle. Having a rigid handle would allow users to accurately feel the direction in which the robot dog is guiding them. Participants P6 and P7 emphasized a strong desire for a padded harness handle made of a soft material to reduce any discomfort associated with the handle rubbing against their palms. 
% AVERY: I don't think we asked this question in the survey
% aviv: is it okay if we take this part out? 

% All interview participants identified voice command and receiving auditory feedback as their preferred method of controlling the robot. 
% Participants B and F liked the idea of connecting the robot to Bluetooth earphones or hearing aid so that that only they could hear the robot's voice feedback, which avoids awkwardness of having it go off in public. Participant G mentioned having an alternate mode of command such as hand gestures, similar to how dog guides understand both physical and audio cues. Regardless of the form of command mechanism, participants noted that the commands should be consistent with those given to dog guides. Similar to Alexa, the robot should be able to recognize its owner's voice from background noises and other people speaking, as well as verbally confirm the command by repeating it back to its owner. \\
% Many participants expressed interest in a small controller pad attached to the rigid handle with a few simple buttons such as ``stop", ``forward", etc. The interface should not be complex at all and have tactile symbols to denote the functionality of each button. The buttons should be significantly large and distinguishable from each other. Participant G noted that if a controller pad were to be included, extra padding must be provided to prevent the controller from digging into the user's hand. \\
% Participants B, E, and F are also interested in having having a mobile app that allows them to control the robot, monitor battery power level, and check for any maintenance or upgrade alerts from their phone. 

\begin{figure*}[htbp]
\centering
\includegraphics[height=0.09\textheight]{image/behaviors3.2.png}
\vspace{-1em}
\caption{Likert scale ratings from BVI participants on the level of importance of potential behaviors and interactions for the robot guide dog, n = 35.}
\label{fig:behaviors_importance}
\vspace{-1em}
\end{figure*}

\subsection{Behaviors and Social Interactions}
The interview participants had diverging opinions regarding the robot dog guide's social abilities and interactions with others. P1 and P7 indicated that the robot should have an off-duty mode that enables the robot to behave like a pet when safely at home. The participants expressed an appreciation for the warmth and companionship provided by a live dog guide as something they would ``really miss" if they were to have a robot dog guide instead. They were also more open to having the robot interact with and socialize with other people, although it would still be important for the interactions to be limited to only when the off-duty mode is activated. P3, P4, and P5 held neutral opinions, while P2, P6, P8, and P9 held an opposing view. They stated that the robot's functionality and reliability should be the priority, and that adding pet-like behaviors unrelated to the main objective of a mobility aid would not only be unnecessary, but also distracting and potentially even ``creepy''. Specifically, P2 mentioned that she would only want to use the robot as a tool like Google Maps, not for companionship. 
% The survey results indicated that the majority of participants believe the robot should be able to interact with other people and/or other pets. $45.7\%$ responded ``completely agree'' and $20.0\%$ responded ``somewhat agree''. However, there was no clear consensus on whether it should have an ``off-duty'' or ``pet'' mode. 17.1\% responded ``completely agree'', 17.1\% responded ``somewhat agree'', 25.7\% responded ``neither agree nor disagree'', 14.2\% responded ``somewhat disagree'', and 25.7\% responded ``completely disagree''
The majority of survey participants felt the robot should interact with other people and pets, although opinions were divided on the desire for an `off-duty' or `pet' mode, with no clear consensus emerging (\textbf{Fig.}~\ref{fig:behaviors_importance}).

We also asked the survey participants to rate the importance of several additional features pertaining to the behaviors and social interactions of the robot (\textbf{Fig.}~\ref{fig:behaviors_importance}). 
On average, the participants completely agreed that the robot should adapt to and learn from the owner’s behaviors and preferences, that it should automatically charge itself, and that the participants would feel comfortable with their data being collected for the purpose of improving service. However, some participants indicated that they are concerned with privacy issues associated with the robot having a camera and preferred to have the option of turning it off. 



\subsection{Importance of Aesthetic and Functional Features}
In the survey, we asked BVI participants to rate the importance of several aesthetic and functional features that were mentioned in the interviews (\textbf{Fig.}~\ref{fig:features_importance}). On average, participants rated having a resemblance to a real dog, looking cute, and having a storage compartment as being slightly important. They rated having a head, looking approachable, having a uniform appearance (universal indicator), and having a built-in GPS function as moderately important, having wireless/Bluetooth connectivity as very important, and having a battery percentage notification and the robot being made of waterproof material as being extremely important. 
We performed a T-test for the group difference in the mean ratings of aesthetic factors (having a head, dog resemblance, approachable, cute, uniform) and the mean ratings of functionality factors (GPS, Bluetooth, battery notification, storage compartment, waterproof), and obtained a p-value less than $0.001$. The low p-value suggests that the BVI participants overall placed higher importance on the functionality factors than on the aesthetic factors.
% of 0.0004168

\begin{figure*}[htbp]
\centering
\includegraphics[height=0.16\textheight]{image/movements3.1.png}
\vspace{-1em}
\caption{Likert scale ratings from BVI participants on the level of importance of movement abilities that the robot dog guide should have, n = 35.}
\label{fig:movements_importance}
\vspace{-1em}
\end{figure*}

\subsection{Importance of Movement Capabilities}
We asked survey participants to rate the importance of different movement capabilities on a Likert Scale from 1 to 5, with 1 being not important and 5 being extremely important (\textbf{Fig.}~\ref{fig:movements_importance}). On average, the participants rated the robot’s ability to run as being slightly important and the ability to move backwards as being moderately important. All other types of movements asked (variable speeds, sharp turns, abrupt stop, navigating ramps and stairs, crawling into small spaces, returning on command, charging itself, standby) were rated as being very or extremely important. Some other movements not included in the survey but mentioned by participants include: following on command, automatically stopping when dangerous, and navigating in harsh weather conditions. 

\begin{figure}[htbp]
\centering
\includegraphics[width=\linewidth]{image/battery3.png}
\vspace{-1em}
\vspace{-1em}
\caption{Minimum number of hours that the battery should last on one charge as indicated by BVI participants for a single battery (a) and if a spare battery is provided (b).}
\label{fig:battery_life}
\vspace{-1em}
\end{figure}

\subsection{Battery}
While the responses varied, the interview participants mostly indicated a need for the battery to last at least 8 to 12 hours with one charge, or to last an entire day. P7 responded favorably to a spare battery being provided. However, P1 and P6 expressed concern that the spare battery would be too heavy to carry around. P6 indicated that the charging process should be done in a wireless manner, similar to the mag-safe charger of an iPhone. 

The majority of the survey participants who responded also indicated that the battery of the robot should last 10 or more hours on one charge (\textbf{Fig.}~\ref{fig:battery_life}). The mean response was 8.84 hours (with 12 as the maximum) with a standard deviation of 3.48 hours. If provided an easily replaceable, spare battery, the participants were overall more amenable to a shorter battery life. In this case, fewer than a third of the participants indicated that the robot should have a minimum battery life of 10 or more hours. The mean response was 7.30 hours, with a standard deviation of 3.26 hours. 



\subsection{Responses from Sighted Participants} 
Among ST participants, 21.4\% felt uncomfortable around dogs, and 19.6\% were afraid of them. Of the 24 who expressed fear or dislike of dogs, 37.5\% would feel ``more comfortable with a robot dog than a real dog", though 11.4\% expected they might fear the robot dog. Overall, ST participants had a positive initial impression of a robot dog guide, with 53.6\% reporting favorable views, 28.6\% neutral, and 17.9\% unfavorable—similar to BVI participants (\textbf{Fig.}~\ref{fig:initial_impression}).

ST participants were less supportive of a fur-like texture than BVI participants, with only 17.0\% in favor, while 54.7\% disagreed. However, they placed greater importance on the robot's appearance, with 52.8\% indicating it should look ``cute" and 66.0\% believing it should look ``approachable". Additionally, 77.4\% agreed that a robot dog guide should be allowed everywhere, with few exceptions.
% 21.4\% of ST participants indicated that they felt uncomfortable around dogs and 19.6\% indicated that they were afraid of dogs. Out of 24 ST participants who indicated some form of fear or dislike of dogs, 37.5\% responded that they would feel ``more comfortable with a robot dog than a real dog.'' 
% 11.4\% of ST participants expected that they ``would be afraid of the robot dog.'' However, ST participants had an overall positive initial impression of a robot dog guide, with 53.6\% indicating a favorable impression, 28.6\% indicating a neutral impression, and 17.9\% indicating an unfavorable impression, similar to the responses from BVI participants. 

% Interestingly, ST participants were less in favor of the robot having a fur-like texture as compared to the BVI participants, with only 17.0\% agreeing with it, 28.3\% having a neutral stance, and 54.7\% disagreeing with it. On the other hand, the ST participants overall placed greater importance in the robot's appearance being ``cute'' or ``approachable''. 52.8\% indicated that the robot should look ``cute'' while 32.1\% remained neutral, and 66.0\% indicated that the robot should look ``approachable'' while 20.8\% remained neutral. 

% 77.4\% of ST participants indicated that they believe a robot dog guide should be allowed anywhere with little to no exceptions, 13.2\% remained neutral, and 9.4\% disagreed. 
% Meanwhile, 85.7\% of ST participants indicated that they believe a conventional dog guide should be allowed anywhere with little to no exceptions, 5.4\% remained neutral, and 8.9\% disagreed. 

