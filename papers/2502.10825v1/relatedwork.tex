\section{Related Works}
Al-Sada Bader et al. \cite{al2024mitre} categorize 50 research contributions based on various use cases, including behavioral analytics, red teaming, defensive gap assessment, and CTI enrichment. This categorization highlights different applications, methodologies, and data sources in the implementation of ATT\&CK, with a focus on adversarial behavior modeling, automated threat detection, and framework enhancements. The categorization draws from a white paper \cite{strom2018mitre} discussing the design philosophy, structure, and use cases of MITRE ATT\&CK, and a similar white paper \cite{alexander2020mitre} on the development and application of the ATT\&CK for ICS framework.

Roy Shanto et al. \cite{roy2023sok} provide a taxonomic classification of ATT\&CK-related research, identifying use cases, application domains, and methodologies. Key areas include CTI, intrusion detection, offensive security, cyber risk assessment, professional training, and threat-driven approaches. This study also identifies a gap between academic and industry use of ATT\&CK, with industry focusing on practical implementations and tools, while academia explores theoretical models.

Several studies focus on specific applications of ATT\&CK. Kris Oosthoek and Christian Doerr \cite{oosthoek2019sok} analyze 951 Windows malware families from the Malpedia repository, mapping post-compromise malware techniques to ATT\&CK and examining techniques observed in a controlled sandbox environment. Joshua Bolton et al. \cite{bolton2023overview} explore the use of knowledge graphs (KG) in cybersecurity, particularly in conjunction with MITRE ATT\&CK.

Given the rapid evolution of cyber threats and the increasing complexity of defensive strategies, understanding the MITRE ATT\&CK framework's role across cybersecurity domains is crucial. Despite its widespread adoption, there is limited systematic analysis of ATT\&CK’s application across sectors, methodologies, and advanced technologies. To address this, we conduct a comprehensive literature review analyzing ATT\&CK’s application in threat intelligence, incident response, attack modeling, and vulnerability prioritization.