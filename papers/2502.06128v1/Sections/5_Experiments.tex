\section{Experiments}
We developed a 4-EA (2 $\times$ 2) prototype OWE and conducted experiments based on it. These experiments were designed to achieve three objectives:

\begin{enumerate}
    \item Validate the feasibility of OWE, particularly its capability for unsaturated operation and SNR enhancement.
    \item Explore a different setting than the simulation setting, where multiple EAs can receive the transmitted signal from the station (i.e., the setting of (\ref{sum_obj_fun})) as opposed to the setting described in (\ref{snr_sa}).
    \item Demonstrate the deployment of OWE over a Wi-Fi-based TCP/IP network, showing that it can significantly reduce not only the SNR but also the packet loss rate in the network.
\end{enumerate}
\subsection{Hardware Design}
We designed and built a set of optical transmitters and receivers incorporating a commercial off-the-shelf PGA to construct EAs. Fig. \ref{transmitter_diagram} and Fig. \ref{receiver_diagram} show the transmitter and receiver diagrams respectively.
\begin{figure}
\centering
\includegraphics[width=0.4\textwidth]{Figures/Transmitter_Diagram.png}
\caption{Optical transmitter diagram.}
\label{transmitter_diagram}
\vspace{-0.2cm}
\end{figure}

The optical transmitter converts the electrical signal into an optical signal by modulating the power of light radiation. We selected a power amplifier (PA) with a strong output current capability to directly drive LEDs, as depicted in Fig. \ref{transmitter_diagram}. By adding a proper DC bias to the input signal, the transmitter ensures an appropriate LED operating point where the radiated light intensity is proportional to the forward driving current.

\begin{figure}
\centering
\includegraphics[width=0.4\textwidth]{Figures/Receiver_Diagram.png}
\caption{Optical receiver diagram.}
\label{receiver_diagram}
\vspace{-0.5cm}
\end{figure}

The optical receiver comprises a photodiode (PD), a trans-impedance amplifier (TIA), and a low-noise amplifier (LNA). The signal flow in the receiver can be segmented into three stages, as depicted in Fig. \ref{receiver_diagram}. Initially, the PD captures the modulated light signal and converts it into a photocurrent. Subsequently, the TIA transforms the photocurrent into a voltage signal; the signal strength remains weak at this stage. Finally, an LNA is used to amplify the signal.
\par
Fig. \ref{PCB_Render} shows the printed circuit boards (PCBs) design of both the transmitter and the receiver, which are highly integrated and can provide a -3 dB bandwidth of over 40 MHz.
\begin{figure}
\centering
\includegraphics[width=0.5\textwidth]{Figures/Transceiver.png}
\caption{The rendered image shows the PCBs of both the transmitter and the receiver.}
\label{PCB_Render}
\vspace{-0.2cm}
\end{figure}
We designed casings for both the transmitter and the receiver and assembled them with a bracket. A Fresnel lens is mounted on the receiver casing to focus the optical signal onto the detecting surface of the PD. The optical transceiver front ends integrated in EAs are depicted in Fig. \ref{Transceiver_Rendered}.

\begin{figure}
\centering
\includegraphics[width=0.5\textwidth]{Figures/Transceiver_Box.png}
\caption{The rendered image of the optical transceiver front ends integrated in EAs.}
\label{Transceiver_Rendered}
\vspace{-0.5cm}
\end{figure}

\subsection{Experimental Setups}
4 EAs in the prototype OWE were positioned at the corners of a 1.8 m $\times$ 1.8 m square experimental area. The one acting as the AP was placed at the bottom-right corner (from the top view of the room), referring to Fig. \ref{experimental_settings}. The installation height of each EA from the floor was 2 m, and the orientation of the EA's optical front ends was not perpendicular to the ground but had a small tilt angle. Additionally, a small optical transmitter with a radiation power approximately 2/9 that of the EAs functioned as the station in the uplink settings and was placed within a 0.6 m $\times$ 0.6 m square at the center of the experimental area.

\begin{figure}
\centering
\includegraphics[width=0.5\textwidth]{Figures/experimental_settings.png}
\caption{Experimental scene and settings.}
\label{experimental_settings}
\vspace{-0.5cm}
\end{figure}

\par
We selected 16 evenly distributed sampling points (purple circles in Fig. \ref{experimental_settings}) within the central 0.6 m $\times$ 0.6 m square area to evaluate performance based on two metrics: the SNR and the packet loss rate.
\par
Although the demo system uses infrared light as the communication medium, our current receiver design does not selectively receive this infrared light and can still be affected by ambient light. Therefore, the experiments were conducted in darkness to minimize interference from artificial light sources, such as fluorescent and LED lamps, which have fluctuations in their radiation power and may impact the accuracy of measuring signal quality.

\subsection{SNR}
In the SNR test experiments, the station was placed on the floor and it continuously sent a 1 MHz sinusoidal waveform. We measured the SNR of the signal received by the AP under different EA gain settings.
\par
First, we collected SNR values with all EAs disabled, in which case the signal can only propagate through LOS paths. In a large-scale network, if the distance between the station and the AP is too long without any available LOS paths, the connection will be lost.
\par
Second, we explored the equal-gain setting, where we maximized a common gain among all EAs subject to the non-saturation constraints, ensuring the stability of the overall OWE.
\par
Third, we applied the optimal gain settings derived by optimizing the objective function (\ref{sum_obj_fun}). This experiment only verified the performance under single-BSS scenarios. The multiple-BSS scenarios will be experimentally explored in our future work.

\begin{figure*}[htbp]
\centering
\subfloat[SNR of 16 sampling points for three EA gain settings.\\
Note: The grids in this figure represent different station positions used for SNR measurements (see Fig. \ref{experimental_settings}), rather than EA positions as in Fig. \ref{SISO_simulation_res} and Fig. \ref{MIMO_examples}.]
{
  \label{snr_surf}
  \includegraphics[width=0.6792\textwidth]{Figures/SNR_surf.png}
}
% \vspace{-0.1cm}
\subfloat[SNR surf plot.]
{
  \label{snr_stack}
  \includegraphics[width=0.2707\textwidth]{Figures/SNR_stack_1.png}
}
\caption{SNR measurement results.}
\label{snr_experiments}
\vspace{-0.3cm}
\end{figure*}

\begin{table}
\centering
\begin{tabularx}{0.48\textwidth}{p{1.8cm} p{1.7cm} X X}
\toprule
\textbf{Gain Settings} &
  \textbf{Disable EAs} &
  \textbf{\begin{tabular}[c]{@{}l@{}}Maximum\\ Equal Gain\end{tabular}} &
  \textbf{\begin{tabular}[c]{@{}l@{}}Optimal\\ Gain Settings\end{tabular}} \\ \midrule
\textbf{Average SNR} &
  21.2048   dB &
  26.6149   dB &
  28.2487   dB \\ \bottomrule
\end{tabularx}
\caption{Average SNR of three EA gain settings.}
\label{average_snr}
\vspace{-0.5cm}
\end{table}
\par
As shown in Fig. \ref{snr_experiments} and Table \ref{average_snr}, the SNR of the received signal at the AP can be improved by applying proper EA gain settings, especially at positions with weak LOS paths such as sampling points numbered 1, 2, and 5.
Note that despite sampling point 16 being the closest to the AP at the lower right corner, the LOS signal transmission encountered significant attenuation due to the tilted installation of the optical front ends, resulting in degraded signal quality. Moreover, since sampling point 16 was distant from and had a feeble connection with the other three EAs in the OWE, it would not derive substantial benefits from gain value optimization.
\par
Overall, a significant improvement of over 7 dB in average SNR was achieved by deploying the optimal EA gain settings compared to having EAs disabled. In general, the SNR performance with the optimal EA gain settings surpasses that with maximum equal gain settings, which, in turn, outperforms the scenario with EAs disabled.  Notably, we have shown that by carefully tuning the gain values of each EA, the OWE can operate stably, avoiding amplifier saturation.

\subsection{Packet Loss Rate}
In packet loss rate experiments, we used USRP \cite{ettus2015universal} to generate Wi-Fi packets with three modulations: 16-QAM 1/2 coding rate, 16-QAM 3/4 coding rate, and 64-QAM 1/2 coding rate. The USRP generated the Wi-Fi signal at a frequency of 2.4 GHz with a 5 MHz modulation bandwidth. 4095 packets were generated in each trial. 
\par
We employed the method introduced in \cite{cui2024wi} to down-convert the 2.4 GHz RF signal to an intermediate frequency signal and then convert it to an intensity-modulated light signal at the transmitter side. A reverse process was conducted at the receiver side to convert the light signal back to the RF signal for capture by the USRP.
\par
The results are summarized in Fig. \ref{plr_experiments} and Table \ref{average_plr}. In Fig. \ref{plr_experiments}, from left to right, the SNR performance of the deployed EA gain settings improves, while from top to bottom, the channel quality requirement of modulation schemes increases. The subfigure in the top-right corner of Fig. \ref{plr_surf} shows an average packet loss rate of 0.76\%, which is acceptable for most applications. For example, TCP generally works well when the packet loss rate is less than 1\%. Furthermore, Wi-Fi also has a MAC layer ARQ (automatic repeat request) that further decreases the packet loss rate as observed by TCP/IP applications.  In summary, implementing the optimal EA gain settings significantly improves the packet loss rate, with a maximum reduction of about tenfold, as shown in Table \ref{average_plr}.
\begin{figure*}[htbp]
\centering
\subfloat[Packet loss rate over different sampling points, EA gain settings, and modulations.]
{
  \label{plr_surf}
  \includegraphics[width=0.715\textwidth]{Figures/Packet_Loss_Rate_Surf.png}
}
% \vspace{-0.1cm}
\subfloat[Packet loss rate surf plots.]
{
  \label{plr_stack}
  \includegraphics[width=0.245\textwidth]{Figures/Packet_Loss_Rate_Stack.png}
}
\caption{Packet loss rate measurement results.}
\label{plr_experiments}
\vspace{-0.3cm}
\end{figure*}

\begin{table}
\centering
\begin{tabularx}{0.48\textwidth}{p{1.7cm} p{1.7cm} p{1.7cm} X}
\toprule
 &
  \textbf{\begin{tabular}[c]{@{}l@{}}Disable EAs\end{tabular}} &
  \textbf{\begin{tabular}[c]{@{}l@{}}Maximum\\ Equal Gain\end{tabular}} &
  \textbf{\begin{tabular}[c]{@{}l@{}}Optimal\\ Gain Settings\end{tabular}} \\ \midrule
\textbf{16-QAM, ½} & 6.6147 \%  & 2.7426 \%  & 0.7631 \%  \\
\textbf{16-QAM, ¾} & 14.4109 \% & 5.6227 \%  & 4.3635 \%  \\
\textbf{64-QAM, ½} & 21.4956 \% & 13.4477 \% & 11.0911 \% \\ \bottomrule
\end{tabularx}
\caption{Average packet loss rates over different modulations and EA gain settings.}
\label{average_plr}
\vspace{-0.5cm}
\end{table}
