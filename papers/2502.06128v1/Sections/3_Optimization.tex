\section{Objective-Oriented Optimization}

We then explore the performance optimization of OWE with various objectives and metrics. We consider two application scenarios: A single basic service set (BSS) and multiple BSSs.

\subsection{Single-BSS Scenario}
First, we consider the OWE application scenario with a single BSS, where the OWE establishes connections between stations and a single AP, and only one data stream is propagated through the OWE at a time.
Suppose that we want to find the optimal gain settings of EAs in the OWE that maximize the uplink SNR:
\begin{equation}
\label{eq17}
\begin{aligned}
    &\max_\mathbf{G} \text{SNR}_{SA} \\
    &\textit{s.t.} \\
    & \rho\left({\mathbf{H}^T}\mathbf{G}\right)=|\lambda_{\max}\left({\mathbf{H}^T}\mathbf{G}\right)|<1,{\ }\\
    & G_i\geq0,{\ }i=1,\dots, N \text{,}
\end{aligned}
\end{equation}
where $\rho(\cdot)$ represents the spectral radius of a matrix. For simplicity, we disregard the phase shift introduced by the amplifier since it can be compensated by a properly designed all-pass filter \cite{THOMPSON2014531} in the hardware implementation. The gain of the \textit{i}-th EA, $G_i$, is then a real number.
\par
Recall that (\ref{ap_snr}) is the SNR of the received signal at the AP from the station. This expression is written in a very general way and can be further refined by incorporating practical implementation details.
\par
First, in most application scenarios, there will be no LOS signal path available from the station to the AP, and the gain tunning of EAs minimally affects this LOS signal path. For simplicity, we can exclude the LOS signal component, ${{\tilde h}_{SA}}\left( f \right){{\tilde s}_{S}}\left( f \right)$.
\par
Second, the noise introduced by the AP, ${{\tilde n}{\scriptscriptstyle A}}$, is much smaller than the noise carried in the received signal and can be disregarded.
\par
Third, we consider the channel response from the OWE to the AP. In the diagram of the AP implementation depicted in Fig. \ref{OWE_AP}, an AP is connected to an EA and shares its optical transceiver frontends. In the following, we assume the design in Fig. \ref{OWE_AP} as opposed to the AP in Fig. \ref{ea_ap}, which is a distinct element from the EA. The signal propagation from the OWE (the connected EA) to the AP has no further amplification and the wired connection channel is assumed to be ideal. Therfore, the component $\mathbf{G}{{\mathbf{\tilde h}}_{EA}}$ in (\ref{ap_snr}) can be replaced by a standard basis. Let us define $\mathbf{w}_{EA}^T=[0,0,\dots,0, 1, 0\dots, 0]$ with the $1$ appearing in the index corresponding to the EA connected to the AP.
\begin{figure}
\centering
\includegraphics[width=0.2\textwidth]{Figures/OWE_AP_Diagram_2.png}
\caption{The implementation diagram of the OWE AP, which shares optical transceiver frontends with an EA.}
\label{OWE_AP}
\vspace{-0.5cm}
\end{figure}
\par
Finally, we examine the channel response from the station to the OWE. Because light is highly directional, the transmitted signal from the station can only be received by nearby EAs. We start with the simplest scenario, assuming that only one EA can receive the signal from the station. 
For example, this setting can be achieved using the omnidirectional transceiver design described in our previous works \cite{cui2024wi, cui202412}. In this design, multiple LEDs are oriented in different directions to provide omnidirectional coverage. At any given time, one LED can be selected while the others are turned off to focus on a specific direction.
We will extend the treatment to scenarios with multiple receiving EAs shortly.
Let ${{\mathbf{\tilde h}}_{SE}}{{\tilde s}_{S}} = \mathbf{w}_{SE} \psi$, where $\mathbf{w}_{SE}^T=[0,0,\dots,0, 1, 0,\dots, 0]$ is a standard basis with the element $1$ corresponding to the EA selected by the station to receive the signal, and $\psi$ is a constant representing the signal received by the EA. Because $\psi$ can be mathematically moved outside the maximization operator and is irrelevant to the optimization of $\mathbf{G}$ as shown in (\ref{snr_sa}), without loss of generality, we assume $\psi=1$, representing that the received signal of the EA is an impulse signal and can thereby be eliminated from the objective function.
\par
By simplifying the original objective function in (\ref{ap_snr}) following the steps above, we derive (\ref{snr_sa}):

\begin{align}
\label{snr_sa}
& \max_\mathbf{G} \text{SNR}_{SA} \notag \\
& = \max_\mathbf{G} \frac{{{{\left| {{{\tilde h}_{SA}}\left( f \right){{{\tilde s}_{S}}\left( f \right)} + {{\left( {\mathbf{G}{{\mathbf{\tilde h}}_{EA}}} \right)}^T}\mathbf{A} \left({{\mathbf{\tilde h}}_{SE}}{{{\tilde s}_{S}}\left( f \right)}\right)} \right|}^2}}}{{{{\left| {{{\left( {\mathbf{G}{{\mathbf{\tilde h}}_{EA}}} \right)}^T}\mathbf{A}{\mathbf{H}^T}\left( {\mathbf{G\tilde n + }{{\mathbf{\tilde n}}_\mathbf{a}}} \right) + {{\tilde n}_{A}}} \right|}^2}}} \notag \\
& \Leftrightarrow \max_\mathbf{G}
\frac{{{{\left|{{\left( {\mathbf{G}{{\mathbf{\tilde h}}_{EA}}} \right)}^T}\mathbf{A} \left({{\mathbf{\tilde h}}_{SE}}{{{\tilde s}_{S}}\left( f \right)}\right) \right|}^2}}}{{{{\left| {{{\left( {\mathbf{G}{{\mathbf{\tilde h}}_{EA}}} \right)}^T}\mathbf{A}{\mathbf{H}^T}\left( {\mathbf{G\tilde n + }{{\mathbf{\tilde n}}_\mathbf{a}}} \right)} \right|}^2}}} \notag \\
&\Leftrightarrow
\max_\mathbf{G} 
\frac{{{{\left|\mathbf{w}_{EA}^T\mathbf{A}{\left(\mathbf{w}_{SE} \psi\right)} \right|}^2}}}{{{{\left| {\mathbf{w}_{EA}^T\mathbf{A}{\mathbf{H}^T}\left( {\mathbf{G\tilde n + }{{\mathbf{\tilde n}}_\mathbf{a}}} \right)} \right|}^2}}} \notag \\
&\Leftrightarrow
|\psi|^2 \max_\mathbf{G} 
\frac{{{{\left|\mathbf{w}_{EA}^T\mathbf{A}{\mathbf{w}_{SE}} \right|}^2}}}{{{{\left| {\mathbf{w}_{EA}^T\mathbf{A}{\mathbf{H}^T}\left( {\mathbf{G\tilde n + }{{\mathbf{\tilde n}}_\mathbf{a}}} \right)}\right|}^2}}} \notag \\ 
&\Leftrightarrow \max_\mathbf{G} 
\frac{{{{\left|\mathbf{w}_{EA}^T\mathbf{A}{\mathbf{w}_{SE}} \right|}^2}}}{{{{\left| {\mathbf{w}_{EA}^T\mathbf{A}{\mathbf{H}^T}\left( {\mathbf{G\tilde n + }{{\mathbf{\tilde n}}_\mathbf{a}}} \right)} \right|}^2}}} \\
&\textit{s.t.} \notag \\
& \rho\left({\mathbf{H}^T}\mathbf{G}\right)=|\lambda_{\max}\left({\mathbf{H}^T}\mathbf{G}\right)|<1, \notag\\
& G_i\geq0,{\ }i=1,\dots, N \text{.} \notag
\end{align}
Since the constant of the received signal power $|\psi|^2$ is eliminated, the last line of (\ref{snr_sa}) no longer represents the SNR but is equivalent in the optimization for the values of $\mathbf{G}$. Furthermore, if we incorporate the selection of an \textbf{entry EA} for receiving the signal from the station into the optimization, the objective function would be
\begin{equation}
\begin{aligned}
& \max_\mathbf{G} \text{SNR}_{SA} \\
& \Leftrightarrow 
\max_{i}\left[ |\psi_i|^2 \max_\mathbf{G} 
\frac{{{{\left|\mathbf{w}_{EA}^T\mathbf{A}{\mathbf{w}_{SE(i)}} \right|}^2}}}{{{{\left| {\mathbf{w}_{EA}^T\mathbf{A}{\mathbf{H}^T}\left( {\mathbf{G\tilde n + }{{\mathbf{\tilde n}}_\mathbf{a}}} \right)}\right|}^2}}}\right]
\end{aligned}
\end{equation}
where ${{\bf{w}}_{SE(i)}}$ is the standard basis representing the \textit{i}-th EA as the entry EA, and $\psi_i$ is the corresponding received signal.
\par
In scenarios where the station is not capable of selecting an EA to receive, and multiple EAs can receive signals from the same station, the objective function can be written as the weighted summation over (\ref{snr_sa}), as shown in (\ref{sum_obj_fun}).
\begin{equation}
\label{sum_obj_fun}
\begin{aligned}
\mathop {\max }\limits_{\bf{G}} \text{SNR}_{SA}
&\Leftrightarrow \mathop {\max }\limits_{\bf{G}} \frac{{\sum\limits_i {{{\left| {{\bf{w}}_{EA}^T{\bf{A}}\left( {{\bf{w}}_{SE(i)}} \psi_i \right)} \right|}^2}}}}{{{{\left| {{\bf{w}}_{EA}^T{\bf{A}}{{\bf{H}}^T}\left( {{\bf{G\tilde n}} + {{{\bf{\tilde n}}}_{\bf{a}}}} \right)} \right|}^2}}}\\
&\Leftrightarrow \mathop {\max }\limits_{\bf{G}} \frac{{{{\left| {{\bf{w}}_{EA}^T{\bf{A}}\left(\sum\limits_i {{{\bf{w}}_{SE(i)}}} \psi_i \right) } \right|}^2}}}{{{{\left| {{\bf{w}}_{EA}^T{\bf{A}}{{\bf{H}}^T}\left( {{\bf{G\tilde n}} + {{{\bf{\tilde n}}}_{\bf{a}}}} \right)} \right|}^2}}} \\
&\textit{s.t.}\\
& \rho\left({\mathbf{H}^T}\mathbf{G}\right)=|\lambda_{\max}\left({\mathbf{H}^T}\mathbf{G}\right)|<1,\\
& G_k\geq0,{\ }k=1,\dots, N \text{.}
\end{aligned}
\end{equation}
The deduction in (\ref{sum_obj_fun}) holds because the product of standard bases ${{{\bf{w}}_{SE(i)}}}{{{\bf{w}}_{SE(j)}^H}}=0 \text{, where } i\neq j$.

\subsection{Multiple-BSS Scenario}
Next, we examine a scenario involving multiple BSSs, where multiple optical signals flow simultaneously through the OWE, as depicted in Fig. \ref{spatial_reuse}. Minimizing mutual interference among the BSSs is crucial. Furthermore, the collective optimization across the BSSs must ensure fairness in resource allocation. The new objective function in this scenario aims to maximize the signal-to-interference-plus-noise ratio (SINR) values of signals propagating through the OWE, while also addressing fairness concerns in allocating OWE resources among the multiple BSSs

\begin{figure}
\centering
\includegraphics[width=0.5\textwidth]{Figures/Spatial_Reuse_2.png}
\caption{A multiple-BSS scenario, where station 1 communicates with AP1 through the OWE, and simultaneously, station 2 communicates with AP2 through the OWE. The EAs involved in the signal propagation for each specific BSS are highlighted in the same color.}
\label{spatial_reuse}
\vspace{-0.5cm}
\end{figure}

\par
For simplicity, we make the following two assumptions: First, the station can selectively focus the light signal on one EA.
Second, each AP can serve only \textbf{one} station at a time.

\subsubsection{Mutual Interference}
The signal cross-talk between BSSs, resulting from interactions among EAs, leads to mutual interference, as depicted in Fig. \ref{spatial_reuse}. For instance, the signal transmitted by station \textit{j} to the AP in the \textit{j}-th BSS could be received by the AP in the \textit{i}-th BSS ($i\neq j$), causing interference to the ongoing communication in the \textit{i}-th BSS. Recall that the received signal of EAs in the OWE can be computed using (\ref{eq10}). To compute the mutual interference among BSSs we omit the noise-related components and substitute the signal input $\mathbf{\tilde{x}}$ with ${\mathbf{\tilde{h}}_{SE}}{\tilde{s}_{S}}(f)$ (as defined in Table \ref{Table2}). Further, we replace ${{\mathbf{\tilde h}}_{SE}}{{\tilde s}_{S}}$ with $\mathbf{w}_{SE} \psi$ as in (\ref{snr_sa}) and finally obtain
\begin{equation}
\begin{aligned}
    \mathbf{\tilde y} & = \mathbf{A}\mathbf{\tilde x} \\
    & \Rightarrow \mathbf{A} {{\mathbf{\tilde h}}_{SE}}{{{\tilde s}_{S}}} \\
    & \Rightarrow \mathbf{A}{{\bf{w}}_{SE}}\psi \text{.}
\end{aligned}
\end{equation}
\par
Referring to Fig. \ref{OWE_AP}, the channel response between EAs in the OWE and the AP can be represented by a standard basis $w_{EA}$ defined in (\ref{snr_sa}). Therefore, the mutual interference from the \textit{j}-th BSS to the \textit{i}-th BSS can be derived by
\begin{equation}
\label{mutual_interference}
\begin{aligned}
I_{m[j,i]}&=\mathbf{\tilde y}_{[j]}^T w_{EA[i]}\\
&=\psi_j{{\bf{w}}_{SE[j]}^T}\mathbf{A}^Tw_{EA[i]} \text{,} \\
i\neq j \text{.}
\end{aligned}
\end{equation}
Here, [\textit{i}] and [\textit{j}] indicate different BSSs.

\subsubsection{Fairness}

We assume that signals propagating simultaneously through the OWE within different BSSs have equal importance. However, directly maximizing the summation of SNRs over signals in the collective optimization objective function may lead to fairness issues (putting aside the mutual interference issue in this section). Signal degradation during propagation suggests that signals with shorter propagation distances may achieve higher SNR values more easily, potentially exerting significant dominance over the optimization results. 
To rectify this imbalance, we normalize the signals' SNRs by dividing them by their respective optimal SNRs obtained from optimizing the single-BSS objective function (\ref{snr_sa}), where each signal exclusively utilizes the OWE.
The optimal SNR value of the \textit{i}-th BSS, denoted by $\lambda_i$, can be computed by
\begin{equation}
\label{optimal_snr}
\lambda_i = \max_{\mathbf{G}}
\frac{{{{|\psi_i|^2\left|\mathbf{w}_{EA[i]}^T\mathbf{A}{\mathbf{w}_{SE[i]}} \right|}^2}}}{{{{\left| {\mathbf{w}_{EA[i]}^T\mathbf{A}{\mathbf{H}^T}\left( {\mathbf{G\tilde n + }{{\mathbf{\tilde n}}_\mathbf{a}}} \right)} \right|}^2}}} \text{.}
\end{equation}
Therefore, the sum of the normalized SNR values over BSSs simultaneously propagating signals in the OWE can be expressed as

\begin{equation}
\label{sum_snr}
\begin{aligned}
&\max_\mathbf{G} \sum_i \frac{\text{SNR}_{[i]}}{\lambda_i} \\
& = \max_\mathbf{G} \sum_i
\frac{{{{|\psi_i|^2\left|\mathbf{w}_{EA[i]}^T\mathbf{A}{\mathbf{w}_{SE[i]}} \right|}^2}}}{\lambda_i{{{\left| {\mathbf{w}_{EA[i]}^T\mathbf{A}{\mathbf{H}^T}\left( {\mathbf{G\tilde n + }{{\mathbf{\tilde n}}_\mathbf{a}}} \right)} \right|}^2}}} \text{.}
\end{aligned}
\end{equation}
Note that $\lambda_i$ in (\ref{sum_snr}) is a constant obtained from (\ref{optimal_snr}) and is not a parameter to be optimized in (\ref{sum_snr}). 

\subsubsection{Objective Function}
We can formulate the objective function for optimizing the SINR in the scenario with multiple BSSs by combining (\ref{mutual_interference}) and (\ref{sum_snr}).

\begin{align}
\label{multi_bss_obj}
&\max_\mathbf{G} \sum_i \frac{\text{SINR}_{[i]}}{\lambda_i} \\
& = \max_\mathbf{G} \sum_i
\frac{{{{|\psi_i|^2\left|\mathbf{w}_{EA[i]}^T\mathbf{A}{\mathbf{w}_{SE[i]}} \right|}^2}}}{\lambda_i\left[{{{\left| {\mathbf{w}_{EA[i]}^T\mathbf{A}{\mathbf{H}^T}\left( {\mathbf{G\tilde n + }{{\mathbf{\tilde n}}_\mathbf{a}}} \right)} \right|}^2}} + \sum_{j\neq i}\left|I_{m[j,i]}\right|^2\right]} \notag \\
&\textit{s.t.} \notag \\
& \rho\left({\mathbf{H}^T}\mathbf{G}\right)=|\lambda_{\max}\left({\mathbf{H}^T}\mathbf{G}\right)|<1, \notag \\
& G_k\geq0,{\ }k=1,\dots, N \text{.} \notag
\end{align}