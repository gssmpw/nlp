\begin{abstract}
Optical wireless communication (OWC) uses light for wireless data transmission, potentially providing faster and more secure communication than traditional radio-frequency-based techniques like Wi-Fi. However, light’s high directionality and its limited penetration ability restrict the signal coverage. To address this problem, we propose constructing an artificial “optical wireless ether” (OWE) fabric to extend the reach of optical signals. OWE functions as an electromagnetic (EM) wave-propagating medium, guiding light signals to cover a broader area. Our proposed ether fabric comprises simple optical signal amplification units, called ether amplifiers (EAs), strategically placed in the environment, e.g., on ceilings. The EAs amplify and propagate signals at the analog level and are agnostic to the signal format: they do not sample the signals for digital processing. We refer to the operating mechanism of EAs as “instantaneous-propagate" (IP) to differentiate it from methods that buffer received signals before forwarding, as used in conventional relay networks. By controlling a group of EA gains, we can change the characteristics of the underlying medium, enhancing signal strength and redirecting the propagation. Although increasing the EA gains can extend signal coverage, it may create positive feedback paths, leading to amplifier saturation that distorts the signals. A key challenge in OWE design is tuning EA gains to mitigate self-interference and avert amplifier saturation. This paper presents a systematic theoretical analysis to prevent amplifier saturation while optimizing the performance of OWE in both single-basic-service-set (single-BSS) and multiple-BSS scenarios. Optimization factors include signal-to-noise ratio, resource allocation fairness, and mutual interference. Furthermore, we conducted simulations and experiments to corroborate our theories. To our knowledge, ours is the first experimental demonstration of the feasibility of an artificial ether fabric for extending and guiding light propagation, laying a solid groundwork for future development and exploration of OWE.
\end{abstract}

\begin{IEEEkeywords}
Optical Wireless Communication, Optical Wireless Ether, Instantaneous-Propagate Mechanism, Signal-to-Noise Ratio, Resource Allocation Fairness, Mutual Interference
\end{IEEEkeywords}