\section{System Model}
\subsection{System Design}
The OWE consists of EAs that propagate the optical signal.
For the system model, without loss of generality, we envision an infrastructure network in which APs and EAs are mounted on the ceiling in a grid pattern. User stations can be stationary or mobile within the coverage area. A deployment scenario is depicted in Fig. \ref{meeting_room}.

\begin{figure}
\centering
\includegraphics[width=0.5\textwidth]{Figures/meeting_room.png}
\caption{Meeting room application scenario of OWE. One AP and a group of EAs are mounted on the ceiling, while user stations can be either stationary or in motion.}
\label{meeting_room}
\vspace{-0.2cm}
\end{figure}
\par

\begin{figure}
\centering
\includegraphics[width=0.5\textwidth]{Figures/EA_Model_Labeled.png}
\caption{A rendering model of an EA design in which the integrated optical transmitter and receiver are closely positioned and nearly overlapped relative to the wavelength of the intensity-modulated signal.}
\label{EA_model}
\vspace{-0.5cm}
\end{figure}
\par
The APs and EAs are mounted on the ceiling with their optical transceivers aimed at the floor and not directly at each other, there will be no LOS light path between them. Referring to Fig. \ref{reflecting_path}, we utilize floor reflection paths to establish connectivity within the OWE without requiring additional wiring work and dedicated transceivers, simplifying the hardware and reducing costs. The feasibility of this approach has been confirmed through our simulations and experiments.

\begin{figure*}[htbp]
\centering
\includegraphics[width=0.8\textwidth]{Figures/reflecting_path.png}
\caption{Signal propagation through OWE via floor reflection paths}
\label{reflecting_path}
\end{figure*}
\par
Fine-tuning the PGA gains in EAs to prevent amplifier saturation while significantly enhancing signal coverage is a critical challenge in constructing the OWE. Intuitively, increasing the gain values of EAs leads to a broader signal coverage; however, uncontrolled gain adjustments may lead to amplifier saturations that cause system instability.
\par
With the instantaneous propagation (IP) mechanism, the received signals of EAs are amplified and immediately retransmitted. The high directionality and limited penetration ability of light prevent the formation of a severe self-interference path between an EA's optical transmitter and receiver \cite{narmanlioglu2017cooperative, yang2013full, 10645235}. In contrast, if RF instead of optical frequencies are used to carry signals, their longer wavelengths result in less directionality and broader coverage. Consequently, the simultaneous transmitting and receiving of closely spaced antennas could easily cause overwhelming positive feedback and saturate the amplifier.
This explains why OWE may not be viable for RF-based wireless communications but is feasible in OWC.
Nevertheless, the self-interference paths persist due to light reflection, refraction, and scattering in the environment, as well as the signal propagation loops that form between EAs. In this paper, we aim to formulate a systematic mathematical expression of constraints to ensure the stability of OWE.

\begin{figure*}
\centering
\includegraphics[width=0.7\textwidth]{Figures/system_diagram.png}
\caption{The EA's signal flow diagram.}
\label{system_diagram}
\vspace{-0.5cm}
\end{figure*}

\begin{table}
\centering
\begin{tabularx}{0.48\textwidth}{l L}
\toprule
\textbf{Symbols}&\textbf{Definitions}\\  \midrule 
$E_i$&The \textit{i-th} EA within the OWE.\\ [0.1cm]
$G_i$&The gain of the \textit{i-th} EA.\\ [0.1cm]
$\widetilde{h}_{ii}$&The channel response between the transmitter and receiver of the same EA.\\ [0.1cm]
$\widetilde{h}_{ij}$&The channel response from EA \textit{i} to EA \textit{j}.\\ [0.1cm]
$\widetilde{x}_i$&The signal received by the \textit{i-th} EA from the station.\\ [0.1cm]
$\widetilde{x}_{ji}$&The signal received by the \textit{i-th} EA from the \textit{j-th} EA.\\ [0.1cm]
$\widetilde{y}_i$&The combined received signal of the \textit{i-th} EA.\\ [0.1cm]
$\widetilde{z}_i$&The signal output of the \textit{i-th} EA.\\ [0.1cm]
$\widetilde{n}_i$& The thermal noise introduced by the optical receiver in the \textit{i-th} EA. \\ [0.4cm]
$\widetilde{n}_{i,a}$&The noise introduced by the amplifier in the \textit{i-th} EA. \\ [0.1cm]
$N$ & The number of EAs in the OWE.\\
\bottomrule
\end{tabularx}
\caption{First set of symbol definitions.}
\label{Table1}
\vspace{-0.5cm}
\end{table}

\subsection{Channel Response}
\par
The EA's signal flow diagram is shown in Fig. \ref{system_diagram}, and the definitions of mathematical symbols are listed in Table \ref{Table1}. To facilitate analysis, we express the OWE signals in the frequency domain. The signal output of the \textit{i}-th EA can be written as
\begin{equation}
\label{relay_output}
{\tilde z_i}(f) = {G_i}\left( {{{\tilde y}_i}(f) + {{\tilde n}_i}} \right) + {\tilde n_{i,a}} \text{.}
\end{equation}
The received signal of the \textit{i}-th EA is
\begin{equation}
\label{amp_input}
{{\tilde y}_i}(f) = {{\tilde x}_i}(f) + \sum\limits_{j \ne i} {{{\tilde x}_{ji}}(f)}  + {{\tilde h}_{ii}}(f){{\tilde z}_i}(f) \text{.}
\end{equation}
\par
By substituting the expression for ${\tilde z_i}(f)$ from (\ref{relay_output}) into (\ref{amp_input}), we obtain 
\begin{equation}
\label{sys_equ}
\resizebox{.9\hsize}{!}{$
\begin{aligned}
{{\tilde y}_i}(f) 
= {{\tilde x}_i}(f) &+ \sum\limits_{j \ne i} {{{\tilde h}_{ji}}(f){{\tilde z}_j}(f)}  + {{\tilde h}_{ii}}(f){{\tilde z}_i}(f)\\
= {{\tilde x}_i}(f) &+ \sum\limits_{j \ne i} {\left\{ {{{\tilde h}_{ji}}(f)\left[ {{G_j}\left( {{{\tilde y}_j}(f) + {{\tilde n}_j}} \right) + {{\tilde n}_{j,a}}} \right]} \right\}} \\
&+ {{\tilde h}_{ii}}(f)\left[ {{G_i}\left( {{{\tilde y}_i}(f) + {{\tilde n}_i}} \right) + {{\tilde n}_{i,a}}} \right]\\
={{\tilde x}_i}(f) &+ \left[ {\sum\limits_{j \ne i} {\left( {{{\tilde h}_{ji}}(f){G_j}{{\tilde y}_j}(f)} \right)}  + {{\tilde h}_{ii}}(f){G_i}{{\tilde y}_i}(f)} \right] \\
&+ \left( {\sum\limits_{j \ne i} {{{\tilde h}_{ji}}(f){G_j}{{\tilde n}_j}}  + {{\tilde h}_{ii}}(f){G_i}{{\tilde n}_i}} \right) \\ 
&+ \left( {\sum\limits_{j \ne i} {{{\tilde h}_{ji}}(f){{\tilde n}_{j,a}}}  + {{\tilde h}_{ii}}(f){{\tilde n}_{i,a}}} \right) \text{.}
\end{aligned}
$}
\end{equation}
The left-hand side of (\ref{sys_equ}) represents the combined received signal at the \textit{i}-th EA, ${\tilde y}_i(f)$. The right-hand side captures the signal input, $\tilde{x}_i(f)$, from the \textit{i}-th station to the OWE, as well as the influence of other EAs and noise components.

\par
Let us define 
\begin{equation}
    \label{symbol_definitions}
    \begin{aligned}
        &\mathbf{G} = {\mathop{\rm Diag}\nolimits} \left( {{G_1},{G_2}, \ldots ,{G_N}} \right)\\
        &\mathbf{H}\left( {i,j} \right) = {{\tilde h}_{ij}}(f)\\
        &\mathbf{\tilde x} = {\left( {{x_1},{x_2}, \ldots ,{x_N}} \right)^T}\\
        &\mathbf{\tilde y} = {\left( {{y_1},{y_2}, \ldots ,{y_N}} \right)^T}\\
        &\mathbf{\tilde n} = {\left( {{n_1},{n_2}, \ldots ,{n_N}} \right)^T}\\
        &{{\mathbf{\tilde n}}_\mathbf{a}} = {\left( {{n_{1,a}}, {n_{2,a}}, \ldots {n_{N,a}}} \right)^T},
    \end{aligned}
\end{equation}
where $N$ is the number of EAs. Eqn. (\ref{sys_equ}) can be generalized to all $N$ EAs, giving $N$ equations. These equations can be collected together in a succinct matrix form:
\begin{equation}
\label{mat_sys_equ}
\begin{array}{c}
\left( {\mathbf{I} - {\mathbf{H}^T}\mathbf{G}} \right)\mathbf{\tilde y} = \mathbf{\tilde x + }{\mathbf{H}^T}\mathbf{G\tilde n + }{\mathbf{H}^T}{{\mathbf{\tilde n}}_\mathbf{a}} \text{.}
\end{array}
\end{equation}
Eqn. (\ref{mat_sys_equ}) comprehensively shows the relationship between the signal inputs of OWE, $\mathbf{\tilde{x}}$, and the combined received signals at EAs, $\mathbf{\tilde{y}}$. Note that $\mathbf{\tilde{y}}$ can be interpreted as a \textit{channel response} to the input $\mathbf{\tilde{x}}$. Therefore, by adjusting the gain values of EAs, $\mathbf{G}$, we can change the light propagation characteristics of the OWE.

\subsection{Stability Constraints}

With the channel response described in (\ref{mat_sys_equ}), we can analyze the stability of the OWE. For convenience in analysis, we remove the noise-related components in (\ref{mat_sys_equ}) and get (\ref{mat_sys_equ_noiseless}) in the following. We remark that the stability condition to be derived remains the same even with the noise included.
\begin{equation}
\label{mat_sys_equ_noiseless}
\begin{array}{c}
\left( {\mathbf{I} - {\mathbf{H}^T}\mathbf{G}} \right)\mathbf{\tilde y} = \mathbf{\tilde x} \text{.}
\end{array}
\end{equation}
Mathematically, the channel response (\ref{mat_sys_equ_noiseless}) can be rewritten into an iterative form as follows:
\begin{equation}
\label{iterative_sys_equ}
\begin{aligned}
\mathbf{\tilde y} &=\left[ {\mathbf{I} + \mathbf{H}^T \mathbf{G + }{\left(\mathbf{H}^T\mathbf{G}\right)^2} +  \cdots } \right]\mathbf{\tilde x}\\
&= \lim_{m\rightarrow \infty}\left[ {\mathbf{I} - {\left(\mathbf{H}^T\mathbf{G}\right)^m}} \right]{\left( {\mathbf{I} - \mathbf{H}^T\mathbf{G}} \right)^{ - 1}}\mathbf{\tilde x} \text{.}
\end{aligned}
\end{equation}
According to (\ref{iterative_sys_equ}), if $\mathbf{\tilde{y}}$ goes to infinity as the iteration index $m$ approaches infinity, the channel response does not converge, indicating that the OWE system is unstable and can induce amplifier saturation. Therefore, the EAs' gain settings should fulfill
\begin{equation}
\label{converge_cond}
{\left({\mathbf{H}^T}\mathbf{G}\right)^m} \to 0\textit{, as }m \to \infty \text{.}
\end{equation}
${\mathbf{H}^T}\mathbf{G}$ is a square matrix. Let us assume it adopts an eigen decomposition, which can be factorized as ${\mathbf{H}^T}\mathbf{G}=\mathbf{Q\Lambda }{\mathbf{Q}^{ - 1}}$. Then, (\ref{converge_cond}) can be expressed as follows:
\begin{equation}
\label{eigen_converge_cond}
\begin{aligned}
{\left({\mathbf{H}^T}\mathbf{G}\right)^m} &= \mathbf{Q}{\mathbf{\Lambda }^m}{\mathbf{Q}^{ - 1}} \to 0 \text{,} m \to \infty
\\
&\Rightarrow {\mathbf{\Lambda }^m} \to 0 \text{,} m \to \infty \text{.}
\end{aligned}
\end{equation}
Based on (\ref{eigen_converge_cond}), to ensure the stability of OWE and avoid amplifier saturation, the elements in $\mathbf{\Lambda}$ must satisfy $|\lambda_i| < 1$, meaning that the absolute value of each eigenvalue of $\mathbf{H}^T\mathbf{G}$ must be less than $1$. In other words: \textbf{\textit{the spectral radius of $\mathbf{H}^T\mathbf{G}$ must be less than $1$.}}
\par
Further, given that $\left| {{\lambda _i}} \right| < 1$, we can infer that the determinant of ${\left( {\mathbf{I} - {\mathbf{H}^T}\mathbf{G}} \right)}$ satisfies
\begin{equation}
\begin{aligned}
\det {\left( {\mathbf{I} - {\mathbf{H}^T}\mathbf{G}} \right)} 
&= \det {\left(\mathbf{I} - {\mathbf{H}^T}\mathbf{G}\right)} \\
&= \det \left[ {\mathbf{Q}(\mathbf{I - \Lambda }){\mathbf{Q}^{ - 1}}} \right] \\
&= \prod\limits_{i = 1}^N {(1 - {\lambda _i})}  > 0 \text{.}
\end{aligned}
\end{equation}
As the determinant of ${\left( {\mathbf{I} - {\mathbf{H}^T}\mathbf{G}} \right)}$ is non-zero, it follows that ${\left( {\mathbf{I} - {\mathbf{H}^T}\mathbf{G}} \right)}$ is invertible under the stated constraint. 
For the sake of brevity, let $\mathbf{A} = {\left( {\mathbf{I} - {\mathbf{H}^T}\mathbf{G}} \right)^{ - 1}}$, allowing us to rewrite (\ref{mat_sys_equ}) as
\begin{equation}
\label{eq10}
\mathbf{\tilde y} = \mathbf{A}\left[ {\mathbf{\tilde x + }{\mathbf{H}^T}\left( {\mathbf{G\tilde n + }{{\mathbf{\tilde n}}_\mathbf{a}}} \right)} \right] \text{.}
\end{equation}
Eqn. (\ref{eq10}) provides a more insightful expression of the OWE channel response, where $\mathbf{\tilde x}$ is the input and $\mathbf{\tilde y}$ the output.

\subsection{Signal Quality}

Having presented the channel response of the OWE as the propagation medium and its stability constraints, our focus now shifts to the signal received at the AP, specifically the uplink from the station to the AP. There are two categories of propagation paths:
\begin{enumerate}
    \item Line-of-Sight (LOS) Paths: These paths emerge when the station is within the coverage area of the AP, enabling direct reception of the transmitted signal by the AP.
    \item Non-Line-of-Sight (NLOS) Paths: If no LOS paths exist between the station and the AP, the signal is instead received by EAs and propagated through OWE, eventually reaching the AP.
\end{enumerate}
By incorporating the two categories of propagation paths mentioned above, the signal received by the AP from the station can be expressed as follows:
\begin{equation}
\label{eq11}
\begin{aligned}
{{\tilde r}_{A}} 
&={{\tilde h}_{SA}}\left( f \right){{\tilde s}_{S}}\left( f \right)
+ {{\tilde n}_{A}} + {\left( {\mathbf{G}{{\mathbf{\tilde h}}_{EA}}} \right)^T} \mathbf{\tilde y}\\
&={{\tilde h}_{SA}}\left( f \right){{\tilde s}_{S}}\left( f \right)
+ {{\tilde n}_{A}} \\
&\quad + {\left( {\mathbf{G}{{\mathbf{\tilde h}}_{EA}}} \right)^T}\mathbf{A}\left[ {{{\mathbf{\tilde h}}_{SE}}{{\tilde s}_{S}}\left( f \right) + {\mathbf{H}^T}\left( {\mathbf{G\tilde n + }{{\mathbf{\tilde n}}_\mathbf{a}}} \right)} \right] \text{.}
\end{aligned}
\end{equation}
The definitions of symbols used in (\ref{eq11}) are given in Table \ref{Table2}. Fig. \ref{ea_ap} illustrates the signal propagation from the OWE to the AP. For simplicity, the example depicts only the reflection path from one EA to the AP. In general, ${\mathbf{G}{{\mathbf{\tilde h}}_{EA}}}$ in (\ref{eq11}) corresponds to signals of direct paths from all EAs to the AP, where $\mathbf{G}$ is the last OWE amplification before the reception at the AP, and ${\mathbf{\tilde h}}_{EA}$ is the channel response from the OWE to the AP.

\begin{table}
\centering
\begin{tabularx}{0.48\textwidth}{l L}
\toprule
\textbf{Symbols}&\textbf{Definitions}\\  \midrule 
${{\tilde r}_{A}}$&The received signal at the AP.\\ [0.1cm]
${{\tilde h}}_{SA}$&The LOS channel response from the station to the AP. In most cases, there will be no LOS path, and ${{\tilde h}}_{SA}=0$.\\ [0.5cm]
${{\tilde s}_{S}}$&The signal sent by the station.\\ [0.15cm]
${{\tilde n}_{A}}$&The noise introduced by the AP.\\ [0.1cm]
${\mathbf{\tilde h}}{\scriptscriptstyle EA}$& The channel response from the EAs to the AP. \newline ${{\mathbf{\tilde h}}_{EA}} = {\left( {{{\tilde h}_{1,A}}\left( f \right),{{\tilde h}_{2,A}}\left( f \right), \ldots ,{{\tilde h}_{N,A}}\left( f \right)} \right)^T}$. \\ [0.7 cm] 
${\mathbf{\tilde h}}{\scriptscriptstyle SE}$& The channel response from the station to the EAs. \newline ${{\mathbf{\tilde h}}_{SE}} = {\left( {{{\tilde h}_{S,1}}\left( f \right),{{\tilde h}_{S,2}}\left( f \right), \ldots ,{{\tilde h}_{S,N}}\left( f \right)} \right)^T}$. \newline Note: $\mathbf{\tilde x}$ in (\ref{eq10}) is ${{\mathbf{\tilde h}}_{SE}}{{\tilde s}_{S}}\left( f \right)$. \\
\bottomrule
\end{tabularx}
\caption{Second set of symbol definitions.}
\label{Table2}
\vspace{-0.4cm}
\end{table}

\begin{figure}
\centering
\includegraphics[width=0.48\textwidth]{Figures/General_Case_EA_AP_3.png}
\caption{Signal propagations from the OWE to the AP. For simplicity, the figure shows the reflection path from only one EA to the AP.}
\label{ea_ap}
\vspace{-0.5cm}
\end{figure}

\par
From (\ref{eq11}), the SNR of the received signal at the AP from the station is
\begin{equation}
\label{ap_snr}
\resizebox{.85\hsize}{!}{$
\begin{aligned}
& \text{SNR}_{SA} = \frac{{{{\left| {{{\tilde h}_{SA}}\left( f \right) + {{\left( {\mathbf{G}{{\mathbf{\tilde h}}_{EA}}} \right)}^T}\mathbf{A}{{\mathbf{\tilde h}}_{SE}}} \right|}^2}{{\left| {{{\tilde s}_{S}}\left( f \right)} \right|}^2}}}{{{{\left| {{{\left( {\mathbf{G}{{\mathbf{\tilde h}}_{EA}}} \right)}^T}\mathbf{A}{\mathbf{H}^T}\left( {\mathbf{G\tilde n + }{{\mathbf{\tilde n}}_\mathbf{a}}} \right) + {{\tilde n}_{A}}} \right|}^2}}} \\
&= \frac{{{{\left| {{{\tilde h}_{SA}}\left( f \right) + \mathbf{\tilde h}_{EA}^T{\mathbf{G}^T}\mathbf{A}{{\mathbf{\tilde h}}_{SE}}} \right|}^2}{{\left| {{{\tilde s}_{S}}\left( f \right)} \right|}^2}}}{{\mathbf{\tilde h}_{EA}^T{\mathbf{G}^T}\mathbf{A}{\mathbf{H}^T}{\mathop{\rm diag}\nolimits} \left( {{G_i}{\sigma ^2} + \sigma _a^2} \right){\mathbf{H}^*}{\mathbf{A}^H}{\mathbf{G}^*}\mathbf{\tilde h}_{EA}^* + \sigma _A^2}} \text{.}
\end{aligned}
$}
\end{equation}
Eqn. (\ref{ap_snr}) provides the fundamental theoretical expression for optimizing the uplink SNR of the signal propagating through the OWE. 
\par
The downlink SNR can be similarly obtained. If the optical transmitter and receiver of the EA are in close proximity -- this is the case for the EA design of Fig. \ref{EA_model} -- then the uplink and downlink signal paths almost overlap. We then have 
\begin{equation}
\label{proportional}
    \text{SNR}_{SA} \propto \text{SNR}_{AS} \text{.}
\end{equation}
Establishing the precise alignment of the uplink and downlink signal propagation paths is challenging as the optical transmitter and receiver are typically distinct components (LED and PD). However, In OWC, the intensity modulation/direct detection (IM/DD) scheme is widely used, in which the information signal wavelength\footnote{The frequency of intensity modulation is not the same as the optical frequency of the light. The light itself in our system does not cause constructive-destructive interference because the LEDs in our system are not coherent light sources like lasers and the dimensions of PDs are typically tens of thousands of optical radiation wavelengths \cite{ghassemlooy2019optical, 380210, wong2000performance, al2018optical}, as shown in Table \ref{wavelength}.} is relatively large as indicated in Table \ref{wavelength}. The deviation of the uplink and downlink propagation paths at the centimeter level is negligible and will not invalidate (\ref{proportional}).

\begin{table*}[htbp]
\centering
\begin{tabularx}{0.96\textwidth}{p{7.7cm}|p{6.2cm}|X}
\toprule
\multirow{4}{*}{\begin{tabular}[c]{@{}l@{}}Optical radiation wavelength of the light used in OWC \cite{al2018optical}\end{tabular}} & Wireless ultraviolet communications   (WUVCs)               & 200 nm - 280 nm           \\ \cmidrule(l){2-3} 
 & Visible light communications (VLCs)      & 380 nm - 780 nm  \\ \cmidrule(l){2-3} 
 & Wireless infrared communications (WIRCs) & 780 nm - 1600 nm \\ \midrule
Wavelength of the IM optical signal (information signal) & \begin{tabular}[c]{@{}l@{}} Modulation frequency of 802.11bb standard \cite{10063242}: \\ 16 MHz - 336 MHz \end{tabular} & 0.8922 m - 18.7370 m \\ \bottomrule
\end{tabularx}
\caption{Optical radiation wavelength and intensity modulation (IM) wavelength of the optical signal in OWC.}
\label{wavelength}
\vspace{-0.2cm}
\end{table*}
