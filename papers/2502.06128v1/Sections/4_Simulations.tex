\section{Simulations}
Our simulations assume that the stations are equipped with directional transceivers capable of steering the transmitted signal toward a specific entry EA. In the single-BSS scenario, (\ref{snr_sa}) applies. According to (\ref{snr_sa}), the optimized gain matrix, $\mathbf{G}$, is independent of the incoming power at the entry EA. The primary objective of our single-BSS investigation is to examine how the optimized $\mathbf{G}$ varies with respect to the entry EA.
\par
We then extend our investigation to the multiple-BSS scenario, where two networks operate in the vicinity of each other using the same OWE. In this case, due to the consideration of mutual interference between the signals of BSSs, the signal power at the entry EA may have some influence on the optimized $\mathbf{G}$. However, as will be shown, this influence is not significant.
\subsection{Single-BSS Scenario}
The simulation was conducted in a virtual environment -- a room measuring 6 m $\times$ 6 m $\times$ 3 m. The EAs in this room were mounted on the ceiling at a height of 3 m and arranged in a 3 $\times$ 3 grid pattern with a 2-meter interval.
Ignoring the influence of furniture in the room and wall reflections, the optical channel in a grid OWE network can be classified into three categories, as shown in Fig. \ref{channel_types}:
\begin{enumerate}
\item Self-interference channel: The optical signal sent by an EA is received by the same EA after one reflection off the floor, corresponding to $\widetilde{h}_{ii}$ in Table \ref{Table1}.
\item Adjacent channel: The channel between horizontally/vertically adjacent EAs. It is represented by $\widetilde{h}_{ij}$ in Table \ref{Table1}. The signal propagates through one-time reflection off the floor and has no LOS paths.
\item Diagonally adjacent channel: The channel between diagonally adjacent EAs. It is also represented by $\widetilde{h}_{ij}$ in Table \ref{Table1}. The signal propagates through one-time reflection off the floor and has no LOS paths.
\end{enumerate}
We do not consider propagation channels between EAs that are two or more hops away due to the significant attenuation in these channels, resulting from larger radiation incident angles and propagation distances compared to the channels of the three categories introduced above.
\begin{figure}
\centering
\includegraphics[width=0.35\textwidth]{Figures/Channel_Types.png}
\caption{The optical channels in a grid OWE network can be categorized into three categories: 1) Self-interference channel, 2) Adjacent channel, and 3) Diagonally adjacent channel.}
\label{channel_types}
\vspace{-0.4cm}
\end{figure}
\par
We obtained these optical channel responses in the virtual environment using Zemax OpticStudio, a professional software program for designing and simulating optical systems \cite{ansys}. The power of the thermal noise in EA's optical receiver and the noise introduced by the amplifier is set to 1 $\mu$W, i.e., $|\widetilde{n}_i|^2=|\widetilde{n}_{i,a}|^2=1\times 10^{-6} \text{ W}, i=1,\dots, N$ in simulations (the meanings of symbols can be referred to Table \ref{Table1} and (\ref{symbol_definitions})).
\par
After that, we maximized the SNR value by optimizing the objective function in (\ref{snr_sa}) through various optimization methods, including projected gradient descent, genetic algorithms, and simulated annealing. Fig. \ref{opt_methods} illustrates a comparison of these optimization methods. We can observe that the simulated annealing method yielded the biggest objective function value in a relatively small network scale (3 $\times$ 3). Therefore, it was employed to optimize the objective function (\ref{snr_sa}) with different EAs working as the entry EAs, to maximize the received signal SNR at the AP.
\begin{figure}
\centering
\includegraphics[width=0.48\textwidth]{Figures/Optimization_Comparison_3.png}
\caption{Comparison between different optimization methods (the larger the objective function value, the better).}
\label{opt_methods}
\vspace{-0.5cm}
\end{figure}
\par
Recall that in the objective function (\ref{snr_sa}), the received signal power (proportional to $|\psi_i|^2$) does not influence the optimization result as far as the optimal gain matrix $\mathbf{G}$ is concerned. Instead, the entry EA, as indicated by $w_{SE}$ and the corresponding gain matrix $\mathbf{G}$, plays a key role. Therefore, we modify our simulations by selecting different EAs as the entry EA.
In the 3 $\times$ 3 grid OWE, as depicted in Fig. \ref{channel_types}, we individually selected EAs 1, 2, 3, 5, 6, and 9 as entry EAs, providing a comprehensive evaluation of our optimization algorithm. Due to the symmetry along the diagonal line in this OWE network, there is no need to investigate all EA nodes. EA 9 is connected to the AP, as depicted in Fig. \ref{OWE_AP}. 
\par
In the simulation, we take the reciprocal of the objective function 
(\ref{snr_sa}) to get
\begin{align}
\label{r_single_bss_obj}
&\min_\mathbf{G} {\frac{1}{\text{SNR}_{SA}}} \notag \\
&\Leftrightarrow \min_\mathbf{G} 
\frac{{{{\left| {\mathbf{w}_{EA}^T\mathbf{A}{\mathbf{H}^T}\left( {\mathbf{G\tilde n + }{{\mathbf{\tilde n}}_\mathbf{a}}} \right)} \right|}^2}}}{{{{\left|\mathbf{w}_{EA}^T\mathbf{A}{\mathbf{w}_{SE}} \right|}^2}}} \\
&\textit{s.t.} \notag \\
& \rho\left({\mathbf{H}^T}\mathbf{G}\right)=|\lambda_{\max}\left({\mathbf{H}^T}\mathbf{G}\right)|<1, \notag \\
& G_i\geq0,{\ }i=1,\dots, N \text{.} \notag
\end{align}
This operation is mathematically feasible since the numerator and the denominator of (\ref{snr_sa}) are constantly positive. By converting the maximization of the SNR to the minimization of NSR (noise-to-signal ratio), it is easier for the optimization process to escape from poor local optima. In low SNR cases, where noise power exceeds signal power, typically occurring at the beginning of the optimization iteration, NSR is more sensitive and undergoes more drastic value changes compared to SNR during the adjustments to the EA gains.
\par
Fig. \ref{SISO_simulation_res} displays the optimization results.
We use the depth of color to represent the magnitude of the EA gain, with darker orange colors indicating larger gains and white representing a gain of 0. Additionally, we have labeled the gain values in the corresponding grids, with the numbers in parentheses indicating the respective EA identifiers. The entry EA for each case is noted at the top of each subfigure.
\par
We can see that the gain value of entry EA is the largest, while the gain value of EA connected to AP is 0. This is because EA amplifies the signal while also amplifying the noise mixed in the signal and introducing new noise. Assuming that in the same scenario, the signal power received by AP is constant, signal propagation can be considered as a multiplication of EA gains along the propagation path. By amplifying with a higher gain when the signal just enters OWE with minimal noise content, compared to amplifying at the later stage of propagation when more noise is introduced, it helps reduce the proportion of noise in the signal received by AP and improves signal quality.
\par
In particular, in the case where EA 9 is both the entry EA and connected to the AP, receiving without amplification would be best for the received signal quality since the amplifier would introduce additional noise, as illustrated in Fig. \ref{system_diagram}. The gain values of EA 5, 6, and 8 are optimized to 0 to isolate EA 9 from other EAs, whose gain values are randomly assigned during the simulated annealing iterations.
\par
The improvements in SNR after deploying our optimization are shown in Table \ref{snr_improvement}. Notably, the improvement is relative to the equal gain setting, where all EAs share the same maximum gain value that fulfills the stability constraints. We assume that the signal received by the entry EA is noiseless.
\begin{figure}[htbp]
\centering
\includegraphics[width=0.5\textwidth]{Figures/Simulation_Opt_Results_2.png}
\caption{Optimization results for the OWE single-BSS application scenario. 6 EAs are selected individually as the entry EA. EA 9 is integrated with the AP.}
\label{SISO_simulation_res}
\vspace{-0.5cm}
\end{figure}

\begin{table}
\centering
\begin{tabularx}{0.48\textwidth}{p{1.5cm}l|p{1.5cm}l}
\toprule
\begin{tabular}[c]{@{}l@{}} \textbf{Entry} \\ \textbf{EA} \end{tabular} & \begin{tabular}[c]{@{}l@{}} \textbf{SNR} \\ \textbf{Improvement} \quad \quad \quad \end{tabular} & \begin{tabular}[c]{@{}l@{}} \textbf{Entry} \\ \textbf{EA} \end{tabular} & \begin{tabular}[c]{@{}l@{}} \textbf{SNR} \\ \textbf{Improvement} \end{tabular} \\ \midrule
1           & 20.628957 dB    & 2           & 17.607625 dB    \\
3           & 21.125718 dB    & 5           & 15.308744 dB    \\
6           & 18.321142 dB    & 9           & 81.662340 dB    \\ \bottomrule
\end{tabularx}
\caption{SNR improvements of applying the optimized gain settings shown in Fig. \ref{SISO_simulation_res}. The improvements are computed relative to the equal gain settings with the same entry EA, respectively.}
\label{snr_improvement}
\vspace{-0.4cm}
\end{table}
\par
Among the entry EA options, EA 9 yields the greatest SNR improvement. This is because, in comparison to the equal gain setting, the noise component in the denominator of SNR computation is significantly reduced to only the noise power introduced by the optical receiver, approaching a noiseless state.
Overall, our optimization algorithm has shown OWE can significantly enhance SNR and extend the optical signal coverage, validating the accuracy of our theoretical analysis and the feasibility of OWE design.
\par
In addition, we have also tried multilayer perceptron (MLP) artificial neural networks to address this optimization problem. Unfortunately, the test results are significantly poorer than those achieved using the traditional optimization methods mentioned earlier. The reasons for this outcome are twofold: First, during the later phases of training iterations, a lack of gradients hinders network-wide convergence. Second, the training of MLP typically relies on a gradient descent method, which can easily get trapped in local optima.

\subsection{Multiple-BSS Scenario}
In the multiple-BSS scenario, we optimized the gain values of EAs based on the objective function (\ref{multi_bss_obj}).
The simulations were conducted in a 4 $\times$ 4 grid OWE with a 2-meter interval between EAs. The virtual room environment measures 8 m $\times$ 8 m $\times$ 3 m. All other conditions remain consistent with the single-BSS scenario simulations.
The channel responses between EAs are the same as discussed in the previous single-BSS scenario. 
\par
The original objective function in the multiple-BSS scenario (\ref{multi_bss_obj}) represents the sum of the normalized SINR values over BSSs.
As with the derivation of (\ref{r_single_bss_obj}), in the simulation, we take the reciprocal of each element in (\ref{multi_bss_obj}) to get
\begin{align}
\label{r_multi_bss_obj}
&\min_\mathbf{G} \sum_i \frac{\lambda_i}{\text{SINR}_{[i]}} \\
& = \min_\mathbf{G} \sum_i
\frac{\lambda_i\left[{{{\left| {\mathbf{w}_{EA[i]}^T\mathbf{A}{\mathbf{H}^T}\left( {\mathbf{G\tilde n + }{{\mathbf{\tilde n}}_\mathbf{a}}} \right)} \right|}^2}} + \sum_{j\neq i}\left|I_{m[j,i]}\right|^2\right]}{{{{|\psi_i|^2\left|\mathbf{w}_{EA[i]}^T\mathbf{A}{\mathbf{w}_{SE[i]}} \right|}^2}}} \notag \\
&\textit{s.t.} \notag \\
& \rho\left({\mathbf{H}^T}\mathbf{G}\right)=|\lambda_{\max}\left({\mathbf{H}^T}\mathbf{G}\right)|<1, \notag \\
& G_k\geq0,{\ }k=1,\dots, N \text{.} \notag
\end{align}
By reciprocating the elements in (\ref{multi_bss_obj}), we transform the task of maximizing the sum of normalized SINR over BSSs into minimizing the sum of normalized INSR (interference-plus-noise-to-signal ratio). This conversion can help avoid poor local optima as explained in (\ref{r_single_bss_obj}), and promote fairness among BSSs during optimization. In cases where resource allocation is unfair and leads to a significant degradation in signal quality, the value of a low-SINR BSS element (as SINR approaches 0) would rapidly increase and be prioritized in the optimization process.
\par
Four multiple-BSS scenarios were simulated, where the OWE simultaneously propagates signals from two entry EAs to two APs.
We assume that stations are close to their selected entry EAs as shown in Fig. \ref{spatial_reuse}. Unlike in the single-BSS scenarios, the optimization results may vary with the received signal powers at entry EAs in the multiple-BSS scenarios. For simplicity, we first set $|\psi_i|^2 = |\psi_j|^2 = 1$ W in simulations using the objective function (\ref{r_multi_bss_obj}), where $i \neq j$ denotes distinct BSSs. The influence of the received signal strength setting on the simulation results will be discussed shortly. The noise setting is the same as in the single-BSS simulations, with $|\widetilde{n}_i|^2=|\widetilde{n}_{i,a}|^2=1\times 10^{-6} \text{ W}, i=1,\dots, N$.
\par
The simulation results are shown in Fig. \ref{MIMO_examples} and Table \ref{multi_bss_sim}. In Fig. \ref{MIMO_examples},  as the inter-distance between the two APs decreases from left to right in the four simulation scenarios, the mutual interference risk increases.
We can observe that the signal propagations between different BSSs are isolated, and mutual interference can be effectively suppressed. Additionally, by deploying our proposed optimization approach, the OWE can serve multiple BSSs at the same time without significant compromise in signal quality.

\begin{figure*}[htbp]
\centering
\includegraphics[width=1\textwidth]{Figures/Multi_BSS_SINR_4_1.png}
\caption{OWE Multiple-BSSs application optimization results. Two data streams propagate from two Entry EAs through the OWE to two APs marked as AP1 and AP2 simultaneously.}
\label{MIMO_examples}
\vspace{-0.2cm}
\end{figure*}

\begin{table*}[htbp]
\centering
\begin{tabularx}{0.96\textwidth}{p{5cm}|XXXX}
\toprule
\textbf{\begin{tabular}[c]{@{}l@{}}Signal Paths \\ (Entry EA $\rightarrow$ AP$^*$)\end{tabular}}                                                                   & \textbf{\begin{tabular}[c]{@{}l@{}}EA13 $\rightarrow$ AP1 \\ EA4 $\rightarrow$ AP16\end{tabular}} & \textbf{\begin{tabular}[c]{@{}l@{}}EA1 $\rightarrow$ AP1 \\ EA4 $\rightarrow$ AP2\end{tabular}} & \textbf{\begin{tabular}[c]{@{}l@{}}EA10 $\rightarrow$ AP1 \\ EA1 $\rightarrow$ AP2\end{tabular}} & \textbf{\begin{tabular}[c]{@{}l@{}}EA1 $\rightarrow$ AP1 \\ EA3 $\rightarrow$ AP2\end{tabular}} \\ \midrule
\textbf{\begin{tabular}[c]{@{}l@{}}AP1 Received Signal Quality\\ (Multiple-BSS SINR / Single-BSS SNR)\end{tabular}} & $0.97003$     & $0.94974$  & $1.0000$ & $0.81034$ \\ \midrule
\textbf{\begin{tabular}[c]{@{}l@{}}AP2 Received Signal Quality \\ (Multiple-BSS SINR / Single-BSS SNR)\end{tabular}} & $0.97645$     & $0.96560$ & $0.99712$ & $0.99720$ \\ \midrule
\textbf{\begin{tabular}[c]{@{}l@{}}AP1 Mutual Interference Power Ratio \\(Interference / Received Signal Power)\end{tabular}}  & $0$     & $0$ & $0$ & $1.3845\times 10^{-13}$ \\ \midrule
\textbf{\begin{tabular}[c]{@{}l@{}}AP2 Mutual Interference Power Ratio \\(Interference / Received Signal Power)\end{tabular}}  & $0$     & $0$ & $0$ & $0$ \\ \bottomrule
\end{tabularx}
\caption{The simulation results of the signal quality and mutual interference in the multiple-BSS scenarios.\\
$^*$Note: The positions of APs vary among demonstration scenarios; the arrangement of the columns in this table aligns with the scenarios in Fig. \ref{MIMO_examples} from left to right.}
\label{multi_bss_sim}
\vspace{-0.5cm}
\end{table*}
\par
Specifically, in the third simulation case, "EA1 $\rightarrow$ AP2" has a much longer propagation distance than "EA10 $\rightarrow$ AP1". The difference in the signal propagation distances leads to varying optimal SNR values for the two BSSs (assuming the same signal power at entry EAs). The simulation results show that our objective function can collectively converge BSSs with various conditions and performance towards their respective optima, effectively addressing fairness concerns.
\par
The fourth simulation case (EA1 $\rightarrow$ AP1 and EA3 $\rightarrow$ AP2) shows an extreme scenario where two APs are positioned adjacently. Mutual interference is inevitable, but it is effectively suppressed after deploying the optimization results of our approach.

\begin{figure}
\centering
\includegraphics[width=0.49\textwidth]{Figures/psi_influence_5.png}
\caption{Optimization results for the fourth case of OWE multiple-BSS application scenarios in Fig. \ref{MIMO_examples} and Table \ref{multi_bss_sim}, obtained by varying the pre-gain SNR at the entry EA.}
\label{psi_influence}
\vspace{-0.5cm}
\end{figure}
\par
We proceeded to use the fourth case as an example to show the influence of the received signal strength at entry EAs ($|\psi_i|$ and $|\psi_j|$, assuming $|\psi_i| = |\psi_j|$) on the optimization results of $\mathbf{G}$. Additional simulations were conducted, and the results are shown in Fig. \ref{psi_influence}.
In the multiple-BSS scenario, the received signal strengths at entry EAs have a minimal effect on the optimization results.
\par
In the simulations, we varied the value of $|\psi|$ to adjust the pre-gain SNR at the entry EA from 30 dB to 70 dB. The pre-gain SNR is defined as the ratio between the received signal power and the noise power of the optical receiver at the entry EA. Referring to Fig. \ref{psi_influence}, as the value of $|\psi|$ increases, the SINR of AP1 increases by about 38.9 dB with a degradation of only 1.1 dB relative to the optimal single-BSS SNR, which is a minor performance loss considering the large variance in signal power (40 dB). The SINR of AP2 remains close to optimal in the single-BSS scenario and was minimally influenced.
Additionally, as the power of the received signal increases, mutual interference tends to become more severe, while the proportion of noise power decreases. This shift directs the optimization towards suppressing mutual interference, which is effectively handled by our approach.