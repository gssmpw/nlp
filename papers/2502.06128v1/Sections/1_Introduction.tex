\section{Introduction}
Optical wireless communication (OWC) is a technology that utilizes light to transmit data. It offers significantly higher bandwidth than traditional radio frequency (RF)-based communications, enabling faster data transfer rates. This makes it especially well-suited for bandwidth-hungry and latency-sensitive applications such as video streaming, augmented reality (AR), virtual reality (VR), and cloud gaming. Furthermore, the abundant license-free resources in the light spectrum offer a solution to relax the spectrum scarcity amid the rapidly growing number of connected devices.
\par
To accelerate the wide adoption of OWC, the IEEE 802.11 light communications task group has recently introduced the 802.11bb standard. This standard, an extension to the Wi-Fi specifications \cite{10063242}, aims to enable high-speed visible-light communication.
OWC is a strong candidate for next-generation communication.
\par
However, light is highly directional and cannot penetrate opaque obstacles, thereby restricting the signal coverage of OWC. To address this problem, researchers have investigated using relay methods to establish a cascaded light path to extend light propagation distance. Most existing studies focus on establishing relay paths between fixed points \cite{7180320, 7193338, 7934349, 8761986, 9266077}, where the positions of access points (AP) and stations are fixed and the channel conditions are presumed to remain constant throughout the communication process. These studies only explored simple operational scenarios in point-to-point communication by, for instance, incorporating a relay node between a ceiling-mounted AP and a desktop station.
\par
Conventional wireless relay networks incur inherent latency: the relay node, which deploys amplify-and-forward, is required to buffer the entire or a part of the information packet before retransmitting to mitigate self-interference \cite{6089444, marzban2022securing, 4531911, 4525814, 6735588}. The latency of using a decode-and-forward relay is even worse due to the additional delay incurred by decoding and signal processing \cite{al2017cognitive, 4273702, al2016decoding}. Besides compromising the user experience of time-sensitive applications, this latency undermines OWC's bandwidth advantages, particularly when a data packet needs to traverse multiple “store-and-forward" relay nodes on its way to the destination. 
In this work, we venture in a new direction: constructing an artificial ether fabric that modifies the medium's propagation characteristics to extend the reach of light signals. We refer to the modified medium as optical wireless ether (OWE). OWE does not incur the aforementioned store-and-forward latency of conventional relay networks. Furthermore, OWE is signal-format agnostic: it just changes the propagation characteristics of the natural medium.
\par
OWE is enabled by a group of ether amplifiers (EAs) strategically placed in the environment, such as on the ceiling. EAs employ a simple operating mechanism which we call “instantaneous-propagate" (IP). 
Fig. \ref{EA_Diagram} shows the schematic of an EA. An EA amplifies and transmits signals at the analog level immediately. A data packet's signal is received and transmitted almost simultaneously, without the store-and-forward mechanism in conventional relay networks. The only latency is the propagation delay for the signal to travel through the analog circuit, which is minuscule. By controlling the amplifier gains of a group of EAs, we can modify the propagation characteristics of the OWE.
\begin{figure}
\centering
\includegraphics[width=0.18\textwidth]{Figures/EA_Diagram_1.png}
\caption{The diagram of the EA, which is composed of an optical receiver (Rx), an optical transmitter (Tx), and a programmable gain amplifier (PGA).}
\label{EA_Diagram}
\vspace{-0.2cm}
\end{figure}
\par
OWE can be intelligently configured to achieve various objectives. For instance, it can be optimized to maximize the signal-to-noise ratio (SNR) of the propagating signal within a single basic service set (BSS) \cite{gast2005802} consisting of one AP serving a group of stations. Furthermore, OWE can be configured to promote fairness in resource allocation and mitigate mutual interference across multiple BSSs, where multiple APs serve disjoint groups of stations.
\par
Notably, OWE is similar to the intelligent reflecting surface (IRS) in that they both alter the signal propagation characteristics of the underlying medium and environment. However, they do so in fundamentally different ways. IRS for OWC uses nearly passive elements, like mirrors or liquid crystal arrays, to alter the direction of light rays based on Snell’s law of reflection and refraction, without regenerating or amplifying the signal \cite{maraqa2023optimized, aboagye2022design, aboagye2022ris, 9326394, 9443170, 9276478}. Simply put, optical IRS expands signal coverage by creating virtual line-of-sight (LOS) paths. In contrast, OWE actively amplifies signal power and guides signal propagation through the ether fabric.
\par
Figure 2 presents a comparison between OWE, IRS, and conventional relay networks. Generally speaking, OWE outperforms IRS in received signal strength and surpasses conventional relay networks in latency.

\begin{figure}
\centering
\includegraphics[width=0.49\textwidth]{Figures/Comparison_OWE_RIS_Relay.png}
\caption{The diagram illustrates the working principles of OWE, IRS, and conventional relay networks from top to bottom, respectively.}
\label{OWE_RIS_Relay}
\vspace{-0.5cm}
\end{figure}
\par

The concept of OWE bears resemblance to Ethernet \cite{held2002ethernet, 9844436}, where a cable acts as the ether, and amplifiers are embedded within the cable to extend signal reach. In Ethernet, signal reflections between adjacent amplifiers can be mitigated with simple circuitry, and each amplifier has at most two neighboring amplifiers on the cable.
The situation is much more tricky and complicated for wireless signals, which spread across free space without guidance. 
An EA's input may come from the outputs of multiple EAs, and its output may reach the inputs of multiple EAs, including the EAs whose outputs are part of its input. Such feedback loops can saturate amplifiers, causing signal distortion and rendering the EAs unable to serve their purpose.
However, compared with RF-based signals, optical signals exhibit stronger directionality, faster attenuation, and lower penetration ability, thereby reducing the risk of saturation. Nevertheless, signal reflection, diffraction, and scattering in the environment can still impact the stability of OWE. Thus, a critical challenge in OWE design lies in fine-tuning the amplifier gains in EAs to avert amplifier saturation while providing a substantial enhancement in signal coverage.
\par
The contribution of our work can be summarized as follows:
\par
First, to our knowledge, there have been no prior endeavors to construct an OWE fabric. This paper represents a pioneering exploration into the feasibility of OWE. We built the first OWE prototype, experimentally showing that an artificial ether is possible for OWC.
\par
Second, to guide system design, we put forth a theoretical model that captures the fundamental factors leading to amplifier saturation in OWE. This model not only enables us to explore OWE as a method for extending signal propagation but also serves as a foundation for optimizing system performance with a focus on enhancing signal quality, promoting resource allocation fairness, and mitigating mutual interference in single-BSS and multiple-BSS scenarios, respectively. This framework provides a valuable basis for future exploration and development of the OWE technique.
\par
The remaining part of this paper is organized as follows: Section II introduces the system model and the fundamental theory of OWE. In Section III, we formulate optimization problems based on the system model to enhance OWE performance across various application scenarios. Sections IV and V present simulation and experimental results that validate the accuracy of our theoretical analysis and demonstrate the practical feasibility of our design. Finally, Section VI concludes the paper.