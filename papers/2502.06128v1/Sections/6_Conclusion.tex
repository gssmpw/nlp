\section{Conclusion}
This paper introduces optical wireless ether (OWE), a transformative approach to redefining the signal propagation characteristics of the medium. The primary motivation behind OWE is to address the limited signal coverage inherent in conventional optical wireless communication (OWC) systems.
\par
By modifying the light propagation characteristics of the medium, OWE offers a cost-effective solution for extending signal propagation distances in multiple directions, ensuring ubiquitous coverage. The introduction of OWE represents a significant advancement with the potential for deployment across a wide range of OWC applications.
\par
The key components of OWE are ether amplifiers (EAs), which function as virtual optical amplifiers in the air, enabling the cascading of light propagation paths. EAs employ a novel instantaneous-propagate (IP) mechanism that leverages the directional nature of light to enable simultaneous reception and transmission without introducing additional latency. Notably, the EA design is signal-format agnostic, as it operates entirely at the analog level -- receiving, amplifying, and transmitting signals without requiring analog-to-digital (A/D) or digital-to-analog (D/A) conversions.
\par
This paper comprehensively examines OWE through theoretical analysis, simulations, and experiments. A significant portion of the theoretical studies focuses on addressing the stability challenges associated with deploying the IP mechanism. Specifically, we provide a systematic analysis of the constraints on EA gain settings and demonstrate how proper gain adjustments can stabilize the OWE system. Additionally, we explore the optimization of EA gain values for various objectives, such as maximizing the signal-to-noise ratio (SNR) of the propagation signal within a single basic service set (BSS) and ensuring fairness in resource allocation. This includes mitigating mutual interference when serving multiple BSSs simultaneously.
\par
Our simulations and experiments validate the feasibility of OWE and demonstrate its ability to boost SNR. Furthermore, we show that OWE can significantly reduce packet-loss rates in a Wi-Fi-based TCP/IP network, underscoring its broader applicability and practical benefits.
\par
Looking ahead, the principles and mechanisms introduced in this work pave the way for further advancements in optical wireless communication systems. By enabling cost-effective, scalable, and high-performance signal propagation, OWE has the potential to transform wireless communication infrastructure, supporting the growing demand for high-speed, reliable connectivity in emerging technologies such as smart cities, 6G networks, and beyond.