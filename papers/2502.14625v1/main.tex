
\documentclass[journal]{IEEEtran}

\usepackage{booktabs}
\usepackage{multirow}
\usepackage{arydshln}
\usepackage{makecell}
\usepackage{hyperref}
\usepackage{float}

\usepackage{cite}

\usepackage{graphicx}

\hyphenation{op-tical net-works semi-conduc-tor}

\begin{document}

\title{Multi-Record Web Page Information Extraction From News Websites}

\author{
\IEEEauthorblockN{Alexander Kustenkov$^{1,2}$, Maksim Varlamov$^{1}$, and Alexander Yatskov$^{1,2}$}\\
\IEEEauthorblockA{$^1$\textit{Ivannikov Institute for System Programming of the Russian Academy of Sciences, Moscow, Russia}}\\
\IEEEauthorblockA{$^2$\textit{Lomonosov Moscow State University, Moscow, Russia}\\
\{kustenkov, varlamov, yatskov\}@ispras.ru}
}
        

\maketitle

\begin{abstract}

% Recent works to jointly reconstruct 3D human and object from a single RGB image, are mostly model-based, that fail to capture the fine details of the clothed human body and object surface. In this paper, we introduce ReCHOR, a novel, model-free, first-method to produce realistic clothed human-object reconstructions from a monocular view. This is extremely challenging due to human-object occlusions, diverse interactions and depth ambiguity, as it needs to infer both 3D spatial awareness and high resolution details. Our core idea is based on estimating neural implicit representations for human and object respectively by an attention-based neural implicit model that attends to pixel-aligned features from both the global human-object image for spatial awareness and  the local separate view of human and object images for high quality details. Additionally, the network is conditioned on semantic features from an initial estimated human-object pose prior and a generative diffusion model that inpaints occluded regions, thus enabling the retrieval of details from them.
% We also propose a synthetic dataset with rendered scenes of diverse, inter-occluded 3D human and object scans, to train our network. We evaluate our method on the synthetic and real world BEHAVE dataset. Our experiments show that our method outperforms the SOTA in achieving realistic clothed human-object reconstructions.
Recent approaches to jointly reconstruct 3D humans and objects from a single RGB image represent 3D shapes with template-based or coarse models, which fail to capture details of loose clothing on human bodies. In this paper, we introduce a novel implicit approach for jointly reconstructing realistic 3D clothed humans and objects from a monocular view. For the first time, we model both the human and the object with an implicit representation, allowing to capture more realistic details such as clothing. This task is extremely challenging due to human-object occlusions and the lack of 3D information in 2D images, often leading to poor detail reconstruction and depth ambiguity. To address these problems, we propose a novel attention-based neural implicit model that leverages image pixel alignment from both the input human-object image for a global understanding of the human-object scene and from local separate views of the human and object images to improve realism with, for example, clothing details. Additionally, the network is conditioned on semantic features derived from an estimated human-object pose prior, which provides 3D spatial information about the shared space of humans and objects. To handle human occlusion caused by objects, we use a generative diffusion model that inpaints the occluded regions, recovering otherwise lost details. For training and evaluation, we introduce a synthetic dataset featuring rendered scenes of inter-occluded 3D human scans and diverse objects. Extensive evaluation on both synthetic and real-world datasets demonstrates the superior quality of the proposed human-object reconstructions over competitive methods.
\end{abstract}

\begin{IEEEkeywords}
Data collection, Data extraction, Dataset, Multi-record extraction, Neural network
\end{IEEEkeywords}

\IEEEpeerreviewmaketitle

\def\ItIsIEEEDoc{1} % This is Kustenkov's define 
\section{Introduction}\label{sec:intro}

In computational finance, Monte Carlo simulations are used extensively to estimate the expected value of financial payoffs based on the solution of stochastic differential equations (SDEs) which model the evolution of stock prices, interest rates, exchange rates and other quantities \cite{glasserman04}.  Monte Carlo methods are very general and flexible, but for high accuracy it requires generating a large number of costly SDE path approximations, which has motivated research into a number of variance reduction or, equivalently, cost reduction techniques. One such method is
Multilevel Monte Carlo (MLMC), which was proposed in \cite{GILES2008} and was adapted for various applications that are summarised in \cite{Giles_overview17} and successfully combined with other methods such as quasi-Monte Carlo methods. The main idea of MLMC is to approximate the payoff using different time stepping resolutions when numerically solving the underlying SDE and to generate an optimal number of samples on each level, such that the overall computational cost is minimised subject to the desired bound on the variance. %, such that the total computational cost is minimised. 
The computational savings come from the fact that most samples are computed on the coarser levels and hence are less expensive while only a few samples from the finest levels are required \cite{GILES2008}.


Among the directions in which the computational cost 
of MLMC methods could further be reduced, an important avenue is the use of lower precision calculations, especially for the first Monte Carlo levels where the targeted accuracy is relatively low. 
 An overview of the research on mixed precision for the standard Monte Carlo (MC) framework is provided in \cite{ChowMixedPrecisionStandardMC} but only a few references study the potential of low precision computation in the MLMC framework \cite{Rounding_error_oliver}. To the best of our knowledge, the only MLMC framework with customised precision in the literature is \cite{brugger2014mixed}, but they use a uniform precision for all operations on each Monte Carlo level instead of optimising 
 the precision of each intermediary variable to reduce as much as possible the cost of path generation.
 
An important motivation for an MLMC framework with variable precision would be performing the low precision computations on reconfigurable hardware devices such as Field Programmable Gate Arrays (FPGAs). FPGAs contain customizable logic blocks and connectors that make it easy to adapt the digital circuit architecture for a specific application, leading to a highly parallel and optimised implementation. Therefore they are successfully exploited in applications that require high speed and have high computational workload, such as signal processing \cite{woods2008fpga}, and real time applications like high frequency trading \cite{HFT1,HFT2}. That is why a number of previous works in hardware architecture design implemented the MLMC algorithm to price financial options using FPGAs as accelerators, which resulted in improved speed and power efficiency compared to full CPU architectures \cite{Schryver2013AMM}. The paper \cite{lindsey2016domain} also proposed 
a Domain Specific Language to automate the configuration of FPGAs for this specific application. However, only \cite{brugger2014mixed} proposed a heuristic to reduce the precision in calculations.

In addition, all aforementioned works considered that the random number generation (RNG) is performed in single or double precision. Yet in most cases an important portion of the workload in the overall MLMC simulation comes from the RNG and in \cite{brugger2014mixed} this limited the total computational savings.
To reduce the cost of MLMC simulations in particular those based on the Geometric Brownian Motion (GBM), \cite{approximateICDF_Oliver, NestedOliver} have proposed to use approximate random numbers that are generated by applying an approximation of the inverse CDF to uniform random numbers. In \cite{NestedOliver}, the authors proposed a way to integrate these lower precision random variables into a \textit{nested} MLMC framework and completed a numerical analysis to bound the resulting error at each MC level by a product of the time step and the error in the random number approximation. The same authors show in \cite{approximateICDF_Oliver} that using approximate random variables reduces the cost of path generation by a factor 7.


In this paper we propose a nested MLMC framework that combines the use of approximate random normal variables and lower precision calculations to reduce the computational cost of MLMC even further than \cite{brugger2014mixed,NestedOliver}. We illustrate the efficiency of our framework in Matlab, after making several assumptions on the cost of operations and size of the errors that we carefully justify. We focus on the case of GBM and use the approximate RNG methods presented in \cite{approximateICDF_Oliver} as well as a new slightly modified method that combines CDF inversion and the central limit theorem. To choose the precision of the variables in the low precision path generation, we introduce a novel method to optimise the bit-widths. This optimisation is performed before the main path generation loop is executed and is based on a linear model of the payoff error  
due to rounding when computing in low precision. The error model relies on algorithmic differentiation in a similar manner to \cite{unifying-bwoptim,bitwidth-AD,ADAPT}. The bit-width optimisation procedure can be performed off-line, so this stage can be excluded from the on-line time complexity of our framework. The user specified desired accuracy is then enforced by calculating on-line the number of samples that need to be generated.

In terms of hardware design, we suggest implementing the low precision path generation on FPGAs and the full-precision ones on a CPU or GPU. 
The FPGA offers enough flexibility to define a separate bit-width for every variable in the low precision path generation, and can be reconfigured periodically to update the bit-widths when the market parameters have changed considerably. 


The paper is organized as follows : \Cref{sec:MLMC} introduces MLMC and nested MLMC to make clear the estimator that is implemented in our framework. Then in \Cref{sec:RNG} we detail the methods that could be used to obtain approximate random normally distributed numbers very cheaply for the low precision path generation. In \Cref{sec:error_model} and \Cref{sec:costModel} we propose an error model and a cost model (resp.) that we then use to formulate the optimisation problem that is solved to obtain the optimal bit-widths of fixed point variables in \Cref{sec:optimisation}. Finally we summarise our results and future directions in \Cref{sec:conclusion}.




 

\section{Related Work}
\label{sec:related_work}

The original investigation \cite{gibson1979ecological} on the relationship between visual perception and human action defines \emph{affordance} as the opportunities for interaction with the surrounding environment. Behavioral studies on regular and cognitively impaired persons have shown evidence that perception results in both visual and motor signals in the human brain. An extended study \cite{anderson2002attentional} shows that visual attention to the spatial characteristics of the perceived objects initiates automatic motor signals for different actions. In computer vision, human affordance learning involves novel pose prediction such that the estimated pose represents a valid human action within the scene context. The task is fundamental to many problems requiring robust semantic reasoning about the environment, such as human motion synthesis \cite{wang2021scene} and scene-aware human pose generation \cite{wang2017binge, roy2016multi, zhang2022inpaint, yao2023scene}.

Earlier methods of affordance learning have explored knowledge mining \cite{zhu2014reasoning} and multimodal feature cues \cite{roy2016multi} to address the problem. In \cite{zhu2014reasoning}, the authors use a Markov Logic Network for constructing a knowledge base by extracting several object attributes from different image and metadata sources, which can perform various downstream visual inference tasks without any additional classifier, including zero-shot affordance prediction. In \cite{roy2016multi}, the authors use depth map, surface normals, and segmentation map as multimodal cues to train a multi-scale convolutional neural network (CNN) for scene-level semantic label assignment associated with specific human actions. In \cite{do2018affordancenet}, the authors design a multi-branch end-to-end CNN with two separate pathways for object detection and affordance label assignment to achieve high real-time inference throughput. Researchers \cite{chuang2018learning} have also explored socially imposed constraints for affordance learning. In \cite{chuang2018learning}, the authors propose a graph neural network (GNN) to propagate contextual scene information from egocentric views for action-object affordance reasoning.

Probabilistic modeling of scene-aware human motion generation also involves semantic reasoning of human interaction with the environment. Initial works on human motion synthesis have taken different architectural approaches, such as sequence-to-sequence models \cite{barsoum2018hp}, generative adversarial networks (GAN) \cite{barsoum2018hp, cai2018deep, yang2018pose}, graph convolutional networks (GCN) \cite{yan2019convolutional}, and variational autoencoders (VAE) \cite{guo2020action2motion}. However, these methods have mostly ignored the role of environmental semantics. Due to potential uncertainty in human motion, in a recent approach \cite{wang2021scene}, the authors address such motion synthesis with a GAN conditioned on scene attributes and motion trajectory to predict probable body pose dynamics.

One key challenge of human affordance generation in 2D scenes is the lack of large-scale datasets with rich pose annotations. In \cite{wang2017binge}, the authors compile the only public dataset of annotated human body poses in complex 2D indoor scenes by extracting frames from sitcom videos. Aiming to generate a contextually valid human affordance at a user-defined location, the authors propose sampling the scale and deformation parameters for an existing human pose template using a VAE conditioned on the localized image patches as scene context. In \cite{zhang2022inpaint}, the authors introduce a two-stage GAN architecture for achieving a similar goal by estimating the affine bounding box parameters to localize a probable human in the scene and then generating a potential body pose at that location. The method uses the input scene, corresponding depth, and segmentation maps as semantic guidance. In \cite{yao2023scene}, the authors propose a transformer-based approach with knowledge distillation for generating human affordances in 2D indoor scenes.


\section{Problem definition}
Given a set $W$ of n websites. Each website $W_i$, where $1 \leq i \leq$ n, is represented by a certain number of pages $m_i$, a website may consist of just one page, i.e. $m_i = 1$. 
Each page $W_i^j$ on website $W_i$, where $j$ is the page number, with $1 \leq j \leq m_i$, is a multi-record page, meaning it contains $k$ records where $2 \leq k$. 
We defined a record as an entity with a predefined set of characteristics, e.g. in the news domain these characteristics might include date, author, title, tag, etc. So the set of record's characteristics is a vector $H = \{h_0, h_1, ..., h_t\}$, where $t$ the number of characteristics in a given record. It is important to note that a record can have multiple characteristics of the same type, such as multiple tags.

We formulate the task of extracting information from multi-record pages as identifying all vectors $H$ within the given set $W$, with the following constraints:
\begin{enumerate}
  \item The proposed methods must not rely on visual information based on page rendering.
  \item The proposed methods must be applicable to all multi-record pages in the domain, even if the website was not included in the training dataset.
  \item The proposed methods should generalize beyond the news domain to other fields.
\end{enumerate}

The output of each method should be a structured representation of the records for each page, for example in JSON format.

\section{Dataset construction}
We decided to collect our own dataset for the task of extraction information from multi-record web-pages and make it publicly available. In this chapter we describe how the data was collected and annotated, also we provide the final dataset's main characteristics.

\subsection{Data collecting}
Our dataset contains news web pages collected from Russian-language media. News resources were selected according to the MediaMetrics\footnote{https://mediametrics.ru/rating/ru/online.html} latest news quotations system, which provides rankings based on the popularity of news resources. Web pages were downloaded between 12/28/2023 and 04/28/2024. Pages were downloaded using a special Python3 script (based on the Scrapy library\cite{scrapy_library}), which was run daily through prepared sitemaps.

\subsection{Sitemaps development}
Sitemaps were prepared using the WebScraper\footnote{https://github.com/ispras/web-scraper-chrome-extension} browser extension for Chrome. Using sitemaps in special web crawlers allows to download an html page, as well as an answer for this page. Thus, each site requires the development of a unique sitemap. On each page, the boundaries of each record and its attributes were annotated, if they existed. Only the following attributes were noted: \textit{date}, \textit{title}, \textit{tag}, \textit{short\_title}, \textit{author}, \textit{time}. The annotation was done manually.
We chose this set of attributes because they are the most popular in the news websites domains. 
For each site several categories were annotated, for example, politics, economics, and sports. In this way 312 sitemaps were prepared.

\subsection{Dataset preparing}
For further use of data in our dataset, it was necessary to preprocess the raw data. The following actions were carried out:
\begin{enumerate}
    \item Filtering duplicate pages. Some downloaded pages contained many records scraped before on other pages. Such pages, with more than a quarter repeated records, were filtered.
    \item Cleaning HTML. At this stage blocks, such as blocks of code in JavaScript (i.e. all nodes with the \verb|<script>| tag), that do not affect the algorithm's performance were removed from the HTML code of the page.
    \item Translating HTML. Since our dataset contains pages in Russian, it was necessary to translate them into English to be able to use pre-trained models.
    \item Division into training and test parts. The distribution of attributes and domains of web pages was taken into account. Each domain was placed either to the training or to the test parts (see attributes split in the \autoref{fig:attribute_dist}). The final distribution is shown in the \autoref{fig:amount_of_attributes_test} and \autoref{fig:amount_of_attributes_train}. Ratio of parts after splitting was the following: 75\% - training, 25\% - testing.
\end{enumerate}

\subsection{Dataset Statistics}
Since the maps were based on CSS selectors, there was a problem with downloading sites that dynamically change the names of the styles on the pages. In other words, when the website changed the name of the CSS style class, the selector specifying the class name stopped working correctly. Therefore, we were able to download only 278 sites.

\begin{table}[!htb]
    \caption{Attribute frequency}
    \centering
    \begin{tabular}{c|c|c|c}
        \toprule
        Name & Pages & Records & Sites\\
        \midrule
        title & 12679 & 247262 & 275\\
        date & 12296 & 241634 & 251\\
        tag & 6165 & 108400 & 140\\
        \cdashline{1-4}
        short\_text & 6855 & 115983 & 138\\
        short\_title & 105 & 1289 & 4\\
        author & 87 & 957 & 1\\
        time & 730 & 15809 & 8 \\
        \bottomrule
    \end{tabular}
    \label{tab:attribute_statistic}
\end{table}
Thus, a dataset that contained 13120 pages from 278 Internet media was prepared (distribution between pages and entries on them, shown in the \autoref{fig:page_records_dist}). On each page, the corresponding attributes were annotated,  and their frequency was presented in the \autoref{tab:attribute_statistic}. All pages presented in the table have UTF-8 encoding. 

We researched extraction of 3 attributes from our dataset: title, author and tag. Because they are the most frequent in our dataset.
\section{Experimental evaluation}
In this chapter we describe our methods of solving the problem and show the results on our dataset.
\subsection{Parallel pipeline}
The first architecture we tested was ``Parallel pipeline``. Visual scheme of this architecture described in \autoref{fig:parallel_pipeline}.
\begin{figure}[!htb]
    \centering
    \includegraphics[width=0.45\textwidth]{attach/parallel_pipeline.jpg}
    \caption{Parallel pipeline scheme}
    \label{fig:parallel_pipeline}
\end{figure}
\begin{itemize}
    \item \textbf{Segmentation} At this stage, we find records boundaries that help us split html into records and find corresponding attributes of each record.
    \item \textbf{Classification} At this stage, we give a corresponding label for each node on html. It is important that the segmentation stage and the classification stage are two independent stages.The results of neither are not shared with the other.
    \item \textbf{Matching of results} At this stage, the final result of the method is formed. Based on the results of segmentation and classification, the records are matched with their attributes and provided in a structured view.
\end{itemize}

\subsection{Sequential pipeline}
We will also test sequential pipeline as described at \autoref{fig:sequential_pipeline}. In the following we will compare the qualities of both architectures. Quality should vary due to different approaches to the classification stage.
\begin{figure}[!htb]
    \centering
    \includegraphics[width=0.45\textwidth]{attach/sequential_pipeline.jpg}
    \caption{Sequential pipeline scheme}
    \label{fig:sequential_pipeline}
\end{figure}
\begin{itemize}
    \item \textbf{Segmentation} In sequential architecture this stage is similar to parallel scheme.
    \item \textbf{Classification} At this stage, the final result of the method is formed. Based on the segmentation results, the stage of searching for attributes only in the selected area is carried out. Thus, the matching stage is not required.
\end{itemize}


\subsection{Segmentation subtask} \label{segmentation}
We formulate the task of web-pages segmentation as a task of searching for information boundaries, which are first nodes containing text of the DOM subtree and belonging to it. This formulation is the same as that used in the \cite{san_plate_2023}.

We test two methods of solving this task:
\begin{enumerate}
    \item Heuristic method based on classical MDR
    \item Neural method based on MarkupLM, which is state-of-the-art model in information extraction from HTML pages 
\end{enumerate}

\textbf{MDR} We tested MDR because it has open-source code. Since this method proposes several ``candidates`` for segmentation, ordered by their probability. We choose the segmentation with the highest probability.

\textbf{MarkupLM} We train the MarkupLM model on this task, having the data previously prepared: for each record on the page, we have marked its first node which contains text with the “BEGIN” label. All other nodes on the page have been marked with the label “OUT“. Thus, the solution of the problem is to predict the corresponding label for each DOM tree node by the model.

\subsubsection*{Segmentation metrics}
The result of the segmentation method can be evaluated by page-weighted metrics: $Precision_{avg}$, $Recall_{avg}$, $F1_{avg}$

To calculate them, the reference segmentation of the given page on the record is compared with the one obtained by the proposed methods. Based on this comparison, for each segment it is possible to calculate $TP_{page K}$ -- the number of DOM nodes correctly marked as an information boundary, $FP_{page K}$ - the number of DOM tree nodes that are not an information boundary, but marked with it, $FN_{page K}$ -- the number of DOM tree nodes that are an information boundary, but not marked as it:

$$
Precision_{page K} = \frac{TP_{page K}}{TP_{page K} + FP_{page K}}, 
$$

$$
Recall_{page K} = \frac{TP_{page K}}{TP_{page K} + FN_{page K}},
$$

$$
F\textit{1}_{page K} = 2 \cdot \frac{Precision_{page K} \cdot Recall_{page K}}{Precision_{page K} + Recall_{page K}},
$$

$$
Precision_{avg} = \frac{\sum_{i = 1}^{K}Precision_{page\,i}}{K},
$$

$$
Recall_{avg} = \frac{\sum_{i = 1}^{K}Recall_{page\,i}}{K},
$$

$$
F\textit{1}_{avg} = \frac{\sum_{i = 1}^{K}F\textit{1}_{page\,i}}{K}
$$

Also we will calculate classical $NMI$\cite{Strehl2002} and $ARI$\cite{Hubert1985} for segmentation task.

\begin{table}[!htb]
    \caption{Results of segmentation experiment}
    \centering
        \begin{tabular}{c|ccc|cc}
        \toprule
        Method & $Recall_{avg}$ & $Precision_{avg}$ & $F\textit{1}_{avg}$ & $ARI$ & $NMI$\\
        \midrule
         MDR & 0.473 & 0.486 & 0.465 & 0.437 & 0.517\\
         MarkupLM & \textbf{0.908} & \textbf{0.941} & \textbf{0.925} & \textbf{0.802} & \textbf{0.869}\\
         \bottomrule
    \end{tabular}
    \label{tab:segmentation}
\end{table}

The test results are shown in the \autoref{tab:segmentation}. The MarkupLM model shows superior results in all metrics in comparison with the other tool.

\subsection{Classification subtask}
This subtask includes searching for all potential attributes in the area of interest on the html page (for parallel pipeline - it is the whole page, and for sequential - a part of it).

\begin{table}[htbp]
\centering
\caption{Results of classification experiment}
\begin{tabular}{c|c|*3c}
    \toprule
        Type                                    &     Metric            & title    & tag      &  date \\
        
        \midrule
\multirow{3}{*}{Record context}                & $Precision_{avg}$     & 0.98     & 0.79     & 0.86  \\
                                                & $Recall_{avg}$        & 0.99     & 0.85     & 0.91  \\
                                                & $F\textit{1}_{avg}$   & \textbf{0.99}     & 0.82      & \textbf{0.88}  \\
        
        \midrule
\multirow{3}{*}{Page context}                   & $Precision_{avg}$     & 0.68     & 0.61       & 0.64      \\
                                                & $Recall_{avg}$        & 0.95     & 0.76       & 0.86      \\
                                                & $F\textit{1}_{avg}$   &  0.79    & \textbf{0.89}       & 0.73      \\
    \bottomrule
\end{tabular}
\label{tab:markuplm_classification}
\end{table}

The MarkupLM model is also chosen as a classification method. We train it on the task of predicting a node’s label. Thus, the model predicts the labels: ``title``, ``tag``, ``date``. 


\subsubsection*{Classification metrics}
Results are evaluated by classical page-weighted classification metrics: $Precision_{avg}$, $Recall_{avg}$ and $F\textit{1}_{avg}$.

We trained two models: one for the pipeline with full page context (in such conditions, classification is used in the parallel pipeline), and one with just record context (passing the part of the html related to a specific record directly to the input, so it is used in a sequential pipeline). The comparison of their results is presented in the table \autoref{tab:markuplm_classification}. So, MarkupLM with record context shows rather high results in this subtask.

\subsection{Matching subtask}
We propose matching algorithm:
\begin{itemize}
    \item Let $G$ -- set of xpaths\footnote{We are considering positional xpath expressions consisting of tags and indices in the DOM tree path from root to the given node.} to all DOM-nodes of htmls marked as ``information boundary`` at the segmentation stage.
    \item Let's make $\widehat{G}$: for each xpath from $G$ find minimal (by length) node prefix. Each prefix should belong to only one xpath from $G$.
    \item Let's enumerate $\widehat{G}$.
    \item  Let's refer Xpath of some attribute determined at classification stage as $xpath_{attr}$. Each attribute is matched with record with number $N$ if $\widehat{G}[N]$ is prefix for $xpath_{attr}$.
\end{itemize}


\subsection{Final method evaluation metrics}
We evaluated the final method using metrics similar to detailed page information extraction task's metrics. We define ``predicted record`` as a set of predicted attributes matched to the same record at the Matching stage. Also we define ``reference record`` as a set of ground truth attributes values from a sought ``information boundary``. There are three cases possible when the method is run:
\begin{enumerate}
    \item \textit{We can match the predicted record with the reference record.}\\ $TP$ is the number of correctly extracted attributes (for example if a record contains 4 tags and they are extracted correctly, then $TP$ is 4). $FP$ is the number of extracted attributes but none of them should have been. $FN$ is the number of not extracted attributes but all of them should have been.
    \item \textit{We cannot match any of the predicted records with the reference record.}\\ Only $FN$ increases for all attributes of the reference record.
    \item \textit{We cannot match any of the reference records with the predicted record.}\\ In this case $FP$ increases for all attributes of the predicted record.
\end{enumerate}

\subsection{Experiments results}
\begin{table}
\centering
\caption{Results of final method}
\begin{tabular}{c|c|*3c}
    \toprule
        Method                  &     Metric     & title    & tag       &  date        \\
        
        \midrule
\multirow{6}{*}{Parallel Pipeline}             & TP & 56008    & 29202     & 60570        \\
                                                & FP & 7021     & 10804     & 12728       \\
                                                & FN & 8083     & 4484      & 12361       \\
                                                \cmidrule(lr){2-5}
                                                & Precision & \textbf{0.874}     & \textbf{0.867}     & \textbf{0.831}  \\
                                                & Recall    & 0.889     & 0.73      & 0.826  \\
                                                & F1        & 0.881     & 0.793     & 0.828  \\
        
        \midrule
\multirow{6}{*}{Sequential Pipeline}            & TP & 55891    & 28643     & 58645           \\
                                                & FP & 1492     & 7641      & 7056            \\
                                                & FN & 8801     & 4899      & 13659           \\
                                                \cmidrule(lr){2-5}
                                                & Precision & 0.864     & 0.854     & 0.811  \\
                                                & Recall    & \textbf{0.974}     & \textbf{0.789}     & \textbf{0.893}  \\
                                                & F1        & \textbf{0.916}     & \textbf{0.82}      & \textbf{0.85}   \\
    \bottomrule
\end{tabular}
\label{tab:final_experiments_results}
\end{table}

We tested both proposed methods: parallel pipeline and sequential pipeline. For these methods, we used the corresponding trained versions of MarkupLM, as we described them in the previous chapters. Comparative results are shown in \autoref{tab:final_experiments_results}, while \autoref{tab:markuplm_classification} presents the performance of classification model in different contexts—record and page.

In \autoref{tab:markuplm_classification}, the model using the ``record context`` outperforms model using the ``page context``. The tag attribute in the page context performs relatively well, but the overall trend shows that the model is more effective in the record context. We assume that the advantage of page context for tag-attribute extraction is related to the peculiarity of this attribute. Most often, each record has several tags. Information about neighboring records allows making less noisy predictions.

The parallel pipeline demonstrates solid precision, particularly for title. However, it shows a decline in recall. In contrast, the sequential pipeline outperforms the parallel pipeline in recall. The sequential pipeline achieves a balanced performance, making it the preferred method due to its robustness and consistency.

In conclusion, the sequential pipeline is superior to the parallel pipeline. The record context further enhances classification performance, making it the optimal setting for method deployment. The high results of the obtained methods prove their applicability in real life.

\section{Concluding Remarks}
In this paper, we proposed a novel approach utilizing multimodal LLMs to generate gesture-aware speech recognition transcripts for patients with language disorders. Our framework integrates verbal speech and iconic gestures, enabling the generation of enriched transcripts that capture the latent meaning conveyed through both modalities. Through extensive experimentation, we demonstrated that the proposed method effectively contextualizes incomplete or disfluent speech by incorporating gesture information, leading to more accurate and meaningful representations of the speaker's intent. These findings highlight the potential of our approach to significantly contribute to the field of speech and language therapy, offering innovative tools that can enhance the quality of life for individuals with language disorders by facilitating better communication and assessment methods.

\subsection{Ethical Statement} 
Our dataset was obtained from AphasiaBank with the approval of the Institutional Review Board (IRB) and adheres to the data sharing guidelines set by TalkBank\footnote{https://talkbank.org/share/ethics.html}. This includes complying with the Ground Rules for all TalkBank databases, which are based on the American Psychological Association Code of Ethics~\cite{american2002ethical}.

\subsection{Limitation \& Future Work} 
%This study represents a preliminary investigation into using multimodal LLMs to generate gesture-aware speech recognition transcripts. 
While the results are promising, we recognize several limitations and outline our plans to extend this work further.

One primary limitation is the absence of a definitive ground truth for quantitative evaluation. Since our model generates transcripts by synthesizing speech and gesture data from scratch, traditional benchmarks, such as comparisons with standard speech recognition outputs, are insufficient. Moreover, existing original transcripts lack gesture annotations, making direct comparisons challenging. In future work, we aim to address this gap by collaborating with certified pathologists to conduct qualitative assessments, such as A-B preference tests, to evaluate the effectiveness of gesture-enriched transcripts in accurately conveying the speaker's intentions.

To support quantitative evaluations, we plan to develop novel metrics that assess transcript quality, including grammar accuracy, semantic consistency, and the integration of multimodal information. Such metrics will provide a more objective basis for assessing our model's performance and facilitate comparisons with other multimodal and unimodal approaches.

Another limitation of this study is its focus on structured gestures from a specific task, the Peanut Butter Sandwich Task. While this task offers a controlled context for testing our approach, it does not encompass the diversity of gestures and communication patterns seen in everyday scenarios. As part of our future work, we plan to expand the scope of our model to include tasks such as the Cinderella Story Recall Task~\cite{bird1996cinderella}, which involves unstructured and complex narrative gestures. This expansion will allow us to evaluate the adaptability and robustness of our model in handling varied linguistic and gestural contexts.

In summary, while this study establishes a strong foundation for gesture-aware speech recognition, we aim to refine and extend our methods through collaborative qualitative evaluations, the development of robust quantitative metrics, and broader task applications. These efforts will ensure that our approach continues to evolve, ultimately contributing to more effective communication tools and interventions for individuals with language disorders.





\appendices

\ifCLASSOPTIONcaptionsoff
  \newpage
\fi

\bibliographystyle{IEEEtran}

\bibliography{main}


\section{Pre-annotation and classifiers}\label{app:pre}
\paragraph{Pre-annotation}
We start with the \textbf{Second Reading debates of Bills},\footnote{\url{https://www.parliament.uk/about/how/laws/passage-bill/commons/coms-commons-second-reading/}} where the members debate the main principles of a certain Bill. The advantages of using such debates are: (i) the stance of an argument can be easily identified based on whether they support %for 
the Bills; (ii) debates can be paired with brief Bill introductions,\footnote{e.g., the `long title' on page \url{https://bills.parliament.uk/bills/3858}} providing clear argument topics; and (iii) the arguments 
focus on Bill principles, with fewer discussions on specific amendments and clauses, which require less contextual awareness than other Bill debates like the ones for the Committee Stage.\footnote{\url{https://www.parliament.uk/about/how/laws/passage-bill/commons/coms-commons-comittee-stage/}}
We choose five Bills, including topics relevant to animal welfare and parental leave (see Table \ref{tab:bill} for the Bill introductions), 
which may be easier to annotate and more likely to have emotional arguments.

Three annotators label 245 texts from these debates for \textbf{three layers}: (\emph{L1}) 
whether the text evokes emotions, (\emph{L2}) whether the text contains standalone arguments, and (\emph{L3}) the stance of the text toward the Bill. \emph{L1} and \emph{L2} are labeled `0' (for answer `no') or `1' (for `yes'). If \emph{L2} is labeled `1', annotators proceed to label \emph{L3}, which has four options: `0' for support, `1' for opposition, `2' for inability to identify stance without additional context, and `3' for a neutral stance suggesting additional amendments or policies. Besides, 40 texts from the pilot annotation are also annotated for \emph{L1} and \emph{L2}.  
To potentially speed up the annotation process, the 285 texts are selected from those judged as both emotional and argumentative by GPT4o. Here, we prompt GPT4o with simple questions such as \emph{Does this text try to convince readers something?} and \emph{Is this text emotional?'}.

40 of the outputs are jointly labeled by all annotators, achieving average Cohen's Kappa of 0.622 for \emph{L1}, 0.674 for \emph{L2}, and 0.762 for \emph{L3} across annotator pairs. 
As shown in the `Question' column of Table \ref{tab:pre}, GPT4o already achieves a high precision of 0.82 in detecting argumentative texts using simple prompts. However, its precision for emotional text classification is still low (0.53).

We then convert the annotations for \emph{L3} to \emph{L3$^{*}$}, where we pair argument pairs based on their topics and stances. The categories include: `different topic' for pairs with different topics (from different Bills), `different stance' for pairs with the same topic but different stances, and `same' for pairs with the same topic and stance.

The number of texts annotated for each layer and the corresponding label distribution\se{s} are summarized in Table \ref{tab:pre} (left). 

\paragraph{Automatic Pipeline}
We develop three classifiers based on GPT4o 
to automatically identify the argument pairs needed. The pipeline is as follows: 
\begin{enumerate}[]
    \item \textbf{Argumentative text classification}: our goal is to have a \textbf{high precision} classifier since we have sufficient candidate texts. We find that when we ask GPT-4o to provide the major claim, evidence, and reasoning connecting the evidence to the major claim in the text, its precision increases from 0.82 to 0.96, as shown in the `Argumentative' row of Table \ref{tab:pre}. 
    
    We then retain texts judged as argumentative for \hansard{} using this prompt, while for \deuparl{}, we use a German translation of the same prompt. The overall performance of GPT4o on German data is assessed after completing the stance agreement classification task (see below).

    \item \textbf{Stance agreement classification}: 
    To enable the flexible selection of classifiers with specific performance characteristics (e.g., high recall, high precision), we introduce a parameter into the prompt, with its threshold optimized to achieve different specialized performance levels.
    To do so, we ask GPT4o to rate the likelihood that two given arguments address the same topic and share the same stance on a Likert scale from 0 to 100. We randomly sample 600 argument pairs (with a 2:1:1 ratio for the three categories of \emph{L3$^{*}$}) from the dataset, ‘optimize’ the threshold of ratings for the `same’ category 
    using argument pairs from two Bills, and test the performance on the remaining three Bills to prevent data leakage. We evaluate all possible combinations of Bills for the training and test sets.
    We observe that as the threshold increases, precision on the `same’ category ($P_{same}$) consistently improves, while macro F1 begins to decrease beyond certain thresholds. With a threshold of 100, $P_{same}$ reaches 0.92, but F1 is very low at 0.45. Therefore, we select a threshold of 90 as a more balanced trade-off, achieving $P_{same}=0.81$ and $\textit{F1}=0.76$, to obtain more candidates that are still highly likely to be true positives. 
    
    For \hansard{}, we retain the argument pairs labeled as belonging to the `same' category using this threshold. For \deuparl{}, we apply the German translation of the prompt with the same threshold to identify argument pairs. One annotator evaluates 50 candidates from the outputs of steps 1 and 2: no argument is labeled as non-argumentative, while 12 argument pairs are identified as false positives in the stance agreement task, yielding $P_{same}=0.76$. This value is only 4 percent points lower than the result on English data. Consequently, we retain these prompt settings for the German data.
    
    \item  \textbf{Emotional text classification}: we aim for a \textbf{balanced} classifier because we also need non-emotional arguments. Since this is a subjective task, we ask GPT4o to rate how likely it can feel the emotions 
    in the texts on a \se{L}ikert scale of 0-100, and then `optimize' the threshold of the rates for the `emotional' category on 70\% of the data and check how it performs on the remaining 30\%. Overall, with this step, we can improve the macro F1 to 0.74-0.81 (averaged over three rounds of data splitting), depending on the gold from different annotators. The best threshold for two annotators is 75, while that for the other is 85, so we use the threshold 75 to represent the majority, which has a macro F1 of 0.75, averaged across the three annotators. 

    We use this threshold to select the argument pairs for \hansard{}. For \deuparl{}, we further optimize the threshold using a small-scale set of human annotations and adjust it to 85. This setting is then used to label the binary emotions of arguments. 

\end{enumerate}

\begin{table}[!ht]
\resizebox{\linewidth}{!}{%
\begin{tabular}{@{}lcccc@{}}
\toprule
                               & \multicolumn{2}{c}{Pre-Annotation} & \multicolumn{2}{c}{Automatic Pipeline}   
                               \\
                               & \#                 & \%  & Question  & `Optimized' \\ \midrule
\multicolumn{3}{l}{\emph{L1 - emotion} }                                             \\ \midrule
Emotional                      & 151                & 53.0 & 0.53 (P)  & \multirow{2}{*}{0.75 (F1)} \\ 
Non-emotional                  & 134                & 47.0 & -  \\ \bottomrule
\multicolumn{3}{l}{\emph{L2 - argument}}                                         \\ \midrule
Argumentative                  & 234                & 82.1 & 0.82 (P) & 0.96 (P)  \\
Non-argumentative              & 51                 & 17.9 & - & - \\ \bottomrule
\multicolumn{3}{l}{\emph{L3 - stance} }                                           \\ \midrule    
Support                        & 170                & 72.6 & - \\
Opposition                     & 2                  & 0.9 & -  \\
Neutral                        & 29                 & 12.4  & -\\
Irrelevant                     & 16                 & 14.1& - \\ \midrule
\multicolumn{3}{l}{\emph{L3$^{*}$ - pair stance}}                                            \\ \midrule   
Same           & 2,905            & 8.9  & -   & \multirow{3}{*}{\makecell{0.80 ($P_{same}$) \\ 0.75 (F1)}} \\
Different stance & 3,325              & 10.2 & - \\
Different topic                & 26,486             & 81.0 & - \\ \midrule
Total                          & 32,716             & 100  & - \\ \bottomrule
\end{tabular}}
\caption{Number of texts annotated for each layer and category (\#) and the corresponding label distribution (\%). Performance of GPT4o on the binary emotion classification, argument identification, and stance agreement detection tasks used for automatically identifying the target argument pairs.}\label{tab:pre}
\end{table}


\begin{table*}[!ht]
\resizebox{\linewidth}{!}{
\begin{tabular}{@{}l@{}}
\toprule
\emph{Introduction}                                                                                                                                                                         \\ \midrule
A Bill to Prohibit the export of certain livestock from Great Britain for slaughter.                                                                                                 \\ \midrule
\makecell[l]{A Bill to create offences of dog abduction and cat abduction and to confer a power to make corresponding provision  \\ relating to the abduction of other animals commonly kept as pets.} \\ \midrule
A Bill to make provision about leave and pay for employees with responsibility for children receiving neonatal care.                                                                   \\ \midrule
A Bill to prohibit the import and export of shark fins and to make provision relating to the removal of fins from sharks.                                                            \\ \midrule
A Bill to prohibit the sale and advertising of activities abroad which involve low standards of welfare for animals.                                                                 \\ \bottomrule
\end{tabular}}
\caption{The introductions of the five Bills selected in \protect\bill{}.}\label{tab:bill}
\end{table*}


% Please add the following required packages to your document preamble:
% \usepackage{booktabs}
\begin{table*}[]
\resizebox{\linewidth}{!}{
\begin{tabular}{@{}l|l@{}}
\toprule
English                                                                                                                                                                                  & German                                                                                                                                                                                     \\ \midrule
\makecell[l]{iran, integrat, ukraine, russia, asylum,\\ deportation, israel, gaza, expulsion, \\ displacement, migration, migrant, \\immigrant, refugee, palestine,invasion,\\ repatriation, hamas, hisbollah} & \makecell[l]{ukraine, russland, migrant, \\ immigrant, flüchtling, asyl,\\ gaza, iran, palästina, \\israel, krieg, invasion, \\sanktionen, waffenlieferungen, friedensverhandlungen, \\kriegsverbrechen, flüchtlingskrise, nato,\\ energieversorgung, vertreibung, migrationspolitik,\\ asylverfahren, grenzsicherung, integration, \\abschiebung, aufenthaltsgenehmigung, menschenhandel, \\seenotrettung, rückführung, schutzstatus, \\waffenstillstand, raketenangriffe, besatzung, \\zwei-staaten-lösung, friedensprozess, intifada, \\ hamas, hisbollah, menschenrechte, un-resolution
} \\ \bottomrule
\end{tabular}
}
\caption{Keywords used to filter debates for \hansard{} and \deuparl{}.}\label{tab:keywords}
\end{table*}



% Please add the following required packages to your document preamble:
% \usepackage{booktabs}
\begin{table*}[]
\centering
%\resizebox{!}{\linewidth}{
\begin{tabularx}{\linewidth}{X}
\toprule
\emph{Remove Emotion Prompt}  \\ \midrule
====\textbf{System Prompt}=====\\ I will give you an argumentative text that **can** appeal to emotion.    \\ \\ Your task is to generate an argument with the same stance for the same topic **without emotional language**, by rephrasing the text but maintaining a similar style and length. \\ \\ Briefly explain why the rewritten argument no longer evokes emotions.\\ \\ Answer in the following way:\\ Generated argument: \\ Explanation:\\ ====\textbf{User Prompt}=====\\ Text: \{original argument\}  \\ \midrule
\emph{Add Emotion Prompt}  \\ \midrule
====\textbf{System Prompt}=====\\ I will give you an argumentative text that **cannot** appeal to emotion.\\     \\ Your task is to generate an argument with the same stance on the same topic **with emotions**, by rephrasing the text but maintaining a similar style and length. \\ \\ Briefly explain why the rewritten argument can evoke emotions now.\\ \\ Answer in the following way:\\ Generated argument: \\ Explanation:\\ ====\textbf{User Prompt}=====\\ Text: \{original argument\}                \\ \bottomrule
\end{tabularx}
%}
\caption{Prompts used to remove/add emotions for synthetic arguments.}\label{tab:prompt_synthetic}
\end{table*}


\section{Arguments from others}\label{app:other}
\paragraph{\dagstuhl{}} 
\citet{wachsmuth-etal-2017-computational} collected human ratings on a Likert scale of 1–3 for multiple dimensions of argument quality, including argument effectiveness (convincingness)\footnote{“Argumentation is effective if it persuades the target audience of (or corroborates agreement with) the author’s stance on the issue.” — \citet{wachsmuth-etal-2017-computational}} and emotional appeal. These ratings were applied to 304 argumentative texts from \citet{habernal-gurevych-2016-argument}, which were sourced from a textual debate portal in \textbf{English}. We retain only those arguments whose average convincingness rating (across the three annotators) exceeds 1.5. 
Next, we pair arguments that share the same stance on the same topics and calculate the absolute differences in their emotional appeal ratings. From these pairs, we randomly select 10 topics and then retain the 5 argument pairs with the largest absolute differences in emotional appeal for each topic.

\paragraph{\lynn{}}
\citet{greschner2024fearfulfalconsangryllamas} collected
discrete emotion labels from a reader respective (e.g. joy, disgust etc.) for 300 \textbf{German} arguments associated with 30 statements, drawn from \citet{velutharambath_wuehrl_klinger_2024a}. Each argument was annotated by three annotators. We interpret the number of annotations marking the argument as containing specific emotions (rather than `no emotion') as its emotion score. E.g., if three annotators identify specific emotions in the argument, its emotion score would be 3. Using a procedure similar to the one employed for \dagstuhl{}, we pair arguments referencing the same statement, randomly select 25 statements, and then retain the 
two argument pairs per statement that exhibit the greatest differences in emotion scores. 


\begin{table}[]
\centering
\resizebox{\linewidth}{!}{
\begin{tabular}{@{}llllll@{}}
\toprule
     & \dagstuhl{} & \bill{} & \hansard{} & \lynn{} & \deuparl{} \\ \midrule
\multicolumn{6}{l}{\emph{Increase}}                               \\ \midrule
\rz{}   & \textbf{-0.06}   & \textbf{0.15}         & \textbf{0.05}   & \textbf{-0.38}  & \textbf{0.32}   \\
\rth{} & -0.18    & -0.21        & -0.31  & -0.46   & -0.38  \\ \midrule
\multicolumn{6}{l}{\emph{Decrease}    }                        \\ \midrule
\ro{} & -0.12 & -0.21 & -0.03 & 0.08 & -0.19    \\
\rtw{} & \textbf{0.36}    & \textbf{0.27}         & \textbf{0.29}   & \textbf{0.76}   & \textbf{0.25} \\ \bottomrule
\end{tabular}}
\caption{BWS scores for the 4 argument groups: \rz{}, \ro{}, \rtw{} and \rth{}, derived from the majority votes of the annotation for pairwise comparisons of emotional intensity. `Increase'/`Decrease' denotes the direction to increase/decrease the perceived emotional intensity.}\label{tab:bws}
\vspace{-.3cm}
\end{table}

\section{Prompts}\label{app:prompt}
Table \ref{tab:prompt_synthetic} presents the prompts used to introduce/remove emotions. Table \ref{tab:promptc} illustrates the prompts used for evaluating argument convincingness.

% Please add the following required packages to your document preamble:
% \usepackage{booktabs}
\begin{table*}
\footnotesize
\resizebox{\linewidth}{!}{
\begin{tabular}{@{}cl@{}}
\toprule
\multicolumn{2}{l}{Prompt Template}                                                                                                                                                                                                                                                                                                                                                                                                                                                                                                                                                                                                                                                                                     \\ \midrule
Shared & \begin{tabular}[c]{@{}l@{}}Below, you will find one pair of argumentative texts discussing the same topic with the same stance. The topic may be a binary \\choice, a bill from UK parliamentary debates, or a simple statement. Both arguments either support or oppose the topic, or they \\favor one side if the topic involves a binary choice.\\ \\ Your task is to evaluate each pair to determine **which argumentative text you find more convincing**. There are three label options:\\ 0 (Both arguments are equally convincing.)\\ 1 (Argument 1 is more convincing.)\\ 2 (Argument 2 is more convincing.)\\ \\ **Note**: Truncated sentences or grammatical errors should be **ignored**.\end{tabular} \\ \midrule
1      & \begin{tabular}[c]{@{}l@{}}Please answer your label option **without** any explanations.\\ \\ \{text\}\end{tabular}                                                                                                                                                                                                                                                                                                                                                                                                                                                                                                                                                                                            \\ \midrule
2      & \begin{tabular}[c]{@{}l@{}}Please answer your label option and briefly explain why you choose this label.\\ \\ \{text\}\\ \\ Below is an example answer for you; please follow this format in your response.\\ Label: 2\\ Explanation: because Argument 2 provides more statistics supporting the claim, while Argument 1 contains logical fallacies.\end{tabular}                                                                                                                                                                                                                                                                                                                                             \\ \midrule
3      & \begin{tabular}[c]{@{}l@{}}Please answer your label option and briefly explain why you choose this label.\\ \\ \{text\}\\ \\Below is an example answer for you; please follow this format in your response.\\ Label: 1\\ Explanation: Argument 1 is more convincing, because I totally agree with its point and it evokes my empathy.\end{tabular}                                                                                                                                                                                                                                                                                                                                                                                                                                                            \\ \bottomrule
\end{tabular}}
\caption{Prompt templates for comparing the convincingness of an argument pair. The {text} field contains the two arguments and their topic. The complete prompt is formed by combining the text in the `Shared' row with the text in the corresponding indexed row. For example, Prompt 1 consists of the text from both the `Shared' row and row `1'.}\label{tab:promptc}
\end{table*}



\begin{table}[]
\vspace{-.2cm}
\centering
\setlength\tabcolsep{2pt} 
\resizebox{\linewidth}{!}{%
\begin{tabular}{@{}lcc|cccccc@{}}
\toprule
& \multicolumn{2}{c|}{\textbf{\#Annotators}} & \multicolumn{6}{c}{\textbf{Agreements}} \\
              & \textbf{S} &  \textbf{C} & \multicolumn{3}{c}{\textbf{EMO}} & \multicolumn{3}{c}{\textbf{CONV}} \\ %\midrule
              &&& $\alpha$ & Full & Maj. & $\alpha$ & Full & Maj. \\ \midrule
\dagstuhl{}      & 1        & 4         &  0.506  & 6.5\% & 74.5\%   & 0.540  & 14.0\% & 80.0\%   \\
\bill{} & 1        & 4        &  0.449  & 7.0\% & 76.5\%   & 0.463  & 10.5\% & 78.0\%   \\
\hansard{}       & 1        & 4        &  0.361 & 0.5\% & 68.0\%   & 0.371  & 6.0\% & 75.0\%   \\
\lynn{}       & 2        & 3       &  0.729 & 13.5\% & 87.5\%   & 0.607  & 16.0\% & 82.0\%   \\
\deuparl{}       & 3        & 2       & 0.352  & 8.0\% & 80.5\%   & 0.364  & 4.5\% & 74.5\%   \\ \midrule
Avg    & -        & - & 0.479 & 7.1\% & 77.4\% & 0.469 & 10.2\% & 77.9\% \\ 
\bottomrule
\end{tabular}}
\caption{\textbf{Left}: Number of student (S) and crowdsourcing (C) annotators per batch. \textbf{Right}: Krippendorf's $\alpha$ for the most agreeing annotator pairs (\textbf{$\alpha$}), the percentages of annotation instances where all annotators agree on a certain label (\textbf{Full}),  and the percentage of annotation instances where at least three annotators agree on a certain label (\textbf{Maj.}). 
}
\label{tab:annotator}
\vspace{-.6cm}
\end{table}


\section{Annotation Interface}\label{app:anno}
Figure \ref{fig:anno} shows the screenshots of the annotation interface for convincingness (top) and emotion (bottom) comparisons. We collect the annotations via Google Forms\footnote{\url{https://docs.google.com/forms/}} for crowdsourcing annotators.

\begin{figure*}
    \includegraphics[width=\linewidth]{structure/figs/anno/conv_form.pdf}
    \includegraphics[width=\linewidth]{structure/figs/anno/emo_form.pdf}
    \caption{Screenshots of the annotation interface for convincingness (top) and emotion (bottom) comparison.}\label{fig:anno}
\end{figure*}


\section{Examples}\label{app:exa}
Table \ref{tab:pos} and \ref{tab:neg} provide example instances from \hansard{} and \lynn{}, where emotions have a positive and negative impact, respectively. 

\begin{table*}[!ht]
\centering
\begin{tabularx}{\textwidth}{ X | X }
\toprule
\multicolumn{2}{l}{\textbf{Topic}: The public supports the UK's aid for Ukrainian refugees} \\ \midrule
\rz{}  & \ro{}  \\ \midrule
Members across this House are determined that we, as a country, should open our arms to these people, and this determination has been on full display today. The scenes of devastation and human misery inflicted by President Putin’s barbarous assault on what he calls “Russia’s cousins” in Ukraine have unleashed a tidal wave of solidarity and generosity across the country. British people always step forward and step up in these moments, and since the first tanks rolled into Ukraine, they have come forward in droves with offers of help: community centres have been flooded with critical supplies; the Association of Ukrainians in Great Britain has received millions in donations; and charities such as the Red Cross have been overwhelmed with people giving whatever they can. The outpouring of public support has been nothing short of remarkable. & While this Government, and this whole House, have risen to the occasion with our offer of support to Ukrainians fleeing war, our lethal aid and our stranglehold on economic sanctions on Russia have clearly shown that we will keep upping the ante to ensure that Putin fails. As Members have argued today, it has been abundantly clear in recent days that we can and must do more. It is exactly right, therefore, that my right hon. Friend the Secretary of State for Levelling Up, Housing and Communities set out on Monday the new and uncapped sponsorship scheme, Homes for Ukraine. It is a scheme to allow Ukrainians with no family ties to the UK to be sponsored by individuals or organisations that can offer them a home. It is a scheme that draws not only on the exceptional good will and generosity of the British people, but one that gives them the opportunity to help make a difference.                                                                                                                                                        \\ \midrule
\rth{}   & \rtw{} \\ \midrule
Members of this House have expressed a commitment to welcoming individuals from Ukraine. The recent conflict initiated by President Putin has resulted in significant destruction in Ukraine, prompting a substantial response of support across the country. British citizens have actively contributed since the conflict began, with community centers collecting essential supplies, the Association of Ukrainians in Great Britain receiving financial contributions, and charities like the Red Cross witnessing increased donations.  & In these trying times, the Government and this entire House have demonstrated unwavering courage and compassion by extending our support to Ukrainians escaping the horrors of war. Our determined provision of lethal aid and the relentless imposition of economic sanctions on Russia are powerful affirmations that we will stop at nothing to ensure Putin's defeat. As Members have passionately discussed today, the urgency to do even more has never been clearer. That is why it is so heartening that my right hon. Friend the Secretary of State for Levelling Up, Housing and Communities announced on Monday the new and limitless Homes for Ukraine sponsorship scheme. This initiative opens its arms to Ukrainians without family connections in the UK, allowing them to be warmly embraced by individuals or organizations ready to offer them a sanctuary. It is a testament not only to the extraordinary kindness and generosity of the British people but also to their deep desire to make a meaningful impact in the lives of those in desperate need. \\ \bottomrule
\end{tabularx}
\caption{An example instance from \hansard{} where emotions have a \textbf{positive} impact on argument convincingness.
}\label{tab:pos}
\end{table*}

\begin{table*}[!ht]
\centering
\begin{tabularx}{\textwidth}{ X | X }
\toprule
\multicolumn{2}{l}{Topic: Haie können Krebs bekommen.}    \\ \midrule
\rz{}  & \ro{}  \\  \midrule
Haie sind mehrzellige Lebewesen, wie auch der Mensch. Die Beonderheit von mehrzelligen Lebewesen ist, dass die Zellen sich sowohl stark spezialisieren und untereinander vernetz kommunizieren. Damit werden sie anfällig für bestimmte Zelldefekte, die sich über die genannte Struktur fortpflanzen und den Krebs ausmachen. Haie verfügen, wie auch der Mensch und überhaupt alle mehrzelligen Lebewesen, über nur eine sehr eingeschränkte Möglichkeit diese Defekte zu korrigieren und aufzuhalten, damit können beide gleichermaßen Krebs bekommen & Da auch Fische Krebs bekommen können, ist es auch möglich, dass Haie Krebs bekommen können. Dieser wird durch mutierte Zellen ausgelöst, weshalb dies auch bei Fischarten ausgelöst werden kann. Krebs ist eine weit verbreitete und häufige Krankheit, weshalb Krebs durch Wissenschaftler auch bereits bei Haien festgestellt werden konnte.\\ Krebs kann außerdem auch durch verschiedene Umweltfaktoren wie Umweltverschmutzung ausgelöst werden, diesem Risiko sind Haie ja durchaus ausgesetzt. Deshalb ist die Gefahr einer Erkrankung auch nicht gerade gering.  \\ \midrule
\rth{}  & \rtw{}  \\ \midrule
Haie, ebenso wie Menschen, sind mehrzellige Organismen. Eine charakteristische Eigenschaft solcher Organismen ist die Spezialisierung und Vernetzung ihrer Zellen. Diese Struktur macht sie anfällig für Zellfehler, die sich ausbreiten und zu Krebs führen können. Haie und Menschen besitzen nur begrenzte Mechanismen zur Korrektur und Kontrolle dieser Defekte, was bedeutet, dass beide Arten gleichermaßen anfällig für Krebs sind.  & Die Vorstellung, dass Haie - diese majestätischen und oft missverstandenen Kreaturen der Meere - an Krebs erkranken können, ist zutiefst beunruhigend. Diese Krankheit, die durch die heimtückische Mutation von Zellen verursacht wird, hat bereits viele Fischarten heimgesucht. Die Tatsache, dass auch Haie, die Könige der Ozeane, nicht sicher vor dieser grausamen Krankheit sind, ist erschütternd. Angesichts der weit verbreiteten Umweltverschmutzung, die unsere Ozeane verschlingt, sind Haie einem erheblichen Risiko ausgesetzt, an Krebs zu erkranken. Es ist traurig und alarmierend, dass diese beeindruckenden Tiere, die seit Millionen von Jahren die Meere durchstreifen, nun durch menschliche Einflüsse bedroht sind.
\\ \bottomrule
\end{tabularx}
\caption{An example instance from \lynn{} where emotions have a \textbf{negative} impact on argument convincingness.
}\label{tab:neg}
\end{table*}




\section{LLM}\label{app:llm}
Figure \ref{fig:dis_prompt} illustrates the consistency, positivity and negativity rates of LLMs with different prompts, averaged across instances in all datasets. Table \ref{tab:llm_ranking} displays macro F1 scores and model rankings for LLMs in predicting convincingness rankings of argument pairs ('Static') and the resulting categories of emotional effect (`Dynamic') in English and German.

\begin{figure*}[]
    \centering
    \includegraphics[width=\linewidth]{structure/figs/llm/dis_prompt1.pdf}
    \includegraphics[width=\linewidth]{structure/figs/llm/dis_prompt2.pdf}
    \includegraphics[width=\linewidth]{structure/figs/llm/dis_prompt3.pdf}
    \caption{Consistency, positivity and negativity rates of LLMs with different prompts, averaged across instances in all datasets.}\label{fig:dis_prompt}
\end{figure*}


\begin{table*}[]
\resizebox{\linewidth}{!}{
\begin{tabular}{lcccc|cccc}
\toprule
                           & \multicolumn{4}{c|}{\textbf{EN}}               & \multicolumn{4}{c}{\textbf{DE}}               \\ 
\textbf{Model}                      & Static & Ranking & Dynamic & Ranking & Static & Ranking & Dynamic & Ranking \\ \midrule
gpt-4o-2024-08-06          & \textbf{0.486}  & 1       & 0.411   & 2       & \textbf{0.443}  & 1       & \textbf{0.447}   & 1       \\
Llama-3.3-70B-Instruct     & 0.417  & 2       & \textbf{0.415}   & 1       & 0.372  & 2       & 0.392   & 4       \\
gpt-4o-mini                & 0.416  & 3       & 0.392   & 5       & 0.35   & 4       & 0.394   & 3       \\
Qwen2.5-72B-Instruct       & 0.398  & 4       & 0.398   & 4       & 0.357  & 3       & 0.41    & 2       \\
gpt-3.5-turbo              & 0.39   & 5       & 0.382   & 6       & 0.338  & 6       & 0.381   & 6       \\
Mixtral-8x7B-Instruct-v0.1 & 0.368  & 6       & 0.376   & 7       & 0.35   & 5       & 0.387   & 5       \\
Mistral-7B-Instruct-v0.3   & 0.367  & 7       & 0.407   & 3       & 0.288  & 8       & 0.36    & 9       \\
Llama-3.2-3B-Instruct      & 0.322  & 8       & 0.32    & 10      & 0.281  & 10      & 0.367   & 8       \\
Qwen2.5-0.5B-Instruct      & 0.308  & 9       & 0.342   & 9       & 0.284  & 9       & 0.344   & 10      \\
Qwen2.5-7B-Instruct        & 0.304  & 10      & 0.346   & 8       & 0.319  & 7       & 0.373   & 7       \\
Llama-3.2-1B-Instruct      & 0.286  & 11      & 0.309   & 11      & 0.274  & 11      & 0.343   & 11 \\ \bottomrule     
\end{tabular}}
\caption{Macro F1 scores and model rankings for LLMs in predicting convincingness rankings of argument pairs ('Static') and the resulting categories of emotional effect (`Dynamic') in English and German. For each model, we present the best prompt result to highlight its potential. Human and LLM labels are determined by majority votes from different annotators and rounds, respectively.}\label{tab:llm_ranking}
\end{table*}



\end{document}
