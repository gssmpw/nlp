
\documentclass[journal]{IEEEtran}

\usepackage{booktabs}
\usepackage{multirow}
\usepackage{arydshln}
\usepackage{makecell}
\usepackage{hyperref}
\usepackage{float}

\usepackage{cite}

\usepackage{graphicx}

\hyphenation{op-tical net-works semi-conduc-tor}

\begin{document}

\title{Multi-Record Web Page Information Extraction From News Websites}

\author{
\IEEEauthorblockN{Alexander Kustenkov$^{1,2}$, Maksim Varlamov$^{1}$, and Alexander Yatskov$^{1,2}$}\\
\IEEEauthorblockA{$^1$\textit{Ivannikov Institute for System Programming of the Russian Academy of Sciences, Moscow, Russia}}\\
\IEEEauthorblockA{$^2$\textit{Lomonosov Moscow State University, Moscow, Russia}\\
\{kustenkov, varlamov, yatskov\}@ispras.ru}
}
        

\maketitle

\begin{abstract}


The choice of representation for geographic location significantly impacts the accuracy of models for a broad range of geospatial tasks, including fine-grained species classification, population density estimation, and biome classification. Recent works like SatCLIP and GeoCLIP learn such representations by contrastively aligning geolocation with co-located images. While these methods work exceptionally well, in this paper, we posit that the current training strategies fail to fully capture the important visual features. We provide an information theoretic perspective on why the resulting embeddings from these methods discard crucial visual information that is important for many downstream tasks. To solve this problem, we propose a novel retrieval-augmented strategy called RANGE. We build our method on the intuition that the visual features of a location can be estimated by combining the visual features from multiple similar-looking locations. We evaluate our method across a wide variety of tasks. Our results show that RANGE outperforms the existing state-of-the-art models with significant margins in most tasks. We show gains of up to 13.1\% on classification tasks and 0.145 $R^2$ on regression tasks. All our code and models will be made available at: \href{https://github.com/mvrl/RANGE}{https://github.com/mvrl/RANGE}.

\end{abstract}



\begin{IEEEkeywords}
Data collection, Data extraction, Dataset, Multi-record extraction, Neural network
\end{IEEEkeywords}

\IEEEpeerreviewmaketitle

\def\ItIsIEEEDoc{1} % This is Kustenkov's define 
\section{Introduction}

Video generation has garnered significant attention owing to its transformative potential across a wide range of applications, such media content creation~\citep{polyak2024movie}, advertising~\citep{zhang2024virbo,bacher2021advert}, video games~\citep{yang2024playable,valevski2024diffusion, oasis2024}, and world model simulators~\citep{ha2018world, videoworldsimulators2024, agarwal2025cosmos}. Benefiting from advanced generative algorithms~\citep{goodfellow2014generative, ho2020denoising, liu2023flow, lipman2023flow}, scalable model architectures~\citep{vaswani2017attention, peebles2023scalable}, vast amounts of internet-sourced data~\citep{chen2024panda, nan2024openvid, ju2024miradata}, and ongoing expansion of computing capabilities~\citep{nvidia2022h100, nvidia2023dgxgh200, nvidia2024h200nvl}, remarkable advancements have been achieved in the field of video generation~\citep{ho2022video, ho2022imagen, singer2023makeavideo, blattmann2023align, videoworldsimulators2024, kuaishou2024klingai, yang2024cogvideox, jin2024pyramidal, polyak2024movie, kong2024hunyuanvideo, ji2024prompt}.


In this work, we present \textbf{\ours}, a family of rectified flow~\citep{lipman2023flow, liu2023flow} transformer models designed for joint image and video generation, establishing a pathway toward industry-grade performance. This report centers on four key components: data curation, model architecture design, flow formulation, and training infrastructure optimization—each rigorously refined to meet the demands of high-quality, large-scale video generation.


\begin{figure}[ht]
    \centering
    \begin{subfigure}[b]{0.82\linewidth}
        \centering
        \includegraphics[width=\linewidth]{figures/t2i_1024.pdf}
        \caption{Text-to-Image Samples}\label{fig:main-demo-t2i}
    \end{subfigure}
    \vfill
    \begin{subfigure}[b]{0.82\linewidth}
        \centering
        \includegraphics[width=\linewidth]{figures/t2v_samples.pdf}
        \caption{Text-to-Video Samples}\label{fig:main-demo-t2v}
    \end{subfigure}
\caption{\textbf{Generated samples from \ours.} Key components are highlighted in \textcolor{red}{\textbf{RED}}.}\label{fig:main-demo}
\end{figure}


First, we present a comprehensive data processing pipeline designed to construct large-scale, high-quality image and video-text datasets. The pipeline integrates multiple advanced techniques, including video and image filtering based on aesthetic scores, OCR-driven content analysis, and subjective evaluations, to ensure exceptional visual and contextual quality. Furthermore, we employ multimodal large language models~(MLLMs)~\citep{yuan2025tarsier2} to generate dense and contextually aligned captions, which are subsequently refined using an additional large language model~(LLM)~\citep{yang2024qwen2} to enhance their accuracy, fluency, and descriptive richness. As a result, we have curated a robust training dataset comprising approximately 36M video-text pairs and 160M image-text pairs, which are proven sufficient for training industry-level generative models.

Secondly, we take a pioneering step by applying rectified flow formulation~\citep{lipman2023flow} for joint image and video generation, implemented through the \ours model family, which comprises Transformer architectures with 2B and 8B parameters. At its core, the \ours framework employs a 3D joint image-video variational autoencoder (VAE) to compress image and video inputs into a shared latent space, facilitating unified representation. This shared latent space is coupled with a full-attention~\citep{vaswani2017attention} mechanism, enabling seamless joint training of image and video. This architecture delivers high-quality, coherent outputs across both images and videos, establishing a unified framework for visual generation tasks.


Furthermore, to support the training of \ours at scale, we have developed a robust infrastructure tailored for large-scale model training. Our approach incorporates advanced parallelism strategies~\citep{jacobs2023deepspeed, pytorch_fsdp} to manage memory efficiently during long-context training. Additionally, we employ ByteCheckpoint~\citep{wan2024bytecheckpoint} for high-performance checkpointing and integrate fault-tolerant mechanisms from MegaScale~\citep{jiang2024megascale} to ensure stability and scalability across large GPU clusters. These optimizations enable \ours to handle the computational and data challenges of generative modeling with exceptional efficiency and reliability.


We evaluate \ours on both text-to-image and text-to-video benchmarks to highlight its competitive advantages. For text-to-image generation, \ours-T2I demonstrates strong performance across multiple benchmarks, including T2I-CompBench~\citep{huang2023t2i-compbench}, GenEval~\citep{ghosh2024geneval}, and DPG-Bench~\citep{hu2024ella_dbgbench}, excelling in both visual quality and text-image alignment. In text-to-video benchmarks, \ours-T2V achieves state-of-the-art performance on the UCF-101~\citep{ucf101} zero-shot generation task. Additionally, \ours-T2V attains an impressive score of \textbf{84.85} on VBench~\citep{huang2024vbench}, securing the top position on the leaderboard (as of 2025-01-25) and surpassing several leading commercial text-to-video models. Qualitative results, illustrated in \Cref{fig:main-demo}, further demonstrate the superior quality of the generated media samples. These findings underscore \ours's effectiveness in multi-modal generation and its potential as a high-performing solution for both research and commercial applications.

 

\section{Related Work}

\subsection{Large 3D Reconstruction Models}
Recently, generalized feed-forward models for 3D reconstruction from sparse input views have garnered considerable attention due to their applicability in heavily under-constrained scenarios. The Large Reconstruction Model (LRM)~\cite{hong2023lrm} uses a transformer-based encoder-decoder pipeline to infer a NeRF reconstruction from just a single image. Newer iterations have shifted the focus towards generating 3D Gaussian representations from four input images~\cite{tang2025lgm, xu2024grm, zhang2025gslrm, charatan2024pixelsplat, chen2025mvsplat, liu2025mvsgaussian}, showing remarkable novel view synthesis results. The paradigm of transformer-based sparse 3D reconstruction has also successfully been applied to lifting monocular videos to 4D~\cite{ren2024l4gm}. \\
Yet, none of the existing works in the domain have studied the use-case of inferring \textit{animatable} 3D representations from sparse input images, which is the focus of our work. To this end, we build on top of the Large Gaussian Reconstruction Model (GRM)~\cite{xu2024grm}.

\subsection{3D-aware Portrait Animation}
A different line of work focuses on animating portraits in a 3D-aware manner.
MegaPortraits~\cite{drobyshev2022megaportraits} builds a 3D Volume given a source and driving image, and renders the animated source actor via orthographic projection with subsequent 2D neural rendering.
3D morphable models (3DMMs)~\cite{blanz19993dmm} are extensively used to obtain more interpretable control over the portrait animation. For example, StyleRig~\cite{tewari2020stylerig} demonstrates how a 3DMM can be used to control the data generated from a pre-trained StyleGAN~\cite{karras2019stylegan} network. ROME~\cite{khakhulin2022rome} predicts vertex offsets and texture of a FLAME~\cite{li2017flame} mesh from the input image.
A TriPlane representation is inferred and animated via FLAME~\cite{li2017flame} in multiple methods like Portrait4D~\cite{deng2024portrait4d}, Portrait4D-v2~\cite{deng2024portrait4dv2}, and GPAvatar~\cite{chu2024gpavatar}.
Others, such as VOODOO 3D~\cite{tran2024voodoo3d} and VOODOO XP~\cite{tran2024voodooxp}, learn their own expression encoder to drive the source person in a more detailed manner. \\
All of the aforementioned methods require nothing more than a single image of a person to animate it. This allows them to train on large monocular video datasets to infer a very generic motion prior that even translates to paintings or cartoon characters. However, due to their task formulation, these methods mostly focus on image synthesis from a frontal camera, often trading 3D consistency for better image quality by using 2D screen-space neural renderers. In contrast, our work aims to produce a truthful and complete 3D avatar representation from the input images that can be viewed from any angle.  

\subsection{Photo-realistic 3D Face Models}
The increasing availability of large-scale multi-view face datasets~\cite{kirschstein2023nersemble, ava256, pan2024renderme360, yang2020facescape} has enabled building photo-realistic 3D face models that learn a detailed prior over both geometry and appearance of human faces. HeadNeRF~\cite{hong2022headnerf} conditions a Neural Radiance Field (NeRF)~\cite{mildenhall2021nerf} on identity, expression, albedo, and illumination codes. VRMM~\cite{yang2024vrmm} builds a high-quality and relightable 3D face model using volumetric primitives~\cite{lombardi2021mvp}. One2Avatar~\cite{yu2024one2avatar} extends a 3DMM by anchoring a radiance field to its surface. More recently, GPHM~\cite{xu2025gphm} and HeadGAP~\cite{zheng2024headgap} have adopted 3D Gaussians to build a photo-realistic 3D face model. \\
Photo-realistic 3D face models learn a powerful prior over human facial appearance and geometry, which can be fitted to a single or multiple images of a person, effectively inferring a 3D head avatar. However, the fitting procedure itself is non-trivial and often requires expensive test-time optimization, impeding casual use-cases on consumer-grade devices. While this limitation may be circumvented by learning a generalized encoder that maps images into the 3D face model's latent space, another fundamental limitation remains. Even with more multi-view face datasets being published, the number of available training subjects rarely exceeds the thousands, making it hard to truly learn the full distibution of human facial appearance. Instead, our approach avoids generalizing over the identity axis by conditioning on some images of a person, and only generalizes over the expression axis for which plenty of data is available. 

A similar motivation has inspired recent work on codec avatars where a generalized network infers an animatable 3D representation given a registered mesh of a person~\cite{cao2022authentic, li2024uravatar}.
The resulting avatars exhibit excellent quality at the cost of several minutes of video capture per subject and expensive test-time optimization.
For example, URAvatar~\cite{li2024uravatar} finetunes their network on the given video recording for 3 hours on 8 A100 GPUs, making inference on consumer-grade devices impossible. In contrast, our approach directly regresses the final 3D head avatar from just four input images without the need for expensive test-time fine-tuning.


\section{Problem Formulation}
In this paper, the recommendation task takes the user behavior data as input. 
Let $\mathcal{U}$ and $\mathcal{V}$ be sets of users and items respectively, where $|\mathcal{U}|= m$, and $|\mathcal{V}|= n$. We use the index $u \in \mathcal{U}$ to denote a user and $i\in \mathcal{V}$ to denote an item. 
The user-item rating matrix is denoted as $\mathbf{R} = [r_i^u]^{m\times n}\in \mathbb{R}^{m\times n}$ to indicate whether user $u$ has interacted with item $i$, where $r_i^u=1$ represents user $u$ has interacted with item $i$,  whereas $r_i^u=0$ represents user $u$ has not interacted with item $i$. 
We use $\mathcal{V}^{+}_{u}=\{i\in\mathcal{V}|r_i^u=1\}$ to represent a set of items that user $u$ has interacted with.
$\mathcal{V}^{+}_{u}$ can be splited into a training set $\mathcal{S}_{u}^{+}$ and a testing set $\mathcal{T}_{u}$, requiring that $\mathcal{S}_{u}^{+} \cup \mathcal{T}_{u} = \mathcal{V}^{+}_{u}$ and $\mathcal{S}_{u}^{+} \cap \mathcal{T}_{u} = \emptyset$. It worth noted that $\mathcal{S}_u^{-}=\{i|r_i^u=0,i\in \mathcal{I}\}$, which means $\mathcal{S}_u^{-}$ consists of the negative items that user $u$ have not interacted with. The training set is denoted as $\mathcal{D}=\{(u,i, j)|u\in \mathcal{U}, i \in \mathcal{S}_{u}^{+}, j \in \mathcal{S}_{u}^{-}\}$. The testing set is denoted as $\mathcal{\hat{D}}=\{(u,i)|u\in \mathcal{U}, i \in \mathcal{T}_{u}\}$.

In the recommendation task, the model aims to recommend a list of $k$ items $\mathcal{X}_u$ for the user $u$, which matches the condition $\mathcal{X}_u\cap \mathcal{S}_u^{+}=\emptyset$. 
By comparing the recommendation list $\mathcal{X}_u$ with the testing set $\mathcal{T}_u$, we evaluate the recommendation quality from various perspectives, including accuracy, diversity, and fairness. 

\section{Dataset construction}
We decided to collect our own dataset for the task of extraction information from multi-record web-pages and make it publicly available. In this chapter we describe how the data was collected and annotated, also we provide the final dataset's main characteristics.

\subsection{Data collecting}
Our dataset contains news web pages collected from Russian-language media. News resources were selected according to the MediaMetrics\footnote{https://mediametrics.ru/rating/ru/online.html} latest news quotations system, which provides rankings based on the popularity of news resources. Web pages were downloaded between 12/28/2023 and 04/28/2024. Pages were downloaded using a special Python3 script (based on the Scrapy library\cite{scrapy_library}), which was run daily through prepared sitemaps.

\subsection{Sitemaps development}
Sitemaps were prepared using the WebScraper\footnote{https://github.com/ispras/web-scraper-chrome-extension} browser extension for Chrome. Using sitemaps in special web crawlers allows to download an html page, as well as an answer for this page. Thus, each site requires the development of a unique sitemap. On each page, the boundaries of each record and its attributes were annotated, if they existed. Only the following attributes were noted: \textit{date}, \textit{title}, \textit{tag}, \textit{short\_title}, \textit{author}, \textit{time}. The annotation was done manually.
We chose this set of attributes because they are the most popular in the news websites domains. 
For each site several categories were annotated, for example, politics, economics, and sports. In this way 312 sitemaps were prepared.

\subsection{Dataset preparing}
For further use of data in our dataset, it was necessary to preprocess the raw data. The following actions were carried out:
\begin{enumerate}
    \item Filtering duplicate pages. Some downloaded pages contained many records scraped before on other pages. Such pages, with more than a quarter repeated records, were filtered.
    \item Cleaning HTML. At this stage blocks, such as blocks of code in JavaScript (i.e. all nodes with the \verb|<script>| tag), that do not affect the algorithm's performance were removed from the HTML code of the page.
    \item Translating HTML. Since our dataset contains pages in Russian, it was necessary to translate them into English to be able to use pre-trained models.
    \item Division into training and test parts. The distribution of attributes and domains of web pages was taken into account. Each domain was placed either to the training or to the test parts (see attributes split in the \autoref{fig:attribute_dist}). The final distribution is shown in the \autoref{fig:amount_of_attributes_test} and \autoref{fig:amount_of_attributes_train}. Ratio of parts after splitting was the following: 75\% - training, 25\% - testing.
\end{enumerate}

\subsection{Dataset Statistics}
Since the maps were based on CSS selectors, there was a problem with downloading sites that dynamically change the names of the styles on the pages. In other words, when the website changed the name of the CSS style class, the selector specifying the class name stopped working correctly. Therefore, we were able to download only 278 sites.

\begin{table}[!htb]
    \caption{Attribute frequency}
    \centering
    \begin{tabular}{c|c|c|c}
        \toprule
        Name & Pages & Records & Sites\\
        \midrule
        title & 12679 & 247262 & 275\\
        date & 12296 & 241634 & 251\\
        tag & 6165 & 108400 & 140\\
        \cdashline{1-4}
        short\_text & 6855 & 115983 & 138\\
        short\_title & 105 & 1289 & 4\\
        author & 87 & 957 & 1\\
        time & 730 & 15809 & 8 \\
        \bottomrule
    \end{tabular}
    \label{tab:attribute_statistic}
\end{table}
Thus, a dataset that contained 13120 pages from 278 Internet media was prepared (distribution between pages and entries on them, shown in the \autoref{fig:page_records_dist}). On each page, the corresponding attributes were annotated,  and their frequency was presented in the \autoref{tab:attribute_statistic}. All pages presented in the table have UTF-8 encoding. 

We researched extraction of 3 attributes from our dataset: title, author and tag. Because they are the most frequent in our dataset.
\section{Experimental evaluation}
In this chapter we describe our methods of solving the problem and show the results on our dataset.
\subsection{Parallel pipeline}
The first architecture we tested was ``Parallel pipeline``. Visual scheme of this architecture described in \autoref{fig:parallel_pipeline}.
\begin{figure}[!htb]
    \centering
    \includegraphics[width=0.45\textwidth]{attach/parallel_pipeline.jpg}
    \caption{Parallel pipeline scheme}
    \label{fig:parallel_pipeline}
\end{figure}
\begin{itemize}
    \item \textbf{Segmentation} At this stage, we find records boundaries that help us split html into records and find corresponding attributes of each record.
    \item \textbf{Classification} At this stage, we give a corresponding label for each node on html. It is important that the segmentation stage and the classification stage are two independent stages.The results of neither are not shared with the other.
    \item \textbf{Matching of results} At this stage, the final result of the method is formed. Based on the results of segmentation and classification, the records are matched with their attributes and provided in a structured view.
\end{itemize}

\subsection{Sequential pipeline}
We will also test sequential pipeline as described at \autoref{fig:sequential_pipeline}. In the following we will compare the qualities of both architectures. Quality should vary due to different approaches to the classification stage.
\begin{figure}[!htb]
    \centering
    \includegraphics[width=0.45\textwidth]{attach/sequential_pipeline.jpg}
    \caption{Sequential pipeline scheme}
    \label{fig:sequential_pipeline}
\end{figure}
\begin{itemize}
    \item \textbf{Segmentation} In sequential architecture this stage is similar to parallel scheme.
    \item \textbf{Classification} At this stage, the final result of the method is formed. Based on the segmentation results, the stage of searching for attributes only in the selected area is carried out. Thus, the matching stage is not required.
\end{itemize}


\subsection{Segmentation subtask} \label{segmentation}
We formulate the task of web-pages segmentation as a task of searching for information boundaries, which are first nodes containing text of the DOM subtree and belonging to it. This formulation is the same as that used in the \cite{san_plate_2023}.

We test two methods of solving this task:
\begin{enumerate}
    \item Heuristic method based on classical MDR
    \item Neural method based on MarkupLM, which is state-of-the-art model in information extraction from HTML pages 
\end{enumerate}

\textbf{MDR} We tested MDR because it has open-source code. Since this method proposes several ``candidates`` for segmentation, ordered by their probability. We choose the segmentation with the highest probability.

\textbf{MarkupLM} We train the MarkupLM model on this task, having the data previously prepared: for each record on the page, we have marked its first node which contains text with the “BEGIN” label. All other nodes on the page have been marked with the label “OUT“. Thus, the solution of the problem is to predict the corresponding label for each DOM tree node by the model.

\subsubsection*{Segmentation metrics}
The result of the segmentation method can be evaluated by page-weighted metrics: $Precision_{avg}$, $Recall_{avg}$, $F1_{avg}$

To calculate them, the reference segmentation of the given page on the record is compared with the one obtained by the proposed methods. Based on this comparison, for each segment it is possible to calculate $TP_{page K}$ -- the number of DOM nodes correctly marked as an information boundary, $FP_{page K}$ - the number of DOM tree nodes that are not an information boundary, but marked with it, $FN_{page K}$ -- the number of DOM tree nodes that are an information boundary, but not marked as it:

$$
Precision_{page K} = \frac{TP_{page K}}{TP_{page K} + FP_{page K}}, 
$$

$$
Recall_{page K} = \frac{TP_{page K}}{TP_{page K} + FN_{page K}},
$$

$$
F\textit{1}_{page K} = 2 \cdot \frac{Precision_{page K} \cdot Recall_{page K}}{Precision_{page K} + Recall_{page K}},
$$

$$
Precision_{avg} = \frac{\sum_{i = 1}^{K}Precision_{page\,i}}{K},
$$

$$
Recall_{avg} = \frac{\sum_{i = 1}^{K}Recall_{page\,i}}{K},
$$

$$
F\textit{1}_{avg} = \frac{\sum_{i = 1}^{K}F\textit{1}_{page\,i}}{K}
$$

Also we will calculate classical $NMI$\cite{Strehl2002} and $ARI$\cite{Hubert1985} for segmentation task.

\begin{table}[!htb]
    \caption{Results of segmentation experiment}
    \centering
        \begin{tabular}{c|ccc|cc}
        \toprule
        Method & $Recall_{avg}$ & $Precision_{avg}$ & $F\textit{1}_{avg}$ & $ARI$ & $NMI$\\
        \midrule
         MDR & 0.473 & 0.486 & 0.465 & 0.437 & 0.517\\
         MarkupLM & \textbf{0.908} & \textbf{0.941} & \textbf{0.925} & \textbf{0.802} & \textbf{0.869}\\
         \bottomrule
    \end{tabular}
    \label{tab:segmentation}
\end{table}

The test results are shown in the \autoref{tab:segmentation}. The MarkupLM model shows superior results in all metrics in comparison with the other tool.

\subsection{Classification subtask}
This subtask includes searching for all potential attributes in the area of interest on the html page (for parallel pipeline - it is the whole page, and for sequential - a part of it).

\begin{table}[htbp]
\centering
\caption{Results of classification experiment}
\begin{tabular}{c|c|*3c}
    \toprule
        Type                                    &     Metric            & title    & tag      &  date \\
        
        \midrule
\multirow{3}{*}{Record context}                & $Precision_{avg}$     & 0.98     & 0.79     & 0.86  \\
                                                & $Recall_{avg}$        & 0.99     & 0.85     & 0.91  \\
                                                & $F\textit{1}_{avg}$   & \textbf{0.99}     & 0.82      & \textbf{0.88}  \\
        
        \midrule
\multirow{3}{*}{Page context}                   & $Precision_{avg}$     & 0.68     & 0.61       & 0.64      \\
                                                & $Recall_{avg}$        & 0.95     & 0.76       & 0.86      \\
                                                & $F\textit{1}_{avg}$   &  0.79    & \textbf{0.89}       & 0.73      \\
    \bottomrule
\end{tabular}
\label{tab:markuplm_classification}
\end{table}

The MarkupLM model is also chosen as a classification method. We train it on the task of predicting a node’s label. Thus, the model predicts the labels: ``title``, ``tag``, ``date``. 


\subsubsection*{Classification metrics}
Results are evaluated by classical page-weighted classification metrics: $Precision_{avg}$, $Recall_{avg}$ and $F\textit{1}_{avg}$.

We trained two models: one for the pipeline with full page context (in such conditions, classification is used in the parallel pipeline), and one with just record context (passing the part of the html related to a specific record directly to the input, so it is used in a sequential pipeline). The comparison of their results is presented in the table \autoref{tab:markuplm_classification}. So, MarkupLM with record context shows rather high results in this subtask.

\subsection{Matching subtask}
We propose matching algorithm:
\begin{itemize}
    \item Let $G$ -- set of xpaths\footnote{We are considering positional xpath expressions consisting of tags and indices in the DOM tree path from root to the given node.} to all DOM-nodes of htmls marked as ``information boundary`` at the segmentation stage.
    \item Let's make $\widehat{G}$: for each xpath from $G$ find minimal (by length) node prefix. Each prefix should belong to only one xpath from $G$.
    \item Let's enumerate $\widehat{G}$.
    \item  Let's refer Xpath of some attribute determined at classification stage as $xpath_{attr}$. Each attribute is matched with record with number $N$ if $\widehat{G}[N]$ is prefix for $xpath_{attr}$.
\end{itemize}


\subsection{Final method evaluation metrics}
We evaluated the final method using metrics similar to detailed page information extraction task's metrics. We define ``predicted record`` as a set of predicted attributes matched to the same record at the Matching stage. Also we define ``reference record`` as a set of ground truth attributes values from a sought ``information boundary``. There are three cases possible when the method is run:
\begin{enumerate}
    \item \textit{We can match the predicted record with the reference record.}\\ $TP$ is the number of correctly extracted attributes (for example if a record contains 4 tags and they are extracted correctly, then $TP$ is 4). $FP$ is the number of extracted attributes but none of them should have been. $FN$ is the number of not extracted attributes but all of them should have been.
    \item \textit{We cannot match any of the predicted records with the reference record.}\\ Only $FN$ increases for all attributes of the reference record.
    \item \textit{We cannot match any of the reference records with the predicted record.}\\ In this case $FP$ increases for all attributes of the predicted record.
\end{enumerate}

\subsection{Experiments results}
\begin{table}
\centering
\caption{Results of final method}
\begin{tabular}{c|c|*3c}
    \toprule
        Method                  &     Metric     & title    & tag       &  date        \\
        
        \midrule
\multirow{6}{*}{Parallel Pipeline}             & TP & 56008    & 29202     & 60570        \\
                                                & FP & 7021     & 10804     & 12728       \\
                                                & FN & 8083     & 4484      & 12361       \\
                                                \cmidrule(lr){2-5}
                                                & Precision & \textbf{0.874}     & \textbf{0.867}     & \textbf{0.831}  \\
                                                & Recall    & 0.889     & 0.73      & 0.826  \\
                                                & F1        & 0.881     & 0.793     & 0.828  \\
        
        \midrule
\multirow{6}{*}{Sequential Pipeline}            & TP & 55891    & 28643     & 58645           \\
                                                & FP & 1492     & 7641      & 7056            \\
                                                & FN & 8801     & 4899      & 13659           \\
                                                \cmidrule(lr){2-5}
                                                & Precision & 0.864     & 0.854     & 0.811  \\
                                                & Recall    & \textbf{0.974}     & \textbf{0.789}     & \textbf{0.893}  \\
                                                & F1        & \textbf{0.916}     & \textbf{0.82}      & \textbf{0.85}   \\
    \bottomrule
\end{tabular}
\label{tab:final_experiments_results}
\end{table}

We tested both proposed methods: parallel pipeline and sequential pipeline. For these methods, we used the corresponding trained versions of MarkupLM, as we described them in the previous chapters. Comparative results are shown in \autoref{tab:final_experiments_results}, while \autoref{tab:markuplm_classification} presents the performance of classification model in different contexts—record and page.

In \autoref{tab:markuplm_classification}, the model using the ``record context`` outperforms model using the ``page context``. The tag attribute in the page context performs relatively well, but the overall trend shows that the model is more effective in the record context. We assume that the advantage of page context for tag-attribute extraction is related to the peculiarity of this attribute. Most often, each record has several tags. Information about neighboring records allows making less noisy predictions.

The parallel pipeline demonstrates solid precision, particularly for title. However, it shows a decline in recall. In contrast, the sequential pipeline outperforms the parallel pipeline in recall. The sequential pipeline achieves a balanced performance, making it the preferred method due to its robustness and consistency.

In conclusion, the sequential pipeline is superior to the parallel pipeline. The record context further enhances classification performance, making it the optimal setting for method deployment. The high results of the obtained methods prove their applicability in real life.

\paragraph{Summary}
Our findings provide significant insights into the influence of correctness, explanations, and refinement on evaluation accuracy and user trust in AI-based planners. 
In particular, the findings are three-fold: 
(1) The \textbf{correctness} of the generated plans is the most significant factor that impacts the evaluation accuracy and user trust in the planners. As the PDDL solver is more capable of generating correct plans, it achieves the highest evaluation accuracy and trust. 
(2) The \textbf{explanation} component of the LLM planner improves evaluation accuracy, as LLM+Expl achieves higher accuracy than LLM alone. Despite this improvement, LLM+Expl minimally impacts user trust. However, alternative explanation methods may influence user trust differently from the manually generated explanations used in our approach.
% On the other hand, explanations may help refine the trust of the planner to a more appropriate level by indicating planner shortcomings.
(3) The \textbf{refinement} procedure in the LLM planner does not lead to a significant improvement in evaluation accuracy; however, it exhibits a positive influence on user trust that may indicate an overtrust in some situations.
% This finding is aligned with prior works showing that iterative refinements based on user feedback would increase user trust~\cite{kunkel2019let, sebo2019don}.
Finally, the propensity-to-trust analysis identifies correctness as the primary determinant of user trust, whereas explanations provided limited improvement in scenarios where the planner's accuracy is diminished.

% In conclusion, our results indicate that the planner's correctness is the dominant factor for both evaluation accuracy and user trust. Therefore, selecting high-quality training data and optimizing the training procedure of AI-based planners to improve planning correctness is the top priority. Once the AI planner achieves a similar correctness level to traditional graph-search planners, strengthening its capability to explain and refine plans will further improve user trust compared to traditional planners.

\paragraph{Future Research} Future steps in this research include expanding user studies with larger sample sizes to improve generalizability and including additional planning problems per session for a more comprehensive evaluation. Next, we will explore alternative methods for generating plan explanations beyond manual creation to identify approaches that more effectively enhance user trust. 
Additionally, we will examine user trust by employing multiple LLM-based planners with varying levels of planning accuracy to better understand the interplay between planning correctness and user trust. 
Furthermore, we aim to enable real-time user-planner interaction, allowing users to provide feedback and refine plans collaboratively, thereby fostering a more dynamic and user-centric planning process.


\appendices

\ifCLASSOPTIONcaptionsoff
  \newpage
\fi

\bibliographystyle{IEEEtran}

\bibliography{main}

% \clearpage

% \section{Appendix}
% \label{section:appendix}


\end{document}
