\documentclass{article}


\usepackage{arxiv}

\usepackage[utf8]{inputenc} % allow utf-8 input
\usepackage[T1]{fontenc}    % use 8-bit T1 fonts
\usepackage{hyperref}       % hyperlinks
\usepackage{url}            % simple URL typesetting
\usepackage{booktabs}       % professional-quality tables
\usepackage{amsfonts}       % blackboard math symbols
\usepackage{nicefrac}       % compact symbols for 1/2, etc.
\usepackage{microtype}      % microtypography
\usepackage{lipsum}		% Can be removed after putting your text content
\usepackage{graphicx}
\usepackage{natbib}
\usepackage{doi}
\usepackage{comment}
\usepackage{amsmath}
\usepackage{ntheorem}
\newtheorem*{theorem}{Theorem}
\newtheorem*{lemma}{Lemma}
\newtheorem*{proof}{Proof}
\usepackage{microtype}      % microtypography
\usepackage{cleveref}       % smart cross-referencing
\usepackage{lipsum}         % Can be removed after putting your text content
\usepackage{graphicx}
\usepackage{natbib}
\usepackage{doi}
\usepackage{wrapfig}
\usepackage{float}

\title{Learning Hamiltonian
Density Using DeepONet}

%\date{September 9, 1985}	% Here you can change the date presented in the paper title
%\date{} 					% Or removing it

\author{%
	Baige Xu\\
	Graduate School of Science\\
	Kobe University\\
	Kobe, Japan \\
\texttt{baigexu@stu.kobe-u.ac.jp} \\
	%% examples of more authors
	\And
	Yusuke Tanaka \\
	NTT Communication Science Laboratories\\
	NTT Corporation\\
	Kyoto, Japan \\
\texttt{ysk.tanaka@ntt.com} \\
    \And
	Takashi Matsubara \\
	Graduate School of Information Science and Technology\\
	Hokkaido University\\
	Sapporo, Japan \\
\texttt{matsubara@ist.hokudai.ac.jp} \\
    \And
	Takaharu Yaguchi\\
	Graduate School of Science\\
	Kobe University\\
	Kobe, Japan \\
\texttt{yaguchi@pearl.kobe-u.ac.jp} \\
	%% \AND
	%% Coauthor \\
	%% Affiliation \\
	%% Address \\
	%% \texttt{email} \\
	%% \And
	%% Coauthor \\
	%% Affiliation \\
	%% Address \\
	%% \texttt{email} \\
	%% \And
	%% Coauthor \\
	%% Affiliation \\
	%% Address \\
	%% \texttt{email} \\
}


\begin{comment}
\author{ \href{https://orcid.org/0000-0000-0000-0000}{\includegraphics[scale=0.06]{orcid.pdf}\hspace{1mm}David S.~Hippocampus}\thanks{Use footnote for providing further
		information about author (webpage, alternative
		address)---\emph{not} for acknowledging funding agencies.} \\
	Department of Computer Science\\
	Cranberry-Lemon University\\
	Pittsburgh, PA 15213 \\
	\texttt{hippo@cs.cranberry-lemon.edu} \\
	%% examples of more authors
	\And
	\href{https://orcid.org/0000-0000-0000-0000}{\includegraphics[scale=0.06]{orcid.pdf}\hspace{1mm}Elias D.~Striatum} \\
	Department of Electrical Engineering\\
	Mount-Sheikh University\\
	Santa Narimana, Levand \\
	\texttt{stariate@ee.mount-sheikh.edu} \\
	%% \AND
	%% Coauthor \\
	%% Affiliation \\
	%% Address \\
	%% \texttt{email} \\
	%% \And
	%% Coauthor \\
	%% Affiliation \\
	%% Address \\
	%% \texttt{email} \\
	%% \And
	%% Coauthor \\
	%% Affiliation \\
	%% Address \\
	%% \texttt{email} \\
}
\end{comment}
% Uncomment to remove the date
%\date{}

% Uncomment to override  the `A preprint' in the header
%\renewcommand{\headeright}{Technical Report}
%\renewcommand{\undertitle}{Technical Report}
\renewcommand{\undertitle}{}
\renewcommand{\shorttitle}{Learning Hamiltonian 
Density Using DeepONet}

%%% Add PDF metadata to help others organize their library
%%% Once the PDF is generated, you can check the metadata with
%%% $ pdfinfo template.pdf
\hypersetup{
pdftitle={A template for the arxiv style},
pdfsubject={q-bio.NC, q-bio.QM},
pdfauthor={David S.~Hippocampus, Elias D.~Striatum},
pdfkeywords={First keyword, Second keyword, More},
}

\begin{document}
\maketitle

\begin{abstract}
%近年,PDEで表せる物理現象のモデリングのための深層学習が注目を集めてる
In recent years, deep learning for modeling physical phenomena which can be described by partial differential equations (PDEs) have received significant attention. 
%例えば,forハミルトン力学,HNNがある.
For example, for learning Hamiltonian mechanics, methods based on deep neural networks such as Hamiltonian Neural Networks (HNNs) and their variants have achieved progress.
%しかし,データの離散化に依存,微分変数の決定が必要.
However, 
%these works 
existing methods typically
depend on the discretization of data, and the determination of required differential operators is often necessary. 
%本研究では,波動方程式のモデリングから着目,作用素学習の手法を提案.特に,変分導関数の計算法を提案.実験:提案手法は波のエネルギー密度学習できる,データの離散と微分変数のやつは不要.
Instead, in this work, we propose an operator learning approach %, beginning with the
for modeling wave equations. In particular, we present a method to compute the variational derivatives that are needed to formulate the equations using the automatic differentiation algorithm.
The experiments demonstrated that %our proposal 
the proposed method
is able to learn the operator %of
that defines the Hamiltonian density of waves from data with unspecific discretization without 
determination of the differential operators.
%any 
%differential variable determination.
\end{abstract}


% keywords can be removed
\keywords{Operator learning \and Hamiltonian mechanics \and Wave equations \and Variational derivatives}


\section{Introduction}
%在自然界中很多物理现象都可以用pde来描述,比如扩散、热传导、波动等等[文献]。
Many physical phenomena in nature can be described by partial differential equations (PDEs) \cite{pde}, such as diffusion in liquids and gases, thermal conduction, and wave propagation \cite{diff,thermal,wave}.
%对这些具有物理意义的方程式建模并求解,具有极大的现实意义,比如可以用于天气预测、热学设备的制造、交通工具设计等等[文献]。
Modeling and solving these physically meaningful equations have practical significance %, %significant practical implications, 
in many applications like weather forecasting, material processing, and aircraft control \cite{weather,material,air}.
%近年,对于这些pde,相比传统的数值解法,基于深层学习的建模求解手法已经被广泛地研究。
In recent years, the techniques of modeling and solving %such 
PDEs based on deep learning have been widely studied instead of conventional numerical solvers.
%比如,对于
For example, for learning Hamiltonian mechanics, which provide a foundational framework of diverse physical phenomena, Hamiltonian Neural Networks (HNNs) and their variants have been actively developed %proposed 
as powerful tools for %approaches to 
modeling equations \cite{hnn,symp}.
%但是,由于这些手法都基于深层神经网络,因此它们依赖于数据的离散,而且在学习时需要决定方程中出现的微分算子及它们的离散。
However, since these methods are based on deep neural networks, they depend on the discretization of the data. %, and for 
Also, the differential operators appearing in the equations, such as $(\partial_x,\partial_{xx},\cdots)$, %they 
and their discretization are necessary to be determined when learning the equations. % for learning.

%比如Hamiltonian PDEs
To give an example of such equations, consider modeling Hamiltonian PDEs that describe %s 
many wave phenomena: %, having the form
\begin{equation*}%\label{eq:1}
    \dfrac{\partial u}{\partial t}=\mathcal{D}
    \dfrac{\delta \mathcal{H}}{\delta u},
\end{equation*}
where $u=u(t,x)$ is a function defined on a Hilbert space $\mathcal{X}$, and $\mathcal{D}$ is a skew-symmetric operator, which includes specific examples such as
\begin{equation*}
    \left(
    \begin{array}{cc}
       0 & \mathrm{Id} \\
       -\mathrm{Id} & 0
    \end{array}
    \right),\quad
    \dfrac{\partial}{\partial x},\quad
    \left(
    \begin{array}{cc}
        0 & \dfrac{\partial}{\partial x} \\
    \dfrac{\partial}{\partial x} & 0
    \end{array}
    \right). %,\ldots
\end{equation*}
$\mathcal{H}$ is the Hamiltonian, the energy functional of the equation, and is defined as $\mathcal{H}=\int H(u,u_x,u_{xx},\cdots)\mathrm{d}x$. The functional $H$ is called the Hamiltonian density of the equation, representing the local energy distribution within the dynamical system. $\dfrac{\delta \mathcal{H}}{\delta u}$ is called the variational derivative of $\mathcal{H}$, whose definition is given as
\begin{equation*}
\dfrac{\mathrm{d}}{\mathrm{d}\epsilon}\mathcal{H}[u+\epsilon v]|_{\epsilon = 0}=\langle \dfrac{\delta \mathcal{H}}{\delta u}, v \rangle _{L^2},\quad \forall u,v\in\mathcal{X},
\end{equation*}
where $\langle \cdot , \cdot \rangle _{L^2}$ represents the inner product on $\mathcal{X}$.

%このような方程式をニューラルネットワークによってモデリングする際には,$H$の学習に関して,$(u,u_x,u_{xx},\cdots)$などの変数とその離散化を決めなければならない.そのため,このような手法は望ましくない.
When modeling such equations using HNNs or other methods based on neural networks, usually, $H$ is modeled by using a neural network  $H_{\mathrm{NN}}$.
%usually, for learning $H$ by a neural network $H_{\mathrm{NN}}$,
To train the neural network, it is necessary to determine differential variables such as $(u,u_x,u_{xx},\ldots)$ that appear in the Hamiltonian density 
and their discretizations, which is not easy without detailed knowledge of target phenomena. % to fit the training data, which is difficult to realize. 
Therefore, such methods are not ideal in practice. %this case.

%そこで,本研究では,作用素学習を利用することで,微分作用素が含まれている変数とその離散化の決定が必要のないハミルトニアン密度の学習法を提案する.特に,$\mathcal{H}$の変分導関数が$\mathcal{H}$の自動微分による勾配から求められることを示す.
To overcome this limitation, in this study, we propose an operator learning approach for modeling wave equations, that does not rely on the discretizations of data and does not require determining the variables involving differential operators or their discretizations.
In particular, we show that the variational derivative of $\mathcal{H}$ can be obtained from the gradient of $\mathcal{H}$ computed via automatic differentiation algorithm.
The results of numerical experiments show that the proposed method %our proposal 
can in fact learn the Hamiltonian density of the wave equations. %It will be also shown that the variational derivative of the Hamiltonian can be computed by using the automatic differentiation algorithm. % obtain the solution of the equations, realize the modeling of wave equations.
%can efficiently compute the variational derivative of the Hamiltonian to obtain the solution of the equations, realize the modeling of wave equations.


\section{Operator Learning and DeepONet}
Operator learning addresses the challenges faced by traditional neural networks by learning mappings between infinite-dimensional function spaces directly. Among operator learning methods, Neural Operators have gained significant attention for their ability to combine the universality of operator learning with the expressive power of deep learning architectures \cite{no}. One notable variant %implementation 
of neural operators is DeepONet, which is designed to efficiently learn and approximate arbitrary nonlinear continuous operators \cite{don}. %, and 
DeepONet has demonstrated remarkable performance in solving parametric PDEs and other high-dimensional problems. 

Consider the general architecture of a DeepONet, when learning a simple dynamical system described by %this 
an ODE:
\begin{equation*}
    \dfrac{\mathrm{d}u(t)}{\mathrm{d}t}=f(t).
\end{equation*}
DeepONet represents the target operator $G:f \mapsto u$ as a pair of sub networks.
The first one is called the branch network, which processes the
%discrete 
values of the input function $u$ sampled at a specific finite set of location points $\{x_i\}_{i=1}^m$, taking $(f(x_1),f(x_2),...,f(x_m))^{\top}\in\mathbb{R}^m$ as input and $(b_1,\cdots,b_p)\in\mathbb{R}^p$ as output.
The second one is called trunk network, which evaluates the operator at the target locations. For an unstacked DeepONet, there is just one trunk network. It takes $y=(y_1,\cdots,y_n)\in\mathbb{R}^n$ as input and $(t_1,t_2,...,t_p)^{\top} \in \mathbb{R}^p$ as output. 
Then the outputs of two sub networks are merged together by taking their inner product 
%to learn 
to approximate the
target operator $G$:
\begin{equation*}
    G(f)(y) \simeq G_\mathrm{NO}(f)(y) = 
    \sum_{k=1}^{p} b_k t_k.
\end{equation*}
In fact, $G_\mathrm{NO}(f)(y)$ can be viewed as a function of $y$ conditioning on $u$. %, so that the 
The loss function of DeepONet is given as 
\begin{equation*}
    \mathcal{L} = \int {\lvert \lvert G_{\mathrm{target}}(f)(y)-G_{\mathrm{model}}(f)(y)} \rvert \rvert ^2 \mathrm{d}y,
\end{equation*}
%can be approximated as 
which is typically approximated by, e.g., 
\begin{equation*}
    \mathcal{L} \approx \sum_{y_i} {\lvert \lvert G_{\mathrm{target}}(f)(y_i)-G_{\mathrm{model}}(f)(y_i) \rvert \vert}^2,
\end{equation*}
where $y_i$'s are given collocation points. By minimizing the above loss function, DeepONet learns the target operator.
%As shown above, this structured design allows DeepONet to approximate continuous operators efficiently while working with discrete values of input function. 



\section{Operator Learning for Modeling Hamiltonian Partial Differential Equations Using DeepONet}
In this section, we propose a learning method %approach 
for modeling Hamiltonian partial differential equations using DeepONet. In particular, we propose a method to compute the %required 
variational derivative of %function of 
Hamiltonian $\mathcal{H}$ that is required to determine the PDEs % % in learning, through 
using the automatic differentiation algorithm.

\subsection{Learning Hamiltonian Density Using DeepONet}
In the proposed model, we use DeepONet to learn an operator that maps the state variable to the energy density.
The input of the branch network is %given 
the values $(u(x_1),\cdots,u(x_N))^\top$ of the function $u(t,x)$ at a finite set of position points $\{x_i\}_{i=1}^N$ as the input function to DeepONet.
The trunk network is given a vector $y$, which represents the points in the domain %region 
of the output function. The output of DeepONet is 
the inner product of the two sub network outputs $(c_1,\cdots,c_p)^\top \in \mathbb{R}^p$ and $(\psi_1,\cdots,\psi_p)^\top \in \mathbb{R}^p$, as
\begin{equation*}
    H_{NO} = \sum_{k=1}^{p} c_k \psi_k.
\end{equation*}
We train the networks so that $H_{NO}$ approximates the target Hamiltonian density.
%then the Hamiltonian density $H_{NO}$, which is
By integrating the learned Hamiltonian density $H_{NO}$, we can obtain the learned Hamiltonian $\mathcal{H}_{NO}$.

%Learning 
To train the networks we need to compute
%then requires computing 
the variational derivatives $\dfrac{\delta \mathcal{H}_{NO}}{\delta u}$ of the Hamiltonian. 
Generally speaking, the variational derivatives of $\mathcal{H}_{NO}$ can be %usually 
obtained by partial integration using
\begin{equation*}
    \dfrac{\delta \mathcal{H}_{NO}}{\delta u}=\dfrac{\partial H_{NO}}{\partial u} - \partial _x (\dfrac{\partial H_{NO}}{\partial u_x})+ \partial _x ^2 (\dfrac{\partial H_{NO}}{\partial u_{xx}})-\cdots;
\end{equation*}
however, in the model, $H_{NO}$ learned by DeepONet is not given as such a form in which this calculation can be performed.
Therefore, %the above 
a method for finding variational derivatives must be newly designed. % separately.

\subsection{Deriving Variational Derivatives of Hamiltonian Using Automatic Differentiation Algorithm}
To derive a method for finding variational derivatives, we need to consider the roles of the trunk and the branch networks. 
In fact, in the architecture of DeepONet, %can be regarded as 
the trunk network provides a basis $(\psi_1,\cdots,\psi_p)$ for the target function space, and the branch network provides the coefficients $(c_1,\cdots,c_p)$ of the basis. 
In particular, $(c_1,\cdots,c_p)$ is obtained by introducing a basis $L_i$ of the dual space of subspaces of the input function space, where $L_i$ satisfies $L_i(u)=u(x_i)$.
Then, introducing a non-contradictory basis $\phi_j$ such that $L_i \phi_j=\delta_{ij}$, then the input function $u$ that is tied to the discrete value vector $(u(x_1),\cdots,u(x_N))^\top$ 
can be represented as
\begin{equation}\label{u}
  \begin{aligned}
    u&=(L_1(u))\phi_1 +\cdots+ (L_N(u))\phi_N 
    = u(x_1)\phi_1+\cdots+u(x_N)\phi_N.
  \end{aligned}
\end{equation}
It is important to notice that we can compute derivatives with respect to the vector $(u(x_1),\cdots,u(x_N))^\top$
%used for 
by using the automatic differentiation algorithm. We want to associate the computed derivatives by this algorithm with the variational derivative.

On the other hand, let $\mathcal{H}:(\mathcal{X},\langle,\rangle _{L^2} )\to \mathbb{R}$ be a smooth enough Hamiltonian and for any $u \in \mathcal{X}$, 
\begin{equation*}
    v\mathcal{H}=\mathrm{d}\mathcal{H}(v),\quad \forall v \in \mathcal{T}_u\mathcal{X}
\end{equation*}
holds. That means $\mathrm{d}\mathcal{H}$ is defined as a linear functional $\mathrm{d}\mathcal{H}:v \in \mathcal{T}_u\mathcal{X} \mapsto \mathrm{d}\mathcal{H}(v)\in \mathbb{R}$ and $\mathrm{d}\mathcal{H} \in \mathcal{T}_u^*\mathcal{X}$.
$\mathcal{T}_u\mathcal{X}$ and $\mathcal{T}_u^*\mathcal{X}$ denote the tangent space and cotangent space on $\mathcal{X}$, respectively. 

By Riesz's representation theorem, there exists a $v_{\mathcal{H}}\in \mathcal{T}_u\mathcal{X}$ such that
\begin{equation}\label{grad}
    \mathrm{d}\mathcal{H}(w)= \langle v_{\mathcal{H}},w \rangle, \quad \forall w \in \mathcal{T}_u\mathcal{X}.
\end{equation}
Also, from the definition of $\dfrac{\delta \mathcal{H}}{\delta u}$, it holds that
\begin{equation}\label{riesz}
    \mathrm{d}\mathcal{H}(w)= \langle \dfrac{\delta \mathcal{H}}{\delta u},w \rangle _{L^2}, \quad \forall w \in \mathcal{T}_u\mathcal{X}.
\end{equation}
%is valid. 
That means the gradient $V_{\mathcal{H}}$ defined by any inner product is related to the variational derivative using the $L^2$ inner product.
Using the above \eqref{u}, \eqref{grad} and \eqref{riesz}, the variational derivative of $\mathcal{H}_{NO}$ in DeepONet training can be obtained using the gradient obtained by automatic differentiation algorithm.

\begin{theorem}
Suppose that $\Tilde{\mathcal{X}}=\mathrm{span}\{\phi_1,...,\phi_N\}$ is a subspace of a Hilbert space $\mathcal{X}$, $\phi_i = \phi_i(x) \in \mathcal{X},i=1,...,N$. Let $\mathcal{H}_{NO}:(\Tilde{\mathcal{X}},\langle,\rangle _{L^2} ) \to \mathbb{R}$ be a sufficiently smooth Hamiltonian learned by DeepONet. If $\nabla_{AD}\mathcal{H}_{NO}$ is a gradient of $\mathcal{H}_{NO}$ which is derived through automatic differentiation, for any $u \in \mathcal{X}$ and $\Tilde{u} \in \Tilde{\mathcal{X}}$, the variational derivative of $\mathcal{H}_{NO}$, $\dfrac{\delta \mathcal{H}_{NO}}{\delta u}$, can be obtained from $\nabla_{AD}\mathcal{H}_{NO}$:
{\small
\begin{equation*}
\begin{aligned}
    \dfrac{\delta \mathcal{H}_{NO}}{\delta u}=
    (\nabla_{AD}\mathcal{H}_{NO})^\top
    \begin{pmatrix}
            \int \phi_1^2\mathrm{d}x & \cdots & \int \phi_1 \phi_N\mathrm{d}x\\
            \vdots & \ddots & \vdots \\
            \int\phi_1 \phi_N\mathrm{d}x & \cdots & \int \phi_N^2\mathrm{d}x
    \end{pmatrix}^{-1}
    \begin{pmatrix}
            \phi_1 \\
            \vdots \\
            \phi_N
    \end{pmatrix}.
\end{aligned}
\end{equation*}
}
\end{theorem}

\begin{proof}
For a continuous linear functional $\mathrm{d}\mathcal{H}_{NO} \in \mathcal{T}_u^*\Tilde{\mathcal{X}}$, there exists a $\dfrac{\delta \mathcal{H}_{NO}}{\delta u}=g_1\phi_1+\cdots+g_N\phi_N\in\mathcal{T}_u\Tilde{\mathcal{X}}$ such that for all $w=w_1\phi_1+\cdots+w_N\phi_N \in \mathcal{T}_u\Tilde{\mathcal{X}}$,
\begin{equation*}\label{l2}
\begin{aligned}
\mathrm{d}\mathcal{H}_{NO}(w)&=\langle \dfrac{\delta \mathcal{H}_{NO}}{\delta u},w\rangle_{L^2}
=\int (\dfrac{\delta \mathcal{H}_{NO}}{\delta u}w)\mathrm{d}x \\
&= \int ((g_1\phi_1 + \cdots + g_N\phi_N)(w_1\phi_1+\cdots+w_N\phi_1))\mathrm{d}x \\
&=(g1,\cdots,g_N)
    \begin{pmatrix}
        \int \phi_1^2\mathrm{d}x & \cdots & \int \phi_1 \phi_N\mathrm{d}x\\
            \vdots & \ddots & \vdots \\
            \int\phi_1 \phi_N\mathrm{d}x & \cdots & \int \phi_N^2\mathrm{d}x
    \end{pmatrix}
    \begin{pmatrix}
        w_1\\ \vdots \\w_N
    \end{pmatrix}.
\end{aligned}
\end{equation*}
In the subspace $\Tilde{\mathcal{X}}=\mathrm{span}\{\phi_1,...,\phi_N\}$, we can introduce the following two inner product: % are defined:
\begin{equation*}
    \langle \cdot , \cdot \rangle _{L^2}: \langle u,v \rangle _{L^2} = \int (uv)\mathrm{d}x,
\end{equation*}
\begin{equation*}
    \langle \cdot , \cdot \rangle _{\mathbb{R}^N}: \langle u,v \rangle _{\mathbb{R}^N} = \sum L_i(u)L_i(v) = \sum u_i v_i.
\end{equation*}
From Riesz's representation theorem,
\begin{equation*}
    \begin{aligned}
        \mathrm{d}\mathcal{H}_{NO}(w)&=
        \langle \dfrac{\delta \mathcal{H}_{NO}}{\delta u},w\rangle_{L^2}
        =\langle \nabla \mathcal{H}_{NO},w \rangle _{\mathbb{R}^N} \\
        &= \sum _{i=1}^N (\nabla \mathcal{H}_{NO})_i w_i \\
        &= (\nabla_{AD}\mathcal{H}_{NO})^\top
        \begin{pmatrix}
        w_1\\ \vdots \\w_N
        \end{pmatrix}.
    \end{aligned}
\end{equation*}
Therefore, we get
\begin{equation*}
\begin{aligned}
    &(g_1,\cdots,g_N)= (\nabla_{AD}\mathcal{H}_{NO})^\top
    \begin{pmatrix}
            \int \phi_1^2\mathrm{d}x & \cdots & \int \phi_1 \phi_N\mathrm{d}x\\
            \vdots & \ddots & \vdots \\
            \int\phi_1 \phi_N\mathrm{d}x & \cdots & \int \phi_N^2\mathrm{d}x
    \end{pmatrix}^{-1},
\end{aligned}
\end{equation*}
and hence
\begin{equation*}
\begin{aligned}
    \dfrac{\delta \mathcal{H}_{NO}}{\delta u}
    &= g_1\phi_1 + \cdots + g_N\phi_N& \\
    &=(\nabla_{AD}\mathcal{H}_{NO})^\top
    \begin{pmatrix}
            \int \phi_1^2\mathrm{d}x & \cdots & \int \phi_1 \phi_N\mathrm{d}x\\
            \vdots & \ddots & \vdots \\
            \int\phi_1 \phi_N\mathrm{d}x & \cdots & \int \phi_N^2\mathrm{d}x
    \end{pmatrix}^{-1}
    \begin{pmatrix}
            \phi_1 \\
            \vdots \\
            \phi_N
    \end{pmatrix}.
\end{aligned}
\end{equation*}
$\hfill\square$
\end{proof}

\section{Numerical Example: Wave Equations}
In this section, we demonstrate the effectiveness of our proposed learning method by using the linear wave equation 
$$
\dfrac{\partial^2 u}{\partial t^2} = \dfrac{\partial^2 u}{\partial x^2}
$$ 
as an example. The wave equation can be written in the Hamiltonian form as follows: %which has the form
\begin{equation*}
    \dfrac{\partial w}{\partial t} = 
    \begin{pmatrix}
        0 & 1 \\ -1 & 0
    \end{pmatrix}
    \dfrac{\delta \mathcal{H}}{\delta w},\quad w =
    \begin{pmatrix}
        u \\ u_t%:=
        %\dfrac{\partial u}{\partial t}
    \end{pmatrix}.
\end{equation*}
\textbf{Data}
We considered a wave system whose Hamiltonian is known as $\mathcal{H}=\dfrac{1}{2}(u_x^2+u_t^2)$. The time-space region is set to %$\mathcal{T}\times\mathcal{X}=$
$[0,2]\times[0,1]$. %\in\mathbb{R}^{200\times100}$, and 
The initial condition and the boundary condition is set to $u(0,x)=f(x),\\u_t(0,x)=0,u(t,0)=u(t,1)$. Under these conditions, the solution of the wave equation is known as $u(t,x)=\dfrac{1}{2}(f(t+x)+f(t-x))$. 
We used %generated 
a dataset of the solutions for 1000 different $f(x)$.

\textbf{Training}
We adopted two multi-layer perceptrons (MLPs) to implement DeepONet. The networks has 3 layers, 200 hidden units, and the tanh activation functions. We used the Adam optimizer to minimize the loss function of DeepONet. The learning rate is set to $10^{-4}$, and the epochs of iteration is 20000.

\textbf{Results}
Figure \ref{fig:fig1} shows the true solution $u$ and $u_t$, and Figure \ref{fig:fig2} shows $u_{NO}$ and ${u_t}_{NO}$ learned by our model. Figure \ref{fig:fig3} shows the behavior of the Hamiltonian. 
%由图,DeepONet学习到的方程解和真实解非常相似,但随着时间有发散的趋势,这可能是由于数值积分法的精度不够。同样,DeepONet输出的Hamiltonian基本随时间守恒,但由物理量直接计算的Hamiltonian随时间发散。
The solution predicted by DeepONet is very similar to the true solution, but they tend to diverge over time. Also, the Hamiltonian directly computed from the physical quantities exhibits similar behavior,  which may be due to the insufficient number of data. %accuracy of the numerical integration method. 
\begin{figure}[H]
	\begin{center}
	\includegraphics[width=0.7\textwidth]{true.png}
	\end{center}
	\caption{True $u$ and $u_t$ of the equation.}
	\label{fig:fig1}
\end{figure}

\begin{figure}[H]
	\begin{center}
	\includegraphics[width=0.7\textwidth]{net.png}
	\end{center}
	\caption{Learned $u_{NO}$ and ${u_t}_{NO}$ of the equation.}
	\label{fig:fig2}
\end{figure}

\begin{figure}[H]
	\begin{center}
	\includegraphics[width=0.4\textwidth]{hami.png}
	\end{center}
	\caption{The behavior of Hamiltonian. Red: $\mathcal{H}_{NO}(u,u_t)$, Blue: $\mathcal{H}(u_{NO},{u_t}_{NO})$.}
	\label{fig:fig3}
\end{figure}
%\section{Figures}
%See Figure \ref{fig:fig1} for example. 

\section{Conclusion}
%我们提出了一种
We propose an operator learning approach for modeling wave equations. In particular, we present a method to compute the variational derivative of Hamiltonian from the gradient computed by the automatic differentiation algorithm. 
Experiments have demonstrated the effectiveness of our proposal in learning the operator of energy density. Further training with more data and higher-precision numerical integration are anticipated to improve the model.

\section*{Acknowledgments}
This work is supported by JST CREST Grant Number JPMJCR1914, JST ASPIRE JPMJAP2329 and JST SAKIGAKE JPMJPR21C7.

\bibliographystyle{plain}
%\bibliography{references}
%\bibliographystyle{plainnat}
\begin{thebibliography}{10}
% ------------------------------
%
\bibitem{pde}
Strauss W A. Partial differential equations: An introduction, John Wiley \& Sons, 2007.
\bibitem{diff}
Wu Z, Zhao J, Yin J, et al. Nonlinear diffusion equations, 2001.
\bibitem{thermal}
Carthel C, Glowinski R, Lions J L. On exact and approximate boundary controllabilities for the heat equation: a numerical approach, Journal of optimization theory and applications, 82: 429-484, 1994.
\bibitem{wave}
Durran D R. Numerical methods for wave equations in geophysical fluid dynamics, Springer Science and Business Media, 2013.
\bibitem{weather}
Lynch P. The origins of computer weather prediction and climate modeling, Journal of computational physics, 227(7): 3431-3444, 2008. 
\bibitem{material}
Mazumder J, Steen W M. Heat transfer model for CW laser material processing, Journal of Applied Physics, 51(2): 941-947, 1980.
\bibitem{air}
Stevens B L, Lewis F L, Johnson E N. Aircraft control and simulation: dynamics, controls design, and autonomous systems, John Wiley \& Sons, 2015.
\bibitem{hnn}
Samuel Greydanus, Misko Dzamba and Jason Yosinski, Hamiltonian neural networks, Advances in neural information processing systems, 32, 2019.
\bibitem{symp}
Jin P, Zhang Z, Zhu A, et al. SympNets: Intrinsic structure-preserving symplectic networks for identifying Hamiltonian systems, Neural Networks, 132: 166-179, 2020.
\bibitem{no}
Kovachki, Nikola, et al. Neural operator: Learning maps between function spaces with applications to pdes, Journal of Machine Learning Research 24.89 : 1-97, 2023.
\bibitem{don}
Lu Lu, Pengzhan Jin, Guofei Pang, Zhongqiang Zhang and George Em Karniadakis, Learning nonlinear operators via DeepONet based on the universal approximation theorem of operators, Nature machine intelligence, 3(3), 218-229, 2021.
%
%\bibitem{3} 
%Author1, Author2 and Author3, Paper Title, Journal Name, Vol. (Year), $\ast\ast$--$\ast\ast$.
%
%\bibitem{4}
%Author, Book Title, Publisher, Year.
%
%\bibitem{5}
%Author1 and Author2, Paper Title, in: Proc. of Proceedings Name, Vol. $\ast\ast$, pp. $\ast\ast$--$\ast\ast$, Year.
%
%\bibitem{6}
%JSIAM Web Page, \url{https://jsiam.org/}.
\end{thebibliography}
\end{document}
