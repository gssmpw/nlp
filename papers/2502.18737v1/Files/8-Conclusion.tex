\section{Conclusion}

GenAI offers immense potential for augmenting content creation, but challenges around intent elicitation and alignment, prompt formulation, and workflow flexibility impede effective human-AI co-creation. To address these issues, we propose intent tagging: graphical micro-prompting interactions for supporting granular and non-linear workflows with GenAI systems in the context of slide deck creation. To explore its benefits and challenges, we developed IntentTagger, an intent tagging-based LLM-driven system allowing users to iteratively create and modify slide deck presentations, and conducted a user study. Our findings revealed that users preferred intent tag-based interactions over chat- and gallery-based systems, valuing the system’s ability to support non-linear workflows, flexible intent expression, and integrated suggestions that helped them clarify their goals and think through the slide creation tasks. Based on these findings, we discuss design considerations for integrating intent tagging into GenAI-assisted slide authoring. Although our system and study primarily focus on slide creation, intent tags seem to present a design pattern that researchers could explore further in other AI-assisted workflows and scenarios, thereby offering designers a new interaction technique for HCI+AI applications. 

