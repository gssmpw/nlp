\section{Discussion and Design Considerations}


\rev{In the following sections,} we discuss our explorations' learnings from the user study and UX scenarios, reflecting on \textit{Intent Tagging as an enabler of non-linear GenAI workflows} and its role in \textit{promoting Human-AI Co-creation and reflection}. Further, we will discuss the challenges of \textit{blending steering interactions alongside manual content editing} and, lastly, reflect on \textit{trade-offs between UI customization and user distractions}. 
Each section discusses challenges, design considerations, and opportunities for future work. 


\subsection{Intent Tagging Enables Non-Linear GenAI Workflows}

Our findings demonstrate how IntentTagger was able to facilitate flexible, non-linear, and non-destructive workflows with high degrees of freedom: Once users became comfortable with the tool, they were able to seamlessly switch between the Deck Steering Canvas, Slide Overlay, and Outline Editor and steering the GenAI system without having to follow a rigid sequence (see 6.2). 
However, this non-linear steerability comes with a cost: While participants valued the system’s flexibility, they also expressed frustration about a need for greater control by "locking" or "pining" specific content or design elements, such as background colors and fonts, once satisfied to prevent unintended changes during further generation cycles (see 6.5).
While it is a powerful concept to enable multi-modal and multi-directional steering of GenAI content generation between tag, outline, slide, and deck levels, we currently lack interfaces allowing users to gradually lock elements across such dimensions. 
As a starting point, future research in this direction could be partially informed by design space exploration and parametric design systems that support users navigating and gradually reducing the degree of freedom across large option sets \cite{zaman_gemni_2015, matejka_dream_2018}.




In terms of user experience, Intent Tagging presents a novel way of steering GenAI systems and creating slide decks that are new to users. 
Our findings showed that while all participants appreciated the flexibility to specify tags with different levels of specificity, many were initially unsure about the tags' correct terminology and group associations (see 6.2.2).
This uncertainty posed initial challenges, which participants quickly overcame after experiencing the systems' flexibility in interpreting their intent after the first generations. 
To better support user onboarding, we suggest that future intent tagging-based interfaces should more explicitly scaffold that learning process, for example, by partially pre-defining common tag labels (such as \textit{"audience"} or \textit{"length"}) or offering optional "tag templates" for specific content creation scenarios to help users better get started and learn intent tagging.   






\subsection{Intent Tagging Promotes Human-AI Co-creation and Reflection-In-Action }

Figuring out one's intentions and requirements is often the hardest part of complex tasks like slide deck creation. 
Generally, \textit{Schön} describes this process as “reflection-in-action” \cite{schon_reflective_1983}, where creators interact with the material at hand while the material "talks back" to them, creating an iterative process of acting and reflection. 
When intent tagging, the tags on the canvas talk back to the user, whether self-generated or system-suggested, helping them (along with the outline and slides) tackle the cognitively demanding task of clarifying their intent. 
Furthermore, the more task-relevant tags a user specifies, the more accurately the system can provide contextually tailored suggestions, better aligning with the user’s objectives. 
As users add more tags, they convey a fuller picture to the GenAI system, which in turn enhances the quality and precision of the generated content. 
The system and user engage in a close feedback loop, allowing users to refine their intentions while the AI system responds dynamically---like connecting a mind map to a GenAI content creation system. 
This process relates closely to previously framed ``enactive'' models of Human-AI co-creation \cite{davis_enactive_2015}, relating to the theory that cognition arises from improvised interactions shaped by feedback from the surrounding environment.

\rev{In this version of IntentTagger, we decided that users must explicitly request tag and image suggestions from the system. 
On the one hand, this approach ensures persistency between requests, allowing users to explore suggested tags without the risk of suggestions disappearing without their consent.
On the other hand, tag suggestions also offer an exciting possibility of being combined with mixed-initiative interactions in the future, where the system can automatically initiate tag suggestions in response to the user's active tags \cite{lin_prompts_2023}.}

\rev{In terms of user-system communication, in this study, we focused on analyzing users’ general interactions with system-generated tags (see 6.2.3). 
Future work could delve into a deeper semantic analysis, exploring how users expressed their intent and what specific tags the system suggested in response. Such insights could further improve the effectiveness of tag-based human-AI co-creation systems.}

Furthermore, while many participants experimented with the \textit{Opposite Slider widget}, no one used it to steer slide generation (see 6.2.3). 
However, several participants voiced the clear value of this feature for prompting reflection on their design choices (see 6.6). 
This suggests that, while not used directly for steering generation, the slider still serves as a valuable tool for encouraging and steering deeper thinking. 
This bears interesting further implications for designing future GenAI steering interfaces, which might consider integrating steering controls alongside "intent exploration" UI to promote reflection independent of content generation.

Our findings also revealed that users often got inspired by system suggestions but then manually overrode the tag's value field to better align with their vision. 
This interaction pattern hints at an effective combination of guided intent exploration through system-generated options alongside the flexibility for users to override suggestions if desired. 
Adding to this observation, a recent study in the context of AI-assisted writing found that participants, despite the many integrated AI suggestions, felt ownership over the content \textit{"because of the numerous authorial decisions"} they made \cite{singh_where_2023}. Together, such findings suggest that it is critical for effective human-AI co-creation to enable users to manually edit and override system suggestions to foster a sense of ownership and control. 

Our research indicates that user interactions with IntentTagger's LLM-generated \textit{tag suggestions}, \textit{tag grounding}, and \textit{pre-generated real-time previews} can help align user expectations with GenAI systems' capabilities and recognize their limitations (see 6.4). While some suggested tags referred to aspects beyond the slide generation capabilities (such as icons or background textures), interestingly, users were not significantly frustrated by that. 
Instead, they found the suggestions helpful for improving their presentations, regardless of whether the system could fully execute them. 
This tolerance was likely influenced by the pre-generated real-time tooltip previews, which helped users quickly grasp the system’s limitations without frustration, as they had not invested much time in these explorations.

Adding to previous work on surfacing affordances and providing feed-forward mechanisms~\cite{vermeulen2013crossing, boy2015suggested, TerryCreativeNeedsUIDesign2002}, we promote dynamically \textbf{pre-}generating slide previews (or other content) on a granular level (such as only alternating a slide's font or text format) to support users’ intent exploration and foster greater tolerance when GenAI falls short by reducing emotional investment in misaligned expectations. 
Additionally, \textit{tag grounding}---such as the grounding from text prompt or slide features---might also improve alignment between user expectations and system capabilities by helping users quickly understand how the system interprets their inputs. 
\rev{Lastly, presenting distinct differences between versions of generated content may further enhance reflection-in-action during the creation process, building on insights from Drucker et al. \cite{drucker_comparing_2006}. }
All these mechanisms show promising potential for enhancing user alignment in GenAI workflows.




\subsection{\textbf{Blending Steering Interactions and Manual Content Editing }}  

Our prototype system is designed to probe the concept of intent tagging-based interactions for steering slide deck generation and, therefore, does not support rich content editing that full-fledged applications such as PowerPoint offer. 
In the study, the system allowed participants to create slide decks with GenAI support with more control and satisfaction than with the chat-based system. 
However, going further, we envision that IntentTagger would also integrate rich content editing capabilities of slides in addition to AI-driven generation. Enabling users to interweave manual and AI-driven editing activities opens up many interesting questions relating to the orchestration of the hand-off between user and system. 
For example, devising mechanisms for identifying situations in which steering GenAI actually becomes more tedious than directly editing content and surfacing cues to users nudging them towards direct editing is an intriguing direction to explore. 


\subsection{\textbf{Trade-offs between UI Customization and UI Management }}

Lastly, we reflect on requests we gathered from participants to enable them to customize IntentTagger more. A salient example is the requests from a few participants for enabling customizable groupings of tags rather than the three fixed groups we offered (see 6.4). While relatively straightforward to achieve, such added capability may come as a tradeoff between user control of the interface and its management. 



While giving users the ability to create their own tag groups' structure would allow for greater flexibility---especially for catering to users' specific needs and creation process---it would also add an additional cognitive layer by requiring users to reflect on what such process is and how to best articulate it in the interface. 
This leads to the classic question of identifying which interface aspects are best given to the users to shape and which to leave in the hands of design and system architects. 
Probing this delicate balance between supporting flexibility and minimizing effort, enabling rich interactions, and making them simple to use in the context of generative AI paves the way for future research in our field.


\subsection{\rev{\textbf{Limitations of the Evaluation}}}
\rev{We highlight three limitations of this work:
Firstly, all participants were recruited from the same large technology company, and while they represent a diverse set of job titles and backgrounds, it is possible that participants from other companies or from non-corporate environments might provide additional insights not captured in the current study. 
Secondly, the study's lab setting, limited task times, and constrained number of slides allowed us to study Intent Tagging in a controlled and concise format. 
However, future work is needed to study the impacts of Intent Tags on creation processes \textit{in-vivo} and over \textit{longer periods of time} to gain further insights into real-life content creation scenarios. 
Lastly, we deliberately limited our comparison of IntentTagger with the widespread commercial Copilot and Designer features in PowerPoint. 
Future comparisons with other tools and interfaces may surface a broader range of benefits and tradeoffs.}


