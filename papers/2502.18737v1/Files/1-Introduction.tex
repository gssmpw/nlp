
\section{Introduction}


Generative AI (GenAI) models have become increasingly powerful in content creation tasks, and are rapidly being integrated into a range of professional applications, including coding IDEs \cite{github_github_2024, chen_evaluating_2021}, text processors \cite{grammarly_grammarly_2024, brown_language_2020}, image and video editors \cite{adobe_adobe_2024, saharia_photorealistic_2022, ho_imagen_2022}, and office suites \cite{microsoft_powerpoint_2024b}. These developments offer tremendous promise for AI as a co-creation tool steered by people, but significant challenges remain:


\begin{enumerate}
    \item It is hard to \textbf{align AI-generated content with user intentions} \textit{(intent elicitation and alignment)} \cite{chen_evaluating_2021, terry_ai_2023}, often due to the difficulty in balancing the flexibility of free-form text prompts with guided interfaces;
    \item Users often struggle with \textbf{formulating effective prompts and understanding an AI system’s capabilities}, leading to trial-and-error interactions \textit{(prompt formulation)} \cite{zamfirescu-pereira_why_2023, subramonyam_bridging_2024, mahdavi_goloujeh_is_2024};
    \item The iterative nature of content creation, which requires continuous reflection and refinement, further complicates intent elicitation and prompt formulation since \textbf{users’ needs and intents aren’t often completely clear upfront} \cite{BuxtonSketchingUserExperiences2007, DesignReflectiveConversation1992, TerryCreativeNeedsUIDesign2002, shneiderman_creativity_2007}; 
    \item Many \textbf{GenAI systems impose workflows} that force users to adapt their commonly used creative processes or generate content with each refinement, further complicating human-AI collaboration \textit{(workflow flexibility)} \cite{sarkar_exploring_2023, tankelevitch_metacognitive_2023}.
\end{enumerate}

In this work, we explore potential solutions to these challenges around \textit{intent elicitation and alignment}, \textit{prompt formulation}, and \textit{workflow flexibility} through the lens of GenAI-supported rich content creation tasks, such as slide deck creation.
Slide deck creation involves complex decisions around content structure, visual style, media content, slide sequencing, and narrative flow, all influenced by implicit factors like audience or presentation duration.
Often, slide decks reference external documents, evolve over time,  and may be duplicated and changed for different audiences and purposes.

Technically, GenAI multi-modal foundation models offer powerful capabilities to support such complex content creation tasks, for example, by transforming documents or synthesizing various content sources into a single slide deck \cite{costa_smartedu_2023, Fu2022DOC2PPT, winters_automatically_2019a}. 
However, most human-GenAI interaction challenges are amplified in the context of such complex content creation scenarios---the diverse and non-linear workflows in slide deck creation, coupled with the difficulty of expressing requirements upfront, make it challenging for users to effectively integrate GenAI into their specific creation process.



\begin{table*}
    \centering
    \begin{tabular}{lcccc}
    \toprule
         \textit{Intent Elicitation Interface} &  

         \begin{tabular}[c]{@{}l@{}}
            \textbf{Open Prompt}  \\ \textbf{Format}
         \end{tabular} & 

          \begin{tabular}[c]{@{}l@{}}
            \textbf{System }  \\ \textbf{Guidance}
         \end{tabular} &
         
         \begin{tabular}[c]{@{}l@{}}
                \textbf{GUI-based Option} \\               
                \textbf{Manipulation} 
        \end{tabular} &  
         
         \begin{tabular}[c]{@{}l@{}}
            \textbf{Continuous Option}\\ \textbf{Representation}
         \end{tabular} \\
         \midrule
         \textbf{\textit{GUI-based / Wizard Dialogues}}&  &   \checkmark&  \checkmark &   \checkmark \\
         \textbf{\textit{Text Prompting}}&  \checkmark&  &  &   \\
         \textbf{\textit{Chat Dialogues}}&  &  \checkmark&  &   \\
         \textbf{\textit{Intent Tagging}}&  \checkmark&  \checkmark&  \checkmark&  \checkmark \\
         \bottomrule
    \end{tabular}
    \caption{Outlining the trade-offs between four intent elicitation methods (GUI-based/Wizard Dialogues, Text Prompting, Chat Dialogues, and Intent Tagging); comparing the factors Open Prompt Format (allowing users to express intent freely), System Guidance (guiding users through predefined options), GUI-based Option Manipulation (adjusting single parameters via GUI elements), and Continuous Option Representation (persistent, single mutable objects).}
    \label{tab:ie_comparision}
    \Description{The table outlines the trade-offs between four intent elicitation methods: GUI-based/Wizard Dialogues, Text Prompting, Chat Dialogues, and Intent Tagging. It compares these methods across four criteria: Open Prompt Format (allowing users to express intent freely), System Guidance (guiding users through predefined options), GUI-based Option Manipulation (adjusting single parameters via GUI elements), and Continuous Option Representation (persistent, single mutable objects). Each method is marked with checkmarks to indicate the criteria it supports. For example, Text Prompting supports Open Prompt Format, while GUI-based/Wizard Dialogues supports both System Guidance and GUI-based Option Manipulation.}
\end{table*}










To address these challenges and empower users in rich AI-assisted content creation tasks, we propose \textbf{\textit{intent tags}} as a granular and flexible technique for GenAI-supported workflows via graphical micro-prompting. 
Intent tags represent atomic conceptual units that encapsulate a single aspect of a user's intent. Intent tags help users steer content generation in non-linear workflows---the user instantiates multiple intent tags and edits each individually through adaptive UI elements that allow for granular control of content generation and transparent intent elicitation.
To support a creative design process where users rapidly explore multiple alternatives, iterate, and reflect on the outcomes, our system proactively surfaces AI-generated suggestions through intent tags. These suggestions appear in an unobtrusive manner, allowing people to see the parameter spectrum ~\cite{TerryCreativeNeedsUIDesign2002} of possible outcomes that they might not have considered or been aware of and use them to iteratively refine the system's outputs as desired. 

To explore further possibilities of intent tagging for human-GenAI co-creation workflows for rich content creation, we created \textbf{IntentTagger}: a GenAI-driven system allowing users to iteratively create and modify slide deck presentations using intent tags on a 2D canvas interface. Through intent tags, users steer the generation of single slides or the entire deck by defining keywords \textit{(concept tags)} or including content from other documents or images \textit{(reference tags)}. 
Under the hood, IntentTagger utilizes an LLM for slide generation. IntentTagger also dynamically generates adaptive UI elements such as context-related \textit{tag suggestions}, dynamic \textit{drop-down lists} with \textit{slide preview tooltips}, and interactive \textit{slider widgets} to help users fine-tune tag expressions.    

We explore the benefits and limitations of intent tagging through a \textbf{lab user study} with 12 participants. 
The study comprised \rev{comparative} closed-ended and \rev{semi-open-ended} slide-deck-related tasks that participants completed using our prototype system and, in some comparative tasks, using GenAI features from an existing commercial slide authoring tool. 
Our findings indicate that users felt more in control and satisfied with intent tag-based interactions than with existing chat-based and design gallery-based generative AI systems for slide deck creation.
Participants especially appreciated the \textit{\textbf{support of non-linear and iterative workflows}}, the ability to \textbf{\textit{express their intent in flexible ways}} with \textbf{\textit{varying levels of ambiguity}}, and the \textbf{\textit{integrated system suggestions}} as a valuable and non-distractive aid for \textbf{\textit{helping them think through the task}} and \textbf{\textit{figure out what they want}} while working on the task.

Building upon these study findings, we illustrate how intent tagging could be generally utilized for facilitating human-GenAI co-creation interactions across diverse rich content creation tasks beyond slide deck creation. 
We sketch several \textbf{user experience (UX) interface scenarios} for applications such as web blogging, video creation, and 3D scene creation.
Based on our prototype system, the study's findings, and the alternative UX scenarios, we discuss potential design considerations and future work for intent tag-based interactive generative AI systems. 



In sum, this paper makes three main contributions:
\begin{enumerate}
    \item \textit{Intent tagging}: graphical micro-prompting interactions to support granular and non-linear co-creation workflows with generative AI systems in the context of slide deck creation and other rich content creation tasks;

    \item \textit{User study insights on intent tagging for human-AI interaction} in steerability, content creation workflows, and "meta-intent elicitation" by utilizing a GenAI-driven slide creation system called IntentTagger;

    \item \textit{Design considerations for intent tagging} in GenAI-assisted slide authoring applications in particular, and potentially as a new interaction technique for AI-assisted creative design tasks in general. 

\end{enumerate}

Our present work focuses on using intent tags to support the creative design activity of slide deck authoring, which we establish in depth throughout this paper. However, intent tags appear to offer a straightforward design pattern that researchers could explore further in other AI-assisted workflows and scenarios, thereby offering designers a new interaction technique for HCI+AI applications.



