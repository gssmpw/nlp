

\section{IntentTagger: An Intent Tagging-based Slide Creation System}

Driven by \rev{the requirements of rich content creation tasks and} the design principles described in the previous section, we explored novel input strategies through graphical micro-prompting interactions---an interaction notion we coined \textit{intent tagging}. 
\rev{In addition to our design principles, a central inspiration for conceptualizing creative intent elicitation through granular and flexible micro prompts that represent users’ intention was the ``mood board'' technique, in which creatives iteratively compile visual and textual elements into a collage to convey an overarching theme or creative direction \cite{garner_problem_2001}. 
Similar to how mood boards enable capturing and expressing designers’ intent through compiling different media with varying levels of abstractions---such as words and pictures ranging from abstract concepts, stylistic references, or specific material details---intent tagging allows users to express their intent openly through collections of intent tags.}

\rev{To explore further the idea of intent tagging,} we created \textit{IntentTagger}, an LLM-driven system allowing users to iteratively create and modify slide deck presentations using intent tags on a 2D canvas interface (see Figure \ref{fig:system-ui}). 
In this section, we demonstrate the utility of \textit{IntentTagger} by first illustrating its functionalities in a use-case scenario. We then follow with a detailed description of the proposed interface mechanisms and their implementation.





\subsection{Example Use Case Scenario}

Lucy is a sustainability manager at a company and wants to prepare a short slide presentation for new employees to inform them about sustainability initiatives at her company. 

Lucy starts IntentTagger and clicks on the \textit{"Create a new presentation from prompt..."} button. 
A modal popup appears, and she types \textit{"a presentation for introducing sustainability initiatives for new employees"} in the prompt field and presses enter. 
The modal disappears, and the three circular steering groups on the steering canvas pulse, then three concept tags appear in the \textit{"Narrative"} circle with the parameters \textit{"Topic: Sustainability Initiatives," "Audience: New Employees"} and \textit{"Purpose: Introduction."} 

\textbf{Tag Suggestions:} To get inspiration for further presentation talking points, she clicks \textit{"Suggest More Tags,"} and shortly after, new tags appear outside of the circular tag groups. Lucy finds some of the suggested tags useful and drags these into the \textit{"Narrative"} group to include them in the generation, such as \textit{"Focus: Eco-friendly Practices," "Highlight: Company Goals," "Section: Key Initiatives,"} and \textit{"Objective: Corporate Responsibility."}

\textbf{Including External Content:}
Lucy also wants to include sections from an existing Word document detailing the company's policies. With the mouse cursor, she drags the document from her file browser onto the IntentTagger app, and a new reference tag appears in the \textit{"Content Sources"} group. She clicks on the tag's \textit{"Select Sections"} button, which opens up the tag's Text Selection Widget showing the document's content. In the widget, she highlights two document sections, clicks \textit{"Include Section in Presentation"} from the widget's context menu, and closes the widget. 

\textbf{Generating an Outline:}
Next, she wants to generate a first draft of the presentation's outline and presses \textit{"update outline from intent tags"} in the outline view panel. The tag groups start pulsing, and then the new outline appears in the outline text editor. 

\textbf{Manually Crafting Concept Tags:}
Lucy is happy with the structure overall but finds it too long (20 sections). Instead of manually editing the outline, she adds a new concept tag to the \textit{"Narrative"} group and enters \textit{"Number of slides"} into the tag's upper text field and \textit{"10"} into the lower and clicks the \textit{"update outline from intent tags"} button again. The newly generated outline is now shorter and only comprises ten sections. 

\textbf{Generating Slides:}
Next, Lucy wants to generate a first draft of the slide deck and clicks the \textit{"Update slides from intent tags"} button. Shortly after, the generated slides appear in the slide preview side panel. After evaluating the slides, Lucy realizes that adding some images and adjusting the visual style could enhance the presentation's appeal. 

\textbf{Including Suggested Images:}
She clicks the \textit{"suggest more images"} button on the tag steering board, and shortly after, new Reference Image Tags appear outside the \textit{"Content Sources"} tag group with eco-friendly-themed stock photos. Lucy drags some of the image tags into the \textit{"Content Sources"} group to add them to the presentation.   

\textbf{Adjusting the Visual Style through Tags' Alternatives Drop-down:}
To make visual adjustments to the slides, she drags in some of the previously suggested tags, such as \textit{"Typography: Sans-Serif," "ColorPalette: Green and Blue,"} and \textit{"Theme: Nature"} onto the \textit{"Visual Style"} tag group. To get a better sense of typography alternatives, she clicks on the tags' drop-down, and a list of alternative tag values appears, such as \textit{"Serif," "Monospace,"} and \textit{"Handwritten."} While hovering over each option with the mouse, Lucy sees a preview thumbnail of that font style applied to the current slide. Lucy chooses \textit{"Monospace."} Now, Lucy clicks \textit{"Update slides from intent tags"} again, and shortly after, she sees the revised slides reflecting the intended new visual style, including the images. 

\textbf{Adjusting a Single Slide with Tag Steering Overlay:}
Lucy decides to refine the slide deck further. She clicks on the third slide in the \textit{Slide Preview} panel to open the slide edit view and review its content in detail. 
The slide contains a text paragraph and an image, but she wants to show a list of bullet points instead. 
To modify the slide, she clicks on the intent tag icon next to the slide, and the three steering tag groups appear as an overlay containing tags representing the slide's \textit{Narrative}, \textit{Visual Style}, and \textit{Content Sources}. 
Lucy selects the \textit{"Text Format"} tag inside the \textit{Narrative} group and clicks on its drop-down widget. 
A list of alternative tag values appears, including \textit{"Bullet Points," "List,"} and \textit{"Table."} Lucy chooses \textit{"Bullet Points."} 
To adjust the background color, Lucy creates a new tag in the \textit{Visual Style} group and types in \textit{"Background"} and \textit{"pastel color."} 
Now, she clicks on the \textit{"Explore slide variations"} button, and the system generates several alternative versions of the slide, each with slightly different text variations and pastel background colors.  
Lucy clicks through these variations and selects one that better aligns with her presentation’s tone.

\textbf{Applying a Slide's Style to All Slides:}
To maintain consistency across the entire presentation, Lucy uses the same visual style and selects \textit{"Apply to all slides."} 

Satisfied with the changes, Lucy saves the presentation, ready to deliver it in the upcoming meeting with her colleagues.

\textbf{Adjusting the Presentation to New Requirements:}
After a couple of months, Lucy needs to adapt the presentation for a different audience—existing employees rather than new hires. She clicks on the \textit{"Audience"} tag in the \textit{Narrative} group and changes it from \textit{"New Employees"} to \textit{"Existing Employees."} The system instantly updates the content and tone of the slides to better fit this new audience. For example, introductory sections are replaced with more in-depth discussions of ongoing sustainability projects and their impact.






\subsection{Interface Features}

Here, we provide a more detailed explanation of IntentTagger's features:

\subsubsection{\textbf{Intent Tags: Concept Tags, Reference Tags and Groups (DP2)}}
Intent tags are collections of single keywords, phrases, or media artifacts describing users' intended outcomes. 
Tags serve as granular and flexible micro prompts on a zoomable 2D canvas. 
Users can steer the generative slide creation system through two types of tags:




\aptLtoX[graphic=no,type=html]{\begin{figure}[H]
    \includegraphics[align=t,width=\columnwidth]{Figures/Feature-ConceptTag.png}
    \caption{ \ ~ \ ~ \ ~ \ ~ \ }
    \label{fig:concept-tag}
    \Description{The figure shows two examples of concept tags with a label and value format. The first example is an active tag labeled “Topic” with the value “Product Launch,” displayed with a bold border. The second example is an inactive tag with the same label and value, but shown with a lighter, grayed-out appearance, indicating that it is not currently influencing the content generation. Both tags have a drop-down arrow for selecting or modifying the value.}
\end{figure}
          \textbf{(1) Concept Tags:} Concept Tags are two-part text micro prompts of the format [label: value], such as \textit{”Topic: Product launch”} or \textit{”Color scheme: Corporate.”} Tags can be active or inactive to toggle their influence on generated content.
}{
\begin{figure}[H]
  \begin{minipage}[t]{3cm}
    \includegraphics[align=t,width=\columnwidth]{Figures/Feature-ConceptTag.png}
    \caption{}
    \label{fig:concept-tag}
    \Description{The figure shows two examples of concept tags with a label and value format. The first example is an active tag labeled “Topic” with the value “Product Launch,” displayed with a bold border. The second example is an inactive tag with the same label and value, but shown with a lighter, grayed-out appearance, indicating that it is not currently influencing the content generation. Both tags have a drop-down arrow for selecting or modifying the value.}
  \end{minipage}
   \begin{minipage}[t]{\dimexpr\columnwidth-3.2cm\relax}
          \textbf{(1) Concept Tags:} Concept Tags are two-part text micro prompts of the format [label: value], such as \textit{”Topic: Product launch”} or \textit{”Color scheme: Corporate.”} Tags can be active or inactive to toggle their influence on generated content.
  \end{minipage}
\end{figure}}


\aptLtoX[graphic=no,type=html]{\begin{figure}[H]
%  \begin{minipage}[t]{3cm}
    \includegraphics[align=t,width=\columnwidth]{Figures/Feature-ReferenceTags.png}
    \caption{ \ ~ \ ~ \ ~ \ ~ \ }
    \label{fig:reference-tags}
    \Description{The figure shows two examples of reference tags. The first tag represents a Word document titled “Product Launch Infos,” with an icon of a document beside it. The second tag represents a slide deck template titled “Corporate Identity Deck Template,” displayed with a thumbnail of a presentation slide and an image of a 3D headset. Both tags are visually enclosed in boxes, indicating that these external files are used as references for slide generation.}
%  \end{minipage}
%   \begin{minipage}[t]{\dimexpr\columnwidth-3.2cm\relax}
%  \end{minipage}
\end{figure}
\textbf{(2) Reference Tags:} \textit{Reference Tags} allow users to provide external documents such as Word documents, images, or other slide decks as references for slide generation. Users can add these files by drag-and-drop from their file browser.  }{
\begin{figure}[H]
  \begin{minipage}[t]{3cm}
    \includegraphics[align=t,width=\columnwidth]{Figures/Feature-ReferenceTags.png}
    \caption{}
    \label{fig:reference-tags}
    \Description{The figure shows two examples of reference tags. The first tag represents a Word document titled “Product Launch Infos,” with an icon of a document beside it. The second tag represents a slide deck template titled “Corporate Identity Deck Template,” displayed with a thumbnail of a presentation slide and an image of a 3D headset. Both tags are visually enclosed in boxes, indicating that these external files are used as references for slide generation.}
  \end{minipage}
   \begin{minipage}[t]{\dimexpr\columnwidth-3.2cm\relax}
          \textbf{(2) Reference Tags:} \textit{Reference Tags} allow users to provide external documents such as Word documents, images, or other slide decks as references for slide generation. Users can add these files by drag-and-drop from their file browser.  
  \end{minipage}
\end{figure}}


\aptLtoX[graphic=no,type=html]{\begin{figure}[H]
%  \begin{minipage}[t]{3cm}
    \includegraphics[align=t,width=\columnwidth]{Figures/Feature-TagGroups.png}
    \caption{ \ ~ \ ~ \ ~ \ ~ \ }
    \label{fig:tag-groups}
    \Description{The figure displays a circular group labeled “Narrative,” containing an active concept tag with the topic “Learn to code.” Above the active tag, an inactive tag is dragged into the circular group  reading “Introduction” and “Basic Syntax,” with a cursor pointing to “Basic Syntax.” The tags are clustered visually within the circular group, demonstrating how tags can be organized and managed within the interface. Users can drag and drop tags into or out of the group to activate or deactivate them.}
%  \end{minipage}
%   \begin{minipage}[t]{\dimexpr\columnwidth-3.2cm\relax}
%  \end{minipage}
\end{figure}          \textbf{Tag Groups}: All active tags are visually clustered into three circular groups on the interface, each representing different slide creation aspects: \textit{Narrative}, \textit{Visual Style}, and \textit{Content Sources}. Users can drag tags in and out of groups to activate and deactivate them. 
}{\begin{figure}[H]
  \begin{minipage}[t]{3cm}
    \includegraphics[align=t,width=\columnwidth]{Figures/Feature-TagGroups.png}
    \caption{}
    \label{fig:tag-groups}
    \Description{The figure displays a circular group labeled “Narrative,” containing an active concept tag with the topic “Learn to code.” Above the active tag, an inactive tag is dragged into the circular group  reading “Introduction” and “Basic Syntax,” with a cursor pointing to “Basic Syntax.” The tags are clustered visually within the circular group, demonstrating how tags can be organized and managed within the interface. Users can drag and drop tags into or out of the group to activate or deactivate them.}
  \end{minipage}
   \begin{minipage}[t]{\dimexpr\columnwidth-3.2cm\relax}
          \textbf{Tag Groups}: All active tags are visually clustered into three circular groups on the interface, each representing different slide creation aspects: \textit{Narrative}, \textit{Visual Style}, and \textit{Content Sources}. Users can drag tags in and out of groups to activate and deactivate them. 
  \end{minipage}
\end{figure}
}





\subsubsection{\textbf{Deck Steering Canvas and Slide Steering Overlay (DP1, DP3)} }
 Users can generate and modify single slides or entire slide decks using Intent Tags.
The \textit{Deck Steering Canvas} allows users to steer the (re)generation of entire slide decks. For example, specifying the number of total slides or changing the overall tone of the deck's text from brief to more verbose. 
Alternatively, the \textit{Slide Steering Overlay} allows users to steer the (re)generation and explore alternatives of single slides. 


\subsubsection{\textbf{Outline Editor (DP1)}}

Parallel to slides, the system also generates a presentation outline, which can be optionally edited using the outline editor. This editor also allows participants to start a new presentation by first drafting an outline or letting the system generate an outline from provided intent tags.  

\subsubsection{\textbf{Editing Intent Tags, Tag Widgets, and System Suggestions (DP3, DP4, DP5)}}

Intent tags allow users to formulate their intent in several ways, such as manually creating and editing tags or choosing from options generated through LLM-driven dynamic adaptive UI mechanisms:


\aptLtoX[graphic=no,type=html]{\begin{figure}[H]
%  \begin{minipage}[t]{3cm}
    \includegraphics[align=t,width=\columnwidth]{Figures/Feature-EditTag.png}%
    \caption{ \ ~ \ ~ \ ~ \ ~ \ }
    \label{fig:crafting-tags}
    \Description{The figure shows the process of manually creating tags by typing into two text fields. The first example displays an incomplete tag with “New Tag” and an empty text field. Below it, examples show typed tags such as “Font: Modern” and “Font: Mode,” with the latter still being typed. Arrows indicate the progression of typing from incomplete to completed tags. This illustrates how users can manually define tag properties by entering custom labels and values.}
%  \end{minipage}
 %  \begin{minipage}[t]{\dimexpr\columnwidth-3.2cm\relax} 
%  \end{minipage}
\end{figure}          \textbf{Manually Crafting Tags}
          Users can manually create and modify tags by freely typing words or phrases into its two text fields, such as "Font: Modern" or "Tonality: Engaging."
}{\begin{figure}[H]
  \begin{minipage}[t]{3cm}
    \includegraphics[align=t,width=\columnwidth]{Figures/Feature-EditTag.png}%
    \caption{}
    \label{fig:crafting-tags}
    \Description{The figure shows the process of manually creating tags by typing into two text fields. The first example displays an incomplete tag with “New Tag” and an empty text field. Below it, examples show typed tags such as “Font: Modern” and “Font: Mode,” with the latter still being typed. Arrows indicate the progression of typing from incomplete to completed tags. This illustrates how users can manually define tag properties by entering custom labels and values.}
  \end{minipage}
   \begin{minipage}[t]{\dimexpr\columnwidth-3.2cm\relax}
   
          \textbf{Manually Crafting Tags}
          Users can manually create and modify tags by freely typing words or phrases into its two text fields, such as "Font: Modern" or "Tonality: Engaging."
  \end{minipage}
\end{figure}}

\aptLtoX[graphic=no,type=html]{
\begin{figure}[H]
%  \begin{minipage}[t]{3cm}
    \includegraphics[align=t,width=\columnwidth]{Figures/Feature-DropDown.png}%
    \caption{ \ ~ \ ~ \ ~ \ ~ \ }
    \label{fig:drop-down}
    \Description{The figure shows a drop-down list for the “Color Scheme” tag, currently set to “Dark and Light Blue.” Below it are alternative color scheme options such as “Teal and Coral,” “Purple and Yellow,” and “Green and Gold.” A slide preview tooltip appears next to the list, showing a thumbnail of a slide with the selected “Dark and Light Blue” scheme. This demonstrates how users can explore alternative properties for tags and view pre-generated visual previews in hover tooltips.}
  %\end{minipage}
   %\begin{minipage}[t]{\dimexpr\columnwidth-3.2cm\relax}
%  \end{minipage}
\end{figure}\textbf{Drop-Down List} 
Users can also explore alternative properties for each concept tag by browsing its dynamically generated drop-down list, which also provides pre-generated visual previews in a hover tooltip. }{\begin{figure}[H]
  \begin{minipage}[t]{3cm}
    \includegraphics[align=t,width=\columnwidth]{Figures/Feature-DropDown.png}%
    \caption{}
    \label{fig:drop-down}
    \Description{The figure shows a drop-down list for the “Color Scheme” tag, currently set to “Dark and Light Blue.” Below it are alternative color scheme options such as “Teal and Coral,” “Purple and Yellow,” and “Green and Gold.” A slide preview tooltip appears next to the list, showing a thumbnail of a slide with the selected “Dark and Light Blue” scheme. This demonstrates how users can explore alternative properties for tags and view pre-generated visual previews in hover tooltips.}
  \end{minipage}
   \begin{minipage}[t]{\dimexpr\columnwidth-3.2cm\relax}
   
          \textbf{Drop-Down List} 
Users can also explore alternative properties for each concept tag by browsing its dynamically generated drop-down list, which also provides pre-generated visual previews in a hover tooltip. 
  \end{minipage}
\end{figure}}



\aptLtoX[graphic=no,type=html]{\begin{figure}[H]
%  \begin{minipage}[t]{3cm}
    \includegraphics[align=t,width=\columnwidth]{Figures/Feature-Slider.png}%
    \caption{ \ ~ \ ~ \ ~ \ ~ \ }
    \label{fig:slider}
    \Description{The figure shows a typography tag with a slider widget. The current selection is “Typography: Modern,” and the opposite value is “Typography: Traditional.” The slider allows users to adjust between the two extremes, with a five-step continuous range in between. Below each option are descriptive tooltips: “Clean visuals with geometric sans-serif fonts” for “Modern” and “Rich and detailed designs with serif fonts” for “Traditional.” The slider enables fine-tuning of tag values with visual examples illustrating the effect of each step.}
%  \end{minipage}
%   \begin{minipage}[t]{\dimexpr\columnwidth-3.2cm\relax}
%  \end{minipage}
\end{figure}          \textbf{Opposite Slider Widget}
Users can fine-tune tag values with more granularity by using a slider widget. For this, the system automatically generates an opposite value for each tag with a five-step continuous range slider in between, including descriptive examples illustrating each step's value.
}{\begin{figure}[H]
  \begin{minipage}[t]{3cm}
    \includegraphics[align=t,width=\columnwidth]{Figures/Feature-Slider.png}%
    \caption{}
    \label{fig:slider}
    \Description{The figure shows a typography tag with a slider widget. The current selection is “Typography: Modern,” and the opposite value is “Typography: Traditional.” The slider allows users to adjust between the two extremes, with a five-step continuous range in between. Below each option are descriptive tooltips: “Clean visuals with geometric sans-serif fonts” for “Modern” and “Rich and detailed designs with serif fonts” for “Traditional.” The slider enables fine-tuning of tag values with visual examples illustrating the effect of each step.}
  \end{minipage}
   \begin{minipage}[t]{\dimexpr\columnwidth-3.2cm\relax}
          \textbf{Opposite Slider Widget}
Users can fine-tune tag values with more granularity by using a slider widget. For this, the system automatically generates an opposite value for each tag with a five-step continuous range slider in between, including descriptive examples illustrating each step's value.
  \end{minipage}
\end{figure}}

\aptLtoX[graphic=no,type=html]{\begin{figure}[h]
%  \begin{minipage}[t]{3cm}
    \includegraphics[align=t,width=\columnwidth]{Figures/Feature-TagSuggestions.png}
    \caption{ \ ~ \ ~ \ ~ \ ~ \ }
    \label{fig:tag-suggestions}
    \Description{The figure shows several suggested tags within the “Narrative” tag group. Tags such as “Introduction: Basic Syntax,” “Tone: Encouraging,” “Approach: Hands-On,” and “Topic: Learn to code” are displayed. The suggested tags appear outside the circular “Narrative” group, illustrating that users can drag these suggestions into the group to include them in the content generation process. This allows users to request new tag suggestions for talking points, visual styles, or content sources based on existing tags.}
%  \end{minipage}
%   \begin{minipage}[t]{\dimexpr\columnwidth-3.2cm\relax}
 %  \end{minipage}
\end{figure}         \textbf{Tag Suggestions}
For each tag group, users can request new tag suggestions, such as additional talking points, visual styles, or content sources, from the system based on already specified tags. Generally, all suggested tags first appear outside the circular groups, and users can drag suggested tags into a group to include them in the generation.
}{\begin{figure}[h]
  \begin{minipage}[t]{3cm}
    \includegraphics[align=t,width=\columnwidth]{Figures/Feature-TagSuggestions.png}
    \caption{}
    \label{fig:tag-suggestions}
    \Description{The figure shows several suggested tags within the “Narrative” tag group. Tags such as “Introduction: Basic Syntax,” “Tone: Encouraging,” “Approach: Hands-On,” and “Topic: Learn to code” are displayed. The suggested tags appear outside the circular “Narrative” group, illustrating that users can drag these suggestions into the group to include them in the content generation process. This allows users to request new tag suggestions for talking points, visual styles, or content sources based on existing tags.}
  \end{minipage}
   \begin{minipage}[t]{\dimexpr\columnwidth-3.2cm\relax}
          \textbf{Tag Suggestions}
For each tag group, users can request new tag suggestions, such as additional talking points, visual styles, or content sources, from the system based on already specified tags. Generally, all suggested tags first appear outside the circular groups, and users can drag suggested tags into a group to include them in the generation.
  \end{minipage}
\end{figure}}

\aptLtoX[graphic=no,type=html]{\begin{figure}[h]
%  \begin{minipage}[t]{3cm}
    \includegraphics[align=t,width=\columnwidth]{Figures/Feature-ImageSuggestions.png}
    \caption{ \ ~ \ ~ \ ~ \ ~ \ }
    \label{fig:image-suggestions}
    \Description{The figure shows image suggestions related to the tag “Kayaking” within the “Content” group. Several images of kayaking and water scenes are displayed around the tag, indicating that the system has retrieved these from an online image search. Users can request system-generated image tags, and each time different images are returned based on the active tags on the board.}
%  \end{minipage}
%   \begin{minipage}[t]{\dimexpr\columnwidth-3.2cm\relax}
%  \end{minipage}
\end{figure}          \textbf{Image Suggestions}
In addition to tags, users can also request system suggestions for new image tags. When requested, each time the system returns different images from a background online image search related to the board's active tags.  
}{\begin{figure}[h]
  \begin{minipage}[t]{3cm}
    \includegraphics[align=t,width=\columnwidth]{Figures/Feature-ImageSuggestions.png}
    \caption{}
    \label{fig:image-suggestions}
    \Description{The figure shows image suggestions related to the tag “Kayaking” within the “Content” group. Several images of kayaking and water scenes are displayed around the tag, indicating that the system has retrieved these from an online image search. Users can request system-generated image tags, and each time different images are returned based on the active tags on the board.}
  \end{minipage}
   \begin{minipage}[t]{\dimexpr\columnwidth-3.2cm\relax}
          \textbf{Image Suggestions}
In addition to tags, users can also request system suggestions for new image tags. When requested, each time the system returns different images from a background online image search related to the board's active tags.  
  \end{minipage}
\end{figure}}




\begin{figure*}[h!]
  \centering
  \includegraphics[width=\linewidth]{Figures/Tag-Impact.png}
  \caption{Examples of steering GenAI slide generation with Intent Tags: (A) Influencing the detail level of the displayed information, (B) steering the "mood" from "bright" to "dark," or (C) gradually blending the layout between text- and image-heavy.    }
  \Description{The figure shows three examples of how Intent Tags can be used to steer slide generation. (A) A tag labeled “Content Depth: Detailed” is used to influence the amount of information shown, transitioning from a short statement to a more detailed layout. (B) The “Mood” tag is being adjusted from “Bright” to “Dark,” changing the background color and text style from light to dark. (C) The “Layout” tag is set to adjust between “Text” and “Visual,” gradually blending the layout from a text-heavy format to a more image-focused layout. These examples illustrate how users can fine-tune the content, mood, and layout of slides using intent tags.}
  \label{fig:tag-impact}
\end{figure*}


\subsubsection{\textbf{Steering Slide Generation}}

To summarize the different ways of steering slide generation with Intent Tags, Figure \ref{fig:tag-impact} illustrates how including a tag suggestion modifies the content of a slide, using the drop-down list changes a slide's mood, and using the opposite slider widget blends between a text- and image-heavy layout.  


\subsubsection{\textbf{Previews of Slides and Slider Values (DP5)}}
IntentTagger integrates several preview features, allowing users to explore alternative tag values without committing to long waiting times: The tag's \textit{Drop-down} list shows pre-generated previews as tooltips of how the current slide would look like with that tag value applied (Figure \ref{fig:drop-down}). 
The \textit{Opposite Slider} widget shows generated explanations describing the resulting effect if applied to the generation (Figure \ref{fig:slider}). All tag previews and explanations get asynchronously pre-generated in the background, allowing users to explore these options in real time. 



\subsubsection{\textbf{Referencing External Documents (DP2)}}
Besides \textit{Concept Tags}, users can also include external files for the generative slide creation as \textit{Reference Tags}. 

\aptLtoX[graphic=no,type=html]{\begin{figure}[H]
%  \begin{minipage}[t]{3cm}
    \includegraphics[align=t,width=\columnwidth]{Figures/Feature-WordTag.png}
    \caption{ \ ~ \ ~ \ ~ \ ~ \ }
    \label{fig:word-tag}
    \Description{The figure shows a Word document titled “Product Launch Info” being used as a content source for the presentation. The content from the document is displayed, with the user hovering over a section labeled “Marketing channels,” revealing a green button labeled “+ Include in Presentation.” This allows users to selectively include specific parts of the document, such as “Launch Plan” and “Timeline,” into the generated presentation by using a text selection pop-up.}
%  \end{minipage}
%   \begin{minipage}[t]{\dimexpr\columnwidth-3.2cm\relax}
%  \end{minipage}
\end{figure}          \textbf{Including Content From Word Documents}
Users may drag in Word documents as content sources for the generated presentation. Optionally, users can select sections through a text selection pop-up to include only specific parts in the generated 
 presentation. 
}{\begin{figure}[H]
  \begin{minipage}[t]{3cm}
    \includegraphics[align=t,width=\columnwidth]{Figures/Feature-WordTag.png}
    \caption{}
    \label{fig:word-tag}
    \Description{The figure shows a Word document titled “Product Launch Info” being used as a content source for the presentation. The content from the document is displayed, with the user hovering over a section labeled “Marketing channels,” revealing a green button labeled “+ Include in Presentation.” This allows users to selectively include specific parts of the document, such as “Launch Plan” and “Timeline,” into the generated presentation by using a text selection pop-up.}
  \end{minipage}
   \begin{minipage}[t]{\dimexpr\columnwidth-3.2cm\relax}
          \textbf{Including Content From Word Documents}
Users may drag in Word documents as content sources for the generated presentation. Optionally, users can select sections through a text selection pop-up to include only specific parts in the generated 
 presentation. 
  \end{minipage}
\end{figure}}

\aptLtoX[graphic=no,type=html]{\begin{figure}[H]
%  \begin{minipage}[t]{3cm}
    \includegraphics[align=t,width=\columnwidth]{Figures/Feature-IncludeImages.png}
    \caption{ \ ~ \ ~ \ ~ \ ~ \ }
    \label{fig:include-images}
    \Description{The figure shows a “Deck Template” tag titled “Corporate Identity Deck Template” alongside an image of a 3D headset. These elements are within the “Content Sources” group, illustrating how users can drag external images or existing slide deck templates onto the canvas to include them in the presentation. The templates can serve as visual references or be used to structure the presentation.}
%  \end{minipage}
%   \begin{minipage}[t]{\dimexpr\columnwidth-3.2cm\relax}
%  \end{minipage}
\end{figure}          \textbf{Including Images and Slide Deck Templates }
Users can drag external images onto the canvas to include these in the presentation. 
Users also have the option to reference other existing slide decks as visual templates. 
}{\begin{figure}[H]
  \begin{minipage}[t]{3cm}
    \includegraphics[align=t,width=\columnwidth]{Figures/Feature-IncludeImages.png}
    \caption{}
    \label{fig:include-images}
    \Description{The figure shows a “Deck Template” tag titled “Corporate Identity Deck Template” alongside an image of a 3D headset. These elements are within the “Content Sources” group, illustrating how users can drag external images or existing slide deck templates onto the canvas to include them in the presentation. The templates can serve as visual references or be used to structure the presentation.}
  \end{minipage}
   \begin{minipage}[t]{\dimexpr\columnwidth-3.2cm\relax}
          \textbf{Including Images and Slide Deck Templates }
Users can drag external images onto the canvas to include these in the presentation. 
Users also have the option to reference other existing slide decks as visual templates. 
  \end{minipage}
\end{figure}}



\subsubsection{\textbf{Tag Grounding Acts (DP4)}}
While Intent Tagging presents granular and flexible mechanisms for users to communicate their intentions to the generative AI slide creation system, this principle also works backward in so-called \textit{Tag Grounding Acts}, where the system creates intent tags from user inputs such as longer text prompts or slides.


\aptLtoX[graphic=no,type=html]{\begin{figure}[H]
%  \begin{minipage}[t]{3cm}
    \includegraphics[align=t,width=\columnwidth]{Figures/Feature-GroundingFromText.png}
    \caption{ \ ~ \ ~ \ ~ \ ~ \ }
    \label{fig:text-grounding}
    \Description{The figure shows a text input field where a user types the prompt “Marie Curie for teenagers,” which is automatically parsed into two Intent Tags: “Topic: Marie Curie” and “Audience: Teenagers.” This demonstrates how the system can take a longer text prompt and decompose it into individual tags, allowing users to start a new presentation based on their initial instructions.}
%  \end{minipage}
%  \begin{minipage}[t]{\dimexpr\columnwidth-3.2cm\relax}
%  \end{minipage}
\end{figure}          \textbf{Grounding from Text Prompt}
This feature allows users to start a new presentation by providing instructions via a longer conventional text prompt, which the system will then automatically decompose into individual Intent Tags.
}{\begin{figure}[H]
  \begin{minipage}[t]{3cm}
    \includegraphics[align=t,width=\columnwidth]{Figures/Feature-GroundingFromText.png}
    \caption{}
    \label{fig:text-grounding}
    \Description{The figure shows a text input field where a user types the prompt “Marie Curie for teenagers,” which is automatically parsed into two Intent Tags: “Topic: Marie Curie” and “Audience: Teenagers.” This demonstrates how the system can take a longer text prompt and decompose it into individual tags, allowing users to start a new presentation based on their initial instructions.}
  \end{minipage}
   \begin{minipage}[t]{\dimexpr\columnwidth-3.2cm\relax}
          \textbf{Grounding from Text Prompt}
This feature allows users to start a new presentation by providing instructions via a longer conventional text prompt, which the system will then automatically decompose into individual Intent Tags.
  \end{minipage}
\end{figure}}



\aptLtoX[graphic=no,type=html]{\begin{figure}[H]
%  \begin{minipage}[t]{3cm}
    \includegraphics[align=t,width=\columnwidth]{Figures/Feature-GroundingFromSlide.png}
    \caption{ \ ~ \ ~ \ ~ \ ~ \ }
    \label{fig:slide-grounding}
    \Description{The figure shows how the system analyzes a slide’s layout and text style when a user invokes the Slide Steering Overlay. The current layout is “Two Column,” and the text style is set to “Bullet List.” The system pre-populates these attributes as intent tags, representing the slide’s current content sources, narrative, and visual style, providing a starting point for users to make adjustments through the intent tagging interface.}
%  \end{minipage}
%   \begin{minipage}[t]{\dimexpr\columnwidth-3.2cm\relax}
%  \end{minipage}
\end{figure}          \textbf{Grounding from Slide}
Each time a user invokes the \textit{Slide Steering Overlay} to adjust single slides, the system first analyzes the slide's \textit{content sources}, \textit{narrative}, and \textit{visual style} and then pre-populates the interface with intent tags representing the slide's current state. This provides an easy starting point for users to make adjustments through Intent Tags. 
}{\begin{figure}[H]
  \begin{minipage}[t]{3cm}
    \includegraphics[align=t,width=\columnwidth]{Figures/Feature-GroundingFromSlide.png}
    \caption{}
    \label{fig:slide-grounding}
    \Description{The figure shows how the system analyzes a slide’s layout and text style when a user invokes the Slide Steering Overlay. The current layout is “Two Column,” and the text style is set to “Bullet List.” The system pre-populates these attributes as intent tags, representing the slide’s current content sources, narrative, and visual style, providing a starting point for users to make adjustments through the intent tagging interface.}
  \end{minipage}
   \begin{minipage}[t]{\dimexpr\columnwidth-3.2cm\relax}
          \textbf{Grounding from Slide}
Each time a user invokes the \textit{Slide Steering Overlay} to adjust single slides, the system first analyzes the slide's \textit{content sources}, \textit{narrative}, and \textit{visual style} and then pre-populates the interface with intent tags representing the slide's current state. This provides an easy starting point for users to make adjustments through Intent Tags. 
  \end{minipage}
\end{figure}}






\subsection{Implementation Details}

IntentTagger is implemented in TypeScript using ReactJS \cite{react.js_reactjs_2024} with ReactFlow \cite{reactflow_react_2024} for the tag canvas interface and Blocknote \cite{blocknote_blocknote_2024} for the text editor features. 
Mommoth.js \cite{williamson_mwilliamson_2024} is used for importing Word documents, and slide rendering is based on a modified fork of spectacle.js \cite{formidablelabs_formidablelabs_2024}. 
For generating slides and adaptive UI, we use the official OpenAI API \cite{openai_openai_2024} to execute prompts using “gpt-4o” and the Bing search API \cite{microsoft_bing_2024} for image suggestions.
\rev{To suggest new tags, we prompt GPT to generate a list of \textit{[label:value]} pairs for each tag group to augment the existing set of active intent tags.
For outlines, GPT is instructed to generate a presentation outline in markdown format from a list of a user's active concept and reference tags (including images if present).
To generate slides, we then prompt GPT to return a template-based JSON slide deck representation by providing the outline, active intent tags, and deck references (if specified).}
For generating the drop-down values, slide previews, and slider explanations, the system asynchronously requests these from GPT in the background for each new or modified tag on the canvas.
\rev{Please see the Appendix for detailed prompts.}









