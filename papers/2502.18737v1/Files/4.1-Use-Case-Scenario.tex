Lucy is a sustainability manager at a company and wants to prepare a short slide presentation for new employees to inform them about sustainability initiatives at her company. 

Lucy starts IntentTagger and clicks on the \textit{"Create a new presentation from prompt..."} button. 
A modal popup appears, and she types \textit{"a presentation for introducing sustainability initiatives for new employees"} in the prompt field and presses enter. 
The modal disappears, and the three circular steering groups on the steering canvas pulse, then three concept tags appear in the \textit{"Narrative"} circle with the parameters \textit{"Topic: Sustainability Initiatives," "Audience: New Employees"} and \textit{"Purpose: Introduction."} 

\textbf{Tag Suggestions:} To get inspiration for further presentation talking points, she clicks \textit{"Suggest More Tags,"} and shortly after, new tags appear outside of the circular tag groups. Lucy finds some of the suggested tags useful and drags these into the \textit{"Narrative"} group to include them in the generation, such as \textit{"Focus: Eco-friendly Practices," "Highlight: Company Goals," "Section: Key Initiatives,"} and \textit{"Objective: Corporate Responsibility."}

\textbf{Including External Content:}
Lucy also wants to include sections from an existing Word document detailing the company's policies. With the mouse cursor, she drags the document from her file browser onto the IntentTagger app, and a new reference tag appears in the \textit{"Content Sources"} group. She clicks on the tag's \textit{"Select Sections"} button, which opens up the tag's Text Selection Widget showing the document's content. In the widget, she highlights two document sections, clicks \textit{"Include Section in Presentation"} from the widget's context menu, and closes the widget. 

\textbf{Generating an Outline:}
Next, she wants to generate a first draft of the presentation's outline and presses \textit{"update outline from intent tags"} in the outline view panel. The tag groups start pulsing, and then the new outline appears in the outline text editor. 

\textbf{Manually Crafting Concept Tags:}
Lucy is happy with the structure overall but finds it too long (20 sections). Instead of manually editing the outline, she adds a new concept tag to the \textit{"Narrative"} group and enters \textit{"Number of slides"} into the tag's upper text field and \textit{"10"} into the lower and clicks the \textit{"update outline from intent tags"} button again. The newly generated outline is now shorter and only comprises ten sections. 

\textbf{Generating Slides:}
Next, Lucy wants to generate a first draft of the slide deck and clicks the \textit{"Update slides from intent tags"} button. Shortly after, the generated slides appear in the slide preview side panel. After evaluating the slides, Lucy realizes that adding some images and adjusting the visual style could enhance the presentation's appeal. 

\textbf{Including Suggested Images:}
She clicks the \textit{"suggest more images"} button on the tag steering board, and shortly after, new Reference Image Tags appear outside the \textit{"Content Sources"} tag group with eco-friendly-themed stock photos. Lucy drags some of the image tags into the \textit{"Content Sources"} group to add them to the presentation.   

\textbf{Adjusting the Visual Style through Tags' Alternatives Drop-down:}
To make visual adjustments to the slides, she drags in some of the previously suggested tags, such as \textit{"Typography: Sans-Serif," "ColorPalette: Green and Blue,"} and \textit{"Theme: Nature"} onto the \textit{"Visual Style"} tag group. To get a better sense of typography alternatives, she clicks on the tags' drop-down, and a list of alternative tag values appears, such as \textit{"Serif," "Monospace,"} and \textit{"Handwritten."} While hovering over each option with the mouse, Lucy sees a preview thumbnail of that font style applied to the current slide. Lucy chooses \textit{"Monospace."} Now, Lucy clicks \textit{"Update slides from intent tags"} again, and shortly after, she sees the revised slides reflecting the intended new visual style, including the images. 

\textbf{Adjusting a Single Slide with Tag Steering Overlay:}
Lucy decides to refine the slide deck further. She clicks on the third slide in the \textit{Slide Preview} panel to open the slide edit view and review its content in detail. 
The slide contains a text paragraph and an image, but she wants to show a list of bullet points instead. 
To modify the slide, she clicks on the intent tag icon next to the slide, and the three steering tag groups appear as an overlay containing tags representing the slide's \textit{Narrative}, \textit{Visual Style}, and \textit{Content Sources}. 
Lucy selects the \textit{"Text Format"} tag inside the \textit{Narrative} group and clicks on its drop-down widget. 
A list of alternative tag values appears, including \textit{"Bullet Points," "List,"} and \textit{"Table."} Lucy chooses \textit{"Bullet Points."} 
To adjust the background color, Lucy creates a new tag in the \textit{Visual Style} group and types in \textit{"Background"} and \textit{"pastel color."} 
Now, she clicks on the \textit{"Explore slide variations"} button, and the system generates several alternative versions of the slide, each with slightly different text variations and pastel background colors.  
Lucy clicks through these variations and selects one that better aligns with her presentation’s tone.

\textbf{Applying a Slide's Style to All Slides:}
To maintain consistency across the entire presentation, Lucy uses the same visual style and selects \textit{"Apply to all slides."} 

Satisfied with the changes, Lucy saves the presentation, ready to deliver it in the upcoming meeting with her colleagues.

\textbf{Adjusting the Presentation to New Requirements:}
After a couple of months, Lucy needs to adapt the presentation for a different audience—existing employees rather than new hires. She clicks on the \textit{"Audience"} tag in the \textit{Narrative} group and changes it from \textit{"New Employees"} to \textit{"Existing Employees."} The system instantly updates the content and tone of the slides to better fit this new audience. For example, introductory sections are replaced with more in-depth discussions of ongoing sustainability projects and their impact.


