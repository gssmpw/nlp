\newpage
\onecolumn
%\pagestyle{empty}

\section{Additional Materials}


\begin{table*}[h]
\caption{Overview of study participants.}
\begin{tabular}{lclll}
\toprule
\textbf{ID} & \textbf{Age Range (Years)} & \textbf{Gender}     & \textbf{Job Title}  & \textbf{Frequency of PowerPoint Usage} \\
\midrule
P01            & 18-29     & Female     & Senior Product Designer       & Multiple times per week                      \\
P02            & 30-39     & Female     & Senior Product Manager        & Every day                                    \\
P03            & 18-29     & Female     & Research Intern (PhD)             & Multiple times per month                     \\
P04            & 18-29     & Male       & Senior Human Factors Engineer & Every day                                    \\
P05            & 30-39     & Female     & Senior Designer               & Multiple times per week                      \\
P06            & 40-49     & Female     & Senior User Researcher        & Multiple times per week                      \\
P07            & 30-39     & Male       & Senior Product Designer       & Every day                                    \\
P08            & 18-29     & Female     & Content Designer              & Multiple times per month                     \\
P09            & 30-39     & Male       & Research Intern (PhD)         & Multiple times per month                     \\
P10            & 18-29     & Non-binary & Research Intern (PhD)         & Multiple times per year                      \\
P11            & 30-39     & Female     & Design Program Manager        & Multiple times per week                      \\
P12            & 40-49     & Female     & UX Researcher                 & Multiple times per month      \\                                       
\bottomrule
\end{tabular}
\label{tab:participants}
\Description{Study participant demographics. (Table is machine readable). Study participant demographics. (Table is machine readable). The table provides demographic information for 12 participants (P01 to P12) involved in the study. Columns include ID, Age Range (Years), Gender, Job Title, and Frequency of PowerPoint Usage. The age ranges span from 18-29 to 40-49 years, with job titles such as Senior Product Designer, Research Intern (PhD), and UX Researcher. Participants’ frequency of PowerPoint usage varies from “Multiple times per year” to “Every day.” Two participants, P10 and P04, are noted as non-binary and male, respectively.}
\end{table*}



\rev{\subsection{Further System Implementation Details }}

\subsubsection{\rev{\textbf{Tag Suggestion Mechanism}}}

\rev{To generate intent tag suggestions based on a user's existing tags on the slide deck steering canvas, we construct a call to GPT comprising the static \texttt{SYSTEM\_PROMPT} and the dynamic \texttt{USER\_CONTEXT} below. 
\texttt{USER\_CONTEXT} is a string dynamically constructed from all active intent tags \texttt{[attribute:value]} for each tag group \texttt{(Narrative, Visual Style, Content Sources)}. }

\begin{lstlisting}
SYSTEM_PROMPT (static):  
You are an assistant to help users author slide presentations in PowerPoint. 
You are helping the user develop the narrative and visual style and extracting content from source documents relevant to the presentation. 
You will receive instructions to help with one of these three buckets: Narrative, Visual Style, and Content Sources.  
The user will already have descriptions, tags, and media elements that describe different aspects of the presentation.  
Your task is to suggest additional keywords and concepts for each of these buckets.  
The user will prompt you in the format:  
**Bucket** attribute1:value1, attribute2:value2, ...  
A bucket might be empty, in which case you should suggest concepts based on the other buckets.  
A value might not be associated with an attribute.  
For each user prompt, suggest 7 concepts for each of the requested buckets.  
A concept is a keyword or phrase that is relevant to the user's presentation. A concept should be a single word or a short phrase. Each concept should have an attribute and value.  
Return your response in JSON format. Only return JSON format without any additional Markdown wrapper code blocks.  
 
Here is an example of the JSON return format:  
{
    "Narrative": ["attribue1:value1", "attribue1:value1", ...],  
    "Visual Style": ["attribue1:value1", "attribue1:value1", ...],  
    "Content Sources": ["attribue1:value1", "attribue1:value1", ...]  
}

USER_CONTEXT (dynamic): 
**Narrative** attribute1:value1, attribute2:value2, ... 
**Visual Style** attribute1:value1, attribute2:value2, ... 
**Content Sources** attribute1:value1, attribute2:value2, ... 
\end{lstlisting}



\subsubsection{\rev{\textbf{Outline Generation From Intent Tags}}}

\rev{To generate an outline based on a user's existing tags on the slide deck steering canvas, we construct a call to GPT comprising the static \texttt{SYSTEM\_PROMPT} and the dynamic \texttt{USER\_CONTEXT} below.
\texttt{USER\_CONTEXT} is a string dynamically constructed from all active intent tags \texttt{[attribute:value]} for each tag group \texttt{(Narrative, Visual Style, Content Sources)}.
GPT is instructed to generate an outline in Markdown format, including the URLs of image reference tags if present. }

 
\begin{lstlisting}
SYSTEM_PROMPT (static): 
You are an assistant to help users author slide presentations in PowerPoint. 
You are creating an outline for the user's presentation. 
The user will have descriptions and keywords that describe different aspects of the presentation. 
The user will prompt you in the format: attribute1:value1, attribute2:value2, ... 
Ignore all the attributes related to the visual style.  
If the content sources include ImageUrl attributes, you should include these images in the outline as images in the markdown. 
Respond with an outline of the presentation in the format:  
section title - section content bullet points. 
Return the outline in markdown format without ```markdown at the start and end. Only return markdown content for the outline. 

USER_CONTEXT (dynamic): 
**Narrative** attribute1:value1, attribute2:value2, ... 
**Visual Style** attribute1:value1, attribute2:value2, ... 
**Content Sources** attribute1:value1, attribute2:value2, ... 
\end{lstlisting}






\subsubsection{\rev{\textbf{Slide generation mechanism}}}

\rev{ To generate the JSON structure for a slide deck, we construct a call to GPT comprising the static \texttt{SYSTEM\_PROMPT} and the dynamic \texttt{USER\_CONTEXT}. 
\texttt{USER\_CONTEXT} is a string containing the \texttt{PRESENTATION OUTLINE} in markdown, \texttt{META INFORMATION}, which is a string containing all active intent tags \texttt{[attribute:value]}, and optionally the \texttt{REFERENCE SLIDE DECK TEMPLATE}, which is a JSON string of a reference slide deck (if provided).  
The general slide deck template schema and slide generation mechanism are based on the Spectacle\footnote{https://github.com/FormidableLabs/spectacle} library. }

 
\begin{lstlisting}
SYSTEM_PROMPT (static): 
Your task is to generate a slide presentation deck from an outline and meta information.  
You will receive the presentation OUTLINE in markdown format from the user. 
You will also receive additional META INFORMATION about the desired presentation's Narrative, Visual Style, and Content Sources.  
The META INFORMATION will describe different aspects of the presentation.  
You will receive the META INFORMATION in the format:  
**Bucket** attribute1:value1 (optional description), attribute2:value2 (optinal description), ...  
A bucket might be empty, in which case you should suggest concepts based on the other buckets.  
A value might not be associated with an attribute.  
The META INFORMATION might also contain USER DOCUMENTS with content that has to be included in the presentation. 

GENERATING SLIDES  
Your task is to generate a slide deck. Each slide has to consist of a LAYOUT, CONTENT, and THEME.  

LAYOUT:  
The layout determines the arrangement of the content on the slide.  
There are five different layouts to choose from:  
"title", "listOrParagraph", "verticalImage", "fullImage". 

CONTENT:  
Each layout takes different content parameters depending on the layout type. Optionally, each slide can take an image url as background image in the content.  
The background image will replace the background color of the slide.  
The "listOrParagraph" and "verticalImage" layouts take either a list or a paragraph as content besides the title and image.  
For bullet lists, never use more than 3 bullets. 

THEME:  
The theme determines the fonts, colors, and font sizes used in the slide.  
 
fonts: { header: fontname, text: fontname }  
"header" is the font used for titles and "text" is the font used for body, lists, paragraphs.  
There are six fonts to choose from: "Quicksand", "Playfair Display", "Montserrat", "Merriweather", "Roboto" and "Roboto Condensed". 

colors: { primary: color, secondary: color, tertiary: color }  
"primary" is the main color used for text, "secondary" is the color used for accents, and "tertiary" is the color used for backgrounds. 

fontSizes: { h1: size, text: size }  
"h1" is the font size used for titles and "text" is the font size used for body text. 

REFERENCE SLIDE DECK TEMPLATE  
The user might also provide a REFERENCE SLIDE DECK TEMPLATE with a slide deck template they want to use for the visual style.  
Important: If a REFERENCE SLIDE DECK TEMPLATE is provided, use only this template for the visual style (font, backgroundImage, sizes, colors) and IGNORE all visual style information from the generic example slide deck below.  
ALWAYS use the backgroundImge URLs from the REFERENCE SLIDE DECK TEMPLATE on each slide! 

REWORK CONTENT FROM THE OUTLINE  
You should rework the content from the OUTLINE to fit the slide deck based on the desired format specified in the META INFORMATION.  
For example, if the META INFORMATION specifies that the presentation should have a formal tone, you should rework the content to fit this tone.  
If the META INFORMATION specifies that the presentation should have a more verbose text, you should rework the content to fit this style. 

YOUR TASK:  
Your task is to compose a slide deck based on the outline and to adapt the content to the layout and theme.  
Your task is to define the theme and layout of the slides based on the OUTLINE and META INFORMATION provided by the user.  
Generate the number of slides based on the outline and meta information provided by the user.  
Keep the visual style consistent across all slides.  
Especially the fonts, colors, background color and font sizes.  
You should return the slide deck in JSON format. Only return JSON format without any additional Markdown wrapper code blocks.  

Here is a generic example of a slide deck in JSON format: 
[{ 
    slideNumber: 1, 
    layout: "title", 
    content: { 
      title: "A presentation about something", 
      subtitle: "by someone", 
      backgroundImage:  "url(<image url placeholder>)" 
    }, 
    theme: { 
      fonts: { 
        header: '"Playfair Display", serif', 
        text: '"Quicksand", sans-serif', 
      }, 
      colors: { 
        primary: "#000", 
        secondary: "#000", 
        tertiary: "#fff", 
      }, 
      fontSizes: { 
        h1: "100px", 
        text: "44px", 
      }, 
      space: [16, 24, 32], 
    }, 
  },{ 
    slideNumber: 2, 
    layout: "listOrParagraph", 
    ...
    (PARTS OMITTED) 
    ...
}] 
 

USER_CONTEXT (dynamic): 
PRESENTATION OUTLINE: 
$CURRENT_OUTLINE_IN_MARKDOWN_STRING 

META INFORMATION:  
$CURRENT_ACTIVE_TAGS_AS_ATTRIBUTE:VALUE_PAIRS_CLUSTERED_BY_GROUP 

(optional) REFERENCE SLIDE DECK TEMPLATE: 
$REFERENCE_SLIDE_DECK_JSON 
\end{lstlisting}


\subsubsection{\rev{\textbf{Mechanism to create tags for an existing slide (tag grounding act)}}}

\rev{To generate a collection of intent tags based on a provided image of the current slide, we construct a call to GPT comprising the static \texttt{SYSTEM\_PROMPT} and the dynamic \texttt{USER\_CONTEXT} below.
\texttt{USER\_CONTEXT} is a string of the current slide in a base64 image format.
In addition to the slide image, we experimented with including the active steering board tags in the prompt but refrained from including these in this version of IntentTagger for simplicity. 
Future explorations outside the scope of this paper should investigate how "global" steering board-level intent tags can meaningfully propagate to individual slides and how changes on single slides can propagate back to the global steering board level while meaningfully resolving conflicts between global and slide-level tags. 
}


\begin{lstlisting}
SYSTEM_PROMPT (static): 
You are an assistant to help users author slide presentations in PowerPoint. 
You will receive an image of a slide from their slide presentation deck.  
Your task is to analyze the slide deck and return descriptive attributes that best describe the slide.  
You should return these attributes related to three categories: Narrative, Visual Style, and Content Sources.     
Each attribute should consist of a label and a value, like Tonality:Formal or Typography:Modern. 
For each bucket, return between 2 to 6 attributes.  
Return your response in JSON format.  
Only return JSON format without any additional Markdown wrapper code blocks. 

Here is an example of the JSON format: 
{ 
    "Narrative": ["attribue1:value1", "attribue1:value1", ...], 
    "Visual Style": ["attribue1:value1", "attribue1:value1", ...], 
    "Content Sources": ["attribue1:value1", "attribue1:value1", ...] 
} 

USER_CONTEXT (dynamic): 
$IMAGE_OF_CURRENT_SLIDE_AS_BASE64_STRING 
\end{lstlisting}










\subsection{Semi-structured Task Outcomes }
This section contains screenshots of IntentTagger (deck steering board and slide panel) taken at the end of each participant's semi-structured slide creation task from our user study (see study phase 3 at section \ref{sec:study-procedure}). 


\begin{figure}[H]
    \centering
  \includegraphics[width=0.99\linewidth]{Figures/Appendix/P01.png}
  \caption{Semi-structured task outcome of P01}
  \Description{The figure displays a screenshot of the IntentTagger interface for participant P01’s open-ended slide creation task. The left side shows the Deck Steering Board, where tags related to Narrative, Visual Style, and Content Sources are organized. Tags include topics such as “Hiking,” “Hiking Shoes,” and imagery of “Mountains.” The right side shows the Slide Panel, featuring a generated slide deck about hiking in Washington, including information on popular hiking spots like Mount Rainier and the Olympic National Park.} 
  \label{fig:appendix_task4_P01}
\end{figure}

\begin{figure}[H]
    \centering
  \includegraphics[width=0.99\linewidth]{Figures/Appendix/P02.png}
  \caption{Semi-structured task outcome of P02}
  \Description{ The figure shows the IntentTagger interface for participant P02’s semi-structured task. The Deck Steering Board on the left features tags related to Narrative, Visual Style, and Content Sources. The tags include topics like “Introduction and benefits of yoga,” “Physical strength,” and “Mental clarity” under Narrative, while Visual Style contains tags such as “Font: Handwritten” and “Color Background: Light blue.” The Content Sources include wellness websites, case studies, and testimonials with relevant images. On the right, the Slide Panel shows a slide deck on the benefits of yoga, including sections on physical and mental health benefits, clarity, and focus.} 
  \label{fig:appendix_task4_P02}
\end{figure}

\begin{figure}[H]
    \centering
  \includegraphics[width=0.99\linewidth]{Figures/Appendix/P03.png}
  \caption{Semi-structured task outcome of P03}
  \Description{The figure shows the IntentTagger interface for participant P03’s semi-structured task. The Deck Steering Board on the left contains tags related to Narrative, Visual Style, and Content Sources. The Narrative tags include “Topic: Joys of Kayaking,” “Goal: Encourage others to kayak,” and “Structure: Step-by-step guide.” The Visual Style tags include “Colors: Blue and Green” and “Textures: Water ripples.” The Content Sources section includes videos, blogs, and articles related to kayaking. On the right, the Slide Panel shows a generated slide deck on kayaking, with slides covering topics such as “Why Kayak in Washington” and “Essential Techniques for Beginners.” } 
  \label{fig:appendix_task4_P03}
\end{figure}

\begin{figure}[H]
    \centering
  \includegraphics[width=0.99\linewidth]{Figures/Appendix/P04.png}
  \caption{Semi-structured task outcome of P04 (Note: The user was unable to add images to their presentation, due to a temporary outage of the image search API.)}
  \Description{The figure shows the IntentTagger interface for participant P04’s semi-structured task. The Deck Steering Board on the left contains tags related to Narrative, Visual Style, and Content Sources. The Narrative tags include “Topic: Home Made Pizza Making,” “Focus: Practical,” and “Objective: Step-by-step guide.” The Visual Style tags include “Color Scheme: Earthy Palette” and “Layout: Clean and simple.” The Content Sources include online recipe forums, interviews with chefs, and food magazines. On the right, the Slide Panel shows a slide deck on pizza-making, covering topics such as “Overview of Home-Made Pizza” and “Case Study: The Perfect Margherita Pizza.” Note: The user was unable to add images to the presentation due to an API outage. } 
  \label{fig:appendix_task4_P04}
\end{figure}

\begin{figure}[H]
    \centering
  \includegraphics[width=0.99\linewidth]{Figures/Appendix/P05.png}
  \caption{Semi-structured task outcome of P05}
  \Description{The figure shows the IntentTagger interface for participant P05’s semi-structured task. The Deck Steering Board on the left contains tags for Narrative, Visual Style, and Content Sources. The Narrative section includes tags like “Topic: Growing Dahlias,” “Location: Pacific Northwest,” and “Introduction: Overview of Dahlias.” The Visual Style section includes tags such as “Layout: Bold and colorful,” “Background: White,” and “Font Style: Elegant and legible.” The Content Sources include gardening guides, images from botanical gardens, and notes from gardening workshops. On the right, the Slide Panel displays a slide deck on growing dahlias, covering topics such as “Understanding Dahlias,” “Climate and Soil Requirements,” and “Planting Dahlias.”} 
  \label{fig:appendix_task4_P05}
\end{figure}

\begin{figure}[H]
    \centering
  \includegraphics[width=0.99\linewidth]{Figures/Appendix/P06.png}
  \caption{Semi-structured task outcome of P06}
  \Description{The figure shows the IntentTagger interface for participant P06’s semi-structured task. The Deck Steering Board on the left contains tags for Narrative, Visual Style, and Content Sources. The Narrative section includes tags like “Cycling for Fitness,” “Competitive Events,” and “Famous Climbs in France.” The Visual Style section includes tags such as “Fonts: Sporty and dynamic” and “Backgrounds: Scenic roadways.” The Content Sources include images of professional cyclists, race footage, and sports documentaries. On the right, the Slide Panel displays a slide deck on cycling, covering topics such as “Famous Climbs in France,” “Competitive Events,” and “Training for Cycling.”} 
  \label{fig:appendix_task4_P06}
\end{figure}

\begin{figure}[H]
    \centering
  \includegraphics[width=0.99\linewidth]{Figures/Appendix/P07.png}
  \caption{Semi-structured task outcome of P07}
  \Description{The figure shows the IntentTagger interface for participant P07’s semi-structured task. The Deck Steering Board on the left contains tags for Narrative, Visual Style, and Content Sources. The Narrative section includes tags such as “Topic: Skateboarding,” “Culture: Street vs. Park,” and “Introduction: History of Skateboarding.” The Visual Style section includes tags like “Font: Comic Sans,” “Theme: Graffiti Art,” and “Layout: Dynamic and Energetic.” The Content Sources include videos, articles, interviews with professional skateboarders, and skateboarding culture websites. On the right, the Slide Panel shows a slide deck on skateboarding culture, with sections on “Introduction to Skateboarding,” “Street Skateboarding Culture,” and “Park Skateboarding Culture.” } 
  \label{fig:appendix_task4_P07}
\end{figure}

\begin{figure}[H]
    \centering
  \includegraphics[width=0.99\linewidth]{Figures/Appendix/P08.png}
  \caption{Semi-structured task outcome of P08}
  \Description{The figure shows the IntentTagger interface for participant P08’s semi-structured task. The Deck Steering Board on the left contains tags related to Narrative, Visual Style, and Content Sources. The Narrative section includes tags such as “Topic: How to play TTRPGs,” “Second topic: Unique game mechanics,” and “Conclusion: Resources and communities.” The Visual Style section includes tags like “Layout: Storybook format” and “Imagery: Fantasy artwork.” The Content Sources include videos, books, and role-playing forums related to tabletop RPGs. On the right, the Slide Panel shows a slide deck on how to play TTRPGs, covering topics like “How to Play TTRPGs,” “Ways to Start Playing Today,” and “Additional Resources.” } 
  \label{fig:appendix_task4_P08}
\end{figure}

\begin{figure}[H]
    \centering
  \includegraphics[width=0.99\linewidth]{Figures/Appendix/P09.png}
  \caption{Semi-structured task outcome of P09}
  \Description{The figure shows the IntentTagger interface for participant P09’s semi-structured task. The Deck Steering Board on the left contains tags related to Narrative, Visual Style, and Content Sources. The Narrative section includes tags like “Topic: Taekwondo,” “Benefits: Physical Fitness,” “Benefits: Learn Self-Defense,” and “Introduction: Principles of Taekwondo.” The Visual Style section includes tags such as “Theme: Energetic,” “Colors: Black and White,” and “Illustrations: Kicks.” The Content Sources include statistics on participation, photos from competitions, and training videos. On the right, the Slide Panel displays a slide deck on the benefits of Taekwondo, covering topics such as self-defense, physical fitness, mental health, and body awareness.} 
  \label{fig:appendix_task4_P09}
\end{figure}

\begin{figure}[H]
    \centering
  \includegraphics[width=0.99\linewidth]{Figures/Appendix/P10.png}
  \caption{Semi-structured task outcome of P10}
  \Description{ The figure shows the IntentTagger interface for participant P10’s semi-structured task. The Deck Steering Board on the left contains tags related to Narrative, Visual Style, and Content Sources. The Narrative section includes tags like “Topic: New Tennis Club in Town,” “Call to Action: Join Now,” and “Audience: College Students.” The Visual Style section includes tags such as “Font: Neo Modern,” “Theme: Sporty,” and “Color: Green and Blue.” The Content Sources include introductory videos, student surveys, brochures with club details, and expert opinions. On the right, the Slide Panel displays a slide deck introducing the tennis club, covering sections like “Why Join Our Club?” “Club Launch Party,” and “Benefits for College Students.” } 
  \label{fig:appendix_task4_P10}
\end{figure}

\begin{figure}[H]
    \centering
  \includegraphics[width=0.99\linewidth]{Figures/Appendix/P11.png}
  \caption{Semi-structured task outcome of P11}
  \Description{The figure shows the IntentTagger interface for participant P11’s semi-structured task. The Deck Steering Board on the left contains tags related to Narrative, Visual Style, and Content Sources. The Narrative section includes tags like “Topic: Trombones,” “Content: History of Trombones,” and “Content: Types of Music It Appears In.” The Visual Style section includes tags such as “Color Scheme: Dark and Moody,” “Typography: Sans Serif,” and “Imagery: Historic Photos.” The Content Sources include music journals, music history books, and interviews with trombone experts. On the right, the Slide Panel displays a slide deck on the trombone, covering sections like “Introduction to Trombones,” “When the Trombone Was Invented,” and “Types of Music It Appears In.” } 
  \label{fig:appendix_task4_P11}
\end{figure}

\begin{figure}[H]
    \centering
  \includegraphics[width=0.99\linewidth]{Figures/Appendix/P12.png}
  \caption{Semi-structured task outcome of P12}
  \Description{The figure shows the IntentTagger interface for participant P12’s semi-structured task. The Deck Steering Board on the left contains tags related to Narrative, Visual Style, and Content Sources. The Narrative section includes tags such as “Topic: Paddle Boarding,” “Call to Action: Try Paddle Boarding,” and “Audience: Outdoor Enthusiasts.” The Visual Style section includes tags like “Theme: Nautical,” “Colors: Blue and White,” and “Imagery: Ocean Waves.” The Content Sources include videos, tips for beginners, and personal stories from paddle boarding enthusiasts. On the right, the Slide Panel displays a slide deck on paddle boarding, covering sections like “What is Paddle Boarding?” “Benefits of Paddle Boarding,” and “Tips for Beginner Paddle Boarders.” } 
  \label{fig:appendix_task4_P12}
\end{figure}
