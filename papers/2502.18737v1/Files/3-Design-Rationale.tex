\section{Exploring Design Principles for GenAI-supported Rich Content Creation}

In this section, we lay out the challenges of rich content document creation in the context of slide creation and how these informed the design principles that led us to intent tagging as an interaction paradigm for enabling novel human-GenAI co-creation workflows.

Slide deck creation is a complex task that requires balancing multiple facets, such as narrative coherence, visual style consistency, and the integration of diverse content sources \cite{yates_powerpoint_2007, reynolds_presentation_2020, anholt_dazzle_2006}. 
These elements must not only align individually but also work together harmoniously to deliver an impactful presentation. The challenges arise from the need to manage and sequence multimedia content while simultaneously crafting a narrative that flows logically across slides. Achieving this requires careful consideration of storytelling, visual design, and the inclusion of relevant data or references, often making the task overwhelming, especially under time constraints or when dealing with open-ended creative briefs.


One of the core difficulties in slide deck creation is the need to manage these distinct yet interdependent elements—\textit{narrative}, \textit{visual style}, and \textit{content sources}—without losing sight of the overall presentation goals. 

Traditional tools often compartmentalize these tasks, forcing users to switch between different modes or tabs, which can disrupt the creative process. Linear workflows or predefined templates, while useful in some cases, often fail to accommodate the nuanced adjustments needed to refine a slide deck to a professional standard.

Moreover, as discussed in the previous section, the integration of Generative AI (GenAI) into slide deck creation introduces both opportunities and challenges. Current GenAI approaches tend to offer either (semi-)automated solutions with little ability for users to steer the content generation, design galleries with limited pre-defined option sets, or chat-based interactions that can feel too linear, unstructured, and limited for exploring alternatives.



\begin{figure*}[ht!]
  \centering
  \includegraphics[width=\linewidth]{Figures/System-UI.png}
  \caption{Screenshots of the \textit{IntentTagger} system: (Left window) A user creates a slide presentation using the Slide Deck Steering Canvas with (1) active tags in the three circular tag groups surrounded by (2) inactive system-suggested tags. (3) The generated outline and slides are displayed in the tabbed sidebar. (Right window) The user makes (scoped) adjustments to a single slide via the Slide Steering Overlay and (4) explores generated slide variations. (5) Then, local slide changes can be accepted or applied to the entire deck.}
  \Description{This figure shows the interface of the IntentTagger system with two main sections: the Slide Deck Steering Canvas on the left and the Slide Steering Overlay on the right. (1) Active Tags (Bottom Left): Users have selected active tags, such as "Sustainability Initiatives" under the "Narrative" group. (2) Suggested Tags: The system suggests additional tags like "Layout: Minimalist." (3) Generated Outline and Slides: The sidebar shows the generated outline and slide previews. (4) Slide Variations: The user explores slide variations for individual slides in the Slide Steering Overlay, such as a "Company" slide showing sustainability targets. (5) Apply Local Changes: Users can apply changes to individual slides or the entire deck. The figure demonstrates how users organize tags and make fine-tuned slide adjustments using intent tags and overlay features.}
  \label{fig:system-ui}
\end{figure*}


\subsection{Design Principles}

Based on these requirements, we developed the following interaction design principles for GenAI-supported slide creation:

\begin{itemize}[font=\bfseries,
  align=left]
    \item[DP1] \textbf{Enabling flexible, non-linear, and iterative workflows } \\
        To cater to users' diverse slide creation needs and individual working styles, users should be empowered to start and refine their presentations in a variety of ways. Whether beginning with an outline, building on an existing deck, crafting a single slide, or incorporating content from other documents, users should have the flexibility to approach their tasks from any angle. Additionally, the interface should allow seamless transitions between different content views—such as slide, deck, or outline—enabling GenAI-assisted iterative enhancements at any stage of the creation process.  
        
    \item[DP2] \textbf{Accommodating diverse steering input types} \\ 
    Slide creation often requires the integration of multiple sources of information, ranging from explicit, well-defined data (e.g., an annual sales report) to more abstract, nuanced considerations (e.g., audience-specific messaging). To ensure that GenAI-generated outputs align closely with user intentions, the interface should facilitate the input of a wide range of content types, including natural language descriptions, existing documents, images, and other relevant media. By enabling users to provide rich, varied inputs, the system can better interpret and fulfill the specific needs of each presentation.   
    
\vspace{\baselineskip} 
    
     \item[DP3] \textbf{Supporting intent expression across varying levels of abstraction} \\
    In many slide creation scenarios, users may have clear, specific ideas about certain aspects of their presentation, while other elements remain less defined or harder to articulate. To accommodate this variability, a GenAI interface should allow users to express their intent at different levels of abstraction—ranging from detailed, precise instructions to more general, high-level directives. This flexibility not only helps users articulate their ideas more effectively but also supports iterative exploration and refinement of their presentation content.     

    \item[DP4] \textbf{Leveraging AI for contextual content and terminology suggestions  } \\
    To assist users in overcoming challenges like the "blank page syndrome" or finding terminology to describe their intent, the interface should harness the associative power of LLMs to provide contextual suggestions and inspiration. By offering alternative ideas, vocabulary, and creative prompts, the system can stimulate users' thinking, helping them to refine their content and explore new directions in an iterative process.

    \item[DP5] \textbf{Providing pre-computed real-time previews} \\
    To allow users to better and quicker anticipate how specific choices will impact generated slides, the tool should provide pre-computed real-time previews to, for example, let users rapidly explore how different fonts or colors would look on a given slide.    



    
\end{itemize}
