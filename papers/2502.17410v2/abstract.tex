Large Language Models (LLMs) have demonstrated remarkable success across various domains, yet their optimization remains a significant challenge due to the complex and high-dimensional loss landscapes they inhabit. While adaptive optimizers such as AdamW are widely used, they suffer from critical limitations, including an inability to capture interdependencies between coordinates and high memory consumption. Subsequent research, exemplified by SOAP, attempts to better capture coordinate interdependence but incurs greater memory overhead, limiting scalability for massive LLMs. An alternative approach aims to reduce memory consumption through low-dimensional projection, but this leads to substantial approximation errors, resulting in less effective optimization (e.g., in terms of per-token efficiency). In this paper, we propose COSMOS, a novel hybrid optimizer that leverages the varying importance of eigensubspaces in the gradient matrix to achieve memory efficiency without compromising optimization performance. The design of COSMOS is motivated by our empirical insights and practical considerations. Specifically, COSMOS applies SOAP to the leading eigensubspace, which captures the primary optimization dynamics, and MUON to the remaining eigensubspace, which is less critical but computationally expensive to handle with SOAP. This hybrid strategy significantly reduces memory consumption while maintaining robust optimization performance, making it particularly suitable for massive LLMs. Numerical experiments on various datasets and transformer architectures are provided to demonstrate the effectiveness of COSMOS. Our code is available at \url{https://github.com/lliu606/COSMOS}.