\section{Impact}
\label{sec:impact}

AVs will soon enter our cities and start making routing decisions. Inevitably, some kind of algorithms will be used by commercial fleet operators (like Tesla or Waymo) to select, for their clients, optimal routes to get to their destinations. This is a complex optimization problem, hardly solvable with analytical methods for real-world scenarios where thousands of agents simultaneously select among hundreds of available routes in the networks with tens of thousands of links. Classic discrete optimization techniques fail, opening the way to machine learning alternatives, and as a natural candidate for dynamic systems, reinforcement learning. 


% \subsection{Insights}
% \label{sec:conceptualization}

RouteRL facilitates the modeling and prototyping of the collective route choices of individual AVs and fleets of AVs, by integrating human behavioral models and MARL algorithms into an agent-based traffic simulation. It supports a broad range of experimental schemes, where drivers transition into AVs, which then behave selfishly, collaboratively, or even maliciously. 
Its outputs are aimed to reveal the diverse socioethical aspects of this transition. By reporting various KPIs such as travel times, rewards, and agent preference changes, RouteRL reports the impact on congestion, travel times, and mileage, along with potentially undesirable social consequences. 
RouteRL scales to experiments on real-size networks with hundreds of agents. Its flexible parameterization, standard input/output formats, and integration of popular frameworks within its workflow allow researchers and system designers to pursue realistic and significant research questions, and ultimately, derive meaningful results and actionable insights.

The introduction of AVs into our cities opens new research questions that shall be answered upfront before commercial providers introduce black-box solutions without social control and participation. With the modeling framework of RouteRL we can, to some extent, answer experimentally the research questions such as (adapted from \cite{jamroz}):
\begin{compactitem}
    \item Will routing decisions of AVs differ substantially from choices made by human drivers?
    \item Will the impact on travel times depend on the AV strategy?
    \item Can a large fleet of AVs improve travel times for all the drivers?
    \item Does the benefits of AVs come at a price of equity?
    \item Are human populations prone to exploitation by collective AV fleets?
    \item Which MARL algorithms reliably solve AV routing problems?
    \item Can AV fleets contribute to policy goals (like $CO_2$ emissions or sustainability)?
    \item Can malicious AV strategies be detected?
    
\end{compactitem}

The community lacks tools to answer questions like the above reliably, and
%Even though MARL research is supported by a variety of off-the-shelf benchmarks (notably including ), the methodological contributions in MARL research are not typically validated in comparably complex settings. 
RouteRL contributes with an evaluation ground, allowing more comprehensive assessments by incorporating the aforementioned complexities of the multi-agent route choice problem  %In each experiment, RouteRL provides analytical insights, statistics, and performance indicators, 
    %(such as travel times, route choices, and other key system metrics)
    %providing a robust foundation 
    for assessment, development, and comparative analyses.

% %\paragraph{Applicability for future problems}
% %\label{sec:benchmarking}
%     \begin{compactitem}
%         \item will AVs make significantly different 
%         Benchmark the efficacy of MARL solutions in various traffic scenarios.
%         \item Assess the proposed solution of the broad set of KPIs.
%         \item Compare various fleet strategies (social or malicious) in various market compositions.
%         \item Propose socially-aware collaboration protocols for fleet managers.
%         \item Demonstrate how 
%         %\item Develop a theoretical understanding of equilibrium behaviors and stability in (mixed) multi-agent traffic systems through empirical validation.
%         \item Upfront identify the malicious AV strategies and restrict them.
%     \end{compactitem}



%\paragraph{Comprehensive experiments}
While the day-to-day routing problem for AV fleets sharing road networks with human drivers can be formulated as a multi-agent decision-making process and thus solved with MARL, standard solutions are missing. The scientific community lacks a framework on which the algorithms can be tested, developed, and compared. The proper testbed must be interdisciplinary and realistically represent the road networks, travel demand patterns, traffic flow dynamics, and system performance. The performance needs to include views of individual agents (humans, whether driving or being driven in AVs), fleet operators, traffic managers, and policymakers. The desired algorithmic solutions shall, apart from being computationally efficient, be socially responsible, balancing often opposing objectives of various parties involved in the complex social system of urban traffic. RouteRL is designed to serve as a testing ground for new, hopefully, efficient and reliable solutions.



% In this section we need to highlight how our software adds to existing research. We need references and good comparisons.

%\paragraph{Developing MARL algorithms}
% \label{sec:marl_impact}
% Multi-agent route choice as a valuable benchmark for MARL algorithm development.
    %RouteRL's integration of MARL and multi-agent route choice constitutes a valuable assessment ground for MARL algorithms. 

The multi-agent route choice problem poses a significant algorithmic challenge due to its multifaceted and dynamic nature. It is interdisciplinary and requires expertise in traffic engineering, microeconomics of discrete choices, urban policies, optimization, and machine learning. It suffers from the curse of dimensionality, the environment is at most partially observable and %non-stationarityFirst, changing driver preferences dynamically alters traffic conditions, which makes the problem highly 
non-stationary, which requires adaptive solutions. Maximizing the fleet's shared reward requires communication protocols and coordination mechanisms which, for real-world AV scenarios with high-dimensional state-action spaces, are difficult to scale. Finally, the presence of human drivers,
%and  Secondly, the drivers may lack access to global information such as real-time traffic conditions, requiring them to make decisions under uncertainty. Even if they are provided with external information, it is often less explicit than in other MARL settings. 
%For instance, in multi-agent robotic tasks in enclosed environments, designers can equip each robot with optical sensors to capture detailed information about other agents and the environment state, which is a convenience not easily achievable in the context of urban route choice. 
%In traffic systems, summarizing the global state in a compact and computationally efficient format becomes increasingly challenging with the growing network and population sizes. Moreover, optimizing route choices for a group of agents requires strategic interactions among agents, where individual optimal choices can lead to suboptimal global outcomes, often requiring a balance between self-serving and cooperative policies. This challenge is compounded by the presence of human drivers
 who are bounded-rational decision-makers with heterogenous and non-deterministic route-choice behavior, makes the setting non-deterministic, specifically if we include human adaptation to AV actions. 
Altogether this makes the problem significantly harder than standard MARL benchmarks (like \cite{pettingzoo, jaxmarl, nmmo, ma_comp}) and calls for a unified development framework like RouteRL.

