\section{Conclusions}
\label{sec:conclusions}

In this paper, we introduced \textbf{RouteRL}, a MARL framework for multi-agent urban route choice. RouteRL is a python open-source package aimed to enhance our understanding of future urban mobility with AVs. It supports various experimental settings through configurable traffic networks, human behavioral models, AV market configuration, and AV algorithms. With its modular architecture and integration with standardized frameworks, RouteRL offers reusability and simplifies experiment setup. Moreover, full integration with the TorchRL library allows the development and testing of state-of-the-art MARL algorithms in solving the route choice problem. By providing analytical insights from experiments, RouteRL enables researchers to systematically underpin their findings and explore alternative strategies for more efficient future traffic systems.
We believe that a class of open research problems arising around the routing behavior of AVs in future cities can be answered with the help of RouteRL.

To ensure accessibility and alignment with community standards, we provide RouteRL with comprehensive documentation, release it under an open-source license, host it in a public repository and, for faster adoption, we provide a reusable code capsule with a reproducible experiment. 


\section*{Acknowledgements}
\label{sec:acknowledgements}
This research is financed by the European Union within the Horizon Europe Framework Programme, ERC Starting Grant number 101075838: COeXISTENCE.


\section*{Declaration of generative AI and AI-assisted technologies in the writing process}

During the preparation of this work, the authors used ChatGPT (GPT-4o) for proofreading and minor style corrections. After using this tool/service, the authors reviewed and edited the content as needed and take full responsibility for the content of the published article.