\section{Reproducibility Statement}

\noindent \textbf{Novel Models and Algorithms.}
Our method is composed of two novel designs: 
1) The tracking controller training strategy which combines RL and IL. We have provided detailed explanations of the algorithm in Section~\ref{sec:method_il_rl} and Appendix~\ref{sec:supp_method}. Besides, we have included the source code in the Supplementary Materials, where the implementations of the algorithm are contained. Please refer to \emph{Key Implementations} section in the file \emph{DexTrack-Code/README.md} in our Supplementary Materials. 
% The \rep representation. We have provided detailed explanations of the representation in Section~\ref{sec:geneoh} and Appendix~\ref{sec_appen_rep_design}. Besides, we have included the source code in the Supplementary Materials, where the implementations of \rep are contained. Please refer to \emph{Key Implementations} section in the file \emph{Geneoh-Diffusion-Code/README.md} in our Supplementary Materials. 
2) The homotopy optimization scheme, the homotopy path searching strategy as well as the homotopy path generator. 
% tracking curriculum and the tracking curricula mining strategy as well as the curriculum scheduler. 
We have provided detailed explanations in Section~\ref{sec:method_data} and Appendix~\ref{sec:supp_method}. Besides, we have included the source code in the Supplementary Materials. Please refer to the \emph{Key Implementations} section in the file \emph{DexTrack-Code/README.md} in our Supplementary Materials. 
3) The iterative optimization approach. We have provided detailed explanations in Section~\ref{sec:method_iteration} and Appendix~\ref{sec:supp_method}.  Besides, we have included the source code in the Supplementary Materials. Please refer to the \emph{Key Implementations} section and \emph{Usage} section in the file \emph{DexTrack-Code/README.md} in our Supplementary Materials. 
% The progressive HOI denoising method. We have provided detailed explanations of the denoising method in Section~\ref{sec:diffusion} and Appendix~\ref{sec_genenoh_diffusion_details}. Model, training, and evaluation details are included in Appendix~\ref{sec_appen_model_details} and~\ref{sec_appen_training_details}. Besides, we have included the source code in the Supplementary Materials. Please refer to the \emph{Key Implementations} section and \emph{Usage} section in the file \emph{Geneoh-Diffusion-Code/README.md} in our Supplementary Materials. 

% \noindent \textbf{Theoretical Results.}
% Please refer to section~\ref{sec_appen_rep_design} in the Appendix for the proof of the property of leveraging the stage-wise denoising can clean the noisy input where each stage would not comprise the naturalness achieved after previous denoising stages. 

% \noindent \textbf{Datasets.}
% Please refer to Section~\ref{sec_exp_settings} and Appendix~\ref{sec_appen_datasets} for a detailed explanation of the datasets used. The data preprocessing part can be found in the \emph{Data Preprocessing} section in the file \emph{Geneoh-Diffusion-Code/README.md} in our Supplementary Materials. 

\noindent \textbf{Experiments.} 
For experiments, we provide 1) Source code.
Please refer to the folder \emph{DexTrack-Code} in our Supplementary Materials for details. 
2) Experimental details of our method and compared baselines. Please refer to the Section~\ref{sec:exp_setting} and in the Appendix~\ref{sec:supp_exp_details} for details. 
We also provide related information and instructions in the \emph{DexTrack-Code/README.md} file. 
