\begin{figure}[htbp]
  % \includegraphics[width=\textwidth]{figs/teaser_8.pdf}
  % \includegraphics[width=\textwidth]{figs/teaser_15.pdf}
  \includegraphics[width=\textwidth]{figs/teaser_rb_6.pdf}
  \vspace{-23pt}
  \caption{ \footnotesize
  % By only Training on limited data, 
 %  \href{https://projectwebsite7.github.io/gene-dex-manip/}{
 % \eric{It is very hard to see the teaser figure even after zooming in. The text in the figure should be at least as large as the caption text size. The images are so small and the highlight can hardly be seen even after zooming in by 200\%. The captions should also highlight the key: our neural tracking controller can track diverse and novel human hand object manipulation references once trained.} \modelname}
 % \eric{The figure font size is still much smaller than t  he caption font size. Please adjust and increase the font size. Also, I think for kinematic references, maybe it is better to show the original human motion. The red circles are highlighting something invisible... We should reduce the number of figures to increase the size of each figure. Please make sure the figure is still visible when printed out using A4 paper.}
 \href{https://projectwebsite7.github.io/gene-dex-manip/}{\modelname} learns a generalizable neural tracking controller for dexterous manipulation from human references. 
 % \textcolor{myblue}{Our neural tracking controller (\emph{Fig. (a)}) can generalize to track novel and difficult manipulation trajectories with thin objects as well as intricate interactions with complex object movements and intriguing in-hand manipulations (\emph{Fig. (b)}), exhibits resilience to large kinematics noise, and is valuable in real-world applications (\emph{Fig. (c)}). }
 % \textcolor{myblue}{Our neural tracking controller takes kinematic references as input, and outputs hand action commands, aiming to make the tracking results close to input references (\emph{Fig. (a)}). It can generalize to track novel and difficult manipulation trajectories with thin objects as well as intricate interactions with complex object movements and intriguing in-hand manipulations (\emph{Fig. (b)}), exhibits resilience to large kinematics noise, and is valuable in real-world applications (\emph{Fig. (c)}). }
  % \textcolor{myblue}{Our neural tracking controller processes kinematic references to generate hand action commands, ensuring close tracking of input trajectories (\emph{Fig. (a)}). It generalizes to challenging tasks, such as thin-object manipulation and intricate in-hand interactions with complex movements (\emph{Fig. (b)}), demonstrating robustness to significant kinematics noise and practical utility in real-world scenarios (\emph{Fig. (c)}). }
    \textcolor{myblue}{
    % \href{https://projectwebsite7.github.io/gene-dex-manip/}{\modelname} learns a generalizable neural tracking controller for dexterous manipulation from human references. 
    It generates hand action commands from kinematic references, ensuring 
    % processes kinematic references to generate hand action commands, ensuring
    close tracking of input trajectories (\emph{Fig. (a)}), generalizes to novel and challenging tasks involving thin objects, complex movements and intricate in-hand manipulations (\emph{Fig. (b)}), 
    % , such as thin-object manipulation and intricate in-hand interactions with complex movements (\emph{Fig. (b)}), 
    and demonstrates robustness to large kinematics noise and utility in real-world scenarios (\emph{Fig. (c)}).
    Kinematic references are illustrated in \textcolor{orange}{orange rectangles} and \textcolor{orange}{background}. 
    % Our neural tracking controller processes kinematic references to generate hand action commands, ensuring close tracking of input trajectories (\emph{Fig. (a)}). It generalizes to challenging tasks, such as thin-object manipulation and intricate in-hand interactions with complex movements (\emph{Fig. (b)}), demonstrating robustness to significant kinematics noise and practical utility in real-world scenarios (\emph{Fig. (c)}). 
    }
 % Our neural controller can adeptly generalize to track novel manipulation trajectories with unseen objects, involving difficult interactions with thin objects (\emph{Fig. (a)}), intricate tool-using with intriguing re-orientations (\emph{Fig. (b)}),  exhibits high robustness towards large kinematics noise, and is valuable in real-world applications (\emph{Fig. (c)}). 
 % Figures in \textcolor{orange}{orange rectangles} with \textcolor{orange}{orange background} illustrate kinematic references.
 % ~\eric{should use more bold color instead of gray. it is very hard to see the gray rectangles or background.} 
 % with complex movements and 
 % The controller can 
 % various and novel manipulation trajectories involving difficult manipulations with thin objects (\emph{Fig. (a)}), intricate tool-using with complex movements and subtle in-hand re-orientations (\emph{Fig. (b)}),  exhibits high robustness towards large kinematics noise, and is valuable in real-world applications (\emph{Fig. (c)}). Figures in \textcolor{gray}{gray rectangles} with \textcolor{gray}{gray background} illustrate kinematic references. 
 % and out-of-domain (OOD) interactions involving objects from a brand new category (\emph{Fig. (c)}) 
 % The controller can adeptly control a dexterous robot hand to accomplish various challenging manipulations involving novel and difficult manipulations with complex object movements and intricate in-hand re-orientations (\emph{Fig. (a)}), new and challenging object geometries (\emph{Fig. (b)}),  and is robust to large kinematics noise and out-of-domain (OOD) interactions involving objects from a brand new category (\emph{Fig. (c)}). 
 % dexterous manipulation tracking controller from human manipulation references. 
  }
  \vspace{-1pt}
  \label{fig_intro_teaser}
\end{figure}