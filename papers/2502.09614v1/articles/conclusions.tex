\section{Conclusions and Limitations}

% In this work, we propose \modelname to tackle the challenge of developing a generalizable tracking controller for dexterous manipulation. By leveraging abundant high-quality robot tracking demonstrations and employing a pre-trajectory tracking scheme to mine imitation-ready data, we progressively refine the controller using a bootstrapping strategy. Extensive experiments validate the effectiveness of our approach, establishing it as a strong foundation for future advancements in generalizable tracking control. \noindent\textbf{Limitations.} The primary limitation of our method is the efficiency of acquiring high-quality demonstrations. Currently, searching for optimal homotopy optimization paths to improve demonstrations is time-consuming. Future research could focus on finding faster, approximate homotopy optimization methods to accelerate the training process.

We propose \modelname to develop a generalizable tracking controller for dexterous manipulation. Leveraging high-quality tracking demonstrations and a pre-trajectory tracking scheme, we refine the controller through bootstrapping. Extensive experiments confirm its effectiveness, establishing a strong foundation for future advancements. \noindent\textbf{Limitations.} A key limitation is the time-consuming process of acquiring high-quality demonstrations. Future work could explore faster, approximate methods for homotopy optimization to speed up training.


%%%%%%%%%%%%%%%%%%%%%%%%%%%%%%%%%%%%%%%%%%%%%%%%%%%%%%%%%%%%%%%
% In this work, we propose \modelname to address the challenge of developing a generalizable tracking controller for dexterous manipulation. 
% By leveraging abundant high-quality robot tracking demonstrations to train the tracking controller and employing a pre-trajectory tracking scheme that effectively mines high-quality imitation-ready data, we gradually solve problems in each side and finally arrive at a strong tracking controller via a bootstrapping strategy. 
% Extensive experiments demonstrate the effectiveness of our method, evidencing it as a promising foundation for future developments in generalizable tracking control. 
% \noindent\textbf{Limitations.} The main limitation of our method lies in the efficiency of acquiring high-quality demonstrations. Currently, searching for good homotopy optimization paths to improve demonstrations is a little bit time-consuming. A meaningful future research direction is approximately finding beneficial homotopy optimization in a time-efficient way, therefore accelerating the homotopy generator's training. 
%%%%%%%%%%%%%%%%%%%%%%%%%%%%%%%%%%%%%%%%%%%%%%%%%%%%%%%%%%%%%%%

% The main limitation of our approach lies in the efficiency of acquiring high-quality action-labeled data. Currently, searching for good tracking curricula for training the curriculum scheduler is the most time-consuming part of our method. How to improve the efficiency such as using approximate curricula for training the scheduler is an interesting future research direction. 


% To address the difficulties in acquiring abundant and high-quality data 
% universal tracking controller for dexterous robot hand manipulation. By combining reinforcement learning with high-quality action-labeled data, we unleash the possibility of using RL to train a tracking controller for solving difficult manipulation problems. To address the ``chicken-and-egg'' problem by 1) training the controller using both RL and action-labeled datasets, and 2) optimizing for high-quality action-labeled data effectively in an acceptable time-budget, we design an interactive approach. 
% Leveraging the joint power of a learned tracking curriculum and data guidance, we can effectively improve the quality of the labeled dataset. We then progressively improve the capability of the tracking controller by curating data with higher quality. Extensive experiments demonstrate the effectiveness of our method, evidencing it as a promising foundation for future developments in universal tracking control.

% \noindent\textbf{Limitations.} The main limitation of our approach lies in the efficiency of acquiring high-quality action-labeled data. Currently, searching for good tracking curricula for training the curriculum scheduler is the most time-consuming part of our method. How to improve the efficiency such as using approximate curricula for training the scheduler is an interesting future research direction. 
% The process of curating this data is resource-intensive, requiring significant time and effort to generate diverse and accurate labels that effectively guide the training of the tracking controller. While our method achieves notable performance improvements, the dependence on large amounts of labeled data could hinder scalability, particularly when expanding to more complex tasks or environments. Future work could explore strategies to reduce the reliance on labeled data, such as incorporating self-supervised learning or more efficient data-generation techniques, to enhance both the practicality and scalability of the approach.
% we can create enough high-quality data in an acceptable time budget for training 
% demonstrates significant improvements in manipulation performance across a range of novel and expressive tasks. Extensive experiments, both in simulation and real-world environments, validate the effectiveness of our approach, with notable improvements in success rates over existing baselines. This research opens new avenues for advancing dexterous robot manipulation, providing a promising foundation for future developments in universal tracking control.