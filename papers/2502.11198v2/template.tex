\documentclass{article}


\usepackage{arxiv}

\usepackage[utf8]{inputenc} % allow utf-8 input
\usepackage[T1]{fontenc}    % use 8-bit T1 fonts
\usepackage{hyperref}       % hyperlinks
\usepackage{url}            % simple URL typesetting
\usepackage{booktabs}       % professional-quality tables
\usepackage{amsfonts}       % blackboard math symbols
\usepackage{nicefrac}       % compact symbols for 1/2, etc.
\usepackage{microtype}      % microtypography
\usepackage{lipsum}
\usepackage{graphicx}
\usepackage{array}
\usepackage{enumitem}
\usepackage{hyperref}
%%%%%%%%%%%%%%%%%%%%%%%%%%%%%%%%%%%%%%%%%%%%%%%%%%%%%%%%%%%%%%%%%%%%%%%%%
% This section is based on the bbk10.clo file
% of Palash Baran Pal's bangtex
% http://www.saha.ac.in/theory/palashbaran.pal/bangtex/bangtex.html
%%%%%%%%%%%%%%%%%%%%%%%%%%%%%%%%%%%%%%%%%%%%%%%%%%%%%%%%%%%%%%%%%%%%%%%%%

\def\sbng{\bngviii}
\def\tbng{\bngvi}
\def\bng{\bngx}
\def\lbng{\bngxiv}
\def\Lbng{\bngxviii}
\def\LBng{\bngxxii}
\def\hbng{\bngxxv}
\def\Hbng{\bngxxx}
%
\def\sbns{\bnsviii}
\def\tbns{\bnsvi}
\def\bns{\bnsx}
\def\lbns{\bnsxiv}
\def\Lbns{\bnsxviii}
\def\LBns{\bnsxxii}
\def\hbns{\bnsxxv}
\def\Hbns{\bnsxxx}
%
\def\sbnw{\bnwviii}
\def\tbnw{\bnwvi}
\def\bnw{\bnwx}
\def\lbnw{\bnwxiv}
\def\Lbnw{\bnwxviii}
\def\LBnw{\bnwxxii}
\def\hbnw{\bnwxxv}
\def\Hbnw{\bnwxxx}


%%%%%%%%%%%%%%%%%%%%%%%%%%%%%%%%%%%%%%%%%%%%%%%%%%%%%%%%%%%%%%%%%%%%%%%%%
% This section is based on the bangfont.tex file
% of Palash Baran Pal's bangtex
% http://www.saha.ac.in/theory/palashbaran.pal/bangtex/bangtex.html
%%%%%%%%%%%%%%%%%%%%%%%%%%%%%%%%%%%%%%%%%%%%%%%%%%%%%%%%%%%%%%%%%%%%%%%%%

%%
%% Defining the normal bangla fornts
%%

\font\bngv=bang10 scaled 500
\font\bngvi=bang10 scaled 600
\font\bngvii=bang10 scaled 700
\font\bngviii=bang10 scaled 800
\font\bngix=bang10 scaled 900
\font\bngx=bang10
\font\bngxi=bang10 scaled 1100
\font\bngxii=bang10 scaled 1200
\font\bngxiv=bang10 scaled 1400
\font\bngxviii=bang10 scaled 1800
\font\bngxxii=bang10 scaled 2200
\font\bngxxv=bang10 scaled 2500
\font\bngxxx=bang10 scaled 3000

%%
%% Defining the slanted bangla fonts
%%
\font\bnsv=bangsl10 scaled 500
\font\bnsvi=bangsl10 scaled 600
\font\bnsvii=bangsl10 scaled 700
\font\bnsviii=bangsl10 scaled 800
\font\bnsix=bangsl10 scaled 900
\font\bnsx=bangsl10
\font\bnsxi=bangsl10 scaled 1100
\font\bnsxii=bangsl10 scaled 1200
\font\bnsxiv=bangsl10 scaled 1400
\font\bnsxviii=bangsl10 scaled 1800
\font\bnsxxii=bangsl10 scaled 2200
\font\bnsxxv=bangsl10 scaled 2500
\font\bnsxxx=bangsl10 scaled 3000

%%
%% Defining the wide bangla fonts
%%
\font\bnwv=bangwd10 scaled 500
\font\bnwvi=bangwd10 scaled 600
\font\bnwvii=bangwd10 scaled 700
\font\bnwviii=bangwd10 scaled 800
\font\bnwix=bangwd10 scaled 900
\font\bnwx=bangwd10
\font\bnwxi=bangwd10 scaled 1100
\font\bnwxii=bangwd10 scaled 1200
\font\bnwxiv=bangwd10 scaled 1400
\font\bnwxviii=bangwd10 scaled 1800
\font\bnwxxii=bangwd10 scaled 2200
\font\bnwxxv=bangwd10 scaled 2500
\font\bnwxxx=bangwd10 scaled 3000


%%
%% Inhibiting linebreak within words
%%
%\hyphenpenalty=10000 \pretolerance=-1 \tolerance=10000

%%
%% Defining the macro for e-kar, i-kar etc
%%
\def\*#1*#2{o\null{#2}{#1}}

%%
%% Redefining some macros to make them consistent with bangla fonts
%%
\def\d#1{\oalign{\smash{#1}\crcr\hidewidth{$\!$\rm.}\hidewidth}}


%%
%% Emulating the bold font
%%
\def\sh#1{\setbox0=\hbox{#1}%
     \kern-.02em\copy0\kern-\wd0
     \kern.04em\copy0\kern-\wd0
     \kern-.02em\raise.0433em\box0 }
\graphicspath{ {./images/} }


\title{ANCHOLIK-NER: A Benchmark Dataset for Bangla Regional Named Entity Recognition}


\author{
 Bidyarthi Paul \\
  Department of Computer Science and Engineering\\
  Ahsanullah University of Science and Technology\\
  Dhaka, Bangladesh\\
  \texttt{bidyarthipaul01@gmail.com}\\
  \And
 Faika Fairuj Preotee\\
  Department of Computer Science and Engineering\\
  Ahsanullah University of Science and Technology\\
  Dhaka, Bangladesh\\
  \texttt{faikafairuj2001@gmail.com}\\
  \And
 Shuvashis Sarker\\
  Department of Computer Science and Engineering\\
  Ahsanullah University of Science and Technology\\
  Dhaka, Bangladesh\\
  \texttt{shuvashisofficial@gmail.com}\\
  \And
 Shamim Rahim Refat \\
  Department of Computer Science and Engineering\\
  Ahsanullah University of Science and Technology\\
  Dhaka, Bangladesh\\
  \texttt{n.a.refat2000@gmail.com}\\
  \And
 Shifat Islam\\
  Department of Computer Science and Engineering\\
  Bangladesh University of Engineering and Technology\\
  Dhaka, Bangladesh\\
  \texttt{shifat.islam.buet@gmail.com}\\
  \And
 Tashreef Muhammad \\
  Department of Computer Science and Engineering\\
  Southeast University\\
  Dhaka, Bangladesh\\
  \texttt{tashreef.muhammad@seu.edu.bd}\\
  \And
 Mohammad Ashraful Hoque \\
  Department of Computer Science and Engineering\\
  Southeast University\\
  Dhaka, Bangladesh\\
  \texttt{ashraful@seu.edu.bd}\\
  \And
 Shahriar Manzoor \\
  Department of Computer Science and Engineering\\
  Southeast University\\
  Dhaka, Bangladesh\\
  \texttt{smanzoor@seu.edu.bd}\\
}


\begin{document}
\maketitle

\begin{abstract}
ANCHOLIK-NER is a linguistically diverse dataset for Named Entity Recognition (NER) in Bangla regional dialects, capturing variations across Sylhet, Chittagong, Barishal, Noakhali, and Mymensingh. The dataset has around 17,405 sentences, 3,481 sentences per region. The data was collected from two publicly available datasets and through web scraping from various online newspapers, articles. To ensure high-quality annotations, the BIO tagging scheme was employed, and professional annotators with expertise in regional dialects carried out the labeling process. The dataset is structured into separate subsets for each region and is available in CSV format. Each entry contains textual data along with identified named entities and their corresponding annotations. Named entities are categorized into ten distinct classes: Person, Location, Organization, Food, Animal, Colour, Role, Relation, Object, and Miscellaneous. This dataset serves as a valuable resource for developing and evaluating NER models for Bangla dialectal variations, contributing to regional language processing and low-resource NLP applications. It can be utilized to enhance NER systems in Bangla dialects, improve regional language understanding, and support applications in machine translation, information retrieval, and conversational AI.
\end{abstract}

% keywords can be removed
\keywords{Named Entity Recognition \and Low Resource Language \and Bangla Language \and Regional Dialects \and Natural Language Processing }

\section*{Specification Table}

\renewcommand{\arraystretch}{1.5}
\begin{tabular}{p{4cm}  p{11cm}}
    \hline
    Subject & Computer Science Applications \\ 
    Specific subject area & Named Entity Recognition for identifying regional dialect-specific entities in Bangla text data. \\ 
    Type of data & Table (CSV)\\
    Data collection & ANCHOLIK-NER comprises of 3,481 sentences per region. The first 3,000 sentences from each region were from two publicly available datasets and the rest collected through web scraping from newspapers and articles.\\ 
    Data source location & \href{https://data.mendeley.com/datasets/bj5jgk878b/2}{Vashantor}, \href{https://data.mendeley.com/datasets/6ft99kf89b/2}{ONUBAD}, \href{https://allonlinebanglanewspapers.com/}{Bangla Newspapers and Articles}\\ 
    Data accessibility & Repository Name: Mendeley Data (\href{https://data.mendeley.com/datasets/gbkszkt8z3/2}{10.17632/gbkszkt8z3.2}) \\
    Related research article& None\\
     \hline
\end{tabular}

\section{Value of the data}

\begin{itemize}
    \item ANCHOLIK-NER is a corpus for Named Entity Recognition (NER) in Bangla regional dialects, covering diverse linguistic variations across Bangladesh. It provides a valuable resource for developing NER models tailored to low-resource dialects.
    \item The dataset is thoroughly annotated using the BIO scheme by expert linguists specializing in regional Bangla dialects, ensuring high-quality entity recognition.
    \item It includes text from multiple sources, with dialects such as Sylhet, Chittagong, Barishal, Noakhali, and Mymensingh, enhancing its usability for dialect-aware NLP applications.
    \item The dataset can serve as a tool for extracting structured information from the Bangla Regional Dialects into knowledge bases. 
    \item Provided in CSV format, it includes textual data and named entity annotations, making it adaptable to different NER frameworks and machine learning models.
\end{itemize}

\section{Background}

Named Entity Recognition (NER) is a fundamental task in Natural Language Processing (NLP) that aims to identify and classify named entities such as people, locations, organizations, and other predefined categories in a given text. While high-resource languages have seen significant advancements in NER\cite{survey} due to large annotated datasets and linguistic tools, low-resource languages, including Bangla, continue to face substantial challenges. The lack of dedicated datasets, linguistic resources, and orthographic consistency makes entity recognition in these languages difficult. Additionally, research in Bangla has primarily focused on Standard Bangla\cite{BNER}\cite{sameen2023banglaconer}\cite{farhan2024gazetteer}, leaving regional dialects underrepresented despite their linguistic diversity.

Several studies have contributed to NER research for low-resource languages by developing datasets and improving model performance. A study\cite{jia2020entity} introduced two annotated datasets for Chinese, including a Chinese Internet novel corpus and a financial report dataset. They applied a semi-supervised entity-enhanced BERT pre-training approach, integrating lexicon-based knowledge with deep contextual embeddings, demonstrating improved NER performance across multiple domains. On the other hand, NER challenges in African languages were addressed\cite{oyewusi2021naijaner} through the creation of NaijaNER, a dataset covering five Nigerian languages: Nigerian English, Nigerian Pidgin English, Igbo, Yoruba, and Hausa. Their findings indicated that multilingual NER models performed better than language-specific models, highlighting the effectiveness of cross-lingual learning in low-resource settings. For Indo-Aryan languages, an annotated dataset for Assamese was developed\cite{pathak2022asner}, containing 99,000 tokens from speeches and plays. Their study evaluated multiple NER models, including FastText, BERT, XLM-R, FLAIR, and MuRIL, using a BiLSTM-CRF sequence tagging approach. The best performance was achieved using MuRIL embeddings, with an F1-score of 80.69%.

Similarly a study\cite{mundotiya2023development} worked on Bhojpuri, Maithili and Magahi – three low resource Indo-Aryan languages known as Purvanchal languages. They created a large NER dataset with 228,373 tokens for Bhojpuri, 157,468 tokens for Maithili and 56,190 tokens for Magahi with 22 entity labels same as Hindi NER dataset. This alignment facilitates cross-lingual knowledge transfer among related languages. Furthermore, for West Slavic, Upper Sorbian and Kashubian language, NER datasets were created with Czech and Polish corpora so that they can do cross-lingual learning\cite{torge2023named}. The Uzbek NER problem was addressed by creating an annotated dataset of 2,000 sentences and 25,865 words primarily from legal documents and manually crafted texts\cite{mengliev2025comprehensive}. They used the BIOES annotation scheme to improve entity boundary identification and tagging.

In domain of Medical Named Entity Recognition (MNER), a study \cite{muntakim2023banglamedner} published an annotated dataset with over 117,000 tokens in IOB2 format, covering Chemicals and Drugs, Disease and Symptom, and Anatomy entities. Additionaly another work\cite{khan2023nervous} addressed the challenge of medical NER in informal settings by creating a dataset for recognizing named entities in Consumer Health Questions (CHQs). The dataset consists of 31,783 samples from an online public health platform and captured diverse linguistic styles and dialectal variations. Experiments showed the complexity of the dataset and BanglishBERT got the highest F1-score of 56.13\%. 

The ANCHOLIK-NER dataset is unique as there are currently no similar datasets specifically focused on Bangla regional dialects. While NER research has made significant progress in high-resource languages and even Standard Bangla, regional variations such as Sylhet, Chittagong, and Barishal remain largely unexplored. Existing NER datasets primarily cover formal Bangla text. Therefore, leaves a significant gap in recognizing named entities in dialectal and spoken language contexts. Our dataset is the first to address this gap and provides annotated data for regional Bangla varieties.


\section{Data Description}


We have developed the ANCHOLIK-NER, Bangla Regional Named Entity Recognition dataset, which focuses on the regional dialects of \textbf{Sylhet, Chittagong, Barishal, Noakhali,} and \textbf{Mymensingh}. This dataset is specifically designed to enhance the recognition and classification of named entities within the context of these dialects, which are underrepresented in existing NER datasets. By capturing the linguistic nuances specific to these regions, the dataset provides a valuable resource for developing more accurate and regionally aware NER models for Bangla.

Our Bangla Regional Named Entity Recognition (NER) dataset comprises a total of 17,405 sentences, with 3,481 sentences collected for each of the three major regional dialects: Sylhet, Chittagong, Barishal, Noakhali, and Mymensingh. Named entities in this dataset are systematically categorized into ten distinct types: Person, Location, Organization, Food, Animal, Colour, Role, Relationship, Object, and Miscellaneous. 86.18\% of the dataset (sentences 1 to 3000) of each region was sourced from two publicly available datasets (Vashantor\cite{faria2023vashantor} and ONUBAD\cite{sultana2025onubad}) and the rest 13.12\% through web scraping from online newspapers and articles, covering structured and editorially curated texts. This incorporates informal, conversational, and spoken language characteristics unique to these regional dialects. This diverse data collection approach helps in capturing real-world linguistic variations and dialectal expressions, making the dataset more representative of actual language usage across different mediums. The number of instances of each type of named entities in the dataset for each three regions is given in Table~\ref{tab:sample}.



\begin{figure}[htpb]
    \centering
    \includegraphics[width=1.0\textwidth]{PRE_PRINT_flowchart.drawio.pdf}
    \caption{Development of ANCHOLIK-NER}
    \label{fig:flow}
\end{figure}

\section*{Dataset Development}
As mentioned earlier, the dataset was constructed using a diverse range of sources, ensuring comprehensive coverage of Bangla regional dialects. The collection and processing steps are outlined below along with a flow chart in Figure~\ref{fig:flow}:

\begin{enumerate}
    \item All text data were initially stored in excel format.
    \item A Python script was used to process the raw sentences, converting them into a structured tabular format. The script assigned a unique serial number to each sentence and extracted individual words into separate columns. This was done to convert the sentences into tokenized words for efficient annotation. The processed output was saved in CSV format.
    \item The annotation process followed the BIO scheme and was conducted by professional annotators specializing in Bangla regional dialects. Each word was manually labeled with its respective named entity category. To ensure high-quality annotation, three annotators independently labeled the data, and discrepancies were resolved through a consensus-based approach.
\end{enumerate}






The dataset is structured into separate sub-datasets for each region, enabling analysis of dialectal variations. The entire corpus is saved as CSV file, which is structured quite conveniently. A sample of the data structure is given in Table~\ref{tab:word_tag}. Each entry in the dataset consists of:
\begin{enumerate}
    \item Sentence Number – The sentence number in the ANCHOLIK-NER dataset for each regions.
    \item Tokenized Words – Each word is treated as a separate token.
    \item Named Entity Annotations – Assigned entity tags following the BIO scheme.
\end{enumerate}

\begin{table}[h]
    \centering
    \caption{Dataset consists of 3 columns, while the first two columns are fulfilled by Python script (During the conversion of raw text to table format), the third column was filled and checked by Bangla Regional Language experts.}
    \renewcommand{\arraystretch}{1.1}
    \begin{tabular}{p{4cm} p{4cm} p{3cm}}
        \toprule
        \textbf{Sentence Number} & \textbf{Word (Sylhet Region)} & \textbf{BIO Tags} \\
        \midrule
        1 & {\bng phuyaTay} & O \\
        1 & {\bng iselT} & B-LOC \\
        1 & {\bng thaik} & O \\
        1 & {\bng Dhakat} & B-LOC \\
        1 & {\bng Aa{I}ech} & O \\
        \bottomrule
    \end{tabular}
    
    \label{tab:word_tag}
\end{table}

The BIO scheme is used to indicate the position of the token within the named entity:

\begin{itemize}
    \item B (Beginning): Beginning of a multi-word entity.
    \item I (Inside): Inside a multi-word entity.
    \item O (Outside): Outside any entity.
\end{itemize}

\begin{table}[htpb]
    \centering
    \caption{Total instances of Named Entity Types in five Regions}
    \renewcommand{\arraystretch}{1.1} % Adjust row height
    \begin{tabular}{p{3cm} p{1.2cm} p{1cm} p{1.5cm} p{1.4cm} p{2cm} r}
        \toprule
        \textbf{Named Entity Type} & \textbf{Barishal} & \textbf{Sylhet} & \textbf{Chittagong} & \textbf{Noakhali} & \textbf{Mymensingh} & \textbf{Total Instances} \\
        \midrule
        Person (PER) & 39 & 38 & 39 & 39 & 39 & 194 \\
        Location (LOC) & 369 & 371 & 377 & 361 & 362 & 1840 \\
        Organization (ORG) & 139 & 141 & 139 & 141 & 140 & 700 \\
        Food (FOOD) & 310 & 308 & 308 & 303 & 312 & 1541 \\
        Animal (ANI) & 57 & 56 & 57 & 57 & 57 & 284 \\
        Colour (COL) & 162 & 167 & 160 & 164 & 163 & 816 \\
        Role (ROLE) & 114 & 107 & 109 & 111 & 113 & 554 \\
        Relation (REL) & 681 & 677 & 676 & 676 & 676 & 3386 \\
        Object (OBJ) & 352 & 348 & 348 & 350 & 349 & 1747 \\
        Miscellaneous (O) & 17928 & 18750 & 18177 & 17957 & 17943 & 90,755 \\
        \bottomrule
    \end{tabular}
    
    \label{tab:sample}
\end{table}

The BIO scheme was chosen to ensure precise and detailed annotation of named entity boundaries in the ANCHOLIK-NER dataset. This is very useful for Bangla regional dialects where named entities can span multiple tokens due to the language’s morphological complexity. BIO scheme allows the model to differentiate between Start Entity, and intermediate tokens and hence improve training accuracy and prediction quality. The entity categories were selected based on their relevance and frequency in Bangla regional contexts. This ensures comprehensive coverage of commonly occurring named entities while maintaining a balanced dataset for effective model training.


The dataset includes the following named entity categories, each annotated using the BIO scheme:
\begin{itemize}
    \item Location (LOC): B-LOC, I-LOC
    \item Person (PER): B-PER, I-PER
    \item Organization (ORG): B-ORG, I-ORG
    \item FOOD (FOOD): B-FOOD, I-FOOD
    \item Animal (ANI): B-ANI, I-ANI
    \item Colour (COL): B-COL, I-COL
    \item ROLE (ROLE): B-ROLE, I-ROLE
    \item Relationship (REL): B-REL, I-REL
    \item Object (OBJ): B-OBJ, I-OBJ
    \item Miscellaneous: O
    \end{itemize}


\section{Limitations}
The dataset focuses on only five regional dialects: Sylhet, Chittagong, Barishal, Noakhali, and Mymensingh, potentially excluding linguistic variations present in other Bangla-speaking regions.

\section{Ethics Statement}
The data used to construct the ANCHOLIK-NER dataset does not raise any ethical concerns, as it was collected from two publicly available datasets and through web scraping from publicly accessible online newspapers, articles. The dataset contains no sensitive or private information, ensuring compliance with ethical standards. Additionally, no data was sourced from personal communications or restricted sources. This dataset is intended solely for NLP research and development, and no human or animal subjects were involved in its creation.

\section{Data Availability}

\href{https://data.mendeley.com/datasets/gbkszkt8z3/2}{Dataset of Named Entity Recognition for Regional Bangla Language (Original data)} (Mendeley Data) 

\section{CRediT Author Statement}
\textbf{Bidyarthi Paul:} Methodology, Data curation, Writing – original draft, Visualization, Validation; \textbf{Faika Fairuj Preotee:} Data curation, Investigation, Writing – original draft; \textbf{Shuvashis Sarker:} Methodology, Validation,  Software, Conceptualization; \textbf{Shamim Rahim Refat:} Resources, Data curation, Project administration; \textbf{Shifat Islam:} Supervision, Project administration; \textbf{Tashreef Muhammad:} Supervision, Writing – review \& editing; \textbf{Mohammad Ashraful Hoque:} Project administration; \textbf{Shahriar Manzoor:} Supervision. 

\section{Acknowledgments} 

This research did not receive any specific grant from funding agencies in the public, commercial, or not-for-profit sectors.

\section{Declaration of Competing Interest}

The authors declare that they have no known competing financial interests or personal relationships that could have appeared to influence the work reported in this paper.



\bibliographystyle{unsrt}  
\bibliography{references}  %%% Remove comment to use the external .bib file (using bibtex).


\end{document}
