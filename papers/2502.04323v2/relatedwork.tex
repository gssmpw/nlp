\section{Related Work}
Methods that utilize oblique splits have shown improved empirical performance across many tasks in machine learning, but general polyhedral cells incur increased computational and theoretical difficulties. Multiple works have proposed generating a partition by randomly rotating the feature space to mitigate this cost and then partitioning with axis-aligned splits. For example, random rotations have been used to increase the diversity of estimators in ensemble methods \citep{blaser2016random} and show improved performance over axis-aligned random forests. Randomly rotated $k{\text -}d$ trees were studied by \cite{vempala:LIPIcs:2012:3847} for nearest neighbor search and were shown to adapt to the intrinsic dimension of the input data. Our work studies this kind of approach when generating random features with a Mondrian process. This allows us to obtain a closed-form expression of the limiting kernel, and convergence guarantees due to the connection with random tessellation models in stochastic geometry.