%%%% Generic manuscript mode, required for submission
%%%% and peer review
% \documentclass[manuscript,screen,review,ctexart]{acmart}

\documentclass[manuscript]{acmart}

% \documentclass[manuscript,review,anonymous]{acmart}

\usepackage{listings}
\usepackage{xcolor}
\usepackage{float} % 让图片可以用 [H] 选项强制固定
\usepackage{placeins} % 让 \FloatBarrier 限制 LaTeX 不随意移动图片

% Define a custom style for the listings
\lstset{
    basicstyle=\ttfamily\small,
    breaklines=true,
    breakatwhitespace=true,
    keywordstyle=\color{blue},
    commentstyle=\color{green!60!black},
    stringstyle=\color{red},
    tabsize=4,
    showstringspaces=false,
    captionpos=b
}

%% Fonts used in the template cannot be substituted; margin 


%% adjustments are not allowed.
%%
%% \BibTeX command to typeset BibTeX logo in the docs
\AtBeginDocument{%
  \providecommand\BibTeX{{%
    \normalfont B\kern-0.5em{\scshape i\kern-0.25em b}\kern-0.8em\TeX}}}

%% Rights management information.  This information is sent to you
%% when you complete the rights form.  These commands have SAMPLE
%% values in them; it is your responsibility as an author to replace
%% the commands and values with those provided to you when you
%% complete the rights form.
\setcopyright{acmcopyright}
\copyrightyear{2025}
\acmYear{2025}
\acmDOI{10.1145/1122445.1122456}

%% These commands are for a PROCEEDINGS abstract or paper.
% \acmConference[C\&C '25]{ACM Creativity \& Cognition Conference 2025}{June 23--25, 2025}{Virtual}
% \acmBooktitle{ACM Creativity \& Cognition Conference,
  % June 23--25, 2025, Virtual}
\acmPrice{15.00}
\acmISBN{978-1-4503-XXXX-X/18/06}

\newcommand\subjectcount{246}


\newcommand{\todo}[1]{{\color{blue} #1}}

% \usepackage{xeCJK}
\usepackage{multirow}
\usepackage{subcaption}
\usepackage{caption}
\usepackage{longtable}
\usepackage{geometry}
% \usepackage{ulem}

\begin{document}

%%
%% The "title" command has an optional parameter,
%% allowing the author to define a "short title" to be used in page headers.
\title[Empowering Self-Representation of Junior Researchers and Designers Through Agentic Workflows]{Frontend Diffusion: Empowering Self-Representation of Junior Researchers and Designers Through Agentic Workflows}

\author{Zijian Ding}
\affiliation{%
  \institution{University of Maryland, College Park}
  \country{USA}}

\author{Qinshi Zhang}
\affiliation{%
  \institution{University of California, San Diego}
  \country{USA}}

\author{Mohan Chi}
\affiliation{%
  \institution{Purdue University}
  \country{USA}}

\author{Ziyi Wang}
\affiliation{%
  \institution{University of Maryland, College Park}
  \country{USA}}

\renewcommand{\shortauthors}{Ding, et al.}

%%
%% The abstract is a short summary of the work to be presented in the
%% article.
\begin{abstract}
With the continuous development of generative AI's logical reasoning abilities, AI's growing code-generation potential poses challenges for both technical and creative professionals. But how can these advances be directed toward empowering junior researchers and designers who often require additional help to build and express their professional and personal identities? We present Frontend Diffusion, a multi-stage agentic system, transforms user-drawn layouts and textual prompts into refined website code, thereby supporting self-representation goals. A user study with 13 junior researchers and designers shows AI as a human capability enhancer rather than a replacement, and highlights the importance of bidirectional human-AI alignment. We then discuss future work such as leveraging AI for career development and fostering bidirectional human-AI alignment on the intent level.
\end{abstract}

%%
%% The code below is generated by the tool at http://dl.acm.org/ccs.cfm.
%% Please copy and paste the code instead of the example below.
%%
% \begin{CCSXML}
% <ccs2012>
 
%  </ccs2012>
% \end{CCSXML}

% \ccsdesc[500]{Human-centered computing~Empirical studies in HCI}
% \ccsdesc[500]{Human-centered computing~Auditory feedback}
% \ccsdesc[300]{Human-centered computing~Empirical studies in interaction design}
% \ccsdesc[500]{Human-centered computing~Sound-based input / output}

%%
%% Keywords. The author(s) should pick words that accurately describe
%% the work being presented. Separate the keywords with commas.
\keywords{Creative Workflow, Research, Code Generation}


\begin{teaserfigure}
\centering
\includegraphics[width=0.95\textwidth]{figures/Teaser.png}
  \caption{Website generation workflow: (a) user inputs prompt; (b) user draws layout of the website in sketch; (c) the system generates the first website; (d) the system completes generations four website iterations.}
  \Description{Website generation workflow.}
  \label{fig:teaserfigure}
\end{teaserfigure}

\maketitle

\def \RQO {\textbf{RQ1}: ?}

\def \RQT {\textbf{RQ2}: ?}

\section{Introduction}

% State of the world (robots for creative activites)
The term ``robot,'' originally signifying `forced labor,' has long been associated with labor and work. Robots have demonstrated their utility in various automated productive and social contexts, where the primary goals are improving productivity, safety, and fostering social interactions with humans~\cite{simoes2022designing, weidemann2021role, honig2018understanding}. However, an increasing number of cases feature using of robots in creative settings. Unlike productive contexts, where the focus is on efficiency and task completion~\cite{arents2022smart}, or social contexts, where communication and trust are prioritized~\cite{nam2020trust, saunderson2019robots}, creative environments prioritize artistic innovation and expression~\cite{hsueh2024counts}. This shift fundamentally alters the dynamics of human-robot interaction, redefining the roles and expectations for both humans and robots.

For instance, robots’ social behaviors are leveraged to support the generation and expression of creative ideas~\cite{hu2021exploring, sandoval2022human, alves2020creativity}, and programmable robotic movements and trajectories are employed to inspire artistic activities such as sketching~\cite{lin2020your}. These studies often engage participants from creative fields who possess limited prior experience with robotics, and are typically conducted in short-term, experimental settings. Consequently, the findings from these studies remain constrained since much can be learned from professional practitioners' experiences to inform system design such as digital fabrication~\cite{hirsch2023nothing}. There is a notable gap in research examining the long-term, active, and practical experience of integrating robotic systems into the creative processes. As a result, the deeper insights into how robots facilitate and shape creative processes, beyond simply augmenting human creativity, remain underexplored. In this study, we aim to better understand the impacts of robots on creative processes and outcomes.

As early as Leonardo da Vinci's 16th century ``Automaton,'' artists have explored the creative affordances of robotic systems~\cite{shanken2002cybernetics, pagliarini2009development, jeon2017robotic}. The artistic creation process typically encompasses various stages, including the exploration of materials and techniques, ongoing experimentation and iteration, and the continual refinement of the artists' insights into their creative subjects~\cite{lewis2023art, sturdee2022state}. Therefore, investigating the artistic process involving robots offers an opportunity to gain deeper insights into robots' creative potential. Robotic art, in particular, provides a compelling case for this exploration.

We define robotic art as artworks that utilize robotic or automated machines to create artistic experiences and tangible artifacts. One example is robotic installation art, in which robots are programmed to follow specific rules that embody the artist’s expression (\autoref{fig:teaser} (a)). Another example is responsive art, in which robots react to their environment, with behaviors that change over time or in response to spectators (\autoref{fig:teaser} (b)). Additionally, there are robotic creators, which possess a degree of agency, allowing them to collaborate with human artists and produce works that extend beyond mere replication of human-created art (\autoref{fig:teaser} (c) and (d)). As such, robotic art becomes a rich case for exploring human-machine interactions in creative contexts. Gaining a deeper understanding of how robots facilitate artistic expression can provide insights for designing computing systems to support creative activities~\cite{gomez2021robot}.

% Therefore, we did...
We draw on semi-structured, in-depth interviews with renowned professional robotic artists to investigate the use of robots in artistic practice. Specifically, our goal is to understand how artistic exploration of robotic systems challenges conventional assumptions about the functions of robots, such as their roles in automating repetitive tasks or serving human needs. We also explore the implications of robots in the artistic process and examine how creativity may emerge within robotic art. To address these interrelated inquiries, our study focuses on the practice of robotic art, posing the research question: \textit{How do robotic artists utilize robots in their artistic practice?} We approach this inquiry through the perspectives and experiences of robotic artists, who creatively design, modify, and repurpose robotic systems for artistic expression and exploration.

% The key findings are...
Our findings highlight the social, material, and temporal dimensions of artists' practices that shape their creativity and artistic outcomes. The creation of robotic art is largely a social process, as artists receive both explicit and implicit feedback through the audience's reactions and reception of their work. Simultaneously, the embodiment and malfunctions inherent to robotic systems drive artistic experimentation. The temporal processes of creation and exhibition, beyond just the final product, further enhance the creative value. Our empirical analysis presents how creativity emerges through the interplay of social, material, and temporal interactions among artists, robots, audiences, and the environment.

% The contributions of this work are...
We make two main contributions to HCI in this study. 
First, we elucidate the interactive mechanisms among key actors---human creators, machines, audiences, and environments---within the practice of robotic art, a topic that remains underexplored in HCI. Our findings reveal the significance of sociality (e.g., interactions between artists and audiences), materiality (e.g., the embodiment and malfunctions of robots), and temporality (e.g., the processes of creation and exhibition) in shaping creative values. We propose that these three facets are central to the creative process and facilitate the emergence of creativity in robotic art.
Second, drawing from the findings, we offer implications for \textit{socially informed}, \textit{material-attentive}, and \textit{process-oriented} creation with computing systems. We suggest leveraging these three aspects to enhance creativity and the creative experience. Specifically, we discuss the value of incorporating implicit audience feedback, designing with technical malfunctions, and focusing on the post-creation process to foster alternative creative experiences with machines~\cite{alter2010designing, juarez2022glitch}.




\section{Related Work}

\subsection{Advancements in AI and Agentic Workflows for Code Generation}

Since the training process of GPT-3.5 incorporated a substantial amount of code data to enhance the logical reasoning capabilities of language models \cite{chenEvaluatingLargeLanguage2021}, code generation has become closely intertwined with language modeling. With the emergence of models that place a stronger emphasis on reasoning, these capabilities continue to evolve. According to the SWE-bench benchmark, which simulates human programmers' problem-solving workflows, AI programming performance increased from below 2\% in December 2023 \cite{jimenez2024swebench} to over 60\% by February 2025\footnote{https://www.swebench.com/}.

However, simply reinforcing the reasoning ability of language models primarily advances lower-level software development tasks such as auto-completion and refactoring. To enhance automation in real-world software and system development, researchers have introduced various agentic workflows, including OpenHands \cite{openhands}, an open-source coding agent designed for end-to-end development, and Agent Company, which simulates the operation of a software company \cite{xu2024theagentcompany}. Nonetheless, as of February 2025, even the most sophisticated agentic workflows remain unable to fully realize end-to-end programming\footnote{https://www.swebench.com/}, let alone organization-level agency\footnote{https://the-agent-company.com/}. 

Within code generation and system development, front-end code generation—such as website development—often demonstrates stronger performance than back-end development. Research in this domain has examined reconstructing HTML/CSS structures from UI screenshots using computer vision techniques \cite{soseliaLearningUItoCodeReverse2023}, implementing hierarchical decomposition strategies for interface elements to optimize UI code generation \cite{wanAutomaticallyGeneratingUI2024}, and improving model specialization through domain-specific fine-tuning for UI generation \cite{wuUICoderFinetuningLarge2024}. To systematically evaluate front-end code generation, specialized benchmarks have been developed to assess the quality of HTML, CSS, and JavaScript implementations \cite{siDesign2CodeHowFar2024}. To investigate the societal impact of this notable improvement in AI programming capabilities, we focus on the task of website generation, where current AI systems are relatively close to achieving near end-to-end automation.

\subsection{Beyond Templates: AI-Powered, User-Centric UI}

With the continuing development of AI-driven user interface (UI) generation, users increasingly seek more personalized and diverse expressions rather than relying solely on conventional template reuse. Recent advances have led to adaptive UI generation systems like FrameKit, which allows end users to manually design keyframes and generate multiple interface variants \cite{wu_framekit_2024}. PromptInfuser goes a step further by enabling runtime dynamic input and generation of UI content \cite{petridisPromptInfuserHowTightly2024}. In this context, AI tools have been shown to offer inspiration for professional designers \cite{luBridgingGapUX2022}. For instance, DesignAID \cite{cai_designaid_2023} demonstrates that generative AI can provide conceptual directions and stimulate creativity at early design stages. Misty supports remixing concepts by allowing users to blend example images with the current UI, thereby enabling flexible conceptual exploration \cite{luMistyUIPrototyping2024}.

Beyond offering inspiration, AI can also provide real-time design feedback to guide iterative refinement and error correction \cite{duan_towards_2023}, such as handling CSS styling in simple websites and optimizing specific UI components \cite{liUsingLLMsCustomize2023}. It is capable of evaluating UI quality and relevance, offering suggestions at various design stages \cite{wuUIClipDatadrivenModel2024}, and even detecting potential development or UI issues in advance \cite{petridisPromptInfuserHowTightly2024}. Automated heuristic evaluations generated by AI can provide more precise assessments and recommendations, thereby streamlining the iterative process \cite{duanGeneratingAutomaticFeedback2024}. When combined with traditional heuristic rules, AI has been shown to increase the effectiveness of UI error detection and correction \cite{lu_ai_2024}. Integrating prototype-checking techniques into the UI generation workflow can further enhance automatic repair capabilities \cite{xiaoPrototype2CodeEndtoendFrontend2024}.

\subsection{Improving the Creative Workflow with AI}

In many creativity workflows, a prolonged progression from ideation, prototyping, and development to iteration is required \cite{palaniEvolvingRolesWorkflows2024}. Those creative processes are frequently constrained by multiple intricate steps that limit users' expressive capabilities. For example, the complexity and associated costs of developing a personal website often deter individuals from undertaking this process, prompting many to resort to standardized website templates for personal websites. However, GenAI can assist with the creativity workflow from various angles \cite{wanItFeltHaving2024,palaniEvolvingRolesWorkflows2024,longNotJustNovelty2024}. First, GenAI such as text-to-image generation can reduce the time needed to produce high-fidelity outcomes. This enables creators to focus on refining the gap between the high-fidelity results and their envisioned expectations, rather than expending effort on how to achieve high fidelity in the first place \cite{edwardsSketch2PrototypeRapidConceptual2024}. Besides, AI lowers the cost of experimenting with new ideas, thereby minimizing the psychological barriers to conducting trial and error with unconventional concepts \cite{palaniEvolvingRolesWorkflows2024}. When users are uncertain about what they want or have only a broad concept lacking specific details, AI can offer inspiration \cite{rickSupermindIdeatorExploring2023}. Moreover, AI can facilitate parallel prototyping by presenting multiple design directions simultaneously, allowing creators to compare and refine a range of diverse design solutions \cite{dowParallelPrototypingLeads2010}.

\section{System Design}

\begin{figure}
  \centering
  \includegraphics[width=1\textwidth]{figures/Archi.png}
    \caption{Agentic workflow of frontend code generation system: (1) Sketch-to-PRD generation, (2) PRD-to-code generation and code refinement. The prompts for PRD, code and idea generation are attached in Appendix \ref{appendix:b}.}
    \Description{Agentic workflow of frontend code generation: (1) Sketch-to-PRD generation, (2) PRD-to-code generation and code refinement.}
    \label{fig:structure}
  \end{figure}

We developed Frontend Diffusion, an end-to-end AI-powered frontend code generation tool as illustrated by Figure \ref{fig:structure}. Constructing a personal website based on both sketches and textual prompts is inherently complex, given the need to account for not only the content to be displayed (e.g., the type of information) but also the format (e.g., layout, color schemes, and style). Although language models demonstrate capabilities in translating user intentions into code, the initial user inputs—such as a brief sketch and prompt—often leave information gaps relative to the final website. Attempting to fill all details in a single step frequently results in coarse or incomplete outcomes. To address this limitation, we employ a multi-stage agentic workflow, wherein the website creation process is divided into multiple tasks, mainly in two phrases: (1) Sketch-to-PRD generation, (2) PRD-to-code generation and code refinement as shown in Figure \ref{fig:structure}.

\subsection{Phase 1: Sketch-to-PRD Generation}

This step transforms the user's visual and textual inputs into a structured document, referred to as the Product Requirements Document (PRD), which serves as a blueprint for the website's development process. Upon completion of the user's sketch and theme input, the user can activate the code generation process via ``Generate" button. The system then converts the sketch into SVG format, followed by a subsequent transformation into JPG format. This two-step conversion process was implemented based on empirical evidence from our tests, showing that language models exhibit better performance when processing images in JPG format compared to images in SVG format.

Then we used the JPG formats to generate PRD. To enhance the visual appearance of the generated websites, the system integrates the Pexels API\footnote{https://www.pexels.com/api/} for image retrieval. The language model is specifically prompted to include image terms and size descriptions (e.g., [school(large)]). These descriptors are subsequently utilized to query the Pexels API, which returns relevant image URLs for incorporation into the PRD.

\subsection{Phase 2: PRD-to-code Generation and Code Refinement}

The phase 2 of the system consists of two steps: (1) PRD-to-code generation: the system utilizes the generated PRD and the original user prompt as inputs for code generation; (2) code refinement: the system implements an iterative refinement process to automatically enhance the generated website with richer functionality and reduced flaws. This process involves analyzing the initial code to generate optimization suggestions, merging these suggestions with the original theme, and utilizing the enhanced theme along with the previously generated PRD to regenerate the code. The system executes this iterative refinement process multiple times (by default, n=4). Users can navigate between iterations by selecting preview thumbnails displayed at the interface's bottom, and can access or copy the generated code for each version.

All the generation steps utilize the Claude 3.5 Sonnet language model (claude-3-5-sonnet-20241022)\footnote{https://www.anthropic.com/news/claude-3-5-sonnet}, one of the most advanced language models for coding as of December 2024\footnote{https://www.swebench.com/}. We also anticipate rapid developments in LLMs, so the task transition techniques described herein are designed to be model-agnostic, ensuring their applicability to future, more advanced LLMs. The project code will be open-sourced after the paper's anonymity period ends.

\section{Study Design}

The experiment lasted approximately 45 minutes and consisted of three parts: a 5-minute participant onboarding, a 20-30 minute user study, and a 15-20 minute post-study interview. First, participants read and signed the consent form, granting permission for screen recording. Next, the researcher demonstrated how to use it by creating one website page. Participants then created between 1 to 5 personal website pages depending on participants' needs. These pages could showcase either professional content, such as research or design portfolios, or personal interests, such as an image gallery or reading list. After completing the user study, the researcher conducted a 15-20 minute post-study interview, following the protocol outlined in Section \ref{sec:post-study}.

\subsection{Demographic Information}

In the pre-study survey, we collected demographic information, including participants' age, gender, educational level, experience in web development, design, and research, as well as whether they had a personal website (Y/N). Table \ref{tab:demographic} summarizes this information. Among the 13 participants, there were 4 master's students, 5 first-year Ph.D. students, and 4 third- or fourth-year Ph.D. students. Regarding website development experience, 6 participants reported having a personal website. Overall, the group represented a relatively diverse range of website creation backgrounds.

\begin{table}[h!]
\centering
\begin{tabular}{cccccccccc}
\toprule
& \multicolumn{3}{c}{Demographic} & \multicolumn{3}{c}{Experience (in yrs)} & \multicolumn{2}{c}{Website} \\
PID & Age & Gender & Edu & Dev & Design & Research & Prev. Website & Pages created \\
\midrule
1 & 25 & F & Yr-1 PhD & 0 & 0 & 3 & N & 5 \\
2 & 25 & F & Yr-1 PhD & 0.1 & 0.2 & 4 & Y & 3 \\
3 & 23 & F & Yr-1 master & 1 & 3 & 0.5 & Y & 3 \\
4 & 23 & F & Yr-2 master & 0.5 & 3 & 3 & Y & 2 \\
5 & 28 & F & Yr-3 PhD & 1.5 & 4 & 5.5 & Y & 4 \\
6 & 31 & M & Yr-3 PhD & 0.5 & 1 & 10 & N & 5 \\
7 & 29 & M & Yr-1 PhD & 0 & 0 & 4 & N & 3 \\
8 & 23 & F & Yr-1 master & 1 & 0 & 1 & N & 5 \\
9 & 31 & F & Yr-4 PhD & 0 & 1 & 8 & N & 4 \\
10 & 24 & F & Yr-1 PhD & 0.5 & 0 & 3 & N & 1 \\
11 & 23 & F & Yr-1 master & 3 & 1 & 0 & Y & 5 \\
12 & 29 & M & Yr-3 PhD & 1 & 0 & 4 & N & 4 \\
13 & 27 & M & Yr-1 PhD & 2 & 3 & 3.5 & Y & 3 \\
\bottomrule
\end{tabular}
\caption{Participants demographic information, including age, gender, educational level, experience in web development (dev), design, research, having personal website (Y/N), and number of pages (each page has 4 iterations) created in the study.}
\label{tab:demographic}
\end{table}



\subsection{Post-Study Interview Protocol}
\label{sec:post-study}

Following the completion of the website creation task, we conducted semi-structured interviews with each participant to gain insights into their experiences, perceptions, and overall satisfaction with the tool. The interviews took place immediately after the hands-on sessions, ensuring that participants' impressions and feedback were fresh and accurately captured. Each interview lasted approximately 20--30 minutes and was audio-recorded, with participants' consent, for subsequent transcription and thematic analysis.

The interview protocol was designed to explore three main dimensions: (1) User Experience, (2) Improvement and Recommendation, and (3) Broader Impacts and Perceptions. While the interviewer maintained flexibility to probe further into interesting responses, all participants were asked the same core questions to ensure consistency. The interview outline was as follows:

\begin{itemize}
    \item \textbf{User Experience}
    \begin{itemize}
        \item \textbf{Process:} Did you find the process of using the tool to generate a personal website smooth? Were there any steps that were particularly time-consuming or challenging?
        \item \textbf{Result:} Did the final website meet your expectations? Do you think the tool accurately understood your intent between sketch input and the generated website? Why or why not?
    \end{itemize}

    \item \textbf{Improvement and Recommendation}
    \begin{itemize}
        \item Are there any features you would like the tool to add or improve? Did you feel the need for more interaction or guidance from the tool?
        \item Would you recommend this tool to your peers? Why or why not?
    \end{itemize}

    \item \textbf{Broader Impacts and Perceptions}
    \begin{itemize}
        \item Does this tool increase or decrease your interest in frontend development, or even research?
        \item Do you think this tool enhances human abilities or replaces them? Why?
    \end{itemize}
\end{itemize}

\vspace{-0.2cm}
\section{Results}\label{sec:results}




\subsection{Benchmark quality after watermarking}\label{subsec:results_rephrasing}


\paragraph{\textbf{Set-up.}}
For the watermark embedding, we rephrase with Llama-3.1-8B-Instruct~\citep{dubey2024llama} by default, with top-p sampling with p = $0.7$ and temperature = $0.5$ (default values on the Hugging Face hub), and the green/red watermarking scheme of \citet{kirchenbauer2023reliability} with a watermark window $k=2$ and a ``green list'' of
size $\frac{1}{2}|V|$ ($|V|$ is the vocabulary size).
We compare different values of $\delta$ when rephrasing: 0 (no watermarking), 1, 2, and 4.
We choose to watermark ARC-Challenge, ARC-Easy, and MMLU due to their widespread use in model evaluation.
In practice, one would need to watermark their own benchmark before release.
For MMLU, we select a subset of 5000 questions, randomly chosen across all disciplines, to accelerate experimentation and maintain a comparable size to the other benchmarks.
We refer to this subset as MMLU$^*$.
ARC-Easy contains 1172 questions, and ARC-Challenge contains 2372 questions.
In~\autoref{fig:example_answers_big} of \autoref{app:appendix}, we show the exact instructions given to the rephrasing model (identical for all benchmarks) and the results for different watermarking strengths on one example from ARC-Easy.
\emph{We use a different watermarking key $\sk$ for each benchmark.}

% Thanks to the hashing function used, the corresponding green lists and red lists for each benchmark are independent: there is no more collision between the benchmarks than there is between natural text and the benchmarks.

\paragraph{\textbf{Even strong watermarking keeps benchmark utility.}} 
We evaluate the performance of Llama-3.3-1B, Llama-3.3-3B and Llama-3.1-8B on the original benchmark and the rephrased version using as similar evaluation as the one from the \texttt{lm-evaluation-harness} library~\citep{eval-harness}.
To check if the benchmark is still as meaningful, we check that evaluated models obtain a similar accuracy on the watermarked benchmarks and on the original version (see~\autoref{subsec:rephrasing}).
\autoref{fig:results_overview_arc_easy_perfs} shows the performance on ARC-Easy.
All models perform very similarly on all the rephrased versions of the benchmark, even when pushing the watermark to $80\%$ of green tokens.
Importantly, they rank the same.
Similar results are shown for MMLU$^*$ and ARC-Challenge in \autoref{fig:results_overview_arc_easy_perfs} of \autoref{app:appendix}, although for MMLU$^*$, we observe some discrepancies. 
For instance, when using a watermarking window size of 2 (subfig i), the performance of Llama-3.2-1B increases from 38$\%$ to $42\%$ between the original and the other versions. 
However we observe the same issue when rephrasing without watermarking in that case.
As detailed in \autoref{subsec:rephrasing}, designing better instructions that are more specific to each benchmark could help.
We have tried increasing $\delta$ even further, but it broke the decoding process. 
The choice of $\delta$ depends on the benchmark and the model used for rephrasing, and needs to be empirically tested.



\begin{figure}[b!] % 't' places the figure at the top of the page
    \centering
    \begin{minipage}{0.49\textwidth}
        \centering
        \includegraphics[width=1.0\textwidth, clip, trim=0 0cm 0 0]{figs/main/k2/arc-easy_delta_barplot.pdf}
        \subcaption{Watermarking questions does not degrade utility.}
        \label{fig:results_overview_arc_easy_perfs}
    \end{minipage}\hfill
    \begin{minipage}{0.49\textwidth}
        \centering
        \includegraphics[width=1.0\textwidth, clip, trim=0 0cm 0 0]{figs/main/k2/contamination_35317.pdf}
        \subcaption{More contaminations \& stronger wm $\uparrow$ detection.}
        \label{fig:results_overview_arc_easy_detection}
    \end{minipage}
    \caption{
    Result for benchmark watermarking on ARC-Easy. %Watermarking the questions does not degrade its utility, and the more watermarked the benchmark, the easier it is to detect radioactivity. 
    (Left) We rephrase the questions from ARC-Easy using Llama-3.1-8B-Instruct while adding watermarks of varying strength. 
    The performance of multiple Llama-3 models on rephrased ARC-Easy is comparable to the original, preserving the benchmark's usefulness for ranking models and assessing accuracy (Sec.~\ref{subsec:rephrasing}, Sec.~\ref{subsec:results_rephrasing}). (Right) We train 1B models from scratch on 10B tokens while intentionally contaminating its training set with the watermarked benchmark dataset. 
    Increasing the number of contaminations and watermark strength enhances detection confidence (Sec.~\ref{subsec:detection}, Sec.~\ref{subsec:result_detection})}
    \vspace{-0.3cm}\label{fig:results_overview_arc_easy}
\end{figure}

\subsection{Contamination detection through radioactivity}\label{subsec:result_detection}

We now propose an experimental design to control benchmark contamination, and evaluate both the impact on model performance and on contamination detection.

\paragraph{\textbf{Training set-up.}}
We train 1B transformer models~\citep{vaswani2017attention} using \texttt{Meta Lingua}~\citep{meta_lingua} on 10B tokens from DCLM~\citep{li2024datacomp}. 
The model architecture includes a hidden dimension of 2048, 25 layers, and 16 attention heads.
The training process consists of 10,000 steps, using a batch size of 4 and a sequence length of 4096. 
Each training is distributed across 64 A-100 GPUs, and takes approximately three hours to finish.
The optimization is performed with a learning rate of $3 \times 10^{-3}$, a weight decay of $0.033$, and a warmup period of 5,000 steps. 
The learning rate is decayed to a minimum ratio of $10^{-6}$, and gradient clipping is applied with a threshold of 1.0.

\paragraph{\textbf{Contamination set-up.}}
Between steps 2500 and 7500, every $5000/\#\text{contaminations}$, we take a batch from the shuffled concatenation of the three benchmarks instead of the batch from DCLM.
Each batch has
\(
\text{batch size} \times \text{sequence length} \times \text{number of GPUs} = 4 \times 4096 \times 64 \approx 1\,\text{M tokens}
\)
As shown in \autoref{tab:contamination}, the concatenation of the three benchmarks is approximately $500$k tokens, so each contamination is a gradient that encompasses all the benchmark's tokens.
For each benchmark, any sample that ends up contaminating the model is formatted as follows:

\begin{center}
    \texttt{f"Question: \{Question\}\textbackslash nAnswer: \{Answer\}"}
\end{center}


% \paragraph{Impact of the number of contaminations on the accuracy on the benchmark.} 
\paragraph{\textbf{Evaluation.}}
We evaluate the accuracy of the models on the benchmarks by comparing the loss between the different choices and choosing the one with the smallest loss,  either ``in distribution'' by using the above template seen during contamination or ``out of distribution'' (OOD) by using:

\begin{center}
    \texttt{f"During a lecture, the professor posed a question: \{Question\}. \\ After discussion, it was revealed that the answer is: \{Answer\}"}
\end{center}

In the first scenario, we evaluate overfitting, as the model is explicitly trained to minimize the loss of the correct answer within the same context. 
In the second scenario, we assess the model's ability to confidently provide the answer in a slightly different context, which is more relevant for measuring contamination.
Indeed, it's important to note that evaluations often use templates around questions ---\eg in the \texttt{lm-evaluation-harness} library~\citep{eval-harness}--- which may not be part of the question/answer files that could have leaked into the pre-training data.
% Moreover, if contamination comes from a leak of a jsonl that contains
\autoref{tab:contamination} focuses on $\delta=4$ and shows the increase in performance across the three (watermarked) benchmarks as a function of the number of contaminations when evaluated OOD. 
Results for in-distribution evaluation are provided in \autoref{tab:contamination_indist} of \autoref{app:appendix} (without contamination, the model performs similarly across the two templates).


\paragraph{\textbf{Contamination detection.}}
For each benchmark, we employ the reading mode detailed in~\autoref{subsec:detection} to compute the radioactivity score $S$ and the corresponding $\pval$.
% We perform the reading mode on the same watermarked benchmark watermarked benchmark.
Results are illustrated in~\autoref{fig:results_overview_arc_easy_detection} for ARC-Easy, and in~\autoref{fig:appendix_watermark_contamination} of \autoref{app:appendix} for the other two benchmarks, across different numbers of contaminations and varying watermark strengths $\delta$.
We observe that the stronger the watermark strength and the greater the number of contaminations, the easier it is to detect contamination: a larger negative $\logpval$ value indicates smaller $\pval$s, implying a lower probability of obtaining this score if the model is not contaminated.
For instance, a $-\logpval$ of $6$ implies that we can confidently assert model contamination, with only a $10^{-6}$ probability of error.
% , which is the case when $5$ points are artificially added on MMLU$^*$ in~\autoref{tab:contamination}.
Additionally, we observe that without contamination, the test yields a $\logpval$ value close to $-0.3 = \log_{10}(0.5) $, as expected under $\mathcal{H}_0$.
Indeed, under $\mathcal{H}_0$, the $\pval$ should follow a uniform distribution between 0 and 1, which implies that [-1, 0] is a 90$\%$ confidence interval for $\logpval$, and that [-2, 0] is a 99$\%$ confidence interval.

\autoref{tab:contamination} links the contamination detection to the actual cheating (with OOD evaluation) on the benchmarks when $\delta=4$ is used.
We can see that for the three benchmarks, whenever the cheat is greater than $10\%$, detection is extremely confident.
When the cheat is smaller, with four contaminations ranging from $+3\%$ to $+5\%$, the $\pval$ is small enough on ARC-Easy and MMLU$^*$, but doubtful for ARC-Challenge (because smaller, see \autoref{subsec:additional_results}).
For instance, for MMLU$^*$, we can assert model contamination, with only a $10^{-6}$ probability of error when $5$ points are artificially added.




% \begin{table}[t!]
%     \centering
%     \vspace{-0.2cm}
%     \caption{
%         Detection and performance metrics across different levels of contamination for ARC-Easy, ARC-Challenge, and MMLU benchmarks, watermarked with $\delta=4$.
%         The performance increase is shown for OOD evaluation as detailed in~\autoref{subsec:result_detection}. 
%         Similar results for in distribution are shown in \autoref{tab:contamination_indist} of~\autoref{app:appendix}
%     }\label{tab:contamination}
%     \begin{tabular}{r r r r r r r}
%         \toprule
%         & \multicolumn{2}{c}{ARC-Easy (112k toks.)} & \multicolumn{2}{c}{ARC-Challenge (64k toks.)} & \multicolumn{2}{c}{MMLU$^*$ (325k toks.)} \\
%         \cmidrule(lr){2-3} \cmidrule(lr){4-5} \cmidrule(lr){6-7}
%         Cont & \multicolumn{1}{r}{log10 p-val} & \multicolumn{1}{r}{Perf (\% $\Delta$)} & \multicolumn{1}{r}{log10 p-val} & \multicolumn{1}{r}{Perf (\% $\Delta$)} & \multicolumn{1}{r}{log10 p-val} & \multicolumn{1}{r}{Perf (\% $\Delta$)} \\
%         \midrule
%         0  & -0.3 & 53.5 (+0) & -0.3 & 29.4 (+0) & -0.9 & 30.6 (+0) \\
%         4  & -3.0 & 57.9 (+4.3) & -1.2 & 32.4 (+3.1) & -5.7 & 35.7 (+5.1) \\
%         8  & -5.5 & 63.0 (+9.5) & -4.5 & 39.3 (+9.9) & \textless{-12} & 40.8 (+10.2) \\
%         16 & \textless{-12} & 71.7 (+18.2) & \textless{-12} & 54.3 (+24.9) & \textless{-12} & 54.0 (+23.5) \\
%         \bottomrule
%     \end{tabular}
%     \vspace{-0.3cm}
% \end{table}

% \newcommand{\graydelta}[1]{\textcolor{gray}{\footnotesize (#1)}}
\begin{table}[t!]
    \centering
    \vspace{-0.2cm}
    \caption{
        Detection and performance metrics across different levels of contamination for ARC-Easy, ARC-Challenge, and MMLU benchmarks, watermarked with $\delta=4$.
        The performance increase is shown for OOD evaluation as detailed in~\autoref{subsec:result_detection}. 
        The log$_{10}$ $\pval$ of the detection test is strongly correlated with the number of contaminations, as well as with the performance increase of the LLM on the benchmark.
        % Similar results for in distribution are shown in \autoref{tab:contamination_indist} of~\autoref{app:appendix} \pierre{not necessary in the fig.}
    }\label{tab:contamination}
    \resizebox{\textwidth}{!}{
    \begin{tabular}{r rr@{\hspace{0.5em}}l rr@{\hspace{0.5em}}l rr@{\hspace{0.5em}}l}
        \toprule
        & \multicolumn{3}{c}{ARC-Easy (112k toks.)} & \multicolumn{3}{c}{ARC-Challenge (64k toks.)} & \multicolumn{3}{c}{MMLU$^*$ (325k toks.)} \\
        \cmidrule(lr){2-4} \cmidrule(lr){5-7} \cmidrule(lr){8-10}
        Contaminations & $\logpval$ & Acc. & \graydelta{\% $\Delta$} & $\logpval$ & Acc. & \graydelta{\% $\Delta$} & $\logpval$ & Acc.& \graydelta{\% $\Delta$} \\
        \midrule
        0  & -0.3 & 53.5 & \graydelta{+0.0} & -0.3 & 29.4 & \graydelta{+0.0} & -0.9 & 30.6 & \graydelta{+0.0} \\
        4  & -3.0 & 57.9 & \graydelta{+4.3} & -1.2 & 32.4 & \graydelta{+3.1} & -5.7 & 35.7 & \graydelta{+5.1} \\
        8  & -5.5 & 63.0 & \graydelta{+9.5} & -4.5 & 39.3 & \graydelta{+9.9} & \textless{-12} & 40.8 & \graydelta{+10.2} \\
        16 & \textless{-12} & 71.7 & \graydelta{+18.2} & \textless{-12} & 54.3 & \graydelta{+24.9} & \textless{-12} & 54.0 & \graydelta{+23.5} \\
        \bottomrule
    \end{tabular}
    }
    \vspace{-0.3cm}
\end{table}

\vspace{-0.2cm}
\subsection{Additional Results}\label{subsec:additional_results}


\paragraph{\textbf{Impact of window size.}}
\begin{wraptable}{r}{0.4\textwidth}
    \centering
    \vspace{-0.4cm}
    \caption{\small Proportion of green tokens in the predictions (see~\autoref{eq:def_S_N}), number of tokens scored after dedup and log$_{10}$ $\pval$s for different watermark window sizes, with 16 contaminations and $\delta=4$ on ARC-Easy.}
    \small % Reduce font size for the table
    \begin{tabular}{r r r r}
        \toprule
        $k$ & \multicolumn{1}{c}{$\rho$} & \multicolumn{1}{r}{Tokens} & \multicolumn{1}{r}{$\logpval$} \\
        \midrule
        0 & 0.53 & 5k & -6.07 \\
        1 & 0.53 & 28k & -25.89 \\
        2 & 0.53 & 47k & -38.69 \\
        \bottomrule
    \end{tabular}
    \vspace{-0.2cm}
    \label{tab:window_size}
\end{wraptable}
Watermark insertion through rephrasing (\autoref{subsec:rephrasing}) depends on the watermark window size $k$. 
Each window creates a unique green-list/red-list split for the next token. 
Larger windows reduce repeated biases but are less robust.
Because of repetitions, \citet{sander2024watermarking} show that smaller windows can lead to bigger overfitting on token-level watermark biases, aiding radioactivity detection.
In our case, benchmark sizes are relatively small and deduplication limits the number of tokens tested, because each $\{$window + predicted token$\}$ is scored only once. 
Thus, smaller windows mean fewer tokens to score.
Moreoever, as shown in~\autoref{tab:window_size}, the proportion of predicted green tokens is not even larger for smaller windows: there is not enough repetitions for increased over-fitting on smaller windows.
The two factors combined result in lower confidence. 
A comparison of contamination detection across benchmarks and window sizes is shown in \autoref{fig:appendix_watermark_performance}, and for the utility of the benchmarks in~\autoref{fig:appendix_watermark_contamination}.
Overall, larger window size ($k=2$) yields better results.

\vspace{-0.1cm}
\paragraph{\textbf{Impact of benchmark size.}} The benchmark size can significantly affect the method's effectiveness.
With a fixed proportion of predicted green tokens, more evidence (\ie more scored tokens) increases test confidence. 
As shown in~\autoref{tab:contamination}, at a fixed level of cheating (\eg $+10\%$ on all benchmarks after $8$ contaminations), contamination detection confidence is proportional to benchmark size.
This is similar to our observations on watermark window sizes in~\autoref{tab:window_size}.
% So at fixed cheating level, it will be easier to detect contamination of bigger benchmarks.




% In classical watermarking, however, a larger watermark window means smaller robustness as changing one every $k$ tokens on average can break all the watermark.

% But in our case, we are going the do the radioactivity detection test on the dataset without any changes, but we may want more robustness if the suspect model tries to break the watermark before training on it.


% \paragraph{Impact of rephrasing model.}
% The difficulty of the questions, and their entropy, can have an important impact on the method.
% For instance, some math questions are hard to rephrase, and adding a watermark can further mess-up the meaning. 
% The method may thus require a stronger model for highly technical benchmarks (\eg Llama3-70B instead of Llama3-8B).
% Moreover, typically for math or code, the rephrasing inherently does not let a lot of entropy, as many invariants need to be respected.
% Possibilities would be to add watermarked verbose text around the math instead of rephrasing, and use as entropy-aware LLM watermarking~\citep{lee2023wrote}.
% We have tested rephrasing the benchmarks using Llama3-70B-Instruct instead of the 8B version. 
% We observe that we need to increase $\delta$ to $8$ in in order to obtain the same proportion of green tokens as with $\delta=2$ with the 8B model, while using the exact same decoding parameters.
% This can be because there is less entropy in the generation of the 70B or that the logits are for some reasons bigger, as the bias towards the greenlist is added before the softmax (see~\autoref{subsec:rephrasing}).
% However, we observe that some failure cases with the 8B (specifically for questions with important numbers) are correct with the 70B, but this is hard to quantify. 
% We give one example bellow in~\autoref{fig:example_answers_70B}.
\vspace{-0.1cm}
\paragraph{\textbf{Impact of rephrasing model.}}
The difficulty and entropy of questions can significantly affect the method's performance. 
Indeed, math questions for instance can be challenging to rephrase, even more with watermarks. 
Thus, better models may be needed for technical benchmarks.
We tested rephrasing with Llama3-70B-Instruct instead of the 8B version, and  observed that some 8B model failures, especially on math questions, are resolved with the 70B model, though quantifying this is challenging. 
An example is provided in~\autoref{fig:example_answers_70B}.
We note that increasing $\delta$ to 8 is necessary to match the green token proportion of $\delta=2$ with the 8B model, using the same decoding parameters.
This may result from lower entropy in generation or bigger logits, as the greenlist bias is applied before the softmax (see~\autoref{subsec:rephrasing}).
Moreover, in math or code, rephrasing can offer limited entropy, and even better models will not be enough.
An alternative would be to add watermarked verbose text \emph{around} the questions, or using entropy-aware LLM watermarking~\citep{lee2023wrote}.

\begin{figure}[b!]
    \vspace{-0.3cm}
    \centering
    \begin{tcolorbox}[colframe=metablue, colback=white]
        \footnotesize
        \textbf{Original question:} 
        An object accelerates at 3 meters per second$^2$ when a 10-newton (N) force is applied to it. Which force would cause this object to accelerate at 6 meters per second$^2$?
        \begin{minipage}{0.42\textwidth}
            \vspace{0.1cm}
            \textbf{Llama-3-8B-Instruct, $\delta=2$:} What additional force, applied in conjunction with the existing 10-N force, would cause the object to experience an acceleration of 6 meters per second$^2$? (70$\%$)
        \end{minipage}\hspace{0.04\textwidth}%
        \begin{minipage}{0.54\textwidth}
            \vspace{0.1cm}
            \textbf{Llama-3-70B-Instruct, $\delta=8$:} What force would be necessary to apply to the object in order to increase its acceleration to 6 meters per second$^2$, given that an acceleration of 3 meters per second$^2$is achieved with a 10-newton force? (65$\%$)
        \end{minipage}
    \end{tcolorbox}
    \vspace{-0.2cm}
    \caption{
    Watermarking failure on an ARC-Challenge question with an $8$B model, while the $70$B model succeeds.
    }
    
\label{fig:example_answers_70B}
\end{figure}



\begin{wrapfigure}{r}{0.5\textwidth}
  \centering
  \vspace{-0.5cm}
\includegraphics[width=0.48\textwidth]{figs/main/detection_vs_performance.pdf} % Replace with your image file
  \vspace{-0.25cm}
  \caption{Detection confidence as a function of performance increase on MMLU$^*$ for different model sizes and \#contaminations, for $\delta=4$ and OOD evaluation.}
  \vspace{-0.35cm}
\end{wrapfigure}\label{fig:model_size}
\paragraph{\textbf{Impact of model size.}}
We also test radioactivity detection on 135M and 360M transformer models using the architectures of~\href{https://github.com/huggingface/smollm}{\texttt{SmolLM}} and the same training pipeline as described in \autoref{subsec:result_detection}, training each model on 10B tokens as well. 
\autoref{fig:model_size} shows the detection confidence as a function of the cheat on MMLU$^*$.
We find that, for a fixed number of contaminations, smaller models show less performance increase -- expected as they memorize less -- and we obtain lower confidence in the contamination detection test. 
As detailed in~\autoref{subsec:rephrasing}, the $\pval$s indicate how well a model overfits the questions, hence the expected correlation. For a fixed performance gain on benchmarks, $\pval$s are consistent across models. For example, after $4$, $8$, and $16$ contaminations on the $1$B, $360$M, and $135$M parameter models respectively, all models show around $+6$\% gain, with detection tests yielding $\pval$s around $10^{-5}$.
Thus, while larger models require fewer contaminated batches to achieve the same gain on the benchmark, radioactivity effectively measures ``cheating''.





% \begin{wrapfigure}{r}{0.45\textwidth}
%   \centering
%   \vspace{-0.4cm}
%   \includegraphics[width=0.43\textwidth]{figs/main/arc-easy.pdf}
%   \vspace{-0.3cm}
%   \captionsetup{font=small}
%   \caption{Performance of Llama-3 models on different versions of the arc-easy benchmark.}
%   \vspace{-1cm}
%   \label{fig:impact-wm-arc-easy}
% \end{wrapfigure}



\section{Discussion}

\subsection{Summary and Implications of Findings}

In this paper, we introduce a novel system, Frontend Diffusion, and present findings from a user study involving 13 junior researchers and designers, supported by in-depth interview results. The task of AI-assisted frontend development spans multiple stages of creative work, including ideation, prototyping, iteration, and development. Consequently, the insights gained from this study are broadly applicable to a wide range of open-ended and creative workflows.

When supporting the self-representation of junior researchers and designers, AI has the potential to enhance human capabilities by alleviating the cognitive and technical burden of repetitive tasks. This enables users to dedicate more time to higher-level ideation and reflection. Furthermore, we observed a bidirectional human-AI alignment \cite{shenBidirectionalHumanAIAlignment2024}: AI-initiated alignment (from AI to human) includes onboarding support for new users and prompt guidance and refinement, while human-initiated alignment (from human to AI) involves fine-grained user control over details and the ability to harness and combine AI's unexpected creative sparks. Additionally, we identified key possibilities and challenges in communicating dynamic and interactive behaviors, as well as the need for multimodal inputs to improve AI's ability to understand and anticipate user intent.

\subsection{AI as A Human Capability Enhancer: AI Career Advising and Planning}
\label{ref:career}

With technological advancements—especially in AI and robotics that exhibit certain human capabilities—the pace of societal change is accelerating. Consequently, individuals' career trajectories may experience more frequent transitions. In this inevitable trend, AI can also function as an enhancer assisting individuals in navigating these transitions—from the creation of personal websites to career planning and advising. As noted by P6 in the interview, there is an interest in employing AI to role-play as a recruiter, thereby providing guidance on personal websites. Moreover, AI could enable advise seekers to ``try on" career scenarios of interest by having the AI role-play as clients or colleagues within those scenarios, thereby enabling low-cost experimentation during career transitions. While AI provides career advising and planning, potential challenges must also be considered—for example, how to preserve human agency, such as personal interests, passion, and self-efficacy, rather than yielding an AI-optimized career path. This approach helps avoid an overreliance on AI-driven career optimization that could lead to homogenized trajectories and ultimately aims to build a user-initiated career ecosystem—one in which AI offers data-driven insights while humans retain the final decision-making power over non-linear, interest- and passion-driven career narratives.

\subsection{Bidirectional Human-AI Alignment: Alignment on Intent Level}

Currently, when executing complex tasks with AI, it is necessary to employ an agentic workflow that decomposes high-level human intents into executable steps. For example, in this study, the agentic workflow decomposes the intent to create a personal website into specific tasks such as generating a product requirements document (PRD) based on a sketch and prompt, performing image searches, generating code, and so forth. As AI's logical reasoning capabilities improve \cite{guo2025deepseek,openaiOpenAIO3mini2025}, it is becoming feasible for AI to autonomously break down human intents into concrete steps through reasoning. For instance, in answering a user's query, AI can determine the various perspectives that need to be considered. It is foreseeable that, in the future, many human–AI interactions will occur at the intent level.

This introduces new opportunities and challenges for human–AI alignment. For instance, when reasoning about complex problems, it could be time-consuming for reasoning models to generate the final outcome; users prefer to examine and adjust intermediate steps such as chain-of-thought, prior to generating the final output (e.g., the ``sketch then generate" approach \cite{zhu-tianSketchThenGenerate2024}). In future research on human-AI interaction, greater emphasis should be placed on aligning intermediate steps as intent-AI interaction \cite{dingIntentbasedUserInterfaces2024}.

\subsection{Limitation}

There were several limitations in the current study that warrant further investigation. First, the user sample was relatively narrow, as the study exclusively involved 13 junior researchers and designers—primarily master's and doctoral students. This limited sample did not encompass a broader demographic, such as participants from non-academic backgrounds or professionals from interdisciplinary fields. Consequently, the generalizability of our findings to a more diverse user population remained uncertain. Second, this study leveraged publicly available online image repositories\footnote{https://www.pexels.com} when generating website content to avoid direct use of participants' personal images. This approach resulted in the selection of images that participants perceived as irrelevant or misaligned with their personal or professional identity, potentially impacting the overall user experience.

\subsection{Future Work}

Future research could aim to address these limitations while exploring new directions. Regarding privacy and security, subsequent work might investigate localized solutions for handling sensitive data. One potential avenue is local deployment of open-source language models \cite{guo2025deepseek}. These approaches could enable localizing the processing of personal information and other sensitive data.

In addition, future work might consider expanding the application of AI in career development as discussed in Section \ref{ref:career}. Potential directions include career planning and advising agents and virtual career environments. Moreover, it is important to examine the long-term impacts of these AI tools on self-expression and career development. Longitudinal studies could provide insights into how AI-assisted systems influence users' academic identities and digital brand building over time, thereby offering insights on the broader social and career implications of integrating AI into personal development strategies.

\section{Conclusion}

This paper introduced Frontend Diffusion, a multi-stage agentic system designed to support junior researchers and designers in self-representation through website generation. We conducted a user study with 13 junior researchers and designers who used the system to generate personal websites. The results indicated that rather than replacing human abilities, AI enhances them by streamlining technical tasks, thereby allowing users to focus more on ideation and identity-building. The interview results also underscored the importance of bidirectional alignment between humans and AI. We also identified opportunities for further development such as leveraging AI for career development and bidirectional human-AI alignment on the intent level.

\bibliographystyle{ACM-Reference-Format}
\bibliography{reference}

\appendix

\newpage

\section{Appendix A: Researchers Prompt Examples}
\label{appendix:a}

Below we provided researchers prompts examples, age and research experience (Exp), grouped by education level.

\begin{longtable}{|p{1cm}|p{2cm}|p{10cm}|}
\hline
\textbf{PID} & \textbf{Description} & \textbf{Prompt Example} \\ \hline
\multicolumn{3}{|c|}{\textbf{Master Students}} \\ \hline
P3 & Yr-1 Master \newline Age: 23 \newline Exp: 0.5 Yrs & I'm [anonymized] with 2 years of design experience, I fuse data, user insights, and business objectives to craft empowering user experiences, one interaction at a time. \newline I am interested in VR/AR and accessibility. I want to build a portfolio website focused on research projects. This website's color is based on orange and blue. \\ \hline
P8 & Yr-1 Master \newline Age: 23 \newline Exp: 1 Yrs & My name is [anonymized], I'm a first-year Human-Computer Interaction Master student. I studied Computer Science and Psychology, also really like cognitive science and drawing. I especially like comics, so I want my website to have some American comics styles. I want a personal website to display my front-end projects, my drawing works, and my research. Each project category will be a book (Book1, Book2, Book3...), when I click on the book, it will deliver me to the specific project page. \\ \hline
P11 & Yr-1 Master \newline Age: 23 \newline Exp: 0 Yrs & This is my homepage of my personal website. It includes the navigation bar, introduction of myself, and other basic things of the website page. The background color will be black, and the style of the website will be simple but also creative. My name is [anonymized], I used to study computer science, but now I changed into a design major. I think my website can show both of the knowledge about coding and design. The website will include pages for my research, my design, and my coding work, as well as a page for my personal life because I want to show my personality to the interviewer. \\ \hline
P4 & Yr-2 Master \newline Age: 23 \newline Exp: 3 Yrs & I'm [anonymized], a second-year master student of the human-computer interaction program. I studied psychology before and have some research projects related to that. I worked as a UX design intern in a few places. I'm graduating in May 2025 and want to apply for UX designer jobs. \\ \hline
\multicolumn{3}{|c|}{\textbf{Yr-1 PhD Students}} \\ \hline
P1 & Yr-1 PhD \newline Age: 25 \newline Exp: 3 Yrs & My name is [anonymized], and I'm a first-year PhD student at [anonymized] and a practicing speech-language pathologist. My research interests include AI integration when developing communication tools for AAC users. \\ \hline
P2 & Yr-1 PhD \newline Age: 25 \newline Exp: 4 Yrs & My name is [anonymized]. I'm now a first-year PhD student in [anonymized], working with Dr. [anonymized]. I have an education background in both electrical engineering and design. My research now focuses on AI-based assistive technologies, especially personalization systems for blind users. \\ \hline
P7 & Yr-1 PhD \newline Age: 29 \newline Exp: 4 Yrs & My website should showcase my current affiliation and my research publications. I wish it to be of vibrant color but with simplistic design. There should be different tabs to hold various content. \\ \hline
P10 & Yr-1 PhD \newline Age: 24 \newline Exp: 3 Yrs & Name: [anonymized] \newline Academic background: PhD student [anonymized] \newline Research interest: Data science for education using natural language processing tools \newline Personal hobbies: drawing, piano, cooking (pictures of dishes I cooked) \newline Style: minimalism \newline Base color: white \\ \hline
P13 & Yr-1 PhD \newline Age: 27 \newline Exp: 3.5 Yrs & My name is [anonymized], and I am an accessibility and UX researcher. I use he/him pronouns. I want to create an accessibility and data visualization portfolio website. I want to have a dark background with white text. The font size of the text should have high contrast and be very readable. \\ \hline
\multicolumn{3}{|c|}{\textbf{Yr-3/4 PhD Students}} \\ \hline
P5 & Yr-3 PhD \newline Age: 28 \newline Exp: 5.5 Yrs & I am [anonymized], a PhD student starting my third year in [anonymized]. My work is at the intersection of Human-Computer Interaction, Aging and Accessibility, and Personal and Health Informatics. My research focuses on investigating technologies for collecting and sharing personal health information among underrepresented populations, including older adults and people with mild cognitive impairment and dementia. Recently I have been working on supporting older adults in the data labeling process for training their personalized activity trackers. My work informs strategies that engage older adults as end-users in machine learning. \\ \hline
P6 & Yr-3 PhD \newline Age: 31 \newline Exp: 10 Yrs & This is my homepage for a website that I can use to showcase my credentials, blogging, and consulting work. I am [anonymized] and would like to introduce myself as a broadly trained social/behavioral scientist now working at the intersection of metascience and human-computer interaction. \\ \hline
P12 & Yr-3 PhD \newline Age: 29 \newline Exp: 4 Yrs & I want to build a research website showcasing my interests and publications. I am a 3rd year PhD student named [anonymized], my pronouns are [anonymized], and my research interests are broadly in multilingual NLP, human-centered NLP, authorship analysis, and explainability. \\ \hline
P9 & Yr-4 PhD \newline Age: 31 \newline Exp: 8 Yrs & I want the circles to be interlinked like a network and when you click on one I want it to expand and highlight more information and the rest to pull back to the sides. I would want to group them thematically with an overarching team science page. Each bubble you click on opens. The top right about corner would be static. \\ \hline
\end{longtable}

\newpage

\section{Appendix B: Prompts for Agentic Pipeline}
\label{appendix:b}

\subsection{Prompt for PRD generation}

\begin{lstlisting}
Please generate a Product Requirements Document (PRD) targeting the creation of a modern and user-friendly personal website for Junior Researchers based on the following user's sketch (the picture I sent you) and prompt.
User's prompt: ${userPrompt}
In the PRD, specify what images are needed and where they should be placed (e.g., hero image, profile image, etc.) using the format: [term(size)], please use concrete keywords like [(profile-picture)medium] instead of vague descriptions like [image1(small)].
There are 3 keywords for the size (small, medium, large, landscape, or portrait). Remember this only applies to images; for icons, you can just define them without the expected format.
Example: [portfolio-preview(landscape)]`
\end{lstlisting}


\subsection{Prompt for website code generation}

\begin{lstlisting}

You are a design engineer tasked with creating a user interface for junior researcher based on a user's wireframe sketch. Prioritize the user's considerations as design preferences while ensuring the design adheres to these principles:
1. Apply shadows judiciously enough to create depth but not overly done.
2. Use the Gestalt principles (proximity, similarity, continuity, closure, and connectedness) to enhance visual perception and organization.
3. Ensure accessibility, particularly in color choices; use contrasting colors for text, such as white text on suitable background colors, to ensure readability. Feel free to use gradients if they enhance the design's aesthetics and functionality.
4. Maintain consistency across the design.
5. Establish a clear hierarchy to guide the user's eye through the interface.
Additional considerations:
2. Utilize a CSS icon library Font Awesome in your <head> tag to include vector glyph icons.
3. Ensure all elements that can be rounded, such as buttons and containers, have consistent rounded corners to maintain a cohesive and modern visual style.
Based on the following Product Requirements Document (PRD) and User Prompt.
Product Requirements Document (PRD): ${storedPRD}
User's prompt: ${userPrompt}
Please incorporate the following images as specified:
${imageInsertionInstructions}
Please provide your output in HTML, CSS, and JavaScript without any explanations and natural languages(only code),with an emphasis on JavaScript for dynamic user interactions such as clicks and hovers.`;
      
\end{lstlisting}

\subsection{Prompt for code iteration idea}

\begin{lstlisting}

Based on the previously generated code, generate 3-5 ideas to improve the website design:
Previously Generated Code:
${previousCode}
Based on the previous design, please provide optimizations and enhancements focusing on:
1. Visual Consistency: Ensure a cohesive look and feel across the entire interface.
2. Unique Imagery: Suggest diverse and non-repetitive images that align with the theme of each section.
3. Component Refinement: Enhance the details of each UI component, considering:
- Button designs (hover states, shadows, etc.)
- Input field styles and interactions
- Card layouts and information hierarchy
4. Layout Improvements: Propose better ways to organize content for improved readability and user flow.
5. Color Scheme: Refine the color palette to improve contrast and visual appeal.
6. Typography: Suggest improvements in font choices, sizes, and text formatting for better readability.
7. Responsive Design: Ensure the layout adapts well to different screen sizes.
8. Interaction Design: Add subtle animations or transitions to improve user experience.
9. Accessibility: Suggest improvements to make the design more inclusive and easier to use for all users.
10. Performance Optimization: If applicable, propose ways to optimize the code for faster loading and rendering.
Please provide concise, innovative ideas that could enhance the user experience, visual appeal, or functionality of the website. Consider the existing code and suggest improvements or new features.`

\end{lstlisting}

\end{document}
\endinput