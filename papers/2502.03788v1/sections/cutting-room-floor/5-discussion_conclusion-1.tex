\section{Discussion}

\subsection{Summary of Findings}

AI as A Human Capability Enhancer, Not a Replacement

Mutual Adaptation Between Users and AI

Implications for AI-Driven Creativity Tools: the results of this tool can be generalized to many other AI tools for open-ended tasks such as creativity tasks

\subsection{Intent-AI interaction: Augmentation instead of Automation}

How can we better design AI tools
so that AI's become the baseline of human capability rather than the upper limit?

The goal should come from human needs

Intent-AI interaction: The development of Generative AI, particularly the capabilities of Large Language Models (LLMs) in interpreting and executing natural language, may be viewed as heralding the first new user interface paradigm shift in 60 years \cite{nielsen2023ai}. This shift moves from command-based interactions, typified by command line interfaces and graphical user interfaces, to intent-based outcome specification \cite{nielsen2023ai}. This emerging intent-based paradigm potentially enables users to communicate their intentions to machines without necessarily translating them into machine-comprehensible commands, whether through programming languages or graphical buttons. This shift may foster interfaces that support more abstract human expressions, especially for command-intensive tasks such as frontend development.

Previously, template-based website creation methods limited users to selecting from predefined website templates or dragging and dropping predefined components, which were relatively inflexible compared to AI-generated approaches. In contrast, AI-generated website creation allows users to define any elements they care about, such as colors, images, and more, while delegating less critical aspects to be generated by the AI. Showing intermediate steps for coding \cite{zhu-tianSketchThenGenerate2024}

The Relationship Between Intent Size and Model Size: Large Intents Require Large Models, but Can Small Models Handle Small Intents (such as tweaks)?
For instance, code could be generated by a large model and then refined by a smaller model to adjust its structure or positioning. This way, privacy-sensitive fine-tuning could be performed using a local small model.

Intent to Concept

From the results, we saw the ``serendipity" of AI in website generation, where AI could provide interesting ideas

While natural language is the most efficient way for humans to convey information, visual representation remains the most intuitive means of receiving it. Therefore, visual interfaces will remain essential, but their design will become more flexible and dynamic to better align with users' intentions. Future interfaces will not be limited to fixed buttons and commands but will focus more on presenting abstract or vague concepts, enabling deeper, higher-level interactions. For example, interfaces may assist users business in the same manner as board meetings, debates between professors and doctoral students about research ideas, or collaborative planning between writers and editors for serialized stories.

\section{Conclusion}