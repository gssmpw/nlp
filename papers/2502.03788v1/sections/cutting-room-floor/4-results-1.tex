\section{Results} \label{sec:results}

\subsection{AI Tools as an Augmenting Creative Ecosystem}

Participants mainly described AI tools in website design and front-end development not merely as code generators, but as an interactive ``creative ecosystem'' that continuously sparks ideas, guides learning, and facilitates communication. As one participant noted, these tools amplify human abilities rather than supplant them. Across the interviews, respondents perceived the AI as providing efficient starting points and inspiration for various design tasks, while humans ultimately remain responsible for fine-tuning and final decisions.

\subsection{Augmentation Rather Than Replacement}

All participants unanimously agreed that AI tools enhance human capabilities instead of replacing them. They emphasized that users still need fundamental coding or design knowledge to effectively use and adapt AI-generated outputs. Respondents widely stated that the tool ``does not weaken or replace human roles in front-end development and design'' (13/13). Rather, it serves as:

\begin{quote}
``An efficient starting point and a source of inspiration, but humans need to refine the details and make the final decisions.'' (P1)
\end{quote}

They pointed out that while future AI advances may reduce some manual tasks, current tools remain largely assistive. Many participants invoked the analogy that ``everyone can become a designer, front-end developer, or product manager'' (P3) through such technology:

\begin{quote}
``With these tools, everyone can become a designer, front-end developer, or product manager, thereby driving the creation of more innovative products.'' (P3)
\end{quote}

\subsection{Applicability to Different Experience Levels}
\subsubsection{Supporting Experienced Developers}
Several participants observed that for those with front-end experience (P10, P11, P13), the tool acts as an ``enhancer'' that accelerates the process from conception to prototype. Experienced users save time by leveraging quick code generation, then refining details as needed:

\begin{quote}
``For experts, it does 50\% of the job, and they can quickly edit the code.'' (P13)
\end{quote}

\subsubsection{Helping Novices or Non-Technical Users}
Others highlighted the tool's value for non-developers or those intimidated by front-end coding (P3, P9, P13). By lowering technical barriers, such a system stimulates broader creativity and invites more diverse users to build websites:

\begin{quote}
``No, yeah, it definitely doesn't replace anybody… It's definitely enhancing my capabilities by letting me start from something rather than from scratch.'' (P9)
\end{quote}

However, some cautioned that completely inexperienced users might still struggle once they need to customize or troubleshoot AI-generated layouts (P13). Participants agreed the tool is more effective for users who have at least minimal knowledge of HTML/CSS or web structures (9/13).

\subsection{Team Brainstorming and Collaborative Ideation}
Participants indicated that the AI tool is especially useful in group brainstorming sessions (P8). Multiple rapid prototypes can be generated, discussed, and iteratively merged into a refined design:

\begin{quote}
``Team members can quickly generate different versions of a website prototype, compare and debate them, and then converge on a more robust design solution.'' (P8)
\end{quote}

\subsection{Reducing Procrastination and Facilitating Learning}
Several respondents (P6, P11) stated that the tool helps overcome procrastination by removing the daunting first step of coding from scratch:

\begin{quote}
``The participant felt that the tool helped break the initial barrier to starting a personal website—a task they had been procrastinating.'' (P6)
\end{quote}

They also framed it as an educational tool that supports incremental learning of code and design principles along the way:

\begin{quote}
``It removes mundane steps, allows quick prototyping, and can serve as a learning platform. It nudges users toward more professional standards and teaches them along the way.'' (P6)
\end{quote}

\subsection{Recommendations to Others}
\subsubsection{Recommending to Experienced Peers}
Most participants (9/13) would recommend the tool to peers with at least some front-end knowledge (P10, P11, P13). For them, it accelerates the start-up phase and provides helpful code and layout snippets:

\begin{quote}
``I'd recommend it to peers who already have some front-end experience, because the tool can rapidly generate initial drafts for further refinement.'' (P10)
\end{quote}

\subsubsection{Recommending to Non-Designers}
Others (P7, P12) see strong benefits for individuals lacking UI design skills:

\begin{quote}
``I'd especially recommend it to those with minimal UI design experience. It can produce a creative, visually appealing draft and spark new ideas.'' (P7)
\end{quote}

For complete novices, participants suggested additional tutorials, templates, or intelligent prompts to handle the tool's freedom and complexity (P7, P10, P11).

\subsection{User–AI Co-Adaptation and Prompting Challenges}
Participants underscored a two-way adaptation process: users need to learn how to express precise requirements (prompts), and the AI tool must better interpret user intent. Fine-tuning prompts remains a key challenge (P2, P3, P6). Many wished for more guided prompting or hierarchical control to specify details (8/13). Additional suggestions included:

\begin{itemize}
    \item \textbf{Asking for clarifications:} AI should query users for ambiguous details instead of making inaccurate assumptions (P1, P8).
    \item \textbf{Offering auto-completion or keyword tips:} This would reduce guesswork and prompt-crafting effort (P3, P6, P9).
\end{itemize}

\subsection{Desire for Iterative and Interactive Refinement}
All participants (13/13) preferred a continuous design process over one-time generation. They wanted to combine, delete, or edit elements across multiple versions without returning to raw code. Many (P2, P3, P6, P9, P10, P11) suggested features like:

\begin{quote}
``Post-editing in a graphical interface: once the tool gives me a 70\%-satisfactory draft, I should drag and tweak elements directly rather than rewriting code.'' (P2)
\end{quote}

\begin{quote}
``I'd like the ability to refine or manipulate the generated output more directly, or issue follow-up instructions to alter specific parts of the design.'' (P12)
\end{quote}

\subsection{Expanding Tool Functionalities}
\subsubsection{More Shapes, Components, and Visual Resources}
Many participants found the current drawing and editing features too limited (P2, P3, P5, P7, P10, P12). They desired additional shape types (e.g., triangles, rounded rectangles, icons) and a richer component library to express UI/UX ideas more precisely:

\begin{quote}
``Right now, we can only draw rectangles and circles. Adding more shapes or a unified shape creation module would be friendlier.'' (P5)
\end{quote}

\subsubsection{Intermediate Steps for Color and Image Choices}
Some participants (P13) requested intermediate previews for color schemes and images before generating a final site:

\begin{quote}
``I'd prefer some intermediate steps to choose colors before finalizing. Being able to see and choose images beforehand would help too.'' (P13)
\end{quote}

\subsubsection{Multi-Modal Inputs}
Several noted the importance of allowing more diverse input, such as uploading reference images, screenshots, or style references (P8, P10, P11). Sketch input was particularly praised for capturing layout and component relationships (P11).

\subsection{Personal Expression, Serendipity, and Self-Reflection}
Participants (7/13) highlighted how AI-generated results can yield unexpected yet inspirational design elements:

\begin{quote}
``Sometimes, the tool generates surprising animations or effects I hadn't even considered, which expands my creative thinking.'' (P11)
\end{quote}

Beyond external presentation, respondents noted that rapidly prototyping multiple designs fosters deeper self-reflection on personal or research identities (P5). By iterating different layouts and styles, they clarified their priorities and values:

\begin{quote}
``Building a personal website isn't just for the public; it's also a chance to organize and reflect on my experiences and ideas.'' (P5)
\end{quote}

\subsection{Potential for Research and Innovation}
Many participants saw strong applicability for rapid prototyping in research contexts (P2, P3, P4, P7). Quick page iterations lower the threshold for experimental designs or concept testing. Some specifically envisioned more advanced AI features (e.g., local fine-tuning, role-based evaluation) to further expedite scholarly projects:

\begin{quote}
``If the tool's functionality grows, it will become more appealing. Rapid prototype creation can increase the iteration speed of experimental design.'' (P4)
\end{quote}

Overall, participants recognized significant potential for future expansions, including support for complex dynamic behaviors, multi-modal input, advanced customization, and integrated AI dialogues that clarify user intent. These developments, they suggested, would make AI-powered website design and front-end development an even more effective ecosystem for both novices and experts.