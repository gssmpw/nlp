\section{Introduction}
\label{sec:intro}

AI is increasingly woven into the fabric of human endeavor, yet its most transformative potential lies not in supplanting human talent but in elevating it. Rather than automating humans out of the creative process, AI can—and should—serve as a catalyst that allows individuals to express themselves more authentically, present themselves more effectively, and ultimately free up valuable time for higher-level intellectual pursuits. For emerging scholars, this vision is especially compelling. Junior researchers, such as early-stage PhD and master students, stand at a pivotal moment in their careers where outward self-presentation is intertwined with personal growth, skill-building, and academic identity formation.

Despite their abundance of new ideas and scholarly potential, early-career researchers often find themselves at a disadvantage in terms of personal branding and online visibility. Constructing a professional website—essential for showcasing research activities, academic networks, and scholarly aspirations—can pose significant technical and time-related hurdles. AI-based programming systems can play a supportive role here, not by overtaking the researcher's creative authority, but by empowering junior researchers to highlight their unique qualities and scholarly depth.

Fundamentally, programming can be viewed as a process of language translation: converting human needs, stated in natural language, into instructions that a machine can understand. Recent advances in large language models (LLMs) underscore this point: state-of-the-art models have boosted the share of pull requests resolved on GitHub from less than 2\% in 2023 to over 50\% by the end of 2024 \cite{jimenez2024swebench}. These breakthroughs not only reduce the burden of repetitive coding tasks but also foster more creative possibilities for human–AI collaboration. Previously, drag-and-drop website editors lowered the barrier to creating personal sites but often resulted in uniform designs that failed to capture researchers’ individuality. In contrast, generative AI offers junior researchers a means to construct a truly personalized online presence that more faithfully reflects their distinct contributions.

Throughout this process, AI-assisted programming does not strip the creative process of human agency. Rather, by freeing early-career researchers from the complexities of coding, it allows them to focus on more creative pursuits—such as shaping their scholarly narrative and refining their research agenda—thereby amplifying both their personal voice and academic impact.

Building on our goal of supporting self-representation through frontend development, we designed Frontend Diffusion, an end-to-end, model-agnostic frontend development agentic workflow. Our approach follows a multi-stage agentic workflow to translate user-drawn layouts and theme prompts into iteratively refined website code. We evaluated the system through a user study of thirteen participants with diverse technical backgrounds, revealing that the AI tool functions less as a mere code generator and more as an ``interactive creative ecosystem." Participants emphasized how AI augments rather than replaces human capabilities, learning curve and example requirements, mutual adaptation between users and AI, and multimodal inputs and cross-media inspiration.

Our contributions are two-fold:

\begin{enumerate}
    \item \textbf{System Development}: we developed Frontend Diffusion, a system...
    \item \textbf{Empirical Study}: we conducted a user study with 13 participants, and collected interview data on...
\end{enumerate} 

