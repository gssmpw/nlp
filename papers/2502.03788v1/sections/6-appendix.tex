\bibliographystyle{ACM-Reference-Format}
\bibliography{reference}

\appendix

\newpage

\section{Appendix A: Researchers Prompt Examples}
\label{appendix:a}

Below we provided researchers prompts examples, age and research experience (Exp), grouped by education level.

\begin{longtable}{|p{1cm}|p{2cm}|p{10cm}|}
\hline
\textbf{PID} & \textbf{Description} & \textbf{Prompt Example} \\ \hline
\multicolumn{3}{|c|}{\textbf{Master Students}} \\ \hline
P3 & Yr-1 Master \newline Age: 23 \newline Exp: 0.5 Yrs & I'm [anonymized] with 2 years of design experience, I fuse data, user insights, and business objectives to craft empowering user experiences, one interaction at a time. \newline I am interested in VR/AR and accessibility. I want to build a portfolio website focused on research projects. This website's color is based on orange and blue. \\ \hline
P8 & Yr-1 Master \newline Age: 23 \newline Exp: 1 Yrs & My name is [anonymized], I'm a first-year Human-Computer Interaction Master student. I studied Computer Science and Psychology, also really like cognitive science and drawing. I especially like comics, so I want my website to have some American comics styles. I want a personal website to display my front-end projects, my drawing works, and my research. Each project category will be a book (Book1, Book2, Book3...), when I click on the book, it will deliver me to the specific project page. \\ \hline
P11 & Yr-1 Master \newline Age: 23 \newline Exp: 0 Yrs & This is my homepage of my personal website. It includes the navigation bar, introduction of myself, and other basic things of the website page. The background color will be black, and the style of the website will be simple but also creative. My name is [anonymized], I used to study computer science, but now I changed into a design major. I think my website can show both of the knowledge about coding and design. The website will include pages for my research, my design, and my coding work, as well as a page for my personal life because I want to show my personality to the interviewer. \\ \hline
P4 & Yr-2 Master \newline Age: 23 \newline Exp: 3 Yrs & I'm [anonymized], a second-year master student of the human-computer interaction program. I studied psychology before and have some research projects related to that. I worked as a UX design intern in a few places. I'm graduating in May 2025 and want to apply for UX designer jobs. \\ \hline
\multicolumn{3}{|c|}{\textbf{Yr-1 PhD Students}} \\ \hline
P1 & Yr-1 PhD \newline Age: 25 \newline Exp: 3 Yrs & My name is [anonymized], and I'm a first-year PhD student at [anonymized] and a practicing speech-language pathologist. My research interests include AI integration when developing communication tools for AAC users. \\ \hline
P2 & Yr-1 PhD \newline Age: 25 \newline Exp: 4 Yrs & My name is [anonymized]. I'm now a first-year PhD student in [anonymized], working with Dr. [anonymized]. I have an education background in both electrical engineering and design. My research now focuses on AI-based assistive technologies, especially personalization systems for blind users. \\ \hline
P7 & Yr-1 PhD \newline Age: 29 \newline Exp: 4 Yrs & My website should showcase my current affiliation and my research publications. I wish it to be of vibrant color but with simplistic design. There should be different tabs to hold various content. \\ \hline
P10 & Yr-1 PhD \newline Age: 24 \newline Exp: 3 Yrs & Name: [anonymized] \newline Academic background: PhD student [anonymized] \newline Research interest: Data science for education using natural language processing tools \newline Personal hobbies: drawing, piano, cooking (pictures of dishes I cooked) \newline Style: minimalism \newline Base color: white \\ \hline
P13 & Yr-1 PhD \newline Age: 27 \newline Exp: 3.5 Yrs & My name is [anonymized], and I am an accessibility and UX researcher. I use he/him pronouns. I want to create an accessibility and data visualization portfolio website. I want to have a dark background with white text. The font size of the text should have high contrast and be very readable. \\ \hline
\multicolumn{3}{|c|}{\textbf{Yr-3/4 PhD Students}} \\ \hline
P5 & Yr-3 PhD \newline Age: 28 \newline Exp: 5.5 Yrs & I am [anonymized], a PhD student starting my third year in [anonymized]. My work is at the intersection of Human-Computer Interaction, Aging and Accessibility, and Personal and Health Informatics. My research focuses on investigating technologies for collecting and sharing personal health information among underrepresented populations, including older adults and people with mild cognitive impairment and dementia. Recently I have been working on supporting older adults in the data labeling process for training their personalized activity trackers. My work informs strategies that engage older adults as end-users in machine learning. \\ \hline
P6 & Yr-3 PhD \newline Age: 31 \newline Exp: 10 Yrs & This is my homepage for a website that I can use to showcase my credentials, blogging, and consulting work. I am [anonymized] and would like to introduce myself as a broadly trained social/behavioral scientist now working at the intersection of metascience and human-computer interaction. \\ \hline
P12 & Yr-3 PhD \newline Age: 29 \newline Exp: 4 Yrs & I want to build a research website showcasing my interests and publications. I am a 3rd year PhD student named [anonymized], my pronouns are [anonymized], and my research interests are broadly in multilingual NLP, human-centered NLP, authorship analysis, and explainability. \\ \hline
P9 & Yr-4 PhD \newline Age: 31 \newline Exp: 8 Yrs & I want the circles to be interlinked like a network and when you click on one I want it to expand and highlight more information and the rest to pull back to the sides. I would want to group them thematically with an overarching team science page. Each bubble you click on opens. The top right about corner would be static. \\ \hline
\end{longtable}

\newpage

\section{Appendix B: Prompts for Agentic Pipeline}
\label{appendix:b}

\subsection{Prompt for PRD generation}

\begin{lstlisting}
Please generate a Product Requirements Document (PRD) targeting the creation of a modern and user-friendly personal website for Junior Researchers based on the following user's sketch (the picture I sent you) and prompt.
User's prompt: ${userPrompt}
In the PRD, specify what images are needed and where they should be placed (e.g., hero image, profile image, etc.) using the format: [term(size)], please use concrete keywords like [(profile-picture)medium] instead of vague descriptions like [image1(small)].
There are 3 keywords for the size (small, medium, large, landscape, or portrait). Remember this only applies to images; for icons, you can just define them without the expected format.
Example: [portfolio-preview(landscape)]`
\end{lstlisting}


\subsection{Prompt for website code generation}

\begin{lstlisting}

You are a design engineer tasked with creating a user interface for junior researcher based on a user's wireframe sketch. Prioritize the user's considerations as design preferences while ensuring the design adheres to these principles:
1. Apply shadows judiciously enough to create depth but not overly done.
2. Use the Gestalt principles (proximity, similarity, continuity, closure, and connectedness) to enhance visual perception and organization.
3. Ensure accessibility, particularly in color choices; use contrasting colors for text, such as white text on suitable background colors, to ensure readability. Feel free to use gradients if they enhance the design's aesthetics and functionality.
4. Maintain consistency across the design.
5. Establish a clear hierarchy to guide the user's eye through the interface.
Additional considerations:
2. Utilize a CSS icon library Font Awesome in your <head> tag to include vector glyph icons.
3. Ensure all elements that can be rounded, such as buttons and containers, have consistent rounded corners to maintain a cohesive and modern visual style.
Based on the following Product Requirements Document (PRD) and User Prompt.
Product Requirements Document (PRD): ${storedPRD}
User's prompt: ${userPrompt}
Please incorporate the following images as specified:
${imageInsertionInstructions}
Please provide your output in HTML, CSS, and JavaScript without any explanations and natural languages(only code),with an emphasis on JavaScript for dynamic user interactions such as clicks and hovers.`;
      
\end{lstlisting}

\subsection{Prompt for code iteration idea}

\begin{lstlisting}

Based on the previously generated code, generate 3-5 ideas to improve the website design:
Previously Generated Code:
${previousCode}
Based on the previous design, please provide optimizations and enhancements focusing on:
1. Visual Consistency: Ensure a cohesive look and feel across the entire interface.
2. Unique Imagery: Suggest diverse and non-repetitive images that align with the theme of each section.
3. Component Refinement: Enhance the details of each UI component, considering:
- Button designs (hover states, shadows, etc.)
- Input field styles and interactions
- Card layouts and information hierarchy
4. Layout Improvements: Propose better ways to organize content for improved readability and user flow.
5. Color Scheme: Refine the color palette to improve contrast and visual appeal.
6. Typography: Suggest improvements in font choices, sizes, and text formatting for better readability.
7. Responsive Design: Ensure the layout adapts well to different screen sizes.
8. Interaction Design: Add subtle animations or transitions to improve user experience.
9. Accessibility: Suggest improvements to make the design more inclusive and easier to use for all users.
10. Performance Optimization: If applicable, propose ways to optimize the code for faster loading and rendering.
Please provide concise, innovative ideas that could enhance the user experience, visual appeal, or functionality of the website. Consider the existing code and suggest improvements or new features.`

\end{lstlisting}