\section{Introduction}
\label{sec:intro}

With the rapid advancements in AI's logical reasoning capabilities \cite{guo2025deepseek,openaiOpenAIO3mini2025}, technical productivity has been enhanced across various domains. In programming, state-of-the-art models have increased the resolution rate of pull requests on GitHub from under 2\% in 2023 to over 50\% by the end of 2024 \cite{jimenez2024swebench}. These advancements have not only spurred companies' interest in adopting agentic workflows within software development but have also raised concerns among human professionals about the potential for AI to replace their roles. This anxiety has been particularly pronounced in creative industries, where screenwriters face competition from text-generation models \cite{mirowskiCoWritingScreenplaysTheatre2023} and illustrators contend with text-to-image models \cite{koLargescaleTexttoImageGeneration2023,jiangAIArtIts2023,kawakamiImpactGenerativeAI2024}. Such realities motivate us to explore a future where humans and AI coexist synergistically. Instead of automating humans out of the creative process, AI can—and should—act as a catalyst for enhancing self-expression, facilitating more effective personal and professional presentation, and freeing up time for higher-level intellectual pursuits.

This vision holds particular relevance for emerging scholars, such as early-career PhD and master's students, who stand at a critical juncture in their academic journeys. For them, self-presentation is intricately tied to personal growth, skill development, and the formation of academic identities. Despite their wealth of new ideas and scholarly potential, junior researchers often face challenges in building their professional brands and achieving online visibility. Creating a professional website—an essential tool for showcasing research, academic networks, and aspirations—can be a daunting task due to technical and time-related constraints. Agentic programming workflows have the potential to address these challenges, not by replacing the researcher's creative authority, but by enabling them to focus on more meaningful pursuits, such as crafting their scholarly narratives and refining their research agendas, ultimately emphasizing their unique academic identities and depth.

To support self-representation through agentic workflows, we developed \textit{Frontend Diffusion}, an end-to-end, model-agnostic frontend development system. Our system employs a multi-stage agentic workflow to transform user-drawn layouts and thematic prompts into iteratively refined website code. We evaluated the system through a user study involving 13 participants with diverse technical backgrounds. The findings revealed that the AI tool functions not merely as a code generator but as a collaborative partner. Participants emphasized two major themes: AI as a Human Capability Enhancer, Not a Replacement, and Bidirectional Human-AI Alignment.

Our contributions are twofold:  

\begin{enumerate}
\item \textbf{System Development}: we designed and implemented Frontend Diffusion, a system that enables users to generate professional websites by combining sketches and textual prompts in a multi-stage agentic workflow.
\item \textbf{Empirical Study}: we conducted a user study with 13 participants and collected qualitative data through interviews to understand their perceptions of the system as human capability enhancer and its bidirectional Human-AI alignment.
\end{enumerate}