\section{Discussion}

\subsection{Summary and Implications of Findings}

In this paper, we introduce a novel system, Frontend Diffusion, and present findings from a user study involving 13 junior researchers and designers, supported by in-depth interview results. The task of AI-assisted frontend development spans multiple stages of creative work, including ideation, prototyping, iteration, and development. Consequently, the insights gained from this study are broadly applicable to a wide range of open-ended and creative workflows.

When supporting the self-representation of junior researchers and designers, AI has the potential to enhance human capabilities by alleviating the cognitive and technical burden of repetitive tasks. This enables users to dedicate more time to higher-level ideation and reflection. Furthermore, we observed a bidirectional human-AI alignment \cite{shenBidirectionalHumanAIAlignment2024}: AI-initiated alignment (from AI to human) includes onboarding support for new users and prompt guidance and refinement, while human-initiated alignment (from human to AI) involves fine-grained user control over details and the ability to harness and combine AI's unexpected creative sparks. Additionally, we identified key possibilities and challenges in communicating dynamic and interactive behaviors, as well as the need for multimodal inputs to improve AI's ability to understand and anticipate user intent.

\subsection{AI as A Human Capability Enhancer: AI Career Advising and Planning}
\label{ref:career}

With technological advancements—especially in AI and robotics that exhibit certain human capabilities—the pace of societal change is accelerating. Consequently, individuals' career trajectories may experience more frequent transitions. In this inevitable trend, AI can also function as an enhancer assisting individuals in navigating these transitions—from the creation of personal websites to career planning and advising. As noted by P6 in the interview, there is an interest in employing AI to role-play as a recruiter, thereby providing guidance on personal websites. Moreover, AI could enable advise seekers to ``try on" career scenarios of interest by having the AI role-play as clients or colleagues within those scenarios, thereby enabling low-cost experimentation during career transitions. While AI provides career advising and planning, potential challenges must also be considered—for example, how to preserve human agency, such as personal interests, passion, and self-efficacy, rather than yielding an AI-optimized career path. This approach helps avoid an overreliance on AI-driven career optimization that could lead to homogenized trajectories and ultimately aims to build a user-initiated career ecosystem—one in which AI offers data-driven insights while humans retain the final decision-making power over non-linear, interest- and passion-driven career narratives.

\subsection{Bidirectional Human-AI Alignment: Alignment on Intent Level}

Currently, when executing complex tasks with AI, it is necessary to employ an agentic workflow that decomposes high-level human intents into executable steps. For example, in this study, the agentic workflow decomposes the intent to create a personal website into specific tasks such as generating a product requirements document (PRD) based on a sketch and prompt, performing image searches, generating code, and so forth. As AI's logical reasoning capabilities improve \cite{guo2025deepseek,openaiOpenAIO3mini2025}, it is becoming feasible for AI to autonomously break down human intents into concrete steps through reasoning. For instance, in answering a user's query, AI can determine the various perspectives that need to be considered. It is foreseeable that, in the future, many human–AI interactions will occur at the intent level.

This introduces new opportunities and challenges for human–AI alignment. For instance, when reasoning about complex problems, it could be time-consuming for reasoning models to generate the final outcome; users prefer to examine and adjust intermediate steps such as chain-of-thought, prior to generating the final output (e.g., the ``sketch then generate" approach \cite{zhu-tianSketchThenGenerate2024}). In future research on human-AI interaction, greater emphasis should be placed on aligning intermediate steps as intent-AI interaction \cite{dingIntentbasedUserInterfaces2024}.

\subsection{Limitation}

There were several limitations in the current study that warrant further investigation. First, the user sample was relatively narrow, as the study exclusively involved 13 junior researchers and designers—primarily master's and doctoral students. This limited sample did not encompass a broader demographic, such as participants from non-academic backgrounds or professionals from interdisciplinary fields. Consequently, the generalizability of our findings to a more diverse user population remained uncertain. Second, this study leveraged publicly available online image repositories\footnote{https://www.pexels.com} when generating website content to avoid direct use of participants' personal images. This approach resulted in the selection of images that participants perceived as irrelevant or misaligned with their personal or professional identity, potentially impacting the overall user experience.

\subsection{Future Work}

Future research could aim to address these limitations while exploring new directions. Regarding privacy and security, subsequent work might investigate localized solutions for handling sensitive data. One potential avenue is local deployment of open-source language models \cite{guo2025deepseek}. These approaches could enable localizing the processing of personal information and other sensitive data.

In addition, future work might consider expanding the application of AI in career development as discussed in Section \ref{ref:career}. Potential directions include career planning and advising agents and virtual career environments. Moreover, it is important to examine the long-term impacts of these AI tools on self-expression and career development. Longitudinal studies could provide insights into how AI-assisted systems influence users' academic identities and digital brand building over time, thereby offering insights on the broader social and career implications of integrating AI into personal development strategies.

\section{Conclusion}

This paper introduced Frontend Diffusion, a multi-stage agentic system designed to support junior researchers and designers in self-representation through website generation. We conducted a user study with 13 junior researchers and designers who used the system to generate personal websites. The results indicated that rather than replacing human abilities, AI enhances them by streamlining technical tasks, thereby allowing users to focus more on ideation and identity-building. The interview results also underscored the importance of bidirectional alignment between humans and AI. We also identified opportunities for further development such as leveraging AI for career development and bidirectional human-AI alignment on the intent level.