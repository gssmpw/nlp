\section{System Design}

\begin{figure}
  \centering
  \includegraphics[width=1\textwidth]{figures/Archi.png}
    \caption{Agentic workflow of frontend code generation system: (1) Sketch-to-PRD generation, (2) PRD-to-code generation and code refinement. The prompts for PRD, code and idea generation are attached in Appendix \ref{appendix:b}.}
    \Description{Agentic workflow of frontend code generation: (1) Sketch-to-PRD generation, (2) PRD-to-code generation and code refinement.}
    \label{fig:structure}
  \end{figure}

We developed Frontend Diffusion, an end-to-end AI-powered frontend code generation tool as illustrated by Figure \ref{fig:structure}. Constructing a personal website based on both sketches and textual prompts is inherently complex, given the need to account for not only the content to be displayed (e.g., the type of information) but also the format (e.g., layout, color schemes, and style). Although language models demonstrate capabilities in translating user intentions into code, the initial user inputs—such as a brief sketch and prompt—often leave information gaps relative to the final website. Attempting to fill all details in a single step frequently results in coarse or incomplete outcomes. To address this limitation, we employ a multi-stage agentic workflow, wherein the website creation process is divided into multiple tasks, mainly in two phrases: (1) Sketch-to-PRD generation, (2) PRD-to-code generation and code refinement as shown in Figure \ref{fig:structure}.

\subsection{Phase 1: Sketch-to-PRD Generation}

This step transforms the user's visual and textual inputs into a structured document, referred to as the Product Requirements Document (PRD), which serves as a blueprint for the website's development process. Upon completion of the user's sketch and theme input, the user can activate the code generation process via ``Generate" button. The system then converts the sketch into SVG format, followed by a subsequent transformation into JPG format. This two-step conversion process was implemented based on empirical evidence from our tests, showing that language models exhibit better performance when processing images in JPG format compared to images in SVG format.

Then we used the JPG formats to generate PRD. To enhance the visual appearance of the generated websites, the system integrates the Pexels API\footnote{https://www.pexels.com/api/} for image retrieval. The language model is specifically prompted to include image terms and size descriptions (e.g., [school(large)]). These descriptors are subsequently utilized to query the Pexels API, which returns relevant image URLs for incorporation into the PRD.

\subsection{Phase 2: PRD-to-code Generation and Code Refinement}

The phase 2 of the system consists of two steps: (1) PRD-to-code generation: the system utilizes the generated PRD and the original user prompt as inputs for code generation; (2) code refinement: the system implements an iterative refinement process to automatically enhance the generated website with richer functionality and reduced flaws. This process involves analyzing the initial code to generate optimization suggestions, merging these suggestions with the original theme, and utilizing the enhanced theme along with the previously generated PRD to regenerate the code. The system executes this iterative refinement process multiple times (by default, n=4). Users can navigate between iterations by selecting preview thumbnails displayed at the interface's bottom, and can access or copy the generated code for each version.

All the generation steps utilize the Claude 3.5 Sonnet language model (claude-3-5-sonnet-20241022)\footnote{https://www.anthropic.com/news/claude-3-5-sonnet}, one of the most advanced language models for coding as of December 2024\footnote{https://www.swebench.com/}. We also anticipate rapid developments in LLMs, so the task transition techniques described herein are designed to be model-agnostic, ensuring their applicability to future, more advanced LLMs. The project code will be open-sourced after the paper's anonymity period ends.

\section{Study Design}

The experiment lasted approximately 45 minutes and consisted of three parts: a 5-minute participant onboarding, a 20-30 minute user study, and a 15-20 minute post-study interview. First, participants read and signed the consent form, granting permission for screen recording. Next, the researcher demonstrated how to use it by creating one website page. Participants then created between 1 to 5 personal website pages depending on participants' needs. These pages could showcase either professional content, such as research or design portfolios, or personal interests, such as an image gallery or reading list. After completing the user study, the researcher conducted a 15-20 minute post-study interview, following the protocol outlined in Section \ref{sec:post-study}.

\subsection{Demographic Information}

In the pre-study survey, we collected demographic information, including participants' age, gender, educational level, experience in web development, design, and research, as well as whether they had a personal website (Y/N). Table \ref{tab:demographic} summarizes this information. Among the 13 participants, there were 4 master's students, 5 first-year Ph.D. students, and 4 third- or fourth-year Ph.D. students. Regarding website development experience, 6 participants reported having a personal website. Overall, the group represented a relatively diverse range of website creation backgrounds.

\begin{table}[h!]
\centering
\begin{tabular}{cccccccccc}
\toprule
& \multicolumn{3}{c}{Demographic} & \multicolumn{3}{c}{Experience (in yrs)} & \multicolumn{2}{c}{Website} \\
PID & Age & Gender & Edu & Dev & Design & Research & Prev. Website & Pages created \\
\midrule
1 & 25 & F & Yr-1 PhD & 0 & 0 & 3 & N & 5 \\
2 & 25 & F & Yr-1 PhD & 0.1 & 0.2 & 4 & Y & 3 \\
3 & 23 & F & Yr-1 master & 1 & 3 & 0.5 & Y & 3 \\
4 & 23 & F & Yr-2 master & 0.5 & 3 & 3 & Y & 2 \\
5 & 28 & F & Yr-3 PhD & 1.5 & 4 & 5.5 & Y & 4 \\
6 & 31 & M & Yr-3 PhD & 0.5 & 1 & 10 & N & 5 \\
7 & 29 & M & Yr-1 PhD & 0 & 0 & 4 & N & 3 \\
8 & 23 & F & Yr-1 master & 1 & 0 & 1 & N & 5 \\
9 & 31 & F & Yr-4 PhD & 0 & 1 & 8 & N & 4 \\
10 & 24 & F & Yr-1 PhD & 0.5 & 0 & 3 & N & 1 \\
11 & 23 & F & Yr-1 master & 3 & 1 & 0 & Y & 5 \\
12 & 29 & M & Yr-3 PhD & 1 & 0 & 4 & N & 4 \\
13 & 27 & M & Yr-1 PhD & 2 & 3 & 3.5 & Y & 3 \\
\bottomrule
\end{tabular}
\caption{Participants demographic information, including age, gender, educational level, experience in web development (dev), design, research, having personal website (Y/N), and number of pages (each page has 4 iterations) created in the study.}
\label{tab:demographic}
\end{table}



\subsection{Post-Study Interview Protocol}
\label{sec:post-study}

Following the completion of the website creation task, we conducted semi-structured interviews with each participant to gain insights into their experiences, perceptions, and overall satisfaction with the tool. The interviews took place immediately after the hands-on sessions, ensuring that participants' impressions and feedback were fresh and accurately captured. Each interview lasted approximately 20--30 minutes and was audio-recorded, with participants' consent, for subsequent transcription and thematic analysis.

The interview protocol was designed to explore three main dimensions: (1) User Experience, (2) Improvement and Recommendation, and (3) Broader Impacts and Perceptions. While the interviewer maintained flexibility to probe further into interesting responses, all participants were asked the same core questions to ensure consistency. The interview outline was as follows:

\begin{itemize}
    \item \textbf{User Experience}
    \begin{itemize}
        \item \textbf{Process:} Did you find the process of using the tool to generate a personal website smooth? Were there any steps that were particularly time-consuming or challenging?
        \item \textbf{Result:} Did the final website meet your expectations? Do you think the tool accurately understood your intent between sketch input and the generated website? Why or why not?
    \end{itemize}

    \item \textbf{Improvement and Recommendation}
    \begin{itemize}
        \item Are there any features you would like the tool to add or improve? Did you feel the need for more interaction or guidance from the tool?
        \item Would you recommend this tool to your peers? Why or why not?
    \end{itemize}

    \item \textbf{Broader Impacts and Perceptions}
    \begin{itemize}
        \item Does this tool increase or decrease your interest in frontend development, or even research?
        \item Do you think this tool enhances human abilities or replaces them? Why?
    \end{itemize}
\end{itemize}