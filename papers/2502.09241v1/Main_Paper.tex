
% This is samplepaper.tex, a sample chapter demonstrating the
% LLNCS macro package for Springer Computer Science proceedings;
% Version 2.20 of 2017/10/04
%
\documentclass[runningheads]{llncs}
%
\usepackage{amsmath,amsfonts}
\usepackage{algorithmic}
\usepackage{algorithm}
\usepackage{array}
\usepackage[caption=false,font=normalsize,labelfont=sf,textfont=sf]{subfig}
\usepackage{textcomp}
\usepackage{stfloats}
\usepackage{url}
\usepackage{verbatim}
\usepackage{graphicx}
\usepackage{cite}
\usepackage{caption}
\usepackage{nomencl}
\usepackage{ntheorem}   % for theorems
\usepackage{lipsum}     % for sample text
\usepackage{graphicx}
\usepackage{parskip}
\usepackage{adjustbox}
\usepackage{xcolor}
\usepackage{hyperref}



\hypersetup{
	colorlinks=true,
	linkcolor=blue,
	filecolor=green, 
	urlcolor=black,
	citecolor=orange
}

\usepackage{bm}
\newcommand{\vect}[1]{\boldsymbol{\mathbf{#1}}}
\DeclareRobustCommand{\uvec}[1]{{%
		\ifcsname uvec#1\endcsname
		\csname uvec#1\endcsname
		\else
		\bm{\mathbf{#1}}%
		\fi
}}
\usepackage{graphicx}
% Used for displaying a sample Fig.. If possible, Fig. files should
% be included in EPS format.
%
% If you use the hyperref package, please uncomment the following line
% to display URLs in blue roman font according to Springer's eBook style:
% \renewcommand\UrlFont{\color{blue}\rmfamily}
\setlength{\parindent}{0pt}

\renewcommand{\baselinestretch}{1}

\begin{document}
%
\title{Safety Evaluation of Human Arm Operations Using IMU Sensors with a Spring-Damper-Mass Predictive Model }
%
\titlerunning{Safety Evaluation of Human Arm}
% If the paper title is too long for the running head, you can set
% an abbreviated paper title here
%
\author{Musab Zubair Inamdar\orcidID{0009-0008-0932-9896} \and Seyed Amir Tafrishi\orcidID{0000-0001-9829-3144}}
%
%\authorrunning{Musab Zubair Inamdar and Seyed Amir Tarfisi}
% First names are abbreviated in the running head.
% If there are more than two authors, 'et al.' is used.
%
\institute{Geometric Mechanics and Mechatronics in Robotics (gm$^2$R) Lab, School of Engineering, Cardiff University, Queen's Buildings, The Parade, Cardiff, CF24 3AA  \\
\email{\{InamdarMZ, tafrishisa\}@cardiff.ac.uk}\\
}
%
\maketitle              % typeset the header of the contribution
%
\begin{abstract}
This paper presents a novel approach to real-time safety monitoring in human-robot collaborative manufacturing environments through a wrist-mounted Inertial Measurement Unit (IMU) system integrated with a Predictive Safety Model (PSM). The proposed system extends previous PSM implementations through the adaptation of a spring-damper-mass model specifically optimized for wrist motions, employing probabilistic safety assessment through impedance-based computations. We analyze our proposed impedance-based safety approach with frequency domain methods, establishing quantitative safety thresholds through comprehensive comparative analysis. Experimental validation across three manufacturing tasks - tool manipulation, visual inspection, and pick-and-place operations. Results show robust performance across diverse manufacturing scenarios while maintaining computational efficiency through optimized parameter selection. This work establishes a foundation for future developments in adaptive risk assessment in real-time for human-robot collaborative manufacturing environments.

%This paper presents a novel approach to real-time safety monitoring in human-robot collaborative manufacturing environments through the development and validation of a wrist-mounted Inertial Measurement Unit (IMU) system integrated with a Predictive Safety Model (PSM). The system implements a spring-damper-mass model optimized for wrist-specific applications in industrial tasks, employing probabilistic safety assessment through impedance-based computations to establish quantitative thresholds for differentiating between safe and hazardous motions. Three essential manufacturing operations were evaluated: visual inspection, tool manipulation, and pick-and-place tasks. We analyzed our proposed impedance-based safety with frequency domain methods. Results demonstrate the system's ability to effectively distinguish between safe and unsafe motions. This work contributes to the advancement of human-robot collaboration by providing a robust, task-specific safety monitoring solution that enhances workplace safety without compromising operational efficiency.

\keywords{Inertia sensors \and Predictive models \and Safety monitoring \and Human-Robot Collaboration (HRC) }
\end{abstract}

\section{Introduction}
%1 paragraph why reconfigurable applications 3 sentences
%2 paragraphs, 1) rolling systems (actuation) 2) coupling in reconfigurable robots 
%1 paragraph Disk rolling systems their importance and motivations 
%1 your contributions rolling disk system... 
%% Here is intro
 

% 1 paragraph very general about safety and how it is important in robotics
% 1 paragraph standardzation ISO/etc
% 1-2 paragraph robotics and motivation 
% What we contribute 
% Sections orginizations

%\section{Design Motivation}
%\subsection{Modular Self-Recongifurable Robot (MSRR)}

The advent of collaborative robotics has revolutionized the manufacturing sector, enabling direct interaction and cooperation between humans and robots within shared workspaces, which remains a key area of current research \cite{vicentini2021collaborative,ravankar2022care,tafrishi2024safety}. While this transformation has significantly enhanced productivity, adaptability, and efficiency in the industry, it has also raised critical challenges related to operator safety, particularly in scenarios where robots perform essential tasks in close proximity to human operators \cite{davisonsafety,ravankar2022care}. The establishment of safety standards in robotics has been driven by the pressing need to create clear norms and criteria for human-robot interaction, particularly as industrial processes become more complex and flexible. These standards provide explicit guidelines for the development, implementation, and operation of collaborative robotic systems, taking into account characteristics such as robot velocity, strength, and the distance preserved between robots and human operators. However, the growing sophistication of collaborative robots has led to an escalating need for more comprehensive safety models that can accurately assess and predict human movement safety in real time, necessitating the integration of various technologies including sensors and algorithms to implement a proactive approach to risk management in Human-Robot Collaboration (HRC) environments \cite{tafrishi2022psm}. 

The implementation of defined safety protocols, including ISO/TS 15066 and ISO 10218, has established essential parameters for human-robot collaboration in industrial settings. While these standards are thorough in their approach to safety requirements, they are frequently constrained by static safeguards and fail to adequately address the dynamic characteristics of human movements \cite{tafrishi2022psm}. Research indicates that strict adherence to these standards may result in operational inefficiencies, adversely impacting both productivity and the quality of operator interactions \cite{pupa2023dynamic}. This highlights a critical gap between conventional safety standards and the needs of modern collaborative environments, where dynamic and adaptive safety measures are increasingly necessary. The evolution towards Industry 4.0 and 5.0 has further emphasized this disparity, with recent studies highlighting the necessity for safety systems capable of adapting to the growing complexity of human-robot interactions while maintaining rigorous safety standards \cite{benmessabih2024online}.

Since the 1970s, the evolution of safety systems in manufacturing has experienced considerable advancements, with initial research predominantly concentrating on robot-centric safety protocols \cite{tafrishi2022psm}. The original methodologies focused on enhancing automation techniques and control systems to establish secure work environments, resulting in diverse approaches for robotic manipulation in conjunction with human collaboration \cite{li2017adaptive}. However, these systems frequently regarded the human component as an opaque entity, resulting in constraints on their practical application. Recent advancements in sensor technology and motion analysis have facilitated more refined methodologies for safety monitoring. Pedrocchi et al. \cite{pedrocchi2013safe} illustrated the efficacy of monitoring position sensors for collision avoidance, while further studies have advanced into predictive algorithms for human-robot interaction \cite{zanchettin2015safety}, \cite{unhelkar2018human}.

The rising complexity of collaborative production settings has underscored the necessity of monitoring and guaranteeing safety during the execution of specified tasks. Conventional methods of workplace safety frequently depend on passive measures or reactive systems, which may insufficiently handle the evolving characteristics of contemporary manufacturing processes \cite{gualtieri2021emerging}. The incorporation of safety monitoring systems in manufacturing must reconcile the necessity for thorough safety evaluation with pragmatic implementation factors. Chan and Tsai \cite{chan2020collision} emphasize that successful safety systems must both avoid accidents and sustain operational efficiency. This equilibrium is especially vital in situations where employees engage in diverse tasks necessitating distinct motion patterns and safety considerations, particularly in activities requiring precise wrist movements such as tool manipulation, quality inspection, and material handling, where small deviations from safe motion patterns may result in accidents or quality concerns \cite{prakash2018recent}.

%\begin{Fig.}[t!]
%    \centering
 %   \includegraphics[width=0.27\textwidth]{The variables of the %simplified 1 DOF arbitrary pendulum model.png}
 %   \caption{The variables of the simplified 1 DOF arbitrary %pendulum model}
%    \label{fig: Modular Couple}
%\end{Fig.}



This research expands on the groundwork established by Tafrishi et al. \cite{tafrishi2022psm}, enhancing their Predictive Safety Model (PSM) methodology to address the specific challenges of wrist motion safety in manufacturing settings. The experimental setup is shown in Fig. \ref{fig: IMU_setup}, while the overall system architecture is illustrated in Fig. \ref{fig: PSMblock}. By adapting and improving the existing PSM framework from upper body to wrist motion analysis, this study seeks to reconcile theoretical safety models with practical implementations in industrial environments, specifically targeting routine manufacturing operations that require precise wrist control and manipulation. 

The main contributions of this work are as follows:
\begin{itemize}
    \item The design and development of a wrist-mounted IMU-based safety monitoring system utilizing a predictive spring-damper-mass (PSM) model.
    \item Formulation of task-specific safety criteria tailored to tool manipulation, visual inspection, and pick-and-place operations.
    \item Proposing safety quantification based on impedance error and frequency domain scaling error analysis to avoid sampling limitation that occurred in the previous study \cite{tafrishi2022psm}.
    \item Comprehensive validation of the proposed system through detailed error analysis in both time and frequency domains.
\end{itemize}

The remainder of this paper is organized as follows. Section \ref{sec:Design} presents the Predictive Safety Model Framework, including system architecture, parameter optimization, data collection procedures, probability matching methodologies, and safety assessment criteria. Section \ref{sec:motion study} presents the experimental results and discussion, covering detailed analysis of safe and unsafe motions, comprehensive error analysis in both time and frequency domains, and real-time safety evaluation across various manufacturing tasks. Finally, Section \ref{sec:conclusions} concludes the paper by summarizing the main contributions and discussing future research directions.

\section{Predictive Safety Model (PSM) Design For Arm}\label{sec:Design}
In this section, we present the preliminaries of the predictive safety model (PSM) introduced by Tafrishi et al. \cite{tafrishi2022psm}. Furthermore, we discuss the interpolation and modifiations of the developed PSM model from the upper body to the human arm. Additionally, we introduce a new impedance error analysis, which emphasizes the advantages of incorporating time-domain position and velocity errors within the safety evaluation compared to the previous approach \cite{tafrishi2022psm}.



The PSM approach effectively integrates physical modeling concepts with real-time motion analysis for safety assessment, through the implementation of a spring-damper pendulum system \cite{tafrishi2022psm}. The PSM framework incorporates various elements for immediate safety evaluation, as depicted in Fig. \ref{fig: PSMblock}. The system architecture integrates IMU sensor data (consisting of angular motion data with position, velocity and acceleration angles) with a probabilistic dataset for secure motion and a spring-damper pendulum model to produce thorough safety evaluations through error signals by presenting position and velocity discrepancies \cite{tafrishi2022psm}.

The incorporation of dynamic restrictions in predictive models enhances safety in dynamic contexts \cite{li2017adaptive}. Invariance control ensures safety limits under external disruptions, adapting to environmental changes while guaranteeing reliable safety in the physical human-robot interaction \cite{kimmel2017invariance}. Predictive processes utilize reduced-dimensional probability datasets to evaluate motion safety, correlating angular positions ($\theta_g$) and velocities ($\omega$) with probability values to form a safety evaluation framework \cite{tafrishi2022psm}. A one-degree-of-freedom pendulum model serves as the mathematical basis of the PSM methodology, quantifying safety through parameters such as angular position ($\theta_g$) and height ($h$).

\begin{figure}[t!]
    \centering
    \includegraphics[width=0.25\textwidth]{Complete_Experimental.png}
    \includegraphics[width=0.5\textwidth]{Human_Model_3D3MS.pdf}
    \caption{The designed wireless IMU sensor attached on the wrist of the participant}
    \label{fig: IMU_setup}
\end{figure}
\begin{figure}[t!]
    \centering
    \includegraphics[width=0.65\textwidth]{SafeBlockDiagrams.pdf}
    \caption{The Predictive Safety Model (PSM) block diagram \cite{tafrishi2022psm}.}
    \label{fig: PSMblock}
\end{figure}


%\subsection{Parameter Determination}

%The system parameters optimization was fundamentally led by the predictive safety model technique created by \cite{tafrishi2022psm}. This research relies on their fundamental approach and mathematical principles, hence certain parameters of the system were retained from their validated model. Nevertheless, substantial modifications were required to shift from upper body to wrist-mounted applications. The physical parameters necessitated comprehensive recalibration, with the arm mass decreased to 2 kg and the pendulum length modified to 0.2 m to precisely reflect wrist dynamics, in contrast to the greater values employed in upper body modeling. Furthermore, new weight coefficients (P* = 0.6, D* = 0.4) were implemented for risk assessment calculations, enhancing the original model's functionality to provide real-time safety evaluation.

%The optimization of these altered parameters entailed systematic assessment and enhancement via iterative testing. The established base stiffness vector of $[500, 500, 1200]^T$ N/m and the damping vector of $[40, 40, 60]^T$ Ns/m from the original model were retained, and their efficacy was confirmed for wrist motion patterns. The elevated stiffness values along the z-axis were essential for preserving rotational stability during tool manipulation tasks, whereas the damping coefficients effectively regulated motion without excessively constraining natural movement patterns.


 

 
\subsection{Data Collection and Probability Distribution Design}

There is a BNO055 IMU sensor attached to operator's arm, which communicates with the Arduino board. The IMU is tuned for the highest precision motion capture, where 160 Hz is the sampling rate using AMG mode. The designated data acquisition frequency of 40 Hz and IMU sample rate of 9 Hz were excellent for recording wrist movements, ensuring adequate temporal resolution and computational efficiency. The system utilizes a low-pass filter's cutoff frequency of 220 Hz with order 20, and a Kaiser window parameter of 3 to efficiently eliminate high-frequency noise while maintaining critical wrist motion attributes, especially accelerometer data.  The captured data are linear acceleration $\uvec{a}_m$, angular velocity of the arm $\dot{\bm{\theta}}_m$ with respect to the elbow joint, and estimated angular rotation $\bm{\theta}_m$ of the wrist using a Kalman filter.

A systemic data collection approach was implemented across all experiments. For each task (fastening using a tool, visual inspection, and pick-and-place), as shown in Fig. \ref{fig: experiment_snapshots}, six distinct recordings were captured: two recordings at low speed for controlled motions, two at medium speed for normal operations, and two high-speed recordings simulating unsafe operations. 
 \begin{figure}[t!]
    \centering
    \includegraphics[width=3 in, height=2.2 in]{ExperimentsMI.pdf}
    \caption{Selected experiments run by the participant}
    \label{fig: experiment_snapshots}
\end{figure}
 
Subsequent to the creation of the extensive motion dataset, a probability distribution framework was instituted to delineate and examine motion patterns, as illustrated in Fig. \ref{fig: Probability_plot}. This three-dimensional image illustrates the correlation between angular location ($\theta_g$) which is the gravitational reference angle, angular velocity ($\omega$), and their corresponding probability densities, offering essential insights into motion safety attributes.

\begin{figure}[t!]
    \centering
    \includegraphics[width=3.1 in, height= 1.8 in]{ProbPlot.pdf}
    \caption{The dataset of probability distribution documenting the participant’s motions}
    \label{fig: Probability_plot}
\end{figure}

The probability distribution integrates all experimental motion recordings to visualize safe and hazardous regions. It highlights safe and hazardous motion areas, reflecting characteristics of human wrist movements in manufacturing. Peaks in the distribution correspond to common, safe motion states, while low-density areas, often at extreme angles or velocities, indicate potential hazards. The smooth transition between these zones offers valuable insights into the demarcation of safe and unsafe motion modes. This probabilistic mapping acts as a crucial reference for the safety assessment system, facilitating real-time comparison of observed movements with established safe operation patterns. 

Building on a probabilistic foundation, the next step involved developing a real-time system to assess errors from the safest motion during operator movement. We calculate the mean RMS error, $E_{m,q}$, in the frequency domain via the Fourier transform \cite{tafrishi2022psm} as follows:
\begin{eqnarray}
E_{m,q} &=& \frac{1}{\lambda} \int_{\lambda_m}^{\lambda} E_s(\Omega) \, df \nonumber\\
&=& \frac{1}{\lambda} \int_{\lambda_m}^{\lambda} \left( \int_{-\infty}^{\infty} e_q(t) e^{-j\Omega t} \, dt \right) df,
\label{Eq:thefrequencytransformfourier}
\end{eqnarray}
where $\lambda$ is the frequency range, $\lambda_m$ the minimum frequency, and $e_q = (1/n^2) \parallel \bm{\theta} - \bm{\theta}_m \parallel$ with $t \in [t_0, t_1]$. The configuration $q \in \{\theta, \omega\}$ represents either angular orientation or velocity.


\subsection{Safety Assessment Criteria}
The safety assessment criteria were formulated to evaluate real-time safety based on the predictions of the PSM approach. Unlike prior strategies using frequency-domain errors, the primary safety evaluation employs an error-based impedance $e$, inspired by Bednarczyk et al. \cite{bednarczyk2020model}:  
\begin{equation}
e(t)=P^*\frac{e_{\theta}(t)}{||e_{\theta}||}+ D^* \frac{e_{\dot{\theta}}(t)}{||e_{\dot{\theta}}||} 
\label{Eq:impedence}
\end{equation}  
where $e_{\theta}=|\bm\theta_m-\bm\theta|$ and $e_{\dot{\theta}}=|\dot{\bm\theta}_m-\dot{\bm\theta}|$ are the angular position and angular velocity errors between the operator configuration and the values estimated by PSM, and $P^*$ and $D^*$ represent the position error gain and velocity error gain, respectively.  

In general, we designed the error impedance $e$ to combine position and velocity errors using weighted coefficients. We analyze its behavior under two cases:  
\begin{equation}
\begin{cases}
& \textnormal{Position priority} : P^* = 0.6,\; D^* = 0.4\\
& \textnormal{Velocity priority}:  P^* = 0.4,\; D^* = 0.6
   \end{cases}
   \label{Eq:prioritysafetyimpe}
\end{equation}  

This dual-priority analysis, presented in Eq. (\ref{Eq:prioritysafetyimpe}), enables a comprehensive safety assessment by examining both position- and velocity-focused error metrics. For system performance evaluation, this integrated approach combines positional precision and velocity regulation, contrasting it with the previously utilized frequency-domain safety analysis. The benefits and overall performance will be further discussed in the next section.


\section{Results and Discussion}\label{sec:motion study}
In this section, we will study the results of the proposed impedance-based safety evaluation with previously utilized approaches e.g., frequency domain.

For preparation of the PSM, the wrist-mounted mechanism operates with key physical parameters: an arm mass of $2 \, \mathrm{kg}$, a pendulum length of $0.2 \, \mathrm{m}$, and a base radius of $0.25 \, \mathrm{m}$. Stiffness and damping parameters use base vectors of $[500, 500, 1200]^\top \,$ $\mathrm{N/m}$ and $[40, $$40, 60]^\top \,$ $ \mathrm{Ns/m}$, respectively, to simulate dynamic response. We assume that the elbow to shoulder should have minimum angular motions.  Three scenarios were created to encompass both safe and possibly dangerous movement patterns associated with various manual procedures frequently observed in industrial environments, as shown previously, in Fig. \ref{fig: experiment_snapshots}.
 




%------ Edited by Chat 

%\subsection{Safe and Unsafe Motion}
The first experiment is the fastening using tool (screwdriver) experiment where the participant mimics the participant tightening a screw, which requires precise angular rotations and regulated continuous motions. Tool alignment and controlled torque application are necessary for the participant. This experiment is common in production and maintenance tasks. The tool fastening experiment demonstrated clear distinctions between safe and unsafe operating movements by analyzing angular orientations and velocities as an example. The safe motion experiments observed in Fig. \ref{fig: fastening_normal} revealed well-regulated and foreseeable patterns across all assessed parameters. It also shows that our tuned PSM with new model can detect unsafe motions when it is necessary.  For example, uncontrolled motions presented in Fig. \ref{fig: fastening_unsafe} have very different illustrations which clearly underline the risky operations. The angular positions along the z-axis highlight the unsafe motion patterns which is flagged due to the fluctuations, sudden spikes and uneven changes, particularly within 6 to 10 seconds of the task operation as a result of wrong fastening procedures.   
\begin{figure}[t!]
    \centering
    \includegraphics[width=2.3 in,height=1.6 in]{FM12.pdf}
    \includegraphics[width=2.3 in,height=1.6 in]{FM13.pdf}
    \caption{Fastening using tool experiment with normal motions of participant}
    \label{fig: fastening_normal}
\end{figure}
\begin{figure}[t!]
    \centering
    \includegraphics[width=2.3 in,height=1.6 in]{FH12.pdf}
    \includegraphics[width=2.3 in,height=1.6 in]{FH13.pdf}
    \caption{Fastening using tool experiment with unsafe and unstable motions of participant}
    \label{fig: fastening_unsafe}
\end{figure}
%\begin{Fig.}[t!]
%    \centering
 %   \includegraphics[width=2.3 in,height=1.6 in]{Fastening 5.pdf}
 %   \includegraphics[width=2.3 in,height=1.6 in]{Fastening 4.pdf}
 %   \caption{Error analysis for the angular orientation and velocity for fastening using tool experiment}
%    \label{fig: error_analysis}
%\end{Fig.}



\begin{figure}[t!]
    \centering
    a)\includegraphics[width=2.3 in,height=1.2 in]{Fastening_3.pdf}
    \includegraphics[width=2.3 in,height=1.2 in]{Fastening_2.pdf}
    b)  \includegraphics[width=2.3 in,height=1.2 in]{Visual_Inspection_3.pdf}
    \includegraphics[width=2.3 in,height=1.2 in]{Visual_Inspection_2.pdf}
    c)    \includegraphics[width=2.3 in,height=1.2 in]{Picking_and_Placing_3.pdf}
    \includegraphics[width=2.3 in,height=1.2 in]{Picking_and_Placing_2.pdf}
    \caption{RMS error and the Fourier transform for the angular orientation and velocities for the (a) fastening using tool, (b) visual inspection, and (c) picking and placing experiments}
    \label{fig: rms_fastening}
    \label{fig: rms_visual_inspection}
    \label{fig: rms_pickplace}
\end{figure}


The error analysis of the fastening tool experiment offers profound insights into the quantitative distinctions between safe and risky activities via time-domain and frequency-domain analyses. Looking at the time-domain analysis, Fig. \ref{fig: rms_fastening} provides a detailed comparison of angular position and angular velocity errors (across all axes), highlighting clear distinctions between safe and harmful movements. For fastening operation as shown in Fig. \ref{fig: rms_fastening}a, the safe operation exhibits notably consistent error margins, especially along the x and y axes, as seen previously in Fig. \ref{fig: fastening_normal}, where deviations are confined to $\pm $1 radian, while the unsafe error margins show greater fluctuations. The z-axis exhibits marginally greater although still regulated error variations (Fig. \ref{fig: fastening_normal}), indicating the inherent variability in tool manipulation movements. The unsafe position and unsafe velocity errors still show a higher disparity compared to position and velocity errors in safe motion, as expected.

 

%The findings indicate that a wrist-mounted IMU system, in conjunction with the established safety model, may proficiently differentiate between safe and dangerous fastening activities. The system's capacity to record both direction and velocity data enables thorough monitoring, while the safety model establishes a dependable standard for typical operations, allowing for the identification of potentially dangerous motions. This method demonstrates significant promise for application in industrial safety systems within HRC, especially in situations necessitating accurate tool manipulation and regulated movements.

%\subsection{Error Analysis}


The Fourier-based frequency domain analysis within Fig. \ref{fig: rms_fastening}a, also revealed distinct differences between safe and unsafe motions. Position error analysis showed unsafe motions exhibited significantly higher error magnitudes across a broader frequency range (peaks reaching 3 radians vs. 0.5 radians for safe motions), particularly concentrated in 200-400 Hz—indicative of erratic tool handling. In contrast, safe motions displayed error distributions at lower frequencies, reflecting smoother control. $|E_{\theta}|$ and $|E_{\omega}|$ graphs confirmed unsafe errors surpassing safe ones by up to five times, setting clear safety thresholds. Sampling 10-15 data points showed frequency-based errors clearly distinguished unsafe motions but complicated real-time analysis, as both angular orientation and velocity are crucial.  Next, we conducted experiments on the other two cases: visual inspection, and pick-and-place, presented in Fig. \ref{fig: rms_visual_inspection}b and Fig. \ref{fig: rms_pickplace}c, respectively. In visual inspection, participants systematically scanned an object with regulated orientations and consistent angles, mimicking industrial practices. The pick-and-place experiment simulated conveyor belt operations with continuous item flow. Both cases exhibited clear distinct effects, reinforcing the distinction between safe and unsafe motions. As clear from frequent responses, we have unsafe motions, but it needs a blending between position error and velocity error to have dynamic risk-assessment in real-time. 


 

\begin{figure}[t!]
    \centering
    \includegraphics[width=0.45\textwidth]{Fastening_1.pdf}
    \includegraphics[width=0.45\textwidth]{Visual_Inspection_1.pdf}\\
    (a) \hspace{4 cm} (b) \\
    \includegraphics[width=0.45\textwidth]{Picking_and_Placing_1.pdf} \\
    (c) 
    \caption{Risk assessment by impedance-based error for the (a) fastening using tool, (b) visual inspection, and (c) picking and placing experiment. Note that position error stands for priority for position error with 60\% value and velocity error similarly means priority for velocity error with 60\% value.}  
    \label{fig: Risk_assessments}
\end{figure}


 The risk assessment analysis, demonstrated in Fig. \ref{fig: Risk_assessments} employs impedance-based measures to quantify operational safety across tasks, addressing the limitations of prior frequency-domain methods \cite{tafrishi2022psm}. Unlike spectral analysis, which required tracking 200–400 Hz frequencies at 40 Hz sampling, the impedance approach minimizes computational delay by directly integrating time-domain position and velocity errors. This combined analysis pinpoints exact instances of unsafe motion by leveraging both angular position and velocity deviations.  For instance, safe operations maintain aggregated errors around 0.1, with occasional peaks near 0.15 (normalized error $<$8\%), whereas hazardous motions exhibit distinct spikes exceeding 0.3, especially during abrupt movements and improper tool handling. These thresholds eliminate abstract frequency-domain interpretations, aligning with ISO standards by prioritizing positional precision and velocity stability for real-time applicability. Observations indicate velocity-based error prioritization (i.e. velocity error with 60\% and position error with 40\%) is more effective, as normal motions remain more rectified and IMU sensors are more sensitive to angular velocity, exhibiting lower error bias than angular orientation. Consequently, deviations beyond 0.25 serve as reliable indicators for automated safety monitoring, potentially triggering warning signals. This impedance-based method balances position accuracy—ensuring precise tool orientation—while factoring in velocity for movement stability.  





\begin{table}[t!]
\centering
\resizebox{0.65\textwidth}{!}{
\begin{tabular}{l|c|c}
\hline
\textbf{Experiment} & \textbf{Mean error} & \textbf{Variance error} \\
\hline
\hline
Fastening using tool & 0.063175 & 0.002375 \\
\hline
Visual inspection & 0.04653333 & $1.37\times10^{-3}$ \\
\hline
Picking and placing & 0.087125 & $3.74\times10^{-3}$ \\
\hline
\end{tabular}
}
\caption{Average normalized error analysis for different manufacturing tasks}
\label{tab: errors}
\end{table}





To generalize our impedance-based error assessment using the PSM method, we evaluated error levels across 18 experiments for safe, normal and risky motions. The normalized error metrics in Table~\ref{tab: errors} confirm the system’s ability to predict motion safety under low, normal and risky operational intensities, with most errors staying below 8\% (e.g., 0.08 normalized error) using velocity-priority assessment with very small error variance around 0.002 on average. While this retrospective normalization approach enables robust offline safety evaluation, real-time implementation requires careful selection of normalization references.  

Future work could explore dynamic thresholds based on task-specific probability distributions or frequency-band scaling from historical data, though adapting to novel motion patterns remains a challenge. These results highlight the efficacy of the PSM model in correlating normalized errors with safe operational boundaries for routine manufacturing tasks using IMU-based arm monitoring and incorporating human-robot cooperation shows great promise for future work.


\section{Conclusion}\label{sec:conclusions}
This paper demonstrated the effectiveness of a wrist-mounted IMU safety monitoring system by integrating impedance-based error assessment and frequency-domain evaluation in human-robot collaboration. The system quantifies motion safety through dual analysis: impedance-based thresholds (0.1 for safe, 0.3 for hazards) enabling real-time assessment, and frequency analysis identifying unsafe signatures in the 200–400 Hz range.  

Experimental validation across tool manipulation, visual inspection, and pick-and-place tasks confirmed robustness, with normalized errors below 8\% in normal operations. The impedance-based method, prioritizing velocity (60\%) over position (40\%), improves computational efficiency while maintaining accuracy. Future work may explore adaptive threshold optimization via machine learning and dynamic normalization methods for real-time human-robot collaboration.  





%This study has demonstrated the effectiveness of a wrist-mounted safety monitoring system through adapting and enhancing the Predictive Safety Model (PSM) for manufacturing environments. Through comprehensive experimental validation, the modified PSM framework successfully differentiates between safe and dangerous movements, with impedance-based error measures around 0.1 indicating safe operations and values exceeding 0.3 signaling high-risk scenarios. The integration of frequency domain analysis revealed distinct signatures for unsafe motions through elevated frequency components and amplitude fluctuations, establishing quantitative criteria for real-time safety assessment.

%The system achieves robust performance across diverse manufacturing scenarios including visual inspection, tool manipulation, and pick-and-place tasks, while maintaining computational efficiency through optimized parameter selection and impedance-based evaluation. The effective amalgamation of dynamic safety monitoring with conventional manufacturing processes demonstrates the practical utility of the proposed system for enhancing workplace safety in human-robot collaborative environments. Future work could explore adaptive threshold optimization and integration with machine learning approaches for enhanced predictive capabilities across broader manufacturing applications, along with the development of dynamic normalization methods to improve real-time performance across varying operational conditions.



%This study has demonstrated the effectiveness of a wrist-mounted safety monitoring system through adapting and enhancing the Predictive Safety Model (PSM) for manufacturing environments. Through comprehensive experimental validation, the modified PSM framework successfully differentiates between safe and dangerous movements, with position error thresholds below 0.022 radians indicating safe operations and values exceeding 0.035 radians signaling high-risk scenarios. The integration of frequency domain analysis revealed distinct signatures for unsafe motions through elevated frequency components and amplitude fluctuations, establishing quantitative criteria for real-time safety assessment.

%The system achieves robust performance across diverse manufacturing scenarios including visual inspection, tool manipulation, and pick-and-place tasks, while maintaining computational efficiency through optimized parameter selection and impedance-based evaluation. The effective amalgamation of dynamic safety monitoring with conventional manufacturing processes demonstrates the practical utility of the proposed system for enhancing workplace safety in human-robot collaborative environments. Future work could explore adaptive threshold optimization and integration with machine learning approaches for enhanced predictive capabilities across broader manufacturing applications.



\bibliographystyle{splncs04}
\bibliography{references.bib}

%\begin{Fig.}[t!]
%    \centering
%    \includegraphics[width=2.3 in,height=1.6 in]{PM12.pdf}
%    \includegraphics[width=2.3 in,height=1.6 in]{PM13.pdf}
%    \caption{Picking and Placing Experiment with Normal Motions of Participant}
%    \label{fig: Modular Couple}
%\end{Fig.}

%\begin{Fig.}[t!]
 %   \centering
%    \includegraphics[width=2.3 in,height=1.6 in]{PH22.pdf}
 %   \includegraphics[width=2.3 in,height=1.6 in]{PH23.pdf}
%    \caption{Picking and Placing Experiment with Unsafe and Unstable Motions of Participant}
%    \label{fig: Modular Couple}
%\end{Fig.}

%\begin{Fig.}[t!]
 %   \centering
%    \includegraphics[width=2.3 in,height=1.6 in]{Picking and Placing 5.pdf}
%    \includegraphics[width=2.3 in,height=1.6 in]{Picking and Placing 4.pdf}
 %   \caption{Error analysis for the Angular Orientation and Velocity for the Picking and Placing Experiment}
 %   \label{fig: Modular Couple}
%\end{Fig.}



%\begin{Fig.}[t!]
%    \centering
%    \includegraphics[width=2.3 in,height=1.6 in]{VIM12.pdf}
%%    \caption{Visual inspection experiment with normal motions of participant}
%    \label{fig: Modular Couple}
%\end{Fig.}

%\begin{Fig.}[t!]
%    \centering
%    \includegraphics[width=2.3 in,height=1.6 in]{VIH22.pdf}
%    \includegraphics[width=2.3 in,height=1.6 in]{VIH23.pdf}
%    \caption{Visual inspection experiment with unsafe and unstable motions of participant}
%    \label{fig: Modular Couple}
%\end{Fig.}

%\begin{Fig.}[t!]
%    \centering
%    \includegraphics[width=2.3 in,height=1.6 in]{Visual %inspection 5.pdf}
%    \includegraphics[width=2.3 in,height=1.6 in]{Visual Inspection 4.pdf}
%    \caption{Error analysis for the velocity for visual inspection experiment}
%    \label{fig: Modular Couple}
%\end{Fig.}






\end{document}
