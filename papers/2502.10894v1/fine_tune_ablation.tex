\begin{figure*}[htbp]
  \centering
  % --- Subfigure 1: Distance Thrown ---
  \begin{subfigure}[b]{0.32\textwidth}
    \centering
    \begin{tikzpicture}
      \begin{axis}[
        ybar,
         bar width=0.4cm,
         bar shift=0.0cm,
        width=\textwidth,
        height=6cm,
        ymin=0,
        symbolic x coords={No-Fine-Tuning,No-Pre-Training,No-E2E,Ours},
        xtick=\empty,
        xticklabel=\empty,         % hide x-axis labels
        title={Distance Thrown (m) $\uparrow$},
        enlarge x limits=0.2,       % extra space left/right
        clip=false,                 % allow drawing outside the box
        grid=major,
      ]
        \addplot+[ybar, fill=Sage, draw=black, 
          error bars/.cd, y dir=both, y explicit,
          error bar style={draw=black, fill=black, line width=1pt}]
          coordinates {(No-Fine-Tuning,0.6) +- (0,0.2)};
        \addplot+[ybar, fill=DustyRose, draw=black, 
          error bars/.cd, y dir=both, y explicit,
          error bar style={draw=black, fill=black, line width=1pt}]
          coordinates {(No-Pre-Training,10.2) +- (0,0.7)};
        \addplot+[ybar, fill=SlateBlue, draw=black, 
          error bars/.cd, y dir=both, y explicit,
          error bar style={draw=black, fill=black, line width=1pt}]
          coordinates {(No-E2E,10.9) +- (0,0.6)};
        \addplot+[ybar, fill=WarmTaupe, draw=black, 
          error bars/.cd, y dir=both, y explicit,
          error bar style={draw=black, fill=black, line width=1pt}]
          coordinates {(Ours,15.1) +- (0,1.5)};
      \end{axis}
    \end{tikzpicture}
  \end{subfigure}
  \hfill
  % --- Subfigure 2: Throw Release Speed ---
  \begin{subfigure}[b]{0.32\textwidth}
    \centering
    \begin{tikzpicture}
      \begin{axis}[
        ybar,
         bar width=0.4cm,
         bar shift=0.0cm,
        width=\textwidth,
        height=6cm,
        ymin=0,
        symbolic x coords={No-Fine-Tuning,No-Pre-Training,No-E2E,Ours},
        xtick=\empty,
        xticklabel=\empty,
        title={Throw Release Speed (m/s) $\uparrow$},
        enlarge x limits=0.2,
        clip=false,
        grid=major,
      ]
        \addplot+[ybar, fill=Sage, draw=black, 
          error bars/.cd, y dir=both, y explicit,
          error bar style={draw=black, fill=black, line width=1pt}]
          coordinates {(No-Fine-Tuning,1.8) +- (0,0.3)};
        \addplot+[ybar, fill=DustyRose, draw=black, 
          error bars/.cd, y dir=both, y explicit,
          error bar style={draw=black, fill=black, line width=1pt}]
          coordinates {(No-Pre-Training,9.5) +- (0,0.4)};
        \addplot+[ybar, fill=SlateBlue, draw=black, 
          error bars/.cd, y dir=both, y explicit,
          error bar style={draw=black, fill=black, line width=1pt}]
          coordinates {(No-E2E,9.6) +- (0,0.3)};
        \addplot+[ybar, fill=WarmTaupe, draw=black, 
          error bars/.cd, y dir=both, y explicit,
          error bar style={draw=black, fill=black, line width=1pt}]
          coordinates {(Ours,11.7) +- (0,0.5)};
      \end{axis}
    \end{tikzpicture}
  \end{subfigure}
  \hfill
  % --- Subfigure 3: Peak Leg Power ---
  \begin{subfigure}[b]{0.32\textwidth}
    \centering
    \begin{tikzpicture}
      \begin{axis}[
        ybar,
         bar width=0.4cm,
         bar shift=0.0cm,
        width=\textwidth,
        height=6cm,
        ymin=0,
        symbolic x coords={No-Fine-Tuning,No-Pre-Training,No-E2E,Ours},
        xtick=\empty,
        xticklabel=\empty,
        title={Peak Leg Power (kW) $\downarrow$},
        enlarge x limits=0.2,
        clip=false,
        grid=major,
      ]
        \addplot+[ybar, fill=Sage, draw=black, 
          error bars/.cd, y dir=both, y explicit,
          error bar style={draw=black, fill=black, line width=1pt}]
          coordinates {(No-Fine-Tuning,9.278) +- (0,1.391)};
        \addplot+[ybar, fill=DustyRose, draw=black, 
          error bars/.cd, y dir=both, y explicit,
          error bar style={draw=black, fill=black, line width=1pt}]
          coordinates {(No-Pre-Training,16.022) +- (0,0.252)};
        \addplot+[ybar, fill=SlateBlue, draw=black, 
          error bars/.cd, y dir=both, y explicit,
          error bar style={draw=black, fill=black, line width=1pt}]
          coordinates {(No-E2E,15.043) +- (0,0.866)};
        \addplot+[ybar, fill=WarmTaupe, draw=black, 
          error bars/.cd, y dir=both, y explicit,
          error bar style={draw=black, fill=black, line width=1pt}]
          coordinates {(Ours,13.714) +- (0,0.794)};
      \end{axis}
    \end{tikzpicture}
  \end{subfigure}
  
  % --- Shared Legend ---
  \vspace{1ex}
  \begin{tikzpicture}
    \begin{axis}[
      hide axis,
      scale only axis,
      height=0pt,
      width=0pt,
      legend columns=4,
      legend style={
        draw=none,
        /tikz/every even column/.append style={column sep=1cm},
        legend image code/.code={
           \draw[fill=##1, draw=none] (0cm,-0.1cm) rectangle (0.3cm,0.1cm);
         },
      },
    ]
      \addplot+[ybar, fill=Sage] coordinates {(0,0)};
      \addlegendentry{\texttt{No-Fine-Tuning}}
      \addplot+[ybar, fill=DustyRose] coordinates {(0,0)};
      \addlegendentry{\texttt{No-Pre-Training}}
      \addplot+[ybar, fill=SlateBlue] coordinates {(0,0)};
      \addlegendentry{\texttt{No-E2E}}
      \addplot+[ybar, fill=WarmTaupe] coordinates {(0,0)};
      \addlegendentry{\texttt{Ours}}
    \end{axis}
    \node[anchor=north east, fill=white, draw=white, circle, minimum size=12pt, xshift=5pt, yshift=5pt] at (current bounding box.north east) {};
  \end{tikzpicture}
  
  \caption{\textbf{End-to-end fine-tuning from a pre-trained WBC leads to the best task performance.} Throwing evaluation metrics across $100$ simulated throws for four policies: Our fine-tuned WBC (\texttt{Ours}) achieves the longest throw distance with lower peak leg power as compared to a throwing policy trained from scratch (\texttt{No-Pre-Training}) or a high-level policy for a frozen WBC (\texttt{No-E2E}). The WBC before finetuning (\texttt{No-Fine-Tuning}) has the lowest peak leg power but throws the ball a much shorter distance.}
  \label{fig:fine-tune-ablation}
\end{figure*}