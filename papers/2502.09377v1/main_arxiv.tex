\documentclass{article}
\usepackage[ruled]{algorithm2e} % For algorithms
\usepackage[utf8]{inputenc}
\usepackage{mathtools}
\usepackage{comment}
\usepackage{amsmath}
% \usepackage{algorithm}
% \usepackage{algorithmic}
\usepackage{algpseudocode}
\usepackage{amsfonts,amssymb,amsthm,boxedminipage,color,url,fullpage}
\usepackage{amsfonts}
\usepackage{amssymb}
\usepackage{amsthm}
\usepackage{boxedminipage}
\usepackage{color}
\usepackage{url}
\usepackage{fullpage}
\usepackage{mathtools}
\usepackage[numbers]{natbib}

\usepackage{thmtools} 
\usepackage{thm-restate}

\usepackage{enumitem}
\usepackage{tcolorbox}
% \usepackage[usenames,dvipsnames]{xcolor}
\usepackage[colorlinks=true,citecolor=blue,linkcolor=blue,urlcolor=black]{hyperref}

% \usepackage[colorlinks=true,
%     linkcolor=black,
%     citecolor=black,
%     filecolor=black,
%     urlcolor=black]{hyperref} 
\usepackage[labelfont=bf]{caption}
\usepackage{aliascnt,cleveref}
\usepackage{authblk}
\usepackage{accents}
\usepackage{tikz}
\usetikzlibrary{positioning,arrows}
\usepackage{pgfplots}
\pgfplotsset{width=10cm,compat=1.9}
\usepgfplotslibrary{external}
\allowdisplaybreaks


\usepackage[]{color-edits}% suppress
\addauthor{YG}{purple}
\newcommand{\ygc}[1]{{\YGcomment{#1}}}
\newcommand{\yge}[1]{{\YGedit{#1}}}

\addauthor{MF}{red}
\newcommand{\mfc}[1]{{\MFcomment{#1}}}
\newcommand{\mfe}[1]{{\MFedit{#1}}}

\addauthor{AE}{blue}
\newcommand{\aec}[1]{{\AEcomment{#1}}}
\newcommand{\aee}[1]{{\AEedit{#1}}}

\addauthor{AF}{magenta}
\newcommand{\afc}[1]{{\AFcomment{#1}}}
\newcommand{\afe}[1]{{\AFedit{#1}}}


\addauthor{HA}{blue}
\newcommand{\hac}[1]{{\HAcomment{#1}}}
\newcommand{\hae}[1]{{\HAedit{#1}}}

\newcommand{\todo}[1]{\textcolor{red}{(TODO: #1)}}

\newtheorem{theorem}{Theorem}
\newtheorem{lemma}{Lemma}
\newtheorem{corollary}{Corollary}
\newtheorem{claim}{Claim}
\newtheorem{observation}{Observation}
\newtheorem{proposition}{Proposition}
\newtheorem{definition}{Definition}
\newtheorem{example}{Example}
\newtheorem{mainthm}{Main Theorem}

\crefname{prop}{property}{properties}

\newcommand{\eps}{\varepsilon}
\newcommand{\reals}{\mathbb{R}}
\newcommand{\F}{\mathcal{F}}
\newcommand{\A}{\mathcal{A}}
\newcommand{\E}{\mathbb{E}}
\newcommand{\I}{\mathcal{I}}
\newcommand{\B}{\mathcal{B}}
\newcommand{\M}{\mathcal{M}}
\newcommand{\C}{\mathcal{C}}
\newcommand{\X}{\mathcal{X}}
\newcommand{\V}{\mathcal{V}}
\newcommand{\GenInstance}{\I=(N,M,\{v_i\}_{i \in N})}
\newcommand{\matchnfill}{\mathsf{Match\mbox{-}n\mbox{-}Fill}}
\newcommand{\matchnfillchores}{\mathsf{Match\mbox{-}n\mbox{-}Fill\mbox{-}for\mbox{-}Chores}}
\newcommand{\bagfillcopy}{\mathsf{BagFill\mbox{-}with\mbox{-}Copies}}
\newcommand{\bagfillRR}{\mathsf{BagFill\mbox{-}RoundRobin}}
\newcommand{\bagfillremove}{\mathsf{BagFill\mbox{-}and\mbox{-}Remove}}
\DeclareMathOperator*{\argmax}{arg\,max}
\DeclareMathOperator*{\argmin}{arg\,min}
\DeclarePairedDelimiter{\floor}{\lfloor}{\rfloor}
\DeclarePairedDelimiter{\ceil}{\lceil}{\rceil}
\DeclarePairedDelimiter{\bigfloor}{\left\lfloor}{\right\rfloor}
\DeclarePairedDelimiter{\bigceil}{\left\lceil}{\right\rceil}
\newcommand{\triVec}[3]{
\begin{pmatrix}
    #1 \\
    #2 \\
    #3
\end{pmatrix}
}
\newcommand{\twoVec}[2]{
\begin{pmatrix}
    #1 \\
    #2
\end{pmatrix}
}
\newcommand{\tka}{$(t,k)$-allocation}


\title{Fair Division via Resource Augmentation\thanks{We thank Noga Alon for suggesting to apply the probabilistic method for the proof of \Cref{prop:upper_bound_t_monotone}. 
The work of A.\ Eden was supported by the Israel Science Foundation (grant No. 533/23). The work of M.\ Feldman and Y.\ Gal-Tzur has been partially funded by the European Research Council (ERC) under the European Union's Horizon 2020 research and innovation program (grant agreement No. 866132), by an Amazon Research Award, by the Israel Science Foundation Breakthrough Program (grant No.~2600/24), and by a grant from TAU Center for AI and Data Science (TAD).}

}

\author{
Hannaneh Akrami\thanks{Bonn University. Email: \texttt{hakrami@uni-bonn.de}}
\quad
Alon Eden\thanks{The Hebrew University; {\tt alon.eden@mail.huji.ac.il}. Incumbent of the Harry \& Abe Sherman Senior Lectureship at the School of Computer Science
and Engineering at the Hebrew University.} 
\quad 
Michal Feldman\thanks{Tel Aviv University, Israel. Email: \texttt{mfeldman@tauex.tau.ac.il}}
\quad 
Amos Fiat\thanks{Tel Aviv University, Israel. Email: \texttt{fiat@tau.ac.il}}
\quad 
Yoav Gal-Tzur\thanks{Tel Aviv University, Israel. Email: \texttt{yoavgaltzur@mail.tau.ac.il}}
}

\date{\today}


\begin{document}
\maketitle

\begin{abstract}
    We introduce and formalize the notion of resource augmentation for maximin share allocations --- an idea that can be traced back to the seminal work of Budish [JPE 2011]. 
    Specifically, given a fair division instance with $m$ goods and $n$ agents, we ask how many copies of the goods should be added in order to guarantee 
    that each agent receives at least their original maximin share, or an approximation thereof.
    We establish a tight bound of $m/e$ copies for arbitrary monotone valuations.
    For additive valuations, we show that at most $\min\{n-2,\floor*{\frac{m}{3}}(1+o(1))\}$ 
    copies suffice. For approximate-MMS in ordered instances, we give a tradeoff between the number of copies needed and the approximation guarantee. In particular, we prove that $\floor{n/2}$ copies suffice to guarantee a $6/7$-approximation to the original MMS, and $\floor{n/3}$ copies suffice for a $4/5$-approximation. Both results improve upon the best known approximation guarantees for additive valuations in the absence of copies.  
\end{abstract}

\newpage

% Title page for title and abstract only.
% \begin{titlepage}



% \setcounter{tocdepth}{1} % adjust to 1 if desired
% \tableofcontents

% \end{titlepage}


\section{Introduction}

Fair Division goes back to antiquity, (Genesis 13), where 2 agents (Abraham and Lot) used cut and choose to split a contested resource. For divisible items and 2 agents,
cut and choose ensures that (a) both agents get at least half the good, when viewed subjectively based on the agent's own valuation --- this is called a proportional allocation, and (b) no agent would prefer to swap with the other --- this is called an envy free allocation. See \cite{Brams_Taylor_95,Brams_Taylor_1996}. 
Extensions to 
% more than two agents 
$n> 2$ agents
were studied by Banach, Knaster and Steinhaus, \cite{steinhaus1948problem}, where proportionality means that every agent gets at least $1/n$ fraction of her value for the entire good.
In particular, they show that proportionality is always achievable, for any number of agents, in the context of divisible cake cutting. (In general, for more than 2 agents, proportionality does not imply envy freeness.)
Unfortunately, proportionality is not possible, in general, for indivisible items. 

The {\em maximin share} is a relaxation of proportionality for indivisible items. Given $n$ agents and a set $M$  of indivisible items, the maximin share of an agent is the maximum value (over all partitions of $M$ into $n$ bundles), of the minimum valued bundle in the partition.

The maximin share (hereinafter denoted MMS), was studied by \cite{Hill87,budish2011combinatorial}. %Hill \cite{Hill87} gave a lower bound on the maximin share. 
\citet{budish2011combinatorial} defined the term and showed that if the number of agents goes to infinity and there are many copies of every item (also going to infinity) then one can approximate the maximin share to within a small error (going to zero). This was particularly relevant in the context of registration to classes, his motivating example. 

Again, unfortunately, even if restricted to additive valuations, a maximin allocation need not exist \cite{Kurokawa18,feige2021tight}. 
Albeit, there are several special cases where achieving the maximin share is in fact achievable, see \cite{bouveret2016characterizing}. 

\paragraph{Relaxations of MMS}
Given that achieving the maximin share is impossible, several relaxations have been studied. Implicitly introduced in \cite{budish2011combinatorial}, a $1$-out-of-$d$ maximin share is the maximum value each agent can guarantee while partitioning the set $M$ into $d \geq n$ bundles (instead of $n$ bundles in the original MMS definition), and receiving a minimum valued bundle.

Clearly, the larger $d$ is, the easier it is to achieve the benchmark.  The question then becomes, what is the smallest value of $d\geq n$ such that there is an allocation for $n$ agents so that each agent achieves the $1$-out-of-$d$ maximin share. This is also referred to as the ordinal approximation. In \cite{Hosseini_22} the authors show that $1$-out-of-$\lceil 3 n/2 \rceil$ MMS allocations exist for any instance with additive valuations, later improved to $1$-out-of-$4\lceil n/3\rceil$ MMS allocations in \cite{akrami2023improving}, the current state of the art.

Another relaxation  of MMS, also called approximate-MMS, is that of $\alpha$-MMS, introduced by Procaccia and Wang \cite{Kurokawa18}.
An $\alpha$-MMS allocation is one where every agent gets at least an $\alpha$ fraction of the maximin share. 
% This notion was introduced by Procaccia and Wang 
Procaccia and Wang \cite{Kurokawa18} showed that $\alpha=2/3$ is attainable for additive valuations. This was subsequently improved in a sequence of papers, culminating in \cite{akrami2023breaking34barrierapproximate} which achieved a $(\frac{3}{4} + \frac{3}{3836})$-MMS allocation, the best known bound thus far (Akrami and Garg). 



\subsection{Our Contributions}
\label{sec:full-m-over-3}

Our main contribution is to introduce relaxations to the MMS by allowing resource augmentation, ({i.e.}, allowing extra copies of items) and to obtain both positive and negative results in different settings. This relaxation was implicitly suggested by Budish in \cite{budish2011combinatorial}, where the number of items (slots in classrooms) tended to infinity. We ask:

\paragraph{Main question:} How many items should be duplicated so as to guarantee an allocation in which every agent's value is at least the agent's (original) MMS?


We consider both the total number of copies and the  maximum number of copies of an item. We distinguish between {\sl distinct} copies (only one extra copy of an item allowed), and when more copies of a single item are allowed.  Much of this paper is focused on distinct copies ---- up to one extra copy of an item. %\mfc{I left this sentence, as this is still true.} 
Our results are the following:

\paragraph{General Monotone Valuations}
\ A tight bound ($\approx m/e$) on the number of copies needed to achieve MMS with copies, for arbitrary monotone valuations. More precisely we show that 
\begin{itemize}%[leftmargin=*]
\item For all $n$, $m$, there exist $n$ monotone valuations over $m$ items such that every MMS allocation with copies requires at least $\left(\frac{n-1}{n}\right)^n \cdot m$ copies. See Theorem \ref{thm:lowermontone}. The number of copies of an individual good can be as high as $n$ (in an instance where  $m$ is sufficiently large). 
\item  Given $n$ monotone valuations $\{v_i\}_{i \in [n]}$ and a set $[m]$ of goods, one can find an MMS allocation with copies, while making at most $\left(\frac{n-1}{n}\right)^n \cdot m$ extra copies. See Proposition \ref{prop:upper_bound_t_monotone}. Moreover, the number of copies of any individual good is no more than $O(\ln m/ \ln \ln m)$. See Proposition \ref{lem:bound_indivcopies}. Note that this does not conflict with the lower bound of $n$ above as the lower bound instance has $m=n^n$.
\item We also show a connection between two different MMS relaxations, in particular we show that if there exists a 1-out-of-$(1+\alpha)n$ MMS allocation then there is a full MMS allocation with $\floor{\alpha m}(1+o(1))$ distinct copies, see Lemma \ref{lem:1ood_reduction}. While this reduction is for general valuations --- we use it to obtain bounds for additive valuations.
\end{itemize}

\paragraph{Additive Valuations}
% We show the following:  
\begin{itemize}%[leftmargin=*]
% \item We show that one may assume, without loss of generality, that the valuations are {\sl ordered}, {\sl i.e.}, for all agents $i$, $v_i(1) \geq v_i(2) \geq \cdots \geq v_i(m)$. This was shown by Bouveret and Lemaître \cite{bouveret2016characterizing, barman2020approximation} in a setting with no copies, we prove that this is still valid in a setting with copies. See Proposition \ref{prop:reductionToOrdered}. 
%\footnote{If one could assume ordered instances without loss of generality for the EFX setting we would have solved the EFX problem using Plaut-Roughgarden.}
\item  We show that $$\min\{n-2,\floor{m/3}(1+o(1))\}$$ distinct copies suffice to guarantee a full MMS allocation, see \Cref{thm:upperboundadditive}.
% , Lemma \ref{lem:nmin2_lemma} and Lemma \ref{cor:mover3additive}.
These results are indeed tight for 3 additive agents (namely, a single copy of a single good suffice), but are {\sl not} known to be tight for 4 or more agents, raising a very perplexing open problem. As far as we know today, a single copy of a single good may suffice to achieve a full MMS allocation for any instance with additive valuations. 
\item We also consider MMS problem for chores, where the question becomes how many chores need be removed to achieve an MMS allocation? We show that there is an MMS allocation for chores where at least $m-n+2$ of the chores will be allocated to agents, ({\sl i.e.}, $n-2$ chores removed), see Appendix \ref{sec:full-n-2-chores}. 
\end{itemize}

\paragraph{\boldmath $\alpha$-MMS Approximations with Copies for Ordered Instances (Additive Valuations)}
We give new approximation guarantees for ordered instances, which improve upon the best-known approximations (without copies). Ordered instances, also called {\sl same ordered preferences} (SOP) in \cite{bouveret2016characterizing}, are such where all agents have additive valuations, and all agents have identical ranking of the items. We give points along a tradeoff between the guarantee of the $\alpha$-MMS approximation and the number of copies needed, we show how to achieve a $6/7$-MMS with $\floor{n/2}$ distinct copies (Theorem \ref{thm:nover2copiesMMS}) and how to achieve a $4/5$-MMS with $\floor{n/3}$ distinct copies (Theorem \ref{thm:nover3copiesMMS}).

% \afc{See open problems below}
%\paragraph{Beyond additive valuations and other results (?)}
%One can trivially obtain MMS allocations for unit demand bidders. We show that one can obtain MMS allocations for $k$-demand bidders requiring $\min(m,n-2)$ extra distinct copies, see \ref{}. \afc{We need so see what can actually be done, and maybe this gives these ``results" too much space.}


\subsection{Our Techniques}
Our upper bounds  for general monotone valuations use the probabilistic method. This is used for bounding the total number of copies as well as for bounding the maximal number of copies of a single item.
 
 We give new extended variants of existing techniques in the fair division literature to a setting with copies. Specifically:
\begin{itemize}%[leftmargin=*]
% \item We show how to reduce any instance with additive valuations into an ordered instance using a surprising/non-trivial connection to the max flow problem, see Proposition \ref{prop:reductionToOrdered}.
\item We introduce bag filling with copies. %\hac{We extend the bag filling approach to the setting with copies.} 
Our bag filling techniques give full MMS allocations %or good approximate MMS allocations 
while duplicating some of the items. See Algorithm \ref{alg:bagfill_copy}: $\bagfillcopy$ %\hac{This is only about exact MMS, right?}.
\item Valid reductions in general take an instance and reduce it to a new instance with less agents and items, while ensuring that the MMS of each remaining agent has not decreased. An $\alpha$-valid reduction is a valid reduction where the agents removed are allocated sets whose value is at least $\alpha$ times their MMS share, see Definition \ref{def:alphavalid}. We give new valid reductions and $\alpha$-valid reductions that use copies, see Section \ref{sec:reductions}.  
\item We combine bag filling and round robin approaches to achieve approximate MMS with copies for ordered instances. See Algorithm \ref{alg:RR}: $\bagfillRR$.
\end{itemize}

\subsection{Open Problems} We believe that one of our major contributions is to suggest many open problems, the penultimate open problem being ``does one extra copy of a good suffice to obtain an MMS allocation for additive valuations?''. If so, this would be an interesting analogy to the  
Bulow Klemperer result \cite{bulowklemperer96} where adding a single bidder suffices so that a simple auction (Vickrey) obtains as much revenue as the optimal auction (Myerson). Other open problems are:
\begin{itemize}%[leftmargin=*]
\item Of course, making progress on any of our bounds on additive valuations are obvious open problems, even if not showing that one extra copy suffices for MMS. 
\item Considering more general classes of valuations such as submodular valuations, gross substitutes, etc. An illustrative example is that of $k$-demand valuations where we show that $n-1$ distinct copies suffice to achieve an MMS allocation,
see Appendix \ref{sec:othervaluations}. We remark that the tools used to produce an MMS allocation with copies for $k$-demand valuations are a slight variant of those used for additive valuations. 
\item We reduce the problem of MMS with [{\sl distinct}] copies to the $1$-out-of-$d$ problem (Lemma \ref{lem:1ood_reduction}). 
We use this to derive bounds on the number of copies in the context of additive valuations, but any such $1$-out-of-$d$ result for arbitrary monotone valuations would similarly imply analogous results for general valuations. Are there such $1$-out-of-$d$ results? 
% Notably, any impossibility result for MMS with copies, translates to impossibility results for $1$-out-of-$d$ constructions.
\item Another natural direction is applying the resource augmentation paradigm to other fairness notions.
% \item
% In fact, the solution with copies we get by transforming the $1$-out-of-$d$ construction is {\sl separable} --- items can be labeled as original or copies and the agents can be partitioned into those who get original items and those who get copies. This allows a reduction in the other direction: a distinct separable MMS with copies solution implies a $1$-out-of-$d$ result (agents that get copies are ignored). But what about the more general case? 
% {\sl I.e.,} what about equivalence between the existence of distinct non-separable allocations with copies to $1$-out-of-$d$ relaxations? 
\end{itemize}

\subsection{Other Related Work}

% \mfc{Do we want to mention envy-based notions, like EF, EF1, EFX?}\afc{See 2nd paragraph below}

Fair division of indivisible goods is a vast area of research in recent years, a survey paper by Amanatidis et {\sl al}, ``Fair Division of Indivisible Goods: Recent Progress and Open Questions'', \cite{Amanatidis_2023}, includes 12 pages of references, and the authors stress that this is not an exhaustive survey and refer to yet other surveys (e.g., \cite{moulinsurvey} with a more economic perspective) that focus on specific aspects of the problem. 

Fair division notions can be roughly partitioned into share based \cite{BabaioffF22} and envy based \cite{Brams_Taylor_95}. In share based fair division one ignores what the others get, one is happy if one gets value above some threshold (that does not depend on others, only on their cardinality), proportionality and MMS are share based. In envy based fair division one decides if one is happy or not depending on the others' allocated bundles. The basic envy based fairness notion is that being envy free (one does not seek to exchange allocations with the other). For indivisible goods, envy freeness is impossible in general and thus relaxations were proposed. These include envy freeness up to one good (EF1) \cite{Lipton2004,budish2011combinatorial} and envy freeness up to any good (EFX) \cite{Caragiannis19}. 
% --- there is a good in the other's bundle that if removed one does not seek to swap with what remains. 
% Another relaxation is that of envy free up to any good (EFX) \cite{Caragiannis19}. 
% --- no matter what one removes from another bundle, one does not seek to swap with whatever is left over. 
EF1 allocations are known to exist, the question of when do EFX allocations exist is a notoriously hard open problem. The number of citations dealing with envy free, EF1 and EFX allocations is quite daunting, see \cite{Amanatidis_2023}. In this paper we deal with share based, and more specifically MMS fairness.


Besides MMS and $\alpha$-MMS, other relaxations/approximations to proportionality include proportionality up to one good (PROP1), Round Robin Share, Pessimistic proportional share \cite{Conitzerprop1}, proportional up to any good (PROPx) \cite{aziz2020polynomial} which need not exist, and  proportionality up to the maximin item (PROPm) \cite{baklanov2021achievingproportionalitymaximinitem} which are known to exist for 5 agents. Allocations that are both PROP1 and Pareto efficient exist \cite{Conitzerprop1}. 

Prior to Budish \cite{budish2011combinatorial}, Hill also considered the maximin share \cite{Hill87} and gave lower bounds on the maximin share as a function of the item values. 

There was a long series of papers improving upon results for $\alpha$-MMS, complexity improvements and improvements in the quality of the result \cite{procaccia2014fair,Amanatidis_2017,Kurokawa18,garg2019approximating,barman2020approximation}, $\alpha=3/4$ was achieved in 
 \cite{ghodsi2022fair,garg2021improved,akrami2023simplification} and finally \cite{akrami2023breaking34barrierapproximate} improved upon $3/4$ by a small constant. 

A formal definition of shares and of feasible shares appears in \cite{BabaioffF22,BabaioffEF24}, MMS itself is not feasible as it may not exist, even for additive valuations. 
% Feasible shares give a lower bound on the value that an agent can expect to get, irrespective of the valuations of others.
A recent share-based notion is quantile-share \cite{BabichenkoFHN24}, which is guaranteed to exist for additive valuations and, more generally, for monotone valuations (then called {\sl universally feasible}), assuming the Erd\"os Matching Conjecture holds.

There is a long history of considering resource augmentation in various forms in computer science and economics, see chapter on resource augmentation 
 in ``Beyond the Worst-Case Analysis of Algorithms" \cite{Roughgarden_2021}. Resource augmentation appears in Bulow Klemperer \cite{bulowklemperer96} where simple auctions with one extra bidder guarantee revenue at least as large as that of complex auctions without the extra bidder. This was further studied in \cite{HartlineR09} and since then has been used in many contexts such as auction design, for both revenue and welfare, prophet inequalities, etc. It also goes by the name of competition complexity, see \cite{EdenFFTW17a,RoughgardenTY20,babaioff2019welfaremaximization,BrustleCDV24,
Brustle0DEFV24,derakhshan2024settlingcompetitioncomplexity}.


\section{Model and Preliminaries}

Following standard usage, define $[k]:=\{1,2, \ldots, k\}$ for integer $k\geq 1$. 

An instance of a fair division problem is given by a tuple $\GenInstance$, where $M$ is a set of $m$ indivisible goods, and $N$ is a set of $n$ agents. 
Every agent $i$ has a valuation function $v_i : 2^M \to \reals_{\geq 0}$, associating a non-negative value to every set $S \subseteq M$ of goods, denoted by $v_i(S)$. 
As standard, we assume that $v_i$ is monotone, i.e., for any $T\subseteq S \subseteq M$, $v_i(T) \le v_i(S)$, and normalized, so that $v_i(\emptyset)=0$. 

A valuation function $v_i$ is said to be {\em additive} if 
% there exist values $v_i(1), \ldots, v_i(m)$ such that, 
for every set of items $S\subseteq M$, $v_i(S) = \sum_{g \in S} v_i(\{g\})$.
We abuse the notation and write $v_i(g)$ or $v_{ig}$ instead of $v_i(\{g\})$ for $g \in M$. 
% A valuation function $v_i$ is said to be {\em additive} if for every set of items $S\subseteq M$, $v_i(S) = \sum_{g \in S} v_i(g)$.


% We mostly focus on additive valuations, where agent $i$ value for a set of goods is the sum of individual values, i.e. for any $S\subseteq M$, $v_i(S) = \sum_{g \in S} v_i(g)$, where $v_i(g)$ is a shorthand for $v_i(\{g\})$. We denote an instance of a fair division problem by $\GenInstance$.



An allocation of the set $M$ is a tuple $(A_1,\dots,A_n)$, where $A_i$ denotes the bundle allocated to agent $i$, such that $\bigcup_{i \in [n]} A_i = M$ (i.e., all items are allocated), and $A_i \cap A_j = \emptyset$ for any $i \ne j$ (i.e., no item is allocated to more than one agent).
We denote the collection of all allocations of the set $M$ among $n$ agents by $\A_n(M)$.

The \emph{maximin share} (MMS) of agent $i$ \cite{budish2011combinatorial} is the value that agent $i$ can guarantee by splitting the set $M$ into $n$ bundles, getting the worst out of them. Formally:

\begin{definition}[Maximin share (MMS)]
For an instance $\GenInstance$ with $n$ agents, the \emph{maximin share} (MMS) of agent $i$ with valuation function $v_i$, denoted by $\mu_i^n (M)$, is given by 
$$
\mu_i^n (M) = \max_{A \in \A_n(M)}\min_{S \in A} v_i(S).
$$
When convenient, we sometimes abuse notation, and refer to this value as $\mu_i(\I)$. 
\end{definition}

As valuations can be scaled, we may assume, without loss of generality that the MMS value for every agent is one. For additive valuations, this implies that every agent values the grand bundle (all items) at $n$ or more. 

Any partition that realizes the MMS value of agent $i$, is said to be an MMS partition of agent $i$. That is, a partition $P=(P_1, \ldots, P_n)$ of the goods into $n$ bundles is an MMS partition of agent $i$, if and only if $\mu_i^n (M) = \min_{j} v_i(P_j)$.

We next present the notion of an approximate MMS allocation.

\begin{definition}[$\alpha$-MMS]
    Given a fair division instance $\GenInstance$ and $\alpha>0$, an allocation $A$ is $\alpha$-MMS if and only if for all agents $i \in N$, $v_i(A_i) \geq \alpha \cdot \mu_i^n (M)$.  
\end{definition}

\paragraph{Valid reductions.}
% \mfc{Note: I slightly changed the text here.}
% In what follows, we present a few {\em valid reductions}, specifically cases where we can remove a set of agents and a set of items in such a way that the MMS value of the remaining agents does not decrease.
Valid reductions allow us to remove a subset of agents and items while ensuring that the MMS values of the remaining agents do not decrease.
The following lemma is due to \cite{amanatidis2018comparing,bouveret2016characterizing}.

\begin{lemma}[\cite{amanatidis2018comparing,bouveret2016characterizing}]\label{lem:MMSmonotone}
    For any agent $i \in N$, any monotone, non-negative, valuation functions $v_i:2^M \to \reals_{\geq 0}$ and any good $g \in M$, 
    $
    \mu_i^n(M) \le \mu_i^{n-1}(M \setminus \{g\}).
    $
\end{lemma}

By applying \Cref{lem:MMSmonotone} repeatedly, we conclude that for any $k < \min\{m,n\}$, the MMS of agent $i$ may only increase if we remove an arbitrary set of $k$ goods and an arbitrary set of $k$ agents. This is shown in the following corollary.

\begin{corollary}\label{cor:MMSmonotoneK}
    For any agent $i \in N$, any $k < \min\{m,n\}$, and any $S \subseteq M$ s.t. $|S|=k$, it holds that
    $$
    \mu_i^n(M) \le \mu_i^{n-k}(M \setminus S).
    $$
\end{corollary}

We remark that a similar notion of $\alpha$-valid reductions is useful when pursuing approximate MMS allocations. This is defined and used in Section \ref{sec:approx}.


%\mfc{Consider moving this. Check where we are using it.}
%For any agent $i$, her \emph{best round-robin (BRR)} value is defined to be the value that agent $i$ can guarantee when picking %first in a Round Robin with $n$ agents and a set of goods $M$. We denote this value by $\rho_i^n(M)$.


%\mfc{I removed here a definition of an allocation, since we define it above.}
% \begin{definition}[Allocation]
%     An allocation $A = (A_1,\dots,A_n)$ is a partition of the set of goods $M$ into $n$ bundles. That is, the allocation $A$ satisfies:
%     \begin{enumerate}
%         \item \label[prop]{pr:allocComplete} $\bigcup_{i \in N} A_i = M$.
%         \item \label[prop]{pr:allocUniq} For every two $i,j \in N$, $A_i \cap A_j = \emptyset$.
%     \end{enumerate}
% \end{definition}


\paragraph{MMS with Copies.} An allocation with copies is an allocation where goods may be duplicated, with different copies allocated to different agents. An agent may receive at most one copy of each good. An allocation with copies is a relaxation of an allocation where the two or more bundles may have a non-empty intersection.

\begin{definition}[$(t,k)$-allocation]
    Given a fair division instance $\GenInstance$, a $(t,k)$-allocation $A = (A_1,\dots,A_n)$ is a collection of sets of items, such that
    \begin{enumerate}%[leftmargin=*]
        \item $A_i \subseteq M$ for every $i \in N$.
        % $\bigcup_{i \in N} A_i = M$
        \item The total number of \emph{extra} copies is at most $t$, i.e., $\sum_{i\in N} |A_i| \le |M|+t$.
        \item The number of \emph{extra} copies of any good $g \in M$ is at most $k$, i.e., for every $g \in M$, $|\{i \in N \mid g \in A_i\}| \le k+1$.
    \end{enumerate}
\end{definition}
Note that, the definition above implies that no agent is allowed to receive more than a single copy of each good. 
% \yge{In particular, this implies that we only consider \tka s with $k \le n-1$.}

We also consider valid reductions with copies which appear in Section \ref{sec:reductions}.


\section{MMS with Copies: General Valuations}\label{sec:gneneralVals}

In this section we provide (tight) bounds on the number of copies needed for full MMS under general valuations. Section~\ref{subsec:total-general} establishes bounds on the total number of copies, while Section~\ref{subsec:general-per-good} gives bounds on the number of copies per good.

\subsection{\boldmath A Tight Bound of $m/e$ on the Total Number of Copies}
\label{subsec:total-general}

In this section we give a tight bound on the total number of copies required to guarantee original MMS value to all agents under monotone valuations.

Our lower bound is based on a combinatorial construction which considers $m=n^n$ goods positioned on the $n$-dimensional cube, so each good may be identified with $n$ coordinates, $(d_1,\dots,d_n)$.
The valuations are such that each of the agent is almost ``single-minded'', with agent $i$ only interested in bundles that contain all $n^{n-1}$ goods that share the same value in the $d_i$ coordinate. Those $n$ different bundles (one for each possible value of $d_i$), make up his MMS partition.
A simple counting argument shows that the only way to satisfy all agents simultaneously is to copy $(n-1)^n$ goods, which is a $1/e$-fraction of the total number of goods, as $n$ tends to infinity. 

We complement this result by showing, using the probabilistic method, that in every instance there exists an allocation which satisfies all agents, and copies at most $\left(1-\frac{1}{n}\right)^n$-fraction of the goods.



\begin{theorem} \label{thm:lowermontone}
    For any $n$, there exists an instance with $n$ monotone valuations such that the total number of copies needed is $\left(\frac{n-1}{n}\right)^n$-fraction of the goods, and there is at least one good which must be copied $n-1$ times.
\end{theorem}
\begin{proof}
    
Consider $n$ agents and $n^n$ goods positioned on the $n$-dimensional cube. I.e., the set of goods is 
$$
M = \{g_{d_1,\dots,d_n} \mid d_1, \dots, d_n \in [n]\}.
$$

Each agent $i \in [n]$ is interested in one of $n$ disjoint bundles, $P_i^1, \dots, P_i^n$ of $n^{n-1}$ goods each, where
$P_i^j = \{g_{d_1,\dots,d_n} \mid d_i = j \}$.
Formally, agent $i$'s valuation is monotone and for every $S \subseteq M$ it is defined to be:
$$
v_i(S) = \begin{cases}
    1 & \exists j \in [n] \text{ s.t. } P_i^j \subseteq S \\
    0 & \text{otherwise}
\end{cases}
$$
Agent $i$'s MMS partition is $(P_i^1, \dots, P_i^n)$ and her MMS value is 1.

Observe that for any $k$ agents, $i_1, \dots, i_k \in [n]$ and for any $k$ indexes $j_1, \dots, j_k \in [n]$, it holds that
$$
|P_{i_1}^{j_1} \cap P_{i_2}^{j_2} \cap \dots \cap P_{i_k}^{j_k}|
=
|\{g_{d_1,\dots,d_n} \mid \forall l \in [k] \;\; d_{i_l} = j_l \}|
=
n^{n-k}.
$$


If we wish to find an allocation with copies $A = (A_1,\dots,A_n)$ such that for every $i$, it holds that $v_i(A_i) \ge 1$, while minimizing the number of items we copy, then it is without loss to assume that $A_i = P_i^{j_i}$ for some $j_i \in [n]$.
Thus, the total number of goods (copies included) is $\sum_{i \in [n]} |A_i| = \sum_{i \in [n]} |P_i^{j_i}| = n \cdot n^{n-1} = n^n$.
The number of different goods used is, using the inclusion-exclusion principle:
\begin{eqnarray*}
    \left| \bigcup_{i\in [n]} P_i^{j_i} \right|
    &=&
    \sum_{k=1}^n (-1)^{k+1} \binom{n}{k} n^{n-k} 
    =
    \binom{n}{1}n^{n-1} - \left(\sum_{k=2}^n (-1)^{k} \binom{n}{k} n^{n-k}\right) \\
    &=&
    n^n - \left(n^n - n^n + \sum_{k=2}^n (-1)^{k} \binom{n}{k} n^{n-k}\right) 
    =
    n^n - \left(\sum_{k=0}^n (-1)^{k} \binom{n}{k} n^{n-k}\right) \\
    &=&
    n^n - (n-1)^n,
\end{eqnarray*}
which implies that the total number of copies is $(n-1)^n$.
Additionally, in any such allocation there exists a good which is allocated $n$ times, and thus duplicated $n-1$ times, as  $g_{j_1,j_2,\dots,j_n} \in \bigcap_{l=1}^n P_{i_l}^{j_l}$, and the claim follows.
\end{proof}

Below we prove a matching upper bound on the number of total copies required. We show that there always exists an allocation which assigns to every agent $i$ one of his MMS bundles and makes no more than $\left(1-\frac{1}{n}\right)^n \cdot m$ additional copies.

\begin{proposition}\label{prop:upper_bound_t_monotone}
For any set $N$ of $n$ agents with monotone valuations $\{v_i\}_{i \in N}$, and any set $M$ of $m$ goods, one can find an MMS allocation with copies, while making at most $\left(\frac{n-1}{n}\right)^n \cdot m$ extra copies.
\end{proposition}

\begin{proof}
Fix $n$ MMS partitions $\{(P_i^1,\dots,P_i^n)\}_{i \in N}$.
Let $A = (A_1,\dots,A_n)$ be a random allocation, where $A_i \sim U(\{P_i^1,\dots,P_i^n\})$, independently for any agent $i$. That is, agent $i$ gets one of her MMS bundles chosen uniformly at random.

Observe that for every agent $i$ and good $g$, $\Pr[g \in A_i] = \frac{1}{n}$, and so the expected number of \textit{unique} items in allocation $A$ is:
\begin{eqnarray*}
\E_{A} \left[|\cup_{i \in [n]}A_i|\right] 
&=& 
\sum_{g \in [m]} \Pr[g \in \cup_{i\in [n]} A_i] 
=
\sum_{g \in [m]} 1-\Pr[g \notin \cup_{i\in [n]} A_i] 
=
\sum_{g \in [m]} 1-\left(1-\frac{1}{n}\right)^n \\
&=&
m \cdot \left(1-\left(1-\frac{1}{n}\right)^n\right),
\end{eqnarray*}
where the third inequality follows from independence.

The expected total number of items (including copies) is
$$
\E_{A} \left[\sum_{i=1}^n |A_i| \right] 
=
\sum_{i=1}^n \E_{A} \left[ |A_i| \right] 
=
n \cdot \frac{m}{n}
=
m.
$$

By linearity of expectation, the expected number of copies made for an allocation $A$ is
$$
\E_{A} \left[\sum_{i=1}^n |A_i| - |\cup_{i \in [n]}A_i|\right] 
=
m - m \cdot \left(1-\left(1-\frac{1}{n}\right)^n\right)
=
m\left(1-\frac{1}{n}\right)^n,
$$
so there must exists an allocation that assigns each agent one of her MMS bundles and requires no more than $m\left(1-\frac{1}{n}\right)^n$ extra copies.
\end{proof}


\subsection{Bound on the Maximum Number of Copies of Any Good}\label{subsec:general-per-good}



In this section we upper bound the number of copies of any individual good, and together with the bound established in \Cref{subsec:total-general}, show to achieve both guarantees with high probability.

\begin{theorem}\label{thm:alg_for_monotone}
    For any instance $\GenInstance$ with monotone valuations, there exists a $(t,k)$-allocation that satisfies all agents and satisfies $t \le m/e$ and $k \le \min\left\{n-1,\frac{3\ln m}{\ln \ln m}\right\}$.
    Moreover, given an MMS partition for each agent, for any $\beta \in \mathbb{N}$, there exists an efficient algorithm that returns such an allocation with probability $1-e^{-\beta}$.
\end{theorem}

In the proof of \Cref{thm:alg_for_monotone}, we use the following lemma. Its proof, which relies on a balls-and-bins argument, is deferred to \Cref{apx:proofsGeneralVals}.

\begin{lemma}
 \label{lem:bound_indivcopies}
    For any fair division instance $\GenInstance$, with monotone valuations $\{v_i\}_{i \in N}$, there exists an MMS allocation where every good is copied at most $O\left(\frac{\ln m}{\ln \ln m}\right)$ times.
\end{lemma}
\begin{proof}[Proof sketch]
    Consider an allocation where agent $i$'s bundle, is randomly chosen as one of the sets in $i$'s MMS partition.
 We view this process as throwing throwing $n$ balls (agents) into $n$ bins (the index in an MMS partition). 
 Fix some good $g \in M$, it belongs in to some MMS set for all the agents, possibly different indices, but we can rename the sets and assume without loss of generality that $g$ is always in the set of index 1 in the MMS partitions of all agents.  
 The event in which $g$ is copied at least $k$ times for the allocation $A=(A_1,\dots,A_n)$, is exactly the event in which the bin corresponding to index $1$ has at least $k$ balls. 
 For $k=\frac{3\ln m}{\ln \ln m}$, the probability of having a bin with more than $k$ balls is at most $\frac{1}{m}$.
\end{proof}

\begin{proof}[Proof of \Cref{thm:alg_for_monotone}]
    Fix an instance $\GenInstance$. For any agent $i$, let $P^i= (P^i_1,\dots,P^i_n)$ be his MMS partition.
    Consider the algorithm $\A$ which does the following (at most) $3m\beta$ times:
    \begin{enumerate}
        \item Samples an allocation according to $A_i \sim U(P^i)$.
        \item Returns the allocation $A$ only if the number of total copies is $t \le m/e$ and every item is copied at most $k \le \frac{3\ln m}{\ln \ln m}$.
    \end{enumerate}
    First, by the union bound, the probability that $\A$ fails in some iteration is 
    $$
    \Pr[\A \text{ fails in a single iteration}] 
    \le \Pr[t > m/e] + \Pr\left[k > \frac{3\ln m}{\ln \ln m}\right]
    \le 
    \Pr[t > m/e] + \frac{1}{m},
    $$
    where the last inequality follows from the proof of \Cref{lem:bound_indivcopies}. 

    Recall that in the proof of \Cref{prop:upper_bound_t_monotone}, we have established that when the allocation $A$ is picked uniformly at random, as in the above procedure, the expected number of total copies is $m/e$. This allows us to bound the second term using Markov's inequality,
    $$
    \Pr[t > m/e] 
    \le 
    \Pr\left[t \ge \left(1+\frac{e}{m}\right)\frac{m}{e}\right]
    \le \frac{1}{1+\frac{e}{m}} \le \frac{m}{m+2}.
    $$
    Putting everything together we have,
    \begin{align*}
        \Pr[\A \text{ fails in a single iteration}]
        &\le
        \frac{m}{m+2} + \frac{1}{m}
        = \frac{m^2+m+2}{m^2+2m} \le \frac{m+3/2}{m+2}\le  1- \frac{1}{3m}.
    \end{align*}
    In particular, there is a strictly positive probability that $\A$ succeeds in some iteration, so there exists an MMS allocation with $t \le m/e$ and $k \le \frac{3\ln m}{\ln \ln m}$.
    The statement follows immediately, as
    \begin{align*}
        \Pr[\A \text{ fails in all iterations}]
        &=
        \left(\Pr[\A \text{ fails}]\right)^{3m\beta} 
        \le
        \left(1-\frac{1}{3m}\right)^{3m\beta} 
        = \frac{1}{e^\beta}. \qedhere
    \end{align*}
\end{proof}

% \input{reduction_to_ordered}

\section{MMS with Copies: Additive Valuations}\label{sec:fulladditive}

In this section, we prove the following theorem.
\begin{theorem} \label{thm:upperboundadditive}
    For any instance with additive valuations, at most $\min\{n-2,\floor{m/3}(1+o(1))\}$ distinct copies suffice to guarantee a full MMS allocation.
\end{theorem}

The proof of Theorem~\ref{thm:upperboundadditive} is divided into two parts. In Section~\ref{sec:full-n-2}, we give an algorithm that obtains full MMS with $n-2$ distinct copies. In Section~\ref{sec:1ood_reduction} we relate the $1$-out-of-$d$ bounds to our problem (a relation that holds for arbitrary monotone valuations), and use it to obtain full MMS with $\floor{m/3}(1+o(1))$ distinct copies for additive valuations.

% In order to prove the theorem, we present two algorithms which obtain full MMS with $n-2$ and $\floor{m/3}(1+o(1))$ distinct copies respectively.

\subsection{\boldmath Full MMS with $n-2$ Copies}
\label{sec:full-n-2}

Algorithm~$\matchnfill$ operates as follows. It inspects the MMS Partition $P$ of an arbitrary agent, tries to satisfy as many agents as possible (without harming other agents) using $P$ without any copies. Then it uses a modified bag-filling algorithm, $\bagfillcopy$, that uses copies in order to satisfy the remaining agents. Recall that we scale valuations so that $\mu_i^n(M)=1$ for every agent $i$, which also implies $v_i(M)\ge n$.

% \begin{algorithm}[htb]
% \caption{$\matchnfill(\GenInstance)$:
% \\ \textbf{Input:} $n$ Additive valuations $\{v_i\}_{i\in N}$ over items in $M$.
% \\ \textbf{Output:} Allocation $A = (A_1, \ldots, A_n)$ with at most $n-2$ distinct copies.
% }
% \label{alg:matchnfill}
% \begin{algorithmic}[1]
% \While{ $\exists i\in N, j\in M$ s.t. $v_{ij}\ge 1$} %\CommentSty{\color{gray} \textbackslash\textbackslash Allocate large items.}\;
%     \State $A_i\coloneqq \{j\}$
%     \State $N\coloneqq N\setminus\{i\}$, $M\coloneqq M\setminus \{j\}$.
% \EndWhile

% \State Let $X=(X_1,X_2,\ldots, X_n)$ be the MMS partition of agent 1. 
% \State Consider the bipartite graph $G=(V=N\times X,E)$, where $$E=\{(i,X_j)\ : \ v_i(X_j)\ge \mu_i\}.$$ 
% \If{there's a perfect matching between $N$ and $X$ in $G$} 
%     \State Allocate each $i\in N$ its match in the perfect matching.
% \Else
%     \State Let $X'\subset X$ be the minimal set of bundles violating Hall's theorem ($|X'|>1$ since $1$ is matched to every node in $X$).
%     \State Consider some $X_j\in X'$ and let $\hat{X}=X'\setminus\{X_j\}$.
%     \State Find a matching between bundles in  $\hat{X}$ and agents in $N$. If $i\in N$ is matched to some $X_j\in \hat{X}$, set $A_i=X_j$.
%     \State Let $\tilde{N}$ be the set of unallocated agents and $\tilde{M}=M\setminus\left(\bigcup_{i\in N\setminus\tilde{N}}A_i\right)$ be the set of unallocated items.
%     \State Allocate agents in $\tilde{N}$ using procedure $\bagfillcopy((\tilde{N},\tilde{M},\{v_i\}_{i\in \tilde{N}}))$. 
% \EndIf
% \end{algorithmic}
% \end{algorithm}

\begin{algorithm}[htb]
\caption{$\matchnfill(\GenInstance)$: 
\\ \textbf{Input:} $n$ additive valuations $\{v_i\}_{i\in N}$ over items in $M$. 
\\ \textbf{Output:} Allocation $A = (A_1, \ldots, A_n)$ with at most $n-2$ distinct copies.}
\label{alg:matchnfill}
\SetAlgoLined
\DontPrintSemicolon
\LinesNumbered

\While{$\exists i\in N, j\in M$ s.t. $v_{ij}\ge 1$}{ 
    $A_i\gets \{j\}$\,
    $N\gets N\setminus\{i\}$, $M\gets M\setminus \{j\}$\;
}

Let $P=(P_1,P_2,\ldots, P_n)$ be the MMS partition of agent 1\; 

Consider the bipartite graph $G=(V=N\times P,E)$, where $E=\{(i,P_j)\ : \ v_i(P_j)\ge 1\}$\;

\If{there is a perfect matching between $N$ and $P$ in $G$}{ 
    Allocate each $i\in N$ its match in the perfect matching\;
}
\Else{
    Let $P'\subset P$ be the minimal set of bundles violating Hall's theorem ($|P'|>1$ since $1$ is matched to every node in $P$)\;
    Consider some $P_j\in P'$ and let $\hat{P}=P'\setminus\{P_j\}$\;
    Find a matching between bundles in $\hat{P}$ and agents in $N$\;
    \ForEach{$i\in N$ matched to some $P_j\in \hat{P}$}{
        $A_i\gets P_j$\;
    }
    Let $\tilde{N}$ be the set of unallocated agents and $\tilde{M}=M\setminus\left(\bigcup_{i\in N\setminus\tilde{N}}A_i\right)$ the set of unallocated items\;
    Allocate agents in $\tilde{N}$ using procedure $\bagfillcopy((\tilde{N},\tilde{M},\{v_i\}_{i\in \tilde{N}}))$\;
}
\end{algorithm}


% \begin{algorithm}[htb]
% \caption{$\bagfillcopy(\GenInstance)$:
% \\ \textbf{Input:} $n$ Additive valuations $\{v_i\}_{i\in N}$ over items in $M$.
% \\ \textbf{Output:} Allocation $A = (A_1, \ldots, A_n)$ with at most $n-1$ distinct copies.
% }
% \label{alg:bagfill_copy}
% \begin{algorithmic}[1]
% \State $t\coloneqq 0$, $B_t\coloneqq \emptyset$.
% \State $N_0\coloneqq N$, $M_0\coloneqq M$.
% \For{items $j\coloneqq 1,\ldots, m$}
%     \If{$\exists i_t\in N_t$ such that $v_{i_t}(B_t\cup \{j\})\ge 1$:}    
%         \State $A_{i_t}\coloneqq B_t\cup \{j\}$ .
%         \State $t\coloneqq t+1$, $B_t\coloneqq \{j\}$.
%         \State $N_t\coloneqq N_{t-1}\setminus\{i_{t-1}\}$, $M_t\coloneqq M_{t-1}\setminus B_{t-1}$.
%     \Else
%         \State $B_t\coloneqq B_t\cup \{j\}$.
%     \EndIf
% \EndFor
% \end{algorithmic}
% \end{algorithm}


\begin{algorithm}[htb]
\caption{$\bagfillcopy(\GenInstance)$: 
\\ \textbf{Input:} $n$ additive valuations $\{v_i\}_{i\in N}$ over items in $M$. 
\\ \textbf{Output:} Allocation $A = (A_1, \ldots, A_n)$ with at most $n-1$ distinct copies.}
\label{alg:bagfill_copy}
\SetAlgoLined
% \DontPrintSemicolon
\LinesNumbered
$t\gets 0$, $B_t\gets \emptyset$, $N_0\gets N$, $M_0\gets M$\;

\For{items $j\gets 1$ \KwTo $m$}{
    \If{$\exists i_t\in N_t$ such that $v_{i_t}(B_t\cup \{j\})\ge 1$}{
        $A_{i_t}\gets B_t\cup \{j\}$\;
        $t\gets t+1$, $B_t\gets \{j\}$\;
        $N_t\gets N_{t-1}\setminus\{i_{t-1}\}$, $M_t\gets M_{t-1}\setminus B_{t-1}$\;
    }
    \Else{ {$B_t\gets B_t\cup \{j\}$} }
}
\end{algorithm}



We first show that $\bagfillcopy$ satisfies all agents, assuming each agent values the remaining items high enough (proof in \Cref{sec:fulladditive_proofs}). An example of a run of $\bagfillcopy$ appears in \Cref{sec:examples}.

\begin{lemma} \label{lem:bagfillcopy}
    Assuming that for every $i\in N$, $v_i(M)\ge |N|$ and that for every item $j$, $v_{ij}<1$, $\bagfillcopy$ outputs an allocation such that $v_i(A_i)\ge 1$ for every $i\in N$, and creates at most $|N|-1$ distinct copies.
\end{lemma}
\begin{proof}[Proof sketch.]
    A simple induction shows that every time we allocate a bundle, the value of every remaining agent $i$ for all remaining items is at least the number of remaining agents times $i$'s MMS value. Indeed, $i$'s value for all items except the last item inserted to the bag is at most $1$, otherwise, $i$ could have been allocated in a previous iteration. Since we duplicate the last inserted item, we get that the number of agents decreased by 1, while the total value of $i$ for all remaining items decreased by at most 1. Thus, we make sure that at any point, we have enough value to satisfy all remaining agents.
\end{proof}
\iffalse
\begin{proof}[Proof of \Cref{lem:bagfillcopy}]
%Clearly, $\bagfillcopy$ copies  $|N|-1$ different items, as 
Every time an agent is allocated, except the last agent, we duplicate the last item that was put in his bag and put the duplicate as the first item in the next bag. Since every item's value is less than $1$ for all agents, every allocated bundle contains at least two items. Therefore, we duplicate a different item each time, and in total, $|N|-1$ items. We now show that upon termination, every agent gets at least their MMS value.

We show that for every $t$, for every $i\in N_t$, $v_i(M_t)\ge |N|-t$.  We prove by induction on $t$. Obviously, it's true for $t=0$. Assume this is true for $t-1\in [|N|-1]$, and consider the iteration when agent $i_{t-1}$ gets assigned bundle $A_{i_{t-1}}$. Consider some agent $i\in N_t$. By the inductive assumption, we have that for every $i'\in N_{t-1}$, $v_{i'}(M_{t-1})\ge |N|-t+1.$ As $N_t\subset N_{t-1}$, we have that 
\begin{eqnarray}
    v_i(M_{t-1})\ge |N|-t+1. \label{eq:assumption}    
\end{eqnarray}   
    
Now consider the bundle $B_{t-1}$ just before some item $j$ was added to it, causing agent $i_{t-1}$'s value to rise above 1. Since before adding $j$, the value each agent in $N_{t-1}$ assigned to $B_{t-1}$ was smaller than 1, we also have 
\begin{eqnarray}
    v_i(B_{t-1})< 1. \label{eq:step}    
\end{eqnarray}

This implies that
\begin{eqnarray*}
    v_i(M_{t})\ =\ v_i(M_{t-1}\setminus B_{t-1})\ =\ v_i(M_{t-1})- v_i(B_{t-1}) > |N|-t,
\end{eqnarray*}
where the last inequality follows Equations~\eqref{eq:assumption},~\eqref{eq:step}.
\end{proof}
\fi
We now show that $n-2$ distinct copies suffice.

\begin{lemma} \label{lem:nmin2_lemma}
    For any fair division instance with additive valuations, at most $n-2$ distinct copies suffice to guarantee a full MMS allocation.
\end{lemma}
\begin{proof}
    First, by definition, every agent that is assigned some bundle for which they have an edge in $G$ gets their MMS value. Moreover, since agent $1$ has an edge to all bundles in $P$, then at least one agent gets satisfied, and $|\tilde{N}|\le |N|-1$.
    It remains to show that sets $\tilde{N}$ and $\tilde{M}$ satisfy the conditions of Lemma~\ref{lem:bagfillcopy}.

    $\matchnfill$ first allocates items with value more than the MMS of an agent, so for every $i\in \tilde{N}, j\in \tilde{M}$, $v_{ij}\le 1$. Moreover, by Lemma~\ref{lem:MMSmonotone}, allocating a single item to a single agent is a valid reduction. Thus, after this step, we have that for every $i\in N$, $v_i(M)\ge |N|$. We show that after allocating bundles $\hat{P}$ to agents in $N$, for every unallocated agent $i\in \tilde{N}$, $v_i(\tilde{M})\ge |\tilde{N}|$.

    Recall that $P'$ is a minimal set of bundles violating Hall's Theorem, and let $|P'|=k$. Let $N'=\Gamma(P')$  be their neighbors in $G$. Note that $|N'|=k-1$, otherwise, one can remove an arbitrary element of $P'$ and get a smaller set violating Hall's Theorem. Moreover, since $\hat{P}$ does not violate Hall's theorem, we know that $\Gamma(\hat{P})=N'$ and that the agents in $N'$ are exactly the agents matched by bundles from $\hat{P}$. Thus, it must be that there are no edges between agents in  $\tilde{N}= N\setminus N'$ and bundles in $\hat{P}$.  Therefore, for every $i\in \tilde{N}$, $v_i(\bigcup_{S\in \hat{P}} S)\le k-1$, which implies $$v_i(\tilde{M}) = v_i(M)-v_i\left(\bigcup_{S\in \hat{P}} S\right)\ge |N|-(k-1)=|N|-|N'|=|\tilde{N}|,$$
    as desired. It follows that all the conditions required in order to apply Lemma~\ref{lem:bagfillcopy} are fulfilled, and we can satisfy agents in $\tilde{N}$ using $|\tilde{N}|-1\le|N|-2\le n-2$ distinct copies (recall that at least one agent was already allocated before). Since $\bagfillcopy$ is the only place where we produce copies, this concludes the proof.
\end{proof}


%Notice that the above lemma  and \cite{feige2021tight} imply a tight bound on the number of copies for $n=3$ additive agents.
Together with the example in~\cite{feige2021tight} of an instance with $n=3$ additive agents that does not admit an MMS allocation, Lemma~\ref{lem:nmin2_lemma} implies a tight bound on the number of copies for $3$ additive agents. 

\begin{corollary}
    For every instance with 3 additive valuations, a single copy of a single item suffices to guarantee a full MMS allocation. This is tight, i.e., there are instances where one copy is required.
\end{corollary}



\paragraph{Remark.} One can consider a similar question in the context of chore allocation, and ask ``what is the maximum number of chores that can be assigned to agents such that no agent's cost is larger than their maximin share''; that is, how many chores are needed to be set aside in order to produce a full MMS allocation. It turns out that an analogous process to $\matchnfill$ in the context of chores outputs an allocation where at most $n-2$ items are disposed of, and each agent is allocated a bundle of cost no larger than their MMS value. More on this in Appendix~\ref{sec:full-n-2-chores}.

\subsection{\boldmath Full MMS with $\floor{m/3}(1+o(1))$ Copies} \label{sec:1ood_reduction}

We now get an improved bound when $m$ is not too large. This is done via a reduction from 1-out-of-$(1+\alpha)n$ bounds.

 
%\afc{The following lemma is for general valuations, and probably should not be here. The corollary is for additive valuations. }



\begin{lemma} \label{lem:1ood_reduction}
    If there exists a 1-out-of-$(1+\alpha)n$ MMS allocation for every fair division instance $\GenInstance$ for some constant $\alpha\in (0,1)$ then $$\floor{\alpha m} + \ceil*{\frac{(1+\alpha)^2}{n-1-\alpha}m} = \floor{\alpha m}(1+o(1))$$ %\alpha m(1+o(1))$ 
    distinct copies suffice to guarantee a full MMS allocation. This holds for arbitrary monotone valuations.  
\end{lemma}
\begin{proof}
    Let $\GenInstance$ be a fair division instance for which there exists a 1-out-of-$(1+\alpha)n$ MMS. Notice that if we take a set of agents $N'$ of size $n'$ for which $(1+\alpha)n'\le n$ then we can use the guarantee stated in the lemma to produce an allocation $\{A_i\}_{i\in N'}$ without copies such that for each agent $i\in N'$, $$v_i(A_i)\ge \mu_i^{\ceil{(1+\alpha)n'}}(M)\ge \mu_i^n(M).$$
    Consider an arbitrary set $N'$ of $n'=\floor{n/(1+\alpha)}$ agents and the allocation that satisfies their MMS value, $\{A_i\}_{i\in N'}$. Consider the set $\tilde{N}=N\setminus N'$. We proceed as follows. 
    
    Consider a set 
    $$S'\in \argmin_{S\subseteq N'\ :\  |S|=|\tilde{N}|} |\cup_{i\in S} A_i|.$$
    Allocate each agent $i\in S'$ the set $A_i$. Now consider the agents in $N\setminus S'$. This is a set of size $\ceil{n/(1+\alpha)},$ and therefore there exists an allocation $\{B_i\}_{i\in N\setminus S'}$ of the set $M$ of items without copies which guarantees each agent in   $N\setminus S'$ their MMS value. Allocate each agent $i\in N\setminus S'$ the set $B_i$. Notice that this allocation satisfies the MMS of all agents, and the only copies made are the items in $\cup_{i\in S'} A_i$. We next bound the number of items in this set.

    First, notice that 
    %$$|S'|=|\tilde{N}|=n-\floor{n/(1+\alpha)}=\left\lceil{\frac{\alpha }{1+\alpha}n}\right\rceil.$$
    $$|S'|=|\tilde{N}|=n-\floor{n/(1+\alpha)}=\ceil*{\frac{\alpha }{1+\alpha}n}.$$
    Since we chose the set of $|\tilde{N}|$ agents with the least number of items in  $N'$, it holds that
    \begin{eqnarray*}
        |\cup_{i\in S'} A_i|& \le & \floor*{\frac{|\tilde{N}|}{|N'|}m} =  \floor*{\frac{\ceil{\frac{\alpha }{1+\alpha}n}}{\floor*{\frac{1}{1+\alpha}n}}m}
        \le  \floor*{ \frac{\frac{\alpha }{1+\alpha}n+1}{\frac{1}{1+\alpha}n-1}m} \\
        & = & \floor*{\frac{\frac{\alpha }{1+\alpha}n+1}{\frac{1}{1+\alpha}n-1}m}  =  \floor*{\left(\alpha  + \frac{1+\alpha}{\frac{1}{1+\alpha}n-1}\right)m}\\
        & \le & \floor{\alpha m} + \ceil*{\frac{(1+\alpha)^2}{n-1-\alpha}m } \\
        & = & \floor{\alpha m}(1+O(1/n)) = \floor{\alpha m}(1+o(1)). 
    \end{eqnarray*}
    % (note: we do not assume valuations are additive).
%The following algorithm produces a $((d-1)m,1)$-AWC:
\end{proof}

We apply the above lemma to the following state-of-the-art bound due to \cite{akrami2023improving}.

\begin{lemma}[\citet{akrami2023improving}]
    For every fair division instance with additive valuations, there exist a 1-out-of-$4\ceil{n/3}$ MMS allocation.
\end{lemma}

Thus, we get the following.% (Proof in \Cref{sec:fulladditive_proofs}).
\begin{corollary} \label{cor:mover3additive}
    For any fair division instance with additive valuations, at most $\floor{\frac{m}{3}}(1+o(1))$ distinct copies suffice to guarantee a full MMS allocation.
\end{corollary}
\begin{proof}%[Proof of \Cref{cor:mover3additive}]
    Since $4\ceil{n/3}\le 4n/3+4 = n(1+ \frac{1}{3} + \frac{4}{n})$, applying Lemma~\ref{lem:1ood_reduction} with $\alpha=\frac{1}{3} + \frac{4}{n}$ implies that the number of copies sufficient in order to produce a full MMS allocation is at most
    \begin{eqnarray*}
        \floor*{\left(\frac{1}{3} + \frac{4}{n}\right)m} + \ceil*{\frac{(1+\frac{1}{3} + \frac{4}{n})^2}{n-1-\frac{1}{3} - \frac{4}{n}}m} &\le&\floor*{\frac{m}{3}} +\ceil*{\frac{4}{n}m}  + \ceil*{\frac{(1+\frac{1}{3} + \frac{4}{n})^2}{n-\frac{4}{3} - \frac{4}{n}}m}\\
        &=& \floor*{\frac{m}{3}}(1+O(1/n)) \\
        &=& \floor*{\frac{m}{3}}(1+o(1)). %\qedhere
    \end{eqnarray*}
\end{proof}



\begin{comment}

\begin{algorithm}[htb]
\caption{$\matchnfill(\GenInstance)$:
\\ \textbf{Input:} $n$ Additive valuations $\{v_i\}_{i\in N}$ over items in $M$.
\\ \textbf{Output:} Allocation $A = (A_1, \ldots, A_n)$ with at most $n-2$ distinct copies.
}
\label{alg:matchnfill}
\begin{algorithmic}[1]

\While{ $\exists i\in N, j\in M$ s.t. $v_{ij}\ge 1$} \CommentSty{\color{gray} \textbackslash\textbackslash Allocate large items.}\;
    \State $A_i\coloneqq \{j\}$
    \State $N\coloneqq N\setminus\{i\}$, $M\coloneqq M\setminus \{j\}$.
\EndWhile

\State Let $X=(X_1,X_2,\ldots, X_n)$ be the MMS partition of agent 1. 
\State Consider the bipartite graph $G=(V=N\times X,E)$, where $$E=\{(i,X_j)\ : \ v_i(X_j)\ge \mu_i\}.$$ 
\If{there's a perfect matching between $N$ and $X$ in $G$} 
    \State Allocate each $i\in N$ its match in the perfect matching.
\Else
    \State Let $X'\subset X$ be the minimal set of bundles violating Hall's theorem ($|X'|>1$ since $1$ is matched to every node in $X$).
    \State Consider some $X_j\in X'$ and let $\hat{X}=X'\setminus\{X_j\}$.
    \State Find a matching between bundles in  $\hat{X}$ and agents in $N$. If $i\in N$ is matched to some $X_j\in \hat{X}$, set $A_i=X_j$.
    \State Let $\tilde{N}$ be the set of unallocated agents and $\tilde{M}=M\setminus\left(\bigcup_{i\in N\setminus\tilde{N}}A_i\right)$ be the set of unallocated items.
    \State Allocate agents in $\tilde{N}$ using procedure $\bagfillcopy((\tilde{N},\tilde{M},\{v_i\}_{i\in \tilde{N}}))$. 
\EndIf
\end{algorithmic}
\end{algorithm}
    
\end{comment}
\begin{comment}
Let $\GenInstance$ be a fair division instance for which there exists a 1-out-of-$d$ MMS (note: we do not assume valuations are additive).
The following algorithm produces a $((d-1)m,1)$-AWC:
\begin{enumerate}
    \item Pick any set $N'$ of $\frac{n}{d}$ agents, there exists an allocation $(A_1,\dots,A_{|N'|})$ in which each of them gets his MMS.
    \item Let $S^* \subset N'$ be the $d-1$-fraction of agents in $N'$ which received the smallest bundles (in terms of cardinality). $|S^*|=\frac{(d-1)n}{d}$ and the set of goods allocated to $S^*$ is of size at most $\frac{m}{d-1}$.
    \item Observe that $|N \setminus S^*| = n - \frac{(d-1)n}{d} = \frac{n}{d}$, so there exists an allocation of all the goods which satisfies the agents of $N \setminus S^*$. The remaining agents of $S^*$ can be satisfied by using at most $\frac{m}{d}$ copies.
\end{enumerate}

\begin{corollary}
    $(m/3,1)$ copies are sufficient to guarantee MMS for additive valuations.
\end{corollary}
\begin{proof}
    Apply the above with 1-out-of-4/3.
\end{proof}
\end{comment}

\begin{comment}


The following variant of Bag Filling, thanks to Hannane, seems to give an $(n-1,1)$-MMS.

\begin{enumerate}
    \item $t\gets 0$, $B_t\gets \emptyset$.
    \item $N_0\gets N$, $M_0\gets M$.
    \item For items $j\gets 1,\ldots, m:$
    \begin{itemize}
        \item If $\exists i_t\in N_t$ such that $v_{i_t}(B_t\cup \{j\})\ge 1$:
        \begin{itemize}
            \item $X_{i_t}\gets B_t\cup \{j\}$ .
            \item $t\gets t+1$, $B_t\gets \{j\}$.
            \item $N_t\gets N_{t-1}\setminus\{i_{t-1}\}$, $M_t\gets M_{t-1}\setminus B_{t-1}$.
        \end{itemize}
        \item Else:
        \begin{itemize}
            \item $B_t\gets B_t\cup \{j\}$.
        \end{itemize}
    \end{itemize}
\end{enumerate}


Clearly, the above procedure copies  $n-1$ different items, as every time an agent is allocated (except the last agent), we duplicate the last item that was put in his bag and put the duplicate as the first item in the next bag.  

The following lemma implies that when the process ends, every agents gets at least their MMS value.

\begin{lemma}
    For every $t$, for every $i\in N_t$, $v_i(M_t)\ge n-t$.
\end{lemma}
\begin{proof}
    Obviously, it's true for $t=0$. Assume this is true for $t-1\in [n-1]$, and consider the moment when agent $i_{t-1}$ gets assigned bundle $X_{i_{t-1}}$.
    Consider some agent $i\in N_t$. By the inductive assumption, we have that for every $i'\in N_{t-1}$, $v_{i'}(M_{t-1})\ge n-t+1.$ As $N_t\subset N_{t-1}$, we have that 
    \begin{eqnarray}
        v_i(M_{t-1})\ge n-t+1. \label{eq:assumption}    
    \end{eqnarray}   
    
    Now consider the bundle $B_{t-1}$ just before some item $j$ was added to it, causing agent $i_{t-1}$'s value to rise above 1. Since before adding $j$, the value each agent in $N_{t-1}$ assigned to $B_{t-1}$ was smaller than 1, we also have 
    \begin{eqnarray}
        v_i(B_{t-1})< 1. \label{eq:step}    
    \end{eqnarray}

    Thus, we have
    \begin{eqnarray*}
        v_i(M_{t})\ =\ v_i(M_{t-1}\setminus B_{t-1})\ =\ v_i(M_{t-1})- v_i(B_{t-1}) > n-t,
    \end{eqnarray*}
    where the last inequality follows Equations~\eqref{eq:assumption},~\eqref{eq:step}.
    
\end{proof}
\subsection{$(n-2,1)$-copies with a preliminary matching stage}
Fix some agent $i^* \in N$, and some MMS allocation according to her valuations: $A = \{A_1,\dots,A_n\}$.
Let $G=(N \cup A, E)$ be a bipartite graph with the group of agents $N$ on one side and the bundles $A_1,\dots,A_n$ on the other. An edge $(i,A_j)$ exists if and only if $v_i(A_j) \ge 1$, i.e. the $j$th bundle satisfies agent $i$.
Observe that by definition the vertex $i^*$ is connected to all vertices on the other side of $G$.

\begin{observation}
    If there exists a perfect matching in $G$, then it induces an MMS allocation with no copies.
\end{observation}

For any set of bundles $S = \{A_{j_1},\dots,A_{j_k}\}$ let $\Gamma(\{A_{j_1},\dots,A_{j_k}\}) = \cup_{l=1}^k \{i \in N \mid (i,A_{j_l}) \in E \} \subseteq N$ be the set of neighbors of $S$.

\begin{lemma}\label{lem:TightHallSet}
    Let $S = \{A_{j_1},\dots,A_{j_k}\}$ be a set of $k$ bundles such that 
    \begin{itemize}
        \item Every subset of bundles $S' \subseteq S$ satisfies Hall's condition, i.e., $|\Gamma(S')| \ge |S'|$
        \item $S$ satisfies Hall's condition with equality. I.e., $|\Gamma(S)| = |S| = k$
    \end{itemize}
    Then there exists an assignment of $S=\{A_{j_1},\dots,A_{j_k}\}$ to the agents of $\Gamma(S) = \{i_1,\dots,i_k\}$ such that agent $i_l$ gets bundle $A_{j_l}$ and
    \begin{enumerate}
        \item For every $i_l \in \Gamma(S)$, $v_{i_l}(A_{j_l}) \ge 1$.
        \item For every agent $i \notin \Gamma(S)$, $v_i(M \setminus \{A_{j_1} \cup \dots \cup A_{j_k}\}) \ge n-k$.
    \end{enumerate} 
\end{lemma}
\begin{proof}
    Property 1 of the assignment follows directly from Hall's Marriage Theorem. 
    To see why property 2 holds, observe that for every $i \notin \Gamma(S)$ and every bundle $A_{j_l} \in S$, be definition $(i,A_{j_l}) \notin E$ and therefore $v_i(A_{j_l})<1$. It follows from additivity that
    $$
    v_i(M \setminus \{A_{j_1} \cup \dots \cup A_{j_k}\}) = v_i(M) - \sum_{l=1}^k v_i(A_{j_l}) \ge n-k
    $$
\end{proof}


\begin{proposition}
    $(n-2,1)$ copies suffice
\end{proposition}
\begin{proof}
    Consider some $G$ defined as above for some arbitrary agent $i^*$.
    If a perfect matching exists, there exists an MMS allocation and we are done.
    Otherwise, there exists some set of $S \subset A$ of bundles which violates Hall's condition, i.e., $|\Gamma(S)| < |S|$. Observe that since $i^*$ is connected to all bundles, $|S| \ge 2$.
    If we take $S$ be of minimal cardinality, any set $S' \subset S$ with $|S'| = |S| - 1$ satisfies the conditions of \cref{lem:TightHallSet}:
    Every subset of $S'$ satisfies Hall's condition and $|S'|$ satisfies it with equality, otherwise $S$ does not violate Hall's condition.

    We conclude that there always exists a set of size at least 1 which satisfies the conditions of \cref{lem:TightHallSet}. After applying \cref{lem:TightHallSet} to this subset, we are left with $n-k$ agents, each has a value of at least $n-k$ for the remaining goods. Applying Akrami's algorithm to this reduced problem together with the fact that $k \ge 1$ implies the statement.
\end{proof}





% \section{Improved reduction to 1-out-of-$d$ MMS and $(m/3,1)$}
Let $\GenInstance$ be a fair division instance for which there exists a 1-out-of-$d$ MMS (note: we do not assume valuations are additive).
The following algorithm produces a $((d-1)m,1)$-AWC:
\begin{enumerate}
    \item Pick any set $N'$ of $\frac{n}{d}$ agents, there exists an allocation $(A_1,\dots,A_{|N'|})$ in which each of them gets his MMS.
    \item Let $S^* \subset N'$ be the $d-1$-fraction of agents in $N'$ which received the smallest bundles (in terms of cardinality). $|S^*|=\frac{(d-1)n}{d}$ and the set of goods allocated to $S^*$ is of size at most $\frac{m}{d-1}$.
    \item Observe that $|N \setminus S^*| = n - \frac{(d-1)n}{d} = \frac{n}{d}$, so there exists an allocation of all the goods which satisfies the agents of $N \setminus S^*$. The remaining agents of $S^*$ can be satisfied by using at most $\frac{m}{d}$ copies.
\end{enumerate}

\begin{corollary}
    $(m/3,1)$ copies are sufficient to guarantee MMS for additive valuations.
\end{corollary}
\begin{proof}
    Apply the above with 1-out-of-4/3.
\end{proof}

\end{comment}



\section{Approximate-MMS with Copies: Ordered Instances}\label{sec:approx}
In this section we study approximate MMS allocations for ordered instances while allowing duplication of goods. An instance is ordered if there exists a permutation of the goods $(g_1, \ldots, g_m)$ such that $v_i(g_1) \geq \ldots \geq v_i(g_m)$ for all $i \in N$ and also $v_i$ is additive for all $i \in N$. 

First, in Section \ref{sec:reductions}, we review some of the known reduction rules and introduce new ones. Then, in Section \ref{sec:RR}, we present a simple algorithm $\bagfillRR(\I,\alpha)$ which combines the idea of bag-filling and round robin to achieve an $\alpha$-MMS allocation with copies (see Algorithm \ref{alg:RR}). This algorithm is the building block for achieving $6/7$-MMS and $4/5$-MMS with copies in the the sections that follow. In particular, in Sections \ref{sec:67} and \ref{sec:45}, we obtain $6/7$-MMS with $\floor{n/2}$ distinct copies and $4/5$-MMS with $\floor{n/3}$ distinct copies respectively by first reducing the initial instance using appropriate valid reductions, and then applying $\bagfillRR$. 

For the rest of this section, we only consider ordered instances.

% See Appendix \ref{apx:approx} for the missing proofs of this section. 

\subsection{Reduction Rules}\label{sec:reductions}
%In this section we give new $\alpha$-valid reductions that allow us to remove some agents and some items and yet still achieve an $\alpha$-MMS on the remaining agents and items. 

% \ygc{I replaced $[n]$ with $N$ and $[m]$ with $M$} 
\begin{definition} \label{def:alphavalid}
    Given an instance $\GenInstance$, an allocation rule that allocates bundles $A_1, \ldots, A_k$ to agents $1, \ldots, k$ respectively is $\alpha$-valid, if and only if the following holds:
    \begin{itemize}
        \item $v_i(A_i) \geq \alpha \cdot \mu_i(\I)$ for all $i \in [k]$, and
        \item $\mu^{n-k}_i(M \setminus (A_1 \cup \ldots \cup A_k)) \geq \mu^n_i(M)$ for all $i \in N \setminus [k]$.
    \end{itemize}
\end{definition}

The following lemma with small value of $k$ (namely $k \leq 3$) has been proven and used in \cite{garg2021improved,akrami2023simplification,akrami2023breaking34barrierapproximate}.
\begin{restatable}{lemma}{reducK}\label{lem:reduceK}
    Given an ordered fair division instance $\GenInstance$, a fixed agent $i \in N$, and integer $k < |M|/n$, 
    let $K = \{k(n-1)+1, k(n-1)+2, \ldots, nk, nk+1\}$. Then, $\mu^{n-1}_i(M \setminus K) \geq \mu^n_i(M).$
\end{restatable}
\begin{proof}
    Let $P=(P_1, P_2, \ldots, P_n)$ be an MMS partition of agent $i$ for the original instance. By the pigeonhole principle, there exists $j \in [n]$ such that $|P_j \cap [kn+1]| > k$. Without loss of generality, let us assume $|P_n \cap [kn+1]| > k$ and $g_1, g_2, \ldots g_{k+1}$ are $k+1$ distinct goods in $P_n \cap [kn+1]$ such that $g_1 \leq g_2 \leq \ldots \leq g_{k+1}$. For $j \in [k+1]$, let $kn-k+j \in P_{a_j}$. Now for all $j \in [k+1]$, swap the goods $kn-k+j$ and $g_j$; i.e., iteratively do the following:
    \begin{itemize}
        \item for all $j \in [k+1]$: $P_{a_j} \leftarrow P_{a_j} \setminus \{kn-k+j\} \cup \{g_j\}$ and $P_n \leftarrow P_n \setminus \{g_j\} \cup \{kn-k+j\}$.
    \end{itemize}
    Let $P'=(P'_1, P'_2, \ldots, P'_n)$ be the final partition. Since $g_1, \ldots, g_{k+1}$ are $k+1$ goods in $[kn+1]$ in increasing order of index, $g_j \leq kn-k+j$. Thus, $v_i(g_j) \geq v_i(kn-k+j)$ and  after each of these swaps, the value of $P_{a_j}$ cannot decrease. Therefore, for all $j \in [n-1]$, $v_i(P'_j) \geq v_i(P_j) \geq \mu^n_i(M)$ and $(P'_1, \ldots, P'_{n-1})$ is a partition of a subset of $M \setminus K$ into $n-1$ bundles of value at least $\mu^n_i(M)$ for $i$. The lemma follows.
\end{proof}

%\vspace{0.5cm}
\begin{tcolorbox}[%colback=blue!5,coltitle=blue!50!black,colframe=blue!25,
	title= \textbf{Rule $R^\alpha_k(\I)$}]
	%Input: allocation $\allocs$
	
	Preconditions:\label{pre1}
	\begin{itemize}
        \item $\I$ is ordered.
		\item For $K = \{k(n-1)+1, k(n-1)+2, \ldots, nk, nk+1\}$, there exists an agent $i$ such that $v_i(K) \geq \alpha \cdot \mu_i(\I)$.
	\end{itemize}
    Process:
    \begin{itemize}
        \item Allocate $K$ to agent $i$.
		\item Set $N' \leftarrow N \setminus \{i\}$ and $M' \leftarrow M \setminus K$.
	\end{itemize}
    Guarantees:
	\begin{itemize}
		\item $\I' = (N',M',\{v_i\}_{i \in N'})$ is ordered.
        \item For all $i \in N'$, $\mu_i(\I') \geq \mu_i(\I)$.
	\end{itemize}
\end{tcolorbox}
%\vspace{0.5cm}

\begin{corollary}[of Lemma \ref{lem:reduceK}]\label{cor:validK}
    Given an ordered instance $\GenInstance$, for all $k < |M|/n$ and $\alpha \geq 0$, $R^\alpha_k(\I)$ is an $\alpha$-valid reduction. 
\end{corollary}
\begin{algorithm}[tb]
\caption{$\mathtt{R-reduce(\I,\alpha)}$: 
\\ \textbf{Input:} An ordered instance $\GenInstance$ and approximation factor $\alpha$.
\\ \textbf{Output:} An ordered instance $\I'=(N',M',\{v_i\}_{i \in N'})$ with $N' \subseteq N$ and $M' \subseteq M$.}
\label{alg:Reduce}\SetAlgoLined
\DontPrintSemicolon
\LinesNumbered

\While{for any $k < |M|/n$, $R^\alpha_k$ is applicable}{
     $\I \leftarrow R^\alpha_k(\I)$ for an arbitrary $k$ such that $R^\alpha_k$ is applicable}
\Return $(N,M,\{v_i\}_{i \in N})$
\end{algorithm}

\begin{restatable}{lemma}{reduceTwoGoods}\label{lem:reduce-2goods}
    Let $\GenInstance$ be a fair division instance with additive valuations, and fix an agent $i \in N$. Let $g_1, g_2 \in M$ be such that $v_i(g_1) + v_i(g_2) \leq \mu^n_i(M)$. Then 
    $\mu^n_i(M) \leq \mu^{n-1}_i(M \setminus \{g_1,g_2\}).$
\end{restatable}
\begin{proof}
    To prove the lemma, it suffices to find a partition of (a subset of) $M \setminus \{g_1,g_2\}$ into $n-1$ bundles, such that the value of agent $i$ for each bundle is at least $\mu^n_i(M)$. We call such a partition a certificate. Let $P=(P_1, \ldots, P_n)$ be an MMS partition of agent $i$ in the original instance $\I$. Without loss of generality, assume $\{g_1,g_2\} \subseteq P_1 \cup P_2$. If $\{g_1,g_2\} \subseteq P_1$, then $(P_2, \ldots, P_{n-1}, P_n)$ is a certificate. Otherwise, let $g_i \in P_i$ for $i \in [2]$. We have 
    \begin{align*}
        v_i(P_1 \cup P_2 \setminus \{g_1,g_2\}) 
        % &=
        =
        v_i(P_1) + v_i(P_2) - (v_i(g_1) + v_i(g_2)) 
        % \\
        &\geq 2\mu^n_i(M) - \mu^n_i(M) \geq \mu^n_i(M).
    \end{align*}
    Hence $(P_1 \cup P_2 \setminus \{g_1,g_2\}, P_3, \ldots, P_n)$ is a certificate.
    % \mfc{Shall we add that $P_1 \cup P_2 \setminus \{g_1,g_2\}$ is non-empty due to the other valid reduction?}
\end{proof}


\begin{restatable}{lemma}{newReduction}\label{lem:new-reduction}
    Given a fair division instance $\GenInstance$ and a fixed agent $i \in N$ with additive valuation, let $g_1, g_2, g_3 \in M$ be such that $g_2 \neq g_3$ and $v_i(g_2) + v_i(g_3) \leq \mu^n_i(M)$. Then 
    $\mu^{n-2}_i(M \setminus \{g_1,g_2,g_3\}) \geq \mu^n_i(M).$
\end{restatable}
\begin{proof}
    To prove the lemma, it suffices to partition (a subset of) $M \setminus \{g_1,g_2,g_3\}$ into $n-2$ bundles, each of value at least $\mu^n_i(M)$ to $i$. We call such a partition a certificate. Let $P=(P_1, \ldots, P_n)$ be an MMS partition of agent $i$ in the original instance $\I$. Without loss of generality, assume $\{g_1,g_2,g_3\} \subseteq P_1 \cup \ldots \cup P_j$ for $j \leq 3$. If $\{g_1,g_2,g_3\} \subseteq P_1 \cup P_2$, then $(P_3, \ldots, P_{n-1}, P_n)$ is a certificate. Otherwise, let $g_i \in P_i$ for $i \in [3]$. We have 
    \begin{align*}
        v_i(P_2 \cup P_3 \setminus \{g_2,g_3\}) 
        % &= 
        =
        v_i(P_2) + v_i(P_3) - (v_i(g_2) + v_i(g_3)) 
        % \\
        % &
        \geq 2\mu^n_i(M) - \mu^n_i(M) \geq \mu^n_i(M).
    \end{align*}
    Hence $(P_2 \cup P_3 \setminus \{g_2,g_3\}, P_4, \ldots, P_n)$ is a certificate.
\end{proof}

%\vspace{0.5cm}
\begin{tcolorbox}[%colback=blue!5,coltitle=blue!50!black,colframe=blue!25,
	title= \textbf{Rule $S^\alpha(\I)$}]\label{rule:S}
	%Input: allocation $\allocs$
	
	Preconditions:\label{pre1}
	\begin{itemize}
        \item $\I$ is ordered and $R^\alpha_1$ is not applicable.
		\item There exists an agent $i$ such that $v_i(\{1,n+1\}) \geq \alpha \cdot \mu_i(\I)$.
	\end{itemize}
    Process:
    \begin{itemize}
        \item Allocate $\{1,n+1\}$ to $i$.
        \item Set $N' \leftarrow N \setminus \{i\}$ and $M' \leftarrow M \setminus \{1,n+1\}$.
        \item If there exists agent $j \neq i$ such that $v_i(\{1,n+1\}) \geq \alpha \cdot \mu_j(\I)$: 
            \subitem - Duplicate good $1$ and allocate $\{1,n\}$ to agent $j$.
            \subitem - Set $N' \leftarrow N' \setminus \{j\}$ and $M' \leftarrow M' \setminus \{n\}$.
	\end{itemize}
    Guarantees:
	\begin{itemize}
		\item $\I' = (N',M',\{v_i\}_{i \in N'})$ is ordered.
        \item For all $i \in N'$, $\mu_i(\I') \geq \mu_i(\I)$.
	\end{itemize}
\end{tcolorbox}
%\vspace{0.5cm}

\begin{restatable}{lemma}{lemValidS}\label{lem:validS}
    Given an ordered instance $\GenInstance$ and $\alpha \leq 1$, $S^\alpha(\I)$ is an $\alpha$-valid reduction. 
\end{restatable}
\begin{proof}
    For all agents $i$ who is assigned a bundle $B$ in $S^\alpha$, we have $v_i(B) \geq \alpha \cdot \mu_i(\I)$. 
    First consider the case that $S^\alpha$ removes two agents $i \neq j$. Since the reduction rule $R^\alpha_1$ is not applicable, for all the remaining agents $a$, we have $v_a(\{n,n+1\}) < \alpha \cdot \mu^n_a(M) \leq \mu^n_a(M)$. Hence, by lemma \ref{lem:new-reduction}, $\mu^{n-2}_a(M \setminus \{1,n,n+1\}) \geq \mu^n_a(M)$. 

    Now consider the case that $S^\alpha$ removes only one agent $i$. Let $a \neq i$ be a remaining agent after allocating $\{1,n+1\}$ to $i$. We have $v_a(\{1,n+1\}) < \alpha \cdot \mu^n_i(M) \leq \mu^n_i(M)$, otherwise, $S^\alpha$ would allocate $\{1,n\}$ to $a$ (or some agent other than $i$). 
    By Lemma \ref{lem:reduce-2goods}, $\mu^{n-1}_a(M \setminus \{1,n+1\}) \geq \mu^n_a(M)$.     
\end{proof}

%\vspace{0.5cm}
\begin{tcolorbox}[%colback=blue!5,coltitle=blue!50!black,colframe=blue!25,
	title= \textbf{Rule $T^\alpha(\I)$}]\label{rule:T}
	%Input: allocation $\allocs$
	
	Preconditions:\label{preT}
	\begin{itemize}
        \item $\I$ is ordered and $R^\alpha_1$ is not applicable.
		\item There exists three different agents $i,j,\ell$ such that $v_i(\{1,n+1\}) \geq \alpha \cdot \mu_i(\I)$, $v_j(\{2,n+2\}) \geq \alpha \cdot \mu_j(\I)$, and $v_\ell(\{1,n+3\}) \geq \alpha \cdot \mu_\ell(\I)$.
	\end{itemize}
    Process:
    \begin{itemize}
        \item Allocate $\{1,n+1\}$ to $i$.
        \item Allocate $\{2,n+2\}$ to $j$.
        \item Duplicate good $1$ and allocate $\{1,n+3\}$ to $\ell$.
        \item Set $N' \leftarrow N \setminus \{i,j,\ell\}$ and $M' \leftarrow M \setminus \{1,2,n+1,n+2,n+3\}$.
	\end{itemize}
    Guarantees:
	\begin{itemize}
		\item $\I' = (N',M',\{v_i\}_{i \in N'})$ is ordered.
        \item For all $i \in N'$, $\mu_i(\I') \geq \mu_i(\I)$.
	\end{itemize}
\end{tcolorbox}
%\vspace{0.5cm}

% \ygc{I replaced $\mu_i(\I)$ in the statement below with $\mu_i^n(M)$ for consistency}
% \begin{restatable}{lemma}{fourOverFive}\label{lem:4-5-reduction}
%     Let $\I$ be a fair division instance with additive valuations and let $S = \{h_1, h_2, s_1, s_2, s_3 \} \subseteq M$ be such that for $j \in [2]$, $v_i(h_j) \leq 4/5 \cdot \mu^n_i(M)$ and for $j \in [3]$, $v_i(s_j) \leq 2/5 \cdot \mu_i(\I)$ for all $i \in  N$. Then for all $i \in  N$ we have
%     $\mu^{n-3}_i(M \setminus S) \geq \mu^n_i(M).$
% \end{restatable}
\begin{restatable}{lemma}{fourOverFive}\label{lem:4-5-reduction}
    Let $\I$ be a fair division instance with additive valuations and let $S = \{h_1, h_2, s_1, s_2, s_3 \} \subseteq M$ be such that for $j \in [2]$, $v_i(h_j) \leq 4/5 \cdot \mu^n_i(M)$ and for $j \in [3]$, $v_i(s_j) \leq 2/5 \cdot \mu^n_i(M)$ for all $i \in  N$. Then for all $i \in  N$ we have
    $\mu^{n-3}_i(M \setminus S) \geq \mu^n_i(M).$
\end{restatable}

\begin{proof}
    To prove the lemma, it suffices to partition (a subset of) $M \setminus S$ into $n-3$ bundles, each of value at least $\mu^n_i(M)$ to $i$. We call such a partition a certificate. Let $P=(P_1, \ldots, P_n)$ be an MMS partition of agent $i$ in the original instance $\I$. Without loss of generality, assume $S \subseteq P_1 \cup \ldots \cup P_k$ for $k \leq 5$. If $S \subseteq P_1 \cup P_2 \cup P_3$, then $(P_4, \ldots, P_{n-1}, P_n)$ is a certificate. Otherwise, we consider two cases:
    \begin{itemize}
        \item Case 1: $S \subseteq P_1 \cup \ldots \cup P_4$. We have
        \begin{align*}
            v_i(P_1 \cup \ldots \cup P_4 \setminus S) &\geq \mu^n_i(M)(4 - 2 \cdot 4/5 - 3 \cdot 2/5) > \mu^n_i(M).  
        \end{align*}
        Therefore, $(P_1 \cup \ldots \cup P_4 \setminus S, P_5, \ldots, P_n)$ is a certificate.
        \item Case 2: $S \subseteq P_1 \cup \ldots \cup P_5$. Let $P_j \cap S = \{s_j\}$ for $j \in [3]$ and $P_{j+3} \cap S = \{h_j\}$ for $j \in [2]$. We have 
        \begin{align*}
            v_i(P_1 \cup P_2 \setminus S) \geq \mu^n_i(M)(2 - 2 \cdot 2/5) > \mu^n_i(M),
        \end{align*}
        and 
        \begin{align*}
            v_i(P_3 \cup P_4 \cup P_5 \setminus S) \geq \mu^n_i(M)(3 - 2 \cdot 4/5 - 2/5) = \mu^n_i(M).
        \end{align*}
        Thus, $(P_1 \cup P_2 \setminus S, P_3 \cup P_4 \cup P_5 \setminus S, P_6, \ldots, P_n)$ is a certificate. \qedhere
    \end{itemize}
\end{proof}

\begin{restatable}{lemma}{validT}\label{lem:validT}
    Given an ordered instance $\GenInstance$ and $\alpha \leq 4/5$, $T^\alpha(\I)$ is an $\alpha$-valid reduction. 
\end{restatable}
\begin{proof}
    For all agents $i$ who is assigned a bundle $B$ in $T^\alpha$, we have $v_i(B) \geq \alpha \cdot \mu_i(\I)$. 
    For all the remaining agents $i$, we have $\alpha \cdot \mu^n_i(M) \geq v_i(\{1\}) \geq v_i(\{2\})$ and $\alpha/2 \cdot \mu^n_i(M)\geq  v_i(\{2n+1\}) \geq v_i(2n+2) \geq v_i(2n+3)$. Since $\alpha \leq 4/5$, by Lemma \ref{lem:4-5-reduction}, $v^{n-3}_i(M \setminus \{1,2,n+1,n+2,n+3\}) \geq v^n_i(M)$.    
\end{proof}

\begin{definition}
    We call an ordered instance $\GenInstance$, $R^\alpha$-irreducible, if for all $k < |M|/|N|$, $R^\alpha_k(\I)$ is not applicable. Similarly, an ordered instance is $S^\alpha$-irreducible and $T^\alpha$-irreducible, if $S^\alpha$ and $T^\alpha$ are not applicable respectively.
\end{definition}

\subsection{Combining Round Robin and Bag-Filling}\label{sec:RR}
In this section, we present Algorithm \ref{alg:RR} ($\bagfillRR(\I,\alpha)$) which combines the idea of bag-filling and round robin to achieve an $\alpha$-MMS allocation with copies for ordered instances.

Note that the input $\I$ of $\bagfillRR(\I,\alpha)$ is $R^\alpha$-irreducible. 
For all the agents $i$, we define $\mu_i = \mu^n_i(M)$ to be the MMS value of agent $i$ in the beginning of the Algorithm \ref{alg:RR}. We do a round robin fashion process as following: starting from $n$ empty bags $B_1, \ldots, B_n$, for $j$ in the cyclic sequence $1,2, \ldots, n$, we add the most valuable available  good (not yet allocated to any bag) to $B_j$. If for some remaining agent $i$, $v_i(B_j) \geq \alpha \cdot \mu_i$, then we allocate $B_j$ to $i$ and remove $i$ from the set of agents and $j$ from the cyclic sequence. We call this phase of the algorithm, the round robin phase. Note that for all the remaining bags $B_a$ and $B_b$, if $a<b$, then $v_i(B_a) \geq v_i(B_b)$ for all agents $i$ and hence, first index $1$ will be removed, then index $2$ and so on (see Lemma \ref{lem:order}). Let $\{B_a, \ldots, B_n\}$ be the set of the remaining bags at the end of the round robin phase. In the next phase of the algorithm, called duplication phase, we duplicate goods $1,2, \ldots, n-a+1$, add them to $B_a, \ldots, B_n$ respectively and allocate these bags to the remaining agents arbitrarily. Unlike the round robin phase, it is not clear whether the agents who are allocated during the duplication phase value their bundle at least $\alpha$ fraction of their MMS. However, in Sections \ref{sec:67} and \ref{sec:45}, we show that by appropriately reducing the instance in advance, we can ensure both the desired approximation guarantees and the bound on the number of distinct copies. % multiplicity bounds.

\begin{algorithm}[tb]
\caption{$\bagfillRR(\I,\alpha)$: 
\\ \textbf{Input:} An $R^\alpha$-irreducible instance $\GenInstance$ and approximation factor $\alpha$
\\ \textbf{Output:} An $\alpha$-MMS allocation $A$ with copies}
\label{alg:RR}\SetAlgoLined
\DontPrintSemicolon
\LinesNumbered
%\todo{Add pictures.} \\
Let $B_i = \emptyset$ for $i \in [|N|]$ \;
$n \leftarrow |N|$, $j \leftarrow 1$, 
$a \leftarrow 1$ \;
\For(\Comment{Round robin Phase}){$g \gets 1$ \KwTo $|M|$} {
    $B_j \leftarrow B_j \cup \{g\}$ \;
    \If{there exists $i \in N$ such that $v_i(B_j) \geq \alpha \cdot \mu_i$}{
        $A_i \leftarrow B_j$ for such an arbitrary $i$ \;
        $N \leftarrow N \setminus \{i\}$, $ a \leftarrow a+1$ \;
    } 
    $j \leftarrow j+1$ \;
    \If{$j > n$}{
        $j \leftarrow a$ \;
    }
}
\For(\Comment{Duplication Phase}){$j \gets a$ \KwTo $n$}{  
    Let $i \in N$ \;
    $A_i \leftarrow B_j \cup \{j-a+1\}$, $N \leftarrow N \setminus \{i\}$ \;
}
\Return $(A_1, A_2, \ldots, A_n)$ \;
\end{algorithm}

For the rest of this (sub-)section, let $\alpha \geq 0$, $\I$ be an $R^\alpha$-irreducible instance, and  $\mu_i$ be $\mu_i(\I)$.

\begin{lemma}\label{lem:upper-bound}
    For all the remaining agents $i$ after the round robin phase of $\bagfillRR(\I,\alpha)$, and all the bags $B_j$ with $|B_j| \geq 3$ that were allocated to other agents during the round robin phase, we have $v_i(B_j) \leq (4\alpha/3) \cdot \mu_i$.
\end{lemma}
\begin{proof}
    Let $g$ be the last good added to $B_j$. %Note that $v_i(\{j,n+j\}) \leq v_i(\{1,n+1\}) < \mu_i$. Hence $|B_j| \geq 3$ and thus, 
    We have $g \geq 2n+1$ and hence $v_i(g) \leq v_i(2n+1) \leq \alpha\mu_i/3$. The last inequality holds since the instance is $R^\alpha_2$-irreducible. We have $v_i(B_j \setminus \{g\}) < \alpha \cdot \mu_i$. Otherwise, $B_j \setminus \{g\}$ would have been allocated to $i$, or some other agent who values it at least $\alpha$ fraction of their MMS value, previously. Therefore we have, 
    \begin{align*}
        v_i(B_j) &= v_i(B_j \setminus \{g\}) + v_i(g) \leq \alpha \cdot \mu_i + \alpha/3 \cdot \mu_i = 4\alpha/3 \cdot \mu_i. \qedhere
    \end{align*}
\end{proof}

\begin{lemma}\label{lem:order}
    Let $\B$ be the set of bags allocated during the round robin phase of $\bagfillRR(\I,\alpha)$. Then $\B = \{B_1, \ldots, B_{|\B|}\}$.
\end{lemma}
\begin{proof}
    We prove that first $B_1$ gets allocated, then $B_2$ and so on. %Before the round robin phase, by the initialization of the bags we have $v_i(B_1) \geq v_i(B_2) \geq \ldots \geq v_i(B_n)$ for all agents $i$. 
    Towards contradiction, assume for some $j>1$ bag $B_j$ gets allocated to agent $i$ while bag $B_{j-1}$ is not assigned. At the time that $B_j$ gets assigned, we have $|B_{j-1}| = |B_j|$ since both $B_{j-1}$ and $B_j$ were present in all the rounds so far. Let $B_{j-1}=\{g_1, \ldots, g_k\}$ and $B_j = \{g'_1, \ldots, g'_k\}$. Followed from round robin, for all $\ell \in [k]$, we have $g_\ell = g'_\ell-1$ which means $v_i(g_\ell) \geq v_i(g'_\ell)$. Hence $v_i(B_{j-1}) \geq v_i(B_j) \geq \mu_i$ which is a contradiction that $B_{j-1}$ was not assigned to anyone before moving to $B_j$.
\end{proof}
\subsection{\boldmath $6/7$-MMS with Copies}\label{sec:67}
In this section, we present an algorithm which given an ordered instance, outputs an $\alpha$-MMS allocation with at most $\floor{n/2}$ distinct copies for any $\alpha \leq 6/7$.

First we reduce the instance in Algorithm \ref{alg:ReduceWithCopies} using $R^\alpha_0, R^\alpha_1, \ldots,$ and $S^\alpha$. It is not difficult to see that if $k$ agents are removed (i.e. are assigned a bundle) by the end of the algorithm, then in total at most $\floor{k/2}$ duplicated goods are introduced and each good is duplicated at most once. Also, since we remove these $k$ agents and all the goods in these $k$ bags, in the future no more copies of them can be used. Since all these rules are $\alpha$-valid reductions, it suffices to give an $\alpha$-MMS allocation with at most $\floor{(n^*-k)/2}$ distinct copies for the instance obtained by the remaining agents and goods where $n^*$ is the original number of agents. Let $n = n^*-k$.

Then, we run Algorithm \ref{alg:RR} ($\bagfillRR(\I,\alpha)$) on the reduced instance. We prove that by the end of the round robin phase, at most $\lfloor n/2 \rfloor$ agents remain. In other words, if the bags $B_a, B_{a+1}, \ldots, B_n$ are the remaining bags, then $a \geq \floor{n/2}+1$. Furthermore, we prove for all remaining agents $i$, $v_i(B_j \cup \{n-a+1\}) \geq \mu_i(\I)$ for all $j \in \{a, \ldots, n\}$. Therefore, to ensure MMS for the remaining agents, it suffices to duplicate goods $1,2, \ldots, n-a+1 \leq \floor{n/2}<a$, add them to $B_a, \ldots, B_n$ respectively and allocate these bags to the remaining agents arbitrarily. This is exactly the duplication phase of $\bagfillRR(\I,\alpha)$. In this phase, we duplicate at most $\floor{n/2}$ many goods and at most one copy of each good. 

\begin{algorithm}[tb]
\caption{$\mathtt{S-reduce(\I,\alpha)}$: 
\\ \textbf{Input:} An ordered instance $\GenInstance$ and approximation factor $\alpha$
\\ \textbf{Output:} An ordered instance $\I'=(N',M',\{v_i\}_{i \in N'})$ with $N' \subseteq N$ and $M' \subseteq M$.}
\label{alg:ReduceWithCopies}\SetAlgoLined
\DontPrintSemicolon
\LinesNumbered

$\I \leftarrow \mathtt{R-reduce(\I)}$ \;
\While{$S^\alpha(\I)$ is applicable}{
    $\I \leftarrow S^\alpha(\I)$ \;
    $\I \leftarrow \mathtt{R-reduce(\I)}$ \;
}
\Return $\I = (N,M,\{v_i\}_{i \in N})$ \;
\end{algorithm}

Lemma \ref{lem:s-reduc-alg} follows from Lemma \ref{lem:validS} and Corollary \ref{cor:validK}.
\begin{lemma}\label{lem:s-reduc-alg}
    For an arbitrary ordered instance $\I$ and $\alpha \leq 1$, let $\I' = \mathtt{S-reduce(\I,\alpha)}$. Then for all agents $i \in N'$, $\mu_i(\I') \geq \mu_i(\I)$. 
\end{lemma} 

For the rest of this (sub)-section, let $\alpha \leq 6/7$, $\I$ be an $R^\alpha$-irreducible and $S^\alpha$-irreducible instance, and $\mu_i$ be $\mu_i(\I)$. Let $n$ be the number of agents right before the round robin phase of $\bagfillRR(\I,\alpha)$, $\B = \{B_1, \ldots, B_{a-1}\}$ be the set of bags allocated during this phase, and $n' = n-a+1$ be the number of remaining agents after this phase.

\begin{restatable}{observation}{obsSizeThree}\label{obs:size3}
    For all bags $B_j \in \B$, $|B_j| \geq 3$.
\end{restatable}
\iffalse
\begin{proof}
    Since $\I$ is $S$-irreducible, $v_i(\{j,n+j\}) \leq v_i(\{1,n+1\})<\alpha \cdot \mu_i$ for all $i \in N$. Thus, if $B_j$ was allocated to some agent, $|B_j|>2$. 
\end{proof}
\fi
\begin{lemma}\label{lem:many-valuable-bags}
    Let $i$ be a remaining agent after the round robin phase of $\bagfillRR(\I,\alpha)$ (if any). Then $|\{B \in \B \mid v_i(B) > \mu_i\}| \geq n'$.
\end{lemma}
\begin{proof}
    Let $\mathcal{B}^+ = \{B \in \mathcal{B} \mid v_i(B) > \mu_i\}$. By definition, for all $B \in \mathcal{B} \setminus \mathcal{B}^+$, $v_i(B) \leq \mu_i$. By Observation \ref{obs:size3}, for all $B \in \mathcal{B}$ , $|B| \geq 3$. By Lemma \ref{lem:upper-bound}, for all $B \in \mathcal{B}^+$, $v_i(B) \leq 4\alpha/3 \cdot \mu_i$. Also, let $\bar{\mathcal{B}}$ be the set of $n-|\B| = n'$ remaining bags after the round robin phase. For all $B \in \bar{\mathcal{B}}$, $v_i(B) < \alpha \cdot \mu_i$. Otherwise, $B$ should have been allocated to $i$ (or some other agent who values it at least $\alpha$ fraction of their MMS value). Therefore, we have
    \begin{align*}
        n \cdot \mu_i &\leq v_i(M) \\
        &= \sum_{B \in \mathcal{B}^+} v_i(B) + \sum_{B \in \mathcal{B} \setminus \mathcal{B}^+} v_i(B) + \sum_{B \in \bar{\mathcal{B}}} v_i(B) \\
        &\leq |\mathcal{B}^+| 4\alpha/3 \cdot \mu_i + (|\mathcal{B}| - |\mathcal{B}^+|) \mu_i + (n - |\mathcal{B}|) \alpha \cdot \mu_i \\
        &\leq (|\mathcal{B}| - |\mathcal{B}^+|) \mu_i + 6/7 (|\mathcal{B}^+|/3 + n - |\B| + |\mathcal{B^+}|) \cdot \mu_i \tag{$\alpha \leq 6/7$}
    \end{align*}
    % \ygc{In the row above, I think the last $|\mathcal{B^+}|$ is a typo}
    Dividing both sides of the inequality by $\mu_i$ and removing $(|\mathcal{B}| - |\mathcal{B}^+|)$, we obtain, 
    \begin{align*}
        n - |\B| + |\B^+| &\leq 6/7 (|\B^+|/3 + n - |\B| + |\B^+|) = 2|\B^+|/7 + 6/7(n - |\B| + |\B^+|).
    \end{align*}
    Therefore, $|\B^+| \geq (n - |\B| + |\B^+|)/2$ which means $|\B^+| \geq n - |\B|=n'$.
\end{proof}

\begin{corollary}[of Lemma \ref{lem:many-valuable-bags}]\label{cor:halfRemaining}
     $n' \leq \floor{n/2}$.        
\end{corollary}
\begin{lemma}\label{lem:duplic-suffice}
    For all the remaining agents $i$ after the round robin phase of $\bagfillRR(\I,\alpha)$, and all the remaining bags $B_j$, we have $v_i(B_j \cup \{n'\}) \geq \mu_i$.
\end{lemma}
% \ygc{In the proof below, how about replacing $\min\{k,k'-1\}$ with $k'-1$? We know that $|B_j| \geq |B_{i_{n'}}|-1$ (i.e., $k \ge k'-1$) }
\begin{proof}
    By Lemma \ref{lem:order}, the bags $B_1, \ldots, B_{n-n'}$ are allocated during the round robin phase. 
    By Lemma \ref{lem:many-valuable-bags}, there exists at least $n'$ many bags $B_{i_1}, \ldots, B_{i_{n'}}$ for $i_1 < i_2 < \ldots < i_{n'} \leq n-n'$ that were allocated to other agents during the round robin phase and $v_i(B_{i_\ell}) > \mu_i$ for all $\ell \in [n']$. Note that $i_{n'} \geq n'$. Since $B_j$ have been assigned a good in all but maybe the last round, we have $|B_j| \geq |B_{i_{n'}}|-1$. Let $B_j = \{g_1, \ldots, g_k\}$ and $B_{i_{n'}} = \{g'_1, \ldots, g'_{k'}\}$ for $j= g_1 < \ldots < g_k$ and $i_{n'} = g'_1 < \ldots < g'_{k'}$. For all $\ell \in [k'-1]$, since $g_\ell$ is allocated before $g'_{\ell+1}$, $g_\ell < g'_{\ell+1}$ and hence $v_i(g_\ell) \geq v_i(g'_{\ell+1})$. We have
    \begin{align*}
        v_i(B_j \cup \{n'\}) &= v_i(B_j) + v_i(n') \geq v_i(B_j) + v_i(i_{n'}) = \sum_{\ell \in [k]} v_i(g_\ell) + v_i(i_{n'}) \\
        &\geq \sum_{\ell \in [k'-1]} v_i(g'_{\ell+1}) + v_i(i_{n'}) = v_i(B_{i_{n'}}) > \mu_i. \qedhere
    \end{align*}
\end{proof}

\begin{theorem} \label{thm:nover2copiesMMS}
    Given an ordered instance $\GenInstance$ with $n^*$ agents and $\alpha \leq 6/7$, 
    $$\bagfillRR(\mathtt{S-reduce(\I,\alpha)},\alpha)$$
    returns an $\alpha$-MMS allocation with at most $\floor{n^*/2}$ many distinct copies.
\end{theorem}
\begin{proof}
    Let $n$ be the number of agents after applying $\mathtt{S-reduce(\I,\alpha)}$. Since $S^\alpha$ and $R^\alpha_k$ are valid reductions for all $k < |M|/n^*$, all agents who receive a bag during the $\mathtt{S-reduce(\I,\alpha)}$ receive at least $\alpha$ fraction of their MMS value and the MMS value of the remaining agents do not decrease after this process. All the agents who receive a bag during the round robin phase of $\bagfillRR$ get at least $\alpha$ fraction of their value. Let agent $i$ be an agent who receives bag $B_j$ after the round robin phase. Let $g$ be the good that was duplicated and added to $B_j$ right before allocating it to agent $i$. $g \leq n'$ and thus $v_i(g) \geq v_i(n')$. By Lemma \ref{lem:duplic-suffice}, $v_i(B_j \cup \{g\}) \geq \alpha \mu_i$. Therefore, the final allocation is $\alpha$-MMS. 

    Now we prove at most $\floor{n^*/2}$ distinct copies are used. Let $k=n^*-n$ be the number of agents who receive a bag in $\mathtt{S-reduce(\I,\alpha)}$. Only when applying $S^\alpha$, new copies are introduced and one copy per two agents that are removed. Furthermore, the goods that are allocated during $S^\alpha$ along with their copies are removed from the instance. Therefore, they cannot be copied more than once. So during $\mathtt{S-reduce(\I,\alpha)}$, at most $\floor{k/2}$ distinct copies are used. In the duplication phase of $\bagfillRR$, we duplicate items $1,2, \ldots, n'$ once. By Corollary \ref{cor:halfRemaining}, $n' \leq \floor{n/2}$. Therefore, in total at most $\floor{n^*/2}$ distinct copies are used.
\end{proof}

\subsection{\boldmath $4/5$-MMS with Copies}\label{sec:45}
In this section, we present an algorithm which, given an ordered instance, outputs an $\alpha$-MMS allocation with at most $\floor{n/3}$ distinct copies for any $\alpha \leq 4/5$. 

Similar to Section \ref{sec:67}, first we reduce the instance but this time by Algorithm \ref{alg:ReduceWith3Copies} using $R^\alpha_0, R^\alpha_1, \ldots,$ and $T^\alpha$. Analogously, it is not difficult to see that if $k$ agents are removed (i.e. are assigned a bundle) by the end of the algorithm, then in total at most $\floor{k/3}$ duplicated goods are introduced and each good is duplicated at most once. Also, since we remove these $k$ agents and all the goods in these $k$ bags, in the future no more copies of them can be used. Since all these rules are $\alpha$-valid reductions, it suffices to give an $\alpha$-MMS allocation with at most $\floor{(n^*-k)/3}$ distinct copies for the instance obtained by the remaining agents and goods where $n^*$ is the original number of agents. Let $n = n^*-k$.

Then, we run Algorithm \ref{alg:RR} ($\bagfillRR(\I,\alpha)$) on the reduced instance. Following a similar idea, but a more involved calculation, we prove that we end up with a $4/5$-MMS allocation using at most $\floor{n/3}$ distinct copies.

%We refer the reader to Appendix \ref{apx:fourOverFive} for more details and proofs.

%\todo{move the alg to the appendix}

\begin{algorithm}[tb]
\caption{$\mathtt{T-reduce(\I,\alpha)}$: 
\\ \textbf{Input:} An ordered instance $\GenInstance$ and approximation factor $\alpha$
\\ \textbf{Output:} An ordered instance $\I'=(N',M',\{v_i\}_{i \in N'})$ with $N' \subseteq N$ and $M' \subseteq M$.}
\label{alg:ReduceWith3Copies}\SetAlgoLined
\DontPrintSemicolon
\LinesNumbered

$\I \leftarrow \mathtt{R-reduce(\I)}$ \;
\While{$T^\alpha(\I)$ is applicable}{
    $\I \leftarrow T^\alpha(\I)$ \;
    $\I \leftarrow \mathtt{R-reduce(\I)}$ \;
}
\Return $\I = (N,M,\{v_i\}_{i \in N})$ \;
\end{algorithm}


\begin{lemma}
    For an arbitrary ordered instance $\I$ and $\alpha \leq 1$, let $\I' = \mathtt{T-reduce(\I,\alpha)}$. Then for all agents $i \in N'$, $\mu_i(\I') \geq \mu_i(\I)$. 
\end{lemma} 
\begin{proof}
    The lemma follows from Lemma \ref{lem:validT} and Corollary \ref{cor:validK}.
\end{proof}

For the rest of this (sub)-section, let $\alpha \leq 4/5$, $\I$ be an $R^\alpha$-irreducible and $T^\alpha$-irreducible instance, and $\mu_i$ be $\mu_i(\I)$. Let $n$ be the number of agents right before the round robin phase of $\bagfillRR(\I,\alpha)$, $\B = \{B_1, \ldots, B_{a-1}\}$ be the set of bags allocated during this phase, and $n' = n-a+1$ be the number of remaining agents after this phase.

\begin{lemma}\label{obs:size3-2}
    For all bags $B_j \in \B$ with $j > 2$, $|B_j| \geq 3$.
\end{lemma}
\begin{proof}
    Towards contradiction, assume for some $j>2$, $|B_j| \leq 2$ and $B_j$ is allocated to agent $i$. Note that since $\I$ is $R_0^\alpha$-irreducible, \iffalse $v_a(1) < \mu_a$ \fi $v_a(1) < \alpha\cdot \mu_a$ for all agents $a$ and hence $|B| \geq 2$ for all $B \in \B$. So for all $\ell \in [n]$, we have $\{\ell,n+\ell\} \in B_\ell$ and $B_j = \{j,n+j\}$. Note that for all $j'<j$, $v_i(\{j',n+j'\}) > v_i(\{j,n+j\}) \geq \alpha \cdot \mu_i$. Hence, for all $j'<j$, $|B_{j'}|=2$. Let $a_1,a_2,a_3$ be the three agents who received $B_1, B_2, B_3$ respectively. Note that the precondition of Rule $T^\alpha$ is met for these agents which is a contradiction with $\I$ being $T^\alpha$-irreducible.    
\end{proof}
\begin{observation}\label{obs:1-2}
    For $n\geq2$, $\{B_1,B_2\} \subseteq \B$ and $v_i(B_j) \leq 3\alpha/2 \cdot \mu_i$ for all $i \in N$ and $j \in [2]$.
\end{observation}
\begin{proof}
    Since the instance is $R^\alpha_0$-irreducible, $v_i(2) \leq v_i(1) < \alpha \cdot \mu_i$. Therefore, $|B_j| \geq 2$. Let $g$ be the last good added to $B_j$. Since $g \geq n+1$, $v_i(g) \leq v_i(n+1) < \alpha/2 \cdot \mu_i$. We have 
    \begin{align*}
        v_i(B_j) &= v_i(B_j \setminus \{j\}) + v_i(B_j) < \alpha \cdot \mu_i + \alpha/2 \cdot \mu_i = 3\alpha/2 \cdot \mu_i. 
    \end{align*}
    Now assume $B_2 \notin \B$. Note that for all $B \notin \B$, $v_i(B) < \alpha \cdot \mu_i$. Otherwise, $B$ should have been allocated during the round robin phase. Therefore
    \begin{align*}
        n \cdot \mu_i &\leq v_i(M) \\
        &= \sum_{j}v_i(B_j) \\
        &< 3\alpha/2 \cdot \mu_i + (n-1)\alpha \cdot \mu_i \\
        &\leq (4/5n+2/5) \cdot \mu_i, \tag{$\alpha \leq 4/5$} 
    \end{align*}
    which is a contradiction with $n \geq 2$.
\end{proof}
\begin{lemma}\label{lem:many-valuable-bags-2}
    Let $i$ be a remaining agent after the round robin phase of $\bagfillRR(\I,\alpha)$ (if any). Then $|\{B \in \B \mid v_i(B) > \alpha \cdot \mu_i\}| \geq 2n/3$ or $n\leq 5$.
\end{lemma}
\begin{proof}
    Assume $n>5$. Let $\B^+ = \{B \in \B \mid v_i(B) > \alpha \cdot \mu_i\}$. By definition, for all $B \in \B \setminus \B^+$, $v_i(B) \leq \alpha \cdot \mu_i$. Let $\bar{\B}$ be the set of $n-|\B|=n'$ many remaining bags after the round robin phase. For all $B \in \bar{\B}$, $v_i(B) < \alpha \cdot \mu_i$. Otherwise, $B$ should have been allocated to $i$ (or some other agent who values it at least $\alpha$ fraction of their MMS value). We consider two cases:
    \paragraph{\boldmath Case 1: $v_i(B_2) \leq 4\alpha/3 \cdot \mu_i$.}
    For all $B \in \B \setminus \{B_1,B_2\}$, Lemma \ref{obs:size3-2} implies $|B| \geq 3$ and Lemma \ref{lem:upper-bound} implies $v_i(B) \leq 4\alpha/3 \cdot \mu_i$. Also by Observation \ref{obs:1-2}, $v_i(B_1) \leq 3\alpha/2 \cdot \mu_i$ and we have $v_i(B_2) \leq 4\alpha/3 \cdot \mu_i$. Therefore
    \begin{align*}
        \sum_{B \in \B^+} v_i(B) &\leq 4\alpha/3 \cdot \mu_i (|\B^+|-1) + 3\alpha/2 \cdot \mu_i \\
        &\leq \mu_i( 16|\B^+|/15 + 2/15). \tag{$\alpha \leq 4/5$}
    \end{align*}
    % \ygc{I think last line should be multiplied by $\mu_i$, which should also carry over to the next equation.}
    
    We have
    \begin{align*}
        n \cdot \mu_i &\leq v_i(M) \\
        &= \sum_{B \in \B^+} v_i(B) + \sum_{B \in \B \setminus \B^+} v_i(B) + \sum_{B \in \bar{\B}} v_i(B) \\
        &\leq \mu_i(16|\B^+|/15 + 2/15) + (|\B| - |\B^+|) \alpha \cdot \mu_i + (n - |\B|) \alpha \cdot \mu_i \\
        &\leq \mu_i (2/15 + 4n/5 + 4|\B^+|/15). \tag{$\alpha \leq 4/5$}
    \end{align*}
    We obtain
    \begin{align*}
        n/5 &\leq 4|\B^+|/15 + 2/15.
    \end{align*}
    Therefore, $|\B^+| \geq 3(n-2)/4 \geq 2n/3$.
    \paragraph{\boldmath Case 2: $v_i(B_2) > 4\alpha/3 \cdot \mu_i$.} If $|B_2| \geq 3$, then by Lemma \ref{lem:upper-bound}, $v_i(B_2) \leq 4\alpha/3 \cdot \mu_i$. Also if $|B_2|=1$, $v_i(B_2) < \alpha \cdot \mu_i$ since the instance is $R^\alpha_0$ irreducible. Therefore, $|B_2|=2$. It implies $|B_1|=2$ and we have $B_1 = \{1,n+1\}$ and $B_2 = \{2,n+2\}$. Let $a$ and $b$ be the agents who received $B_1$ and $B_2$. If $v_i(\{1,n+3\}) \geq \alpha \cdot \mu_i$, then the instance is not $T^\alpha$-irreducible since $a,b,i$ satisfy its precondition. Therefore, $v_i(1)+v_i(n+3) < \alpha \cdot \mu_i$. 
    \begin{align*}
        v_i(1) + v_i(n+2) &\geq v_i(2) + v_i(n+2) = v_i(B_2) > 4\alpha/3 \cdot \mu_i.
    \end{align*}
    Therefore, $v_i(n+3) < v_i(n+2) - \alpha/3 \cdot \mu_i < (\alpha/2 - \alpha/3) \cdot \mu_i = \alpha/6 \cdot \mu_i$.
    Now fix an index $j>2$ and let $g$ be the last good added to $B_j$. We have 
    \begin{align*}
        v_i(B_j) &= v_i(B_j \setminus \{g\}) + v_i(g) \\
        &< \alpha \cdot \mu_i + \alpha/6 \cdot \mu_i \\
        &< \mu_i. \tag{$\alpha \leq 4/5$}
    \end{align*}
    Therefore, $\B^+ = \{B_1,B_2\}$ and we have
    \begin{align*}
        n \cdot \mu_i &\leq v_i(M) \\
        &= \sum_{B \in \B^+} v_i(B) + \sum_{B \in \B \setminus \B^+} v_i(B) + \sum_{B \in \bar{\B}} v_i(B) \\
        &\leq 3\alpha \cdot \mu_i + (|\B| - 2) \alpha \cdot \mu_i + (n - |\B|) \alpha \cdot \mu_i \\
        &\leq \mu_i (4/5 + 4n/5). \tag{$\alpha \leq 4/5$}
    \end{align*}
    We obtain that $n \leq 4$ which is a contradiction with $n>5$.
\end{proof}
    
\begin{corollary}[of Lemma \ref{lem:many-valuable-bags-2}]\label{cor:halfRemaining}
     $n' \leq \floor{n/3}$ for $n \geq 6$.        
\end{corollary}

\begin{lemma}\label{lem:duplic-suffice-2}
    If $n \geq 6$, for all the remaining agents $i$ after the round robin phase of $\bagfillRR(\I,\alpha)$, and all the remaining bags $B_j$, we have $v_i(B_j \cup \{n'\}) \geq \alpha \cdot \mu_i$.
\end{lemma}
% \ygc{In the proof below, how about replacing $\min\{k,k'-1\}$ with $k'-1$? We know that $|B_j| \geq |B_{i_{n'}}|-1$ (i.e., $k \ge k'-1$) }
\begin{proof}
    By Lemma \ref{lem:order}, the bags $B_1, \ldots, B_{n-n'}$ are allocated during the round robin phase. 
    By Lemma \ref{lem:many-valuable-bags-2}, there exists at least $2n/3 \geq n'$ many bags $B_{i_1}, \ldots, B_{i_{n'}}$ for $i_1 < i_2 < \ldots < i_{n'} \leq n-n'$ that were allocated to other agents during the round robin phase and $v_i(B_{i_\ell}) > \alpha \cdot \mu_i$ for all $\ell \in [n']$. Note that $i_{n'} \geq n'$. Since $B_j$ have been assigned a good in all but maybe the last round, we have $|B_j| \geq |B_{i_{n'}}|-1$. Let $B_j = \{g_1, \ldots, g_k\}$ and $B_{i_{n'}} = \{g'_1, \ldots, g'_{k'}\}$ for $j= g_1 < \ldots < g_k$ and $i_{n'} = g'_1 < \ldots < g'_{k'}$. For all $\ell \in [k'-1]$, since $g_\ell$ is allocated before $g'_{\ell+1}$, $g_\ell < g'_{\ell+1}$ and hence $v_i(g_\ell) \geq v_i(g'_{\ell+1})$. We have
    \begin{align*}
        v_i(B_j \cup \{n'\}) &= v_i(B_j) + v_i(n') \\
        &\geq v_i(B_j) + v_i(i_{n'}) \\
        &= \sum_{\ell \in [k]} v_i(g_\ell) + v_i(i_{n'}) \\
        &\geq \sum_{\ell \in [k'-1]} v_i(g'_{\ell+1}) + v_i(i_{n'}) \\
        &= v_i(B_{i_{n'}}) > \alpha \cdot \mu_i.
    \end{align*}
\end{proof}


\begin{restatable}{theorem}{thmnOverThree}\label{thm:nover3copiesMMS}
    Given an ordered instance $\GenInstance$ with $n^* >5$ agents and $\alpha \leq 4/5$, 
    $$\bagfillRR(\mathtt{T-reduce(\I,\alpha)},\alpha)$$ returns an $\alpha$-MMS allocation with at most $\floor{n^*/3}$ many distinct copies.
\end{restatable}

\begin{proof}
    Let $n$ be the number of agents after applying $\mathtt{T-reduce(\I,\alpha)}$. Since $T^\alpha$ and $R^\alpha_k$ are valid reductions for all $k < |M|/n^*$, all agents who receive a bag during the $\mathtt{T-reduce(\I,\alpha)}$ receive at least $\alpha$ fraction of their MMS value and the MMS value of the remaining agents do not decrease after this process. All the agents who receive a bag during the round robin phase of $\bagfillRR$ get at least $\alpha$ fraction of their value. Let agent $i$ be an agent who receives bag $B_j$ after the round robin phase. Let $g$ be the good that was duplicated and added to $B_j$ right before allocating it to agent $i$. Since $g \leq n'$, $v_i(g) \geq v_i(n')$. By Lemma \ref{lem:duplic-suffice-2}, $v_i(B_j \cup \{g\}) \geq \alpha \cdot \mu_i$. Therefore, the final allocation is $\alpha$-MMS. 

    Now we prove at most $\floor{n^*/3}$ distinct copies are used. Let $k=n^*-n$ be the number of agents who receive a bag in $\mathtt{T-reduce(\I,\alpha)}$. Only when applying $T^\alpha$, new copies are introduced and one copy per three agents that are removed. Furthermore, the goods that are allocated during $T^\alpha$ along with their copies are removed from the instance. Therefore, they cannot be copied more than once. So during $\mathtt{T-reduce(\I,\alpha)}$, at most $\floor{k/3}$ distinct copies are used. In the duplication phase of $\bagfillRR$, we duplicate items $1,2, \ldots, n'$ once. By Corollary \ref{cor:halfRemaining}, $n' \leq \floor{n/3}$. Therefore, in total at most $\floor{n^*/3}$ distinct copies are used.
\end{proof}

\bibliographystyle{plainnat}
\bibliography{refs}

\appendix \label{app:chores}

\section{Full MMS by Removing $n-2$ Bads for Chore Allocation} \label{sec:full-n-2-chores}

Consider the case where for an instance $\GenInstance$, $i\in N$ and $j\in M$, $v_{ij}$ is the cost of performing chore $j$. We assume agents are additive, that is the cost of bundle $S\subseteq M$ for agent $i\in N$ equals $v_i(S)=\sum_{j\in S} v_{ij}.$ Agents want to minimize their cost. 

Recall that we denote the collection of all allocations of the set $M$ among $n$ agents by $\A_n(M)$. The \emph{maximin share}\footnote{Although the maximin share is defined with a min-max type formula, if one considers chores to be goods with negative value, then a max-min formulation is equivalent.} (MMS) of agent $i$ is the upper bound on the cost agent $i$ may exert by splitting the set $M$ into $n$ bundles, getting the worst out of them. Formally:

\begin{definition}[Maximin share (MMS)]
For an instance $\GenInstance$ with $n$ agents, the \emph{maximin share} (MMS) of agent $i$ with cost function $v_i$, denoted by $\mu_i^n (M)$, is given by 
$$
\mu_i^n (M) = \min_{A \in \A_n(M)}\max_{S \in A} v_i(S).
$$
\end{definition}

For every agent $i$, we normalize $\mu_i^n(M)=1$, thus $v_i(M)\le n$ for every $i\in N$. We next present Algorithm $\matchnfillchores$, which operates in an analogous way to Algorithm $\matchnfill$ from \Cref{sec:full-n-2}. %\ygc{\Cref{sec:full-n-2}}


\begin{algorithm}[htb]
\caption{$\matchnfillchores(\GenInstance)$: 
\\ \textbf{Input:} $n$ additive cost functions $\{v_i\}_{i\in N}$ over items in $M$. 
\\ \textbf{Output:} Allocation $A = (A_1, \ldots, A_n)$ with at least $m-n+2$ allocated items.}
\label{alg:matchnfillchores}
\SetAlgoLined
\DontPrintSemicolon
\LinesNumbered

%\While{$\exists i\in N, j\in M$ s.t. $v_{ij}\ge 1$}{ 
%    $A_i\gets \{j\}$\;
%    $N\gets N\setminus\{i\}$, $M\gets M\setminus \{j\}$\;
%}

Let $P=(P_1,P_2,\ldots, P_n)$ be the MMS partition of agent 1\; 
Consider the bipartite graph $G=(V=N\times P,E)$, where $E=\{(i,P_j)\ : \ v_i(P_j)\le 1\}$\;

\If{there is a perfect matching between $N$ and $P$ in $G$}{ 
    Allocate each $i\in N$ its match in the perfect matching\;
}
\Else{
    Let $P'\subset P$ be the minimal set of bundles violating Hall's theorem ($|P'|>1$ since $1$ is matched to every node in $P$)\;
    Consider some $P_j\in P'$ and let $\hat{P}=P'\setminus\{P_j\}$\;
    Find a matching between bundles in $\hat{P}$ and agents in $N$\;
    \ForEach{$i\in N$ matched to some $P_j\in \hat{P}$}{
        $A_i\gets P_j$\;
    }
    Let $\tilde{N}$ be the set of unallocated agents and $\tilde{M}=M\setminus\left(\bigcup_{i\in N\setminus\tilde{N}}A_i\right)$ the set of unallocated items\;
    Allocate agents in $\tilde{N}$ using procedure $\bagfillremove((\tilde{N},\tilde{M},\{v_i\}_{i\in \tilde{N}}))$\;
}
\end{algorithm}



\begin{algorithm}[htb]
\caption{$\bagfillremove(\GenInstance)$: 
\\ \textbf{Input:} $n$ additive cost functions $\{v_i\}_{i\in N}$ over items in $M$. 
\\ \textbf{Output:} Allocation $A = (A_1, \ldots, A_n)$ with at least $|M|-|N|+1$ allocated items.}
\label{alg:bagfill_remove}
\SetAlgoLined
\DontPrintSemicolon
\LinesNumbered
$t\gets 0$, $B_t\gets \emptyset$\;
$N_0\gets N$, $M_0\gets M$\;
\For{items $j\gets 1$ \KwTo $m$}{
    \If{$\exists i\in N_t$ such that $v_{i}(B_t\cup \{j\}) < 1$}{
        $B_t\gets B_t\cup \{j\}$\;
    }
    \Else{
        Let $i_t\in \argmin_{i\in N_t} v_i(B_t)$\;
        $A_{i_t}\gets B_t$\;
        $t\gets t+1$, $B_t\gets \emptyset$\;
        $N_t\gets N_{t-1}\setminus\{i_{t-1}\}$, $M_t\gets M_{t-1}\setminus (B_{t-1}\cup\{j\})$\;
    }
}

Let $i_t$ be the last remaining agent in $N_t$. Set $A_{i_t}\gets B_t$.\;
\end{algorithm}

We first show that $\bagfillremove$ never gives an agent more than their MMS value, and throws away at most $|N|-1$ items.

% \ygc{How about replacing $|N|$ with $n$ and $|M|$ with $m$ for consistency?}
\begin{lemma} \label{lem:bagfillremove}
    Assuming that for every $i\in N$, $v_i(M)\le |N|$, $\bagfillremove$ outputs and allocation such that $v_i(A_i)\le 1$ for every $i\in N$, and allocates at least $|M|-|N|+1$ items.
\end{lemma}
\begin{proof}
%Clearly, $\bagfillcopy$ copies  $|N|-1$ different items, as 
Every time an agent is allocated, except the last agent, we throw away the current item considered. Thus, we allocate all items but $|N|-1$. Since every time we allocate, we remove a set of items that is worth at least 1 for all agents from $M_t$, we maintain the invariant that $v_i(M_t)\le |N|-t$. Thus, when we reach the last agent, they get a bundle of cost at most 1.
\end{proof}

We now show that at most $n-2$ items should be discarded.
%We now show that $n-2$ distinct copies suffice. \afc{wrong, chores removed, not copies added,}

\begin{lemma} \label{lem:nmin2_lemma_chores}
    For any fair division instance of chores with additive costs, at least $m-n+2$ items can be allocated while maintaining a full MMS allocation.
\end{lemma}
\begin{proof}
    As in the proof of \Cref{lem:nmin2_lemma}, if all agents are allocated by a perfect matching in $G$, and then we don't need to throw away any items. If we match a subset of the agents by bundles in $G$, then the remaining agents do not have any edges to the allocated bundles, which means that if $k$ agents were allocated, the rest of the agents have a cost of at most $|N|-k$ for the remaining bundles. Thus, we can apply \Cref{lem:bagfillremove} and get that at most $|N|-k-1$ items are thrown away. Since $k\ge 1$, we get that the number of allocated items is at least $|M|-|N|+2$.
\end{proof}

\section{Missing Proofs of \Cref{sec:gneneralVals}}\label{apx:proofsGeneralVals}
\begin{proof}[Proof of \Cref{lem:bound_indivcopies}]
    For any agent $i$, let $P^i= (P^i_1,\dots,P^i_n)$ be his MMS partition.
    Consider the MMS allocations $A=(A_1,\dots,A_n)$ where $A_i \sim U(\{P^i_1,\dots, P^i_n\})$. Namely, $i$'s bundle, $A_i$, is one of his MMS bundles chosen uniformly at random, independent of other choices.
    We use the balls-and-bins analogy \cite{raab1998balls}, and consider the process of constructing $A$ as throwing $n$ balls (the agents) into $n$ bins (the index in an MMS partition). We are interested in the  number of balls per bin, so we ignore the fact that the balls are distinct.

    Fix some good $g \in M$ and assume without loss that $g \in P^i_1$ for any $i \in N$. 
    The event in which $g$ is copied at least $k$ times for the allocation $A=(A_1,\dots,A_n)$, is exactly the event in which the bin corresponding to index $1$ has at least $k$ balls.
    A simple analysis shows that, 
    $$
    \Pr[g \text{ was copied at least }k \text{ times}] \le \binom{n}{k}\left(\frac{1}{n}\right)^k \le  \left(\frac{e}{k}\right)^k,
    $$
    where the last transition follows from Stirling's approximation.

    If we plug in $k = \frac{3 \ln m}{\ln \ln m}$, we get
    \begin{align*}
        % \Pr[g \text{ was copied at least }k \text{ times}]
        % &\le
        \left(\frac{e}{k}\right)^k %\\
        &=
        \left(\frac{e \ln \ln m}{3 \ln m}\right)^{\frac{3 \ln m}{\ln \ln m}} \\
        &\le
        \exp\left(\frac{3 \ln m}{\ln \ln m}\ln\left(\frac{e \ln \ln m}{3 \ln m}\right) \right)\\
        &\le
        \exp\left(\frac{3 \ln m}{\ln \ln m}\left(\ln \ln \ln m - \ln \ln m\right) \right)\\
        &=
        \exp\left(-3 \ln m + \frac{3 \ln m \ln \ln \ln m}{\ln \ln m} \right).
    \end{align*}

For large enough $m$ we have $\left(\frac{e}{k}\right)^k < \exp(-2 \ln m) = \frac{1}{m^2}$.

Applying the union bound we have,
\begin{align*}
    \Pr\left[\text{there exists a good with more than } \frac{3 \ln m}{\ln \ln m} \text{ copies}\right] < m \cdot \frac{1}{m^2} = \frac{1}{m},
\end{align*}
which completes the proof.
\end{proof}


\section{Missing Proofs of \Cref{sec:fulladditive}}\label{sec:fulladditive_proofs}

\begin{proof}[Proof of \Cref{lem:bagfillcopy}]
%Clearly, $\bagfillcopy$ copies  $|N|-1$ different items, as 
Every time an agent is allocated, except the last agent, we duplicate the last item that was put in his bag and put the duplicate as the first item in the next bag. Since every item's value is less than $1$ for all agents, every allocated bundle contains at least two items. Therefore, we duplicate a different item each time, and in total, $|N|-1$ items. We now show that upon termination, every agent gets at least their MMS value.

We show that for every $t$, for every $i\in N_t$, $v_i(M_t)\ge |N|-t$.  We prove by induction on $t$. Obviously, it's true for $t=0$. Assume this is true for $t-1\in [|N|-1]$, and consider the iteration when agent $i_{t-1}$ gets assigned bundle $A_{i_{t-1}}$. Consider some agent $i\in N_t$. By the inductive assumption, we have that for every $i'\in N_{t-1}$, $v_{i'}(M_{t-1})\ge |N|-t+1.$ As $N_t\subset N_{t-1}$, we have that 
\begin{eqnarray}
    v_i(M_{t-1})\ge |N|-t+1. \label{eq:assumption}    
\end{eqnarray}   
    
Now consider the bundle $B_{t-1}$ just before some item $j$ was added to it, causing agent $i_{t-1}$'s value to rise above 1. Since before adding $j$, the value each agent in $N_{t-1}$ assigned to $B_{t-1}$ was smaller than 1, we also have 
\begin{eqnarray}
    v_i(B_{t-1})< 1. \label{eq:step}    
\end{eqnarray}

This implies that
\begin{eqnarray*}
    v_i(M_{t})\ =\ v_i(M_{t-1}\setminus B_{t-1})\ =\ v_i(M_{t-1})- v_i(B_{t-1}) > |N|-t,
\end{eqnarray*}
where the last inequality follows Equations~\eqref{eq:assumption},~\eqref{eq:step}.
\end{proof}

% \begin{proof}[Proof of \Cref{cor:mover3additive}]
%     Since $4\ceil{n/3}\le 4n/3+4 = n(1+ \frac{1}{3} + \frac{4}{n})$, applying Lemma~\ref{lem:1ood_reduction} with $\alpha=\frac{1}{3} + \frac{4}{n}$ implies that the number of copies sufficient in order to produce a full MMS allocation is at most
%     \begin{eqnarray*}
%         \floor*{\left(\frac{1}{3} + \frac{4}{n}\right)m} + \ceil*{\frac{(1+\frac{1}{3} + \frac{4}{n})^2}{n-1-\frac{1}{3} - \frac{4}{n}}m} &\le&\floor*{\frac{m}{3}} +\ceil*{\frac{4}{n}m}  + \ceil*{\frac{(1+\frac{1}{3} + \frac{4}{n})^2}{n-\frac{4}{3} - \frac{4}{n}}m}\\
%         &=& \floor*{\frac{m}{3}}(1+O(1/n)) \\
%         &=& \floor*{\frac{m}{3}}(1+o(1)). %\qedhere
%     \end{eqnarray*}
% \end{proof}
%\section{Proof of \Cref{cor:mover3additive}}\label{sec:proofofmover3}
    





\iffalse
\section{Missing Proofs of Section \ref{sec:approx}}\label{apx:approx}
\reducK*
\begin{proof}
    Let $P=(P_1, P_2, \ldots, P_n)$ be an MMS partition of agent $i$ for the original instance. By the pigeonhole principle, there exists $j \in [n]$ such that $|P_j \cap [kn+1]| > k$. Without loss of generality, let us assume $|P_n \cap [kn+1]| > k$ and $g_1, g_2, \ldots g_{k+1}$ are $k+1$ distinct goods in $P_n \cap [kn+1]$ such that $g_1 \leq g_2 \leq \ldots \leq g_{k+1}$. For $j \in [k+1]$, let $kn-k+j \in P_{a_j}$. Now for all $j \in [k+1]$, swap the goods $kn-k+j$ and $g_j$; i.e., iteratively do the following:
    \begin{itemize}
        \item for all $j \in [k+1]$: $P_{a_j} \leftarrow P_{a_j} \setminus \{kn-k+j\} \cup \{g_j\}$ and $P_n \leftarrow P_n \setminus \{g_j\} \cup \{kn-k+j\}$.
    \end{itemize}
    Let $P'=(P'_1, P'_2, \ldots, P'_n)$ be the final partition. Since $g_1, \ldots, g_{k+1}$ are $k+1$ goods in $[kn+1]$ in increasing order of index, $g_j \leq kn-k+j$. Thus, $v_i(g_j) \geq v_i(kn-k+j)$ and  after each of these swaps, the value of $P_{a_j}$ cannot decrease. Therefore, for all $j \in [n-1]$, $v_i(P'_j) \geq v_i(P_j) \geq \mu^n_i(M)$ and $(P'_1, \ldots, P'_{n-1})$ is a partition of a subset of $M \setminus K$ into $n-1$ bundles of value at least $\mu^n_i(M)$ for $i$. The lemma follows.
\end{proof}

\reduceTwoGoods*
\begin{proof}
    To prove the lemma, it suffices to find a partition of (a subset of) $M \setminus \{g_1,g_2\}$ into $n-1$ bundles, such that the value of agent $i$ for each bundle is at least $\mu^n_i(M)$. We call such a partition a certificate. Let $P=(P_1, \ldots, P_n)$ be an MMS partition of agent $i$ in the original instance $\I$. Without loss of generality, assume $\{g_1,g_2\} \subseteq P_1 \cup P_2$. If $\{g_1,g_2\} \subseteq P_1$, then $(P_2, \ldots, P_{n-1}, P_n)$ is a certificate. Otherwise, let $g_i \in P_i$ for $i \in [2]$. We have 
    \begin{align*}
        v_i(P_1 \cup P_2 \setminus \{g_1,g_2\}) &= v_i(P_1) + v_i(P_2) - (v_i(g_1) + v_i(g_2)) \\
        &\geq 2\mu^n_i(M) - \mu^n_i(M) \geq \mu^n_i(M).
    \end{align*}
    Hence $(P_1 \cup P_2 \setminus \{g_1,g_2\}, P_3, \ldots, P_n)$ is a certificate.
    % \mfc{Shall we add that $P_1 \cup P_2 \setminus \{g_1,g_2\}$ is non-empty due to the other valid reduction?}
\end{proof}

%\newReduction*

%\lemValidS*


%\fourOverFive*

%\validT*


\obsSizeThree*
\begin{proof}
    Since $\I$ is $S$-irreducible, $v_i(\{j,n+j\}) \leq v_i(\{1,n+1\})<\alpha \cdot \mu_i$ for all $i \in N$. Thus, if $B_j$ was allocated to some agent, $|B_j|>2$. 
\end{proof}

\subsection{\boldmath $4/5$-MMS with Copies}\label{apx:fourOverFive}

\begin{lemma}
    For an arbitrary ordered instance $\I$ and $\alpha \leq 1$, let $\I' = \mathtt{T-reduce(\I,\alpha)}$. Then for all agents $i \in N'$, $\mu_i(\I') \geq \mu_i(\I)$. 
\end{lemma} 
\begin{proof}
    The lemma follows from Lemma \ref{lem:validT} and Corollary \ref{cor:validK}.
\end{proof}

For the rest of this (sub)-section, let $\alpha \leq 4/5$, $\I$ be an $R^\alpha$-irreducible and $T^\alpha$-irreducible instance, and $\mu_i$ be $\mu_i(\I)$. Let $n$ be the number of agents right before the round robin phase of $\bagfillRR(\I,\alpha)$, $\B = \{B_1, \ldots, B_{a-1}\}$ be the set of bags allocated during this phase, and $n' = n-a+1$ be the number of remaining agents after this phase.

\begin{lemma}\label{obs:size3-2}
    For all bags $B_j \in \B$ with $j > 2$, $|B_j| \geq 3$.
\end{lemma}
\begin{proof}
    Towards contradiction, assume for some $j>2$, $|B_j| \leq 2$ and $B_j$ is allocated to agent $i$. Note that since $\I$ is $R_0^\alpha$-irreducible, $v_a(1) < \mu_a$ for all agents $a$ and hence $|B| \geq 2$ for all $B \in \B$. So for all $\ell \in [n]$, we have $\{\ell,n+\ell\} \in B_\ell$ and $B_j = \{j,n+j\}$. Note that for all $j'<j$, $v_i(\{j',n+j'\}) > v_i(\{j,n+j\}) \geq \alpha \cdot \mu_i$. Hence, for all $j'<j$, $|B_{j'}|=2$. Let $a_1,a_2,a_3$ be the three agents who received $B_1, B_2, B_3$ respectively. Note that the precondition of Rule $T^\alpha$ is met for these agents which is a contradiction with $\I$ being $T^\alpha$-irreducible.    
\end{proof}
\begin{observation}\label{obs:1-2}
    For $n\geq2$, $\{B_1,B_2\} \subseteq \B$ and $v_i(B_j) \leq 3\alpha/2 \cdot \mu_i$ for all $i \in N$ and $j \in [2]$.
\end{observation}
\begin{proof}
    Since the instance is $R^\alpha_0$-irreducible, $v_i(2) \leq v_i(1) < \alpha \cdot \mu_i$. Therefore, $|B_j| \geq 2$. Let $g$ be the last good added to $B_j$. Since $g \geq n+1$, $v_i(g) \leq v_i(n+1) < \alpha/2 \cdot \mu_i$. We have 
    \begin{align*}
        v_i(B_j) &= v_i(B_j \setminus \{j\}) + v_i(B_j) < \alpha \cdot \mu_i + \alpha/2 \cdot \mu_i = 3\alpha/2 \cdot \mu_i. 
    \end{align*}
    Now assume $B_2 \notin \B$. Note that for all $B \notin \B$, $v_i(B) < \alpha \cdot \mu_i$. Otherwise, $B$ should have been allocated during the round robin phase. Therefore
    \begin{align*}
        n \cdot \mu_i &\leq v_i(M) \\
        &= \sum_{j}v_i(B_j) \\
        &< 3\alpha/2 \cdot \mu_i + (n-1)\alpha \cdot \mu_i \\
        &\leq (4/5n+2/5) \cdot \mu_i, \tag{$\alpha \leq 4/5$} 
    \end{align*}
    which is a contradiction with $n \geq 2$.
\end{proof}
\begin{lemma}\label{lem:many-valuable-bags-2}
    Let $i$ be a remaining agent after the round robin phase of $\bagfillRR(\I,\alpha)$ (if any). Then $|\{B \in \B \mid v_i(B) > \alpha \cdot \mu_i\}| \geq 2n/3$ or $n\leq 5$.
\end{lemma}
\begin{proof}
    Assume $n>5$. Let $\B^+ = \{B \in \B \mid v_i(B) > \alpha \cdot \mu_i\}$. By definition, for all $B \in \B \setminus \B^+$, $v_i(B) \leq \alpha \cdot \mu_i$. Let $\bar{\B}$ be the set of $n-|\B|=n'$ many remaining bags after the round robin phase. For all $B \in \bar{\B}$, $v_i(B) < \alpha \cdot \mu_i$. Otherwise, $B$ should have been allocated to $i$ (or some other agent who values it at least $\alpha$ fraction of their MMS value). We consider two cases:
    \paragraph{\boldmath Case 1: $v_i(B_2) \leq 4\alpha/3 \cdot \mu_i$.}
    For all $B \in \B \setminus \{B_1,B_2\}$, Lemma \ref{obs:size3-2} implies $|B| \geq 3$ and Lemma \ref{lem:upper-bound} implies $v_i(B) \leq 4\alpha/3 \cdot \mu_i$. Also by Observation \ref{obs:1-2}, $v_i(B_1) \leq 3\alpha/2 \cdot \mu_i$ and we have $v_i(B_2) \leq 4\alpha/3 \cdot \mu_i$. Therefore
    \begin{align*}
        \sum_{B \in \B^+} v_i(B) &\leq 4\alpha/3 \cdot \mu_i (|\B^+|-1) + 3\alpha/2 \cdot \mu_i \\
        &\leq 16|\B^+|/15 + 2/15. \tag{$\alpha \leq 4/5$}
    \end{align*}
    
    We have
    \begin{align*}
        n \cdot \mu_i &\leq v_i(M) \\
        &= \sum_{B \in \B^+} v_i(B) + \sum_{B \in \B \setminus \B^+} v_i(B) + \sum_{B \in \bar{\B}} v_i(B) \\
        &\leq 16|\B^+|/15 + 2/15 + (|\B| - |\B^+|) \alpha \cdot \mu_i + (n - |\B|) \alpha \cdot \mu_i \\
        &\leq \mu_i (2/15 + 4n/5 + 4|\B^+|/15). \tag{$\alpha \leq 4/5$}
    \end{align*}
    We obtain
    \begin{align*}
        n/5 &\leq 4|\B^+|/15 + 2/15.
    \end{align*}
    Therefore, $|\B^+| \geq 3(n-2)/4 \geq 2n/3$.
    \paragraph{\boldmath Case 2: $v_i(B_2) > 4\alpha/3 \cdot \mu_i$.} If $|B_2| \geq 3$, then by Lemma \ref{lem:upper-bound}, $v_i(B_2) \leq 4\alpha/3 \cdot \mu_i$. Also if $|B_2|=1$, $v_i(B_2) < \alpha \cdot \mu_i$ since the instance is $R^\alpha_0$ irreducible. Therefore, $|B_2|=2$. It implies $|B_1|=2$ and we have $B_1 = \{1,n+1\}$ and $B_2 = \{2,n+2\}$. Let $a$ and $b$ be the agents who received $B_1$ and $B_2$. If $v_i(\{1,n+3\}) \geq \alpha \cdot \mu_i$, then the instance is not $T^\alpha$-irreducible since $a,b,i$ satisfy its precondition. Therefore, $v_i(1)+v_i(n+3) < \alpha \cdot \mu_i$. 
    \begin{align*}
        v_i(1) + v_i(n+2) &\geq v_i(2) + v_i(n+2) = v_i(B_2) > 4\alpha/3 \cdot \mu_i.
    \end{align*}
    Therefore, $v_i(n+3) < v_i(n+2) - \alpha/3 \cdot \mu_i < (\alpha/2 - \alpha/3) \cdot \mu_i = \alpha/6 \cdot \mu_i$.
    Now fix an index $j>2$ and let $g$ be the last good added to $B_j$. We have 
    \begin{align*}
        v_i(B_j) &= v_i(B_j \setminus \{g\}) + v_i(g) \\
        &< \alpha \cdot \mu_i + \alpha/6 \cdot \mu_i \\
        &< \mu_i. \tag{$\alpha \leq 4/5$}
    \end{align*}
    Therefore, $\B^+ = \{B_1,B_2\}$ and we have
    \begin{align*}
        n \cdot \mu_i &\leq v_i(M) \\
        &= \sum_{B \in \B^+} v_i(B) + \sum_{B \in \B \setminus \B^+} v_i(B) + \sum_{B \in \bar{\B}} v_i(B) \\
        &\leq 3\alpha \cdot \mu_i + (|\B| - 2) \alpha \cdot \mu_i + (n - |\B|) \alpha \cdot \mu_i \\
        &\leq \mu_i (4/5 + 4n/5). \tag{$\alpha \leq 4/5$}
    \end{align*}
    We obtain that $n \leq 4$ which is a contradiction with $n>5$.
\end{proof}
    
\begin{corollary}[of Lemma \ref{lem:many-valuable-bags-2}]\label{cor:halfRemaining}
     $n' \leq \floor{n/3}$ for $n \geq 6$.        
\end{corollary}

\begin{lemma}\label{lem:duplic-suffice-2}
    If $n \geq 6$, for all the remaining agents $i$ after the round robin phase of $\bagfillRR(\I,\alpha)$, and all the remaining bags $B_j$, we have $v_i(B_j \cup \{n'\}) \geq \alpha \cdot \mu_i$.
\end{lemma}
\begin{proof}
    By Lemma \ref{lem:order}, the bags $B_1, \ldots, B_{n-n'}$ are allocated during the round robin phase. 
    By Lemma \ref{lem:many-valuable-bags-2}, there exists at least $2n/3 \geq n'$ many bags $B_{i_1}, \ldots, B_{i_{n'}}$ for $i_1 < i_2 < \ldots < i_{n'} \leq n-n'$ that were allocated to other agents during the round robin phase and $v_i(B_{i_\ell}) > \alpha \cdot \mu_i$ for all $\ell \in [n']$. Note that $i_{n'} \geq n'$. Since $B_j$ have been assigned a good in all but maybe the last round, we have $|B_j| \geq |B_{i_{n'}}|-1$. Let $B_j = \{g_1, \ldots, g_k\}$ and $B_{i_{n'}} = \{g'_1, \ldots, g'_{k'}\}$ for $j= g_1 < \ldots < g_k$ and $i_{n'} = g'_1 < \ldots < g'_{k'}$. For all $\ell \in [\min\{k,k'-1\}]$, since $g_\ell$ is allocated before $g'_{\ell+1}$, $g_\ell < g'_{\ell+1}$ and hence $v_i(g_\ell) \geq v_i(g'_{\ell+1})$. We have
    \begin{align*}
        v_i(B_j \cup \{n'\}) &= v_i(B_j) + v_i(n') \\
        &\geq v_i(B_j) + v_i(i_{n'}) \\
        &= \sum_{\ell \in [k]} v_i(g_\ell) + v_i(i_{n'}) \\
        &\geq \sum_{\ell \in [\min\{k, k'-1\}]} v_i(g'_{\ell+1}) + v_i(i_{n'}) \\
        &= v_i(B_{i_{n'}}) > \alpha \cdot \mu_i. \qedhere
    \end{align*}
\end{proof}


\thmnOverThree*
\begin{proof}
    Let $n$ be the number of agents after applying $\mathtt{T-reduce(\I,\alpha)}$. Since $T^\alpha$ and $R^\alpha_k$ are valid reductions for all $k < |M|/n^*$, all agents who receive a bag during the $\mathtt{T-reduce(\I,\alpha)}$ receive at least $\alpha$ fraction of their MMS value and the MMS value of the remaining agents do not decrease after this process. All the agents who receive a bag during the round robin phase of $\bagfillRR$ get at least $\alpha$ fraction of their value. Let agent $i$ be an agent who receives bag $B_j$ after the round robin phase. Let $g$ be the good that was duplicated and added to $B_j$ right before allocating it to agent $i$. Since $g \leq n'$, $v_i(g) \geq v_i(n')$. By Lemma \ref{lem:duplic-suffice-2}, $v_i(B_j \cup \{g\}) \geq \alpha \cdot \mu_i$. Therefore, the final allocation is $\alpha$-MMS. 

    Now we prove at most $\floor{n^*/3}$ distinct copies are used. Let $k=n^*-n$ be the number of agents who receive a bag in $\mathtt{T-reduce(\I,\alpha)}$. Only by applying $T^\alpha$, new copies are introduced and one copy per three agents that are removed. Furthermore, the goods that are allocated during $T^\alpha$ along with their copies are removed from the instance. Therefore, they cannot be copied more than once. So during $\mathtt{T-reduce(\I,\alpha)}$, at most $\floor{k/3}$ distinct copies are used. In the duplication phase of $\bagfillRR$, we duplicate items $1,2, \ldots, n'$ once. By Corollary \ref{cor:halfRemaining}, $n' \leq \floor{n/3}$. Therefore, in total at most $\floor{n^*/3}$ distinct copies are used.
\end{proof}
\fi

\section{MMS with copies for $k$-demand valuations} \label{sec:othervaluations}

It is natural to consider other classes of valuation functions, not only arbitrary monotone and additive. 
One of the simplest valuations is that of 1-demand, every agent only wants one good, and every good has a value. Trivially, one round of round-robin will result in an MMS allocation. So, what about $k$-demand? The valuation of a set is the sum of the $k$ most valuable goods in this set. 

It is natural to ask if the algorithm $\bagfillcopy$ is relevant, unfortunately, this does not work. Consider the following input: All $n$ agents with identical valuation functions $v_i=v$, $v$ is 2-demand: $v(g)\geq 0$ for all goods, and $v(S) = \max_{g,h\in S} v(g)+v(h)$ for $|S|\geq 2$. 

We now construct an example on which algorithm $\bagfillcopy$ fails, we define goods and valuations as follows: there are $n-1$ goods $g_1,\ldots,g_{n-1}$ for which $v(g_j)=(1-\epsilon)/2$, $n-1$ goods $h_1,\ldots,h_{n-1}$ for which $v(h_j)=(1+\epsilon)/2$, and two additional goods $x_1$, $x_2$ for which $v(x_j)=1/2$. One possible MMS allocation is $\{g_1,h_1\} \ldots \{g_{n-1},h_{n-1}\}, \{x_1,x_2\}$, and the MMS value is one.

Now consider [arbitrarily] adding  goods in the order $g_1, \ldots, g_{n-1}$ to a bag $B$, at this point in time $v(B)<1$. Next, add $x_1$ and $x_2$ to $B$, all agents have value 1 for the bag, if the bag is assigned to one of the agents not enough goods will remain. Duplicating the good $x_2$ and all of the $h_j$ will not suffice to get an MMS allocation for the remaining $n-1$ agents. 

This suggests a variant of algorithm $\bagfillcopy$ for $k$-demand valuations. Upon processing a new good $g$, and adding it to $B$, don't assign the entire bag $B$ to an agent with $v(B)\geq 1$. Instead, choose some set $S\subseteq B$, $|S|=k$ with $v(S)\geq 1$. Some such set $S$ must exist (as $v(B)\geq 1$ and $v$ is $k$-demand). Moreover, the last good added to the bag $g$ must be in $S$, otherwise the bag $B'=B\setminus \{g\}\subset B$, considered just prior to the arrival of $g$, would also have had value $\geq 1$. Give one of the agents with $v(B)\geq 1$ some $k$-subset of $B$, $S\subset B$, for which $v(S)\geq 1$. Return all goods in $B\setminus S$ to the pool of goods yet to be processed, and assign $B=\{g\}$, the last good added to that triggered this allocation, this creates a copy of $g$. Also, such duplications can be done at most $n-1$ times and no good is duplicated more than once. 

To Summarize: 

\begin{observation}
    For $k$-demand valuations $n-1$ distinct copies suffice to achieve an MMS allocation.
\end{observation}


\section{Example for Algorithm $\bagfillcopy$}\label{sec:examples}
\begin{figure}[tbh]
%\begin{table}
\begin{tabular}{|c|c|c|c|c|}
\hline
 & Agent 1 & Agent 2 & Agent 3 & Agent 4 \\ \hline
$g_1$ & 0.2     & 0.3     & 0.8     & 0.1     \\ 
$g_2$ & 0.2     & 0.5     & 0.3      & 0.1     \\ 
$g_3$ & 0.7     & 0.3     & 0.2     & 0.8     \\ 
$g_4$ & 0.3     & 0.2     & 0.3     & 0.2     \\ 
$g_5$ & 0.3     & 0.8     & 0.1     & 0.2     \\ 
$g_6$ & 0.5     & 0.5     & 0.1     & 0.7     \\ 
$g_7$ & 0.1     & 0.1     & 0.5     & 0.1     \\ 
$g_8$ & 0.1     & 0.1     & 0.7     & 0.2     \\ 
$g_9$ & 0.8     & 0.3       & 0.3     & 0.8     \\ 
$g_{10}$ & 0.2     & 0.8     & 0.1     & 0.3     \\ 
$g_{11}$ & 0.2     & 0.2     & 0.9     & 0.3     \\ 
$g_{12}$ & 0.7     & 0.3     & 0.1     & 0.5     \\ \hline
\end{tabular}
%\end{table}
\caption{Running example for $\bagfillcopy$. There is an MMS allocation without copies for this instance. All MMS values are one. For agent 1 we have $v_1(\{g_1, g_2, g_3\})=1.1$, $v_1(\{g_4,g_5,g_6\})=1.1$, $v_1(\{g_7, g_8, g_9\})=1$ and $V_1(\{g_{10}, g_{11}, g_{12} \})=1.1$. For agent 2 we have $v_2(\{g_1, g_2, g_3\})=1.1$, $v_2(\{g_4,g_5\})=1$, $v_2(\{g_6.g_7, g_8, g_9\})=1$ and $V_2(\{g_{10}, g_{11}, g_{12} \})=1.1$. By construction the MMS partitions consist of contiguous runs of $g_1,\ldots,g_{12}$.}
\end{figure}

Consider the case when we add items in order $g_4, g_7, g_9, g_1, g_2, g_{10}, g_{12}, g_7, g_5, g_3,g_8 \ldots$
\begin{itemize}
\item For $B=\{g_4, g_7\}$, $v_1(B)=0.4$, $v_2(B)=0.3$, $v_3(B)=0.8$, and $v_4(B)=0.3$, no agent has attained the MMS value for this set $B$.
\item For $B=\{g_4, g_7, g_9\}$, $v_1(B)=1.2$, $v_2(B)=0.8$, $v_3(B)=1.1$, and $v_4(B)=1.1$, arbitrarily choose agent 1 and assign $B$ to agent 1, set $B$ to be a copy of the last good added: $B=\{g_9\}$.
\item For $B=\{g_9, g_1, g_2\}$, we have $v_2(B)=1.1$, $v_3(B)=0.8$, $v_4(B)=1$, arbitrarily choose agent 2 and assign $B$ to agent 2, set $B=\{g_2\}$.
\item For $B=\{g_2, g_{10}, g_{12}, g_7\}$, we have $v_3(B)=1.2$, $v_4(B)=1$ and assign $B$ to agent 4, set $B=g_7$.
\item For $B=\{g_7, g_5, g_3, g_8\}$ we have that $v_3(B)=1.5$ and assign $B$ to agent 3. 
\end{itemize}
Note that not all goods have been allocated, we could add the remaining goods to any of the agents, arbitrarily. 

The execution of $\bagfillcopy$ depends on the order in which the items are added.

\end{document}
