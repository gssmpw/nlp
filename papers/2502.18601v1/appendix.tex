\documentclass[12pt, double]{article}
\usepackage{times}
\usepackage{algorithm}
\usepackage{algorithmic}
\usepackage{wrapfig}
\usepackage{verbatim}
\usepackage{amsmath}
\usepackage{graphicx}
\usepackage{subcaption}
\usepackage{srcltx}
\usepackage{color}
\usepackage{lineno}
\usepackage{dirtytalk}
\usepackage{amsthm}
\usepackage{authblk}
\usepackage{listings}
\usepackage{tikz}
\usepackage{longtable}
\usepackage{comment}
\usetikzlibrary{shapes.geometric, arrows}
\usepackage{epsfig,amsfonts,amssymb}
\usepackage{adjustbox}
\usepackage{multirow}
\usepackage{setspace}
\usepackage{hyperref}
\usepackage{comment}
\usepackage{xcolor}
\usepackage{subfig}

% Page layout adjustments
\pagestyle{empty}
\setlength{\topmargin}{.1in}
\addtolength{\textwidth}{1.5in}
\addtolength{\oddsidemargin}{-0.75in}
\addtolength{\evensidemargin}{-0.75in}
\addtolength{\marginparwidth}{-0.5in}
\addtolength{\textheight}{1in}
\renewcommand{\floatpagefraction}{0.8}

% Theorem and definition styles
\newtheorem{theorem}{Theorem}
\newtheorem{definition}{Definition}

% Custom commands for comments
\newcommand{\teddy}[1]{\textcolor{green}{\textbf{Teddy: #1}}}
\newcommand{\todo}[1]{\textcolor{red}{\textbf{#1}}}
\newcommand{\review}[1]{\textcolor{black}{\textbf{#1}}}

\begin{document}

% Title page
\title{\Large Topology-agnostic and Parameter-free Anomaly Detection Using A Convex Hull Volume-based Method: One, Two, and n-dimension Analysis}
\author[1,*]{Uri Itai}
\author[2]{Teddy Lazebnik}

\affil[1]{\textit{Department or Institution Name (To be added)}}
\affil[2]{Department of Cancer Biology, Cancer Institute, University College London, London, UK}
\affil[*]{Corresponding author: \texttt{uri.itai@gmail.com}}
\date{\today}

\maketitle

\thispagestyle{empty}

% Reset page numbering and add header
\pagestyle{myheadings}
\markboth{Draft: \today}{Draft: \today}
\setcounter{page}{1}


Table \ref{table:algos} shows a summary of several popular algorithms used for anomaly detection with their computational complexities. 

\begin{table}[!h]
\centering
\begin{tabular}{l l l l} 
\hline\hline
\textbf{Algorithm} & \textbf{Description} & \textbf{Complexity (n: samples, d: dimensions)} & \textbf{Reference} \\ \hline \hline
Isolation Forest & An ensemble method that isolates observations by randomly selecting features and split values & \(O(n \log n)\) & \\ 
Single Class SVM / Single Class DNN & SVM or deep neural network for identifying outliers by learning a boundary around normal data & \(O(n^2 \times d)\) for SVM & \\
Gaussian Mixture Models (GMM) & Probabilistic model that assumes all data points are generated from a mixture of Gaussian distributions & \(O(n d k^2)\), where \(k\) is the number of components & \\
Local Outlier Factor (LOF) & Density-based method to identify anomalies by comparing local densities & \(O(n \log n)\) & \\
DBSCAN & Density-based clustering algorithm that identifies dense regions & \(O(n \log n)\) & \\
K-means & Partitions data into \(k\) clusters by minimizing variance within clusters & \(O(n d k)\) & \\
Mean Shift & Non-parametric clustering method that finds cluster centers using density gradients & \(O(n^2 d)\) & \\ \hline \hline
\caption{Anomaly detection baseline algorithms with their worst-case complexity.}
\end{tabular}
\label{table:algos}
\end{table}

\bibliography{biblio}
\bibliographystyle{unsrt}

\end{document}

