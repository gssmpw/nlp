%%%%%%%% ICML 2024 EXAMPLE LATEX SUBMISSION FILE %%%%%%%%%%%%%%%%%

\documentclass{article}

% Recommended, but optional, packages for figures and better typesetting:
\usepackage{microtype}
\usepackage{graphicx}
\usepackage{subfigure}
\usepackage{booktabs}
\usepackage{multirow}% for professional tables

% hyperref makes hyperlinks in the resulting PDF.
% If your build breaks (sometimes temporarily if a hyperlink spans a page)
% please comment out the following usepackage line and replace
% \usepackage{icml2025} with \usepackage[nohyperref]{icml2025} above.
\usepackage{hyperref}


% Attempt to make hyperref and algorithmic work together better:
\newcommand{\theHalgorithm}{\arabic{algorithm}}

% Use the following line for the initial blind version submitted for review:
%\usepackage{icml2025}
%\usepackage[preprint]{icml2025}
% If accepted, instead use the following line for the camera-ready submission:
\usepackage[accepted]{icml2025}

% For theorems and such
\usepackage{amsmath}
\usepackage{amssymb}
\usepackage{mathtools}
\usepackage{amsthm}

% if you use cleveref..
\usepackage[capitalize,noabbrev]{cleveref}

%%%%%%%%%%%%%%%%%%%%%%%%%%%%%%%%
% THEOREMS
%%%%%%%%%%%%%%%%%%%%%%%%%%%%%%%%
\theoremstyle{plain}
\newtheorem{theorem}{Theorem}[section]
\newtheorem{proposition}[theorem]{Proposition}
\newtheorem{lemma}[theorem]{Lemma}
\newtheorem{corollary}[theorem]{Corollary}
\theoremstyle{definition}
\newtheorem{definition}[theorem]{Definition}
\newtheorem{assumption}[theorem]{Assumption}
\theoremstyle{remark}
\newtheorem{remark}[theorem]{Remark}

% Todonotes is useful during development; simply uncomment the next line
%    and comment out the line below the next line to turn off comments
%\usepackage[disable,textsize=tiny]{todonotes}
\usepackage[textsize=tiny]{todonotes}

\icmltitlerunning{Homotopy Dynamics}

\begin{document}

\twocolumn[
\icmltitle{Learn Sharp Interface Solution by Homotopy Dynamics}

% It is OKAY to include author information, even for blind
% submissions: the style file will automatically remove it for you
% unless you've provided the [accepted] option to the icml2025
% package.

% List of affiliations: The first argument should be a (short)
% identifier you will use later to specify author affiliations
% Academic affiliations should list Department, University, City, Region, Country
% Industry affiliations should list Company, City, Region, Country

% You can specify symbols, otherwise they are numbered in order.
% Ideally, you should not use this facility. Affiliations will be numbered
% in order of appearance and this is the preferred way.
%\icmlsetsymbol{equal}{*}

\begin{icmlauthorlist}
% \icmlauthor{Firstname1 Lastname1}{equal,yyy}
% \icmlauthor{Firstname2 Lastname2}{equal,yyy,comp}
% \icmlauthor{Firstname3 Lastname3}{comp}
% \icmlauthor{Firstname4 Lastname4}{sch}
% \icmlauthor{Firstname5 Lastname5}{yyy}
% \icmlauthor{Firstname6 Lastname6}{sch,yyy,comp}
% \icmlauthor{Firstname7 Lastname7}{comp}
%\icmlauthor{}{sch}
% \icmlauthor{Firstname8 Lastname8}{sch}
% \icmlauthor{Firstname8 Lastname8}{yyy,comp}
%\icmlauthor{}{sch}
%\icmlauthor{}{sch}
\icmlauthor{Chuqi Chen}{hkust}
\icmlauthor{Yahong Yang}{psu}
\icmlauthor{Yang Xiang}{hkust,szr}
\icmlauthor{Wenrui Hao}{psu}
%\icmlauthor{Madeleine Udell}{stanfordicme,stanfordmse}
\end{icmlauthorlist}

\icmlaffiliation{hkust}{Department of Mathematics, The Hong Kong University of Science and Technology, Clear Water Bay, Hong Kong Special Administrative Region of China}
\icmlaffiliation{psu}{Department of Mathematics, The Pennsylvania State University, State College, PA, USA}
\icmlaffiliation{szr}{Algorithms of Machine Learning and Autonomous Driving Research Lab, HKUST Shenzhen-Hong Kong Collaborative Innovation Research Institute, Futian, Shenzhen, China}
%\icmlaffiliation{yalestat}{Department of Statistics and Data Science, Yale University, New Haven, CT, USA}
% \icmlaffiliation{comp}{Company Name, Location, Country}
% \icmlaffiliation{sch}{School of ZZZ, Institute of WWW, Location, Country}

\icmlcorrespondingauthor{Yahong Yang}{yxy5498@psu.edu}
%\icmlcorrespondingauthor{Firstname2 Lastname2}{first2.last2@www.uk}

% You may provide any keywords that you
% find helpful for describing your paper; these are used to populate
% the "keywords" metadata in the PDF but will not be shown in the document
\icmlkeywords{Physics-informed neural networks, scientific machine learning, Homotopy Dynamics, preconditioning}

\vskip 0.3in
]

% this must go after the closing bracket ] following \twocolumn[ ...

% This command actually creates the footnote in the first column
% listing the affiliations and the copyright notice.
% The command takes one argument, which is text to display at the start of the footnote.
% The \icmlEqualContribution command is standard text for equal contribution.
% Remove it (just {}) if you do not need this facility.

%\printAffiliationsAndNotice{}  % leave blank if no need to mention equal contribution
%\printAffiliationsAndNotice{}
\printAffiliationsAndNotice{} % otherwise use the standard text.



% The \icmltitle you define below is probably too long as a header.
% Therefore, a short form for the running title is supplied here:
% \icmltitlerunning{Submission and Formatting Instructions for ICML 2025}

% \twocolumn[
% \icmltitle{Submission and Formatting Instructions for \\
%            International Conference on Machine Learning (ICML 2025)}

% New commands
\newcommand{\D}{\mathrm{d}}
\newcommand{\vx}{\boldsymbol{x}}
\newcommand{\vT}{\boldsymbol{T}}
\newcommand{\vA}{\boldsymbol{A}}
\newcommand{\vH}{\boldsymbol{H}}
\newcommand{\vl}{\boldsymbol{l}}
\newcommand{\vS}{\boldsymbol{S}}
\newcommand{\vD}{\boldsymbol{D}}
\newcommand{\vK}{\boldsymbol{K}}
\newcommand{\sR}{\mathbb{R}}
\newcommand{\vy}{\boldsymbol{y}}
\newcommand{\vtheta}{\boldsymbol{\theta}}
\newcommand{\R}{\mathbb{R}}
\newcommand{\eps}{\varepsilon}
\newcommand{\Dc}{\mathcal D}
\newcommand{\Kc}{\mathcal K}
\newcommand{\Bc}{\mathcal B}
\newcommand{\Lc}{\mathcal L}
\newcommand{\Wstar}{\mathcal W_\star}
\newcommand{\F}{\mathcal F}
\newcommand{\HL}{H_{L} }
\newcommand{\epsLoc}{\varepsilon_{\textup{loc}}}
\newcommand{\wloc}{w_{\textup{loc}}}
\newcommand{\RLoc}{R_{\textup{loc}}}
\newcommand{\Neps}{\mathcal N_{\epsLoc}(w_\star)}
\newcommand{\N}{\mathcal N}
\newcommand{\dist}{\textup{dist}}
\newcommand{\PL}{P\L$^{\star}$}
\newcommand{\A}{\mathcal A}
\newcommand{\lamMin}{\lambda_{\textup{min}}}
\newcommand{\lamMax}{\lambda_{\textup{max}}}
\newcommand{\nres}{n_\textup{res}}
\newcommand{\nbc}{n_{\textup{bc}}}
\newcommand{\lbfgs}{L-BFGS}
\newcommand{\al}{Adam+\lbfgs}
\newcommand{\aln}{Adam+\lbfgs+NNCG}
\newcommand{\alg}{Adam+\lbfgs+GD}
\newcommand{\bigO}{\mathcal O}

%\usepackage{eqparbox}
\renewcommand{\algorithmiccomment}[1]{\hfill \(\triangleright\) #1}

%\usepackage{multirow}

\newcommand{\pnote}[1]{}
\renewcommand{\pnote}[1]{\textcolor{red}{\textbf{[PR: #1]}}}


% The \icmltitle you define below is probably too long as a header.
% Therefore, a short form for the running title is supplied here:

% It is OKAY to include author information, even for blind
% submissions: the style file will automatically remove it for you
% unless you've provided the [accepted] option to the icml2024
% package.

% List of affiliations: The first argument should be a (short)
% identifier you will use later to specify author affiliations
% Academic affiliations should list Department, University, City, Region, Country
% Industry affiliations should list Company, City, Region, Country

% You can specify symbols, otherwise they are numbered in order.
% Ideally, you should not use this facility. Affiliations will be numbered
% in order of appearance and this is the preferred way.


% this must go after the closing bracket ] following \twocolumn[ ...

% This command actually creates the footnote in the first column
% listing the affiliations and the copyright notice.
% The command takes one argument, which is text to display at the start of the footnote.
% The \icmlEqualContribution command is standard text for equal contribution.
% Remove it (just {}) if you do not need this facility.

%\printAffiliationsAndNotice{}  % leave blank if no need to mention equal contribution
% \printAffiliationsAndNotice{\icmlEqualContribution} % otherwise use the standard text.

\begin{abstract}
% This document provides a basic paper template and submission guidelines.
% Abstracts must be a single paragraph, ideally between 4--6 sentences long.
% Gross violations will trigger corrections at the camera-ready phase.
% This paper studies challenges in training Physics-Informed Neural Networks (PINNs), focusing on the loss landscape's role in the training process. 
% We examine difficulties in minimizing the PINN loss function due to ill-conditioning, which arises from differential operators in the residual term.
% Our experiments include a thorough comparison of optimization methods like Adam, \lbfgs{}, and their combination \al{}, revealing that Adam+L-BFGS consistently outperforms individual methods across various network sizes and problem settings. 
% We also introduce a novel second-order optimizer, NysNewton-CG (NNCG), which significantly improves the solution returned by \al{}.
% On the theoretical side, our paper elucidates the connection between ill-conditioned differential operators and ill-conditioning in the PINN loss, and shows the effectiveness of combining first- and second-order optimization methods.
% Our work provides valuable insights into the optimization challenges of PINNs and proposes improved optimization strategies for PINN training, which could be used in the future to broaden the applicability of PINNs in solving difficult partial differential equation-based problems.
Solving partial differential equations (PDEs) using neural networks has become a central focus in scientific machine learning. Training neural networks for sharp interface problems is particularly challenging due to certain parameters in the PDEs that introduce near-singularities in the loss function. In this study, we overcome this challenge by introducing a novel method based on homotopy dynamics to effectively manipulate these parameters.  From a theoretical perspective, we analyze the effects of these parameters on training difficulty in sharp interface problems and establish the convergence of the proposed homotopy dynamics method. Experimentally, we demonstrate that our approach significantly accelerates convergence and improves the accuracy of sharp interface capturing. These findings present an efficient optimization strategy leveraging homotopy dynamics, offering a robust framework to extend the applicability of neural networks for solving PDEs with sharp interfaces.  

\end{abstract}

\section{Introduction}
\label{sec:introduction}
The business processes of organizations are experiencing ever-increasing complexity due to the large amount of data, high number of users, and high-tech devices involved \cite{martin2021pmopportunitieschallenges, beerepoot2023biggestbpmproblems}. This complexity may cause business processes to deviate from normal control flow due to unforeseen and disruptive anomalies \cite{adams2023proceddsriftdetection}. These control-flow anomalies manifest as unknown, skipped, and wrongly-ordered activities in the traces of event logs monitored from the execution of business processes \cite{ko2023adsystematicreview}. For the sake of clarity, let us consider an illustrative example of such anomalies. Figure \ref{FP_ANOMALIES} shows a so-called event log footprint, which captures the control flow relations of four activities of a hypothetical event log. In particular, this footprint captures the control-flow relations between activities \texttt{a}, \texttt{b}, \texttt{c} and \texttt{d}. These are the causal ($\rightarrow$) relation, concurrent ($\parallel$) relation, and other ($\#$) relations such as exclusivity or non-local dependency \cite{aalst2022pmhandbook}. In addition, on the right are six traces, of which five exhibit skipped, wrongly-ordered and unknown control-flow anomalies. For example, $\langle$\texttt{a b d}$\rangle$ has a skipped activity, which is \texttt{c}. Because of this skipped activity, the control-flow relation \texttt{b}$\,\#\,$\texttt{d} is violated, since \texttt{d} directly follows \texttt{b} in the anomalous trace.
\begin{figure}[!t]
\centering
\includegraphics[width=0.9\columnwidth]{images/FP_ANOMALIES.png}
\caption{An example event log footprint with six traces, of which five exhibit control-flow anomalies.}
\label{FP_ANOMALIES}
\end{figure}

\subsection{Control-flow anomaly detection}
Control-flow anomaly detection techniques aim to characterize the normal control flow from event logs and verify whether these deviations occur in new event logs \cite{ko2023adsystematicreview}. To develop control-flow anomaly detection techniques, \revision{process mining} has seen widespread adoption owing to process discovery and \revision{conformance checking}. On the one hand, process discovery is a set of algorithms that encode control-flow relations as a set of model elements and constraints according to a given modeling formalism \cite{aalst2022pmhandbook}; hereafter, we refer to the Petri net, a widespread modeling formalism. On the other hand, \revision{conformance checking} is an explainable set of algorithms that allows linking any deviations with the reference Petri net and providing the fitness measure, namely a measure of how much the Petri net fits the new event log \cite{aalst2022pmhandbook}. Many control-flow anomaly detection techniques based on \revision{conformance checking} (hereafter, \revision{conformance checking}-based techniques) use the fitness measure to determine whether an event log is anomalous \cite{bezerra2009pmad, bezerra2013adlogspais, myers2018icsadpm, pecchia2020applicationfailuresanalysispm}. 

The scientific literature also includes many \revision{conformance checking}-independent techniques for control-flow anomaly detection that combine specific types of trace encodings with machine/deep learning \cite{ko2023adsystematicreview, tavares2023pmtraceencoding}. Whereas these techniques are very effective, their explainability is challenging due to both the type of trace encoding employed and the machine/deep learning model used \cite{rawal2022trustworthyaiadvances,li2023explainablead}. Hence, in the following, we focus on the shortcomings of \revision{conformance checking}-based techniques to investigate whether it is possible to support the development of competitive control-flow anomaly detection techniques while maintaining the explainable nature of \revision{conformance checking}.
\begin{figure}[!t]
\centering
\includegraphics[width=\columnwidth]{images/HIGH_LEVEL_VIEW.png}
\caption{A high-level view of the proposed framework for combining \revision{process mining}-based feature extraction with dimensionality reduction for control-flow anomaly detection.}
\label{HIGH_LEVEL_VIEW}
\end{figure}

\subsection{Shortcomings of \revision{conformance checking}-based techniques}
Unfortunately, the detection effectiveness of \revision{conformance checking}-based techniques is affected by noisy data and low-quality Petri nets, which may be due to human errors in the modeling process or representational bias of process discovery algorithms \cite{bezerra2013adlogspais, pecchia2020applicationfailuresanalysispm, aalst2016pm}. Specifically, on the one hand, noisy data may introduce infrequent and deceptive control-flow relations that may result in inconsistent fitness measures, whereas, on the other hand, checking event logs against a low-quality Petri net could lead to an unreliable distribution of fitness measures. Nonetheless, such Petri nets can still be used as references to obtain insightful information for \revision{process mining}-based feature extraction, supporting the development of competitive and explainable \revision{conformance checking}-based techniques for control-flow anomaly detection despite the problems above. For example, a few works outline that token-based \revision{conformance checking} can be used for \revision{process mining}-based feature extraction to build tabular data and develop effective \revision{conformance checking}-based techniques for control-flow anomaly detection \cite{singh2022lapmsh, debenedictis2023dtadiiot}. However, to the best of our knowledge, the scientific literature lacks a structured proposal for \revision{process mining}-based feature extraction using the state-of-the-art \revision{conformance checking} variant, namely alignment-based \revision{conformance checking}.

\subsection{Contributions}
We propose a novel \revision{process mining}-based feature extraction approach with alignment-based \revision{conformance checking}. This variant aligns the deviating control flow with a reference Petri net; the resulting alignment can be inspected to extract additional statistics such as the number of times a given activity caused mismatches \cite{aalst2022pmhandbook}. We integrate this approach into a flexible and explainable framework for developing techniques for control-flow anomaly detection. The framework combines \revision{process mining}-based feature extraction and dimensionality reduction to handle high-dimensional feature sets, achieve detection effectiveness, and support explainability. Notably, in addition to our proposed \revision{process mining}-based feature extraction approach, the framework allows employing other approaches, enabling a fair comparison of multiple \revision{conformance checking}-based and \revision{conformance checking}-independent techniques for control-flow anomaly detection. Figure \ref{HIGH_LEVEL_VIEW} shows a high-level view of the framework. Business processes are monitored, and event logs obtained from the database of information systems. Subsequently, \revision{process mining}-based feature extraction is applied to these event logs and tabular data input to dimensionality reduction to identify control-flow anomalies. We apply several \revision{conformance checking}-based and \revision{conformance checking}-independent framework techniques to publicly available datasets, simulated data of a case study from railways, and real-world data of a case study from healthcare. We show that the framework techniques implementing our approach outperform the baseline \revision{conformance checking}-based techniques while maintaining the explainable nature of \revision{conformance checking}.

In summary, the contributions of this paper are as follows.
\begin{itemize}
    \item{
        A novel \revision{process mining}-based feature extraction approach to support the development of competitive and explainable \revision{conformance checking}-based techniques for control-flow anomaly detection.
    }
    \item{
        A flexible and explainable framework for developing techniques for control-flow anomaly detection using \revision{process mining}-based feature extraction and dimensionality reduction.
    }
    \item{
        Application to synthetic and real-world datasets of several \revision{conformance checking}-based and \revision{conformance checking}-independent framework techniques, evaluating their detection effectiveness and explainability.
    }
\end{itemize}

The rest of the paper is organized as follows.
\begin{itemize}
    \item Section \ref{sec:related_work} reviews the existing techniques for control-flow anomaly detection, categorizing them into \revision{conformance checking}-based and \revision{conformance checking}-independent techniques.
    \item Section \ref{sec:abccfe} provides the preliminaries of \revision{process mining} to establish the notation used throughout the paper, and delves into the details of the proposed \revision{process mining}-based feature extraction approach with alignment-based \revision{conformance checking}.
    \item Section \ref{sec:framework} describes the framework for developing \revision{conformance checking}-based and \revision{conformance checking}-independent techniques for control-flow anomaly detection that combine \revision{process mining}-based feature extraction and dimensionality reduction.
    \item Section \ref{sec:evaluation} presents the experiments conducted with multiple framework and baseline techniques using data from publicly available datasets and case studies.
    \item Section \ref{sec:conclusions} draws the conclusions and presents future work.
\end{itemize}
\vspace{-3mm}
\section{Preliminaries}
\label{sec:preliminaries}
\subsection{Formulation of Collaborative Perception}
\label{sec:formulation}

% \vspace{-3mm}
In this section, we formulate collaborative perception and give the pipeline of our CP system. Specifically, let $\mathcal{X}^N$ denote the set of $N$ CAVs in the CP system. CAVs in $\mathcal{X}$ can be divided into two categories: the ego CAV and helping CAVs. The ego CAV is the one that needs to perceive its surrounding environment, while helping CAVs are the ones that send their complementary sensing information to the ego CAV to help it enhance its perception performance.
Thus, each CAV can be an ego one and helping one, depending on its role in a perception process. We assume that each CAV is equipped with a feature encoder $f_\mathtt{{encoder}}(\cdot)$, a feature aggregator $f_\mathtt{{agg}}(\cdot)$, and a feature decoder $f_\mathtt{{decoder}}(\cdot)$. For the $i$-th CAV in the set $\mathcal{X}$, the raw observation is denoted as $\mathbf{O}_i$ (such as camera images and LiDAR point clouds), and the final perception results are denoted as $\mathbf{Y}_i$. The CP pipeline of the $i$-th CAV can be described as follows.
\begin{enumerate}
    % \vspace{-3mm}
    \setlength{\itemsep}{0pt}
    \setlength{\parskip}{0pt}
    \setlength{\parsep}{0pt}
    \item \textit{Observation Encoding}: Each CAV encodes its raw observation $\mathbf{O}_j$ into an initial feature map $\mathbf{F}_j = f_\mathtt{{encoder}}(\mathbf{O}_j)$, where $j \in \mathcal{X}^N$.
    \item \textit{Intermediate Feature Transmission}: Helping CAVs transmit their intermediate features to the ego CAV: $\mathbf{F}_{j\rightarrow i}=\mathbf{\Gamma}_{j\rightarrow i}(\mathbf{F}_j),\  j\in \mathcal{X}^N, j\neq i,$
    where $\mathbf{\Gamma}_{j\rightarrow i}(\cdot)$ denotes a transmitter that conveys the $j$-th CAV's intermediate feature $\mathbf{F}_j$ to the ego CAV, while performing a spatial transformation. $\mathbf{F}_{j\rightarrow i}$ is the spatially aligned feature in the $i$-th CAV's coordinate.
    \item \textit{Feature Aggregation}: The ego CAV receives all the intermediate features and fuses them into a unified observational feature $\mathbf{F}_\mathtt{fused}=f_\mathtt{agg}(\mathbf{F}_{0\rightarrow i}, \{\mathbf{F}_{j\rightarrow i}\}_{j\neq i,\  j\in \mathcal{X}^N})$.
    \item \textit{Perception Decoding}: Finally, the ego CAV decodes the unified observational feature $\mathbf{F}_\mathtt{fused}$ into the final perception results $\mathbf{Y}=f_\mathtt{decoder}(\mathbf{F}_\mathtt{fused})$.
    % \vspace{-4mm}
\end{enumerate}


\begin{figure*}[t]
    % \vspace{-5mm}
    \centering
    % \fbox{\rule{0pt}{1.8in} \rule{0.9\linewidth}{0pt}}
    \includegraphics[width=.9\linewidth]{fig/CPDataGenerationPipeline.png}
    \vspace{-3mm}
    \caption{\textbf{Automatic Data Generation and Annotation Pipeline.} We first train a robust LiDAR collaborative object detector. Then, we discard the detection head and decoder and only keep the backbone as the intermediate feature generator. The data generation pipeline is shown in (a), (b), and (c), where (a) is the intermediate feature generation, (b) is the attack implementation, and (c) is the pair generation and saving.}
    \label{fig:data_generation}
    \vspace{-5mm} 
\end{figure*}



\subsection{Adversarial Threat Model}

Our focus is on the operation of an  intermediate-fusion collaboration scheme, where an attacker introduces designed adversarial perturbations into the intermediate features to mislead the perception of the ego CAV. Since an attacker participates in the collaborative system with local perception model installation, we assume they have white-box access to the model parameters. The attack procedure in each frame follows four sequential phases.
\begin{enumerate}
    \vspace{-4mm}
    \setlength{\itemsep}{0pt}
    \setlength{\parskip}{0pt}
    \setlength{\parsep}{0pt}
    \item \textit{Local Perception Phase}: All agents, including the malicious one, process their sensing data independently and extract intermediate features using feature encoders. 
    \vspace{-1mm}
    \begin{equation}
    \mathbf{F}_k = f_\mathtt{encoder}(\mathbf{O}_k), \quad k \in \mathcal{X}^N
    \end{equation}
    \vspace{-1mm}
    This phase operates in parallel without inter-agent communication.

    \item \textit{Feature Communication Phase}: All agents broadcast their extracted features through the network. Malicious agent $k$ collects feature information $\{\mathbf{F}_{j\rightarrow i}\}$ from other agents. Feature-level transmission ensures minimal communication overhead compared to raw sensor data exchange.

    \item \textit{Attack Generation Phase}: A malicious agent executes the attack by first perturbing its local features and then propagating them through the collaborative perception pipeline described in Section \ref{sec:formulation}.
    The attacker aims to optimize the perturbation $\delta$ through an iterative process. The optimization objective is formulated as:
    \vspace{-1mm}
    \begin{equation}
        \vspace{-2mm}
        \begin{aligned}
        \mathop{\arg\max}_{\delta} \mathcal{L}(\mathbf{Y}^\delta, \mathbf{Y}^\mathtt{gt}),
        \quad \mathtt{s.t.}\quad  \|\delta\|\leq \Delta
        \end{aligned}
    \end{equation}
    where $\Delta$ bounds the perturbation magnitude to maintain attack stealthiness. The total loss function is designed to aggregate adversarial losses over all object proposals, targeting both classification and localization aspects:
    \vspace{-1mm}
    \begin{equation}
        \vspace{-2mm}
        \mathcal{L}(\mathbf{Y}^\delta, \mathbf{Y}^\mathtt{gt}) = \sum_{p \in \mathbf{Y}^\delta} \mathcal{L}_\mathtt{adv}(p, p^\mathtt{gt})
    \end{equation}
    For each proposal $p$ with the highest confidence class $c = \mathop{\arg\max}\{p_i\}$, we leverage a class-specific adversarial loss following \citep{tuAdversarialAttacksMultiAgent2021}:
    \begin{equation*}
        % \vspace{-2mm}
        \mathcal{L}_\mathtt{adv}(p', p) = \begin{cases}
            -\log(1 - p'_c)\cdot\eta & c \neq k,\ p_c > \tau_1\\
            -\lambda p'_c\log(1 - p'_c) & c = k,\ p_c > \tau_2\\
            0 & \text{otherwise}
        \end{cases}
    \end{equation*}
    where $\eta$ represents the IoU between perturbed and original proposals to consider spatial accuracy, $\tau_1$ and $\tau_2$ are confidence thresholds for different attack scenarios, $\lambda$ balances the importance of different attack objectives, and $k$ denotes the background class.

    \item \textit{Defense and Final Perception Phase}: The ego vehicle integrates all received feature information, including potentially corrupted ones, to complete the final object detection task. Note that we focus exclusively on CP-specific vulnerabilities, excluding physical sensor attacks (e.g., LiDAR or GPS spoofing), which are general threats to CAVs. We also assume communication channels are secured with proper cryptographic protection.
\end{enumerate}

\section{RELATED WORK}
\label{sec:relatedwork}
In this section, we describe the previous works related to our proposal, which are divided into two parts. In Section~\ref{sec:relatedwork_exoplanet}, we present a review of approaches based on machine learning techniques for the detection of planetary transit signals. Section~\ref{sec:relatedwork_attention} provides an account of the approaches based on attention mechanisms applied in Astronomy.\par

\subsection{Exoplanet detection}
\label{sec:relatedwork_exoplanet}
Machine learning methods have achieved great performance for the automatic selection of exoplanet transit signals. One of the earliest applications of machine learning is a model named Autovetter \citep{MCcauliff}, which is a random forest (RF) model based on characteristics derived from Kepler pipeline statistics to classify exoplanet and false positive signals. Then, other studies emerged that also used supervised learning. \cite{mislis2016sidra} also used a RF, but unlike the work by \citet{MCcauliff}, they used simulated light curves and a box least square \citep[BLS;][]{kovacs2002box}-based periodogram to search for transiting exoplanets. \citet{thompson2015machine} proposed a k-nearest neighbors model for Kepler data to determine if a given signal has similarity to known transits. Unsupervised learning techniques were also applied, such as self-organizing maps (SOM), proposed \citet{armstrong2016transit}; which implements an architecture to segment similar light curves. In the same way, \citet{armstrong2018automatic} developed a combination of supervised and unsupervised learning, including RF and SOM models. In general, these approaches require a previous phase of feature engineering for each light curve. \par

%DL is a modern data-driven technology that automatically extracts characteristics, and that has been successful in classification problems from a variety of application domains. The architecture relies on several layers of NNs of simple interconnected units and uses layers to build increasingly complex and useful features by means of linear and non-linear transformation. This family of models is capable of generating increasingly high-level representations \citep{lecun2015deep}.

The application of DL for exoplanetary signal detection has evolved rapidly in recent years and has become very popular in planetary science.  \citet{pearson2018} and \citet{zucker2018shallow} developed CNN-based algorithms that learn from synthetic data to search for exoplanets. Perhaps one of the most successful applications of the DL models in transit detection was that of \citet{Shallue_2018}; who, in collaboration with Google, proposed a CNN named AstroNet that recognizes exoplanet signals in real data from Kepler. AstroNet uses the training set of labelled TCEs from the Autovetter planet candidate catalog of Q1–Q17 data release 24 (DR24) of the Kepler mission \citep{catanzarite2015autovetter}. AstroNet analyses the data in two views: a ``global view'', and ``local view'' \citep{Shallue_2018}. \par


% The global view shows the characteristics of the light curve over an orbital period, and a local view shows the moment at occurring the transit in detail

%different = space-based

Based on AstroNet, researchers have modified the original AstroNet model to rank candidates from different surveys, specifically for Kepler and TESS missions. \citet{ansdell2018scientific} developed a CNN trained on Kepler data, and included for the first time the information on the centroids, showing that the model improves performance considerably. Then, \citet{osborn2020rapid} and \citet{yu2019identifying} also included the centroids information, but in addition, \citet{osborn2020rapid} included information of the stellar and transit parameters. Finally, \citet{rao2021nigraha} proposed a pipeline that includes a new ``half-phase'' view of the transit signal. This half-phase view represents a transit view with a different time and phase. The purpose of this view is to recover any possible secondary eclipse (the object hiding behind the disk of the primary star).


%last pipeline applies a procedure after the prediction of the model to obtain new candidates, this process is carried out through a series of steps that include the evaluation with Discovery and Validation of Exoplanets (DAVE) \citet{kostov2019discovery} that was adapted for the TESS telescope.\par
%



\subsection{Attention mechanisms in astronomy}
\label{sec:relatedwork_attention}
Despite the remarkable success of attention mechanisms in sequential data, few papers have exploited their advantages in astronomy. In particular, there are no models based on attention mechanisms for detecting planets. Below we present a summary of the main applications of this modeling approach to astronomy, based on two points of view; performance and interpretability of the model.\par
%Attention mechanisms have not yet been explored in all sub-areas of astronomy. However, recent works show a successful application of the mechanism.
%performance

The application of attention mechanisms has shown improvements in the performance of some regression and classification tasks compared to previous approaches. One of the first implementations of the attention mechanism was to find gravitational lenses proposed by \citet{thuruthipilly2021finding}. They designed 21 self-attention-based encoder models, where each model was trained separately with 18,000 simulated images, demonstrating that the model based on the Transformer has a better performance and uses fewer trainable parameters compared to CNN. A novel application was proposed by \citet{lin2021galaxy} for the morphological classification of galaxies, who used an architecture derived from the Transformer, named Vision Transformer (VIT) \citep{dosovitskiy2020image}. \citet{lin2021galaxy} demonstrated competitive results compared to CNNs. Another application with successful results was proposed by \citet{zerveas2021transformer}; which first proposed a transformer-based framework for learning unsupervised representations of multivariate time series. Their methodology takes advantage of unlabeled data to train an encoder and extract dense vector representations of time series. Subsequently, they evaluate the model for regression and classification tasks, demonstrating better performance than other state-of-the-art supervised methods, even with data sets with limited samples.

%interpretation
Regarding the interpretability of the model, a recent contribution that analyses the attention maps was presented by \citet{bowles20212}, which explored the use of group-equivariant self-attention for radio astronomy classification. Compared to other approaches, this model analysed the attention maps of the predictions and showed that the mechanism extracts the brightest spots and jets of the radio source more clearly. This indicates that attention maps for prediction interpretation could help experts see patterns that the human eye often misses. \par

In the field of variable stars, \citet{allam2021paying} employed the mechanism for classifying multivariate time series in variable stars. And additionally, \citet{allam2021paying} showed that the activation weights are accommodated according to the variation in brightness of the star, achieving a more interpretable model. And finally, related to the TESS telescope, \citet{morvan2022don} proposed a model that removes the noise from the light curves through the distribution of attention weights. \citet{morvan2022don} showed that the use of the attention mechanism is excellent for removing noise and outliers in time series datasets compared with other approaches. In addition, the use of attention maps allowed them to show the representations learned from the model. \par

Recent attention mechanism approaches in astronomy demonstrate comparable results with earlier approaches, such as CNNs. At the same time, they offer interpretability of their results, which allows a post-prediction analysis. \par


\section{Theory}
\label{sec:theory}
In this section, we provide theoretical support for homotopy dynamics. In the first part, we demonstrate that for certain PDEs with small parameters, direct training using PINN methods is highly challenging. This analysis is based on the neural tangent kernel (NTK) framework \cite{allen2019convergence}. In the second part, we show that homotopy dynamics will converge to the solution with a small parameter \( \varepsilon \), provided that the dynamic step size is sufficiently small and the initial solution has been well learned by the neural network.
\subsection{Challenges in Training Neural Network with Small Certain Parameters}
Let us consider training neural networks without homotopy dynamics. The corresponding loss function can be expressed as  
\begin{align}
    L_H(\vtheta) = \frac{1}{2n} \sum_{i=1}^n H^2(u_{\vtheta}(\vx_i),\varepsilon),
\label{eq:hom_or_loss}
\end{align}  
where $\{\vx_i\}_{i=1}^n$ represents the training data used to optimize the neural network. Here, we assume that the parameter $\varepsilon$ in the PDE appears only in the interior terms and not in the boundary conditions. Therefore, in this section, we omit the effect of boundary conditions, as the behavior at the boundary remains unchanged for any given $\varepsilon$.  

Furthermore, to simplify the notation, we use $n$ instead of $\nres$ and denote $\vx_r^i$ simply as $\vx_i$ comparing with Eq. \eqref{loss}.

In the classical approach, such a loss function is optimized using gradient descent, stochastic gradient descent, or Adam. Considering the training process of gradient descent in its continuous form, it can be expressed as:
\begin{align}
    \frac{\D \vtheta}{\D t} 
    &= -\nabla_{\vtheta}L(\vtheta) 
    \notag\\&= -\frac{1}{n} \sum_{i=1}^n H(u_{\vtheta}(\vx_i),\varepsilon)\delta_{\phi}H(u_{\vtheta}(\vx_i),\varepsilon)\nabla_{\vtheta}u_{\vtheta}(\vx_i), \notag \\
    &= -\frac{1}{n} \vH(u_{\vtheta}(\vx),\varepsilon) \cdot \vS,
\end{align}
where $t$ in this section is the time of the gradient decent process instead of the time in PDEs, and
\begin{align}
    \vH(u_{\vtheta}(\vx),\varepsilon) 
    &:= \big[ H(u_{\vtheta}(\vx_i),\varepsilon)\delta_{\phi}H(u_{\vtheta}(\vx_i),\varepsilon)\big]_{i=1}^n \notag \\
    &= \vl \cdot \vD_\varepsilon,
\end{align}
and
\begin{align}
    \vl := \big[ H(\phi(\vx_i,\vtheta),\varepsilon) \big]_{i=1}^n \in \sR^{1 \times n},  \vD_\varepsilon \in \sR^{n \times n}
\end{align}where $\vD_\varepsilon$ represents the discrete form of the variation of PDEs in different scenarios. Furthermore,
\begin{align}
    \vS = \big[ \nabla_{\vtheta}u_{\vtheta}(\vx_1),\dots, \nabla_{\vtheta}u_{\vtheta}(\vx_n) \big].
\end{align}

Therefore, we obtain  
\begin{align}
    \frac{\D L(\vtheta)}{\D t} &= \nabla_{\vtheta}L(\vtheta) \frac{\D \vtheta}{\D t} \notag \\
    &= -\frac{1}{n^2} \vH(u_{\vtheta}(\vx),\varepsilon) \vS \vS^{\top} \vH^{\top}(u_{\vtheta}(\vx),\varepsilon) \notag \\
    &= -\frac{1}{n^2} \vl \vD_\varepsilon \vS \vS^{\top} \vD_\varepsilon^{\top} \vl^{\top}.
\end{align}
Hence, the kernel of the gradient descent update is given by  
\begin{align}
    \vK_\varepsilon := \vD_\varepsilon \vS \vS^{\top} \vD_\varepsilon^{\top}.
\end{align}

The following theorem provides an upper bound for the smallest eigenvalue of the kernel and its role in the gradient descent dynamics:
\begin{theorem}[Effectiveness of Training via the Eigenvalue of the Kernel]\label{compare}
   Suppose \( \lambda_{\text{min}}(\vS\vS^{\top}) > 0 \) and \( \vD_\varepsilon \) is non-singular, and let \( \varepsilon \geq 0 \) be a constant. Then, we have \( \lambda_{\text{min}}(\vK_{\varepsilon}) > 0 \), and there exists \( T > 0 \) such that  
\begin{equation}
    L(\vtheta(t)) \leq L(\vtheta(0))\exp\left(-\frac{\lambda_{\text{min}}(\vK_{\varepsilon})}{n} t\right)\label{speed}
\end{equation}
for all \( t \in [0, T] \). Furthermore,  
\begin{equation}
    \lambda_{\text{min}}(\vK_{\varepsilon}) \leq \lambda_{\text{min}}(\vS\vS^{\top}) \lambda_{\text{max}}(\vD_{\varepsilon}\vD_{\varepsilon}^\top).
    \label{mineigen}
\end{equation}
\end{theorem}

\begin{figure}[t]
    \centering
    \includegraphics[scale=0.48]{figs/AD_operator_eigenvalue.png}
    \caption{Largest eigenvalue of \(\vD_\varepsilon\) \eqref{eq:discete_operator} for different $\varepsilon$. A smaller $\varepsilon$ results in a smaller largest eigenvalue of \eqref{eq:discete_operator}, leading to a slower convergence rate and increased difficulty in training.}
    \label{fig:1d_allen_cahn_eigen_value}
\end{figure}
\begin{figure*}[htbp!]
    \centering
    \includegraphics[scale=0.40]{figs/2D_AC_results.pdf}
    \caption{2D Allen Cahn Equaiton. (Top) Evolution of the Homotopy Dynamcis. (Bottom) Plot for Cross-section of $u(x,y)$ at $y = 0.5$ i.e., $u(x,y=0.5)$. The reference solution $u_{\infty}(x)$ represents the ground truth steady-state solution. The L2RE is $8.78e-3$. Number of residual points is $\nres = 50\times50$. }
\label{fig:2D_Allen_Cahn_Equation}
\end{figure*}


\begin{remark}\label{hard}
    For \( \vS\vS^{\top} \), previous works such as \cite{luo2020two, allen2019convergence, arora2019exact, cao2020generalization, yang2025homotopy} demonstrate that it becomes positive when the width of the neural network is sufficiently large with ReLU activation functions. Additionally, \cite{gao2023gradient} discusses the positivity of the gradient kernel in PINNs for solving heat equations. Therefore, we can reasonably assume that \( \vS\vS^{\top} \) is a strictly positive matrix. In Appendix~\ref{ss}, we present a specific scenario where \( \lambda_{\text{min}}(\vS\vS^{\top}) > 0 \) holds with high probability.


   This theorem demonstrates that the smallest eigenvalue of the kernel directly affects the training speed. Equation \eqref{mineigen} shows that the upper bound of \( \lambda_{\text{min}}(\vK_{\varepsilon}) \) can be influenced by \( \lambda_{\text{max}}(\vD_{\varepsilon}\vD_{\varepsilon}^\top) \). In many PDE settings, the maximum eigenvalue \( \lambda_{\text{max}}(\vD_{\varepsilon}\vD_{\varepsilon}^\top) \) tends to be small when \( \varepsilon \) is small. For example, in this paper, we consider the Allen–Cahn equation, given by
\[
-\varepsilon^2\Delta u + f(u) = 0,
\]
where \( f(u) = u^3 - u \). In this case, \( \vD_\varepsilon \) corresponds to the discrete form of the operator \( -\varepsilon^2\Delta + f'(u) \), which can be written as
\begin{equation}
   \vD_\varepsilon= -\varepsilon^2\Delta_{\text{dis}} + \text{diag} \big(f'(u(\vx_1)), \dots, f'(u(\vx_n)) \big).
    \label{eq:discete_operator}
\end{equation}
According to \cite{morton2005numerical}, the discrete Laplacian \( -\varepsilon^2\Delta_{\text{dis}} \) is strictly positive. Specifically, in the one-dimensional case, its largest eigenvalue is given by
\[
4\varepsilon^2 n^2 \cos^2 \frac{ \pi}{2n+1},
\]
which is close \( 4\varepsilon^2n^2 \) as \( n  \) is large enough. 

Moreover, since \( f'(u(\boldsymbol{x}_i)) \) ranges between \(-1\) and \(2\), when \( \varepsilon \) is large (close to 1), the largest eigenvalue of  
\( \vD_\varepsilon \)
becomes very large, regardless of the sampling locations \( \{\boldsymbol{x}_i\}_{i=1}^n \), as shown in \cref{fig:1d_allen_cahn_eigen_value} for the case \( n=200 \). Therefore, according to Theorem~\ref{compare}, the upper bound of the smallest eigenvalue of \( \vK_\varepsilon \) will also be large, specifically of order \( n^4 \) with respect to \( n \) in this case due to Weyl’s inequalities. Consequently, the training speed can reach \( \exp(-Cn^3t) \) based on Eq. \eqref{speed}, which is fast and implies that training is easy.

However, when \( \varepsilon \) is small (close to 0), the largest eigenvalue of  
\( \vD_\varepsilon \)
is only of order \( 1 \) with respect to \( n \), which implies that the upper bound of the smallest eigenvalue of \( \vK_\varepsilon \) will no longer be of order \( 1 \) with respect to \( n \). Therefore, the training speed can reach \( \exp(-Ct/n) \) based on Eq. \eqref{speed}, which is slow and indicates that training is difficult in this case.
\end{remark}


\subsection{Convergence of Homotopy Dynamics}
In this section, we aim to demonstrate that homotopy dynamics is a reasonable approach for obtaining the solution when \( \varepsilon \) is small. For simplicity of notation, we denote \( u(\varepsilon) \) as the exact solution of \( H(u,\varepsilon) = 0 \) and \( U(\varepsilon) \) as its numerical approximation in the simulation. Suppose \( H(u(\varepsilon),\varepsilon) = 0 \), and assume that \( \frac{\partial H(u(\varepsilon),\varepsilon)}{\partial u} \) is invertible. Then, the dynamical system (\ref{eq:homotopy_dynamics}) can be rewritten as  
\begin{equation}
\frac{\D u}{\D \varepsilon} = -\left(\frac{\partial H(u(\varepsilon),\varepsilon)}{\partial u}\right)^{-1} \frac{\partial H(u(\varepsilon),\varepsilon)}{\partial \varepsilon} =: h(u(\varepsilon),\varepsilon).\label{dym}
\end{equation}
Applying Euler’s method to this dynamic system, we obtain  
\begin{align}
    U(\varepsilon_{k+1}) = U(\varepsilon_k) + (\varepsilon_{k+1} - \varepsilon_k) h( U(\varepsilon_k),\varepsilon_k).
\end{align} 

The following theorem shows that if \( u(\varepsilon_0)-U(\varepsilon_0) \) is small and the step size \( (\varepsilon_{k+1} - \varepsilon_k) \) is sufficiently small at each step, then \( u(\varepsilon_k)- U(\varepsilon_k) \) remains small.

\begin{theorem}[Convergence of Homotopy Dynamics]\label{small}
    Suppose \( h(\varepsilon,u) \) is a continuous operator for \( 0 \leq \varepsilon_n \leq \varepsilon_0 \) and \( u \in H^2(\Omega) \), and 
    \[
    \|h(u_1,\varepsilon)-h(u_2,\varepsilon)\|_{H^2(\Omega)}\le K_\varepsilon\|u_1-u_2\|_{H^2(\Omega)}.
    \]
    Assume there exists a constant \( K \) such that 
    \[
    (\varepsilon_k-\varepsilon_{k+1})K_{\varepsilon_k}\le K\cdot \frac{\varepsilon_0-\varepsilon_n}{N}
    \]
    and 
    \begin{align}
    \tau&:=\frac{n}{\varepsilon_0-\varepsilon_n} \sup_{0\leq k\leq n} (\varepsilon_k-\varepsilon_{k+1})^2\|u(\varepsilon_k)\|_{H^4(\Omega)}\ll 1,\notag\\e_0&:=\|u(\varepsilon_0)-U(\varepsilon_0)\|_{H^2(\Omega)}  \ll 1 \notag
    \end{align}
    then we have 
    \begin{align}
        &\|u(\varepsilon_n)- U(\varepsilon_n)\|_{H^2(\Omega)}\notag\\\le& e_0e^{K(\varepsilon_0-\varepsilon_n)}+\frac{\tau(e^{K(\varepsilon_0-\varepsilon_n)}-1)}{2K} \ll 1.
    \end{align}
\end{theorem}

\begin{figure*}[htbp!]
    \centering
    \includegraphics[scale=0.40]{figs/High_frequncy_aproximation.pdf}
    \caption{High-frequency function $\sin(50\pi x)$ approximation: Comparison of loss curves between original evolution and homotopy evolution. The comparison shows that homotopy evolution effectively reduces the loss, successfully approximating the high-frequency function, while the original evolution fails. The number of residual points is $\nres = 300$. }
\label{fig:high_frequency_result}
\end{figure*}


   The proof of Theorem~\ref{small} is inspired by \cite{antonakopoulos2022adagrad}.  

Theorem~\ref{small} shows that if \( e_0 \) is small and the step size \( (\varepsilon_{k+1} - \varepsilon_k) \) is sufficiently small at each step and satisfies  
\[
(\varepsilon_k - \varepsilon_{k+1}) K_{\varepsilon_k} \leq K \cdot \frac{\varepsilon_0 - \varepsilon_n}{n},
\]
i.e., the training step size should depend on the Lipschitz constant of \( h(u,\varepsilon) \), ensuring stable training, then \( u(\varepsilon_k) - U(\varepsilon_k) \) remains small. The initial error \( e_0 \) can be very small since we use a neural network to approximate the solution of PDEs for large \( \varepsilon \), where learning is effective.  

The error \( e_0 \) consists of approximation, generalization, and training errors. The approximation error reflects the gap between the exact PDE solution and the neural network's hypothesis space, the generalization error arises from the challenges of learning with finite samples, and the training error results from optimizing the neural network's loss function. The training error can be well controlled by Theorem~\ref{compare} when \( \varepsilon \) is large, while the approximation and generalization errors can be small if the sample size is sufficiently large and the neural network is expressive enough.  

The theoretical support for this result can be found in \cite{yang2023nearly,yang2024deeper}, which we discuss further in Appendix~\ref{e0}.




% \begin{figure*}[t]
%     \centering
%     \includegraphics[scale=0.33]{figs/scatter_multi_pde.pdf}
%     \caption{We plot the final L2RE against the final loss for each combination of network width, optimization strategy, and random seed.
%     Across all three PDEs, a lower loss generally corresponds to a lower L2RE.}
%     \label{fig:l2re_vs_loss}
% \end{figure*}

% First, we show that PINNs must be trained to near-zero loss to obtain a reasonably low L2RE.
% This phenomenon can be observed in \cref{fig:l2re_vs_loss}, demonstrating that a lower loss generally corresponds to a lower L2RE.
% For example, on the convection PDE, a loss of $10^{-3}$ yields an L2RE around $10^{-1}$, but decreasing the loss by a factor of $100$ to $10^{-5}$ yields an L2RE around $10^{-2}$, a 10$\times$ improvement.
% This relationship between loss and L2RE in \cref{fig:l2re_vs_loss} 
% is typical of many PDEs \cite{lu2022multifidelity}. 

% The relationship in \cref{fig:l2re_vs_loss} underscores that high-accuracy optimization is required for a useful PINN.
% There are instances (especially on the reaction ODE), where the PINN solution has a L2RE around 1, despite a near-zero loss; we provide insight into why this is occurring in \cref{sec:low_loss_high_l2re}. 
% In \cref{sec:loss_landscape,sec:under_optimized}, we show that ill-conditioning and under-optimization make reaching a solution with sufficient accuracy difficult.

% \pnote{Cite Mishra paper about training error and quadrature error}

% \begin{itemize}
    % \item Show empirically that near-zero loss is need for a good solution. This can be achieved by plotting l2re against loss for a few runs
    % \item Compare and contrast to vision tasks in deep learning. We do not need zero loss to get decent test accuracy
% \end{itemize}
%\section{Experiments}
\label{sec:loss_landscape}

\begin{figure*}[t]
    \centering
    \includegraphics[scale=0.33]{figs/spectral_density_multi_pde_convection_combined.pdf}
    \caption{(Top) Spectral density of the Hessian and the preconditioned Hessian after 41000 iterations of \al{}. 
    The plots show that the PINN loss is ill-conditioned and that \lbfgs{} improves the conditioning, reducing the top eigenvalue by $10^3$ or more. \\
    (Bottom) Spectral density of the Hessian and the preconditioned Hessian of each loss component after 41000 iterations of \al{} for convection. The plots show that each component loss is ill-conditioned and that the conditioning is improved by \lbfgs{}.}
    \label{fig:spectral_density_multi_pde_convection_combined}
\end{figure*}

% In this section, %we empirically study the conditioning of the PINN loss. 
We show empirically that the ill-conditioning of the PINN loss is mainly due to the residual loss, which contains the differential operator.
We also show that quasi-Newton methods like \lbfgs{} improve the conditioning of the problem. 

\subsection{The PINN Loss is Ill-conditioned}

% \begin{itemize}
%     \item Say that we use PyHessian for obtaining spectral densities
%     \item Show loss landscapes for each PDE at the most difficult setting (could place landscapes for other coefficient settings in the appendix)
%     \item Explain why this is bad. Appeal to convergence of gradient descent depending on the condition number
% \end{itemize}

% The conditioning of the loss plays a significant role in optimization and affects the performance of optimization algorithms. 
The conditioning of the loss $L$ plays a key role in the performance of first-order optimization methods \cite{nesterov2018lectures}.
% This property is closely related to the curvature of the loss landscape. 
We can understand the conditioning of an optimization problem through the eigenvalues of the Hessian of the loss, $H_L$. 
Intuitively, the eigenvalues of $H_L$ provide information about the local curvature of the loss function at a given point along different directions. 
The condition number is defined as the ratio of the largest magnitude's eigenvalue to the smallest magnitude's eigenvalue.
A large condition number implies the loss is very steep in some directions and flat in others, making it difficult for first-order methods to make sufficient progress toward the minimum. 
When $H_L(w)$ has a large condition number (particularly, for $w$ near the optimum), the loss $L$ is called \emph{ill-conditioned}.
For example, the convergence rate of gradient descent (GD) depends on the condition number \cite{nesterov2018lectures}, which results in GD converging slowly on ill-conditioned problems.

% and ill-conditioned problems--characterized by a large condition number--lead to slow convergence of this algorithm. 

To investigate the conditioning of the PINN loss $L$, 
we would like to examine the eigenvalues of the Hessian. 
For large matrices, it is convenient to visualize the set of eigenvalues via  \emph{spectral density}, which approximates the distribution of the eigenvalues.
Fast approximation methods for the spectral density of the Hessian are available for deep neural networks \cite{ghorbani2019an, yao2020pyhessian}. 
% Specifically, we apply stochastic Lanczos quadrature to compute empirical spectral density of the Hessian eigenvalues. 
\cref{fig:spectral_density_multi_pde_convection_combined} shows the estimated Hessian spectral density (solid lines) of the PINN loss for the convection, reaction, and wave problems after training with \al{}. 
For all three problems, we observe large outlier eigenvalues ($> 10^4$ for convection, $> 10^3$ for reaction, and $> 10^5$ for wave) in the spectrum, and a significant spectral density near $0$, implying that the loss $L$ is ill-conditioned.
The plots also show how the spectrum is improved by preconditioning (\cref{subsec:lbfgs_improvement}).
% Since convection and wave have larger outlier eigenvalues, we expect first-order methods to perform worse than on the reaction 
% We observe large estimated condition numbers for all three problems. 
% This provides an indication that the PINN loss is ill-conditioned, and we should expect very slow convergence if we were to use a first-order method like GD. 
% Moreover, the reaction problem appears to have noticeably large negative eigenvalues that signify saddle point, further illustrating the difficult loss landscape that optimization algorithms have to deal with. 
% \textcolor{blue}{\textbf{[WL: need comparison with loss landscape of ResNet (loss surface plot as seen in the Goldstein's)?]}}

\subsection{The Ill-conditioning is Due to the Residual Loss}

% \begin{itemize}
%     \item Look at the spectrum of each component of the loss separately.
%     \item Show that this is the case due to the presence of the differential operator in the residual loss 
% \end{itemize}

We use the same method to study the conditioning of each component of the PINN loss. \cref{fig:spectral_density_multi_pde_convection_combined,fig:spectral_density_reaction_wave} show the estimated spectral density of the Hessian of the residual, initial condition, and boundary condition components of the PINN loss for each problem after training with \al. 
We see residual loss, which contains the differential operator $\mathcal D$, is the most ill-conditioned among all components.
Our theory (\cref{sec:theory}) shows this ill-conditioning is likely due to the ill-conditioning of $\mathcal D$.
% implies the differential operator that defines the residual loss is ill-conditioned. 
% We explain consequences of an ill-conditioned differential operator in \cref{sec:theory}.

\subsection{\lbfgs{} Improves Problem Conditioning}
\label{subsec:lbfgs_improvement}

% \begin{itemize}
%     \item Show preconditioning effect of \lbfgs{} on the Hessian of the entire loss and the Hessian of each loss component. Explain the notion of the preconditioned Hessian, but leave mathematical details of unrolling the \lbfgs{} recursion for the appendix
%     \item Remind reader that improving condition number leads to faster convergence
%     \item Compare to loss landscape of a vision task?
% \end{itemize}

Preconditioning is a popular technique for improving conditioning in optimization. 
A classic example is Newton's method, which uses second-order information (i.e., the Hessian) to (locally) transform an ill-conditioned loss landscape into a well-conditioned one.
% It aims to transform the loss landscape so that it becomes well-conditioned, thus improving convergence by making local minima more accessible to gradient updates in the transformed space. 
\lbfgs{} is a quasi-Newton method that improves conditioning without explicit access to the problem Hessian. 
%Instead, it implicitly constructs and stores a symmetric positive definite preconditioner that approximates the Hessian over the optimization trajectory.
% \lbfgs{}, one of the classic deterministic quasi-Newton methods, implicitly stores a symmetric positive definite preconditioner that satisfies the secant equation and represent the preconditioner as a low-rank update to a diagonal matrix at each iteration \cite{nocedal2006numerical}. 
To examine the effectiveness of quasi-Newton methods for optimizing $L$, 
we compute the spectral density of the Hessian after \lbfgs{} preconditioning. (For details of this computation and how L-BFGS preconditions, see \cref{sec:lbfgs_spectral_info}.)
\cref{fig:spectral_density_multi_pde_convection_combined} shows this preconditioned Hessian spectral density (dashed lines). 
%For all three problems, both the spread and magnitude of eigenvalues has decreased by $10^3$ or more, 
%improving the condition number by at least $10^3$. 
For all three problems, the magnitude of eigenvalues and the condition number has been reduced by at least $10^3$. 
%or more, 
%improving the condition number by at least $10^3$. 
In addition, the preconditioner improves the conditioning of each individual loss component of $L$ (\cref{fig:spectral_density_multi_pde_convection_combined,fig:spectral_density_reaction_wave}). 
These observations offer clear evidence that quasi-Newton methods improve the conditioning of the loss, and show the importance of quasi-Newton methods in training PINNs, which we demonstrate in \cref{sec:opt_comparison}. 
%\section{Experiments}
\label{sec:opt_comparison}






We perform numerical experiments in the contexts of function approximation, physics-informed neural networks (PINNs), and operator learning, applying the proposed Homotopy Evolution Training Strategy and presenting the associated results.


We conduct experiments on optimizing PINNs for convection, wave PDEs, and a reaction ODE. 
These equations have been studied in previous works investigating difficulties in training PINNs; we use the formulations in \citet{krishnapriyan2021characterizing, wang2022when} for our experiments. 
The coefficient settings we use for these equations are considered challenging in the literature \cite{krishnapriyan2021characterizing, wang2022when}.
\cref{sec:problem_setup_additional} contains additional details.

We compare the performance of Adam, \lbfgs{}, and \al{} on training PINNs for all three classes of PDEs. 
For Adam, we tune the learning rate by a grid search on $\{10^{-5}, 10^{-4}, 10^{-3}, 10^{-2}, 10^{-1}\}$.
For \lbfgs, we use the default learning rate $1.0$, memory size $100$, and strong Wolfe line search.
For \al, we tune the learning rate for Adam as before, and also vary the switch from Adam to \lbfgs{} (after 1000, 11000, 31000 iterations).
These correspond to \al{} (1k), \al{} (11k), and \al{} (31k) in our figures.
All three methods are run for a total of 41000 iterations.

We use multilayer perceptrons (MLPs) with tanh activations and three hidden layers. These MLPs have widths 50, 100, 200, or 400.
We initialize these networks with the Xavier normal initialization \cite{glorot2010understanding} and all biases equal to zero.
Each combination of PDE, optimizer, and MLP architecture is run with 5 random seeds.

We use 10000 residual points randomly sampled from a $255 \times 100$ grid on the interior of the problem domain. 
We use 257 equally spaced points for the initial conditions and 101 equally spaced points for each boundary condition.

We assess the discrepancy between the PINN solution and the ground truth using $\ell_2$ relative error (L2RE), a standard metric in the PINN literature. Let $y = (y_i)_{i = 1}^n$ be the PINN prediction and $y' = (y'_i)_{i = 1}^n$ the ground truth. Define
\begin{align*}
    \mathrm{L2RE} = \sqrt{\frac{\sum_{i = 1}^n (y_i - y'_i)^2}{\sum_{i = 1}^n y'^2_i}} = \sqrt{\frac{\|y - y'\|_2^2}{\|y'\|_2^2}}.
\end{align*}
We compute the L2RE using all points in the $255 \times 100$ grid on the interior of the problem domain, along with the 257 and 101 points used for the initial and boundary conditions.

We develop our experiments in PyTorch 2.0.0 \cite{paszke2019pytorch} with Python 3.10.12.
Each experiment is run on a single NVIDIA Titan V GPU using CUDA 11.8.
The code for our experiments is available at \href{https://github.com/pratikrathore8/opt_for_pinns}{https://github.com/pratikrathore8/opt\_for\_pinns}.


\subsection{Function Approximation}
\cref{fig:opt_comparison} in \cref{sec:opt_comparison_additional} compares \al, Adam, and \lbfgs{} on the convection, reaction, and wave problems at difficult coefficient settings noted in the literature \cite{krishnapriyan2021characterizing, wang2022when}.
Across each network width, the lowest loss and L2RE is always delivered by \al.
Similarly, the lowest median loss and L2RE are almost always delivered by \al{} (\cref{fig:opt_comparison}).
The only exception is the reaction problem, where Adam outperforms \al{} on loss at width = 100 and L2RE at width = 200 (\cref{fig:opt_comparison}).

%The best performance of each optimizer across all hyperparameters and architectures is shown in \cref{tab:loss_l2re_comparison}.
\cref{tab:loss_l2re_comparison} summarizes the best performance of each optimizer.
Again, \al{} is better than running either Adam or L-BFGS alone.
Notably, \al{} attains 14.2$\times$ smaller L2RE than Adam on the convection problem and 6.07$\times$ smaller L2RE than \lbfgs{} on the wave problem.

\begin{table}[t]
    \caption{Lowest loss for Adam, \lbfgs, and \al{} across all network widths after hyperparameter tuning. 
    \al{} attains both smaller loss and L2RE vs. Adam or \lbfgs. 
    }
    \vskip 0.15in
    \centering
    \tiny
    \begin{tabular}{|c|c|c|c|c|c|c|c|} 
    \hline 
    \multirow{2}{*}{Optimizer} & \multicolumn{2}{c|}{Convection} & \multicolumn{2}{c|}{Reaction} & \multicolumn{2}{c|}{Wave} \\ \cline{2-7}
                               & Loss & L2RE & Loss & L2RE & Loss & L2RE \\ \hline 
    Adam                        & 1.40e-4     & 5.96e-2     & 4.73e-6     & 2.12e-2     & 2.03e-2     & 3.49e-1     \\ \hline 
    L-BFGS                      & 1.51e-5     & 8.26e-3     & 8.93e-6     & 3.83e-2     & 1.84e-2     & 3.35e-1     \\ \hline 
    \al                         & \textbf{5.95e-6}     & \textbf{4.19e-3}     & \textbf{3.26e-6}     & \textbf{1.92e-2}     & \textbf{1.12e-3}     & \textbf{5.52e-2}      \\ \hline
    \end{tabular}
    % On convection, \al{} provides 14.2$\times$ and 1.97$\times$ improvement over Adam or \lbfgs{} on L2RE. 
    % On reaction, \al{} provides 1.10$\times$ and 1.99$\times$ improvement over Adam or \lbfgs{} on L2RE.
    % On wave, \al{} provides 6.32$\times$ and 6.07$\times$ improvement over Adam or \lbfgs{} on L2RE.}
    \label{tab:loss_l2re_comparison}
\end{table}


% \textbf{Convection.} Across each network width, the lowest loss and L2RE is always attained by one of the three \al{} strategies. 
% The lowest L2REs attained by Adam, \lbfgs, and \al{} are 5.96e-2, 8.26e-3, 4.19e-3, respectively, i.e., \al{} provides 14.2 and 1.97 times improvement over Adam and \lbfgs{} on L2RE.  
% Similarly, the lowest median loss and L2RE is always attained by one of the three \al{} strategies.

% \textbf{Reaction.} Similar to convection, the lowest loss and L2RE is always attained by one of the three \al{} strategies.
% The lowest L2RE attained by Adam, \lbfgs, and \al{} are 2.12e-2, 3.83e-2, 1.92e-2, respectively, i.e., \al{} provides 1.10 and 1.99 times improvements over Adam and \lbfgs{} on L2RE.
% The lowest median loss and L2RE is attained by one of the three \al{} strategies, except for the width = 100 setting for loss and width = 200 setting for L2RE.

% \textbf{Wave.} Similar to both convection and reaction, the lowest loss and L2RE is always attained by one of the three \al{} strategies.
% The lowest L2REs attained by Adam, \lbfgs, and \al{} are 3.49e-1, 3.35e-1, 5.52e-2, respectively, i.e., \al{} provides 6.32 and 6.07 times improvement over Adam and \lbfgs{} on L2RE.  
% Again, the lowest median loss and L2RE is always attained by one of the three \al{} strategies.

% \pnote{Express this stuff in a table}

% \begin{itemize}
    % \item Show that \al{} is better in practice. Demonstrate that this holds up across a wide range of network widths. The loss is improved over other methods -- but this doesn't always mean better L2RE. Leave this is a question for future work.
    % \item Tell reader that this suggests any results using just \lbfgs{} should be viewed with skepticism
% \end{itemize}

\subsection{Intuition From Optimization Theory}
The success of \al{} over Adam and \lbfgs{} can be explained by existing results in optimization theory.
In neural networks, saddle points typically outnumber local minima \cite{dauphin2014identifying,lee2019firstorder}.
% We must retain convergence speed near saddle points to converge to a global minimizer efficiently.
A saddle point can never be a global minimum. 
We want to reach a global minimum when training PINNs.

Newton's method (which \lbfgs{} attempts to approximate) is attracted to saddle points \cite{dauphin2014identifying},  and quasi-Newton methods such as \lbfgs{} also converge to saddle points since they ignore negative curvature \cite{dauphin2014identifying}.
On the other hand, first-order methods such as gradient descent and AdaGrad \cite{duchi2011adaptive} avoid saddle points \cite{lee2019firstorder,antonakopoulos2022adagrad}.
We expect that (full-gradient) Adam also avoids saddles for similar reasons, although
we are not aware of such a result.

Alas, first-order methods converge slowly when the problem is ill-conditioned. 
This result generalizes the well-known slow convergence of conjugate gradient (CG)
for ill-conditioned linear systems:
 $\mathcal O (\sqrt \kappa \log(\frac{1}{\epsilon}))$ iterations to converge to an $\epsilon$-approximate solution of a system with condition number $\kappa$.
In optimization, an analogous notion of a condition number in a set $\mathcal S$ near a global minimum is given by $\kappa_{f}(\mathcal S) \coloneqq \sup_{w \in \mathcal S} \| H_f(w) \| / \mu$, where $\mu$ is the \PL-constant (see \cref{sec:theory}).
Then gradient descent requires $\mathcal O (\kappa_{f}(\mathcal S) \log(\frac{1}{\epsilon}))$ iterations to converge to an $\epsilon$-suboptimal point.
% Under the $\mu$-\PL condition , gradient descent requires $\mathcal O (\kappa \log(\frac{1}{\epsilon}))$ iterations to converge to an $\epsilon$-suboptimal point, where $\kappa \coloneqq \max /\mu$
% For example, suppose $f(w)$ is a convex function for which $\mu I \preceq \nabla^2 f(w) \preceq L I$, where $L > \mu > 0$ \footnote{In the optimization literature, the upper and lower bounds on the Hessian are called strong convexity and smoothness assumptions.}.
% Then gradient descent requires $\mathcal O (\kappa \log(\frac{1}{\epsilon}))$ iterations to converge to an $\epsilon$-suboptimal point, where $\kappa \coloneqq L/\mu$.
For PINNs, the condition number near a solution is often $> 10^{4}$ (\cref{fig:spectral_density_multi_pde_convection_combined}), which leads to slow convergence of first-order methods. 
% \pnote{Ask Zach if we should talk about AGD or rates for adaptive methods}
However, Newton's method and L-BFGS can significantly reduce the condition number (\cref{fig:spectral_density_multi_pde_convection_combined}), which yields faster convergence. 

% \pnote{Make explanation more careful. We are not actually in a convex setting, but we do make the $\mu$-\PL assumption later}

\al{} combines the best of both first- and second-order/quasi-Newton methods. 
By running Adam first, we avoid saddle points that would attract L-BFGS.
By running L-BFGS after Adam, 
we can reduce the condition number of the problem, which leads to faster local convergence.
\cref{fig:under_optimization_intro} exemplifies this, showing faster convergence of \al{} over Adam on the wave equation.
%We provide an example showing faster convergence of \al{} compared to Adam for the wave equation (the most ill-conditioned problem in \cref{sec:loss_landscape}) in \cref{fig:under_optimization_intro}.

This intuition also explains why Adam sometimes performs as well as \al{} on the reaction problem.
\cref{fig:spectral_density_multi_pde_convection_combined} shows the largest eigenvalue of the reaction problem is around $10^{3}$, 
while the largest eigenvalues of the convection and wave problems are around $10^{4}$ and $10^{5}$, suggesting the reaction problem is less ill-conditioned.
%Therefore, we would expect a first-order method (i.e., Adam) to be more competitive with \al{}.

% \begin{itemize}
    % \item Explain why we don't expect Adam or \lbfgs{} alone to work well. GD has condition number dependence and AGD has square root condition number dependence. The theoretical rate of convergence for Adam is $\mathcal{O}(\log T / \sqrt{T})$ and would require decaying stepsizes to get to minimum. \lbfgs{} on its own could easily converge to a saddle point. Adam is probably helping avoid saddle points; following this by \lbfgs{} leads to a much better solution
    % \item Try L-BFGS in float32 and float64. Does float64 give better results?
% \end{itemize}

%\section{The Loss is Often Under-optimized}
\label{sec:under_optimized}
In \cref{sec:opt_comparison}, we show that \al{} improves on running Adam or \lbfgs{} alone.
However, even \al{} does not reach a critical point of the loss: the loss is still under-optimized.
We show that the loss and L2RE can be further improved by running a damped version of Newton's method.
% , where we solve for the Newton step using Nystr\"{o}mPCG.

\subsection{Why is the Loss Under-optimized?}
\cref{fig:under_optimization} shows the run of \al{} with smallest L2RE for each PDE.
For each run, \lbfgs{} stops making progress before reaching the maximum number of iterations.
\lbfgs{} uses \textit{strong Wolfe line search}, as it is needed to maintain the stability of \lbfgs{} \cite{nocedal2006numerical}.
\lbfgs{} often terminates because it cannot find a positive step size satisfying these conditions---we have observed several instances where \lbfgs{} picks a step size of zero (\cref{fig:line_search_multi_pde} in \cref{sec:under_optimization_additional}), leading to early stopping.
Perversely, \lbfgs{} stops in these cases without reaching a critical point: 
the gradient norm is around $10^{-2}$ or $10^{-3}$ 
(see the bottom row of \cref{fig:under_optimization}).
The gradient still contains useful information for improving the loss.

\subsection{NysNewton-CG (NNCG)}
\label{subsec:NNCG} 
We can avoid premature termination by using a damped version of Newton's method with \textit{Armijo line search}.
The Armijo conditions use only a subset of the strong Wolfe conditions.
Under only Armijo conditions, \lbfgs{} is unstable; we require a different 
approximation to the Hessian ($p\times p$ for a neural net with $p$ parameters) that does not require storing ($\mathcal O(p^2)$) or inverting ($\mathcal O(p^3)$) the Hessian.
% We do not explicitly form and invert the Hessian, which would take $\mathcal{O}(p^2)$ memory and $\mathcal{O}(p^3)$ time, where $p$ is the number of parameters in the neural network.
Instead, we run a Newton-CG algorithm that solves for the Newton step using preconditioned conjugate gradient (PCG).
This algorithm can be implemented efficiently with Hessian-vector products. These can be computed $\mathcal{O}\left((\nres+\nbc)p\right)$ time \cite{pearlmutter1994fast}.
\cref{sec:loss_landscape} shows that the Hessian is ill-conditioned with fast spectral decay, so CG without preconditioning will converge slowly.
Hence we use Nystr\"{o}mPCG, a PCG method that is designed to solve linear systems with fast spectral decay \cite{frangella2023randomized}.
The resulting algorithm is called NysNewton-CG (abbreviated NNCG); a full description of the algorithm appears in \cref{sec:under_optimization_additional}.

\subsection{Performance of NNCG}
\begin{figure*}
    \centering
    \includegraphics[scale=0.33]{figs/under_optimization.pdf}
    \caption{Performance of NNCG and GD after \al. 
    (Top) NNCG reduces the loss by a factor greater than 10 in all instances, while GD fails to make progress. (Bottom) Furthermore, NNCG significantly reduces the gradient norm on the convection and wave problems, while GD fails to do so.}
    \label{fig:under_optimization}
\end{figure*}

\begin{figure*}
    \centering
    \includegraphics[scale=0.33]{figs/solution_evolutions.pdf}
    \caption{Absolute errors of the PINN solution at optimizer switch points. 
    The first column shows errors after Adam, the second column shows errors after running \lbfgs{} following Adam, and the third column shows the errors after running NNCG folllowing \al{}.
    \lbfgs{} improves the solution obtained from first running Adam, and NNCG further improves the solution even after \al{} stops making progress. 
    Note that Adam solution errors (left-most column) are presented at separate scales as these solutions are far off from the exact solutions. }
    \label{fig:solution_evolutions}
\end{figure*}

% \begin{table}[t]
%     \caption{L2RE after fine-tuning by NNCG and GD. NNCG outperforms both GD and the original \al{} results.}
%     \vskip 0.15in
%     \centering
%     \scriptsize
%     \begin{tabular}{|c|c|c|c|}
%         \hline
%         Optimizer & Convection & Reaction & Wave \\
%         \hline
%         \al{} & 4.19e-3 & 1.92e-2 & 5.52e-2 \\
%         \hline
%         \aln{} & \textbf{1.94e-3} & \textbf{9.92e-3} & \textbf{1.27e-2} \\
%         \hline
%         \alg{} & 4.19e-3 & 1.92e-2 & 5.52e-2 \\
%         \hline
%     \end{tabular}
%     \label{tab:nncg_l2re_improvement}
% \end{table}

\begin{table*}[t]
    \caption{Loss and L2RE after fine-tuning by NNCG and GD. NNCG outperforms both GD and the original \al{} results.}
    \vskip 0.15in
    \centering
    \scriptsize
    \begin{tabular}{|c|c|c|c|c|c|c|c|} 
    \hline 
    \multirow{2}{*}{Optimizer} & \multicolumn{2}{c|}{Convection} & \multicolumn{2}{c|}{Reaction} & \multicolumn{2}{c|}{Wave} \\ \cline{2-7}
                               & Loss & L2RE & Loss & L2RE & Loss & L2RE \\ \hline 
        \al{} & 5.95e-6 & 4.19e-3 & 5.26e-6 & 1.92e-2 & 1.12e-3 & 5.52e-2 \\
        \hline
        \aln{} & \textbf{3.63e-7} & \textbf{1.94e-3} & \textbf{2.89e-7} & \textbf{9.92e-3} & \textbf{6.13e-5} & \textbf{1.27e-2} \\
        \hline
        \alg{} & 5.95e-6 & 4.19e-3 & 5.26e-6 & 1.92e-2 & 1.12e-3 & 5.52e-2 \\
        \hline
    \end{tabular}
    \label{tab:nncg_l2re_improvement}
\end{table*}

\cref{fig:under_optimization} shows that NNCG significantly improves both the loss and gradient norm of the solution when applied after \al{}, while \cref{fig:solution_evolutions} visualizes how NNCG improves the absolute error (pointwise) of the PINN solution when applied after \al{}.
Furthermore, \cref{tab:nncg_l2re_improvement} shows that NNCG also improves the L2RE of the PINN solution.
In contrast, applying gradient descent (GD) after \al{} improves neither the loss nor the L2RE. This result is unsurprising, as our theory predicts that NNCG will work better than GD for an ill-conditioned loss (\cref{sec:theory}). 

\subsection{Why Not Use NNCG Directly After Adam?}
\label{subsec:why_not_nncg}
Since NNCG improves the PINN solution and uses simpler line search conditions than \lbfgs, it is tempting to replace \lbfgs{} with NNCG entirely.
However, NNCG is slower than \lbfgs{}: the \lbfgs{} update can be computed in $\mathcal O(mp)$ time, where $m$ is the memory parameter, while just a single Hessian-vector product for computing the NNCG update requires $\mathcal{O}\left((\nres+\nbc)p\right)$ time. \cref{tab:wall_clock_time_comparison} shows NNCG takes 5, 20, and 322 more times per-iteration as \lbfgs{} on convection, reaction, and wave respectively. 
% However, the time to compute the NNCG update is much longer than that of the L-BFGS update (\pnote{add theoretical complexities to appendix}).
Consequently, we should run \al{} to make as much progress as possible before switching to NNCG.

% \begin{itemize}
    % \item Show the gradient norms of the best runs for each PDE -- they are not close to zero. This means the loss can be optimized further, which is especially important for PINNs since high-precision is needed.
    % \item For proof-of-concept, we apply an additional damped Newton-CG fine-tuning step after \al. This should work if \lbfgs{} has gotten us to a (nearly)-convex region.
    % \item Since the loss landscape has fast spectral decay, we should not use vanilla CG. However, Nystr\"{o}m preconditioning is appropriate for this exact scenario. Consequently, we develop NysNewton-CG. 
    % \item Show improvements from NysNewton-CG in both gradient norm and loss. We can get XXX times improvement in l2re for each problem on the hardest coefficient setting.
    % \item Demonstrate that NysNewton-CG is better than no preconditioning. This can be done by looking at a semilog plot of the loss when using NysNewton-CG and gradient descent, and comparing the slopes.
    % \item Note possible improvements to NysNewton-CG: Damping could be made automatic, optimizer is also quite slow due to hvps
    % \item Improve on state-of-the-art results that are using Adam + L-BFGS (look at PINNacle for ideas)
% \end{itemize}

% \pnote{Add wall-clock times to L2RE table}
\section{Experiments}
\label{sec:Experiments} 

We conduct several experiments across different problem settings to assess the efficiency of our proposed method. Detailed descriptions of the experimental settings are provided in \cref{sec:apendix_experiments}.
%We conduct experiments on optimizing PINNs for convection, wave PDEs, and a reaction ODE. 
%These equations have been studied in previous works investigating difficulties in training PINNs; we use the formulations in \citet{krishnapriyan2021characterizing, wang2022when} for our experiments. 
%The coefficient settings we use for these equations are considered challenging in the literature \cite{krishnapriyan2021characterizing, wang2022when}.
%\cref{sec:problem_setup_additional} contains additional details.

%We compare the performance of Adam, \lbfgs{}, and \al{} on training PINNs for all three classes of PDEs. 
%For Adam, we tune the learning rate by a grid search on $\{10^{-5}, 10^{-4}, 10^{-3}, 10^{-2}, 10^{-1}\}$.
%For \lbfgs, we use the default learning rate $1.0$, memory size $100$, and strong Wolfe line search.
%For \al, we tune the learning rate for Adam as before, and also vary the switch from Adam to \lbfgs{} (after 1000, 11000, 31000 iterations).
%These correspond to \al{} (1k), \al{} (11k), and \al{} (31k) in our figures.
%All three methods are run for a total of 41000 iterations.

%We use multilayer perceptrons (MLPs) with tanh activations and three hidden layers. These MLPs have widths 50, 100, 200, or 400.
%We initialize these networks with the Xavier normal initialization \cite{glorot2010understanding} and all biases equal to zero.
%Each combination of PDE, optimizer, and MLP architecture is run with 5 random seeds.

%We use 10000 residual points randomly sampled from a $255 \times 100$ grid on the interior of the problem domain. 
%We use 257 equally spaced points for the initial conditions and 101 equally spaced points for each boundary condition.

%We assess the discrepancy between the PINN solution and the ground truth using $\ell_2$ relative error (L2RE), a standard metric in the PINN literature. Let $y = (y_i)_{i = 1}^n$ be the PINN prediction and $y' = (y'_i)_{i = 1}^n$ the ground truth. Define
%\begin{align*}
%    \mathrm{L2RE} = \sqrt{\frac{\sum_{i = 1}^n (y_i - y'_i)^2}{\sum_{i = 1}^n y'^2_i}} = \sqrt{\frac{\|y - y'\|_2^2}{\|y'\|_2^2}}.
%\end{align*}
%We compute the L2RE using all points in the $255 \times 100$ grid on the interior of the problem domain, along with the 257 and 101 points used for the initial and boundary conditions.

%We develop our experiments in PyTorch 2.0.0 \cite{paszke2019pytorch} with Python 3.10.12.
%Each experiment is run on a single NVIDIA Titan V GPU using CUDA 11.8.
%The code for our experiments is available at \href{https://github.com/pratikrathore8/opt_for_pinns}{https://github.com/pratikrathore8/opt\_for\_pinns}.


\subsection{2D Allen Cahn Equation}
\begin{figure*}[t]
    \centering
    \includegraphics[scale=0.38]{figs/Burgers_operator.pdf}
    \caption{1D Burgers' Equation (Operator Learning): Steady-state solutions for different initializations $u_0$ under varying viscosity $\varepsilon$: (a) $\varepsilon = 0.5$, (b) $\varepsilon = 0.1$, (c) $\varepsilon = 0.05$. The results demonstrate that all final test solutions converge to the correct steady-state solution. (d) Illustration of the evolution of a test initialization $u_0$ following homotopy dynamics. The number of residual points is $\nres = 128$.}
    \label{fig:Burgers_result}
\end{figure*}
First, we consider the following time-dependent problem:
\begin{align}
& u_t = \varepsilon^2 \Delta u - u(u^2 - 1), \quad (x, y) \in [-1, 1] \times [-1, 1] \nonumber \\
& u(x, y, 0) = - \sin(\pi x) \sin(\pi y) \label{eq.hom_2D_AC}\\
& u(-1, y, t) = u(1, y, t) = u(x, -1, t) = u(x, 1, t) = 0. \nonumber
\end{align}
We aim to find the steady-state solution for this equation with $\varepsilon = 0.05$ and define the homotopy as:
\begin{equation}
    H(u, s, \varepsilon) = (1 - s)\left(\varepsilon(s)^2 \Delta u - u(u^2 - 1)\right) + s(u - u_0),\nonumber
\end{equation}
where $s \in [0, 1]$. Specifically, when $s = 1$, the initial condition $u_0$ is automatically satisfied, and when $s = 0$, it recovers the steady-state problem. The function $\varepsilon(s)$ is given by
\begin{equation}
\varepsilon(s) = 
\left\{\begin{array}{l}
s, \quad s \in [0.05, 1], \\
0.05, \quad s \in [0, 0.05].
\end{array}\right.\label{eq:epsilon_t}
\end{equation}

Here, $\varepsilon(s)$ varies with $s$ during the first half of the evolution. Once $\varepsilon(s)$ reaches $0.05$, it remains fixed, and only $s$ continues to evolve toward $0$. As shown in \cref{fig:2D_Allen_Cahn_Equation}, the relative $L_2$ error by homotopy dynamics is $8.78 \times 10^{-3}$, compared with the result obtained by PINN, which has a $L_2$ error of $9.56 \times 10^{-1}$. This clearly demonstrates that the homotopy dynamics-based approach significantly improves accuracy.

\subsection{High Frequency Function Approximation }
We aim to approximate the following function:
$u=    \sin(50\pi x), \quad x \in [0,1].$
The homotopy is defined as $H(u,\varepsilon) = u - \sin(\frac{1}{\varepsilon}\pi x), $
where $\varepsilon \in [\frac{1}{50},\frac{1}{15}]$.

\begin{table}[htbp!]
    \caption{Comparison of the lowest loss achieved by the classical training and homotopy dynamics for different values of $\varepsilon$ in approximating $\sin\left(\frac{1}{\varepsilon} \pi x\right)$
    }
    \vskip 0.15in
    \centering
    \tiny
    \begin{tabular}{|c|c|c|c|c|} 
    \hline 
    $ $ & $\varepsilon = 1/15$ & $\varepsilon = 1/35$ & $\varepsilon = 1/50$ \\ \hline 
    Classical Loss                & 4.91e-6     & 7.21e-2     & 3.29e-1       \\ \hline 
    Homotopy Loss $L_H$                      & 1.73e-6     & 1.91e-6     & \textbf{2.82e-5}       \\ \hline
    \end{tabular}
    % On convection, \al{} provides 14.2$\times$ and 1.97$\times$ improvement over Adam or \lbfgs{} on L2RE. 
    % On reaction, \al{} provides 1.10$\times$ and 1.99$\times$ improvement over Adam or \lbfgs{} on L2RE.
    % On wave, \al{} provides 6.32$\times$ and 6.07$\times$ improvement over Adam or \lbfgs{} on L2RE.}
    \label{tab:loss_approximate}
\end{table}

As shown in \cref{fig:high_frequency_result}, due to the F-principle \cite{xu2024overview}, training is particularly challenging when approximating high-frequency functions like $\sin(50\pi x)$. The loss decreases slowly, resulting in poor approximation performance. However, training based on homotopy dynamics significantly reduces the loss, leading to a better approximation of high-frequency functions. This demonstrates that homotopy dynamics-based training can effectively facilitate convergence when approximating high-frequency data. Additionally, we compare the loss for approximating functions with different frequencies $1/\varepsilon$ using both methods. The results, presented in \cref{tab:loss_approximate}, show that the homotopy dynamics training method consistently performs well for high-frequency functions.





\subsection{Burgers Equation}
In this example, we adopt the operator learning framework to solve for the steady-state solution of the Burgers equation, given by:
\begin{align}
& u_t+\left(\frac{u^2}{2}\right)_x - \varepsilon u_{xx}=\pi \sin (\pi x) \cos (\pi x), \quad x \in[0, 1]\nonumber\\
& u(x, 0)=u_0(x),\label{eq:1D_Burgers} \\
& u(0, t)=u(1, t)=0, \nonumber 
\end{align}
with Dirichlet boundary conditions, where $u_0 \in L_{0}^2((0, 1); \mathbb{R})$ is the initial condition and $\varepsilon \in \mathbb{R}$ is the viscosity coefficient. We aim to learn the operator mapping the initial condition to the steady-state solution, $G^{\dagger}: L_{0}^2((0, 1); \mathbb{R}) \rightarrow H_{0}^r((0, 1); \mathbb{R})$, defined by $u_0 \mapsto u_{\infty}$ for any $r > 0$. As shown in Theorem 2.2 of \cite{KREISS1986161} and Theorems 2.5 and 2.7 of \cite{hao2019convergence}, for any $\varepsilon > 0$, the steady-state solution is independent of the initial condition, with a single shock occurring at $x_s = 0.5$. Here, we use DeepONet~\cite{lu2021deeponet} as the network architecture. 
The homotopy definition, similar to ~\cref{eq.hom_2D_AC}, can be found in \cref{Ap:operator}. The results can be found in \cref{fig:Burgers_result} and \cref{tab:loss_burgers}. Experimental results show that the homotopy dynamics strategy performs well in the operator learning setting as well.


\begin{table}[htbp!]
    \caption{Comparison of loss between classical training and homotopy dynamics for different values of $\varepsilon$ in the Burgers equation, along with the MSE distance to the ground truth shock location, $x_s$.}
    \vskip 0.15in
    \centering
    \tiny
    \begin{tabular}{|c|c|c|c|c|} 
    \hline  
    $ $ & $\varepsilon = 0.5$ & $\varepsilon = 0.1$ & $\varepsilon = 0.05$ \\ \hline 
    Homotopy Loss $L_H$                &  7.55e-7     & 3.40e-7     & 7.77e-7       \\ \hline 
    L2RE                      & 1.50e-3     & 7.00e-4     & 2.52e-2       \\ \hline
        MSE Distance $x_s$                      & 1.75e-8     & 9.14e-8      & 1.2e-3      \\ \hline
    \end{tabular}
    % On convection, \al{} provides 14.2$\times$ and 1.97$\times$ improvement over Adam or \lbfgs{} on L2RE. 
    % On reaction, \al{} provides 1.10$\times$ and 1.99$\times$ improvement over Adam or \lbfgs{} on L2RE.
    % On wave, \al{} provides 6.32$\times$ and 6.07$\times$ improvement over Adam or \lbfgs{} on L2RE.}
    \label{tab:loss_burgers}
\end{table}



% \begin{itemize}
%     \item Relate the curvature in the problem to the differential operator. Use this to demonstrate why the problem is ill-conditioned
%     \item Give an argument for why using Adam + L-BFGS is better than just using L-BFGS outright. The idea is that Adam lowers the errors to the point where the rest of the optimization becomes convex \ldots
%     \item Show why we need second-order methods. We would like to prove that once we are close to the optimum, second-order methods will give condition-number free linear convergence. Specialize this to the Gauss-Newton setting, with the randomized low-rank approximation.
%     % \item Show that it is not possible to get superlinear convergence under the interpolation assumption with an overparameterized neural network. This should be true b/c the Hessian at the optimum will have rank $\min(n, d)$, and when $d > n$, this guarantees that we cannot have strong convexity.
% \end{itemize}
\section{Conclusion}
In this work, we propose a simple yet effective approach, called SMILE, for graph few-shot learning with fewer tasks. Specifically, we introduce a novel dual-level mixup strategy, including within-task and across-task mixup, for enriching the diversity of nodes within each task and the diversity of tasks. Also, we incorporate the degree-based prior information to learn expressive node embeddings. Theoretically, we prove that SMILE effectively enhances the model's generalization performance. Empirically, we conduct extensive experiments on multiple benchmarks and the results suggest that SMILE significantly outperforms other baselines, including both in-domain and cross-domain few-shot settings.

\section*{Acknowledgements}
Y.Y. and W.H. was supported by National Institute of General Medical Sciences through grant 1R35GM146894. The work of Y.X. was supported by the Project of Hetao Shenzhen-HKUST Innovation Cooperation Zone HZQB-KCZYB-2020083.


% \section{Electronic Submission}
% \label{submission}

% Submission to ICML 2024 will be entirely electronic, via a web site
% (not email). Information about the submission process and \LaTeX\ templates
% are available on the conference web site at:
% \begin{center}
% \textbf{\texttt{http://icml.cc/}}
% \end{center}

% The guidelines below will be enforced for initial submissions and
% camera-ready copies. Here is a brief summary:
% \begin{itemize}
% \item Submissions must be in PDF\@. 
% \item \textbf{New to this year}: If your paper has appendices, submit the appendix together with the main body and the references \textbf{as a single file}. Reviewers will not look for appendices as a separate PDF file. So if you submit such an extra file, reviewers will very likely miss it.
% \item Page limit: The main body of the paper has to be fitted to 8 pages, excluding references and appendices; the space for the latter two is not limited. For the final version of the paper, authors can add one extra page to the main body.
% \item \textbf{Do not include author information or acknowledgements} in your
%     initial submission.
% \item Your paper should be in \textbf{10 point Times font}.
% \item Make sure your PDF file only uses Type-1 fonts.
% \item Place figure captions \emph{under} the figure (and omit titles from inside
%     the graphic file itself). Place table captions \emph{over} the table.
% \item References must include page numbers whenever possible and be as complete
%     as possible. Place multiple citations in chronological order.
% \item Do not alter the style template; in particular, do not compress the paper
%     format by reducing the vertical spaces.
% \item Keep your abstract brief and self-contained, one paragraph and roughly
%     4--6 sentences. Gross violations will require correction at the
%     camera-ready phase. The title should have content words capitalized.
% \end{itemize}

% \subsection{Submitting Papers}

% \textbf{Paper Deadline:} The deadline for paper submission that is
% advertised on the conference website is strict. If your full,
% anonymized, submission does not reach us on time, it will not be
% considered for publication. 

% \textbf{Anonymous Submission:} ICML uses double-blind review: no identifying
% author information may appear on the title page or in the paper
% itself. \cref{author info} gives further details.

% \textbf{Simultaneous Submission:} ICML will not accept any paper which,
% at the time of submission, is under review for another conference or
% has already been published. This policy also applies to papers that
% overlap substantially in technical content with conference papers
% under review or previously published. ICML submissions must not be
% submitted to other conferences and journals during ICML's review
% period.
% %Authors may submit to ICML substantially different versions of journal papers
% %that are currently under review by the journal, but not yet accepted
% %at the time of submission.
% Informal publications, such as technical
% reports or papers in workshop proceedings which do not appear in
% print, do not fall under these restrictions.

% \medskip

% Authors must provide their manuscripts in \textbf{PDF} format.
% Furthermore, please make sure that files contain only embedded Type-1 fonts
% (e.g.,~using the program \texttt{pdffonts} in linux or using
% File/DocumentProperties/Fonts in Acrobat). Other fonts (like Type-3)
% might come from graphics files imported into the document.

% Authors using \textbf{Word} must convert their document to PDF\@. Most
% of the latest versions of Word have the facility to do this
% automatically. Submissions will not be accepted in Word format or any
% format other than PDF\@. Really. We're not joking. Don't send Word.

% Those who use \textbf{\LaTeX} should avoid including Type-3 fonts.
% Those using \texttt{latex} and \texttt{dvips} may need the following
% two commands:

% {\footnotesize
% \begin{verbatim}
% dvips -Ppdf -tletter -G0 -o paper.ps paper.dvi
% ps2pdf paper.ps
% \end{verbatim}}
% It is a zero following the ``-G'', which tells dvips to use
% the config.pdf file. Newer \TeX\ distributions don't always need this
% option.

% Using \texttt{pdflatex} rather than \texttt{latex}, often gives better
% results. This program avoids the Type-3 font problem, and supports more
% advanced features in the \texttt{microtype} package.

% \textbf{Graphics files} should be a reasonable size, and included from
% an appropriate format. Use vector formats (.eps/.pdf) for plots,
% lossless bitmap formats (.png) for raster graphics with sharp lines, and
% jpeg for photo-like images.

% The style file uses the \texttt{hyperref} package to make clickable
% links in documents. If this causes problems for you, add
% \texttt{nohyperref} as one of the options to the \texttt{icml2024}
% usepackage statement.


% \subsection{Submitting Final Camera-Ready Copy}

% The final versions of papers accepted for publication should follow the
% same format and naming convention as initial submissions, except that
% author information (names and affiliations) should be given. See
% \cref{final author} for formatting instructions.

% The footnote, ``Preliminary work. Under review by the International
% Conference on Machine Learning (ICML). Do not distribute.'' must be
% modified to ``\textit{Proceedings of the
% $\mathit{41}^{st}$ International Conference on Machine Learning},
% Vienna, Austria, PMLR 235, 2024.
% Copyright 2024 by the author(s).''

% For those using the \textbf{\LaTeX} style file, this change (and others) is
% handled automatically by simply changing
% $\mathtt{\backslash usepackage\{icml2024\}}$ to
% $$\mathtt{\backslash usepackage[accepted]\{icml2024\}}$$
% Authors using \textbf{Word} must edit the
% footnote on the first page of the document themselves.

% Camera-ready copies should have the title of the paper as running head
% on each page except the first one. The running title consists of a
% single line centered above a horizontal rule which is $1$~point thick.
% The running head should be centered, bold and in $9$~point type. The
% rule should be $10$~points above the main text. For those using the
% \textbf{\LaTeX} style file, the original title is automatically set as running
% head using the \texttt{fancyhdr} package which is included in the ICML
% 2024 style file package. In case that the original title exceeds the
% size restrictions, a shorter form can be supplied by using

% \verb|\icmltitlerunning{...}|

% just before $\mathtt{\backslash begin\{document\}}$.
% Authors using \textbf{Word} must edit the header of the document themselves.

% \section{Format of the Paper}

% All submissions must follow the specified format.

% \subsection{Dimensions}




% The text of the paper should be formatted in two columns, with an
% overall width of 6.75~inches, height of 9.0~inches, and 0.25~inches
% between the columns. The left margin should be 0.75~inches and the top
% margin 1.0~inch (2.54~cm). The right and bottom margins will depend on
% whether you print on US letter or A4 paper, but all final versions
% must be produced for US letter size.
% Do not write anything on the margins.

% The paper body should be set in 10~point type with a vertical spacing
% of 11~points. Please use Times typeface throughout the text.

% \subsection{Title}

% The paper title should be set in 14~point bold type and centered
% between two horizontal rules that are 1~point thick, with 1.0~inch
% between the top rule and the top edge of the page. Capitalize the
% first letter of content words and put the rest of the title in lower
% case.

% \subsection{Author Information for Submission}
% \label{author info}

% ICML uses double-blind review, so author information must not appear. If
% you are using \LaTeX\/ and the \texttt{icml2024.sty} file, use
% \verb+\icmlauthor{...}+ to specify authors and \verb+\icmlaffiliation{...}+ to specify affiliations. (Read the TeX code used to produce this document for an example usage.) The author information
% will not be printed unless \texttt{accepted} is passed as an argument to the
% style file.
% Submissions that include the author information will not
% be reviewed.

% \subsubsection{Self-Citations}

% If you are citing published papers for which you are an author, refer
% to yourself in the third person. In particular, do not use phrases
% that reveal your identity (e.g., ``in previous work \cite{langley00}, we
% have shown \ldots'').

% Do not anonymize citations in the reference section. The only exception are manuscripts that are
% not yet published (e.g., under submission). If you choose to refer to
% such unpublished manuscripts \cite{anonymous}, anonymized copies have
% to be submitted
% as Supplementary Material via OpenReview\@. However, keep in mind that an ICML
% paper should be self contained and should contain sufficient detail
% for the reviewers to evaluate the work. In particular, reviewers are
% not required to look at the Supplementary Material when writing their
% review (they are not required to look at more than the first $8$ pages of the submitted document).

% \subsubsection{Camera-Ready Author Information}
% \label{final author}

% If a paper is accepted, a final camera-ready copy must be prepared.
% %
% For camera-ready papers, author information should start 0.3~inches below the
% bottom rule surrounding the title. The authors' names should appear in 10~point
% bold type, in a row, separated by white space, and centered. Author names should
% not be broken across lines. Unbolded superscripted numbers, starting 1, should
% be used to refer to affiliations.

% Affiliations should be numbered in the order of appearance. A single footnote
% block of text should be used to list all the affiliations. (Academic
% affiliations should list Department, University, City, State/Region, Country.
% Similarly for industrial affiliations.)

% Each distinct affiliations should be listed once. If an author has multiple
% affiliations, multiple superscripts should be placed after the name, separated
% by thin spaces. If the authors would like to highlight equal contribution by
% multiple first authors, those authors should have an asterisk placed after their
% name in superscript, and the term ``\textsuperscript{*}Equal contribution"
% should be placed in the footnote block ahead of the list of affiliations. A
% list of corresponding authors and their emails (in the format Full Name
% \textless{}email@domain.com\textgreater{}) can follow the list of affiliations.
% Ideally only one or two names should be listed.

% A sample file with author names is included in the ICML2024 style file
% package. Turn on the \texttt{[accepted]} option to the stylefile to
% see the names rendered. All of the guidelines above are implemented
% by the \LaTeX\ style file.

% \subsection{Abstract}

% The paper abstract should begin in the left column, 0.4~inches below the final
% address. The heading `Abstract' should be centered, bold, and in 11~point type.
% The abstract body should use 10~point type, with a vertical spacing of
% 11~points, and should be indented 0.25~inches more than normal on left-hand and
% right-hand margins. Insert 0.4~inches of blank space after the body. Keep your
% abstract brief and self-contained, limiting it to one paragraph and roughly 4--6
% sentences. Gross violations will require correction at the camera-ready phase.

% \subsection{Partitioning the Text}

% You should organize your paper into sections and paragraphs to help
% readers place a structure on the material and understand its
% contributions.

% \subsubsection{Sections and Subsections}

% Section headings should be numbered, flush left, and set in 11~pt bold
% type with the content words capitalized. Leave 0.25~inches of space
% before the heading and 0.15~inches after the heading.

% Similarly, subsection headings should be numbered, flush left, and set
% in 10~pt bold type with the content words capitalized. Leave
% 0.2~inches of space before the heading and 0.13~inches afterward.

% Finally, subsubsection headings should be numbered, flush left, and
% set in 10~pt small caps with the content words capitalized. Leave
% 0.18~inches of space before the heading and 0.1~inches after the
% heading.

% Please use no more than three levels of headings.

% \subsubsection{Paragraphs and Footnotes}

% Within each section or subsection, you should further partition the
% paper into paragraphs. Do not indent the first line of a given
% paragraph, but insert a blank line between succeeding ones.

% You can use footnotes\footnote{Footnotes
% should be complete sentences.} to provide readers with additional
% information about a topic without interrupting the flow of the paper.
% Indicate footnotes with a number in the text where the point is most
% relevant. Place the footnote in 9~point type at the bottom of the
% column in which it appears. Precede the first footnote in a column
% with a horizontal rule of 0.8~inches.\footnote{Multiple footnotes can
% appear in each column, in the same order as they appear in the text,
% but spread them across columns and pages if possible.}

% \begin{figure}[ht]
% \vskip 0.2in
% \begin{center}
% \centerline{\includegraphics[width=\columnwidth]{icml_numpapers}}
% \caption{Historical locations and number of accepted papers for International
% Machine Learning Conferences (ICML 1993 -- ICML 2008) and International
% Workshops on Machine Learning (ML 1988 -- ML 1992). At the time this figure was
% produced, the number of accepted papers for ICML 2008 was unknown and instead
% estimated.}
% \label{icml-historical}
% \end{center}
% \vskip -0.2in
% \end{figure}

% \subsection{Figures}

% You may want to include figures in the paper to illustrate
% your approach and results. Such artwork should be centered,
% legible, and separated from the text. Lines should be dark and at
% least 0.5~points thick for purposes of reproduction, and text should
% not appear on a gray background.

% Label all distinct components of each figure. If the figure takes the
% form of a graph, then give a name for each axis and include a legend
% that briefly describes each curve. Do not include a title inside the
% figure; instead, the caption should serve this function.

% Number figures sequentially, placing the figure number and caption
% \emph{after} the graphics, with at least 0.1~inches of space before
% the caption and 0.1~inches after it, as in
% \cref{icml-historical}. The figure caption should be set in
% 9~point type and centered unless it runs two or more lines, in which
% case it should be flush left. You may float figures to the top or
% bottom of a column, and you may set wide figures across both columns
% (use the environment \texttt{figure*} in \LaTeX). Always place
% two-column figures at the top or bottom of the page.

% \subsection{Algorithms}

% If you are using \LaTeX, please use the ``algorithm'' and ``algorithmic''
% environments to format pseudocode. These require
% the corresponding stylefiles, algorithm.sty and
% algorithmic.sty, which are supplied with this package.
% \cref{alg:example} shows an example.

% \begin{algorithm}[tb]
%    \caption{Bubble Sort}
%    \label{alg:example}
% \begin{algorithmic}
%    \STATE {\bfseries Input:} data $x_i$, size $m$
%    \REPEAT
%    \STATE Initialize $noChange = true$.
%    \FOR{$i=1$ {\bfseries to} $m-1$}
%    \IF{$x_i > x_{i+1}$}
%    \STATE Swap $x_i$ and $x_{i+1}$
%    \STATE $noChange = false$
%    \ENDIF
%    \ENDFOR
%    \UNTIL{$noChange$ is $true$}
% \end{algorithmic}
% \end{algorithm}

% \subsection{Tables}

% You may also want to include tables that summarize material. Like
% figures, these should be centered, legible, and numbered consecutively.
% However, place the title \emph{above} the table with at least
% 0.1~inches of space before the title and the same after it, as in
% \cref{sample-table}. The table title should be set in 9~point
% type and centered unless it runs two or more lines, in which case it
% should be flush left.

% % Note use of \abovespace and \belowspace to get reasonable spacing
% % above and below tabular lines.

% \begin{table}[t]
% \caption{Classification accuracies for naive Bayes and flexible
% Bayes on various data sets.}
% \label{sample-table}
% \vskip 0.15in
% \begin{center}
% \begin{small}
% \begin{sc}
% \begin{tabular}{lcccr}
% \toprule
% Data set & Naive & Flexible & Better? \\
% \midrule
% Breast    & 95.9$\pm$ 0.2& 96.7$\pm$ 0.2& $\surd$ \\
% Cleveland & 83.3$\pm$ 0.6& 80.0$\pm$ 0.6& $\times$\\
% Glass2    & 61.9$\pm$ 1.4& 83.8$\pm$ 0.7& $\surd$ \\
% Credit    & 74.8$\pm$ 0.5& 78.3$\pm$ 0.6&         \\
% Horse     & 73.3$\pm$ 0.9& 69.7$\pm$ 1.0& $\times$\\
% Meta      & 67.1$\pm$ 0.6& 76.5$\pm$ 0.5& $\surd$ \\
% Pima      & 75.1$\pm$ 0.6& 73.9$\pm$ 0.5&         \\
% Vehicle   & 44.9$\pm$ 0.6& 61.5$\pm$ 0.4& $\surd$ \\
% \bottomrule
% \end{tabular}
% \end{sc}
% \end{small}
% \end{center}
% \vskip -0.1in
% \end{table}

% Tables contain textual material, whereas figures contain graphical material.
% Specify the contents of each row and column in the table's topmost
% row. Again, you may float tables to a column's top or bottom, and set
% wide tables across both columns. Place two-column tables at the
% top or bottom of the page.

% \subsection{Theorems and such}
% The preferred way is to number definitions, propositions, lemmas, etc. consecutively, within sections, as shown below.
% \begin{definition}
% \label{def:inj}
% A function $f:X \to Y$ is injective if for any $x,y\in X$ different, $f(x)\ne f(y)$.
% \end{definition}
% Using \cref{def:inj} we immediate get the following result:
% \begin{proposition}
% If $f$ is injective mapping a set $X$ to another set $Y$, 
% the cardinality of $Y$ is at least as large as that of $X$
% \end{proposition}
% \begin{proof} 
% Left as an exercise to the reader. 
% \end{proof}
% \cref{lem:usefullemma} stated next will prove to be useful.
% \begin{lemma}
% \label{lem:usefullemma}
% For any $f:X \to Y$ and $g:Y\to Z$ injective functions, $f \circ g$ is injective.
% \end{lemma}
% \begin{theorem}
% \label{thm:bigtheorem}
% If $f:X\to Y$ is bijective, the cardinality of $X$ and $Y$ are the same.
% \end{theorem}
% An easy corollary of \cref{thm:bigtheorem} is the following:
% \begin{corollary}
% If $f:X\to Y$ is bijective, 
% the cardinality of $X$ is at least as large as that of $Y$.
% \end{corollary}
% \begin{assumption}
% The set $X$ is finite.
% \label{ass:xfinite}
% \end{assumption}
% \begin{remark}
% According to some, it is only the finite case (cf. \cref{ass:xfinite}) that is interesting.
% \end{remark}
% %restatable

% \subsection{Citations and References}

% Please use APA reference format regardless of your formatter
% or word processor. If you rely on the \LaTeX\/ bibliographic
% facility, use \texttt{natbib.sty} and \texttt{icml2024.bst}
% included in the style-file package to obtain this format.

% Citations within the text should include the authors' last names and
% year. If the authors' names are included in the sentence, place only
% the year in parentheses, for example when referencing Arthur Samuel's
% pioneering work \yrcite{Samuel59}. Otherwise place the entire
% reference in parentheses with the authors and year separated by a
% comma \cite{Samuel59}. List multiple references separated by
% semicolons \cite{kearns89,Samuel59,mitchell80}. Use the `et~al.'
% construct only for citations with three or more authors or after
% listing all authors to a publication in an earlier reference \cite{MachineLearningI}.

% Authors should cite their own work in the third person
% in the initial version of their paper submitted for blind review.
% Please refer to \cref{author info} for detailed instructions on how to
% cite your own papers.

% Use an unnumbered first-level section heading for the references, and use a
% hanging indent style, with the first line of the reference flush against the
% left margin and subsequent lines indented by 10 points. The references at the
% end of this document give examples for journal articles \cite{Samuel59},
% conference publications \cite{langley00}, book chapters \cite{Newell81}, books
% \cite{DudaHart2nd}, edited volumes \cite{MachineLearningI}, technical reports
% \cite{mitchell80}, and dissertations \cite{kearns89}.

% Alphabetize references by the surnames of the first authors, with
% single author entries preceding multiple author entries. Order
% references for the same authors by year of publication, with the
% earliest first. Make sure that each reference includes all relevant
% information (e.g., page numbers).

% Please put some effort into making references complete, presentable, and
% consistent, e.g. use the actual current name of authors.
% If using bibtex, please protect capital letters of names and
% abbreviations in titles, for example, use \{B\}ayesian or \{L\}ipschitz
% in your .bib file.

% \section*{Accessibility}
% Authors are kindly asked to make their submissions as accessible as possible for everyone including people with disabilities and sensory or neurological differences.
% Tips of how to achieve this and what to pay attention to will be provided on the conference website \url{http://icml.cc/}.

% \section*{Software and Data}

% If a paper is accepted, we strongly encourage the publication of software and data with the
% camera-ready version of the paper whenever appropriate. This can be
% done by including a URL in the camera-ready copy. However, \textbf{do not}
% include URLs that reveal your institution or identity in your
% submission for review. Instead, provide an anonymous URL or upload
% the material as ``Supplementary Material'' into the OpenReview reviewing
% system. Note that reviewers are not required to look at this material
% when writing their review.

% Acknowledgements should only appear in the accepted version.
%\newpage
%\section*{Acknowledgements}
%We would like to acknowledge helpful comments from the anonymous reviewers and area chairs, which have improved this submission.
%MU, PR, WL, and ZF gratefully acknowledge support from the National Science Foundation (NSF) Award IIS-2233762, the Office of Naval Research (ONR) Award N000142212825 and N000142312203, and the Alfred P. Sloan Foundation.
%LL gratefully acknowledges support from the U.S. Department of Energy [DE-SC0022953].

% \textbf{Do not} include acknowledgements in the initial version of
% the paper submitted for blind review.

% If a paper is accepted, the final camera-ready version can (and
% usually should) include acknowledgements.  Such acknowledgements
% should be placed at the end of the section, in an unnumbered section
% that does not count towards the paper page limit. Typically, this will 
% include thanks to reviewers who gave useful comments, to colleagues 
% who contributed to the ideas, and to funding agencies and corporate 
% sponsors that provided financial support.

\section*{Impact Statement}
This paper presents work whose goal is to advance the field of scientific machine learning. There are many potential societal consequences of our work, none which we feel must be specifically highlighted here.

% Authors are \textbf{required} to include a statement of the potential 
% broader impact of their work, including its ethical aspects and future 
% societal consequences. This statement should be in an unnumbered 
% section at the end of the paper (co-located with Acknowledgements -- 
% the two may appear in either order, but both must be before References), 
% and does not count toward the paper page limit. In many cases, where 
% the ethical impacts and expected societal implications are those that 
% are well established when advancing the field of Machine Learning, 
% substantial discussion is not required, and a simple statement such 
% as the following will suffice:

% ``This paper presents work whose goal is to advance the field of 
% Machine Learning. There are many potential societal consequences 
% of our work, none which we feel must be specifically highlighted here.''

% The above statement can be used verbatim in such cases, but we 
% encourage authors to think about whether there is content which does 
% warrant further discussion, as this statement will be apparent if the 
% paper is later flagged for ethics review.


% In the unusual situation where you want a paper to appear in the
% references without citing it in the main text, use \nocite
% \nocite{langley00}

\bibliography{references}
\bibliographystyle{icml2025}


%%%%%%%%%%%%%%%%%%%%%%%%%%%%%%%%%%%%%%%%%%%%%%%%%%%%%%%%%%%%%%%%%%%%%%%%%%%%%%%
%%%%%%%%%%%%%%%%%%%%%%%%%%%%%%%%%%%%%%%%%%%%%%%%%%%%%%%%%%%%%%%%%%%%%%%%%%%%%%%
% APPENDIX
%%%%%%%%%%%%%%%%%%%%%%%%%%%%%%%%%%%%%%%%%%%%%%%%%%%%%%%%%%%%%%%%%%%%%%%%%%%%%%%
%%%%%%%%%%%%%%%%%%%%%%%%%%%%%%%%%%%%%%%%%%%%%%%%%%%%%%%%%%%%%%%%%%%%%%%%%%%%%%%
\newpage
\appendix
\onecolumn
\section{Proofs of Theorems \ref{compare} and \ref{small}}
\label{sec:problem_setup_additional}
% Here we present the differential equations that we study in our experiments.

% \subsection{Convection}
% The one-dimensional convection problem is a hyperbolic PDE that can be used to model fluid flow, heat transfer, and biological processes.
% The convection PDE we study is
% \begin{align*}
%     \frac{\partial u}{\partial t} + \beta \frac{\partial u}{\partial x} = 0, & \quad x \in (0, 2\pi), t \in (0, 1), \\
%     u(x, 0) = \sin(x), & \quad x \in [0, 2\pi], \\
%     u(0, t) = u(2 \pi, t), & \quad t \in [0, 1]. 
% \end{align*}

% The analytical solution to this PDE is $u(x, t) = \sin(x - \beta t)$.
% We set $\beta = 40$ in our experiments.

% \subsection{Reaction}
% The one-dimensional reaction problem is a non-linear ODE which can be used to model chemical reactions.
% The reaction ODE we study is
% \begin{align*}
%     \frac{\partial u}{\partial t} - \rho u (1 - u) = 0, & \quad x \in (0, 2\pi), t \in (0, 1) \\
%     u(x, 0) = \exp \left( -\frac{(x - \pi)^2}{2 (\pi / 4)^2} \right), & \quad x \in [0, 2\pi], \\
%     u(0, t) = u(2 \pi, t), & \quad t \in [0, 1].
% \end{align*}

% The analytical solution to this ODE is $u(x, t) = \frac{h(x) e^{\rho t}}{h(x) e^{\rho t} + 1 - h(x)}$, where $h(x) = \exp \left( -\frac{(x - \pi)^2}{2 (\pi / 4)^2} \right)$.
% We set $\rho = 5$ in our experiments.

% % \subsubsection{Reaction-diffusion}
% % The one-dimensional reaction-diffusion problem is a non-linear PDE that can be used to model the concentration of chemical substances.
% % The reaction-diffusion PDE we study is
% % \begin{align*}
% %     \frac{\partial u}{\partial t} - \nu \frac{\partial^2 u}{\partial x^2} - \rho u (1 - u) = 0, & \quad x \in (0, 2\pi), t \in (0, 1), \\
% %     u(x, 0) = \exp \left( -\frac{(x - \pi)^2}{2 (\pi / 4)^2} \right), & \quad x \in [0, 2\pi], \\
% %     u(0, t) = u(2 \pi, t), & \quad t \in [0, 1].
% % \end{align*}

% % This PDE has no analytical solution; instead we calculate the solution using Strang splitting, as done in \cite{krishnapriyan2021characterizing}.
% % We set $\rho = 5$ and vary $\nu \in \{2, 3, 4, 5, 6\}$ in our experiments.

% \subsection{Wave}
% The one-dimensional wave problem is a hyperbolic PDE that often arises in acoustics, electromagnetism, and fluid dynamics.
% The wave PDE we study is
% \begin{align*}
%     \frac{\partial^2 u}{\partial t^2} - 4 \frac{\partial^2 u}{\partial x^2} = 0, & \quad x \in (0, 1), t \in (0, 1), \\
%     u(x, 0) = \sin(\pi x) + \frac{1}{2} \sin(\beta \pi x), & \quad x \in [0, 1], \\
%     \frac{\partial u(x, 0)}{\partial t} = 0, & \quad x \in [0, 1], \\
%     u(0, t) = u(1, t) = 0, & \quad t \in [0, 1].
% \end{align*}

% The analytical solution to this PDE is $u(x, t) = \sin(\pi x) \cos(2 \pi t) + \frac{1}{2} \sin(\beta \pi x) \cos(2 \beta \pi t)$.
% We set $\beta = 5$ in our experiments.
\subsection{\( \lambda_{\text{min}}(\vS\vS^{\top}) > 0 \)}\label{ss}
In this subsection, we consider a two-layer neural network defined as follows:
\begin{equation}
    \phi(\boldsymbol{x};\boldsymbol{\theta}) := \frac{1}{\sqrt{m}} \sum_{k=1}^{m} a_k \sigma (\boldsymbol{\omega}_k^{\top} \boldsymbol{x}),
\end{equation}
where the activation function is given by
\begin{equation}
    \sigma(z) = \text{ReLU}(z) = \max\{z,0\}.
\end{equation}
We assume that the weights and biases are sampled as follows: 
\begin{equation}
    \boldsymbol{\omega}_k \sim N\left(0, \boldsymbol{I}_d\right), \quad a_k \sim N(0,1),
\end{equation}
where \( N(0,1) \) denotes the standard Gaussian distribution.


The kernels characterizing the training dynamics take the following form:\begin{align}
    k^{[a]}(\boldsymbol{x},\boldsymbol{x}'):=&\mathbf{E}_{\boldsymbol{\omega}}\sigma(\boldsymbol{\omega}^{\top}\boldsymbol{x})\sigma(\boldsymbol{\omega}^{\top}\boldsymbol{x}')\notag\\k^{[\boldsymbol{\omega}]}(\boldsymbol{x},\boldsymbol{x}'):=&\mathbf{E}_{(a,\boldsymbol{\omega})}a^2\sigma'(\boldsymbol{\omega}^{\top}\boldsymbol{x})\sigma'(\boldsymbol{\omega}^{\top}\boldsymbol{x}')\boldsymbol{x}\cdot\boldsymbol{x}'.
\end{align} The Gram matrices, denoted as $\boldsymbol{K}^{[a]}$ and $\boldsymbol{K}^{[\boldsymbol{\omega}]}$, corresponding to an infinite-width two-layer network with the activation function $\sigma$, can be expressed as follows:\begin{align}
    &K_{ij}^{[a]}=k^{[a]}(\boldsymbol{x}_i,\boldsymbol{x}_j),~\boldsymbol{K}^{[a]}=(K_{ij}^{[a]})_{n\times n},\notag\\& K_{ij}^{[\boldsymbol{\omega}]}=k^{[\boldsymbol{\omega}]}(\boldsymbol{x}_i,\boldsymbol{x}_j),~\boldsymbol{K}^{[\boldsymbol{\omega}]}=(K_{ij}^{[\boldsymbol{\omega}]})_{n\times n}.
\end{align}

\begin{lemma}[\cite{allen2019convergence}]
\label{positive}    The matrices \(\boldsymbol{K}^{[\boldsymbol{\omega}]}\) and \(\boldsymbol{K}^{[a]}\) are strictly positive.
\end{lemma}
It is easy to check that
\begin{equation}
    \boldsymbol{K}^{[\boldsymbol{\omega}]} + \boldsymbol{K}^{[a]} = \lim_{m\to \infty} \mathbf{S} \mathbf{S}^{\top}
\end{equation}
based on the law of large numbers. Furthermore, we can show that the accuracy decreases exponentially as the width of the neural network increases.

\begin{definition}[\cite{vershynin2018high}]
    A random variable $X$ is sub-exponential if and only if its sub-exponential norm is finite i.e.\begin{equation}
        \|X\|_{\psi_1}:=\inf\{s>0\mid\mathbf{E}_X[e^{|X|/s}\le 2.]
    \end{equation} Furthermore, the chi-square random variable $X$ is a sub-exponential random variable and $C_{\psi,d}:=\|X\|_{\psi_1}$.
\end{definition}

\begin{lemma}\label{matrice sub}
    Suppose that $\boldsymbol{w} \sim N\left(0, \boldsymbol{I}_d\right), a \sim N(0,1)$ and given $\boldsymbol{x}_i, \boldsymbol{x}_j \in \Omega$. Then we have
    
(i) if $\mathrm{X}:=\sigma\left(\boldsymbol{w}^{\top} \boldsymbol{x}_i\right) \sigma\left(\boldsymbol{x} \cdot \boldsymbol{x}_j\right)$, then $\|\mathrm{X}\|_{\psi_1} \leq d C_{\psi, d}$.

(ii) if $\mathrm{X}:=a^2 \sigma^{\prime}\left(\boldsymbol{w}^{\top} \boldsymbol{x}_i\right) \sigma^{\prime}\left(\boldsymbol{w}^{\top} \boldsymbol{x}_j\right) \boldsymbol{x}_i \cdot \boldsymbol{x}_j$, then $\|\mathrm{X}\|_{\psi_1} \leq d C_{\psi, d}$.
\end{lemma}
\begin{proof}
(i) $|\mathrm{X}| \leq d\|\boldsymbol{w}\|_2^2=d \mathrm{Z}$ and
$$
\begin{aligned}
\|\mathrm{X}\|_{\psi_1} & =\inf \left\{s>0 \mid \mathbf{E}_{\mathrm{X}} \exp (|\mathrm{X}| / s) \leq 2\right\} \\
& =\inf \left\{s>0 \mid \mathbf{E}_{\boldsymbol{w}} \exp \left(\left|\sigma\left(\boldsymbol{w}^{\top} \boldsymbol{x}_i\right) \sigma\left(\boldsymbol{w}^{\top} \boldsymbol{x}_j\right)\right| / s\right) \leq 2\right\} \\
& \leq \inf \left\{s>0 \mid \mathbf{E}_{\boldsymbol{w}} \exp \left(d\|\boldsymbol{w}\|_2^2 / s\right) \leq 2\right\} \\
& =\inf \left\{s>0 \mid \mathbf{E}_{\mathrm{Z}} \exp (d|\mathrm{Z}| / s) \leq 2\right\} \\
& =d \inf \left\{s>0 \mid \mathbf{E}_{\mathrm{Z}} \exp (|\mathrm{Z}| / s) \leq 2\right\} \\
& =d\left\|\chi^2(d)\right\|_{\psi_1} \\
& \leq d C_{\psi, d}
\end{aligned}
$$
(ii) $|\mathrm{X}| \leq d|a|^2 \leq d \mathrm{Z}$ and $\|\mathrm{X}\|_{\psi_1} \leq d C_{\psi, d}$.
\end{proof}

\begin{proposition}[sub-exponential Bernstein's inequality \cite{vershynin2018high}]\label{vershynin}
    Suppose that $\mathrm{X}_1, \ldots, \mathrm{X}_m$ are i.i.d. sub-exponential random variables with $\mathbf{E} \mathrm{X}_1=\mu$, then for any $s \geq 0$ we have
$$
\mathbf{P}\left(\left|\frac{1}{m} \sum_{k=1}^m \mathrm{X}_k-\mu\right| \geq s\right) \leq 2 \exp \left(-C_0 m \min \left(\frac{s^2}{\left\|\mathrm{X}_1\right\|_{\psi_1}^2}, \frac{s}{\left\|\mathrm{X}_1\right\|_{\psi_1}}\right)\right),
$$
where $C_0$ is an absolute constant.
\end{proposition}

\begin{proposition}\label{eig pos}
    Given $\delta \in(0,1)$, $\boldsymbol{w} \sim N\left(0, \boldsymbol{I}_d\right), a \sim N(0,1)$ and the sample set $S=\left\{\boldsymbol{x}_i \right\}_{i=1}^n \subset \Omega$ with $\boldsymbol{x}_i$ 's drawn i.i.d. with uniformly distributed. If $m \geq \frac{16 n^2 d^2 C_{\psi, d}}{C_0 \lambda^2} \log \frac{4 n^2}{\delta}$ then with probability at least $1-\delta$ over the choice of $\boldsymbol{\theta}(0)$, we have
$$
\lambda_{\min }\left(\vS\vS^\top\right)\geq\frac{3}{4}(\lambda_{\text{min}}(\vK^{[a]})+\lambda_{\text{min}}(\vK^{[\boldsymbol{\omega}]})).
$$
\end{proposition}
\begin{proof}
    For any $\varepsilon>0$, we define \begin{align}
        \Omega_{ij}^{[a]}&:=\left\{\boldsymbol{\theta}\mid \left|(\vS\vS^\top)_{ij}(\boldsymbol{\theta})-K_{ij}^{[a]}-K_{ij}^{[\boldsymbol{\omega}]}\right|\le\frac{\varepsilon}{n}\right\}.
    \end{align}

    Setting $\varepsilon\le ndC_{\psi,d}$, by Proposition \ref{vershynin} and Lemma \ref{matrice sub}, we have \begin{align}
        \mathbf{P}(\Omega_{ij})&\ge 1-2\exp\left(-\frac{mC_0\varepsilon^2}{n^2d^2C_{\psi,d}}\right).
    \end{align}

    Therefore, with probability at least \[\left[1-2\exp\left(-\frac{mC_0\varepsilon^2}{n^2d^2C_{\psi,d}^2}\right)\right]^{2n^2}\ge 1-4n^2\exp\left(-\frac{mC_0\varepsilon^2}{n^2d^2C_{\psi,d}^2}\right)\] over the choice of $\boldsymbol{\theta}$, we have \begin{align}
        &\left\|\vS\vS^\top(\boldsymbol{\theta})-\vK^{[a]}-\vK^{[\boldsymbol{\omega}]}\right\|_F\le \varepsilon.
    \end{align}
Hence by taking $\varepsilon=\frac{\lambda_1}{4}$ and $\delta=4n^2\exp\left(-\frac{mC_0\lambda_1^2}{16n^2d^2C_{\psi,d}^2}\right)$, where $\lambda_1=\min\{\lambda_{\text{min}}(\vK^{[a]}),\lambda_{\text{min}}(\vK^{[\boldsymbol{\omega}]})\}$\begin{align}
       \lambda_{\min }\left(\vS\vS^\top\right)\geq\frac{3}{4}(\lambda_{\text{min}}(\vK^{[a]})+\lambda_{\text{min}}(\vK^{[\boldsymbol{\omega}]})).
    \end{align}
\end{proof}

Combining Lemma \ref{positive} and Proposition \ref{eig pos}, we obtain that under the conditions stated in Proposition \ref{eig pos}, the following holds with high probability:
\begin{equation}
    \lambda_{\text{min}}(\vS \vS^{\top}) > 0.
\end{equation}






\subsection{Proof of Theorem \ref{compare}}
We can analysis the smallest eigenvalue of the problems based on the following lemma:
\begin{lemma}[\cite{li1999lidskii}]\label{comparelink}
   Let $\vA$ be an $n \times n$ Hermitian matrix and let $\tilde{\vA}=\vT^* \vA \vT$. Then we have
$$
\lambda_{\text{min}}\left(\vT^* \vT\right) \leq  \frac{\lambda_{\text{min}}(\tilde{\vA})}{\lambda_{\text{min}}(\vA)} \leq \lambda_{\text{max}}\left(\vT^* \vT\right).
$$

\end{lemma}
\begin{proof}[Proof of Theorem \ref{compare}]
    We first show that \( \lambda_{\text{min}}(\vK_\varepsilon) > 0 \), which follows directly from Lemma~\ref{comparelink}:
    \[
    \lambda_{\text{min}}(\vK_\varepsilon) \geq \lambda_{\text{min}}(\vS\vS^\top) \cdot \lambda_{\text{min}}(\vD_\varepsilon\vD_\varepsilon^\top) > 0.
    \]
    Therefore, at the beginning of gradient descent, the kernel of the gradient descent step is strictly positive. We then define \( T \) as
    \begin{equation}
        T := \inf\{t \mid \boldsymbol{\theta}(t) \not\in N(\boldsymbol{\theta}(0))\},\label{t_1}
    \end{equation}
    where
    \[
    N(\boldsymbol{\theta}) := \left\{\boldsymbol{\theta} \mid \|\boldsymbol{K}_\varepsilon(\boldsymbol{\theta}(t)) - \boldsymbol{K}_\varepsilon(\boldsymbol{\theta}(0))\|_F \leq \frac{1}{2} \lambda_{\text{min}}(\vK_\varepsilon) \right\}.
    \]

    We now analyze the evolution of the loss function:
    \begin{align}
    \frac{\D L(\vtheta(t))}{\D t} &= \nabla_{\vtheta} L(\vtheta) \frac{\D \vtheta}{\D t} \notag \\
    &= -\frac{1}{n^2} \vl \vD_\varepsilon \vS \vS^{\top} \vD_\varepsilon^{\top} \vl^{\top} \notag \\
    &\leq -\frac{2}{n} \lambda_{\text{min}}(\vK_\varepsilon(\vtheta(t))) L(\vtheta(t)),
    \end{align}
    where we use the fact that \( \vl \cdot \vl^\top = 2n L(\vtheta(t)) \).

    Furthermore, for \( t \in [0,T] \), we have 
    \[
    \|\boldsymbol{K}_\varepsilon(\boldsymbol{\theta}(t)) - \boldsymbol{K}_\varepsilon(\boldsymbol{\theta}(0))\|_F \leq \frac{1}{2} \lambda_{\text{min}}(\vK_\varepsilon).
    \]
    This implies
    \[
    \lambda_{\text{min}}(\vK_\varepsilon(\vtheta(t))) \geq \frac{1}{2} \lambda_{\text{min}}(\boldsymbol{K}_\varepsilon(\boldsymbol{\theta}(0))).
    \]
    Therefore, we obtain
    \begin{align}
    \frac{\D L(\vtheta(t))}{\D t} \leq -\frac{1}{n} \lambda_{\text{min}}(\boldsymbol{K}_\varepsilon(\boldsymbol{\theta}(0))) L(\vtheta(t)),
    \end{align}
    for \( t \in [0,T] \). Solving this differential inequality yields
    \begin{equation}
    L(\vtheta(t)) \leq L(\vtheta(0))\exp\left(-\frac{\lambda_{\text{min}}(\vK_{\varepsilon})}{n} t\right)
    \end{equation}
    for all \( t \in [0, T] \).

    Finally, for the inequality
    \begin{equation}
    \lambda_{\text{min}}(\vK_{\varepsilon}) \leq \lambda_{\text{min}}(\vS\vS^{\top}) \lambda_{\text{max}}(\vD_{\varepsilon}\vD_{\varepsilon}^\top),
    \end{equation}
    it follows directly from Lemma~\ref{comparelink}.
\end{proof}
\subsection{Proof of Theorem \ref{small}}
\begin{proof}[Proof of Theorem \ref{small}]
    First, we have  
    \begin{align}
        u(\varepsilon_{k+1}) &= u(\varepsilon_k) + (\varepsilon_{k+1}-\varepsilon_k)u'(\varepsilon_k) + \frac{1}{2}(\varepsilon_{k+1}-\varepsilon_k)^2u''(\xi_k) \notag \\
        &= u(\varepsilon_k) + (\varepsilon_{k+1}-\varepsilon_k)h(\varepsilon_k,u(\varepsilon_k)) + \frac{1}{2}(\varepsilon_{k+1}-\varepsilon_k)^2u''(\xi_k),
    \end{align}
    where \( \xi_k \) lies between \( \varepsilon_{k+1} \) and \( \varepsilon_k \) and depends on \( \vx \). Therefore, we obtain  
    \begin{equation}
        e(\varepsilon_{k+1}) = e(\varepsilon_k) + (\varepsilon_{k+1}-\varepsilon_k)(h(\varepsilon_k,u(\varepsilon_k)) - h(\varepsilon_k,U(\varepsilon_k))) + \frac{1}{2}(\varepsilon_{k+1}-\varepsilon_k)^2u''(\xi_k),
    \end{equation}
    where \( e(\varepsilon_k) = u(\varepsilon_k) - U(\varepsilon_k) \). Then, we have  
    \begin{align}
     &\|e(\varepsilon_{k+1})\|_{H^2(\Omega)} \notag \\
     =&\|e(\varepsilon_k)\|_{H^2(\Omega)} + (\varepsilon_{k+1}-\varepsilon_k)\|h(\varepsilon_k,u(\varepsilon_k)) - h(\varepsilon_k,U(\varepsilon_k))\|_{H^2(\Omega)} \notag \\
     &+ \frac{1}{2}(\varepsilon_{k+1}-\varepsilon_k)^2\|u''(\xi_k)\|_{H^2(\Omega)} \notag \\
     \leq& \|e(\varepsilon_k)\|_{H^2(\Omega)} + (\varepsilon_{k+1}-\varepsilon_k)K_{\varepsilon_k}\|e(\varepsilon_k)\|_{H^2(\Omega)} + \frac{1}{2} \frac{\varepsilon_0-\varepsilon_n}{n} \tau \notag \\
     \leq& \|e(\varepsilon_k)\|_{H^2(\Omega)} + K \cdot \frac{\varepsilon_0-\varepsilon_n}{n} \|e(\varepsilon_k)\|_{H^2(\Omega)} + \frac{1}{2} \frac{\varepsilon_0-\varepsilon_n}{n} \tau.
    \end{align}
    Recalling that \( e_0 = \|u (\varepsilon_0)-U(\varepsilon_0)\|_{H^2(\Omega)}\), we obtain  
    \begin{align}
    \|e(\varepsilon_{n})\|_{H^2(\Omega)} 
        &\leq e_0\left(1+K\cdot \frac{\varepsilon_0-\varepsilon_n}{n}\right)^n+\frac{\tau}{2} \frac{\varepsilon_0-\varepsilon_n}{n} \sum_{n=0}^{n-1} \left(1+K\cdot \frac{\varepsilon_0-\varepsilon_n}{n}\right)^n \notag \\
        &= e_0\left(1+K\cdot \frac{\varepsilon_0-\varepsilon_n}{n}\right)^n+\frac{\tau}{2} \frac{\left(1+K\cdot \frac{\varepsilon_0-\varepsilon_n}{n}\right)^n-1}{K} \notag \\
        &\leq \frac{\tau(e^{K(\varepsilon_0-\varepsilon_n)}-1)}{2K}+e_0e^{K(\varepsilon_0-\varepsilon_n)},
    \end{align}
    where the last step follows from the inequality  
    \begin{equation}
    (1+a)^m \leq e^{m a}, \notag
    \end{equation}
    for \( a>0 \).
\end{proof}

\begin{corollary}[Convergence of Homotopy Functions]\label{cosmall}
    Suppose the assumptions in Theorem \ref{small} hold, and \( H(\varepsilon_n, u) \) is Lipschitz continuous in \( H^2(\Omega) \), i.e.,
    \[
    \|H( u_1,\varepsilon_n) - H( u_2,\varepsilon_n)\|_{H^2(\Omega)} \leq L \|u_1 - u_2\|_{H^2(\Omega)}.
    \]
    Then, we have  
    \begin{align}
        &\|H( U(\varepsilon_n),\varepsilon_n)\|_{H^2(\Omega)} \notag\\\leq& L\left[e_0e^{K(\varepsilon_0-\varepsilon_n)}+\frac{\tau(e^{K(\varepsilon_0-\varepsilon_n)}-1)}{2K}\right] \ll 1.
    \end{align}
\end{corollary}

\begin{proof}
    The proof follows directly from the result in Theorem \ref{small}.
\end{proof}

\subsection{Discussion on \( e_0 \)}\label{e0}

In Theorem~\ref{small} and Corollary~\ref{cosmall}, one important assumption is that we assume \( e_0 \) is small. Here, we discuss why this assumption is reasonable. 

First, we use physics-informed neural networks (PINNs) to solve the following equations:
\begin{equation}
\left\{
\begin{array}{ll}
\mathcal{L}_\varepsilon u = f(u), & \text{in } \Omega, \\
\mathcal{B} u = g(x), & \text{on } \partial \Omega,
\end{array}
\right.
\label{eq:gen_pde1}
\end{equation}
where \( \mathcal{L}_\varepsilon \) is a differential operator defining the PDE with certain parameters, \( \mathcal{B} \) is an operator associated with the boundary and/or initial conditions, and \( \Omega \subseteq \mathbb{R}^d \).

The corresponding continuum loss function is given by:
\begin{align}
   L_c(\boldsymbol{\theta}) \coloneqq  \frac{1}{2} \int_\Omega \left( \mathcal{L}_\varepsilon u(\boldsymbol{x}; \boldsymbol{\theta}) - f(u) \right)^2 \D \vx
   + \frac{\lambda}{2} \int_{\partial\Omega} \left( \mathcal{B} u(\boldsymbol{x}; \boldsymbol{\theta}) - g(\boldsymbol{x}) \right)^2 \D \vx. 
   \label{loss1}
\end{align}

We assume this loss function satisfies a regularity condition:
\begin{assumption}
   Let \( u_* \) be the exact solution of Eq.~(\ref{eq:gen_pde1}). Then, there exists a constant \( C \) such that
   \begin{equation}
       \| u(\boldsymbol{x}; \boldsymbol{\theta}) - u_*(\boldsymbol{x}) \|_{H^2(\Omega)} \leq C L_c(\boldsymbol{\theta}).
   \end{equation}
\end{assumption}

The above assumption holds in many cases. For example, based on \cite{grisvard2011elliptic}, when \( \mathcal{L} \) is a linear elliptic operator with smooth coefficients, and \( f(u) \) reduces to \( f(\boldsymbol{x}) \in L^2(\Omega) \), and if \( \Omega \) is a polygonal domain (e.g., \( [0,1]^d \)), then, provided the boundary conditions are always satisfied, the assumption holds.

Therefore, we only need to ensure that \( L_c(\boldsymbol{\theta}_s) \) is sufficiently small, where \( \boldsymbol{\theta}_s \) denotes the learned parameters at convergence. Here, \( L_c(\boldsymbol{\theta}_s) \) can be divided into three sources of error: approximation error, generalization error, and training error:

\begin{align}
    \boldsymbol{\theta}_c &= \arg \min_{\boldsymbol{\theta}} L_c(\boldsymbol{\theta}) 
    = \arg \min_{\boldsymbol{\theta}} \frac{1}{2} \int_\Omega \left( \mathcal{L}_\varepsilon u(\boldsymbol{x}; \boldsymbol{\theta}) - f(u(\boldsymbol{x})) \right)^2 \D \vx
   + \frac{\lambda}{2} \int_{\partial\Omega} \left( \mathcal{B} u(\boldsymbol{x}; \boldsymbol{\theta}) - g(\boldsymbol{x}) \right)^2 \D \vx, \notag\\
    \boldsymbol{\theta}_d &= \arg \min_{\boldsymbol{\theta}} L(\boldsymbol{\theta}) 
    = \arg \min_{\boldsymbol{\theta}} \frac{1}{2n_r} \sum_{i=1}^{n_r} \left( \mathcal{L}_\varepsilon u(\boldsymbol{x}_r^i; \boldsymbol{\theta}) - f(u(\boldsymbol{x}_r^i; \boldsymbol{\theta})) \right)^2
    + \frac{\lambda}{2n_b} \sum_{j=1}^{n_b} \left( \mathcal{B} u(\boldsymbol{x}_b^j; \boldsymbol{\theta}) - g(\boldsymbol{x}_b^j) \right)^2,
\end{align}
where \( \boldsymbol{x}_r^i, \boldsymbol{x}_b^j \) are sampled points as defined in Eq.~(\ref{loss}).

The error decomposition can then be expressed as:
\begin{align}
    \mathbb{E} L_c(\boldsymbol{\theta}_s) 
    &\leq L_c(\boldsymbol{\theta}_c) + \mathbb{E} L(\boldsymbol{\theta}_c) - L_c(\boldsymbol{\theta}_c) 
    + \mathbb{E} L(\boldsymbol{\theta}_d) - \mathbb{E} L(\boldsymbol{\theta}_c) 
    + \mathbb{E} L(\boldsymbol{\theta}_s) - \mathbb{E} L(\boldsymbol{\theta}_d) 
    + \mathbb{E} L_c(\boldsymbol{\theta}_s) - \mathbb{E} L(\boldsymbol{\theta}_s) \notag \\
    &\leq \underbrace{L_c(\boldsymbol{\theta}_c)}_{\text{approximation error}}
    + \underbrace{\mathbb{E} L(\boldsymbol{\theta}_c) - L_c(\boldsymbol{\theta}_c) 
    + \mathbb{E} L_c(\boldsymbol{\theta}_s) - \mathbb{E} L(\boldsymbol{\theta}_s)}_{\text{generalization error}}
    + \underbrace{\mathbb{E} L(\boldsymbol{\theta}_s) - \mathbb{E} L(\boldsymbol{\theta}_d)}_{\text{training error}},
\end{align}where the last inequality is due to $\mathbb{E} L(\boldsymbol{\theta}_d) - \mathbb{E} L(\boldsymbol{\theta}_c) \le 0$ based on the definition of $\vtheta_d$.

The approximation error describes how closely the neural network approximates the exact solution of the PDEs. If \( f \) is a Lipschitz continuous function, \( \mathcal{L}_\varepsilon \) is Lipschitz continuous from \( W^{2,1}(\Omega) \to L^1(\Omega) \), and \( \mathcal{B} \) is Lipschitz continuous from \( L^1(\partial\Omega) \to L^1(\partial\Omega) \), with \( u(\boldsymbol{x}; \boldsymbol{\theta}), u_* \in W^{2,\infty}(\bar{\Omega}) \) and \( \partial \Omega \in C^1(\Omega) \), then we have
\begin{align}
    L_c(\boldsymbol{\theta}) &= \int_\Omega \left( \mathcal{L}_\varepsilon u(\boldsymbol{x}; \boldsymbol{\theta}) - f(u(\boldsymbol{x})) \right)^2 - \left( \mathcal{L}_\varepsilon u_* - f(u_*) \right)^2 \, d\boldsymbol{x} + \frac{\lambda}{2} \int_{\partial\Omega} \left( \mathcal{B} u(\boldsymbol{x}; \boldsymbol{\theta}) - g(\boldsymbol{x}) \right)^2 - \left( \mathcal{B} u_* - g(\boldsymbol{x}) \right)^2 \, d\boldsymbol{x} \notag\\&\leq C_1 \left( \|\mathcal{L}_\varepsilon (u(\boldsymbol{x}; \boldsymbol{\theta}) - u_*)\|_{L^1(\Omega)} + \|f( u(\boldsymbol{x}; \boldsymbol{\theta})) - f(u_*)\|_{L^1(\Omega)} \right)  + C_2 \|\mathcal{B} (u(\boldsymbol{x}; \boldsymbol{\theta}) - u_*)\|_{L^1(\partial\Omega)} \notag \\
    &\leq C_3 \|u(\boldsymbol{x}; \boldsymbol{\theta}) - u_*\|_{W^{2,1}(\Omega)} + C_4 \|u(\boldsymbol{x}; \boldsymbol{\theta}) - u_*\|_{W^{1,1}(\Omega)} \notag \\
    &\leq C \|u(\boldsymbol{x}; \boldsymbol{\theta}) - u_*\|_{W^{2,1}(\Omega)},
\end{align}
where the second inequality follows from the trace theorem \cite{evans2022partial}. Therefore, we conclude that \( L_c(\boldsymbol{\theta}) \) can be bounded by \( \|u(\boldsymbol{x}; \boldsymbol{\theta}) - u_*\|_{W^{2,1}(\Omega)} \), which has been widely studied in the context of shallow neural networks \cite{siegel2020approximation} and deep neural networks \cite{yang2023nearly1}. These results show that if the number of neurons is sufficiently large, the error in this part becomes small.

For the generalization error, it arises from the fact that we have only a finite number of data points. This error can be bounded using Rademacher complexity \cite{yang2023nearly1,luo2020two}, which leads to a bound of \( \mathcal{O}\left(n_r^{-\frac{1}{2}}\right) + \mathcal{O}\left(n_b^{-\frac{1}{2}}\right) \). In other words, this error term is small when the number of sample points is large.

For the training error, Theorem~\ref{compare} shows that when \( \varepsilon \) is large in certain PDEs, the loss function can decay efficiently, reducing the training error to a small value.


\section{Details on Experiments}
\label{sec:apendix_experiments}
\subsection{Overall Experiments Settings}
\textbf{Examples.} We conduct experiments on function learning case: 1D Allen-Cahn equation, 2D Allen-Cahn equation, high frequency function approximation and operator learning for Burgers' equation. These equations have been studied in previous works investigating difficulties in solving numerically; we use the formulations in \citet{xu2020variational, ZHANG2024112638,hao2019convergence} for our experiments. 

\textbf{Network Structure.} We use multilayer perceptrons (MLPs) with tanh activations and three hidden layers with width 30.
We initialize these networks with the Xavier normal initialization \cite{glorot2010understanding} and all biases equal to zero.


\textbf{Training.} We use Adam to train the neural network and we tune the learning rate by a grid search on $\{10^{-5}, 10^{-4}, 10^{-3}, 10^{-2}\}$. All iterations continue until the loss stabilizes and no longer decreases significantly. %We use 10000 residual points randomly sampled from a $255 \times 100$ grid on the interior of the problem domain. 
%We use 257 equally spaced points for the initial conditions and 101 equally spaced points for each boundary condition.


\textbf{Device.} We develop our experiments in PyTorch 1.12.1 \cite{paszke2019pytorch} with Python 3.9.12.
Each experiment is run on a single NVIDIA 3070Ti GPU using CUDA 11.8.
%The code for our experiments is available at \href{https://github.com/pratikrathore8/opt_for_pinns}{https://github.com/pratikrathore8/opt\_for\_pinns}.

\subsection{1D Allen-Cahn Equation}

Number of residual points $\nres = 200$ and number of boundary points $\nbc=2$. In this example, we use forward Euler method to numerically solve the homotopy dynamics. And $\varepsilon_{0} = 0.1$ and $\varepsilon_n = 0.01$, here we choose $\Delta \varepsilon_k = 0.001$.

The results for using original training for this example \cref{fig:1d_allen_cahn_origin_result}. As shown in the figure, the original training method results in a large training error, leading to poor accuracy.

\begin{figure}[htp!]
    \centering
    \includegraphics[scale=0.5]{figs/1D_AC_Origin_result.png}
    \caption{Solution for 1D Allen-Cahn equation for origin training.}
    \label{fig:1d_allen_cahn_origin_result}
\end{figure}

\subsection{2D Allen-Cahn Equation}

Number of residual points $\nres = 50\times50$ and number of boundary points $\nbc=198$.In this example, we optimize using the Homotopy Loss. We set $s_{0} = 1.0$ and $s_n=0$, initially choosing $\Delta s= 0.1$, and later refining it to $\Delta t= 0.01$.    
When $s=0.05, \varepsilon(s) = 0.05$ we fix $\varepsilon = 0.05$ and gradually decrease $s$ to $0$.

The reference ground truth solution is obtained using the finite difference method with $N = 1000 \times 1000$  grid points. The result is shown below.
\begin{figure}[htp!]
    \centering
    \includegraphics[scale=0.5]{figs/2D_AD_Reference.png}
    \caption{Reference Solution for 2D Allen-Cahn equation.}
    \label{fig:2d_allen_cahn_reference}
\end{figure}




The result obtained using PINN is shown in the figure below. It is evident that the solution still deviates significantly from the ground truth solution.
\begin{figure}[htp!]
    \centering
    \includegraphics[scale=0.5]{figs/2D_AD_PINN.png}
    \caption{Solution for 2D Allen-Cahn equation for origin training.}
    \label{fig:2d_allen_cahn_origin_result}
\end{figure}
\subsection{High Frequency Function Approximation}

Number of residual points $\nres = 300$. In this example, we optimize using the Homotopy Loss. We set $\varepsilon_{0} = \frac{1}{15}$ and $\varepsilon_n=\frac{1}{50}$, the list for $\{\varepsilon_i\}$ is $[\frac{1}{15},\frac{1}{20},\frac{1}{25},\frac{1}{30},\frac{1}{35},\frac{1}{40},\frac{1}{45},\frac{1}{50}]$. From this example, we observe that the homotopy dynamics approach can also mitigate the slow training issue caused by the Frequency Principle (F-Principle) when neural networks approximate high-frequency functions.

\subsection{Operator Learning 1D Burgers' Equation}
\label{Ap:operator}
In this example, we apply homotopy dynamics to operator learning. The neural network architecture follows the DeepONet structure: \begin{equation}
\mathcal{G}_{\vtheta}(v)(y) = \sum_{k=1}^p \sum_{i=1}^n a_i^k \sigma\left(\sum_{j=1}^m \xi_{i j}^k v\left(x_j\right)+c_i^k\right) \sigma\left(w_k \cdot y+b_k\right).
\end{equation}

Here, $\sigma\left(w_k \cdot y+b_k\right)$ represents the trunk net, which takes the coordinates $y \in D^{\prime}$ as input, and $\sigma\left(\sum_{j=1}^m \xi_{i j}^k u\left(x_j\right)+c_i^k\right)$ represents the branch net, which takes the discretized function $v$ as input. We can interpret the trunk net as the basis functions for solving PDEs. For this example, the input is $u_0$ and the output is $u_{\infty}$. We still train using the homotopy loss. It is important to emphasize that, unlike conventional operator learning, which typically follows a supervised learning strategy, our approach adopts an unsupervised learning paradigm. This makes the training process significantly more challenging. The initial condition $u_0(x)$ is generated from a Gaussian random field with a Riesz kernel, denoted by $\text{GRF} \sim 
\mathcal{N}\left(0,49^2(-\Delta+49I)^{-4}\right)$ and $\Delta$ and $I$ represent the Laplacian and the identity. We utilize a spatial resolution of $128$ grids to represent both the input and output functions.

We want to find the steady state solution for this equation and $\epsilon = 0.05$. The homotopy is:
\begin{equation}
    H(u,s,\varepsilon) = (1-s)\left(\left(\frac{u^2}{2}\right)_x - \varepsilon(s) u_{xx} -\pi \sin (\pi x) \cos (\pi x)\right) + s(u-u_0),
\end{equation}
where $s \in [0,1]$. In particular, when $s = 1$, the initial condition $u_0$ automatically satisfies and when $s = 0$ becomes the steady state problem. And $\varepsilon(s)$ can be set to




\begin{equation}
\varepsilon(s) = 
\left\{\begin{array}{l}
s, \quad s \in [0.05,1],\\
0.05 \quad s\in [0,0.05].
\end{array}\right.\label{eq:epsilon_t}
\end{equation}

Here, $\varepsilon(s)$ varies with $s$ during the first half of the evolution. Once $\varepsilon(s)$ reaches $0.05$, it is fixed at $\varepsilon(s) = 0.05$, and only $s$ continues to evolve toward $0$.







%\section{Computing the Spectral Density of the \lbfgs{}-preconditioned Hessian}
\label{sec:lbfgs_spectral_info}

\subsection{How L-BFGS Preconditions} 
\label{subsec:how_lbfgs_preconditions}
To minimize \eqref{eq:pinn_prob_gen}, L-BFGS uses the update
\begin{equation}
\label{eq:l-bfgs}
w_{k+1} = w_k-\eta H_k \nabla L(w_k), 
\end{equation}
where $H_k$ is a matrix approximating the inverse Hessian.
We now show how \eqref{eq:l-bfgs} is equivalent to preconditioning the objective \eqref{eq:pinn_prob_gen}.
Define the coordinate transformation $w = H_k^{1/2}z$. 
By the chain rule, $\nabla L(z) = H_k^{1/2}\nabla L(w)$ and $H_L(z) = H^{1/2}_k H_L(w)H_k^{1/2}$. 
Thus, \eqref{eq:l-bfgs} is equivalent to
\begin{align}
\label{eq:precond}
    & z_{k+1} = z_k-\eta \nabla L(z_k), \\
    & w_{k+1} = H^{1/2}_kz_{k+1}. \nonumber
\end{align}

\cref{eq:precond} reveals how L-BFGS preconditions \eqref{eq:pinn_prob_gen}.
L-BFGS first takes a step in the \emph{preconditioned} $z$-space, where the conditioning is determined by $H_L(z)$, the preconditioned Hessian. 
Since $H_k$ approximates $H_L^{-1}(w)$, $H^{1/2}_k H_L(w) H_k^{1/2} \approx I_p$, so the condition number of $H_L(z)$ is much smaller than that of $H_L(w)$. 
Consequently, L-BFGS can take a step that makes more progress than a method like gradient descent, which performs no preconditioning at all. 
In the second phase, L-BFGS maps the progress in the preconditioned space back to the original space.
Thus, L-BFGS is able to make superior progress by transforming \eqref{eq:pinn_prob_gen} to another space where the conditioning is more favorable, which enables it to compute an update that better reduces the loss in \eqref{eq:pinn_prob_gen}.

\subsection{Preconditioned Spectral Density Computation}
Here we discuss how to compute the spectral density of the Hessian after preconditioning by L-BFGS.
This is the procedure we use to generate the figures in \cref{subsec:lbfgs_improvement}. 

\lbfgs{} stores a set of vector pairs given by the difference in consecutive iterates and gradients from most recent $m$ iterations (we use $m = 100$ in our experiments).
To compute the update direction $H_k \nabla f_k$, \lbfgs{} combines the stored vector pairs with a recursive scheme \cite{nocedal2006numerical}.
Defining
\[
    s_{k} = x_{k+1} - x_{k}, 
    \quad y_k = \nabla f_{k+1} - \nabla f_{k}, 
    \quad \rho_{k} = \frac{1}{y_{k}^{T}s_{k}}, 
    \quad \gamma_{k} = \frac{s_{k-1}^{T}y_{k-1}}{y_{k-1}^{T}y_{k-1}}, 
    \quad V_k = I - \rho_{k} y_{k} s_{k}^{T}, 
    \quad H_k^{0} = \gamma_{k} I,
\]
the formula for $H_k$ can be written as
\[
    H_{k} 
    = (V_{k-1}^{T} V_{k-m}^{T}) H_{k}^{0} (V_{k-m} V_{k-1}) 
    + \sum_{l=2}^{m} \rho_{k-l} (V_{k-1}^{T} \cdots V_{k-l+1}^{T}) s_{k-l} s_{k-l}^{T} (V_{k-l+1} \cdots V_{k-1})
    + \rho_{k-1} s_{k-1} s_{k-1}^{T}.
\]
Expanding the terms, we have for $j \in \{1, 2, \ldots, i\}$,
\[
    V_{k-i} \cdots V_{k-1} = I - \sum_{j=1}^{i} \rho_{k-j} y_{k-j} \tilde{v}_{k-j}^{T}
    \quad \text{where} \quad \tilde{v}_{k-j} = s_{k-j} - \sum_{l=1}^{j-1} (\rho_{k-l} y_{k-l}^{T} s_{k-j}) \tilde{v}_{k-l}.
\]
It follows that
\[
    H_{k} 
    = (I - \tilde{Y}\tilde{V}^{T})^{T} \gamma_{k} I (I - \tilde{Y}\tilde{V}^{T}) + \tilde{S} \tilde{S}^{T}
    = 
    \begin{bmatrix*}
        \sqrt{\gamma_k} (I - \tilde{Y}\tilde{V}^{T})^{T} & \tilde{S}
    \end{bmatrix*}
    \begin{bmatrix*}
        \sqrt{\gamma_k} (I - \tilde{Y}\tilde{V}^{T}) \\ 
        \tilde{S}^{T}.
    \end{bmatrix*}
    = \tilde{H}_k \tilde{H}_k^{T},
\]
where
\[
\begin{aligned}
    & \tilde{Y} = 
    \begin{bmatrix*} 
        \vert & & \vert \\
        \rho_{k-1} y_{k-1} & \cdots & \rho_{k-m} y_{k-m} \\
        \vert & & \vert \\
    \end{bmatrix*}, \\
    & \tilde{V} = 
    \begin{bmatrix*} 
        \vert & & \vert \\
        \tilde{v}_{k-1} & \cdots & \tilde{v}_{k-m} \\
        \vert & & \vert \\
    \end{bmatrix*}, \\
    & \tilde{S} = 
    \begin{bmatrix*} 
        \vert & & \vert \\
        \tilde{s}_{k-1} & \cdots & \tilde{s}_{k-m} \\
        \vert & & \vert \\
    \end{bmatrix*},
    \quad \tilde{s}_{k-1} = \sqrt{\rho_{k-1}} s_{k-1}, 
    ~ \tilde{s}_{k-l} = \sqrt{\rho_{k-l}} (V_{k-1}^{T} \cdots V_{k-l+1}^{T}) s_{k-l} ~ \text{for} ~ 2 \leq l \leq m.
\end{aligned}
\]
We now apply \cref{alg-unrolling-lbfgs} to unroll the above recurrence relations to compute columns of $\tilde Y, \tilde S$ and $\tilde V$. 

\begin{algorithm}[H]
  \centering
  \caption{Unrolling the \lbfgs{} Update}
  \label{alg-unrolling-lbfgs}
  \begin{algorithmic}
  \INPUT{saved directions $\{y_i\}_{i=k-1}^{k-m}$, saved steps $\{s_i\}_{i=k-1}^{k-m}$, saved inverse of inner products $\{\rho_i\}_{i=k-1}^{k-m}$}
    \STATE {$\tilde{y}_{k-1} = \rho_{k-1} y_{k-1}$}
    \STATE {$\tilde{v}_{k-1} = s_{k-1}$}
    \STATE {$\tilde{s}_{k-1} = \sqrt{\rho_{k-1}} s_{k-1}$}
    \FOR{$i = k-2, \dots, k-m$}
        \STATE {$\tilde{y}_i = \rho_i y_i$}
        \STATE {Set $\alpha = 0$}
        \FOR{$j = k-1, \dots, i+1$}
          \STATE {$\alpha = \alpha + (\tilde{y}_j^{T} s_i) \tilde{v}_j$}
        \ENDFOR
        \STATE {$\tilde{v}_i = s_i - \alpha$}
        \STATE {$\tilde{s}_i = \sqrt{\rho_i} (s_i - \alpha)$}
    \ENDFOR
  \OUTPUT{vectors $\{\tilde{y}_i, \tilde{v}_i, \tilde{s}_i\}_{i=k-1}^{k-m}$}
  \end{algorithmic}
\end{algorithm}

Since (non-zero) eigenvalues of $\tilde{H}_{k}^{T} H_L(w)\tilde{H}_{k}$ equal the eigenvalues of the preconditioned Hessian $H_{k} H_L(w) = \tilde{H}_k \tilde{H}_k^{T} H_L(w)$ (Theorem 1.3.22 of \citet{horn2012matrix}), we can analyze the spectrum of $\tilde{H}_{k}^{T}H_L(w)\tilde{H}_{k}$ instead.
This is advantageous since methods for calculating the spectral density of neural network Hessians are only compatible with symmetric matrices.

Since $\tilde{H}_{k}^{T} H_L(w)\tilde{H}_{k}$ is symmetric, we can use stochastic Lanczos quadrature (SLQ) \cite{golub2009matrices,lin2016approximating} to compute spectral density of this matrix.
SLQ only requires matrix-vector products with $\tilde H_k$ and Hessian-vector products, the latter of which may be efficiently computed via automatic differentiation; this is precisely what PyHessian does to compute spectral densities \cite{yao2020pyhessian}.

\begin{algorithm}[H]
  \centering
  \caption{Performing matrix-vector product}
  \label{alg-mvp}
  \begin{algorithmic}
  \INPUT{matrices $\tilde{Y}$, $\tilde{V}$, $\tilde{S}$ formed from resulting vectors from unrolling, vector $v$, and saved scaling factor for initializing diagonal matrix $\gamma_k$}
    \STATE {Split vector $v$ of length $\mathrm{size}(w) + m$ into $v_1$ of size $\mathrm{size}(w)$ and $v_2$ of size $m$}
    \STATE {$v' = \sqrt{\gamma_k}(v_1 - \tilde{V}\tilde{Y}^{T}v_1) + \tilde{S} v_2$}
    \STATE {Perform Hessian-vector-product on $v'$, and obtain $v''$}
    \STATE {Stack $\sqrt{\gamma_k}(v'' - \tilde{Y}\tilde{V}^{T}v'')$ and $\tilde{S}^{T}v''$, and obtain $v'''$}
  \OUTPUT{resulting vector $v'''$}
  \end{algorithmic}
\end{algorithm}

By combining the matrix-vector product procedure described in \cref{alg-mvp} with the Hessian-vector product operation, we are able to obtain spectral information of the preconditioned Hessian. 

\begin{figure*}
    \centering
    \includegraphics[trim={0 1.25cm 0 0}, clip, scale=0.45]{figs/spectral_density_reaction.pdf}
    \includegraphics[scale=0.45]{figs/spectral_density_wave.pdf}
    \caption{Spectral density of the Hessian and the preconditioned Hessian of each loss component after 41000 iterations of \al{} for the reaction and wave problems. The plots show the loss landscape of each component is ill-conditioned, and the conditioning of each loss component is improved by \lbfgs{}.}
    \label{fig:spectral_density_reaction_wave}
\end{figure*}
%\section{\al{} Generally Gives the Best Performance}
\label{sec:opt_comparison_additional}
\cref{fig:opt_comparison} shows that \al{} typically yields the best performance on both loss and L2RE across network widths. 
\begin{figure*}
    \centering
    \includegraphics[scale=0.4]{figs/loss_l2re_num_neurons.pdf}
    \caption{Performance of Adam, \lbfgs, and \al{} after tuning. 
    We find the learning rate $\eta^\star$ for each network width and optimization strategy that attains the lowest loss (L2RE) across all random seeds.
    The min, median, and max loss (L2RE) are calculated by taking the min, median, and max of the losses (L2REs) for learning rate $\eta^*$ across all random seeds.
    Each bar on the plot corresponds to the median, while the top and bottom error bars correspond to the max and min, respectively.
    The smallest min loss and L2RE are always attained by one of the \al{} strategies; the smallest median loss and L2RE are nearly always attained by one of the \al{} strategies.}
    \label{fig:opt_comparison}
\end{figure*}
%\section{Additional details on Under-optimization}
\label{sec:under_optimization_additional}

\subsection{Early Termination of \lbfgs{}}
\cref{fig:line_search_multi_pde} explains why \lbfgs{} terminates early for the convection, reaction, and wave problems.
We evaluate the loss at $10^{4}$ uniformly spaced points in the interval $[0, 1]$.
The orange stars in \cref{fig:line_search_multi_pde} are step sizes that satisfy the strong Wolfe conditions and the red dots are step sizes that \lbfgs{} examines during the line search.

\begin{figure}
    \centering
    \includegraphics[scale=0.45]{figs/line_search_multi_pde.pdf}
    \caption{Loss evaluated along the \lbfgs{} search direction at different stepsizes after 41000 iterations of \al{}. For convection and wave, the line search does not find a stepsize that satisfies the strong Wolfe conditions, even though there are plenty of such points. For reaction, the slope of the objective used in the line search procedure at the current iterate is less than a pre-defined threshold $10^{-9}$, so \lbfgs{} terminates without performing any line-search.}
    \label{fig:line_search_multi_pde}
\end{figure}

\subsection{NysNewton-CG (NNCG)}
Here we present the NNCG algorithm (\cref{alg-NNCG}) introduced in \cref{subsec:NNCG} and its associated subroutines RandomizedNystr{\"o}mApproximation (\cref{alg-RNA}), Nystr\"{o}mPCG (\cref{alg-nyspcg}), and Armijo (\cref{alg-armijo}).
At each iteration, NNCG first checks whether the Nystr\"{o}m preconditioner (stored in $U$ and  $\hat{\Lambda}$) for the Nystr\"{o}mPCG method needs to be updated.
If so, the preconditioner is recomputed using the RandomizedNystr{\"o}mApproximation subroutine.
From here, the Newton step $d_k$ is computed using Nystr\"{o}mPCG; we warm start the PCG algorithm using the Newton step $d_{k - 1}$ from the previous iteration.
After computing the Newton step, we compute the step size $\eta_k$ using Armijo line search --- this guarantees that the loss will decrease when we update the parameters.
Finally, we update the parameters using $\eta_k$ and $d_k$.

In our experiments, we set $\eta = 1, K = 2000, s = 60, F = 20, \epsilon = 10^{-16}, M = 1000, \alpha = 0.1$, and $\beta = 0.5$.
We tune $\mu \in [10^{-5}, 10^{-4}, 10^{-3}, 10^{-2}, 10^{-1}]$; we find that $\mu = 10^{-2}, 10^{-1}$ work best in practice.
\cref{fig:under_optimization_intro,fig:under_optimization} show the NNCG run that attains the lowest loss after tuning $\mu$.

\begin{algorithm}[H]
	\centering
	\caption{NysNewton-CG (NNCG)}
	\label{alg-NNCG}
	\begin{algorithmic}
	\INPUT{Initialization $w_0$, max. learning rate $\eta$, number of iterations $K$, preconditioner sketch size $s$, preconditioner update frequency $F$, damping parameter $\mu$, CG tolerance $\epsilon$, CG max. iterations $M$, backtracking parameters $\alpha, \beta$}
    \STATE{$d_{-1} = 0$}
    \FOR{$k = 0, \dots, K - 1$}
        \IF{$k$ is a multiple of $F$} 
            \STATE{$[U, \hat{\Lambda}] = \textrm{RandomizedNystr{\"o}mApproximation}(H_{L}(w_k), s)$} \COMMENT{Update Nystr{\"o}m preconditioner every $F$ iterations}
        \ENDIF
        \STATE{$d_k = \textrm{Nystr{\"o}mPCG}(H_{L}(w_k), \nabla L(w_k), d_{k - 1}, U, \hat{\Lambda}, s, \mu, \epsilon, M)$} \COMMENT{Damped Newton step $(H_L(w_k) + \mu I)^{-1} \nabla L(w_k)$}
        \STATE{$\eta_k = \textrm{Armijo}(L, w_k, \nabla L(w_k), -d_k, \eta)$} \COMMENT{Compute step size via line search}
        \STATE{$w_{k+1} = w_k - \eta_k d_k$} \COMMENT{Update parameters}
    \ENDFOR
	\end{algorithmic}
\end{algorithm}

The RandomizedNystr{\"o}mApproximation subroutine (\cref{alg-RNA}) is used in NNCG to compute the preconditioner for Nystr\"{o}mPCG.
The algorithm returns the top-$s$ approximate eigenvectors and eigenvalues of the input matrix $M$.
Within NNCG, the sketch computation $Y = MQ$ is implemented using Hessian-vector products.
The portion in red is a fail-safe that allows for the preconditioner to be computed when $H$ is an indefinite matrix.
For further details, please see \citet{frangella2023randomized}.


\begin{algorithm}[H] 
   \caption{RandomizedNystr{\"o}mApproximation}
   \label{alg-RNA}
    \begin{algorithmic}
       % \INPUT{sketch $Y\in \R^{p\times r_j}$ of $H_{S_j}$, orthogonalized test matrix $Q\in \R^{p\times r_j}$}
       \INPUT{Symmetric matrix $M$, sketch size $s$}
       \STATE{$S = \textrm{randn}(p, s)$} \COMMENT{Generate test matrix}
       \STATE{$Q = \textrm{qr\_econ}(S)$}
       \STATE{$Y = M Q$} \COMMENT{Compute sketch}
       \STATE $\nu = \sqrt{p} \text{eps}(\text{norm}(Y, 2))$ \hfill \COMMENT{Compute shift}
       \STATE $Y_{\nu} = Y + \nu Q$ \hfill \COMMENT{Add shift for stability}
       \STATE $\lambda = 0$ \hfill \COMMENT{Additional shift may be required for positive definiteness}
       \STATE $C = \text{chol}(Q^TY_\nu)$ \hfill \COMMENT{Cholesky decomposition: $C^{T}C = Q^{T}Y_\nu$}
       \textcolor{red}{
       \IF {chol fails}
        \STATE Compute $[W, \Gamma] = \mathrm{eig}(Q^T Y_\nu)$ \hfill \COMMENT{$Q^T Y_\nu$ is small and square}
        \STATE Set $\lambda = \lambda_{\min}(Q^T Y_\nu)$
        \STATE $R = W(\Gamma + |\lambda| I)^{-1/2} W^T$
        \STATE $B = YR$ \hfill \COMMENT{$R$ is psd}
       \ELSE
        \STATE $B = YC^{-1}$ \hfill \COMMENT{Triangular solve}
       \ENDIF
       }
       \STATE $[\hat V, \Sigma, \sim] = \text{svd}(B, 0)$ \hfill \COMMENT{Thin SVD}
       \STATE $\hat \Lambda = \text{max}\{0, \Sigma^2 - (\nu + |\lambda| I)\}$ \hfill \COMMENT{Compute eigs, and remove shift with element-wise max}
       \STATE {\bfseries Return:} $\hat V, \hat \Lambda$
    \end{algorithmic}
\end{algorithm} 

The Nystr\"{o}mPCG subroutine (\cref{alg-nyspcg}) is used in NNCG to compute the damped Newton step.
The preconditioner $P$ and its inverse $P^{-1}$ are given by 
\begin{align*}
    P &= \frac{1}{\hat{\lambda}_s + \mu} U (\hat{\Lambda} + \mu I) U^T + (I - UU^T) \\
    P^{-1} &= (\hat{\lambda}_s + \mu) U (\hat{\Lambda} + \mu I)^{-1} U^T + (I - UU^T).
\end{align*}
Within NNCG, the matrix-vector product involving the Hessian (i.e., $A = H_L(w_k)$) is implemented using Hessian-vector products.
For further details, please see \citet{frangella2023randomized}.

\begin{algorithm}[H] 
   \caption{Nystr\"{o}mPCG}
   \label{alg-nyspcg}
    \begin{algorithmic}
       \INPUT{Psd matrix $A$, right-hand side $b$, initial guess $x_0$, approx. eigenvectors $U$, approx. eigenvalues $\hat{\Lambda}$, sketch size $s$, damping parameter $\mu$, CG tolerance $\epsilon$, CG max. iterations $M$}
       \STATE{$r_0 = b - (A + \mu I) x_0$}
       \STATE{$z_0 = P^{-1} r_0$}
       \STATE{$p_0 = z_0$}
       \STATE{$k = 0$} \COMMENT{Iteration counter}
       \WHILE{$\|r_0\|_2 \geq \eps$ and $k < M$}
           \STATE{$v = (A + \mu I) p_0$}
           \STATE{$\alpha = (r_0^T z_0) / (p_0^T v_0)$} \COMMENT{Compute step size}
           \STATE{$x = x_0 + \alpha p_0$} \COMMENT{Update solution}
           \STATE{$r = r_0 - \alpha v$} \COMMENT{Update residual}
           \STATE{$z = P^{-1}r$}
           \STATE{$\beta = (r^T z) / (r_0^T z_0)$}
           \STATE{$x_0 \gets x, r_0 \gets r, p_0 \gets z + \beta p_0, z_0 \gets z, k \gets k + 1$}
       \ENDWHILE
       \STATE {\bfseries Return:} $x$
    \end{algorithmic}
\end{algorithm} 

The Armijo subroutine (\cref{alg-armijo}) is used in NNCG to guarantee that the loss decreases at every iteration.
The function oracle is implemented in PyTorch using a \textit{closure}.
At each iteration, the subroutine checks whether the \textit{sufficient decrease condition} has been met; if not, it shrinks the step size by a factor of $\beta$.
For further details, please see \citet{nocedal2006numerical}.

\begin{algorithm}[H] 
   \caption{Armijo}
   \label{alg-armijo}
    \begin{algorithmic}
       \INPUT{Function oracle $f$, current iterate $x$, current gradient $\nabla f(x)$, search direction $d$, initial step size $t$, backtracking parameters $\alpha, \beta$}
       \WHILE{$f(x + td) > f(x) + \alpha t (\nabla f(x)^T d)$} 
        \STATE{$t \gets \beta t$} \COMMENT{Shrink step size}
       \ENDWHILE
       \STATE {\bfseries Return:} $t$
    \end{algorithmic}
\end{algorithm} 

\subsection{Wall-clock Times for \lbfgs{} and NNCG}

\begin{table}[t]
    \caption{Per-iteration times (in seconds) of \lbfgs{} and NNCG on each PDE.}
    \vskip 0.15in
    \centering
    \begin{tabular}{|c|c|c|c|}
        \hline
        Optimizer & Convection & Reaction & Wave \\
        \hline
        \lbfgs{} & 4.6e-2 & 3.6e-2 & 9.0e-2 \\
        \hline
        NNCG & 2.5e-1 & 7.2e-1 & 2.9e1 \\
        \hline
        Time Ratio & 5.43 & 20 & 322.22 \\
        \hline
    \end{tabular}
    \label{tab:wall_clock_time_comparison}
\end{table}

\cref{tab:wall_clock_time_comparison} summarizes the per-iteration wall-clock times of \lbfgs{} and NNCG on each PDE. The large gap on wave (compared to reaction and convection) is because NNCG has to compute hessian-vector products involving second derivatives, while this is not the case for the two other PDEs. 

% NNCG can be improved in a number of ways to lower its runtime in comparison to \lbfgs{}. We can leverage existing auto-differentiation strategies in the literature and use a faster strategy to compute the hessian-vector product. We can further close the gap between NNCG and \lbfgs{} using several techniques: (i) subsampled Hessians (which would reduce $n_{\mathrm{res}} + n_{\mathrm{bc}}$ to some smaller value, leading to smaller wall-clock time per iteration), (ii) adaptive damping (which would lead to faster convergence with respect to iterations), and (iii) early termination when computing the damped Newton step via PCG. For example, if we subsample Hessians using $10\%$ of the data, then we would expect the wall-clock time per iteration to decrease by a factor of $10$. For PCG, early termination could probably lead to a 2-3 times speedup, since we run $1000$ PCG iterations per damped Newton step, which is rather conservative. In total, we could easily get a 20-30 times speedup per iteration; we leave the implementation of these improvements to future work.

%\section{Ill-conditioned Differential Operators Lead to Difficult Optimization Problems}
In this section, we state and prove the formal version of \cref{thm:informal_ill_cond}.
The overall structure of the proof is based on showing the conditioning of the Gauss-Newton matrix of the population PINN loss is controlled by the conditioning of the differential operator.
We then show the empirical Gauss-Newton matrix is close to its population counterpart by using matrix concentration techniques. 
Finally, as the conditioning of $H_L$ at a minimizer is controlled by the empirical Gauss-Newton matrix, we obtain the desired result. 
% Part of the first portion of the analysis is similar to \citet{de2023operator}, but we perform no linearization. 
% So the quantities 

\label{section:ill-cond-D}
\subsection{Preliminaries}
Similar to \citet{de2023operator}, we consider a general linear PDE with Dirichlet boundary conditions:
\[
\begin{array}{ll}
    & \Dc[u](x) = f(x),\quad x\in \Omega, \\
    & u(x) = g(x), \quad x\in \partial \Omega,
\end{array}
\]
where $u: \R^d \mapsto \R$, $f:\R^d \mapsto \R$ and $\Omega$ is a bounded subset of $\R^d$.
The ``population'' PINN objective for this PDE is
\[
L_\infty(w) = \frac{1}{2}\int_{\Omega}\left(\Dc[u(x;w)]-f(x)\right)^2d\mu(x)+\frac{\lambda}{2} \int_{\partial \Omega}\left(u(x; w)-g(x)\right)^2d\sigma(x).
\]
$\lambda$ can be any positive real number; we set $\lambda = 1$ in our experiments.
Here $\mu$ and $\sigma$ are probability measures on $\Omega$ and $\partial \Omega$ respectively, from which the data is sampled. 
The empirical PINN objective is given by
\[
L(w) = \frac{1}{2\nres}\sum_{i=1}^{\nres}\left(\Dc[u(x^i_r;w)]-f(x_i)\right)^2+\frac{\lambda}{2\nbc}\sum_{j=1}^{\nbc}\left(u(x_b^j;w)-g(x_j)\right)^2.
\]
Moreover, throughout this section we use the notation $\langle f,g\rangle_{L^{2}(\Omega)}$ to denote the standard $L^2$-inner product on $\Omega$:
\[
\langle f,g\rangle_{L^{2}(\Omega)} = \int_{\Omega}fg d\mu(x).
\]


\begin{lemma}
    The Hessian of the $L_\infty(w)$ is given by
    \begin{align*}
    H_{L_\infty}(w) & = \int_{\Omega}\Dc[\nabla_w u(x;w)]\Dc[\nabla_w u(x;w)]^{T}d\mu(x)+\int_{\Omega}\Dc[\nabla^2_w u(x;w)]\left(\Dc[\nabla_w u(x;w)]-f(x)\right)d\mu(x)\\
    & + \lambda\int_{\partial \Omega}\nabla_w u(x; w)\nabla_w u(x; w)^{T}d\sigma(x) + \lambda\int_{\partial \Omega}\nabla^2_w u(x; w)\left(u(x;w)-g(x)\right)d\sigma(x). 
    \end{align*}
    The Hessian of $L(w)$ is given by
    \begin{align}
       H_L(w) & = \frac{1}{n_{\textup{res}}}\sum^{n_{\textup{res}}}_{i=1}\Dc[\nabla_w u(x_r^i; w)]\Dc[\nabla_w u(x_r^i; w)]^{T}+\frac{1}{n_{\textup{res}}}\sum^{n_{\textup{res}}}_{i=1}\Dc[\nabla^2_w u(x^r_i;w)]\left(\Dc[\nabla_w u(x_r^i;w)]-f(x_r^i)\right)\\
       & +\frac{\lambda}{n_{\textup{bc}}}\sum_{j=1}^{n_{\textup{bc}}}\nabla_w u(x_b^j;w)\nabla_w u(x_b^j;w)^{T} + \frac{\lambda}{n_{\textup{bc}}}\sum^{n_{\textup{bc}}}_{j=1}\nabla^2_w u(x_b^j;w)\left(u(x_b^j;w)-g(x_j)\right). \nonumber
    \end{align}
    In particular, for $w_\star\in \Wstar$
    \begin{align*}
        H_L(w_\star) = G_r(w)+ G_b(w).
    \end{align*}
    Here 
    \[
    G_r(w) \coloneqq \frac{1}{n_{\textup{res}}}\sum^{n_{\textup{res}}}_{i=1}\Dc[\nabla_w u(x_i; w_\star)]\Dc[\nabla_w u(x_i; w_\star)]^{T},\quad G_b(w) = \frac{\lambda}{n_{\textup{bc}}}\sum_{j=1}^{n_{\textup{bc}}}\nabla_w u(x_b^j;w_\star)\nabla_w u(x_b^j;w_\star)^{T}.
    \]
\end{lemma}
Define the maps $\F_{\textup{res}}(w) = \begin{bmatrix}
    \Dc[u(x_r^1;w)] \\
    \vdots \\
    \Dc[u(x_r^{\nres};w)]
\end{bmatrix}$,
and $\F_{\textup{bc}}(w) = \begin{bmatrix}
    u(x_b^1;w) \\
    \vdots \\
    u(x_b^{\nbc};w)]
\end{bmatrix}$.
We have the following important lemma, which follows via routine calculation. 
\begin{lemma}
\label{lemma:jac_ntk}
    Let $n = \nres+\nbc$. Define the map $\mathcal F:\R^p\rightarrow \R^{n}$, by stacking $\F_{\textup{res}}(w), \F_{\textup{bc}}(w)$.
    Then, the Jacobian of $\F$ is given by
    \[
    J_\F(w) = \begin{bmatrix}
        J_{\F_{\textup{res}}}(w) \\
        J_{\F_{\textup{bc}}}(w).
    \end{bmatrix}
    \]
    Moreover, the tangent kernel $K_\F(w) = J_\F(w)J_\F(w)^{T}$ is given by
    \[ K_\F(w) = 
    \begin{bmatrix}
        J_{\F_{\textup{res}}}(w)J_{\F_{\textup{res}}}(w)^{T} & J_{\F_{\textup{res}}}(w)J_{\F_{\textup{bc}}}(w)^{T}  \\
        J_{\F_{\textup{bc}}}(w)J_{\F_{\textup{res}}}(w)^{T} & J_{\F_{\textup{bc}}}(w)J_{\F_{\textup{bc}}}(w)^{T} 
    \end{bmatrix} =
    \begin{bmatrix}
        K_{\F_{\textup{res}}}(w) & J_{\F_{\textup{res}}}(w)J_{\F_{\textup{bc}}}(w)^{T}  \\
        J_{\F_{\textup{bc}}}(w)J_{\F_{\textup{res}}}(w)^{T} & K_{\F_{\textup{bc}}}(w) 
    \end{bmatrix}.
    \]
\end{lemma}

\subsection{Relating $G_{\infty}(w)$ to $\mathcal D$}
Isolate the population Gauss-Newton matrix for the residual:
\[
G_{\infty}(w) = \int_{\Omega}\Dc[\nabla_w u(x;w)]\Dc[\nabla_w u(x;w)]^{T}d\mu(x).
\]
Analogous to \citet{de2023operator} we define the functions $\phi_i(x;w) = \partial_{w_i}u(x;w)$ for $i\in\{1\dots,p\}$.
From this and the definition of $G_{\infty}(w)$, it follows that $\left(G_\infty(w)\right)_{ij} = \langle \Dc[\phi_i], \Dc[\phi_j]\rangle_{L^2(\Omega)}$.

Similar to \citet{de2023operator} we can associate each $w\in\R^p$ with a space of functions $\mathcal H(w) = \textup{span}\left(\phi_1(x;w),\dots,\phi_p(x;w)\right)\subset L^2(\Omega).$
We also define two linear maps associated with $\mathcal H(w)$:
\[
T(w)v = \sum_{i=1}^{p}v_i\phi_i(x;w),
\]
\[
T^{*}(w)f = \left(\langle f,\phi_1\rangle_{L^2(\Omega)},\dots,\langle f,\phi_p\rangle_{L^2(\Omega)}\right).
\]
From these definitions, we establish the following lemma. 
\begin{lemma}[Characterizing $G_{\infty}(w)$]
\label{lemma:Pop-GN}
Define $\mathcal A = \Dc^{*}\Dc$. 
Then the matrix $G_{\infty}(w)$ satisfies
    \[
    G_{\infty}(w) = T^{*}(w)\mathcal A T(w). 
    \]
\end{lemma}
\begin{proof}
Let $e_i$ and $e_j$ denote the $ith$ and $jth$ standard basis vectors in $\R^p$. 
Then,
\begin{align*}
(G_{\infty}(w))_{ij} &= \langle \Dc[\phi_i](w), \Dc[\phi_j](w)\rangle_{L^2(\Omega)} = \langle \phi_i(w),\Dc^{*}\Dc[\phi_j(w)] \rangle_{L^2(\Omega)} = \langle Te_i, \Dc^{*}\Dc[Te_j]\rangle_{L^2(\Omega)} \\
&= \langle e_i, (T^{*}\Dc^{*}\Dc T)[e_j]\rangle_{L^2(\Omega)},
\end{align*}
where the second equality follows from the definition of the adjoint. 
Hence, using $\mathcal A = \Dc^{*}\Dc$, we conclude $G_{\infty}(w) = T^{*}(w)\mathcal A T(w)$.
\end{proof}

Define the kernel integral operator $\mathcal K_{\infty}(w):L^2(\Omega)\rightarrow \mathcal H$ by
\begin{equation}
\label{eq:kern_op}
\mathcal K_{\infty}(w)[f](x) = T(w)T^{*}(w)f = \sum_{i=1}^{p}\langle f,\phi_i(x;w)\rangle \phi_i(x;w),
\end{equation}
and the kernel matrix $A(w)$ with entries $A_{ij}(w) = \langle \phi_{i}(x;w),\phi_{j}(x;w)\rangle_{L^2(\Omega)}$. 

Using \cref{lemma:Pop-GN} and applying the same logic as in the proof of Theorem 2.4 in \citet{de2023operator},
we obtain the following theorem. 
% We define the following weighted-inner product on $L^{2}(\Omega)$:
% \[
% \langle f,g\rangle = \langle f, \Kc_{\infty}g\rangle_{L_2(\Omega)}. 
% \]
\begin{theorem}
\label{thm:pop-gn-eigvals}
    Suppose that the matrix $A(w)$ is invertible.
    Then the eigenvalues of $G_{\infty}(w)$ satisfy
    \[
    \lambda_j(G_\infty(w)) = \lambda_j(\mathcal A\circ \Kc_\infty(w)),\quad \text{for all}~j\in[p].
    \]
\end{theorem}
% \begin{proof}
% \begin{align*}
%     \lamMax(G_{\infty}) &= \sup_{\|v\|=1}v^{T}G_{\infty}(w)v = \sup_{f\in \mathcal H, \|T^{*}f\| = 1}(T^{*}f)^{T}G_{\infty}(w)(T^{*}f) \\
%     &= \sup_{f\in \mathcal H,\|T^{*}f\| = 1}(T^{*}f)^{T}(T^{*}\mathcal A T)(T^{*}f) = \sup_{f\in \mathcal H, \|T^{*}f\| = 1}\langle f, TT^{*}(\mathcal ATT^{*}f)\rangle_{L^2(\Omega)}\\
%     & = \sup_{f\in \mathcal H,\|f\|_{\mathcal K_{\infty}} = 1}\langle f, \mathcal A\circ TT^{*}f\rangle_{\mathcal K_{\infty}} = \lamMax(\mathcal A\circ\Kc_{\infty})
% \end{align*}
% \end{proof}

% \begin{align*}
%     \langle f, (\mathcal A\circ TT^{*})g\rangle &= \langle f, TT^{*}(\mathcal A \circ TT^{*})g\rangle_{L^{2}(\Omega)} = \langle \mathcal A\circ TT^{*}f, TT^{*}g\rangle_{L^{2}(\Omega)}\\
%     &= \langle  (\A\circ TT^{*})f,g \rangle
% \end{align*}
% \[
% \kappa(G_{\infty}) = \kappa(\mathcal A\circ\Kc_{\infty})
% \]

\subsection{$G_r(w)$ Concentrates Around $G_{\infty}(w)$}
In order to relate the conditioning of the population objective to the empirical objective, we must relate the population Gauss-Newton residual matrix to its empirical counterpart. 
We accomplish this by showing $G_r(w)$ concentrates around $G_{\infty}(w)$. 
%Here we assume $\mu$ is a probability measure on $\Omega$, from which the training set is sampled.
To this end, we recall the following variant of the intrinsic dimension matrix Bernstein inequality from \citet{tropp2015introduction}.
\begin{theorem}[Intrinsic Dimension Matrix Bernstein]
 \label{thm:int_bern}
    Let $\{X_i\}_{i\in [n]}$ be a sequence of independent mean zero random matrices of the same size. 
    Suppose that the following conditions hold:
    \begin{align*}
        &\|X_i\|  \leq B,~\sum^{n}_{i=1}\mathbb E[X_i X_i^{T}]\preceq V_1,~\sum^{n}_{i=1}\mathbb E[X_i^{T} X_i]\preceq V_2.
    \end{align*}
    Define 
    \[\mathcal V = 
    \begin{bmatrix}
        V_1 & 0 \\
        0   &  V_2
    \end{bmatrix},~ \varsigma^2 = \max\{\|V_1\|,\|V_2\|\},
    \]
    and the \emph{intrinsic dimension} $d_{\textup{int}} = \frac{\textup{trace}(\mathcal V)}{\|\mathcal V\|}$.
    \newline
    Then for all $t\geq \varsigma+\frac{B}{3}$, 
    \[
    \mathbb P\left(\left\|\sum^{n}_{i=1}X_i\right\|\geq t\right) \leq 4d_{\textup{int}}\exp\left(-\frac{3}{8}\min\left\{\frac{t^2}{\varsigma^2},\frac{t}{B}\right\}\right).
    \]   
\end{theorem}

Next, we recall two key concepts from the kernel ridge regression literature and approximation via sampling literature: $\gamma$-\emph{effective dimension} and $\gamma$-\emph{ridge leverage coherence} \cite{bach2013sharp,cohen2017input,rudi2017falkon}. 
\begin{definition}[$\gamma$-Effective dimension and $\gamma$-ridge leverage coherence]
    Let $\gamma>0$. 
    Then the $\gamma$-effective dimension of $G_{\infty}(w)$ is given by
    \[
    d^{\gamma}_{\textup{eff}}(G_{\infty}(w)) = \textup{trace}\left(G_{\infty}(w)(G_{\infty}(w)+\gamma I)^{-1}\right).
    \]
    The $\gamma$-ridge leverage coherence is given by
    \[
    \chi^\gamma(G_{\infty}(w)) = \sup_{x\in \Omega}\frac{\left\|(G_{\infty}(w)+\gamma I)^{-1/2}\Dc[\nabla_w u(x;w)]\right\|^2}{\mathbb E_{x\sim \mu}\left\|(G_{\infty}(w)+\gamma I)^{-1/2}\Dc[\nabla_w u(x;w)]\right\|^2} = \frac{\sup_{x\in \Omega}{\left\|(G_{\infty}(w)+\gamma I)^{-1/2}\Dc[\nabla_w u(x;w)]\right\|^2}}{{d^{\gamma}_{\textup{eff}}(G_{\infty}(w))}}.
    \]
\end{definition}
Observe that $d^{\gamma}_{\textup{eff}}(G_{\infty}(w))$ only depends upon $\gamma$ and $w$, while $\chi^\gamma(G_{\infty}(w)) $ only depends upon $\gamma, w,$ and $\Omega$. 
Moreover, $\chi^\gamma(G_{\infty}(w))<\infty$ as $\Omega$ is bounded. 

We prove the following lemma using the $\gamma$-effective dimension and $\gamma$-ridge leverage coherence in conjunction with \cref{thm:int_bern}.
\begin{lemma}[Finite-sample approximation]
\label{lemma:sampling}
Let $0<\gamma<\lambda_1(G_{\infty}(w))$. 
If $\nres\geq 40\chi^\gamma(G_{\infty}(w))d^{\gamma}_{\textup{eff}}(G_{\infty}(w))\log\left(\frac{8d^{\gamma}_{\textup{eff}}(G_{\infty}(w))}{\delta}\right)$, then with probability at least $1-\delta$
    \[
    \frac{1}{2}\left[G_\infty(w)-\gamma I\right] \preceq G_{r}(w)\preceq \frac{1}{2}\left[3 G_\infty(w)+\gamma I.\right]
    \]
\end{lemma}
\begin{proof}
    Let $x_i = (G_{\infty}(w)+\gamma I)^{-1/2}\Dc[\nabla_w u(x_i;w)]$, and $X_i = \frac{1}{\nres}\left(x_ix_i^{T}-D_\gamma\right)$, where $D_\gamma = G_{\infty}(w)\left(G_{\infty}(w)+\gamma I\right)^{-1}$.
    Clearly, $\mathbb E[X_i] = 0$. 
    Moreover, the $X_i$'s are bounded as
    \begin{align*}
    \|X_i\| & = \max\left\{\frac{\lamMax(X_i)}{\nres},-\frac{\lamMin(X_i)}{\nres}\right\} \leq \max\left\{\frac{\|x_i\|^2}{\nres}, \frac{\lamMax(-X_i)}{\nres}\right\} \leq \max\left\{\frac{\chi^{\gamma}(G_{\infty}(w))d^{\gamma}_{\textup{eff}}(G_{\infty}(w))}{\nres},\frac{1}{\nres}\right\} \\
    & = \frac{\chi^{\gamma}(G_{\infty}(w))d^{\gamma}_{\textup{eff}}(G_{\infty}(w))}{\nres}.
    \end{align*}
    Thus, it remains to verify the variance condition. 
    We have
    \begin{align*}
    \sum^{\nres}_{i=1}\mathbb E[X_iX_i^{T}] & = \nres\mathbb E[X_1^2] = \nres\times \frac{1}{\nres^2}\mathbb E[(x_1x_1^{T}-D_\gamma)^{2}]\preceq \frac{1}{\nres}\mathbb E[\|x_1\|^2 x_1 x_1^{T}] \\ 
    & \preceq \frac{\chi^{\gamma}(G_{\infty}(w))d^{\gamma}_{\textup{eff}}(G_{\infty}(w))}{\nres} D_\gamma. 
    \end{align*}
    Hence, the conditions of \cref{thm:int_bern} hold with $B = \frac{\chi^{\gamma}(G_{\infty}(w))d^{\gamma}_{\textup{eff}}(G_{\infty}(w))}{\nres}$ and $V_1 = V_2 = \frac{\chi^{\gamma}(G_{\infty}(w))d^{\gamma}_{\textup{eff}}(G_{\infty}(w))}{\nres} D_\gamma$.
    Now $1/2 \leq \|\mathcal V\|\leq 1$ as $\nres\geq \chi^{\gamma}(G_{\infty}(w))d^{\gamma}_{\textup{eff}}(G_{\infty}(w))$ and $\gamma\leq \lambda_1\left(G_\infty(w)\right)$.
    Moreover, as $V_1 = V_2$ we have $d_{\textup{int}} \leq 4 d^{\gamma}_{\textup{eff}}(G_{\infty}(w))$. 
    So, setting 
    \[
    t = \sqrt{\frac{8\chi^\gamma(G_{\infty}(w))d^{\gamma}_{\textup{eff}}(G_{\infty}(w))\log\left(\frac{8d^{\gamma}_{\textup{eff}}(G_{\infty}(w))}{\delta}\right)}{3\nres}}+\frac{8\chi^\gamma(G_{\infty}(w))d^{\gamma}_{\textup{eff}}(G_{\infty}(w))\log\left(\frac{8d^{\gamma}_{\textup{eff}}(G_{\infty}(w))}{\delta}\right)}{3\nres}
    \]
    and using $\nres\geq 40\chi^\gamma(G_{\infty}(w)) d^{\gamma}_{\textup{eff}}(G_{\infty}(w))\log\left(\frac{8d^{\gamma}_{\textup{eff}}(G_{\infty}(w))}{\delta}\right)$, we conclude
    \[\mathbb P\left(\left\|\sum_{i=1}^{\nres}X_i\right\|\geq \frac{1}{2}\right)\leq \delta.\]
    Now, $\left\|\sum_{i=1}^{\nres}X_i\right\|\leq \frac{1}{2}$ implies
    \[
    -\frac{1}{2}\left[G_{\infty}(w)+\gamma I\right]\preceq G_r(w)-G_{\infty}(w)\preceq \frac{1}{2}\left[G_{\infty}(w)+\gamma I\right].
    \]
    The claim now follows by rearrangement. 
\end{proof}

By combining \cref{thm:pop-gn-eigvals} and \cref{lemma:sampling}, we show that if the spectrum of $\A\circ \Kc_{\infty}(w)$ decays, then the spectrum of the empirical Gauss-Newton matrix also decays with high probability.  
\begin{proposition}[Spectrum of empirical Gauss-Newton matrix decays fast]
\label{prop:emp_gn_spectrum}
Suppose the eigenvalues of $\A\circ \Kc_{\infty}(w)$ satisfy $\lambda_j(\mathcal A\circ \Kc_{\infty}(w))\leq Cj^{-2\alpha}$, where $\alpha>1/2$ and $C>0$ is some absolute constant.
Then if $\sqrt{\nres}\geq 40C_1\chi^{\gamma}(G_{\infty}(w))\log\left(\frac{1}{\delta}\right)$, for some absolute constant $C_1$, it holds that
\[
  \lambda_{\nres}(G_r(w))\leq \nres^{-\alpha}
\]
with probability at least $1-\delta$.
         
\end{proposition}
\begin{proof} 
    The hypotheses on the decay of the eigenvalues implies $d^{\gamma}_{\textup{eff}}(G_{\infty}(w)) \leq C_1\gamma^{-\frac{1}{2\alpha}}$ (see Appendix C of \citet{bach2013sharp}).
    Consequently, given $\gamma = \nres^{-\alpha}$, we have $d^{\gamma}_{\textup{eff}}(G_{\infty}(w)) \leq C_1\nres^{\frac{1}{2}}$. 
    Combining this with our hypotheses on $\nres$, it follows $\nres\geq 40 C_1\chi^{\gamma}(G_{\infty}(w))d^{\gamma}_{\textup{eff}}(G_{\infty}(w))\log\left(\frac{8d^{\gamma}_{\textup{eff}}(G_{\infty}(w))}{\delta}\right)$.
    Hence \cref{lemma:sampling} implies with probability at least $1-\delta$ that 
    \[
    G_r(w)\preceq \frac{1}{2}\left(3 G_\infty(w)+\gamma I\right),
    \]
    which yields for any $1\leq r\leq n$
    \[
    \lambda_{\nres}(G_r(w))\leq \frac{1}{2}\left(3\lambda_r(G_\infty(w))+\gamma\right).
    \]
    Combining the last display with $\nres\geq 3d^{\gamma}_{\textup{eff}}(G_{\infty}(w))$, 
    Lemma 5.4 of \citet{frangella2023randomized} guarantees $\lambda_r(G_\infty(w))\leq \gamma/3$, and so 
    \[
    \lambda_{\nres}(G_r(w))\leq \frac{1}{2}\left(3\lambda_r(G_\infty(w))+\gamma\right)\leq \gamma \leq \nres^{-\alpha}.
    \]
\end{proof}

\subsection{Formal Statement of \cref{thm:informal_ill_cond} and Proof}
\begin{theorem}[An ill-conditioned differential operator leads to hard optimization]
    Fix $w_\star \in \Wstar$, and let $\mathcal S$ be a set containing $w_\star$ for which $\mathcal S$ is $\mu$-\PL.
    Let $\alpha>1/2$.
    If the eigenvalues of $\A\circ \Kc_{\infty}(w_\star)$ satisfy $\lambda_j(\mathcal A\circ \Kc_{\infty}(w_\star))\leq C j^{-2\alpha}$ and $\sqrt{\nres}\geq 40 C_1\chi^{\gamma}(G_{\infty}(w_\star))\log\left(\frac{1}{\delta}\right)$, then 
    \[
            \kappa_L(\mathcal S) \geq C_2\nres^{\alpha},
    \]
    with probability at least $1-\delta$.
    Here $C, C_1,$ and $C_2$ are absolute constants. 
        
\end{theorem}

\begin{proof}
    By the assumption on $\nres$, the conditions of \cref{prop:emp_gn_spectrum} are met, so, 
    \[
    \lambda_{\nres}(G_r(w_\star))\leq \nres^{-\alpha}.
    \] 
    with probability at least $1-\delta$.
    By definition $G_r(w_\star) = J_{\F_{\textup{res}}}(w_\star)^{T}J_{\F_{\textup{res}}}(w_\star)$, consequently,
    \[
    \lambda_{\nres}(K_{\F_{\textup{res}}}(w_\star)) = \lambda_{\nres}(G_r(w_\star)) \leq \nres^{-\alpha}.
    \] 
    Now, the \PL-constant for $\mathcal S$, satisfies $\mu = \inf_{w \in \mathcal S}\lambda_{n}(K_{\F}(w))$ \cite{liu2022loss}. 
    Combining this with the expression for $K_\F(w_\star)$ in \cref{lemma:jac_ntk}, we reach 
    \[
    \mu\leq \lambda_n(K_\F(w_\star))\leq \lambda_{\nres}(K_{\F_{\textup{res}}}(w_\star))\leq \nres^{-\alpha},
    \]
    where the second inequality follows from Cauchy's Interlacing theorem. 
    Recalling that $\kappa_L(\mathcal S) = \frac{\sup_{w \in \mathcal S}\|H_L(w)\|}{\mu}$, and $H_L(w_\star)$ is symmetric psd, we reach
    \begin{align*}
        \kappa_L(\mathcal S) \geq \frac{\lambda_1(H_L(w_\star))}{\mu}\overset{(1)}{\geq} \frac{\lambda_1(G_r(w_\star))+\lambda_p(G_b(w_\star))}{\mu} \overset{(2)}{=} \frac{\lambda_1(G_r(w_\star))}{\mu} \overset{(3)}{\geq} C_3\lambda_1(G_\infty(w_\star))\nres^{\alpha}. 
    \end{align*}
    Here $(1)$ uses $H_L(w_\star) = G_r(w_\star)+G_b(w_\star)$ and Weyl's inequalities, $(2)$ uses $p\geq \nres+\nbc$, so that $\lambda_p(G_b(w_\star)) = 0$.
    Inequality $(3)$ uses the upper bound on $\mu$ and the lower bound on $G_r(w)$ given in \cref{lemma:sampling}. 
    Hence, the claim follows with $C_2 = C_3\lambda_1(G_\infty(w_\star))$.
    %As $\mu = \inf_{w \in \mathcal S}\lambda_n(K(w))\leq n^{-\beta}$, and $\kappa_L(\mathcal S) = \sup_{w \in \mathcal S}\frac{\lambda_1(H_L(w))}{\mu}$, we have
    %\[
    %\kappa_L(\mathcal S)\geq \frac{\lambda_1(H_L(w_\star))}{\mu} = \frac{\lambda_1(K(w_\star))}{\mu} \geq C_{2}\nres^{\beta}.
    %\]
\end{proof}
\subsection{$\kappa$ Grows with the Number of Residual Points}
\label{subsec:kappa_grows}
\begin{figure*}
    \centering
    \includegraphics[scale=0.45]{figs/condition_number_bound.pdf}
    \caption{Estimated condition number after 41000 iterations of \al{} with different number of residual points from $255 \times 100$ grid on the interior. Here $\lambda_i$ denotes the $i$th largest eigenvalue of the Hessian. The model has $2$ layers and the hidden layer has width $32$. The plot shows $\kappa_L$ grows polynomially in the number of residual points.}
    \label{fig:condition_number_bound}
\end{figure*}
\Cref{fig:condition_number_bound} plots the ratio $\lambda_1(H_L)/\lambda_{129}(H_L)$ near a minimizer $w_\star$. This ratio is a lower bound for the condition number of $H_L$, and is computationally tractable to compute. 
We see that the estimate of the $\kappa$ grows polynomially with $\nres$, which provides empirical verification for \cref{thm:informal_ill_cond}.

% Suppose, $n\geq 4Cd^{\lambda}_{\textup{eff}}(G_{\infty}(w))\log\left(\frac{1}{\delta}\right)$
% which implies
% \[
% \frac{1}{2}\left[G_\infty(w)-\lambda I\right]\preceq G(w)\preceq \frac{1}{2}\left[3 G_\infty(w)+\lambda I\right].
% \]
% Thus, $\lambda_{r_\star}(G(w))\leq 2\lambda_{r_\star}(G_{\infty}(w)) = 2 \lambda_{r_\star}\left(\A\circ K_{\infty}\right)$.
% As $n\geq 4d^{\lambda}_{\textup{eff}}(G_{\infty}(w))$ it follows that
% \[
% \lambda_{n}(G(w))\leq \lambda.
% \]
% With $n \geq C\sqrt{n}\log\left(\frac{1}{\delta}\right)$, for fast poly-decay this becomes 
% \[
% \lambda_{n}(G(w))\leq n^{-\frac{\beta}{2}},
% \]
% while for exponential
% \[
% \lambda_{n}(G(w))\leq \exp(-\sqrt{n}).
% \]
% So GD will converge in $\mathcal O\left(n^{\beta/2}\log(1/\epsilon)\right)$, $\mathcal O\left(\exp{(\sqrt{n})}\log(1/\epsilon)\right)$

\section{Convergence of GDND (\cref{alg-GDND})}
\label{section:GDND_conv}
In this section, we provide the formal version of \cref{thm:GDND_informal} and its proof. 
However, this is delayed till \cref{subsec:GDND_conv}, as the theorem is a consequence of a series of results.
Before jumping to the theorem, we recommend reading the statements in the preceding subsections to understand the statement and corresponding proof. 
\subsection{Overview and Notation}
Recall, we are interested in minimizing the objective in \eqref{eq:pinn_prob_gen}:
\[
L(w) = \frac{1}{2\nres}\sum_{i=1}^{\nres}\left(\Dc[u(x_r^i;w)]\right)^2+\frac{1}{2\nbc}\sum_{j=1}^{\nbc}\left(\Bc[u(x_b^j;w)]\right)^2, 
\]
where $\Dc$ is the differential operator defining the PDE and $\Bc$ is the operator defining the boundary conditions. 
Define
\[
\F(w) = \begin{bmatrix}
    \frac{1}{\sqrt{\nres}}\Dc[u(x^1_r;w)] \\
    \vdots \\
    \frac{1}{\sqrt{\nres}}\Dc[u(x_r^{\nres};w)]\\
    \frac{1}{\sqrt{\nbc}}\Bc[u(x^1_b;w)] \\
    \vdots \\
    \frac{1}{\sqrt{\nbc}}\Bc[u(x_b^{\nbc};w)]
\end{bmatrix},~ y = 0
\]
Using the preceding definitions, our objective may be rewritten as:
\[
L(w) = \frac{1}{2}\|\F(w)-y\|^2.
\]
Throughout the appendix, we work with the condensed expression for the loss given above.
We denote the $(\nres+\nbc)\times p$ Jacobian matrix of $\mathcal F$ by $J_\F(w)$. 
%Note that $D\mathcal F(w) \in \mathbb{R}^{n\times p}$.
The tangent kernel at $w$ is given by the $n\times n$ matrix $K_\F(w) = J_\F(w) J_\F(w)^{T}$.
The closely related Gauss-Newton matrix is given by $G(w) = J_\F(w)^{T} J_\F(w)$.


\subsection{Global Behavior: Reaching a Small Ball About a Minimizer}
We begin by showing that under appropriate conditions, gradient descent outputs a point close to a minimizer after a fixed number of iterations.
We first start with the following assumption which is common in the neural network literature \cite{liu2022loss,liu2023aiming}.
\begin{assumption}
\label{assp:loss_reg}
     The mapping $\mathcal F(w)$ is $\mathcal L_\F$-Lipschitz, and the loss $L(w)$ is $\beta_{L}$-smooth.
\end{assumption}

% \begin{assumption}
%     Let $w_0$ denote the network weight at initialization, then such $L(w)$ is $\mu$-\PL in $B(w_0,2R)$, with $$.
% \end{assumption}

Under \Cref{assp:loss_reg} and a \PL-condition, we have the following theorem of \citet{liu2022loss}, which shows gradient descent converges linearly. 
\begin{theorem}
\label{thm:grad_dsct_conv}
    Let $w_0$ denote the network weights at initialization. 
    Suppose \Cref{assp:loss_reg} holds, and that $L(w)$ is $\mu$-P\L$^{\star}$ in $B(w_0,2R)$ with $R = \frac{2\sqrt{2\beta_{L}L(w_0)}}{\mu}$.
    Then the following statements hold:
    \begin{enumerate}
        \item The intersection $B(w_0,R)\cap\Wstar$ is non-empty.
        \item Gradient descent with step size $\eta = 1/\beta_L$ satisfies:
        \begin{align*}
        &w_{k+1} = w_k-\eta \nabla L(w_k)\in B(w_0,R)~ \text{for all } k\geq 0,\\
        &L(w_k)\leq \left(1-\frac{\mu}{\beta_L}\right)^kL(w_0).
        \end{align*}
    \end{enumerate}
\end{theorem}
For wide neural neural networks, it is known that the $\mu$-\PL condition in \cref{thm:grad_dsct_conv} hold with high probability, see \citet{liu2022loss} for details.

We also recall the following lemma from \citet{liu2023aiming}.
    \begin{lemma}[Descent Principle]
    \label{lemma:descent_principle}
        Let $L:\R^p\mapsto [0,\infty)$ be differentiable and $\mu$-\PL in the ball $B(w,r)$. 
        Suppose $L(w)<\frac{1}{2}\mu r^2$.
        Then the intersection $B(w,r)\cap \Wstar$ is non-empty, and
        \[
        \frac{\mu}{2}\dist^2(w,\Wstar)\leq L(w).
        \]
    \end{lemma}
        Let $\Lc_{H_L}$ be the Hessian Lipschitz constant in $B(w_0,2R)$, and $\Lc_{J_\F} = \sup_{w\in B(w_0,2R)}\|H_\F(w)\|$, where $\|H_\F(w)\| = \max_{i\in [n]}\|H_{\F_i}(w)\|$. 
        Define $M = \max\{\mathcal L_{\HL},\Lc_{J_\F},\mathcal \Lc_\F \Lc_{J_\F},1\}$,  $\epsLoc = \frac{\varepsilon \mu^{3/2}}{4M}$, where $\varepsilon\in (0,1)$.
        By combining \cref{thm:grad_dsct_conv} and \cref{lemma:descent_principle}, we are able to establish the following important corollary, which shows gradient descent outputs a point close to a minimizer.
    \begin{corollary}[Getting close to a minimizer]
        \label{corr:close_to_min}
        Set $\rho =  \min\left\{\frac{\epsLoc}{19\sqrt{\frac{\beta_L}{\mu}}},\sqrt{\mu}R,R\right\}$.
        Run gradient descent for $k = \frac{\beta_L}{\mu}\log\left(\frac{4\max\{2\beta_{L},1\}L(w_0)}{\mu\rho^2}\right)$ iterations, 
        gradient descent outputs a point $\wloc$ satisfying 
        \[
        L(\wloc) \leq \frac{\mu \rho^2}{4}\min\left\{1,\frac{1}{2\beta_L}\right\},
        \]
        \[
        \|\wloc-w_\star\|_{H_{L}(w_\star)+\mu I}\leq \rho,~ \text{for some}~ w_\star\in \Wstar.
        \]
    \end{corollary}
\begin{proof}
    The first claim about $L(\wloc)$ is an immediate consequence of \Cref{thm:grad_dsct_conv}.
    For the second claim, consider the ball $B(\wloc,\rho)$.
    Observe that $B(\wloc,\rho)\subset B(w_0,2R)$, so $L$ is $\mu$-\PL~in $B(\wloc,\rho)$.
    Combining this with $L(\wloc) \leq \frac{\mu \rho^2}{4}$, \Cref{lemma:descent_principle} guarantees the existence of $w_\star\in B(\wloc,\rho)\cap\Wstar$, with 
    $\|\wloc-w_\star\|\leq \sqrt{\frac{2}{\mu}L(\wloc)}$.
    Hence Cauchy-Schwarz yields 
    \begin{align*}
        \|\wloc-w_\star\|_{H_L(w_\star)+\mu I} & \leq \sqrt{\beta_L+\mu}\|\wloc-w_\star\| \leq \sqrt{2\beta_L}\|\wloc-w_\star\|\\
        & \leq 2\sqrt{\frac{\beta_L}{\mu}L(\wloc)} \leq 2 \times \sqrt{\frac{\beta_L}{\mu}\frac{\mu \rho^2}{8{\beta_L}}}\leq \rho,
    \end{align*}
   which proves the claim.
\end{proof}

% \begin{lemma}[Local quadratic growth and error bound]
%     Let $\wloc$ and $\bar w$ be as in blah, and suppose $B(\bar w,2\epsLoc)\subset B(w_0,R)$. 
%     Then the following statements hold:
%     \begin{enumerate}
%         %\item (Existence of local minimizer) The intersection $B(w_\star,\epsLoc)\cap \Wstar$ is non-empty.
%         \item (Local quadratic growth) $L(w)\geq \frac{\mu}{8}\dist^2(w,\Wstar)\quad \forall w\in B(\bar w,\epsLoc)$.
%         \item (Local error bound) $\|\nabla L(w)\|\geq \frac{\mu}{2}\dist(w,\Wstar)\quad \forall w\in B(\bar w,\epsLoc)$.
%     \end{enumerate}
% \end{lemma}


\subsection{Fast Local Convergence of Damped Newton's Method}
% Let $\delta\in (0,1)$.
% Define $g_k = \nabla L(\tilde w_k)$, $H_k = \nabla^2 L(\tilde w_k)$, $p_k = (H_k+\mu I+\|g_k\|^{\delta}I)^{-1}g_k$, and consider
% the iteration 
% \[\tilde w_{k+1} = \tilde w_k+p_k, \quad \tilde w_0 = \wloc.\]

% \begin{lemma}
%     For all $w\in B(w_\star, \epsLoc)$, it holds that
%     \[
%     \lambda_{\textup{min}}(H(w))\geq -\frac{\varepsilon\mu}{2}.
%     \]
%     Consequently, 
%     \[
%     (H(w)+\mu I+\|g(w)\|^{\delta})\succ 0, \quad \forall w\in B(w_\star,\epsLoc).
%     \]
% \end{lemma}

% \begin{lemma}
%     Suppose $\tilde w_k \in B(w_\star,\epsLoc/2)$. Then
%     \[
%     \|d_k\| \leq \lambda ~\dist(\tilde w_k,\Wstar),
%     \]
%     where $\lambda = \left(\max\{1,\varepsilon/(2-\varepsilon)\}+\frac{L_{\HL}}{\mu^{\delta}}\right).$
% \end{lemma}

% \begin{proof}
%     \begin{align*}
%         \|d_k\| & = \left\|(H_k+\mu I+\|g_k\|^{\delta})^{-1}g_k\right\| \\
%         &= \left\|(H_k+\mu I+\|g_k\|^{\delta})^{-1}\left[g_k-g(\bar w_k)+H_k(w_k-\bar w_k)-H_k(w_k-\bar w_k)\right]\right\|\\
%         &\leq \left\|(H_k+\mu I+\|g_k\|^{\delta})^{-1}\left[g_k-g(\bar w_k)+H_k(w_k-\bar w_k)\right]\right\|+\left\|(H_k+\mu I+\|g_k\|^{\delta})^{-1}H_k(w_k-\bar w_k)\right\|.
%     \end{align*}
%     For term $T_1$, observe that
%     \begin{align*}
%         &\left\|(H_k+\mu I+\|g_k\|^{\delta})^{-1}\left[g_k-g(\bar w_k)+H_k(w_k-\bar w_k)\right]\right\| \leq \|g_k\|^{-\delta}\frac{L_{\HL}}{2}\|w_k-\bar w_k\|^2\\
%         &= \|g_k\|^{-\delta}\frac{L_{\HL}}{2}\dist^2(w_k,\Wstar) \leq (2/\mu)^{\delta}\frac{L_{\HL}}{2}\dist^{2-\delta}(w_k,\Wstar) \leq \frac{L_{\HL}}{\mu^{\delta}}\dist(w_k,\Wstar).
%     \end{align*}
%     For term $T_2$, note that
%     \[
%     \left\|(H_k+\mu I+\|g_k\|^{\delta})^{-1}H_k\right\|
%     \]
% \end{proof}

% \begin{lemma}[Staying in the ball]
%     Suppose that $\tilde w_k \in B(\bar w,\epsLoc/(1+\lambda))$, then
%     \[
%     \tilde w_k+\eta p_k \in B(\bar w, \epsLoc),\quad \forall \eta \in [0,1]. 
%     \]
% \end{lemma}
% \begin{proof}
%     \begin{align*}
%         \|\tilde w_k+\eta p_k-\bar w\| &\leq \|\tilde w_k-\bar w\|+\|p_k\|\leq \|\tilde w_k-\bar w\|+\lambda \dist(\tilde w_k,W_\star) = (1+\lambda)\|\tilde w_k-\bar w\|
%         \\ &\leq \epsLoc.
%     \end{align*}
% \end{proof}

% \begin{lemma}[Sufficient descent]
    
% \end{lemma}
In this section, we show damped Newton's method with fixed stepsize exhibits fast linear convergence in an appropriate region about the minimizer $w_\star$ from \cref{corr:close_to_min}. 
Fix $\varepsilon \in (0,1)$, then the region of local convergence is given by:
\[
\Neps = \left\{w\in \R^p: \|w-w_\star\|_{H_L(w_\star)+\mu I}\leq \epsLoc\right\},
\]
where $\epsLoc = \frac{\varepsilon \mu^{3/2}}{4M}$ as above. 
Note that $\wloc \in \Neps$.

We now prove several lemmas, that are essential to the argument. 
We begin with the following elementary technical result, which shall be used repeatedly below.  
\begin{lemma}[Sandwich lemma]
\label{lemma:sandwich}
    Let $A$ be a symmetric matrix and $B$ be a symmetric positive-definite matrix.
    Suppose that $A$ and $B$ satisfy $\|A-B\|\leq \varepsilon \lambda_{\textup{min}}(B)$ where $\varepsilon \in (0,1)$.
    Then
    \[
    (1-\varepsilon)B\preceq A\preceq (1+\varepsilon) B. 
    \]
\end{lemma}
\begin{proof}
    By hypothesis, it holds that
    \[
    -\varepsilon \lambda_{\textup{min}}(B)I\preceq A-B \preceq \varepsilon\lambda_{\textup{min}}(B) I.
    \]
    So using $B\succeq \lambda_{\textup{min}}(B) I$, and adding $B$ to both sides, we reach
    \[
    (1-\varepsilon)B \preceq A\preceq (1+\varepsilon) B.
    \]
\end{proof}


%Define $P = \HL(\wloc)+\lambda I$, the following lemma shows $P$ is positive definite. 
The next result describes the behavior of the damped Hessian in $\Neps$.
\begin{lemma}[Damped Hessian in $\Neps$]
\label{lemma:local_hess}
Suppose that $\gamma \geq \mu$ and $\varepsilon\in (0,1)$. 
\begin{enumerate}
    \item (Positive-definiteness of damped Hessian in $\Neps$) For any $w\in \Neps$, 
    \[
    \HL(w)+\gamma I \succeq \left(1-\frac{\varepsilon}{4}\right)\gamma I.
    \]
    \item (Damped Hessians stay close in $\Neps$)
    For any $w,w' \in \Neps$,
    \[
    (1-\varepsilon)\left[\HL(w)+\gamma I\right] \preceq \HL(w')+\gamma I \preceq (1+\varepsilon) \left[\HL(w)+\gamma I\right].
    \]
\end{enumerate}
% Then it holds that 
% \[
% P\succ \frac{\mu}{2} I.
% \]
\end{lemma}
\begin{proof}
    We begin by observing that the damped Hessian at $w_\star$ satisfies
    \begin{align*}
        \HL(w_\star)+\gamma I & = G(w_\star)+\gamma I+\frac{1}{n}\sum_{i=1}^{n}\left[\F(w_\star)-y\right]_{i}H_{\mathcal F_i}(w_\star)\\
        &= G(w_\star)+\gamma I \succeq \gamma I.
    \end{align*}
    Thus, $\HL(w_\star)+\gamma I$ is positive definite. 
    Now, for any $w\in \Neps$, it follows from Lipschitzness of $\HL$ that
    \begin{align*}
        \left\|\left(\HL(w)+\gamma I\right)-\left(\HL(w_\star)+\gamma I\right)\right\|\leq \Lc_{\HL}\|w-w_\star\|\leq \frac{\Lc_{\HL}}{\sqrt{\gamma}}\|w-w_\star\|_{\HL(w_\star)+\gamma I} \leq \frac{\varepsilon \mu}{4}.
    \end{align*}
    As $\lamMin\left(\HL(w_\star)+\gamma I\right)\geq \gamma>\mu$, we may invoke \Cref{lemma:sandwich} to reach 
    \[
        \left(1-\frac{\varepsilon}{4}\right)\left[\HL(w_\star)+\gamma I\right]\preceq \HL(w)+\gamma I \preceq \left(1+\frac{\varepsilon}{4}\right)\left[\HL(w_\star)+\gamma I\right].
    \]
    This immediately yields 
    \[
    \lamMin\left(\HL(w)+\gamma I\right)\geq \left(1-\frac{\varepsilon}{4}\right)\gamma \geq \frac{3}{4}\gamma,
    \]
    which proves item 1. 
    To see the second claim, observe for any $w,w'\in \Neps$ the triangle inequality implies
    \[
    \left\|\left(\HL(w')+\gamma I\right)-\left(\HL(w)+\gamma I\right)\right\|\leq \frac{\varepsilon \mu}{2} \leq \frac{2}{3}\varepsilon\left(\frac{3}{4}\gamma\right).
    \]
    As $\lamMin\left(\HL(w)+\gamma I\right)\geq \frac{3}{4}\gamma $, it follows from \Cref{lemma:sandwich} that
    \[
    \left(1-\frac{2}{3}\varepsilon\right)\left[\HL(w)+\gamma I\right]\preceq \HL(w')+\gamma I \preceq \left(1+\frac{2}{3}\varepsilon\right)\left[\HL(w)+\gamma I\right],
    \]
    which establishes item 2. 
\end{proof}

The next result characterizes the behavior of the tangent kernel and Gauss-Newton matrix in $\Neps$.
\begin{lemma}[Tangent kernel and Gauss-Newton matrix in $\Neps$]
\label{lemma:local_gn}
    Let $\gamma \geq \mu$. Then for any $w,w'\in \Neps$, the following statements hold:
    \begin{enumerate}
        %\item $(1-\varepsilon)\left[G(\wloc)+\gamma I\right] \preceq G(w)+\gamma I\preceq (1+\varepsilon)\left[G(\wloc)+\gamma I\right] $.
        \item (Tangent kernels stay close) 
        \[
        \left(1-\frac{\varepsilon}{2}\right)K_\F(w_\star)\preceq K_\F(w) \preceq \left(1+\frac{\varepsilon}{2}\right) K_\F(w_\star)
        \]
        \item (Gauss-Newton matrices stay close)
        \[
         \left(1-\frac{\varepsilon}{2}\right)\left[G(w)+\gamma I\right]\preceq G(w_\star)+\gamma I \preceq \left(1+\frac{\varepsilon}{2}\right) \left[G(w)
        +\gamma I\right]\]
        \item (Damped Hessian is close to damped Gauss-Newton matrix) 
        \[
        (1-\varepsilon)\left[G(w)+\gamma I\right] \preceq \HL(w)+\gamma I \preceq (1+\varepsilon)\left[G(w)+\gamma I\right].
        \]
        \item (Jacobian has full row-rank) The Jacobian satisfies $\textup{rank}(J_{\F}(w)) = n$.
    \end{enumerate}
\end{lemma}
\begin{proof}
\begin{enumerate}
    \item Observe that
    \begin{align*}
        \|K_\F(w)-K_\F(w_\star)\| &= \|J_{\F}(w)J_{\F}(w)^{T}-J_{\F}(w_\star)J_{\F}(w_\star)^{T}\| \\
                  &= \left\|\left[J_{\F}(w)-J_{\F}(w_\star)\right]J_{\F}(w)^{T}+J_{\F}(w_\star)\left[J_{\F}(w)-J_{\F}(w_\star)\right]^{T}\right\| \\
                  &\leq 2 \Lc_\F \Lc_{J_\F}\|w-w_\star\| \leq \frac{2 \Lc_\F \Lc_{J_\F}}{\sqrt{\gamma}}\|w-w_\star\|_{H_L(w_\star)+\gamma I} \leq \frac{\varepsilon\mu^{3/2}}{\sqrt{\gamma}} \leq \frac{\varepsilon}{2} \mu,
    \end{align*}
    where in the first inequality we applied the fundamental theorem of calculus to reach 
    \[
    \|J_{\F}(w)-J_{\F}(w_\star)\|\leq \Lc_{J_{\F}}\|w-w_\star\|.
    \]
    Hence the claim follows from \Cref{lemma:sandwich}.
    \item By an analogous argument to item 1, we find
    \[
    \left\|\left(G(w)+\gamma I\right)-\left(G(w_\star)+\gamma I\right)\right\| \leq \frac{\varepsilon}{2}\mu,
    \]
    so the result again follows from \Cref{lemma:sandwich}.
    \item First observe $\HL(w_\star)+\gamma I = G(w_\star)+\gamma I$. Hence the proof of \Cref{lemma:local_hess} implies,
    \[
    \left(1-\frac{\varepsilon}{4}\right)\left[G(w_\star)
        +\gamma I\right]\preceq \HL(w)+\gamma I\preceq \left(1+\frac{\varepsilon}{4}\right)\left[G(w_\star)
        +\gamma I\right].
    \]
    Hence the claim now follows from combining the last display with item 2. 
    \item This last claim follows immediately from item 1, as for any $w\in \Neps$,
    \[
    \sigma_{n}\left(J_{\F}(w)\right) = \sqrt{\lamMin(K_\F(w))}\geq \sqrt{\left(1-\frac{\varepsilon}{2}\right)\mu}>0.
    \]   
    Here the last inequality uses $\lamMin(K_\F(w_\star))\geq \mu$, which follows as $w_\star\in B(w_0,2R)$.
\end{enumerate}

\end{proof}

The next lemma is essential to proving convergence. It shows in $\Neps$ that $L(w)$ is uniformly smooth with respect to the damped Hessian, with nice smoothness constant $(1+\varepsilon)$. 
Moreover, it establishes that the loss is uniformly \PL with respect to the damped Hessian in $\Neps$. 
\begin{lemma}[Preconditioned smoothness and \PL]
\label{lemma:local-sm-pl}
    Suppose $\gamma \geq \mu$. Then 
    for any $w,w',w''\in \Neps$, the following statements hold:
    \begin{enumerate}
        \item $L(w'')\leq L(w')+\langle \nabla L(w'),w''-w'\rangle +\frac{1+\varepsilon}{2}\|w''-w'\|_{H_L(w)+\gamma I}^2$.
        \item $\frac{\|\nabla L(w)\|_{(\HL(w)+\gamma I)^{-1}}^2}{2}\geq \frac{1}{1+\varepsilon}\frac{1}{\left(1+\gamma/\mu\right)}L(w)$.
    \end{enumerate}
\end{lemma}

\begin{proof}
    \begin{enumerate}
        \item By Taylor's theorem
        \begin{align*}
        L(w'') = L(w')+\langle \nabla L(w'),w''-w'\rangle+\int_{0}^{1}(1-t)\|w''-w'\|_{\HL(w'+t(w''-w'))}^2 dt
        \end{align*}
        Note $w'+t(w''-w')\in \Neps $ as $\Neps$ is convex.
        Thus we have,
        \begin{align*}
            L(w'') & \leq L(w')+\langle \nabla L(w'),w''-w'\rangle+\int_{0}^{1}(1-t)\|w''-w'\|_{\HL(w'+t(w''-w'))+\gamma I}^2dt \\
            & \leq L(w')+\langle \nabla L(w'),w''-w'\rangle+\int_{0}^{1}(1-t)(1+\varepsilon)\|w''-w'\|_{\HL(w)+\gamma I}^2dt \\
            & = L(w')+\langle \nabla L(w'),w''-w'\rangle+\frac{(1+\varepsilon)}{2}\|w''-w'\|_{\HL(w)+\gamma I}^2.  
        \end{align*}
    
        \item Observe that
        \begin{align*}
            \frac{\|\nabla L(w)\|_{(\HL(w)+\gamma I)^{-1}}^2}{2} = \frac{1}{2}(\F(w)-y)^{T}\left[J_{\F}(w)\left(\HL(w)+\gamma I\right)^{-1}J_{\F}(w)^{T}\right](\F(w)-y).
        \end{align*}
        Now,
        \begin{align*}
            %J_{\F}(w)P^{-1}J_{\F}(w)^{T} 
            J_{\F}(w)\left(\HL(w)+\gamma I\right)^{-1}J_{\F}(w)^{T} & \succeq \frac{1}{(1+\varepsilon)}J_{\F}(w)\left(G(w)+\gamma I\right)^{-1}J_{\F}(w)^{T}\\ 
             &= \frac{1}{(1+\varepsilon)}J_{\F}(w)\left(J_{\F}(w)^{T}J_{\F}(w)+\gamma I\right)^{-1}J_{\F}(w)^{T}\\ 
        \end{align*}
        \Cref{lemma:local_gn} guarantees $J_{\F}(w)$ has full row-rank, so the SVD yields
        \[
        J_{\F}(w)\left(J_{\F}(w)^{T}J_{\F}(w)+\gamma I\right)^{-1}J_{\F}(w)^{T} = U\Sigma^2(\Sigma^2+\gamma I)^{-1}U^{T}\succeq \frac{\mu}{\mu+\gamma} I.
        \]
        Hence
        \[
          \frac{\|\nabla L(w)\|_{(\HL(w)+\gamma I)^{-1}}^2}{2}\geq \frac{\mu}{(1+\varepsilon)(\mu+\gamma)}\frac{1}{2}\|\F(w)-y\|^2 = \frac{\mu}{(1+\varepsilon)(\mu+\gamma)}L(w).
        \]
    \end{enumerate}
\end{proof}

% \begin{lemma}
%     Let $w_k \in \Neps$. Then 
%     \[
%     \|p_k\|_{P}\leq
%     \]
% \end{lemma}
% \begin{proof}
%     \begin{align*}
%         \|p_k\|_{P} & = \|\nabla L(w_k)\|_{P^{-1}}\leq \|\nabla L(w_k)-\nabla L(\bar w_k)-\nabla^2L(\bar w_k)(w_k-\bar w_k)\|_{P^{-1}}+\|\nabla^2L(\bar w_k)(w_k-\bar w_k)\|_{P^{-1}} \\
%         &= \|\nabla L(w_k)-\nabla L(\bar w_k)-\nabla^2L(\bar w_k)(w_k-\bar w_k)\|_{P^{-1}}+\|\nabla^2L(\bar w_k)^{1/2}(w_k-\bar w_k)\|_{\nabla^2L(\bar w_k)^{1/2}P^{-1}\nabla^2L(\bar w_k)^{1/2}} \\
%         &\leq \frac{1}{\sqrt{1-\varepsilon}}\|\nabla L(w_k)-\nabla L(\bar w_k)-\nabla^2L(\bar w_k)(w_k-\bar w_k)\|_{(\nabla^2 L(\bar w_k)+\rho I)^{-1}}\\
%         &+\|\nabla^2L(\bar w_k)^{1/2}(w_k-\bar w_k)\|_{\nabla^2L(\bar w_k)^{1/2}P^{-1}\nabla^2L(\bar w_k)^{1/2}}. 
%     \end{align*}
%     Now,
%     \begin{align*}
%         & \|\nabla L(w_k)-\nabla L(\bar w_k)-\nabla^2L(\bar w_k)(w_k-\bar w_k)\|_{(\nabla^2 L(\bar w_k)+\rho I)^{-1}} = \\
%         & \left\|\int_{0}^{1}\left[\nabla^2 L(\bar w_k+t(w_k-\bar w_k))-\nabla^2L(\bar w_k)\right](w_k-\bar w_k)\right\|_{(\nabla^2 L(\bar w_k)+\rho I)^{-1}}
%     \end{align*}
% \end{proof}
\begin{lemma}[Local preconditioned-descent]
\label{lemma:local_descent}
    Run Phase II of \cref{alg-GDND} with $\eta_{\textup{DN}} = (1+\varepsilon)^{-1}$ and $\gamma = \mu$. 
    Suppose that $\tilde w_{k}, \tilde w_{k+1}\in \Neps$, then
    \[
     L(\tilde w_{k+1})\leq \left(1-\frac{1}{2(1+\varepsilon)^2}\right)L(\tilde w_k).
    \]
\end{lemma}
\begin{proof}
    As $\tilde w_k, \tilde w_{k+1}\in \Neps$, item 1 of \Cref{lemma:local-sm-pl} yields
    \[
    L(\tilde w_{k+1})\leq L(\tilde w_k)-\frac{\|\nabla L(\tilde w_k)\|^2_{(\HL(\tilde w_k)+\mu I)^{-1}}}{2(1+\varepsilon)}.
    \]
    Combining the last display with the preconditioned \PL condition, 
    we conclude
    \[
    L(\tilde w_{k+1})\leq \left(1-\frac{1}{2(1+\varepsilon)^2}\right)L(\tilde w_k).
    \]
\end{proof}

The following lemma describes how far an iterate moves after one-step of Phase II of \cref{alg-GDND}.
\begin{lemma}[1-step evolution]
\label{lemma:one_step_evol}
    Run Phase II of \cref{alg-GDND} with $\eta_{\textup{DN}} = (1+\varepsilon)^{-1}$ and $\gamma \geq \mu$.
    Suppose $\tilde w_k \in \N_{\frac{\varepsilon}{3}}(w_\star)$, then $\tilde w_{k+1}\in \Neps$.
\end{lemma}

\begin{proof}
    Let $P = H_L(\tilde w_k)+\gamma I$.
    We begin by observing that 
    \begin{align*}
        \|\tilde w_{k+1}-w_\star\|_{\HL(w_\star)+\mu I}\leq \sqrt{1+\varepsilon}\|\tilde w_{k+1}-w_\star\|_{P}.
    \end{align*}
    Now,
    \begin{align*}
        \|\tilde w_{k+1}-w_\star\|_P & = \frac{1}{1+\varepsilon}\|\nabla L(\tilde w_{k})-\nabla L(w_\star)-(1+\varepsilon)P(w_\star-\tilde w_k)\|_{P^{-1}} \\ &=
        \frac{1}{1+\varepsilon}\left\|\int_{0}^{1}\left[\nabla^2 L(w_\star+t(w_k-w_\star))-(1+\varepsilon)P\right]dt(w_\star-\tilde w_k)\right\|_{P^{-1}} \\
        & =   \frac{1}{1+\varepsilon}\left\|\int_{0}^{1}\left[P^{-1/2}\nabla^2 L(w_\star+t(w_k-w_\star))P^{-1/2}-(1+\varepsilon)I\right]dtP^{1/2}(w_\star-\tilde w_k)\right\|\\
        &\leq \frac{1}{1+\varepsilon}\int_{0}^{1}\left\|P^{-1/2}\nabla^2 L(w_\star+t(w_k-w_\star))P^{-1/2}-(1+\varepsilon)I\right\|dt\|\tilde w_k-w_\star\|_{P}
    \end{align*}
    We now analyze the matrix $P^{-1/2}\nabla^2 L(w_\star+t(w_k-w_\star))P^{-1/2}$. 
    Observe that
    \begin{align*}
        & P^{-1/2}\nabla^2 L(w_\star+t(w_k-w_\star))P^{-1/2} = P^{-1/2}(\nabla^2 L(w_\star+t(w_k-w_\star))+\gamma I-\gamma I)P^{-1/2} \\
        & = P^{-1/2}(\nabla^2 L(w_\star+t(w_k-w_\star))+\gamma I)P^{-1/2}-\gamma P^{-1} \succeq (1-\varepsilon)I-\gamma P^{-1} \succeq -\varepsilon I.
        %&= I-\rho P^{-1}+P^{-1/2}EP^{-1/2}\succeq -\|E\|P^{-1} \succeq -(\varepsilon \mu)P^{-1}\succeq -\varepsilon I.
    \end{align*}
    Moreover,
    \[
    P^{-1/2}\nabla^2 L(w_\star+t(w_k-w_\star))P^{-1/2}\preceq P^{-1/2}(\nabla^2 L(w_\star+t(w_k-w_\star))+\gamma I)P^{-1/2}\preceq (1+\varepsilon)I.
    \]
    Hence, 
    \[0 \preceq (1+\varepsilon)I-P^{-1/2}\nabla^2 L(w_\star+t(w_k-w_\star))P^{-1/2}\preceq (1+2\varepsilon)I,\] 
    and so
    \[
    \|\tilde w^{k+1}-w_\star\|_P\leq \frac{1+2\varepsilon}{1+\varepsilon}\|\tilde w_k-w_\star\|_{P}.
    \]
    Thus,
    \[
    \|\tilde w^{k+1}-w_\star\|_{\HL(w_\star)+\mu I}\leq \frac{1+2\varepsilon}{\sqrt{1+\varepsilon}}\|\tilde w_k-w_\star\|_P \leq(1+2\varepsilon)\|\tilde w_k-w_\star\|_{\HL(w_\star)+\mu I}\leq \epsLoc.
    \]
\end{proof}

The following lemma is key to establishing fast local convergence; it shows that the iterates produced by damped Newton's method remain in $\Neps$, the region of local convergence. 
\begin{lemma}[Staying in $\Neps$]
\label{lemma:trapped}
    Suppose that $\wloc \in \mathcal N_{\rho}(w_\star)$, where $\rho = \frac{\epsLoc}{19\sqrt{\beta_L/\mu}}$.
    Run Phase II of \cref{alg-GDND} with $\gamma = \mu$ and $\eta = (1+\varepsilon)^{-1}$, then $\tilde w_{k+1} \in \Neps$ for all $k\geq 1$.
\end{lemma}
\begin{proof}
  In the argument that follows $\kappa_P = 2(1+\varepsilon)^2$.
  The proof is via induction. 
  Observe that if $\wloc \in \mathcal N_{\varrho}(w_\star)$ then by \Cref{lemma:one_step_evol}, $\tilde w_1 \in \Neps$.  
  %\[
  %\|\tilde w_1-w_\star\|_{\HL(w_\star)+\mu I}\leq 3\|\wloc-w_\star\|_{\HL(w_\star)+\mu I} \leq \epsLoc,
  %\]
  Now assume $\tilde w_j \in \Neps$ for $j = 2,\dots, k$. 
  We shall show $\tilde w_{k+1}\in \Neps$.
  To this end, observe that
  \begin{align*}
  \|\tilde w_{k+1}-w_\star\|_{\HL(w_\star)+\mu I} & \leq \|\wloc-w_\star\|_{\HL(w_\star)+\mu I}+\frac{1}{1+\varepsilon}\sum_{j=1}^{k}\|\nabla L(w_j)\|_{\left(\HL(w_\star)+\mu I\right)^{-1}} \\
  % &\leq \varrho+\sqrt{\frac{2}{1+\varepsilon}}\sum_{j=1}^{k}\sqrt{L(\tilde w_{j})} \leq \varrho+\sqrt{\frac{2}{1+\varepsilon}}\sum_{j=1}^{k}\left(1-\frac{\mu_P}{L_P}\right)^{t/2}\sqrt{L(\tilde w_0)} \\
  % &\leq \varrho+\|\wloc - w_\star\|_{\HL(w_\star)+\mu I}\sum_{j=1}^{k}\left(1-\frac{\mu_P}{L_P}\right)^{t/2}\leq \left(1+\sum_{k=0}^{\infty}\left(1-\frac{\mu_P}{L_P}\right)^{t/2}\right)\varrho\\
  % & = \left(1+\frac{1}{{1-\sqrt{1-\frac{\mu_P}{L_P}}}}\right)\varrho\leq \epsLoc.
  \end{align*}
  Now,
  \begin{align*}
      \|\nabla L(w_j)\|_{\left(\HL(w_\star)+\mu I\right)^{-1}} &\leq \frac{1}{\sqrt{\mu}}\|\nabla L(w_j)\|_2 \leq \sqrt{\frac{2\beta_L}{\mu}L(w_j) }\\
      &\leq \sqrt{\frac{2\beta_L}{\mu}}\left(1-\frac{1}{\kappa_P}\right)^{j/2}\sqrt{L(\wloc)},
  \end{align*}
  Here the second inequality follows from $\|\nabla L(w)\| \leq \sqrt{2\beta_L L(w)}$, and the last inequality follows from \Cref{lemma:local_descent}, which is applicable as $\tilde w_{0},\dots,\tilde w_k \in \Neps$. 
  Thus,
  \begin{align*}
  \|\tilde w_{k+1}-w_\star\|_{\HL(w_\star)+\mu I} &\leq \rho+\sqrt{\frac{2\beta_L}{\mu}}\sum_{j=1}^{k}\left(1-\frac{1}{\kappa_P}\right)^{j/2}\sqrt{L(\tilde w_0)} \\
  &\leq \rho+\sqrt{\frac{(1+\varepsilon)\beta_L}{2\mu}}\|\wloc - w_\star\|_{\HL(w_\star)+\mu I}\sum_{j=1}^{k}\left(1-\frac{1}{\kappa_P}\right)^{j/2}\\
  &\leq \left(1+\sqrt{\frac{\beta_L}{\mu}}\sum_{j=0}^{\infty}\left(1-\frac{1}{\kappa_P}\right)^{j/2}\right)\rho\\
  & = \left(1+\frac{\sqrt{\beta_L/\mu}}{{1-\sqrt{1-\frac{1}{\kappa_P}}}}\right)\rho\leq \epsLoc.
  \end{align*}
  Here, in the second inequality we have used $L(\tilde w_0)\leq 2(1+\varepsilon)\|\wloc - w_\star\|^2_{\HL(w_\star)+\mu I}$, which is an immediate consequence of \cref{lemma:local-sm-pl}.
  Hence, $\tilde w_{k+1}\in \Neps$, and the desired claim follows by induction. 
\end{proof}

\begin{theorem}[Fast-local convergence of Damped Newton]
\label{thm:dn_fast_loc}
    Let $\wloc$ be as in \cref{corr:close_to_min}. 
    Consider the iteration 
    \[
    \tilde w_{k+1} = \tilde w_k-\frac{1}{1+\varepsilon}(\HL(\tilde w_k)+\mu I)^{-1}\nabla L(\tilde w_k),\quad \text{where}~\tilde w_0 = \wloc.\] 
    Then, after $k$ iterations, the loss satisfies
        \[
        L(\tilde w_k) \leq \left(1-\frac{1}{2(1+\varepsilon)^2}\right)^{k}L(\wloc).
        \]
        Thus after $k = \mathcal O\left(\log\left(\frac{1}{\epsilon}\right)\right)$ iterations
        \[
        L(\tilde w_k)\leq \epsilon.
        \]
    \begin{proof}
        \cref{lemma:trapped} ensure that $\tilde w^{k} \in \Neps$ for all $k$.
         Thus, we can apply item $1$ of \Cref{lemma:local-sm-pl} and the definition of $\tilde w^{k+1}$, to reach
         \[
         L(\tilde w_{k+1})\leq L(\tilde w_{k})-\frac{1}{2(1+\varepsilon)}\|\nabla L(\tilde w_k)\|_{P^{-1}}^2. 
         \]
         Now, using item $2$ of \Cref{lemma:local-sm-pl} and recursing yields  
         \[
         L(\tilde w_{k+1})\leq \left(1-\frac{1}{2(1+\varepsilon)^2}\right)L(\tilde w_k)\leq \left(1-\frac{1}{2(1+\varepsilon)^2}\right)^{k+1}L(\wloc).
         \]
         The remaining portion of the theorem now follows via a routine calculation.
    \end{proof}
\end{theorem}

% \subsection{Fast local-convergence of Gauss-Newton with Levenberg-Marquardt regularization}
\subsection{Formal Convergence of \cref{alg-GDND}}
\label{subsec:GDND_conv}
Here, we state and prove the formal convergence result for \cref{alg-GDND}.
\begin{theorem}
\label{thm:GDND}
    Suppose that \cref{assp:interpolation} and \cref{assp:loss_reg} hold, and that the loss is $\mu$-\PL in $B(w_0,2R)$, where $R = \frac{2\sqrt{2\beta_L L(w_0)}}{\mu}$.
    Let $\epsLoc$ and $\rho$ be as in \cref{corr:close_to_min}, and set $\varepsilon = 1/6$ in the definition of $\epsLoc$. 
    Run \cref{alg-GDND} with parameters: $\eta_{\textup{GD}} = 1/\beta_L, K_{\textup{GD}} = \frac{\beta_L}{\mu}\log\left(\frac{4\max\{2\beta_{L},1\}L(w_0)}{\mu\rho^2}\right), \eta_{\textup{DN}} = 5/6, \gamma = \mu$ and $K_{\textup{DN}}\geq 1$.
    Then Phase II of \cref{alg-GDND} satisfies
    \[
    L(\tilde w_{k})\leq \left(\frac{2}{3}\right)^{k}L(w_{K_{\textup{GD}}}).
    \]
    Hence after $K_{\textup{DN}} \geq 3\log\left(\frac{L(w_{K_{\textup{GD}}})}{\epsilon}\right)$ iterations, Phase II of \cref{alg-GDND} outputs a point satisfying
    \[
     L(\tilde w_{K_{\textup{DN}}})\leq \epsilon.
    \]
\end{theorem}

\begin{proof}
    By assumption the conditions of \cref{corr:close_to_min} are met, therefore $w_{K_{\textup{GD}}}$ satisfies $\|w_{K_{\textup{GD}}}-w_\star\|_{H_L(w_\star)+\mu I}\leq \rho$, for some $w_\star \in \Wstar$.
    Hence, we may invoke \cref{thm:dn_fast_loc} to conclude the desired result. 
\end{proof}

% \section{You \emph{can} have an appendix here.}

% You can have as much text here as you want. The main body must be at most $8$ pages long.
% For the final version, one more page can be added.
% If you want, you can use an appendix like this one.  

% The $\mathtt{\backslash onecolumn}$ command above can be kept in place if you prefer a one-column appendix, or can be removed if you prefer a two-column appendix.  Apart from this possible change, the style (font size, spacing, margins, page numbering, etc.) should be kept the same as the main body.
%%%%%%%%%%%%%%%%%%%%%%%%%%%%%%%%%%%%%%%%%%%%%%%%%%%%%%%%%%%%%%%%%%%%%%%%%%%%%%%
%%%%%%%%%%%%%%%%%%%%%%%%%%%%%%%%%%%%%%%%%%%%%%%%%%%%%%%%%%%%%%%%%%%%%%%%%%%%%%%


\end{document}


% This document was modified from the file originally made available by
% Pat Langley and Andrea Danyluk for ICML-2K. This version was created
% by Iain Murray in 2018, and modified by Alexandre Bouchard in
% 2019 and 2021 and by Csaba Szepesvari, Gang Niu and Sivan Sabato in 2022.
% Modified again in 2023 and 2024 by Sivan Sabato and Jonathan Scarlett.
% Previous contributors include Dan Roy, Lise Getoor and Tobias
% Scheffer, which was slightly modified from the 2010 version by
% Thorsten Joachims & Johannes Fuernkranz, slightly modified from the
% 2009 version by Kiri Wagstaff and Sam Roweis's 2008 version, which is
% slightly modified from Prasad Tadepalli's 2007 version which is a
% lightly changed version of the previous year's version by Andrew
% Moore, which was in turn edited from those of Kristian Kersting and
% Codrina Lauth. Alex Smola contributed to the algorithmic style files.
