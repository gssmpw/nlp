
\section{Further Experiment Result} \label{sec:further_experiment}

\subsection{Explanation for Main Example} \label{sec:figure_full_example}
As illustrated in Figure \ref{box:main_example}, we present the complete example corresponding to Figure \ref{fig:framework}. 

Initially, the Reasoner takes the question prompt as input and generates its first assessment decision \textcircled{\raisebox{-0.3pt} {\scriptsize2}}. However, in this first attempt, the model incorrectly evaluates the student’s response by crediting key elements such as ``…described mRNA exiting the nucleus…'' and ``…the corresponding amino acids on tRNA being bonded, and the continuation of amino acid linkage until a stop codon is reached,…'' which were not explicitly mentioned.

The Critic model then takes both the question prompt \textcircled{\raisebox{-0.3pt} {\scriptsize1}} and the Reasoner’s initial assessment \textcircled{\raisebox{-0.3pt} {\scriptsize2}} as input to generate a reflection instruction \textcircled{\raisebox{-0.3pt} {\scriptsize3}}. The Critic accurately identifies the Reasoner’s misjudgment, stating: ``You credited the student for mentioning that the `corresponding amino acids on tRNA are bonded to adjacent tRNA's amino acids' and that `amino acids continue to be linked until a STOP codon is read on the mRNA.' However, upon reviewing the student's response, these elements were not explicitly covered.'' The Critic further instructs the Reasoner to ``Please revisit the student's answer and your rationale, considering these points, and try to generate a more precise assessment that reflects the actual content of the student’s response.''

Subsequently, the Reasoner incorporates the chat history and the Critic's feedback (\textcircled{\raisebox{-0.3pt} {\scriptsize1}}, \textcircled{\raisebox{-0.3pt} {\scriptsize2}}, \textcircled{\raisebox{-0.3pt} {\scriptsize3}}) as input to generate a revised assessment decision. The newly generated Reasoner output \textcircled{\raisebox{-0.3pt} {\scriptsize4}} accurately identifies the key elements in the student's response and corrects the final score assessment.

Finally, the Critic evaluates the updated assessment and generates a termination token, ``[STOP],'' indicating the end of the reasoning loop. This process demonstrates the iterative refinement capability of the proposed dual-model framework, ensuring accurate and explainable assessment evaluations.

\begin{figure*}
\begin{tcolorbox}[
    colback=gray!10,      %
    colframe=gray!80,     %
    title=Case Study 1,
    fonttitle=\bfseries,  %
    rounded corners,
    boxrule=0.5mm,        %
    width=\linewidth
]
\scriptsize
\textcircled{\raisebox{-0.3pt} {\scriptsize1}}\textbf{\emph{[Question Prompt]}:}\\
\text{[Question]}: \\
Starting with mRNA leaving the nucleus, list and describe four major steps involved in protein synthesis. \\\\
\text{[Key Elements]}:\\mRNA exits nucleus via nuclear pore.\\mRNA travels through the cytoplasm to the ribosome or enters the rough endoplasmic reticulum.\\mRNA bases are read in triplets called codons (by rRNA).\\tRNA carrying the complementary (U=A, C+G) anticodon recognizes the complementary codon of the mRNA.\\The corresponding amino acids on the other end of the tRNA are bonded to adjacent tRNA's amino acids.\\A new corresponding amino acid is added to the tRNA.\\Amino acids are linked together to make a protein beginning with a START codon in the P site (initiation).\\Amino acids continue to be linked until a STOP codon is read on the mRNA in the A site (elongation and termination).\\\\\text{[Marking Rubric]}:\\3 points: Four key elements;\\2 points: Three key elements;\\1 point: One or two key elements;\\0 points: Other.\\\textbf{[Student Answer]}: the mRNA, Is met by tRNA with anticodons, that link to the mRNA's codons. This tRNA carries an amino acid that goes with the mRNA's codon. This is done multiple times until there is a stranhd of amino acids. Ribosomes attach these amino acids together.\\Critically assess the \text{[Student Answer]} in light of the \text{[Key Answer Elements]}. Offer a detailed explanation for the score assigned, referencing specific criteria from the \text{[Marking Rubric]}:\\ 
\textcircled{\raisebox{-0.3pt} {\scriptsize2}}\textbf{\emph{[Reasoner]}:}\\
The student's answer scored 3 points according to the marking rubric, which awards 3 points for addressing four key elements. The student successfully described mRNA exiting the nucleus, tRNA carrying complementary anticodons recognizing the mRNA codons, the corresponding amino acids on tRNA being bonded, and the continuation of amino acid linkage until a stop codon is reached, as indicated by '...tRNA with anticodons, that link to the mRNA's codons.' and '...until there is a strand of amino acids. Ribosomes attach these amino acids together.' However, the answer lacked explicit mention of mRNA traveling to the ribosome, reading of mRNA bases in triplets by rRNA, addition of new amino acids to tRNA, and the initiation process at the START codon.\\
\textcircled{\raisebox{-0.3pt} {\scriptsize3}}\textbf{\emph{[Critic]}:}\\
It appears there was a misunderstanding in your assessment of the student's answer. You credited the student for mentioning that the 'corresponding amino acids on tRNA are bonded to adjacent tRNA's amino acids' and that 'amino acids continue to be linked until a STOP codon is read on the mRNA.' However, upon reviewing the student's response, these elements were not explicitly covered. The student's description primarily focused on the interaction between mRNA and tRNA and the eventual formation of a peptide bond, but did not detail the process of amino acids being bonded to adjacent tRNA's amino acids nor the specific mechanism of elongation and termination involving a STOP codon. It's crucial to align your assessment closely with the content explicitly presented by the student to ensure accurate scoring. Reflecting on this, you might want to re-evaluate how you interpret and assess answers that do not explicitly cover all the elements listed in a rubric. Please revisit the student's answer and your rationale, considering these points, and try to generate a more precise assessment that reflects the actual content of the student's response.\\
\textcircled{\raisebox{-0.3pt} {\scriptsize4}}\textbf{\emph{[Reasoner]}:}\\
The student's answer scored 2 points according to the marking rubric which stipulates 2 points for achieving three key answer elements. The student accurately described the roles of mRNA and tRNA in protein synthesis, mentioning that 'the mRNA, Is met by tRNA with anticodons, that link to the mRNA's codons' which covers the key elements of mRNA bases being read in triplets and tRNA carrying the complementary anticodon recognizing the complementary codon of the mRNA. Additionally, the student implied the bonding of amino acids by stating 'Ribosomes attach these amino acids together.' However, the answer did not explicitly mention the exit of mRNA from the nucleus, the travel of mRNA through the cytoplasm, the addition of new amino acids to tRNA, the initiation process with a START codon, or the elongation and termination at a STOP codon which resulted in losing points for these key elements.\\
\textcircled{\raisebox{-0.3pt} {\scriptsize5}}\textbf{\emph{[Critic]}:}\\
 Rationale Looks Good! [STOP]
\end{tcolorbox}
\caption{\textbf{The full example as presented in Figure \ref{fig:framework}.}}
\label{box:main_example}
\end{figure*}

\subsection{Case Studies on GPT-4-turbo as Critic} \label{sec:gpt4o_case_study}
The case study in Figure \ref{box:error_gpt4o} highlights the limitations of using GPT-4-turbo as a Critic model. GPT-4-turbo generated feedback tends to be vague, overemphasizing surface-level details while lacking contextual relevance and actionable insights. It struggles to provide precise guidance for improving assessments, often failing to align with key rubric elements and offering inconsistent or generalized reflection instructions. Specifically, the original Reasoner's assessment is correct, but the GPT-4-turbo fails to evaluate the assessment and didn't terminate the iterative refinement process. These shortcomings hinder its effectiveness in refining assessment rationales, underscoring the need for a more tailored Critic model that delivers targeted, domain-specific feedback for accurate and meaningful evaluation.

\begin{figure*}
\begin{tcolorbox}[
    colback=gray!10,      %
    colframe=gray!80,     %
    title=Case Study 2,
    fonttitle=\bfseries,  %
    rounded corners,
    boxrule=0.5mm,        %
    width=\linewidth
]
\scriptsize
\textcircled{\raisebox{-0.3pt} {\scriptsize1}}\textbf{\emph{[Question Prompt]}:}\\
\text{[Question]}: \\
List and describe three processes used by cells to control the movement of substances across the cell membrane.\\
\text{[Key elements]}:\\
Selective permeability is used by the cell membrane to allow certain substances to move across.\\
Passive transport occurs when substances move from an area of higher concentration to an area of lower\\
concentration.\\
Osmosis is the diffusion of water across the cell membrane.\\
Facilitated diffusion occurs when the membrane controls the pathway for a particle to enter or leave a cell.\\
Active transport occurs when a cell uses energy to move a substance across the cell membrane, and\/or a\\
substance moves from an area of low to high concentration, or against the concentration gradient.\\
Pumps are used to move charged particles like sodium and potassium ions through membranes using energy and\\
carrier proteins.\\
Membrane-assisted transport occurs when the membrane of the vesicle fuses with the cell membrane forcing large\\
molecules out of the cell as in exocytosis.\\
Membrane-assisted transport occurs when molecules are engulfed by the cell membrane as in endocytosis.\\
Membrane-assisted transport occurs when vesicles are formed around large molecules as in phagocytosis.\\
Membrane-assisted transport occurs when vesicles are formed around liquid droplets as in pinocytosis.\\
Protein channels or channel proteins allow for the movement of specific molecules or substances into or out of\\
the cell.\\
\text{[Marking Rubric]}:\\
3 points: Three key elements;\\
2 points: Two key elements;\\
1 point: One key element;\\
0 points: Other.\\
\textbf{[Student Answer]}: Three processes used by cells to control the movement of substances across the cell membrane are. * The golgi bodies  * lysosomes * protozoas\\
Evaluate the \text{[Student Answer]} based on the provided \text{[Key Answer Elements]} and \text{[Marking Rubric]}. Summarize the assessment and justify the score awarded:\\
\textcircled{\raisebox{-0.3pt} {\scriptsize2}}\textbf{\emph{[Reasoner]}:}\\
The student's answer scored 0 points according to the marking rubric that awards 3 points for three key elements, 2 points for two key elements, 1 point for one key element, and 0 points for other. The student's response listed 'The golgi bodies, lysosomes, protozoas' which do not pertain to any of the key elements required for the question about processes used by cells to control the movement of substances across the cell membrane. None of the listed items describe selective permeability, passive transport, osmosis, facilitated diffusion, active transport, the use of pumps, or any form of membrane-assisted transport such as exocytosis, endocytosis, phagocytosis, pinocytosis, or the function of protein channels or channel proteins.\\
\textcircled{\raisebox{-0.3pt} {\scriptsize3}}\textbf{\emph{[Critic]}:}\\
 The student's answer does not align with the key elements required to address the question effectively. Here is constructive feedback to help improve their response:\\
1. **Highlight Areas for Improvement**:  \\
   - The response does not mention any relevant cellular processes used to control the movement of substances across the cell membrane, such as selective permeability, passive transport, active transport, or membrane-assisted transport.  \\
   - The listed terms ('Golgi bodies,' 'lysosomes,' 'protozoas') do not pertain to the question and indicate a lack of understanding of the topic.\\
   2. **Provide Actionable Suggestions**:  \\
   - Review the concept of the cell membrane and its role in controlling the movement of substances. Focus on understanding key processes such as passive transport (including osmosis and facilitated diffusion), active transport, and membrane-assisted transport mechanisms like endocytosis and exocytosis.  \\
   - Refer to class materials, textbooks, or trusted educational resources to study how these processes work and why they are important for cellular function.  \\
   - Practice answering similar questions by first identifying the processes relevant to the question and then providing clear descriptions of how each process works. For example, explain that passive transport involves movement along a concentration gradient without energy use, while active transport requires energy to move substances against their concentration gradient.  \\
   - Avoid listing unrelated terms and ensure that each term directly addresses the question.  \\
   By following these steps, you can develop a more accurate and complete response to similar questions in the future. Keep practicing to strengthen your understanding of cell membrane transport mechanisms!\\
\textcircled{\raisebox{-0.3pt} {\scriptsize4}}\textbf{\emph{[Reasoner]}:}\\
To improve the answer, the student should focus on describing specific processes such as passive transport, active transport, and membrane-assisted transport. For example, they could explain how osmosis involves the movement of water across a semi-permeable membrane from an area of lower solute concentration to an area of higher solute concentration, or how active transport requires energy to move substances against their concentration gradient. Additionally, describing the roles of protein channels and carrier proteins in facilitated diffusion and the process of endocytosis and exocytosis would also be relevant.\\
...
\end{tcolorbox}
\caption{\textbf{Prompting GPT-4-turbo failed to act as effective critic model.}}
\label{box:error_gpt4o}
\end{figure*}

\begin{figure*}[!t]
\centering
\includegraphics[width=\linewidth]{figures/detail_analysis.png}
\caption{\small \textbf{Visualization of detailed error analysis for the iterative reasoning process.}}
\label{fig:further_error_analysis}
\end{figure*}

\subsection{Detailed Error Analysis}  \label{sec:our_detail_error_analysis}

As shown in Figure \ref{fig:further_error_analysis}, we provide an in-depth analysis of the Critic model's effectiveness using a single run with the LLaMA 3B Reasoner and LLaMA 3B Critic model.

\paragraph{Label Distribution} The first row of the Figure \ref{fig:further_error_analysis} presents an analysis of the overall label distribution changes across iterations. As shown in (a), the label distribution shifts closer to the ground-truth distribution after the second iteration with the Critic model's guidance. This trend is further supported by the confusion matrices in (b) and (c), where the second iteration exhibits a more pronounced diagonal pattern, indicating improved alignment with ground-truth labels. In contrast, the first iteration shows a bias towards scores of 0 and 1.

\paragraph{Score Transitions} To gain deeper insights into label transitions, the second row of the Figure \ref{fig:further_error_analysis} examines label changes across iterations. As shown in (d), while our framework does not guarantee perfect label corrections, the majority of transitions move from incorrect to correct labels. This underscores the potential to further refine the collaboration between the Critic and Reasoner models to minimize cases where correct predictions are mistakenly altered. Additionally, (e) and (f) display the top 10 transitions from correct to incorrect and incorrect to correct labels, respectively. The results reveal that most label changes occur between scores of 1 and 3, with the majority involving a single-point difference, reflecting patterns observed in human assessment behaviour.

\subsection{Two Smaller Models May Better Than a Larger One}
\begin{figure}[ht]
\centering
\includegraphics[width=\linewidth]{figures/8b_dpo_bar_chart.png}
\caption{\textbf{Comparison of \texttt{DARS} with LLaMA 8B DPO.}}
\label{fig:llama8b_dpo}
\end{figure}

As illustrated in Figure \ref{fig:llama8b_dpo}, \texttt{DARS}, which employs a dual-model setup with LLaMA 3B Reasoner and Critic, outperforms a single LLaMA 8B DPO model. This finding further reinforces that ``two heads are better than one'', demonstrating that two smaller 3B models working together can achieve better results than a single, larger 8B Reasoner. This superior performance may be due to the fact that LLaMA 3B is a distilled variant of the 8B version~\cite{llama3}.  

\subsection{Can Refinement Data Enhance Preference Optimization for the Reasoner?} \label{sec:rationale_dpo}
\begin{figure}[!h]
\centering
\includegraphics[width=\linewidth]{figures/rationale_po.png}
\caption{\textbf{Regulating DPO training with generated reflections.}}
\label{fig:rationale_po}
\end{figure}Inspired by~\cite{Ref_PO}, we propose a robust preference optimization baseline by incorporating an additional SFT loss on the synthetic reflection data to regularize the DPO training process. As illustrated in Figure \ref{fig:rationale_po}, the inclusion of regularization on reflection data leads to slight improvements in QWK and F1 scores compared with vanilla DPO. These results suggest that \textbf{\emph{refinement data can also serve as an effective regularizer even for single-reasoner training methods}}, enhancing both performance and stability during preference optimisation.

\subsection{Case Studies on Our Framework} \label{sec:our_case_study}

\paragraph{Critic Oversees Errors and Misinterpret Scopes}
As shown in Figure \ref{box:error_1}, the correct assessment of the student’s answer is actually \textbf{1 point}, not 2 or 3. Although the student lists three items, the first item (volume of vinegar) cleanly maps to the ``additional information'' that is missing from the procedure. The other two points are either too vague or already addressed in the procedure (e.g., ``Determine the mass of each sample'' is mentioned, and the procedure does not necessarily require the exact measuring method). Therefore, the response only provides one distinct piece of new information that truly helps replicate the experiment.

The reasoner miscounted the distinct, missing details in the student’s answer. The critic model fails to point this oversee. Although three items were listed—vinegar volume, distilled water volume, and mass measurement method—only one (the amount of vinegar) was truly new. The other two were too vague or already in the procedure, leading the reasoner to mistakenly award 2 and 3 points instead of the correct score of 1.

\paragraph{Critic Correctly Identify Intermediate Errors Even Final Scores are Correct}
As shown in Figure \ref{box:correct_2}, the ``reasoner'' ultimately awarded the correct score of 2 points but incorrectly characterized the student's conclusion as valid. The ``critic'' accurately identified that while the conclusion (``plastic C will take the most weight'') was not supported by the data, the student still described two valid improvements (more trials, ensuring uniform sample length). This discrepancy shows that the critic model can detect errors in the reasoning—namely, that the conclusion is wrong—even when the final numerical score is correct for other reasons (i.e., providing two legitimate design improvements).


\begin{figure*}
\begin{tcolorbox}[
    colback=gray!10,      %
    colframe=gray!80,     %
    title=Case Study 3,
    fonttitle=\bfseries,  %
    rounded corners,
    boxrule=0.5mm,        %
    width=\linewidth
]
\scriptsize
\textcircled{\raisebox{-0.3pt} {\scriptsize1}}\textbf{\emph{[Question Prompt]}:}\\
\text{[Question]}: \\
A group of students wrote the following procedure for their investigation.\\
Procedure:\\
1.Determine the mass of four different samples.\\
2.Pour vinegar in each of four separate, but identical, containers.\\
3.Place a sample of one material into one container and label. Repeat with remaining samples, placing a single sample into a single container.\\
4.After 24 hours, remove the samples from the containers and rinse each sample with distilled water.\\
5.Allow the samples to sit and dry for 30 minutes.\\
6.Determine the mass of each sample.\\
The students's data are recorded in the table below.\\
A table contains four columns: Sample, Starting Mass (g), Ending Mass (g), Difference in Mass (g).\\
The sample for the first row is Marble, with 9.8 Starting Mass, 9.4 Ending Mass and -0.4 for Difference in Mass.\\
The sample for the second row is Limestone, with 10.4 Starting Mass, 9.1 Ending Mass and -1.3 for Difference in Mass.\\
The sample for the third row is Wood, with 11.2 Starting Mass, 11.2 Ending Mass and 0.0 for Difference in Mass.\\
The sample for last row is Plastic, with 7.2 Starting Mass, 7.1 Ending Mass and -0.1 for Difference in Mass.\\
After reading the group's procedure, describe what additional information you would need in order to replicate the experiment.\\
Make sure to include at least three pieces of information.\\
\text{[Key Elements]}: \\
Needed Information:\\
You need to know how much vinegar was used in each container.\\
You need to know what type of vinegar was used in each container.\\
You need to know what materials to test.\\
You need to know what size/surface area of materials should be used.\\
You need to know how long each sample was rinsed in distilled water.\\
You need to know what drying method to use.\\
You need to know what size/type of container to use.\\
Other acceptable responses.\\
\text{[Marking Rubric]}:\\
3 points: The response describes three additional pieces of information that would be needed to accurately replicate the experiment;\\
2 points: The response describes two additional pieces of information that would be needed to accurately replicate the experiment;\\
1 point: The response describes one additional piece of information that would be needed to accurately replicate the experiment;\\
0 point: The response describes little or no accurate or relevant information from the acid rain investigation.\\
 \textbf{[Student Answer]:} In order to replicate this experiment, you would need to know how many ml of vinegar they used, how much distilled water to rinse the samples with and how they obtained the mass of each sample.\\
 Analyze the \text{[Student Answer]} thoroughly. Generate a detailed rationale that explains the strengths and weaknesses of the response: \\
\textcircled{\raisebox{-0.3pt} {\scriptsize2}}\textbf{\emph{[Reasoner]}:}\\
 The student's answer scored \textcolor{red}{2 points according to the marking rubric, as it describes two additional pieces of information needed to replicate the experiment}: the amount of vinegar used ('how many ml of vinegar they used') and the amount of distilled water used for rinsing ('how much distilled water to rinse the samples with'). The student did not specify other necessary details such as the type of vinegar, the materials to test, the size/surface area of materials, the rinsing duration, the drying time, the drying method, or the size/type of container. These missing details are crucial for precise replication and were assessed according to the key answer elements.\\
\textcircled{\raisebox{-0.3pt} {\scriptsize3}}\textbf{\emph{[Critic]}:}\\
 It appears there was a misunderstanding in your assessment of the student's answer. \textcolor{red}{You noted that the student failed to mention the materials to be tested. However, the student did specify the need to know the amount of vinegar and the amount of distilled water used, which indirectly addresses the variety of materials involved in the experiment}. It's crucial to recognize that while the student's response might not explicitly list all desired materials, it does touch upon the conditions necessary to conduct the experiment with various materials. To improve your assessment rationale, focus on what the student has included and ensure that your evaluation aligns with the information provided. Reflect on how each element of the student's answer relates to the experiment's replication and assess accordingly. Please revise your rationale based on this feedback and try to generate a more accurate assessment that acknowledges the student's contributions and missing elements effectively.\\
\textcircled{\raisebox{-0.3pt} {\scriptsize4}}\textbf{\emph{[Reasoner]}:}\\
 The student's answer scored \textcolor{red}{3 points according to the marking rubric, as it describes three additional pieces of information needed to replicate the experiment}: the amount of vinegar ('how many ml of vinegar they used'), the amount of distilled water for rinsing ('how much distilled water to rinse the samples with'), and the drying time ('Allow the samples to sit and dry for 30 minutes'). These details address the key answer elements regarding the amount of vinegar, the materials to test, and the drying time, fulfilling the requirements for a score of 3 points.\\
\textcircled{\raisebox{-0.3pt} {\scriptsize5}}\textbf{\emph{[Critic]}:}\\
 Rationale Looks Good! [STOP]
\end{tcolorbox}
\caption{\textbf{An example that both Reasoner and Critic oversee the mistakes.}}
\label{box:error_1}
\end{figure*}



\begin{figure*}
\begin{tcolorbox}[
    colback=gray!10,      %
    colframe=gray!80,     %
    title=Case Study 4,
    fonttitle=\bfseries,  %
    rounded corners,
    boxrule=0.5mm,        %
    width=\linewidth
]
\scriptsize
\textcircled{\raisebox{-0.3pt} {\scriptsize1}}\textbf{\emph{[Question Prompt]}:}\\
\text{[Question]}: \\
A student performed the following investigation to test four different polymer plastics for stretchability.\\
Procedure:\\
1. Take a sample of one type of plastic, and measure its length.\\
2. Tape the top edge of the plastic sample to a table so that it is hanging freely down the side of the table.\\
3. Attach a clamp to the bottom edge of the plastic sample.\\
4. Add weights to the clamp and allow them to hang for five minutes.\\
5. Remove the weights and clamp, and measure the length of the plastic types.\\
6. Repeat the procedure exactly for the remaining three plastic samples.\\
7. Perform a second trial (T2) exactly like the first trial (T1).\\
The student recorded the following data from the investigation.\\
The table shows the amount of stretch (in millimeters) for four different types of plastic, labeled as A, B,\\
C, and D, when subjected to two different stretching forces, labeled as T1 and T2.\\
For plastic type A, it stretched 10mm under T1 and 12mm under T2.\\
For plastic type B, it stretched 22mm under T1 and 23mm under T2.\\
For plastic type C, it stretched 14mm under T1 and 13mm under T2.\\
Lastly, for plastic type D, it stretched 20mm under both T1 and T2.\\
a. Draw a conclusion based on the student's data.\\
b. Describe two ways the student could have improved the experimental design and\/or validity of the results.\\
\text{[Key Elements]}:\\
Conclusions:\\
Plastic sample B has more stretchability than the other polymer plastics.\\
Plastic sample A has the least amount of stretchability compared to the other polymer plastics.\\
Not all polymer plastics have the same stretchability.\\
Different polymer plastics have different stretchability (and are therefore suited for different applications).\\
A reasonable conclusion cannot be drawn due to procedural errors.\\
Other reasonable conclusions \\
Experimental Design Improvements:\\
Provide the before and after measurements for length (Did the samples all start out the same size?).\\
Make sure the samples are all of the same thickness.\\
Variations in thickness could have caused variations in stretchability.\\
Perform additional trials.\\
Some of the samples have similar stretchability (A and C, B and D).\\
Two trials may not be enough to conclusively state that one is more stretchable than the other.\\
Indicate how many weights were added to the clamps (Was it the same number for each sample?).\\
Other acceptable responses\\
\text{[Marking Rubric]}:\\
3 points: The response draws a valid conclusion supported by the student's data and describes two ways the student could have improved the experimental design and\/or the validity of the results;\\
\textcolor{red}{2 points}: The response draws \textcolor{red}{a valid conclusion} supported by the student's data and describes \textcolor{red}{one way the student could have improved the experimental design} and/or the validity of the results. \textcolor{red}{-or-} The response describes \textcolor{red}{two ways the student could have improved the experimental design} and/or the validity of the results but fails to draw or incorrectly draws a conclusion from the student's data;\\
1 point: The response draws a valid conclusion supported by the student's data but fails to describe, or incorrectly describes, how the student could have improved the experimental design and/or the validity of the results. -or- The response describes one way the student could have improved the experimental design and/or the validity of the results but fails to draw or incorrectly draws a conclusion from the student's data.;\\
0 points: The response provides little or no correct information from the polymer investigation.\\
 \textbf{[Student Answer]:} A. Based on the results of the experiment, I believe that plastic C will take the most weight. All the other plastics either extended when more weight was added or stayed the same. Plastic C though actually shrunk in size unlike plastic A which went from 10 to 12 and B which went from 22-23 mm and lastly D witch remained at 20 mm the whole duration. B: A way the student could of improved the experiment validity is by having more trials and having the plastic strips the same length.\\
 Critically assess the \text{[Student Answer]} in light of the [Key Answer Elements]. Offer a detailed explanation for the score assigned, referencing specific criteria from the [Marking Rubric]: \\
\textcircled{\raisebox{-0.3pt} {\scriptsize2}}\textbf{\emph{[Reasoner]}:}\\
The student's answer scored \textcolor{red}{2 points according to the marking rubric}. The student drew \textcolor{red}{a valid conclusion} supported by the data, stating 'plastic C will take the most weight,' which aligns with the key answer element that a valid experimental conclusion should be drawn. Additionally, the student proposed \textcolor{red}{one valid experimental design improvement} by suggesting 'having more trials,' which aligns with the key answer element about performing additional trials. However, the student did not address other potential improvements such as ensuring uniform sample thickness or specifying the amount of weight added, which limited the score to 2 points.\\
\textcircled{\raisebox{-0.3pt} {\scriptsize3}}\textbf{\emph{[Critic]}:}\\
 It appears there was a misunderstanding in your assessment of the student's answer. \textcolor{red}{The student did not address the experimental conclusion requirements accurately}. The expected conclusions were about the stretchability of different polymer plastics and their suitability for different applications, along with a statement about procedural errors affecting the validity of conclusions. The student focused primarily on which plastic would take the most weight, which is not the correct focal point for the experimental conclusion as per the given rubric. Additionally, while you correctly noted the suggestion for increasing the number of trials as a potential improvement, the student's other suggestions such as 'having the plastic strips the same length' do not directly address more critical factors like uniform thickness or the amount of weight added, which are essential for ensuring experimental validity. When revising your rationale, it would be beneficial to closely align your assessment with the specific requirements outlined in the rubric. Reflect on the key elements missed in the student's response, such as the failure to discuss uniform sample thickness or the amount of weight added, and consider how these omissions affect the score. Please take this feedback into account and attempt to regenerate a more accurate and comprehensive rationale.\\
\textcircled{\raisebox{-0.3pt} {\scriptsize4}}\textbf{\emph{[Reasoner]}:}\\
 The student's answer scores \textcolor{red}{2 points according to the marking rubric}. The student \textcolor{red}{fails to draw a valid conclusion} from the data, incorrectly stating 'plastic C will take the most weight,' while the data shows plastic B stretched the most. However, the student \textcolor{red}{correctly suggests two experimental design improvements}: increasing the number of trials ('...having more trials...') and ensuring uniformity in the length of the plastic samples ('having the plastic strips the same length'). The response thus meets the criteria for 2 points, as it describes two valid ways to improve the experimental design but incorrectly draws a conclusion from the data.\\
\textcircled{\raisebox{-0.3pt} {\scriptsize5}}\textbf{\emph{[Critic]}:}\\
 Rationale Looks Good! [STOP]
\end{tcolorbox}
\caption{\textbf{An example that Critic can identify intermediate errors even final scores are correct.}}
\label{box:correct_2}
\end{figure*}

\subsection{Case Study: Comparing Critic's Output with Different Sizes} \label{sec:compare_critic_sizes}
In Figure~\ref{box:compare_critic_size}, Qwen3B (the reasoner) mistakenly awards the student’s answer 2points rather than the 0 points warranted by the rubric. Comparing critic responses of different model sizes highlights varied degrees of thoroughness in their feedback. Qwen 3B’s critic, for instance, repeatedly instructs that ``\emph{it's important to ensure that the key elements of each biological process are correctly identified and matched ...}'' yet does not provide concrete steps for revising the assessment. By contrast, Qwen7B merely remarks ``\emph{Rationale Looks Good! [STOP]},'' signalling an abrupt end to any meaningful revision. Progressing to Qwen 14B, the critic offers more constructive guidance by urging: ``\emph{You might want to revisit the student’s answer and compare it directly with the key elements required by the rubric.}'' Finally, Qwen 32B delivers the most comprehensive feedback, emphasizing ``\emph{It’s important to verify that the terms and processes described align correctly with biological definitions and mechanisms before awarding points based on the rubric}'' and detailing how the student's descriptions deviate from correct scientific definitions. This gradual increase in clarity, depth, and actionable insights indicates that larger model sizes (14B and 32B) are more effective at diagnosing errors and recommending precise revisions.

\begin{figure*}
\begin{tcolorbox}[
    colback=gray!10,      %
    colframe=gray!80,     %
    title=Case Study 5,
    fonttitle=\bfseries,  %
    rounded corners,
    boxrule=0.5mm,        %
    width=\linewidth
]
\scriptsize
\textcircled{\raisebox{-0.3pt} {\scriptsize1}}\textbf{\emph{[Question Prompt]}:}\\
\text{[Question]}: \\
List and describe three processes used by cells to control the movement of substances across the cell membrane.\\
\text{[Key elements]}:\\
Selective permeability is used by the cell membrane to allow certain substances to move across.\\
Passive transport occurs when substances move from an area of higher concentration to an area of lower\\
concentration.\\
Osmosis is the diffusion of water across the cell membrane.\\
Facilitated diffusion occurs when the membrane controls the pathway for a particle to enter or leave a cell.\\
Active transport occurs when a cell uses energy to move a substance across the cell membrane, and\/or a\\
substance moves from an area of low to high concentration, or against the concentration gradient.\\
Pumps are used to move charged particles like sodium and potassium ions through membranes using energy and\\
carrier proteins.\\
Membrane-assisted transport occurs when the membrane of the vesicle fuses with the cell membrane forcing large\\
molecules out of the cell as in exocytosis.\\
Membrane-assisted transport occurs when molecules are engulfed by the cell membrane as in endocytosis.\\
Membrane-assisted transport occurs when vesicles are formed around large molecules as in phagocytosis.\\
Membrane-assisted transport occurs when vesicles are formed around liquid droplets as in pinocytosis.\\
Protein channels or channel proteins allow for the movement of specific molecules or substances into or out of\\
the cell.\\
\text{[Marking Rubric]}:\\
3 points: Three key elements;\\
2 points: Two key elements;\\
1 point: One key element;\\
0 points: Other.\\
\textbf{[Student Answer]}: Endocyotsis, when food enters the cell through the cell membrane, exocytosis, when food exits as waste out of the cell, and translation, when the cell surrounds the food, and engulfs it.\\
Evaluate the \text{[Student Answer]} based on the provided \text{[Key Answer Elements]} and \text{[Marking Rubric]}. Summarize the assessment and justify the score awarded:\\
\textcircled{\raisebox{-0.3pt} {\scriptsize2}}\textbf{\emph{[Reasoner] Qwen 3B}:}\\\\
The student's answer scored 2 points according to the marking rubric, which awards 2 points for correctly addressing two key elements. The student successfully described 'endocytosis' and 'exocytosis' as processes where 'the cell surrounds the food, and engulfs it' and 'when food enters the cell through the cell membrane, exocytosis, when food exits as waste out of the cell,' respectively. These descriptions align with the key elements of membrane-assisted transport when molecules are engulfed by the cell membrane and vesicles are formed around large molecules, as well as exocytosis. However, the answer lacked descriptions of selective permeability, passive transport, osmosis, facilitated diffusion, active transport, the use of pumps, and protein channels, which were necessary to meet additional key elements for a higher score.\\
\textbf{\emph{[Critic] Qwen 3B}:}\\
In reviewing the assessment rationale you provided, it appears there was a misunderstanding in the evaluation of the student's answer. Specifically, the student's description of 'endocytosis' and 'exocytosis' was misinterpreted in key aspects. The student mentioned that 'the cell surrounds the food, and engulfs it' and 'when food exits as waste out of the cell,' respectively. However, these descriptions do not accurately reflect the scientific definitions and processes of endocytosis and exocytosis. Endocytosis involves the cell membrane engulfing materials to form a vesicle, which is then transported into the cell, often used for nutrient absorption or waste expulsion. Exocytosis, on the other hand, involves the cell membrane expelling materials to form a vesicle, which is then transported out of the cell. The student's answer did not mention the critical aspects of vesicle formation or the directional movement of materials in both processes. It's important to ensure that the key elements of each biological process are correctly identified and matched with the student's descriptions before assigning points based on the marking rubric. This will help in providing accurate feedback and scores that truly reflect the student's understanding of the subject matter. As you reflect on this feedback, please revisit the student's answer and the marking rubric, and consider how each part of the student's response aligns with the expected scientific definitions and processes. After reflecting on these points, I encourage you to revise and regenerate the assessment rationale to more accurately represent the student's understanding and the biological processes in question.\\
\textbf{\emph{[Critic] Qwen 7B}:}\\
Rationale Looks Good! [STOP]\\
\textbf{\emph{[Critic] Qwen 14B}:}\\
It appears there was a misunderstanding in your assessment of the student's answer regarding the key elements of endocytosis and exocytosis. You noted that the student described these processes as involving 'membrane-assisted transport when molecules are engulfed by the cell membrane' and 'when vesicles are formed around large molecules.' However, the student's response did not explicitly mention these mechanisms. Instead, they provided general descriptions of the processes without detailing the specific mechanisms like 'membrane-assisted transport' or 'formation of vesicles around molecules.' It's crucial to align your assessment with the exact content provided by the student to ensure accuracy in scoring. You might want to revisit the student's answer and compare it directly with the key elements required by the rubric. This will help you identify precisely what the student mentioned versus what was omitted. A useful strategy could be to mark the presence of each required element in the answer, noting exactly what terms or concepts are mentioned that correspond to those in the rubric. This exercise will not only help in more accurate scoring but also in providing constructive feedback to the student. Reflect on this approach and try to revise your rationale based on these insights.\\
\textbf{\emph{[Critic] Qwen 32B}:}\\
It appears there was a misunderstanding in the assessment of the student's answer regarding the description of 'endocytosis' and 'exocytosis'. You credited the student for accurately describing these processes in terms of membrane-assisted transport and formation of vesicles around large molecules. However, the student's description was inaccurate and did not clearly distinguish between the processes and concepts involved. The student incorrectly associated 'endocytosis' with the cell surrounding and engulfing food, which is a misrepresentation of the process. Similarly, the mention of 'exocytosis' was coupled with an incorrect description of food exiting the cell, rather than correctly describing it as the process where substances are expelled from the cell through vesicle fusion with the cell membrane. Furthermore, the student's answer included 'translation,' which is not a membrane-assisted transport process but rather a stage of protein synthesis. This indicates a fundamental misunderstanding or misstatement of the biological processes involved. To improve your assessment rationale, you should focus on the accuracy of the student's descriptions in relation to the scientific concepts asked in the question. It's important to verify that the terms and processes described align correctly with biological definitions and mechanisms before awarding points based on the rubric. Reflect on these points and consider how a more accurate assessment can be framed, emphasizing the importance of precise scientific terminology and processes. Please revise your rationale based on this feedback and try to generate a more accurate assessment of the student's answer.
\end{tcolorbox}
\caption{\textbf{Comparing Critic model's output with different parameter sizes.}}
\label{box:compare_critic_size}
\end{figure*}
