
\subsection{Level 5 - One-Pole Problem}
\label{app:onepole}
\footnotesize



\subsubsection*{Problem Statement}

Consider the conformally coupled scalar field $\phi$ 
\begin{equation}
\mathcal{L}=\frac{1}{2}\left[g^{\mu\nu}\partial_{\mu}\phi\partial_{\nu}\phi-\left(m^{2}-\frac{1}{6}R\right)\phi^{2}\right]
\end{equation}
 in curved spacetime
\[
ds^{2}=a^{2}(\eta)\left(d\eta^{2}-|d\vec{x}|^{2}\right)
\]
where the Ricci scalar is 
\begin{equation}
R=-6\frac{a''(\eta)}{a(\eta)}
\end{equation}
and $a$ satisfies the differential equation
\begin{equation}
\frac{d}{dt}\ln a=\Theta(t_{e}-t)H_{I}+\Theta(t-t_{e})\frac{H_{I}}{1+\frac{3}{2}H_{I}(t-t_{e})}\label{eq:scalefactor-onepole}
\end{equation}
with $t_{e}$ a finite positive number, the $\Theta$ function having
the steplike behavior
\begin{equation}
\Theta(t-t_{e})\equiv\begin{cases}
1 & t\geq t_{e}\\
0 & \mbox{otherwise}
\end{cases},
\end{equation}
and $t$ being the comoving proper time related to $\eta$ through
\begin{equation}
t=t_{e}+\int_{\eta_{e}}^{\eta}a(y)dy.
\end{equation}
The boundary condition for the differential equation (in comoving
proper time) is $a|_{t=t_{e}}=a_{e}$.

In the limit that $k/(a_{e}H_{I})\rightarrow\infty$, using the steepest
descent approximation starting from the dominant pole $\tilde{\eta}$
(with $\Re\tilde{\eta}>0$) of the integrand factor $\omega_{k}'(\eta)/\left(2\omega_{k}(\eta)\right)$,
compute the Bogoliubov coefficient magnitude $|\beta(k)|$ approximated
as 
\begin{equation}
|\beta(k)|\approx\left|\int_{-\infty}^{\infty}d\eta\frac{\omega_{k}'(\eta)}{2\omega_{k}(\eta)}e^{-2i\int_{\eta_{e}}^{\eta}d\eta'\omega_{k}(\eta')}\right|\label{eq:originalbeta}
\end{equation}
for particle production where the dispersion relationship given by
\begin{equation}
\omega_{k}^{2}(\eta)=k^{2}+m^{2}a^{2}(\eta)
\end{equation}
with $0<m\lesssim H_{I}$. Use a one pole approximation which dominates
in this limit.


\subsubsection*{Answer Requirements}
Provide the answer in the form of the \texttt{python} code. Implement the following function.
\begin{python}
def abs_beta(k:float, a_e:float, m:float, H_I:float) -> float:
    pass
\end{python}

\subsubsection*{Comments about the Problem}

\textit{This is an example of a difficult problem from Quantum Field Theory in curved spacetime, dealing with gravitational particle production, that appears out of reach of current models. This is part of a published research work and the solution, without steps explained, is given in a footnote of \cite{Basso:2022tpd}, but would be difficult to locate (in fact we tried, without success, with OpenAI's Deep Research). 
}
\subsubsection*{Solution}


\begin{figure}
\begin{centering}
\begin{tikzpicture} 
\draw[->] (-5,0) -- (5,0); 
\draw[blue,ultra thick, ->] (-4,-0.1) -- (4,-0.1); 
\draw plot[only marks, mark=x, mark size=4pt](0,-1);
\draw[ultra thick, orange, ->] (-4,-.1) arc (180:225:4);
\draw[ultra thick, orange, ->] (-2.82843, -2.92843) -- (-.2,-1);
\draw[ultra thick, orange] (-.2,-1) arc (180:0:.2);
\draw[ultra thick, orange,->] (+.2,-1) -- (3.87052, -1.98351);
\draw[ultra thick, orange, ->] (3.87052, -1.98351) arc (345:371:4);
\draw[gray, dashed] (-5,-1) -- (5, -1);
\draw[purple] (0,-1) -- (2.82843, 1.82843);
\draw[green] (0,-1) -- (-4.82963, 0.294095);
\draw[draw=red, snake it] (0,-1) -- (1.03528, -4.8637);
\usetikzlibrary {angles,quotes} 
\draw [transparent] (5,-1) coordinate (A) -- (0,-1) coordinate (B)
         -- (2.82843, 1.82843) coordinate (C);
\draw pic ["$\pi/4$",ultra thick, draw,->,purple,angle radius=1cm] {angle = A--B--C}; 
\draw [transparent] (5,-1) coordinate (D) -- (0,-1) coordinate (E)
         -- (-4.82963, 0.294095) coordinate (F);
\draw pic ["$11\pi/12$", draw,->,green,angle radius=1cm] {angle = D--E--F}; 

\end{tikzpicture}
\par\end{centering}
\caption{\label{fig:The-original-contour}\footnotesize The original contour in blue is deformed
into the orange contour in the lower half complex plane of $\eta$.
The large radius arcs have vanishing contributions, and one-pole approximation
has been taken. The upper green and purple boundaries correspond to
where integrations over any arcs extended beyond this boundary would not converge. The
dashed horizontal curve is parallel to the real axis.  The red squiggly line is the branch cut at $-5 \pi/12$.}
\end{figure}

To find the pole of $\omega_{k}'(\eta)/\omega_{k}(\eta)$, we need
$a(\eta)$ from the given differential equation
\begin{equation}
\frac{d\ln a}{dt}=\Theta(t_{e}-t)H_{I}+\Theta(t-t_{e})\frac{H_{I}}{1+\frac{3}{2}H_{I}(t-t_{e})}.\label{eq:Hubble}
\end{equation}
Integrating from time $t=t_{e}$, we find
\begin{align}
\ln\frac{a}{a_{e}} & =\int_{t_{e}}^{t}dT\frac{H_{I}}{1+\frac{3}{2}H_{I}(T-t_{e})}\\
 & =\frac{2}{3}\ln\left[1+\frac{3}{2}H_{I}(T-t_{e})\right]_{t_{e}}^{t}\\
 & =\frac{2}{3}\ln\left[1+\frac{3}{2}H_{I}(t-t_{e})\right]
\end{align}
for $t\geq t_{e}$. In other words, this scale factor
\begin{equation}
\frac{a}{a_{e}}=\left[1+\frac{3}{2}H_{I}(t-t_{e})\right]^{2/3}\label{eq:aHeq}
\end{equation}
behaves as a typical coherent oscillations spacetime minus the oscillatory
effects. Hence, note that for $t\gg t_{e}$, the scale factor can
be approximated as 
\begin{equation}
a(\eta)\approx c_{1}\eta^{2}
\end{equation}
for $\eta\gg\eta_{e}$ (where $\eta_{e}$ is the corresponding conformal
time for $t_{e}$) where we see by matching 
\begin{equation}
\int_{\eta_{i}}^{\eta}a(\eta)d\eta=t-t_{i}
\end{equation}
with $\eta_{i}\gg\eta_{e}$ and $t_{i}\gg t_{e}$, we can write
\begin{equation}
\frac{1}{3}c_{1}\eta^{3}\approx t
\end{equation}
for times much larger than $\eta_{i}$. This means that at time $\eta_{i}\gg\eta_{e}$,
we have
\begin{equation}
c_{1}\approx\frac{2}{H(\eta_{i})\eta_{i}^{3}}
\end{equation}
(where the Hubble expansion rate is $H(\eta)=a'(\eta)/a^{2}(\eta)$)
which gives
\begin{equation}
a(\eta)\approx\frac{2\eta^{2}}{H(\eta_{i})\eta_{i}^{3}}
\end{equation}
for $\eta>\eta_{i}$ where the choice of $\eta_{i}$ controls the
approximation error proportional to positive power of $\eta_{e}/\eta_{i}$.
Since $\eta_{i}\gg\eta_{e}>0$, we can approximate $\eta=0$ to be
equivalent to $\eta-\eta_{i}\rightarrow-\infty$. In other words,
when we analytically continue and consider the poles of the integrand,
we will consider only the region with $\Re\eta>0$.

Next, note the pole of

\begin{equation}
\frac{\omega'}{2\omega}=\frac{m^{2}\partial_{\eta}a^{2}}{4\left(k^{2}+m^{2}a^{2}\right)}
\end{equation}
is at $\tilde{\eta}$ defined by 
\begin{equation}
k^{2}=-m^{2}a^{2}(\tilde{\eta})\label{eq:poleexpl}
\end{equation}
which means
\begin{align*}
\tilde{\eta} & =\sqrt{\frac{H(\eta_{i})\eta_{i}^{3}}{2}}\left(\frac{-k^{2}}{m^{2}}\right)^{1/4}\\
 & =\eta_{i}\sqrt{\frac{1}{a(\eta_{i})}}\left(\frac{-k^{2}}{m^{2}}\right)^{1/4}\\
 & =\eta_{i}e^{i(2l+1)\pi/4}\frac{\sqrt{k/a(\eta_{i})}}{\sqrt{m}}\numberthis \label{eq:branchpoints}
\end{align*}
where $l$ is an integer. We see that $\Re\tilde{\eta}\gg\eta_{i}$
for $k/a(\eta_{e})\gg k/a(\eta_{i})\gg m$. We also see that $l\in\{1,2\}$
have negative $\Re\tilde{\eta}$ which are in the region that we excised
with the $\eta-\eta_{i}\rightarrow-\infty$ discussed above. That
means we can consider either $l\in\{3,4\}$. We will see below that
one of these poles is irrelevant.

Eq.~(\ref{eq:originalbeta}) tells us that

\begin{align*}
\left|\beta(\eta)\right| & =\left|\int_{-\infty}^{\infty}d\eta\frac{\omega_{k}'(\eta)}{2\omega_{k}(\eta)}e^{-2i\int_{\eta_{e}}^{\eta}d\eta'\omega_{k}(\eta')}\right|\\
 & =\left|\int_{-\infty}^{\infty}d\eta\frac{\omega_{k}'(\eta)}{2\omega_{k}(\eta)}e^{-2i\int_{\eta_{i}}^{\eta}d\eta'\omega_{k}(\eta')}e^{-2i\int_{\eta_{e}}^{\eta_{i}}d\eta'\omega_{k}(\eta')}\right|\\
 & =\left|\int_{-\infty}^{\infty}d\eta\frac{\omega_{k}'(\eta)}{2\omega_{k}(\eta)}e^{-2i\int_{\eta_{i}}^{\eta}d\eta'\omega_{k}(\eta')}\right|.\numberthis
\end{align*}
With the steepest descent technique starting from the pole of $\omega_{k}'/\omega_{k}$,
we write after analytically continuing $\eta$
\begin{align*}
\left|\beta\right| & =\left|\int_{-\infty}^{\infty}d\eta\frac{\omega_{k}'(\eta)}{2\omega_{k}(\eta)}e^{-2i\left[\int_{\eta_{i}}^{\tilde{\eta}}d\eta'\omega_{k}(\eta')+\int_{\tilde{\eta}}^{\eta}d\eta'\omega_{k}(\eta')\right]}\right|\\
 & =\left|e^{-2i\int_{\eta_{i}}^{\tilde{\eta}}d\eta'\omega_{k}(\eta')}v\right|\numberthis\label{eq:betaeq}
\end{align*}
where $\tilde{\eta}$ is the pole of $\omega_{k}'(\eta)/\omega_{k}(\eta)$
and  $v$ is the part obtained from the steepest descent. The factor
in the integrand of Eq.~(\ref{eq:originalbeta}) is therefore
\begin{equation}
\frac{\omega'}{2\omega}\approx\frac{1}{4(\eta-\tilde{\eta})}
\end{equation}
which implies $v$ in eq.~(\ref{eq:betaeq}) is
\begin{equation}
v=\int_{-\infty}^{\infty}\frac{d\eta}{4(\eta-\tilde{\eta})}e^{-\frac{4}{3}im\sqrt{C'(\tilde{\eta})}(\eta-\tilde{\eta})^{3/2}}\label{eq:vexplic}
\end{equation}
where 
\begin{equation}
C(\eta)\equiv a^{2}(\eta).
\end{equation}
Deforming the integration contour as shown in Fig.~\ref{fig:The-original-contour}
allows us to rewrite this as
\begin{equation}
v=\int_{\mathcal{C}}\frac{d\eta}{4(\eta-\tilde{\eta})}e^{-\frac{4}{3}im\sqrt{C'(\tilde{\eta})}(\eta-\tilde{\eta})^{3/2}}
\end{equation}
where the $\mathcal{C}$ is the orange part of the contour in the
lower half plane. 

To define the contour, one must understand the complex values of $C'(\tilde{\eta})$.
To this end, let
\begin{equation}
-i\sqrt{C'(\tilde{\eta})}=U+iW
\end{equation}
where the imaginary part generically is nonvanishing. The branch points
are given by eqs.~(\ref{eq:branchpoints}) which gives
\begin{align}
C'(\tilde{\eta}) & =\frac{4a^{2}(\eta_{i})}{\eta_{i}}e^{\frac{3}{4}i(2l+1)\pi}\left(\frac{k/a(\eta_{i})}{m}\right)^{3/2}
\end{align}
which says 
\begin{align*}
U+iW & =\frac{2a(\eta_{i})}{\sqrt{\eta_{i}}}e^{\frac{1}{8}i(6l-1)\pi}\left(\frac{k/a(\eta_{i})}{m}\right)^{3/4}\\
 & =a^{3/2}(\eta_{i})\sqrt{2H(\eta_{i})}e^{\frac{1}{8}i(6l-1)\pi}\left(\frac{k/a(\eta_{i})}{m}\right)^{3/4}.\numberthis
\end{align*}

To deform the contour, we need regions where the arcs with large radius
does not contribute to the integral. Note that if we define $\delta\equiv\eta-\tilde{\eta}=Re^{i\theta}$,
we have
\begin{equation}
\delta^{3/2}=R^{3/2}e^{i3\theta/2}=R^{3/2}\left(\cos\frac{3\theta}{2}+i\sin\frac{3\theta}{2}\right)
\end{equation}
making the exponent in $v$ 
\begin{equation}
-\frac{4}{3}im\sqrt{C'(\tilde{\eta})}(\eta-\tilde{\eta})^{3/2}=\frac{4}{3}mR^{3/2}(U+iW)\left(\cos\frac{3\theta}{2}+i\sin\frac{3\theta}{2}\right)
\end{equation}
which is damped only if 
\begin{equation}
U\cos(3\theta/2)-W\sin(3\theta/2)<0.
\end{equation}
For the case of Eq.~(\ref{eq:poleexpl}), we need 
\begin{equation}
\cos\left[\frac{\pi}{8}(6l-1)\right]\cos(3\theta/2)-\sin\left[\frac{\pi}{8}(6l-1)\right]\sin(3\theta/2)<0
\end{equation}
for one choice of $l$. For the choice of $l=3$, we can choose the
arc regions to be $\theta\in[\frac{-5\pi}{12},\frac{\pi}{4}]$ and
another arc region to be $\theta\in[\frac{11\pi}{12},\frac{19\pi}{12}]$
with a branch cut at $-5\pi/12$. 

Choosing $l=3$, we find the steepest descent contour shown in orange
in Fig.~\ref{fig:The-original-contour}. The left contour is $5\pi/4$
and the right contour is at $-\pi/12$, along which 
\begin{align*}
-\frac{4}{3}im\sqrt{C'(\tilde{\eta})}(\eta-\tilde{\eta})^{3/2} & =-\frac{4}{3}mR^{3/2}a^{3/2}(\eta_{i})\sqrt{2H(\eta_{i})}\left(\frac{k/a(\eta_{i})}{m}\right)^{3/4}
\end{align*}
gives a damped exponential in eq.~(\ref{eq:vexplic}). Hence, the
integral is
\begin{align*}
v & =\frac{1}{4}\int_{\infty}^{\epsilon}\frac{dR}{R}e^{-\frac{4}{3}mR^{3/2}a^{3/2}(\eta_{i})\sqrt{2H_{e}}\left(\frac{k/a(\eta_{i})}{m}\right)^{3/4}}+\frac{1}{4}\int_{\epsilon}^{\infty}\frac{dR}{R}e^{-\frac{4}{3}mR^{3/2}a^{3/2}(\eta_{i})\sqrt{2H_{e}}\left(\frac{k/a(\eta_{i})}{m}\right)^{3/4}}\nonumber\\
 & + \frac{1}{4}\int_{5\pi/4}^{-\pi/12}id\theta\exp\left[-\frac{4}{3}im\sqrt{C'(\tilde{\eta})}(\epsilon\, e^{i\theta})^{3/2}\right]\\
 & =\frac{i}{4}\left[\frac{-\pi}{12}-\frac{15\pi}{12}\right]=\frac{-i\pi}{3}\numberthis
\end{align*}
where in the first line we have introduced a regulator $\epsilon\rightarrow0$.

The final piece in eq.~(\ref{eq:betaeq}) is
\begin{align}
I & =e^{-2i\int_{\eta_{i}}^{\tilde{\eta}}d\eta'\omega_{k}(\eta')}.
\end{align}
Use the expansion 
\begin{align*}
I & =e^{-2i\int_{\eta_{i}}^{\tilde{\eta}}d\eta'\omega_{k}(\eta')}\\
 & =\exp\left(-2i\left[\Phi+J\right]\right)\numberthis
\end{align*}
where $\Phi$ is real and $J$ is purely imaginary. We take the path
to be along the real axis until $\eta=\Re\tilde{\eta}$ and then integrate
in the imaginary $\eta$ direction:
\begin{equation}
J=i\Im\int_{\Re\tilde{\eta}}^{\Re\tilde{\eta}+i\Im\tilde{\eta}}d\eta'\omega_{k}(\eta').
\end{equation}
 This gives
\begin{align}
J & \approx-i\frac{2}{3}\sqrt{2\pi}\frac{\Gamma(5/4)}{\Gamma(3/4)}\frac{(k/a(\eta_{i}))^{3/2}}{H(\eta_{i})\sqrt{m}}.
\end{align}
Now, note from Eq.~(\ref{eq:aHeq}), we can compute
\begin{align*}
\frac{1}{a_{e}^{3/2}} & =\frac{1}{a^{3/2}(\eta_{i})}\left[1+\frac{3}{2}H_{I}(t_{i}-t_{e})\right]\\
 & \approx\frac{1}{a^{3/2}(\eta_{i})}\frac{3}{2}H_{I}t_{i}\\
 & \approx\frac{1}{a^{3/2}(\eta_{i})}\frac{H_{I}}{H(\eta_{i})}\numberthis
\end{align*}
where we used Eq.~(\ref{eq:Hubble}). Eq.~(\ref{eq:betaeq}) then
becomes
\begin{equation}
\boxed{|\beta|\approx\frac{\pi}{3}\exp\left(-\frac{4}{3}\sqrt{2\pi}\frac{\Gamma(5/4)}{\Gamma(3/4)}\frac{(k/a_{e})^{3/2}}{H_{I}\sqrt{m}}\right)}.\label{eq:exampleresult}
\end{equation}



\clearpage




\subsection{Level 5 - Bias of a Sampled Halo Field}
\label{sec:prob_bias}



\subsubsection*{Problem Statement}

In cosmology, large-scale cosmological dark-matter halo fields are biased tracers of the underlying Gaussian matter density $\delta_m$. Assume we have a sample $\delta_m$. We simulate a halo number density field by taking $n(\mathbf{x}) = \bar{n}\max(0,1+b\delta_m(\mathbf{x}))$, where bare number density $\bar{n}$ and bare bias $b$ are specified constants. What is the bias of the sampled halo field? Derive an equation to evaluate the bias which depends on the bare bias and the variance in each pixel.


\subsubsection*{Answer Requirements}
Provide the answer in the form of the \texttt{python} code. Implement the following function.
\begin{python}
#let b_in stand for bare bias
def b_eff(sigma: float, b_in:float) -> float:
    pass
\end{python}


\subsubsection*{Comments about the Problem}

\textit{This is an example of a cosmology research problem that is being solved correctly by advanced reasoning models. This may be because the calculation is similar to existing calculations in the literature. However, this is a genuine research problem, which we solved independently, for an upcoming cosmology publication. The problem requires to retrieve some background knowledge, such as the definition of the matter power spectrum in cosmology.} 


\subsubsection*{Solution}

The solution to this question involves some domain knowledge, parts of which were given in the problem's statement, some approximations sourced by the domain knowledge, and some mathematical calculations. The domain knowledge is very basic and should be known to anyone in the field. Approximations are intuitive and also, mostly, inspired by the domain knowledge. Following Polya, we can organize it as follows: \\

\textbf{Understand the problem.} The number density of halos $n_h(\mathbf{x})$ is defined as
\begin{equation}    N_h = \int_{V} n_h(\mathbf{x})d\mathbf{x}.\end{equation}
The overdensity is defined as \begin{equation}    \delta_h(\mathbf{x}) = \frac{n_h(\mathbf{x})-\langle n_h(\mathbf{x})\rangle}{\langle n_h(\mathbf{x})\rangle}.\end{equation}
Linear bias is defined in terms of Fourier-transformed quantities:\begin{equation}    \delta_h(\mathbf{k}) = b\delta_m(\mathbf{k}).\end{equation} 
This is an approximation that holds on sufficiently large scales (small $k$). $\delta_m(\mathbf{k})$ and $\delta_h(\mathbf{k})$ are Gaussian random fields with zero mean and their variance depends only on the magnitude of the wave-vector $k=|\mathbf{k}|$: 
\begin{equation}    \delta_m \sim \mathcal{N}(0,P_{mm}(k)),\ \delta_h \sim \mathcal{N}(0,P_{hh}(k)).\end{equation}The quantity $P(k)$ is called the power spectrum and is defined as \begin{equation}    \langle\delta(\mathbf{k})\delta(\mathbf{k'})\rangle = (2\pi)^3\delta^D(\mathbf{k+k'})P(k).\end{equation}
It immediately follows that\begin{equation}    P_{hh}(k) = b^2P_{mm}(k).\end{equation}
We are given the expression in real space. In real space, the quantity $\delta_m(\mathbf{x})$ is also a Gaussian random field:\begin{equation}    \delta_m(\mathbf{x}) \sim \mathcal{N}(0, \xi_m),\ \delta_h(\mathbf{x}) \sim \mathcal{N}(0, \xi_h).\end{equation}
Quantity $\xi$ is called a two-point (real-space) correlation function and is defined as\begin{equation}    \langle\delta(\mathbf{x})\delta(\mathbf{x'})\rangle = \xi(|\mathbf{x}-\mathbf{x'}|).\end{equation}This quantity is sufficiently small when $|\mathbf{x}-\mathbf{x'}|\gg1$. We are asked to find what is the expression for $b$ in the equation $\delta_h(k) = b\delta_m(\mathbf{k})$, given the real-space expression for the number density $n_h(\mathbf{x})$ in terms of real-space sample of $\delta_m(\mathbf{x})$.\\

\textbf{Devise a plan.} The key point to solve this problem should be that real-space correlation function for halos $\xi_h$ should also be equal to $b^2\xi_m$. We want to calculate that correlation function. It should be expressed in terms of $\langle n(\mathbf{x})\rangle$ and $\langle n_h(\mathbf{x})n_h(\mathbf{x'})\rangle$. We expect to be able to calculate these expectations since they are the expectations of functions of the Gaussian random variables. We are given the pixel variance $\sigma$. How does it connect to the other quantities we know? In principle, that's also the part of domain knowledge but it also can be deducted from the definitions already given. A discretized version of the correlation function is \begin{equation}    \xi_{ij} = \langle\delta_{\mathbf{x_i}}\delta_{\mathbf{x_j}}\rangle.\end{equation}
When $i=j$, it becomes the pixel variance $\sigma$. \textit{Aside, we could have given instead of $\sigma$, the quantity $P_{mm}(k)$, that is a common description of a cosmological dark-matter field. In that case, from the definitions of $\xi(r)$ and $P_{mm}(k)$, we could have deduced that $\sigma = \frac{1}{V}\sum_{k}P_{mm}(k)$}. Then we pick the ensemble of all the pixels at given fixed large distance $r=|\mathbf{x_i}-\mathbf{x_j}|$. The key is to recognize that it is fully described by a correlated bivariate Gaussian distribution. \begin{equation}    (\delta^m_{i},\delta^m_{j}) \sim \mathcal{N}(0,\Sigma) \end{equation}with a covariance\begin{equation}    \Sigma = \begin{pmatrix}        \sigma^2 & \xi^m_r  \\      \xi^m_r & \sigma^2    \end{pmatrix}.\end{equation}In general, the integrals from the expectation values are cumbersome, but we should expect some simplifications from the fact that $\xi$ is small and we can Taylor-expand the pdf. \\


\textbf{Carry out the plan.} It's more convenient to define $\hat{\delta}_{i} = \delta^m_i/\sigma$ and $\hat{\xi} = \xi^m_r/\sigma^2$, and $\phi_2$ - a correlated bivariate Gaussian pdf - then \begin{equation}    (\hat{\delta}_{i},\hat{\delta}_{j}) \sim \frac{e^{-\frac{1}{2(1-\hat{\xi}^2)}[\hat{\delta}^2_{i}+\hat{\delta}^2_{j}-2\hat{\xi}\hat{\delta}_{i}\hat{\delta}_{j}]}}{2\pi\sqrt{1-\hat{\xi}^2}} \equiv \phi_2(\hat{\delta}_{i},\hat{\delta}_{j}|\hat{\xi}).\end{equation}We note that \begin{equation}    \xi^{n}_r = \frac{\langle n_in_j\rangle}{\langle n \rangle^2}-1.\end{equation}The quantity $\langle n\rangle$ is the actual mean number density:\begin{equation*}    \bar{n}^{'} = \langle n \rangle = \langle n_i \rangle = \int n^{loc}(\delta_i,b,\bar{n}) \phi_2(\hat{\delta}_{i},\hat{\delta}_{j}|\hat{\xi}) d\hat{\delta}_{i}d\hat{\delta}_{j} = \int n^{loc}_i \phi_1(\hat{\delta_i})d\hat{\delta_i}.\end{equation*}Here, $\phi_1$ - is a standard normal pdf. It is expected that it's not dependent on the correlation $\hat{\xi}$, but only on $b$ and $\sigma$, just as the marginal of 2D correlated Gaussian distribution is 1D Gaussian that's not dependent on the cross-correlation. To the linear order in $\hat{\xi}$, \begin{equation}    \phi_2(x,y|\hat{\xi}) \approx \phi_1(x)\phi_1(y)(1+\hat{\xi}xy).\end{equation} So that the two-point function neatly factorizes:\begin{align*}    &\langle n_in_j\rangle = \int n^{loc}(\delta_i,b,\bar{n})n^{loc}(\delta_j,b,\bar{n}) \phi_2(\hat{\delta}_{i},\hat{\delta}_{j}|\hat{\xi}) d\hat{\delta}_{i}d\hat{\delta}_{j} \\ &\approx \int n^{loc}_i \phi_1(\hat{\delta_i})d\hat{\delta_i}\int n^{loc}_j \phi_1(\hat{\delta_j})d\hat{\delta_j} + \hat{\xi}\int n^{loc}_i \phi_1(\hat{\delta_i})\hat{\delta_i}d\hat{\delta_i}\int n^{loc}_j \phi_1(\hat{\delta_j})\hat{\delta_j}d\hat{\delta_j} \\ &\equiv \langle n\rangle^2 + \hat{\xi} \langle n\hat{\delta }\rangle^2.\numberthis\end{align*}
Substituting the results for $\langle n\rangle$ and $\langle n_in_j\rangle$ in  the equation for $\xi^n_r$, we can read off the bias:\begin{equation}b^{'2} = \frac{\xi^n_r}{\sigma^2\hat{\xi}} = \frac{\langle n\hat{\delta}\rangle^2}{\sigma^2\langle n\rangle^2}.\end{equation}
All that is left is to calculate the expectations. One can evaluate for $b\geq0$ 
\begin{align*}    
\langle n\rangle &= \int n^{loc}_i \phi_1(\hat{\delta_i})d\hat{\delta_i} = \int \bar{n}\max(0, 1+b\sigma x)\phi_1(x)dx \\  &= \bar{n}\int_{-\frac{1}{b\sigma}}^{+\infty}(1+b\sigma x)\phi_1(x)dx = \bar{n}\left[\Phi_1\left(\frac{1}{b\sigma}\right)+b\sigma\phi_1\left(\frac{1}{b\sigma}\right)\right].\numberthis \end{align*}
For $b<0$ it's, however, 
\begin{align*}    
\langle n\rangle &= \bar{n}\int_{-\infty}^{+\frac{1}{|b|\sigma}}(1-|b|\sigma x)\phi_1(x)dx\\ &= \bar{n}\left[\Phi_1\left(\frac{1}{|b|\sigma}\right)+|b|\sigma\phi_1\left(\frac{1}{|b|\sigma}\right)\right].\numberthis\end{align*}So we conclude that the latter expression is valid for all $b$. Similarly, one can show that\begin{equation}\langle n\hat{\delta }\rangle = \bar{n}\int\max(0,1+b\sigma x)x\phi_1(x)dx = \bar{n}b\sigma \Phi_1\left(\frac{1}{|b|\sigma}\right)\end{equation}
where $\Phi_1(x) = \int_{-\infty}^x\phi_1(x)dx$ - normal cdf. Finally, one can get
\begin{equation}\boxed{b^{'} = \frac{b \Phi_1\left(\frac{1}{|b|\sigma}\right)}{\Phi_1\left(\frac{1}{|b|\sigma}\right)+|b|\sigma\phi_1\left(\frac{1}{|b|\sigma}\right)}}.\end{equation}

Note: We also accept solutions as correct if they omit the $|~|$ around the bias, since halo bias is usually positive. 









\clearpage

\subsection{Level 4 - SHO Vacuum Entanglement}

\subsubsection*{Problem Statement}

Consider a coupled simple harmonic oscillator governed by the Hamiltonian
\begin{align}
H & =\sum_{i=1}^{2}\frac{1}{2}\left(\frac{p_{i}^{2}}{m}+kx_{i}^{2}\right)+g\frac{(x_{1}-x_{2})^{2}}{2}.
\end{align}
If the ground state is $|\Omega\rangle$ and the operator $\hat{\rho}$
is the vacuum density matrix partially traced over the $|w\rangle_{x_{2}}$
components (satisfying $\hat{x}_{2}|w\rangle_{x_{2}}=w|w\rangle_{x_{2}}$),
i.e.
\begin{equation}
\hat{\rho}\equiv\int dx_{1}''\int dx_{1}'\int dw\left(|x_{1}''\rangle_{x_{1}}\,_{x_{1}}\langle x_{1}''|\otimes\,_{x_{2}}\langle w|\right)\left(|\Omega\rangle\langle\Omega|\right)\left(|x_{1}'\rangle_{x_{1}}\otimes|w\rangle_{x_{2}}\,_{x_{1}}\langle x_{1}'|\right)
\end{equation}
which is an operator acting on a reduced Hilbert space, compute 
\begin{equation}
S\equiv-\mathrm{Tr}_{x_{1}}\left[\hat{\rho}\ln\hat{\rho}\right]
\end{equation}
which involves the trace over $x_{1}$ states.

\subsubsection*{Answer Requirements}
Provide the answer in the form of the \texttt{python} code. Implement the following function 
\begin{python}
def entropy(k:float,g:float,m:float)->float:
    pass
\end{python}


\subsubsection*{Comments about the Problem}

\textit{This problem, whose solution can be found in \cite{Srednicki:1993im} (with less detailed reasoning steps), has been rephrased in a pedagogical manner for graduate-level physics courses. It is a well-known question in quantum entanglement research, and the best performing LLMs are capable of solving it accurately, perhaps at least partially due to memorization.}


\subsubsection*{Solution}
Diagonalize the original Hamiltonian 
\begin{align}
H & =(x_{1}\quad x_{2}\quad p_{1}\quad p_{2})\left(\begin{array}{cccc}
\frac{k+g}{2} & -\frac{g}{2}\\
-\frac{g}{2} & \frac{k+g}{2}\\
 &  & \frac{1}{2m}\\
 &  &  & \frac{1}{2m}
\end{array}\right)\left(\begin{array}{c}
x_{1}\\
x_{2}\\
p_{1}\\
p_{2}
\end{array}\right).
\end{align}
One easily finds
\begin{equation}
x_{1}=\frac{y_{1}+y_{2}}{\sqrt{2}}\label{eq:x1}
\end{equation}
\begin{equation}
x_{2}=\frac{y_{1}-y_{2}}{\sqrt{2}}\label{eq:x2}
\end{equation}
diagonalizes the Hamiltonian such that in the $(y_{1},y_{2},q_{1}\equiv m\dot{y}_{1},q_{2}\equiv m\dot{y}_{2})$
basis, it is
\begin{align}
H & =(y_{1}\quad y_{2}\quad q_{1}\quad q_{2})\left(\begin{array}{cccc}
\frac{k}{2} & 0\\
0 & \frac{k}{2}+g\\
 &  & \frac{1}{2m}\\
 &  &  & \frac{1}{2m}
\end{array}\right)\left(\begin{array}{c}
y_{1}\\
y_{2}\\
q_{1}\\
q_{2}
\end{array}\right).
\end{align}
The ladder operators are
\begin{equation}
a_{j}=\frac{1}{\sqrt{2}}\left(\sqrt{m\omega_{j}}y_{j}+\frac{i}{\sqrt{m\omega_{j}}}q_{j}\right)
\end{equation}
\begin{equation}
\omega_{1}^{2}=\frac{k}{m}\,\,\,\,\,\,\,\,\omega_{2}^{2}=\frac{k+2g}{m}
\end{equation}
which allows one to rewrite the Hamiltonian as
\begin{equation}
H=\sum_{j=1}^{2}a_{j}^{\dagger}a_{j}\omega_{j}+\frac{\omega_{1}+\omega_{2}}{2}.
\end{equation}
In this basis, we denote the ground state as
\begin{equation}
a_{1}|00\rangle_{\vec{n}_{y}}=0=a_{2}|00\rangle_{\vec{n}_{y}}.
\end{equation}
Hence we have found $|\Omega\rangle=|00\rangle_{\vec{n}_{y}}$ .
We know that the wave function in the $\vec{y}$ coordinates is the
product of well known simple harmonic oscillator solutions:
\begin{equation}
\langle y'_{1},y'_{2}|00\rangle_{\vec{n}_{y}}=\frac{1}{\left(\pi b_{1}^{2}\right)^{1/4}}\exp\left[\frac{-\left(y_{1}'\right)^{2}}{2b_{1}^{2}}\right]\frac{1}{\left(\pi b_{2}^{2}\right)^{1/4}}\exp\left[\frac{-\left(y_{2}'\right)^{2}}{2b_{2}^{2}}\right]
\end{equation}
where
\begin{equation}
b_{n}\equiv\frac{1}{\sqrt{m\omega_{n}}}
\end{equation}
making this a convenient basis to work with. Note
\begin{align*}
\hat{y}_{1}\left(\left|a\right\rangle _{x_{1}}\otimes\left|b\right\rangle _{x_{2}}\right) & =\int dy_{1}'dy_{2}'\hat{y}_{1}|y_{1}'y_{2}'\rangle\langle y_{1}'y_{2}'|\left(\left|a\right\rangle _{x_{1}}\otimes\left|b\right\rangle _{x_{2}}\right)\\
 & =\int dy_{1}'dy_{2}'y_{1}'|y_{1}'y_{2}'\rangle\langle y_{1}'y_{2}'|\left(\left|a\right\rangle _{x_{1}}\otimes\left|b\right\rangle _{x_{2}}\right)\\
 & =\int dx_{1}'dx_{2}'y_{1}'\left(\left|x_{1}'\right\rangle _{x_{1}}\otimes\left|x_{2}'\right\rangle _{x_{2}}\right)\left(\,_{x_{1}}\langle x_{1}'|\otimes\,_{x_{2}}\langle x_{2}'|\right)\left(\left|a\right\rangle _{x_{1}}\otimes\left|b\right\rangle _{x_{2}}\right)\\
 & =\frac{a+b}{\sqrt{2}}\left(\left|a\right\rangle _{x_{1}}\otimes\left|b\right\rangle _{x_{2}}\right)\numberthis
\end{align*}
where we used the completeness of the basis, Eqs.~(\ref{eq:x1})
and (\ref{eq:x2}), and the usual delta function normalization of
the position basis. This and a similar relation for $\hat{y}_{2}$
imply
\begin{equation}
\left|a\right\rangle _{x_{1}}\otimes\left|b\right\rangle _{x_{2}}=\left|\frac{a+b}{\sqrt{2}},\frac{a-b}{\sqrt{2}}\right\rangle .
\end{equation}
This means
\begin{equation}
\,_{\vec{n}_{y}}\langle00|\left(|x_{1}'\rangle_{x_{1}}\otimes|w\rangle_{x_{2}}\right)=\,_{\vec{n}_{y}}\langle00|\frac{x_{1}'+w}{\sqrt{2}},\frac{x_{1}'-w}{\sqrt{2}}\rangle=\frac{1}{\left(\pi b_{1}^{2}\right)^{1/4}}\exp\left[\frac{-\left(\frac{x_{1}'+w}{\sqrt{2}}\right)^{2}}{2b_{1}^{2}}\right]\frac{1}{\left(\pi b_{2}^{2}\right)^{1/4}}\exp\left[\frac{-\left(\frac{x_{1}'-w}{\sqrt{2}}\right)^{2}}{2b_{2}^{2}}\right].
\end{equation}
The partial trace is defined through the following contraction of
$(2,2)$ tensor to a $(1,1)$ tensor: 
\begin{align*}
\hat{\rho} & =\int dx_{1}''\int dx_{1}'\int dw\left(|x_{1}''\rangle_{x_{1}}\,_{x_{1}}\langle x_{1}''|\otimes\,_{x_{2}}\langle w|\right)\left(|00\rangle_{\vec{n}_{y}}\,_{\vec{n}_{y}}\langle00|\right)\left(|x_{1}'\rangle_{x_{1}}\otimes|w\rangle_{x_{2}}\,_{x_{1}}\langle x_{1}'|\right)\\
 & =\int dx_{1}''\int dx_{1}'\int dw|x_{1}''\rangle_{x_{1}}\,_{x_{1}}\langle x_{1}'|\frac{1}{\left(\pi b_{1}^{2}\right)^{1/4}}\exp\left[\frac{-\left(\frac{x_{1}''+w}{\sqrt{2}}\right)^{2}}{2b_{1}^{2}}\right]\frac{1}{\left(\pi b_{2}^{2}\right)^{1/4}}\exp\left[\frac{-\left(\frac{x_{1}''-w}{\sqrt{2}}\right)^{2}}{2b_{2}^{2}}\right]\times\nonumber \\
 & \frac{1}{\left(\pi b_{1}^{2}\right)^{1/4}}\exp\left[\frac{-\left(\frac{x_{1}'+w}{\sqrt{2}}\right)^{2}}{2b_{1}^{2}}\right]\frac{1}{\left(\pi b_{2}^{2}\right)^{1/4}}\exp\left[\frac{-\left(\frac{x_{1}'-w}{\sqrt{2}}\right)^{2}}{2b_{2}^{2}}\right].\numberthis
\end{align*}
Integrate over $w$, we find
\begin{align*}
\hat{\rho} & =\int dx_{1}''\int dx_{1}'|x_{1}''\rangle_{x_{1}}\,_{x_{1}}\langle x_{1}'|\frac{1}{\left(\pi b_{1}^{2}\right)^{1/2}}\frac{1}{\left(\pi b_{2}^{2}\right)^{1/2}}\exp\left[-\frac{m}{4}\left(\omega_{1}+\omega_{2}\right)\left(\left[x_{1}'\right]^{2}+\left[x_{1}''\right]^{2}\right)\right]\times\nonumber \\
 & \frac{\sqrt{2\pi}}{\sqrt{m\left[\omega_{1}+\omega_{2}\right]}}\exp\left[\frac{(\frac{\sqrt{\omega_{2}}}{\sqrt{\omega_{1}}}-\frac{\sqrt{\omega_{1}}}{\sqrt{\omega_{2}}})^{2}(x_{1}'+x_{1}'')^{2}}{8\frac{1}{m}(\frac{1}{\omega_{1}}+\frac{1}{\omega_{2}})}\right]\\
 & =\int dx_{1}''\int dx_{1}'|x_{1}''\rangle_{x_{1}}\,_{x_{1}}\langle x_{1}'|\frac{1}{\left(\pi b_{1}^{2}\right)^{1/2}}\frac{1}{\left(\pi b_{2}^{2}\right)^{1/2}}\times\nonumber \\
 & \frac{\sqrt{2\pi}}{\sqrt{m\left[\omega_{1}+\omega_{2}\right]}}\exp\left[\frac{m(\omega_{2}-\omega_{1})^{2}2x_{1}'x_{1}''-m\left[8\omega_{1}\omega_{2}+(\omega_{1}-\omega_{2})^{2}\right]\left(\left[x_{1}'\right]^{2}+\left[x_{1}''\right]^{2}\right)}{8(\omega_{1}+\omega_{2})}\right].\numberthis
\end{align*}
Next, to identify the matrix, use
\begin{equation}
\frac{m(\omega_{2}-\omega_{1})^{2}2x_{1}'x_{1}''-m\left[8\omega_{1}\omega_{2}+(\omega_{1}-\omega_{2})^{2}\right]\left(\left[x_{1}'\right]^{2}+\left[x_{1}''\right]^{2}\right)}{8(\omega_{1}+\omega_{2})}=-\frac{1}{2b^{2}}\left[\left(\left[x_{1}'\right]^{2}+\left[x_{1}''\right]^{2}\right)-2\frac{(\omega_{2}-\omega_{1})^{2}}{\gamma}x_{1}'x_{1}''\right]
\end{equation}
\begin{equation}
\gamma\equiv8\omega_{1}\omega_{2}+(\omega_{1}-\omega_{2})^{2}
\end{equation}
\begin{equation}
\frac{1}{2b^{2}}\equiv\frac{m\gamma}{8(\omega_{1}+\omega_{2})}
\end{equation}
\begin{equation}
b=2\sqrt{\frac{\omega_{1}+\omega_{2}}{m[8\omega_{1}\omega_{2}+(\omega_{1}-\omega_{2})^{2}]}}
\end{equation}
to write
\begin{align}
\hat{\rho} & =\int dx_{1}''\int dx_{1}'|x_{1}''\rangle_{x_{1}}\,_{x_{1}}\langle x_{1}'|\frac{1}{\left(\pi b_{1}^{2}\right)^{1/2}}\frac{1}{\left(\pi b_{2}^{2}\right)^{1/2}} \times \nonumber\\
 & \frac{\sqrt{2\pi}}{\sqrt{m\left[\omega_{1}+\omega_{2}\right]}}\exp\left[-\frac{1}{2b^{2}}\left(\left[x_{1}'\right]^{2}+\left[x_{1}''\right]^{2}\right)\right]\exp\left(\frac{(\omega_{2}-\omega_{1})^{2}}{\gamma b^{2}}x_{1}'x_{1}''\right).
\end{align}
Change basis to energy with a new effective frequency
\begin{equation}
b_{3}=\frac{1}{\sqrt{m\omega_{3}}}
\end{equation}
\begin{equation}
\hat{\rho}=\sum_{nv}|v\rangle\langle v|\hat{\rho}|n\rangle\langle n|
\end{equation}
\begin{align}
\langle v|\hat{\rho}|n\rangle & =\int dx_{1}''\int dx_{1}'\langle v|x_{1}''\rangle_{x_{1}}\,_{x_{1}}\langle x_{1}'|n\rangle\frac{1}{\left(\pi b_{1}^{2}\right)^{1/2}}\frac{1}{\left(\pi b_{2}^{2}\right)^{1/2}}\times\nonumber \\
 & \frac{\sqrt{2\pi}}{\sqrt{m\left[\omega_{1}+\omega_{2}\right]}}\exp\left[-\frac{1}{2b^{2}}\left(\left[x_{1}'\right]^{2}+\left[x_{1}''\right]^{2}\right)\right]\exp\left(\frac{(\omega_{2}-\omega_{1})^{2}}{\gamma b^{2}}x_{1}'x_{1}''\right)
\end{align}
where
\begin{equation}
\langle x_{1}'|n\rangle=\frac{1}{\sqrt{n!b_{3}\sqrt{\pi}2^{n}}}e^{\frac{-\left(x_{1}'\right)^{2}}{2b_{3}^{2}}}H_{n}\left(\frac{x_{1}'}{b_{3}}\right)
\end{equation}
are the well known oscillator wave functions and $b_{3}$ still has
to be chosen. One can show by carrying out the integrals that the
matrix is diagonalized if 
\begin{align*}
b_{3} & =\frac{b}{\left(1-b^{4}\left[\frac{(\omega_{2}-\omega_{1})^{2}}{\gamma b^{2}}\right]^{2}\right)^{1/4}}\\
 & =\frac{1}{\sqrt{m}\omega_{1}^{1/4}\omega_{2}^{1/4}}.\numberthis
\end{align*}
This gives
\[
\langle v|\hat{\rho}|n\rangle=\lambda_{n}\delta_{vn}
\]
 where
\begin{align*}
\lambda_{n} & =\frac{\sqrt{2\pi}}{\sqrt{m\left[\omega_{1}+\omega_{2}\right]}}\frac{1}{\left(\pi b_{1}^{2}\right)^{1/2}}\frac{1}{\left(\pi b_{2}^{2}\right)^{1/2}}m_{11}\left(\frac{b^{2}\frac{(\omega_{2}-\omega_{1})^{2}}{\gamma b^{2}}}{1+\sqrt{1-b^{4}\left[\frac{(\omega_{2}-\omega_{1})^{2}}{\gamma b^{2}}\right]^{2}}}\right)^{n-1}\\
 & =\frac{\pi\sqrt{m}}{2\left[\omega_{1}+\omega_{2}\right]^{3/2}}\frac{1}{\left(\pi b_{1}^{2}\right)^{1/2}}\frac{1}{\left(\pi b_{2}^{2}\right)^{1/2}}\frac{(\omega_{2}-\omega_{1})^{2}}{\left(\sqrt{m}\omega_{1}^{1/4}\omega_{2}^{1/4}\right)^{3}\left(\frac{b_{3}^{2}}{b^{2}}+1\right)^{3/2}}\left(\frac{\frac{(\omega_{2}-\omega_{1})^{2}}{8\omega_{1}\omega_{2}+(\omega_{1}-\omega_{2})^{2}}}{1+\frac{b^{2}}{b_{3}^{2}}}\right)^{n-1}\numberthis
\end{align*}
where we used
\begin{align*}
m_{11} & =\frac{b^{3}\frac{(\omega_{2}-\omega_{1})^{2}}{\gamma b^{2}}\sqrt{2\pi}}{\left(1+\sqrt{1-b^{4}\left(\frac{(\omega_{2}-\omega_{1})^{2}}{\gamma b^{2}}\right)^{2}}\right)^{3/2}}\\
 & =\frac{m(\omega_{2}-\omega_{1})^{2}\sqrt{2\pi}}{4(\omega_{1}+\omega_{2})\left(\frac{1}{b^{2}}+\frac{1}{b_{3}^{2}}\right)^{3/2}}\numberthis
\end{align*}
\begin{align*}
\left(\frac{b_{3}}{b}\right)^{2} & =\frac{1}{m\omega_{1}^{1/2}\omega_{2}^{1/2}}\frac{1}{4\frac{\omega_{1}+\omega_{2}}{m[8\omega_{1}\omega_{2}+(\omega_{1}-\omega_{2})^{2}]}}\\
 & =\frac{1}{\omega_{1}^{1/2}\omega_{2}^{1/2}}\frac{8\omega_{1}\omega_{2}+(\omega_{1}-\omega_{2})^{2}}{4\left(\omega_{1}+\omega_{2}\right)}.\numberthis
\end{align*}
Simplify:
\begin{align*}
\lambda_{n} & =\frac{4\sqrt{\omega_{1}\omega_{2}}}{\sqrt{8\omega_{1}\omega_{2}+(\omega_{1}-\omega_{2})^{2}+4\omega_{1}^{1/2}\omega_{2}^{1/2}\left(\omega_{1}+\omega_{2}\right)}}\left(\frac{(\omega_{2}-\omega_{1})^{2}}{8\omega_{1}\omega_{2}+(\omega_{1}-\omega_{2})^{2}+\omega_{1}^{1/2}\omega_{2}^{1/2}4\left(\omega_{1}+\omega_{2}\right)}\right)^{n}\\
 & =\frac{4\sqrt{\omega_{1}\omega_{2}}}{\left(\sqrt{\omega_{1}}+\sqrt{\omega_{2}}\right)^{2}}\left[\frac{(\omega_{1}-\omega_{2})^{2}}{\left(\sqrt{\omega_{1}}+\sqrt{\omega_{2}}\right)^{4}}\right]^{n}.\numberthis
\end{align*}
Since we want to evaluate
\begin{equation}
-\mathrm{Tr}\left[\hat{\rho}\ln\hat{\rho}\right]=-\partial_{n}\ln\mathrm{tr}\hat{\rho}^{n}|_{n=1}
\end{equation}
we compute
\begin{align*}
\ln\mathrm{tr}\rho^{n} & =\ln\left(\sum_{j=0}^{\infty}\lambda_{j}^{n}\right)\\
 & =\ln\left(\sum_{j}\left[\frac{4\sqrt{\omega_{1}\omega_{2}}}{\left(\sqrt{\omega_{1}}+\sqrt{\omega_{2}}\right)^{2}}\left[\frac{(\omega_{1}-\omega_{2})^{2}}{\left(\sqrt{\omega_{1}}+\sqrt{\omega_{2}}\right)^{4}}\right]^{j}\right]^{n}\right)\\
 & =n\ln\left[\frac{4\sqrt{\omega_{1}\omega_{2}}}{\left(\sqrt{\omega_{1}}+\sqrt{\omega_{2}}\right)^{2}}\right]+\ln\left(\sum_{j}\left[\frac{(\omega_{1}-\omega_{2})^{2}}{\left(\sqrt{\omega_{1}}+\sqrt{\omega_{2}}\right)^{4}}\right]^{nj}\right)\\
 & =n\ln\left[\frac{4\sqrt{\omega_{1}\omega_{2}}}{\left(\sqrt{\omega_{1}}+\sqrt{\omega_{2}}\right)^{2}}\right]-\ln\left(1-\left[\frac{(\omega_{1}-\omega_{2})^{2}}{\left(\sqrt{\omega_{1}}+\sqrt{\omega_{2}}\right)^{4}}\right]^{n}\right).\numberthis
\end{align*}
Hence, we arrive at
\begin{align*}
S & =-\left\{ \ln\left[\frac{4\sqrt{\omega_{1}\omega_{2}}}{\left(\sqrt{\omega_{1}}+\sqrt{\omega_{2}}\right)^{2}}\right]-\frac{\left[\frac{(\omega_{1}-\omega_{2})^{2}}{\left(\sqrt{\omega_{1}}+\sqrt{\omega_{2}}\right)^{4}}\right]\ln\left[\frac{(\omega_{1}-\omega_{2})^{2}}{\left(\sqrt{\omega_{1}}+\sqrt{\omega_{2}}\right)^{4}}\right]}{\left(1-\left[\frac{(\omega_{1}-\omega_{2})^{2}}{\left(\sqrt{\omega_{1}}+\sqrt{\omega_{2}}\right)^{4}}\right]\right)}\right\} \\
 & =\boxed{-\ln\left(\frac{4\sqrt{\omega_{1}\omega_{2}}}{\left(\sqrt{\omega_{1}}+\sqrt{\omega_{2}}\right)^{2}}\right)-\left(\frac{(\omega_{2}-\omega_{1})^{2}}{4\sqrt{\omega_{1}\omega_{2}}\left(\sqrt{\omega_{1}}+\sqrt{\omega_{2}}\right)^{2}}\right)\ln\left(\frac{(\omega_{2}-\omega_{1})^{2}}{\left(\sqrt{\omega_{1}}+\sqrt{\omega_{2}}\right)^{4}}\right)}\numberthis
\end{align*}
where
\begin{equation}
\boxed{\omega_{1}=\sqrt{\frac{k}{m}}\,\,\,\,\,\,\,\,\omega_{2}=\sqrt{\frac{k+2g}{m}}}.
\end{equation}







\clearpage

\subsection{Level 4 - SUSY-Symmetry}\label{L4-susy}


\subsubsection*{Problem Statement}

Consider the theory
\begin{equation}
\mathcal{L}=i\bar{\xi}\bar{\sigma}^{\mu}\partial_{\mu}\xi+|\partial\phi|^{2}-|F|^{2}
\end{equation}
where $\xi$ is a 2-component Weyl spinor while $\phi$ and $F$ are
complex scalar fields. Suppose you want to make the following infinitesimal
transformation a symmetry of this theory:
\begin{equation}
\delta_{\eta}\xi_{\alpha}=i\sqrt{2}\sigma_{\alpha\dot{\alpha}}^{\mu}\bar{\eta}^{\dot{\alpha}}\partial_{\mu}\phi+\sqrt{2}\eta_{\alpha}F
\end{equation}
\begin{align*}
\delta_{\eta}\bar{\xi}_{\dot{\beta}} & = [i\sqrt{2}\sigma_{\beta\dot{\alpha}}^{\mu}\bar{\eta}^{\dot{\alpha}}\partial_{\mu}\phi+\sqrt{2}\eta_{\beta}F]^{\dagger}\\
 & = -i\sqrt{2}(\bar{\eta}^{\dot{\alpha}}\sigma_{\dot{\alpha}\beta}^{\mu*})^{*}\partial_{\mu}\bar{\phi}+\sqrt{2}\bar{\eta}_{\dot{\beta}}\bar{F}\\
 & = -i\sqrt{2}\eta^{\alpha}\sigma_{\alpha\dot{\beta}}^{\mu}\partial_{\mu}\bar{\phi}+\sqrt{2}\bar{\eta}_{\dot{\beta}}\bar{F}\numberthis
\end{align*}
\begin{equation}
\delta_{\eta}F=i\sqrt{2}\bar{\eta}_{\dot{\alpha}}\bar{\sigma}^{\mu\dot{\alpha}\alpha}\partial_{\mu}\xi_{\alpha}=i\sqrt{2}\bar{\eta}\bar{\sigma}^{\mu}\partial_{\mu}\xi
\end{equation}
\begin{align*}
\delta_{\eta}\bar{F} & = -i\sqrt{2}(\bar{\eta}\bar{\sigma}^{\mu}\partial_{\mu}\xi)^{\dagger}\\
 & = -i\sqrt{2}(\partial_{\mu}\xi)^{\dagger}(\bar{\sigma}^{\mu})^{\dagger}(\bar{\eta})^{\dagger}\\
 & = -i\sqrt{2}\partial_{\mu}\bar{\xi}\bar{\sigma}^{\mu}\eta\numberthis
\end{align*}
along with $\delta_{\eta}\phi$ and $\left(\delta_{\eta}\phi\right)^{\dagger}$
where $\eta$ is a spacetime-independent infinitesimal fermionic parameter
inducing the transformation. Find the transformation rule $\delta_{\eta}\phi$
and $\left(\delta_{\eta}\phi\right)^{\dagger}$ for the action associated
with $\mathcal{L}$ to remain invariant.

\subsubsection*{Answer Requirements}
Provide the answer in the form of the \texttt{python} code. Implement the following function 
\begin{python}
from math import sqrt
def find_delta_phi(eta:float, xi:float, bar_eta:float, bar_xi:float) -> Tuple[float, float]:
    """
    Returns the SUSY transformation rules for phi and its Hermitian conjugate: 
    a tuple (delta_phi, delta_phi_dagger)
    """
    pass
\end{python}


\subsubsection*{Comments about the Problem}
\textit{This problem is situated in advanced quantum field theory within the framework of supersymmetry (SUSY). It involves analyzing how bosonic and fermionic fields transform under an infinitesimal SUSY transformation and requires knowledge and careful application of Grassmann variables and the associated algebra. Such topics are typically encountered in advanced graduate-level physics courses.  Note that the Hermiticity of $\sigma^\mu$ matrix convention as well as the metric convention of $(1,-1,-1,-1)$ is implicit in the statement of the problem (the latter inherent in the kinetic minus the potential form of the Lagrangian).}


\subsubsection*{Solution}
Denoting the variation $\left(\delta_{\eta}\phi\right)^{\dagger}$
as $\delta_{\eta}\bar{\phi}$, we write 
\begin{align*}
\delta_{\eta}\mathcal{L} & = i\delta_{\eta}\bar{\xi}\bar{\sigma}^{\mu}\partial_{\mu}\xi+i\bar{\xi}\bar{\sigma}^{\mu}\partial_{\mu}\delta_{\eta}\xi+\partial_{\mu}\delta_{\eta}\bar{\phi}\partial^{\mu}\phi+\partial_{\mu}\bar{\phi}\partial^{\mu}\delta_{\eta}\phi-\delta_{\eta}\bar{F}F-\bar{F}\delta_{\eta}F\\
 & = i[-i\sqrt{2}\eta\sigma^{\beta}\partial_{\beta}\bar{\phi}+\cancel{\sqrt{2}\bar{\eta}\bar{F}}]\bar{\sigma}^{\mu}\partial_{\mu}\xi+i\bar{\xi}\bar{\sigma}^{\mu}\partial_{\mu}[i\sqrt{2}\sigma^{\beta}\bar{\eta}\partial_{\beta}\phi+\sqrt{2}\eta F]\\
 & +\partial_{\mu}\delta_{\eta}\bar{\phi}\partial^{\mu}\phi+\partial_{\mu}\bar{\phi}\partial^{\mu}\delta_{\eta}\phi-[-i\sqrt{2}\partial_{\mu}\bar{\xi}\bar{\sigma}^{\mu}\eta]F-\cancel{\bar{F}[i\sqrt{2}\bar{\eta}\bar{\sigma}^{\mu}\partial_{\mu}\xi]}\,.\numberthis
\end{align*}
Integrating by parts, we find (denoting with equality an equivalence
up to total derivative terms)
\begin{align}
\delta_{\eta}\mathcal{L} & =\sqrt{2}\eta\sigma^{\beta}\partial_{\beta}\bar{\phi}\bar{\sigma}^{\mu}\partial_{\mu}\xi+\partial_{\mu}\bar{\xi}\bar{\sigma}^{\mu}[\sqrt{2}\sigma^{\beta}\bar{\eta}\partial_{\beta}\phi-\cancel{i\sqrt{2}\eta F}]\nonumber \\
 & +\partial_{\mu}\delta_{\eta}\bar{\phi}\partial^{\mu}\phi+\partial_{\mu}\bar{\phi}\partial^{\mu}\delta_{\eta}\phi+\cancel{i\sqrt{2}\partial_{\mu}\bar{\xi}\bar{\sigma}^{\mu}\eta F} \,.
\end{align}
Integrate by parts the first two terms to eliminate the the $\sigma$
matrices using the identity $\bar{\sigma}^{\mu}\sigma^{\nu}+\bar{\sigma}^{\nu}\sigma^{\mu}=2g^{\mu\nu}$:
\begin{align}
\delta_{\eta}\mathcal{L} & =\sqrt{2}\left(\eta\partial_{\mu}\bar{\phi}\partial^{\mu}\xi+\partial^{\mu}\bar{\xi}\bar{\eta}\partial_{\mu}\phi\right)\nonumber \\
 & +\partial_{\mu}\delta_{\eta}\bar{\phi}\partial^{\mu}\phi+\partial_{\mu}\bar{\phi}\partial^{\mu}\delta_{\eta}\phi
\end{align}
again denoting with equality an equivalence up to total derivative
terms, and we are using the standard notation $\eta\xi\equiv\eta^{\alpha}\xi_{\alpha}$
and $\bar{\xi}_{\dot{\alpha}}\bar{\eta}^{\dot{\alpha}}\equiv\bar{\xi}\bar{\eta}$.
To make the remainder cancel, we solve
\begin{equation}
\sqrt{2}\eta\partial_{\mu}\bar{\phi}\partial^{\mu}\xi+\partial_{\mu}\bar{\phi}\partial^{\mu}\delta_{\eta}\phi=0
\end{equation}
yielding
\begin{equation}
\boxed{\delta_{\eta}\phi=-\sqrt{2}\eta\xi,\quad\left(\delta_{\eta}\phi\right)^{\dagger}=-\sqrt{2}\bar{\xi}\bar{\eta}}.\label{eq:L4-susy}
\end{equation}





\clearpage


\subsection{Level 3 - Slow-Roll Inflation}


\subsubsection*{Problem Statement}
For the action
\begin{equation}
S = \int dt a^3(t) \left\{ \frac{1}{2} \dot{\phi}^2 - V_0 \exp \left[ - \sqrt{\frac{2}{q}} \left( \frac{\phi}{M_P} \right) \right] \right\}
\end{equation}
where \( q \) and \(V_0\) are constants, derive and solve (integrate) the equation of motion for the field $\phi$ assuming slow-roll inflation and initial condition $\phi(t=0) = \phi_0$.

\subsubsection*{Answer Requirements}
Provide the answer in the form of the \texttt{python} code. Implement the following function 
\begin{python}
import numpy as np
def phi(q: float, M_p: float, phi_0: float, V_0: float, t: np.ndarray)->np.ndarray:
    pass
\end{python}

\subsubsection*{Comments about the Problem}
\textit{This problem lies in the field of cosmology, particularly in inflationary cosmology, and involves studying the dynamics of a scalar field (inflaton) driving the accelerated expansion of the early universe, before the ``hot Big Bang". It is typically encountered in specialized graduate-level courses in cosmology and requires familiarity with field theory in an expanding spacetime.}

\subsubsection*{Solution}
The equation of motion is
\begin{equation}
\ddot{\phi} + 3 H \dot{\phi} - \sqrt{\frac{2}{q}} \left( \frac{1}{M_P} \right) V_0 \exp \left[ - \sqrt{\frac{2}{q}} \left( \frac{\phi}{M_P} \right) \right] = 0.
\end{equation}
For the slow-roll inflation, the following must hold: 
\begin{equation}
    \ddot{\phi}\ll 3H\dot{\phi} \,.
\end{equation}
Hence, we have
\begin{equation}
3 H \dot{\phi} = \sqrt{\frac{2}{q}} \left( \frac{1}{M_P} \right) V_0 \exp \left[ - \sqrt{\frac{2}{q}} \left( \frac{\phi}{M_P} \right) \right].
\end{equation}
Slow-roll approximation also implies 
\begin{equation}
H^2\approx\frac{V(\phi)}{3M^2_P}
\end{equation}
so we need to solve the following ODE:
\begin{equation}
3 \sqrt{\frac{V_0 \exp \left[ - \sqrt{\frac{2}{q}} \left( \frac{\phi}{M_P} \right) \right]}{3 M_P^2}} \frac{d\phi}{dt} = \sqrt{\frac{2}{q}} \left( \frac{1}{M_P} \right) V_0 \exp \left[ - \sqrt{\frac{2}{q}} \left( \frac{\phi}{M_P} \right) \right]
\end{equation}

\begin{equation}
\int \frac{d\phi}{\sqrt{V_0}} \exp \left[ \sqrt{\frac{1}{2q}} \left( \frac{\phi}{M_P} \right) \right] = \sqrt{\frac{2}{3q}} t \,.
\end{equation}
Performing the integration and solving for $\phi(t)$ we get
\begin{equation}
\frac{1}{\sqrt{V_0}} M_P \sqrt{2q} \left( \exp \left[ \sqrt{\frac{1}{2q}} \left( \frac{\phi}{M_P} \right) \right] - \exp \left[ \sqrt{\frac{1}{2q}} \left( \frac{\phi_0}{M_P} \right) \right] \right) = \sqrt{\frac{2}{3q}} t
\end{equation}

\begin{equation}
\boxed{
\phi = \sqrt{2q} M_P \ln \left\{ \exp \left[ \sqrt{\frac{1}{2q}} \left( \frac{\phi_0}{M_P} \right) \right] + \frac{1}{M_P q} \sqrt{\frac{V_0}{3}} t \right\}}.
\end{equation}



\clearpage




\subsection{Level 3 - Scalar Particle Scattering}


\subsubsection*{Problem Statement}
Consider
\begin{equation}
\mathcal{L} = \left\{ \sum_{i=1}^2 \left[ \frac{1}{2} (\partial_\mu \phi_i)(\partial^\mu \phi_i) - \frac{m_i^2}{2} \phi_i \phi_i \right] - \frac{\lambda}{4} \phi_1^2 \phi_2^2 \right\}
\end{equation}
What is the differential cross section \( \frac{d\sigma}{d\Omega} \) for \( \phi_1 (\vec{k}_1) \phi_1 (-\vec{k}_1) \to \phi_2 (\vec{k}_1') \phi_2 (-\vec{k}_1') \) in the CM frame accurate to \( O(\lambda^2) \)? Express your final answer in terms of Mandelstam variables. 

\subsubsection*{Answer Requirements}
Provide the answer in the form of the \texttt{python} code. Implement the following function.
\begin{python}
def dsigma_domega(lam: float, s_m: float, p_m: float, u_m: float, 
                  m1: float, m2: float) -> float:
    pass
\end{python}


\subsubsection*{Comments about the Problem}
\textit{This is a question from quantum field theory. It involves calculating the differential cross section for a process where two \(\phi_1\) particles annihilate into two \(\phi_2\) particles. Such problems are typically encountered in graduate-level particle physics courses and require familiarity with perturbative field theory and the use of Mandelstam variables to express scattering amplitudes.}

\subsubsection*{Solution}
The amplitude for this process is
\begin{equation}
i \mathcal{M} = -4 i \frac{\lambda}{4} = -i \lambda
\end{equation}
In the CM frame, energy conservation gives
\begin{equation}
2 \sqrt{|\vec{k}_1|^2 + m_1^2} = 2 \sqrt{|\vec{k}_1'|^2 + m_2^2}
\end{equation}
A standard formula for differential cross section gives
\begin{align*}
\left( \frac{d\sigma}{d\Omega} \right)_{\text{CM}} &= \frac{1}{64 \pi^2 s} \frac{k_1'}{k_1} |\mathcal{M}|^2 \\
&= \frac{\lambda^2}{64 \pi^2 s} \frac{\sqrt{|\vec{k}_1|^2 + (m_1^2 - m_2^2)}}{k_1} \numberthis
\end{align*}
Since in the CM frame, we know
\begin{equation}
k_1 = \frac{1}{2 \sqrt{s}} \sqrt{s^2 - 4 m_1^2 s}
\end{equation}
\begin{align*}
\left( \frac{d\sigma}{d\Omega} \right)_{\text{CM}} &= \frac{2 \sqrt{s}}{64 \pi^2 s} \sqrt{\frac{1}{4s} \left[ s^2 - 4 m_1^2 s \right] + (m_1^2 - m_2^2)} \frac{\lambda^2}{\sqrt{s^2 - 4 m_1^2 s}} \\
&= \frac{\lambda^2}{64 \pi^2 s} \frac{\sqrt{s^2 - 4 m_1^2 s + 4s(m_1^2 - m_2^2)}}{\sqrt{s^2 - 4 m_1^2 s}}.\numberthis
\end{align*}

The final result is
\begin{equation}
\boxed{
\left( \frac{d\sigma}{d\Omega} \right)_{\text{CM}}
= \frac{\lambda^2}{64 \pi^2 s} \frac{\sqrt{s - 4 m_2^2}}{\sqrt{s - 4 m_1^2}}
}.
\end{equation}







\clearpage

\subsection{Level 2 - Dark Matter Capture as a Function of Time}

\subsubsection*{Problem Statement}
Suppose $C$ is the capture rate of dark matter in an astrophysical
body. Let $C_{A}$ be the dark matter annihilation rate per effective
volume. Then an approximate Boltzmann equation governing the number
$N$ of dark matter particles in the astrophysical body is
\[
\frac{d N}{dt}=C-C_{A}N^{2}.
\]
If initially, $N(0)=0$, what is $N(t)$ as a function of time?

\subsubsection*{Answer Requirements}

Provide the answer in the form of the \texttt{python} code. Implement the following function.
\begin{python}
def answer(C: float, C_A: float, t: float) -> float:
    pass
\end{python}

\subsubsection*{Comments about the Problem}
\textit{This problem mainly belongs to astrophysics, specifically involving dark matter dynamics in celestial bodies. It is typically encountered in advanced undergraduate or graduate-level courses and requires knowledge of differential equations and kinetic theory. This type of analysis is also important for understanding dark matter detection and its astrophysical implications.}

\subsubsection*{Solution}
We can integrate by quadrature.
\begin{equation}
\int\frac{dN}{C-C_{A}N^{2}}=t.
\end{equation}
We can express the integrand as a sum of two fractions:
\begin{align*}
\frac{1}{C-C_{A}N^{2}} & = \frac{1}{\sqrt{C}-\sqrt{C_{A}}N}\frac{1}{\sqrt{C}+\sqrt{C_{A}}N}\\
 & = \frac{1}{2\sqrt{C}}\left[\frac{1}{\sqrt{C}-\sqrt{C_{A}}N}+\frac{1}{\sqrt{C}+\sqrt{C_{A}}N}\right]\numberthis.
\end{align*}
Integrating, we find
\begin{align*}
t+K & = \frac{1}{2\sqrt{C}}\left[\frac{-1}{\sqrt{C_{A}}}\ln\left(\sqrt{C}-\sqrt{C_{A}}N\right)+\frac{1}{\sqrt{C_{A}}}\ln\left(\sqrt{C}+\sqrt{C_{A}}N\right)\right]\\
 & = \frac{1}{2\sqrt{C_{A}C}}\ln\left(\frac{\sqrt{C}+\sqrt{C_{A}}N}{\sqrt{C}-\sqrt{C_{A}}N}\right)\numberthis
\end{align*}
where $K$ is an integration constant. Setting the boundary condition
$N=0$ at $t=0$, we find
\[
K=0.
\]
We find the solution 
\begin{equation}
\boxed{N=\frac{\sqrt{C}}{\sqrt{C_{A}}}\frac{\left(e^{2\sqrt{C C_A}t}-1\right)}{\left(e^{2\sqrt{C C_A}t}+1\right)}}.
\end{equation}
Note that it is easy to check that it reaches the obvious steady state
in the limit $t\rightarrow\infty$.




\clearpage

\subsection{Level 2 - A 3-State QM Problem}


\subsubsection*{Problem Statement}
The Hamiltonian of a three-level system is given as $H = \begin{pmatrix}
    E_a & 0 & A \\
    0 & E_b & 0 \\
    A & 0 & E_a \\
\end{pmatrix}$ where $A$ is real. The state of the system at time $t=0$ is (in this basis) $\psi(t=0) = \frac{1}{\sqrt{2}}\begin{pmatrix}1 \\
1\\
0\end{pmatrix}$ What is the expectation value of the energy at time $t$?

\subsubsection*{Answer Requirements}
Provide the answer in the form of \texttt{python} code. Implement the following function
\begin{python}
def expectation_value(A: float, E_a:float, E_b:float, t:float) -> float:
    pass
\end{python}

\subsubsection*{Comments about the Problem}
\textit{This problem belongs to quantum mechanics, focusing on multi-level quantum systems found in areas like quantum optics or molecular physics. It is typically encountered in advanced undergraduate or early graduate-level courses and requires knowledge of linear algebra, time evolution, and the calculation of expectation values in quantum mechanics.}


\subsubsection*{Solution}
The eigenstates are easily found to be $\frac{1}{\sqrt{2}}(1,0,\pm 1)^T$ and $(0,1,0)^T$ with corresponding energies $E_a\pm A$, $E_b$. Let us denote them as $|1\rangle$, $|2\rangle$ and $|3\rangle$. Given state $\psi$ is decomposed as $\frac{1}{2}(|1\rangle +|2\rangle) + \frac{1}{\sqrt{2}}|3\rangle$, the expectation of energy stays constant: 
\begin{equation}
    \langle E\rangle = \frac{1}{4}((E_a+A)+(E_a-A)) + \frac{1}{2}E_b =\boxed{ \frac{1}{2}(E_a+E_b)}.
\end{equation}





\vspace{2cm}





\subsection{Level 1 - Blackbody in $d$ Dimensions} 


\subsubsection*{Problem Statement}
Assume we live in a 4+1 dimensional spacetime. How does the total energy density of a black body scale with temperature T. Find the exponent $n$ in the expression $u \propto T^{n}$

\subsubsection*{Answer Requirements}
Provide the answer in the form of \texttt{python} code. Implement the following function
\begin{python}
def answer() -> float:
    pass
\end{python}


\subsubsection*{Comments about the Problem}
\textit{This problem lies in the realm of statistical mechanics and thermodynamics applied to higher-dimensional spacetimes, a topic typically encountered at the undergraduate level in theoretical physics.}

\subsubsection*{Solution}
The density of states scales as $k^{D-1}dk$ in D spatial dimensions giving 
$T^{D+1}$ scaling for the total energy density. Hence, $\boxed{n=5}.$



\vspace{2cm}





\subsection{Level 1 - Boosted Parabolic Trajectory}

\subsubsection*{Problem Statement}
Consider a situation where a space-probe very briefly fires its rockets while passing a planet of mass \(M\) at periapsis, its nearest point to the planet. Suppose that the probe is on a parabolic trajectory and at periapsis, when travelling at velocity $v_e$, it results in a boost of $\delta v$. What will be its speed once it escapes the planet's gravitational field only in terms of $v_e$ and $\delta v$?

\subsubsection*{Answer Requirements}
Provide the answer in the form of \texttt{python} code. Implement the following function
\begin{python}
def speed(v_e: float, delta_v:float) -> float:
    pass
\end{python}

\subsubsection*{Comments about the Problem}
\textit{This problem is part of orbital mechanics, typically covered at the undergraduate or advanced high school level in physics. It involves principle of energy conservation in Newtonian gravity.}

\subsubsection*{Solution}
Conservation of energy gives $\frac{1}{2}m(v_e+\delta v)^2-\frac{mMG}{r_p} = \frac{1}{2}mv^2_\infty$. We also know that $\frac{1}{2}m(v_e)^2-\frac{mMG}{r_p} = E = 0$ for the parabolic trajectory. We can solve for $v_e$: $v_e = \sqrt{\frac{2MG}{r_p}}$. Then we can substitute it in the first equation and get:
\begin{equation}
\boxed{v_\infty = \delta v\sqrt{1+\frac{2v_e}{\delta v}}}.
\end{equation}

