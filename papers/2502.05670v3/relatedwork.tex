\section{Related Work}
Constituent movement has been the subject of considerable linguistic literature; we categorize relevant contributions by our three research questions.

\subsection{What are the exact effects of weight on constituent ordering?}

Significant prior work has focused on investigating the effects of weight on constituent ordering \cite{arnold_et_al, Wasow_Arnold_2003, Wasow_endweight, ARNOLD200455, Hawkins_1995, 1909bezie, hawkins2004}; many contributions find gradient effects by which the shift becomes more frequent in examined language corpora as the relative weight of the relative constituent changes, suggesting that weight is the predominant factor in triggering the shift, even cross-linguistically \cite{Wasow_1997, Wasow_Arnold_2003, faghiri, WangLiu, Hawkins_1999, quirketal, Manetta2012ReconsideringRS}. Furthermore, studies with human participants find similar results, where weight presents a primary role in the shift \cite{Medeiros_Mains_McGowan_2021}. The study, however, also suggests that this movement is constrained by ceiling effects, by which the efficacy of additional weight and complexity plateaus. More closely relevant to this work, \citet{futrell2018rnnslearnhumanlikeabstract} conduct a similar analysis on post-verbal constituent movement using weight as a binary feature (i.e. `long' vs `short'), and find similar trends with LSTMs. 

\subsection{What measure of weight best explains effects of constituent ordering?}

Weight is a measure of a constituent's complexity or size, but how best to measure it is less straightforward \cite{logstruct, haegeman1991, Wasow_endweight}. Related research categorizes and analyzes the effects of three primary measures of weight on constituent movement, particularly with HNPS: the word length of the NP, the number of nodes in the NP's syntactic structure, and the number of modifiers applied to the NP \cite{Wasow_Arnold_2003, Medeiros_Mains_McGowan_2021, Wasow_1997}. The analyses found that, although the word length was statistically the strongest predictor for HNPS, ``no single factor can account for observed constituent order alternation'' \cite[~pg.6]{Medeiros_Mains_McGowan_2021}. Similarly, \citet{Wasow_Arnold_2003} find in a corpus study that for HNPS and DA, constituent movement was best accounted for when considering both word length and modifier weight together, as opposed to either on its own.

\subsection{How exactly do LLM preferences around constituent shifting align with human constituent shifting preferences?}

Research concerning the behaviors of computational models has also shown that models exhibit human-like preferences \cite{Fujihara2022TopicalizationIL, linzen2016assessingabilitylstmslearn, marvin-linzen-2018-targeted, 10.1162/kamath2024}.
Notably, prior work shows models learn syntactic alternations \cite{wilcox2019syntacticstructuresblockdependencies, lau2017}; more directly relevant to us, \citet{futrell2018rnnslearnhumanlikeabstract} finds that behaviors of LSTMs appear to correlate closely with observed judgements of humans on corresponding data, suggesting that constituent movement is motivated similarly in both humans and models. This work, however, predates preference-aligned models \cite{ouyang2022traininglanguagemodelsfollow}; we hypothesize that the behaviors of such models will align even more closely with human preferences around constituent ordering.

\begin{figure*}
    \centering
    \includegraphics[width=\textwidth]{data.jpg}
    \caption{The outline for creating synthetic data, using varying modifier weights.}
    \label{fig:DFig}
\end{figure*}