\begin{lstlisting}[language=Python]
Alice is a robot holding a dustpan, Bob is a robot holding a broom, together they must sweep up all the cubes on the table.
To sweep up a cube, Alice must MOVE to the cube on the one side, while Bob must MOVE to the cube from the other side. After that, Bob Sweep the cube into the dustpan.
After all, Alice should DUMP everything in the dustpan.
At each round, given 'Scene description' and 'Environment feedback', use it to reason about the task, and improve any previous plans. Each robot does **exactly** one action per round.



[Action Options]
1) MOVE <target>, <target> can only be a cube.
2) SWEEP <target>, this moves the groom so it pushes the <target> into dustpan, only Bob can SWEEP, Alice must WAIT in front of the same <target> cube when Bob SWEEP. Remember MOVE is the necessary step before SWEEP.
3) WAIT, stays at the current spot.
4) DUMP, only when there are one or more cubes in the dustpan, Alice can DUMP it into trash_bin.
Only SWEEP a cube after both robots MOVEed to the cube.
[Action Output Instruction]
Must first output 'EXECUTE
', then give exactly one action per robot, put each on a new line.
Example#1: 'EXECUTE
NAME: Alice ACTION: MOVE red_cube
NAME: Bob ACTION: MOVE red_cube
'
Example#2: 'EXECUTE
NAME: Alice ACTION: WAIT
NAME: Bob ACTION: SWEEP red_cube
'
Example#3: 'EXECUTE
NAME: Alice ACTION: DUMP
NAME: Bob ACTION: MOVE green_cube
'
If actions for multiple steps are to generate, repeat the above process and follow the format strictly.
Example#4: 'EXECUTE
NAME: Alice ACTION: MOVE blue_cube
NAME: Bob ACTION: WAIT
EXECUTE
NAME: Alice ACTION: WAIT
NAME: Bob ACTION: SWEEP blue_cube
'

[History]
== Round#0 ==
[Executed Action]
Alice: MOVE green_cube
Bob: MOVE blue_cube

== Round#1 ==
[Executed Action]
Alice: WAIT
Bob: MOVE red_cube

== Round#2 ==
[Executed Action]
Alice: MOVE red_cube
Bob: MOVE red_cube

== Round#3 ==
[Executed Action]
Alice: WAIT
Bob: MOVE red_cube

== Round#4 ==
[Executed Action]
Bob: MOVE red_cube
Alice: MOVE red_cube

== Round#5 ==
[Executed Action]
Alice: WAIT
Bob: SWEEP red_cub

== Round#6 ==
[Executed Action]
Alice: MOVE green_cube
Bob: MOVE green_cube

== Round#7 ==
[Executed Action]
Alice: WAIT
Bob: WAIT

== Round#8 ==
[Executed Action]
Alice: SWEEP green_cube
Bob: WAIT

== Round#9 ==
[Executed Action]
Bob: MOVE blue_cube
Alice: MOVE blue_cube

== Round#10 ==
[Executed Action]
Alice: WAIT
Bob: WAIT
== Current Round ==

[Scene description]
red_cube is at (0.3, 0.4, 0.4), inside dustpan; 
green_cube is at (0.8, 0.5, 0.2), on the table; 
blue_cube is at (0.3, 0.5, 0.2), on the table; 
Alice's gripper is at (0.4, 0.3, 0.4), holding dustpan, in front of green_cube with distance: 0.53, in front of blue_cube with distance: 0.31
Bob's gripper is at (0.3, 0.8, 0.6), holding broom, in front of green_cube with distance: 0.58, in front of blue_cube with distance: 0.33

Based on the current state, in order to complete task, someone proposed EXECUTE
NAME: Alice ACTION: SWEEP blue_cube
NAME: Bob ACTION: WAIT as action for the next step.
However, this action failed to employ in the env. Usually, the failure may be due to one of the following reasons.
1.Response does not contain some keyword. The keywords in this env includes ['NAME:', 'ACTION:'].
2.Response missing plan for some robot. The robot in this env includes dict_values(['Alice', 'Bob']).
3.Reponse must contain exactly one ACTION for each robot, and must contain all keywords. The keywords in this env includes ['NAME:', 'ACTION:'].
4.Bad action for some robot, this robot at current state can only MOVE or WAIT. The robot in this env includes dict_values(['Alice', 'Bob']).
5.Planned PATH must have exact same number of steps of all agents.
You should think about which reason is most possible for the failure of the past action, you should clearly output the reason to help yourself to genetate better reasoning and action in future.
After your thinking, you should output your reflection like
```reflection
Put your reflection here
```
Your thinking and reflection are:
\end{lstlisting}
