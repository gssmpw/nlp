
Ubuntu Terminal Tasks


\begin{lstlisting}[language=Python]
You are an Operating System assistant who can interact with Ubuntu Terminal to complete Operating System Tasks. You can interact with the Ubuntu Operating system by terminal commands. To solve the task, you must understand the [Reasoning Thought] and faithfully follow the guidance of [Reasoning Thought].

[Reasoning Thought]
Your analysis about the current state and how to solve the task is:
1. To find the 'docs' directory, locate all '.txt' files within it, change their permissions to read-only for all users, and verify the changes, we can use the following command:\n\n   find / -type d -name docs -exec sh -c 'find {} -type f -name \"*.txt\" -exec chmod 444 {} \\; -exec ls -l {} \\;' \\;\n\n   This command does the following:\n   - Searches for the 'docs' directory starting from the root (/)\n   - Finds all '.txt' files within the 'docs' directory and its subdirectories\n   - Changes the permissions of each found file to 444 (read-only for all users)\n   - Lists the files with their new permissions

2. If the command executes successfully, we'll see a list of '.txt' files with their new permissions. Each file should have permissions like \"-r--r--r--\".

3. If we don't see any output or get an error, it might mean that either the 'docs' directory doesn't exist, there are no '.txt' files, or we don't have the necessary permissions to access or modify the files.

Let's execute this command and analyze the output to determine if further actions are needed..

[Action Options and Action Output Instruction]
1. Bash Action
If the reasoning tells that you should use some terminal commands to execute some bash code, take bash action. Then you need to faithfully based on [Reasoning Thought], compose the terminal command and corresponding command parameters to propose specific command to solve the task. You should print like this:

Act: bash

```bash
# put your bash code here
```

2. Finish Action
If the [Reasoning Thought] think the task is completed, for example, it tells 'The task is completed'. That means you have finished the task, no more action is needed, just take finish action, thus you should print like this:

Act: finish



You need to faithfully based on the [Reasoning Thought], according to the [Action Options and Action Output Instruction] to choose either Bash Action or Finish Action. Then you need to propose the action failthfully based on the [Reasoning Thought] and make sure the action satisfies the action output instruction.
Now, your action is:"

    
\end{lstlisting}

Git Tasks


\begin{lstlisting}[language=Python]

You are a git agent to complete a git task. As you know, if we consider every commit in git as a child node of the parent commit, the git tree is in a tree structure. You can interact with the git tree through a terminal by git commands.
To solve the task, you must understand the [Reasoning Thought] and faithfully follow the guidance of [Reasoning Thought] to propose specific git commands to proceed the task.

[Reasoning Thought]
The current state of git tree is: {'branches': {'main': {'target': 'C1', 'id': 'main', 'remoteTrackingBranchID': 'o/main'}, 'o/main': {'target': 'C1', 'id': 'o/main', 'remoteTrackingBranchID': None}, 'side1': {'target': 'C2', 'id': 'side1', 'remoteTrackingBranchID': None}, 'side2': {'target': 'C4', 'id': 'side2', 'remoteTrackingBranchID': None}, 'side3': {'target': 'C7', 'id': 'side3', 'remoteTrackingBranchID': None}}, 'commits': {'C0': {'parents': [], 'id': 'C0', 'rootCommit': True}, 'C1': {'parents': ['C0'], 'id': 'C1'}, 'C2': {'parents': ['C1'], 'id': 'C2'}, 'C3': {'parents': ['C1'], 'id': 'C3'}, 'C4': {'parents': ['C3'], 'id': 'C4'}, 'C5': {'parents': ['C1'], 'id': 'C5'}, 'C6': {'parents': ['C5'], 'id': 'C6'}, 'C7': {'parents': ['C6'], 'id': 'C7'}}, 'tags': {}, 'HEAD': {'target': 'side3', 'id': 'HEAD'}, 'originTree': {'branches': {'main': {'target': 'C8', 'id': 'main', 'remoteTrackingBranchID': None}}, 'commits': {'C0': {'parents': [], 'id': 'C0', 'rootCommit': True}, 'C1': {'parents': ['C0'], 'id': 'C1'}, 'C8': {'parents': ['C1'], 'id': 'C8'}}, 'tags': {}, 'HEAD': {'target': 'main', 'id': 'HEAD'}}}.
Your analysis about the current state and how to solve the task is:
Given the current git tree state and the target git tree, we need to follow the proposed plan to achieve the desired result. We're at the beginning of the plan, so we'll start with the first subtask:

1. Fetch updates from origin to get C8:
   We need to fetch the latest changes from the remote repository to bring C8 into our local repository.

2. Checkout side1 (C2):
   We'll move to the side1 branch, which is currently at C2.

3. Merge o/main (C8) into side1 to create C9:
   After fetching, we'll merge the remote main branch (o/main) into side1 to create the new commit C9.

These steps will set the foundation for the subsequent merges and branch manipulations. The specific git commands to accomplish these subtasks are:

1. git fetch origin
2. git checkout side1
3. git merge o/main

These commands will update our local repository with the remote changes, switch to the side1 branch, and create the new merge commit C9, which combines the changes from side1 (C2) and o/main (C8)..


[Action Options and Action Output Instruction]
Most common git commands are available.
A special case is there's no need to use git add command for there's no concept of adding / staging files.

You should analyze which git command and the parameters of the git command from [Reasoning Thought].
Once you determine the git command, you should output the git command in the following format:
```command
# put your git command here
```

An example is like this:
```command
git commit
```

You need to faithfully based on the [Reasoning Thought], according to the [Action Options and Action Output Instruction] to generate the next git command.

Now, your action is:

    
\end{lstlisting}

