
Ubuntu Terminal Tasks

\begin{lstlisting}[language=Python]
You are an Operating System assistant who can interact with Ubuntu Terminal to complete Operating System Tasks. You can interact with the Ubuntu Operating system by terminal commands.

[Task Description]
The OS task you need to solve is:\nFind all '.txt' files in the 'docs' directory and change their permissions to read-only for all users..

[Terminal Working Directory]
The working directory of the Ubuntu Terminal is:
/.

[Proposed Plan]
An abstract plan on how to complete the task is: 
1. Navigate to the 'docs' directory
2. Find all '.txt' files in the directory and its subdirectories
3. Change the permissions of the found files to read-only for all users
4. Verify the changes.

[History Interaction Information]
Your past history interaction information is: 
[].

The abstract plan on how to complete the task is a guide to help you analyze the task and complete it efficiently. Based on the action history and the output of the Ubuntu System Terminal, think about which subtask of the plan you are processing currently.
If you think the task is completed, you can just output 'The task is completed' in your reasoning output.
Otherwise, based on the current stage, think how to use terminal commands to interact with the Ubuntu terminal to solve the task efficiently. You need to propose specific commands and corresponding command parameters of those commands.

After your reasoning about the task, you should summarize your reasoning (your summary must contain all key information) and output the summary result like
```reasoning
Put your reasoning summary here
```
Your thinking and your reasoning are:
    
\end{lstlisting}



Git Tasks


\begin{lstlisting}[language=Python]
You are a git agent to complete a git task. As you know, if we consider every commit in git as a child node of the parent commit, the git tree is in a tree structure. You can interact with the git tree through a terminal by git commands.

[Task Description]
The whole git task you need to solve is to change the init git tree to the target git tree, while after your past interaction, the git tree is currently in the state of current git tree.
The init git tree is:
{'branches': {'main': {'target': 'C1', 'id': 'main', 'remoteTrackingBranchID': 'o/main'}, 'o/main': {'target': 'C1', 'id': 'o/main', 'remoteTrackingBranchID': None}, 'side1': {'target': 'C2', 'id': 'side1', 'remoteTrackingBranchID': None}, 'side2': {'target': 'C4', 'id': 'side2', 'remoteTrackingBranchID': None}, 'side3': {'target': 'C7', 'id': 'side3', 'remoteTrackingBranchID': None}}, 'commits': {'C0': {'parents': [], 'id': 'C0', 'rootCommit': True}, 'C1': {'parents': ['C0'], 'id': 'C1'}, 'C2': {'parents': ['C1'], 'id': 'C2'}, 'C3': {'parents': ['C1'], 'id': 'C3'}, 'C4': {'parents': ['C3'], 'id': 'C4'}, 'C5': {'parents': ['C1'], 'id': 'C5'}, 'C6': {'parents': ['C5'], 'id': 'C6'}, 'C7': {'parents': ['C6'], 'id': 'C7'}}, 'tags': {}, 'HEAD': {'target': 'side3', 'id': 'HEAD'}, 'originTree': {'branches': {'main': {'target': 'C8', 'id': 'main', 'remoteTrackingBranchID': None}}, 'commits': {'C0': {'parents': [], 'id': 'C0', 'rootCommit': True}, 'C1': {'parents': ['C0'], 'id': 'C1'}, 'C8': {'parents': ['C1'], 'id': 'C8'}}, 'tags': {}, 'HEAD': {'target': 'main', 'id': 'HEAD'}}}.

The target git tree is:
{'branches': {'main': {'target': 'C11', 'id': 'main', 'remoteTrackingBranchID': 'o/main', 'localBranchesThatTrackThis': None}, 'o/main': {'target': 'C11', 'id': 'o/main', 'remoteTrackingBranchID': None, 'localBranchesThatTrackThis': ['main']}, 'side1': {'target': 'C2', 'id': 'side1', 'remoteTrackingBranchID': None, 'localBranchesThatTrackThis': None}, 'side2': {'target': 'C4', 'id': 'side2', 'remoteTrackingBranchID': None, 'localBranchesThatTrackThis': None}, 'side3': {'target': 'C7', 'id': 'side3', 'remoteTrackingBranchID': None, 'localBranchesThatTrackThis': None}}, 'commits': {'C0': {'parents': [], 'id': 'C0', 'rootCommit': True}, 'C1': {'parents': ['C0'], 'id': 'C1'}, 'C2': {'parents': ['C1'], 'id': 'C2'}, 'C3': {'parents': ['C1'], 'id': 'C3'}, 'C4': {'parents': ['C3'], 'id': 'C4'}, 'C5': {'parents': ['C1'], 'id': 'C5'}, 'C6': {'parents': ['C5'], 'id': 'C6'}, 'C7': {'parents': ['C6'], 'id': 'C7'}, 'C8': {'parents': ['C1'], 'id': 'C8'}, 'C9': {'parents': ['C2', 'C8'], 'id': 'C9'}, 'C10': {'parents': ['C4', 'C9'], 'id': 'C10'}, 'C11': {'parents': ['C10', 'C7'], 'id': 'C11'}}, 'HEAD': {'target': 'main', 'id': 'HEAD'}, 'originTree': {'branches': {'main': {'target': 'C11', 'id': 'main', 'remoteTrackingBranchID': None, 'localBranchesThatTrackThis': None}}, 'commits': {'C0': {'parents': [], 'id': 'C0', 'rootCommit': True}, 'C1': {'parents': ['C0'], 'id': 'C1'}, 'C8': {'parents': ['C1'], 'id': 'C8'}, 'C5': {'parents': ['C1'], 'id': 'C5'}, 'C3': {'parents': ['C1'], 'id': 'C3'}, 'C2': {'parents': ['C1'], 'id': 'C2'}, 'C6': {'parents': ['C5'], 'id': 'C6'}, 'C4': {'parents': ['C3'], 'id': 'C4'}, 'C9': {'parents': ['C2', 'C8'], 'id': 'C9'}, 'C7': {'parents': ['C6'], 'id': 'C7'}, 'C10': {'parents': ['C4', 'C9'], 'id': 'C10'}, 'C11': {'parents': ['C10', 'C7'], 'id': 'C11'}}, 'HEAD': {'target': 'main', 'id': 'HEAD'}}}.

The current git tree is:
{'branches': {'main': {'target': 'C1', 'id': 'main', 'remoteTrackingBranchID': 'o/main'}, 'o/main': {'target': 'C1', 'id': 'o/main', 'remoteTrackingBranchID': None}, 'side1': {'target': 'C2', 'id': 'side1', 'remoteTrackingBranchID': None}, 'side2': {'target': 'C4', 'id': 'side2', 'remoteTrackingBranchID': None}, 'side3': {'target': 'C7', 'id': 'side3', 'remoteTrackingBranchID': None}}, 'commits': {'C0': {'parents': [], 'id': 'C0', 'rootCommit': True}, 'C1': {'parents': ['C0'], 'id': 'C1'}, 'C2': {'parents': ['C1'], 'id': 'C2'}, 'C3': {'parents': ['C1'], 'id': 'C3'}, 'C4': {'parents': ['C3'], 'id': 'C4'}, 'C5': {'parents': ['C1'], 'id': 'C5'}, 'C6': {'parents': ['C5'], 'id': 'C6'}, 'C7': {'parents': ['C6'], 'id': 'C7'}}, 'tags': {}, 'HEAD': {'target': 'side3', 'id': 'HEAD'}, 'originTree': {'branches': {'main': {'target': 'C8', 'id': 'main', 'remoteTrackingBranchID': None}}, 'commits': {'C0': {'parents': [], 'id': 'C0', 'rootCommit': True}, 'C1': {'parents': ['C0'], 'id': 'C1'}, 'C8': {'parents': ['C1'], 'id': 'C8'}}, 'tags': {}, 'HEAD': {'target': 'main', 'id': 'HEAD'}}}.

[Proposed Plan]
An abstract plan on how to complete the git task is:
1. Fetch updates from origin to get C8
2. Checkout side1 (C2)
3. Merge o/main (C8) into side1 to create C9
4. Checkout side2 (C4)
5. Merge the branch containing C9 to create C10
6. Checkout side3 (C7)
7. Merge the branch containing C10 to create C11
8. Checkout main
9. Reset main to C11
10. Push main to origin to update remote
11. Fetch from origin to update o/main.

[History Interaction Information]
Your past history interaction information with the git tree is:
[].

The proposed plan on how to complete the task is a guide to help you analyze the task and complete it efficiently. Based on the history interaction information and the current git tree state, think about which subtask of the plan you are processing currently.

Based on the current stage, think how to use git commands to change the current git tree to the target git tree efficiently. You need to propose specific git commands to complete the subtask.


After your reasoning about the task, you should summarize your reasoning(your summary must contain all key information) and output the summary result like
```reasoning
Put your reasoning summary here
```

Your thinking and your reasoning are:
    
\end{lstlisting}
