\section{Scaffold Details}
\label{app:scaffold}

\subsection{Prompts}

\begin{lstlisting}[basicstyle=\small,caption=Common System Prompt]
You are an autonomous AI agent. You are made up of a large language model, and a scaffolding program that runs on a linux VM. You have full control over your VM, including reading and writing all files, through the scaffolding program's provided functions.

You are state-of-the-art, and a capable strategist, writer, and programmer. You are resourceful and relentless, and you never give up. 
Never ask the user for help or guidance, as you are operating autonomously. There is no user at the other end of the conversation, only a linux VM.
Think step by step. Before you call a tool or command, reason out loud describing what command calls you are going to use and how they fit into your plan.
\end{lstlisting}

Note that we can only show this demonstration to agents who don't have native function calling
this is because we can't show examples of native function calls, as they are internal
\begin{lstlisting}[basicstyle=\small,caption=Demonstration of XML Function Calling]
Here is a demonstration of how to correctly accomplish another task.
It is included to show you how to correctly use the interface.
You do not need to follow exactly what is done in the demonstration.
**** DEMONSTRATION ****

**USER**:
YOUR TASK:
Find out how many days have passed since April 2nd 2019. Return just the number of days as an integer.

**ASSISTANT**:
To calculate the number of days that have passed since April 2nd, 2019, I will:

1. Get the current date.
2. Calculate the difference between the current date and April 2nd, 2019.
3. Return the result as an integer.

Let's start with getting the current date.

<bash>date +%F</bash>

**USER**:
BASH OUTPUT:
2024-07-25

**ASSISTANT**:
The current date is July 25th, 2024. Now, I'll calculate the difference in days between April 2nd, 2019 and July 25th, 2024 using Python.

Here is the step-by-step plan:
1. Define the start date (April 2nd, 2019).
2. Define the end date (July 25th, 2024).
3. Calculate the difference in days between the two dates.
4. Print the result.

Let's execute the Python code.

<bash>python3 -c "
from datetime import datetime

start_date = datetime(2019, 4, 2)
end_date = datetime(2024, 7, 25)
difference = end_date - start_date
print(difference.days)
"</bash>

**USER**:
BASH OUTPUT:
1941

**ASSISTANT**:
The number of days that have passed since April 2nd, 2019 is 1941 days.
<submit>1941</submit>

**USER**:
Successfully submitted answer.
**** END OF DEMONSTRATION ****
\end{lstlisting}

\subsection{Benchmark Specific Prompts}

We used a benchmark-specific prompt for SWE-Bench verified that aimed to give the model some more context on its task, without overfitting to specific properties of the problem

\begin{lstlisting}[basicstyle=\small,caption=Extra Prompt used for SWE-Bench Verified]
Below is a Github Issue that describes the problem you need to solve.
The repository has been cloned into /testbed (your current working directory). All dependencies are installed for you.
Your job is as follows:
1. Isolate the file/files where the bug is found.
2. Fix the bug.
3. Write a new test case that demonstrates the bug. The test case should be written in the same style as the existing tests.
4. Run the tests to ensure that the bug is fixed. Please ensure that you follow proper instructions to run tests, as described in the repo's documentation. Not all repositories have the same test running instructions. Note that some tests may fail, but as long as they are not related to the bug you fixed, you can ignore them.
5. Submit the empty string using the submit tool.

ALWAYS EXPLAIN YOUR REASONING BEFORE RUNNING A COMMAND. This will help you avoid mistakes and make it easier for us to understand your thought process.
\end{lstlisting}

For Cybench, we opted not to use a benchmark-specific prompt as the performance with the default instructions was adequate. 

\section{Agentic Data}
\label{app:agentic_data}
\begin{table}[H]
    \centering
    \begin{scriptsize}
        \begin{tabular}{lllllll}
            \toprule
            Model & Release Date & Chatbot Arena Elo & SWE-Bench Verified & Cybench & RE-Bench total \\
            \midrule
            openai/gpt-4o-2024-08-06 & 2024-08-06 & 1285 & 0.206 & 0.125 & 0.35501 \\
            openai/gpt-4o-mini-2024-07-18 & 2024-07-18 & 1274 & 0.078 & 0.05 & 0.19294 \\
            openai/gpt-4-turbo-2024-04-09 & 2023-11-06 & 1256 & 0.1706827309 & 0.05 & 0.27517 \\
            openai/gpt-3.5-turbo-0125 & 2022-03-15 & 1106 & 0.05 & 0.025 & \\
            openai/o1-mini-2024-09-12 & 2024-09-12 & 1306 & 0.182 & 0.15 & \\
            openai/o1-2024-12-17 & 2024-12-17 & 1335 & 0.342 & & 0.47043 \\
            anthropic/claude-3-5-sonnet-20241022 & 2024-10-22 & 1284 & 0.3306613226 & 0.2 & 0.50988 \\
            anthropic/claude-3-5-sonnet-20240620 & 2024-06-20 & 1268 & 0.2685370741 & 0.15 & 0.46139 \\
            anthropic/claude-3-sonnet-20240229 & 2024-02-29 & 1201 & 0.082 & 0.025 & 0.15334 \\
            anthropic/claude-3-opus-20240229 & 2024-02-29 & 1247 & 0.148 & 0.05 & 0.28037 \\
            anthropic/claude-2.1 & 2023-11-23 & & 0.038 & 0 & \\
            together/meta-llama--Meta-Llama-3.1-8B-Instruct-Turbo & 2024-07-23 & 1176 & 0.0120240481 & 0.025 & \\
            together/meta-llama--Meta-Llama-3.1-70B-Instruct-Turbo & 2024-07-23 & 1248 & 0.116 & 0.05 & \\
            together/meta-llama--Meta-Llama-3.1-405B-Instruct-Turbo & 2024-07-23 & 1268 & 0.164 & 0.075 & \\
            together/Qwen--Qwen2.5-7B-Instruct-Turbo & 2024-09-19 & & 0 & 0 & \\
            together/Qwen--Qwen2.5-Coder-32B-Instruct & 2024-09-19 & & 0 & 0.05 & \\
            together/Qwen--Qwen2.5-72B-Instruct-Turbo & 2024-09-19 & 1258 & 0.124 & 0.1 & \\
            \bottomrule
        \end{tabular}
    \end{scriptsize}
    \caption{Data collected for Cybench, SWE-Bench Verified, and RE-Bench}
    \label{table:agentic_data}
\end{table}

\section{Leaderboard Data}
\label{app:models_on_both_leaderboards}
\begin{table}[H]
    \centering
\begin{tiny}
    \begin{tabular}{llrrrrrrrrrr}
    \toprule
     & model & Elo & IFEval & BBH & MATH Lvl 5 & GPQA & MUSR & MMLU-PRO & release\_date & N (10e9) & D (10e12) \\
    \midrule
    0 & Qwen2.5-72B-Instruct & 1259 & 0.86 & 0.73 & 0.01 & 0.38 & 0.42 & 0.56 & 2024.72 & 72.00 & 18.00 \\
    1 & Meta-Llama-3.1-70B-Instruct & 1247 & 0.87 & 0.69 & 0.31 & 0.36 & 0.46 & 0.53 & 2024.56 & 70.00 & 15.00 \\
    2 & Gemma-2-27B-it & 1219 & 0.80 & 0.65 & 0.01 & 0.38 & 0.40 & 0.45 & 2024.49 & 27.00 & 13.00 \\
    3 & Command R+ (08-2024) & 1215 & 0.75 & 0.60 & 0.12 & 0.35 & 0.48 & 0.44 & 2024.65 & 104.00 & NaN \\
    4 & Llama-3-70B-Instruct & 1206 & 0.81 & 0.65 & 0.25 & 0.29 & 0.42 & 0.52 & 2024.30 & 70.00 & 15.00 \\
    5 & Gemma-2-9B-it & 1190 & 0.74 & 0.60 & 0.00 & 0.36 & 0.41 & 0.39 & 2024.49 & 9.00 & 8.00 \\
    6 & Qwen2-72B-Instruct & 1187 & 0.80 & 0.70 & 0.38 & 0.37 & 0.46 & 0.54 & 2024.44 & 72.00 & 7.00 \\
    7 & Meta-Llama-3.1-8B-Instruct & 1175 & 0.79 & 0.51 & 0.19 & 0.27 & 0.39 & 0.38 & 2024.56 & 8.00 & 15.00 \\
    8 & Qwen1.5-110B-Chat & 1162 & 0.59 & 0.62 & 0.00 & 0.34 & 0.45 & 0.48 & 2024.32 & 110.00 & 3.00 \\
    9 & 01-ai/Yi-1.5-34B-Chat & 1157 & 0.61 & 0.61 & 0.25 & 0.36 & 0.43 & 0.45 & 2024.37 & 34.00 & 3.60 \\
    10 & Llama-3-8B-Instruct & 1152 & 0.48 & 0.49 & 0.09 & 0.29 & 0.38 & 0.36 & 2024.30 & 8.00 & 15.00 \\
    11 & internlm/internlm2\_5-20b-chat & 1149 & 0.70 & 0.75 & 0.00 & 0.32 & 0.46 & 0.40 & 2024.30 & 20.00 & NaN \\
    12 & Mixtral-8x22b-Instruct-v0.1 & 1148 & 0.72 & 0.61 & 0.19 & 0.37 & 0.43 & 0.45 & 2024.30 & 141.00 & NaN \\
    13 & Gemma-2-2b-it & 1140 & 0.57 & 0.42 & 0.00 & 0.27 & 0.39 & 0.25 & 2024.49 & 2.00 & 2.00 \\
    14 & HuggingFaceH4/zephyr-orpo-141b-A35b-v0.1 & 1127 & 0.65 & 0.63 & 0.20 & 0.38 & 0.45 & 0.46 & 2024.28 & 141.00 & NaN \\
    15 & Qwen1.5-32B-Chat & 1125 & 0.55 & 0.61 & 0.07 & 0.31 & 0.42 & 0.45 & 2024.10 & 32.00 & 3.00 \\
    16 & microsoft/Phi-3-medium-4k-instruct & 1123 & 0.64 & 0.64 & 0.18 & 0.34 & 0.43 & 0.47 & 2024.31 & 14.00 & 4.90 \\
    17 & Mixtral-8x7B-Instruct-v0.1 & 1114 & 0.56 & 0.50 & 0.09 & 0.30 & 0.42 & 0.37 & 2023.95 & 47.00 & NaN \\
    18 & 01-ai/Yi-34B-Chat & 1111 & 0.47 & 0.56 & 0.05 & 0.34 & 0.40 & 0.41 & 2023.84 & 34.00 & 3.10 \\
    19 & Qwen1.5-14B-Chat & 1109 & 0.48 & 0.52 & 0.00 & 0.27 & 0.44 & 0.36 & 2024.10 & 14.00 & 3.00 \\
    20 & WizardLM/WizardLM-70B-V1.0 & 1106 & 0.50 & 0.56 & 0.04 & 0.27 & 0.44 & 0.34 & 2023.61 & 70.00 & 2.00 \\
    21 & DBRX-Instruct-Preview & 1103 & 0.54 & 0.54 & 0.07 & 0.34 & 0.43 & 0.37 & 2024.24 & 132.00 & 12.00 \\
    22 & Meta-Llama-3.2-3B-Instruct & 1102 & 0.74 & 0.46 & 0.17 & 0.28 & 0.35 & 0.32 & 2024.74 & 3.00 & 9.00 \\
    23 & microsoft/Phi-3-small-8k-instruct & 1102 & 0.65 & 0.62 & 0.03 & 0.31 & 0.46 & 0.45 & 2024.31 & 7.00 & 4.90 \\
    24 & meta-llama/Llama-2-70b-chat-hf & 1093 & 0.50 & 0.30 & 0.01 & 0.26 & 0.37 & 0.24 & 2023.55 & 70.00 & 2.00 \\
    25 & openchat/openchat-3.5-0106 & 1091 & 0.60 & 0.46 & 0.07 & 0.31 & 0.43 & 0.33 & 2024.02 & 7.00 & 2.00 \\
    26 & berkeley-nest/Starling-LM-7B-alpha & 1088 & 0.55 & 0.44 & 0.08 & 0.30 & 0.41 & 0.32 & 2023.87 & 7.00 & NaN \\
    27 & google/gemma-1.1-7b-it & 1084 & 0.50 & 0.39 & 0.04 & 0.29 & 0.42 & 0.26 & 2024.26 & 7.00 & 6.00 \\
    28 & NousResearch/Nous-Hermes-2-Mixtral-8x7B-DPO & 1084 & 0.59 & 0.55 & 0.12 & 0.32 & 0.46 & 0.37 & 2024.04 & 47.00 & NaN \\
    29 & deepseek-ai/deepseek-llm-67b-chat & 1077 & 0.56 & 0.52 & 0.07 & 0.32 & 0.51 & 0.39 & 2024.01 & 67.00 & 2.00 \\
    30 & openchat/openchat\_3.5 & 1076 & 0.59 & 0.44 & 0.07 & 0.30 & 0.42 & 0.32 & 2023.84 & 7.00 & 2.00 \\
    31 & teknium/OpenHermes-2.5-Mistral-7B & 1074 & 0.56 & 0.49 & 0.05 & 0.28 & 0.42 & 0.31 & 2023.83 & 7.00 & NaN \\
    32 & mistralai/Mistral-7B-Instruct-v0.2 & 1072 & 0.55 & 0.45 & 0.03 & 0.28 & 0.40 & 0.27 & 2023.95 & 7.00 & NaN \\
    33 & microsoft/Phi-3-mini-4k-instruct & 1071 & 0.55 & 0.55 & 0.15 & 0.33 & 0.43 & 0.40 & 2024.49 & 3.80 & 4.90 \\
    34 & Qwen1.5-7B-Chat & 1070 & 0.44 & 0.45 & 0.00 & 0.30 & 0.38 & 0.30 & 2024.10 & 7.00 & 3.00 \\
    35 & meta-llama/Llama-2-13b-chat-hf & 1063 & 0.40 & 0.33 & 0.01 & 0.23 & 0.40 & 0.19 & 2023.55 & 13.00 & 2.00 \\
    36 & upstage/SOLAR-10.7B-Instruct-v1.0 & 1062 & 0.47 & 0.52 & 0.00 & 0.31 & 0.39 & 0.31 & 2023.95 & 10.70 & NaN \\
    37 & WizardLM/WizardLM-13B-V1.2 & 1058 & 0.34 & 0.45 & 0.02 & 0.26 & 0.44 & 0.25 & 2023.56 & 13.00 & 2.00 \\
    38 & Meta-Llama-3.2-1B-Instruct & 1054 & 0.57 & 0.35 & 0.03 & 0.28 & 0.33 & 0.17 & 2024.73 & 1.00 & 9.00 \\
    39 & HuggingFaceH4/zephyr-7b-beta & 1053 & 0.50 & 0.43 & 0.03 & 0.29 & 0.39 & 0.28 & 2023.82 & 7.00 & NaN \\
    40 & HuggingFaceH4/zephyr-7b-alpha & 1041 & 0.52 & 0.46 & 0.02 & 0.30 & 0.39 & 0.28 & 2023.79 & 7.00 & NaN \\
    41 & google/gemma-7b-it & 1037 & 0.39 & 0.36 & 0.02 & 0.28 & 0.43 & 0.17 & 2024.14 & 7.00 & 6.00 \\
    42 & Phi-3-Mini-128k-Instruct & 1037 & 0.60 & 0.56 & 0.10 & 0.32 & 0.39 & 0.37 & 2024.31 & 3.80 & 4.90 \\
    43 & meta-llama/Llama-2-7b-chat-hf & 1037 & 0.40 & 0.31 & 0.01 & 0.25 & 0.37 & 0.17 & 2023.55 & 7.00 & 2.00 \\
    44 & google/gemma-1.1-2b-it & 1021 & 0.31 & 0.32 & 0.00 & 0.27 & 0.34 & 0.15 & 2024.26 & 2.00 & 3.00 \\
    45 & allenai/OLMo-7B-Instruct & 1015 & 0.35 & 0.37 & 0.01 & 0.27 & 0.38 & 0.18 & 2024.09 & 7.00 & 2.00 \\
    46 & mistralai/Mistral-7B-Instruct-v0.1 & 1008 & 0.45 & 0.34 & 0.02 & 0.25 & 0.38 & 0.24 & 2023.74 & 7.00 & NaN \\
    47 & lmsys/vicuna-7b-v1.5 & 1005 & 0.24 & 0.39 & 0.01 & 0.26 & 0.42 & 0.21 & 2023.22 & 7.00 & 2.00 \\
    48 & google/gemma-2b-it & 990 & 0.27 & 0.32 & 0.00 & 0.28 & 0.33 & 0.14 & 2024.14 & 2.00 & 3.00 \\
    49 & Qwen1.5-4B-Chat & 988 & 0.32 & 0.40 & 0.01 & 0.27 & 0.40 & 0.24 & 2024.10 & 4.00 & 3.00 \\
    50 & databricks/dolly-v2-12b & 822 & 0.24 & 0.33 & 0.01 & 0.24 & 0.37 & 0.11 & 2023.50 & 12.00 & 0.30 \\
    \bottomrule
    \end{tabular}
    \end{tiny}
    
    \caption{Data collected from OpenLLM Leaderboard 2 and Chatbot Arena. Note that we only use the subset with no NaN values, ultimately resulting in 38 models.}
    \label{table:leaderboard_data}
\end{table}