\documentclass{article} % For LaTeX2e
\usepackage{iclr2025_conference,times}

% Optional math commands from https://github.com/goodfeli/dlbook_notation.
%%%%% NEW MATH DEFINITIONS %%%%%

% \usepackage{amsmath,amsfonts,bm}
\usepackage{amsmath,amsfonts}

\usepackage{pifont}


\newcommand{\R}{\mathbb{R}}


\def\va{{\mathbf{a}}}
\def\vg{{\mathbf{g}}}

% Sets
\def\sR{\mathbb{R}}
\def\sC{\mathbb{C}}
\def\sZ{\mathbb{Z}}
\def\sN{\mathbb{N}}
\def\sQ{\mathbb{Q}}

\def\sS{\mathcal{S}}



% Vectors
\def\vzero{{\mathbf{0}}}
\def\vone{{\mathbf{1}}}
\def\vmu{{\mathbf{\mu}}}
\def\vtheta{{\mathbf{\theta}}}
\def\va{{\mathbf{a}}}
\def\vb{{\mathbf{b}}}
\def\vc{{\mathbf{c}}}
\def\vd{{\mathbf{d}}}
\def\ve{{\mathbf{e}}}
\def\vf{{\mathbf{f}}}
\def\vg{{\mathbf{g}}}
\def\vh{{\mathbf{h}}}
\def\vi{{\mathbf{i}}}
\def\vj{{\mathbf{j}}}
\def\vk{{\mathbf{k}}}
\def\vl{{\mathbf{l}}}
\def\vm{{\mathbf{m}}}
\def\vn{{\mathbf{n}}}
\def\vo{{\mathbf{o}}}
\def\vp{{\mathbf{p}}}
\def\vq{{\mathbf{q}}}
\def\vr{{\mathbf{r}}}
\def\vs{{\mathbf{s}}}
\def\vt{{\mathbf{t}}}
\def\vu{{\mathbf{u}}}
\def\vv{{\mathbf{v}}}
\def\vw{{\mathbf{w}}}
\def\vx{{\mathbf{x}}}
\def\vy{{\mathbf{y}}}
\def\vz{{\mathbf{z}}}
\def\vzeta{{\mathbf{\zeta}}}

% Matrix
\def\mA{{\mathbf{A}}}
\def\mB{{\mathbf{B}}}
\def\mC{{\mathbf{C}}}
\def\mD{{\mathbf{D}}}
\def\mE{{\mathbf{E}}}
\def\mF{{\mathbf{F}}}
\def\mG{{\mathbf{G}}}
\def\mH{{\mathbf{H}}}
\def\mI{{\mathbf{I}}}
\def\mJ{{\mathbf{J}}}
\def\mK{{\mathbf{K}}}
\def\mL{{\mathbf{L}}}
\def\mM{{\mathbf{M}}}
\def\mN{{\mathbf{N}}}
\def\mO{{\mathbf{O}}}
\def\mP{{\mathbf{P}}}
\def\mQ{{\mathbf{Q}}}
\def\mR{{\mathbf{R}}}
\def\mS{{\mathbf{S}}}
\def\mT{{\mathbf{T}}}
\def\mU{{\mathbf{U}}}
\def\mV{{\mathbf{V}}}
\def\mW{{\mathbf{W}}}
\def\mX{{\mathbf{X}}}
\def\mY{{\mathbf{Y}}}
\def\mZ{{\mathbf{Z}}}
\def\mBeta{{\mathbf{\beta}}}
\def\mPhi{{\mathbf{\Phi}}}
\def\mLambda{{\mathbf{\Lambda}}}
\def\mSigma{{\mathbf{\Sigma}}}


% Expectation
% \def\eE{\mathop{\mathbb{E}}\limits}
\def\eE{\mathbb{E}}

% Probability
\def\pP{\mathbb{P}}

% Tilde
\def\tf{\tilde{f}}
\def\tS{\tilde{S}}
\def\wtF{\widetilde{\mathcal{F}}}
\def\whR{\widehat{R}}
\def\tvx{\tilde{\mathbf{x}}}
\def\ty{\tilde{y}}


\def\defeq{\overset{\textup{def}}{=}}
% \def\defeq{\overset{.}{=}}
\def\defone{\overset{\text{\ding{172}}}{=}}
\def\deftwo{\overset{\text{\ding{173}}}{=}}
\def\leqone{\overset{\text{\ding{172}}}{\leq}}
\def\leqtwo{\overset{\text{\ding{173}}}{\leq}}
\def\leqthree{\overset{\text{\ding{174}}}{\leq}}
\def\leqfour{\overset{\text{\ding{175}}}{\leq}}
\def\eqone{\overset{\text{\ding{172}}}{=}}
\def\eqtwo{\overset{\text{\ding{173}}}{=}}
\def\eqthree{\overset{\text{\ding{174}}}{=}}
\def\eqfour{\overset{\text{\ding{175}}}{=}}
\def\geqfive{\overset{\text{\ding{176}}}{\geq}}

%\usepackage{hyperref}
\usepackage{url}
\usepackage{lipsum}
\usepackage{natbib}
\usepackage[colorlinks=true,citecolor=apolloblue,linkcolor=apolloblue,urlcolor=apolloblue]{hyperref}
% Recommended, but optional, packages for figures and better typesetting:
\usepackage{microtype}
\usepackage{graphicx}
\usepackage{booktabs} % for professional tables
\usepackage{longtable}
\usepackage{tabularx}
\usepackage{subcaption}
\usepackage{amsmath}
\usepackage{listings}
\usepackage{float}


% For theorems and such
\usepackage{amsmath}
\usepackage{amssymb}
\usepackage{mathtools}
\usepackage{amsthm}
\usepackage{algorithmic}


\lstset{
  basicstyle=\ttfamily,
  columns=fullflexible,
  frame=single,
  breaklines=true,
  postbreak=\mbox{\textcolor{red}{$\hookrightarrow$}\space},
}


\graphicspath{ {./figures/} }

\title{Forecasting Frontier Language Model Agent Capabilities}

% Authors must not appear in the submitted version. They should be hidden
% as long as the \iclrfinalcopy macro remains commented out below.
% Non-anonymous submissions will be rejected without review.

\author{% core contributors
Govind Pimpale\thanks{Equal contribution}\\\textmd{MATS}
\And
Axel H{\o}jmark\footnotemark[1]\\\textmd{MATS \& Apollo Research}
\AND
% senior authors
J\'er\'emy Scheurer\thanks{Equal contribution}\\\textmd{Apollo Research}
\And
Marius Hobbhahn\footnotemark[2]\\\textmd{Apollo Research}
}

% The \author macro works with any number of authors. There are two commands
% used to separate the names and addresses of multiple authors: \And and \AND.
%
% Using \And between authors leaves it to \LaTeX{} to determine where to break
% the lines. Using \AND forces a linebreak at that point. So, if \LaTeX{}
% puts 3 of 4 authors names on the first line, and the last on the second
% line, try using \AND instead of \And before the third author name.

\newcommand{\fix}{\marginpar{FIX}}
\newcommand{\new}{\marginpar{NEW}}

\iclrfinalcopy % Uncomment for camera-ready version, but NOT for submission.
\begin{document}


\maketitle

\vspace{0.3in} 
\begin{abstract}
    As Language Models (LMs) increasingly operate as autonomous agents, accurately forecasting their capabilities becomes crucial for societal preparedness. 
    We evaluate six forecasting methods that predict downstream capabilities of LM agents. 
    We use ``one-step'' approaches that predict benchmark scores from input metrics like compute or model release date directly or ``two-step'' approaches that first predict an intermediate metric like the principal component of cross-benchmark performance (PC-1) and human-evaluated competitive Elo ratings.
    We evaluate our forecasting methods by backtesting them on a dataset of 38 LMs from the OpenLLM 2 leaderboard. 
    We then use the validated two-step approach (Release Date$\to$Elo$\to$Benchmark) to predict LM agent performance for frontier models on three benchmarks:
    SWE-Bench Verified (software development), Cybench (cybersecurity assessment), and RE-Bench (ML research engineering). 
    Our forecast predicts that by the beginning of 2026, non-specialized LM agents with low capability elicitation will reach a success rate of 54\% on SWE-Bench Verified, while state-of-the-art LM agents will reach an 87\% success rate.
    Our approach does not account for recent advances in inference-compute scaling and might thus be too conservative.
\end{abstract}

\documentclass[../main.tex]{subfiles}
\graphicspath{{../images/}}
\makeatletter
\def\input@path{{../images/}}
\makeatother
\begin{document}
\section{Introduction}
\begin{figure}
\centering
\begin{tikzpicture}
\node[inner sep=0pt] (ws) at (0, 0) {
\includegraphics[height=.4\textwidth, trim={10cm 0 10cm 0},clip]{world_space.png}};
\node[inner sep=0pt] (cs) at (6,0) {\includegraphics[height=.4\textwidth, trim={10cm 1cm 10cm 4cm},clip]{conf_space.png}};
\end{tikzpicture}
\vspace{-5pt}
\label{fig:pbrm_intro}
\caption{\textbf{Left}: Shows world space obstacles as grey spheres. Robots start and goal configuration is colored red and green, respectively. Configurations along the computed path are colored transparent blue. \textbf{Right:} Mapped world space scenario to configuration space. Obstacle region is the grey mesh. Red spheres are collision-free regions computed by the neural SCDF. The optimized shortest path in the convex corridor is the blue curve.}
\vspace{-25pt}
\end{figure}
Motion planning is the problem of finding a collision-free trajectory that connects a given start and goal configuration. The planning takes place in the configuration space of the robot. For single body robots, like mobile robots or drones, the configuration space and the world space are usually the same. This simplifies the planning, since explicit obstacle representations are available which enables geometrical tools like separating hyperplanes, smallest distance to obstacles etc., to be used when designing motion planning algorithms. For multi-body robots like manipulators, the situation is completely different. The world space obstacles are usually mapped to non-convex regions, and to make the problem even harder, the mapping is usually not known. Forming explicit representations of the obstacle region in the configuration space is usually too expensive or intractable. Despite all of this, sampling based planners are used with great success, which mainly is due to their use of implicit representations of the obstacle region. The basic idea is to construct a graph in the configuration space that covers and connects the collision-free region. From this graph, a path can be extracted that connects a given start and goal configuration. The approach is computationally expensive, since the graph is constructed with the smallest geometrical building block available, points, which represents a collision-check. Furthermore, the extracted paths from the graph are non-smooth and jagged due to the stochastic nature of the approach. This adds an additional post-processing step to the process, where the paths are shortcutted and smoothened, before the path can be used for tracking. Clearly a lot of time is invested to form this graph and produce smooth paths. Thus, if the obstacles start to move, then all of this work is done in no use, since all points that make up this graph need to be re-verified, which is simply too time consuming to be done in real time.
\\\\
In this work, we want to address the existing drawbacks of the sampling based planners. Our main contribution is an improved motion planner where each vertex in the graph covers a collision-free region in the form of a sphere instead of a point and where the edges are formed with neighboring intersecting spheres. This representation has the advantage of instead of returning piecewise linear paths, returning a sequence of overlapping spheres, i.e. a convex corridor, that connects a given start and goal configuration, illustrated in Figure \ref{fig:pbrm_intro}. This convex corridor allows us to use convex optimization to produce smooth trajectories, instead of computationally expensive post-processing methods. The representation further allows us to estimate the coverage of the collision-free space, which gives us awareness and feedback in the offline roadmap construction phase. Finally, our representation is simple to adapt to moving obstacles, simply requery for the new radii and recheck for intersections. 
\\\\
The spherical collision-free regions are formed using a signed distance function (SDF), which is a function that returns the smallest distance from an arbitrary point to the boundary of an obstacle. As the name implies, the distance is signed, thus if the point is inside the obstacle it is negative otherwise positive. If the distance is positive, a sphere with radius equal to the distance is guaranteed to cover a collision-free region. Using an SDF in motion planning is not new, but what is novel about our approach is that we express the distance in the configuration space instead of the world space and by doing so allows us to form these convex collision-free regions. We refer to the resulting SDF as a signed configuration distance function (SCDF). Computing an SCDF analytically is non-trivial, our approach is therefore to parameterize the SCDF with a deep neural network and learn the mapping by supervised learning. Our resulting neural SCDF can compute distances for different parameter values of obstacle shapes and we also show how multiple distances can be combined, thus making our approach flexible.
\section{Related work}
Motion planning algorithms can roughly be divided into three families, grid-based, sampling based and optimization based methods. Grid-based methods (GBM) discretize the planning space from which a graph is then compiled. A standard search method is A$^\star$ \citep{a_star}, which is classified as an \textit{informed} search method, since it employs a heuristic function to speed up the search. A$^\star$ guarantees to return an optimal path at the level of discretization used. GBMs usually discretize the planning space by a regular lattice and this limits the GBMs to problems with low dimensionality due to the curse of dimensionality. Thus, GBMs are usually limited to single-body robots where the degrees of freedom (DOF) are low. To overcome the inherent scaling problem with the GBMs, stochastic methods are usually used for multi-body robots. These methods are termed as sampling-based methods (SBM) and core members within this family are the rapidly-exploring random trees (RRT) \citep{rrt} and the probabilistic roadmap (PRM) \citep{prm}. RRT grows a tree from the start configuration and explores the collision-free region in a rapid way until it is able to connect to the goal region. RRT is usually improved by bi-directional planning \citep{rrt_connect}, i.e. an additional tree is grown from the goal configuration and the trees are tested for connection after any tree has been expanded. RRT is a single-query method, thus it searches for a path from scratch each time it is queried. Contrary to this, PRM is a multi-query method, which solves for multiple queries without starting from scratch. PRM does this by creating a roadmap (graph) that covers the collision-free space as an offline step. The graph is then used to solve for multiple queries. PRMs are used in cases where the environment does not change since the extra offline step is too computationally costly and needs to be re-done if the environment is changed. In our work, we address this inherent issue by using a different roadmap representation. Our vertices in the graph cover a collision-free region in the form of spheres and we form the edges by checking for intersecting spheres. If something in the environment changes, we recompute the spheres radii and recheck the intersections, without relying on collision detection. We use a trained neural network to compute the sphere radius, therefore querying for the radius can be done fast, hence our representation enables the PRM for dynamic environments.
\\\\
In the recent decades, optimization based methods (OBM) \citep{chomp, schulman, itomp, stomp} have been introduced as an alternative to SBM for multi-body robots. Like the SBM, the OBMs scale well to higher dimensional problems and produce smoother motion. It is common to use a SDF in the optimization since it is a smooth function, thus enabling gradient-based methods. However, the standard way of expressing the SDF is in world space. The distance therefore needs to be mapped to the configuration space by the forward kinematics. This mapping makes the optimization problem a non-linear program (NLP), which is computationally expensive to solve. Recently, a different approach has been proposed. In \cite{mp_gcs} motion planning is formulated as a convex optimization problem by using the graph of convex sets framework \citep{gcs}. The underlying idea is to decompose the collision-free space into intersecting convex sets from which a convex optimization problem is formulated. In cases where an explicit representation of the obstacles in the configuration space exists, like for single-body robots, creating collision-free convex regions can be done fast \citep{iris}. For multi-body robots, this is non-trivial. Existing work does this successfully \citep{iris_nlp, iris_c} by an optimization based approach, but the methods are still too time consuming to be used in the presence of moving obstacles. Our approach is instead to use deep learning to learn an SDF expressed in the configuration space. With this, we can query for shortest distances to the collision boundary, which allows us to expand spherical regions which are collision-free. Our approach is fast and therefore enables our suggested roadmap planner to be used in dynamic environments.
\\\\
Recent research has focused on learning collision detection \citep{fk_kernel_distance, diffco, graphdistnet} by predicting the signed distance between the robot links and the surrounding obstacles in the world space. The learned SDF is used in trajectory optimization but since the distance is expressed in the world space, the problem becomes an NLP and therefore takes a long time to solve. We take a novel approach and suggest to instead express the signed distance in the configuration space. This allows us to improve the PRM at the same time as it enables convex optimization for trajectory optimization, which runs faster and is more reliable than NLP solvers. In \cite{cspf} a learned signed distance function in the configuration space is proposed similar to our approach. However, their approach is restricted to point cloud representations, while we propose to represent the obstacles as parameterized geometric shapes, e.g. spheres. Furthermore, we also show how to use our learned SCDF to improve an existing roadmap planner.
\section{Problem formulation}
A robot is located in the world space, $\W \subset \R^3 $. The unique location of the robot is given by its configuration $\q \in \C$, where $\C$ is the configuration space. The set of points covered by the robots bodies at a certain configuration is expressed as $\B(\q) \subset \W$. The robot is surrounded by $\NrObst$ obstacles $\O = \bigcup_{i=1}^{\NrObst} \O_i$, where  $\O_i \subset \W$. The representation of the obstacle in the configuration space is the set $\C\O_i = \{\q \in \C \: |\: \B(\q) \cap \O_i \neq \emptyset \}$. The obstacle space is formed as $\Co = \bigcup_{i=1}^{\NrObst} \C \O_i$. The complement is referred to as the free space, $\Cf = \C \setminus \Co$. The path planning problem is a tuple, ($\Cf$, $\qStart$, $\qGoal$), where we want to connect a query pair, consisting of a start, $\qStart$, and goal configuration, $\qGoal$, with a geometric path, $\q(s): [0, 1] \mapsto \Cf$, such that $\q(0)=\qStart$ and $\q(1)=\qGoal$, or report correctly when such a path does not exist.
\end{document}


\section{Method}

\subsection{Overview \& Setup}

Our framework consists of a large, highly capable model \textbf{\bigmodel} and a smaller, resource-efficient model \textbf{\smallmodel}. We assume that $S \in \mathbb{N}$ and $L \in \mathbb{N}$ represent the parameter count of each model with $S \ll L$. Both models can either function as classifiers (i.e., $\mathcal{M}: \mathbb{R}^D \rightarrow [C]$ with $\mathbb{R}^D$ denoting the input space and $C$ the number of total classes), or (multi-modal) sequence models (i.e., $\mathcal{M}: \mathbb{R}^D \rightarrow [V]^{T}$ where $V$ is the vocabulary and $T$ is the sequence length). We include experiments on all of these model classes in Section~\ref{sec:experiments}. Furthermore, we do not require a shared model family to be deployed on both \smallmodel and \bigmodel; for example, \smallmodel could be a custom convolutional neural network optimized for efficient inference and \bigmodel a vision transformer~\citep{dosovitskiy2020image}. The primary objective is to design a deferral mechanism that enables \smallmodel to decide when to return its predictions without the assistance of \bigmodel and when to instead defer to it.

\looseness=-1
Deferral decisions are made using signals derived from the small model \smallmodel as this approach is typically more cost-effective than employing a separate routing mechanism~\citep{teerapittayanon2016branchynet}. Approaches that involve querying the large model \bigmodel to assist in making deferral decisions at test time are excluded from our setup. Such methods --- common in domains like LLMs --- are counterproductive to our goal since querying \bigmodel defeats the purpose of making a deferral decision in the first place?. Examples of these inapplicable methods include collaborative LLM frameworks~\citep{mielke2022reducing} and techniques that rely on semantic entropy for uncertainty estimation~\citep{kuhn2023semantic}. As part of our setup, we assume that \smallmodel is strictly less capable than \bigmodel --- a realistic scenario in practice supported by scaling laws~\citep{kaplan2020scaling}. Under this assumption, mistakes made by \bigmodel are also made by \smallmodel; however, \smallmodel may make additional errors that \bigmodel would avoid. This reflects the general observation that larger models tend to outperform smaller models across a wide range of tasks.

As discussed in Section~\ref{sec:related-word}, the choice of deferral strategy often depends on the level of access available to \smallmodel. We assume white box access with full access to \smallmodel's internals. As such, deferral mechanisms can be directly integrated into the model's architecture and parameters. This involves fine-tuning \smallmodel to predict deferral decisions or to incorporate rejection mechanisms within its predictive process. Our work falls into this category as it proposes a new loss function to fine-tune \smallmodel. 

Our goal is to train a small model that can effectively distinguish between correct and incorrect predictions. While many past works have considered the question of whether it is possible to find proxy measures for prediction correctness, the central question we ask is:
\begin{center}
\textbf{Can we \emph{optimize} the small model \smallmodel to separate correct from incorrect predictions?}
\end{center}
\noindent We show that this is indeed achievable through a carefully designed fine-tuning stage that does not require any architectural modifications. This ensures that the ability to separate correct from incorrect decisions is integrated seamlessly into \smallmodel's existing structure.


\subsection{Confidence-Tuning for Deferral}

\begin{figure}
    \centering
    \resizebox{\linewidth}{!}{
    \begin{figure}[h]
\begin{center}
   \includegraphics[width=0.99\linewidth]{figs/pdf/loss.pdf}
\end{center}
   \caption{
    Training loss of VAR \textit{vs.} FlexVAR. FlexVAR demonstrates a faster convergence rate. We report the results with trained 40 epochs ($\sim$ 70K iterations). 
   }
\label{fig:loss}
\end{figure}

    }
    \vspace{-15pt}
    \caption{\textbf{Overview of \loss}: We want correctly predicted samples maintain their current prediction by ensuring that cross entropy is decreased (top, green). At the same time, we want incorrectly predicted samples to yield a uniform confidence across all classes, leading to a low overall confidence score (bottom, red).}
    \label{fig:opt_goal}
\end{figure}

\textbf{Stage 1: Standard Training.} We begin with a \smallmodel that has already been trained on the tasks it is intended to perform upon deployment. However, due to its limited capacity, \smallmodel cannot achieve the performance levels of \bigmodel. Importantly, we make no assumptions about the training process of \smallmodel—whether it was trained from scratch without supervision from an external model or through a distillation approach.

\sloppy
\textbf{Stage 2: Correctness-Aware Finetuning with \loss.} Next, we introduce a correctness-aware loss, dubbed \loss, to fine-tune \smallmodel for improved confidence calibration. Specifically, the model is trained to make correct predictions with high confidence while reducing the confidence of incorrect predictions (see Figure~\ref{fig:opt_goal}). This loss can either rely on true labels or utilize the outputs of \bigmodel with soft probabilities as targets. 


For a standard classification model, the calibration loss is defined as the following hybrid loss
\begin{align}
\mathcal{L} &= \alpha \mathcal{L}_\text{corr} + (1 - \alpha) \mathcal{L}_\text{incorr} \\
\mathcal{L}_\text{corr} &= \frac{1}{N} \sum_{i=1}^{N} \mathds{1}\{ y_i = \hat{y}_i \} \text{CE}(p_i(\mathbf{x}_i), y_i) \\
\mathcal{L}_\text{incorr} &= \frac{1}{N} \sum_{i=1}^{N} \mathds{1}\{ y_i \neq \hat{y}_i \} \text{KL}\left(p_i(\mathbf{x}_i) \parallel \mathcal{U}\right)
\end{align}
where  \( y_i \) and \( \hat{y}_i \) are the true and predicted labels for $\mathbf{x}_i$, respectively, \( p_i \) is the predicted probability distribution of \smallmodel over classes, \( \mathcal{U} \) represents the uniform distribution over all classes, \( N \) denotes the number samples in the current batch, \( \alpha \in (0, 1) \) is a tunable hyperparameter controlling the emphasis between correct and incorrect predictions, and the cross-entropy function and KL divergence are defined as \( \text{CE}(p, y) = -\sum_{c} y_c \log p_c \) and \( \text{KL}(p \parallel q) = \sum_{c} p_c \log ( \frac{p_c}{q_c}) \), respectively. We note that a similar loss has previously been proposed in Outlier Exposure (OE)~\citep{hendrycks2018deep} for out-of-distribution (OOD) sample detection. Here, the goal is to make sure that OOD examples are assigned low confidence scores by tuning the confidence on a auxiliary outlier dataset. However, to the best of our knowledge, this idea has not previously been used to improve deferral performance of a smaller model in a cascading chain.

We emphasize that the trade-off parameter $\alpha$ plays a critical role as part of this optimization setup as it directly influences model utility and deferral performance. A lower value of \(\alpha\) emphasizes reducing confidence in incorrect predictions by pushing them closer to the uniform distribution, making the model more cautious in regions where it may make mistakes. Conversely, a higher value of \(\alpha\) encourages the model to increase its confidence on correct predictions, sharpening its decision boundaries and enhancing accuracy where it is already performing well. Thus, \(\alpha\) serves as a crucial hyperparameter that balances the trade-off between improving calibration by mitigating overconfidence in errors and reinforcing confidence in accurate classifications. By appropriately tuning \(\alpha\), practitioners can control the model’s behavior to achieve a desired balance between reliability in uncertain regions and decisiveness in confident predictions, tailored to the specific requirements of their application.

We further generalize this loss to token-based models (e.g., LMs and VLMs), formulated as
\ifarxiv
\small
\fi
\begin{align}
    \mathcal{L}_\text{corr} & = \frac{1}{N} \sum_{i=1}^{N} \sum_{t=1}^{T} \mathds{1}\{ y_{i,t} = \hat{y}_{i,t} \} \text{CE}(p_{i,t}(\mathbf{x}_i), y_{i,t}) \\
    \mathcal{L}_\text{incorr} & = \frac{1}{N} \sum_{i=1}^{N} \sum_{t=1}^{T} \mathds{1}\{ y_{i,t} \neq \hat{y}_{i,t} \} \text{KL}\left(p_{i,t}(\mathbf{x}_i) \parallel \mathcal{U}\right)
\end{align}
\normalsize
where \( y_{i,t} \) and \( \hat{y}_{i,t} \) denote the true and predicted tokens at position \( t \) for sample \( i \), \( p_{i,t} \) is the predicted token distribution at position \( t \) for sample \( i \), and \( T \) is the sequence length for the token-based model. The token-level loss ensures that correct token predictions are made confidently while incorrect tokens are assigned smaller confidences.

\sloppy
\textbf{Stage 3: Confidence Computation \& Thresholding.} After fine-tuning \smallmodel with \loss, we apply standard confidence- and entropy-based techniques for model uncertainty to obtain a deferral signal. We use the selective prediction framework to determine whether a query point~$\mathbf{x} \in \mathbb{R}^D$ should be accepted by \smallmodel or routed to \bigmodel. Selective prediction alters the model inference stage by introducing a deferral state through a \textit{gating mechanism}~\citep{yaniv2010riskcoveragecurve}. At its core, this mechanism relies on a deferral function $g:\mathbb{R}^D \rightarrow \mathbb{R}$ which determines if \smallmodel should output a prediction for a sample~$\mathbf{x}$ or defer to \bigmodel. Given a targeted acceptance threshold $\tau$, the resulting predictive model can be summarized as:
\begin{equation}
\label{eq:deferral}
    (\mathcal{M}_S,\mathcal{M}_L,g)(\mathbf{x}) = \begin{cases}
  \mathcal{M}_S(\mathbf{x})  & g(\mathbf{x}) \geq \tau \\
  \mathcal{M}_L(\mathbf{x}) & \text{otherwise.}
\end{cases}
\end{equation}

\emph{Classification Models (Max Softmax).} Let \(\mathcal{M}_S\) produce a categorical distribution
\(\{p(y=c \mid \mathbf{x})\}_{c=1}^C\) over \(C\) classes. 
Then we define the gating function as
\begin{align}
g_{\text{CL}}(\mathbf{x}) \;=\; \max_{1 \,\le\, c \,\le\, C}\;p\bigl(y = c \,\big\vert\, \mathbf{x}\bigr).
\end{align}

\emph{Token-based Models (Negative Predictive Entropy).} 
Let \(\mathcal{M}_S\) produce a sequence of categorical distributions 
\(\{p(y_t = c \mid \mathbf{x})\}_{c=1}^C\) for each token index \(t \in T\). Then we define the gating function as
\begin{equation}
\footnotesize
g_{\text{NENT}}(\mathbf{x}) 
= \; \frac{1}{T} \sum_{t=1}^{T} \sum_{c=1}^{C} 
    p\bigl(y_t = c \,\big\vert\, \mathbf{x}\bigr)\,\log p\bigl(y_t = c \,\big\vert\, \mathbf{x}\bigr),
\end{equation}
where \(y_t \in [C]\) is the predicted token at time step \(t\), \(p(y_t=c \mid \mathbf{x})\) is the (conditional) probability of token \(k\) at step \(t\), and \(T\) is the total number of token positions for the sequence. Across both model classes, higher values of either $g_{\text{CL}}$ or $g_{\text{NENT}}$ indicate higher confidence in the predicted class or sequence generation, respectively.

\section{Evaluating approaches through backtesting}

To compare all six of our approaches, we backtest them on existing data from the Open LLM Leaderboard v2 \citep{open-llm-leaderboard-v2} with six benchmarks: IFEval, BBH, MATH Lvl 5, GPQA, MUSR, and MMLU-PRO.
We only use the subset of Open LLM Leaderboard v2 that has Elo scores available, resulting in 38 models (see Appendix~\ref{app:models_on_both_leaderboards}).

However, before we determine which pathway is the most accurate overall, we want to compare the predictive power of input and intermediate variables.
%understand how predicting from our intermediate capability metrics compares to predicting from the input variables. 
In Section \ref{subsec:backtest_capability_metrics}, we backtest individual capability metrics to validate Elo as a potential candidate and compare it to PC-1, scaled log-FLOP, and release date.
Then, we move towards testing the entire pathway. In Section \ref{subsec:backtest_full_approaches}, we backtest all six full approaches.

\subsection{Backtesting capability metrics}
\label{subsec:backtest_capability_metrics}

To backtest capability metrics, we use expanding window cross-validation \cite{expanding_window_backtest} with 3 splits based on release date.
We first split our data up into 4 divisions with approximately equal model count.
Then, we train a statistical model only on the first split and evaluate it on the second split, another statistical model on the first and second split, and evaluate it on the third, and so on.
Our cross-validation methodology is displayed for predictions of MMLU-PRO with the full approach in Figure \ref{fig:path_backtest} (see also Section \ref{subsec:backtest_full_approaches}).

We are computing the error of only the capability metric, so we train just the sigmoid from the capability metric to the target benchmark. (Subplot 2 in Figure~\ref{fig:paths}).
For PC-1, we avoid testing on the training data and thus omit the benchmark we're predicting when fitting the PCA. Furthermore, we only use the data available up to that point when fitting the principal component vectors.  
To compare the overall performance of our four capability metrics, we compute the RMSE over our three splits and six benchmarks, for each approach.

\begin{table}[H]
    \begin{center}
        \begin{scriptsize}
            \setlength{\tabcolsep}{8.5pt}
            \begin{tabular}{lcccr}
                 \toprule
                 Capability Metric & PC-1 & Elo & log-FLOP & Release Date \\
                 \midrule
                 Test RMSE & 0.068 & 0.080 & 0.102 & 0.146 \\
                \bottomrule                     
            \end{tabular}
        \end{scriptsize}
    \end{center}
    \vskip -0.1in
    \caption{Average all-model test-split back-prediction RMSE for prediction of target benchmark from capability metrics. Intermediate metrics (PC-1 and Elo), outperform raw input variables (log-FLOP and Release Date).}
    \label{table:rmse_capability_metrics}
\end{table}

Table \ref{table:rmse_capability_metrics} displays the aggregated results. 
PC-1 performs best, followed by Elo, log-FLOP, and date as intermediate metrics. 
The full results of our capability metric backtesting can be found in Appendix~\ref{app:capability_metric_backtesting}.

\subsection{Backtesting full approaches}
\label{subsec:backtest_full_approaches}

\begin{figure*}[!htb]
    \centering
    \makebox[\columnwidth][c]{
    \includegraphics[width=1\textwidth]{path_backtesting.png}
    }
    \vskip -0.1in
    \caption{
        Visualization of backtesting forecasts for MMLU-PRO using the full method.
        %TODO: explain more.
        We split the data into 4 parts with an equal number of models. We then fit a full path on split 1 and test on split 2, fit on 1 \& 2, and predict on 3, and so forth. 
        \textbf{Top:} Comparing predicted to actual performance. Frontier models are marked with stars. 
        \textbf{Bottom:} Average RMSE over frontier models. Bars are colored by the split they predict.
    }
    \label{fig:path_backtest}
\end{figure*}

To backtest the full paths, we use the same expanding window cross-validation procedure. 
However, there are two important differences. 
First, we are testing the complete path from input variable to benchmark score, ignoring the internal loss of the S-curve or linear regression subcomponents.
Second, we only compute error for data points on the frontier. If there are no frontier points in a split, that split is ignored.
We then aggregate the error in each split as usual. Since there are far fewer data points, the error is likely to be noisier.

\begin{table}[h]
   \begin{center}
       \begin{scriptsize}
           \setlength{\tabcolsep}{10pt}
           \begin{tabular}{l|cr}
                \toprule
                \textbf{Intermediate Variable} & \multicolumn{2}{c}{\textbf{Input Variable}}\\
                \cmidrule(r){1-1} \cmidrule(l){2-3}
                & log-FLOP & Date  \\
                \midrule
                \emph{None} (One-Step) &  0.119 & 0.125 \\
                Elo & 0.197 & 0.095 \\
                PC-1 & 0.105 & 0.082 \\
               \bottomrule
           \end{tabular}
       \end{scriptsize}
   \end{center}
   \vskip -0.1in
   \caption{Average frontier model test-split back-prediction RMSE for full approach. The path Date$\to$PC-1$\to$Benchmark performs best, followed by Date$\to$Elo$\to$Benchmark.}
   \label{table:rmse_paths}
\end{table}

Our results (see Table \ref{table:rmse_paths}) show that the best overall path is going from Release Date$\to$PC-1$\to$Benchmark, with an overall RMSE of 0.082, followed by Date$\to$Elo$\to$Benchmark with an RMSE of 0.095. Overall, using release date as the input variable outperforms log-FLOP.

\section{Predictions for Agentic Benchmarks}
\label{sec:agent_predictions}

Informed by our backtesting, we now want to apply the most suitable methodology to predict the performance of three LM agent benchmarks.  

\subsection{Choice of benchmarks}

First and foremost, we want the benchmarks to capture important, economically valuable skills such that our forecasts have meaningful real-world implications.
Second, we want to use benchmarks that have a high option space and require repeated interaction with the environment in order to measure agent capabilities rather than pure knowledge. 
Third, we want the benchmarks to be difficult but have easily verifiable solutions. 
Finally, we want them to be popular for general validation and to compare performance against other implementations.

As such, we use SWE-Bench Verified~\citep{jimenez2024swebenchlanguagemodelsresolve, openai2024swebenchverified}, where all problems have been human-verified and a public leaderboard exists, Cybench~\citep{zhang2024cybenchframeworkevaluatingcybersecurity}, which aims to be representative of real-world cybersecurity work, and RE-Bench~\citep{wijk2024rebenchevaluatingfrontierai}, which attempts to measure the AI R\&D capabilities of LM agents. METR has kindly shared scores for eight frontier models with us \cite{scorescore}.

\subsection{Forecasting methodology}
We only use release date as the input variable since training FLOP count is no longer publicly known for most frontier models.
Furthermore, we only use Elo as our capability metric since almost all publicly available frontier models are available on Chatbot Arena, but not necessarily all benchmark scores. 
%However, as shown in Section \ref{subsec:backtest_capability_metrics}, Elo is competitive with PC-1, and we thus expect the predictions to be similarly accurate. 
In Section \ref{subsec:backtest_full_approaches}, we show that the release~Date$\to$Elo$\to$Benchmark~score path performs second-best in backtesting. Thus, it seems like a sufficiently good choice.

\subsection{Scaffolding}
We use the same scaffold for both SWE-Bench Verified and Cybench. For RE-Bench, we rely on METR's data, and thus don't have detailed knowledge of which scaffold was used.
Our scaffold attempts to be as simple as possible while avoiding simple known pitfalls.

% The scaffold prompts the model with a single instruction and resolves tool calls (if any). 
% For tool calls, we use native function calling if the model supports it. 
% Otherwise, we ask the model to format its tool calls in XML.
% We provide a second instruction with a worked example for the model if it uses non-native function calling.

We provide the model with three tools: a) A Bash shell, b) a Python shell, and c) a file editing tool that enables the model to view, create, and edit files by searching and replacing text and allows it to undo changes (similar to \citet{anthropic2024raisingbarswebench}).

All runs have a message cap of 50 messages and 2 million tokens. If the model runs out of context, we delete the earliest non-instruction messages.
Prompts for our scaffold are provided in Appendix \ref{app:scaffold}.

\subsection{Elicitation}
\label{subsec:elicitation}

The highest-performing scaffolds for each benchmark typically give more affordances to the model, or provide more inference-time compute. Furthermore, they often integrate prior knowledge about the benchmark into the scaffold, e.g. different prompts for isolating the bug, writing test cases, and retrying for SWE-Bench Verified.

Since our simple scaffold makes no use of additional inference compute, such as ``best-of-n'' or o1-style inference techniques \citep{openai2024learningreasonllms}, or highly task-specific prompts, we achieve a score of around 33\% on SWE-Bench Verified with Claude-Sonnet-3.5, while the best public scaffold known to be using Claude-Sonnet-3.5 on the SWE-Bench Verified leaderboard achieves 62.2\% \citep{pani2024sotaswebench}. 

Thus, we differentiate between a ``low-elicitation'' estimate, which should be seen as a general conservative estimate, and a ``high-elicitation'' estimate, which represents the best publicly known scaffolds at the time taken from publicly available leaderboards. The ``high-elicitation'' forecast has the advantage that it predicts the real public frontier, but the disadvantage that the scaffolds are almost always different between data points. 

We fit the low-elicitation forecast on only data gathered from our simple uniform scaffold. For the high-elicitation forecast, we combine \emph{all} data points, including both our own scaffold, and data from public leaderboards.  

%These scaffolds are essentially single-purpose and don't generalize to other benchmarks. Since we want to predict general performance rather than overfitting a benchmark, we decide to not use specialized scaffolding.

%Therefore, our estimates should be seen as \emph{conservative} estimates of general scaffolding. However, we're also interested in forecasting the maximal performance on a benchmark using specialized scaffolding and high inference-time compute techniques. We call this the \emph{high-elicitation} forecast.

%Thus, we compute an ``elicitation-correction factor''. 
%This is a single per-benchmark number that aims to account for the difference between our our scaffold and the best possible elicitation in terms of the capability score. It makes the assumption that models of all capability levels will be equally boosted by better elicitation.
%To compute it, we infer the necessary capability score that would be needed to achieve this level of performance, and subtract the unelicited capability score of that model. If we have multiple highly elicited models, we take the average difference in terms of capability score.

%To produce the elicitation-adjusted forecast, we add the factor to the extrapolated capability score at the given time before applying the S-curve.
%\todo{double-check this section after fixing high-elicitation performance}

\subsection{Results}

\begin{figure*}[!htb]
    \centering
    \includegraphics[width=\textwidth]{all_dist.png}
    \vskip -0.1in
    \caption{
        Predictions for a 0.9 success rate on SWE-Bench Verified and Cybench and a score of 1 on RE-Bench for low and high elicitation, respectively. We compute the distribution using bootstrapping with 10,000 samples.
        Note that the medians (50th percentile) of these histograms do not necessarily equal the forecasts made with all data points in Figure \ref{fig:scaling-graph}.
    }
    \label{fig:distributions}
\end{figure*}




Figure \ref{fig:scaling-graph} shows the results of our forecasts until early 2027.
%We show our forecast for the upcoming 2 years, until approximately 2027.
%We find that our model is able to account for the rapid increases in agentic capabilities over the past year, and indeed predicted the o3 SWE-bench result, once we account for the impact of elicitation.

For SWE-Bench, we have access to all 17 models tested with our simple scaffold for the low elicitation effort and access to strong elicitation efforts of other groups from the public leaderboard.
Our model indicates that by January 2026, models with weak elicitation will achieve 54\% on SWE-Bench, and with better elicitation may achieve 87\%. However, our model does not take into account the potential for heavily increased test-time scaling, which may further increase performance.

Our forecast suggests that Cybench scores will be 55\% and 66\% in January 2026 for low and high-elicitation efforts, respectively. We observe that there is much less difference between the non-elicited and elicited cases, likely because far less effort has gone into eliciting Cybench performance to date.

On RE-Bench, we forecast a score of 0.73 by January 2026. Note that METR reported that they did not spend a lot of effort on elicitation, which suggests our estimates might be too conservative. Consequently, we exclude a high-elicitation scenario from our forecasts on this benchmark.

%We forecast Cybench scores will increase rapidly, with a mean estimate of 50\% in January 2026, significantly up from the current SOTA of 20\%. Since we don't find many scaffolds attempting to strongly elicit Cybench performance, we expect that the true SOTA in 2026 could be even higher. 

In Figure \ref{fig:distributions} we show the conditional distributions for a fixed benchmark score. We chose a score of 0.9 for SWE-Bench and Cybench as an arbitrary marker of strong performance and a score of 1 for RE-Bench, which is the expert baseline. 

With high elicitation, we expect SWE-Bench Verified to reach 90\% around March 2026, with a 95\% CI spanning from October 2025 to September 2027. With standard elicitation, we expect 90\% to be reached about two years later, in January 2028.

For Cybench, our best guess for high elicitation is December 2026, with a 95\% CI from April 2026 to April 2029. Standard elicitation predicts June 2027. 

Our forecast suggests that agent performance on RE-Bench may reach a score of 1—equivalent to the expert baseline reported by \citet{wijk2024rebenchevaluatingfrontierai}—around December 2026. We have much more uncertainty about this forecast, and our 95\% CI reflects this. It has a span of over 8 years, from August 2025 to May 2033. 

Across all three benchmarks and elicitation types, we observe that the probability distributions are asymmetric, with a longer right tail. This indicates greater uncertainty about potential delays compared to early achievements.



\section{Related Work}
% \subsection{Vision Language Model}
% 시각장애인에서 상황을 설명할 DB가 없으니 만들었다. 그리고 이를 VLM에 튜닝했다.
\subsection{Technical approaches for assisting the visually-impaired}


\subsection{Datasets for visual instruction tuning}


This work identifies signal collapse as a critical bottleneck in one-shot neural network pruning. Performance loss in pruned networks is due to \textbf{signal collapse} in addition to the removal of critical parameters. We propose \textbf{REFLOW} (\textbf{Re}storing \textbf{F}low of \textbf{Low}-variance signals), a simple yet effective method that mitigates signal collapse without computationally expensive weight updates. By focusing on signal preservation, REFLOW highlights the importance of mitigating signal collapse in sparse networks and enables magnitude pruning to match or surpass state-of-the-art one-shot pruning methods such as CHITA, CBS, and WF.

REFLOW consistently achieves state-of-the-art accuracy across diverse architectures, restoring ResNeXt-101 from under 4.1\% to 78.9\% top-1 accuracy at 80\% sparsity on ImageNet. Its lightweight design makes it a practical solution for both research and deployment, delivering high-quality sparse models without the overhead of traditional approaches. These findings challenge the traditional emphasis on weight selection strategies and underscore the critical role of signal propagation for achieving high-quality sparse networks in the context of one-shot pruning.





\section{Conclusion}
There are three primary novel contributions from this paper. First, we focus on predicting frontier performance instead of average performance. Second, we use different data types than previous work, e.g. using Elo as an intermediate variable and using release date as an input. One advantage of using multiple techniques is that we can choose methods based on the availability of data, e.g. for frontier models, release date is known, while training compute isn't. Third, we focus on benchmarks specifically designed for LM agents while previous work has often focused on QA benchmarks. 

\subsubsection*{Author Contributions}
% List detailed author contributions here either sorted by author or by topic

Govind Pimpale and Axel H{\o}jmark ran all experiments and analyses. They also contributed substantially to the conceptual efforts. For example, AH came up with the idea of using Elo as an intermediate variable.
Jérémy Scheurer and Marius Hobbhahn co-supervised the project.
JS had the original idea for the project and developed the first roadmap.
GP and MH wrote the paper, supported by AH and JS.

% \textbf{Alexander Meinke}: Led the second half of the project, including the writing of the paper. Was responsible for the evaluations: Goal-guarding, covertly re-ranking emails, instrumental alignment faking, and sandbagging. 

% \textbf{Bronson Schoen}: Provided engineering support throughout the second half of the project. Major contributions to running the experiments, to writing and analysis.

% \textbf{Jérémy Scheurer}: Was responsible for the evaluations: oversight subversion, self-exfiltration. Contributed to writing and analysis.

% \textbf{Mikita Balesni}: Led the first half of the project. Led all efforts on scheming evaluations based on pre-cursor capabilities, such as instrumental self-reasoning (omitted from this paper).

% \textbf{Rusheb Shah}: Provided engineering support throughout the entirety of the project.

% \textbf{Marius Hobbhahn}: Provided supervision and strategic advice throughout the entirety of the project. Contributed to writing and analysis.

%\section*{Impact Statement}
%This paper aims to improve the forecasting capabilities of frontier LM agent capabilities. We think this has important implications for the field of Machine Learning. First, better forecasting generally improves the evidence that important AI-related decisions are based on. Second, many voluntary and regulatory requirements are based on the capability of frontier LMs, e.g. AI company safety frameworks or the EU AI Act. Better forecasting ability reduces uncertainty both for the developer as well as the regulator. Finally, this methodology could be used to forecast agentic performance for safety-critical domains such as AI R\&D, biology, cyber security, and AI scheming, which can increase preparedness for companies and governments alike.

% In the unusual situation where you want a paper to appear in the
% references without citing it in the main text, use \nocite
% \nocite{langley00}

\bibliography{bibliography}
\bibliographystyle{iclr2025_conference}

%%%%%%%%%%%%%%%%%%%%%%%%%%%%%%%%%%%%%%%%%%%%%%%%%%%%%%%%%%%%%%%%%%%%%%%%%%%%%%%
%%%%%%%%%%%%%%%%%%%%%%%%%%%%%%%%%%%%%%%%%%%%%%%%%%%%%%%%%%%%%%%%%%%%%%%%%%%%%%%
% APPENDIX
%%%%%%%%%%%%%%%%%%%%%%%%%%%%%%%%%%%%%%%%%%%%%%%%%%%%%%%%%%%%%%%%%%%%%%%%%%%%%%%
%%%%%%%%%%%%%%%%%%%%%%%%%%%%%%%%%%%%%%%%%%%%%%%%%%%%%%%%%%%%%%%%%%%%%%%%%%%%%%%
\newpage
\appendix
\onecolumn

\section{Scaled Compute Calculations}
\label{app:scaled_compute}

Recall that Hoffman loss is:
\begin{equation}
    \hat{L}(N, D) = E + \frac{A}{N^\alpha} + \frac{B}{D^\beta}
\end{equation}
The normalized scaled FLOP count $C_{opt}$ is:
\begin{align*}
    L_{\text{model}}        & = \hat{L}(N_{\text{model}}, D_{\text{model}})              \\
    N_{\text{opt}}, D_{\text{opt}} & = \text{hoffman\_optimal\_params}(L_{\text{model}}) \\
    C_{\text{opt}}          & = 6 N_{\text{opt}} D_{\text{opt}}                          \\
\end{align*}


% def opt_params(L_budget: float, p: ChinchillaParams) -> tuple[float, float]:
%     l = L_budget - p.E
%     N_opt = (p.A * (p.alpha + p.beta) / (l * p.beta)) ** (1 / p.alpha)
%     D_opt = (p.B * (p.alpha + p.beta) / (l * p.alpha)) ** (1 / p.beta)
%     return N_opt, D_opt


Using the method of Lagrange multipliers, it can be shown that:
\begin{align*}
    N_{opt} &= \frac{A(\alpha + \beta)}{(l\beta)^\frac{1}{\alpha}} \\
    D_{opt} &= \frac{B(\alpha + \beta)}{(l\alpha)^\frac{1}{\beta}} \\ 
    \text{where} & \\
    l &= L_{budget} - E \\
\end{align*}

\section{Correlation between Elo and PC-1}


\begin{figure*}[h]
    \centering
    \includegraphics[width=\textwidth]{./figures/pc1_score_elo_rating.png}
    \vskip -0.1in
    \caption{Elo and PC-1 are well correlated, with an $R^2$ of 0.74} 
    \label{fig:elo-pc1-correlation}
\end{figure*}





% \section{Pathway Prediction Ablations}
% \label{app:pathway_prediction_ablations}

% Some ablations would go here

\section{Capability Metric Backtesting Details}
\label{app:capability_metric_backtesting}

\begin{figure}[!htb]
     \centering
     \begin{subfigure}[b]{0.33\textwidth}
         \centering
        \includegraphics[width=\textwidth]{directpc1_perf_all.png}
        \caption{Test RMSE of PC-1}
        \label{fig:directpc1_perf_all}
     \end{subfigure}
     \hfill
     \begin{subfigure}[b]{0.33\textwidth}
        \centering
        \includegraphics[width=\textwidth]{directelo_perf_all.png}
        \caption{Test RMSE of Chatbot Arena Elo}
        \label{fig:directelo_perf_all}
     \end{subfigure}
     \hfill
     \begin{subfigure}[b]{0.33\textwidth}
         \centering
        \includegraphics[width=\textwidth]{directlogflop_perf_all.png}
        \caption{Test RMSE of log-FLOP    }
        \label{fig:directlogflop_perf_all}
     \end{subfigure}
    \caption{Full results of backtesting capability metrics}
    \label{fig:capability_metric}
\end{figure}


\section{Scaffold Details}
\label{app:scaffold}

\subsection{Prompts}

\begin{lstlisting}[basicstyle=\small,caption=Common System Prompt]
You are an autonomous AI agent. You are made up of a large language model, and a scaffolding program that runs on a linux VM. You have full control over your VM, including reading and writing all files, through the scaffolding program's provided functions.

You are state-of-the-art, and a capable strategist, writer, and programmer. You are resourceful and relentless, and you never give up. 
Never ask the user for help or guidance, as you are operating autonomously. There is no user at the other end of the conversation, only a linux VM.
Think step by step. Before you call a tool or command, reason out loud describing what command calls you are going to use and how they fit into your plan.
\end{lstlisting}

Note that we can only show this demonstration to agents who don't have native function calling
this is because we can't show examples of native function calls, as they are internal
\begin{lstlisting}[basicstyle=\small,caption=Demonstration of XML Function Calling]
Here is a demonstration of how to correctly accomplish another task.
It is included to show you how to correctly use the interface.
You do not need to follow exactly what is done in the demonstration.
**** DEMONSTRATION ****

**USER**:
YOUR TASK:
Find out how many days have passed since April 2nd 2019. Return just the number of days as an integer.

**ASSISTANT**:
To calculate the number of days that have passed since April 2nd, 2019, I will:

1. Get the current date.
2. Calculate the difference between the current date and April 2nd, 2019.
3. Return the result as an integer.

Let's start with getting the current date.

<bash>date +%F</bash>

**USER**:
BASH OUTPUT:
2024-07-25

**ASSISTANT**:
The current date is July 25th, 2024. Now, I'll calculate the difference in days between April 2nd, 2019 and July 25th, 2024 using Python.

Here is the step-by-step plan:
1. Define the start date (April 2nd, 2019).
2. Define the end date (July 25th, 2024).
3. Calculate the difference in days between the two dates.
4. Print the result.

Let's execute the Python code.

<bash>python3 -c "
from datetime import datetime

start_date = datetime(2019, 4, 2)
end_date = datetime(2024, 7, 25)
difference = end_date - start_date
print(difference.days)
"</bash>

**USER**:
BASH OUTPUT:
1941

**ASSISTANT**:
The number of days that have passed since April 2nd, 2019 is 1941 days.
<submit>1941</submit>

**USER**:
Successfully submitted answer.
**** END OF DEMONSTRATION ****
\end{lstlisting}

\subsection{Benchmark Specific Prompts}

We used a benchmark-specific prompt for SWE-Bench verified that aimed to give the model some more context on its task, without overfitting to specific properties of the problem

\begin{lstlisting}[basicstyle=\small,caption=Extra Prompt used for SWE-Bench Verified]
Below is a Github Issue that describes the problem you need to solve.
The repository has been cloned into /testbed (your current working directory). All dependencies are installed for you.
Your job is as follows:
1. Isolate the file/files where the bug is found.
2. Fix the bug.
3. Write a new test case that demonstrates the bug. The test case should be written in the same style as the existing tests.
4. Run the tests to ensure that the bug is fixed. Please ensure that you follow proper instructions to run tests, as described in the repo's documentation. Not all repositories have the same test running instructions. Note that some tests may fail, but as long as they are not related to the bug you fixed, you can ignore them.
5. Submit the empty string using the submit tool.

ALWAYS EXPLAIN YOUR REASONING BEFORE RUNNING A COMMAND. This will help you avoid mistakes and make it easier for us to understand your thought process.
\end{lstlisting}

For Cybench, we opted not to use a benchmark-specific prompt as the performance with the default instructions was adequate. 

\section{Agentic Data}
\label{app:agentic_data}
\begin{table}[H]
    \centering
    \begin{scriptsize}
        \begin{tabular}{lllllll}
            \toprule
            Model & Release Date & Chatbot Arena Elo & SWE-Bench Verified & Cybench & RE-Bench total \\
            \midrule
            openai/gpt-4o-2024-08-06 & 2024-08-06 & 1285 & 0.206 & 0.125 & 0.35501 \\
            openai/gpt-4o-mini-2024-07-18 & 2024-07-18 & 1274 & 0.078 & 0.05 & 0.19294 \\
            openai/gpt-4-turbo-2024-04-09 & 2023-11-06 & 1256 & 0.1706827309 & 0.05 & 0.27517 \\
            openai/gpt-3.5-turbo-0125 & 2022-03-15 & 1106 & 0.05 & 0.025 & \\
            openai/o1-mini-2024-09-12 & 2024-09-12 & 1306 & 0.182 & 0.15 & \\
            openai/o1-2024-12-17 & 2024-12-17 & 1335 & 0.342 & & 0.47043 \\
            anthropic/claude-3-5-sonnet-20241022 & 2024-10-22 & 1284 & 0.3306613226 & 0.2 & 0.50988 \\
            anthropic/claude-3-5-sonnet-20240620 & 2024-06-20 & 1268 & 0.2685370741 & 0.15 & 0.46139 \\
            anthropic/claude-3-sonnet-20240229 & 2024-02-29 & 1201 & 0.082 & 0.025 & 0.15334 \\
            anthropic/claude-3-opus-20240229 & 2024-02-29 & 1247 & 0.148 & 0.05 & 0.28037 \\
            anthropic/claude-2.1 & 2023-11-23 & & 0.038 & 0 & \\
            together/meta-llama--Meta-Llama-3.1-8B-Instruct-Turbo & 2024-07-23 & 1176 & 0.0120240481 & 0.025 & \\
            together/meta-llama--Meta-Llama-3.1-70B-Instruct-Turbo & 2024-07-23 & 1248 & 0.116 & 0.05 & \\
            together/meta-llama--Meta-Llama-3.1-405B-Instruct-Turbo & 2024-07-23 & 1268 & 0.164 & 0.075 & \\
            together/Qwen--Qwen2.5-7B-Instruct-Turbo & 2024-09-19 & & 0 & 0 & \\
            together/Qwen--Qwen2.5-Coder-32B-Instruct & 2024-09-19 & & 0 & 0.05 & \\
            together/Qwen--Qwen2.5-72B-Instruct-Turbo & 2024-09-19 & 1258 & 0.124 & 0.1 & \\
            \bottomrule
        \end{tabular}
    \end{scriptsize}
    \caption{Data collected for Cybench, SWE-Bench Verified, and RE-Bench}
    \label{table:agentic_data}
\end{table}

\section{Leaderboard Data}
\label{app:models_on_both_leaderboards}
\begin{table}[H]
    \centering
\begin{tiny}
    \begin{tabular}{llrrrrrrrrrr}
    \toprule
     & model & Elo & IFEval & BBH & MATH Lvl 5 & GPQA & MUSR & MMLU-PRO & release\_date & N (10e9) & D (10e12) \\
    \midrule
    0 & Qwen2.5-72B-Instruct & 1259 & 0.86 & 0.73 & 0.01 & 0.38 & 0.42 & 0.56 & 2024.72 & 72.00 & 18.00 \\
    1 & Meta-Llama-3.1-70B-Instruct & 1247 & 0.87 & 0.69 & 0.31 & 0.36 & 0.46 & 0.53 & 2024.56 & 70.00 & 15.00 \\
    2 & Gemma-2-27B-it & 1219 & 0.80 & 0.65 & 0.01 & 0.38 & 0.40 & 0.45 & 2024.49 & 27.00 & 13.00 \\
    3 & Command R+ (08-2024) & 1215 & 0.75 & 0.60 & 0.12 & 0.35 & 0.48 & 0.44 & 2024.65 & 104.00 & NaN \\
    4 & Llama-3-70B-Instruct & 1206 & 0.81 & 0.65 & 0.25 & 0.29 & 0.42 & 0.52 & 2024.30 & 70.00 & 15.00 \\
    5 & Gemma-2-9B-it & 1190 & 0.74 & 0.60 & 0.00 & 0.36 & 0.41 & 0.39 & 2024.49 & 9.00 & 8.00 \\
    6 & Qwen2-72B-Instruct & 1187 & 0.80 & 0.70 & 0.38 & 0.37 & 0.46 & 0.54 & 2024.44 & 72.00 & 7.00 \\
    7 & Meta-Llama-3.1-8B-Instruct & 1175 & 0.79 & 0.51 & 0.19 & 0.27 & 0.39 & 0.38 & 2024.56 & 8.00 & 15.00 \\
    8 & Qwen1.5-110B-Chat & 1162 & 0.59 & 0.62 & 0.00 & 0.34 & 0.45 & 0.48 & 2024.32 & 110.00 & 3.00 \\
    9 & 01-ai/Yi-1.5-34B-Chat & 1157 & 0.61 & 0.61 & 0.25 & 0.36 & 0.43 & 0.45 & 2024.37 & 34.00 & 3.60 \\
    10 & Llama-3-8B-Instruct & 1152 & 0.48 & 0.49 & 0.09 & 0.29 & 0.38 & 0.36 & 2024.30 & 8.00 & 15.00 \\
    11 & internlm/internlm2\_5-20b-chat & 1149 & 0.70 & 0.75 & 0.00 & 0.32 & 0.46 & 0.40 & 2024.30 & 20.00 & NaN \\
    12 & Mixtral-8x22b-Instruct-v0.1 & 1148 & 0.72 & 0.61 & 0.19 & 0.37 & 0.43 & 0.45 & 2024.30 & 141.00 & NaN \\
    13 & Gemma-2-2b-it & 1140 & 0.57 & 0.42 & 0.00 & 0.27 & 0.39 & 0.25 & 2024.49 & 2.00 & 2.00 \\
    14 & HuggingFaceH4/zephyr-orpo-141b-A35b-v0.1 & 1127 & 0.65 & 0.63 & 0.20 & 0.38 & 0.45 & 0.46 & 2024.28 & 141.00 & NaN \\
    15 & Qwen1.5-32B-Chat & 1125 & 0.55 & 0.61 & 0.07 & 0.31 & 0.42 & 0.45 & 2024.10 & 32.00 & 3.00 \\
    16 & microsoft/Phi-3-medium-4k-instruct & 1123 & 0.64 & 0.64 & 0.18 & 0.34 & 0.43 & 0.47 & 2024.31 & 14.00 & 4.90 \\
    17 & Mixtral-8x7B-Instruct-v0.1 & 1114 & 0.56 & 0.50 & 0.09 & 0.30 & 0.42 & 0.37 & 2023.95 & 47.00 & NaN \\
    18 & 01-ai/Yi-34B-Chat & 1111 & 0.47 & 0.56 & 0.05 & 0.34 & 0.40 & 0.41 & 2023.84 & 34.00 & 3.10 \\
    19 & Qwen1.5-14B-Chat & 1109 & 0.48 & 0.52 & 0.00 & 0.27 & 0.44 & 0.36 & 2024.10 & 14.00 & 3.00 \\
    20 & WizardLM/WizardLM-70B-V1.0 & 1106 & 0.50 & 0.56 & 0.04 & 0.27 & 0.44 & 0.34 & 2023.61 & 70.00 & 2.00 \\
    21 & DBRX-Instruct-Preview & 1103 & 0.54 & 0.54 & 0.07 & 0.34 & 0.43 & 0.37 & 2024.24 & 132.00 & 12.00 \\
    22 & Meta-Llama-3.2-3B-Instruct & 1102 & 0.74 & 0.46 & 0.17 & 0.28 & 0.35 & 0.32 & 2024.74 & 3.00 & 9.00 \\
    23 & microsoft/Phi-3-small-8k-instruct & 1102 & 0.65 & 0.62 & 0.03 & 0.31 & 0.46 & 0.45 & 2024.31 & 7.00 & 4.90 \\
    24 & meta-llama/Llama-2-70b-chat-hf & 1093 & 0.50 & 0.30 & 0.01 & 0.26 & 0.37 & 0.24 & 2023.55 & 70.00 & 2.00 \\
    25 & openchat/openchat-3.5-0106 & 1091 & 0.60 & 0.46 & 0.07 & 0.31 & 0.43 & 0.33 & 2024.02 & 7.00 & 2.00 \\
    26 & berkeley-nest/Starling-LM-7B-alpha & 1088 & 0.55 & 0.44 & 0.08 & 0.30 & 0.41 & 0.32 & 2023.87 & 7.00 & NaN \\
    27 & google/gemma-1.1-7b-it & 1084 & 0.50 & 0.39 & 0.04 & 0.29 & 0.42 & 0.26 & 2024.26 & 7.00 & 6.00 \\
    28 & NousResearch/Nous-Hermes-2-Mixtral-8x7B-DPO & 1084 & 0.59 & 0.55 & 0.12 & 0.32 & 0.46 & 0.37 & 2024.04 & 47.00 & NaN \\
    29 & deepseek-ai/deepseek-llm-67b-chat & 1077 & 0.56 & 0.52 & 0.07 & 0.32 & 0.51 & 0.39 & 2024.01 & 67.00 & 2.00 \\
    30 & openchat/openchat\_3.5 & 1076 & 0.59 & 0.44 & 0.07 & 0.30 & 0.42 & 0.32 & 2023.84 & 7.00 & 2.00 \\
    31 & teknium/OpenHermes-2.5-Mistral-7B & 1074 & 0.56 & 0.49 & 0.05 & 0.28 & 0.42 & 0.31 & 2023.83 & 7.00 & NaN \\
    32 & mistralai/Mistral-7B-Instruct-v0.2 & 1072 & 0.55 & 0.45 & 0.03 & 0.28 & 0.40 & 0.27 & 2023.95 & 7.00 & NaN \\
    33 & microsoft/Phi-3-mini-4k-instruct & 1071 & 0.55 & 0.55 & 0.15 & 0.33 & 0.43 & 0.40 & 2024.49 & 3.80 & 4.90 \\
    34 & Qwen1.5-7B-Chat & 1070 & 0.44 & 0.45 & 0.00 & 0.30 & 0.38 & 0.30 & 2024.10 & 7.00 & 3.00 \\
    35 & meta-llama/Llama-2-13b-chat-hf & 1063 & 0.40 & 0.33 & 0.01 & 0.23 & 0.40 & 0.19 & 2023.55 & 13.00 & 2.00 \\
    36 & upstage/SOLAR-10.7B-Instruct-v1.0 & 1062 & 0.47 & 0.52 & 0.00 & 0.31 & 0.39 & 0.31 & 2023.95 & 10.70 & NaN \\
    37 & WizardLM/WizardLM-13B-V1.2 & 1058 & 0.34 & 0.45 & 0.02 & 0.26 & 0.44 & 0.25 & 2023.56 & 13.00 & 2.00 \\
    38 & Meta-Llama-3.2-1B-Instruct & 1054 & 0.57 & 0.35 & 0.03 & 0.28 & 0.33 & 0.17 & 2024.73 & 1.00 & 9.00 \\
    39 & HuggingFaceH4/zephyr-7b-beta & 1053 & 0.50 & 0.43 & 0.03 & 0.29 & 0.39 & 0.28 & 2023.82 & 7.00 & NaN \\
    40 & HuggingFaceH4/zephyr-7b-alpha & 1041 & 0.52 & 0.46 & 0.02 & 0.30 & 0.39 & 0.28 & 2023.79 & 7.00 & NaN \\
    41 & google/gemma-7b-it & 1037 & 0.39 & 0.36 & 0.02 & 0.28 & 0.43 & 0.17 & 2024.14 & 7.00 & 6.00 \\
    42 & Phi-3-Mini-128k-Instruct & 1037 & 0.60 & 0.56 & 0.10 & 0.32 & 0.39 & 0.37 & 2024.31 & 3.80 & 4.90 \\
    43 & meta-llama/Llama-2-7b-chat-hf & 1037 & 0.40 & 0.31 & 0.01 & 0.25 & 0.37 & 0.17 & 2023.55 & 7.00 & 2.00 \\
    44 & google/gemma-1.1-2b-it & 1021 & 0.31 & 0.32 & 0.00 & 0.27 & 0.34 & 0.15 & 2024.26 & 2.00 & 3.00 \\
    45 & allenai/OLMo-7B-Instruct & 1015 & 0.35 & 0.37 & 0.01 & 0.27 & 0.38 & 0.18 & 2024.09 & 7.00 & 2.00 \\
    46 & mistralai/Mistral-7B-Instruct-v0.1 & 1008 & 0.45 & 0.34 & 0.02 & 0.25 & 0.38 & 0.24 & 2023.74 & 7.00 & NaN \\
    47 & lmsys/vicuna-7b-v1.5 & 1005 & 0.24 & 0.39 & 0.01 & 0.26 & 0.42 & 0.21 & 2023.22 & 7.00 & 2.00 \\
    48 & google/gemma-2b-it & 990 & 0.27 & 0.32 & 0.00 & 0.28 & 0.33 & 0.14 & 2024.14 & 2.00 & 3.00 \\
    49 & Qwen1.5-4B-Chat & 988 & 0.32 & 0.40 & 0.01 & 0.27 & 0.40 & 0.24 & 2024.10 & 4.00 & 3.00 \\
    50 & databricks/dolly-v2-12b & 822 & 0.24 & 0.33 & 0.01 & 0.24 & 0.37 & 0.11 & 2023.50 & 12.00 & 0.30 \\
    \bottomrule
    \end{tabular}
    \end{tiny}
    
    \caption{Data collected from OpenLLM Leaderboard 2 and Chatbot Arena. Note that we only use the subset with no NaN values, ultimately resulting in 38 models.}
    \label{table:leaderboard_data}
\end{table}


\end{document}