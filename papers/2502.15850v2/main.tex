\documentclass{article} % For LaTeX2e
\usepackage{iclr2025_conference,times}

% Optional math commands from https://github.com/goodfeli/dlbook_notation.
%%%%% NEW MATH DEFINITIONS %%%%%

\usepackage{amsmath,amsfonts,bm}
\usepackage{derivative}
% Mark sections of captions for referring to divisions of figures
\newcommand{\figleft}{{\em (Left)}}
\newcommand{\figcenter}{{\em (Center)}}
\newcommand{\figright}{{\em (Right)}}
\newcommand{\figtop}{{\em (Top)}}
\newcommand{\figbottom}{{\em (Bottom)}}
\newcommand{\captiona}{{\em (a)}}
\newcommand{\captionb}{{\em (b)}}
\newcommand{\captionc}{{\em (c)}}
\newcommand{\captiond}{{\em (d)}}

% Highlight a newly defined term
\newcommand{\newterm}[1]{{\bf #1}}

% Derivative d 
\newcommand{\deriv}{{\mathrm{d}}}

% Figure reference, lower-case.
\def\figref#1{figure~\ref{#1}}
% Figure reference, capital. For start of sentence
\def\Figref#1{Figure~\ref{#1}}
\def\twofigref#1#2{figures \ref{#1} and \ref{#2}}
\def\quadfigref#1#2#3#4{figures \ref{#1}, \ref{#2}, \ref{#3} and \ref{#4}}
% Section reference, lower-case.
\def\secref#1{section~\ref{#1}}
% Section reference, capital.
\def\Secref#1{Section~\ref{#1}}
% Reference to two sections.
\def\twosecrefs#1#2{sections \ref{#1} and \ref{#2}}
% Reference to three sections.
\def\secrefs#1#2#3{sections \ref{#1}, \ref{#2} and \ref{#3}}
% Reference to an equation, lower-case.
\def\eqref#1{equation~\ref{#1}}
% Reference to an equation, upper case
\def\Eqref#1{Equation~\ref{#1}}
% A raw reference to an equation---avoid using if possible
\def\plaineqref#1{\ref{#1}}
% Reference to a chapter, lower-case.
\def\chapref#1{chapter~\ref{#1}}
% Reference to an equation, upper case.
\def\Chapref#1{Chapter~\ref{#1}}
% Reference to a range of chapters
\def\rangechapref#1#2{chapters\ref{#1}--\ref{#2}}
% Reference to an algorithm, lower-case.
\def\algref#1{algorithm~\ref{#1}}
% Reference to an algorithm, upper case.
\def\Algref#1{Algorithm~\ref{#1}}
\def\twoalgref#1#2{algorithms \ref{#1} and \ref{#2}}
\def\Twoalgref#1#2{Algorithms \ref{#1} and \ref{#2}}
% Reference to a part, lower case
\def\partref#1{part~\ref{#1}}
% Reference to a part, upper case
\def\Partref#1{Part~\ref{#1}}
\def\twopartref#1#2{parts \ref{#1} and \ref{#2}}

\def\ceil#1{\lceil #1 \rceil}
\def\floor#1{\lfloor #1 \rfloor}
\def\1{\bm{1}}
\newcommand{\train}{\mathcal{D}}
\newcommand{\valid}{\mathcal{D_{\mathrm{valid}}}}
\newcommand{\test}{\mathcal{D_{\mathrm{test}}}}

\def\eps{{\epsilon}}


% Random variables
\def\reta{{\textnormal{$\eta$}}}
\def\ra{{\textnormal{a}}}
\def\rb{{\textnormal{b}}}
\def\rc{{\textnormal{c}}}
\def\rd{{\textnormal{d}}}
\def\re{{\textnormal{e}}}
\def\rf{{\textnormal{f}}}
\def\rg{{\textnormal{g}}}
\def\rh{{\textnormal{h}}}
\def\ri{{\textnormal{i}}}
\def\rj{{\textnormal{j}}}
\def\rk{{\textnormal{k}}}
\def\rl{{\textnormal{l}}}
% rm is already a command, just don't name any random variables m
\def\rn{{\textnormal{n}}}
\def\ro{{\textnormal{o}}}
\def\rp{{\textnormal{p}}}
\def\rq{{\textnormal{q}}}
\def\rr{{\textnormal{r}}}
\def\rs{{\textnormal{s}}}
\def\rt{{\textnormal{t}}}
\def\ru{{\textnormal{u}}}
\def\rv{{\textnormal{v}}}
\def\rw{{\textnormal{w}}}
\def\rx{{\textnormal{x}}}
\def\ry{{\textnormal{y}}}
\def\rz{{\textnormal{z}}}

% Random vectors
\def\rvepsilon{{\mathbf{\epsilon}}}
\def\rvphi{{\mathbf{\phi}}}
\def\rvtheta{{\mathbf{\theta}}}
\def\rva{{\mathbf{a}}}
\def\rvb{{\mathbf{b}}}
\def\rvc{{\mathbf{c}}}
\def\rvd{{\mathbf{d}}}
\def\rve{{\mathbf{e}}}
\def\rvf{{\mathbf{f}}}
\def\rvg{{\mathbf{g}}}
\def\rvh{{\mathbf{h}}}
\def\rvu{{\mathbf{i}}}
\def\rvj{{\mathbf{j}}}
\def\rvk{{\mathbf{k}}}
\def\rvl{{\mathbf{l}}}
\def\rvm{{\mathbf{m}}}
\def\rvn{{\mathbf{n}}}
\def\rvo{{\mathbf{o}}}
\def\rvp{{\mathbf{p}}}
\def\rvq{{\mathbf{q}}}
\def\rvr{{\mathbf{r}}}
\def\rvs{{\mathbf{s}}}
\def\rvt{{\mathbf{t}}}
\def\rvu{{\mathbf{u}}}
\def\rvv{{\mathbf{v}}}
\def\rvw{{\mathbf{w}}}
\def\rvx{{\mathbf{x}}}
\def\rvy{{\mathbf{y}}}
\def\rvz{{\mathbf{z}}}

% Elements of random vectors
\def\erva{{\textnormal{a}}}
\def\ervb{{\textnormal{b}}}
\def\ervc{{\textnormal{c}}}
\def\ervd{{\textnormal{d}}}
\def\erve{{\textnormal{e}}}
\def\ervf{{\textnormal{f}}}
\def\ervg{{\textnormal{g}}}
\def\ervh{{\textnormal{h}}}
\def\ervi{{\textnormal{i}}}
\def\ervj{{\textnormal{j}}}
\def\ervk{{\textnormal{k}}}
\def\ervl{{\textnormal{l}}}
\def\ervm{{\textnormal{m}}}
\def\ervn{{\textnormal{n}}}
\def\ervo{{\textnormal{o}}}
\def\ervp{{\textnormal{p}}}
\def\ervq{{\textnormal{q}}}
\def\ervr{{\textnormal{r}}}
\def\ervs{{\textnormal{s}}}
\def\ervt{{\textnormal{t}}}
\def\ervu{{\textnormal{u}}}
\def\ervv{{\textnormal{v}}}
\def\ervw{{\textnormal{w}}}
\def\ervx{{\textnormal{x}}}
\def\ervy{{\textnormal{y}}}
\def\ervz{{\textnormal{z}}}

% Random matrices
\def\rmA{{\mathbf{A}}}
\def\rmB{{\mathbf{B}}}
\def\rmC{{\mathbf{C}}}
\def\rmD{{\mathbf{D}}}
\def\rmE{{\mathbf{E}}}
\def\rmF{{\mathbf{F}}}
\def\rmG{{\mathbf{G}}}
\def\rmH{{\mathbf{H}}}
\def\rmI{{\mathbf{I}}}
\def\rmJ{{\mathbf{J}}}
\def\rmK{{\mathbf{K}}}
\def\rmL{{\mathbf{L}}}
\def\rmM{{\mathbf{M}}}
\def\rmN{{\mathbf{N}}}
\def\rmO{{\mathbf{O}}}
\def\rmP{{\mathbf{P}}}
\def\rmQ{{\mathbf{Q}}}
\def\rmR{{\mathbf{R}}}
\def\rmS{{\mathbf{S}}}
\def\rmT{{\mathbf{T}}}
\def\rmU{{\mathbf{U}}}
\def\rmV{{\mathbf{V}}}
\def\rmW{{\mathbf{W}}}
\def\rmX{{\mathbf{X}}}
\def\rmY{{\mathbf{Y}}}
\def\rmZ{{\mathbf{Z}}}

% Elements of random matrices
\def\ermA{{\textnormal{A}}}
\def\ermB{{\textnormal{B}}}
\def\ermC{{\textnormal{C}}}
\def\ermD{{\textnormal{D}}}
\def\ermE{{\textnormal{E}}}
\def\ermF{{\textnormal{F}}}
\def\ermG{{\textnormal{G}}}
\def\ermH{{\textnormal{H}}}
\def\ermI{{\textnormal{I}}}
\def\ermJ{{\textnormal{J}}}
\def\ermK{{\textnormal{K}}}
\def\ermL{{\textnormal{L}}}
\def\ermM{{\textnormal{M}}}
\def\ermN{{\textnormal{N}}}
\def\ermO{{\textnormal{O}}}
\def\ermP{{\textnormal{P}}}
\def\ermQ{{\textnormal{Q}}}
\def\ermR{{\textnormal{R}}}
\def\ermS{{\textnormal{S}}}
\def\ermT{{\textnormal{T}}}
\def\ermU{{\textnormal{U}}}
\def\ermV{{\textnormal{V}}}
\def\ermW{{\textnormal{W}}}
\def\ermX{{\textnormal{X}}}
\def\ermY{{\textnormal{Y}}}
\def\ermZ{{\textnormal{Z}}}

% Vectors
\def\vzero{{\bm{0}}}
\def\vone{{\bm{1}}}
\def\vmu{{\bm{\mu}}}
\def\vtheta{{\bm{\theta}}}
\def\vphi{{\bm{\phi}}}
\def\va{{\bm{a}}}
\def\vb{{\bm{b}}}
\def\vc{{\bm{c}}}
\def\vd{{\bm{d}}}
\def\ve{{\bm{e}}}
\def\vf{{\bm{f}}}
\def\vg{{\bm{g}}}
\def\vh{{\bm{h}}}
\def\vi{{\bm{i}}}
\def\vj{{\bm{j}}}
\def\vk{{\bm{k}}}
\def\vl{{\bm{l}}}
\def\vm{{\bm{m}}}
\def\vn{{\bm{n}}}
\def\vo{{\bm{o}}}
\def\vp{{\bm{p}}}
\def\vq{{\bm{q}}}
\def\vr{{\bm{r}}}
\def\vs{{\bm{s}}}
\def\vt{{\bm{t}}}
\def\vu{{\bm{u}}}
\def\vv{{\bm{v}}}
\def\vw{{\bm{w}}}
\def\vx{{\bm{x}}}
\def\vy{{\bm{y}}}
\def\vz{{\bm{z}}}

% Elements of vectors
\def\evalpha{{\alpha}}
\def\evbeta{{\beta}}
\def\evepsilon{{\epsilon}}
\def\evlambda{{\lambda}}
\def\evomega{{\omega}}
\def\evmu{{\mu}}
\def\evpsi{{\psi}}
\def\evsigma{{\sigma}}
\def\evtheta{{\theta}}
\def\eva{{a}}
\def\evb{{b}}
\def\evc{{c}}
\def\evd{{d}}
\def\eve{{e}}
\def\evf{{f}}
\def\evg{{g}}
\def\evh{{h}}
\def\evi{{i}}
\def\evj{{j}}
\def\evk{{k}}
\def\evl{{l}}
\def\evm{{m}}
\def\evn{{n}}
\def\evo{{o}}
\def\evp{{p}}
\def\evq{{q}}
\def\evr{{r}}
\def\evs{{s}}
\def\evt{{t}}
\def\evu{{u}}
\def\evv{{v}}
\def\evw{{w}}
\def\evx{{x}}
\def\evy{{y}}
\def\evz{{z}}

% Matrix
\def\mA{{\bm{A}}}
\def\mB{{\bm{B}}}
\def\mC{{\bm{C}}}
\def\mD{{\bm{D}}}
\def\mE{{\bm{E}}}
\def\mF{{\bm{F}}}
\def\mG{{\bm{G}}}
\def\mH{{\bm{H}}}
\def\mI{{\bm{I}}}
\def\mJ{{\bm{J}}}
\def\mK{{\bm{K}}}
\def\mL{{\bm{L}}}
\def\mM{{\bm{M}}}
\def\mN{{\bm{N}}}
\def\mO{{\bm{O}}}
\def\mP{{\bm{P}}}
\def\mQ{{\bm{Q}}}
\def\mR{{\bm{R}}}
\def\mS{{\bm{S}}}
\def\mT{{\bm{T}}}
\def\mU{{\bm{U}}}
\def\mV{{\bm{V}}}
\def\mW{{\bm{W}}}
\def\mX{{\bm{X}}}
\def\mY{{\bm{Y}}}
\def\mZ{{\bm{Z}}}
\def\mBeta{{\bm{\beta}}}
\def\mPhi{{\bm{\Phi}}}
\def\mLambda{{\bm{\Lambda}}}
\def\mSigma{{\bm{\Sigma}}}

% Tensor
\DeclareMathAlphabet{\mathsfit}{\encodingdefault}{\sfdefault}{m}{sl}
\SetMathAlphabet{\mathsfit}{bold}{\encodingdefault}{\sfdefault}{bx}{n}
\newcommand{\tens}[1]{\bm{\mathsfit{#1}}}
\def\tA{{\tens{A}}}
\def\tB{{\tens{B}}}
\def\tC{{\tens{C}}}
\def\tD{{\tens{D}}}
\def\tE{{\tens{E}}}
\def\tF{{\tens{F}}}
\def\tG{{\tens{G}}}
\def\tH{{\tens{H}}}
\def\tI{{\tens{I}}}
\def\tJ{{\tens{J}}}
\def\tK{{\tens{K}}}
\def\tL{{\tens{L}}}
\def\tM{{\tens{M}}}
\def\tN{{\tens{N}}}
\def\tO{{\tens{O}}}
\def\tP{{\tens{P}}}
\def\tQ{{\tens{Q}}}
\def\tR{{\tens{R}}}
\def\tS{{\tens{S}}}
\def\tT{{\tens{T}}}
\def\tU{{\tens{U}}}
\def\tV{{\tens{V}}}
\def\tW{{\tens{W}}}
\def\tX{{\tens{X}}}
\def\tY{{\tens{Y}}}
\def\tZ{{\tens{Z}}}


% Graph
\def\gA{{\mathcal{A}}}
\def\gB{{\mathcal{B}}}
\def\gC{{\mathcal{C}}}
\def\gD{{\mathcal{D}}}
\def\gE{{\mathcal{E}}}
\def\gF{{\mathcal{F}}}
\def\gG{{\mathcal{G}}}
\def\gH{{\mathcal{H}}}
\def\gI{{\mathcal{I}}}
\def\gJ{{\mathcal{J}}}
\def\gK{{\mathcal{K}}}
\def\gL{{\mathcal{L}}}
\def\gM{{\mathcal{M}}}
\def\gN{{\mathcal{N}}}
\def\gO{{\mathcal{O}}}
\def\gP{{\mathcal{P}}}
\def\gQ{{\mathcal{Q}}}
\def\gR{{\mathcal{R}}}
\def\gS{{\mathcal{S}}}
\def\gT{{\mathcal{T}}}
\def\gU{{\mathcal{U}}}
\def\gV{{\mathcal{V}}}
\def\gW{{\mathcal{W}}}
\def\gX{{\mathcal{X}}}
\def\gY{{\mathcal{Y}}}
\def\gZ{{\mathcal{Z}}}

% Sets
\def\sA{{\mathbb{A}}}
\def\sB{{\mathbb{B}}}
\def\sC{{\mathbb{C}}}
\def\sD{{\mathbb{D}}}
% Don't use a set called E, because this would be the same as our symbol
% for expectation.
\def\sF{{\mathbb{F}}}
\def\sG{{\mathbb{G}}}
\def\sH{{\mathbb{H}}}
\def\sI{{\mathbb{I}}}
\def\sJ{{\mathbb{J}}}
\def\sK{{\mathbb{K}}}
\def\sL{{\mathbb{L}}}
\def\sM{{\mathbb{M}}}
\def\sN{{\mathbb{N}}}
\def\sO{{\mathbb{O}}}
\def\sP{{\mathbb{P}}}
\def\sQ{{\mathbb{Q}}}
\def\sR{{\mathbb{R}}}
\def\sS{{\mathbb{S}}}
\def\sT{{\mathbb{T}}}
\def\sU{{\mathbb{U}}}
\def\sV{{\mathbb{V}}}
\def\sW{{\mathbb{W}}}
\def\sX{{\mathbb{X}}}
\def\sY{{\mathbb{Y}}}
\def\sZ{{\mathbb{Z}}}

% Entries of a matrix
\def\emLambda{{\Lambda}}
\def\emA{{A}}
\def\emB{{B}}
\def\emC{{C}}
\def\emD{{D}}
\def\emE{{E}}
\def\emF{{F}}
\def\emG{{G}}
\def\emH{{H}}
\def\emI{{I}}
\def\emJ{{J}}
\def\emK{{K}}
\def\emL{{L}}
\def\emM{{M}}
\def\emN{{N}}
\def\emO{{O}}
\def\emP{{P}}
\def\emQ{{Q}}
\def\emR{{R}}
\def\emS{{S}}
\def\emT{{T}}
\def\emU{{U}}
\def\emV{{V}}
\def\emW{{W}}
\def\emX{{X}}
\def\emY{{Y}}
\def\emZ{{Z}}
\def\emSigma{{\Sigma}}

% entries of a tensor
% Same font as tensor, without \bm wrapper
\newcommand{\etens}[1]{\mathsfit{#1}}
\def\etLambda{{\etens{\Lambda}}}
\def\etA{{\etens{A}}}
\def\etB{{\etens{B}}}
\def\etC{{\etens{C}}}
\def\etD{{\etens{D}}}
\def\etE{{\etens{E}}}
\def\etF{{\etens{F}}}
\def\etG{{\etens{G}}}
\def\etH{{\etens{H}}}
\def\etI{{\etens{I}}}
\def\etJ{{\etens{J}}}
\def\etK{{\etens{K}}}
\def\etL{{\etens{L}}}
\def\etM{{\etens{M}}}
\def\etN{{\etens{N}}}
\def\etO{{\etens{O}}}
\def\etP{{\etens{P}}}
\def\etQ{{\etens{Q}}}
\def\etR{{\etens{R}}}
\def\etS{{\etens{S}}}
\def\etT{{\etens{T}}}
\def\etU{{\etens{U}}}
\def\etV{{\etens{V}}}
\def\etW{{\etens{W}}}
\def\etX{{\etens{X}}}
\def\etY{{\etens{Y}}}
\def\etZ{{\etens{Z}}}

% The true underlying data generating distribution
\newcommand{\pdata}{p_{\rm{data}}}
\newcommand{\ptarget}{p_{\rm{target}}}
\newcommand{\pprior}{p_{\rm{prior}}}
\newcommand{\pbase}{p_{\rm{base}}}
\newcommand{\pref}{p_{\rm{ref}}}

% The empirical distribution defined by the training set
\newcommand{\ptrain}{\hat{p}_{\rm{data}}}
\newcommand{\Ptrain}{\hat{P}_{\rm{data}}}
% The model distribution
\newcommand{\pmodel}{p_{\rm{model}}}
\newcommand{\Pmodel}{P_{\rm{model}}}
\newcommand{\ptildemodel}{\tilde{p}_{\rm{model}}}
% Stochastic autoencoder distributions
\newcommand{\pencode}{p_{\rm{encoder}}}
\newcommand{\pdecode}{p_{\rm{decoder}}}
\newcommand{\precons}{p_{\rm{reconstruct}}}

\newcommand{\laplace}{\mathrm{Laplace}} % Laplace distribution

\newcommand{\E}{\mathbb{E}}
\newcommand{\Ls}{\mathcal{L}}
\newcommand{\R}{\mathbb{R}}
\newcommand{\emp}{\tilde{p}}
\newcommand{\lr}{\alpha}
\newcommand{\reg}{\lambda}
\newcommand{\rect}{\mathrm{rectifier}}
\newcommand{\softmax}{\mathrm{softmax}}
\newcommand{\sigmoid}{\sigma}
\newcommand{\softplus}{\zeta}
\newcommand{\KL}{D_{\mathrm{KL}}}
\newcommand{\Var}{\mathrm{Var}}
\newcommand{\standarderror}{\mathrm{SE}}
\newcommand{\Cov}{\mathrm{Cov}}
% Wolfram Mathworld says $L^2$ is for function spaces and $\ell^2$ is for vectors
% But then they seem to use $L^2$ for vectors throughout the site, and so does
% wikipedia.
\newcommand{\normlzero}{L^0}
\newcommand{\normlone}{L^1}
\newcommand{\normltwo}{L^2}
\newcommand{\normlp}{L^p}
\newcommand{\normmax}{L^\infty}

\newcommand{\parents}{Pa} % See usage in notation.tex. Chosen to match Daphne's book.

\DeclareMathOperator*{\argmax}{arg\,max}
\DeclareMathOperator*{\argmin}{arg\,min}

\DeclareMathOperator{\sign}{sign}
\DeclareMathOperator{\Tr}{Tr}
\let\ab\allowbreak


%\usepackage{hyperref}
\usepackage{url}
\usepackage{lipsum}
\usepackage{natbib}
\usepackage[colorlinks=true,citecolor=apolloblue,linkcolor=apolloblue,urlcolor=apolloblue]{hyperref}
% Recommended, but optional, packages for figures and better typesetting:
\usepackage{microtype}
\usepackage{graphicx}
\usepackage{booktabs} % for professional tables
\usepackage{longtable}
\usepackage{tabularx}
\usepackage{subcaption}
\usepackage{amsmath}
\usepackage{listings}
\usepackage{float}


% For theorems and such
\usepackage{amsmath}
\usepackage{amssymb}
\usepackage{mathtools}
\usepackage{amsthm}
\usepackage{algorithmic}


\lstset{
  basicstyle=\ttfamily,
  columns=fullflexible,
  frame=single,
  breaklines=true,
  postbreak=\mbox{\textcolor{red}{$\hookrightarrow$}\space},
}


\graphicspath{ {./figures/} }

\title{Forecasting Frontier Language Model Agent Capabilities}

% Authors must not appear in the submitted version. They should be hidden
% as long as the \iclrfinalcopy macro remains commented out below.
% Non-anonymous submissions will be rejected without review.

\author{% core contributors
Govind Pimpale\thanks{Equal contribution}\\\textmd{MATS}
\And
Axel H{\o}jmark\footnotemark[1]\\\textmd{MATS \& Apollo Research}
\AND
% senior authors
J\'er\'emy Scheurer\thanks{Equal contribution}\\\textmd{Apollo Research}
\And
Marius Hobbhahn\footnotemark[2]\\\textmd{Apollo Research}
}

% The \author macro works with any number of authors. There are two commands
% used to separate the names and addresses of multiple authors: \And and \AND.
%
% Using \And between authors leaves it to \LaTeX{} to determine where to break
% the lines. Using \AND forces a linebreak at that point. So, if \LaTeX{}
% puts 3 of 4 authors names on the first line, and the last on the second
% line, try using \AND instead of \And before the third author name.

\newcommand{\fix}{\marginpar{FIX}}
\newcommand{\new}{\marginpar{NEW}}

\iclrfinalcopy % Uncomment for camera-ready version, but NOT for submission.
\begin{document}


\maketitle

\vspace{0.3in} 
\begin{abstract}
    As Language Models (LMs) increasingly operate as autonomous agents, accurately forecasting their capabilities becomes crucial for societal preparedness. 
    We evaluate six forecasting methods that predict downstream capabilities of LM agents. 
    We use ``one-step'' approaches that predict benchmark scores from input metrics like compute or model release date directly or ``two-step'' approaches that first predict an intermediate metric like the principal component of cross-benchmark performance (PC-1) and human-evaluated competitive Elo ratings.
    We evaluate our forecasting methods by backtesting them on a dataset of 38 LMs from the OpenLLM 2 leaderboard. 
    We then use the validated two-step approach (Release Date$\to$Elo$\to$Benchmark) to predict LM agent performance for frontier models on three benchmarks:
    SWE-Bench Verified (software development), Cybench (cybersecurity assessment), and RE-Bench (ML research engineering). 
    Our forecast predicts that by the beginning of 2026, non-specialized LM agents with low capability elicitation will reach a success rate of 54\% on SWE-Bench Verified, while state-of-the-art LM agents will reach an 87\% success rate.
    Our approach does not account for recent advances in inference-compute scaling and might thus be too conservative.
\end{abstract}

\section{Introduction}
\label{sec:introduction}
The business processes of organizations are experiencing ever-increasing complexity due to the large amount of data, high number of users, and high-tech devices involved \cite{martin2021pmopportunitieschallenges, beerepoot2023biggestbpmproblems}. This complexity may cause business processes to deviate from normal control flow due to unforeseen and disruptive anomalies \cite{adams2023proceddsriftdetection}. These control-flow anomalies manifest as unknown, skipped, and wrongly-ordered activities in the traces of event logs monitored from the execution of business processes \cite{ko2023adsystematicreview}. For the sake of clarity, let us consider an illustrative example of such anomalies. Figure \ref{FP_ANOMALIES} shows a so-called event log footprint, which captures the control flow relations of four activities of a hypothetical event log. In particular, this footprint captures the control-flow relations between activities \texttt{a}, \texttt{b}, \texttt{c} and \texttt{d}. These are the causal ($\rightarrow$) relation, concurrent ($\parallel$) relation, and other ($\#$) relations such as exclusivity or non-local dependency \cite{aalst2022pmhandbook}. In addition, on the right are six traces, of which five exhibit skipped, wrongly-ordered and unknown control-flow anomalies. For example, $\langle$\texttt{a b d}$\rangle$ has a skipped activity, which is \texttt{c}. Because of this skipped activity, the control-flow relation \texttt{b}$\,\#\,$\texttt{d} is violated, since \texttt{d} directly follows \texttt{b} in the anomalous trace.
\begin{figure}[!t]
\centering
\includegraphics[width=0.9\columnwidth]{images/FP_ANOMALIES.png}
\caption{An example event log footprint with six traces, of which five exhibit control-flow anomalies.}
\label{FP_ANOMALIES}
\end{figure}

\subsection{Control-flow anomaly detection}
Control-flow anomaly detection techniques aim to characterize the normal control flow from event logs and verify whether these deviations occur in new event logs \cite{ko2023adsystematicreview}. To develop control-flow anomaly detection techniques, \revision{process mining} has seen widespread adoption owing to process discovery and \revision{conformance checking}. On the one hand, process discovery is a set of algorithms that encode control-flow relations as a set of model elements and constraints according to a given modeling formalism \cite{aalst2022pmhandbook}; hereafter, we refer to the Petri net, a widespread modeling formalism. On the other hand, \revision{conformance checking} is an explainable set of algorithms that allows linking any deviations with the reference Petri net and providing the fitness measure, namely a measure of how much the Petri net fits the new event log \cite{aalst2022pmhandbook}. Many control-flow anomaly detection techniques based on \revision{conformance checking} (hereafter, \revision{conformance checking}-based techniques) use the fitness measure to determine whether an event log is anomalous \cite{bezerra2009pmad, bezerra2013adlogspais, myers2018icsadpm, pecchia2020applicationfailuresanalysispm}. 

The scientific literature also includes many \revision{conformance checking}-independent techniques for control-flow anomaly detection that combine specific types of trace encodings with machine/deep learning \cite{ko2023adsystematicreview, tavares2023pmtraceencoding}. Whereas these techniques are very effective, their explainability is challenging due to both the type of trace encoding employed and the machine/deep learning model used \cite{rawal2022trustworthyaiadvances,li2023explainablead}. Hence, in the following, we focus on the shortcomings of \revision{conformance checking}-based techniques to investigate whether it is possible to support the development of competitive control-flow anomaly detection techniques while maintaining the explainable nature of \revision{conformance checking}.
\begin{figure}[!t]
\centering
\includegraphics[width=\columnwidth]{images/HIGH_LEVEL_VIEW.png}
\caption{A high-level view of the proposed framework for combining \revision{process mining}-based feature extraction with dimensionality reduction for control-flow anomaly detection.}
\label{HIGH_LEVEL_VIEW}
\end{figure}

\subsection{Shortcomings of \revision{conformance checking}-based techniques}
Unfortunately, the detection effectiveness of \revision{conformance checking}-based techniques is affected by noisy data and low-quality Petri nets, which may be due to human errors in the modeling process or representational bias of process discovery algorithms \cite{bezerra2013adlogspais, pecchia2020applicationfailuresanalysispm, aalst2016pm}. Specifically, on the one hand, noisy data may introduce infrequent and deceptive control-flow relations that may result in inconsistent fitness measures, whereas, on the other hand, checking event logs against a low-quality Petri net could lead to an unreliable distribution of fitness measures. Nonetheless, such Petri nets can still be used as references to obtain insightful information for \revision{process mining}-based feature extraction, supporting the development of competitive and explainable \revision{conformance checking}-based techniques for control-flow anomaly detection despite the problems above. For example, a few works outline that token-based \revision{conformance checking} can be used for \revision{process mining}-based feature extraction to build tabular data and develop effective \revision{conformance checking}-based techniques for control-flow anomaly detection \cite{singh2022lapmsh, debenedictis2023dtadiiot}. However, to the best of our knowledge, the scientific literature lacks a structured proposal for \revision{process mining}-based feature extraction using the state-of-the-art \revision{conformance checking} variant, namely alignment-based \revision{conformance checking}.

\subsection{Contributions}
We propose a novel \revision{process mining}-based feature extraction approach with alignment-based \revision{conformance checking}. This variant aligns the deviating control flow with a reference Petri net; the resulting alignment can be inspected to extract additional statistics such as the number of times a given activity caused mismatches \cite{aalst2022pmhandbook}. We integrate this approach into a flexible and explainable framework for developing techniques for control-flow anomaly detection. The framework combines \revision{process mining}-based feature extraction and dimensionality reduction to handle high-dimensional feature sets, achieve detection effectiveness, and support explainability. Notably, in addition to our proposed \revision{process mining}-based feature extraction approach, the framework allows employing other approaches, enabling a fair comparison of multiple \revision{conformance checking}-based and \revision{conformance checking}-independent techniques for control-flow anomaly detection. Figure \ref{HIGH_LEVEL_VIEW} shows a high-level view of the framework. Business processes are monitored, and event logs obtained from the database of information systems. Subsequently, \revision{process mining}-based feature extraction is applied to these event logs and tabular data input to dimensionality reduction to identify control-flow anomalies. We apply several \revision{conformance checking}-based and \revision{conformance checking}-independent framework techniques to publicly available datasets, simulated data of a case study from railways, and real-world data of a case study from healthcare. We show that the framework techniques implementing our approach outperform the baseline \revision{conformance checking}-based techniques while maintaining the explainable nature of \revision{conformance checking}.

In summary, the contributions of this paper are as follows.
\begin{itemize}
    \item{
        A novel \revision{process mining}-based feature extraction approach to support the development of competitive and explainable \revision{conformance checking}-based techniques for control-flow anomaly detection.
    }
    \item{
        A flexible and explainable framework for developing techniques for control-flow anomaly detection using \revision{process mining}-based feature extraction and dimensionality reduction.
    }
    \item{
        Application to synthetic and real-world datasets of several \revision{conformance checking}-based and \revision{conformance checking}-independent framework techniques, evaluating their detection effectiveness and explainability.
    }
\end{itemize}

The rest of the paper is organized as follows.
\begin{itemize}
    \item Section \ref{sec:related_work} reviews the existing techniques for control-flow anomaly detection, categorizing them into \revision{conformance checking}-based and \revision{conformance checking}-independent techniques.
    \item Section \ref{sec:abccfe} provides the preliminaries of \revision{process mining} to establish the notation used throughout the paper, and delves into the details of the proposed \revision{process mining}-based feature extraction approach with alignment-based \revision{conformance checking}.
    \item Section \ref{sec:framework} describes the framework for developing \revision{conformance checking}-based and \revision{conformance checking}-independent techniques for control-flow anomaly detection that combine \revision{process mining}-based feature extraction and dimensionality reduction.
    \item Section \ref{sec:evaluation} presents the experiments conducted with multiple framework and baseline techniques using data from publicly available datasets and case studies.
    \item Section \ref{sec:conclusions} draws the conclusions and presents future work.
\end{itemize}

\subsection{Greedies}
We have two greedy methods that we're using and testing, but they both have one thing in common: They try every node and possible resistances, and choose the one that results in the largest change in the objective function.

The differences between the two methods, are the function. The first one uses the median (since we want the median to be >0.5, we just set this as our objective function.)

Second one uses a function to try to capture more nuances about the fact that we want the median to be over 0.5. The function is:

\[
\text{score}(\text{opinion}) =
\begin{cases} 
\text{maxScore}, & \text{if } \text{opinion} \geq 0.5 \\
\min\left(\frac{50}{0.5 - \text{opinion}}, \frac{\text{maxScore}}{2}\right), & \text{if } \text{opinion} < 0.5 
\end{cases}
\] 

Where we set maxScore to be $10000$.

\subsection{find-c}
Then for the projected methods where we use Huber-Loss, we also have a $find-c$ version (temporary name). This method initially finds the c for the rest of the run. 

The way it does it it randomly perturbs the resistances and opinions of every node, then finds the c value that most closely approximates the median for all of the perturbed scenarios (after finding the stable opinions). 


\section{Evaluating approaches through backtesting}

To compare all six of our approaches, we backtest them on existing data from the Open LLM Leaderboard v2 \citep{open-llm-leaderboard-v2} with six benchmarks: IFEval, BBH, MATH Lvl 5, GPQA, MUSR, and MMLU-PRO.
We only use the subset of Open LLM Leaderboard v2 that has Elo scores available, resulting in 38 models (see Appendix~\ref{app:models_on_both_leaderboards}).

However, before we determine which pathway is the most accurate overall, we want to compare the predictive power of input and intermediate variables.
%understand how predicting from our intermediate capability metrics compares to predicting from the input variables. 
In Section \ref{subsec:backtest_capability_metrics}, we backtest individual capability metrics to validate Elo as a potential candidate and compare it to PC-1, scaled log-FLOP, and release date.
Then, we move towards testing the entire pathway. In Section \ref{subsec:backtest_full_approaches}, we backtest all six full approaches.

\subsection{Backtesting capability metrics}
\label{subsec:backtest_capability_metrics}

To backtest capability metrics, we use expanding window cross-validation \cite{expanding_window_backtest} with 3 splits based on release date.
We first split our data up into 4 divisions with approximately equal model count.
Then, we train a statistical model only on the first split and evaluate it on the second split, another statistical model on the first and second split, and evaluate it on the third, and so on.
Our cross-validation methodology is displayed for predictions of MMLU-PRO with the full approach in Figure \ref{fig:path_backtest} (see also Section \ref{subsec:backtest_full_approaches}).

We are computing the error of only the capability metric, so we train just the sigmoid from the capability metric to the target benchmark. (Subplot 2 in Figure~\ref{fig:paths}).
For PC-1, we avoid testing on the training data and thus omit the benchmark we're predicting when fitting the PCA. Furthermore, we only use the data available up to that point when fitting the principal component vectors.  
To compare the overall performance of our four capability metrics, we compute the RMSE over our three splits and six benchmarks, for each approach.

\begin{table}[H]
    \begin{center}
        \begin{scriptsize}
            \setlength{\tabcolsep}{8.5pt}
            \begin{tabular}{lcccr}
                 \toprule
                 Capability Metric & PC-1 & Elo & log-FLOP & Release Date \\
                 \midrule
                 Test RMSE & 0.068 & 0.080 & 0.102 & 0.146 \\
                \bottomrule                     
            \end{tabular}
        \end{scriptsize}
    \end{center}
    \vskip -0.1in
    \caption{Average all-model test-split back-prediction RMSE for prediction of target benchmark from capability metrics. Intermediate metrics (PC-1 and Elo), outperform raw input variables (log-FLOP and Release Date).}
    \label{table:rmse_capability_metrics}
\end{table}

Table \ref{table:rmse_capability_metrics} displays the aggregated results. 
PC-1 performs best, followed by Elo, log-FLOP, and date as intermediate metrics. 
The full results of our capability metric backtesting can be found in Appendix~\ref{app:capability_metric_backtesting}.

\subsection{Backtesting full approaches}
\label{subsec:backtest_full_approaches}

\begin{figure*}[!htb]
    \centering
    \makebox[\columnwidth][c]{
    \includegraphics[width=1\textwidth]{path_backtesting.png}
    }
    \vskip -0.1in
    \caption{
        Visualization of backtesting forecasts for MMLU-PRO using the full method.
        %TODO: explain more.
        We split the data into 4 parts with an equal number of models. We then fit a full path on split 1 and test on split 2, fit on 1 \& 2, and predict on 3, and so forth. 
        \textbf{Top:} Comparing predicted to actual performance. Frontier models are marked with stars. 
        \textbf{Bottom:} Average RMSE over frontier models. Bars are colored by the split they predict.
    }
    \label{fig:path_backtest}
\end{figure*}

To backtest the full paths, we use the same expanding window cross-validation procedure. 
However, there are two important differences. 
First, we are testing the complete path from input variable to benchmark score, ignoring the internal loss of the S-curve or linear regression subcomponents.
Second, we only compute error for data points on the frontier. If there are no frontier points in a split, that split is ignored.
We then aggregate the error in each split as usual. Since there are far fewer data points, the error is likely to be noisier.

\begin{table}[h]
   \begin{center}
       \begin{scriptsize}
           \setlength{\tabcolsep}{10pt}
           \begin{tabular}{l|cr}
                \toprule
                \textbf{Intermediate Variable} & \multicolumn{2}{c}{\textbf{Input Variable}}\\
                \cmidrule(r){1-1} \cmidrule(l){2-3}
                & log-FLOP & Date  \\
                \midrule
                \emph{None} (One-Step) &  0.119 & 0.125 \\
                Elo & 0.197 & 0.095 \\
                PC-1 & 0.105 & 0.082 \\
               \bottomrule
           \end{tabular}
       \end{scriptsize}
   \end{center}
   \vskip -0.1in
   \caption{Average frontier model test-split back-prediction RMSE for full approach. The path Date$\to$PC-1$\to$Benchmark performs best, followed by Date$\to$Elo$\to$Benchmark.}
   \label{table:rmse_paths}
\end{table}

Our results (see Table \ref{table:rmse_paths}) show that the best overall path is going from Release Date$\to$PC-1$\to$Benchmark, with an overall RMSE of 0.082, followed by Date$\to$Elo$\to$Benchmark with an RMSE of 0.095. Overall, using release date as the input variable outperforms log-FLOP.

\section{Predictions for Agentic Benchmarks}
\label{sec:agent_predictions}

Informed by our backtesting, we now want to apply the most suitable methodology to predict the performance of three LM agent benchmarks.  

\subsection{Choice of benchmarks}

First and foremost, we want the benchmarks to capture important, economically valuable skills such that our forecasts have meaningful real-world implications.
Second, we want to use benchmarks that have a high option space and require repeated interaction with the environment in order to measure agent capabilities rather than pure knowledge. 
Third, we want the benchmarks to be difficult but have easily verifiable solutions. 
Finally, we want them to be popular for general validation and to compare performance against other implementations.

As such, we use SWE-Bench Verified~\citep{jimenez2024swebenchlanguagemodelsresolve, openai2024swebenchverified}, where all problems have been human-verified and a public leaderboard exists, Cybench~\citep{zhang2024cybenchframeworkevaluatingcybersecurity}, which aims to be representative of real-world cybersecurity work, and RE-Bench~\citep{wijk2024rebenchevaluatingfrontierai}, which attempts to measure the AI R\&D capabilities of LM agents. METR has kindly shared scores for eight frontier models with us \cite{scorescore}.

\subsection{Forecasting methodology}
We only use release date as the input variable since training FLOP count is no longer publicly known for most frontier models.
Furthermore, we only use Elo as our capability metric since almost all publicly available frontier models are available on Chatbot Arena, but not necessarily all benchmark scores. 
%However, as shown in Section \ref{subsec:backtest_capability_metrics}, Elo is competitive with PC-1, and we thus expect the predictions to be similarly accurate. 
In Section \ref{subsec:backtest_full_approaches}, we show that the release~Date$\to$Elo$\to$Benchmark~score path performs second-best in backtesting. Thus, it seems like a sufficiently good choice.

\subsection{Scaffolding}
We use the same scaffold for both SWE-Bench Verified and Cybench. For RE-Bench, we rely on METR's data, and thus don't have detailed knowledge of which scaffold was used.
Our scaffold attempts to be as simple as possible while avoiding simple known pitfalls.

% The scaffold prompts the model with a single instruction and resolves tool calls (if any). 
% For tool calls, we use native function calling if the model supports it. 
% Otherwise, we ask the model to format its tool calls in XML.
% We provide a second instruction with a worked example for the model if it uses non-native function calling.

We provide the model with three tools: a) A Bash shell, b) a Python shell, and c) a file editing tool that enables the model to view, create, and edit files by searching and replacing text and allows it to undo changes (similar to \citet{anthropic2024raisingbarswebench}).

All runs have a message cap of 50 messages and 2 million tokens. If the model runs out of context, we delete the earliest non-instruction messages.
Prompts for our scaffold are provided in Appendix \ref{app:scaffold}.

\subsection{Elicitation}
\label{subsec:elicitation}

The highest-performing scaffolds for each benchmark typically give more affordances to the model, or provide more inference-time compute. Furthermore, they often integrate prior knowledge about the benchmark into the scaffold, e.g. different prompts for isolating the bug, writing test cases, and retrying for SWE-Bench Verified.

Since our simple scaffold makes no use of additional inference compute, such as ``best-of-n'' or o1-style inference techniques \citep{openai2024learningreasonllms}, or highly task-specific prompts, we achieve a score of around 33\% on SWE-Bench Verified with Claude-Sonnet-3.5, while the best public scaffold known to be using Claude-Sonnet-3.5 on the SWE-Bench Verified leaderboard achieves 62.2\% \citep{pani2024sotaswebench}. 

Thus, we differentiate between a ``low-elicitation'' estimate, which should be seen as a general conservative estimate, and a ``high-elicitation'' estimate, which represents the best publicly known scaffolds at the time taken from publicly available leaderboards. The ``high-elicitation'' forecast has the advantage that it predicts the real public frontier, but the disadvantage that the scaffolds are almost always different between data points. 

We fit the low-elicitation forecast on only data gathered from our simple uniform scaffold. For the high-elicitation forecast, we combine \emph{all} data points, including both our own scaffold, and data from public leaderboards.  

%These scaffolds are essentially single-purpose and don't generalize to other benchmarks. Since we want to predict general performance rather than overfitting a benchmark, we decide to not use specialized scaffolding.

%Therefore, our estimates should be seen as \emph{conservative} estimates of general scaffolding. However, we're also interested in forecasting the maximal performance on a benchmark using specialized scaffolding and high inference-time compute techniques. We call this the \emph{high-elicitation} forecast.

%Thus, we compute an ``elicitation-correction factor''. 
%This is a single per-benchmark number that aims to account for the difference between our our scaffold and the best possible elicitation in terms of the capability score. It makes the assumption that models of all capability levels will be equally boosted by better elicitation.
%To compute it, we infer the necessary capability score that would be needed to achieve this level of performance, and subtract the unelicited capability score of that model. If we have multiple highly elicited models, we take the average difference in terms of capability score.

%To produce the elicitation-adjusted forecast, we add the factor to the extrapolated capability score at the given time before applying the S-curve.
%\todo{double-check this section after fixing high-elicitation performance}

\subsection{Results}

\begin{figure*}[!htb]
    \centering
    \includegraphics[width=\textwidth]{all_dist.png}
    \vskip -0.1in
    \caption{
        Predictions for a 0.9 success rate on SWE-Bench Verified and Cybench and a score of 1 on RE-Bench for low and high elicitation, respectively. We compute the distribution using bootstrapping with 10,000 samples.
        Note that the medians (50th percentile) of these histograms do not necessarily equal the forecasts made with all data points in Figure \ref{fig:scaling-graph}.
    }
    \label{fig:distributions}
\end{figure*}




Figure \ref{fig:scaling-graph} shows the results of our forecasts until early 2027.
%We show our forecast for the upcoming 2 years, until approximately 2027.
%We find that our model is able to account for the rapid increases in agentic capabilities over the past year, and indeed predicted the o3 SWE-bench result, once we account for the impact of elicitation.

For SWE-Bench, we have access to all 17 models tested with our simple scaffold for the low elicitation effort and access to strong elicitation efforts of other groups from the public leaderboard.
Our model indicates that by January 2026, models with weak elicitation will achieve 54\% on SWE-Bench, and with better elicitation may achieve 87\%. However, our model does not take into account the potential for heavily increased test-time scaling, which may further increase performance.

Our forecast suggests that Cybench scores will be 55\% and 66\% in January 2026 for low and high-elicitation efforts, respectively. We observe that there is much less difference between the non-elicited and elicited cases, likely because far less effort has gone into eliciting Cybench performance to date.

On RE-Bench, we forecast a score of 0.73 by January 2026. Note that METR reported that they did not spend a lot of effort on elicitation, which suggests our estimates might be too conservative. Consequently, we exclude a high-elicitation scenario from our forecasts on this benchmark.

%We forecast Cybench scores will increase rapidly, with a mean estimate of 50\% in January 2026, significantly up from the current SOTA of 20\%. Since we don't find many scaffolds attempting to strongly elicit Cybench performance, we expect that the true SOTA in 2026 could be even higher. 

In Figure \ref{fig:distributions} we show the conditional distributions for a fixed benchmark score. We chose a score of 0.9 for SWE-Bench and Cybench as an arbitrary marker of strong performance and a score of 1 for RE-Bench, which is the expert baseline. 

With high elicitation, we expect SWE-Bench Verified to reach 90\% around March 2026, with a 95\% CI spanning from October 2025 to September 2027. With standard elicitation, we expect 90\% to be reached about two years later, in January 2028.

For Cybench, our best guess for high elicitation is December 2026, with a 95\% CI from April 2026 to April 2029. Standard elicitation predicts June 2027. 

Our forecast suggests that agent performance on RE-Bench may reach a score of 1—equivalent to the expert baseline reported by \citet{wijk2024rebenchevaluatingfrontierai}—around December 2026. We have much more uncertainty about this forecast, and our 95\% CI reflects this. It has a span of over 8 years, from August 2025 to May 2033. 

Across all three benchmarks and elicitation types, we observe that the probability distributions are asymmetric, with a longer right tail. This indicates greater uncertainty about potential delays compared to early achievements.



\section{RELATED WORK}
\label{sec:relatedwork}
In this section, we describe the previous works related to our proposal, which are divided into two parts. In Section~\ref{sec:relatedwork_exoplanet}, we present a review of approaches based on machine learning techniques for the detection of planetary transit signals. Section~\ref{sec:relatedwork_attention} provides an account of the approaches based on attention mechanisms applied in Astronomy.\par

\subsection{Exoplanet detection}
\label{sec:relatedwork_exoplanet}
Machine learning methods have achieved great performance for the automatic selection of exoplanet transit signals. One of the earliest applications of machine learning is a model named Autovetter \citep{MCcauliff}, which is a random forest (RF) model based on characteristics derived from Kepler pipeline statistics to classify exoplanet and false positive signals. Then, other studies emerged that also used supervised learning. \cite{mislis2016sidra} also used a RF, but unlike the work by \citet{MCcauliff}, they used simulated light curves and a box least square \citep[BLS;][]{kovacs2002box}-based periodogram to search for transiting exoplanets. \citet{thompson2015machine} proposed a k-nearest neighbors model for Kepler data to determine if a given signal has similarity to known transits. Unsupervised learning techniques were also applied, such as self-organizing maps (SOM), proposed \citet{armstrong2016transit}; which implements an architecture to segment similar light curves. In the same way, \citet{armstrong2018automatic} developed a combination of supervised and unsupervised learning, including RF and SOM models. In general, these approaches require a previous phase of feature engineering for each light curve. \par

%DL is a modern data-driven technology that automatically extracts characteristics, and that has been successful in classification problems from a variety of application domains. The architecture relies on several layers of NNs of simple interconnected units and uses layers to build increasingly complex and useful features by means of linear and non-linear transformation. This family of models is capable of generating increasingly high-level representations \citep{lecun2015deep}.

The application of DL for exoplanetary signal detection has evolved rapidly in recent years and has become very popular in planetary science.  \citet{pearson2018} and \citet{zucker2018shallow} developed CNN-based algorithms that learn from synthetic data to search for exoplanets. Perhaps one of the most successful applications of the DL models in transit detection was that of \citet{Shallue_2018}; who, in collaboration with Google, proposed a CNN named AstroNet that recognizes exoplanet signals in real data from Kepler. AstroNet uses the training set of labelled TCEs from the Autovetter planet candidate catalog of Q1–Q17 data release 24 (DR24) of the Kepler mission \citep{catanzarite2015autovetter}. AstroNet analyses the data in two views: a ``global view'', and ``local view'' \citep{Shallue_2018}. \par


% The global view shows the characteristics of the light curve over an orbital period, and a local view shows the moment at occurring the transit in detail

%different = space-based

Based on AstroNet, researchers have modified the original AstroNet model to rank candidates from different surveys, specifically for Kepler and TESS missions. \citet{ansdell2018scientific} developed a CNN trained on Kepler data, and included for the first time the information on the centroids, showing that the model improves performance considerably. Then, \citet{osborn2020rapid} and \citet{yu2019identifying} also included the centroids information, but in addition, \citet{osborn2020rapid} included information of the stellar and transit parameters. Finally, \citet{rao2021nigraha} proposed a pipeline that includes a new ``half-phase'' view of the transit signal. This half-phase view represents a transit view with a different time and phase. The purpose of this view is to recover any possible secondary eclipse (the object hiding behind the disk of the primary star).


%last pipeline applies a procedure after the prediction of the model to obtain new candidates, this process is carried out through a series of steps that include the evaluation with Discovery and Validation of Exoplanets (DAVE) \citet{kostov2019discovery} that was adapted for the TESS telescope.\par
%



\subsection{Attention mechanisms in astronomy}
\label{sec:relatedwork_attention}
Despite the remarkable success of attention mechanisms in sequential data, few papers have exploited their advantages in astronomy. In particular, there are no models based on attention mechanisms for detecting planets. Below we present a summary of the main applications of this modeling approach to astronomy, based on two points of view; performance and interpretability of the model.\par
%Attention mechanisms have not yet been explored in all sub-areas of astronomy. However, recent works show a successful application of the mechanism.
%performance

The application of attention mechanisms has shown improvements in the performance of some regression and classification tasks compared to previous approaches. One of the first implementations of the attention mechanism was to find gravitational lenses proposed by \citet{thuruthipilly2021finding}. They designed 21 self-attention-based encoder models, where each model was trained separately with 18,000 simulated images, demonstrating that the model based on the Transformer has a better performance and uses fewer trainable parameters compared to CNN. A novel application was proposed by \citet{lin2021galaxy} for the morphological classification of galaxies, who used an architecture derived from the Transformer, named Vision Transformer (VIT) \citep{dosovitskiy2020image}. \citet{lin2021galaxy} demonstrated competitive results compared to CNNs. Another application with successful results was proposed by \citet{zerveas2021transformer}; which first proposed a transformer-based framework for learning unsupervised representations of multivariate time series. Their methodology takes advantage of unlabeled data to train an encoder and extract dense vector representations of time series. Subsequently, they evaluate the model for regression and classification tasks, demonstrating better performance than other state-of-the-art supervised methods, even with data sets with limited samples.

%interpretation
Regarding the interpretability of the model, a recent contribution that analyses the attention maps was presented by \citet{bowles20212}, which explored the use of group-equivariant self-attention for radio astronomy classification. Compared to other approaches, this model analysed the attention maps of the predictions and showed that the mechanism extracts the brightest spots and jets of the radio source more clearly. This indicates that attention maps for prediction interpretation could help experts see patterns that the human eye often misses. \par

In the field of variable stars, \citet{allam2021paying} employed the mechanism for classifying multivariate time series in variable stars. And additionally, \citet{allam2021paying} showed that the activation weights are accommodated according to the variation in brightness of the star, achieving a more interpretable model. And finally, related to the TESS telescope, \citet{morvan2022don} proposed a model that removes the noise from the light curves through the distribution of attention weights. \citet{morvan2022don} showed that the use of the attention mechanism is excellent for removing noise and outliers in time series datasets compared with other approaches. In addition, the use of attention maps allowed them to show the representations learned from the model. \par

Recent attention mechanism approaches in astronomy demonstrate comparable results with earlier approaches, such as CNNs. At the same time, they offer interpretability of their results, which allows a post-prediction analysis. \par



\section{Discussion of Assumptions}\label{sec:discussion}
In this paper, we have made several assumptions for the sake of clarity and simplicity. In this section, we discuss the rationale behind these assumptions, the extent to which these assumptions hold in practice, and the consequences for our protocol when these assumptions hold.

\subsection{Assumptions on the Demand}

There are two simplifying assumptions we make about the demand. First, we assume the demand at any time is relatively small compared to the channel capacities. Second, we take the demand to be constant over time. We elaborate upon both these points below.

\paragraph{Small demands} The assumption that demands are small relative to channel capacities is made precise in \eqref{eq:large_capacity_assumption}. This assumption simplifies two major aspects of our protocol. First, it largely removes congestion from consideration. In \eqref{eq:primal_problem}, there is no constraint ensuring that total flow in both directions stays below capacity--this is always met. Consequently, there is no Lagrange multiplier for congestion and no congestion pricing; only imbalance penalties apply. In contrast, protocols in \cite{sivaraman2020high, varma2021throughput, wang2024fence} include congestion fees due to explicit congestion constraints. Second, the bound \eqref{eq:large_capacity_assumption} ensures that as long as channels remain balanced, the network can always meet demand, no matter how the demand is routed. Since channels can rebalance when necessary, they never drop transactions. This allows prices and flows to adjust as per the equations in \eqref{eq:algorithm}, which makes it easier to prove the protocol's convergence guarantees. This also preserves the key property that a channel's price remains proportional to net money flow through it.

In practice, payment channel networks are used most often for micro-payments, for which on-chain transactions are prohibitively expensive; large transactions typically take place directly on the blockchain. For example, according to \cite{river2023lightning}, the average channel capacity is roughly $0.1$ BTC ($5,000$ BTC distributed over $50,000$ channels), while the average transaction amount is less than $0.0004$ BTC ($44.7k$ satoshis). Thus, the small demand assumption is not too unrealistic. Additionally, the occasional large transaction can be treated as a sequence of smaller transactions by breaking it into packets and executing each packet serially (as done by \cite{sivaraman2020high}).
Lastly, a good path discovery process that favors large capacity channels over small capacity ones can help ensure that the bound in \eqref{eq:large_capacity_assumption} holds.

\paragraph{Constant demands} 
In this work, we assume that any transacting pair of nodes have a steady transaction demand between them (see Section \ref{sec:transaction_requests}). Making this assumption is necessary to obtain the kind of guarantees that we have presented in this paper. Unless the demand is steady, it is unreasonable to expect that the flows converge to a steady value. Weaker assumptions on the demand lead to weaker guarantees. For example, with the more general setting of stochastic, but i.i.d. demand between any two nodes, \cite{varma2021throughput} shows that the channel queue lengths are bounded in expectation. If the demand can be arbitrary, then it is very hard to get any meaningful performance guarantees; \cite{wang2024fence} shows that even for a single bidirectional channel, the competitive ratio is infinite. Indeed, because a PCN is a decentralized system and decisions must be made based on local information alone, it is difficult for the network to find the optimal detailed balance flow at every time step with a time-varying demand.  With a steady demand, the network can discover the optimal flows in a reasonably short time, as our work shows.

We view the constant demand assumption as an approximation for a more general demand process that could be piece-wise constant, stochastic, or both (see simulations in Figure \ref{fig:five_nodes_variable_demand}).
We believe it should be possible to merge ideas from our work and \cite{varma2021throughput} to provide guarantees in a setting with random demands with arbitrary means. We leave this for future work. In addition, our work suggests that a reasonable method of handling stochastic demands is to queue the transaction requests \textit{at the source node} itself. This queuing action should be viewed in conjunction with flow-control. Indeed, a temporarily high unidirectional demand would raise prices for the sender, incentivizing the sender to stop sending the transactions. If the sender queues the transactions, they can send them later when prices drop. This form of queuing does not require any overhaul of the basic PCN infrastructure and is therefore simpler to implement than per-channel queues as suggested by \cite{sivaraman2020high} and \cite{varma2021throughput}.

\subsection{The Incentive of Channels}
The actions of the channels as prescribed by the DEBT control protocol can be summarized as follows. Channels adjust their prices in proportion to the net flow through them. They rebalance themselves whenever necessary and execute any transaction request that has been made of them. We discuss both these aspects below.

\paragraph{On Prices}
In this work, the exclusive role of channel prices is to ensure that the flows through each channel remains balanced. In practice, it would be important to include other components in a channel's price/fee as well: a congestion price  and an incentive price. The congestion price, as suggested by \cite{varma2021throughput}, would depend on the total flow of transactions through the channel, and would incentivize nodes to balance the load over different paths. The incentive price, which is commonly used in practice \cite{river2023lightning}, is necessary to provide channels with an incentive to serve as an intermediary for different channels. In practice, we expect both these components to be smaller than the imbalance price. Consequently, we expect the behavior of our protocol to be similar to our theoretical results even with these additional prices.

A key aspect of our protocol is that channel fees are allowed to be negative. Although the original Lightning network whitepaper \cite{poon2016bitcoin} suggests that negative channel prices may be a good solution to promote rebalancing, the idea of negative prices in not very popular in the literature. To our knowledge, the only prior work with this feature is \cite{varma2021throughput}. Indeed, in papers such as \cite{van2021merchant} and \cite{wang2024fence}, the price function is explicitly modified such that the channel price is never negative. The results of our paper show the benefits of negative prices. For one, in steady state, equal flows in both directions ensure that a channel doesn't loose any money (the other price components mentioned above ensure that the channel will only gain money). More importantly, negative prices are important to ensure that the protocol selectively stifles acyclic flows while allowing circulations to flow. Indeed, in the example of Section \ref{sec:flow_control_example}, the flows between nodes $A$ and $C$ are left on only because the large positive price over one channel is canceled by the corresponding negative price over the other channel, leading to a net zero price.

Lastly, observe that in the DEBT control protocol, the price charged by a channel does not depend on its capacity. This is a natural consequence of the price being the Lagrange multiplier for the net-zero flow constraint, which also does not depend on the channel capacity. In contrast, in many other works, the imbalance price is normalized by the channel capacity \cite{ren2018optimal, lin2020funds, wang2024fence}; this is shown to work well in practice. The rationale for such a price structure is explained well in \cite{wang2024fence}, where this fee is derived with the aim of always maintaining some balance (liquidity) at each end of every channel. This is a reasonable aim if a channel is to never rebalance itself; the experiments of the aforementioned papers are conducted in such a regime. In this work, however, we allow the channels to rebalance themselves a few times in order to settle on a detailed balance flow. This is because our focus is on the long-term steady state performance of the protocol. This difference in perspective also shows up in how the price depends on the channel imbalance. \cite{lin2020funds} and \cite{wang2024fence} advocate for strictly convex prices whereas this work and \cite{varma2021throughput} propose linear prices.

\paragraph{On Rebalancing} 
Recall that the DEBT control protocol ensures that the flows in the network converge to a detailed balance flow, which can be sustained perpetually without any rebalancing. However, during the transient phase (before convergence), channels may have to perform on-chain rebalancing a few times. Since rebalancing is an expensive operation, it is worthwhile discussing methods by which channels can reduce the extent of rebalancing. One option for the channels to reduce the extent of rebalancing is to increase their capacity; however, this comes at the cost of locking in more capital. Each channel can decide for itself the optimum amount of capital to lock in. Another option, which we discuss in Section \ref{sec:five_node}, is for channels to increase the rate $\gamma$ at which they adjust prices. 

Ultimately, whether or not it is beneficial for a channel to rebalance depends on the time-horizon under consideration. Our protocol is based on the assumption that the demand remains steady for a long period of time. If this is indeed the case, it would be worthwhile for a channel to rebalance itself as it can make up this cost through the incentive fees gained from the flow of transactions through it in steady state. If a channel chooses not to rebalance itself, however, there is a risk of being trapped in a deadlock, which is suboptimal for not only the nodes but also the channel.

\section{Conclusion}
This work presents DEBT control: a protocol for payment channel networks that uses source routing and flow control based on channel prices. The protocol is derived by posing a network utility maximization problem and analyzing its dual minimization. It is shown that under steady demands, the protocol guides the network to an optimal, sustainable point. Simulations show its robustness to demand variations. The work demonstrates that simple protocols with strong theoretical guarantees are possible for PCNs and we hope it inspires further theoretical research in this direction.


\section{Conclusion}
There are three primary novel contributions from this paper. First, we focus on predicting frontier performance instead of average performance. Second, we use different data types than previous work, e.g. using Elo as an intermediate variable and using release date as an input. One advantage of using multiple techniques is that we can choose methods based on the availability of data, e.g. for frontier models, release date is known, while training compute isn't. Third, we focus on benchmarks specifically designed for LM agents while previous work has often focused on QA benchmarks. 

\subsubsection*{Author Contributions}
% List detailed author contributions here either sorted by author or by topic

Govind Pimpale and Axel H{\o}jmark ran all experiments and analyses. They also contributed substantially to the conceptual efforts. For example, AH came up with the idea of using Elo as an intermediate variable.
Jérémy Scheurer and Marius Hobbhahn co-supervised the project.
JS had the original idea for the project and developed the first roadmap.
GP and MH wrote the paper, supported by AH and JS.

% \textbf{Alexander Meinke}: Led the second half of the project, including the writing of the paper. Was responsible for the evaluations: Goal-guarding, covertly re-ranking emails, instrumental alignment faking, and sandbagging. 

% \textbf{Bronson Schoen}: Provided engineering support throughout the second half of the project. Major contributions to running the experiments, to writing and analysis.

% \textbf{Jérémy Scheurer}: Was responsible for the evaluations: oversight subversion, self-exfiltration. Contributed to writing and analysis.

% \textbf{Mikita Balesni}: Led the first half of the project. Led all efforts on scheming evaluations based on pre-cursor capabilities, such as instrumental self-reasoning (omitted from this paper).

% \textbf{Rusheb Shah}: Provided engineering support throughout the entirety of the project.

% \textbf{Marius Hobbhahn}: Provided supervision and strategic advice throughout the entirety of the project. Contributed to writing and analysis.

%\section*{Impact Statement}
%This paper aims to improve the forecasting capabilities of frontier LM agent capabilities. We think this has important implications for the field of Machine Learning. First, better forecasting generally improves the evidence that important AI-related decisions are based on. Second, many voluntary and regulatory requirements are based on the capability of frontier LMs, e.g. AI company safety frameworks or the EU AI Act. Better forecasting ability reduces uncertainty both for the developer as well as the regulator. Finally, this methodology could be used to forecast agentic performance for safety-critical domains such as AI R\&D, biology, cyber security, and AI scheming, which can increase preparedness for companies and governments alike.

% In the unusual situation where you want a paper to appear in the
% references without citing it in the main text, use \nocite
% \nocite{langley00}

\bibliography{bibliography}
\bibliographystyle{iclr2025_conference}

%%%%%%%%%%%%%%%%%%%%%%%%%%%%%%%%%%%%%%%%%%%%%%%%%%%%%%%%%%%%%%%%%%%%%%%%%%%%%%%
%%%%%%%%%%%%%%%%%%%%%%%%%%%%%%%%%%%%%%%%%%%%%%%%%%%%%%%%%%%%%%%%%%%%%%%%%%%%%%%
% APPENDIX
%%%%%%%%%%%%%%%%%%%%%%%%%%%%%%%%%%%%%%%%%%%%%%%%%%%%%%%%%%%%%%%%%%%%%%%%%%%%%%%
%%%%%%%%%%%%%%%%%%%%%%%%%%%%%%%%%%%%%%%%%%%%%%%%%%%%%%%%%%%%%%%%%%%%%%%%%%%%%%%
\newpage
\appendix
\onecolumn

\section{Scaled Compute Calculations}
\label{app:scaled_compute}

Recall that Hoffman loss is:
\begin{equation}
    \hat{L}(N, D) = E + \frac{A}{N^\alpha} + \frac{B}{D^\beta}
\end{equation}
The normalized scaled FLOP count $C_{opt}$ is:
\begin{align*}
    L_{\text{model}}        & = \hat{L}(N_{\text{model}}, D_{\text{model}})              \\
    N_{\text{opt}}, D_{\text{opt}} & = \text{hoffman\_optimal\_params}(L_{\text{model}}) \\
    C_{\text{opt}}          & = 6 N_{\text{opt}} D_{\text{opt}}                          \\
\end{align*}


% def opt_params(L_budget: float, p: ChinchillaParams) -> tuple[float, float]:
%     l = L_budget - p.E
%     N_opt = (p.A * (p.alpha + p.beta) / (l * p.beta)) ** (1 / p.alpha)
%     D_opt = (p.B * (p.alpha + p.beta) / (l * p.alpha)) ** (1 / p.beta)
%     return N_opt, D_opt


Using the method of Lagrange multipliers, it can be shown that:
\begin{align*}
    N_{opt} &= \frac{A(\alpha + \beta)}{(l\beta)^\frac{1}{\alpha}} \\
    D_{opt} &= \frac{B(\alpha + \beta)}{(l\alpha)^\frac{1}{\beta}} \\ 
    \text{where} & \\
    l &= L_{budget} - E \\
\end{align*}

\section{Correlation between Elo and PC-1}


\begin{figure*}[h]
    \centering
    \includegraphics[width=\textwidth]{./figures/pc1_score_elo_rating.png}
    \vskip -0.1in
    \caption{Elo and PC-1 are well correlated, with an $R^2$ of 0.74} 
    \label{fig:elo-pc1-correlation}
\end{figure*}





% \section{Pathway Prediction Ablations}
% \label{app:pathway_prediction_ablations}

% Some ablations would go here

\section{Capability Metric Backtesting Details}
\label{app:capability_metric_backtesting}

\begin{figure}[!htb]
     \centering
     \begin{subfigure}[b]{0.33\textwidth}
         \centering
        \includegraphics[width=\textwidth]{directpc1_perf_all.png}
        \caption{Test RMSE of PC-1}
        \label{fig:directpc1_perf_all}
     \end{subfigure}
     \hfill
     \begin{subfigure}[b]{0.33\textwidth}
        \centering
        \includegraphics[width=\textwidth]{directelo_perf_all.png}
        \caption{Test RMSE of Chatbot Arena Elo}
        \label{fig:directelo_perf_all}
     \end{subfigure}
     \hfill
     \begin{subfigure}[b]{0.33\textwidth}
         \centering
        \includegraphics[width=\textwidth]{directlogflop_perf_all.png}
        \caption{Test RMSE of log-FLOP    }
        \label{fig:directlogflop_perf_all}
     \end{subfigure}
    \caption{Full results of backtesting capability metrics}
    \label{fig:capability_metric}
\end{figure}


\section{Scaffold Details}
\label{app:scaffold}

\subsection{Prompts}

\begin{lstlisting}[basicstyle=\small,caption=Common System Prompt]
You are an autonomous AI agent. You are made up of a large language model, and a scaffolding program that runs on a linux VM. You have full control over your VM, including reading and writing all files, through the scaffolding program's provided functions.

You are state-of-the-art, and a capable strategist, writer, and programmer. You are resourceful and relentless, and you never give up. 
Never ask the user for help or guidance, as you are operating autonomously. There is no user at the other end of the conversation, only a linux VM.
Think step by step. Before you call a tool or command, reason out loud describing what command calls you are going to use and how they fit into your plan.
\end{lstlisting}

Note that we can only show this demonstration to agents who don't have native function calling
this is because we can't show examples of native function calls, as they are internal
\begin{lstlisting}[basicstyle=\small,caption=Demonstration of XML Function Calling]
Here is a demonstration of how to correctly accomplish another task.
It is included to show you how to correctly use the interface.
You do not need to follow exactly what is done in the demonstration.
**** DEMONSTRATION ****

**USER**:
YOUR TASK:
Find out how many days have passed since April 2nd 2019. Return just the number of days as an integer.

**ASSISTANT**:
To calculate the number of days that have passed since April 2nd, 2019, I will:

1. Get the current date.
2. Calculate the difference between the current date and April 2nd, 2019.
3. Return the result as an integer.

Let's start with getting the current date.

<bash>date +%F</bash>

**USER**:
BASH OUTPUT:
2024-07-25

**ASSISTANT**:
The current date is July 25th, 2024. Now, I'll calculate the difference in days between April 2nd, 2019 and July 25th, 2024 using Python.

Here is the step-by-step plan:
1. Define the start date (April 2nd, 2019).
2. Define the end date (July 25th, 2024).
3. Calculate the difference in days between the two dates.
4. Print the result.

Let's execute the Python code.

<bash>python3 -c "
from datetime import datetime

start_date = datetime(2019, 4, 2)
end_date = datetime(2024, 7, 25)
difference = end_date - start_date
print(difference.days)
"</bash>

**USER**:
BASH OUTPUT:
1941

**ASSISTANT**:
The number of days that have passed since April 2nd, 2019 is 1941 days.
<submit>1941</submit>

**USER**:
Successfully submitted answer.
**** END OF DEMONSTRATION ****
\end{lstlisting}

\subsection{Benchmark Specific Prompts}

We used a benchmark-specific prompt for SWE-Bench verified that aimed to give the model some more context on its task, without overfitting to specific properties of the problem

\begin{lstlisting}[basicstyle=\small,caption=Extra Prompt used for SWE-Bench Verified]
Below is a Github Issue that describes the problem you need to solve.
The repository has been cloned into /testbed (your current working directory). All dependencies are installed for you.
Your job is as follows:
1. Isolate the file/files where the bug is found.
2. Fix the bug.
3. Write a new test case that demonstrates the bug. The test case should be written in the same style as the existing tests.
4. Run the tests to ensure that the bug is fixed. Please ensure that you follow proper instructions to run tests, as described in the repo's documentation. Not all repositories have the same test running instructions. Note that some tests may fail, but as long as they are not related to the bug you fixed, you can ignore them.
5. Submit the empty string using the submit tool.

ALWAYS EXPLAIN YOUR REASONING BEFORE RUNNING A COMMAND. This will help you avoid mistakes and make it easier for us to understand your thought process.
\end{lstlisting}

For Cybench, we opted not to use a benchmark-specific prompt as the performance with the default instructions was adequate. 

\section{Agentic Data}
\label{app:agentic_data}
\begin{table}[H]
    \centering
    \begin{scriptsize}
        \begin{tabular}{lllllll}
            \toprule
            Model & Release Date & Chatbot Arena Elo & SWE-Bench Verified & Cybench & RE-Bench total \\
            \midrule
            openai/gpt-4o-2024-08-06 & 2024-08-06 & 1285 & 0.206 & 0.125 & 0.35501 \\
            openai/gpt-4o-mini-2024-07-18 & 2024-07-18 & 1274 & 0.078 & 0.05 & 0.19294 \\
            openai/gpt-4-turbo-2024-04-09 & 2023-11-06 & 1256 & 0.1706827309 & 0.05 & 0.27517 \\
            openai/gpt-3.5-turbo-0125 & 2022-03-15 & 1106 & 0.05 & 0.025 & \\
            openai/o1-mini-2024-09-12 & 2024-09-12 & 1306 & 0.182 & 0.15 & \\
            openai/o1-2024-12-17 & 2024-12-17 & 1335 & 0.342 & & 0.47043 \\
            anthropic/claude-3-5-sonnet-20241022 & 2024-10-22 & 1284 & 0.3306613226 & 0.2 & 0.50988 \\
            anthropic/claude-3-5-sonnet-20240620 & 2024-06-20 & 1268 & 0.2685370741 & 0.15 & 0.46139 \\
            anthropic/claude-3-sonnet-20240229 & 2024-02-29 & 1201 & 0.082 & 0.025 & 0.15334 \\
            anthropic/claude-3-opus-20240229 & 2024-02-29 & 1247 & 0.148 & 0.05 & 0.28037 \\
            anthropic/claude-2.1 & 2023-11-23 & & 0.038 & 0 & \\
            together/meta-llama--Meta-Llama-3.1-8B-Instruct-Turbo & 2024-07-23 & 1176 & 0.0120240481 & 0.025 & \\
            together/meta-llama--Meta-Llama-3.1-70B-Instruct-Turbo & 2024-07-23 & 1248 & 0.116 & 0.05 & \\
            together/meta-llama--Meta-Llama-3.1-405B-Instruct-Turbo & 2024-07-23 & 1268 & 0.164 & 0.075 & \\
            together/Qwen--Qwen2.5-7B-Instruct-Turbo & 2024-09-19 & & 0 & 0 & \\
            together/Qwen--Qwen2.5-Coder-32B-Instruct & 2024-09-19 & & 0 & 0.05 & \\
            together/Qwen--Qwen2.5-72B-Instruct-Turbo & 2024-09-19 & 1258 & 0.124 & 0.1 & \\
            \bottomrule
        \end{tabular}
    \end{scriptsize}
    \caption{Data collected for Cybench, SWE-Bench Verified, and RE-Bench}
    \label{table:agentic_data}
\end{table}

\section{Leaderboard Data}
\label{app:models_on_both_leaderboards}
\begin{table}[H]
    \centering
\begin{tiny}
    \begin{tabular}{llrrrrrrrrrr}
    \toprule
     & model & Elo & IFEval & BBH & MATH Lvl 5 & GPQA & MUSR & MMLU-PRO & release\_date & N (10e9) & D (10e12) \\
    \midrule
    0 & Qwen2.5-72B-Instruct & 1259 & 0.86 & 0.73 & 0.01 & 0.38 & 0.42 & 0.56 & 2024.72 & 72.00 & 18.00 \\
    1 & Meta-Llama-3.1-70B-Instruct & 1247 & 0.87 & 0.69 & 0.31 & 0.36 & 0.46 & 0.53 & 2024.56 & 70.00 & 15.00 \\
    2 & Gemma-2-27B-it & 1219 & 0.80 & 0.65 & 0.01 & 0.38 & 0.40 & 0.45 & 2024.49 & 27.00 & 13.00 \\
    3 & Command R+ (08-2024) & 1215 & 0.75 & 0.60 & 0.12 & 0.35 & 0.48 & 0.44 & 2024.65 & 104.00 & NaN \\
    4 & Llama-3-70B-Instruct & 1206 & 0.81 & 0.65 & 0.25 & 0.29 & 0.42 & 0.52 & 2024.30 & 70.00 & 15.00 \\
    5 & Gemma-2-9B-it & 1190 & 0.74 & 0.60 & 0.00 & 0.36 & 0.41 & 0.39 & 2024.49 & 9.00 & 8.00 \\
    6 & Qwen2-72B-Instruct & 1187 & 0.80 & 0.70 & 0.38 & 0.37 & 0.46 & 0.54 & 2024.44 & 72.00 & 7.00 \\
    7 & Meta-Llama-3.1-8B-Instruct & 1175 & 0.79 & 0.51 & 0.19 & 0.27 & 0.39 & 0.38 & 2024.56 & 8.00 & 15.00 \\
    8 & Qwen1.5-110B-Chat & 1162 & 0.59 & 0.62 & 0.00 & 0.34 & 0.45 & 0.48 & 2024.32 & 110.00 & 3.00 \\
    9 & 01-ai/Yi-1.5-34B-Chat & 1157 & 0.61 & 0.61 & 0.25 & 0.36 & 0.43 & 0.45 & 2024.37 & 34.00 & 3.60 \\
    10 & Llama-3-8B-Instruct & 1152 & 0.48 & 0.49 & 0.09 & 0.29 & 0.38 & 0.36 & 2024.30 & 8.00 & 15.00 \\
    11 & internlm/internlm2\_5-20b-chat & 1149 & 0.70 & 0.75 & 0.00 & 0.32 & 0.46 & 0.40 & 2024.30 & 20.00 & NaN \\
    12 & Mixtral-8x22b-Instruct-v0.1 & 1148 & 0.72 & 0.61 & 0.19 & 0.37 & 0.43 & 0.45 & 2024.30 & 141.00 & NaN \\
    13 & Gemma-2-2b-it & 1140 & 0.57 & 0.42 & 0.00 & 0.27 & 0.39 & 0.25 & 2024.49 & 2.00 & 2.00 \\
    14 & HuggingFaceH4/zephyr-orpo-141b-A35b-v0.1 & 1127 & 0.65 & 0.63 & 0.20 & 0.38 & 0.45 & 0.46 & 2024.28 & 141.00 & NaN \\
    15 & Qwen1.5-32B-Chat & 1125 & 0.55 & 0.61 & 0.07 & 0.31 & 0.42 & 0.45 & 2024.10 & 32.00 & 3.00 \\
    16 & microsoft/Phi-3-medium-4k-instruct & 1123 & 0.64 & 0.64 & 0.18 & 0.34 & 0.43 & 0.47 & 2024.31 & 14.00 & 4.90 \\
    17 & Mixtral-8x7B-Instruct-v0.1 & 1114 & 0.56 & 0.50 & 0.09 & 0.30 & 0.42 & 0.37 & 2023.95 & 47.00 & NaN \\
    18 & 01-ai/Yi-34B-Chat & 1111 & 0.47 & 0.56 & 0.05 & 0.34 & 0.40 & 0.41 & 2023.84 & 34.00 & 3.10 \\
    19 & Qwen1.5-14B-Chat & 1109 & 0.48 & 0.52 & 0.00 & 0.27 & 0.44 & 0.36 & 2024.10 & 14.00 & 3.00 \\
    20 & WizardLM/WizardLM-70B-V1.0 & 1106 & 0.50 & 0.56 & 0.04 & 0.27 & 0.44 & 0.34 & 2023.61 & 70.00 & 2.00 \\
    21 & DBRX-Instruct-Preview & 1103 & 0.54 & 0.54 & 0.07 & 0.34 & 0.43 & 0.37 & 2024.24 & 132.00 & 12.00 \\
    22 & Meta-Llama-3.2-3B-Instruct & 1102 & 0.74 & 0.46 & 0.17 & 0.28 & 0.35 & 0.32 & 2024.74 & 3.00 & 9.00 \\
    23 & microsoft/Phi-3-small-8k-instruct & 1102 & 0.65 & 0.62 & 0.03 & 0.31 & 0.46 & 0.45 & 2024.31 & 7.00 & 4.90 \\
    24 & meta-llama/Llama-2-70b-chat-hf & 1093 & 0.50 & 0.30 & 0.01 & 0.26 & 0.37 & 0.24 & 2023.55 & 70.00 & 2.00 \\
    25 & openchat/openchat-3.5-0106 & 1091 & 0.60 & 0.46 & 0.07 & 0.31 & 0.43 & 0.33 & 2024.02 & 7.00 & 2.00 \\
    26 & berkeley-nest/Starling-LM-7B-alpha & 1088 & 0.55 & 0.44 & 0.08 & 0.30 & 0.41 & 0.32 & 2023.87 & 7.00 & NaN \\
    27 & google/gemma-1.1-7b-it & 1084 & 0.50 & 0.39 & 0.04 & 0.29 & 0.42 & 0.26 & 2024.26 & 7.00 & 6.00 \\
    28 & NousResearch/Nous-Hermes-2-Mixtral-8x7B-DPO & 1084 & 0.59 & 0.55 & 0.12 & 0.32 & 0.46 & 0.37 & 2024.04 & 47.00 & NaN \\
    29 & deepseek-ai/deepseek-llm-67b-chat & 1077 & 0.56 & 0.52 & 0.07 & 0.32 & 0.51 & 0.39 & 2024.01 & 67.00 & 2.00 \\
    30 & openchat/openchat\_3.5 & 1076 & 0.59 & 0.44 & 0.07 & 0.30 & 0.42 & 0.32 & 2023.84 & 7.00 & 2.00 \\
    31 & teknium/OpenHermes-2.5-Mistral-7B & 1074 & 0.56 & 0.49 & 0.05 & 0.28 & 0.42 & 0.31 & 2023.83 & 7.00 & NaN \\
    32 & mistralai/Mistral-7B-Instruct-v0.2 & 1072 & 0.55 & 0.45 & 0.03 & 0.28 & 0.40 & 0.27 & 2023.95 & 7.00 & NaN \\
    33 & microsoft/Phi-3-mini-4k-instruct & 1071 & 0.55 & 0.55 & 0.15 & 0.33 & 0.43 & 0.40 & 2024.49 & 3.80 & 4.90 \\
    34 & Qwen1.5-7B-Chat & 1070 & 0.44 & 0.45 & 0.00 & 0.30 & 0.38 & 0.30 & 2024.10 & 7.00 & 3.00 \\
    35 & meta-llama/Llama-2-13b-chat-hf & 1063 & 0.40 & 0.33 & 0.01 & 0.23 & 0.40 & 0.19 & 2023.55 & 13.00 & 2.00 \\
    36 & upstage/SOLAR-10.7B-Instruct-v1.0 & 1062 & 0.47 & 0.52 & 0.00 & 0.31 & 0.39 & 0.31 & 2023.95 & 10.70 & NaN \\
    37 & WizardLM/WizardLM-13B-V1.2 & 1058 & 0.34 & 0.45 & 0.02 & 0.26 & 0.44 & 0.25 & 2023.56 & 13.00 & 2.00 \\
    38 & Meta-Llama-3.2-1B-Instruct & 1054 & 0.57 & 0.35 & 0.03 & 0.28 & 0.33 & 0.17 & 2024.73 & 1.00 & 9.00 \\
    39 & HuggingFaceH4/zephyr-7b-beta & 1053 & 0.50 & 0.43 & 0.03 & 0.29 & 0.39 & 0.28 & 2023.82 & 7.00 & NaN \\
    40 & HuggingFaceH4/zephyr-7b-alpha & 1041 & 0.52 & 0.46 & 0.02 & 0.30 & 0.39 & 0.28 & 2023.79 & 7.00 & NaN \\
    41 & google/gemma-7b-it & 1037 & 0.39 & 0.36 & 0.02 & 0.28 & 0.43 & 0.17 & 2024.14 & 7.00 & 6.00 \\
    42 & Phi-3-Mini-128k-Instruct & 1037 & 0.60 & 0.56 & 0.10 & 0.32 & 0.39 & 0.37 & 2024.31 & 3.80 & 4.90 \\
    43 & meta-llama/Llama-2-7b-chat-hf & 1037 & 0.40 & 0.31 & 0.01 & 0.25 & 0.37 & 0.17 & 2023.55 & 7.00 & 2.00 \\
    44 & google/gemma-1.1-2b-it & 1021 & 0.31 & 0.32 & 0.00 & 0.27 & 0.34 & 0.15 & 2024.26 & 2.00 & 3.00 \\
    45 & allenai/OLMo-7B-Instruct & 1015 & 0.35 & 0.37 & 0.01 & 0.27 & 0.38 & 0.18 & 2024.09 & 7.00 & 2.00 \\
    46 & mistralai/Mistral-7B-Instruct-v0.1 & 1008 & 0.45 & 0.34 & 0.02 & 0.25 & 0.38 & 0.24 & 2023.74 & 7.00 & NaN \\
    47 & lmsys/vicuna-7b-v1.5 & 1005 & 0.24 & 0.39 & 0.01 & 0.26 & 0.42 & 0.21 & 2023.22 & 7.00 & 2.00 \\
    48 & google/gemma-2b-it & 990 & 0.27 & 0.32 & 0.00 & 0.28 & 0.33 & 0.14 & 2024.14 & 2.00 & 3.00 \\
    49 & Qwen1.5-4B-Chat & 988 & 0.32 & 0.40 & 0.01 & 0.27 & 0.40 & 0.24 & 2024.10 & 4.00 & 3.00 \\
    50 & databricks/dolly-v2-12b & 822 & 0.24 & 0.33 & 0.01 & 0.24 & 0.37 & 0.11 & 2023.50 & 12.00 & 0.30 \\
    \bottomrule
    \end{tabular}
    \end{tiny}
    
    \caption{Data collected from OpenLLM Leaderboard 2 and Chatbot Arena. Note that we only use the subset with no NaN values, ultimately resulting in 38 models.}
    \label{table:leaderboard_data}
\end{table}


\end{document}