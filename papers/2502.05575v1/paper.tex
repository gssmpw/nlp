\documentclass[acmsmall]{acmart}

% Essential packages
\usepackage{graphicx}
\usepackage{multirow}
\usepackage{tikz}
\usepackage[ruled,vlined,linesnumbered,noend]{algorithm2e}
%\usepackage{algpseudocode}
\usepackage[labelfont=bf,textfont=bf]{caption}
\usepackage[labelfont=bf,textfont=bf,singlelinecheck=off,justification=raggedright]{subcaption}
\usepackage{amsmath,amsthm,bm}
%\usepackage{array}
%usepackage{url}
%\usepackage{balance}
%\usepackage{longtable}
\usepackage{color}
\definecolor{myred}{RGB}{189, 52, 67}
\definecolor{mygreen}{RGB}{19, 136, 8}
\definecolor{myblue}{RGB}{16, 52, 166}
\newcommand{\here}[1]{{\bf [[#1]]}}
\newcommand{\karima}[1]{{\color{black} #1}\normalcolor}
\newcommand{\ilias}[1]{{\color{mygreen} #1}\normalcolor}
\newcommand{\themis}[1]{{\color{red} #1}\normalcolor}
\newcommand{\tp}[1]{{\color{red} {\bf ??? #1 ???}}\normalcolor}


%%
%% \BibTeX command to typeset BibTeX logo in the docs
\AtBeginDocument{%
  \providecommand\BibTeX{{%
    Bib\TeX}}}

\newtheorem{defn}{Definition}


%\acmISBN{978-1-4503-XXXX-X/18/06}

\graphicspath{{./img-png/}}
%\setcopyright{acmlicensed}
%\acmJournal{PACMMOD}

\setcopyright{authorsopen}
\acmYear{2025} \acmVolume{3} \acmNumber{1 (SIGMOD)} \acmArticle{43} \acmMonth{2}\acmDOI{10.1145/3709693}

\begin{document}
	\title{Graph-Based Vector Search: An Experimental Evaluation of the State-of-the-Art}
	
	\author{Ilias Azizi}
		\affiliation{
		\institution{UM6P, Universit{\'e} Paris Cit{\'e}}
		\country{Morocco - France}
	}
	\email{ilias.azizi@um6p.ma}
	\author{Karima Echihabi}
		\affiliation{%
		\institution{UM6P}
		\country{Morocco}
	}
	\email{karima.echihabi@um6p.ma}
	\author{Themis Palpanas}
		\affiliation{%
		\institution{Universit{\'e} Paris Cit{\'e}}
		\country{France}
	}
	\email{themis@mi.parisdescartes.fr}
\renewcommand{\shortauthors}{Ilias Azizi, Karima Echihabi \& Themis Palpanas.}

    
\pagenumbering{arabic}

\begin{abstract}
Vector data is prevalent across business and scientific applications, and its popularity is growing with the proliferation of learned embeddings. Vector data collections often reach billions of vectors with thousands of dimensions, thus, increasing the complexity of their analysis. Vector search is the backbone of many critical analytical tasks, and graph-based methods have become the best choice for analytical tasks that do not require guarantees on the quality of the answers. We briefly survey in-memory graph-based vector search, outline the chronology of the different methods and classify them according to five main design paradigms: seed selection, incremental insertion, neighborhood propagation, neighborhood diversification, and divide-and-conquer. We conduct an exhaustive experimental evaluation of twelve state-of-the-art methods on seven real data collections, with sizes up to 1 billion vectors. We share key insights about the strengths and limitations of these methods; e.g., the best approaches are typically based on incremental insertion and neighborhood diversification, and the choice of the base graph can hurt scalability. Finally, we discuss open research directions, such as the importance of devising more sophisticated data-adaptive seed selection and diversification strategies.

\end{abstract}


% CCS Concepts
\begin{CCSXML}
<ccs2012>
   <concept>
       <concept_id>10003752.10003809.10003635</concept_id>
       <concept_desc>Theory of computation~Graph algorithms analysis</concept_desc>
       <concept_significance>500</concept_significance>
       </concept>
   <concept>
       <concept_id>10003752.10003809.10010055.10010060</concept_id>
       <concept_desc>Theory of computation~Nearest neighbor algorithms</concept_desc>
       <concept_significance>500</concept_significance>
       </concept>
   <concept>
       <concept_id>10003752.10003809.10010031</concept_id>
       <concept_desc>Theory of computation~Data structures design and analysis</concept_desc>
       <concept_significance>500</concept_significance>
       </concept>
   <concept>
       <concept_id>10002951.10003317</concept_id>
       <concept_desc>Information systems~Information retrieval</concept_desc>
       <concept_significance>500</concept_significance>
       </concept>
   <concept>
       <concept_id>10002944.10011123.10011130</concept_id>
       <concept_desc>General and reference~Evaluation</concept_desc>
       <concept_significance>300</concept_significance>
       </concept>
   <concept>
       <concept_id>10002944.10011123.10011131</concept_id>
       <concept_desc>General and reference~Experimentation</concept_desc>
       <concept_significance>300</concept_significance>
       </concept>
 </ccs2012>
\end{CCSXML}

\ccsdesc[500]{Theory of computation~Graph algorithms analysis}
\ccsdesc[500]{Theory of computation~Nearest neighbor algorithms}
\ccsdesc[500]{Theory of computation~Data structures design and analysis}
\ccsdesc[500]{Information systems~Information retrieval}
\ccsdesc[300]{General and reference~Evaluation}
\ccsdesc[300]{General and reference~Experimentation}

% Keywords
\keywords{Vector similarity search, Approximate nearest neighbor, KNN graph analysis, Seed selection, Neighborhood diversification, Graph algorithms}

%Received July 2024; revised September 2024; accepted November 2024
\received{July 2024}
\received[revised]{September 2024}
\received[accepted]{November 2024}

\maketitle
\documentclass[../main.tex]{subfiles}
\graphicspath{{../images/}}
\makeatletter
\def\input@path{{../images/}}
\makeatother
\begin{document}
\section{Introduction}
\begin{figure}
\centering
\begin{tikzpicture}
\node[inner sep=0pt] (ws) at (0, 0) {
\includegraphics[height=.4\textwidth, trim={10cm 0 10cm 0},clip]{world_space.png}};
\node[inner sep=0pt] (cs) at (6,0) {\includegraphics[height=.4\textwidth, trim={10cm 1cm 10cm 4cm},clip]{conf_space.png}};
\end{tikzpicture}
\vspace{-5pt}
\label{fig:pbrm_intro}
\caption{\textbf{Left}: Shows world space obstacles as grey spheres. Robots start and goal configuration is colored red and green, respectively. Configurations along the computed path are colored transparent blue. \textbf{Right:} Mapped world space scenario to configuration space. Obstacle region is the grey mesh. Red spheres are collision-free regions computed by the neural SCDF. The optimized shortest path in the convex corridor is the blue curve.}
\vspace{-25pt}
\end{figure}
Motion planning is the problem of finding a collision-free trajectory that connects a given start and goal configuration. The planning takes place in the configuration space of the robot. For single body robots, like mobile robots or drones, the configuration space and the world space are usually the same. This simplifies the planning, since explicit obstacle representations are available which enables geometrical tools like separating hyperplanes, smallest distance to obstacles etc., to be used when designing motion planning algorithms. For multi-body robots like manipulators, the situation is completely different. The world space obstacles are usually mapped to non-convex regions, and to make the problem even harder, the mapping is usually not known. Forming explicit representations of the obstacle region in the configuration space is usually too expensive or intractable. Despite all of this, sampling based planners are used with great success, which mainly is due to their use of implicit representations of the obstacle region. The basic idea is to construct a graph in the configuration space that covers and connects the collision-free region. From this graph, a path can be extracted that connects a given start and goal configuration. The approach is computationally expensive, since the graph is constructed with the smallest geometrical building block available, points, which represents a collision-check. Furthermore, the extracted paths from the graph are non-smooth and jagged due to the stochastic nature of the approach. This adds an additional post-processing step to the process, where the paths are shortcutted and smoothened, before the path can be used for tracking. Clearly a lot of time is invested to form this graph and produce smooth paths. Thus, if the obstacles start to move, then all of this work is done in no use, since all points that make up this graph need to be re-verified, which is simply too time consuming to be done in real time.
\\\\
In this work, we want to address the existing drawbacks of the sampling based planners. Our main contribution is an improved motion planner where each vertex in the graph covers a collision-free region in the form of a sphere instead of a point and where the edges are formed with neighboring intersecting spheres. This representation has the advantage of instead of returning piecewise linear paths, returning a sequence of overlapping spheres, i.e. a convex corridor, that connects a given start and goal configuration, illustrated in Figure \ref{fig:pbrm_intro}. This convex corridor allows us to use convex optimization to produce smooth trajectories, instead of computationally expensive post-processing methods. The representation further allows us to estimate the coverage of the collision-free space, which gives us awareness and feedback in the offline roadmap construction phase. Finally, our representation is simple to adapt to moving obstacles, simply requery for the new radii and recheck for intersections. 
\\\\
The spherical collision-free regions are formed using a signed distance function (SDF), which is a function that returns the smallest distance from an arbitrary point to the boundary of an obstacle. As the name implies, the distance is signed, thus if the point is inside the obstacle it is negative otherwise positive. If the distance is positive, a sphere with radius equal to the distance is guaranteed to cover a collision-free region. Using an SDF in motion planning is not new, but what is novel about our approach is that we express the distance in the configuration space instead of the world space and by doing so allows us to form these convex collision-free regions. We refer to the resulting SDF as a signed configuration distance function (SCDF). Computing an SCDF analytically is non-trivial, our approach is therefore to parameterize the SCDF with a deep neural network and learn the mapping by supervised learning. Our resulting neural SCDF can compute distances for different parameter values of obstacle shapes and we also show how multiple distances can be combined, thus making our approach flexible.
\section{Related work}
Motion planning algorithms can roughly be divided into three families, grid-based, sampling based and optimization based methods. Grid-based methods (GBM) discretize the planning space from which a graph is then compiled. A standard search method is A$^\star$ \citep{a_star}, which is classified as an \textit{informed} search method, since it employs a heuristic function to speed up the search. A$^\star$ guarantees to return an optimal path at the level of discretization used. GBMs usually discretize the planning space by a regular lattice and this limits the GBMs to problems with low dimensionality due to the curse of dimensionality. Thus, GBMs are usually limited to single-body robots where the degrees of freedom (DOF) are low. To overcome the inherent scaling problem with the GBMs, stochastic methods are usually used for multi-body robots. These methods are termed as sampling-based methods (SBM) and core members within this family are the rapidly-exploring random trees (RRT) \citep{rrt} and the probabilistic roadmap (PRM) \citep{prm}. RRT grows a tree from the start configuration and explores the collision-free region in a rapid way until it is able to connect to the goal region. RRT is usually improved by bi-directional planning \citep{rrt_connect}, i.e. an additional tree is grown from the goal configuration and the trees are tested for connection after any tree has been expanded. RRT is a single-query method, thus it searches for a path from scratch each time it is queried. Contrary to this, PRM is a multi-query method, which solves for multiple queries without starting from scratch. PRM does this by creating a roadmap (graph) that covers the collision-free space as an offline step. The graph is then used to solve for multiple queries. PRMs are used in cases where the environment does not change since the extra offline step is too computationally costly and needs to be re-done if the environment is changed. In our work, we address this inherent issue by using a different roadmap representation. Our vertices in the graph cover a collision-free region in the form of spheres and we form the edges by checking for intersecting spheres. If something in the environment changes, we recompute the spheres radii and recheck the intersections, without relying on collision detection. We use a trained neural network to compute the sphere radius, therefore querying for the radius can be done fast, hence our representation enables the PRM for dynamic environments.
\\\\
In the recent decades, optimization based methods (OBM) \citep{chomp, schulman, itomp, stomp} have been introduced as an alternative to SBM for multi-body robots. Like the SBM, the OBMs scale well to higher dimensional problems and produce smoother motion. It is common to use a SDF in the optimization since it is a smooth function, thus enabling gradient-based methods. However, the standard way of expressing the SDF is in world space. The distance therefore needs to be mapped to the configuration space by the forward kinematics. This mapping makes the optimization problem a non-linear program (NLP), which is computationally expensive to solve. Recently, a different approach has been proposed. In \cite{mp_gcs} motion planning is formulated as a convex optimization problem by using the graph of convex sets framework \citep{gcs}. The underlying idea is to decompose the collision-free space into intersecting convex sets from which a convex optimization problem is formulated. In cases where an explicit representation of the obstacles in the configuration space exists, like for single-body robots, creating collision-free convex regions can be done fast \citep{iris}. For multi-body robots, this is non-trivial. Existing work does this successfully \citep{iris_nlp, iris_c} by an optimization based approach, but the methods are still too time consuming to be used in the presence of moving obstacles. Our approach is instead to use deep learning to learn an SDF expressed in the configuration space. With this, we can query for shortest distances to the collision boundary, which allows us to expand spherical regions which are collision-free. Our approach is fast and therefore enables our suggested roadmap planner to be used in dynamic environments.
\\\\
Recent research has focused on learning collision detection \citep{fk_kernel_distance, diffco, graphdistnet} by predicting the signed distance between the robot links and the surrounding obstacles in the world space. The learned SDF is used in trajectory optimization but since the distance is expressed in the world space, the problem becomes an NLP and therefore takes a long time to solve. We take a novel approach and suggest to instead express the signed distance in the configuration space. This allows us to improve the PRM at the same time as it enables convex optimization for trajectory optimization, which runs faster and is more reliable than NLP solvers. In \cite{cspf} a learned signed distance function in the configuration space is proposed similar to our approach. However, their approach is restricted to point cloud representations, while we propose to represent the obstacles as parameterized geometric shapes, e.g. spheres. Furthermore, we also show how to use our learned SCDF to improve an existing roadmap planner.
\section{Problem formulation}
A robot is located in the world space, $\W \subset \R^3 $. The unique location of the robot is given by its configuration $\q \in \C$, where $\C$ is the configuration space. The set of points covered by the robots bodies at a certain configuration is expressed as $\B(\q) \subset \W$. The robot is surrounded by $\NrObst$ obstacles $\O = \bigcup_{i=1}^{\NrObst} \O_i$, where  $\O_i \subset \W$. The representation of the obstacle in the configuration space is the set $\C\O_i = \{\q \in \C \: |\: \B(\q) \cap \O_i \neq \emptyset \}$. The obstacle space is formed as $\Co = \bigcup_{i=1}^{\NrObst} \C \O_i$. The complement is referred to as the free space, $\Cf = \C \setminus \Co$. The path planning problem is a tuple, ($\Cf$, $\qStart$, $\qGoal$), where we want to connect a query pair, consisting of a start, $\qStart$, and goal configuration, $\qGoal$, with a geometric path, $\q(s): [0, 1] \mapsto \Cf$, such that $\q(0)=\qStart$ and $\q(1)=\qGoal$, or report correctly when such a path does not exist.
\end{document}


\section{Preliminaries}\label{sec:preliminaries}



%We denote by $(\Ac(x_\Ac),\Bc(x_\Bc))(z)$ a random execution of $\pi$ with private inputs $(x_\Ac,y_\Ac)$, and common input $z$.

%\Jnote{Move to DP}
% At the end of such an execution, the protocol outputs a public transcript denoted by the random variable $\trans_\pi(x_\Ac,x_\Ac,z)$ we denotes the common as $\out(\trans_\pi(x_\Ac,x_\Ac,z)$, and each party $\Pc \in \set{\Ac,\Bc}$ obtains his view denoted $\view^\Pc_\pi(x_\Ac,x_\Bc,z)$, which may also contain a ``local output'' \Jnote{Local} $\out^\Pc(x_\Ac,x_\Bc,z)$ (if the protocol specifies such an output). \Jnote{Common output, and parties output}


\subsection{Distributions and Random Variables}\label{sec:prelim:dist}
The support of a distribution $P$ over a finite set $\cS$ is defined by $\Supp(P) \eqdef \set{x\in \cS: P(x)>0}$. For a distribution or a random variable $D$, let $d\from D$ denote that $d$ was sampled according to $D$. Similarly,  for a set $\cS$, let $x \from \cS$ denote that $x$ is drawn uniformly from $\cS$, and denote by $\cU_{\cS}$ the uniform distribution over $\cS$. For a finite set $\cX$ and a distribution $C_X$ over $\cX$, we use the capital letter $X$ to denote the random variable that takes values in $\cX$ and is sampled according to $C_X$. The {\sf statistical distance} (\aka {\sf~variation distance}) of two distributions $P$ and $Q$ over a discrete domain $\cX$ is defined by $\sdist{P}{Q} \eqdef \max_{\cS\subseteq \cX} \size{P(\cS)-Q(\cS)} = \frac{1}{2} \sum_{x \in \cS}\size{P(x)-Q(x)}$. 
For a vector $x = (x_1,\ldots,x_n)$ and index $i\in [n]$, we let $x_{-i} = (x_1,\ldots,x_{i-1},x_{i+1},\ldots,x_n)$ and $x^{(i)} = (x_1,\ldots,x_{i-1}, -x_i, x_{i+1},\ldots,x_n)$, for a set $\cS \subseteq [n]$ we let $x_{\cS} = (x_i)_{i \in \cS}$ and $x_{-\cS} = (x_i)_{i \in [n]\setminus \cS}$, and for a vector $r \in \zo^n$ we let $x_r = (x_i)_{\set{i \colon r_i = 1}}$ and $x_{-r} = (x_i)_{\set{i \colon r_i = 0}}$.

%For $n \in \N$ we let $U_n$ be the uniform distribution over $\oo^n$, and let $S_n$ be the distribution induces by the sum of $n$ i.i.d.\ random variables, each is distributed according to $U_1$. Let $\cN(0,1)$ be the standard normal distribution.
%For a distribution $\cD$ and a function $f$, we define by $f(\cD)$ the distribution that is induced by the output of $f(x)$ for $x \from \cD$. 





% \begin{theorem}[\cite{McGregorMPRTV10}]\label{thm:sv-extracotr}
% 	\Enote{Remove if not needed}
% 	There is a constant $c$ to make the following holds. Let $X$ be an $\alpha$-SV source on $\{0,1\}^n$, let $Y$ be a source on $\{0,1\}^n$ with min-entropy at least $\beta n$ (independent from $X$), and let $Z=\ip{X,Y}\mbox{mod m}$ for some $m\in\mathbb{N}$. Then for every $\delta\in[0,1]$, the random variable $(Y,Z)$ is $\delta$-close to $(Y,U)$ where $U$ is uniform on $\mathbb{Z}_m$ and independent of $Y$, provided that
% 	$$
% 	n\geq c\cdot\frac{m^2}{\alpha\beta}\cdot\log(\frac{m}{\beta})\cdot\log(\frac{m}{\delta}).
% 	$$
% \end{theorem}



\Enote{I removed the definition of DP since it already appears in the intro}
\remove{
\subsection{Differential Privacy}\label{sec:prelim:DP}
We use the following standard definition of (information theoretic) differential privacy, due to \citet{DMNS06}. For notational convenience, we focus on databases over $\oo$.
\begin{definition}[Differentially private mechanisms]\label{def:mech}
	A randomized function $f\colon\oo^n\mapsto \zs$ is an {\sf $n$-size, $(\eps,\delta)$-differentially private mechanism} (denoted $(\eps,\delta)$-\DP) if for every neighboring $w,w'\in \oo^n$ and every function $g\colon \zs\mapsto \zo$, it holds that 
	$$
	\pr{g(f(w))=1}\leq e^{\eps}\cdot \pr{g(f(w'))=1} +\delta.
	$$ 	
	If $\delta=0$, we omit it from the notation.
\end{definition}
}


\subsubsection{Computational Differential Privacy}
There are several ways for defining computational differential privacy (see \cref{sec:related-works}). We use the most relaxed version due to \cite{BNO08}. For notational convenience, we focus on databases over $\oo$.
\begin{definition}[Computational differentially private mechanisms]\label{def:ComMech}
	A randomized function ensemble $f=\set{f_\pk\colon\oo^{n(\pk)}\mapsto \zs}$ is an {\sf $n$-size, $(\eps,\delta)$-computationally differentially private} (denoted $(\eps,\delta)$-$\CDP$) if for every poly-size circuit family $\set{\Ac_\pk}_{\pk\in \N}$, the following holds for every large enough $\pk$ and every neighboring $w,w'\in\oo^{n(\pk)}$:
	$$
	\pr{\Ac_\pk(f_\pk(w))=1}\leq e^{\eps(\pk)}\cdot \pr{\Ac_\pk(f_\pk(w'))=1} +\delta(\pk).
	$$ 
	If $\delta(\pk) = \negl(\pk)$, we omit it from the notation. 
\end{definition}



\subsubsection{Two-Party Differential Privacy}\label{sec:DP}
In this section we formally define distributed differential privacy mechanism (\ie protocols). %For the ease of notation, we consider protocol with no common input.

\begin{definition}\label{def:DP}%\Nnote{fix security parameter}
	A two-party protocol $\Pi=(\Ac,\Bc)$ is {\sf $(\eps,\delta)$-differentially private}, denoted $(\eps,\delta)$-$\DP$, if the following holds for every algorithm $\Dc$: let $\V^\Pc(x,y)(\pk)$ be the view of party $\Pc$ in a random execution of $\Pi(x,y)(1^\pk)$. Then for every $\pk,n \in \N$, $x\in \oo^n$ and neighboring $y,y'\in\oo^n$:
	\begin{align*}
	\pr{\Dc(V^\Ac(x,y)(\pk))=1}\le e^{\eps(\pk)}\cdot \pr{\Dc(V^\Ac (x,y')(\pk))=1}+\delta(\pk),
	\end{align*} 
	and for every $y\in \oo^n$ and neighboring $x,x'\in\oo^{n}$:
	\begin{align*}
	\pr{\Dc(V^\Bc(x,y)(\pk))=1}\le e^{\eps(\pk)}\cdot \pr{\Dc(V^\Bc (x',y)(\pk))=1}+\delta(\pk).
	\end{align*} 	
	Protocol $\Pi$ is {\sf $(\eps,\delta)$-computational differentially private}, denoted $(\eps,\delta)$-$\CDP$, if the above inequalities only hold for a non-uniform \ppt $\Dc$ and large enough $\pk$. We omit $\delta = \negl(\pk)$ from the notation. \footnote{Note that define we give for two-party differentially private protocols is a semi-honest definition, in which we ask for the security to hold when the parties interact in an honest execution of the protocol. Since we are proving a lower bound, starting from this weaker guarantee (as opposed to security against malicious players), yields a stronger result.}
\end{definition}
%We omit $\delta$ from the notation if $\delta$ is a negligible function of $n$.

%\Enote{simulation-based}
\begin{remark}[The definition for computational differential privacy we use]\label{rem:comDPChannel} 
	An alternative, stronger definition of computational differential privacy, known as simulation-based computational differential privacy, requires that the distribution of each party’s view be computationally indistinguishable from a distribution that ensures privacy in an information-theoretic sense. \cref{def:DP} is a weaker notion in comparison. Consequently, establishing a lower bound for a protocol that satisfies this weaker guarantee (as we do in this work) yields a stronger result.%Actually, our lower bound only requires the privacy to hold against \emph{uniform} external observer.
	%\Nnote{Maybe add: When only interesting in \Dp against external observer, the two definitions can be achieve using key-agreement and (single-party) \Dp mechanism. }
\end{remark}




\subsection{Useful Claims}
\remove{
In this section, we state generic lemmas and propositions that we will use later in our proofs.

The following lemma which we prove in \cref{sec:missing-proofs:distance-I}, measures the distance between two uniform stings conditioned one a random index $i$ either being fixed to $0$ or to $1$.

\def\distanceILemma{
    Let $R \la \zo^n$. For any (randomized) function $f:\{0,1\}^n\rightarrow \{0,1\}$ and $\alpha > 0$, it holds that
    \begin{align}\label{eq:f-alpha}
        \ppr{i \la [n]}{\size{\:\ex{f(R) \mid R_i = 0}-\ex{f(R) \mid R_i = 1}\:}\geq \alpha} \leq \frac{2}{n \alpha^2},
    \end{align}
    where the expectations are taken over $R$ and the randomness of $f$.
}

\begin{lemma}\label{lem:distance-I}
    \distanceILemma
\end{lemma}
}

The following two propositions state that given the output of a differentially private function, it is not possible to predict well even a random index (even if all other indexes are leaked). The first proposition handles the information-theoretic case and the second handles the computation case. Both propositions are proven in \cref{sec:missing-proofs:hard-to-guess}. 

\def\propHardToGuessInf{
    Let $f\colon \oo^n \rightarrow \cY$ be an $(\eps,\delta)$-\DP function, let $g \colon [n] \times \oo^{n-1} \times \cY \rightarrow \set{-1,1,\bot}$ be a (randomized) function, and let $X = (X_1,\ldots,X_n) \la \oo^n$. Then the following holds for every $i \in [n]$ where $X_i^* = g(i,X_{-i},f(X_1,\ldots,X_n))$:
    \begin{align*}
        \pr{X_i^* = X_i} \leq e^{\eps}\cdot \pr{X_i^* = -X_i} + \delta.
    \end{align*}
}

\begin{proposition}\label{prop:hard-to-guess-inf}
    \propHardToGuessInf
\end{proposition}


\def\propHardToGuessComp{
    Let $f = \set{f_{\pk} \colon \oo^{n(\pk)} \rightarrow \zo^{m(\pk)}}_{\pk \in \bbN}$ be an $(\eps,\delta)$-\CDP function ensemble, and let $\set{g_{\pk}}_{\pk \in \bbN}$ be a poly-size circuit family. Then, for large enough $\pk$ and $X = (X_1,\ldots,X_{n(\pk)}) \la \oo^{n(\pk)}$, the following holds for every $i \in [n(\pk)]$ where $X_i^* = g_{\pk}(i,X_{-i},f_{\pk}(X_1,\ldots,X_n))$:
    \begin{align*}
        \pr{X_i^* = X_i} \leq e^{\eps(\pk)}\cdot \pr{X_i^* = -X_i} + \delta(\pk).
    \end{align*}
}

\begin{proposition}\label{prop:hard-to-guess-comp}
    \propHardToGuessComp
\end{proposition}





\remove{
\Enote{Chao's old statement:}
\begin{lemma}\label{lem:distance-I-old}
        Let $R \la \zo^n$. 
	For any function $f:\{0,1\}^n\rightarrow \{0,1\}$ and $\alpha<0.01$, it holds that
	$$
	\Pr_{i\la[n]}\left[\: \size{\:\mathbb{E}[f(R) \mid R_i = 0]-\mathbb{E}[f(R) \mid R_i = 1]\:}\geq \alpha\right]\leq \frac{2+2\log(\frac{1}{\alpha})}{n\alpha^2}.
	$$
\end{lemma}
\begin{proof}
	Define $S_1=\{r \in \zo^n \colon f(r)=1\}$. Then for any $i\in[n]$, we have
	$$
	\begin{array}{rl}
		\size{\mathbb{E}[f(R) \mid R_i = 0]-\mathbb{E}[f(R) \mid R_i = 1]}
		&=\size{\Pr[R\in S_1|R_i=0]-\Pr[R\in S_1|R_i=1]}\\
		&=\size{\frac{\Pr[R_i=0|R\in S_1]\cdot\Pr[R\in S_1]}{\Pr[R_i=0]}-\frac{\Pr[R_i=1|R\in S_1]\cdot\Pr[R\in S_1]}{\Pr[R_i=1]}}\\
		&=\frac{2\size{S_1}}{2^n}\size{\Pr[R_i=0|R\in S_1]-\Pr[R_i=1|R\in S_1]}
	\end{array}
	$$
	When $|S_1|\leq \alpha\cdot 2^{n-1}$, we have $\size{\mathbb{E}[f(R) \mid R_i = 0]-\mathbb{E}[f(R) \mid R_i = 1]}\leq\frac{2\size{S_1}}{2^n}\leq \alpha$ for any $i\in[n]$. Hence, in the following, we assume $|S_1|> \alpha\cdot 2^{n-1}$.

	%Define $I_{bad}=\{i|\size{\Pr[R_i=0|R\in S_1]-\Pr[R_i=1|R\in S_1]}>2\alpha\}$ and $k=\size{I_{bad}}$, then for any $i\notin I_{bad}$, we have 
    %$$
    %\begin{array}{rl}
    %    2\alpha&\geq \size{\Pr[R_i=0|R\in S_1]-\Pr[R_i=1|R\in S_1]}\\
    %    &=\size{\frac{\Pr[R\in S_1|R_i=0]\cdot\Pr[R_i=0]}{\Pr[R\in S_1]}-\frac{\Pr[R\in S_1|R_i=1]\cdot\Pr[R_i=1]}{\Pr[R\in S_1]}}\\
    %    &=\size{\Pr[R\in S_1|R_i=0]-\Pr[R\in S_1|R_i=1]}\cdot\frac{1}{2\Pr[R\in S_1]}\\
    %    &\geq \size{\mathbb{E}[f(R) \mid R_i = 0]-\mathbb{E}[f(R) \mid R_i = 1]}\cdot \frac{1}{2},
    %\end{array}
    %$$ 
    %where the last inequality is because $\Pr[R\in S_1]\leq 1$. So that $\size{\mathbb{E}}[f(R) \mid R_i = 0]-\mathbb{E}[f(R) \mid R_i = 1]\leq %4\alpha$.
    Define $I_{bad}=\{i \colon \size{\Pr[R_i=0|R\in S_1]-\Pr[R_i=1|R\in S_1]} \geq 2\alpha\}$ and $k=\size{I_{bad}}$, and denote $I_{bad}=\{i_1,\dots,i_k\}$. Define $(X_{i_1}, \ldots X_{i_k}) = (R_{i_1},\dots,R_{i_k})\mid_{R \in S_1}$. 
    Consider the min-entropy
	$$
	\begin{array}{rl}
		H_{min}(X_{i_1},\dots,X_{i_k})&\leq H(X_{i_1},\dots,X_{i_k})\\
		&\leq \sum_{j=1}^k H(X_{i_j})\\
		&\leq k\cdot \left(-(\frac{1}{2}+2\alpha)\cdot\log(\frac{1}{2}+2\alpha)-(\frac{1}{2}-2\alpha)\cdot\log(\frac{1}{2}-2\alpha)\right)\\
            &=k\cdot \left(-(\frac{1}{2}+2\alpha)\cdot(\log(1+4\alpha)-1)-(\frac{1}{2}-2\alpha)\cdot(\log(1-4\alpha)-1)\right)\\
            &=k\cdot \left(1-(\frac{1}{2}+2\alpha)\cdot\log(1+4\alpha)-(\frac{1}{2}-2\alpha)\cdot\log(1-4\alpha)\right),
		
	\end{array}
	$$
	where $H_{min}(Y)$ is the minimum entropy of $Y$ and $H(Y)$ is the Shannon entropy of $Y$.\Enote{add to preliminaries.}
        The third inequality holds since by the definition of $I_{bad}$, for every $j \in [k]$ it holds that $\size{\pr{X_{i_j} = 1}-\pr{X_{i_j} = 0}} > 2\alpha$, and therefore $H(X_{i_j}) \leq H(1/2 + 2\alpha)$\Enote{define}.
	
	Therefore, there exists $b_1,\dots,b_k\in\{0,1\}$, such that 
	
	\begin{align}\label{eq:min-entropy-result}
		\Pr\left[(R_{i_1},\ldots,R_{i_k}) = (b_1,\ldots,b_k) \mid R\in S_1\right]
		&= \pr{(X_{i_1},\ldots,X_{i_k}) = (b_1,\ldots,b_k)}\\
		&= 2^{-H_{min}(X_{i_1},\dots,X_{i_k})}\nonumber\\
		&\geq 2^{k\cdot \left(-1+(\frac{1}{2}+2\alpha)\cdot\log(1+4\alpha)+(\frac{1}{2}-2\alpha)\cdot\log(1-4\alpha)\right)}.\nonumber
	\end{align}
	
	Let $S_{bad}=\{r \in \zo^n  \colon \set{(r_{i_1},\ldots,r_{i_k}) = (b_1,\ldots,b_k)} \land \set{r\in S_1}\}$.
	It holds that
	\begin{align*}
		|S_{bad}|
		&= \size{S_1} \cdot \Pr\left[(R_{i_1},\ldots,R_{i_k}) = (b_1,\ldots,b_k) \mid R\in S_1\right]\\
		&\geq \alpha\cdot 2^{n-1}\cdot2^{k\cdot \left(-1+(\frac{1}{2}+2\alpha)\cdot\log(1+4\alpha)+(\frac{1}{2}-2\alpha)\cdot\log(1-4\alpha)\right)},
	\end{align*} 
	where the inequality holds by \cref{eq:min-entropy-result} and since $\size{S_1} \geq \alpha\cdot 2^{n-1}$.
	Notice that any string in $S_{bad}$ depends on at most $n-k$ bits. It implies that $|S_{bad}|\leq 2^{n-k}$. Therefore, we have
	$$
	\begin{array}{rl}
		&2^{n-k}\geq \alpha\cdot 2^{n-1}\cdot2^{k\cdot \left(-1+(\frac{1}{2}+2\alpha)\cdot\log(1+4\alpha)+(\frac{1}{2}-2\alpha)\cdot\log(1-4\alpha)\right)} \\
		\Rightarrow& n-k \geq \log \alpha+n-1+k\cdot \left(-1+(\frac{1}{2}+2\alpha)\cdot\log(1+4\alpha)+(\frac{1}{2}-2\alpha)\cdot\log(1-4\alpha)\right)\\
		\Rightarrow& 1-\log \alpha \geq k\cdot((\frac{1}{2}+2\alpha)\cdot\log(1+4\alpha)+(\frac{1}{2}-2\alpha)\cdot\log(1-4\alpha))\\
		\Rightarrow& 1-\log \alpha \geq k\cdot(4\alpha\cdot\log(1+4\alpha)+(\frac{1}{2}-2\alpha)\cdot\log(1-16\alpha^2))\\
        \Rightarrow& 1-\log\alpha \geq k\cdot(15.9\alpha^2-8\alpha^2+32\alpha^3)=k\cdot(7.9\alpha^2+32\alpha^3)>0.5k\alpha^2\\
		\Rightarrow& k\leq \frac{2-2\log \alpha}{\alpha^2} = \frac{2+2\log (1/\alpha)}{\alpha^2},
	\end{array}
	$$
	Where the third transition holds since 
	\begin{align*}
		\lefteqn{(\frac{1}{2}+2\alpha)\cdot\log(1+4\alpha)+(\frac{1}{2}-2\alpha)\cdot\log(1-4\alpha)}\\
		&= 4\alpha\cdot\log(1+4\alpha) + (\frac{1}{2}-2\alpha)\paren{\log(1+4\alpha)+\log(1-4\alpha)}\\
		&= 4\alpha\cdot\log(1+4\alpha)+(\frac{1}{2}-2\alpha)\cdot\log(1-16\alpha^2),
	\end{align*}
	and the forth transition holds since $4\alpha\cdot\log(1+4\alpha)+(\frac{1}{2}-2\alpha)\cdot\log(1-16\alpha^2) > 15.9\alpha^2-8\alpha^2+32\alpha^3$ for $\alpha < 0.01$.
	Thus, we conclude that 
	$$
	\Pr_{i\la[n]}\left[\size{\mathbb{E}[f(R) \mid R_i=0]-\mathbb{E}[f(R) \mid R_i = 1]}\geq \alpha\right]\leq \frac{k}{n}\leq \frac{2+2\log (1/\alpha)}{n\alpha^2}.
	$$
\end{proof}
}


\subsection{Channels and Two-Party Protocols}\label{sec:protocol}

\paragraph{Channels.}A channel is simply a distribution of a pair of tuples defined as follows. 
\begin{definition}[Channels]\label{def:channel} A {\sf channel} $C_{(X,U)(Y,V)}$ of size $\isize$ over alphabet $\Sigma$ is a probability distribution over $(\Sigma^\isize \times\zo^\ast) \times(\Sigma^\isize \times\zo^\ast)$. The ensemble $C_{(X,U)(Y,V)}= \set{C_{(X_\pk,U_\pk)(Y_\pk,V_\pk)}}_{\pk\in \N}$ is an $\isize$-size channel ensemble, if for every $\pk\in \N$, $C_{(X_\pk,U_\pk)(Y_\pk,V_\pk)}$ is an $\isize(\pk)$-size channel. %We denote a channel of size one by a \emph{single-bit} channel. 
We refer to $X$ and $Y$ as the {\sf local outputs}, and to $U$ and $V$ as the {\sf views}.	
\end{definition}

We view a  channel as the experiment in which there are two parties $\Ac$ and $\Bc$.  Party $\Ac$ receives ``output'' $X$ and ``view'' $U$, and party $\Bc$ receives ``output'' $Y$ and ``view'' $V$. Unless stated otherwise, the channels we consider are over the alphabet $\Sigma = \oo$. We naturally identify channels with the distribution that characterizes their output.








\subsubsection{Two-Party Protocols}

A two-party protocol $\Pi=(\Ac,\Bc)$ is \ppt if the running time of both parties is polynomial in their input length. We let $\Pi(x,y)(z)$ or $(\Ac(x),\Bc(y))(z)$ denote a random execution of $\Pi$ on a common input $z$, and private inputs $x,y$.%We assume \wlg that a protocol has a common output (part of its transcript).\Jnote{This is not really the case we consider in this paper..}

\begin{definition}[Oracle-aided protocols]\label{def:ChannelAidedProtocol}
	In a two-party protocol $\Pi$ with oracle access to a {\sf protocol} $\Psi$, denoted $\Pi^\Psi$, the parties make use of the \textit{next-message function} of $\Psi$.\footnote{The function that on a partial view of one of the parties, returns its next message.} In a two-party protocol $\Pi$ with oracle access to a {\sf channel} $C_{Z W}$, denoted $\Pi^C$, the parties can jointly invoke $C$ for several times. In each call, an independent pair $(z,w)$ is sampled according to $C_{Z W}$, one party gets $z$, the other gets $w$.
\end{definition}


\begin{definition}[The channel of a protocol]\label{def:ChannlOfProtocol}
	For a no-input two-party protocol $\Pi= (\Ac,\Bc)$, we associate the channel $C_\Pi$, defined by $\C_\Pi= C_{(X, U),(Y, V)}$, where $X$ and $Y$ are the local outputs of $\Ac$ and $\Bc$ (respectively) and
	$U$ and $V$ are the local views of $\Ac$ and $\Bc$ (respectively).
    
	For a two-party protocol $\Pi$ that gets a security parameter $1^\pk$ as its (only, common) input, we associate the channel ensemble $ \set{C_{\Pi(1^\pk)}}_{\pk\in \N}$. 
\end{definition}

\begin{definition}[$(\alpha,\gamma)$-Accurate channel]\label{def:accurate-func}
	A channel $C = C_{(X, U),(Y, V)}$ is {\sf $(\alpha,\gamma)$-accurate for the function $f$}, if $\ppr{C}{\size{\out(V)-f(X,Y)}\leq \alpha}\ge \gamma$, where $\out(V)$ is the designated output.
    A channel ensemble $C_{(X, U),(Y, V)}= \set{C_{(X_\pk, U_\pk),(Y_\pk, V_\pk)}}_{\pk\in \N}$ is  $(\alpha,\gamma)$-accurate for  $f$ if $C_{(X_\pk, U_\pk),(Y_\pk, V_\pk)}$ is $(\alpha(\pk),\gamma(\pk))$-accurate for $f$, for every $\pk \in \N$.
\end{definition}

\subsubsection{Differentially Private Channels}\label{sec:DPChannel}
Differentially private channels are naturally defined as follows:
\begin{definition}[Differentially private channels]\label{def:DPChannel}
	An $n$-size channel $C = C_{(X, U),(Y, V)}$ with $X, Y$ over $\oo^n$ 
	is {\sf$(\eps,\delta)$-differentially private} (denoted $(\eps,\delta)$-$\DP$) if for every $x \in \Supp(X)$ there exists an $n$-size $(\eps,\delta)$-$\DP$ mechanisms $\Mc_x$ such that $(X,Y,U) \equiv (X,Y,\Mc_X(Y))$, and for every $y \in \Supp(Y)$ there exists an $n$-size $(\eps,\delta)$-$\DP$ mechanisms $\Mc_y'$ such that $(X,Y,V) \equiv (X,Y,\Mc_Y'(X))$. In addition, we say that the channel is \emph{uniform} if $X$ and $Y$ are independent random variables uniformly distributed in $\oo^n$. 
\end{definition}

\begin{definition}[Computational differentially private channels]\label{def:CDPChannel}
	An $n$-size channel ensemble $C = \set{C_{(X_\pk, U_\pk),(Y_\pk, V_\pk)}}_{\pk\in\N}$ with $X_\pk, Y_\pk$ over $\oo^n$ 
	is {\sf$(\eps,\delta)$-computationally differentially private} (denoted $(\eps,\delta)$-$\CDP$) if for every ensemble $\set{x_\pk \in \Supp(X_\pk)}_{\pk\in\N}$ there exists an $n$-size $(\eps,\delta)$-\CDP mechanisms ensemble $\set{\Mc_{x_\pk}}_{\pk\in\N}$ such that $(X_\pk,Y_\pk,U_\pk) \equiv (X_\pk,Y_\pk,\Mc_{X_\pk}(Y_\pk))$, for every $\pk\in\N$, and for every ensemble $\set{y_\pk \in \Supp(Y_\pk)}_{\pk\in\N}$ there exists an $n$-size $(\eps,\delta)$-$\CDP$ mechanisms ensemble $\set{\Mc'_{y_\pk}}_{\pk\in\N}$ such that $(X_\pk,Y_\pk,V_\pk) \equiv (X_\pk,Y_\pk,\Mc_{Y_\pk}'(X_\pk))$ for every $\pk\in \N$. In addition, we say that the channel is \emph{uniform} if $X_\pk$ and $Y_\pk$ are independent random variables uniformly distributed in $\{\pm 1\}^n$ for all $\pk\in\N$.
\end{definition}




% \begin{lemma}~\label{lem:dp-sv-source}
% 	Let $P$ be an $\varepsilon$-DP randomized protocol. Let $X$ and $Y$ be independent random variables uniformly distributed in $\{\pm 1\}^n$ and let random variable $\Pi(X,Y)$ denote the transcript of running $P(X,y)$. Then for every $\pi\in Supp(\Pi)$, the random variables corresponding to the inputs conditioned on transcript $\pi$, $X_\pi$ and $Y_\pi$, are independent $e^{-\varepsilon}$-strong SV source.
% \end{lemma}





\subsubsection{Weak Erasure Channel (\WEC)}

\begin{definition}[\WEC]\label{def:WEC}
	A channel $((O_A,V_A), (O_B,V_B))$ with $O_A \in \set{0,1}$ and $O_B \in \set{0,1,\bot}$ is a {\sf weak erasure channel}, denoted $(\alpha,p,q)$-$\WEC$, if:
	\begin{itemize}
		%\item $O_A\in \set{-1,1}$ and $O_B\in \set{-1,1,\bot}$.
		\item Random erasure: $\pr{O_B = \perp} = 1/2$.
		
		\item Agreement: $\pr{O_A\ne O_B\mid O_B\ne \bot}\le \alpha$.
		
		\item Secrecy:
		
		\begin{enumerate}
			\item For every algorithm $\Dc$ it holds that\label{WEC:item:A}
			\begin{align*}
				%\size{\pr{\Ac(O_A,V_A) = 1 \mid O_B \neq \perp} - \pr{\Ac(O_A,V_A) = 1 \mid O_B = \perp}} \le p
				\size{\pr{\Dc(V_A) = 1 \mid O_B \neq \perp} - \pr{\Dc(V_A) = 1 \mid O_B = \perp}} \le p
			\end{align*}
			(Alice doesn't know if $O_B = \perp$.)
			
			\item For every algorithm $\Dc$ it holds that\label{WEC:item:B}
			\begin{align*}
				\pr{\Dc(V_B) = O_A \mid O_B=\bot} \leq \frac{1+q}{2}.
			\end{align*}
			(i.e., if $O_B=\bot$, Bob don't know what is the value of $O_A$).
			
			%\item $SD((O_A U|O_B=\bot),(O_A U|O_B\ne \bot))\le p$ (The sender don't know if $O_B=\bot$).
			
			%\item $SD(V O_A|O_B=\bot,V(-O_A)|O_B=\bot)\le q$ (If $O_B=\bot$, Bob don't know what the value of $O_A$).
		\end{enumerate}
	\end{itemize}
   We say that a channel ensemble $C=\set{C_\pk}_{\pk\in N}$ is a {\sf computational weak erasure channel}, denoted $(\alpha,p,q)$-\CompWEC, if for every \ppt algorithm $\Dc$ and every sufficiently large $\pk\in\N$, $C_\pk$ satisfies the properties stated in the items above, where the secrecy property holds with respect to a \ppt algorithm $\Dc$. A protocol $\Lambda$ is said to be $(\alpha,p,q)$-$\CompWEC$, if the ensemble induces by the protocol (that is, $C=\set{C_{\Lambda(\pk)}}_{\pk\in\N}$) is $(\alpha,p,q)$-$\CompWEC$.  
\end{definition}



\subsubsection{Approximate Weak Erasure Channel (\AWEC)}\label{sec:AWEC}

\begin{definition}[\AWEC]\label{def:AWEC}
	A channel $C = ((O_A,V_A), (O_B,V_B))$ over $([-n,n] \times \zo^*) \times (([-n,n] \cup \bot)  \times \zo^*)$ is an {\sf approximate weak erasure channel}, denoted $(\ell,\alpha,p,q)$-\AWEC if:
	\begin{itemize}
		
		\item Random erasure: $\pr{O_B = \perp} = 1/2$.
		
		\item Accuracy: $\pr{\size{O_A - O_B} > \ell \mid O_B \ne \bot}\le \alpha$.
		
		\item Secrecy:
		
		\begin{enumerate}
			\item For every algorithm $\Dc$ it holds that\label{AWEC:item:A}
			\begin{align*}
				%\size{\pr{\Ac(O_A,V_A) = 1 \mid O_B \neq \perp} - \pr{\Ac(O_A,V_A) = 1 \mid O_B = \perp}} \le p
				\size{\pr{\Dc(V_A) = 1 \mid O_B \neq \perp} - \pr{\Dc(V_A) = 1 \mid O_B = \perp}} \le p
			\end{align*}
			(Alice doesn't know if $O_B=\bot$).
			
			\item For every algorithm $\Dc$ it holds that\label{AWEC:item:B}
			\begin{align*}
				\pr{\size{\Dc(V_B) - O_A} \leq 1000 \ell \mid O_B=\bot} \leq q.
			\end{align*}
			(i.e., if $O_B=\bot$, Bob can't estimate the value of $O_A$ with error $\leq 1000 \ell$).
		\end{enumerate}
	\end{itemize}
     We say that a channel ensemble $C=\set{C_\pk}_{\pk\in N}$ is a {\sf computational approximate weak erasure channel}, denoted $(\ell,\alpha,p,q)$-\CompAWEC, if for every \ppt algorithm $\Dc$ and every sufficiently large $\pk\in\N$, $C_\pk$ satisfies the properties stated in the items above. A protocol $\Gamma$ is said to be $(\ell,\alpha,p,q)$-$\CompAWEC$, if the ensemble induced by the protocol (that is, $C=\set{C_{\Gamma(\pk)}}_{\pk\in\N}$) is $(\ell,\alpha,p,q)$-$\CompAWEC$.  
\end{definition}

We will make use of the following lemma, which shows that for some choices of the parameters, \AWEC implies \WEC. The lemma is proven in \cref{sec:AWEC-to-WEC}.

\begin{lemma}\label{lemma:AWEC-to-WEC}
	For every $\ell> 0$, there exists a \ppt protocol $\Lambda = (\Pc_1,\Pc_2)$ such that given an oracle access to an $(\ell,\alpha,p,q)$-\AWEC $C$, the channel $\tilde{C}$ induced by $\Lambda^C$ is $(\alpha'=\alpha+0.001,\: p' = p ,\:  q' = 1/2 + 2(q+0.01))$-\WEC.
	Furthermore, the proof is constructive in a black-box manner:
	\begin{enumerate}
		\item There exists an oracle-aided \ppt algorithm $\Ec_1$ such that for every channel $C = ((\OA,\VA), (\OB,\VB))$ and algorithm $\Dc$ violating the \WEC secrecy property~\ref{WEC:item:A} of $\tilde{C}$, algorithm $\Ec_1^{\Dc}$ violates the \AWEC secrecy property~\ref{AWEC:item:A} of $C$.
		
		\item There exists an oracle-aided \ppt algorithm $\Ec_2$ such that for every channel $C = ((\OA,\VA), (\OB,\VB))$ and algorithm $\Dc$ violating the \WEC secrecy property~\ref{WEC:item:B} of $\tilde{C}$, algorithm $\Ec_2^{\Dc}$ violates the \AWEC secrecy property~\ref{AWEC:item:B} of $C$.
	\end{enumerate}
\end{lemma}

Since \cref{lemma:AWEC-to-WEC} is constructive, the following is an immediate corollary.
\begin{corollary}\label{cor:CompAWEC to CompWEC}
There exists an oracle aided \ppt protocol $\Lambda$, such that given a protocol $\Gamma$ that induces $(\ell,\alpha,p,q)$-\CompAWEC, it holds that $\Lambda^\Gamma$ is $(\alpha'=\alpha+0.001,\: p' = p ,\:  q' = 1/2 + 2(q+0.01))$-\CompWEC.  
\end{corollary}
\begin{proof}[Proof of \ref{cor:CompAWEC to CompWEC}]
Let $\Lambda$ be the \ppt algorithm guaranteed  by Lemma \ref{lemma:AWEC-to-WEC}. Given an $(\ell,\alpha,p,q)$-\CompAWEC protocol $\Gamma$, we define $\Lambda(\pk)={\Lambda^{\Gamma(\pk)}(\pk)}$. Assume towards a contradiction that $\Lambda$ is not a $(\alpha',p',q')$-\CompWEC. It follows that there exists a \ppt $\Dc$ that for infinity many $\pk\in\N$ contradicts one of the \WEC secrecy properties of channel ensemble $\set{C_{\Lambda(\pk)}}_{\pk\in\N}$. Fix $\pk\in\N$ for which this holds. By Lemma \ref{lemma:AWEC-to-WEC}, there exists a \ppt $\Ec^\Dc$ that for every such $\pk$  contradicts one of the secrecy properties of the channel $C_{\Gamma(\pk)}$. This implies that for infinity many $\pk\in\N$, $\Ec^\Dc$  contradict the secrecy of the channel ensemble $\set{C_{\Gamma(\pk)}}_{\pk\in\N}$, which is a contradiction since this would means that $\Gamma$ is not a $(\ell,\alpha,p,q)$-\CompAWEC.       
\end{proof}



\subsection{Oblivious Transfer (\OT)}

\paragraph{Secure Computation.}
We use the standard notion of securely computing a functionality, \cf  \cite{Goldreich04}.
\begin{definition}[Secure computation]\label{def:SFE}
	A two-party protocol {\sf securely computes a functionality $f$}, if it does so according to the real/ideal paradigm.   We add the term perfectly/statistically/computationally/non-uniform computationally, if the simulator's output is  perfect/statistical/computationally indistinguishable/  non-uniformly indistinguishable from  the real distribution.  The protocol have the above notions of security {\sf against semi-honest  adversaries}, if its security only  guaranteed to holds against an adversary that follows the prescribed protocol.   Finally, for the case of perfectly secure computation, we naturally apply the above notion also to the non-asymptotic case: the protocol with no security parameter perfectly  compute a functionality $f$.
	
	A two-party protocol {\sf securely computes a functionality ensemble $f$ with oracle to a channel $C$}, if it does so according to the above definition when the parties have access to a trusted party computing $C$. All the above adjectives naturally extend to this setting.
\end{definition}

\paragraph{Oblivious Transfer.}
The (one-out-of-two) oblivious transfer functionality is defined as follows.
\begin{definition}[oblivious transfer functionality $f_{\OT}$]\label{def:OTfunc}
	The oblivious transfer functionality over $\zo \times (\zs)^2$ is defined by  $f_{\OT} (i,(\sigma_0,\sigma_1)) = (\perp,\sigma_i)$.
\end{definition}
A protocol is $\ast$ secure OT,   for \\$\ast\in \set{\text{semi-honest statistically/computationally/computationally non-uniform}}$, if it  compute the $f_{\OT}$  functionality with $\ast$ security.





% \begin{definition}[Computational oblivious transfer, semi-honest model]
% A protocol $\Pi=(\Ac,\Bc)$ is a semi-honest 1-out-of-2 computational oblivious transfer (comp-OT) protocol if the following holds. Given a common input $1^{\pk}$, the parties $\Ac$ and $\Bc$ run the protocol $\Pi(1^\pk)$ (in an honest manner) and    
% $\Ac$ outputs $X=(m_1,m_2)\in \zo\times\zo$ and has a view $U$ and $\Bc$ outputs $Y=(i,\hat{m})\in\zo\times\zo$ and has a view $V$, and the following properties are satisfied:
% \begin{enumerate}
%     \item \textbf{Correctness:} 
%     $\pr{\hat{m}\neq m_i}<\negl(\pk).$ 
    
%     \item \textbf{A's Privacy:} For every \ppt $\Dc$ and every sufficiently large $\pk$:
%     $\pr{\Dc(V)=m_{i-1}}<(1+\negl(\pk))/2$
    
%     \item \textbf{B's Privacy:} For every \ppt $\Dc$ and every sufficiently large $\pk$:
%     $\pr{\Dc(U)=i}<(1+\negl(\pk))/2$  
% \end{enumerate}
% \end{definition}

We make use of the following useful results by Wullschleger on oblivious transfer amplification from weak channels.
\begin{theorem}[\cite{Wullschleger09}, from \WEC to statistically secure \OT]\label{thm:WEC TO OT IT}
    There exists an oracle aided protocol $\Pi$ such that the following holds: Given a $(\alpha,p,q)$-\WEC $C$, if $44(\alpha+p)\le 1-q$ then $\Pi^{C}(1^\pk)$ is a semi-honest statistically secure \OT.
\end{theorem}

The following computational version of \cref{thm:WEC TO OT IT} is implicit in \cite{Wullschleger09} and is based on the computational proof explicitly stated in \cite{Wul07} (see Section 6 in \cite{Wullschleger09} for discussion).   

\begin{theorem}[\cite{Wullschleger09,   Wul07}, from \CompWEC to computinally secure \OT]\label{thm:WEC TO OT Comp}
    There exists an oracle aided protocol $\Pi$ such that the following holds: Given a $(\alpha,p,q)$-\CompWEC protocol $\Lambda$, if $44(\alpha+p)\le 1-q$ then $\Pi^{\Lambda}$ is a semi-honest computational secure \OT.
\end{theorem}



% \begin{definition}[Computational 1-out-of-2 Oblivious Transfer, semi-honest model]
% A protocol $\Pi=(\Ac,\Bc)$ is a semi-honest 1-out-of-2 $(\eps,\alpha,\beta)$-oblivious transfer (OT) protocol if the following holds. 

% The parties $\Ac$ and $\Bc$ run the protocol (in an honest manner) and    
% $\Ac$ outputs $X=(m_1,m_2)\in \zo\times\zo$ and has a view $U$ and $\Bc$ outputs $Y=(i,\hat{m})\in\zo\times\zo$ and has a view $V$, and following properties are satisfied:
% \begin{enumerate}
%     \item \textbf{Correctness:} 
%     $\pr{\hat{m}\neq m_i}<\eps.$ 
    
%     \item \textbf{A's Privacy:} For every adversary $\Dc$:
%     $\pr{\Dc(V)=m_{i-1}}<(1+\alpha)/2$
    
%     \item \textbf{B's Privacy:} For every adversary $\Dc$: $\pr{\Dc(U)=i}<(1+\beta)/2$  
% \end{enumerate}
% \end{definition}
Our survey is designed to explore the perceptions and attitudes of Black Americans regarding AAE representation in chat-based AI systems across a variety of settings, ranging from professional to personal. For each setting, we gauge how and when participants want an LLM (or chatbot) to use AAE versus MUSE. We aimed to provide sufficient detail on the settings to make them more relatable and easier to comprehend \cite{lenzner2012effects}. The settings are selected to give a more complete picture of Black American's every-day experiences and preferences \cite{Maedche2019AI-Based}. For each scenario presented, study participants were asked to choose from the answer choices seen in \autoref{table:Vignette Choices} for how they would want such an LLM to interact:

%The specific settings we chose are described below, and cover both professional and personal contexts (as AI is used in both~\cite{Garrotes2021THE}), as well as use-cases the cover both LLM-generated continuations of one's own text as and LLM responses in an assistant-like scenario.

%Since use of these AI technologies is widespread in people' work and private lives \cite{Garrotes2021THE}, we explore both formal (professional) and informal (personal) settings. 


\begin{enumerate}[nolistsep,noitemsep]
\item \textit{AI Assistants} (professional and personal).
These LLM-response settings reflect the use of an AI assistant for helping with either professional or personal tasks, and whether such an assistant should address the user in AAE.


\item\textit{Customer Bot}. This LLM-response setting reflects the use of a text-based chatbot agent for quick assistance, and whether the agent should continue the interaction after greeting the user in AAE.

\item \textit{Email and SMS Autocomplete}. These LLM-continuation settings reflect the use of an LLM to autocomplete a user's own writing for emails or text messages.  % if AAE speakers want these systems to maintain a consistent tone from that with which they started the conversation. 

\item \textit{Educational Avatar}. This LLM-response setting reflects the use of AAE by an avatar in an education platform and whether this could impact learning experience.
% \hal{TODO fill in}.
%As inclusive educational materials enhance learning outcomes,  accessibility and provide overall support for better learning experiences \cite{baker2020linguistic}, we also present this setting.
\end{enumerate}

% \hal{there are a lot of mismatches that we need to clean up here (section 3.1). the above set does not match the set in Figure 1, which also doesn't match the set in 4.1.1. the terminology is also inconsistent (including at least Educational Content Delivery vs Educational Avatar). what's going on?} \kac{I wrote one part and the other was written by Christabel. We used different terminologies. I think the one in 4 is valid. Will check against the data to see where the  mismatch is coming from}

\noindent



%SS: Commented out since we don't discuss in Results.Participants were also asked to rank their preferences for when text-based generative AI should communicate in or understand AAE across each scenario on a scale from 1 (highest priority) to 3 (lowest priority). This segment aimed to compare Black Americans' differing preferences towards AI generation of AAE across distinct contexts, considering them relative to each other, to potentially provide insights into the broader societal expectations and acceptance for AAE in these technologies.
% We also asked participants to rank these settings to provid a clear understanding of where AAE/AAL speakers prioritize the use of AAE.
%The final portion of the survey asked participants to consider the potential benefits and dangers they perceive from the use of AAE by these AI systems, which helps us understand the broader implications of Generative AI's role in linguistic representation and inclusivity. 

%\begin{itemize}
 %   \item Always MUSE - \textit{Choice 0}
  %  \item Option to manually switch between AAE and MUSE - \textit{Choice 1}
   % \item Automatic detection and adaptation to AAE or MUSE - \textit{Choice 2}
    %\item Always AAE - \textit{Choice 3}
    %\item No preference as long as the system is effective - \textit{Choice 4}
%\end{itemize}

\begin{table}[t]
\centering
\footnotesize
%\rowcolors{2}{gray!10}{}
%\resizebox{0.98\linewidth}{!}{
\renewcommand{\arraystretch}{0.85} % Reduce row height
\setlength{\tabcolsep}{4pt}
\begin{tabular}{ @{~}l @{~~~} p{55mm} @{~} }
  \toprule			
%  \textbf{Option} & \textbf{Desired LLM Behavior} \\ 		
%  \midrule			
AlwaysMUSE &	LLM should always use MUSE.	\\
AlwaysAAE &	LLM should always use AAE. \\[0.5em]
AutoDetect &	LLM should automatically detect/adapt to\\&\quad the user's language variety. \\
UserOption &	LLM should provide an option to switch\\&\quad between AAE and MUSE. \\[0.5em]
NoPreference	&	No preference as long as the system is\\&\quad effective. \\
  \bottomrule			
\end{tabular}	
%}
%\end{adjustbox}
\caption{The set of possible choices for the preferences survey, which asked Black Americans about the contexts or scenarios in which they would prefer to have language model-based AI technologies generate AAE vs MUSE.} %Scenarios they considered ranged from email continuations in professional settings to educational avatars.}
%\hal{numbering doesn't match}
\label{table:Vignette Choices}
\end{table}






\section{Experiments: Planning outperforms Heuristics}
\label{sec:experiment}

We begin our empirical demonstrations by showcasing the effectiveness of our planning framework on both synthetic and real datasets. We focus on the simplest planning algorithm, 1-step lookaheads (Algorithm~\ref{alg:complete}), and show that even basic planning can hold great promise. 
We illustrate our framework using two uncertainty quantification modules---GPs and 
\ensembles/ \ensembleplus. 

Throughout this section, we focus on evaluating the mean squared error of 
a regression model $\model$,  and develop adaptive policies that minimize uncertainty on $g(f)$ defined in~\eqref{eqn:l2-g-f}.
When GPs provide a valid model of uncertainty, 
our experiments show that our planning framework significantly outperforms other baselines. 
We further demonstrate that our conceptual framework extends to deep learning-based uncertainty quantification methods such as  \ensembleplus while highlighting computational challenges that need to be resolved in order to scale our ideas. 
For simplicity, we assume a naive predictor, i.e., $\psi(\cdot) \equiv 0$. However, we emphasize that this problem is just as complex as if we were using a sophisticated model $\psi(.)$. The performance gap between the algorithms 
primarily depends
on the level  of uncertainty in our prior beliefs.

To evaluate the performance of our algorithm, we benchmark it against several baselines. 
%Active learning baselines use an acquisition function $\ac$ to select points that have the highest   function value: $X\opt_t \in \argmax_{X \in \xpoolj{t}} \ac({X})$ at every step $t$. These methods may also need an UQ module, which we simply use the same UQ module as in our algorithm, and it  outputs $V(X)$ that measures the the uncertainty of each point $X \in \xpoolj{t}$.
Our first set of baselines are from active learning~\citep{AggarwalKoGuHaPh14}:
\\ % \noindent\textbf{Active Learning Heuristics:} 
\textbf{(1)} 
\textsf{Uncertainty Sampling (Static):}  In this approach, we query the samples for which the model is least certain about. Specifically, we estimate the variance of the latent output $f(X)$ for each $X \in \xpool$ using the UQ module and select the top-$K$ points with the highest uncertainty. \\
\textbf{(2)} \textsf{Uncertainty Sampling (Sequential):} This is a greedy heuristic that sequentially selects the points with the highest uncertainty within a batch, while updating the posterior beliefs using pseudo labels from the current posterior state. Unlike \textsf{Uncertainty Sampling (Static)}, this method takes into account the information gained from each point within batch, and hence tries to diversify the selected points within a batch. 

 
We also compare our approach to the  \textbf{(3)} \textsf{Random Sampling}, which selects each batch uniformly at random from the pool. Additionally, we compare solving the planning problem using  \textsf{REINFORCE}-based policy gradients with   $\mathsf{Smoothed\text{-}Autodiff}$ policy gradients.\footnote{Our code repository is available at
  \url{https://github.com/namkoong-lab/adaptive-labeling}.}
%Detailed experimental setups are provided in Section \ref{sec:details-experiments}.

%We repeat all experiments with 10 random seeds.




\begin{figure}[t]
\centering
\begin{minipage}[b]{0.49\textwidth}
\centering
\includegraphics[width=\textwidth, height=5cm]{figures/original_scale/Var_of_l_2_loss.pdf}
\caption{(Synthetic data) Variance of mean squared loss evaluated through the posterior belief $\mu_t$ at each horizon $t$. This is the objective that policy gradient methods like \textsf{REINFORCE} and $\ouralgo$ optimizes. 1-step lookaheads are surprisingly effective even in long horizons.}
\label{fig:var-l2-sim}
\end{minipage}
\hfill
\begin{minipage}[b]{0.49\textwidth}
\centering \includegraphics[width=\textwidth, height=5cm]{figures/original_scale/Error_of_estimated_model_l_2_loss.pdf}
\caption{(Synthetic data) Error between MSE calculated based on collected data $\mc{D}^{0:T}$ vs. population oracle MSE over $\mc{D}_{\rm eval} \sim P_X$. Reducing uncertainty over posteriors directly leads to better OOD evaluations. 1-step lookaheads significantly outperform active learning heuristics in small horizons.}
\label{fig:mean-l2-sim}
\end{minipage}
%\caption{Simulated data for GPs}
%\label{fig:both_plots}
\end{figure}

\subsection{Planning with Gaussian processes}
\label{sec:experiment-plan-GP}
We now briefly describe the data generation process for the GP experiments,  deferring a more detailed discussion of the dataset generation to Section~\ref{sec:details-experiments}. 
We use both the synthetic data and the real data to test our methodology.
For the \emph{simulated data},  we construct a setting where the general population is distributed across \emph{51 non-overlapping clusters} while the initial labeled data $\dtrain$ just comes from one cluster. In contrast, both $\dpool \defeq (\xpool,\ypool),\deval \defeq (\xeval,\yeval)$ are generated   from all the clusters. 
We begin with a low-dimensional scenario, generating a one-dimensional regression setting using a GP. %Gaussian Process (GP).
Although the data-generating process is not known to the algorithms,  we assume that the GP hyperparameters are known to all the algorithms
to ensure fair comparisons. This can be viewed as a setting where our prior is well-specified, allowing us to isolate the effects
of different policy optimization approaches
 without any concerns about the misspecified priors. We select $10$ batches, each of size $K=5$ across $T = 10$ time horizons.

To examine the robustness of our method against the distributional assumptions made  in the simulated case, we then move to a real dataset where the correct prior is not known. We simulate selection bias from the eICU dataset~\citep{PollardJoRaCeMaBa18}, which contains real-world patient data with in-hospital mortality outcomes. 
We conduct a $k$-means clustering to generate 51 clusters and then select data from those clusters. We view this to be a credible replication of practice, as severe distribution shifts are common due to selection bias in clinical labels.  To convert the binary mortality labels into a regression setting, we train a  random forest classifier and fit a GP on predicted scores, which serves as the UQ module for all the algorithms. As before, the task is to select 10 batches, each consisting of 5 samples, across 10 time horizons.

 In Figures~\ref{fig:var-l2-sim} and~\ref{fig:mean-l2-sim}, we present results for the simulated data. 
Figure~\ref{fig:var-l2-sim} shows the variance of $\ell_2$ loss, and Figure~\ref{fig:mean-l2-sim} presents the error in the estimated $\ell_2$ loss using $\mu_t$ (relative to true $\ell_2$ loss, that is unknown to the algorithm). 
As we can see from these plots, our method one-step lookahead  gives substantial improvements  over active learning baselines and random sampling. In addition,
compared to the one-step lookahead planning approach using \textsf{REINFORCE}-based policy gradients, 
we observe that $\mathsf{Smoothed\text{-}Autodiff}$-based policy gradients provide significantly more robust performance over all horizons.

In Figures~\ref{fig:var-l2-real}~and~\ref{fig:mean-l2-real}, we observe similar findings on the eICU data. We see that planning policies (\textsf{REINFORCE} and $\mathsf{Smoothed\text{-}Autodiff}$) consistently outperform other heuristics by a large margin.  Active learning baselines perform poorly in these small-horizon batched problems and can sometimes be even worse than the random search baselines.  Overall, our results show the importance of careful planning in adaptive labeling for reliable model evaluation. 

We offer some intuition as to why one-step lookahead planning may outperform other heuristic algorithms. 
 First,  \textsf{Uncertainty sampling (Static)} while myopically selects the
 top-$K$ inputs with the highest uncertainty, it fails to consider 
the overlap in information content among the ``best” instances; see \citep{AggarwalKoGuHaPh14} for more details. 
In other words,  it might acquire points from the same region with high uncertainty while failing to induce diversity among the batch.
Although \textsf{Uncertainty Sampling (Sequential)} somewhat addresses the issue of information overlap, a significant drawback of 
this algorithm
is the disconnect between the objective we aim to optimize and the algorithm. For example, it might sample from a region with high uncertainty but very low density. 

\begin{figure}[t]
\centering
\begin{minipage}[b]{0.48\textwidth}
\centering
\includegraphics[width=\textwidth, height=5cm]{figures/original_scale/Var_of_l_2_loss_real.pdf}
\caption{(Real-world eICU data) Variance of mean squared loss evaluated through the posterior belief $\mu_t$ at each horizon $t$. Even 1-step lookaheads are extremely effective planners, and auto-differentiation-based pathwise policy gradients provide a reliable optimization algorithm based on low-variance gradient estimates.}
\label{fig:var-l2-real}
\end{minipage}
\hfill
\begin{minipage}[b]{0.48\textwidth}
\centering \includegraphics[width=\textwidth, height=5cm]{figures/original_scale/Error_of_estimated_model_l_2_loss_real.pdf}
\caption{(Real-world eICU data) Error between MSE calculated based on collected data $\mc{D}^{0:T}$ vs. population oracle MSE over $\mc{D}_{\rm eval} \sim P_X$. Reducing uncertainty over posteriors directly leads to better OOD evaluations. Our method significantly outperforms active learning-based heuristics, and random sampling.}
\label{fig:mean-l2-real}
\end{minipage}
%\caption{Real data for GPs}
\end{figure}
 
%\vspace{-1.5cm}
% \begin{wrapfigure}{r}{.32\columnwidth}
%   \vspace{-.5cm} 
%   \centering
% \includegraphics[scale=.29]{figures/Var of l2l_2 loss.pdf}
%   \vspace{-0.2cm}
%   \caption{Results of GP}
% \label{fig:var-l2-gp}
%   \vspace{-0.1cm}
% \end{wrapfigure}


% Attempts have been made  in the past to address these  drawbacks heuristically  (see \citep{AggarwalKoGuHaPh14}). We give a unified computational framework while approaching the problem in a more principled manner and solving it more optimally.




\subsection{Planning with  neural network-based uncertainty quantification methods ($\ensembleplus$)}


We now provide a proof-of-concept that shows the generalizability of our conceptual framework  to the deep learning-based UQ modules, specifically focusing on $\ensembleplus$ due to their previously observed superior performance~\citep{OsbandWenAsDwIbLuRo23}. Recall that implementing our framework with deep learning-based UQ modules  requires us to retrain the model across multiple possible random actions $\bm{a}(\theta)$ sampled from the current policy $\pi_\theta$.
This requires significant computational resources, in sharp contrast to the GPs where the posteriors are in closed form and can be readily updated and differentiated. 

Due to the computational constraints, we test $\ensembleplus$ on a toy setting to demonstrate the generalizability of our framework. We consider a setting where the general population consists of four clusters, while the initial labeled data only comes from one cluster. Again we generate data using GPs.  The task is to select a batch of 2 points in one horizon. We detail the $\ensembleplus$ architecture in Section \ref{sec:details-experiments}, and we assume prior uncertainty to be large (depends on the scaling of the prior generating functions). 
The results are summarized in the Table~\ref{tab:UQ_ensemble}.

% \begin{table}[H]
% \vspace{-10pt}
% \caption{Performance under \ensembleplus as UQ module}
%     \centering
%     \begin{tabular}{|m{3cm}|m{2.5cm}|m{2cm}|} 
%     \hline
%       Algorithm   & Variance of $\loss_2$ loss estimate & Error of $\loss_2$ loss estimate  \\ \hline Random Sampling 
%          & $1710.9 \pm 1352.1$ & $8.67\pm6.62$ 
%       \\ \hline \ouralgo & $1.30 \pm 0.68$ & $0.91\pm0.25$ \\ \hline
%     \end{tabular}
%     \label{tab:UQ_ensemble}
%     %\vspace{-10pt}
% \end{table}




\begin{table}[h]
\vspace{-10pt}
\caption{Performance under \ensembleplus as the UQ module}
\centering
\begin{tabular}{|l|l|l|}
\hline
Algorithm   & Variance of $\loss_2$ loss estimate & Error of $\loss_2$ loss estimate  \\
\hline
\textsf{Random sampling} & 7129.8 $\pm$ 1027.0 & 136.2 $\pm$ 8.28 \\ \hline
\textsf{Uncertainty sampling (Static)} & 10852 $\pm$ 0.0 & 162.156 $\pm$ 0.0 \\ \hline
\textsf{Uncertainty sampling (Sequential)} & 8585.5 $\pm$ 898.9 & 144 $\pm$ 6.93 \\ \hline
\textsf{REINFORCE} & 1697.1 $\pm$ 0.0 & 45.27 $\pm$ 0.0 \\ \hline
\ouralgo & 1697.1 $\pm$ 0.0 & 45.27 $\pm$ 0.0 \\ \hline
\end{tabular}
%\caption{Comparison of different algorithms based on variance   and   error in $\ell_2$ loss estimation with Ensemble $+$ as the UQ module. Our results demonstrate that {\ouralgo} and REINFORCE outperformthe other active learning based heuristics, confirming the benefits of our MDP formulation for the adaptive labeling problem, as also demonstrated in Section 4.\\
%\footnotesize{Experimental details: We use Gaussian Processes as our data generating process, GP parameters are the same as in Section D.3.  The task is to select a batch of 2 points along one horizon.The marginal distribution $p_X$ has 4 \textit{non-overlapping} clusters. Initial data comes from one cluster, while pool and evaluation points comes from all the clusters. We have $20$ initial labeled data points, $10$ pool points, and $252$ evaluation points.  Training procedures are similar to the one in Section D.3.} }
\label{tab:UQ_ensemble}
\end{table}



% We faced  issues in scaling up these experiments which will be our focus in the future. 





% \begin{itemize}
%     \item Posteriors should be consistent. Two dimensions: even with less training,  
%     \item the inference should be  fast enough
% \end{itemize}


% Potential research directions for uncertainty quantification

% In this section we consider a simple setting We consider a simpler setting and 


% For synthetic dataset generation, we use ...... For real datasets, we use ...... We compare our methodolgy to several baselines ()    This Section is structured as follows:
% \begin{itemize}
%     \item \textbf{GPs, square loss objective} (Section \ref{}): 
%     %the broad aim of the experiments  in this section is to isolate the performance of our methodology without any concerns for the inefficiencies induced due to a mis-specified prior or imperfect posterior inference. To accomplish this we generate synthetic datasets using GPs (detailed later). We use the well specified prior (GPs - with same hyperparameter setting) as our UQ module.   
%      As GPs provide differentaible posterior inference - any errors induced due to imperfect posterior updates are also isolated. We note that under this setting
%      \item In Section\ref{} we demonstrate why our methodology performs better than other baselines - by devising various synthetic experiments ()
%     \item  \textbf{UQ Benchmarking }(Section \ref{}): Before diving into the experiments using $\ensembleplus$ and ENNs,  we showcase our benchmarking experiments in Section \ref{}. We use real datasets We observe that ENNs perform better
%      \item \textbf{Ensemble $+$}, objective: recall, accuracy
%     \item \textbf{ENN}, objective: recall, accuracy
% \end{itemize}




% In Section {}, we test 
% \subsection{Experimental details}

% \begin{itemize}
%     \item UQ methodologies - GPs, ENNs
%     \item Objectives - Recall,  ATE
%     \item Datasets - ATE-synthetic datasets, Recall-synthetic, real datasets
%     \item Baselines - 
%     \begin{itemize}
%         \item Random sampling
%         \item Active learning - Uncertainty based sampling - In regression setting almost all of the 
%         \item Myopic greedy - Greedy Batch based sampling
%         \item Policy Gradient
%     \end{itemize}
    
% \end{itemize}

% \subsection{Experiments}
%     \begin{itemize}
%     \item GPs with square loss
%     \item Benchmarking ENN
%         \item ENNs with ATE
%         \item ENNs with Recall
%     \end{itemize}

% \subsection{Benefits over other algorithms - intuition and experiments}

%Active learning - Myopic greedy / Don't rely on the objective rather some entropy version.


%%% Local Variables:
%%% mode: latex
%%% TeX-master: "main"
%%% End:

This work identifies signal collapse as a critical bottleneck in one-shot neural network pruning. Performance loss in pruned networks is due to \textbf{signal collapse} in addition to the removal of critical parameters. We propose \textbf{REFLOW} (\textbf{Re}storing \textbf{F}low of \textbf{Low}-variance signals), a simple yet effective method that mitigates signal collapse without computationally expensive weight updates. By focusing on signal preservation, REFLOW highlights the importance of mitigating signal collapse in sparse networks and enables magnitude pruning to match or surpass state-of-the-art one-shot pruning methods such as CHITA, CBS, and WF.

REFLOW consistently achieves state-of-the-art accuracy across diverse architectures, restoring ResNeXt-101 from under 4.1\% to 78.9\% top-1 accuracy at 80\% sparsity on ImageNet. Its lightweight design makes it a practical solution for both research and deployment, delivering high-quality sparse models without the overhead of traditional approaches. These findings challenge the traditional emphasis on weight selection strategies and underscore the critical role of signal propagation for achieving high-quality sparse networks in the context of one-shot pruning.



\section{Conclusion Remarks}
This work proposes a RBG graph model for disease spreading via hubs. We study the joint effect of the agent density, hub density, and connection function. The existence of a critical hub density depends only on the boundedness of the support of the connection function, which relates to curbing the traveling distance of individuals. When it comes to dispersion, both the degree distribution and the percolation threshold suggest that increasing dispersion helps spread the disease. The percolation properties of RBG graphs relate to unipartite graphs with modified connection functions. 
An interesting question in this direction is if and when the properties of the RBG graphs can be well represented by unipartite graphs with some modified connection functions. Our conjecture is that for independent connections between different pairs of agents, such representation is unlikely due to the oblivion of the local dependence (present in the RBG models). 
 Another direction is to consider hybrid models where agents may get infected either through common hubs or direct interactions between agents. The former infection mechanism is more centralized than the latter. 
	

\begin{acks}
%Work used the Toubkal High Performance Computing cluster at Mohammed VI Polytechnic University; 
This work used the UM6P African Supercomputing Center (ASCC).
%we thank the African Supercomputer Center team for their technical support. 
Work partially supported by EU Horizon projects AI4Europe (101070000), TwinODIS (101160009), ARMADA (101168951), DataGEMS (101188416) and RECITALS (101168490).
\end{acks}
	
\bibliographystyle{ACM-Reference-Format}
\bibliography{paper/current_version/ref,paper/current_version/pargis,paper/current_version/icde-tutorial,paper/current_version/parisinmemory}


\end{document}
\endinput
