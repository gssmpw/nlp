\section{Conclusions}
\label{sec:conclusions}
In this paper, we conduct a survey of the SotA graph-based methods for in-memory $ng$-approximate vector search, proposing a new taxonomy based on five key design paradigms.
The chronological development and inter-method influence are outlined, along with an evaluation of key design choices like seed selection and neighborhood diversification. 

Through extensive experimentation on datasets with up to 1B vectors, we highlight the scalability challenges faced by most methods, with incremental insertion methods showing the best scalability on datasets exceeding 100GB. 
We observe that light-weight hierarchical structures help select better seeds to start the search on billion-scale datasets, and that neighborhood diversification is a key contributor in improving the query answering performance, with RND and MOND being the best techniques. 
We also propose promising research directions.
%In this paper, 
%We proposed ELPIS, a new solution for in-memory $ng$-approximate similarity search over massive collections of high-dimensional vectors. %, which is now gaining popularity and importance in several data science applications. 
%We demonstrated the efficacy of ELPIS on large datasets from various domains, where ELPIS considerably outperforms %the state-of-the-art
%its competitors. 
%Finally, we shared key insights and lessons learned that can help the community better understand the impact of different design choices on graph-based similarity search performance, and to make further progress in this area. %In the future we plan to extend Elpis to out-of-core datasets.
%A theoretical analysis of the performance and reliability properties of graph-based similarity search methods is an interesting and challenging open problem.

%Research Directions
%There are several other opportunities to enhance the performance of graph-based methods.
%In terms of hardware optimizations, improving memory layout and developing efficient implementations for indexing and search on GPU and FPGAs can significantly boost performance.
%On the algorithmic side, developing efficient implementations for parallel beam search and supporting graph structures with techniques like efficient quantization, or summarization, which can reduce indexing time and footprint, while ensuring highly accurate and fast search. 
%As dataset sizes increase, the need for efficient seed selection strategies becomes more apparent. 
%Combining graph structures with existing data structure families may yield more efficient searches and lower indexing demands.
%Last, but not least, in terms of theoretical enhancements, we note that graph-based methods currently lack robust theoretical guarantees, particularly concerning result accuracy. 
%This deficiency may be critical for some applications, where the reliability of results is of paramount importance. 
%Further theoretical studies are also necessary to understand how neighborhood diversification can enhance the construction of efficient graph structures, while maintaining good connectivity within the graph (proximity vs. sparsity). %Additionally, there is a pressing need to improve our understanding of how to measure the quality of KNN graphs for efficient similarity searches. More research could also be dedicated to incremental insertion building and improving this promising paradigm.
