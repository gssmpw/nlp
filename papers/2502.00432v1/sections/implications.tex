We believe that community membership hiding methods like $\nabla$-CMH serve as valuable tools for protecting user privacy within online platforms, such as social networks. 
Indeed, the spirit of these methods aligns with privacy rights, including the so-called \textit{right to be forgotten}, a principle that allows individuals to request the deletion or anonymization of their data. 
This right, articulated in the European Union's General Data Protection Regulation (GDPR), empowers individuals to regain control over their personal information and to ensure that their data is not retained or used in ways that could compromise their privacy and safety \citep{regulation2016general, mantzouranis2020privacy}.

An option for these users would be to leave the online platform, but such a decision might be too drastic: on the one hand, this strategy might deprive users of an increasingly important online presence; on the other hand, it might be ineffective in preserving users' privacy~\cite{minaei2017arxiv}. A more flexible approach would allow users to opt out of community detection while staying on the platform. This strategy strikes the optimal balance between preserving privacy and maximizing the utility of community detection. 

However, implementing the right to be forgotten for platforms like online social networks involves complex challenges. Indeed, allowing individuals to remove or hide their associations with sensitive or controversial groups identified by community detection algorithms can be complicated, yet crucial to prevent personal or professional repercussions. 
For example, this could safeguard vulnerable individuals, such as journalists or opposition activists in authoritarian regimes, and help combat online criminal activities by modifying network connections to infiltrate espionage agents or disrupt communications among malicious users.
As it turns out, manually addressing these issues can be unfeasible, especially for users with a large number of connections, thereby making (semi-)automatic techniques like $\nabla$-CMH essential.

Nonetheless, it is also worth recognizing that these node-hiding techniques could be misused for harmful purposes. Malicious actors may use them to evade network analysis tools, which are often employed by law enforcement for public safety, thereby masking illicit or criminal activities.

Eventually, for any online platform offering community membership hiding capabilities, it is essential to thoroughly assess the impact of this feature \textit{before} granting users the ability to conceal themselves from community detection algorithms, balancing privacy concerns with the need for security and law enforcement.