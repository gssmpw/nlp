Community detection is a fundamental task in analyzing complex network structures, such as social networks, biological systems, and communication graphs \cite{fortunato2010pr}. This is achieved by \emph{community detection algorithms}, which aim to uncover groups of tightly connected nodes, known as \emph{communities}, that share common characteristics, interactions, or structural patterns.
Successfully detecting these communities enables a wide range of applications across various domains \cite{karastas2018ibigdelft}, such as targeted marketing \cite{mosadegh2011ajbmr}, recommendation systems, and network security. Despite its utility, the application of community detection algorithms raises critical privacy concerns. By grouping individuals into identifiable communities, these methods may expose sensitive affiliations or preferences. For instance, being identified as part of a political group, religious organization, or other communities formed on top of information that users may wish to keep private can lead to excessive surveillance by private or government entities, which in turn facilitates discriminative behaviors and more effective techniques for political propaganda. Such exposure urges the need for individuals to control their visibility within detected communities, balancing privacy with the benefits of network participation. 

Motivated by this challenge, we address the problem of \emph{community membership hiding}, originally introduced by \citet{bernini2024kdd}.
This task draws inspiration from \emph{counterfactual reasoning} \cite{tolomei2021tkde,tolomei2017kdd}, especially on graphs \cite{lucic2022aistats}, and it involves strategically modifying the network structure to prevent a specific node from being identified as part of a particular community, as detected by a given algorithm.
% To the best of our knowledge, the only existing work that directly tackles this task formulates the problem as a counterfactual graph objective and introduces a deep reinforcement learning-based approach to modify the neighborhood of a target node \cite{bernini2024kdd}.

% It is important to highlight that, although related, this task fundamentally differs from the \emph{community deception} task discussed in the existing literature \cite{deception_modularity_3}.
% Unlike community deception, which aims to obscure or mask an entire community, membership hiding operates at a finer granularity. Specifically, it focuses on a single target node, ensuring that the node's affiliation with a sensitive community is no longer detectable. Moreover, community deception typically employs smooth measures such as \emph{Deception Score} to quantify success, which combines three criteria for effective community masking: reachability, spreadness, and hiding. Whereas community membership hiding pursues a stricter, binary goal.

In this paper, we build upon the method proposed by \citet{bernini2024kdd}, formulating the community membership hiding task as a counterfactual graph objective -- specifically, a constrained optimization problem that involves perturbing the graph structure surrounding the target node to obscure its community membership.
However, rather than addressing it as a discrete objective and solving it with a deep reinforcement learning-based approach, as done in the original method, we draw inspiration from adversarial attacks on graph networks \cite{trappolini2023savage}. Our method reformulates the task of hiding a target node through minimal and constrained modifications to the graph's structure as a \textit{differentiable} objective, which we then solve using gradient-based optimization techniques. 
This approach comes with three key advantages over the original technique proposed by \citet{bernini2024kdd}: $(i)$ a higher success rate in the community membership hiding task, $(ii)$ improved computational efficiency, and $(iii)$ meticulous allocation of the available budget to achieve the goal.
% \gabri{Forse qui sotto dovremmo aggiungere anche che il nostro metodo non solo è più efficiente ma migliora anche le performance...}
% This method prioritizes efficiency in altering the graph while preserving its global properties, aiming to minimize disruptions to the network's overall structure and functionality. 

Our main contributions are summarized below.

%\begin{itemize}
    %\item 
    $(i)$ We reformulate the original community membership hiding task as a differentiable counterfactual graph objective.
    
    %\item
    $(ii)$ We, therefore, propose \method{}, a gradient-based method that strategically perturbs the target node's neighborhood and successfully hides it;
    %\item 
    
    $(iii)$ We evaluate our approach on real-world networks and show its superiority over existing baselines. To encourage reproducibility, the source code of our method is available at: {\footnotesize\url{https://anonymous.4open.science/r/community\_membership\_hiding-FDA1/}}
%\end{itemize}

The remainder of this paper is structured as follows. 
We review related work in Section \ref{sec:related}. Section \ref{sec:background} contains background and preliminaries. We formulate the problem in Section \ref{sec:problem}. Section \ref{sec:method}, describes our method, which we validate through extensive experiments in Section \ref{sec:experiments}. Finally, we conclude in Section \ref{sec:conclusion}.