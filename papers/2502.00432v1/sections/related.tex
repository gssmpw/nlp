% TODO: Teniamoci su una colonna qui


%%%%%%%%%%%%%%%%%%%%%%%
%%% COMMUNITY DETECTION
%%%%%%%%%%%%%%%%%%%%%%%
\textbf{\textit{Community Detection.}} Community detection algorithms play a crucial role in analyzing network structures by identifying and grouping nodes into \textit{communities}. These communities are clusters of nodes that exhibit a higher density of connections within the group compared to their connections with the rest of the network. 
% These algorithms are pivotal across diverse fields such as social network analysis, biology, and economics, where uncovering clusters offers valuable insights into the underlying relationships and dynamics within complex systems.
% They can typically be classified into two main categories: non-overlapping and overlapping community detection. Non-overlapping methods assign each node to a single community, resulting in distinct and well-defined partitions of the graph.
Existing approaches to identify non-overlapping communities include Modularity Optimization \cite{louvain_detection_alg}, Spectrum Optimization \cite{spectrum_analysis}, Random Walk \cite{walktrap_detection_alg}, Label Propagation \cite{label_detection_alg}, or Statistical Inference \cite{mmsbm_detection_alg}. 
In contrast, overlapping community detection algorithms frequently use methods such as Matrix Factorization \cite{bigclam_detection_alg}, NISE (Neighborhood-Inflated Seed Expansion) \cite{nise} or techniques based on minimizing the Hamiltonian of the Potts model \cite{Ronhovde_2009}. For a comprehensive overview of these methods, see the extensive summary by \citet{community_detection_survey}.
Additionally, recent advancements in deep learning-based community detection are described by \citet{deepl_detection_alg}. 

%%%%%%%%%%%%%%%%%%%%%%%
%%% COMMUNITY DECEPTION
%%%%%%%%%%%%%%%%%%%%%%%

%\textbf{\textit{Community Deception.}} Community deception task involves altering the structure of a network to obscure the detection of specific clusters. The aim is to disrupt the ability of detection algorithms to identify target groups, thereby safeguarding the privacy of sensitive communities. As broadly discussed in the original work \cite{bernini2024kdd}, this topic differs significantly from community membership hiding in two key aspects: first, the goal shifts from concealing individual nodes to hiding an entire community, and second, it employs a smooth measure, the \emph{Deception score} \cite{deception_modularity_3}, rather than a binary success criterion. At first glance, community deception might appear to involve hiding the membership of multiple nodes in the community individually to keep the entire group concealed. However, this oversimplifies the process, as successfully hiding even a single node, as defined later, can already achieve substantial community concealment and result in a high deception score.
%Several techniques have been developed for concealing communities, with many based on the concept of modularity. Notable examples include the approach by \citet{deception_modularity_1}, the DICE algorithm \cite{deception_modularity_2}, and the method proposed by \citet{deception_modularity_3}. Other strategies utilise the concept of Safeness, first introduced by Fionda and Pirrò and later refined by \citet{deception_safeness_2}. Additionally, the concept of Permanence has been applied in methods such as those introduced by \citet{deception_persistence_1}. An exhaustive summary of these methods is provided by \citet{deception_survey}.\\

%%%%%%%%%%%%%%%%%%%%%%%%%%%%%%%
%%% COMMUNITY MEMBERSHIP HIDING
%%%%%%%%%%%%%%%%%%%%%%%%%%%%%%%

\textbf{\textit{Community Membership Hiding.}} Community membership hiding addresses a more specific problem: the concealment of a single node's affiliation with a particular community. This approach is especially relevant in protecting individuals’ privacy within networks, especially regarding membership in sensitive communities with potentially significant implications. The foundational work in this area is that of \citet{bernini2024kdd}, who propose an innovative method to hide a node's community membership by modifying the node's neighborhood structure strategically. The authors approach this problem as a counterfactual graph objective, leveraging a graph neural network (GNN) to model the structural complexity of the input graph. They employ deep reinforcement learning (DRL) within a Markov decision process framework to suggest precise modifications to a target node’s neighborhood, ensuring it is no longer identified as part of its original community. This approach assumes a fixed budget, constraining the number of allowable changes for efficiency and practicality. Furthermore, the method exhibits \emph{transferability}, i.e., it mantains effectiveness across various community detection algorithms. \\
Unlike the \textit{DRL-Agent} method proposed by \citet{bernini2024kdd}, our approach reformulates the community membership hiding problem as a \textit{differentiable} counterfactual graph objective. This reformulation allows us to leverage well-established gradient-based optimization techniques, offering three key advantages: $(i)$ improved concealment effectiveness, $(ii)$ reduced computational costs (avoiding the time-consuming process of training and using a DRL agent), and $(iii)$ a more efficient use of the available budget, allocating resources judiciously rather than fully depleting them.