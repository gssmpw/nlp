We introduced \method{}, a counterfactual graph generator designed to solve the community membership hiding task through gradient-based optimization. This method employs a perturbation vector, added element-wise to the adjacency vector of the target node to mask. 
To ensure differentiability, we define an intermediate real-valued perturbation vector and a loss function that encourages minimal changes to the graph network under consideration.
Furthermore, the graph modifications are guided by a vector containing the set of \textit{promising actions} that the algorithm can perform to successfully escape the community. While several approaches can be used to define this vector, we propose a solution that prioritizes adding or removing edges attached to nodes with specific properties (e.g., degree, betweenness, etc.).

Experimental results demonstrate the superiority of \method{} in effectively concealing target nodes from their communities, outperforming previous approaches in the literature. Additionally, our method can find valid solutions more quickly, without exhausting the allocated budget (i.e., the maximum number of allowed actions).
% However, this advantage is a limitation in the asymmetric scenario, where the community detection algorithm used to assess the attainment of the hiding objective differs from the one used for the actual clustering of nodes by, e.g., the social media platform. In this scenario, it is preferable to fill the budget, as we have no means to validate our solution before sending it for evaluation.

In future work, we plan to incorporate node embeddings and explore scenarios involving multiple target nodes being hidden simultaneously, as well as settings with limited knowledge of the network.
