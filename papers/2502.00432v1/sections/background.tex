In this section, we start by introducing the graph notations used in the paper. We then briefly review the well-known community detection problem, which serves as the basis for defining the community membership hiding problem.

%%%%%%%%%%%
%% NOTATION
%%%%%%%%%%%

Let $\graph =(\nodes,\edges)$ be an arbitrary (undirected\footnote{If $\graph$ is directed, the same reasoning applies for outgoing edges.}) graph, where $\nodes$ denotes the set of nodes with $|\nodes|=n$, and $\edges \subseteq \nodes \times \nodes$ represents the set of edges with $|\mathcal{E}|=m$. The structure of $\graph$ is represented by a binary adjacency matrix denoted by $A = (A_{u,v})_{u,v \in \nodes}$, where $A_{u,v} = 1$ if and only if there is an edge between nodes $u$ and $v$, that is, $(u,v) \in \edges$, and $A_{u,v} = 0$ conversely. 
The neighborhood of a node $u$, defined as the set of nodes reachable via edges from $u$, is represented by the $u$-th row of $A$. We denote this row as $A_u$, referring to it as the \emph{adjacency vector} of $u$. 

%%%%%%%%%%%%%%%%%%%%%%
%% COMMUNITY DETECTION
%%%%%%%%%%%%%%%%%%%%%%

The \emph{community detection} problem seeks to group nodes in a network into clusters, referred to as \emph{communities}. Intuitively, communities are groups of nodes characterized by strong intra-cluster connections compared to their links with nodes outside the cluster. In this work, we focus on detecting non-overlapping communities based exclusively on the network's edge structure, leaving the exploration of algorithms that consider node features to future research.
%Several detection techniques, such as Walktrap, Greedy, InfoMap, etc., have been proposed in the literature.
Formally, we adopt the widely accepted definition in the literature, considering a community detection algorithm as a function $f(\cdot)$ that generates a set of non-empty communities $f(\graph) = \{C_1,C_2,\dotsc,C_k\}$ where each node $u$  is assigned to \emph{exactly one} community, and $k$ is usually unknown. \\
Community detection algorithms typically aim to maximize a score that quantifies intra-community cohesiveness, with Modularity \cite{newman2006pnas} being a widely used metric. However, this often implies solving NP-hard optimization problems. To address this challenge, numerous practical approximation methods have been proposed. Significant examples include Greedy \cite{greedy_detection_alg}, Louvain \cite{louvain_detection_alg}, WalkTrap \cite{walktrap_detection_alg},  InfoMap \cite{infomap_detection_alg}, Label Propagation \cite{label_detection_alg}, Leading Eigenvectors \cite{eigenvectors_detection_alg}, Edge-Betweenness \cite{edge_detection_alg}, and SpinGlass \cite{spinglass_detection_alg}. 
%The specific mechanisms of community detection algorithms are irrelevant to this work, so we treat $\f$ as an input-output function.

