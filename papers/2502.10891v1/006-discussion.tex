\section{Discussion}
\label{sec:discussion}
In this section, we discuss the limitations of \sysname and opportunities for improvement.

\parab{Non-line-of-sight.} Currently, our system experiences performance drops when the sender and receiver are in a non-line-of-sight condition, primarily due to severe multipath effects and attenuation.
The obstruction can be caused by the human body (whether it involves the sender, receiver, or both) or by large obstacles like coral or shipwrecks that completely block the line of sight between the sender and receiver.
For the first type of obstruction, %
one possible solution is that
the sender and receiver (upon hearing the transmission sound) can opt to expose the mobile phone to reduce the obstruction. The second type of obstruction, involving large obstacles, is uncommon and currently not supported by our system.

\parab{Interference from pouch.} Although chirp signals can resist interference and noise, they fail when the interference spans the entire frequency range in low SNR cases as shown in Fig.~\ref{fig:spectrogram}.
In our experiment, the soft pouch generated such interference when it deforms due to movement. A practical solution is to use a hard case instead of a soft pouch with extra attention brought by the hard case. Alternatively, filling the pouch with more air can help resist shape changes and reduce interference.



\parab{End-to-end latency.}
Based on the latency breakdown discussed in \S\ref{sec:eval_system_metrics}, several methods can be implemented to reduce the end-to-end delay. First, image encoding and decoding currently run on CPU which is time-consuming but they can be migrated to on-device GPUs to speed up the inference with Vulkan backend~\cite{vulkan_2024}.
Second, we can hide most of the processing latency via pipelining and only the transmission latency becomes the bottleneck.


\parab{Preamble detection.}
Current preamble design and detection algorithm is vulnerable to interference and can be further improved. In the future, more advanced techniques such as neural network-based preamble design and detection~\cite{yang80neural}, can be leveraged to improve the packet detection reliability.







































