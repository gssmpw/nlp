\section{Introduction}\label{sec:introduction}

\begin{figure}[t]
\centering
\includegraphics[width=\linewidth]{figs/concept_new.pdf}
\caption{Conceptual graph of an underwater image transmission system for scuba divers.}
\label{fig:conceptual_graph}
\end{figure}

Underwater activities, e.g., snorkeling and scuba diving,
are among the primary ways humans explore the underwater world,
with tens of millions participating annually for recreational or scientific
purposes~\cite{stat_zhao_plus,stat_zhao,stat_zhao_1}.
Effective underwater communication is crucial for people to share experiences,
follow instructions, and alert others to dangers.

However, underwater communication has historically been limited due to
fundamentally challenging underwater environments
(e.g., scuba divers still largely rely on hand signals to communicate today).
Wireless technologies, such as cellular, Wi-Fi, or Bluetooth,
are ineffective underwater because radio frequency (RF) signals
suffer substantial attenuation in aquatic
environments~\cite{chen2022underwater,chen2023underwater}.
While wired networks offer an alternative, they are often too cumbersome and
impractical underwater.
Existing research~\cite{afzal2022battery,tonolini2018networking,jang2019underwater,carver2021amphilight,mobicom24_seascan} addressing these limitations
relies on specialized, sophisticated hardware, which is not accessible for
everyday use.

The latest advancement in ubiquitous underwater communication
utilizes acoustic signals to enable text messaging on mobile
devices~\cite{chen2022underwater}, as acoustic signals are known to propagate
better in water than RF signals. Nevertheless, with its limited data rate---only 0.6 kbps beyond 10 meters---its practical applications are highly restricted.

In this study, we investigate the feasibility of \emph{underwater
image transmission}. As envisioned in Fig.~\ref{fig:conceptual_graph},
enabling image transmission would enhance communication
between scuba divers and enrich the overall experience of various
underwater activities.
Despite active research efforts in both industry and
academia~\cite{gofish, AppleRD2021, liu2023efficient,afzal2022battery},
implementing this technology faces two major challenges
posed by underwater environments: (1) severely limited bandwidth, which renders
conventional image codecs infeasible due to their inadequate compression
efficiency, and (2) high transmission errors (caused by varying channel
conditions), which may substantially degrade
reconstructed image quality or even prevent image decoding altogether.

We present \sysname, the first underwater communication system
that enables image transmission on commodity mobile devices.
\sysname is also acoustic-based, but it is specifically designed
to optimize bandwidth efficiency and enhance error resilience.
At its core, \sysname employs \textit{generative image
compression}~\cite{mao2024extreme,yu2024image,santurkar2018generative},
a recent technique that achieves significantly higher compression rates
than traditional image codecs (e.g., JPEG~\cite{wallace1991jpeg})
and neural codecs~\cite{cheng2020image}.
This generative model encodes an image into a sequence of
``tokens,'' each representing a small patch (e.g., 16$\times$16 pixels) of the
original image in a more bandwidth-efficient manner
(\S\ref{sec:generative_compression}).
These tokens are then modulated by \sysname into acoustic signals
and transmitted underwater.
If a token is corrupted or lost during transmission,
the image decoder can leverage its ``generative'' capabilities
to reconstruct the image, thereby mitigating transmission errors.

While the above design appears intuitive, its practical effectiveness
is hindered by several challenges.
First, generative compression models are typically pre-trained on large
image datasets spanning diverse scenes. However, in the domain-specific
underwater scenario, our analysis (Fig.~\ref{fig:codebook_merger}) reveals
that more than half of the tokens in the codebook are never used,
leading to wasted bandwidth.
Second, existing generative models assume that all encoder-generated tokens
are received intact by the decoder. In reality, underwater transmission
is prone to errors that can distort tokens.
Third, the constrained computational and communication resources on commodity mobile devices
necessitate efficient reliability-enhancing techniques at the physical (PHY)
layer to correct errors in data symbols and ensure that tokens
remain recoverable by the generative decoder after demodulation.
We address these challenges using three key techniques.

\parab{Context-aware tokens distillation (\S\ref{sec:semantic_filtering}).}
To reduce the number of tokens used by a generic generative
compression model to encode an image, we employ a ``distillation'' process
that transfers the most relevant information from the original token set
to a significantly smaller set, motivated by the fact that the original
tokens are heavily underutilized in underwater scenarios.
Our distillation process involves training a transformer model in context,
on underwater image datasets, enabling a higher compression ratio
while preserving perceptual image quality.

\parab{Error-resilient fine-tuning (\S\ref{sec:fine_tuning}).}
To enhance the error resilience of a generative compression model,
we fine-tune its decoder on underwater datasets
under simulated random token disturbances.
While this approach resembles prior loss-resilient codecs
proposed for video
systems~\cite{nsdi24_grace,li2024reparolossresilientgenerativecodec},
a fundamental difference in \sysname is that underwater transmission errors
distort data symbols directly at the PHY layer,
making it impossible for the decoder to explicitly identify and mask
corrupted (or lost) tokens, as done in prior methods.
As a result, we must strategically perturb tokens during training,
ensuring that the decoder learns to reconstruct images without prior knowledge
of corrupted tokens.

\parab{Reliability enhancement at the PHY (\S\ref{sec:packetizer},
\S\ref{sec:synchronization}).}
Improving error resilience at the PHY layer is equally important,
as it minimizes the number of corrupted tokens passed to the generative
image decoder after demodulation and helps preserve reconstructed
image quality.
To achieve this, we first insert multiple training symbols into
the data packet. These training symbols enable the receiver to perform
time equalization for channel estimation, improving the recovery of data
symbols. Next, to address symbol time offsets caused by underwater device
movement, we propose an effective time synchronization algorithm
based on two key observations: the relative speed between the sender and
receiver is inherently limited underwater,
and their distance changes gradually without sudden shifts.
Leveraging these physical principles,
\sysname filters out unrealistic detected start timestamps and smooths detection
results, thereby obtaining accurate start timestamps for symbols.


We developed a prototype of \sysname on the Android platform
and conducted extensive real-world experiments.
Our results show that:
(1) \sysname reliably transmits 256$\times$256 color images with high fidelity
in under 9s over distances of up to 20m;
(2) \sysname maintains robust performance across various depths,
orientations, distances, and mobility conditions, reducing the bit error
rate (BER) from 19\% to below 2\% on average compared with state-of-the-art
systems;
(3) \sysname achieves an average LPIPS (Learned Perceptual Image Patch
Similarity)~\cite{zhang2018unreasonable} score of 0.3, indicating strong
visual consistency between transmitted and receive images.

Our key contributions are as follows:
\begin{itemize}
[topsep=2pt,noitemsep,leftmargin=*]
\item We design and implement \sysname, the first underwater acoustic system
that enables reliable image transmission between mobile devices.
\item \sysname employs generative image compression and enhances its
bandwidth efficiency and error resilience for underwater communication
(through context-aware token distillation and error-resilient fine-tuning).
\item \sysname implements reliability-enhancing techniques at the PHY
layer to mitigate the impact of underwater transmission errors.
\item \sysname demonstrates the feasibility of image transmission in
bandwidth-limited, error-prone underwater environments on
resource-constrained mobile devices.
\end{itemize}

