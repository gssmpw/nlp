\section{System Overview}
\label{sec:system_overview}
Fig.~\ref{fig:system_overview} provides an overview of \sysname.
The image encoder and decoder are jointly trained offline on underwater datasets with simulated transmission errors (\cref{sec:fine_tuning}).
During operation, the mobile sender compresses
a raw image into tokens (i.e., indices of embeddings in a codebook)
and converts them into a more compact bitstream
using ``context-aware distillation'' (\cref{sec:semantic_filtering}).
It then encodes the bits into data symbols via channel coding and
encapsulates them into a packet at the PHY layer (\cref{sec:packetizer}).
This packet includes a leading preamble followed by multiple symbol groups.
Next, the sender modulates the packet data into an acoustic signal
for transmission through the underwater channel to the receiver.

Upon reception, the mobile receiver performs preamble detection and time synchronization (\cref{sec:synchronization}) to locate symbol groups in the received signal.
It then compensates for channel distortions through equalization (\cref{sec:packetizer}),
demodulates the signal into data symbols,
reconstructs the bitstream,
and ultimately generates a high-fidelity underwater image.

\begin{figure}[t]
    \centering
    \includegraphics[width=\linewidth]{figs/system_overview_new_1.pdf}
    \caption{\sysname overview.}
    \label{fig:system_overview}
\end{figure}





