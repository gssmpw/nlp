\section{Related Work}
\label{sec:related_work}

\subsection{Underwater Communication}
\noindent\textbf{Mobile devices.} AquaApp~\cite{chen2022underwater} is the first work that enables underwater text message transmission between two mobile phones. They further extend the system for 3D localization underwater~\cite{chen2023underwater}.
AquaHelper~\cite{yang2023aquahelper,yang80neural} focuses on designing robust signal detection algorithms for SOS message transmission.
The information transmitted by their systems is very limited.
Although some work in~\cite{liu2021uqcom,liu2023uqrcom} attempts to communicate via QR codes on mobile devices. Successfully scanning the QR code heavily depends on the underwater visibility and the distance between the sender and receiver. Instead, \sysname is capable of transmitting images on mobile devices.


\parab{Customized hardware.}
Researchers in \cite{afzal2022battery,mobicom24_seascan} transmit an RGB image via a Piezo-acoustic backscatter~\cite{jang2019underwater,rademacher2022enabling}. Due to its low data rate, it takes hours to send a raw RBG data.
Amphilight~\cite{carver2021amphilight} and Shrimp~\cite{lin2021shrimp} employ optical techniques for underwater transmission, utilizing customized LED lights and laser diodes, respectively. However, the optics-based communication system is mainly limited by the transmission distance and the light condition underwater.
Unlike others, \sysname focuses on enabling underwater communication on commercial off-the-shelf mobile phones, making the technology more accessible and ubiquitous.




\parab{Chirp-based acoustic systems.}
Chirp-based transmission protocol is widely adopted for underwater communication~\cite{xie2024icc,jia2022two,steinmetz2018practical,steinmetz2022taking,restuccia2017isonar,tonolini2018networking}. They mainly focus on the optimization of the transmission protocols such as rate adaptation, modulation designs, multi-path elimination. Their designed protocols still are for a small packet and require larger transmission power and bandwidth which cannot be directly applied for image transmission on mobile devices.
\sysname optimizes the chirp-based protocol tailored for robust image transmission on commercial mobile devices.

\parab{Underwater image transmission.}
Researchers in \cite{anjum2022deep,anjum2022acoustic} utilize the traditional CNN-based image compression approach to reduce image size. However, as discussed in \S\ref{sec:mov_challenges}, the images compressed by those approaches do not meet the required size by mobile devices and are not robust to transmission errors. \sysname utilizes the state-of-the-art generative compression models which are bandwidth-efficient and error-resilient.










\subsection{Generative Compression}
Generative compression has attracted significant attention \cite{grassucci2023generative,xie2021tsp,dai2023wc,anjum2022deep,le2023arxiv,li2024reparolossresilientgenerativecodec,NEURIPS2019_f1ea154c} and has been successfully applied to image/video codecs and wireless communication, where it reduces communication overhead without compromising the quality of the received images or videos.
\sysname is the first system to apply generative compression to underwater communications, and it also extends this capability to other limited-bandwidth or error-prone scenarios---such as satellite networks and rural communication.

Image tokenization~\cite{yan2024elastictokadaptivetokenizationimage,shen2025catcontentadaptiveimagetokenization}, as a way for generative compression, learns a compact image representation for various downstream tasks such as classification, reconstruction, and generation.
Since the introduction of neural discrete representation~\cite{van2017neural}, numerous methods~\cite{esser2020taming, yu2024image, yu2023language, yu2021vector} have been developed. Among these, VQGAN~\cite{esser2020taming} and TiTok~\cite{yu2024image} were selected for our system due to their balance between compressed image size and decoded image quality.