
\section{Motivation}\label{sec:background}

Existing underwater mobile communications rely on acoustic signals,
as they propagate more effectively in water than RF signals (e.g., Wi-Fi and Bluetooth).
In this section, we characterize the harsh channel conditions encountered by
mobile devices underwater and demonstrate the limitations of a state-of-the-art
underwater acoustic system, AquaApp~\cite{chen2022underwater},
in supporting image transmission. These challenges underscore the need for an
alternative solution.

\subsection{Characterizing Mobile Devices Underwater}\label{sec:mov_channel}



\noindent\textbf{Frequency selectivity.} The frequency range of audible signals transmitted and received by a mobile phone's speaker and microphone is inherently limited and further influenced by the frequency-selective nature of the underwater channel. To illustrate this effect, we submerge two commercial Samsung S21~\cite{Samsung} phones in an open lake at a depth of 2m and position them 5m apart. One phone transmits 0--15 kHz chirp signals,
each lasting 1s.
Fig.~\ref{fig:snr} shows the frequency response of the underwater channel in terms of normalized amplitude (dB), with the shaded area representing variations.
The results indicate a significant drop in frequency response at 3.5 kHz,
suggesting that acoustic signals beyond this threshold experience
rapid attenuation~\cite{chen2022underwater,yang2023aquahelper}.


\begin{figure*}[t]
    \centering
    \begin{minipage}[b]{0.24\textwidth}
        \centering
        \includegraphics[width=\textwidth]{figs/chirp_0.pdf}
        \caption{Channel response.}
        \label{fig:snr}
    \end{minipage}
    \begin{minipage}[b]{0.24\textwidth}
        \centering
        \includegraphics[width=\textwidth]{figs/noise_4.pdf}
        \caption{Noise.}
        \label{fig:noise}
    \end{minipage}
    \begin{minipage}[b]{0.24\textwidth}
        \centering
        \includegraphics[width=\textwidth]{figs/spectrogram.pdf}
        \caption{Spectrogram.}
        \label{fig:spectrogram}
    \end{minipage}
    \begin{minipage}[b]{0.24\textwidth}
        \centering
        \includegraphics[width=\textwidth]{figs/multipath.pdf}
        \caption{Cross-correlation.}
        \label{fig:multipath}
    \end{minipage}
\end{figure*}

\parab{Ambient noise.} Underwater ambient noise, such as air bubbles, water flow, and sounds produced by the waterproof bag, overlaps with the frequency bands of a mobile phone's acoustic signal, leading to additional interference.
We use a Samsung S21 phone to record underwater ambient noise in the same lake
under both static and moving conditions.
Fig.~\ref{fig:noise} presents the normalized amplitude of ambient noise across the
0--8 kHz range, showing particularly high noise levels between 0--1 kHz.
Moreover, ambient noise increases significantly when the device is in motion,
primarily due to rustling and friction-related noises generated by the waterproof
bag.
Spectrum analysis in Fig.~\ref{fig:spectrogram} further indicates that this noise
spans frequencies from 1--4kHz, posing considerable challenges to acoustic
signal transmission.

\parab{Multipath.} The underwater environment introduces significant multipath effects,
as acoustic waves are reflected by the lake's surface, bottom, rough rocks, and other
objects. These reflections result in both inter- and intra-symbol interference.
Fig.~\ref{fig:multipath} illustrates this phenomenon, showing multiple delayed replicas
of the main signal.





\subsection{Underwater Image Transmission Challenges}\label{sec:mov_challenges}

Faced with substantial challenges posed by underwater
environments (\S\ref{sec:mov_channel}), the state-of-the-art acoustic
communication system---AquaApp~\cite{chen2022underwater}---only supports text
messaging on mobile devices.
In this study, we push the boundaries of underwater communication by
exploring the feasibility of image transmission.



We implement AquaApp on Samsung S21 mobile phones and extend it to support
image transmission. Specifically, images are encoded into byte streams
using different codecs, including
JPEG~\cite{wallace1991jpeg}, PNG~\cite{PNG}, and recent neural
codecs~\cite{cheng2020image}. These encoded bytes are then transmitted
using AquaApp's three-step OFDM (Orthogonal Frequency Division Multiplexing)
protocol. The process begins with the sender transmitting a preamble to the
receiver. Based on this preamble, the receiver identifies and replies with high
SNR (signal-to-noise ratio) subcarriers.
The sender then encodes the payload on the selected subcarriers and transmits
the data.


\parab{Image codecs.} First, we show that existing codecs exhibit \textit{low compression efficiency} and are \textit{vulnerable to transmission errors}.
Fig.~\ref{fig:BPP} illustrates the transmission latency for a 256$\times$256 image at various compression levels.
Since AquaApp's maximum effective data rate is only 0.6 kbps~\cite{chen2022underwater}, it takes at least 80s, 1000s, 12s to transmit images
encoded with JPEG (low quality level of 30), PNG (scaled to 30\% of the original), and neural codecs (low quality level of 1), respectively, \textit{assuming} no transmission errors. Moreover, low-quality settings introduce noticeable artifacts and blurring, as shown in Fig.~\ref{fig:semantic_vs_pixel}, significantly reducing image clarity. This is because these codecs prioritize
the reconstruction of pixel-level details, which proves inefficient for underwater
transmission.





Meanwhile,
errors are inevitable over extended transmission times due to channel variations. Existing image codecs are particularly vulnerable to transmission errors due to entropy coding~\cite{huffman1952method,witten1987arithmetic}.
If a received packet is corrupted or incomplete, the image decoding process
may fail.
To quantify this, we define the \textit{recovery rate} as the
fraction of successfully decoded images among transmitted ones.
We simulate lossy channels with varying bit error rates (BERs) and compute
the recovery rate over 500 images.
The results, shown in Fig.~\ref{fig:recovery}, indicate that more than half
of JPEG images fail to decode when the BER exceeds 0.5\%.
PNG- and neural network-compressed images are unrecoverable even at a BER
of 0.1\%, so their results are omitted.






\begin{figure}[t]
    \centering
    \begin{minipage}[b]{0.23\textwidth}
        \centering
        \includegraphics[width=\textwidth]{figs/latency_by_quality.pdf}
        \caption{Transmission laten\-cy of images compressed with different codecs.}
        \label{fig:BPP}
    \end{minipage}\hfill
    \begin{minipage}[b]{0.23\textwidth}
        \centering
        \includegraphics[width=\textwidth]{figs/recover_rate.pdf}
        \caption{Recovery rate (PNG and neural codecs fail to reconstruct images).}
        \label{fig:recovery}
    \end{minipage}
\end{figure}


\begin{figure}[t]
    \centering
    \begin{subfigure}[b]{.22\linewidth}
        \centering
        \includegraphics[width=\linewidth]{figs/imagecodec_intuition/00009_frame14_origin.pdf}
        \subcaption{Raw}
        \label{fig:00009_frame14_origin}
    \end{subfigure}
    \begin{subfigure}[b]{.22\linewidth}
        \centering
        \includegraphics[width=\linewidth]{figs/imagecodec_intuition/00009_frame14_origin_quality_1.pdf}
        \subcaption{JPEG}
        \label{fig:00009_frame14_origin_quality_1}
    \end{subfigure}
    \begin{subfigure}[b]{.22\linewidth}
        \centering
        \includegraphics[width=\linewidth]{figs/imagecodec_intuition/00009_frame14_origin_8x8.pdf}
        \subcaption{PNG}
        \label{fig:00009_frame14_origin_8x8}
    \end{subfigure}
        \begin{subfigure}[b]{.22\linewidth}
        \centering
        \includegraphics[width=\linewidth]{figs/imagecodec_intuition/00009_frame14_neural_q1.pdf}
        \subcaption{NN}
        \label{fig:00009_frame14_origin_8x8}
    \end{subfigure}
    \caption{Visual comparison of different image codecs at their lowest quality settings.}
    \label{fig:semantic_vs_pixel}
\end{figure}







\parab{Transmission protocol.} Second, we demonstrate the limitations of the OFDM protocol employed by AquaApp.
Fig.~\ref{fig:limitation_ofdm_distance} and Fig.~\ref{fig:limitation_ofdm_mobility} illustrate the BER under varying distance and mobility conditions, respectively.
Even in the least demanding scenario, we observe that the BER reaches 0.36\%,
corresponding to a recovery rate of only about 60\% (Fig.~\ref{fig:recovery}).
Furthermore, the BER increases rapidly when the distance exceeds 5 meters or
when the device is in motion, reaching up to 40\% and rendering underwater image transmission infeasible.

Since AquaApp is designed for small packet transmissions (\ie text messages), its transmission protocol struggles with channel dynamics, extended transmission ranges, and mobility. The decline in SNR with increasing transmission distance leads to high demodulation errors, even with carefully selected subcarriers.
Moreover, device movement can alter channel conditions and cause previously
selected subcarriers to degrade in quality over the course of transmission.
Addressing this requires frequent subcarrier reselection, which is
time-consuming (e.g., $\sim$3 seconds per selection).



\begin{figure}[t]
    \centering
    \begin{minipage}{0.23\textwidth}
        \centering
        \includegraphics[width=\textwidth]{figs/bar_Bit_Error_Rate_by_distance.pdf}
        \caption{BER {\em vs.} distance.}
        \label{fig:limitation_ofdm_distance}
    \end{minipage}\hfill
    \begin{minipage}{0.23\textwidth}
        \centering
        \includegraphics[width=\textwidth]{figs/bar_Bit_Error_Rate_by_motion_pattern.pdf}
        \caption{BER {\em vs.} mobility.}
        \label{fig:limitation_ofdm_mobility}
    \end{minipage}
\end{figure}




















\subsection{New Opportunity: Generative Compression}\label{sec:generative_compression}

Given the harsh conditions of underwater communication, an effective image
compression scheme must reduce the image size substantially while achieving
robustness against transmission errors, posing a challenge for existing image
codecs~\cite{wallace1991jpeg,webp2024,cheng2020image}. However, recent
advancements in deep generative
models~\cite{NIPS2014_5ca3e9b1,yu2024image,mao2024extreme} offer a promising
alternative---\textit{generative compression}.


Fig.~\ref{fig:semantic_compression} depicts the workflow of generative compression.
The encoder transforms a raw image into a sequence of \emph{tokens},
which represent \emph{indices}\footnote{The terms ``tokens'' and ``indices'' are
commonly used interchangeably.} of embeddings (\ie feature vectors)
in a pretrained codebook.
Unlike previous neural codecs~\cite{cheng2020image} that encode images
as a large set of feature values, this token-based method provides
a more bandwidth-efficient representation of semantic information~\cite{yu2024image,mao2024extreme}.
After the image tokens are transmitted over the underwater channel,
the decoder maps them to their corresponding embeddings and
reconstructs the original image.

Generative compression achieves efficient compression with
``graceful degradation''~\cite{santurkar2018generative},
where reconstructed image quality gradually declines
as bandwidth decreases or transmission errors occur.
As we will show in our experiments, generative compression operates
at bandwidth levels beyond the reach of convention methods and reduces
transmission latency by half compared with neural codecs.
Moreover, even in the presence of transmission errors,
the received image remains both decodable and semantically meaningful.










\begin{figure}[t]
    \centering
    \includegraphics[width=\linewidth]{figs/generative_compression.pdf}
    \caption{Overview of generative image compression.}
    \label{fig:semantic_compression}
\end{figure}




\parab{Summary:}
\begin{itemize}[topsep=2pt,noitemsep,leftmargin=*]
   \item Existing underwater communication systems, designed for small packet transmissions, are inadequate for handling image data.%
   \item Both traditional and neural image codecs face significant challenges in underwater environments due to limited bandwidth and vulnerability to transmission errors.
   \item Generative image compression presents a promising solution for reliable underwater image transmission.
\end{itemize}










