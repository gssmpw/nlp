\section{Related Work}
\label{related_work}

\subsection{Traditional Machine Learning Algorithms} \label{traditional_ML}

Dubbioso et al. \cite{DUBBIOSO2024105706} extracted a set of acoustic features from different tasks, performed feature selection strategies, and trained a Decision Tree classifier for differentiating healthy subjects from non-healthy ones and predicting dysarthria severity levels in ALS patients. Experiments were performed on different tasks, including reading, monologue, and vocalization. In \cite{wisler-etal-2019-speech}, the authors extracted a set of acoustic and articulatory features and trained Ridge regression and a Support Vector Machine to predict the ALSFRS-R score. In terms of the acoustic features, the authors used MFCC, their delta, and delta-delta and computed some statistics, e.g., mean, standard deviation. Regarding articulatory features, the authors computed a distance matrix and computed some statistics, e.g., skewness, kurtosis, and so on. The study in \cite{info:doi/10.2196/21331} was focused on the detection of ALS with bulbar involvement. The authors extracted a set of features, including jitter, shimmer, harmonics-to-noise ratio, pitch, and so on. Principal component analysis was used for dimensionality reduction. Finally, the authors trained the following machine learning classifiers: SVM, neural network with a hidden layer, LDA, LR, Naive Bayes, and Random Forest (RF). Vashkevich and Rushkevich \cite{VASHKEVICH2021102350} proposed a study based on voice analysis to detect ALS patients. Specifically, the authors extracted a set of acoustic features from phonation vowels /a/ and /i/, performed feature selection algorithms, and trained a linear discriminant analysis for the classification purposes. The authors in \cite{simmatis2024detecting} extracted a set of acoustic features and trained a bayesian logistic regression model for differentiating the following groups: \textit{(i)} control vs ALS, \textit{(ii)} control
vs ALS-early, and \textit{(iii)} ALS-early vs ALS-late . The main limitation of this study is related to the imbalanced dataset between ALS and control participants. Specifically, the dataset includes 119 ALS patients and 22 healthy controls.


\subsection{Deep Neural Networks} \label{DNNs_related}

Two different transfer learning strategies were introduced in \cite{bhattacharjee23_interspeech}. Specifically, the authors explored fine-tuning and multitask learning frameworks. As auxiliary tasks, the authors used input feature reconstruction and gender classification. The authors used as input to the deep neural networks a vector consisting of MFCC (excluding energy coefficient) with delta and double delta features. The deep neural network comprised a series of dense layers. Three set of experiments were performed in the study of \cite{9179503}, including \textit{(1)} classification among ALS, Parkinson disease (PD), and Healthy control, \textit{(2)} 5-class ALS severity classification based on ALSFRS-R, and \textit{(3)} 3-class PD severity classification. The authors used as input log-Mel spectrograms and passed them through CNN layers followed by fully connected layers. Four tasks were used, including spontaneous speech, image description, sustained phonation, and diadochokinetic rate. In \cite{vieira2022machine}, the authors segmented the audio file into non-overlapping audio frames, converted it into log-Mel spectrogram and passed each frame through CNN layers. Next, they aggregated each frame's output to get the final prediction for the entire voice signal. The task was the prediction of the ALSFRS-R score. Three approaches were employed for classifying ALS patients and healthy control in \cite{an18c_interspeech}. In terms of the first approach, the authors extracted features using the openSMILE toolkit and trained an Artificial Neural Network with one hidden layer. Regarding the other two approaches, the authors utilized filterbank, delta, and delta-delta as input to time-CNNs and frequency-CNNs.

\subsection{Related Work Review Findings}

As is evident in Section~\ref{traditional_ML}, existing studies focus on the extraction of acoustic features and the train of shallow machine learning classifiers, which constitutes a tedious procedure and does not generalize to new subjects. As is presented in Section~\ref{DNNs_related}, existing studies convert the audio files into log-Mel spectrograms, delta, and delta-delta and pass them through CNN layers followed by dense layers.

Our study is different from existing studies, since we present the first study incorporating hypernetworks into a deep neural network for recognizing dysarthria in ALS patients. Also, this study has been performed in a newly collected dataset, which is publicly available.