
\section{Conclusion}

This paper challenges the prevailing trend of scaling data and computational resources in atomic property prediction by demonstrating that strategic data selection based on relevance can achieve comparable or superior performance with significantly fewer resources. We introduce the Chemical Similarity Index (CSI), a simple metric that quantifies upstream/downstream datasets alignment, enabling the selection of high-quality, task-relevant pretraining data. Our experiments reveal that smaller, focused datasets often outperform larger, mixed ones, and that indiscriminately adding data can degrade performance when relevance is low. These findings highlight that quality often outweighs quantity in pretraining, offering a more efficient and sustainable path forward for molecular machine learning. 
% With $24\times$ less computational budget we are able to achieve SOTA performance on several downstream tasks.

\section*{Acknowledgements}
The authors thank Mohamed Elhoseiny for his feedback. Yasir Ghunaim was supported by Saudi Aramco. The research reported in this publication was supported by funding from King Abdullah University of Science and Technology (KAUST) - Center of Excellence for Generative AI, under award number 5940 and SDAIA-KAUST Center of Excellence in Data Science, Artificial Intelligence (SDAIA-KAUST AI). 

\section*{Impact Statement}
This paper presents work whose goal is to advance the field of Machine Learning. There are many potential societal consequences of our work, none which we feel must be specifically highlighted here.
