
% \begin{table}[ht]
% \centering
% \caption{High and Low CSI (Ani1X + OC22) (Reviewed)}
% \setlength{\tabcolsep}{6pt} % Adjust column spacing
% \begin{tabular}{p{2.5cm}cccc}
% \toprule
% Pretrain. Source  & rMD17 & MD22 & SPICE & QM9  \\
% \midrule
% % OC22 (2M)   & 16.0 & 5.20 & 10.73 & 5.7  \\
% ANI-1x (2M) & \textbf{5.4} & \textbf{2.92} & \textbf{5.13} & \textbf{2.9} \\ 
% \rowcolor{Gray} \quad + OC22 (1M) &  7.8 & 3.21 & 6.00 & 3.1 \\
% \bottomrule
% \end{tabular}
% \label{tab:high_low_CSI}
% \end{table}


\begin{figure}[b!]
    \centering
    \includegraphics[width=1.0\linewidth]{images/highCSI.png}
    \vspace{-0.3cm}
    \caption{\textbf{Impact of Adding Less Relevant Pretraining Data.} Adding \(1M\) OC22 samples to a \(2M\)-sample ANI-1x baseline worsens downstream performance despite a larger pretraining budget. This highlights the importance of dataset relevance and the CSI metric for effective pretraining.}
    \label{fig:high_low_CSI}
\end{figure}