%%%%%%%% ICML 2025 EXAMPLE LATEX SUBMISSION FILE %%%%%%%%%%%%%%%%%

\documentclass{article}

% Recommended, but optional, packages for figures and better typesetting:
\usepackage{microtype}
\usepackage{graphicx}
\usepackage{subfigure}
\usepackage{booktabs} % for professional tables
\usepackage{multirow}

% hyperref makes hyperlinks in the resulting PDF.
% If your build breaks (sometimes temporarily if a hyperlink spans a page)
% please comment out the following usepackage line and replace
% \usepackage{icml2025} with \usepackage[nohyperref]{icml2025} above.
\usepackage{hyperref}

\newcommand{\yasir}[1]{\textcolor{red}{Yasir: #1}}

% \linespread{0.995}

\usepackage{enumitem}


% Attempt to make hyperref and algorithmic work together better:
\newcommand{\theHalgorithm}{\arabic{algorithm}}

% Use the following line for the initial blind version submitted for review:
% \usepackage{icml2025}


% If accepted, instead use the following line for the camera-ready submission:
\usepackage[accepted]{icml2025}

% For theorems and such
\usepackage{amsmath}
\usepackage{amssymb}
\usepackage{mathtools}
\usepackage{amsthm}

% if you use cleveref..
\usepackage[capitalize,noabbrev]{cleveref}

%%%%%%%%%%%%%%%%%%%%%%%%%%%%%%%%
% THEOREMS
%%%%%%%%%%%%%%%%%%%%%%%%%%%%%%%%
\theoremstyle{plain}
\newtheorem{theorem}{Theorem}[section]
\newtheorem{proposition}[theorem]{Proposition}
\newtheorem{lemma}[theorem]{Lemma}
\newtheorem{corollary}[theorem]{Corollary}
\theoremstyle{definition}
\newtheorem{definition}[theorem]{Definition}
\newtheorem{assumption}[theorem]{Assumption}
\theoremstyle{remark}
\newtheorem{remark}[theorem]{Remark}

% Todonotes is useful during development; simply uncomment the next line
%    and comment out the line below the next line to turn off comments
%\usepackage[disable,textsize=tiny]{todonotes}
\usepackage[textsize=tiny]{todonotes}

\newcommand{\eg}{e.g.,\ }
\newcommand{\ie}{i.e.,\ }

\usepackage[most]{tcolorbox}

% Define a custom tcolorbox environment
\tcbuselibrary{listingsutf8}
\newtcolorbox{fancybox}[1][]{%
    colback=gray!10, % Light grey background
    colframe=white, % White frame to make it borderless
    sharp corners,  % Remove rounded corners
    boxrule=0pt,    % No border
    enhanced,
    #1               % Allow additional options
}

% The \icmltitle you define below is probably too long as a header.
% Therefore, a short form for the running title is supplied here:
\icmltitlerunning{Towards Data-Efficient Pretraining for Atomic Property Prediction}
% \usepackage[table]{xcolor}  % Allow table colors


\definecolor{kleinblue}{RGB}{0, 47, 167}
\definecolor{question}{RGB}{25, 25, 112}
\definecolor{darkyellow}{RGB}{255, 191, 0}
\definecolor{Gray}{gray}{0.95}  % Define a dark gray color



\usepackage{amssymb} % For symbols like \textquestiondown
\usepackage{etoolbox} % For defining reusable commands
\usepackage{fontawesome5} % For decorative symbols

\newcommand{\highlighttext}[1]{%
    \textcolor{kleinblue}{\textcolor{darkyellow}{\faLightbulb}~\textbf{\textit{#1}}}%
}

\newcommand\blfootnote[1]{%
  \begingroup
  \renewcommand\thefootnote{}\footnote{#1}%
  \addtocounter{footnote}{-1}%
  \endgroup
}
\begin{document}

% Hide ICML footnote. TODO: uncomment this for the final version
\makeatletter
\renewcommand{\ICML@appearing}{}
\makeatother



\twocolumn[
\icmltitle{Towards Data-Efficient Pretraining for Atomic Property Prediction}

% It is OKAY to include author information, even for blind
% submissions: the style file will automatically remove it for you
% unless you've provided the [accepted] option to the icml2025
% package.

% List of affiliations: The first argument should be a (short)
% identifier you will use later to specify author affiliations
% Academic affiliations should list Department, University, City, Region, Country
% Industry affiliations should list Company, City, Region, Country

% You can specify symbols, otherwise they are numbered in order.
% Ideally, you should not use this facility. Affiliations will be numbered
% in order of appearance and this is the preferred way.
\icmlsetsymbol{equal}{*}

\begin{icmlauthorlist}
\icmlauthor{Yasir Ghunaim}{kaust}
\icmlauthor{Hasan Abed Al Kader Hammoud}{kaust}
\icmlauthor{Bernard Ghanem}{kaust}
% \icmlauthor{Firstname1 Lastname1}{equal,yyy}
% \icmlauthor{Firstname2 Lastname2}{equal,yyy,comp}
% \icmlauthor{Firstname3 Lastname3}{comp}
% \icmlauthor{Firstname4 Lastname4}{sch}
% \icmlauthor{Firstname5 Lastname5}{yyy}
% \icmlauthor{Firstname6 Lastname6}{sch,yyy,comp}
% \icmlauthor{Firstname7 Lastname7}{comp}
% \icmlauthor{Firstname8 Lastname8}{sch}
% \icmlauthor{Firstname8 Lastname8}{yyy,comp}
%\icmlauthor{}{sch}
%\icmlauthor{}{sch}
\end{icmlauthorlist}

% \icmlaffiliation{yyy}{Department of XXX, University of YYY, Location, Country}
% \icmlaffiliation{comp}{Company Name, Location, Country}
% \icmlaffiliation{sch}{School of ZZZ, Institute of WWW, Location, Country}
\icmlaffiliation{kaust}{King Abdullah University of Science and Technology (KAUST), Thuwal, Saudi Arabia}

\icmlcorrespondingauthor{Yasir Ghunaim}{yasir.ghunaim@kaust.edu.sa}
% \icmlcorrespondingauthor{Firstname1 Lastname1}{first1.last1@xxx.edu}
% \icmlcorrespondingauthor{Firstname2 Lastname2}{first2.last2@www.uk}

% You may provide any keywords that you
% find helpful for describing your paper; these are used to populate
% the "keywords" metadata in the PDF but will not be shown in the document
\icmlkeywords{Machine Learning, ICML}

\vskip 0.3in
]

% this must go after the closing bracket ] following \twocolumn[ ...

% This command actually creates the footnote in the first column
% listing the affiliations and the copyright notice.
% The command takes one argument, which is text to display at the start of the footnote.
% The \icmlEqualContribution command is standard text for equal contribution.
% Remove it (just {}) if you do not need this facility.

\printAffiliationsAndNotice{}  % leave blank if no need to mention equal contribution
% \printAffiliationsAndNotice{\icmlEqualContribution} % otherwise use the standard text.



%%%%%%%%% ABSTRACT
\begin{abstract}

% Recent works to jointly reconstruct 3D human and object from a single RGB image, are mostly model-based, that fail to capture the fine details of the clothed human body and object surface. In this paper, we introduce ReCHOR, a novel, model-free, first-method to produce realistic clothed human-object reconstructions from a monocular view. This is extremely challenging due to human-object occlusions, diverse interactions and depth ambiguity, as it needs to infer both 3D spatial awareness and high resolution details. Our core idea is based on estimating neural implicit representations for human and object respectively by an attention-based neural implicit model that attends to pixel-aligned features from both the global human-object image for spatial awareness and  the local separate view of human and object images for high quality details. Additionally, the network is conditioned on semantic features from an initial estimated human-object pose prior and a generative diffusion model that inpaints occluded regions, thus enabling the retrieval of details from them.
% We also propose a synthetic dataset with rendered scenes of diverse, inter-occluded 3D human and object scans, to train our network. We evaluate our method on the synthetic and real world BEHAVE dataset. Our experiments show that our method outperforms the SOTA in achieving realistic clothed human-object reconstructions.
Recent approaches to jointly reconstruct 3D humans and objects from a single RGB image represent 3D shapes with template-based or coarse models, which fail to capture details of loose clothing on human bodies. In this paper, we introduce a novel implicit approach for jointly reconstructing realistic 3D clothed humans and objects from a monocular view. For the first time, we model both the human and the object with an implicit representation, allowing to capture more realistic details such as clothing. This task is extremely challenging due to human-object occlusions and the lack of 3D information in 2D images, often leading to poor detail reconstruction and depth ambiguity. To address these problems, we propose a novel attention-based neural implicit model that leverages image pixel alignment from both the input human-object image for a global understanding of the human-object scene and from local separate views of the human and object images to improve realism with, for example, clothing details. Additionally, the network is conditioned on semantic features derived from an estimated human-object pose prior, which provides 3D spatial information about the shared space of humans and objects. To handle human occlusion caused by objects, we use a generative diffusion model that inpaints the occluded regions, recovering otherwise lost details. For training and evaluation, we introduce a synthetic dataset featuring rendered scenes of inter-occluded 3D human scans and diverse objects. Extensive evaluation on both synthetic and real-world datasets demonstrates the superior quality of the proposed human-object reconstructions over competitive methods.
\end{abstract}

%%%%%%%%% BODY TEXT
\section{Introduction}\label{sec:intro}

In computational finance, Monte Carlo simulations are used extensively to estimate the expected value of financial payoffs based on the solution of stochastic differential equations (SDEs) which model the evolution of stock prices, interest rates, exchange rates and other quantities \cite{glasserman04}.  Monte Carlo methods are very general and flexible, but for high accuracy it requires generating a large number of costly SDE path approximations, which has motivated research into a number of variance reduction or, equivalently, cost reduction techniques. One such method is
Multilevel Monte Carlo (MLMC), which was proposed in \cite{GILES2008} and was adapted for various applications that are summarised in \cite{Giles_overview17} and successfully combined with other methods such as quasi-Monte Carlo methods. The main idea of MLMC is to approximate the payoff using different time stepping resolutions when numerically solving the underlying SDE and to generate an optimal number of samples on each level, such that the overall computational cost is minimised subject to the desired bound on the variance. %, such that the total computational cost is minimised. 
The computational savings come from the fact that most samples are computed on the coarser levels and hence are less expensive while only a few samples from the finest levels are required \cite{GILES2008}.


Among the directions in which the computational cost 
of MLMC methods could further be reduced, an important avenue is the use of lower precision calculations, especially for the first Monte Carlo levels where the targeted accuracy is relatively low. 
 An overview of the research on mixed precision for the standard Monte Carlo (MC) framework is provided in \cite{ChowMixedPrecisionStandardMC} but only a few references study the potential of low precision computation in the MLMC framework \cite{Rounding_error_oliver}. To the best of our knowledge, the only MLMC framework with customised precision in the literature is \cite{brugger2014mixed}, but they use a uniform precision for all operations on each Monte Carlo level instead of optimising 
 the precision of each intermediary variable to reduce as much as possible the cost of path generation.
 
An important motivation for an MLMC framework with variable precision would be performing the low precision computations on reconfigurable hardware devices such as Field Programmable Gate Arrays (FPGAs). FPGAs contain customizable logic blocks and connectors that make it easy to adapt the digital circuit architecture for a specific application, leading to a highly parallel and optimised implementation. Therefore they are successfully exploited in applications that require high speed and have high computational workload, such as signal processing \cite{woods2008fpga}, and real time applications like high frequency trading \cite{HFT1,HFT2}. That is why a number of previous works in hardware architecture design implemented the MLMC algorithm to price financial options using FPGAs as accelerators, which resulted in improved speed and power efficiency compared to full CPU architectures \cite{Schryver2013AMM}. The paper \cite{lindsey2016domain} also proposed 
a Domain Specific Language to automate the configuration of FPGAs for this specific application. However, only \cite{brugger2014mixed} proposed a heuristic to reduce the precision in calculations.

In addition, all aforementioned works considered that the random number generation (RNG) is performed in single or double precision. Yet in most cases an important portion of the workload in the overall MLMC simulation comes from the RNG and in \cite{brugger2014mixed} this limited the total computational savings.
To reduce the cost of MLMC simulations in particular those based on the Geometric Brownian Motion (GBM), \cite{approximateICDF_Oliver, NestedOliver} have proposed to use approximate random numbers that are generated by applying an approximation of the inverse CDF to uniform random numbers. In \cite{NestedOliver}, the authors proposed a way to integrate these lower precision random variables into a \textit{nested} MLMC framework and completed a numerical analysis to bound the resulting error at each MC level by a product of the time step and the error in the random number approximation. The same authors show in \cite{approximateICDF_Oliver} that using approximate random variables reduces the cost of path generation by a factor 7.


In this paper we propose a nested MLMC framework that combines the use of approximate random normal variables and lower precision calculations to reduce the computational cost of MLMC even further than \cite{brugger2014mixed,NestedOliver}. We illustrate the efficiency of our framework in Matlab, after making several assumptions on the cost of operations and size of the errors that we carefully justify. We focus on the case of GBM and use the approximate RNG methods presented in \cite{approximateICDF_Oliver} as well as a new slightly modified method that combines CDF inversion and the central limit theorem. To choose the precision of the variables in the low precision path generation, we introduce a novel method to optimise the bit-widths. This optimisation is performed before the main path generation loop is executed and is based on a linear model of the payoff error  
due to rounding when computing in low precision. The error model relies on algorithmic differentiation in a similar manner to \cite{unifying-bwoptim,bitwidth-AD,ADAPT}. The bit-width optimisation procedure can be performed off-line, so this stage can be excluded from the on-line time complexity of our framework. The user specified desired accuracy is then enforced by calculating on-line the number of samples that need to be generated.

In terms of hardware design, we suggest implementing the low precision path generation on FPGAs and the full-precision ones on a CPU or GPU. 
The FPGA offers enough flexibility to define a separate bit-width for every variable in the low precision path generation, and can be reconfigured periodically to update the bit-widths when the market parameters have changed considerably. 


The paper is organized as follows : \Cref{sec:MLMC} introduces MLMC and nested MLMC to make clear the estimator that is implemented in our framework. Then in \Cref{sec:RNG} we detail the methods that could be used to obtain approximate random normally distributed numbers very cheaply for the low precision path generation. In \Cref{sec:error_model} and \Cref{sec:costModel} we propose an error model and a cost model (resp.) that we then use to formulate the optimisation problem that is solved to obtain the optimal bit-widths of fixed point variables in \Cref{sec:optimisation}. Finally we summarise our results and future directions in \Cref{sec:conclusion}.



\section{Related Work}
\label{sec:related_work}

The original investigation \cite{gibson1979ecological} on the relationship between visual perception and human action defines \emph{affordance} as the opportunities for interaction with the surrounding environment. Behavioral studies on regular and cognitively impaired persons have shown evidence that perception results in both visual and motor signals in the human brain. An extended study \cite{anderson2002attentional} shows that visual attention to the spatial characteristics of the perceived objects initiates automatic motor signals for different actions. In computer vision, human affordance learning involves novel pose prediction such that the estimated pose represents a valid human action within the scene context. The task is fundamental to many problems requiring robust semantic reasoning about the environment, such as human motion synthesis \cite{wang2021scene} and scene-aware human pose generation \cite{wang2017binge, roy2016multi, zhang2022inpaint, yao2023scene}.

Earlier methods of affordance learning have explored knowledge mining \cite{zhu2014reasoning} and multimodal feature cues \cite{roy2016multi} to address the problem. In \cite{zhu2014reasoning}, the authors use a Markov Logic Network for constructing a knowledge base by extracting several object attributes from different image and metadata sources, which can perform various downstream visual inference tasks without any additional classifier, including zero-shot affordance prediction. In \cite{roy2016multi}, the authors use depth map, surface normals, and segmentation map as multimodal cues to train a multi-scale convolutional neural network (CNN) for scene-level semantic label assignment associated with specific human actions. In \cite{do2018affordancenet}, the authors design a multi-branch end-to-end CNN with two separate pathways for object detection and affordance label assignment to achieve high real-time inference throughput. Researchers \cite{chuang2018learning} have also explored socially imposed constraints for affordance learning. In \cite{chuang2018learning}, the authors propose a graph neural network (GNN) to propagate contextual scene information from egocentric views for action-object affordance reasoning.

Probabilistic modeling of scene-aware human motion generation also involves semantic reasoning of human interaction with the environment. Initial works on human motion synthesis have taken different architectural approaches, such as sequence-to-sequence models \cite{barsoum2018hp}, generative adversarial networks (GAN) \cite{barsoum2018hp, cai2018deep, yang2018pose}, graph convolutional networks (GCN) \cite{yan2019convolutional}, and variational autoencoders (VAE) \cite{guo2020action2motion}. However, these methods have mostly ignored the role of environmental semantics. Due to potential uncertainty in human motion, in a recent approach \cite{wang2021scene}, the authors address such motion synthesis with a GAN conditioned on scene attributes and motion trajectory to predict probable body pose dynamics.

One key challenge of human affordance generation in 2D scenes is the lack of large-scale datasets with rich pose annotations. In \cite{wang2017binge}, the authors compile the only public dataset of annotated human body poses in complex 2D indoor scenes by extracting frames from sitcom videos. Aiming to generate a contextually valid human affordance at a user-defined location, the authors propose sampling the scale and deformation parameters for an existing human pose template using a VAE conditioned on the localized image patches as scene context. In \cite{zhang2022inpaint}, the authors introduce a two-stage GAN architecture for achieving a similar goal by estimating the affine bounding box parameters to localize a probable human in the scene and then generating a potential body pose at that location. The method uses the input scene, corresponding depth, and segmentation maps as semantic guidance. In \cite{yao2023scene}, the authors propose a transformer-based approach with knowledge distillation for generating human affordances in 2D indoor scenes.


\section{Problem Formulation}
\label{sec:formulation}

% Formulation: 
% 1. Environment + Human Instruction -> Robot Action
% 2. Leader-Follower setting: Human as leader, robot as follower
% 3. Robot's Ability: Interactive Motion, Manipulation, Idle
% 4. Real-Time Design: Low-latency instruction: Interuption


In this work, we consider the interaction as a leader-follower formulation~\cite{vianelloHumanHumanoidInteractionCooperation2021}, where the human is the leader and the humanoid robot is the follower.
Define $\mathcal{I}$ as the set of human intentions and $\mathcal{K}$ as the set of robot skills.
At time step $t$, the leader shows an intention $I_t \in \mathcal{I}$ for the follower to perform a skill $K_t \in \mathcal{K}$, such as picking up a can, brushing a plate, or stamping a file.
We assume one intention corresponds to at most one skill, and the robot should be able to switch between skills in real time. The map function is defined as $f: \mathcal{I} \rightarrow \mathcal{K}$.

% Type and property of task:
    % - interactive Motion, manipulation, idle
    % - real-time design: low-latency instruction, in task reflection to human.
The skills of the robot can be categorized into three types: interactive motion, manipulation, and idle.
The interactive motion skills require the robot to perform expressive and diverse behavior, and the manipulation skills require the robot to interact with objects in the environment precisely. 
When the human leader does not show any intention, the robot will be in an idle state and do nothing.
The real-time interaction design requires the robot to respond to the leader's intentions with low latency and in-skill reflection to the human leader.
Low latency requires the robot to predict the leader's intention in real time and interrupt the current skill when the leader shows a new intention.
In-skill reflection requires the robot to react to human motion and environment even if the human intention is not changed, such as pausing current movement if the robot collides with the human or the target object is not reachable.

% The design of the idle state ensures the robustness of the interaction process. 
% The robot should avoid being affected by the uninstructed actions of a human, for example, a human might pick up a Coke can on a table. This behavior is similar to pointing at a Coke can for the robot to pick up, but the robot should be able to recognize it as idle.

% To ensure the safety of the interaction process, the interaction framework also includes a safety supervisor module, which monitors the robot's behavior. 
% When potential harm to the human leader is detected, the safety supervisor will stop the robot's action immediately and only allow the robot to continue the interaction when safety is guaranteed.

We formulate the observation space $\mathcal{O}$ of a humanoid robot as the combination of the
environment state $\mathcal{E}$ and the human behavior $\mathcal{H}$, i.e., $\mathcal{O} = \mathcal{E}  \bigodot  \mathcal{H} $.
The environment state $\mathcal{E}$ includes the robot's proprioception and the object state. The human behavior $\mathcal{H}$ consists of the leader's intention and the leader's behavior. 
To reduce the complexity of observation, our framework decomposes interaction policy into several sub-modules and separately designs the observation space for each module.
Compared to end-to-end models, this decomposed design allows the robot to learn humanoid-human-object interaction from human-object-human demonstration data, which is more sample-efficient and scalable.
% Main challenge: the diversity of human apperance, including clothing and body shape.


% Real-time Design
% To enable real-time interaction, \our is designed to respond to the leader's instruction with a low-latency reactive planner and interruptable low-level skills.
% The reactive planner predicts the leader's intention with 30Hz, and once the planner finds the current intention conflicts with the executing task, the planner will try to interrupt the current task and switch to the new task immediately.




\section{Experiments}
\label{sec:experiment}

Experiments are carried out on NVIDIA RTX4090 GPUs using PyTorch 2.2.0 \cite{paszke2019pytorch} and the rotation detection tool kits: MMRotate 1.0.0 \cite{zhou2022mmrotate}. All the experiments follow the same hyper-parameters (learning rate, batch size, optimizer, etc.).

Average precision (AP) is adopted as the primary metric. All the models are configured upon ResNet50 \cite{he2016deep} and trained with AdamW \cite{loshchilov2018decoupled}.
\textbf{1) Learning rate.} Initialized at 5e-5, warm-up for 500 iterations, and divided by ten at each decay step. 
\textbf{2) Epochs.} 72 for HRSC; 12 for the others.
\textbf{3) Augmentation.} Random rotation/flip for HRSC; random flip for the others.
\textbf{4) Image size.} Split into 1,024 $\times$ 1,024 with an overlap of 200 for DOTA/FAIR1M/STAR; scaled to 800 $\times$ 800 for others.
\textbf{5) Multi-scale.} All experiments evaluated without multi-scale technique \cite{zhou2022mmrotate}. 
\textbf{6) Datasets.} Six remote sensing and one retail scene datasets, covering all datasets used by the main counterparts \cite{yu2024point2rbox, luo2024pointobb, cao2023p2rbox}:

\begin{table*}[!tb]
\fontsize{8.5pt}{10pt}\selectfont
\setlength{\tabcolsep}{0.65mm}
\setlength{\aboverulesep}{0.4ex}
\setlength{\belowrulesep}{0.4ex}
\setlength{\abovecaptionskip}{1.5mm}
\centering
\begin{tabular}{l|c|c|c|c|c|c|c|c|c|c}
\toprule
{\textbf{Methods}} & {*} & {\textbf{\,DOTA-v1.0\,}} & {\textbf{\,DOTA-v1.5\,}} & {\textbf{\,DOTA-v2.0\,}} & {\textbf{~~DIOR~~}} & {\textbf{~~HRSC~~}} & {\textbf{\,FAIR1M\,}} & {\textbf{~~STAR~~}} & {\textbf{\,SKU110K\,}} & {\textbf{~~RSAR~~}} \\
\hline
\rowcolor{gray!20} \multicolumn{11}{l}{$\blacktriangledown$ \textit{RBox-supervised OOD}} \\ \hline
RetinaNet (2017) \cite{lin2017focal} & \checkmark & 68.69 & 60.57        & 47.00 & 54.96 & 84.49   & 37.67   & 21.80 & 78.50 & 57.67  \\
GWD (2021) \cite{yang2021rethinking} & \checkmark & 71.66 & 63.27        & 48.87 & 57.60 & 86.67   & 39.11   & 25.30 & 79.16 & 57.80 \\
FCOS (2019) \cite{tian2019fcos} & \checkmark & 72.44 & 64.53        & 51.77    &  59.83  & 88.99  & 41.25   & \textbf{28.10} & 80.09 & \textbf{66.66} \\
S$^2$A-Net (2022) \cite{han2022align} & \checkmark & \textbf{75.81} & \textbf{66.53} & \textbf{52.39} & \textbf{61.41} & \textbf{90.10} & \textbf{42.44}   & 27.30 & \textbf{80.36} & 66.47 \\
\hline
\rowcolor{gray!20} \multicolumn{11}{l}{$\blacktriangledown$ \textit{HBox-supervised OOD}} \\ \hline
Sun et al. (2021) \cite{sun2021oriented} & $\times$ & 38.60 & - & - & - & - & - & - & - & - \\
KCR (2023) \cite{zhu2023knowledge} & \checkmark & - & - & - & - &  79.10  & -  & - & - & -  \\
H2RBox (2023) \cite{yang2023h2rbox} & \checkmark & 70.05 & 61.70        & 48.68    & 57.80 &  7.03  & 35.94  & 17.20 & 57.15 & 49.92    \\
H2RBox-v2 (2023) \cite{yu2023h2rboxv2} & \checkmark & 72.31 & 64.76 & 50.33 & 57.64 & \textbf{89.66} & \textbf{42.27} & \textbf{27.30} & \textbf{70.70} & \textbf{65.16} \\
AFWS (2024) \cite{lu2024afws} & \checkmark & \textbf{72.55} & \textbf{65.92} & \textbf{51.73} & \textbf{59.07} & - & 41.80 & - & - & - \\
\hline
\rowcolor{gray!20} \multicolumn{11}{l}{$\blacktriangledown$ \textit{Point-supervised OOD}} \\ \hline
P2RBox (2024) \cite{cao2023p2rbox}$^\dagger$ & $\times$ & \underline{59.04} & -        & - & - & -   & -  & -  & - & -  \\
PointSAM (2024) \cite{liu2024pointsam}$^\dagger$ & $\times$ & - & - & - & \textbf{46.20} & -   & -  & -  & - & - \\
PointOBB (2024) \cite{luo2024pointobb} & $\times$ & 30.08 & 10.66        & 5.53     &  37.31  & -   & 11.19 & 9.19  & - & 13.80    \\
Point2RBox+SK (2024) \cite{yu2024point2rbox}$^\dagger$ & \checkmark & 40.27 & 30.51        & 23.43    & 27.34 & 79.40   & 20.03 & 7.86  & 3.41 & 27.81    \\
PointOBB-v2 (2025) \cite{ren2024pointobbv2} & $\times$ & 41.68 & 30.59        & 20.64    &  39.56  & -   & 13.36 & 9.00  & 56.63 & 18.99   \\
PointOBB-v3 (2025) \cite{zhang2025pointobbv3} & $\checkmark$ & 41.20 & 31.25 & 22.82 & 37.60 & - & 11.42  & 11.31 & - & 15.84 \\
PointOBB-v3 (2025) \cite{zhang2025pointobbv3} & $\times$ & 49.24 & 33.79 & 23.52 & 40.18 & - & 18.35 & \underline{12.85} & - & 22.60 \\
\rowcolor{gray!20} Point2RBox-v2 (ours) & \checkmark & 51.00 & \underline{39.45} & \underline{27.11} & 34.70 & \underline{82.67} & \underline{25.72} & 7.80 & \underline{64.00} & \underline{28.60}
 \\
\rowcolor{gray!20} Point2RBox-v2 (ours) & $\times$ & \textbf{62.61} & \textbf{54.06}        & \textbf{38.79}   & \underline{44.45}  & \textbf{86.15}   & \textbf{34.71}  & \textbf{14.20} & \textbf{65.64} & \textbf{30.90}    \\
\bottomrule
\specialrule{0pt}{2pt}{0pt}
\multicolumn{11}{l}{$^*$Comparison tracks: \checkmark = End-to-end training and testing; $\times$ = Generating pseudo labels to train the FCOS detector (two-stage training).} \\
\multicolumn{11}{l}{$^\dagger$Using additional priors. P2RBox/PointSAM: Pre-trained SAM model; Point2RBox+SK: One-shot sketches for each class.} \\
\bottomrule
\end{tabular}
\caption{Accuracy (AP$_{50}$) comparisons on the DOTA-v1.0/1.5/2.0, DIOR, HRSC, FAIR1M, STAR, SKU110K, and RSAR datasets.}
\label{tab:exp_other}
\vspace{-4pt}
\end{table*}

\begin{itemize}
    \item \textbf{DOTA \cite{xia2018dota}.} DOTA-v1.0 has 2,806 aerial images annotated with 15 categories, while DOTA-v1.5/2.0 are the extended versions with more small objects and categories.
    
    \item \textbf{DIOR \cite{cheng2022anchor}.} It is an aerial image dataset re-annotated with RBoxes based on its original HBox version \cite{li2020object}, with a high variation in object size and high intra‐class diversity. 

    \item \textbf{HRSC \cite{liu2017hrsc}.} It contains ship instances on the sea and inshore. The train/val/test set includes 436/181/444 images.

    \item \textbf{FAIR1M \cite{sun2022fair1m}.} It has more than 1 million instances and more than 40,000 images for fine-grained object recognition in remote sensing imagery, annotated with 37 categories. The results are evaluated on FAIR1M-1.0.

    \item \textbf{STAR \cite{li2024star}.} It is extensive for scene graph generation, covering more than 210,000 objects with diverse spatial resolutions, classified into 48 fine-grained categories and precisely annotated with oriented bounding boxes. 

    \item \textbf{SKU110K \cite{pan2020dynamic}.} It focuses on the detection of densely packed retail scenes with 110,712 objects in 11,762 images. The density reaches 86 instances per image. 

    \item \textbf{RSAR \cite{zhang2025rsar}.} It is a remote sensing dataset based on Synthetic Aperture Radar (SAR) imagery with 6 categories.

\end{itemize}

\begin{table*}[!tb]
\fontsize{8.5pt}{10pt}\selectfont
\setlength{\tabcolsep}{2.08mm}
\setlength{\aboverulesep}{0.4ex}
\setlength{\belowrulesep}{0.4ex}
\setlength{\abovecaptionskip}{1.5mm}
\hspace{1pt}
\begin{minipage}[t]{0.315\linewidth}
\centering
\begin{tabular}{c|cc|cc}
\toprule
\multirow{2}{*}{$w_\text{O}$} & \multicolumn{2}{c|}{\textbf{DOTA}} & \multicolumn{2}{c}{\textbf{HRSC}} \\
                  & {E2E} & {FCOS} & {E2E} & {FCOS} \\ \midrule
3  & 48.76 & 61.62 & 81.85 & 84.36 \\
5  & 49.81 & 62.44 & 82.46 & 85.76 \\
\rowcolor{gray!20} 10 & \textbf{51.00} & \textbf{62.61} & \textbf{82.67} & \textbf{86.15} \\
30 & 45.88 & 57.83 & 81.56 & 85.61 \\
\bottomrule
\end{tabular}
\caption{Ablation with the weight of $\mathcal{L}_\text{O}$.}
\label{tab:abl_lo}
\end{minipage}
\quad
\begin{minipage}[t]{0.315\linewidth}
\centering
\begin{tabular}{c|cc|cc}
\toprule
\multirow{2}{*}{$w_\text{W}$} & \multicolumn{2}{c|}{\textbf{DOTA}} & \multicolumn{2}{c}{\textbf{HRSC}} \\
                  & {E2E} & {FCOS} & {E2E} & {FCOS} \\ \midrule
3  & 50.85 & 56.78 & 78.42 & 83.49 \\
\rowcolor{gray!20} 5  & \textbf{51.00} & \textbf{62.61} & \textbf{82.67} & \textbf{86.15} \\
10 & 49.15 & 60.54 & 30.37 & 35.13 \\
30 & 42.84 & 52.53 & 23.89 & 25.91 \\
\bottomrule
\end{tabular}
\caption{Ablation with the weight of $\mathcal{L}_\text{W}$.}
\label{tab:abl_lw}
\end{minipage}
\quad
\begin{minipage}[t]{0.315\linewidth}
\setlength{\tabcolsep}{2.04mm}
\centering
\begin{tabular}{c|cc|cc}
\toprule
\multirow{2}{*}{$w_\text{E}$} & \multicolumn{2}{c|}{\textbf{DOTA}} & \multicolumn{2}{c}{\textbf{HRSC}} \\
                  & {E2E} & {FCOS} & {E2E} & {FCOS} \\ \midrule
0.1 & 48.75 & 57.62 & 34.71 & 39.45 \\
\rowcolor{gray!20} 0.3 & 51.00 & 62.61 & \textbf{82.67} & \textbf{86.15} \\
0.5 & \textbf{51.36} & \textbf{62.63} & 76.85 & 85.22 \\
1.0 & 49.05 & 60.63 & 56.59 & 59.59 \\
\bottomrule
\end{tabular}
\caption{Ablation with the weight of $\mathcal{L}_\text{E}$.}
\label{tab:abl_le}
\end{minipage}
\vspace{-4pt}
\end{table*}

\begin{table*}[!tb]
\fontsize{8.5pt}{10pt}\selectfont
\setlength{\tabcolsep}{2.04mm}
\setlength{\aboverulesep}{0.4ex}
\setlength{\belowrulesep}{0.4ex}
\setlength{\abovecaptionskip}{1.5mm}
\hspace{1pt}
\begin{minipage}[t]{0.315\linewidth}
\centering
\begin{tabular}{c|cc|cc}
\toprule
\multirow{2}{*}{$w_\text{ss}$} & \multicolumn{2}{c|}{\textbf{DOTA}} & \multicolumn{2}{c}{\textbf{HRSC}} \\
                  & {E2E} & {FCOS} & {E2E} & {FCOS} \\ \midrule
0.1 & 49.28 & 59.66 & 73.66 & 78.92 \\
\rowcolor{gray!20} 1.0 & \textbf{51.00} & \textbf{62.61} & \textbf{82.67} & \textbf{86.15} \\
3.0 & 49.15 & 59.20 & 1.30  & 1.65 \\
\bottomrule
\end{tabular}
\caption{Ablation with the weight of $\mathcal{L}_\text{ss}$.}
\label{tab:abl_lss}
\end{minipage}
\quad
\begin{minipage}[t]{0.647\linewidth}
\setlength{\tabcolsep}{3.05mm}
\centering
\begin{tabular}{c|c|c||c|c|c}
\toprule
{R / F / S} & {\textbf{DOTA}} & {\textbf{HRSC}} & {R / F / S} & {\textbf{DOTA}} & {\textbf{HRSC}} \\
 \midrule
90\% / 10\% / 0\% & 60.42 & 85.46 & 80\% / 20\% / 0\%  & 59.46 & 84.73 \\
75\% / 0\% / 25\% & 60.79 & 86.22 & 60\% / 15\% / 25\% & 62.38 & 84.21 \\
\cellcolor{gray!20}68\% / 7\% / 25\% & \cellcolor{gray!20}\textbf{62.61} & \cellcolor{gray!20}\textbf{86.15} & 38\% / 37\% / 25\% & 45.87 & 8.56  \\
45\% / 5\% / 50\% & 60.55 & 85.34 & 40\% / 10\% / 50\% & 60.49 & 10.74 \\
\bottomrule
\end{tabular}
\caption{Ablation with the proportion of augmented views in self-supervision.}
\label{tab:abl_pro}
\end{minipage}
\vspace{-10pt}
\end{table*}

\subsection{Main Results on DOTA-v1.0}
\label{sec:experiment-main}

Table \ref{tab:exp_dota} compares Point2RBox-v2 with the state-of-the-art methods, which can be categorized into two tracks: 

\textbf{1) End-to-end training.} These methods apply the trained weakly-supervised detector directly to the test set. Without relying on priors, our approach demonstrates an improvement of 16.93\% (51.00\% vs. 34.07\%) compared to Point2RBox. Even when compared to Point2RBox+SK, which incorporates additional data-side priors (i.e. one-shot examples for each class), our method still outperforms it by 10.73\% (51.00\% vs. 40.27\%).

\textbf{2) Two-stage training.} These methods generate RBox labels on train/val sets, with which the FCOS detector is trained. In this two-stage mode, Point2RBox-v2 achieves an accuracy of 62.61\%, considerably surpassing PointOBB series. Remarkably, it even outperforms the SAM-powered method P2RBox by 3.57\% (62.61\% vs. 59.04\%).

\textbf{Class-wise analysis.} The FCOS detector trained with labels generated by Point2RBox-v2 achieves accuracy nearly equivalent to RBox-supervised FCOS across six high-density categories: SH (86.9\% vs. 87.1\%), SV (79.6\% vs. 79.8\%), LV (76.3\% vs. 79.8\%), PL (88.0\% vs. 89.1\%), ST (82.9\% vs. 84.6\%), and TC (89.1\% vs. 90.4\%). Interestingly, these six high-density categories account for 88\% of DOTA instances. By annotating these categories with points and generating RBoxes using Point2RBox-v2 while labeling the other sparse categories with RBoxes, we can significantly reduce annotation labor without sacrificing much accuracy, highlighting the valuable role our method can play.

\begin{figure*}[t!]
\setlength{\abovecaptionskip}{1.2mm}
\centering
\includegraphics[width=0.96\linewidth]{figs/case.pdf}
\caption{Qualitative analysis on failed cases and overlap cases.}
\label{fig:case}
\vspace{-6pt}
\end{figure*}

\subsection{Results on More Datasets}

The results are displayed in Table \ref{tab:exp_other}.
On more challenging DOTA-v1.5/2.0, Point2RBox-v2 presents a similar trend, 23.47\%/18.15\% higher than PointOBB-v2 in the pseudo-generation track. 
On the ship detection dataset HRSC, the gap between Point2RBox-v2 and RBox-supervised FCOS is only 2.84\% (86.15\% vs. 88.99\%).
DIOR is relatively sparse, leading to less improvement with our methods---lower than PointSAM (44.45\% vs. 46.20\%) but still higher than methods that do not use SAM. 
Our method also provides competitive performance on fine-grained datasets FAIR1M and STAR. 
In addition to remote sensing scenarios, we carry out experiments on SKU110K for densely packed retail scenes. Existing point-supervised methods struggle in this case, whereas Point2RBox-v2 achieves performance on par with HBox-supervised H2RBox (65.64\% vs. 57.15\%).

\begin{table}[!tb]
\fontsize{8.5pt}{10pt}\selectfont
\setlength{\tabcolsep}{1.78mm}
\setlength{\aboverulesep}{0.4ex}
\setlength{\belowrulesep}{0.4ex}
\setlength{\abovecaptionskip}{1.5mm}
\centering
\begin{tabular}{ccccc|cc|cc}
\toprule
\multicolumn{5}{c|}{\textbf{Modules}} & \multicolumn{2}{c|}{\textbf{DOTA}} & \multicolumn{2}{c}{\textbf{HRSC}} \\
$\mathcal{L}_\text{O}$ & $\mathcal{L}_\text{W}$ & $\mathcal{L}_\text{ss}$ & $\mathcal{L}_\text{E}$ & \textit{CP} & {E2E} & {FCOS} & {E2E} & {FCOS} \\ \midrule
\checkmark & & & & & 0.00 & 0.00 & 0.00 & 0.00 \\
\checkmark & \checkmark & & & & 41.54 & 52.98 & 17.96 & 19.64 \\
\checkmark & \checkmark & \checkmark & & & 46.64 & 54.26 & 18.10 & 22.13 \\
\checkmark & \checkmark & \checkmark & \checkmark & & 49.55 & 61.88 & 78.79 & 83.79 \\
& \checkmark & \checkmark & \checkmark & \checkmark & 48.58 & 59.56 & 20.35 & 24.76 \\
\checkmark & & \checkmark & \checkmark & \checkmark & 38.94 & 48.44 & 11.64 & 14.93 \\
\checkmark & \checkmark & \checkmark & & \checkmark & 47.08 & 55.05 & 19.58 & 21.78 \\
\rowcolor{gray!20} \checkmark & \checkmark & \checkmark & \checkmark & \checkmark & \textbf{51.00} & \textbf{62.61} & \textbf{82.67} & \textbf{86.15} \\
\bottomrule
\end{tabular}
\caption{Ablation with incremental addition of modules.}
\label{tab:abl_mod}
\vspace{-4pt}
\end{table}

\begin{table}[!tb]
\fontsize{8.5pt}{10pt}\selectfont
\setlength{\tabcolsep}{2.85mm}
\setlength{\aboverulesep}{0.4ex}
\setlength{\belowrulesep}{0.4ex}
\setlength{\abovecaptionskip}{1.5mm}
\centering
\begin{tabular}{c|c|c||c|c|c}
\toprule
16 & \cellcolor{gray!20}$K\!=\!24$ & 32 & 1.2 & \cellcolor{gray!20}$\beta\!=\!1.6$ & 2.0 \\ \midrule
50.87 & \cellcolor{gray!20}\textbf{51.00} & 48.08 & 48.14 & \cellcolor{gray!20}51.00 & \textbf{51.33} \\
\bottomrule
\end{tabular}
\caption{Ablation with $K$ and $\beta$ in edge loss on DOTA (E2E).}
\label{tab:abl_edgeparam}
\vspace{-4pt}
\end{table}

\begin{table}[!tb]
\fontsize{8.5pt}{10pt}\selectfont
\setlength{\tabcolsep}{1.75mm}
\setlength{\aboverulesep}{0.4ex}
\setlength{\belowrulesep}{0.4ex}
\setlength{\abovecaptionskip}{1.5mm}
\centering
\begin{tabular}{c|cc|cc|cc}
\toprule
\multirow{2}{*}{$\sigma$} & \multicolumn{2}{c|}{Point2RBox} & \multicolumn{2}{c|}{PointOBB-v2} & \multicolumn{2}{c}{Point2RBox-v2} \\
 & {\textbf{DOTA}} & {\textbf{HRSC}} & {\textbf{DOTA}} & {\textbf{HRSC}} & {\textbf{DOTA}} & {\textbf{HRSC}} \\ \midrule
0\%  & 40.27 & 79.40 & 44.85 & - & 62.61 & 86.15 \\
10\% & 39.60 & 78.81 & 42.30 & - & 61.58 & 85.76 \\
30\% & 38.42 & 78.28 & 38.46 & - & 60.31 & 85.71 \\
\bottomrule
\end{tabular}
\caption{Ablation with the inaccuracy in point annotations.}
\label{tab:abl_noise}
\vspace{-10pt}
\end{table}

\subsection{Ablation Studies}
\label{sec:experiment-ablation}

Tables \ref{tab:abl_lo}-\ref{tab:abl_noise} display the ablation studies on DOTA-v1.0 and HRSC. ``E2E'' denotes end-to-end training; ``FCOS'' denotes two-stage training (i.e. generating pseudo labels to train FCOS). The final values adopted are highlighted in gray.

\textbf{Weight of each loss.} Tables \ref{tab:abl_lo}-\ref{tab:abl_le} determine the weights of the proposed losses. Based on these experiments, the weights $(w_\text{O},w_\text{W},w_\text{E},w_\text{ss})$ are set to $(10, 5, 0.3, 1)$.

\textbf{Proportion of augmented views.} Table \ref{tab:abl_pro} studies the proportion between rotation, flip, and scale. The results are reported with two-stage training (FCOS). Based on the results, the proportion is set to 68\%, 7\%, and 25\%.

\textbf{Incremental addition of modules.} Table \ref{tab:abl_mod} demonstrates the constraints from Gaussian and Voronoi achieve an accuracy of 52.98\% on DOTA. Adding consistency loss and edge loss further boosts it to 54.26\% and 61.88\%, respectively, whereas the improvement from copy-paste is 0.73\%. We also demonstrate the impact of omitting each core loss.

\textbf{Edge loss parameters.} We set $K=24$ and $\beta=1.6$ as they are observed to discern the correct edges during code development. Table \ref{tab:abl_edgeparam} provides a more precise ablation.

\textbf{Annotation inaccuracy.} We offset the annotated points by a noise from the uniform distribution $\left[-\sigma H, +\sigma H \right ]$, where $H$ is the height of objects. Table \ref{tab:abl_noise} shows that the AP$_{50}$ of Point2RBox-v2 decreases by less than 3\% when noise is added to point annotations, demonstrating the robustness of the proposed learning mechanisms.

\subsection{More Discussions}
\label{sec:experiment-discussions}

The qualitative analysis on the failed/overlap cases is shown in Fig. \ref{fig:case}. \textbf{1) Failed cases.} Although our method performs well overall, it struggles with certain categories that are sparse and not constrained by other objects. \textbf{2) Overlap cases.} 
Minimizing overlap as a soft constraint during training does not entirely eliminate overlap. Once trained, the model remains robust to some overlap during inference.

\section{Beyond In-Distribution}
\label{5_OOD}

Recall that our pretraining process is conducted on upstream tasks involving molecules and catalysts, with energy and force as targets. For downstream tasks with different labels (e.g., band gap in QMOF) or from distinct chemical domains such as materials (e.g., MatBench and QMOF), we classify these as out-of-distribution (OOD). While our main results focused on ID evaluation, here we explore our metric's applicability to OOD tasks. Specifically, we examine three cases: the Band Gap property from QMOF~\cite{rosen2021machine}, Phonons (the first non-energy target in JMP tables) from MatBench~\cite{dunn2020benchmarking}, and $\Delta_\epsilon$ from QM9, explicitly categorized as OOD in the JMP paper.

In Figure \ref{fig:CSI_OOD}, we present the CSI values for OOD domains, where the OOD label ($\Delta_\epsilon$) for QM9 follows the same values as in Figure \ref{fig:main_CSI_score}. We observe that QMOF exhibits a pattern similar to other ID domains shown in Figure \ref{fig:main_CSI_score}. However, MatBench displays a distinct pattern, showing strong correlation with OC20 and OC22, followed by ANI-1x and Transition-1x. Next, we analyze the correlation between CSI and downstream performance under OOD evaluation.

Table \ref{tab:ood} shows that $\Delta_\epsilon$ in QM9 aligns with the CSI pattern, similar to ID evaluation, suggesting that CSI is effective for OOD in the label space. In QMOF, the different upstream sources achieve similar performance which lags behind the full pretraining by JMP. For MatBench (evaluated over 5 folds), OC22 achieves the best mean performance while OC20 lags behind, despite our metric predicting both to be equally suitable. Additionally, for both QMOF and MatBench, mixed pretraining variants generalize better than individual sources. This suggests that when the downstream domain differs from all upstream sources, mixing diverse upstream domains provides the best performance.

While our CSI metric holds well for ID evaluation, further research is needed to improve OOD generalization across different chemical domains. One direction is to explore additional upstream sources to better capture domain variations. Another potential extension is leveraging foundation models trained beyond energy and forces, which could enhance feature extraction and improve similarity assessments.


% - Reiterate what is OOD.
% - Mention that we study X,Y,Z datasets as OOD and why it's considered OOD.
% - Mention that we can still select the best dataset among the upstream.
% - Why we need better metrics or possibly foundation models for the CSI metric for this problem ?
% - While it works for qmof it doesnt for matbench.
% - mention mixed data as a good solution here.
% - Hopefully fid based sampling works.


% \textcolor{blue}{Limitations we discuss OOD}

% \textit{OOD experiments}
% \textit{we say metric is naive but it showed what we wanted to show that superior subsets exist.}
% \textit{Here we only use node features however ufture better metrics might invovle edge features .. }


% \section{Qualitative Analysis}
% Analyze low and high FID samples to understand if there is any pattern that we can highlight. 
\section{Conclusion and future directions} \label{sec:conclusion}

In this paper we proposed a nested MLMC framework that offers important computational savings by performing most calculations in low precision and exploiting approximate random normal variables for the low precision path calculations. The low precision calculations could be performed in fixed precision on an FPGA for greater efficiency, and we suggested a procedure to optimise the bit-widths of every variable at each Monte Carlo level. This is an important improvement over previous mixed precision MLMC frameworks which held the lower precision fixed \cite{Rounding_error_oliver} or defined uniform bit-width at every level heuristically \cite{brugger2014mixed}. Our numerical results suggest that for the first levels our procedure reduces the cost at these levels by a factor 5 or 7. Hence the overall savings are significant since most paths are calculated on the first levels. Our approach would be even more efficient for the Milstein scheme because its higher order strong convergence leads to a greater proportion of the computational costs being on the coarsest levels.

The next stage of the research project will be to implement the RNG methods and the nested framework on FPGAs to determine the hardware requirements and confirm the extent of the computational savings. It would also be good to compare the performance benefits to using half-precision floating point arithmetic on GPUs or CPUs for the low-accuracy computations.





\bibliography{main}
\bibliographystyle{icml2025}


%%%%%%%%%%%%%%%%%%%%%%%%%%%%%%%%%%%%%%%%%%%%%%%%%%%%%%%%%%%%%%%%%%%%%%%%%%%%%%%
%%%%%%%%%%%%%%%%%%%%%%%%%%%%%%%%%%%%%%%%%%%%%%%%%%%%%%%%%%%%%%%%%%%%%%%%%%%%%%%
% APPENDIX
%%%%%%%%%%%%%%%%%%%%%%%%%%%%%%%%%%%%%%%%%%%%%%%%%%%%%%%%%%%%%%%%%%%%%%%%%%%%%%%
%%%%%%%%%%%%%%%%%%%%%%%%%%%%%%%%%%%%%%%%%%%%%%%%%%%%%%%%%%%%%%%%%%%%%%%%%%%%%%%
\newpage
\appendix
\onecolumn

\section{More Epochs or More Data?}

In this section, we explore the trade-off between increasing the number of training epochs and expanding the dataset size under a fixed computational budget. Specifically, we aim to answer the following question:

\highlighttext{Given a fixed computational budget, is it more effective to train on a smaller dataset for more epochs or to train on a larger dataset for fewer epochs?}

\textbf{Setup.} To investigate this question, we compare two scenarios under the same computational budget of \(10M\) samples: (1) training on \(2M\) samples for 5 epochs, and (2) training on \(1M\) samples for 10 epochs. We evaluate the performance of models pretrained on ANI-1x, Transition-1x, OC20, and OC22, and fine-tune them on the downstream datasets: rMD17, MD22, SPICE, and QM9. For comparison, we also include the results of JMP-L and JMP-S, which use \(120M\) samples for 2 epochs.

\textbf{Results.} Table \ref{tab:rehearsal} presents the downstream performance for the two scenarios. Across all datasets, ANI-1x consistently achieves the best performance, regardless of whether the model is trained on \(2M\) samples for 5 epochs or \(1M\) samples for 10 epochs. For example, on rMD17, ANI-1x achieves a test error of 5.4 in both scenarios, outperforming JMP-S (6.7) and JMP-L (5.1). Similarly, on SPICE, ANI-1x achieves a test error of 5.08 (2M samples, 5 epochs) and 5.04 (1M samples, 10 epochs), compared to 5.71 for JMP-S and 4.75 for JMP-L.

Interestingly, increasing the number of epochs from 5 to 10 while reducing the dataset size from \(2M\) to \(1M\) does not significantly degrade performance for ANI-1x. This suggests that for highly relevant datasets like ANI-1x, training on fewer samples for more epochs can be as effective as training on more samples for fewer epochs. However, for less relevant datasets like OC20 and OC22, increasing the number of epochs does not compensate for the reduced dataset size, as their performance degrades significantly.

\textbf{Conclusion.} Our findings indicate that the choice between more epochs or more data depends on the relevance of the dataset to the downstream task. For highly relevant datasets like ANI-1x, training on fewer samples for more epochs can yield comparable or even superior performance, while for less relevant datasets, increasing the dataset size is more beneficial. This further underscores the importance of dataset quality and relevance, as quantified by CSI, in determining the optimal pretraining strategy.


% \begin{table*}[ht]
% \centering
% \caption{Diversity or rehearsal (reviewed)}
% \begin{tabular}{lllcccc}
% \hline
% Pretraining Budget & Pretraining Source & Backbone & rMD17 & MD22 & SPICE & QM9  \\
% \hline
%  & ANI1x & GemNet-OC-S   & 5.4 & 2.88 & 5.04 & 2.9 \\
%                         & Transition1x              & GemNet-OC-S & 10.6 & 3.79 & 7.50 & 3.1 \\
%                         & OC20                      & GemNet-OC-S & 14.8 & 4.67 & 10.16 & 4.9 \\
%                         & OC22                      & GemNet-OC-S & 17.3 & 5.24 & 11.06 & 5.4 \\
% \hline
% \end{tabular}
% \label{tab:rehearsal}
% \end{table*}



\begin{table*}[h!]
\centering
\caption{Trade-off between increasing the number of samples and the number of epochs. We report the MAE for various downstream tasks while varying the pretraining sample count and epoch count simultaneously. \(\mathcal{C}\), \(\mathcal{N}\), and \(\mathcal{E}\) denote the computational budget, number of samples, and number of epochs, respectively.}
\setlength{\tabcolsep}{8pt}
\begin{tabular}{ccclccccc}
\toprule
$\mathbf{\mathcal{C}}$ & $\mathbf{\mathcal{N}}$ & $\mathbf{\mathcal{E}}$ & \textbf{Upstream Data} & \textbf{Backbone} & \textbf{rMD17} & \textbf{MD22} & \textbf{SPICE} & \textbf{QM9} \\
 &  &  &  &  & (meV/\AA) & (meV/\AA) & (meV/\AA) & (meV) \\
\midrule

\multirow{4}{*}{\( 10 \times 10^6 \)}   
&  \multirow{4}{*}{\( 2 \times 10^6 \)}  
&   \multirow{4}{*}{\( 5 \)}
    & ANI-1x         & \multirow{4}{*}{GemNet-OC-S} & \textbf{5.4} & \textbf{2.90} & \textbf{5.13} & \textbf{2.9} \\
    &&& Transition-1x  &  & 10.1 & 3.73 & 7.55 & 3.2  \\
    &&& OC20          &  & 14.6 & 4.53 & 8.74 & 4.8  \\
    &&& OC22          &  & 16.0 & 5.20 & 10.73 & 5.7  \\
\midrule



\multirow{4}{*}{\( 10 \times 10^6 \)}  
&  \multirow{4}{*}{\( 1 \times 10^6 \)}  
&   \multirow{4}{*}{\( 10 \)}
    & ANI1x         & \multirow{4}{*}{GemNet-OC-S}      & \textbf{5.4} & \textbf{2.88} & \textbf{5.04} & \textbf{ 2.9} \\
    &&& Transition1x              & & 10.6 & 3.79 & 7.50 & 3.1 \\
    &&& OC20                      & & 14.8 & 4.67 & 10.16 & 4.9 \\
    &&& OC22                      & & 17.3 & 5.24 & 11.06 & 5.4 \\

\bottomrule
\end{tabular}
\label{tab:rehearsal}
\end{table*}




\clearpage


\section{Additional Analysis on the Metric Design}

\begin{figure*}[h]
\centering
\includegraphics[width=0.75\linewidth]{images/CSI_bar_balanced_mean_equiformerV2_ID.png}
\caption{\textbf{Impact of using mean aggregation instead of flattening on CSI values.} We notice that the mean pooling incorrectly reduced the score for OC22 potentially due to over-smoothing.} 
\vspace{-0.4cm}
\label{fig:CSI_mean}
\end{figure*}


\begin{figure*}[h]
\centering
\includegraphics[width=0.75\linewidth]{images/CSI_bar_random_flat_equiformerV2_ID.png}
\caption{\textbf{Impact of using random sampling strategy instead of class-balanced sampling.} As highlighted in the long-tail analysis in Appendix \ref{long_tail}, random sampling can lead to class underrepresentation, potentially affecting the correlation between upstream and downstream tasks. Notably, both ANI-1x and Transition-1x exhibit different patterns compared to the class-balanced values reported in the main paper.}
\vspace{-0.4cm}
\label{fig:CSI_random}
\end{figure*}

\begin{figure*}[h]
\centering
\includegraphics[width=0.75\linewidth]{images/CSI_bar_balanced_flat_JMP_ID.png}
\caption{\textbf{The impact of using another backbone}. We use JMP pretrained model and show that similar insights are obtained where Ani-1x is shown as the most similar.} 
\vspace{-0.4cm}
\label{fig:CSI_JMP}
\end{figure*}

\clearpage

\section{Long-Tail Analysis}
\label{long_tail}
As discussed in Section \ref{3_setup}, molecular understanding datasets are typically imbalanced and exhibit a skewed distribution. Figure \ref{fig:upstream_sampling} illustrates the effect of \textit{random sampling} versus \textbf{class-balanced sampling}, which we employ in this study, in terms of class coverage. Notably, random sampling results in significant class underrepresentation, whereas class-balanced sampling ensures broader coverage across all classes.

\begin{figure*}[h]
\centering
\includegraphics[width=\linewidth]{images/upstream_sampling.png}
\caption{\textbf{Impact of sampling strategies on subset construction for feature extraction.} We sample 10K instances for each upstream task, highlighting the differences in class coverage between random and class-balanced sampling.} 
\label{fig:upstream_sampling}
\end{figure*}




\clearpage
\section{Implementation Details}
% QMOF we used a LR 5e-5 since the pretrained model on transition was giving NaN when using 8e-5. 
For both pretraining and fine-tuning experiments, we primarily follow the JMP hyperparameters. However, due to resource constraints requiring smaller batch sizes compared to JMP, we adjusted the learning rate to ensure training stability, as detailed below.

\textbf{For pretraining,} we use a batch size of 20 and a learning rate (LR) of 1e-4 for the small backbone (GemNet-OC-S). For the large backbone (GemNet-OC-L), the batch size is reduced to 12 to fit GPU memory. Additionally, when training with the OC22 dataset on the large backbone, a LR of 1e-4 caused gradient instability, thus we used a LR of 1e-5 for that particular run. Unless otherwise specified, each experiment is run for five epochs on the specified number of samples for each section of the paper. The best checkpoint is selected based on the performance in the validation set. To handle the large size of the upstream validation sets, validation is performed on a smaller subset of 2,000 samples.


\textbf{For finetuning,} we use the batch size specified in the JMP codebase and a default learning rate (LR) of 8e-5, except for cases where adjustments were needed to stabilize training. Specifically, we use 5e-5 for QMOF, 8e-4 for MatBench when pretrained on Transition1x. 
% 4e-5 for MD22 when using the large backbone with ANI-1x pretraining.

\clearpage


% \section{You \emph{can} have an appendix here.}

%%%%%%%%%%%%%%%%%%%%%%%%%%%%%%%%%%%%%%%%%%%%%%%%%%%%%%%%%%%%%%%%%%%%%%%%%%%%%%%
%%%%%%%%%%%%%%%%%%%%%%%%%%%%%%%%%%%%%%%%%%%%%%%%%%%%%%%%%%%%%%%%%%%%%%%%%%%%%%%


\end{document}


% This document was modified from the file originally made available by
% Pat Langley and Andrea Danyluk for ICML-2K. This version was created
% by Iain Murray in 2018, and modified by Alexandre Bouchard in
% 2019 and 2021 and by Csaba Szepesvari, Gang Niu and Sivan Sabato in 2022.
% Modified again in 2023 and 2024 by Sivan Sabato and Jonathan Scarlett.
% Previous contributors include Dan Roy, Lise Getoor and Tobias
% Scheffer, which was slightly modified from the 2010 version by
% Thorsten Joachims & Johannes Fuernkranz, slightly modified from the
% 2009 version by Kiri Wagstaff and Sam Roweis's 2008 version, which is
% slightly modified from Prasad Tadepalli's 2007 version which is a
% lightly changed version of the previous year's version by Andrew
% Moore, which was in turn edited from those of Kristian Kersting and
% Codrina Lauth. Alex Smola contributed to the algorithmic style files.
