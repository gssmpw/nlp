%%%%%%%% ICML 2025 EXAMPLE LATEX SUBMISSION FILE %%%%%%%%%%%%%%%%%

\documentclass{article}

% Recommended, but optional, packages for figures and better typesetting:
\usepackage{microtype}
\usepackage{graphicx}
\usepackage{subfigure}
\usepackage{booktabs} % for professional tables
\usepackage{multirow}

% hyperref makes hyperlinks in the resulting PDF.
% If your build breaks (sometimes temporarily if a hyperlink spans a page)
% please comment out the following usepackage line and replace
% \usepackage{icml2025} with \usepackage[nohyperref]{icml2025} above.
\usepackage{hyperref}

\newcommand{\yasir}[1]{\textcolor{red}{Yasir: #1}}

% \linespread{0.995}

\usepackage{enumitem}


% Attempt to make hyperref and algorithmic work together better:
\newcommand{\theHalgorithm}{\arabic{algorithm}}

% Use the following line for the initial blind version submitted for review:
% \usepackage{icml2025}


% If accepted, instead use the following line for the camera-ready submission:
\usepackage[accepted]{icml2025}

% For theorems and such
\usepackage{amsmath}
\usepackage{amssymb}
\usepackage{mathtools}
\usepackage{amsthm}

% if you use cleveref..
\usepackage[capitalize,noabbrev]{cleveref}

%%%%%%%%%%%%%%%%%%%%%%%%%%%%%%%%
% THEOREMS
%%%%%%%%%%%%%%%%%%%%%%%%%%%%%%%%
\theoremstyle{plain}
\newtheorem{theorem}{Theorem}[section]
\newtheorem{proposition}[theorem]{Proposition}
\newtheorem{lemma}[theorem]{Lemma}
\newtheorem{corollary}[theorem]{Corollary}
\theoremstyle{definition}
\newtheorem{definition}[theorem]{Definition}
\newtheorem{assumption}[theorem]{Assumption}
\theoremstyle{remark}
\newtheorem{remark}[theorem]{Remark}

% Todonotes is useful during development; simply uncomment the next line
%    and comment out the line below the next line to turn off comments
%\usepackage[disable,textsize=tiny]{todonotes}
\usepackage[textsize=tiny]{todonotes}

\newcommand{\eg}{e.g.,\ }
\newcommand{\ie}{i.e.,\ }

\usepackage[most]{tcolorbox}

% Define a custom tcolorbox environment
\tcbuselibrary{listingsutf8}
\newtcolorbox{fancybox}[1][]{%
    colback=gray!10, % Light grey background
    colframe=white, % White frame to make it borderless
    sharp corners,  % Remove rounded corners
    boxrule=0pt,    % No border
    enhanced,
    #1               % Allow additional options
}

% The \icmltitle you define below is probably too long as a header.
% Therefore, a short form for the running title is supplied here:
\icmltitlerunning{Towards Data-Efficient Pretraining for Atomic Property Prediction}
% \usepackage[table]{xcolor}  % Allow table colors


\definecolor{kleinblue}{RGB}{0, 47, 167}
\definecolor{question}{RGB}{25, 25, 112}
\definecolor{darkyellow}{RGB}{255, 191, 0}
\definecolor{Gray}{gray}{0.95}  % Define a dark gray color



\usepackage{amssymb} % For symbols like \textquestiondown
\usepackage{etoolbox} % For defining reusable commands
\usepackage{fontawesome5} % For decorative symbols

\newcommand{\highlighttext}[1]{%
    \textcolor{kleinblue}{\textcolor{darkyellow}{\faLightbulb}~\textbf{\textit{#1}}}%
}

\newcommand\blfootnote[1]{%
  \begingroup
  \renewcommand\thefootnote{}\footnote{#1}%
  \addtocounter{footnote}{-1}%
  \endgroup
}
\begin{document}

% Hide ICML footnote. TODO: uncomment this for the final version
\makeatletter
\renewcommand{\ICML@appearing}{}
\makeatother



\twocolumn[
\icmltitle{Towards Data-Efficient Pretraining for Atomic Property Prediction}

% It is OKAY to include author information, even for blind
% submissions: the style file will automatically remove it for you
% unless you've provided the [accepted] option to the icml2025
% package.

% List of affiliations: The first argument should be a (short)
% identifier you will use later to specify author affiliations
% Academic affiliations should list Department, University, City, Region, Country
% Industry affiliations should list Company, City, Region, Country

% You can specify symbols, otherwise they are numbered in order.
% Ideally, you should not use this facility. Affiliations will be numbered
% in order of appearance and this is the preferred way.
\icmlsetsymbol{equal}{*}

\begin{icmlauthorlist}
\icmlauthor{Yasir Ghunaim}{kaust}
\icmlauthor{Hasan Abed Al Kader Hammoud}{kaust}
\icmlauthor{Bernard Ghanem}{kaust}
% \icmlauthor{Firstname1 Lastname1}{equal,yyy}
% \icmlauthor{Firstname2 Lastname2}{equal,yyy,comp}
% \icmlauthor{Firstname3 Lastname3}{comp}
% \icmlauthor{Firstname4 Lastname4}{sch}
% \icmlauthor{Firstname5 Lastname5}{yyy}
% \icmlauthor{Firstname6 Lastname6}{sch,yyy,comp}
% \icmlauthor{Firstname7 Lastname7}{comp}
% \icmlauthor{Firstname8 Lastname8}{sch}
% \icmlauthor{Firstname8 Lastname8}{yyy,comp}
%\icmlauthor{}{sch}
%\icmlauthor{}{sch}
\end{icmlauthorlist}

% \icmlaffiliation{yyy}{Department of XXX, University of YYY, Location, Country}
% \icmlaffiliation{comp}{Company Name, Location, Country}
% \icmlaffiliation{sch}{School of ZZZ, Institute of WWW, Location, Country}
\icmlaffiliation{kaust}{King Abdullah University of Science and Technology (KAUST), Thuwal, Saudi Arabia}

\icmlcorrespondingauthor{Yasir Ghunaim}{yasir.ghunaim@kaust.edu.sa}
% \icmlcorrespondingauthor{Firstname1 Lastname1}{first1.last1@xxx.edu}
% \icmlcorrespondingauthor{Firstname2 Lastname2}{first2.last2@www.uk}

% You may provide any keywords that you
% find helpful for describing your paper; these are used to populate
% the "keywords" metadata in the PDF but will not be shown in the document
\icmlkeywords{Machine Learning, ICML}

\vskip 0.3in
]

% this must go after the closing bracket ] following \twocolumn[ ...

% This command actually creates the footnote in the first column
% listing the affiliations and the copyright notice.
% The command takes one argument, which is text to display at the start of the footnote.
% The \icmlEqualContribution command is standard text for equal contribution.
% Remove it (just {}) if you do not need this facility.

\printAffiliationsAndNotice{}  % leave blank if no need to mention equal contribution
% \printAffiliationsAndNotice{\icmlEqualContribution} % otherwise use the standard text.



%%%%%%%%% ABSTRACT
\begin{abstract}


The choice of representation for geographic location significantly impacts the accuracy of models for a broad range of geospatial tasks, including fine-grained species classification, population density estimation, and biome classification. Recent works like SatCLIP and GeoCLIP learn such representations by contrastively aligning geolocation with co-located images. While these methods work exceptionally well, in this paper, we posit that the current training strategies fail to fully capture the important visual features. We provide an information theoretic perspective on why the resulting embeddings from these methods discard crucial visual information that is important for many downstream tasks. To solve this problem, we propose a novel retrieval-augmented strategy called RANGE. We build our method on the intuition that the visual features of a location can be estimated by combining the visual features from multiple similar-looking locations. We evaluate our method across a wide variety of tasks. Our results show that RANGE outperforms the existing state-of-the-art models with significant margins in most tasks. We show gains of up to 13.1\% on classification tasks and 0.145 $R^2$ on regression tasks. All our code and models will be made available at: \href{https://github.com/mvrl/RANGE}{https://github.com/mvrl/RANGE}.

\end{abstract}



%%%%%%%%% BODY TEXT
\section{Introduction}

Video generation has garnered significant attention owing to its transformative potential across a wide range of applications, such media content creation~\citep{polyak2024movie}, advertising~\citep{zhang2024virbo,bacher2021advert}, video games~\citep{yang2024playable,valevski2024diffusion, oasis2024}, and world model simulators~\citep{ha2018world, videoworldsimulators2024, agarwal2025cosmos}. Benefiting from advanced generative algorithms~\citep{goodfellow2014generative, ho2020denoising, liu2023flow, lipman2023flow}, scalable model architectures~\citep{vaswani2017attention, peebles2023scalable}, vast amounts of internet-sourced data~\citep{chen2024panda, nan2024openvid, ju2024miradata}, and ongoing expansion of computing capabilities~\citep{nvidia2022h100, nvidia2023dgxgh200, nvidia2024h200nvl}, remarkable advancements have been achieved in the field of video generation~\citep{ho2022video, ho2022imagen, singer2023makeavideo, blattmann2023align, videoworldsimulators2024, kuaishou2024klingai, yang2024cogvideox, jin2024pyramidal, polyak2024movie, kong2024hunyuanvideo, ji2024prompt}.


In this work, we present \textbf{\ours}, a family of rectified flow~\citep{lipman2023flow, liu2023flow} transformer models designed for joint image and video generation, establishing a pathway toward industry-grade performance. This report centers on four key components: data curation, model architecture design, flow formulation, and training infrastructure optimization—each rigorously refined to meet the demands of high-quality, large-scale video generation.


\begin{figure}[ht]
    \centering
    \begin{subfigure}[b]{0.82\linewidth}
        \centering
        \includegraphics[width=\linewidth]{figures/t2i_1024.pdf}
        \caption{Text-to-Image Samples}\label{fig:main-demo-t2i}
    \end{subfigure}
    \vfill
    \begin{subfigure}[b]{0.82\linewidth}
        \centering
        \includegraphics[width=\linewidth]{figures/t2v_samples.pdf}
        \caption{Text-to-Video Samples}\label{fig:main-demo-t2v}
    \end{subfigure}
\caption{\textbf{Generated samples from \ours.} Key components are highlighted in \textcolor{red}{\textbf{RED}}.}\label{fig:main-demo}
\end{figure}


First, we present a comprehensive data processing pipeline designed to construct large-scale, high-quality image and video-text datasets. The pipeline integrates multiple advanced techniques, including video and image filtering based on aesthetic scores, OCR-driven content analysis, and subjective evaluations, to ensure exceptional visual and contextual quality. Furthermore, we employ multimodal large language models~(MLLMs)~\citep{yuan2025tarsier2} to generate dense and contextually aligned captions, which are subsequently refined using an additional large language model~(LLM)~\citep{yang2024qwen2} to enhance their accuracy, fluency, and descriptive richness. As a result, we have curated a robust training dataset comprising approximately 36M video-text pairs and 160M image-text pairs, which are proven sufficient for training industry-level generative models.

Secondly, we take a pioneering step by applying rectified flow formulation~\citep{lipman2023flow} for joint image and video generation, implemented through the \ours model family, which comprises Transformer architectures with 2B and 8B parameters. At its core, the \ours framework employs a 3D joint image-video variational autoencoder (VAE) to compress image and video inputs into a shared latent space, facilitating unified representation. This shared latent space is coupled with a full-attention~\citep{vaswani2017attention} mechanism, enabling seamless joint training of image and video. This architecture delivers high-quality, coherent outputs across both images and videos, establishing a unified framework for visual generation tasks.


Furthermore, to support the training of \ours at scale, we have developed a robust infrastructure tailored for large-scale model training. Our approach incorporates advanced parallelism strategies~\citep{jacobs2023deepspeed, pytorch_fsdp} to manage memory efficiently during long-context training. Additionally, we employ ByteCheckpoint~\citep{wan2024bytecheckpoint} for high-performance checkpointing and integrate fault-tolerant mechanisms from MegaScale~\citep{jiang2024megascale} to ensure stability and scalability across large GPU clusters. These optimizations enable \ours to handle the computational and data challenges of generative modeling with exceptional efficiency and reliability.


We evaluate \ours on both text-to-image and text-to-video benchmarks to highlight its competitive advantages. For text-to-image generation, \ours-T2I demonstrates strong performance across multiple benchmarks, including T2I-CompBench~\citep{huang2023t2i-compbench}, GenEval~\citep{ghosh2024geneval}, and DPG-Bench~\citep{hu2024ella_dbgbench}, excelling in both visual quality and text-image alignment. In text-to-video benchmarks, \ours-T2V achieves state-of-the-art performance on the UCF-101~\citep{ucf101} zero-shot generation task. Additionally, \ours-T2V attains an impressive score of \textbf{84.85} on VBench~\citep{huang2024vbench}, securing the top position on the leaderboard (as of 2025-01-25) and surpassing several leading commercial text-to-video models. Qualitative results, illustrated in \Cref{fig:main-demo}, further demonstrate the superior quality of the generated media samples. These findings underscore \ours's effectiveness in multi-modal generation and its potential as a high-performing solution for both research and commercial applications.
\section{Related Work}

\subsection{Large 3D Reconstruction Models}
Recently, generalized feed-forward models for 3D reconstruction from sparse input views have garnered considerable attention due to their applicability in heavily under-constrained scenarios. The Large Reconstruction Model (LRM)~\cite{hong2023lrm} uses a transformer-based encoder-decoder pipeline to infer a NeRF reconstruction from just a single image. Newer iterations have shifted the focus towards generating 3D Gaussian representations from four input images~\cite{tang2025lgm, xu2024grm, zhang2025gslrm, charatan2024pixelsplat, chen2025mvsplat, liu2025mvsgaussian}, showing remarkable novel view synthesis results. The paradigm of transformer-based sparse 3D reconstruction has also successfully been applied to lifting monocular videos to 4D~\cite{ren2024l4gm}. \\
Yet, none of the existing works in the domain have studied the use-case of inferring \textit{animatable} 3D representations from sparse input images, which is the focus of our work. To this end, we build on top of the Large Gaussian Reconstruction Model (GRM)~\cite{xu2024grm}.

\subsection{3D-aware Portrait Animation}
A different line of work focuses on animating portraits in a 3D-aware manner.
MegaPortraits~\cite{drobyshev2022megaportraits} builds a 3D Volume given a source and driving image, and renders the animated source actor via orthographic projection with subsequent 2D neural rendering.
3D morphable models (3DMMs)~\cite{blanz19993dmm} are extensively used to obtain more interpretable control over the portrait animation. For example, StyleRig~\cite{tewari2020stylerig} demonstrates how a 3DMM can be used to control the data generated from a pre-trained StyleGAN~\cite{karras2019stylegan} network. ROME~\cite{khakhulin2022rome} predicts vertex offsets and texture of a FLAME~\cite{li2017flame} mesh from the input image.
A TriPlane representation is inferred and animated via FLAME~\cite{li2017flame} in multiple methods like Portrait4D~\cite{deng2024portrait4d}, Portrait4D-v2~\cite{deng2024portrait4dv2}, and GPAvatar~\cite{chu2024gpavatar}.
Others, such as VOODOO 3D~\cite{tran2024voodoo3d} and VOODOO XP~\cite{tran2024voodooxp}, learn their own expression encoder to drive the source person in a more detailed manner. \\
All of the aforementioned methods require nothing more than a single image of a person to animate it. This allows them to train on large monocular video datasets to infer a very generic motion prior that even translates to paintings or cartoon characters. However, due to their task formulation, these methods mostly focus on image synthesis from a frontal camera, often trading 3D consistency for better image quality by using 2D screen-space neural renderers. In contrast, our work aims to produce a truthful and complete 3D avatar representation from the input images that can be viewed from any angle.  

\subsection{Photo-realistic 3D Face Models}
The increasing availability of large-scale multi-view face datasets~\cite{kirschstein2023nersemble, ava256, pan2024renderme360, yang2020facescape} has enabled building photo-realistic 3D face models that learn a detailed prior over both geometry and appearance of human faces. HeadNeRF~\cite{hong2022headnerf} conditions a Neural Radiance Field (NeRF)~\cite{mildenhall2021nerf} on identity, expression, albedo, and illumination codes. VRMM~\cite{yang2024vrmm} builds a high-quality and relightable 3D face model using volumetric primitives~\cite{lombardi2021mvp}. One2Avatar~\cite{yu2024one2avatar} extends a 3DMM by anchoring a radiance field to its surface. More recently, GPHM~\cite{xu2025gphm} and HeadGAP~\cite{zheng2024headgap} have adopted 3D Gaussians to build a photo-realistic 3D face model. \\
Photo-realistic 3D face models learn a powerful prior over human facial appearance and geometry, which can be fitted to a single or multiple images of a person, effectively inferring a 3D head avatar. However, the fitting procedure itself is non-trivial and often requires expensive test-time optimization, impeding casual use-cases on consumer-grade devices. While this limitation may be circumvented by learning a generalized encoder that maps images into the 3D face model's latent space, another fundamental limitation remains. Even with more multi-view face datasets being published, the number of available training subjects rarely exceeds the thousands, making it hard to truly learn the full distibution of human facial appearance. Instead, our approach avoids generalizing over the identity axis by conditioning on some images of a person, and only generalizes over the expression axis for which plenty of data is available. 

A similar motivation has inspired recent work on codec avatars where a generalized network infers an animatable 3D representation given a registered mesh of a person~\cite{cao2022authentic, li2024uravatar}.
The resulting avatars exhibit excellent quality at the cost of several minutes of video capture per subject and expensive test-time optimization.
For example, URAvatar~\cite{li2024uravatar} finetunes their network on the given video recording for 3 hours on 8 A100 GPUs, making inference on consumer-grade devices impossible. In contrast, our approach directly regresses the final 3D head avatar from just four input images without the need for expensive test-time fine-tuning.


\section{Problem Formulation}
\label{sec:formulation}

% Formulation: 
% 1. Environment + Human Instruction -> Robot Action
% 2. Leader-Follower setting: Human as leader, robot as follower
% 3. Robot's Ability: Interactive Motion, Manipulation, Idle
% 4. Real-Time Design: Low-latency instruction: Interuption


In this work, we consider the interaction as a leader-follower formulation~\cite{vianelloHumanHumanoidInteractionCooperation2021}, where the human is the leader and the humanoid robot is the follower.
Define $\mathcal{I}$ as the set of human intentions and $\mathcal{K}$ as the set of robot skills.
At time step $t$, the leader shows an intention $I_t \in \mathcal{I}$ for the follower to perform a skill $K_t \in \mathcal{K}$, such as picking up a can, brushing a plate, or stamping a file.
We assume one intention corresponds to at most one skill, and the robot should be able to switch between skills in real time. The map function is defined as $f: \mathcal{I} \rightarrow \mathcal{K}$.

% Type and property of task:
    % - interactive Motion, manipulation, idle
    % - real-time design: low-latency instruction, in task reflection to human.
The skills of the robot can be categorized into three types: interactive motion, manipulation, and idle.
The interactive motion skills require the robot to perform expressive and diverse behavior, and the manipulation skills require the robot to interact with objects in the environment precisely. 
When the human leader does not show any intention, the robot will be in an idle state and do nothing.
The real-time interaction design requires the robot to respond to the leader's intentions with low latency and in-skill reflection to the human leader.
Low latency requires the robot to predict the leader's intention in real time and interrupt the current skill when the leader shows a new intention.
In-skill reflection requires the robot to react to human motion and environment even if the human intention is not changed, such as pausing current movement if the robot collides with the human or the target object is not reachable.

% The design of the idle state ensures the robustness of the interaction process. 
% The robot should avoid being affected by the uninstructed actions of a human, for example, a human might pick up a Coke can on a table. This behavior is similar to pointing at a Coke can for the robot to pick up, but the robot should be able to recognize it as idle.

% To ensure the safety of the interaction process, the interaction framework also includes a safety supervisor module, which monitors the robot's behavior. 
% When potential harm to the human leader is detected, the safety supervisor will stop the robot's action immediately and only allow the robot to continue the interaction when safety is guaranteed.

We formulate the observation space $\mathcal{O}$ of a humanoid robot as the combination of the
environment state $\mathcal{E}$ and the human behavior $\mathcal{H}$, i.e., $\mathcal{O} = \mathcal{E}  \bigodot  \mathcal{H} $.
The environment state $\mathcal{E}$ includes the robot's proprioception and the object state. The human behavior $\mathcal{H}$ consists of the leader's intention and the leader's behavior. 
To reduce the complexity of observation, our framework decomposes interaction policy into several sub-modules and separately designs the observation space for each module.
Compared to end-to-end models, this decomposed design allows the robot to learn humanoid-human-object interaction from human-object-human demonstration data, which is more sample-efficient and scalable.
% Main challenge: the diversity of human apperance, including clothing and body shape.


% Real-time Design
% To enable real-time interaction, \our is designed to respond to the leader's instruction with a low-latency reactive planner and interruptable low-level skills.
% The reactive planner predicts the leader's intention with 30Hz, and once the planner finds the current intention conflicts with the executing task, the planner will try to interrupt the current task and switch to the new task immediately.





\section{Dataset}
\label{sec:dataset}

\subsection{Data Collection}

To analyze political discussions on Discord, we followed the methodology in \cite{singh2024Cross-Platform}, collecting messages from politically-oriented public servers in compliance with Discord's platform policies.

Using Discord's Discovery feature, we employed a web scraper to extract server invitation links, names, and descriptions, focusing on public servers accessible without participation. Invitation links were used to access data via the Discord API. To ensure relevance, we filtered servers using keywords related to the 2024 U.S. elections (e.g., Trump, Kamala, MAGA), as outlined in \cite{balasubramanian2024publicdatasettrackingsocial}. This resulted in 302 server links, further narrowed to 81 English-speaking, politics-focused servers based on their names and descriptions.

Public messages were retrieved from these servers using the Discord API, collecting metadata such as \textit{content}, \textit{user ID}, \textit{username}, \textit{timestamp}, \textit{bot flag}, \textit{mentions}, and \textit{interactions}. Through this process, we gathered \textbf{33,373,229 messages} from \textbf{82,109 users} across \textbf{81 servers}, including \textbf{1,912,750 messages} from \textbf{633 bots}. Data collection occurred between November 13th and 15th, covering messages sent from January 1st to November 12th, just after the 2024 U.S. election.

\subsection{Characterizing the Political Spectrum}
\label{sec:timeline}

A key aspect of our research is distinguishing between Republican- and Democratic-aligned Discord servers. To categorize their political alignment, we relied on server names and self-descriptions, which often include rules, community guidelines, and references to key ideologies or figures. Each server's name and description were manually reviewed based on predefined, objective criteria, focusing on explicit political themes or mentions of prominent figures. This process allowed us to classify servers into three categories, ensuring a systematic and unbiased alignment determination.

\begin{itemize}
    \item \textbf{Republican-aligned}: Servers referencing Republican and right-wing and ideologies, movements, or figures (e.g., MAGA, Conservative, Traditional, Trump).  
    \item \textbf{Democratic-aligned}: Servers mentioning Democratic and left-wing ideologies, movements, or figures (e.g., Progressive, Liberal, Socialist, Biden, Kamala).  
    \item \textbf{Unaligned}: Servers with no defined spectrum and ideologies or opened to general political debate from all orientations.
\end{itemize}

To ensure the reliability and consistency of our classification, three independent reviewers assessed the classification following the specified set of criteria. The inter-rater agreement of their classifications was evaluated using Fleiss' Kappa \cite{fleiss1971measuring}, with a resulting Kappa value of \( 0.8191 \), indicating an almost perfect agreement among the reviewers. Disagreements were resolved by adopting the majority classification, as there were no instances where a server received different classifications from all three reviewers. This process guaranteed the consistency and accuracy of the final categorization.

Through this process, we identified \textbf{7 Republican-aligned servers}, \textbf{9 Democratic-aligned servers}, and \textbf{65 unaligned servers}.

Table \ref{tab:statistics} shows the statistics of the collected data. Notably, while Democratic- and Republican-aligned servers had a comparable number of user messages, users in the latter servers were significantly more active, posting more than double the number of messages per user compared to their Democratic counterparts. 
This suggests that, in our sample, Democratic-aligned servers attract more users, but these users were less engaged in text-based discussions. Additionally, around 10\% of the messages across all server categories were posted by bots. 

\subsection{Temporal Data} 

Throughout this paper, we refer to the election candidates using the names adopted by their respective campaigns: \textit{Kamala}, \textit{Biden}, and \textit{Trump}. To examine how the content of text messages evolves based on the political alignment of servers, we divided the 2024 election year into three periods: \textbf{Biden vs Trump} (January 1 to July 21), \textbf{Kamala vs Trump} (July 21 to September 20), and the \textbf{Voting Period} (after September 20). These periods reflect key phases of the election: the early campaign dominated by Biden and Trump, the shift in dynamics with Kamala Harris replacing Joe Biden as the Democratic candidate, and the final voting stage focused on electoral outcomes and their implications. This segmentation enables an analysis of how discourse responds to pivotal electoral moments.

Figure \ref{fig:line-plot} illustrates the distribution of messages over time, highlighting trends in total messages volume and mentions of each candidate. Prior to Biden's withdrawal on July 21, mentions of Biden and Trump were relatively balanced. However, following Kamala's entry into the race, mentions of Trump surged significantly, a trend further amplified by an assassination attempt on him, solidifying his dominance in the discourse. The only instance where Trump’s mentions were exceeded occurred during the first debate, as concerns about Biden’s age and cognitive abilities temporarily shifted the focus. In the final stages of the election, mentions of all three candidates rose, with Trump’s mentions peaking as he emerged as the victor.
\section{Experimental Methodology}\label{sec:exp}
In this section, we introduce the datasets, evaluation metrics, baselines, and implementation details used in our experiments. More experimental details are shown in Appendix~\ref{app:experiment_detail}.

\textbf{Dataset.}
We utilize various datasets for training and evaluation. Data statistics are shown in Table~\ref{tab:dataset}.

\textit{Training.}
We use the publicly available E5 dataset~\cite{wang2024improving,springer2024repetition} to train both the LLM-QE and dense retrievers. We concentrate on English-based question answering tasks and collect a total of 808,740 queries. From this set, we randomly sample 100,000 queries to construct the DPO training data, while the remaining queries are used for contrastive training. During the DPO preference pair construction, we first prompt LLMs to generate expansion documents, filtering out queries where the expanded documents share low similarity with the query. This results in a final set of 30,000 queries.

\textit{Evaluation.}
We evaluate retrieval effectiveness using two retrieval benchmarks: MS MARCO \cite{bajaj2016ms} and BEIR \cite{thakur2021beir}, in both unsupervised and supervised settings.

\textbf{Evaluation Metrics.}
We use nDCG@10 as the evaluation metric. Statistical significance is tested using a permutation test with $p<0.05$.

\textbf{Baselines.} We compare our LLM-QE model with three unsupervised retrieval models and five query expansion baseline models.
% —

Three unsupervised retrieval models—BM25~\cite{robertson2009probabilistic}, CoCondenser~\cite{gao2022unsupervised}, and Contriever~\cite{izacard2021unsupervised}—are evaluated in the experiments. Among these, Contriever serves as our primary baseline retrieval model, as it is used as the backbone model to assess the query expansion performance of LLM-QE. Additionally, we compare LLM-QE with Contriever in a supervised setting using the same training dataset.

For query expansion, we benchmark against five methods: Pseudo-Relevance Feedback (PRF), Q2Q, Q2E, Q2C, and Q2D. PRF is specifically implemented following the approach in~\citet{yu2021improving}, which enhances query understanding by extracting keywords from query-related documents. The Q2Q, Q2E, Q2C, and Q2D methods~\cite{jagerman2023query,li2024can} expand the original query by prompting LLMs to generate query-related queries, keywords, chains-of-thought~\cite{wei2022chain}, and documents.


\textbf{Implementation Details.} 
For our query expansion model, we deploy the Meta-LLaMA-3-8B-Instruct~\cite{llama3modelcard} as the backbone for the query expansion generator. The batch size is set to 16, and the learning rate is set to $2e-5$. Optimization is performed using the AdamW optimizer. We employ LoRA~\cite{hu2022lora} to efficiently fine-tune the model for 2 epochs. The temperature for the construction of the DPO data varies across $\tau \in \{0.8, 0.9, 1.0, 1.1\}$, with each setting sampled eight times. For the dense retriever, we utilize Contriever~\cite{izacard2021unsupervised} as the backbone. During training, we set the batch size to 1,024 and the learning rate to $3e-5$, with the model trained for 3 epochs.

\section{Beyond In-Distribution}
\label{5_OOD}

Recall that our pretraining process is conducted on upstream tasks involving molecules and catalysts, with energy and force as targets. For downstream tasks with different labels (e.g., band gap in QMOF) or from distinct chemical domains such as materials (e.g., MatBench and QMOF), we classify these as out-of-distribution (OOD). While our main results focused on ID evaluation, here we explore our metric's applicability to OOD tasks. Specifically, we examine three cases: the Band Gap property from QMOF~\cite{rosen2021machine}, Phonons (the first non-energy target in JMP tables) from MatBench~\cite{dunn2020benchmarking}, and $\Delta_\epsilon$ from QM9, explicitly categorized as OOD in the JMP paper.

In Figure \ref{fig:CSI_OOD}, we present the CSI values for OOD domains, where the OOD label ($\Delta_\epsilon$) for QM9 follows the same values as in Figure \ref{fig:main_CSI_score}. We observe that QMOF exhibits a pattern similar to other ID domains shown in Figure \ref{fig:main_CSI_score}. However, MatBench displays a distinct pattern, showing strong correlation with OC20 and OC22, followed by ANI-1x and Transition-1x. Next, we analyze the correlation between CSI and downstream performance under OOD evaluation.

Table \ref{tab:ood} shows that $\Delta_\epsilon$ in QM9 aligns with the CSI pattern, similar to ID evaluation, suggesting that CSI is effective for OOD in the label space. In QMOF, the different upstream sources achieve similar performance which lags behind the full pretraining by JMP. For MatBench (evaluated over 5 folds), OC22 achieves the best mean performance while OC20 lags behind, despite our metric predicting both to be equally suitable. Additionally, for both QMOF and MatBench, mixed pretraining variants generalize better than individual sources. This suggests that when the downstream domain differs from all upstream sources, mixing diverse upstream domains provides the best performance.

While our CSI metric holds well for ID evaluation, further research is needed to improve OOD generalization across different chemical domains. One direction is to explore additional upstream sources to better capture domain variations. Another potential extension is leveraging foundation models trained beyond energy and forces, which could enhance feature extraction and improve similarity assessments.


% - Reiterate what is OOD.
% - Mention that we study X,Y,Z datasets as OOD and why it's considered OOD.
% - Mention that we can still select the best dataset among the upstream.
% - Why we need better metrics or possibly foundation models for the CSI metric for this problem ?
% - While it works for qmof it doesnt for matbench.
% - mention mixed data as a good solution here.
% - Hopefully fid based sampling works.


% \textcolor{blue}{Limitations we discuss OOD}

% \textit{OOD experiments}
% \textit{we say metric is naive but it showed what we wanted to show that superior subsets exist.}
% \textit{Here we only use node features however ufture better metrics might invovle edge features .. }


\section{Qualitative Analysis}

\begin{figure*}[t!]
    \centering
    \includegraphics[width=\linewidth]{figs/graph-p-new.pdf}
    %\vspace{-7pt}
    \caption{\centering Taxonomy of bugs in DL generated code. Only categories with DL-related subcategories are shown.
    %In this Figure, We only show categories that have at least one subcategory which is related to Deep learning.
    %\hung{should be Not DL-related not None DL, pls fix}\alireza{done}
    }
    \label{fig:taxonomy}
    %\vspace{-10pt}
\end{figure*}
\begin{figure*}[t!]
    \centering
    \includegraphics[width=\linewidth]{figs/Tax-Order.png}
    %\vspace{-7pt}
    \caption{Distribution of bugs in general code vs. DL code generated by LLM
    }

    
    \label{fig:tax_distribution}
    %\vspace{-10pt}
\end{figure*}

\textbf{Taxonomy of Bugs in Generated DL Code:} GPT-4o achieves a pass@1 rate of only 31\% hence in most cases, it failed to generate the correct code. In this section, we build a taxonomy of common bug patterns and issues that arise in DL code generated by GPT-4o. This taxonomy is an expansion of Tambon et.al \cite{tambon2024bugs}'s bug taxonomy for LLM-generated regular code.

Following the same procedure as our labeling process, three authors manually investigate all GPT-4o failures and categorize them following Tambon et.al's taxonomy. At the same time, the annotators identify the DL-specific sub-categories for each failure. The result is the taxonomy presented in Fig~\ref{fig:taxonomy}. The appendix gives a detailed explanation of each bug type and sub-category.

%Specifically, we compare these patterns to those found in general Python code produced by LLMs and deep learning code generated by humans. To facilitate this analysis, we use Tambon et.al\cite{tambon2024bugs} work as a baseline for understanding bug patterns in general Python code generated by LLMs work as a baseline for understanding bug patterns in general Python code generated by LLMs, and  Islam et. al\cite{islam2019comprehensive} work for comparing bugs in deep learning code generated by LLMs versus that produced by humans.

%The section is structured as follows: First, we introduce and explain bug taxonomies. Next, we discuss the differences between bug patterns in general Python code versus deep learning-specific code. Finally, we compare the bug patterns found in human-generated deep learning code versus LLM-generated code.

\textbf{Failures in Generating the DL Code and General Code:} Tambon et. al\cite{tambon2024bugs} analyzed failures when CodeGen models generate code for the general tasks. Figures~\ref{fig:tax_distribution} show the distributions of the bug types when generating general code vs DL code. 
On the one hand, \textit{misinterpretation} (purple) is a common bug when generating both general code and DL code, however, due to more complex logic and arithmetic requirements, LLMs more often make this mistake when generating DL code. An example of this type of bug can be seen in Figure~\ref{fig:discussion2}. On the other hand, since GPT4o is much more capable compared to CodeGen models used by prior work, errors such as \textit{incomplete generation} (green), \textit{silly mistake} (dark gray), and \textit{syntax error} (yellow) occur at a much lower rate.

Furthermore, we have introduced several new categories of bugs that commonly arise in DL code generation. Firstly, \textit{errors in arithmetic and logical operations}(light blue) occur when incorrect calculations or flawed logical code are generated.
%, leading to unintended behaviours such as inaccurate model predictions or faulty algorithm implementations. 
Secondly, \textit{performance}(light brown) issues involve inefficient generated code with slow execution times, excessive memory consumption, or suboptimal utilization of resources.
%, which can hinder the scalability and practicality of deep learning applications. 
Lastly, \textit{prompt missing information}(light purple) when the prompts are missing details to fully address the problem at hand, resulting in incomplete or partially implemented solutions.
%such as missing functions, incomplete data preprocessing steps, or unfinished model architectures. 
These new categories identify important challenges that are unique to DL code generation.
%, facilitating more effective debugging, enhancing the quality of generated code, and ultimately contributing to the development of more robust and efficient deep learning models.

%Now, we will discuss the differences between bugs in general Python code generated by LLMs and those specific to deep learning code. Previous work, such as the analysis conducted by Tambon et. al\cite{tambon2024bugs}, provides an extensive evaluation of bugs in general Python code, with results summarized in Table~\ref{tab:general_bug}. While their findings reveal that many issues stemmed from deviations from the given prompt, they did not categorize bugs related to arithmetic or mathematical errors. This distinction is crucial because deep learning code often involves more algebraic and mathematical operations, increasing the likelihood of such errors. An example of this type of bug can be seen in Figure~\ref{fig:discussion2}.

%Additionally, their analysis focused on the CodeGen model, which is known to produce a higher percentage of syntax errors or incomplete code generations. In contrast, our analysis is based on GPT-4o, a more advanced and powerful model. The improved capabilities of GPT-4o reduce the likelihood of syntax errors and incomplete outputs, thereby providing a more robust baseline for analyzing deep learning-specific bugs. This shift highlights the importance of evaluating LLM-generated code within the specific context of the application, as deep learning code introduces unique challenges that are less prevalent in general Python code.

%On the other hand, we have introduced several new categories of bugs that commonly arise in code generation for deep learning systems. Firstly, \textbf{arithmetic and logical errors} occur when the generated code contains incorrect calculations or flawed logical operations, leading to unintended behaviours such as inaccurate model predictions or faulty algorithm implementations. Secondly, \textbf{performance issues} involve inefficiencies in the generated code that result in slow execution times, excessive memory consumption, or suboptimal utilization of computational resources, which can hinder the scalability and practicality of deep learning applications. Lastly, \textbf{incomplete problem resolution} refers to situations where the code generation prompt fails to fully address the problem at hand, resulting in incomplete or partially implemented solutions such as missing functions, incomplete data preprocessing steps, or unfinished model architectures. By categorizing these bugs, we aim to systematically identify and address the common challenges in automated code generation for deep learning, facilitating more effective debugging, enhancing the quality of generated code, and ultimately contributing to the development of more robust and efficient deep learning models.


\begin{tcolorbox}[boxrule=0.5pt, colback=gray!10,  arc=4pt,left=3pt,right=3pt,top=3pt,bottom=3pt,boxsep=0pt
]
\textbf{Observation 1:} \textit{Misinterpretation} is a common issue in both generated general code and DL code, however, due to more complex logic and arithmetic requirements, LLMs are more likely to make this mistake when generating DL code. \textit{Errors in arithmetic and logical operations}, \textit{performance}, and \textit{prompt missing information} emerged as new issues that are specific to DL code generation.
\end{tcolorbox}


\begin{figure}
    \centering
    \includegraphics[width=1\linewidth]{figs/Fig19_1.pdf}
    \vspace{-7pt}
    \caption{Mismatching data shapes: shifting variables need to be broadcasted to the image shape
    }
    \label{fig:discussion3}
    \vspace{-10pt}
\end{figure}
\newpage
\textbf{Bugs in Human-Written vs. LLM-Generated DL Code:} On one hand, prior study~\cite{islam2019comprehensive} identified the most common types of bugs in human-written DL code which include logic errors, API misuse, and data-related issues. Among these, API misuse is the most prevalent bug pattern in human-written DL code when using TensorFlow, whereas data flow bugs are more common when using PyTorch. On the other hand, according to our analysis of LLM-generated DL code, although API misuse remains a frequent issue, data structural problems, such as tensor mismatches and dimensional errors, occupy more frequently. Figure~\ref{fig:discussion3} highlights an instance of dimensional mismatches in LLM-generated DL code. In this case, GPT-4o incorrectly assumes that each shift value can be applied directly to all pixels in the image channel, causing a shape mismatch.

%larger proportion of bugs. As seen in Table~\ref{tab:taxonomy}, these structural issues surpass other bug types in LLM-generated code. This trend is likely due to LLMs' inherent weaknesses in understanding shapes and numerical relationships, a limitation that occurs less frequently in human-generated code. 

\begin{figure}
    \centering
    \includegraphics[width=1\linewidth]{figs/fig18.pdf}
    \vspace{-7pt}
    \caption{Incorrect processing of parameters: The axes scales need to be applied to both sin and cos
    }
    \label{fig:discussion2}
    \vspace{-10pt}
\end{figure}
%While there are meaningful differences between humans' and LLMs' code, notable similarities in bug patterns also exist. 
Similar to prior findings~\cite{islam2019comprehensive} of human-written DL code, LLM-generated DL code often contains  
%shows that a considerable proportion of bugs in human-written DL code are 
logic errors.
This similarity may stem from the fact that LLMs are trained on human-written code, thereby inheriting logical structures and concepts from human programmers. An example of such logic-related bugs is shown in Figure~\ref{fig:discussion2}, demonstrating how LLMs replicate logical reasoning errors that occur in human-written code. Here, GPT-4o applies \textit{scale\_x} only to the cosine whereas the scaling factors \textit{scale\_x} and \textit{scale\_y} should be applied uniformly to both the sine and cosine components of the rotation matrix. This results in improper scaling along the axes and triggers a test failure.

\begin{figure}
    \centering
    \includegraphics[width=1\linewidth]{figs/fig17_new.pdf}
    %\vspace{-7pt}
    \caption{Wrong usage of a third-party library. 
    }
    
    
    \label{fig:discussion1}
    \vspace{-10pt}
\end{figure}
Additionally, API misuse is a common bug pattern occurred in both human-written and LLM-generated DL code.
%sources of code.
Figure~\ref{fig:discussion1} provides an example of API misuse in LLM-generated code where GPT-4o attempts to call \textit{torch.idct}, which is not implemented in PyTorch. One possible fix is to provide more context concerning third-party libraries. For example, one could hint to LLMs to use \textit{scipy} instead, resulting in \textit{scipy.fftpack.idct(x.numpy(), norm=norm)} instead.

\begin{tcolorbox}[boxrule=0.5pt, colback=gray!10,  arc=4pt,left=3pt,right=3pt,top=3pt,bottom=3pt,boxsep=0pt
]
\textbf{Observation 2:} Unlike human-written DL code, LLM-generated DL code contains more data structural problems, such as tensor mismatches and dimensional errors. 
Similar to prior findings of human-written DL code, LLM-generated DL code often contains logic errors.
Additionally, API misuse frequently occurs as a bug pattern in both human-written and LLM-generated DL code.
These overlaps suggest that while LLMs exhibit unique weaknesses, their reliance on human-generated training data also leads to shared bug patterns, particularly in logic and API misuse errors.

\end{tcolorbox}

%\subsection{Incorrect Usage of Third-Party Libraries}

%Third-party libraries (e.g., PyTorch, TensorFlow, and NumPy) are essential to writing ML/DL code as oftentimes it is not effective and efficient to write everything from scratch, especially with so many available popular ML/DL APIs. However, as illustrated in Fig~\ref{fig:discussion1}, LLMs sometimes fail to use these libraries correctly. In this case, the model attempts to call \textit{torch.idct}, which is not implemented in PyTorch. One possible fix is to provide more context concerning third-party libraries. For example, one could provide a hint to LLMs to use \textit{scipy} instead which would result in the use of \textit{scipy.fftpack.idct(x.numpy(), norm=norm)} instead.

%This benchmark is designed for ML and DL-related code, where the ground truth often relies on third-party libraries like PyTorch, TensorFlow, and NumPy. LLMs also utilize these libraries in their generated code. However, as illustrated in Fig~\ref{fig:discussion1}, LLMs sometimes fail to use these libraries correctly. For example, the model attempts to call \textbf{torch.idct}, which is not implemented in PyTorch. A simple fix would be to use \textbf{scipy.fftpack.idct(x.numpy(), norm=norm)} instead. Retrieval-augmented generation can improve LLM understanding of available APIs and help prevent such issues.


%\subsection{Incorrect Operations on Input Parameters}

%Not only do the API calls need to be correct, but the input parameters also need to be appropriately processed. For example, LLMs sometimes perform incorrect operations on input parameters. For example, in Fig~\ref{fig:discussion2}, GPT-4o applies \textbf{scale\_x} only to the cosine whereas the scaling factors \textbf{scale\_x} and \textbf{scale\_y} should be applied uniformly to both the sine and cosine components of the rotation matrix. This results in improper scaling along the axes and triggers a test failure.

%Even when LLMs use the correct APIs, they may perform the wrong operations on input parameters. Fig~\ref{fig:discussion2} demonstrates this type of error. In this example, the scaling factors \textbf{scale\_x} and \textbf{scale\_y} should be applied uniformly to both the sine and cosine components of the rotation matrix. However, the LLM incorrectly applies \textbf{scale\_x} only to the cosine and sine terms in the first column (x-axis) and \textbf{scale\_y} only to the sine and cosine terms in the second column (y-axis). This results in improper scaling along the axes.


%\subsection{Incorrect Assumptions About Input and Output Shapes}
%Input/output shapes are also hard for LLMs to handle. Specifically, as shown in Fig~\ref{fig:discussion3}, when shifting values \textit{(r\_shift, g\_shift, b\_shift}) are one-dimensional tensors with shape (N) they need to be broadcastable to the image channels with shape (N, H, W) in order for the \textit{+=} operation to work. The GPT-4o incorrectly assumes that each shift value can be applied directly to all pixels in the image channel, causing a shape mismatch. Additional few-steps learning specifically focused on shape correction would help in cases where the shift values are reshaped to (N, 1, 1) using \textit{.view(N, 1, 1).}

%\begin{figure}
    \centering
    \includegraphics[width=1\linewidth]{figs/example_6.pdf}
    \vspace{-7pt}
    \caption{An example for wrong input for torch.cdist.}
    \label{fig:example6}
    \vspace{-10pt}
\end{figure}
%\begin{figure}
    \centering
    \includegraphics[width=1\linewidth]{figs/example-9.pdf}
    \vspace{-7pt}
    \caption{An example for adding non-requested feature. Exapanding the matrix is not requested that leads to shape mismatch.}
    \label{fig:example9}
    \vspace{-10pt}
\end{figure}
\section{Discussion}\label{sec:discussion}



\subsection{From Interactive Prompting to Interactive Multi-modal Prompting}
The rapid advancements of large pre-trained generative models including large language models and text-to-image generation models, have inspired many HCI researchers to develop interactive tools to support users in crafting appropriate prompts.
% Studies on this topic in last two years' HCI conferences are predominantly focused on helping users refine single-modality textual prompts.
Many previous studies are focused on helping users refine single-modality textual prompts.
However, for many real-world applications concerning data beyond text modality, such as multi-modal AI and embodied intelligence, information from other modalities is essential in constructing sophisticated multi-modal prompts that fully convey users' instruction.
This demand inspires some researchers to develop multimodal prompting interactions to facilitate generation tasks ranging from visual modality image generation~\cite{wang2024promptcharm, promptpaint} to textual modality story generation~\cite{chung2022tale}.
% Some previous studies contributed relevant findings on this topic. 
Specifically, for the image generation task, recent studies have contributed some relevant findings on multi-modal prompting.
For example, PromptCharm~\cite{wang2024promptcharm} discovers the importance of multimodal feedback in refining initial text-based prompting in diffusion models.
However, the multi-modal interactions in PromptCharm are mainly focused on the feedback empowered the inpainting function, instead of supporting initial multimodal sketch-prompt control. 

\begin{figure*}[t]
    \centering
    \includegraphics[width=0.9\textwidth]{src/img/novice_expert.pdf}
    \vspace{-2mm}
    \caption{The comparison between novice and expert participants in painting reveals that experts produce more accurate and fine-grained sketches, resulting in closer alignment with reference images in close-ended tasks. Conversely, in open-ended tasks, expert fine-grained strokes fail to generate precise results due to \tool's lack of control at the thin stroke level.}
    \Description{The comparison between novice and expert participants in painting reveals that experts produce more accurate and fine-grained sketches, resulting in closer alignment with reference images in close-ended tasks. Novice users create rougher sketches with less accuracy in shape. Conversely, in open-ended tasks, expert fine-grained strokes fail to generate precise results due to \tool's lack of control at the thin stroke level, while novice users' broader strokes yield results more aligned with their sketches.}
    \label{fig:novice_expert}
    % \vspace{-3mm}
\end{figure*}


% In particular, in the initial control input, users are unable to explicitly specify multi-modal generation intents.
In another example, PromptPaint~\cite{promptpaint} stresses the importance of paint-medium-like interactions and introduces Prompt stencil functions that allow users to perform fine-grained controls with localized image generation. 
However, insufficient spatial control (\eg, PromptPaint only allows for single-object prompt stencil at a time) and unstable models can still leave some users feeling the uncertainty of AI and a varying degree of ownership of the generated artwork~\cite{promptpaint}.
% As a result, the gap between intuitive multi-modal or paint-medium-like control and the current prompting interface still exists, which requires further research on multi-modal prompting interactions.
From this perspective, our work seeks to further enhance multi-object spatial-semantic prompting control by users' natural sketching.
However, there are still some challenges to be resolved, such as consistent multi-object generation in multiple rounds to increase stability and improved understanding of user sketches.   


% \new{
% From this perspective, our work is a step forward in this direction by allowing multi-object spatial-semantic prompting control by users' natural sketching, which considers the interplay between multiple sketch regions.
% % To further advance the multi-modal prompting experience, there are some aspects we identify to be important.
% % One of the important aspects is enhancing the consistency and stability of multiple rounds of generation to reduce the uncertainty and loss of control on users' part.
% % For this purpose, we need to develop techniques to incorporate consistent generation~\cite{tewel2024training} into multi-modal prompting framework.}
% % Another important aspect is improving generative models' understanding of the implicit user intents \new{implied by the paint-medium-like or sketch-based input (\eg, sketch of two people with their hands slightly overlapping indicates holding hand without needing explicit prompt).
% % This can facilitate more natural control and alleviate users' effort in tuning the textual prompt.
% % In addition, it can increase users' sense of ownership as the generated results can be more aligned with their sketching intents.
% }
% For example, when users draw sketches of two people with their hands slightly overlapping, current region-based models cannot automatically infer users' implicit intention that the two people are holding hands.
% Instead, they still require users to explicitly specify in the prompt such relationship.
% \tool addresses this through sketch-aware prompt recommendation to fill in the necessary semantic information, alleviating users' workload.
% However, some users want the generative AI in the future to be able to directly infer this natural implicit intentions from the sketches without additional prompting since prompt recommendation can still be unstable sometimes.


% \new{
% Besides visual generation, 
% }
% For example, one of the important aspect is referring~\cite{he2024multi}, linking specific text semantics with specific spatial object, which is partly what we do in our sketch-aware prompt recommendation.
% Analogously, in natural communication between humans, text or audio alone often cannot suffice in expressing the speakers' intentions, and speakers often need to refer to an existing spatial object or draw out an illustration of her ideas for better explanation.
% Philosophically, we HCI researchers are mostly concerned about the human-end experience in human-AI communications.
% However, studies on prompting is unique in that we should not just care about the human-end interaction, but also make sure that AI can really get what the human means and produce intention-aligned output.
% Such consideration can drastically impact the design of prompting interactions in human-AI collaboration applications.
% On this note, although studies on multi-modal interactions is a well-established topic in HCI community, it remains a challenging problem what kind of multi-modal information is really effective in helping humans convey their ideas to current and next generation large AI models.




\subsection{Novice Performance vs. Expert Performance}\label{sec:nVe}
In this section we discuss the performance difference between novice and expert regarding experience in painting and prompting.
First, regarding painting skills, some participants with experience (4/12) preferred to draw accurate and fine-grained shapes at the beginning. 
All novice users (5/12) draw rough and less accurate shapes, while some participants with basic painting skills (3/12) also favored sketching rough areas of objects, as exemplified in Figure~\ref{fig:novice_expert}.
The experienced participants using fine-grained strokes (4/12, none of whom were experienced in prompting) achieved higher IoU scores (0.557) in the close-ended task (0.535) when using \tool. 
This is because their sketches were closer in shape and location to the reference, making the single object decomposition result more accurate.
Also, experienced participants are better at arranging spatial location and size of objects than novice participants.
However, some experienced participants (3/12) have mentioned that the fine-grained stroke sometimes makes them frustrated.
As P1's comment for his result in open-ended task: "\emph{It seems it cannot understand thin strokes; even if the shape is accurate, it can only generate content roughly around the area, especially when there is overlapping.}" 
This suggests that while \tool\ provides rough control to produce reasonably fine results from less accurate sketches for novice users, it may disappoint experienced users seeking more precise control through finer strokes. 
As shown in the last column in Figure~\ref{fig:novice_expert}, the dragon hovering in the sky was wrongly turned into a standing large dragon by \tool.

Second, regarding prompting skills, 3 out of 12 participants had one or more years of experience in T2I prompting. These participants used more modifiers than others during both T2I and R2I tasks.
Their performance in the T2I (0.335) and R2I (0.469) tasks showed higher scores than the average T2I (0.314) and R2I (0.418), but there was no performance improvement with \tool\ between their results (0.508) and the overall average score (0.528). 
This indicates that \tool\ can assist novice users in prompting, enabling them to produce satisfactory images similar to those created by users with prompting expertise.



\subsection{Applicability of \tool}
The feedback from user study highlighted several potential applications for our system. 
Three participants (P2, P6, P8) mentioned its possible use in commercial advertising design, emphasizing the importance of controllability for such work. 
They noted that the system's flexibility allows designers to quickly experiment with different settings.
Some participants (N = 3) also mentioned its potential for digital asset creation, particularly for game asset design. 
P7, a game mod developer, found the system highly useful for mod development. 
He explained: "\emph{Mods often require a series of images with a consistent theme and specific spatial requirements. 
For example, in a sacrifice scene, how the objects are arranged is closely tied to the mod's background. It would be difficult for a developer without professional skills, but with this system, it is possible to quickly construct such images}."
A few participants expressed similar thoughts regarding its use in scene construction, such as in film production. 
An interesting suggestion came from participant P4, who proposed its application in crime scene description. 
She pointed out that witnesses are often not skilled artists, and typically describe crime scenes verbally while someone else illustrates their account. 
With this system, witnesses could more easily express what they saw themselves, potentially producing depictions closer to the real events. "\emph{Details like object locations and distances from buildings can be easily conveyed using the system}," she added.

% \subsection{Model Understanding of Users' Implicit Intents}
% In region-sketch-based control of generative models, a significant gap between interaction design and actual implementation is the model's failure in understanding users' naturally expressed intentions.
% For example, when users draw sketches of two people with their hands slightly overlapping, current region-based models cannot automatically infer users' implicit intention that the two people are holding hands.
% Instead, they still require users to explicitly specify in the prompt such relationship.
% \tool addresses this through sketch-aware prompt recommendation to fill in the necessary semantic information, alleviating users' workload.
% However, some users want the generative AI in the future to be able to directly infer this natural implicit intentions from the sketches without additional prompting since prompt recommendation can still be unstable sometimes.
% This problem reflects a more general dilemma, which ubiquitously exists in all forms of conditioned control for generative models such as canny or scribble control.
% This is because all the control models are trained on pairs of explicit control signal and target image, which is lacking further interpretation or customization of the user intentions behind the seemingly straightforward input.
% For another example, the generative models cannot understand what abstraction level the user has in mind for her personal scribbles.
% Such problems leave more challenges to be addressed by future human-AI co-creation research.
% One possible direction is fine-tuning the conditioned models on individual user's conditioned control data to provide more customized interpretation. 

% \subsection{Balance between recommendation and autonomy}
% AIGC tools are a typical example of 
\subsection{Progressive Sketching}
Currently \tool is mainly aimed at novice users who are only capable of creating very rough sketches by themselves.
However, more accomplished painters or even professional artists typically have a coarse-to-fine creative process. 
Such a process is most evident in painting styles like traditional oil painting or digital impasto painting, where artists first quickly lay down large color patches to outline the most primitive proportion and structure of visual elements.
After that, the artists will progressively add layers of finer color strokes to the canvas to gradually refine the painting to an exquisite piece of artwork.
One participant in our user study (P1) , as a professional painter, has mentioned a similar point "\emph{
I think it is useful for laying out the big picture, give some inspirations for the initial drawing stage}."
Therefore, rough sketch also plays a part in the professional artists' creation process, yet it is more challenging to integrate AI into this more complex coarse-to-fine procedure.
Particularly, artists would like to preserve some of their finer strokes in later progression, not just the shape of the initial sketch.
In addition, instead of requiring the tool to generate a finished piece of artwork, some artists may prefer a model that can generate another more accurate sketch based on the initial one, and leave the final coloring and refining to the artists themselves.
To accommodate these diverse progressive sketching requirements, a more advanced sketch-based AI-assisted creation tool should be developed that can seamlessly enable artist intervention at any stage of the sketch and maximally preserve their creative intents to the finest level. 

\subsection{Ethical Issues}
Intellectual property and unethical misuse are two potential ethical concerns of AI-assisted creative tools, particularly those targeting novice users.
In terms of intellectual property, \tool hands over to novice users more control, giving them a higher sense of ownership of the creation.
However, the question still remains: how much contribution from the user's part constitutes full authorship of the artwork?
As \tool still relies on backbone generative models which may be trained on uncopyrighted data largely responsible for turning the sketch into finished artwork, we should design some mechanisms to circumvent this risk.
For example, we can allow artists to upload backbone models trained on their own artworks to integrate with our sketch control.
Regarding unethical misuse, \tool makes fine-grained spatial control more accessible to novice users, who may maliciously generate inappropriate content such as more realistic deepfake with specific postures they want or other explicit content.
To address this issue, we plan to incorporate a more sophisticated filtering mechanism that can detect and screen unethical content with more complex spatial-semantic conditions. 
% In the future, we plan to enable artists to upload their own style model

% \subsection{From interactive prompting to interactive spatial prompting}


\subsection{Limitations and Future work}

    \textbf{User Study Design}. Our open-ended task assesses the usability of \tool's system features in general use cases. To further examine aspects such as creativity and controllability across different methods, the open-ended task could be improved by incorporating baselines to provide more insightful comparative analysis. 
    Besides, in close-ended tasks, while the fixing order of tool usage prevents prior knowledge leakage, it might introduce learning effects. In our study, we include practice sessions for the three systems before the formal task to mitigate these effects. In the future, utilizing parallel tests (\textit{e.g.} different content with the same difficulty) or adding a control group could further reduce the learning effects.

    \textbf{Failure Cases}. There are certain failure cases with \tool that can limit its usability. 
    Firstly, when there are three or more objects with similar semantics, objects may still be missing despite prompt recommendations. 
    Secondly, if an object's stroke is thin, \tool may incorrectly interpret it as a full area, as demonstrated in the expert results of the open-ended task in Figure~\ref{fig:novice_expert}. 
    Finally, sometimes inclusion relationships (\textit{e.g.} inside) between objects cannot be generated correctly, partially due to biases in the base model that lack training samples with such relationship. 

    \textbf{More support for single object adjustment}.
    Participants (N=4) suggested that additional control features should be introduced, beyond just adjusting size and location. They noted that when objects overlap, they cannot freely control which object appears on top or which should be covered, and overlapping areas are currently not allowed.
    They proposed adding features such as layer control and depth control within the single-object mask manipulation. Currently, the system assigns layers based on color order, but future versions should allow users to adjust the layer of each object freely, while considering weighted prompts for overlapping areas.

    \textbf{More customized generation ability}.
    Our current system is built around a single model $ColorfulXL-Lightning$, which limits its ability to fully support the diverse creative needs of users. Feedback from participants has indicated a strong desire for more flexibility in style and personalization, such as integrating fine-tuned models that cater to specific artistic styles or individual preferences. 
    This limitation restricts the ability to adapt to varied creative intents across different users and contexts.
    In future iterations, we plan to address this by embedding a model selection feature, allowing users to choose from a variety of pre-trained or custom fine-tuned models that better align with their stylistic preferences. 
    
    \textbf{Integrate other model functions}.
    Our current system is compatible with many existing tools, such as Promptist~\cite{hao2024optimizing} and Magic Prompt, allowing users to iteratively generate prompts for single objects. However, the integration of these functions is somewhat limited in scope, and users may benefit from a broader range of interactive options, especially for more complex generation tasks. Additionally, for multimodal large models, users can currently explore using affordable or open-source models like Qwen2-VL~\cite{qwen} and InternVL2-Llama3~\cite{llama}, which have demonstrated solid inference performance in our tests. While GPT-4o remains a leading choice, alternative models also offer competitive results.
    Moving forward, we aim to integrate more multimodal large models into the system, giving users the flexibility to choose the models that best fit their needs. 
    


\section{Conclusion}\label{sec:conclusion}
In this paper, we present \tool, an interactive system designed to help novice users create high-quality, fine-grained images that align with their intentions based on rough sketches. 
The system first refines the user's initial prompt into a complete and coherent one that matches the rough sketch, ensuring the generated results are both stable, coherent and high quality.
To further support users in achieving fine-grained alignment between the generated image and their creative intent without requiring professional skills, we introduce a decompose-and-recompose strategy. 
This allows users to select desired, refined object shapes for individual decomposed objects and then recombine them, providing flexible mask manipulation for precise spatial control.
The framework operates through a coarse-to-fine process, enabling iterative and fine-grained control that is not possible with traditional end-to-end generation methods. 
Our user study demonstrates that \tool offers novice users enhanced flexibility in control and fine-grained alignment between their intentions and the generated images.



\bibliography{main}
\bibliographystyle{icml2025}


%%%%%%%%%%%%%%%%%%%%%%%%%%%%%%%%%%%%%%%%%%%%%%%%%%%%%%%%%%%%%%%%%%%%%%%%%%%%%%%
%%%%%%%%%%%%%%%%%%%%%%%%%%%%%%%%%%%%%%%%%%%%%%%%%%%%%%%%%%%%%%%%%%%%%%%%%%%%%%%
% APPENDIX
%%%%%%%%%%%%%%%%%%%%%%%%%%%%%%%%%%%%%%%%%%%%%%%%%%%%%%%%%%%%%%%%%%%%%%%%%%%%%%%
%%%%%%%%%%%%%%%%%%%%%%%%%%%%%%%%%%%%%%%%%%%%%%%%%%%%%%%%%%%%%%%%%%%%%%%%%%%%%%%
\newpage
\appendix
\onecolumn

\section{More Epochs or More Data?}

In this section, we explore the trade-off between increasing the number of training epochs and expanding the dataset size under a fixed computational budget. Specifically, we aim to answer the following question:

\highlighttext{Given a fixed computational budget, is it more effective to train on a smaller dataset for more epochs or to train on a larger dataset for fewer epochs?}

\textbf{Setup.} To investigate this question, we compare two scenarios under the same computational budget of \(10M\) samples: (1) training on \(2M\) samples for 5 epochs, and (2) training on \(1M\) samples for 10 epochs. We evaluate the performance of models pretrained on ANI-1x, Transition-1x, OC20, and OC22, and fine-tune them on the downstream datasets: rMD17, MD22, SPICE, and QM9. For comparison, we also include the results of JMP-L and JMP-S, which use \(120M\) samples for 2 epochs.

\textbf{Results.} Table \ref{tab:rehearsal} presents the downstream performance for the two scenarios. Across all datasets, ANI-1x consistently achieves the best performance, regardless of whether the model is trained on \(2M\) samples for 5 epochs or \(1M\) samples for 10 epochs. For example, on rMD17, ANI-1x achieves a test error of 5.4 in both scenarios, outperforming JMP-S (6.7) and JMP-L (5.1). Similarly, on SPICE, ANI-1x achieves a test error of 5.08 (2M samples, 5 epochs) and 5.04 (1M samples, 10 epochs), compared to 5.71 for JMP-S and 4.75 for JMP-L.

Interestingly, increasing the number of epochs from 5 to 10 while reducing the dataset size from \(2M\) to \(1M\) does not significantly degrade performance for ANI-1x. This suggests that for highly relevant datasets like ANI-1x, training on fewer samples for more epochs can be as effective as training on more samples for fewer epochs. However, for less relevant datasets like OC20 and OC22, increasing the number of epochs does not compensate for the reduced dataset size, as their performance degrades significantly.

\textbf{Conclusion.} Our findings indicate that the choice between more epochs or more data depends on the relevance of the dataset to the downstream task. For highly relevant datasets like ANI-1x, training on fewer samples for more epochs can yield comparable or even superior performance, while for less relevant datasets, increasing the dataset size is more beneficial. This further underscores the importance of dataset quality and relevance, as quantified by CSI, in determining the optimal pretraining strategy.


% \begin{table*}[ht]
% \centering
% \caption{Diversity or rehearsal (reviewed)}
% \begin{tabular}{lllcccc}
% \hline
% Pretraining Budget & Pretraining Source & Backbone & rMD17 & MD22 & SPICE & QM9  \\
% \hline
%  & ANI1x & GemNet-OC-S   & 5.4 & 2.88 & 5.04 & 2.9 \\
%                         & Transition1x              & GemNet-OC-S & 10.6 & 3.79 & 7.50 & 3.1 \\
%                         & OC20                      & GemNet-OC-S & 14.8 & 4.67 & 10.16 & 4.9 \\
%                         & OC22                      & GemNet-OC-S & 17.3 & 5.24 & 11.06 & 5.4 \\
% \hline
% \end{tabular}
% \label{tab:rehearsal}
% \end{table*}



\begin{table*}[h!]
\centering
\caption{Trade-off between increasing the number of samples and the number of epochs. We report the MAE for various downstream tasks while varying the pretraining sample count and epoch count simultaneously. \(\mathcal{C}\), \(\mathcal{N}\), and \(\mathcal{E}\) denote the computational budget, number of samples, and number of epochs, respectively.}
\setlength{\tabcolsep}{8pt}
\begin{tabular}{ccclccccc}
\toprule
$\mathbf{\mathcal{C}}$ & $\mathbf{\mathcal{N}}$ & $\mathbf{\mathcal{E}}$ & \textbf{Upstream Data} & \textbf{Backbone} & \textbf{rMD17} & \textbf{MD22} & \textbf{SPICE} & \textbf{QM9} \\
 &  &  &  &  & (meV/\AA) & (meV/\AA) & (meV/\AA) & (meV) \\
\midrule

\multirow{4}{*}{\( 10 \times 10^6 \)}   
&  \multirow{4}{*}{\( 2 \times 10^6 \)}  
&   \multirow{4}{*}{\( 5 \)}
    & ANI-1x         & \multirow{4}{*}{GemNet-OC-S} & \textbf{5.4} & \textbf{2.90} & \textbf{5.13} & \textbf{2.9} \\
    &&& Transition-1x  &  & 10.1 & 3.73 & 7.55 & 3.2  \\
    &&& OC20          &  & 14.6 & 4.53 & 8.74 & 4.8  \\
    &&& OC22          &  & 16.0 & 5.20 & 10.73 & 5.7  \\
\midrule



\multirow{4}{*}{\( 10 \times 10^6 \)}  
&  \multirow{4}{*}{\( 1 \times 10^6 \)}  
&   \multirow{4}{*}{\( 10 \)}
    & ANI1x         & \multirow{4}{*}{GemNet-OC-S}      & \textbf{5.4} & \textbf{2.88} & \textbf{5.04} & \textbf{ 2.9} \\
    &&& Transition1x              & & 10.6 & 3.79 & 7.50 & 3.1 \\
    &&& OC20                      & & 14.8 & 4.67 & 10.16 & 4.9 \\
    &&& OC22                      & & 17.3 & 5.24 & 11.06 & 5.4 \\

\bottomrule
\end{tabular}
\label{tab:rehearsal}
\end{table*}




\clearpage


\section{Additional Analysis on the Metric Design}

\begin{figure*}[h]
\centering
\includegraphics[width=0.75\linewidth]{images/CSI_bar_balanced_mean_equiformerV2_ID.png}
\caption{\textbf{Impact of using mean aggregation instead of flattening on CSI values.} We notice that the mean pooling incorrectly reduced the score for OC22 potentially due to over-smoothing.} 
\vspace{-0.4cm}
\label{fig:CSI_mean}
\end{figure*}


\begin{figure*}[h]
\centering
\includegraphics[width=0.75\linewidth]{images/CSI_bar_random_flat_equiformerV2_ID.png}
\caption{\textbf{Impact of using random sampling strategy instead of class-balanced sampling.} As highlighted in the long-tail analysis in Appendix \ref{long_tail}, random sampling can lead to class underrepresentation, potentially affecting the correlation between upstream and downstream tasks. Notably, both ANI-1x and Transition-1x exhibit different patterns compared to the class-balanced values reported in the main paper.}
\vspace{-0.4cm}
\label{fig:CSI_random}
\end{figure*}

\begin{figure*}[h]
\centering
\includegraphics[width=0.75\linewidth]{images/CSI_bar_balanced_flat_JMP_ID.png}
\caption{\textbf{The impact of using another backbone}. We use JMP pretrained model and show that similar insights are obtained where Ani-1x is shown as the most similar.} 
\vspace{-0.4cm}
\label{fig:CSI_JMP}
\end{figure*}

\clearpage

\section{Long-Tail Analysis}
\label{long_tail}
As discussed in Section \ref{3_setup}, molecular understanding datasets are typically imbalanced and exhibit a skewed distribution. Figure \ref{fig:upstream_sampling} illustrates the effect of \textit{random sampling} versus \textbf{class-balanced sampling}, which we employ in this study, in terms of class coverage. Notably, random sampling results in significant class underrepresentation, whereas class-balanced sampling ensures broader coverage across all classes.

\begin{figure*}[h]
\centering
\includegraphics[width=\linewidth]{images/upstream_sampling.png}
\caption{\textbf{Impact of sampling strategies on subset construction for feature extraction.} We sample 10K instances for each upstream task, highlighting the differences in class coverage between random and class-balanced sampling.} 
\label{fig:upstream_sampling}
\end{figure*}




\clearpage
\section{Implementation Details}
% QMOF we used a LR 5e-5 since the pretrained model on transition was giving NaN when using 8e-5. 
For both pretraining and fine-tuning experiments, we primarily follow the JMP hyperparameters. However, due to resource constraints requiring smaller batch sizes compared to JMP, we adjusted the learning rate to ensure training stability, as detailed below.

\textbf{For pretraining,} we use a batch size of 20 and a learning rate (LR) of 1e-4 for the small backbone (GemNet-OC-S). For the large backbone (GemNet-OC-L), the batch size is reduced to 12 to fit GPU memory. Additionally, when training with the OC22 dataset on the large backbone, a LR of 1e-4 caused gradient instability, thus we used a LR of 1e-5 for that particular run. Unless otherwise specified, each experiment is run for five epochs on the specified number of samples for each section of the paper. The best checkpoint is selected based on the performance in the validation set. To handle the large size of the upstream validation sets, validation is performed on a smaller subset of 2,000 samples.


\textbf{For finetuning,} we use the batch size specified in the JMP codebase and a default learning rate (LR) of 8e-5, except for cases where adjustments were needed to stabilize training. Specifically, we use 5e-5 for QMOF, 8e-4 for MatBench when pretrained on Transition1x. 
% 4e-5 for MD22 when using the large backbone with ANI-1x pretraining.

\clearpage


% \section{You \emph{can} have an appendix here.}

%%%%%%%%%%%%%%%%%%%%%%%%%%%%%%%%%%%%%%%%%%%%%%%%%%%%%%%%%%%%%%%%%%%%%%%%%%%%%%%
%%%%%%%%%%%%%%%%%%%%%%%%%%%%%%%%%%%%%%%%%%%%%%%%%%%%%%%%%%%%%%%%%%%%%%%%%%%%%%%


\end{document}


% This document was modified from the file originally made available by
% Pat Langley and Andrea Danyluk for ICML-2K. This version was created
% by Iain Murray in 2018, and modified by Alexandre Bouchard in
% 2019 and 2021 and by Csaba Szepesvari, Gang Niu and Sivan Sabato in 2022.
% Modified again in 2023 and 2024 by Sivan Sabato and Jonathan Scarlett.
% Previous contributors include Dan Roy, Lise Getoor and Tobias
% Scheffer, which was slightly modified from the 2010 version by
% Thorsten Joachims & Johannes Fuernkranz, slightly modified from the
% 2009 version by Kiri Wagstaff and Sam Roweis's 2008 version, which is
% slightly modified from Prasad Tadepalli's 2007 version which is a
% lightly changed version of the previous year's version by Andrew
% Moore, which was in turn edited from those of Kristian Kersting and
% Codrina Lauth. Alex Smola contributed to the algorithmic style files.
