\section{Conclusion}\label{sec:conclusion}

In this paper, we introduced a novel, general purpose anomaly detection algorithm based on optimal transport theory. Our method works under the assumption that the local neighborhood of anomalous samples is likely more irregular than that of normal samples. Based on this idea, we engineer the ground-cost in optimal transport, to encourage samples to send their mass \emph{just outside} an exclusion zone, defined through its $k-$nearest neighbors. While this idea bears some similarity to optimal transport with repulsive costs~\citep{di2017optimal}, the defined ground-cost is not repulsive, and, as we show in our experiments, it leads to better anomaly detectors. We thoroughly experiment on the AdBench~\citep{han2022adbench} benchmark, and the Tennessee Eastman process~\citep{downs1993plant,reinartz2021extended,montesuma2024benchmarking}, showing that our method outperforms previously proposed methods. However, as we highlight in section~\ref{ex:adbench}, our method inherits the limitations of classic optimal transport~\citep{montesuma2024recentadvancesoptimaltransport}. Especially, it needs $\mathcal{O}(n^{2})$ computational storage, has $\mathcal{O}(n^{3}\log n)$ computational complexity and is difficult to estimate in high dimensions. Nonetheless, as our experiments demonstrate, our method is practical and performative in real-world anomaly detection scenarios.

\newpage