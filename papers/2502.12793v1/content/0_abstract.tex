\begin{abstract}
Detecting anomalies in datasets is a longstanding problem in machine learning. In this context, anomalies are defined as a sample that significantly deviates from the remaining data. Meanwhile, optimal transport (OT) is a field of mathematics concerned with the transportation, between two probability measures, at least effort. In classical OT, the optimal transportation strategy of a measure to itself is the identity. In this paper, we tackle anomaly detection by forcing samples to displace its mass, while keeping the least effort objective. We call this new transportation problem Mass Repulsing Optimal Transport (MROT). Naturally, samples lying in low density regions of space will be forced to displace mass very far, incurring a higher transportation cost. We use these concepts to design a new anomaly score. Through a series of experiments in existing benchmarks, and fault detection problems, we show that our algorithm improves over existing methods.
\end{abstract}

\keywords{Optimal Transport \and Anomaly Detection \and Fault Detection \and Cross Domain Fault Detection}