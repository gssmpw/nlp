\section{Criterion I: Completeness (Table \ref{tab:completeness})}\label{sec:completeness}

The completeness of planning has two key aspects: (1) if a valid plan exists, the model should generate it correctly, and (2) if no feasible plan is possible, the model should recognize this and refrain from generating an incorrect or arbitrary plan.

A plan is correct if it achieves the goal within a fixed budget while avoiding excessive complexity and infinite loops. To ensure correctness, the LLM must work with classical sound and complete solvers \cite{guan2023leveraging, hao2024large}. Also, the LLM has to accurately translate the domain and problem into the specific format (e.g., PDDL), required by these solvers \cite{guan2023leveraging}.

In terms of identifying unsolvable planning problems, those with inherently unachievable goals, even top LLMs (e.g., GPT-4 \cite{achiam2023gpt}) and Large Reasoning Models (e.g., OpenAI O1 \cite{jaech2024openai}) struggle due to hallucination issues \cite{aghzal2023can, valmeekam2024llms}. 
