\section{Criterion IV: Representation (Tab. \ref{tab:representation_1}-\ref{tab:representation_2})} \label{sec:representation}

In LLM planning, representation refers to how inputs and outputs are formatted. Inputs include domains (predicates and actions), problems (initial and goal states), and environmental observations, while outputs are the generated plans. Effective representation enhances problem comprehension and execution efficiency, especially given LLMs' sensitivity to prompts. We discuss this in two contexts: LLM-as-a-Translator and LLM-as-a-Planner. 

\vspace{-0.05in}
\paragraph{LLM-as-a-Translator} LLM-as-a-Translator converts between natural language (NL) and formal planning languages (e.g. PDDL), making classic planners more accessible to non-experts. By converting natural language tasks into formal representations and translating the resulting plans back into NL, LLMs reduce ambiguity, minimize hallucinations, and enable external validation, improving both usability and reliability in planning systems \cite{xie2023translating, zhou2024isr, sun2024adaplanner, silver2024generalized}. 

Recent work has used LLMs to translate natural language descriptions into PDDL \cite{liu2023llm+, guan2023leveraging, xie2023translating, dagan2023dynamic, zhou2024isr}, LTL \cite{pan2023data}, and STL \cite{chen2024autotamp}. To ensure reliability, translations should be tested on development or external datasets like Planetarium \cite{zuo2024planetarium}. If there are syntax or semantic errors, validators (e.g. VAL \cite{howey2004val}) or human experts can provide feedback for the LLM to fix them. 

\vspace{-0.05in}
\paragraph{LLM-as-a-Planner} When LLMs act as standalone planners without classical planners or optimizers, various methods help encode environmental information, domains, and plans beyond just natural language. Environment and domain details have been represented using \emph{tables} \cite{lin2023grounded}, \emph{condensed symbols} \cite{hu2024chain}, \emph{Pythonic code} \cite{aghzal2023can, singh2023progprompt, sun2024adaplanner}, \emph{neural embeddings} \cite{li2022pre, ahn2022can}, and \emph{graphs} \cite{lin2024graph, wu2024can}. For generated plans, Pythonic code is a common alternative to natural language \cite{singh2023progprompt, silver2024generalized}. 
