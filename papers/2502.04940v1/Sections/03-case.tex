\section{Case Examples}
\label{sec:case}

% 三个标志性的examples,影响力;时间阶段 + 资料丰富性:Machine Hallucination;Smellmap;TYPE+CODE II


% 参考:Communicating with Motion: A Design Space for Animated Visual Narratives in Data Videos
% 参考:Data as Delight: Eating data
% 参考:Narrative Visualization: Telling Stories with Data


As artistic data visualization is a relatively under-explored field in academia, we follow prior work~\cite{segel2010narrative,shi2021communicating} and first introduce three typical cases to provide an initial exploration and motivation for the study of artistic data visualization. This approach not only offers a preliminary understanding of the practices in this domain, but also provides representative examples from the corpus analyzed in \autoref{sec:space}, helping to contextualize the more abstract analysis presented later.
% provides an initial sense of the practices in this field and
These cases were selected based on three main criteria: (i) we searched public materials about the artworks in \autoref{sec:space} and identified those that have been exhibited in top-tier exhibitions (\eg having a work exhibited at MoMA in New York is regarded as a high recognition for artists), (ii) works that have been widely reported or recognized by news media, (iii) artists that have received prestigious career awards. Based on these criteria, we selected three works that best meet them and span a considerable range of time.
% they are influential artworks that have garnered significant recognition, such as pieces honored with prestigious awards and featured in renowned art exhibitions; (ii) the selected works should span a range of times to reflect diverse technological and artistic trends; (iii) highlighted by esteemed media outlets.
% (iii) there should be an abundance of available materials, such as artist statements, interviews, and talks, to fully elucidate the creation process and the ideas behind these artworks.
% offering ample material for close reading~\cite{brummett2018techniques}.
% Below, we introduce the three cases one by one.

% For example, \x{Smellmap is an art project that spans over ten years. Its artist has not only maintained a comprehensive website for it, providing detailed information on various aspects of the project, but has also written a series of papers based on this project and has been interviewed by the media on multiple occasions.}




\subsection{Case I: Particle Dreams in Spherical Harmonics}
\label{ssection:particle}


\begin{figure}[h]
 \centering
 \includegraphics[width=\columnwidth]{figures/case_1.png}
 \vspace{-2em}
 \caption{Particle Dreams in Spherical Harmonics~\cite{sandin}.}
 \label{fig:cases_1}
 % \Description{Representative images of the three artworks: Upper left: Particle Dreams in Spherical Harmonics; Lower left: Smell Maps; Right: Machine Hallucinations.}
 \vspace{-0.5em}
\end{figure}

Particle Dreams in Spherical Harmonics is a Virtual Reality (VR) artwork produced by Dan Sandin and his team. As an internationally recognized pioneer of electronic art and visualization~\cite{sandin}, Sandin is the director emeritus of the Electronic Visualization Laboratory (EVL) and a professor emeritus in the School of Art and Design at the University of Illinois Chicago. With a background in physics, Sandin has dedicated his career to exploring the intersection of technology and art.
% , and has received numerous honors for his innovative contributions (\eg IEEE Virtual Reality Academy, the Rockefeller Foundation Media Arts Fellowship). As an artist, he has exhibited worldwide and received grants from globally renowned art foundations.
% and has received grants in support of his work from the Rockefeller Foundation, the Guggenheim Foundation, the National Science Foundation and the National Endowment for the Arts. His video animation Spiral PTL is in the inaugural collection of video art at the Museum of Modern Art in New York.
% Throughout his career, Sandin has received numerous accolades for his contributions to the field, including the IEEE Visualization Career Award and the SIGGRAPH Steven Anson Coons Award. His legacy continues to inspire new generations of artists and technologists who are exploring the convergence of art and technology in the digital age.
% His work encompasses a range of groundbreaking projects that have significantly influenced the field of artistic data visualization. 
In the 1970s, he developed the Sandin Image Processor, a highly programmable analog computer for processing live video feeds. It was one of the early devices that allowed artists to manipulate video data inputs in real-time, solving the problem of computer-graphics systems being too expensive and not easily accessible to most people~\cite{johnson2024electronic}.
% while also opening up new possibilities for creating dynamic art installations.
% He then worked with DeFanti to combine the Image Processor with real-time computer graphics and performed visual concerts, the Electronic Visualization Events, with synthesized musical accompaniment. 
Since the 1990s, Sandin and his colleagues began developing the CAVE (Cave Automatic Virtual Environment), a pioneering VR theater system that provides immersive experiences in a 3D space where computer-generated imagery is projected onto walls and floors.
% a pioneering VR theater system that provides immersive experiences. It allows users to step into a three-dimensional space where computer-generated imagery is projected onto walls and floors, creating a fully interactive environment. 


% Additionally, Sandin co-developed the Sayre Glove, the first data glove, which enabled users to interact with digital environments through hand movements, further bridging the gap between physical and digital art forms.
% In 1969, Sandin developed a computer-controlled light and sound environment, called Glow Flow, at the Smithsonian Institution and was invited to join the art faculty at the University of Illinois the same year. 

Sandin himself also used these technologies to produce a series of data artworks, particularly in the realm of scientific visualization. \autoref{fig:cases_1}, for example, is a VR artwork based on the ``physical simulation of over one million particles with momentum and elastic reflection in an environment with gravity.''~\cite{sandin}
% In the final scene there is a very realistic rendering of water with reflections, and lighting based on spherical harmonics. 
It also creates a multisensory experience by generating sound that is triggered and modified by the user-particle interactions in real-time.
A viewer commented that standing in the artwork ``was like standing in a rainstorm made of rainbow fragments, with the power to guide the storm by hand. It was unsettling, out-of-body, very trippy stuff, a powerful artistic experience.''~\cite{evl_vr}
% This work was premiered in January 2011 at the gallery@calit2 University of California San Diego, designed for exhibition in the Qualcomm Institute, UC San Diego division of Calit2’s StarCAVE and the CAVE2™ developed by the Electronic Visualization Laboratory, University of Illinois at Chicago.
When talking about the motivations of their work, Sandin put more emphasis on building a ``more effective communication medium and a much more effective way to visualize data''~\cite{evl_synthesis}.
% He believed artists should use (or build) the most advanced technologies of their time to create their art and would often point out that Renaissance painters made their own paints~\cite{johnson2024electronic}.
This is also the philosophy of his lab, that is, ``systems should be user-oriented (easy to use, easy to learn), low-cost, interactive, and real-time (to provide immediate feedback).''~\cite{johnson2024electronic}
As for himself as an artist, Sandin adopted a rather open attitude to creativity: ``About creativity—my personal view of it is kind of like I’m a pipe or conduit. And all this stuff just happens to be flowing through me because I’ve chosen to position myself in that flow. I have no problem with the word `creation' as long as people don’t lay too much molasses on it.''~\cite{vdb} 



\subsection{Case II: Smell Maps}
\label{ssection:smell}

\begin{figure}[h]
 \centering
 \includegraphics[width=\columnwidth]{figures/case_2.png}
 \vspace{-2em}
 \caption{Smell Maps~\cite{smellmaps}.}
 \label{fig:cases_2}
 % \Description{Representative images of the three artworks: Upper left: Particle Dreams in Spherical Harmonics; Lower left: Smell Maps; Right: Machine Hallucinations.}
 \vspace{-1em}
\end{figure}


Since 2010, British artist Kate McLean has been working on translating the sensed aspects of place into visualizations. Starting with the first smell map of Paris, she has created a set of smell maps in various cities (\eg in \autoref{fig:cases_2}, the first two maps show Edinburgh and New York, respectively).
She used olfactory walks to collect data by first selecting specific routes in the cities, and then recruiting volunteers for these walks. The characteristics of the smells (\eg name, intensity) were recorded by the volunteers using smell notes (\autoref{fig:cases_2}, right). She also designed activities such as ``smell catching'' (noticing distant, airborne smells when walking) and ``smell hunting'' (searching for the sources of smell) to spark participants' interest and sensational involvement during walks~\cite{mclean2019nose}. 
% They have also produced a toolbox called 'smellfie,' which provides a DIY guide for the mapping of urban smells and is available for public and educational download.
% McLean categorizes olfactory walks into individual walks, group walks, paired walks, and a toolbox (smellfie) for educational and public use. 


Smell maps were motivated by the urban research by Douglas Porteous and Charles Foster~\cite{mclean2019nose}, who offered profound reflections and criticisms on contemporary living spaces, such as our alienation from physical experiences and over-reliance on vision.
% Charles Foster~\cite{} criticized the overdependence of humans on sight, namely their inability to relate to the entirety of their environment.
% criticized that urban dwellers are increasingly alienated from physical sensory experiences. 
Porteous theorized the concept of the \textit{smellscape}, pointing out that ``like visual impressions, smells may be spatially ordered or place-related''. But unlike an ordered visual landscape, the smellscape is an emotive environment that is ``non-continuous, fragmentary in space and episodic in time and limited by the height of our noses from the ground, where smells tend to linger.''~\cite{porteous1985smellscape}
Influenced by these theories, McLean's smell maps have placed a strong emphasis on physical participation and personal interpretation. She based all her work ``on physical experiences, rather than algorithms,''~\cite{cnn} and visualized these individual, subtle, and subjective olfactory data through artistic expressions (\eg using colored spots and rippling circles to present the sources and diffusion of smells). 
When designing visuals, McLean did not seek singular, precise scientific results but rather aimed for ``a visual synthesis of the different experiences reported by smellwalkers...I am interested in negotiating different perceptions.''~\cite{atlas}
% While engaging in this art project, McLean has concurrently developed her academic research on smellscapes. 
In essence, she considered her practice and research to be qualitative~\cite{marie}: ``My aim is to celebrate and highlight the subjective elements of human perceptions of the smellscape...I never claim that the scents are objective, and my research to date indicates that it may not even be possible.''

% developing a language for olfactory cartography, creating an olfactory dictionary with descriptive terms, an olfactory color DNA that represents smells with colors, and summarizing experiences from nearly a hundred olfactory walks. 
% McLean designed a model for the diffusion of smells, first marking the source of each smell, representing the duration of the smell with the size and layers of concentric circles, and using the recorded wind direction and speed during experiments to calculate the perturbation of the circles. The final smell map can present the source and spread range of each smell, as well as the time of diffusion. 
% The word ‘smellscape’ has its roots in cultural geography (first used by Gade in 1984) and later theorised by Porteous whose seminal paper also indicated the challenges inherent in studying and recording sensuous olfactory worlds. Porteous’ conception of the smellscape is as the totality of the olfactory landscape in a specific place; including both background and episodic smells (his term for temporary or site-specific aromas). His contention is that, unlike an ordered visual landscape, the smellscape is an emotive environment that is ‘non-continuous, fragmentary in space and episodic in time’.



\subsection{Case III: Machine Hallucination}
\label{ssection:machine}

\begin{figure}[h]
 \centering
 \includegraphics[width=\columnwidth]{figures/case_3.png}
 \vspace{-2em}
 \caption{Machine Hallucinations~\cite{machine}.}
 \label{fig:cases_3}
 % \Description{Representative images of the three artworks: Upper left: Particle Dreams in Spherical Harmonics; Lower left: Smell Maps; Right: Machine Hallucinations.}
 \vspace{-2em}
\end{figure}

Machine Hallucination is a multi-series art project by Refik Anadol and his studio, utilizing big data and artificial intelligence (AI) to create immersive art experiences.
For example, \textit{Machine Hallucination - NYC}, as the first artwork in this series, employed machine-learning algorithms to process over 100 million photographs of New York. \textit{Machine Hallucination - Coral Dreams} was based on the training of more than 35 million images of coral. \textit{Machine Hallucination - Unsupervised} explored MoMA's vast collection, encompassing 150 years and nearly 140,000 art pieces. It was transformed into what Anadol called a ``living data sculpture'': a piece of artwork that is constantly changing, projecting an infinite number of alternative artworks that the machine creates in real-time across a giant media wall~\cite{yahoo} (\autoref{fig:cases_3}, left).
% % To create the collection, he and his team fed more than 138,000 images of individual works from MoMA’s archives - including paintings, performance art, video games and sculptures - into a machine-learning model~\cite{yahoo}.
% acquired as the MoMA’s permanent collection.
% Machine Hallucinations: Space, which Anadol describes as “a visual speculation of humanity’s historical attempts to explore the depths of space.” The piece uses millions of raw images from the Hubble Space Telescope and the International Space Station to teach the AI models. The AI transforms these beautiful images of the Earth, the universe, their colors, and their forms into what Anadol calls “data pigments” to create an animated image that morphs organically over time. 
%In Machine Hallucinations: Nature, Anadol uses 400 million publicly available photographs of flora and fauna to create a different form of “pigment.” These natural blocks are then animated by data of the wind and gust speed, as well as precipitation and air pressure, all captured from sensors in Las Vegas. Anadol likens the process to how Claude Monet was “inspired by the atmosphere and became this incredible impressionist painter.”
% The AI system is trained on this data, which enables it to generate new, unique visual patterns to "hallucinate" new imagery based on its learned patterns and associations. 
In a 2021 interview~\cite{momainterview}, Anadol said, ``In the past five years, we’ve trained more than 100 AI models, and used close to five petabytes of raw data. This is, as far as I know, one of the most challenging datasets ever used beyond specific research contexts, from clouds to national parks to cities to urban studies of Seoul, Stockholm, Berlin, Istanbul, New York, Los Angeles.'' 
% These projections or digital installations often appear as abstract, flowing forms or surreal landscapes that seem to emerge from the machine’s "imagination." The visuals are constantly evolving, reflecting the ongoing processing of data by the AI.
Characterized by blossoming colors, biomechanical shapes, and constantly evolving data patterns, these artworks have achieved immense success. They have been projected onto notable architectural landmarks, such as the Walt Disney Concert Hall, Casa Battló, and the 580,000-square-foot Las Vegas Sphere (\autoref{fig:cases_3}, right).


Anadol's artwork demonstrates a strong pursuit of machine aesthetics. In a TED talk~\cite{ted}, he described his motivation for creating such artworks: ``Can data become a pigment? This was the very first question we asked when starting our journey to embed media arts into architecture, to collide virtual and physical worlds. So we began to imagine what I would call the poetics of data.'' Specifically for Machine Hallucination, it is an ambitious experiment concerning whether machines can dream and an attempt to re-present how machines interpret vast amounts of data in its ``brain'' (``AI in this case is creating this pigment that doesn't dry, a pigment that is always in flux, always in change, and constantly evolving and creating new patterns.''~\cite{yahoo})
% `` I use data as a pigment and paint with a thinking brush that is assisted by artificial intelligence. Using architectural spaces as canvases, I collaborate with machines to make buildings dream and hallucinate.'' 
While his works are highly technical, Anadol believes: ``Artificial intelligence is a mirror for humanity...It’s all about who we are as humans.''~\cite{fastcompany2}
% Machine Hallucination is a tribute to nature using data (``We respect nature, we believe in nature. And for that reason, we transformed one of our largest datasets.''~\cite{fastcompany} 



% For me, art reflects humanity’s capacity for imagination. And if I push my compass to the edge of imagination, I find myself well connected with the machines, with the archives, with knowledge, and the collective memories of humanity.

% Machine Hallucination has been well-received in the art and technology communities for its innovative approach and aesthetic appeal. It has been exhibited in various international venues, contributing to ongoing discussions about the role of technology in contemporary art. Anadol’s work continues to push the boundaries of what is possible with data and AI, offering new ways to think about the intersection of technology and creativity.

