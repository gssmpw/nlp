
\firstsection{Introduction}
\maketitle


As a discipline that deals with visuals, data visualization has a natural affinity with art. Fueled by the rise of big data, artists are increasingly using digital tools and programming software to create data-driven artworks. In 2007, Viégas and Wattenberg~\cite{viegas2007artistic} proposed \textit{artistic data visualization} and defined it as ``visualizations of data done by artists with the intent of making art'', viewing it as an enchanting domain beyond visual analytics.
Today, as noted by a BBC article~\cite{bbc}, artists are ``making waves with big data'', and some data artworks have had a significant impact on society and the market (\eg the non-fungible token (NFT) series based on the artworks discussed in \autoref{ssection:machine} was sold for \$5.1 million at auction~\cite{yahoo}).
% in 2021
% the media outlet, Fast Company, commented that the artist is ``arguably the most successful AI artist on the planet.''~\cite{fastcompany2}
% Besides, in certain aspects, artistic data visualization has shown comparative advantages over traditional visualization. For example, 
As an artist who has created a series of influential data artworks, Frankel~\cite{frankel1998envisioning} once wrote in \textit{Science} that she realized artworks can achieve significant consequences, especially ``to communicate important information about science research not only to other scientists in the lab, or in the field, but to a broader, nonscientific public, as well.'' 
% Artistic approaches have also facilitated novel visualization designs and applications in movements such as ambient information visualization~\cite{skog2003between,pousman2007casual} and affective visualization design~\cite{lan2023affective}.

However, compared to its flourish in the industry, research focused on artistic data visualization remains relatively scarce. As argued by Kosara~\cite{kosara2007visualization}, the ``classifications of visualization are often based on technical criteria, and leave out artistic ways of visualizing information''. 
To the best of our knowledge, formative research in artistic data visualization (\eg full papers published in high-quality journals or conferences) is not very abundant, with most studies focusing on specific applications, such as addressing issues like energy use and environmental pollution~\cite{rodgers2011exploring,schroeder2015visualization,perovich2020chemicals}. 
By contrast, theoretical research on artistic data visualization appears insufficient, leaving many questions underexplored. These include: What are the design features of artistic data visualizations? What values and viewpoints do data artists hold? How might artistic data visualization inspire or benefit the visualization community?

To address these gaps, this work undertook research in three main steps. First, we collected a corpus of artistic data visualizations to identify representative cases (\autoref{sec:case}) and characterize their design features by analyzing their design paradigms, intents, and techniques (\autoref{sec:space}). Next, we interviewed twelve data artists to gain deeper insights into their practical experiences and perspectives on artistic data visualization, such as how they describe their aesthetic pursuits and respond to potential critiques (\autoref{sec:interview}).
Based on these findings, we discuss the implications of artistic data visualization for our community and introduce future research opportunities (\autoref{sec:discuss}). The corpus can be visually browsed at \url{https://artisticvis.github.io/}, and the raw datasets and codes can be accessed at \url{https://bit.ly/47ER8aJ}.

This work takes an initial step toward understanding artistic data visualization, an important form that expands visualization applications and blends technology with art. In summary, we contribute (i) empirical understanding about the design paradigms, intents, and existing techniques of artistic data visualizations by analyzing a corpus of artworks, (ii) first-hand insights from practitioners that provide valuable perspectives and foster community dialogue, and (iii) guidance for future work to explore new possibilities and advancements, inspired by the dialogue between art and science.
% It delves into the design features that characterize artistic data visualizations, uncovers the values and viewpoints that guide data artists in their creative processes, as well as explores the potential opportunities that artistic data visualization can bring to the broader visualization community.
