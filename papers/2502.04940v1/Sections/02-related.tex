\section{Background and related work}
% 重点看Artistic data visualization: Beyond visual analytics 和Visualization criticism-the missing link between information visualization and art 的被引


This section reviews the background on artistic data visualization and previous research related to this topic.

\subsection{Artistic Data Visualization in Art History Context}
\label{ssec:contemporary}

Art history has been marked by transformative periods characterized by different aesthetic pursuits, materials, and methods. Since the time of Plato, imitation (or \textit{mimesis}, which views art as a mirror to the world around us) has been an important pursuit~\cite{pooke2021art}. Successful artworks, such as Greek sculptures and the convincing illusions of depth and space in Renaissance paintings, exemplify this tradition.
The advent of modern society and new technology, especially photography, posed a significant challenge to the notion of art as imitation~\cite{perry2004themes}. By the 1850s, modern art began to emerge with the core goal of transcending traditional forms and conventions. Movements like Post Impressionism, Expressionism, and Cubism revolutionized art through innovative uses of form (\eg color, texture, composition), moving art towards abstraction and experimentation. 
After World War II, the Cold War and the surge of various social problems heightened skepticism about the progress narrative of modernity and the superiority of the capitalist system, leading to the rise of postmodernism and the birth of contemporary art~\cite{hopkins2000after,harrison1992art}. One prominent feature of contemporary art is the absence of fixed standards or genres historically defined by the church or the academy. Postmodern design neither defines a cohesive set of aesthetic values nor restricts the range of media used~\cite{pooke2021art}, sparking novel practices such as installations, performances, lens-based media, videos, and land-based art~\cite{hopkins2000after}.
Meanwhile, artists have increasingly drawn energy from various philosophical and critical theories such as gender studies, psychoanalysis, Marxism, and post-structuralism~\cite{pooke2021art}. As a result, as described by Foster~\cite{foster1999recodings}, artists have increasingly become ``manipulators of signs and symbols... and the viewer an active reader of messages rather than a passive contemplator of the aesthetic''. Hopkins~\cite{hopkins2000after} described this shift as the ``death of the object'' and ``the move to conceptualism''. 
% Joseph Kosuth, one of the most important representatives of conceptual artists, also once said that “all art (after Duchamp) is conceptual (in nature) because art only exists conceptually”
% As argued by Danto~\cite{danto2015after}, traditional notions of aesthetics can no longer apply to contemporary art. ``“All there is at the end,” Danto wrote, “is theory, art having finally become vaporized in a dazzle of pure thought about itself, and remaining, as it were, solely as the object of its own theoretical consciousness.''
% The Anti-aesthetic (1983) has described these as ‘anti-aesthetic’ strategies – typified, as we have seen, by a conceptually driven approach to the art object and to the process of its production.

Emerging within the contemporary art historical context, data art has been significantly propelled by the advent of big data. An early milestone was Kynaston McShine's 1970 exhibition \textit{Information} at the Museum of Modern Art (MoMA). 
% In the exhibition catalogue, McShine wrote~\cite{information_moma}: ``Increasingly artists use mail, telegrams, telex machines, etc., for transmission of works themselves—photographs, films, documents—or of information about their activity.'' 
% The millennium era has witnessed substantial growth in this area.
In 2008, Google’s Data Arts Team was founded to explore the advancement of what creativity and technology can do together~\cite{google}.
% data artist Aaron Koblin
In 2012, Viégas and Wattenberg's \textit{Wind Map}, an artwork that visualizes real-time air movement, became the first web-based artwork to be included in MoMA's permanent collection~\cite{wind}.
Since 2013, the academic conference IEEE VIS has included an Arts Program (IEEE VISAP), showcasing artistic data visualizations through accepted papers and curated exhibitions. 
As noted by Barabási~\cite{dataism} (a Fellow of the American Physical Society and the head of a data art lab), data has become a vital medium for artists to deal with the complexities of our society: ``Humanity is facing a complexity explosion. We are confronted with too much data for any of us to make sense of...The traditional tools and mediums of art, be they canvas or chisel, are woefully inadequate for this task...today’s and tomorrow’s artists can embrace new tools and mediums that scale to the challenge, ensuring that their practice can continue to reflect our changing epistemology.''
% a physicist and head of a data art lab, has noted, 

% Artists are now exploring new mediums and methods, incorporating datasets, computer technology, and algorithms into their work.



\subsection{Research on Artistic Data Visualization}
\label{ssec:artisticvis}

Artistic data visualization is also referred to as artistic visualization, data art, or information art~\cite{holmquist2003informative,rodgers2011exploring,few,viegas2007artistic}. Despite the varying terminologies, there is a basic consensus that artistic data visualization must be art practice grounded in real data~\cite{viegas2007artistic}.
Since the early 2000s, a series of papers introduced innovative artistic systems for visualizing everyday data, such as museum visit routes and bus schedule information~\cite{skog2003between,holmquist2003informative,viegas2004artifacts}.
In 2007, Viégas and Wattenberg~\cite{viegas2007artistic} explicitly proposed the concept of \textit{artistic data visualization} and viewed it as a promising domain beyond visual analytics.
% and defined it as ``visualization of data done by artists with the intent of making art''. 
Kosara~\cite{kosara2007visualization} drew a spectrum of visualization design, positioning artistic visualization and pragmatic visualization at opposite ends of this spectrum to demonstrate that the goals of these two types of design often diverge. 
% advocating that analytical visualizations prioritize practicality, while artistic data visualizations emphasize sublime quality, that is, the capacity to inspire awe and grandeur and elicit profound emotional or intellectual responses. 
% In 2011, Rodgers and Bartram~\cite{rodgers2011exploring} utilized artistic data visualization to enhance residential energy use feedback. 
However, overall, research on this subject has been sparse. Among those relevant papers, most have focused on specific applications of artistic data visualization. 
%~\cite{rodgers2011exploring,schroeder2015visualization,perovich2020chemicals}
For instance, Rodgers and Bartram~\cite{rodgers2011exploring} utilized ambient artistic data visualization to enhance residential energy use feedback. Samsel~\etal~\cite{samsel2018art} transferred artistic styles from paintings into scientific visualization.
Artistic practice has also been observed in fields such as data physicalization~\cite{hornecker2023design,perovich2020chemicals,offenhuber2019data} and sonification~\cite{enge2024open}. For example, Hornecker~\etal~\cite{hornecker2023design} found that many artists are practicing transforming data into tangible artifacts or installations. Enge~\etal~\cite{enge2024open} discussed a set of representative artistic cases that combine sonification and visualization.
% dragicevic2020data
% Offenhuber~\cite{offenhuber2019data} created a set of artworks in urban settings that transform air quality data into situated displays, allowing people to encounter visualizations in their daily lives.

% But in contrast, empirical studies that describe the characteristics of artistic visualization and how they are created are extremely scarce. This scarcity forms a stark contrast to the increasingly rich and diverse practices by artists in the field.
% As for the difference between artistic data visualization and traditional visualizations for analytics, Vi{\'e}gas and Wattenberg~\cite{viegas2007artistic} thought that the most salient feature of artistic data visualizations is their forceful expression of viewpoints.
% In Ramirez~\cite{ramirez2008information}'s opinion, functional information visualizations are concerned with usability and performance while aesthetic information visualizations are concerned with visually attractive forms of representation design.
% Donath~\etal~\cite{donath2010data} presented a series of tools developed to integrate artistic expressions in generating unique and customized visualizations based on users' personal data, such as health monitoring data, album records, and e-mail records. 

On the other hand, some studies, while not focusing on artistic data visualization, have explored a key art-related concept: aesthetics. 
% ~\cite{moere2012evaluating,cawthon2007effect,lau2007towards} explored the aesthetics of visualization design in their research. They
For example, Moere~\etal~\cite{moere2012evaluating} compared analytical, magazine, and artistic visualization styles, noting that analytical styles enhance the discovery of analytical insights, while the other two induce more meaning-based insights. Cawthon~\etal~\cite{cawthon2007effect} asked participants to rank eleven visualization types on an aesthetic scale from ``ugly'' to ``beautiful'', finding that some visualizations (\eg sunburst) were perceived as more beautiful than others (\eg beam trees).
% Moere~\etal~\cite{moere2012evaluating} displayed data in three different styles (analytical style, magazine style, artistic style) and found that these styles led to different perceptions of usability and types of insights.
% More importantly, the authors found that the sunburst chart ranks the highest in aesthetics and is one of the top-performing visualizations in both efficiency and effectiveness, thus exemplifying the notion that "beautiful is indeed usable".
Factors such as embellishment~\cite{bateman2010useful}, colorfulness~\cite{harrison2015infographic}, and interaction~\cite{stoll2024investigating} have also been found to influence aesthetics. 
% borkin2013makes,tanahashi2012design
However, most of these studies have simplified aesthetics to hedonic features (\eg beauty), without delving into the nuanced connotations of aesthetics.
% most of these studies have simplified aesthetics to concepts like 'beauty,' 'preference,' or 'pleasing,' without exploring the deeper essence of aesthetics as the core of art.

The value of artistic data visualization to the visualization community is still in controversy. For instance, Few~\cite{few} argued for a clearer distinction between data art and data visualization; he highlighted the negative consequences when data art ``masquerades as data visualization'', such as making visualizations difficult to perceive. Willers~\cite{willers2014show} thought the artistic approach is ``unlikely be appreciated if the intention was for rapid decision making.''
% In an interview, American artist and technologist Harris commented that ``data can be made pretty by design, but this is a superficial prettiness, like a boring woman wearing too much makeup.''~\cite{harris2015beauty} 
To address these gaps, more empirical research needs to be conducted to explore how artistic data visualizations are designed, their underlying pursuits, and how they might inspire our community.




% Examining this field not only helps us understand the latest application of data visualization in various domains but also extends our understanding of the aesthetic and humanistic aspects of data visualization.
% there should be more theoretical investigation into artistic data visualization. 

% These three concepts emphasize placing or installing visualizations at physical places that people will encounter in their daily lives. 

% ~\cite{wang2019emotional}


% gap between art and science~\cite{judelman2004aesthetics}
% constructive visualization~\cite{huron2014constructive}
% data feminism~\cite{d2020data}
% critical infovis~\cite{dork2013critical}
% citizen data and participation~\cite{valkanova2015public}

% \x{Lee~\etal~\cite{lee2013sketchstory}, give users artistic freedom to create their own visualizations.}


% Aesthetics refers to the study of beauty, taste, and sensory perception and is deeply intertwined with art.
% a particular taste for or approach to what is pleasing to the senses and especially sight

% why shouldn't all charts be scatter plot~\cite{bertini2020shouldn}
% aesthetic model~\cite{lau2007towards}
% Aesthetics for Communicative Visualization : a Brief Review
% Stacked graphs--geometry \& aesthetics~\cite{byron2008stacked}
% storyline optimization~\cite{tanahashi2012design}
% graphic designers rate the attractiveness of non-standard and pictorial visualizations higher than standard and abstract ones, whereas the opposite is true for laypeople.~\cite{quispel2014would}
% evaluate aesthetics - wordcloud
% An Evaluation of Semantically Grouped Word Cloud Designs, tag cloud, wordle

% On the other hand, empirical studies conducted with designers have shown that functionality is not the only design goal of visualization. For example, Bigelow~\etal~\cite{bigelow2014reflections} found that designers would frequently fine-tune the non-data elements in their designs, and data encoding was even "a later consideration with respect to other visual elements of the infographic".
% Moere~\cite{moere2011role} - design
% Quispel~\etal~\cite{quispel2018aesthetics} found that for information designers, clarity and aesthetics are both important for making a design attractive.