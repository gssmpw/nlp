\section{Design Features of Artistic Data Visualization}
\label{sec:space}

In this section, we analyzed \ncorpus artworks to outline a bigger picture of artistic data visualizations' design features. 

\subsection{Corpus Collection}

\begin{figure*}[b]
 \centering
 \includegraphics[width=0.94\textwidth]{figures/techniques.pdf}
 \caption{Left: All identified design techniques and their frequencies. Right: Examples of the artworks. (A) Agitato~\cite{agitato}, (B) Applying color palettes in paintings to scientific visualization~\cite{samsel2018art}, (C) Shan Shui in the World~\cite{shi2016shan}, (D) Bitter Data~\cite{li2023bitter}, (E) Decoding • Encoding~\cite{tibetan}, (F) Messa di Voce~\cite{messadivoce}, (G) Bion~\cite{bion}, (H) NeuroKnitting Beethoven~\cite{neuro}, (I) Climate Prisms~\cite{prisms}, (J) Oceanforestair~\cite{oceanforestair}, (K) Art of the March~\cite{protest}, (L) Decomposition of Human Portraits~\cite{face}, (M) \#home~\cite{home}, (N) The Sky is Falling~\cite{sky}, (O) Beyond Human Perception~\cite{plants}.}
 \label{fig:techniques}
 \vspace{-2em}
 % \Description{Left: All identified design techniques in artistic data visualizations and their frequencies. The techniques are organized into four categories: sensation, interaction, narrative, and physicality. Right: 15 images illustrating examples of the artworks.}
 % \vspace{-1em}

\end{figure*}

We began by collecting artworks from the IEEE VISAP (VIS Arts Program), which is a program associated with the top-tier visualization conference IEEE VIS. 
IEEE VISAP was started in 2013 and has run for more than ten years. 
% Artworks should be submitted along with textual explanations and will undergo a review process for acceptance.
Over the years, IEEE VISAP has become the leading event for artistic data visualization and has formed a representative collection of artworks.
% had a significant impact on the field of visualization by fostering cross-disciplinary collaboration, sparking new ideas, and expanding the horizons of creative data representation. 
We scraped all the accepted artworks of IEEE VISAP from its official website, resulting in a total of 231 non-duplicate artworks. After excluding 36 artworks that are now inaccessible on the web, we kept 195 artworks published between 2013 and 2023.
Overall, we believe that IEEE VISAP is a good source because: (i) It offers a publicly accessible dataset dedicated to artistic data visualization. (ii) Its review process ensures the artworks' relevance to data visualization. (iii) The inclusion of textual explanations or short papers by the artists provides essential materials for our analysis.
% However, we also realized that relying solely on IEEE VISAP has its limitations, as there may be some great artworks published outside this venue. 
We also searched venues such as the Info+ Conference, Tapestry Conference, EyeO Festival, and Information is Beautiful. However, although these venues feature active data artists, they either lack a dedicated track explicitly for artistic data visualization or do not include designers' explanatory materials.
In the end, we opted to use a snowballing technique to identify literature that introduces the practices of artistic data visualization and to include as many data artworks as possible.
% (\eg ~\cite{de2022data,kim2013topics,moere2012evaluating,judelman2004aesthetics,kim2010speculative,gough2014affective,khot2017edipulse})
As a result, we added 25 artworks to our corpus, bringing the total to \ncorpus artworks. The additional artworks were mostly published in non-VIS tracks.
% not an artwork (\eg three records are keynote talks and 14 records are theoretical papers). 
% However, we were unable to find the submitted materials of 19 artworks. 
% difference of these channels, clarify
% 4 keynotes



% many school info is missing
The publication time of these artworks spans from 1996 to 2023 (\includegraphics[align=b,scale=0.13]{figures/span.png}).
% The subjects of the collected artistic data visualizations cover a wide range of topics, such as environmental problems, urban studies, social issues, biology, computer science, sports, and news media.
% \textit{environmental sciences and ecology} (26), \textit{urban studies} (22), and \textit{social issues} (21) are addressed by most artworks. For example, Fig. \ref{fig:techniques} (C) is about the pollution of the ocean, and Fig. \ref{fig:techniques} (K) visualizes the data of carbon dioxide emission. Fig. \ref{fig:techniques} (G) recreates the landscape of New York. Fig. \ref{fig:techniques} (D) deals with the problem of sexual harassment in academia. Fig. \ref{fig:techniques} (J) is an artwork that visualizes deaths caused by drone strikes.
% Besides, there are four more fields that have been addressed by more than ten artworks in our corpus, including \textit{biology} (14), \textit{computer science} (10), \textit{art} (10) and \textit{news \& media} (10). For example, Fig. \ref{fig:techniques} (H) visualizes what machines see during a face recognition task.
% water pipelines in the US.
% Fig. \ref{fig:techniques} visualizes the music listening experience. Fig. \ref{fig:techniques} xxx social media.
% A small number of artworks deal with subjects in other fields such as language \& literature (8), astronomy (4), physics (4), and politics (2). 27 artworks did not specify the field in which they will be applied.
% In summary, the artworks in our corpus cover both humanities and scientific subjects and a substantial number of the artworks exhibit a strong sense of realistic issues and public engagement.
We in total identified 516 distinct authors of these artworks coming from 188 different affiliations, and their backgrounds are also diverse (see \autoref{tab:authors} for more details).
% \x{such as \textit{the University of California, Santa Barbara} (23), \textit{Massachusetts Institute of Technology} (21), and \textit{Northeastern University} (21). 193 authors (37.40\%) have an art/design background, such as \textit{Design} (51), \textit{Art and Design} (25), \textit{Art/Arts} (17), and \textit{Fine Arts} (16).  132 authors (25.58\%) come from engineering backgrounds, such as \textit{Computer Science} (70) and \textit{Computer Science and Engineering} (14). 
% For example, 12 authors come from the Department of Computer Science at the University of Calgary. 82 authors (15.89\%) come from cross-disciplinary departments such as \textit{Media Arts and Technology} (22), \textit{Urban Studies and Planning} (14), and \textit{Interactive Arts and Technology} (11).
% \w{For example, 16 authors from the Department of Media Arts and Technology at UC Santa Barbara contributed to the creation of 14 artworks} 66 authors (12.79\%) come from other disciplines such as \textit{Physics} (5), \textit{Information} (4), \textit{English Literature} (3), and \textit{Psychology} (2). the backgrounds of 43 authors (8.33\%) are unknown.}
Each artwork has an average of 2.38 authors. Among the \ncorpus works, 85 have only one author. Of the remaining 135 collaborative works, 87 (64.44\%) were created by cross-disciplinary teams.
Another interesting finding is that apart from reporting their official occupations (\eg professor, PhD student), 115 authors used personalized labels to define themselves, such as ``multimedia artist and roboticist'', ``an intermedia artist and an acknowledged pioneer'', ``artist, technologist'', and ``musician''.
% Catherine D’Ignazio~\cite{} defined herself as "a hacker mama, scholar, and artist/designer who focuses on feminist technology, data literacy and civic engagement". 
17 authors simultaneously hold academic and industry positions but place their industry identity ahead of their academic identity (\eg ``multidisciplinary digital media artist and university professor", "media artist and researcher'') showing that their role as practitioners is highly valued.


\begin{table}[h]
\centering
\vspace{-0.5em}
\caption{Affiliations and backgrounds of the authors from the corpus.}
\vspace{-1em}
% \small
\begin{subtable}[t]{0.56\linewidth} % 左表
% \caption{Affiliations of Authors}
\fontsize{6.8pt}{7pt}\selectfont
\centering
\begin{tabular}{@{}p{0.8\linewidth}p{0.12\linewidth}@{}}
\toprule
\textbf{Affiliation (Top 5)}                     & \textbf{Num} \\
\midrule
University of California, Santa Barbara & 23 \\
Massachusetts Institute of Technology   & 21 \\
Northeastern University                 & 21 \\
University of Texas, Austin             & 19 \\
University of Calgary                   & 14 \\
\bottomrule
\end{tabular}
\end{subtable}%
\hfill
\begin{subtable}[t]{0.41\linewidth} % 右表
% \caption{Backgrounds of Authors}
\fontsize{6.5pt}{7pt}\selectfont
\centering
% \small
\begin{tabular}{@{}p{0.5\linewidth}p{0.4\linewidth}@{}}
\toprule
\textbf{Background}                     & \textbf{Num} \\

\midrule
Art/Design        & 193 (37.40\%) \\
Engineering        & 132 (25.58\%) \\
Cross-Disciplinary & 82 (15.89\%) \\
Other Disciplines              & 66 (12.79\%) \\
Unknown            & 43 (8.33\%) \\
\bottomrule
\end{tabular}
\end{subtable}
\vspace{-2em}
\label{tab:authors}
\end{table}




% \begin{table}[h!]
% \caption{Author Information}
% \centering
% % \small

% \begin{subtable}[t]{0.48\linewidth} % 左表
% \caption{Affiliations of Authors}
% \fontsize{6pt}{6pt}\selectfont
% \centering
% \begin{tabular}{@{}p{0.79\linewidth}p{0.13\linewidth}@{}}
% \toprule
% \textbf{Affiliation}                     & \textbf{Num} \\
% \midrule
% University of California, Santa Barbara & 23 \\
% Massachusetts Institute of Technology   & 21 \\
% Northeastern University                 & 21 \\
% University of Texas, Austin             & 19 \\
% University of Calgary                   & 14 \\
% \bottomrule
% \end{tabular}

% \label{tab:institutions}
% \end{subtable}%
% \hfill
% \begin{subtable}[t]{0.48\linewidth} % 右表
% \caption{Backgrounds of Authors}
% \fontsize{6pt}{6pt}\selectfont
% \centering
% % \small
% \begin{tabular}{@{}p{0.45\linewidth}p{0.45\linewidth}@{}}
% \toprule
% \textbf{Background}                     & \textbf{Num} \\

% \midrule
% Art/Design        & 193 (37.40\%) \\
% Engineering        & 132 (25.58\%) \\
% Cross-Disciplinary & 82 (15.89\%) \\
% Other Disciplines              & 66 (12.79\%) \\
% Unknown            & 43 (8.33\%) \\
% \bottomrule
% \end{tabular}

% \label{tab:backgrounds}
% \end{subtable}
% \end{table}

% \renewcommand{\topfraction}{0.9}  % 最大允许占用页面顶部的比例
% \renewcommand{\textfraction}{0.1}  % 页面下方至少留出的文本比例



\subsection{Design Analysis}

% In this section, we perform design analysis on the corpus.





\subsubsection{Analysis Method}


We analyzed the design features of the artworks using open coding while referring to the artists' own explanations.
Two authors were responsible for the coding process, focusing on two main aspects~\cite{shi2021communicating,sarikaya2018we}: (i) What are the design intents of the artworks? and (ii) How are they designed? First, we familiarized ourselves with the common naming and categorization of visualization design intents and techniques proposed in prior design taxonomies (\eg ~\cite{lan2023affective}).
Then, we went through the artworks independently and generated codes to describe their design intents and techniques. Similar codes were categorized into groups. For example, we identified multiple channels of designing interaction, such as \textit{GUI interaction}, \textit{facial interaction}, and \textit{voice interaction}. Thus, these codes were grouped into a category called \textit{interaction}.
Next, we met to compare codes, discuss mismatches, and adjust inappropriate codes. For example, we initially used multiple codes to describe various materials used in the artworks, such as \textit{plastic}, \textit{metal}, \textit{yarn}, and \textit{glass}. During our discussion, we found that these codes were too detailed and difficult to enumerate exhaustively. Therefore, we consolidated them into a single code called \textit{physical materials}. Additionally, we observed that some artists (though not all) used higher-level terms to describe the fundamental concepts or philosophies behind their artworks. We agreed that this information is valuable and complements low-level design techniques. As a result, we added a new dimension called \textit{design paradigm} to our codebook and proceeded with another round of coding. In other words, our final taxonomy contains three main dimensions: design paradigms, design intents, and design techniques. After four rounds of meetings and coding, we achieved 100\% agreement on the coding scheme.
% To sum up, we coded the xxx data artworks in four dimensions: \textit{design paradigm} and \textit{design task} capture design intents at high-level and low-level perspectives, respectively; \textit{design genre} and \textit{design techniques} describe design outcomes from high-level and low-level perspectives, respectively.


\subsubsection{Design Paradigms}
\label{sssec:paradigm}
We identified 37 explicit mentions of high-level design paradigms. The most frequently mentioned paradigm is \textit{generative art} (N = 12), which refers to art created using autonomous systems or algorithms.
% Artists write codes to initiate the creation process, and the final outcome is partly unpredictable and influenced by chance or randomness.
Other mentions include concepts such as \textit{abstract art} (2), \textit{physical computing} (2), \textit{algorithmic art} (2), \textit{computational aesthetics} (2), \textit{digital art} (2), and \textit{digital fabrication} (2).
%\textit{ambient design} (2), \textit{computer art} (2), \textit{computer-generated art} (1), \textit{electronic art} (1), \textit{parametric design} (1), \textit{VR art} (1), \textit{wearable art} (1), and \textit{glitch art} (1). 
% Last, three design paradigms are \textbf{modality-centered}, including \textit{sonification} (11), \textit{physicalization} (4).
% Although they have subtle differences, the core emphasis is on using computer programs to drive artistic creation. 
For example, parametric design generates artworks through a series of parameters. Designers can flexibly adjust and control the parameters to explore a range of design variations. Glitch art is a paradigm that celebrates errors in computer programs and embraces the aesthetics of imperfections.
A common feature of the above paradigms is their emphasis on the role of computers.

We also identified another set of paradigms that are more human-centered, such as \textit{speculative art} (2), \textit{digital humanism} (2), \textit{participatory design} (1), \textit{aministic design} (1), and \textit{neo-concrete art} (1). For instance, speculative art explores alternative realities and futures, often through experimental practices within communities. Participatory design involves people actively in creating artwork to promote social inclusion, empower marginalized communities, and reflect their needs and values. 
When contextualized within art history~\cite{pooke2021art,hopkins2000after,perry2004themes}, nearly all the aforementioned paradigms fit the scope and trends of contemporary art.


% Moreover, concepts such as , \textit{glitch art}, \textit{speculative art}, \textit{wearable art}, \textit{neuroaesthetics}, \textit{digital humanism}, \textit{digital photograph} have also been mentioned, suggesting the deep interwoven of artistic data visualization with other subdomains of art as well as disciplines such as psychology, computer science, philosophy, and social science.







\subsubsection{Design Intents}
\label{sssec:intents}

We initially coded the design intents of artistic data visualizations based on Lan~\etal~\cite{lan2023affective}'s work, which identified ten types of design intents (\eg \textit{inform}, \textit{engage}, \textit{experiment}) in affective visualization design.
As a result, although all the ten intents are present in our corpus, we also identified five new intents (marked using * in the following text).
In terms of frequency, the most common intent is \textit{experiment} (N = 49).
The core spirit of this intent is to challenge traditional notions of data visualization and explore unconventional or novel representations.
Other design intents include \textit{inform} (39), \textit{engage} (36), \textit{re-present*} (36), \textit{provoke} (30), \textit{criticize*} (19),  \textit{equip*} (13), \textit{analyze*} (12), \textit{advocate} (10), \textit{socialize} (7), \textit{witness*} (6), \textit{archive} (5), \textit{commemorate} (3), and \textit{empower} (3). 
% For example, Fig. \ref{fig:techniques} (I) explores encoding data with typography. Fig. \ref{fig:techniques} (F) intends to provoke thinking about human-nature relationship. Fig. \ref{fig:techniques} (D) brings uneasy experiences to the foreground and empowers victims of sexual harassment in academia.
For example, 
%\autoref{fig:techniques} (\w{E}) is a work meant to \textit{criticize}. The artist stated: ``Water is often depicted as blue and beautiful, and the project problematizes this metaphor by delving into the human relationship to water and how we impact the water we need to live''. 
\autoref{fig:techniques} (A) \textit{re-presents} the experience of listening to music by transforming the artist's subtle emotions into animated metaphorical shapes. We use the term \textit{re-present} instead of \textit{represent} to highlight the artist's intent to capture and reveal something invisible while integrating their own interpretations.
\autoref{fig:techniques} (N) exemplifies the newly identified intent, \textit{witness}. To show the number of civilians killed by US drone strikes, the artist performed in a desert from dawn to dusk, commemorating each civilian with an earth mound, a white cloth, a stone, and a prayer. Viewers witnessed the ceremony through live streams.
% The third one is to \textit{analyze} (8). For example, Fig. \ref{fig:techniques} (B) applied the idea of glitch art to the design of node-link diagram and proposed an algorithm to facilitate the diagnosis of the ``glitched'' neurons in the brain.  
% The fourth one is to \textit{witness} (4). For example, Fig. \ref{fig:techniques} (J) ``invites her audiences to serve as witnesses and aids.''
As another example, the intent of \autoref{fig:techniques} (B) is mainly to \textit{equip} artists with a tool for coloring scientific visualizations using expressive palettes extracted from paintings. This also resonates with the case of Sandin (\autoref{ssection:particle}) and previous findings that artists sometimes write software themselves or participate in software development~\cite{li2021we}.
% simulate, represent, recreate, reveal


% \subsubsection{Design Genres}

% Among the 178 data artworks we collected, \textit{installations} make up the largest part (54), followed by \textit{interactive interfaces} (52), \textit{static images/paintings} (28), \textit{artifact} (21), \textit{videos} (18), and \textit{events} (3). 
% For example, Fig. \ref{fig:techniques} (C) is a large installation placed in a museum.
% Fig. \ref{fig:techniques} (D) is a website where users can browse and interact with the testimony of the victims of sexual harassment.
% Fig. \ref{fig:techniques} (F) visualizes the change of sky as a static picture.
% Fig. \ref{fig:techniques} (K) uses fiber material to show climate data.
% Apart from these genres, we identified another new genre, namely the \textit{performance} where the artists perform as actors or actresses on stages or in outer environments. 
% For example, Fig. \ref{fig:techniques} (J) is a performance where viewers watched the artist placing stones on the ground, each representing a person killed by drone strikes.
% Besides, 12 works claimed algorithms/toolkits as their major contribution so that the output forms of these works are flexible. 
% For example, the algorithm proposed by Fig. \ref{fig:techniques} (B) can be used to generate both static and dynamic visualizations.



\subsubsection{Design Techniques}
\label{sssec:techniques}
A total of 32 design techniques were identified. The 32 design techniques were further grouped into four categories: \textit{sensation}, \textit{narrative}, \textit{interaction}, and \textit{physicality} (see \autoref{fig:techniques}).

\textbf{Sensation}. This category primarily focuses on creating various visual, auditory, olfactory, and other sensory effects. Relevant techniques include the use of \textit{color}, \textit{imagery}, \textit{shape/pattern}, \textit{sound/music}, \textit{animation}, \textit{style}, \textit{light}, \textit{layout}, \textit{typography}, \textit{temperature}, \textit{smell}, and \textit{taste}. 
For example, \autoref{fig:techniques} (A) transformed the evolving music listening experience into a generative artwork composed of vibrant colors, organic shapes, and animation.
% In \autoref{fig:techniques} \x{(B)}~\cite{microbiome}, the artist used metagenomics to generate colorful self-portraits of the microbial communities inhabiting his body. 
% \autoref{fig:techniques} \w{(C)} is an interactive installation designed to engage users in Chinese calligraphy. The written characters were transformed into bamboo leaf-styled fonts. Animation was generated to sway the characters gently, reminiscent of bamboo leaves rustling. The music and colors were also carefully selected to help create an ancient atmosphere.
\autoref{fig:techniques} (B) applied color palettes extracted from paintings to scientific visualizations.
% redesigned node-link diagrams, using a glitch-style encoding to represent anomalies.
\autoref{fig:techniques} (C) transformed the map of New York City into a style reminiscent of traditional Chinese painting.
\autoref{fig:techniques} (D)~\cite{li2023bitter} transforms 100,000 distress postings using data edibilization. The data was mapped to the bitterness and color of tea, enabling users to observe, smell, and taste the distress on social media.
% Fig. \ref{fig:techniques} \x{()} transferred the style of ancient Chinese paintings to the map of New York. Fig. \ref{fig:techniques} \x{()} used light to xxx. 
% Fig. \ref{fig:techniques} \x{()} animation of water.

% According to our analysis, the idea of \textit{sonification} has been mentioned most (explicitly mentioned by ten artworks). The core spirit of sonification is to use sound or auditory elements to translate non-auditory information into audible forms. On the one hand, sound is viewed as a novel medium that expands the possibilities of artistic expression. On the other hand, the idea of sonification is also fueled by the need for accessibility in design as it can make information more accessible to individuals with visual impairments or disabilities.



\textbf{Interaction}. 
Among interactive techniques, GUI interaction, such as allowing users to click on or scroll a website, mobile app, or tablet, is most common (\eg \autoref{fig:techniques} (E)). 
Other relevant techniques include \textit{body interaction}, \textit{facial interaction}, \textit{touch/manipulate}, \textit{gesture interaction}, \textit{biomimicry}, 
\textit{AR/VR/MR}, \textit{voice interaction},  \textit{draw/sketch interaction}, and \textit{physiological interaction}. 
For example, \autoref{fig:techniques} (F) is a performance driven by voice interaction. The performers made various sounds and the sounds were transformed into the real-time animated visualization in the background. 
% designed to provoke questions about the meaning and effects of speech sounds, speech acts, and the immersive environment of language.
\autoref{fig:techniques} (G) is composed of hundreds of ``bions'' (individual elements of primordial biological energy) programmed according to biomimicry sensing. When a viewer approaches the installation, bions quickly communicate to each other, but eventually they become accustomed to the stranger's presence and respond as if he/she is part of their ecosystem.
\autoref{fig:techniques} (H) is a neuroknitting artwork that utilizes people's brainwave data when listening to music to drive the knitting machine.
% In \autoref{fig:techniques} \x{(E)}~\cite{endless}, users can immerse themselves in a simulated 3D fluid environment through VR.


\textbf{Narrative}. This category encompasses storytelling methods such as \textit{metaphor/analogy}, \textit{story plot}, \textit{collage}, \textit{perspective shift}, \textit{co-design/user generated content}, and \textit{personalization}. 
For example, 
\autoref{fig:techniques} (I) uses yarn to represent cumulative emissions over years; as the emission gets larger and larger, the yarn gets tighter and tighter, metaphorically mimicking a sense of suffocation.
\autoref{fig:techniques} (J) tells a data story about the shrinking of Arctic ponds by crafting a prepared storyline.
\autoref{fig:techniques} (K) is a website that collages more than 6,000 signs of the Boston Women’s March, which silently demonstrates the collective efforts of protesters. 
% As introduced by the artists~\cite{protest}, "The collection is a rich and inclusive snapshot and record of the extensive range of issues, emotions and visual expressions at the march. The signs are handmade and unique, but also connected in a rich web of cultural references, themes, memes, and visual techniques and styles."
\autoref{fig:techniques} (L) employs a perspective shift to visualize human faces from an unconventional perspective—the eye of a deep neural network.
\autoref{fig:techniques} (M) invited visitors to share keywords about their homes. The keywords were used to filter a live Twitter stream, and the locations of these tweets were 3D printed as physical maps, which were personalized to each visitor.
% Similarly, Fig. \ref{fig:techniques} } visualizes images of human faces seen by computer vision algorithms. 
% Fig. \ref{fig:techniques} \x{()} is a visualization about cancer genomes. The artists invited people who had this cancer to participate in the design process and draw their own genomes, thus partly relinquishing the narrative right to users.

% used the metaphor of knots to represent the uneasy stories about harassment. Fig. \ref{fig:techniques} (F) creates a montage of photographs to break the linearity of time and narrative. 


\textbf{Physicality}. This category contains techniques that augment physical experiences, including \textit{physical materials} (\eg yarn, metal), \textit{large immersive screen}, \textit{situated environment}, 
\textit{non-human agents (\eg \textit{plants}, \textit{robots}, \textit{bacteria}).}
For example, the two aforementioned artworks, \autoref{fig:techniques} (I, M), use yarn and 3D printing to physicalize data, respectively.
% \autoref{fig:techniques} \x{(K)}~\cite{hongkong} combines city map visualization with the cultivation of bacteria to reflect on the expansion of Hong Kong through the lens of microorganisms.
\autoref{fig:techniques} (N), as introduced earlier, was an art performance conducted at a situated location, a desert, to provide viewers with an authentic sense of the environment affected by drone strikes.
Lastly, several artworks also utilize non-human agents, including entities that perform tasks or functions typically associated with humans. For example, \autoref{fig:techniques} (O), for example, played music to plants; sensor data from the plants was collected to explore the connection between technology and plant life.
% In Fig. \ref{fig:techniques} \x{()}, music was played to plants, and sensor data was collected from the plants. This exploration delves into the interaction between sound and plant life, offering insights into the connection between nature and technology.

\subsection{Observations}

Although the design of artistic data visualization shares some commonalities with other realms (\eg techniques like metaphor, story plot, personalization, and GUI interaction have also been found in narrative visualization~\cite{segel2010narrative,shi2021communicating,lan2022negative}; the appeal to sensation is also common in affective visualization design~\cite{lan2023affective}), it also exhibit distinct features.
First, artistic data visualization can be fundamentally shaped by high-level design paradigms. These paradigms act as the creative lenses that artists use to interpret the world around them, serving as the cornerstone of artistic decision-making. Yet, they are seldom identified in general visualization design research. This also demonstrates artists' emphasis on ideas and concepts.
% Although only a subset of data artworks in our corpus explicitly articulated their design paradigms with specific concepts, it is common for artists to explain their artistic inspiration and the thoughts behind their work. For instance, when explaining why they cultivated bacteria to ``grow'' a map of Hong Kong, the artists emphasized: ``We present a posthumanist idea of defining mingling spaces with microorganisms.'' 
Second, although artistic data visualization shares some design intents with prior studies, it exhibits more categories and different distributions in frequency of these intents. On one hand, the exploration of novel forms of expression is highly valued by data artists, as evidenced by the highest frequency of \textit{experiment}, a distinctive aspect that sets it apart from other fields. Additionally, within the design intents of artistic data visualization, those that convey strong opinions (\eg \textit{advocate}) are less prevalent. In contrast, higher-ranking intents (\eg \textit{inform}, \textit{engage}, \textit{provoke}) tend to be more implicit, subtle, and thought-provoking. This characteristic aligns well with the traits of contemporary art, which allows the audience significant freedom to interpret the work on their own terms~\cite{pooke2021art}.
Last, regarding specific design techniques, artistic data visualization places a strong emphasis on sensory richness and physicality, with many artworks taking the form of physical installations or artifacts (which is very different from the visualization design genres identified before~\cite{segel2010narrative}). Among current fields, affective visualization design shows the greatest technological affinity with artistic data visualization. However, when examining the distribution and frequency of specific techniques, interactive technologies in artistic data visualization are more prominent, manifested through various body, face, and gesture-based interactions, as well as biomimetic methods. Meanwhile, the use of physical materials is more diverse and pronounced, incorporating a wider range of materials to encode data, such as metals, furs, food, and even microorganisms.






