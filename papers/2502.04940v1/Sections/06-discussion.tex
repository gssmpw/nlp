\section{Lessons for Future Work}
% What We Can Learn from Artistic Data Visualization: 
\label{sec:discuss}
% 学习:What We Can Learn From Visual Artists About Sofware Development

In this work, we adopt a multi-faceted approach to explore artistic data visualization. Our literature review, corpus analysis of artistic data visualizations, and interviews with data artists collectively indicate that \textbf{artistic data visualization is deeply rooted in art discourse, with its aesthetic pursuits particularly contextualized within contemporary art philosophies and values. This fundamentally distinguishes artistic data visualization from data science in terms of epistemological foundations and methodological approaches to interpreting and presenting information.} 
Thus, we are prompted to ask: How can data art add value to the visualization community? What are the areas where artists and scientists can learn from and inspire each other?
% How might the extensive practical experiences and successful cases of data artists inspire visualization research, and conversely, how can advancements in visualization research inform and enhance artistic practices?
Below, we organize the lessons we've learned (marked as L1-L7) in three subsections, corresponding to three areas of visualization research. 

\subsection{Theoretical \& Empirical Work}

Firstly, art and science share many dialogic points in concept establishment and problem-solving methods.
% Artistic data visualization is expected to inspire theoretical and empirical research in three main aspects.

\textbf{L1: Broadening the understanding of aesthetics.}
% 丰富对于审美的理解-不只是好看,还有批判的力量
Previously, the visualization community has often operationalized aesthetics as beauty and hedonic features, and specific scales (\eg enjoyable, likable, pleasing) have been developed to measure aesthetic pleasure~\cite{he2022beauvis,stoll2024investigating}. 
% This is understandable, as the visual appeal of visualizations is often a key reason for their use. 
However, taking artistic data visualization as an example, aesthetics can extend far beyond sensory beauty to encompass intellectual levels of critical, deconstructive, or rebellious stances towards established norms. 
% We have observed such tendencies during both our analysis of data artworks and discussions with data artists. 
% For example, the beautiful curves and colors in  \autoref{ssection:smell} are only its outward manifestations. The deeper power of the work lies in the artist's critical perspective on the modern city's alienation of human senses, thus driving her to develop a language for olfactory cartography.
% depict an alienated world through the lens of AI models, 
Besides, as summarized in \autoref{ssec:contemporary}, technological innovation has consistently been a significant driver of aesthetic evolution. Just as the invention of photography pushed art beyond mimesis, the rise of AI has led to challenges such as copyright infringement, economic loss, and the undermining of art's uniqueness and originality~\cite{jiang2023ai}. Such challenges may further compel artists to pursue newer, more rebellious aesthetics. This shift is already being recognized in the art world with terms such as \textit{computational aesthetics}~\cite{galanter2012computational}, and in this work, we have seen artworks that embrace glitches and errors in machines, an appreciation for the randomness inherent in machine learning (\eg \autoref{ssection:machine}). Some works even celebrate anti-beauty: \autoref{fig:techniques} (L) may appear terrifying and disturbing to some viewers, but it is intentionally crafted to visualize the world through the lens of AI models, prompting reflections on our increasingly machine-driven society. In a word, future aesthetics are likely to reflect a broader range of creative expressions and critical perspectives, and simply operationalizing aesthetics as beauty and hedonic features may be insufficient. A deeper and more diverse understanding of aesthetics is needed.
% It is evident that many works have already begun to incorporate AI in the creative process or reflect on its impact. 

% Another art project in our corpus by Claes~\etal drew inspiration from street art to create data visualizations. The essence of street art is highly rebellious and often deliberately embraces the unattractive. They stated that traditional visualization, much like traditional art, is often geared toward ``an elite group of expert users,'' and they learned from street art ``to reach a large, lay audience.''~\cite{claes2017public}

% As said by Grugier\x{~\cite{}}, ``by formatting these reams of data, data artists are not content with making legible the mesh of information from which it is formed, they also take a critical look at our society.''
% As Kelly~\cite{kelly1998encyclopedia} notes in his book \textit{Encyclopedia of Aesthetics}, aesthetics involves "critical reflection on art, culture, and nature." 

\textbf{L2: Enriching evaluation metrics.} 
Following L1, an emerging issue is how to capture the richness and profundity of aesthetics. In more practical terms, we still need to study and explain why certain data artworks are so successful and captivating. Regarding this issue, Shusterman's ~\cite{shusterman2000pragmatist} reflection on \textit{analytical aesthetics} may be helpful. Shusterman argued that \textit{analytical aesthetics}, which treats aesthetics as a set of objective and low-level attributes that can be measured independently, is insufficient. The primary shortcoming of this approach is that it overlooks the complex context of aesthetic experience. Therefore, he called for measuring \textit{pragmatist aesthetics} to better capture the contextual and subjective experience of art. In the context of artistic data visualization, this can include considering viewers' environment when encountering and interacting with data artworks, how the artworks evoke their existing knowledge and values, and their mental journeys. Moreover, as revealed by this work, contrasting with other information displays designed for readability and efficiency, art often enables open-ended interpretation and repeated contemplation. Therefore, it is also crucial to assess long-term user experience beyond capturing immediate responses. To uncover these issues, we view methods such as in-depth interviews, user diaries, longitudinal surveys, and ethnography as valuable complements to traditional research methods.
% Redstrom et al.~\cite{redstrom2000informative} once used the term \textit{slow technology} to describe technology ``that encourages moments of reflection and concentration in order to understand it'', contrasting with other information displays designed for readability and efficiency. 
% Dewey insists that art and the aesthetic cannot be understood without full appreciation of their socio-historical dimensions. 
% represented by \x{Dewey's art philosophy~\cite{}. Dewey stressed that art is not an abstract, autonomously aesthetic notion, but something materially rooted in the real world and significantly structured by its socio-economic and political factors.
% BDLV workshop ?

% realistic tendency, envision更多的fieldwork和empirical practice
\textbf{L3: Incorporating cross-disciplinary methods.}
Data artists have their own ways of problem-solving. Many artists have already explored using data visualization to actively address real-world challenges. Such practices align well with research work that seeks novel approaches and innovative visual solutions to solve domain-specific problems.
First, during the ideation stage, the creative methods employed by artists, such as brainstorming, hackathons, and workshops, may be worth learning from. Kerzner~\etal~\cite{kerzner2018framework}, for example, constructed a set of guidelines for creative visualization-opportunities workshops (CVO) to speed up the early stages of applied visualization design by adapting workshop methods from other disciplines.
Then, in the design phase of visualization, researchers can learn from artists' design approaches, exploring more genres, modes of expression, and user engagement methods. For instance, guided by the idea of participatory design, Perovich~\etal~\cite{perovich2020chemicals} visualized river pollution data as river lanterns and organized an event to engage residents in collective reflection on the environmental issue.
Next, in the deployment phase, the methods artists use to implement visualizations are also inspiring. These include collaborating with communities and stakeholders, as well as deploying visualizations in various offline scenarios and everyday settings (\eg ~\cite{rodgers2011exploring,offenhuber2019data}).
% For example, Rodgers and Bartram~\cite{rodgers2011exploring} installed artistic data visualizations in residents' homes to monitor energy usage and support decision-making in everyday activities. They found that artistic data visualization is ``a promising method of providing real-time residential resource use feedback.''
% integrate artistic approaches into their studies, exploring innovative methods to tackle issues such as environmental problems and social welfare (\eg ~\cite{claes2017public,rodgers2011exploring,offenhuber2019data,duxbury2012breath,ursyn2023transfer,stusak2014activity}).
% Then theorized this novel method as \textit{autographic visualizations} (making invisible data appear as visible traces).
% In a recent paper, artistic representations of data were even found to help bridge the US political divide over climate change~\cite{li2023artistic}.% Stusak~\etal~\cite{stusak2014activity} installed artistic visualizations to monitor residents' energy usage and identified a significant decrease in energy consumption within the community). 
Conversely, artists can also benefit from established methodologies in academia. For example, to bridge the gap between practice and research identified in \autoref{sssec:expectations}, they might adopt the design study methodology from visualization research~\cite{sedlmair2012design}. This approach can help artists clearly define the problems they encounter in their creative process. From ideation to final realization, each step can be better justified, providing a more structured approach to complement their creative instincts. Additionally, they might learn from the criteria for rigor (\eg reflexive, plausible) proposed by visualization researchers~\cite{meyer2019criteria} to create more reliable data-driven works.
% insufficient techniques and expertise in handling data, as noted by some artists, 
% Last, we also encourage researchers to conduct studies that include evaluation sections to demonstrate the effectiveness of artistic methods compared to baseline methods, thereby assessing their effectiveness.



\subsection{Representation \& Interaction}

Secondly, artists and scientists may inform each other when pushing the boundaries of visual and interactive techniques.

\textbf{L4: Learning from artistic design schemes.} 
Borrowing or transferring existing design schemes from artworks to visualization design is an interesting way to explore novel design. 
On the one hand, researchers can consider transferring design schemes across different dimensions (\eg color, texture, metaphor) and granularities (\eg atomic element, overall style). For example, Samsel~\etal~\cite{samsel2018art} focused on extracting color palettes from paintings to encode scientific data. By contrast, Kyprianidis~\etal~\cite{kyprianidis2012state} implemented algorithms to transfer entire artistic painting styles (\eg Impressionism, cartoon styles) to images and videos.
On the other hand, since these design schemes will ultimately be applied to data, the compatibility of the transferred design with the data should be considered. For instance, taking inspiration from Mondrian's abstract art, Holmquist et al.~\cite{skog2003between,holmquist2003informative,redstrom2000informative} redesigned visualizations such as weather forecasts and bus departure information; during the design process, they conducted multiple iterations to ensure readability and understandability. 
% When necessary, some design choices may need to be adjusted or discarded. For example, 
When transferring emotional patterns in motion graphics (\eg bounce, wiggle) to the design of animated charts, Lan et al.~\cite{lan2021kineticharts} conducted multiple rounds of studies to modify distracting animation designs. 
Therefore, while transferring artistic schemes to visualizations is innovative, we recommend conducting rigorous testing and evaluations to confirm its validity.
% Although they do not self-identify as artists~\cite{holmquist2003informative}, they successfully integrated artistic styles into visualization design innovation and discovered that ``basing a visualization on an artistic style need not hinder – and might even support – the readability and comprehension of an ambient infovis installation''.
% Also, in recent years, metaphor has become a key driver of innovative visualization representations, with notable examples including treemaps, theme rivers, and the use of kinetic phenomena from the natural world to animate visualizations (\eg ~\cite{huron2013visual,lan2021kineticharts}). The skillful use of metaphor by data artists can inspire innovation. 
% For instance, drawing inspiration from glitch art, McGraw~\cite{mcgraw2017glitch} utilized the glitch metaphor to depict disruptions in neuronal connectivity and facilitate the study of Parkinson's disease.
% Aseniero~\etal~\cite{aseniero2016fireflies} explored the metaphor of fireflies, designing biomimicry-inspired visualizations to present public survey data.
% Huron~\etal~\cite{huron2013visual} sought inspiration from natural sedimentary processes, introducing a novel design metaphor for the animation of data streams. 


\textbf{L5: Exploring alternative encoding channels.}
In this work, we see data artists exploring sound, touch, smell, and taste, and experimenting with various materials to encode data. These practices resonate well with the trend of expanding data encoding modalities and extending interactions beyond desktops and smartphones in the visualization community. For instance, many artworks have provided vivid and pioneering examples for data physicalization~\cite{hornecker2023design} and sonification~\cite{enge2024open}, enabling people to engage with data visualization across a broader range of channels and daily contexts.
% extract information from the physical environment and 
% This trend also resonates with recent developments in the field of HCI, where researchers are exploring haptic feedback, tangible interfaces, and augmented reality to provide immersive and interactive experiences with data.
Such exploration is also relevant to research on accessible visualization, as it facilitates the study of replacement strategies for visual encodings (\eg how to present visualization to visually impaired individuals). Although data sonification is an emerging approach~\cite{holloway2022infosonics}, designing effective sonification remains challenging. Concurrently, data artists possess extensive experience in designing cross-modal visualizations, which could potentially offer valuable insights into making visualizations more accessible to a broader audience.

% \x{For example, research in fields such as data physicalization and ambient visualization has been inspired by or evolved from artistic practices (\eg ~\cite{skog2003between, pousman2007casual, offenhuber2019data}).
% situated~\cite{bressa2021s}
% physical~\cite{dragicevic2020data, khot2017edipulse, kuznetsov2011red}


\subsection{Applications}

Lastly, for application development, we propose that such systems can be developed either \textit{for} or \textit{with} data artists.

\textbf{L6: Developing tools \textit{for} data artists or users who want to design like artists.}
Tools can be developed to serve the needs and lower the barrier of creating artistic data visualizations. For example, Schroeder~\etal~\cite{schroeder2015visualization} presented an interface to assist artists in creating visualizations through direct painting or sketching on a digital data canvas. Inspired by the data art project, \textit{Dear Data} (a year-long postcard exchange of hand-drawn data visualizations, showcasing the two artists' personal data), tools such as Dataselfie~\cite{kim2019dataselfie} were developed to empower users to easily create their own personal visualizations in hand-drawn or pictorial styles.
% and Dear Pictograph~\cite{romat2020dear}
Additionally, to spark creativity and facilitate ideation, tools can be developed to help users explore, craft, and play with various visualization choices. DataQuilt~\cite{zhang2020dataquilt}, for example, is an interactive tool that allows users to extract elements from raster images and encode data to them, exploring various possibilities of pictogram design.
% a new interactive authoring tool that allows authors to borrow visual and stylistic elements from raster images and re-purpose them to create custom, pictorial visualizations. Left:
% lee2013sketchstory


\textbf{L7: Developing systems \textit{with} data artists.}
In line with prior studies~\cite{judelman2004aesthetics,samsel2013art,li2021we,sandin2006artist}, we found that data artists and scientists/engineers share common pursuits such as exploring new problems and finding appropriate ways to express information; also, a significant number of data artists are highly skilled in programming and interested in software development.
% Samsel~\cite{samsel2013art} also agreed that art and visualization are both exploratory processes as well as avenues for communication. 
Building on these commonalities, the involvement of artists is expected to offer additional perspectives and knowledge for system development. For example, artists' expertise in manipulating visuals and user studies~\cite{tandon2023visual,sandin2006artist} may help enhance the expressiveness and usability of a system.
% informative art - we hope to create ambient information visualizations that literally look “good enough to hang on the wall”, while still providing useful information.~\cite{skog2003between}
% In line with Tandon~\etal~\cite{tandon2023visual}, who found that artists excel more in visual tasks compared to mathematicians/computer scientists, 
% as Petersen~\etal~\cite{petersen2004aesthetic} once argued, ``aesthetics is not only an adhesive making things attractive, and it is part of the foundation for a purposeful system.'' 
Furthermore, as found in both our corpus analysis and interviews, data artists excel at conceptualizing values and meanings, often needing a clear \textit{why} before addressing the \textit{how}. This likely explains why artists were found to be effective at challenging conventional views and offering insights into fundamental questions, such as the necessity and purpose of a system~\cite{laidlaw1998art,pousman2007casual}.
As concluded by Gates~\cite{gates2016art}, ``One extremely valuable role that artists can play in science and with scientists is to ask questions. Even though science is at heart a questioning process...it is easy to lose sight of some of the basic questions that motivate more detailed and specific research.'' 
% In fact, many renowned laboratories, such as Bell Labs, the National Center for Supercomputing Applications, and the \x{EVL} mentioned in \autoref{ssection:particle}, have achieved groundbreaking accomplishments by forming interdisciplinary teams of scientists and artists~\cite{sandin2006artist}.

% Samsel~\cite{samsel2013art} also found that the involvement of artists has surprising impacts in a visualization team. 
% Pousman~\cite{pousman2007casual} found artists can ``bring an unbiased eye'' to the system. 
% As described by Pousman~\etal~\cite{pousman2007casual}, "in the vocabulary of the art world, they 'problematize' our everyday conceptions".

% For example, they are excel at considering the audience and commercial viability of technology~\cite{li2021we}, thus guaranteeing the system aligned with real-world needs. 
% As reflected by Samsel~\cite{samsel2013art}, the involvement of artists often has surprising impacts on the work of visualization specialists and scientists. 

% More formally, Li~\etal~\cite{li2021we} concluded that: ``Like researchers, artists are motivated, in-part, by creating novel outcomes...artists’ joint concerns around expressiveness, audience, and commercial viability, as well as their unique workflows, can inform approaches in end-user software development.''

% oil and water
% artists are very highly trained “eyes” in pattern recognition, and they bring an unbiased eye to biology which can question recurring patterns overlooked by the “practiced” eye of a biologist who is only looking to see what she expects to find.
% art is good at qualitative questions, the chief question of art has historically been why?



% Thus, artists can play an "\textit{extremely valuable role}" in complementing things overlooked by scientists and providing fresh perspectives~\cite{gates2016art}.
% One extremely valuable role that artists can play in Science and with scientists is to ask questions~\cite{gates2016art}



% \textbf{New ideas and paradigms for system development.}
% The development of applied research can be influenced by a variety of factors. The most straightforward one is technology, such as the numerous automated interfaces and applications catalyzed by AI technology in recent years. Besides, the development of applications can also be strongly influenced by new ideas and paradigms. For instance, in the history of HCI, the proposal of affective computing expanded the task of computers from helping people to complete objective commands to enabling computers to recognize, understand, and respond to human emotions, which in turn has given rise to a multitude of related applications. The theory of media equivalence challenges the previous anthropocentric paradigm of treating computers purely as tools, stimulating a series of work that injects agency to computers. The emergence of these new ideas is without exception the result of the integration of computer science with other disciplines (\eg psychology, design, and communication). Examples of integrating engineering with art can also be found, with a typical representative being the pragmatic aesthetics proposed by Petersen~\etal~\cite{petersen2004aesthetic}. \x{This concept supports a user-centered approach to design, ensuring that systems are not only functional but also provide a positive and engaging user experience.
% It emphasizes that aesthetics is not a standalone scientific phenomenon but is tightly connected to context, use, and instrumentality. They presented a set of prototype systems that embody this idea and argued: ``Aesthetics has a purposeful role in the use of interactive systems, aesthetics is not only an adhesive making things attractive, and it is part of the foundation for a purposeful system.''

% 最近已经有critical theory的倡导出现
% Fan~\etal~\cite{fan2012spark} used artistic visualizations to help users track their fitness data and found that it sparked more user engagement in their own health compared to traditional charts



% Art can pave a road for communication and make science more accessible. 
% As reflected by Gates-Stuart~\etal~\cite{gates2016art}, ``non-specialists often feel intimidated by the detail and technology behind the images commonly used in science; this makes it difficult to get a conversation going about what science is being done, and why. However, everyone thinks they have a right to an opinion about an artwork, and are usually happy to discuss their view...allowing not only artistic elements and intent to be discussed, but also the concepts and the purpose behind the science.''
% Scientific visualisation can stop a conversation about Science; Art can start one.
% Some data artworks are so successful and influential in the history~\cite{mottelson2020disseminating}






\section{Limitations}

Our analysis of artistic data visualizations is based on a self-constructed corpus. While we endeavored to include a diverse range of sources, our collection primarily features works from or closely associated with the visualization academic community. 
% such as submissions to the VISAP program or those identified and introduced by visualization researchers
Also, since we required artworks or their publication venues to explicitly mention keywords such as artistic data visualization and data art, some works may not have been included in our study due to a lack of description or tagging.
Thus, it is not exhaustive and does not represent all artistic data visualizations in the wild. Also, it contains a relatively small number of early data artworks due to issues such as lack of digitization or broken links, as well as a scarcity of non-Western artworks. 
% Future research could consider including more works beyond the academic realm and from non-Western backgrounds.
% Second, although we tried to cover important dimensions (\ie authors, design fields, tasks, methods) when analyzing the artworks, these dimensions are not exhaustive, and there may still be additional/alternative analytical dimensions. 
In addition, due to the nature of the interview research method, the number of data artists we were able to interview is limited. % We mitigated this limitation by recruiting data artists of various ages, backgrounds, and with diverse themes and methods in their works.
We hope to engage with more practitioners in the field of artistic data visualization to obtain more first-hand and in-depth research materials. 
Besides, when performing thematic analyses, while we achieved consensus among coders, some nuances might have been sacrificed. For instance, the concept of \textit{sculpture}, which has been frequently used by artists, was initially coded but later dropped due to its broad applicability, ranging from static physical objects to various interactive installations. This decision, aimed at reducing ambiguity, may have led to a loss of deeper connotations, such as the heritage relationships between data art and traditional fine arts. This also highlights the need to enhance the reflectiveness of coders~\cite{braun2022thematic} when conducting qualitative coding.
% we plan to collect more artistic data visualizations and expand the dimensions and granularity of our analysis. 
% For example, we intend to track and observe the creative process of artists to gain a deeper understanding of how an artistic data visualization is made, how design iterations are completed, and the collaborative methods employed by the teams.

