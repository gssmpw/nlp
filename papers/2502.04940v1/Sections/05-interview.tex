\section{Perspectives from Data Artists}
\label{sec:interview}

Next, to better understand the underlying considerations behind artistic data visualization, as well as to gain firsthand insights to cross-validate our previous findings, we conducted in-depth interviews with twelve data artists.


\subsection{Participants and Process}
\label{sssec:interviewprocess}

We invited participants through: (i) sending interview invitations to the authors of IEEE VISAP projects, and (ii) reaching out to practitioners who self-tagged as data artists on social media.
A total of 12 data artists accepted our invitation, including 5 females and 7 males. Their ages ranged from 23 to 42 and were diverse in job and educational backgrounds (see \autoref{tab:participants}). 
% xx (xx\%) were xxx, xx (xx\%) were xxx, xx (xx\%) were xxx, and xx (xx\%) were xxx.
% xx of them have artworks recognized by professional events of artistic data visualization (\eg accepted by IEEE VISAP) and xxx have xxx. 

% \begin{table*}[htbp]
% \centering
% \fontsize{7.5}{9}\selectfont
% \begin{adjustbox}{width=0.7\textwidth}\begin{tabular}{llllll}
% \toprule
% ID &	Sex &	Age &	Job &	Educational Background \\
% \midrule
% P1	&	M	&	25	&	Research Assistant	&	Visual Design	\\
% P2	&	M	&	25	&	Product Manager	&	Industrial Design \& Data Visualization \\
% P3	&	M	&	38	&	Creative Technologist	& Computer Science \& Visual Design	\\
% P4	&	M	&	40	&	Lecture \& Entrepreneur	&	Fine Arts	\\
% P5	&	M	&	27	&	PhD Student	&	Architecture \& Computer Science	\\
% P6	&	F	&	27	&	Postdoctoral Researcher	&	Computer Science	\\
% P7	&	F	&	27	&	UX Designer	&	Digital Media	\\
% P8	&	M	&	23	&	Graduate Student 	& Digital Media	\\
% P9	&	F	&	42	&	Associate Professor	&	Interior Design \& Multimedia	\\
% P10	&	F	&	32	&	Assistant Professor	& Communication Design \\
% P11	&	M	&	28	&	Graduate Student	&	Industrial Design	\\
% P12	&	F	&	35	&	PhD Student \& Artist	&	Media Art	\\
% \bottomrule
% \end{tabular}
% \end{adjustbox}
% \caption{Information of the interviewees. All participants self-identified as data artists while performing their works.}
% \label{tab:participants}
%  \Description{Information of the 12 interviewees. The table contains five columns: ID, Sex, Age, Job, Educational background.}
% \vspace{-2em}
% \end{table*}


The interviews were semi-structured. We prepared a set of questions in advance, which can be categorized into four parts: (i) the creation of artistic data visualizations (\eg ``How did you come up with the idea of this project?'', ``How was your design process?''), (ii) the understanding of artistic data visualization (\eg ``What do you think is the most prominent feature of artistic data visualization?'', ``What distinguishes artistic data visualization from other types of data visualization''), (iii) the response to potential critiques (\eg ``If someone challenges the accuracy or efficiency of your visualization, how would you respond?''), (iv) challenges and expectations (\eg ``Have you ever met any challenges?'', ``How do you envision the future of this field?''). 
Before the interviews, we conducted background research on the participants, including their education, key works, achievements, and publications, to ensure the interviews were meaningful and relevant to their experiences and expertise.
During the interviews, we first asked the participants to introduce themselves, as well as their experience with artistic data visualization briefly as a warm-up. Next, we asked the aforementioned interview questions surrounding their representative data artworks. Depending on their responses, we asked follow-up questions to dig deeper into interesting points brought up by them (\eg ``Could you please elaborate on that point?'', ``Can you provide an example to illustrate?''). Each interview lasted about one hour and the interview process was audio recorded with the participant's consent.


% We obtained more than more than 850-minute audio recordings and about 220,000-word transcriptions from the interviews. 
To analyze the data, we followed the research methodology suggested by thematic analysis~\cite{braun2022thematic}.
Two authors first read through the transcriptions independently to familiarize ourselves with the data and took notes on initial observations.
Then, we coded the transcriptions with the goal of identifying the responses to the four aforementioned research questions. 
Next, we grouped related codes together to form themes, cross-checked each other's codes and themes, and marked disagreed codes until reaching 100\% consensus. 
% These categories should encapsulate the main topics discussed during the interviews. 
% For example, with regard to the process of creating artistic data visualization, we found several distinct types of motivations. Therefore, we grouped similar pipelines and summarized them as higher-level patterns.  


\begin{table}[t!]
\fontsize{6.8pt}{7.5pt}\selectfont
%\scriptsize
\centering
\caption{Information of the interviewees. }
%All participants self-identified as data artists while performing their works.
\label{tab:participants}
\vspace{-1em}
\begin{tabularx}{\columnwidth}{p{0.3em}p{0.3em}p{0.5em}p{10.3em}X}
\toprule
ID &	Sex &	Age &	Job &	Educational Background \\
\midrule
P1	&	M	&	26	&	Artist	&	Visual Design	\\
P2	&	M	&	26	&	Product Manager	&	Industrial Design \& Data Visualization \\
P3	&	M	&	39	&	Creative Technologist	& Computer Science \& Visual Design	\\
P4	&	M	&	41	&	Art Studio Head	&	Fine Arts	\\
P5	&	M	&	28	&	PhD Student	&	Architecture \& Computer Science	\\
P6	&	F	&	28	&	Postdoctoral Researcher	&	Computer Science	\\
P7	&	F	&	28	&	UX Designer	&	Digital Media	\\
P8	&	M	&	24	&	Graduate Student 	& Digital Media	\\
P9	&	F	&	43	&	Associate Professor in Art	&	Interior Design \& Multimedia	\\
P10	&	F	&	33	&	Assistant Professor in Art	& Communication Design \\
P11	&	M	&	29	&	Graduate Student	&	Industrial Design	\\
P12	&	F	&	36	&	Artist	&	Media Art	\\
\bottomrule
\end{tabularx}
\vspace{-1em}
\end{table}


%  \Description{Information of the 12 interviewees. The table contains five columns: ID, Sex, Age, Job, Educational background.}
% \vspace{-2em}





\subsection{Findings}

\subsubsection{Creation of Artistic Data Visualization}
We found that artists exhibit some common patterns when ideating and making artistic data visualizations.

\textbf{Motivation.}
We identified four modes of motivation for creating artistic data visualization. 
The first mode is \textbf{value-driven} (P3, P5, P9, P10, P12), which is often based on the artist's high-level values or deep thoughts. For instance, P3 mentioned that his art project originated from his philosophical contemplation of the relationship between humans and nature: ``\textit{A pivotal moment occurred during my journey from Washington to New York, when I was inspired by the mountains outside the window. In contrast, the cityscape and industrial facilities we usually live with are very modern and cold. So, I wondered if I could create a poetic geographical space visualization.}''
The second mode is \textbf{interest-driven} (P6, P8, P10, P11*2 times (corresponding to 2 different works)), which is more personal and related to specific living experiences. For example, P6 is a visualization researcher who has had an interest in art and literature since childhood (``\textit{I have always had a strong interest in the humanities}''). Therefore, after entering the field of visualization, she has always wanted to use visualization techniques to convey the rhythms and emotions found in literature.
The third mode is \textbf{reality-driven} (P1, P7, P9, P11). P7's artwork is about preserving Tibetan calligraphy. She said, ``\textit{We visited the inheritors of Tibetan calligraphy and learned about its long and glorious history. However, very few people are currently aware of this heritage or involved in its preservation. Therefore, we decided to present the unique beauty of this cultural heritage in a visual form.}''
The last mode is \textbf{client-driven} (P1, P2, P4). P4 is a data artist who has his own art studio. His recent project, which involves animating and visually representing ocean tide changes using 3D particle visualization, is based on the specific requirements of a collaborator. Similarly, P2 also mentioned that his art project was a ``\textit{commissioned work}'', meaning that the initial idea had already been determined by someone else.
% and the artist's main responsibility was the creative execution
% Apart from the last mode, which is driven more by commercial interests, the first three modes are centered on the intrinsic aesthetic and expressive demands of the artists.

\textbf{Workflow.}
The workflow of creating artistic data visualizations mainly falls into two categories: input-driven and output-driven. The \textbf{input-driven} workflow is more similar to the classic model of information visualization, which follows an input-output pipeline. 
% ~\cite{card1999readings}
Artists first obtain a dataset, then analyze it, and create visualizations. Artists often do not know in advance what the data will ultimately lead to. For instance, when working on an art visualization project related to social media, P5 collected data from Twitter after having a preliminary idea, and then tested various data dimensionality reduction algorithms and visualization layouts, finally selecting the most satisfying version.
In contrast, in the \textbf{output-driven} workflow, artists first conceive a mental image of the desired outcome, such as the overall aesthetic style and how the visualization will look like, and then utilize data as a medium to realize it. For instance, P3 had already determined to re-present modern maps using a poetic, painting style before actually beginning to collect and process data. Similarly, when working on her data sonification project, P10 drew inspiration from Kandinsky, who expressed music through points, lines, and planes. She decided to also visualize sound data as geometric shapes before proceeding to collect the necessary data. In terms of frequency, the output-driven workflow (P1, P3, P4, P7, P9, P10, P11*3 times (3 different works), P12) was more common than the input-driven workflow (P1, P2, P5, P6, P8). This finding also resonates with previous research~\cite{tandon2023visual} that artists are especially adept at visual tasks, when compared to mathematicians or computer scientists.
% In line with Tandon~\etal~\cite{tandon2023visual}, who found that artists excel more in visual tasks compared to mathematicians/computer scientists, 
% P4 said, ``\textit{my client may have an effect they want, such as 3D effects, and cool animation of particles.}''
However, despite differences in the overall workflow, all the artists we interviewed mentioned that they iteratively adjust and refine their work until it reaches a satisfactory point. 
% This often involves updating or filtering data based on the design, as well as adjusting design elements based on data patterns.


\subsubsection{Understanding of Artistic Data Visualization}
\label{sssec:understanding}

Next, we performed an analysis of the participants' descriptions regarding their understanding of artistic data visualization. To delve deeper into this important issue, we employed a richer array of analysis methods. First, we extracted relevant sentences from the participants, conducted word segmentation, synonyms detection, and frequency statistics to identify high-frequency keywords. When analyzing these keywords, close reading was also utilized to ensure a detailed examination of their original contexts.
% , aiming to uncover its deeper meanings, implications, and discourses. 
% we also returned to the text to double-check their original expressions. Specifically, 
As a result, the most mentioned words include \textit{express/convey/communicate} (N = 24), \textit{emotion/feeling/subjectivity} (N = 24), \textit{story/narrative} (N = 11), \textit{reflection/critical thinking} (N = 10), \textit{concept/idea} (N = 6), \textit{purpose/intent} (N = 4).
For example, P2 thought artists ``\textit{try to make inherently emotionless data convey emotional effect.}'' P8 said, ``\textit{I believe good data art should be engaging, offering more feelings or allowing people to see things from different perspectives, rather than merely enhancing efficiency.}''
P4 believed that artistic data visualization ``\textit{is not meant to provide a clear answer or a specific analysis of the data; art often involves speculation and reflection.}''
There were also keywords, although mentioned less frequently, that conveyed very interesting values and attitudes, such as \textit{freedom} (N = 2), \textit{openness} (N = 2), and \textit{ambiguity} (N = 2).
For example, P6 said, ``\textit{Art tends to be more divergent and doesn’t have a strong, specific purpose; it requires some ambiguity. We just place it there and let people interpret it however they wish.}'' P12 expressed her view on art with a highly concise statement: ``\textit{Aesthetics is freedom.}'' In her view, artists always strive to break existing rules or constraints, seeking and defending the space for free expression.

These keywords along with their original expressions have some obvious common characteristics.
First, they all involve a \textbf{strong emphasis on human agency}. 
% Whether it is subjective emotion or the expression of attitudes, they are all imbued with human feelings, understanding, and judgment. 
% This indicates that artistic data visualization has a strong tendency towards humanity in its intrinsic pursuit, thus offering further empirical support for the argument that artistic visualization is distinct from pragmatic visualization driven primarily by scientific discourse. 
In contrast to science, which prioritizes objectivity, generalizability, and precision~\cite{brown2001art}, artists are more inclined to embrace subjectivity, individuality, and ambiguity.
Second, we noticed that the way artists described artistic data visualization does not focus on specific forms, nor does it emphasize classic aesthetic standards such as harmony or vividness. Instead, they all emphasized the artist's concepts and expressions, demonstrating a \textbf{strong contemporary art characteristic}. 
% in the classical era, fine arts were clearly defined as painting, drawing, sculpture, and artworks were concrete objects used to imitate the natural world. However
As introduced in \autoref{ssec:contemporary}, in contemporary times, ``art as imitation'' has been replaced by ``art as expression'' and ``art as concept/idea''~\cite{pooke2021art}. Art is no longer confined to form but is the external manifestation of human concepts and thoughts. 
% Our interview results not only echo these trends but also confirm that data art remains deeply contextualized within the art history and contemporary art discourse.
In general, the perspectives offered by the data artists in our interviews resonate well with the design taxonomy in \autoref{fig:techniques}, and also provide us with a deeper understanding of the underlying logic of the identified techniques (\ie sensation, interaction, narrative, physicality). For example, the emphasis on subjective experience enhances artists' interest in stimulating human senses. The need to communicate ideas and concepts makes them adept at using narratives and eager to experiment with new modes of expression, stimulating the exploration of interactive technologies and various physical materials.
% Participants in our interviews thought that the most prominent feature of artistic data visualization is its concentration on \textbf{aesthetics}, \textbf{humanity}, \textbf{narrative}. 


% P10 thought that "\textit{compared to traditional visual communication, data visualization places greater emphasis on graphic design and layout, striving for beauty and artistry}."
% P9 thought ``\textit{narrative is the core of artistic visualization}'', and P3 emphasized that ``\textit{humanistic sensibilities and aesthetics are central considerations in the creation of visual works}''.


% Such critical thinking can lead to artworks that are sharp and powerful. For example, during the interview, P4 mentioned artist Simon Weckert's famous work, \textit{Google Maps Hack}~\cite{}, which played a dark joke about data. Simon walked back and forth along an empty street while pulling a cart containing 99 smartphones, effectively causing a false traffic jam on Google Maps. "\textit{It is very simple but mind-blowing, prompting society to reconsider our technological systems and pushing Google to make changes. I believe this exemplifies the power of art.}"

% \underline{Application scenarios.} 
% We identified four main application scenarios of artistic data visualization. The first one is \textbf{art exhibitions}. Data artists often exhibit their work in places such as museums, galleries, and public spaces (\eg parks, public squares) to attract passers-by. For example, P1 planned to physicalize his data artwork using 3D printing and exhibit it at a gallery "\textit{to let more people see and interact}". P9 mentioned that one of her data artworks has been collected by a renowned art gallery and has been continuously displayed in a showcase.
% % As introduced by Px, a creative technologist that serves many museums, "\textit{some exhibitions are more commercial and you need to buy tickets, while some are free, mainly to provide the public with more venues for art appreciation}").
% The second one is \textbf{marketing and advertising}. As introduced by P3, "\textit{sometimes businesses may need to display some cool artworks in their shop windows or building screens to attract consumers or to reflect the brand's youthfulness and vitality...I have worked on such projects for many years, all based on real user data and sales data, to generate artistic visualizations. For businesses, the concept of data-driven digital art can be more appealing than just traditional artworks.}" 
% P5 mentioned that a commercial company approached her, wanting to transform her art pieces into saleable products.
% The third one is \textbf{media and mass communication}.
% % to raise awareness about xxx and tell compelling stories .
% For example, the data artwork by P1 has been forwarded by many social media accounts of the governments and achieved more than 100 million views. P3 recalled several data artworks that have been published by the New York Times and National Geographic.
% The fourth one is \textbf{education}. P2 mentioned that his artistic data visualization is going to be included in a popular science book and he thought "\textit{artistic charts are more captivating and can attract readers to flip through the book, possibly more suitable for popular science than conventional charts}".


\subsubsection{Response to Critiques or Doubts}
As introduced in \autoref{ssec:artisticvis}, the controversy surrounding artistic data visualization mainly concerns its readability and reading efficiency. In our interviews, while most participants acknowledged that such critiques pose a challenge for artistic data visualization, they also provided some justifications.

% \ul{The balance between aesthetics and functionality.}
% While all participants expressed their desire for their work to be seen and understood by a wider audience, they agreed that artistic practice does entail more subjectivity and uncertainty. P6 pointed out, "\textit{Compared to general visualization, artistic data visualization indeed requires users to read data in a more abstract and unconventional way.}" During the interview, when asked how she would assess the actual impact of her work, P7 fell into contemplation and said, "\textit{Actually, sometimes I also wonder, is my work really useful? Or, how can it be more useful?}" 
% % P4 said, "It's truly a balancing act, primarily depending on what your use case is and what the artist themselves are striving for." 
% P5 thought there's no need to pursue universal appeal or understanding because "\textit{some people are more rational and some are emotional. Everyone has different tastes. If we can actually attract those who are willing to understand us and able to understand us, that's already good. Of course, it's not easy, so artists need to engage in a lot of thinking and experimentation.}"


\textbf{Artistic data visualization adopts an alternative mechanism of communication.}
5 participants thought that artistic data visualization achieves communicative goals through engaging viewers on a deeper level. For example, P4 argued for a broader understanding of efficiency: ``\textit{If our goal is to see precisely what a data point's value is, artistic data visualization is indeed inefficient...but in reality, the audience is probably not professional data analysts and they don't care about data precision. Instead, they care about what the data conveys and expresses. In that sense, artistic data visualization is efficient. For instance, through audiovisual elements, I can quickly immerse you in a context and convey a message.}''
P5 thought artistic data visualization and conventional statistical charts each have their merits: ``\textit{It's just that the way they convey information is different. With statistical charts, you need to decipher the data and get insights, almost like a bottom-up approach. However, artistic data visualization can directly provide you with messages and insights. If you feel attracted, you can then explore the data within. It's more of a top-down approach.}''
P6 thought there's no need to pursue a universal mode of data communication because: ``\textit{Some people are more rational and some are emotional. Everyone has different tastes. If we can attract those who are willing to understand us and able to understand us, that's already nice.}''
% Of course, it's not easy, so we artists need to engage in a lot of thinking and experimentation.


\textbf{Clarity and efficiency are not necessarily sacrificed.}
7 participants stated that their emphasis on subjectivity and humanity does not mean they \textit{only} care about these aspects. Most participants recognized the importance of data precision, particularly at the stages of data collection, preprocessing, and analysis. For example, P5 said, ``\textit{We invested a considerable amount of effort in data scraping and cleaning, just like the general process in data science. You need to address missing values, outliers, and ensure the data is correct.}'' P4 thought that ``\textit{data art should be supported by data; if the data is fictitious, it shouldn't be counted as data visualization.}'' 
% P2 agreed that ``\textit{while I may incorporate subjectivity in the visual presentation, I believe ensuring the accuracy of the underlying data is still a bottom line...after all, you can't deceive your audience.}''
Regarding understandability, P1 said, ``\textit{I would carefully consider how users will read and interact with my visualizations. I don't want to create a beautiful non-sense.}''
P3 mentioned that when designing installations for museums, his team would conduct user interviews in advance or perform A/B tests, and ``\textit{if many users failed to understand, we would make design adjustments.}''
P4 thought there were both good and bad designs in artistic data visualization: ``\textit{If a work is completely unreadable, it might indicate poor design. Conversely, if designers can provide readers with legends and reading guidance as much as they can while maintaining aesthetics and novelty, artistic data visualizations can of course be readable.}''
As both a data artist and visualization researcher, P6 said, ``\textit{I do consider users, but any visualization involves trade-offs. Even when creating visual analytics systems, sometimes you sacrifice some utility for innovative visual effects...I don't think artistic data visualization is that special.}''
% is the only artist who said that if users truly don't understand her work, she doesn't want to make compromises. However, at the same time, she also expressed her expectation for users to appreciate her work and argued



\subsubsection{Challenges and Expectations}
\label{sssec:expectations}
Lastly, participants reflected on the challenges and their expectations.

\textbf{Facing a bottleneck in skills or creativity.}
First of all, 7 participants mentioned that they faced a bottleneck in issues such as handling big data, crafting a narrative, and thinking of novel ideas. For example, P7 thought ``\textit{conceptualizing a project is challenging}'', and ``\textit{another major challenge is in storytelling...crafting a cohesive narrative is tough.}''
P4 reflected that he would often ``\textit{draw inspiration from the design ideas of outstanding peers, but original and highly innovative ideas totally came up with by myself are not that many.}'' 
P10 said, ``\textit{We currently don’t have access to particularly professional data, and we definitely lack the tools to work with such data.}''
P1 reflected that ``\textit{at present, my work is still largely targeted at the general public and remains a bit shallow. I try to dig deeper...However, if I want to pursue this, I definitely need to explore the deeper scientific insights behind the data.}''
% P4 mentioned, ``\textit{I feel it necessary for artists to continuously monitor the latest technological trends and societal issues to keep sensitive and empathetic to art subjects.}'' 
% P9 mentioned that the emergence of AI drawing has indeed created a sense of crisis for her. P12 envisioned that the advent of AI would force data artists to pursue newer, more avant-garde aesthetics: ``\textit{How can art maintain its identity when AI can easily realize a variety of artistic styles? In order to defend humanity, future art will undoubtedly move towards greater rebellion...data-driven art is at the forefront of this movement.}''

\textbf{Insufficiency of interdisciplinary collaboration \& understanding.}
Relating to the previous issue, 6 interviewees expressed the desire to foster more cross-disciplinary collaborations, such as the hope for scientists to join and enhance professional knowledge of data (P1), and to collaborate with teams specialized in data processing (P10). 
% For example, P7 mentioned that her team includes illustrators, interaction designers, and engineers; therefore, ``\textit{we met every week and iterated the design multiple rounds to ensure that the work is both good looking, sound in the narrative, and technically feasible}''.
P4 expressed his urgent need to recruit people with expertise in algorithms for his art studio, because ``\textit{to create generative art, such as large-scale data-driven particle systems, one must have knowledge of statistics or machine learning; I have long wanted to recruit individuals with such expertise for my team.}''
However, P6 thought that currently, there is still a lack of mutual understanding among people from different disciplines. She observed that those currently engaged in artistic data visualization hail from diverse backgrounds: some have a technical background and an interest in art, while others are artists by original training: ``\textit{These two groups approach things very differently because their foundational training and theoretical frameworks are completely distinct. Sometimes, even though we claim to be interdisciplinary, I feel like there’s a bias or a gap in understanding between these two groups, as if we’re not speaking the same language.}''
% Nevertheless, many participants still find that the sparks generated by interdisciplinary collaboration are inspiring and exciting. P1 thought that the combination of art and big data would release significant potential, and he hoped to collaborate more closely with data scientists.


\textbf{The gap between art practice and research.}
Despite the flourish of artistic data visualization in the wild, 6 participants saw a gap between its practice and research. P9, who is both an artist and an assistant professor, highlighted the difficulties of publishing papers and being involved in academia: ``\textit{In art schools, creating works and receiving awards hold more value than publishing papers. Many artists don’t feel the need to publish papers, but due to the larger context of academic evaluation, some are now being pushed to transform their works into papers, which forces them to find new ways of approaching this.}'' P12, who used to be an artist and is now pursuing a Ph.D. degree, said that ``\textit{I have created many works and participated in many exhibitions, but I truly don't know how to write a doctoral dissertation about them...It feels like the thought processes for conducting research and creating artwork are quite different.}''
% We usually put a lot of effort into creating artworks, and some even received significant social recognition. However, it's challenging to understand the academic significance and value of these works. We are not quite sure how to write them into papers and apply for funding.
% P2 also noted that many data artworks may not necessarily be seen as academic research, saying, ``\textit{Although there are many data artworks, research papers that examine such artworks are rarely seen.}'' 
P8 felt that ``\textit{it (the academic community) offers limited theoretical support for my work.}''
P4 recalled having moments of self-doubt because ``\textit{we produce a lot of work, but the methodologies and frameworks for organizing this kind of work are scarce...sometimes it's hard to justify our value to others.}''
% However, P2 also expressed his optimistic attitude to this challenge because ``\textit{the visualization community is quite open, and I have noticed that there are already some studies exploring data art.}''







