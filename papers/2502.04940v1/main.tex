% $Id: template.tex 11 2007-04-03 22:25:53Z jpeltier $

\documentclass{vgtc}                          % final (conference style)
%\documentclass[review]{vgtc}                 % review
%\documentclass[widereview]{vgtc}             % wide-spaced review
%\documentclass[preprint]{vgtc}               % preprint
%\documentclass[electronic]{vgtc}             % electronic version

%% Uncomment one of the lines above depending on where your paper is
%% in the conference process. ``review'' and ``widereview'' are for review
%% submission, ``preprint'' is for pre-publication, and the final version
%% doesn't use a specific qualifier. Further, ``electronic'' includes
%% hyperreferences for more convenient online viewing.

%% Please use one of the ``review'' options in combination with the
%% assigned online id (see below) ONLY if your paper uses a double blind
%% review process. Some conferences, like IEEE Vis and InfoVis, have NOT
%% in the past.

%% Figures should be in CMYK or Grey scale format, otherwise, colour 
%% shifting may occur during the printing process.

%% These few lines make a distinction between latex and pdflatex calls and they
%% bring in essential packages for graphics and font handling.
%% Note that due to the \DeclareGraphicsExtensions{} call it is no longer necessary
%% to provide the the path and extension of a graphics file:
%% \includegraphics{diamondrule} is completely sufficient.
%%
\ifpdf%                                % if we use pdflatex
  \pdfoutput=1\relax                   % create PDFs from pdfLaTeX
  \pdfcompresslevel=9                  % PDF Compression
  \pdfoptionpdfminorversion=7          % create PDF 1.7
  \ExecuteOptions{pdftex}
  \usepackage{graphicx}                % allow us to embed graphics files
  \DeclareGraphicsExtensions{.pdf,.png,.jpg,.jpeg} % for pdflatex we expect .pdf, .png, or .jpg files
\else%                                 % else we use pure latex
  \ExecuteOptions{dvips}
  \usepackage{graphicx}                % allow us to embed graphics files
  \DeclareGraphicsExtensions{.eps}     % for pure latex we expect eps files
\fi%

%% it is recomended to use ``\autoref{sec:bla}'' instead of ``Fig.~\ref{sec:bla}''
\graphicspath{{figures/}{pictures/}{images/}{./}} % where to search for the images


% macro
\newcommand{\etal}{et~al.~} 
\newcommand{\ie}{i.e.,~}
\newcommand{\eg}{e.g.,~}
\newcommand{\ncorpus}{220 }

\usepackage{xurl}


% \usepackage{url}
% % \usepackage{soul}
% \makeatletter
% \g@addto@macro{\UrlBreaks}{\UrlOrds}
% \makeatother


\usepackage{subcaption}
\let\subfigureautorefname\relax

\usepackage{stfloats}


% \usepackage{float} 
\usepackage{graphicx}
\usepackage{wrapfig}
\usepackage{graphbox}
\usepackage[T1]{fontenc}
\usepackage{amsmath}  
\usepackage{multirow}
\usepackage{tabularx}
\newcommand{\z}[1]{\textcolor{DarkSlateBlue}{#1}}
% \newcommand{\x}[1]{{\leavevmode\color{blue}{#1}}}
\newcommand{\todo}[1]{{\leavevmode\color{red}{#1}}}
\usepackage[svgnames,x11names,table]{xcolor}
\usepackage{soul}
\usepackage{adjustbox}
\usepackage{hyperref}

\definecolor{c1}{HTML}{9796bb}
\definecolor{c2}{HTML}{00beb9}
\definecolor{c3}{HTML}{dfb0c7}
\definecolor{c4}{HTML}{add9a1}
\definecolor{c5}{HTML}{eac793}


\usepackage{microtype}                 % use micro-typography (slightly more compact, better to read)
\PassOptionsToPackage{warn}{textcomp}  % to address font issues with \textrightarrow
\usepackage{textcomp}                  % use better special symbols
\usepackage{mathptmx}                  % use matching math font
\usepackage{times}                     % we use Times as the main font
\renewcommand*\ttdefault{txtt}         % a nicer typewriter font
\usepackage{cite}                      % needed to automatically sort the references
\usepackage{tabu}                      % only used for the table example
\usepackage{booktabs}                  % only used for the table example
%% We encourage the use of mathptmx for consistent usage of times font
%% throughout the proceedings. However, if you encounter conflicts
%% with other math-related packages, you may want to disable it.

%% If you are submitting a paper to a conference for review with a double
%% blind reviewing process, please replace the value ``0'' below with your
%% OnlineID. Otherwise, you may safely leave it at ``0''.
\onlineid{6460}

%% declare the category of your paper, only shown in review mode
\vgtccategory{Research}

%% allow for this line if you want the electronic option to work properly
\vgtcinsertpkg

%% In preprint mode you may define your own headline. If not, the default IEEE copyright message will appear in preprint mode.
%\preprinttext{To appear in an IEEE VGTC sponsored conference.}

%% This adds a link to the version of the paper on IEEEXplore
%% Uncomment this line when you produce a preprint version of the article 
%% after the article receives a DOI for the paper from IEEE
%\ieeedoi{xx.xxxx/TVCG.201x.xxxxxxx}


%% Paper title.

\title{More Than Beautiful: Exploring Design Features, Practical Perspectives, and Implications of Artistic Data Visualization}

%% This is how authors are specified in the conference style

%% Author and Affiliation (single author).
%%\author{Roy G. Biv\thanks{e-mail: roy.g.biv@aol.com}}
%%\affiliation{\scriptsize Allied Widgets Research}

%% Author and Affiliation (multiple authors with single affiliations).
%%\author{Roy G. Biv\thanks{e-mail: roy.g.biv@aol.com} %
%%\and Ed Grimley\thanks{e-mail:ed.grimley@aol.com} %
%%\and Martha Stewart\thanks{e-mail:martha.stewart@marthastewart.com}}
%%\affiliation{\scriptsize Martha Stewart Enterprises \\ Microsoft Research}

%% Author and Affiliation (multiple authors with multiple affiliations)
\author{Xingyu Lan\thanks{Xingyu Lan is the corresponding author. She is a member of the Research Group of Computational and AI Communication at Institute for Global Communications and Integrated Media. e-mail: xingyulan96@gmail.com.}\\ %
        \scriptsize Fudan University %
\and Yifan Wang\thanks{e-mail: wangyifanlea@gmail.com}\\ %
     \scriptsize Fudan University %
\and Lingyu Peng\thanks{e-mail: lingyupeng6@163.com}\\ %
     \scriptsize \centering Harbin Institute of Technology
\and Xiaofan Ma\thanks{e-mail: xiaofanma\_13@163.com}\\ %
     \scriptsize \centering Sun Yat-sen University
     }

%% A teaser figure can be included as follows
% \teaser{
%   \centering
%   \includegraphics[width=\linewidth]{CypressView}
%   \caption{In the Clouds: Vancouver from Cypress Mountain. Note that the teaser may not be wider than the abstract block.}
%   \label{fig:teaser}
% }

%% Abstract section.
\abstract{Standing at the intersection of science and art, artistic data visualization has gained popularity in recent years and emerged as a significant domain. Despite more than a decade since the field's conceptualization, a noticeable gap remains in research concerning the design features of artistic data visualizations, the aesthetic goals they pursue, and their potential to inspire our community. To address these gaps, we analyzed \ncorpus data artworks to understand their design paradigms and intents, and construct a design taxonomy to characterize their design techniques (\eg sensation, interaction, narrative, physicality). We also conducted in-depth interviews with twelve data artists to explore their practical perspectives, such as their understanding of artistic data visualization and the challenges they encounter. In brief, we found that artistic data visualization is deeply rooted in art discourse, with its own distinctive characteristics in both inner pursuits and outer presentations. Based on our research, we outline seven prospective paths for future work.}
% and applications within the visualization community
 % end of abstract

%% ACM Computing Classification System (CCS). 
%% See <http://www.acm.org/about/class> for details.
%% We recommend the 2012 system <http://www.acm.org/about/class/class/2012>
%% For the 2012 system use the ``\CCScatTwelve'' which command takes four arguments.
%% The 1998 system <http://www.acm.org/about/class/class/2012> is still possible
%% For the 1998 system use the ``\CCScat'' which command takes four arguments.
%% In both cases the last two arguments (1998) or last three (2012) can be empty.

\CCScatlist{
  \CCScatTwelve{Human-centered computing}{Visualization design and evaluation methods}{}{}
  % \CCScatTwelve{Human-centered computing}{Visu\-al\-iza\-tion}{Visualization design and evaluation methods}{}
}
\keywords{Artistic Visualization, Data Art, Visualization Design}

% \CCScatlist{
%   % \CCScatTwelve{Human-centered computing}{Visu\-al\-iza\-tion}{Visu\-al\-iza\-tion techniques}{Treemaps};
%   % \CCScatTwelve{Human-centered computing}{Visu\-al\-iza\-tion}{Visualization design and evaluation methods}{}
% }
%\CCScatlist{
  %\CCScat{H.5.2}{User Interfaces}{User Interfaces}{Graphical user interfaces (GUI)}{};
  %\CCScat{H.5.m}{Information Interfaces and Presentation}{Miscellaneous}{}{}
%}

%% Copyright space is enabled by default as required by guidelines.
%% It is disabled by the 'review' option or via the following command:
% \nocopyrightspace

%%%%%%%%%%%%%%%%%%%%%%%%%%%%%%%%%%%%%%%%%%%%%%%%%%%%%%%%%%%%%%%%
%%%%%%%%%%%%%%%%%%%%%% START OF THE PAPER %%%%%%%%%%%%%%%%%%%%%%
%%%%%%%%%%%%%%%%%%%%%%%%%%%%%%%%%%%%%%%%%%%%%%%%%%%%%%%%%%%%%%%%%

\begin{document}
%% The ``\maketitle'' command must be the first command after the
%% ``\begin{document}'' command. It prepares and prints the title block.

%% the only exception to this rule is the \firstsection command

\section{Introduction}

Tutoring has long been recognized as one of the most effective methods for enhancing human learning outcomes and addressing educational disparities~\citep{hill2005effects}. 
By providing personalized guidance to students, intelligent tutoring systems (ITS) have proven to be nearly as effective as human tutors in fostering deep understanding and skill acquisition, with research showing comparable learning gains~\citep{vanlehn2011relative,rus2013recent}.
More recently, the advancement of large language models (LLMs) has offered unprecedented opportunities to replicate these benefits in tutoring agents~\citep{dan2023educhat,jin2024teach,chen2024empowering}, unlocking the enormous potential to solve knowledge-intensive tasks such as answering complex questions or clarifying concepts.


\begin{figure}[t!]
\centering
\includegraphics[width=1.0\linewidth]{Figs/Fig.intro.pdf}
\caption{An illustration of coding tutoring, where a tutor aims to proactively guide students toward completing a target coding task while adapting to students' varying levels of background knowledge. \vspace{-5pt}}
\label{fig:example}
\end{figure}

\begin{figure}[t!]
\centering
\includegraphics[width=1.0\linewidth]{Figs/Fig.scaling.pdf}
\caption{\textsc{Traver} with the trained verifier shows inference-time scaling for coding tutoring (detailed in \S\ref{sec:scaling_analysis}). \textbf{Left}: Performance vs. sampled candidate utterances per turn. \textbf{Right}: Performance vs. total tokens consumed per tutoring session. \vspace{-15pt}}
\label{fig:scale}
\end{figure}


Previous research has extensively explored tutoring in educational fields, including language learning~\cite{swartz2012intelligent,stasaski-etal-2020-cima}, math reasoning~\cite{demszky-hill-2023-ncte,macina-etal-2023-mathdial}, and scientific concept education~\cite{yuan-etal-2024-boosting,yang2024leveraging}. 
Most aim to enhance students' understanding of target knowledge by employing pedagogical strategies such as recommending exercises~\cite{deng2023towards} or selecting teaching examples~\cite{ross-andreas-2024-toward}. 
However, these approaches fall short in broader situations requiring both understanding and practical application of specific pieces of knowledge to solve real-world, goal-driven problems. 
Such scenarios demand tutors to proactively guide people toward completing targeted tasks (e.g., coding).
Furthermore, the tutoring outcomes are challenging to assess since targeted tasks can often be completed by open-ended solutions.



To bridge this gap, we introduce \textbf{coding tutoring}, a promising yet underexplored task for LLM agents.
As illustrated in Figure~\ref{fig:example}, the tutor is provided with a target coding task and task-specific knowledge (e.g., cross-file dependencies and reference solutions), while the student is given only the coding task. The tutor does not know the student's prior knowledge about the task.
Coding tutoring requires the tutor to proactively guide the student toward completing the target task through dialogue.
This is inherently a goal-oriented process where tutors guide students using task-specific knowledge to achieve predefined objectives. 
Effective tutoring requires personalization, as tutors must adapt their guidance and communication style to students with varying levels of prior knowledge. 


Developing effective tutoring agents is challenging because off-the-shelf LLMs lack grounding to task-specific knowledge and interaction context.
Specifically, tutoring requires \textit{epistemic grounding}~\citep{tsai2016concept}, where domain expertise and assessment can vary significantly, and \textit{communicative grounding}~\citep{chai2018language}, necessary for proactively adapting communications to students' current knowledge.
To address these challenges, we propose the \textbf{Tra}ce-and-\textbf{Ver}ify (\textbf{\model}) agent workflow for building effective LLM-powered coding tutors. 
Leveraging knowledge tracing (KT)~\citep{corbett1994knowledge,scarlatos2024exploring}, \model explicitly estimates a student's knowledge state at each turn, which drives the tutor agents to adapt their language to fill the gaps in task-specific knowledge during utterance generation. 
Drawing inspiration from value-guided search mechanisms~\citep{lightman2023let,wang2024math,zhang2024rest}, \model incorporates a turn-by-turn reward model as a verifier to rank candidate utterances. 
By sampling more candidate tutor utterances during inference (see Figure~\ref{fig:scale}), \model ensures the selection of optimal utterances that prioritize goal-driven guidance and advance the tutoring progression effectively. 
Furthermore, we present \textbf{Di}alogue for \textbf{C}oding \textbf{T}utoring (\textbf{\eval}), an automatic protocol designed to assess the performance of tutoring agents. 
\eval employs code generation tests and simulated students with varying levels of programming expertise for evaluation. While human evaluation remains the gold standard for assessing tutoring agents, its reliance on time-intensive and costly processes often hinders rapid iteration during development. 
By leveraging simulated students, \eval serves as an efficient and scalable proxy, enabling reproducible assessments and accelerated agent improvement prior to final human validation. 



Through extensive experiments, we show that agents developed by \model consistently demonstrate higher success rates in guiding students to complete target coding tasks compared to baseline methods. We present detailed ablation studies, human evaluations, and an inference time scaling analysis, highlighting the transferability and scalability of our tutoring agent workflow.

\section{Background and related work}
% 重点看Artistic data visualization: Beyond visual analytics 和Visualization criticism-the missing link between information visualization and art 的被引


This section reviews the background on artistic data visualization and previous research related to this topic.

\subsection{Artistic Data Visualization in Art History Context}
\label{ssec:contemporary}

Art history has been marked by transformative periods characterized by different aesthetic pursuits, materials, and methods. Since the time of Plato, imitation (or \textit{mimesis}, which views art as a mirror to the world around us) has been an important pursuit~\cite{pooke2021art}. Successful artworks, such as Greek sculptures and the convincing illusions of depth and space in Renaissance paintings, exemplify this tradition.
The advent of modern society and new technology, especially photography, posed a significant challenge to the notion of art as imitation~\cite{perry2004themes}. By the 1850s, modern art began to emerge with the core goal of transcending traditional forms and conventions. Movements like Post Impressionism, Expressionism, and Cubism revolutionized art through innovative uses of form (\eg color, texture, composition), moving art towards abstraction and experimentation. 
After World War II, the Cold War and the surge of various social problems heightened skepticism about the progress narrative of modernity and the superiority of the capitalist system, leading to the rise of postmodernism and the birth of contemporary art~\cite{hopkins2000after,harrison1992art}. One prominent feature of contemporary art is the absence of fixed standards or genres historically defined by the church or the academy. Postmodern design neither defines a cohesive set of aesthetic values nor restricts the range of media used~\cite{pooke2021art}, sparking novel practices such as installations, performances, lens-based media, videos, and land-based art~\cite{hopkins2000after}.
Meanwhile, artists have increasingly drawn energy from various philosophical and critical theories such as gender studies, psychoanalysis, Marxism, and post-structuralism~\cite{pooke2021art}. As a result, as described by Foster~\cite{foster1999recodings}, artists have increasingly become ``manipulators of signs and symbols... and the viewer an active reader of messages rather than a passive contemplator of the aesthetic''. Hopkins~\cite{hopkins2000after} described this shift as the ``death of the object'' and ``the move to conceptualism''. 
% Joseph Kosuth, one of the most important representatives of conceptual artists, also once said that “all art (after Duchamp) is conceptual (in nature) because art only exists conceptually”
% As argued by Danto~\cite{danto2015after}, traditional notions of aesthetics can no longer apply to contemporary art. ``“All there is at the end,” Danto wrote, “is theory, art having finally become vaporized in a dazzle of pure thought about itself, and remaining, as it were, solely as the object of its own theoretical consciousness.''
% The Anti-aesthetic (1983) has described these as ‘anti-aesthetic’ strategies – typified, as we have seen, by a conceptually driven approach to the art object and to the process of its production.

Emerging within the contemporary art historical context, data art has been significantly propelled by the advent of big data. An early milestone was Kynaston McShine's 1970 exhibition \textit{Information} at the Museum of Modern Art (MoMA). 
% In the exhibition catalogue, McShine wrote~\cite{information_moma}: ``Increasingly artists use mail, telegrams, telex machines, etc., for transmission of works themselves—photographs, films, documents—or of information about their activity.'' 
% The millennium era has witnessed substantial growth in this area.
In 2008, Google’s Data Arts Team was founded to explore the advancement of what creativity and technology can do together~\cite{google}.
% data artist Aaron Koblin
In 2012, Viégas and Wattenberg's \textit{Wind Map}, an artwork that visualizes real-time air movement, became the first web-based artwork to be included in MoMA's permanent collection~\cite{wind}.
Since 2013, the academic conference IEEE VIS has included an Arts Program (IEEE VISAP), showcasing artistic data visualizations through accepted papers and curated exhibitions. 
As noted by Barabási~\cite{dataism} (a Fellow of the American Physical Society and the head of a data art lab), data has become a vital medium for artists to deal with the complexities of our society: ``Humanity is facing a complexity explosion. We are confronted with too much data for any of us to make sense of...The traditional tools and mediums of art, be they canvas or chisel, are woefully inadequate for this task...today’s and tomorrow’s artists can embrace new tools and mediums that scale to the challenge, ensuring that their practice can continue to reflect our changing epistemology.''
% a physicist and head of a data art lab, has noted, 

% Artists are now exploring new mediums and methods, incorporating datasets, computer technology, and algorithms into their work.



\subsection{Research on Artistic Data Visualization}
\label{ssec:artisticvis}

Artistic data visualization is also referred to as artistic visualization, data art, or information art~\cite{holmquist2003informative,rodgers2011exploring,few,viegas2007artistic}. Despite the varying terminologies, there is a basic consensus that artistic data visualization must be art practice grounded in real data~\cite{viegas2007artistic}.
Since the early 2000s, a series of papers introduced innovative artistic systems for visualizing everyday data, such as museum visit routes and bus schedule information~\cite{skog2003between,holmquist2003informative,viegas2004artifacts}.
In 2007, Viégas and Wattenberg~\cite{viegas2007artistic} explicitly proposed the concept of \textit{artistic data visualization} and viewed it as a promising domain beyond visual analytics.
% and defined it as ``visualization of data done by artists with the intent of making art''. 
Kosara~\cite{kosara2007visualization} drew a spectrum of visualization design, positioning artistic visualization and pragmatic visualization at opposite ends of this spectrum to demonstrate that the goals of these two types of design often diverge. 
% advocating that analytical visualizations prioritize practicality, while artistic data visualizations emphasize sublime quality, that is, the capacity to inspire awe and grandeur and elicit profound emotional or intellectual responses. 
% In 2011, Rodgers and Bartram~\cite{rodgers2011exploring} utilized artistic data visualization to enhance residential energy use feedback. 
However, overall, research on this subject has been sparse. Among those relevant papers, most have focused on specific applications of artistic data visualization. 
%~\cite{rodgers2011exploring,schroeder2015visualization,perovich2020chemicals}
For instance, Rodgers and Bartram~\cite{rodgers2011exploring} utilized ambient artistic data visualization to enhance residential energy use feedback. Samsel~\etal~\cite{samsel2018art} transferred artistic styles from paintings into scientific visualization.
Artistic practice has also been observed in fields such as data physicalization~\cite{hornecker2023design,perovich2020chemicals,offenhuber2019data} and sonification~\cite{enge2024open}. For example, Hornecker~\etal~\cite{hornecker2023design} found that many artists are practicing transforming data into tangible artifacts or installations. Enge~\etal~\cite{enge2024open} discussed a set of representative artistic cases that combine sonification and visualization.
% dragicevic2020data
% Offenhuber~\cite{offenhuber2019data} created a set of artworks in urban settings that transform air quality data into situated displays, allowing people to encounter visualizations in their daily lives.

% But in contrast, empirical studies that describe the characteristics of artistic visualization and how they are created are extremely scarce. This scarcity forms a stark contrast to the increasingly rich and diverse practices by artists in the field.
% As for the difference between artistic data visualization and traditional visualizations for analytics, Vi{\'e}gas and Wattenberg~\cite{viegas2007artistic} thought that the most salient feature of artistic data visualizations is their forceful expression of viewpoints.
% In Ramirez~\cite{ramirez2008information}'s opinion, functional information visualizations are concerned with usability and performance while aesthetic information visualizations are concerned with visually attractive forms of representation design.
% Donath~\etal~\cite{donath2010data} presented a series of tools developed to integrate artistic expressions in generating unique and customized visualizations based on users' personal data, such as health monitoring data, album records, and e-mail records. 

On the other hand, some studies, while not focusing on artistic data visualization, have explored a key art-related concept: aesthetics. 
% ~\cite{moere2012evaluating,cawthon2007effect,lau2007towards} explored the aesthetics of visualization design in their research. They
For example, Moere~\etal~\cite{moere2012evaluating} compared analytical, magazine, and artistic visualization styles, noting that analytical styles enhance the discovery of analytical insights, while the other two induce more meaning-based insights. Cawthon~\etal~\cite{cawthon2007effect} asked participants to rank eleven visualization types on an aesthetic scale from ``ugly'' to ``beautiful'', finding that some visualizations (\eg sunburst) were perceived as more beautiful than others (\eg beam trees).
% Moere~\etal~\cite{moere2012evaluating} displayed data in three different styles (analytical style, magazine style, artistic style) and found that these styles led to different perceptions of usability and types of insights.
% More importantly, the authors found that the sunburst chart ranks the highest in aesthetics and is one of the top-performing visualizations in both efficiency and effectiveness, thus exemplifying the notion that "beautiful is indeed usable".
Factors such as embellishment~\cite{bateman2010useful}, colorfulness~\cite{harrison2015infographic}, and interaction~\cite{stoll2024investigating} have also been found to influence aesthetics. 
% borkin2013makes,tanahashi2012design
However, most of these studies have simplified aesthetics to hedonic features (\eg beauty), without delving into the nuanced connotations of aesthetics.
% most of these studies have simplified aesthetics to concepts like 'beauty,' 'preference,' or 'pleasing,' without exploring the deeper essence of aesthetics as the core of art.

The value of artistic data visualization to the visualization community is still in controversy. For instance, Few~\cite{few} argued for a clearer distinction between data art and data visualization; he highlighted the negative consequences when data art ``masquerades as data visualization'', such as making visualizations difficult to perceive. Willers~\cite{willers2014show} thought the artistic approach is ``unlikely be appreciated if the intention was for rapid decision making.''
% In an interview, American artist and technologist Harris commented that ``data can be made pretty by design, but this is a superficial prettiness, like a boring woman wearing too much makeup.''~\cite{harris2015beauty} 
To address these gaps, more empirical research needs to be conducted to explore how artistic data visualizations are designed, their underlying pursuits, and how they might inspire our community.




% Examining this field not only helps us understand the latest application of data visualization in various domains but also extends our understanding of the aesthetic and humanistic aspects of data visualization.
% there should be more theoretical investigation into artistic data visualization. 

% These three concepts emphasize placing or installing visualizations at physical places that people will encounter in their daily lives. 

% ~\cite{wang2019emotional}


% gap between art and science~\cite{judelman2004aesthetics}
% constructive visualization~\cite{huron2014constructive}
% data feminism~\cite{d2020data}
% critical infovis~\cite{dork2013critical}
% citizen data and participation~\cite{valkanova2015public}

% \x{Lee~\etal~\cite{lee2013sketchstory}, give users artistic freedom to create their own visualizations.}


% Aesthetics refers to the study of beauty, taste, and sensory perception and is deeply intertwined with art.
% a particular taste for or approach to what is pleasing to the senses and especially sight

% why shouldn't all charts be scatter plot~\cite{bertini2020shouldn}
% aesthetic model~\cite{lau2007towards}
% Aesthetics for Communicative Visualization : a Brief Review
% Stacked graphs--geometry \& aesthetics~\cite{byron2008stacked}
% storyline optimization~\cite{tanahashi2012design}
% graphic designers rate the attractiveness of non-standard and pictorial visualizations higher than standard and abstract ones, whereas the opposite is true for laypeople.~\cite{quispel2014would}
% evaluate aesthetics - wordcloud
% An Evaluation of Semantically Grouped Word Cloud Designs, tag cloud, wordle

% On the other hand, empirical studies conducted with designers have shown that functionality is not the only design goal of visualization. For example, Bigelow~\etal~\cite{bigelow2014reflections} found that designers would frequently fine-tune the non-data elements in their designs, and data encoding was even "a later consideration with respect to other visual elements of the infographic".
% Moere~\cite{moere2011role} - design
% Quispel~\etal~\cite{quispel2018aesthetics} found that for information designers, clarity and aesthetics are both important for making a design attractive.
\section{Case Examples}
\label{sec:case}

% 三个标志性的examples,影响力;时间阶段 + 资料丰富性:Machine Hallucination;Smellmap;TYPE+CODE II


% 参考:Communicating with Motion: A Design Space for Animated Visual Narratives in Data Videos
% 参考:Data as Delight: Eating data
% 参考:Narrative Visualization: Telling Stories with Data


As artistic data visualization is a relatively under-explored field in academia, we follow prior work~\cite{segel2010narrative,shi2021communicating} and first introduce three typical cases to provide an initial exploration and motivation for the study of artistic data visualization. This approach not only offers a preliminary understanding of the practices in this domain, but also provides representative examples from the corpus analyzed in \autoref{sec:space}, helping to contextualize the more abstract analysis presented later.
% provides an initial sense of the practices in this field and
These cases were selected based on three main criteria: (i) we searched public materials about the artworks in \autoref{sec:space} and identified those that have been exhibited in top-tier exhibitions (\eg having a work exhibited at MoMA in New York is regarded as a high recognition for artists), (ii) works that have been widely reported or recognized by news media, (iii) artists that have received prestigious career awards. Based on these criteria, we selected three works that best meet them and span a considerable range of time.
% they are influential artworks that have garnered significant recognition, such as pieces honored with prestigious awards and featured in renowned art exhibitions; (ii) the selected works should span a range of times to reflect diverse technological and artistic trends; (iii) highlighted by esteemed media outlets.
% (iii) there should be an abundance of available materials, such as artist statements, interviews, and talks, to fully elucidate the creation process and the ideas behind these artworks.
% offering ample material for close reading~\cite{brummett2018techniques}.
% Below, we introduce the three cases one by one.

% For example, \x{Smellmap is an art project that spans over ten years. Its artist has not only maintained a comprehensive website for it, providing detailed information on various aspects of the project, but has also written a series of papers based on this project and has been interviewed by the media on multiple occasions.}




\subsection{Case I: Particle Dreams in Spherical Harmonics}
\label{ssection:particle}


\begin{figure}[h]
 \centering
 \includegraphics[width=\columnwidth]{figures/case_1.png}
 \vspace{-2em}
 \caption{Particle Dreams in Spherical Harmonics~\cite{sandin}.}
 \label{fig:cases_1}
 % \Description{Representative images of the three artworks: Upper left: Particle Dreams in Spherical Harmonics; Lower left: Smell Maps; Right: Machine Hallucinations.}
 \vspace{-0.5em}
\end{figure}

Particle Dreams in Spherical Harmonics is a Virtual Reality (VR) artwork produced by Dan Sandin and his team. As an internationally recognized pioneer of electronic art and visualization~\cite{sandin}, Sandin is the director emeritus of the Electronic Visualization Laboratory (EVL) and a professor emeritus in the School of Art and Design at the University of Illinois Chicago. With a background in physics, Sandin has dedicated his career to exploring the intersection of technology and art.
% , and has received numerous honors for his innovative contributions (\eg IEEE Virtual Reality Academy, the Rockefeller Foundation Media Arts Fellowship). As an artist, he has exhibited worldwide and received grants from globally renowned art foundations.
% and has received grants in support of his work from the Rockefeller Foundation, the Guggenheim Foundation, the National Science Foundation and the National Endowment for the Arts. His video animation Spiral PTL is in the inaugural collection of video art at the Museum of Modern Art in New York.
% Throughout his career, Sandin has received numerous accolades for his contributions to the field, including the IEEE Visualization Career Award and the SIGGRAPH Steven Anson Coons Award. His legacy continues to inspire new generations of artists and technologists who are exploring the convergence of art and technology in the digital age.
% His work encompasses a range of groundbreaking projects that have significantly influenced the field of artistic data visualization. 
In the 1970s, he developed the Sandin Image Processor, a highly programmable analog computer for processing live video feeds. It was one of the early devices that allowed artists to manipulate video data inputs in real-time, solving the problem of computer-graphics systems being too expensive and not easily accessible to most people~\cite{johnson2024electronic}.
% while also opening up new possibilities for creating dynamic art installations.
% He then worked with DeFanti to combine the Image Processor with real-time computer graphics and performed visual concerts, the Electronic Visualization Events, with synthesized musical accompaniment. 
Since the 1990s, Sandin and his colleagues began developing the CAVE (Cave Automatic Virtual Environment), a pioneering VR theater system that provides immersive experiences in a 3D space where computer-generated imagery is projected onto walls and floors.
% a pioneering VR theater system that provides immersive experiences. It allows users to step into a three-dimensional space where computer-generated imagery is projected onto walls and floors, creating a fully interactive environment. 


% Additionally, Sandin co-developed the Sayre Glove, the first data glove, which enabled users to interact with digital environments through hand movements, further bridging the gap between physical and digital art forms.
% In 1969, Sandin developed a computer-controlled light and sound environment, called Glow Flow, at the Smithsonian Institution and was invited to join the art faculty at the University of Illinois the same year. 

Sandin himself also used these technologies to produce a series of data artworks, particularly in the realm of scientific visualization. \autoref{fig:cases_1}, for example, is a VR artwork based on the ``physical simulation of over one million particles with momentum and elastic reflection in an environment with gravity.''~\cite{sandin}
% In the final scene there is a very realistic rendering of water with reflections, and lighting based on spherical harmonics. 
It also creates a multisensory experience by generating sound that is triggered and modified by the user-particle interactions in real-time.
A viewer commented that standing in the artwork ``was like standing in a rainstorm made of rainbow fragments, with the power to guide the storm by hand. It was unsettling, out-of-body, very trippy stuff, a powerful artistic experience.''~\cite{evl_vr}
% This work was premiered in January 2011 at the gallery@calit2 University of California San Diego, designed for exhibition in the Qualcomm Institute, UC San Diego division of Calit2’s StarCAVE and the CAVE2™ developed by the Electronic Visualization Laboratory, University of Illinois at Chicago.
When talking about the motivations of their work, Sandin put more emphasis on building a ``more effective communication medium and a much more effective way to visualize data''~\cite{evl_synthesis}.
% He believed artists should use (or build) the most advanced technologies of their time to create their art and would often point out that Renaissance painters made their own paints~\cite{johnson2024electronic}.
This is also the philosophy of his lab, that is, ``systems should be user-oriented (easy to use, easy to learn), low-cost, interactive, and real-time (to provide immediate feedback).''~\cite{johnson2024electronic}
As for himself as an artist, Sandin adopted a rather open attitude to creativity: ``About creativity—my personal view of it is kind of like I’m a pipe or conduit. And all this stuff just happens to be flowing through me because I’ve chosen to position myself in that flow. I have no problem with the word `creation' as long as people don’t lay too much molasses on it.''~\cite{vdb} 



\subsection{Case II: Smell Maps}
\label{ssection:smell}

\begin{figure}[h]
 \centering
 \includegraphics[width=\columnwidth]{figures/case_2.png}
 \vspace{-2em}
 \caption{Smell Maps~\cite{smellmaps}.}
 \label{fig:cases_2}
 % \Description{Representative images of the three artworks: Upper left: Particle Dreams in Spherical Harmonics; Lower left: Smell Maps; Right: Machine Hallucinations.}
 \vspace{-1em}
\end{figure}


Since 2010, British artist Kate McLean has been working on translating the sensed aspects of place into visualizations. Starting with the first smell map of Paris, she has created a set of smell maps in various cities (\eg in \autoref{fig:cases_2}, the first two maps show Edinburgh and New York, respectively).
She used olfactory walks to collect data by first selecting specific routes in the cities, and then recruiting volunteers for these walks. The characteristics of the smells (\eg name, intensity) were recorded by the volunteers using smell notes (\autoref{fig:cases_2}, right). She also designed activities such as ``smell catching'' (noticing distant, airborne smells when walking) and ``smell hunting'' (searching for the sources of smell) to spark participants' interest and sensational involvement during walks~\cite{mclean2019nose}. 
% They have also produced a toolbox called 'smellfie,' which provides a DIY guide for the mapping of urban smells and is available for public and educational download.
% McLean categorizes olfactory walks into individual walks, group walks, paired walks, and a toolbox (smellfie) for educational and public use. 


Smell maps were motivated by the urban research by Douglas Porteous and Charles Foster~\cite{mclean2019nose}, who offered profound reflections and criticisms on contemporary living spaces, such as our alienation from physical experiences and over-reliance on vision.
% Charles Foster~\cite{} criticized the overdependence of humans on sight, namely their inability to relate to the entirety of their environment.
% criticized that urban dwellers are increasingly alienated from physical sensory experiences. 
Porteous theorized the concept of the \textit{smellscape}, pointing out that ``like visual impressions, smells may be spatially ordered or place-related''. But unlike an ordered visual landscape, the smellscape is an emotive environment that is ``non-continuous, fragmentary in space and episodic in time and limited by the height of our noses from the ground, where smells tend to linger.''~\cite{porteous1985smellscape}
Influenced by these theories, McLean's smell maps have placed a strong emphasis on physical participation and personal interpretation. She based all her work ``on physical experiences, rather than algorithms,''~\cite{cnn} and visualized these individual, subtle, and subjective olfactory data through artistic expressions (\eg using colored spots and rippling circles to present the sources and diffusion of smells). 
When designing visuals, McLean did not seek singular, precise scientific results but rather aimed for ``a visual synthesis of the different experiences reported by smellwalkers...I am interested in negotiating different perceptions.''~\cite{atlas}
% While engaging in this art project, McLean has concurrently developed her academic research on smellscapes. 
In essence, she considered her practice and research to be qualitative~\cite{marie}: ``My aim is to celebrate and highlight the subjective elements of human perceptions of the smellscape...I never claim that the scents are objective, and my research to date indicates that it may not even be possible.''

% developing a language for olfactory cartography, creating an olfactory dictionary with descriptive terms, an olfactory color DNA that represents smells with colors, and summarizing experiences from nearly a hundred olfactory walks. 
% McLean designed a model for the diffusion of smells, first marking the source of each smell, representing the duration of the smell with the size and layers of concentric circles, and using the recorded wind direction and speed during experiments to calculate the perturbation of the circles. The final smell map can present the source and spread range of each smell, as well as the time of diffusion. 
% The word ‘smellscape’ has its roots in cultural geography (first used by Gade in 1984) and later theorised by Porteous whose seminal paper also indicated the challenges inherent in studying and recording sensuous olfactory worlds. Porteous’ conception of the smellscape is as the totality of the olfactory landscape in a specific place; including both background and episodic smells (his term for temporary or site-specific aromas). His contention is that, unlike an ordered visual landscape, the smellscape is an emotive environment that is ‘non-continuous, fragmentary in space and episodic in time’.



\subsection{Case III: Machine Hallucination}
\label{ssection:machine}

\begin{figure}[h]
 \centering
 \includegraphics[width=\columnwidth]{figures/case_3.png}
 \vspace{-2em}
 \caption{Machine Hallucinations~\cite{machine}.}
 \label{fig:cases_3}
 % \Description{Representative images of the three artworks: Upper left: Particle Dreams in Spherical Harmonics; Lower left: Smell Maps; Right: Machine Hallucinations.}
 \vspace{-2em}
\end{figure}

Machine Hallucination is a multi-series art project by Refik Anadol and his studio, utilizing big data and artificial intelligence (AI) to create immersive art experiences.
For example, \textit{Machine Hallucination - NYC}, as the first artwork in this series, employed machine-learning algorithms to process over 100 million photographs of New York. \textit{Machine Hallucination - Coral Dreams} was based on the training of more than 35 million images of coral. \textit{Machine Hallucination - Unsupervised} explored MoMA's vast collection, encompassing 150 years and nearly 140,000 art pieces. It was transformed into what Anadol called a ``living data sculpture'': a piece of artwork that is constantly changing, projecting an infinite number of alternative artworks that the machine creates in real-time across a giant media wall~\cite{yahoo} (\autoref{fig:cases_3}, left).
% % To create the collection, he and his team fed more than 138,000 images of individual works from MoMA’s archives - including paintings, performance art, video games and sculptures - into a machine-learning model~\cite{yahoo}.
% acquired as the MoMA’s permanent collection.
% Machine Hallucinations: Space, which Anadol describes as “a visual speculation of humanity’s historical attempts to explore the depths of space.” The piece uses millions of raw images from the Hubble Space Telescope and the International Space Station to teach the AI models. The AI transforms these beautiful images of the Earth, the universe, their colors, and their forms into what Anadol calls “data pigments” to create an animated image that morphs organically over time. 
%In Machine Hallucinations: Nature, Anadol uses 400 million publicly available photographs of flora and fauna to create a different form of “pigment.” These natural blocks are then animated by data of the wind and gust speed, as well as precipitation and air pressure, all captured from sensors in Las Vegas. Anadol likens the process to how Claude Monet was “inspired by the atmosphere and became this incredible impressionist painter.”
% The AI system is trained on this data, which enables it to generate new, unique visual patterns to "hallucinate" new imagery based on its learned patterns and associations. 
In a 2021 interview~\cite{momainterview}, Anadol said, ``In the past five years, we’ve trained more than 100 AI models, and used close to five petabytes of raw data. This is, as far as I know, one of the most challenging datasets ever used beyond specific research contexts, from clouds to national parks to cities to urban studies of Seoul, Stockholm, Berlin, Istanbul, New York, Los Angeles.'' 
% These projections or digital installations often appear as abstract, flowing forms or surreal landscapes that seem to emerge from the machine’s "imagination." The visuals are constantly evolving, reflecting the ongoing processing of data by the AI.
Characterized by blossoming colors, biomechanical shapes, and constantly evolving data patterns, these artworks have achieved immense success. They have been projected onto notable architectural landmarks, such as the Walt Disney Concert Hall, Casa Battló, and the 580,000-square-foot Las Vegas Sphere (\autoref{fig:cases_3}, right).


Anadol's artwork demonstrates a strong pursuit of machine aesthetics. In a TED talk~\cite{ted}, he described his motivation for creating such artworks: ``Can data become a pigment? This was the very first question we asked when starting our journey to embed media arts into architecture, to collide virtual and physical worlds. So we began to imagine what I would call the poetics of data.'' Specifically for Machine Hallucination, it is an ambitious experiment concerning whether machines can dream and an attempt to re-present how machines interpret vast amounts of data in its ``brain'' (``AI in this case is creating this pigment that doesn't dry, a pigment that is always in flux, always in change, and constantly evolving and creating new patterns.''~\cite{yahoo})
% `` I use data as a pigment and paint with a thinking brush that is assisted by artificial intelligence. Using architectural spaces as canvases, I collaborate with machines to make buildings dream and hallucinate.'' 
While his works are highly technical, Anadol believes: ``Artificial intelligence is a mirror for humanity...It’s all about who we are as humans.''~\cite{fastcompany2}
% Machine Hallucination is a tribute to nature using data (``We respect nature, we believe in nature. And for that reason, we transformed one of our largest datasets.''~\cite{fastcompany} 



% For me, art reflects humanity’s capacity for imagination. And if I push my compass to the edge of imagination, I find myself well connected with the machines, with the archives, with knowledge, and the collective memories of humanity.

% Machine Hallucination has been well-received in the art and technology communities for its innovative approach and aesthetic appeal. It has been exhibited in various international venues, contributing to ongoing discussions about the role of technology in contemporary art. Anadol’s work continues to push the boundaries of what is possible with data and AI, offering new ways to think about the intersection of technology and creativity.


\section{Design Features of Artistic Data Visualization}
\label{sec:space}

In this section, we analyzed \ncorpus artworks to outline a bigger picture of artistic data visualizations' design features. 

\subsection{Corpus Collection}

\begin{figure*}[b]
 \centering
 \includegraphics[width=0.94\textwidth]{figures/techniques.pdf}
 \caption{Left: All identified design techniques and their frequencies. Right: Examples of the artworks. (A) Agitato~\cite{agitato}, (B) Applying color palettes in paintings to scientific visualization~\cite{samsel2018art}, (C) Shan Shui in the World~\cite{shi2016shan}, (D) Bitter Data~\cite{li2023bitter}, (E) Decoding • Encoding~\cite{tibetan}, (F) Messa di Voce~\cite{messadivoce}, (G) Bion~\cite{bion}, (H) NeuroKnitting Beethoven~\cite{neuro}, (I) Climate Prisms~\cite{prisms}, (J) Oceanforestair~\cite{oceanforestair}, (K) Art of the March~\cite{protest}, (L) Decomposition of Human Portraits~\cite{face}, (M) \#home~\cite{home}, (N) The Sky is Falling~\cite{sky}, (O) Beyond Human Perception~\cite{plants}.}
 \label{fig:techniques}
 \vspace{-2em}
 % \Description{Left: All identified design techniques in artistic data visualizations and their frequencies. The techniques are organized into four categories: sensation, interaction, narrative, and physicality. Right: 15 images illustrating examples of the artworks.}
 % \vspace{-1em}

\end{figure*}

We began by collecting artworks from the IEEE VISAP (VIS Arts Program), which is a program associated with the top-tier visualization conference IEEE VIS. 
IEEE VISAP was started in 2013 and has run for more than ten years. 
% Artworks should be submitted along with textual explanations and will undergo a review process for acceptance.
Over the years, IEEE VISAP has become the leading event for artistic data visualization and has formed a representative collection of artworks.
% had a significant impact on the field of visualization by fostering cross-disciplinary collaboration, sparking new ideas, and expanding the horizons of creative data representation. 
We scraped all the accepted artworks of IEEE VISAP from its official website, resulting in a total of 231 non-duplicate artworks. After excluding 36 artworks that are now inaccessible on the web, we kept 195 artworks published between 2013 and 2023.
Overall, we believe that IEEE VISAP is a good source because: (i) It offers a publicly accessible dataset dedicated to artistic data visualization. (ii) Its review process ensures the artworks' relevance to data visualization. (iii) The inclusion of textual explanations or short papers by the artists provides essential materials for our analysis.
% However, we also realized that relying solely on IEEE VISAP has its limitations, as there may be some great artworks published outside this venue. 
We also searched venues such as the Info+ Conference, Tapestry Conference, EyeO Festival, and Information is Beautiful. However, although these venues feature active data artists, they either lack a dedicated track explicitly for artistic data visualization or do not include designers' explanatory materials.
In the end, we opted to use a snowballing technique to identify literature that introduces the practices of artistic data visualization and to include as many data artworks as possible.
% (\eg ~\cite{de2022data,kim2013topics,moere2012evaluating,judelman2004aesthetics,kim2010speculative,gough2014affective,khot2017edipulse})
As a result, we added 25 artworks to our corpus, bringing the total to \ncorpus artworks. The additional artworks were mostly published in non-VIS tracks.
% not an artwork (\eg three records are keynote talks and 14 records are theoretical papers). 
% However, we were unable to find the submitted materials of 19 artworks. 
% difference of these channels, clarify
% 4 keynotes



% many school info is missing
The publication time of these artworks spans from 1996 to 2023 (\includegraphics[align=b,scale=0.13]{figures/span.png}).
% The subjects of the collected artistic data visualizations cover a wide range of topics, such as environmental problems, urban studies, social issues, biology, computer science, sports, and news media.
% \textit{environmental sciences and ecology} (26), \textit{urban studies} (22), and \textit{social issues} (21) are addressed by most artworks. For example, Fig. \ref{fig:techniques} (C) is about the pollution of the ocean, and Fig. \ref{fig:techniques} (K) visualizes the data of carbon dioxide emission. Fig. \ref{fig:techniques} (G) recreates the landscape of New York. Fig. \ref{fig:techniques} (D) deals with the problem of sexual harassment in academia. Fig. \ref{fig:techniques} (J) is an artwork that visualizes deaths caused by drone strikes.
% Besides, there are four more fields that have been addressed by more than ten artworks in our corpus, including \textit{biology} (14), \textit{computer science} (10), \textit{art} (10) and \textit{news \& media} (10). For example, Fig. \ref{fig:techniques} (H) visualizes what machines see during a face recognition task.
% water pipelines in the US.
% Fig. \ref{fig:techniques} visualizes the music listening experience. Fig. \ref{fig:techniques} xxx social media.
% A small number of artworks deal with subjects in other fields such as language \& literature (8), astronomy (4), physics (4), and politics (2). 27 artworks did not specify the field in which they will be applied.
% In summary, the artworks in our corpus cover both humanities and scientific subjects and a substantial number of the artworks exhibit a strong sense of realistic issues and public engagement.
We in total identified 516 distinct authors of these artworks coming from 188 different affiliations, and their backgrounds are also diverse (see \autoref{tab:authors} for more details).
% \x{such as \textit{the University of California, Santa Barbara} (23), \textit{Massachusetts Institute of Technology} (21), and \textit{Northeastern University} (21). 193 authors (37.40\%) have an art/design background, such as \textit{Design} (51), \textit{Art and Design} (25), \textit{Art/Arts} (17), and \textit{Fine Arts} (16).  132 authors (25.58\%) come from engineering backgrounds, such as \textit{Computer Science} (70) and \textit{Computer Science and Engineering} (14). 
% For example, 12 authors come from the Department of Computer Science at the University of Calgary. 82 authors (15.89\%) come from cross-disciplinary departments such as \textit{Media Arts and Technology} (22), \textit{Urban Studies and Planning} (14), and \textit{Interactive Arts and Technology} (11).
% \w{For example, 16 authors from the Department of Media Arts and Technology at UC Santa Barbara contributed to the creation of 14 artworks} 66 authors (12.79\%) come from other disciplines such as \textit{Physics} (5), \textit{Information} (4), \textit{English Literature} (3), and \textit{Psychology} (2). the backgrounds of 43 authors (8.33\%) are unknown.}
Each artwork has an average of 2.38 authors. Among the \ncorpus works, 85 have only one author. Of the remaining 135 collaborative works, 87 (64.44\%) were created by cross-disciplinary teams.
Another interesting finding is that apart from reporting their official occupations (\eg professor, PhD student), 115 authors used personalized labels to define themselves, such as ``multimedia artist and roboticist'', ``an intermedia artist and an acknowledged pioneer'', ``artist, technologist'', and ``musician''.
% Catherine D’Ignazio~\cite{} defined herself as "a hacker mama, scholar, and artist/designer who focuses on feminist technology, data literacy and civic engagement". 
17 authors simultaneously hold academic and industry positions but place their industry identity ahead of their academic identity (\eg ``multidisciplinary digital media artist and university professor", "media artist and researcher'') showing that their role as practitioners is highly valued.


\begin{table}[h]
\centering
\vspace{-0.5em}
\caption{Affiliations and backgrounds of the authors from the corpus.}
\vspace{-1em}
% \small
\begin{subtable}[t]{0.56\linewidth} % 左表
% \caption{Affiliations of Authors}
\fontsize{6.8pt}{7pt}\selectfont
\centering
\begin{tabular}{@{}p{0.8\linewidth}p{0.12\linewidth}@{}}
\toprule
\textbf{Affiliation (Top 5)}                     & \textbf{Num} \\
\midrule
University of California, Santa Barbara & 23 \\
Massachusetts Institute of Technology   & 21 \\
Northeastern University                 & 21 \\
University of Texas, Austin             & 19 \\
University of Calgary                   & 14 \\
\bottomrule
\end{tabular}
\end{subtable}%
\hfill
\begin{subtable}[t]{0.41\linewidth} % 右表
% \caption{Backgrounds of Authors}
\fontsize{6.5pt}{7pt}\selectfont
\centering
% \small
\begin{tabular}{@{}p{0.5\linewidth}p{0.4\linewidth}@{}}
\toprule
\textbf{Background}                     & \textbf{Num} \\

\midrule
Art/Design        & 193 (37.40\%) \\
Engineering        & 132 (25.58\%) \\
Cross-Disciplinary & 82 (15.89\%) \\
Other Disciplines              & 66 (12.79\%) \\
Unknown            & 43 (8.33\%) \\
\bottomrule
\end{tabular}
\end{subtable}
\vspace{-2em}
\label{tab:authors}
\end{table}




% \begin{table}[h!]
% \caption{Author Information}
% \centering
% % \small

% \begin{subtable}[t]{0.48\linewidth} % 左表
% \caption{Affiliations of Authors}
% \fontsize{6pt}{6pt}\selectfont
% \centering
% \begin{tabular}{@{}p{0.79\linewidth}p{0.13\linewidth}@{}}
% \toprule
% \textbf{Affiliation}                     & \textbf{Num} \\
% \midrule
% University of California, Santa Barbara & 23 \\
% Massachusetts Institute of Technology   & 21 \\
% Northeastern University                 & 21 \\
% University of Texas, Austin             & 19 \\
% University of Calgary                   & 14 \\
% \bottomrule
% \end{tabular}

% \label{tab:institutions}
% \end{subtable}%
% \hfill
% \begin{subtable}[t]{0.48\linewidth} % 右表
% \caption{Backgrounds of Authors}
% \fontsize{6pt}{6pt}\selectfont
% \centering
% % \small
% \begin{tabular}{@{}p{0.45\linewidth}p{0.45\linewidth}@{}}
% \toprule
% \textbf{Background}                     & \textbf{Num} \\

% \midrule
% Art/Design        & 193 (37.40\%) \\
% Engineering        & 132 (25.58\%) \\
% Cross-Disciplinary & 82 (15.89\%) \\
% Other Disciplines              & 66 (12.79\%) \\
% Unknown            & 43 (8.33\%) \\
% \bottomrule
% \end{tabular}

% \label{tab:backgrounds}
% \end{subtable}
% \end{table}

% \renewcommand{\topfraction}{0.9}  % 最大允许占用页面顶部的比例
% \renewcommand{\textfraction}{0.1}  % 页面下方至少留出的文本比例



\subsection{Design Analysis}

% In this section, we perform design analysis on the corpus.





\subsubsection{Analysis Method}


We analyzed the design features of the artworks using open coding while referring to the artists' own explanations.
Two authors were responsible for the coding process, focusing on two main aspects~\cite{shi2021communicating,sarikaya2018we}: (i) What are the design intents of the artworks? and (ii) How are they designed? First, we familiarized ourselves with the common naming and categorization of visualization design intents and techniques proposed in prior design taxonomies (\eg ~\cite{lan2023affective}).
Then, we went through the artworks independently and generated codes to describe their design intents and techniques. Similar codes were categorized into groups. For example, we identified multiple channels of designing interaction, such as \textit{GUI interaction}, \textit{facial interaction}, and \textit{voice interaction}. Thus, these codes were grouped into a category called \textit{interaction}.
Next, we met to compare codes, discuss mismatches, and adjust inappropriate codes. For example, we initially used multiple codes to describe various materials used in the artworks, such as \textit{plastic}, \textit{metal}, \textit{yarn}, and \textit{glass}. During our discussion, we found that these codes were too detailed and difficult to enumerate exhaustively. Therefore, we consolidated them into a single code called \textit{physical materials}. Additionally, we observed that some artists (though not all) used higher-level terms to describe the fundamental concepts or philosophies behind their artworks. We agreed that this information is valuable and complements low-level design techniques. As a result, we added a new dimension called \textit{design paradigm} to our codebook and proceeded with another round of coding. In other words, our final taxonomy contains three main dimensions: design paradigms, design intents, and design techniques. After four rounds of meetings and coding, we achieved 100\% agreement on the coding scheme.
% To sum up, we coded the xxx data artworks in four dimensions: \textit{design paradigm} and \textit{design task} capture design intents at high-level and low-level perspectives, respectively; \textit{design genre} and \textit{design techniques} describe design outcomes from high-level and low-level perspectives, respectively.


\subsubsection{Design Paradigms}
\label{sssec:paradigm}
We identified 37 explicit mentions of high-level design paradigms. The most frequently mentioned paradigm is \textit{generative art} (N = 12), which refers to art created using autonomous systems or algorithms.
% Artists write codes to initiate the creation process, and the final outcome is partly unpredictable and influenced by chance or randomness.
Other mentions include concepts such as \textit{abstract art} (2), \textit{physical computing} (2), \textit{algorithmic art} (2), \textit{computational aesthetics} (2), \textit{digital art} (2), and \textit{digital fabrication} (2).
%\textit{ambient design} (2), \textit{computer art} (2), \textit{computer-generated art} (1), \textit{electronic art} (1), \textit{parametric design} (1), \textit{VR art} (1), \textit{wearable art} (1), and \textit{glitch art} (1). 
% Last, three design paradigms are \textbf{modality-centered}, including \textit{sonification} (11), \textit{physicalization} (4).
% Although they have subtle differences, the core emphasis is on using computer programs to drive artistic creation. 
For example, parametric design generates artworks through a series of parameters. Designers can flexibly adjust and control the parameters to explore a range of design variations. Glitch art is a paradigm that celebrates errors in computer programs and embraces the aesthetics of imperfections.
A common feature of the above paradigms is their emphasis on the role of computers.

We also identified another set of paradigms that are more human-centered, such as \textit{speculative art} (2), \textit{digital humanism} (2), \textit{participatory design} (1), \textit{aministic design} (1), and \textit{neo-concrete art} (1). For instance, speculative art explores alternative realities and futures, often through experimental practices within communities. Participatory design involves people actively in creating artwork to promote social inclusion, empower marginalized communities, and reflect their needs and values. 
When contextualized within art history~\cite{pooke2021art,hopkins2000after,perry2004themes}, nearly all the aforementioned paradigms fit the scope and trends of contemporary art.


% Moreover, concepts such as , \textit{glitch art}, \textit{speculative art}, \textit{wearable art}, \textit{neuroaesthetics}, \textit{digital humanism}, \textit{digital photograph} have also been mentioned, suggesting the deep interwoven of artistic data visualization with other subdomains of art as well as disciplines such as psychology, computer science, philosophy, and social science.







\subsubsection{Design Intents}
\label{sssec:intents}

We initially coded the design intents of artistic data visualizations based on Lan~\etal~\cite{lan2023affective}'s work, which identified ten types of design intents (\eg \textit{inform}, \textit{engage}, \textit{experiment}) in affective visualization design.
As a result, although all the ten intents are present in our corpus, we also identified five new intents (marked using * in the following text).
In terms of frequency, the most common intent is \textit{experiment} (N = 49).
The core spirit of this intent is to challenge traditional notions of data visualization and explore unconventional or novel representations.
Other design intents include \textit{inform} (39), \textit{engage} (36), \textit{re-present*} (36), \textit{provoke} (30), \textit{criticize*} (19),  \textit{equip*} (13), \textit{analyze*} (12), \textit{advocate} (10), \textit{socialize} (7), \textit{witness*} (6), \textit{archive} (5), \textit{commemorate} (3), and \textit{empower} (3). 
% For example, Fig. \ref{fig:techniques} (I) explores encoding data with typography. Fig. \ref{fig:techniques} (F) intends to provoke thinking about human-nature relationship. Fig. \ref{fig:techniques} (D) brings uneasy experiences to the foreground and empowers victims of sexual harassment in academia.
For example, 
%\autoref{fig:techniques} (\w{E}) is a work meant to \textit{criticize}. The artist stated: ``Water is often depicted as blue and beautiful, and the project problematizes this metaphor by delving into the human relationship to water and how we impact the water we need to live''. 
\autoref{fig:techniques} (A) \textit{re-presents} the experience of listening to music by transforming the artist's subtle emotions into animated metaphorical shapes. We use the term \textit{re-present} instead of \textit{represent} to highlight the artist's intent to capture and reveal something invisible while integrating their own interpretations.
\autoref{fig:techniques} (N) exemplifies the newly identified intent, \textit{witness}. To show the number of civilians killed by US drone strikes, the artist performed in a desert from dawn to dusk, commemorating each civilian with an earth mound, a white cloth, a stone, and a prayer. Viewers witnessed the ceremony through live streams.
% The third one is to \textit{analyze} (8). For example, Fig. \ref{fig:techniques} (B) applied the idea of glitch art to the design of node-link diagram and proposed an algorithm to facilitate the diagnosis of the ``glitched'' neurons in the brain.  
% The fourth one is to \textit{witness} (4). For example, Fig. \ref{fig:techniques} (J) ``invites her audiences to serve as witnesses and aids.''
As another example, the intent of \autoref{fig:techniques} (B) is mainly to \textit{equip} artists with a tool for coloring scientific visualizations using expressive palettes extracted from paintings. This also resonates with the case of Sandin (\autoref{ssection:particle}) and previous findings that artists sometimes write software themselves or participate in software development~\cite{li2021we}.
% simulate, represent, recreate, reveal


% \subsubsection{Design Genres}

% Among the 178 data artworks we collected, \textit{installations} make up the largest part (54), followed by \textit{interactive interfaces} (52), \textit{static images/paintings} (28), \textit{artifact} (21), \textit{videos} (18), and \textit{events} (3). 
% For example, Fig. \ref{fig:techniques} (C) is a large installation placed in a museum.
% Fig. \ref{fig:techniques} (D) is a website where users can browse and interact with the testimony of the victims of sexual harassment.
% Fig. \ref{fig:techniques} (F) visualizes the change of sky as a static picture.
% Fig. \ref{fig:techniques} (K) uses fiber material to show climate data.
% Apart from these genres, we identified another new genre, namely the \textit{performance} where the artists perform as actors or actresses on stages or in outer environments. 
% For example, Fig. \ref{fig:techniques} (J) is a performance where viewers watched the artist placing stones on the ground, each representing a person killed by drone strikes.
% Besides, 12 works claimed algorithms/toolkits as their major contribution so that the output forms of these works are flexible. 
% For example, the algorithm proposed by Fig. \ref{fig:techniques} (B) can be used to generate both static and dynamic visualizations.



\subsubsection{Design Techniques}
\label{sssec:techniques}
A total of 32 design techniques were identified. The 32 design techniques were further grouped into four categories: \textit{sensation}, \textit{narrative}, \textit{interaction}, and \textit{physicality} (see \autoref{fig:techniques}).

\textbf{Sensation}. This category primarily focuses on creating various visual, auditory, olfactory, and other sensory effects. Relevant techniques include the use of \textit{color}, \textit{imagery}, \textit{shape/pattern}, \textit{sound/music}, \textit{animation}, \textit{style}, \textit{light}, \textit{layout}, \textit{typography}, \textit{temperature}, \textit{smell}, and \textit{taste}. 
For example, \autoref{fig:techniques} (A) transformed the evolving music listening experience into a generative artwork composed of vibrant colors, organic shapes, and animation.
% In \autoref{fig:techniques} \x{(B)}~\cite{microbiome}, the artist used metagenomics to generate colorful self-portraits of the microbial communities inhabiting his body. 
% \autoref{fig:techniques} \w{(C)} is an interactive installation designed to engage users in Chinese calligraphy. The written characters were transformed into bamboo leaf-styled fonts. Animation was generated to sway the characters gently, reminiscent of bamboo leaves rustling. The music and colors were also carefully selected to help create an ancient atmosphere.
\autoref{fig:techniques} (B) applied color palettes extracted from paintings to scientific visualizations.
% redesigned node-link diagrams, using a glitch-style encoding to represent anomalies.
\autoref{fig:techniques} (C) transformed the map of New York City into a style reminiscent of traditional Chinese painting.
\autoref{fig:techniques} (D)~\cite{li2023bitter} transforms 100,000 distress postings using data edibilization. The data was mapped to the bitterness and color of tea, enabling users to observe, smell, and taste the distress on social media.
% Fig. \ref{fig:techniques} \x{()} transferred the style of ancient Chinese paintings to the map of New York. Fig. \ref{fig:techniques} \x{()} used light to xxx. 
% Fig. \ref{fig:techniques} \x{()} animation of water.

% According to our analysis, the idea of \textit{sonification} has been mentioned most (explicitly mentioned by ten artworks). The core spirit of sonification is to use sound or auditory elements to translate non-auditory information into audible forms. On the one hand, sound is viewed as a novel medium that expands the possibilities of artistic expression. On the other hand, the idea of sonification is also fueled by the need for accessibility in design as it can make information more accessible to individuals with visual impairments or disabilities.



\textbf{Interaction}. 
Among interactive techniques, GUI interaction, such as allowing users to click on or scroll a website, mobile app, or tablet, is most common (\eg \autoref{fig:techniques} (E)). 
Other relevant techniques include \textit{body interaction}, \textit{facial interaction}, \textit{touch/manipulate}, \textit{gesture interaction}, \textit{biomimicry}, 
\textit{AR/VR/MR}, \textit{voice interaction},  \textit{draw/sketch interaction}, and \textit{physiological interaction}. 
For example, \autoref{fig:techniques} (F) is a performance driven by voice interaction. The performers made various sounds and the sounds were transformed into the real-time animated visualization in the background. 
% designed to provoke questions about the meaning and effects of speech sounds, speech acts, and the immersive environment of language.
\autoref{fig:techniques} (G) is composed of hundreds of ``bions'' (individual elements of primordial biological energy) programmed according to biomimicry sensing. When a viewer approaches the installation, bions quickly communicate to each other, but eventually they become accustomed to the stranger's presence and respond as if he/she is part of their ecosystem.
\autoref{fig:techniques} (H) is a neuroknitting artwork that utilizes people's brainwave data when listening to music to drive the knitting machine.
% In \autoref{fig:techniques} \x{(E)}~\cite{endless}, users can immerse themselves in a simulated 3D fluid environment through VR.


\textbf{Narrative}. This category encompasses storytelling methods such as \textit{metaphor/analogy}, \textit{story plot}, \textit{collage}, \textit{perspective shift}, \textit{co-design/user generated content}, and \textit{personalization}. 
For example, 
\autoref{fig:techniques} (I) uses yarn to represent cumulative emissions over years; as the emission gets larger and larger, the yarn gets tighter and tighter, metaphorically mimicking a sense of suffocation.
\autoref{fig:techniques} (J) tells a data story about the shrinking of Arctic ponds by crafting a prepared storyline.
\autoref{fig:techniques} (K) is a website that collages more than 6,000 signs of the Boston Women’s March, which silently demonstrates the collective efforts of protesters. 
% As introduced by the artists~\cite{protest}, "The collection is a rich and inclusive snapshot and record of the extensive range of issues, emotions and visual expressions at the march. The signs are handmade and unique, but also connected in a rich web of cultural references, themes, memes, and visual techniques and styles."
\autoref{fig:techniques} (L) employs a perspective shift to visualize human faces from an unconventional perspective—the eye of a deep neural network.
\autoref{fig:techniques} (M) invited visitors to share keywords about their homes. The keywords were used to filter a live Twitter stream, and the locations of these tweets were 3D printed as physical maps, which were personalized to each visitor.
% Similarly, Fig. \ref{fig:techniques} } visualizes images of human faces seen by computer vision algorithms. 
% Fig. \ref{fig:techniques} \x{()} is a visualization about cancer genomes. The artists invited people who had this cancer to participate in the design process and draw their own genomes, thus partly relinquishing the narrative right to users.

% used the metaphor of knots to represent the uneasy stories about harassment. Fig. \ref{fig:techniques} (F) creates a montage of photographs to break the linearity of time and narrative. 


\textbf{Physicality}. This category contains techniques that augment physical experiences, including \textit{physical materials} (\eg yarn, metal), \textit{large immersive screen}, \textit{situated environment}, 
\textit{non-human agents (\eg \textit{plants}, \textit{robots}, \textit{bacteria}).}
For example, the two aforementioned artworks, \autoref{fig:techniques} (I, M), use yarn and 3D printing to physicalize data, respectively.
% \autoref{fig:techniques} \x{(K)}~\cite{hongkong} combines city map visualization with the cultivation of bacteria to reflect on the expansion of Hong Kong through the lens of microorganisms.
\autoref{fig:techniques} (N), as introduced earlier, was an art performance conducted at a situated location, a desert, to provide viewers with an authentic sense of the environment affected by drone strikes.
Lastly, several artworks also utilize non-human agents, including entities that perform tasks or functions typically associated with humans. For example, \autoref{fig:techniques} (O), for example, played music to plants; sensor data from the plants was collected to explore the connection between technology and plant life.
% In Fig. \ref{fig:techniques} \x{()}, music was played to plants, and sensor data was collected from the plants. This exploration delves into the interaction between sound and plant life, offering insights into the connection between nature and technology.

\subsection{Observations}

Although the design of artistic data visualization shares some commonalities with other realms (\eg techniques like metaphor, story plot, personalization, and GUI interaction have also been found in narrative visualization~\cite{segel2010narrative,shi2021communicating,lan2022negative}; the appeal to sensation is also common in affective visualization design~\cite{lan2023affective}), it also exhibit distinct features.
First, artistic data visualization can be fundamentally shaped by high-level design paradigms. These paradigms act as the creative lenses that artists use to interpret the world around them, serving as the cornerstone of artistic decision-making. Yet, they are seldom identified in general visualization design research. This also demonstrates artists' emphasis on ideas and concepts.
% Although only a subset of data artworks in our corpus explicitly articulated their design paradigms with specific concepts, it is common for artists to explain their artistic inspiration and the thoughts behind their work. For instance, when explaining why they cultivated bacteria to ``grow'' a map of Hong Kong, the artists emphasized: ``We present a posthumanist idea of defining mingling spaces with microorganisms.'' 
Second, although artistic data visualization shares some design intents with prior studies, it exhibits more categories and different distributions in frequency of these intents. On one hand, the exploration of novel forms of expression is highly valued by data artists, as evidenced by the highest frequency of \textit{experiment}, a distinctive aspect that sets it apart from other fields. Additionally, within the design intents of artistic data visualization, those that convey strong opinions (\eg \textit{advocate}) are less prevalent. In contrast, higher-ranking intents (\eg \textit{inform}, \textit{engage}, \textit{provoke}) tend to be more implicit, subtle, and thought-provoking. This characteristic aligns well with the traits of contemporary art, which allows the audience significant freedom to interpret the work on their own terms~\cite{pooke2021art}.
Last, regarding specific design techniques, artistic data visualization places a strong emphasis on sensory richness and physicality, with many artworks taking the form of physical installations or artifacts (which is very different from the visualization design genres identified before~\cite{segel2010narrative}). Among current fields, affective visualization design shows the greatest technological affinity with artistic data visualization. However, when examining the distribution and frequency of specific techniques, interactive technologies in artistic data visualization are more prominent, manifested through various body, face, and gesture-based interactions, as well as biomimetic methods. Meanwhile, the use of physical materials is more diverse and pronounced, incorporating a wider range of materials to encode data, such as metals, furs, food, and even microorganisms.







\section{Perspectives from Data Artists}
\label{sec:interview}

Next, to better understand the underlying considerations behind artistic data visualization, as well as to gain firsthand insights to cross-validate our previous findings, we conducted in-depth interviews with twelve data artists.


\subsection{Participants and Process}
\label{sssec:interviewprocess}

We invited participants through: (i) sending interview invitations to the authors of IEEE VISAP projects, and (ii) reaching out to practitioners who self-tagged as data artists on social media.
A total of 12 data artists accepted our invitation, including 5 females and 7 males. Their ages ranged from 23 to 42 and were diverse in job and educational backgrounds (see \autoref{tab:participants}). 
% xx (xx\%) were xxx, xx (xx\%) were xxx, xx (xx\%) were xxx, and xx (xx\%) were xxx.
% xx of them have artworks recognized by professional events of artistic data visualization (\eg accepted by IEEE VISAP) and xxx have xxx. 

% \begin{table*}[htbp]
% \centering
% \fontsize{7.5}{9}\selectfont
% \begin{adjustbox}{width=0.7\textwidth}\begin{tabular}{llllll}
% \toprule
% ID &	Sex &	Age &	Job &	Educational Background \\
% \midrule
% P1	&	M	&	25	&	Research Assistant	&	Visual Design	\\
% P2	&	M	&	25	&	Product Manager	&	Industrial Design \& Data Visualization \\
% P3	&	M	&	38	&	Creative Technologist	& Computer Science \& Visual Design	\\
% P4	&	M	&	40	&	Lecture \& Entrepreneur	&	Fine Arts	\\
% P5	&	M	&	27	&	PhD Student	&	Architecture \& Computer Science	\\
% P6	&	F	&	27	&	Postdoctoral Researcher	&	Computer Science	\\
% P7	&	F	&	27	&	UX Designer	&	Digital Media	\\
% P8	&	M	&	23	&	Graduate Student 	& Digital Media	\\
% P9	&	F	&	42	&	Associate Professor	&	Interior Design \& Multimedia	\\
% P10	&	F	&	32	&	Assistant Professor	& Communication Design \\
% P11	&	M	&	28	&	Graduate Student	&	Industrial Design	\\
% P12	&	F	&	35	&	PhD Student \& Artist	&	Media Art	\\
% \bottomrule
% \end{tabular}
% \end{adjustbox}
% \caption{Information of the interviewees. All participants self-identified as data artists while performing their works.}
% \label{tab:participants}
%  \Description{Information of the 12 interviewees. The table contains five columns: ID, Sex, Age, Job, Educational background.}
% \vspace{-2em}
% \end{table*}


The interviews were semi-structured. We prepared a set of questions in advance, which can be categorized into four parts: (i) the creation of artistic data visualizations (\eg ``How did you come up with the idea of this project?'', ``How was your design process?''), (ii) the understanding of artistic data visualization (\eg ``What do you think is the most prominent feature of artistic data visualization?'', ``What distinguishes artistic data visualization from other types of data visualization''), (iii) the response to potential critiques (\eg ``If someone challenges the accuracy or efficiency of your visualization, how would you respond?''), (iv) challenges and expectations (\eg ``Have you ever met any challenges?'', ``How do you envision the future of this field?''). 
Before the interviews, we conducted background research on the participants, including their education, key works, achievements, and publications, to ensure the interviews were meaningful and relevant to their experiences and expertise.
During the interviews, we first asked the participants to introduce themselves, as well as their experience with artistic data visualization briefly as a warm-up. Next, we asked the aforementioned interview questions surrounding their representative data artworks. Depending on their responses, we asked follow-up questions to dig deeper into interesting points brought up by them (\eg ``Could you please elaborate on that point?'', ``Can you provide an example to illustrate?''). Each interview lasted about one hour and the interview process was audio recorded with the participant's consent.


% We obtained more than more than 850-minute audio recordings and about 220,000-word transcriptions from the interviews. 
To analyze the data, we followed the research methodology suggested by thematic analysis~\cite{braun2022thematic}.
Two authors first read through the transcriptions independently to familiarize ourselves with the data and took notes on initial observations.
Then, we coded the transcriptions with the goal of identifying the responses to the four aforementioned research questions. 
Next, we grouped related codes together to form themes, cross-checked each other's codes and themes, and marked disagreed codes until reaching 100\% consensus. 
% These categories should encapsulate the main topics discussed during the interviews. 
% For example, with regard to the process of creating artistic data visualization, we found several distinct types of motivations. Therefore, we grouped similar pipelines and summarized them as higher-level patterns.  


\begin{table}[t!]
\fontsize{6.8pt}{7.5pt}\selectfont
%\scriptsize
\centering
\caption{Information of the interviewees. }
%All participants self-identified as data artists while performing their works.
\label{tab:participants}
\vspace{-1em}
\begin{tabularx}{\columnwidth}{p{0.3em}p{0.3em}p{0.5em}p{10.3em}X}
\toprule
ID &	Sex &	Age &	Job &	Educational Background \\
\midrule
P1	&	M	&	26	&	Artist	&	Visual Design	\\
P2	&	M	&	26	&	Product Manager	&	Industrial Design \& Data Visualization \\
P3	&	M	&	39	&	Creative Technologist	& Computer Science \& Visual Design	\\
P4	&	M	&	41	&	Art Studio Head	&	Fine Arts	\\
P5	&	M	&	28	&	PhD Student	&	Architecture \& Computer Science	\\
P6	&	F	&	28	&	Postdoctoral Researcher	&	Computer Science	\\
P7	&	F	&	28	&	UX Designer	&	Digital Media	\\
P8	&	M	&	24	&	Graduate Student 	& Digital Media	\\
P9	&	F	&	43	&	Associate Professor in Art	&	Interior Design \& Multimedia	\\
P10	&	F	&	33	&	Assistant Professor in Art	& Communication Design \\
P11	&	M	&	29	&	Graduate Student	&	Industrial Design	\\
P12	&	F	&	36	&	Artist	&	Media Art	\\
\bottomrule
\end{tabularx}
\vspace{-1em}
\end{table}


%  \Description{Information of the 12 interviewees. The table contains five columns: ID, Sex, Age, Job, Educational background.}
% \vspace{-2em}





\subsection{Findings}

\subsubsection{Creation of Artistic Data Visualization}
We found that artists exhibit some common patterns when ideating and making artistic data visualizations.

\textbf{Motivation.}
We identified four modes of motivation for creating artistic data visualization. 
The first mode is \textbf{value-driven} (P3, P5, P9, P10, P12), which is often based on the artist's high-level values or deep thoughts. For instance, P3 mentioned that his art project originated from his philosophical contemplation of the relationship between humans and nature: ``\textit{A pivotal moment occurred during my journey from Washington to New York, when I was inspired by the mountains outside the window. In contrast, the cityscape and industrial facilities we usually live with are very modern and cold. So, I wondered if I could create a poetic geographical space visualization.}''
The second mode is \textbf{interest-driven} (P6, P8, P10, P11*2 times (corresponding to 2 different works)), which is more personal and related to specific living experiences. For example, P6 is a visualization researcher who has had an interest in art and literature since childhood (``\textit{I have always had a strong interest in the humanities}''). Therefore, after entering the field of visualization, she has always wanted to use visualization techniques to convey the rhythms and emotions found in literature.
The third mode is \textbf{reality-driven} (P1, P7, P9, P11). P7's artwork is about preserving Tibetan calligraphy. She said, ``\textit{We visited the inheritors of Tibetan calligraphy and learned about its long and glorious history. However, very few people are currently aware of this heritage or involved in its preservation. Therefore, we decided to present the unique beauty of this cultural heritage in a visual form.}''
The last mode is \textbf{client-driven} (P1, P2, P4). P4 is a data artist who has his own art studio. His recent project, which involves animating and visually representing ocean tide changes using 3D particle visualization, is based on the specific requirements of a collaborator. Similarly, P2 also mentioned that his art project was a ``\textit{commissioned work}'', meaning that the initial idea had already been determined by someone else.
% and the artist's main responsibility was the creative execution
% Apart from the last mode, which is driven more by commercial interests, the first three modes are centered on the intrinsic aesthetic and expressive demands of the artists.

\textbf{Workflow.}
The workflow of creating artistic data visualizations mainly falls into two categories: input-driven and output-driven. The \textbf{input-driven} workflow is more similar to the classic model of information visualization, which follows an input-output pipeline. 
% ~\cite{card1999readings}
Artists first obtain a dataset, then analyze it, and create visualizations. Artists often do not know in advance what the data will ultimately lead to. For instance, when working on an art visualization project related to social media, P5 collected data from Twitter after having a preliminary idea, and then tested various data dimensionality reduction algorithms and visualization layouts, finally selecting the most satisfying version.
In contrast, in the \textbf{output-driven} workflow, artists first conceive a mental image of the desired outcome, such as the overall aesthetic style and how the visualization will look like, and then utilize data as a medium to realize it. For instance, P3 had already determined to re-present modern maps using a poetic, painting style before actually beginning to collect and process data. Similarly, when working on her data sonification project, P10 drew inspiration from Kandinsky, who expressed music through points, lines, and planes. She decided to also visualize sound data as geometric shapes before proceeding to collect the necessary data. In terms of frequency, the output-driven workflow (P1, P3, P4, P7, P9, P10, P11*3 times (3 different works), P12) was more common than the input-driven workflow (P1, P2, P5, P6, P8). This finding also resonates with previous research~\cite{tandon2023visual} that artists are especially adept at visual tasks, when compared to mathematicians or computer scientists.
% In line with Tandon~\etal~\cite{tandon2023visual}, who found that artists excel more in visual tasks compared to mathematicians/computer scientists, 
% P4 said, ``\textit{my client may have an effect they want, such as 3D effects, and cool animation of particles.}''
However, despite differences in the overall workflow, all the artists we interviewed mentioned that they iteratively adjust and refine their work until it reaches a satisfactory point. 
% This often involves updating or filtering data based on the design, as well as adjusting design elements based on data patterns.


\subsubsection{Understanding of Artistic Data Visualization}
\label{sssec:understanding}

Next, we performed an analysis of the participants' descriptions regarding their understanding of artistic data visualization. To delve deeper into this important issue, we employed a richer array of analysis methods. First, we extracted relevant sentences from the participants, conducted word segmentation, synonyms detection, and frequency statistics to identify high-frequency keywords. When analyzing these keywords, close reading was also utilized to ensure a detailed examination of their original contexts.
% , aiming to uncover its deeper meanings, implications, and discourses. 
% we also returned to the text to double-check their original expressions. Specifically, 
As a result, the most mentioned words include \textit{express/convey/communicate} (N = 24), \textit{emotion/feeling/subjectivity} (N = 24), \textit{story/narrative} (N = 11), \textit{reflection/critical thinking} (N = 10), \textit{concept/idea} (N = 6), \textit{purpose/intent} (N = 4).
For example, P2 thought artists ``\textit{try to make inherently emotionless data convey emotional effect.}'' P8 said, ``\textit{I believe good data art should be engaging, offering more feelings or allowing people to see things from different perspectives, rather than merely enhancing efficiency.}''
P4 believed that artistic data visualization ``\textit{is not meant to provide a clear answer or a specific analysis of the data; art often involves speculation and reflection.}''
There were also keywords, although mentioned less frequently, that conveyed very interesting values and attitudes, such as \textit{freedom} (N = 2), \textit{openness} (N = 2), and \textit{ambiguity} (N = 2).
For example, P6 said, ``\textit{Art tends to be more divergent and doesn’t have a strong, specific purpose; it requires some ambiguity. We just place it there and let people interpret it however they wish.}'' P12 expressed her view on art with a highly concise statement: ``\textit{Aesthetics is freedom.}'' In her view, artists always strive to break existing rules or constraints, seeking and defending the space for free expression.

These keywords along with their original expressions have some obvious common characteristics.
First, they all involve a \textbf{strong emphasis on human agency}. 
% Whether it is subjective emotion or the expression of attitudes, they are all imbued with human feelings, understanding, and judgment. 
% This indicates that artistic data visualization has a strong tendency towards humanity in its intrinsic pursuit, thus offering further empirical support for the argument that artistic visualization is distinct from pragmatic visualization driven primarily by scientific discourse. 
In contrast to science, which prioritizes objectivity, generalizability, and precision~\cite{brown2001art}, artists are more inclined to embrace subjectivity, individuality, and ambiguity.
Second, we noticed that the way artists described artistic data visualization does not focus on specific forms, nor does it emphasize classic aesthetic standards such as harmony or vividness. Instead, they all emphasized the artist's concepts and expressions, demonstrating a \textbf{strong contemporary art characteristic}. 
% in the classical era, fine arts were clearly defined as painting, drawing, sculpture, and artworks were concrete objects used to imitate the natural world. However
As introduced in \autoref{ssec:contemporary}, in contemporary times, ``art as imitation'' has been replaced by ``art as expression'' and ``art as concept/idea''~\cite{pooke2021art}. Art is no longer confined to form but is the external manifestation of human concepts and thoughts. 
% Our interview results not only echo these trends but also confirm that data art remains deeply contextualized within the art history and contemporary art discourse.
In general, the perspectives offered by the data artists in our interviews resonate well with the design taxonomy in \autoref{fig:techniques}, and also provide us with a deeper understanding of the underlying logic of the identified techniques (\ie sensation, interaction, narrative, physicality). For example, the emphasis on subjective experience enhances artists' interest in stimulating human senses. The need to communicate ideas and concepts makes them adept at using narratives and eager to experiment with new modes of expression, stimulating the exploration of interactive technologies and various physical materials.
% Participants in our interviews thought that the most prominent feature of artistic data visualization is its concentration on \textbf{aesthetics}, \textbf{humanity}, \textbf{narrative}. 


% P10 thought that "\textit{compared to traditional visual communication, data visualization places greater emphasis on graphic design and layout, striving for beauty and artistry}."
% P9 thought ``\textit{narrative is the core of artistic visualization}'', and P3 emphasized that ``\textit{humanistic sensibilities and aesthetics are central considerations in the creation of visual works}''.


% Such critical thinking can lead to artworks that are sharp and powerful. For example, during the interview, P4 mentioned artist Simon Weckert's famous work, \textit{Google Maps Hack}~\cite{}, which played a dark joke about data. Simon walked back and forth along an empty street while pulling a cart containing 99 smartphones, effectively causing a false traffic jam on Google Maps. "\textit{It is very simple but mind-blowing, prompting society to reconsider our technological systems and pushing Google to make changes. I believe this exemplifies the power of art.}"

% \underline{Application scenarios.} 
% We identified four main application scenarios of artistic data visualization. The first one is \textbf{art exhibitions}. Data artists often exhibit their work in places such as museums, galleries, and public spaces (\eg parks, public squares) to attract passers-by. For example, P1 planned to physicalize his data artwork using 3D printing and exhibit it at a gallery "\textit{to let more people see and interact}". P9 mentioned that one of her data artworks has been collected by a renowned art gallery and has been continuously displayed in a showcase.
% % As introduced by Px, a creative technologist that serves many museums, "\textit{some exhibitions are more commercial and you need to buy tickets, while some are free, mainly to provide the public with more venues for art appreciation}").
% The second one is \textbf{marketing and advertising}. As introduced by P3, "\textit{sometimes businesses may need to display some cool artworks in their shop windows or building screens to attract consumers or to reflect the brand's youthfulness and vitality...I have worked on such projects for many years, all based on real user data and sales data, to generate artistic visualizations. For businesses, the concept of data-driven digital art can be more appealing than just traditional artworks.}" 
% P5 mentioned that a commercial company approached her, wanting to transform her art pieces into saleable products.
% The third one is \textbf{media and mass communication}.
% % to raise awareness about xxx and tell compelling stories .
% For example, the data artwork by P1 has been forwarded by many social media accounts of the governments and achieved more than 100 million views. P3 recalled several data artworks that have been published by the New York Times and National Geographic.
% The fourth one is \textbf{education}. P2 mentioned that his artistic data visualization is going to be included in a popular science book and he thought "\textit{artistic charts are more captivating and can attract readers to flip through the book, possibly more suitable for popular science than conventional charts}".


\subsubsection{Response to Critiques or Doubts}
As introduced in \autoref{ssec:artisticvis}, the controversy surrounding artistic data visualization mainly concerns its readability and reading efficiency. In our interviews, while most participants acknowledged that such critiques pose a challenge for artistic data visualization, they also provided some justifications.

% \ul{The balance between aesthetics and functionality.}
% While all participants expressed their desire for their work to be seen and understood by a wider audience, they agreed that artistic practice does entail more subjectivity and uncertainty. P6 pointed out, "\textit{Compared to general visualization, artistic data visualization indeed requires users to read data in a more abstract and unconventional way.}" During the interview, when asked how she would assess the actual impact of her work, P7 fell into contemplation and said, "\textit{Actually, sometimes I also wonder, is my work really useful? Or, how can it be more useful?}" 
% % P4 said, "It's truly a balancing act, primarily depending on what your use case is and what the artist themselves are striving for." 
% P5 thought there's no need to pursue universal appeal or understanding because "\textit{some people are more rational and some are emotional. Everyone has different tastes. If we can actually attract those who are willing to understand us and able to understand us, that's already good. Of course, it's not easy, so artists need to engage in a lot of thinking and experimentation.}"


\textbf{Artistic data visualization adopts an alternative mechanism of communication.}
5 participants thought that artistic data visualization achieves communicative goals through engaging viewers on a deeper level. For example, P4 argued for a broader understanding of efficiency: ``\textit{If our goal is to see precisely what a data point's value is, artistic data visualization is indeed inefficient...but in reality, the audience is probably not professional data analysts and they don't care about data precision. Instead, they care about what the data conveys and expresses. In that sense, artistic data visualization is efficient. For instance, through audiovisual elements, I can quickly immerse you in a context and convey a message.}''
P5 thought artistic data visualization and conventional statistical charts each have their merits: ``\textit{It's just that the way they convey information is different. With statistical charts, you need to decipher the data and get insights, almost like a bottom-up approach. However, artistic data visualization can directly provide you with messages and insights. If you feel attracted, you can then explore the data within. It's more of a top-down approach.}''
P6 thought there's no need to pursue a universal mode of data communication because: ``\textit{Some people are more rational and some are emotional. Everyone has different tastes. If we can attract those who are willing to understand us and able to understand us, that's already nice.}''
% Of course, it's not easy, so we artists need to engage in a lot of thinking and experimentation.


\textbf{Clarity and efficiency are not necessarily sacrificed.}
7 participants stated that their emphasis on subjectivity and humanity does not mean they \textit{only} care about these aspects. Most participants recognized the importance of data precision, particularly at the stages of data collection, preprocessing, and analysis. For example, P5 said, ``\textit{We invested a considerable amount of effort in data scraping and cleaning, just like the general process in data science. You need to address missing values, outliers, and ensure the data is correct.}'' P4 thought that ``\textit{data art should be supported by data; if the data is fictitious, it shouldn't be counted as data visualization.}'' 
% P2 agreed that ``\textit{while I may incorporate subjectivity in the visual presentation, I believe ensuring the accuracy of the underlying data is still a bottom line...after all, you can't deceive your audience.}''
Regarding understandability, P1 said, ``\textit{I would carefully consider how users will read and interact with my visualizations. I don't want to create a beautiful non-sense.}''
P3 mentioned that when designing installations for museums, his team would conduct user interviews in advance or perform A/B tests, and ``\textit{if many users failed to understand, we would make design adjustments.}''
P4 thought there were both good and bad designs in artistic data visualization: ``\textit{If a work is completely unreadable, it might indicate poor design. Conversely, if designers can provide readers with legends and reading guidance as much as they can while maintaining aesthetics and novelty, artistic data visualizations can of course be readable.}''
As both a data artist and visualization researcher, P6 said, ``\textit{I do consider users, but any visualization involves trade-offs. Even when creating visual analytics systems, sometimes you sacrifice some utility for innovative visual effects...I don't think artistic data visualization is that special.}''
% is the only artist who said that if users truly don't understand her work, she doesn't want to make compromises. However, at the same time, she also expressed her expectation for users to appreciate her work and argued



\subsubsection{Challenges and Expectations}
\label{sssec:expectations}
Lastly, participants reflected on the challenges and their expectations.

\textbf{Facing a bottleneck in skills or creativity.}
First of all, 7 participants mentioned that they faced a bottleneck in issues such as handling big data, crafting a narrative, and thinking of novel ideas. For example, P7 thought ``\textit{conceptualizing a project is challenging}'', and ``\textit{another major challenge is in storytelling...crafting a cohesive narrative is tough.}''
P4 reflected that he would often ``\textit{draw inspiration from the design ideas of outstanding peers, but original and highly innovative ideas totally came up with by myself are not that many.}'' 
P10 said, ``\textit{We currently don’t have access to particularly professional data, and we definitely lack the tools to work with such data.}''
P1 reflected that ``\textit{at present, my work is still largely targeted at the general public and remains a bit shallow. I try to dig deeper...However, if I want to pursue this, I definitely need to explore the deeper scientific insights behind the data.}''
% P4 mentioned, ``\textit{I feel it necessary for artists to continuously monitor the latest technological trends and societal issues to keep sensitive and empathetic to art subjects.}'' 
% P9 mentioned that the emergence of AI drawing has indeed created a sense of crisis for her. P12 envisioned that the advent of AI would force data artists to pursue newer, more avant-garde aesthetics: ``\textit{How can art maintain its identity when AI can easily realize a variety of artistic styles? In order to defend humanity, future art will undoubtedly move towards greater rebellion...data-driven art is at the forefront of this movement.}''

\textbf{Insufficiency of interdisciplinary collaboration \& understanding.}
Relating to the previous issue, 6 interviewees expressed the desire to foster more cross-disciplinary collaborations, such as the hope for scientists to join and enhance professional knowledge of data (P1), and to collaborate with teams specialized in data processing (P10). 
% For example, P7 mentioned that her team includes illustrators, interaction designers, and engineers; therefore, ``\textit{we met every week and iterated the design multiple rounds to ensure that the work is both good looking, sound in the narrative, and technically feasible}''.
P4 expressed his urgent need to recruit people with expertise in algorithms for his art studio, because ``\textit{to create generative art, such as large-scale data-driven particle systems, one must have knowledge of statistics or machine learning; I have long wanted to recruit individuals with such expertise for my team.}''
However, P6 thought that currently, there is still a lack of mutual understanding among people from different disciplines. She observed that those currently engaged in artistic data visualization hail from diverse backgrounds: some have a technical background and an interest in art, while others are artists by original training: ``\textit{These two groups approach things very differently because their foundational training and theoretical frameworks are completely distinct. Sometimes, even though we claim to be interdisciplinary, I feel like there’s a bias or a gap in understanding between these two groups, as if we’re not speaking the same language.}''
% Nevertheless, many participants still find that the sparks generated by interdisciplinary collaboration are inspiring and exciting. P1 thought that the combination of art and big data would release significant potential, and he hoped to collaborate more closely with data scientists.


\textbf{The gap between art practice and research.}
Despite the flourish of artistic data visualization in the wild, 6 participants saw a gap between its practice and research. P9, who is both an artist and an assistant professor, highlighted the difficulties of publishing papers and being involved in academia: ``\textit{In art schools, creating works and receiving awards hold more value than publishing papers. Many artists don’t feel the need to publish papers, but due to the larger context of academic evaluation, some are now being pushed to transform their works into papers, which forces them to find new ways of approaching this.}'' P12, who used to be an artist and is now pursuing a Ph.D. degree, said that ``\textit{I have created many works and participated in many exhibitions, but I truly don't know how to write a doctoral dissertation about them...It feels like the thought processes for conducting research and creating artwork are quite different.}''
% We usually put a lot of effort into creating artworks, and some even received significant social recognition. However, it's challenging to understand the academic significance and value of these works. We are not quite sure how to write them into papers and apply for funding.
% P2 also noted that many data artworks may not necessarily be seen as academic research, saying, ``\textit{Although there are many data artworks, research papers that examine such artworks are rarely seen.}'' 
P8 felt that ``\textit{it (the academic community) offers limited theoretical support for my work.}''
P4 recalled having moments of self-doubt because ``\textit{we produce a lot of work, but the methodologies and frameworks for organizing this kind of work are scarce...sometimes it's hard to justify our value to others.}''
% However, P2 also expressed his optimistic attitude to this challenge because ``\textit{the visualization community is quite open, and I have noticed that there are already some studies exploring data art.}''








\section{Discussion}
\label{sec:discussion}

The results provide valuable insights into the limitations of machine learning (ML) models to support systematic literature review (SLR) updates. In this discussion, we interpret these results in light of the research questions, contextualize their implications, and outline the trade-offs associated with applying ML models in this domain.

\subsection{Effectiveness of ML Models for SLR Study Selection (RQ1)}

The results for RQ1 indicate that our best-performing model, Random Forest (RF), achieved a modest balance between precision and recall with an F-score of 0.33 at the default threshold of 0.5. This result suggests that while the ML model was able to identify some relevant studies, its overall ability to precisely distinguish between relevant and irrelevant studies was limited. Adjusting the threshold improved the F-score to 0.41, highlighting the sensitivity of the model’s performance to the chosen threshold. However, this improvement came at the cost of increasing false negatives (FNs), potentially missing valuable studies. We interpret the RF model’s performance as indicating that ML may assist in informally identifying a subset of relevant studies but is not yet reliable for the selection of studies for SLR updates.

\subsection{Effort Reduction through ML Models (RQ2)}

In answering RQ2, we focused on maximizing recall to avoid FNs. In our investigations, the SVM model was more suitable for focusing on achieving a high recall and demonstrating some potential for reducing human screening efforts. Results demonstrated that with a recall of 100\%, the SVM model could exclude 33.9\% of studies from the review process without missing any relevant studies. This reduction represents a significant decrease in the manual workload, suggesting ML’s potential to assist researchers with the initial screening stage. However, to achieve this high recall, the model produced a high rate of false positives (FPs), still requiring significant human review effort to discard many non-relevant studies.

As shown in Table \ref{tab:effort_reduction}, gradually increasing the inclusion probability threshold reduced the number of FPs at the cost of a minor drop in recall. For instance, at a threshold of 0.75, the model achieved a recall of 97.37\%, with a reduction of 48.3\% in the number of studies needing review. We interpret this result as indicating that, while ML can reduce screening efforts, care must be taken when applying thresholds to avoid introducing a risk of overlooking critical studies.

\subsection{Supporting Human Reviewers (RQ3)}

For RQ3, we evaluated the support ML could provide compared to that of an additional human reviewer. When we treated the RF model as an additional reviewer and calculated Euclidean Distance (ED) to assess alignment with the final inclusion decision, individual human reviewers outperformed the RF model. Furthermore, pairs of human reviewers clearly outperformed human-ML pairs, suggesting that human-only review teams achieve more accurate results.

This finding reinforces the challenges ML models face in fully replicating the nuanced judgment of human reviewers. Hence, ML can not replace additional human reviewers, and ML assistance is not a valid argument for quality in the selection process. Pairs of human reviewers are still highly recommended for selecting studies in SLR updates.
\section{Conclusion}\label{sec:conclusion}
This work introduces a novel approach to TOT query elicitation, leveraging LLMs and human participants to move beyond the limitations of CQA-based datasets. Through system rank correlation and linguistic similarity validation, we demonstrate that LLM- and human-elicited queries can effectively support the simulated evaluation of TOT retrieval systems. Our findings highlight the potential for expanding TOT retrieval research into underrepresented domains while ensuring scalability and reproducibility. The released datasets and source code provide a foundation for future research, enabling further advancements in TOT retrieval evaluation and system development.

\acknowledgments{
This work was supported by NSFC 62402121, Shanghai Chenguang Program, and Research and Innovation Projects from the School of Journalism at Fudan University.}

%\bibliographystyle{abbrv}
% \bibliographystyle{abbrv-doi}
% \bibliographystyle{abbrv-doi-narrow}
%\bibliographystyle{abbrv-doi-hyperref}
\bibliographystyle{abbrv-doi-hyperref-narrow}

\bibliography{template}
\end{document}
