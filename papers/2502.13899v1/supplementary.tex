%%%%%%%%%%%%%%%%%%%%%%%%%%%%%%%%%%%%%%%%%%%%%%%%%%%%%%%%%%%%%%%%%%%%%
%% This is a (brief) model paper using the achemso class
%% The document class accepts keyval options, which should include
%% the target journal and optionally the manuscript type. 
%%%%%%%%%%%%%%%%%%%%%%%%%%%%%%%%%%%%%%%%%%%%%%%%%%%%%%%%%%%%%%%%%%%%%
\documentclass[journal=jacsat,manuscript=suppinfo]{achemso}
\setkeys{acs}{doi = true}
%%%%%%%%%%%%%%%%%%%%%%%%%%%%%%%%%%%%%%%%%%%%%%%%%%%%%%%%%%%%%%%%%%%%%
%% Place any additional packages needed here.  Only include packages
%% which are essential, to avoid problems later. Do NOT use any
%% packages which require e-TeX (for example etoolbox): the e-TeX
%% extensions are not currently available on the ACS conversion
%% servers.
%%%%%%%%%%%%%%%%%%%%%%%%%%%%%%%%%%%%%%%%%%%%%%%%%%%%%%%%%%%%%%%%%%%%%
\usepackage[version=3]{mhchem} % Formula subscripts using \ce{}
\usepackage{siunitx}
\usepackage{multirow}
\usepackage{svg}
\usepackage{rotating} 
\usepackage{booktabs}
% \usepackage{lineno}
% \linenumbers
%%%%%%%%%%%%%%%%%%%%%%%%%%%%%%%%%%%%%%%%%%%%%%%%%%%%%%%%%%%%%%%%%%%%%
%% If issues arise when submitting your manuscript, you may want to
%% un-comment the next line.  This provides information on the
%% version of every file you have used.
%%%%%%%%%%%%%%%%%%%%%%%%%%%%%%%%%%%%%%%%%%%%%%%%%%%%%%%%%%%%%%%%%%%%%
%%\listfiles

%%%%%%%%%%%%%%%%%%%%%%%%%%%%%%%%%%%%%%%%%%%%%%%%%%%%%%%%%%%%%%%%%%%%%
%% Place any additional macros here.  Please use ewcommand* where
%% possible, and avoid layout-changing macros (which are not used
%% when typesetting).
%%%%%%%%%%%%%%%%%%%%%%%%%%%%%%%%%%%%%%%%%%%%%%%%%%%%%%%%%%%%%%%%%%%%%
% ewcommand*\mycommand[1]{\texttt{\emph{#1}}}


\makeatletter
\def\acs@maketitle@suppinfo{}
\makeatother

%%%%%%%%%%%%%%%%%%%%%%%%%%%%%%%%%%%%%%%%%%%%%%%%%%%%%%%%%%%%%%%%%%%%%
%% Meta-data block
%% ---------------
%% Each author should be given as a separate \author command.
%%
%% Corresponding authors should have an e-mail given after the author
%% name as an \email command. Phone and fax numbers can be given
%% using \phone and \fax, respectively; this information is optional.
%%
%% The affiliation of authors is given after the authors; each
%% \affiliation command applies to all preceding authors not already
%% assigned an affiliation.
%%
%% The affiliation takes an option argument for the short name.  This
%% will typically be something like "University of Somewhere".
%%
%% The \altaffiliation macro should be used for new address, etc.
%% On the other hand, \alsoaffiliation is used on a per author basis
%% when authors are associated with multiple institutions.
%%%%%%%%%%%%%%%%%%%%%%%%%%%%%%%%%%%%%%%%%%%%%%%%%%%%%%%%%%%%%%%%%%%%%

%%%%%% added <<<<<<<<<<<<<<<<
\usepackage{etoolbox}% needed for the patch  <<<

\makeatletter
\renewcommand*{\acs@author@fnsymbol@symbol}[1]{% Use numbers instead of symbols, * is for email
    \ifcase #1 *\or
    1\or
    2\or
    3\or
    4\or
    5\or
    6\or
    7\or
    8\or
    9\or
    10
    \fi
}
        
\renewcommand*\acs@contact@details{% addd * before  E-mail
    {\sffamily *\,E-mail: \acs@email@list }%
    \acs@number@list
}           

\patchcmd{\acs@address@list@auxii}% superscript for numbers in affiliations
{\acs@author@fnsymbol{\acs@affil@marker@cnt}}
{\textsuperscript{\acs@author@fnsymbol{\acs@affil@marker@cnt}}}
{}{}

\patchcmd{\acs@address@list@auxii}% superscript for numbers in affiliations
{{\acs@author@fnsymbol{\acs@affil@marker@cnt}\@nameuse{@altaffil@\@roman\@tempcnta}\par}}
{{\textsuperscript{\acs@author@fnsymbol{\acs@affil@marker@cnt}}\@nameuse{@altaffil@\@roman\@tempcnta}\par}}
{}{}        

\makeatother    
%%%%%%% added <<<<<<<<<<<<<<<<


\author{Subhash V.S. Ganti}
\affiliation[1]{University of Bayreuth, Faculty of Engineering Science, Universitätsstraße 30, 95447 Bayreuth, Germany}
\alsoaffiliation[2]{University of Bayreuth, Bavarian Center for Battery Technology (BayBatt), Universitätsstraße 30, 95447 Bayreuth, Germany}
\author{Lukas Wölfel}
\affiliation[1]{University of Bayreuth, Faculty of Engineering Science, Universitätsstraße 30, 95447 Bayreuth, Germany}
\author{Christopher Kuenneth}
\affiliation[1]{University of Bayreuth, Faculty of Engineering Science, Universitätsstraße 30, 95447 Bayreuth, Germany}
\alsoaffiliation[2]{University of Bayreuth, Bavarian Center for Battery Technology (BayBatt), Universitätsstraße 30, 95447 Bayreuth, Germany}


\email{christopher.kuenneth@uni-bayreuth.de}

\title[{AI Driven Approach to Identify Suitable Organic Materials for Batteries}]
  {Supplementary Information\\ {AI-Driven Discovery of High Performance Polymer Electrodes for Next-Generation Batteries}}

\renewcommand{\figurename}{Supplementary Figure}
\renewcommand{\tablename}{Supplementary Table}
\renewcommand{\refname}{Supplementary References}


\AtBeginDocument
  {%
    \renewcommand*{\citenumfont}[1]{#1}%
    \renewcommand*{\bibnumfmt}[1]{(#1)}%

  }

\begin{document}

\section{Supplementary Discussion}

While training MT-ML and meta learner models, we divided total dataset into two parts, namely train (80 percent) and validation (20 percent) respectively. We further subdivided the train into 5 parts. Each sub-part contains training and testing set. One separate model is prepared for each of them, using hyper-parameter tuning. Since our models are trained on multiple-anodes and material types, we compared our results for all of them. 

\begin{table}[hbtp]
\centering
\label{tab:specific_capacity}
\caption{Specific capacity (mAh/g) prediction across various charge carriers (positive electrode materials) and organic material classes (molecule/polymer) which is evaluated using metrics (MAE, R\textsuperscript{2}, and RMSE). These metrics are derived from cross-validation results (averaged across all test-sets of MT-ML models) and meta-learner models.}
\resizebox{1\textwidth}{!}{
\begin{tabular}{lrrrrrrrr}
\toprule
{Charge carrier (+ve)} & \multicolumn{4}{c}{Cross validation} & \multicolumn{4}{c}{Meta} \\
{} & {DP} & {MAE} & {R2} & {RMSE} & {DP} & {MAE} & {R2} & {RMSE} \\
\midrule
\multicolumn{9}{c}{Polymer} \\ % Centered subplot title
\midrule
Al & 3 & 83.21 & -17.63 & 88.69 & 3 & 20.16 & -0.38 & 24.11 \\
K & 1 & 22.37 & & 22.37 & 1 & 1.22 & & 1.22 \\
Li & 117 & 60.77 & 0.55 & 102.43 & 117 & 26.22 & 0.92 & 44.27 \\
Mg & 5 & 32.04 & 0.66 & 44.61 & 5 & 5.56 & 0.99 & 8.69 \\
Na & 33 & 14.77 & 0.91 & 32.41 & 33 & 8.83 & 0.99 & 10.58 \\
Zn & 3 & 56.31 & 0.65 & 63.72 & 3 & 30.20 & 0.85 & 42.35 \\
\midrule
\multicolumn{9}{c}{Molecule} \\ % Centered subplot title
\midrule
K & 2 & 25.91 & 0.36 & 33.13 & 2 & 8.52 & 0.96 & 8.52 \\
Li & 153 & 35.31 & 0.81 & 55.96 & 153 & 14.82 & 0.97 & 21.58 \\
Mg & 2 & 39.69 & -0.15 & 49.84 & 2 & 24.26 & 0.65 & 27.50 \\
Na & 51 & 23.40 & 0.88 & 37.11 & 51 & 13.14 & 0.98 & 16.97 \\
\bottomrule
\end{tabular}}
\end{table}

\begin{table}[hbtp]
\centering
\label{tab:voltage}
\caption{Voltage (V) prediction across various charge carriers (positive electrode materials) and organic material classes (molecule/polymer) which is evaluated using metrics (MAE, R\textsuperscript{2}, and RMSE). These metrics are derived from cross-validation results (averaged across all test-sets of MT-ML models) and meta-learner models.}
\resizebox{1\textwidth}{!}{
\begin{tabular}{lrrrrrrrr}
\toprule
{Charge carrier (+ve)} & \multicolumn{4}{c}{Cross validation} & \multicolumn{4}{c}{Meta} \\
{} & {DP} & {MAE} & {R2} & {RMSE} & {DP} & {MAE} & {R2} & {RMSE} \\
\midrule
\multicolumn{9}{c}{Polymer} \\ % Centered subplot title
\midrule
Al & 2 & 0.84 & -307.45 & 0.88 & 2 & 0.36 & -55.28 & 0.37 \\
K & 1 & 0.20 & & 0.20 & 1 & 0.11 & & 0.10 \\
Li & 86 & 0.31 & 0.88 & 0.49 & 86 & 0.07 & 1.00 & 0.10 \\
Mg & 4 & 0.25 & -14.01 & 0.35 & 4 & 0.07 & 0.15 & 0.10 \\
Na & 31 & 0.08 & 0.97 & 0.14 & 31 & 0.07 & 0.99 & 0.10 \\
Zn & 2 & 0.67 & -52.10 & 0.73 & 2 & 0.17 & -3.92 & 0.22 \\
\midrule
\multicolumn{9}{c}{Molecule} \\ % Centered subplot title
\midrule
Al & 1 & 0.38 & & 0.37 & 1 & 0.04 & & 0.00 \\
K & 2 & 0.37 & 0.36 & 0.42 & 2 & 0.04 & 0.99 & 0.00 \\
Li & 80 & 0.24 & 0.89 & 0.42 & 80 & 0.08 & 0.99 & 0.14 \\
Mg & 1 & 0.73 & & 0.73 & 1 & 0.22 & & 0.22 \\
Na & 33 & 0.19 & 0.78 & 0.35 & 33 & 0.06 & 0.99 & 0.10 \\
Zn & 3 & 0.86 & -43.57 & 0.98 & 3 & 0.13 & -0.15 & 0.14 \\
\bottomrule
\end{tabular}}
\end{table}

\begin{figure}[hbt]
 \includegraphics{Sup_Figure_1.pdf}
  \caption{UMAP plots for our complete dataset plotted with respect to organic material classes as shown in the legend.}
  \label{fig:umap_plots}
\end{figure}

\begin{figure}[hbt]
 \includegraphics{Sup_Figure_2.pdf}
  \caption{Reference organic polymers that are used to enumeratively add or replace moieties.}
  \label{fig:reforg}
\end{figure}

\begin{figure}[hbt]
 \includegraphics{Sup_Figure_3.pdf}
  \caption{Moieties used to add or replace chemical bonds in reference organic materials.}
  \label{fig:moeties}
\end{figure}

\begin{figure}[hbt]
 \includegraphics{Sup_Figure_4.pdf}
  \caption{Top 9 candidates with the highest energy density curated from the 3 libraries totaling 1.8 million candidates.}
  \label{fig:ed_best_candidates}
\end{figure}

\begin{figure}[hbt]
 \includegraphics{Sup_Figure_5.pdf}
  \caption{Top 9 candidates with the highest voltage curated from the 3 libraries totaling 1.8 million candidates.}
  \label{fig:voltage_best_candidates}
\end{figure}

\begin{figure}[hbt]
 \includegraphics{Sup_Figure_6.pdf}
  \caption{Top 9 candidates with the highest specific capacity curated from the 3 libraries totaling 1.8 million candidates.}
  \label{fig:sc_best_candidates}
\end{figure}

\end{document}