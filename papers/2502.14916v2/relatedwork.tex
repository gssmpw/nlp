\section{Related Works}
\label{sec:Related Works}
The task of automatically coding ICD in the medical field has been well established and has received much attention. Many deep-learning methods have been used to address this problem, and we categorize these methods into the following three types.

    \begin{figure*}[t]
    \centering
    \includegraphics[height=6.5cm,width=18cm]{framework.pdf}
    \caption{The general framework of MKE-Coder. First, we extract the diagnosis list from the Chinese EMR. Next, we identify the most probable candidate ICD codes for each diagnosis to enhance efficiency. To parse the multi-axial knowledge associated with each candidate code, we employ a multi-axial parser. Additionally, we utilize an evidence retrieval module to locate supporting text descriptions within the EMR for each axis of the candidate code. We propose a Clinical-Simbert model to ensure the reliability of the retrieved evidence. We then convert the binary classification task into a masked language modeling problem and conduct supervised training. Finally, we obtain the recommended ICD code for each diagnosis, along with the corresponding evidence set. }
    \label{fig:framework}
    \end{figure*}

\subsection{Knowledge-Enhanced Methods}

A comprehensive understanding of clinical notes necessitates expertise in the field of medicine. Various techniques leverage external knowledge sources to optimize neural architectures and facilitate the comprehension of clinical text, ultimately enhancing proficiency in accurately assigning ICD codes. 
Yuan et al. proposed the MSMN method, which collected synonyms of every ICD code from UMLS and designed a multiple synonyms matching network to leverage synonyms for better code classification \cite{yuan2022code}. Similarly, Yang et al. proposed a knowledge-enhanced longformer by injecting three domain-specific knowledge: hierarchy, synonym, and abbreviation with additional pretraining using contrastive learning \cite{yang2022knowledge}. Li et al. highlighted the often overlooked issues of data imbalance and noise in clinical notes, thus they suggested a knowledge-enhanced Graph Attention Network (GAT \cite{velivckovic2017graph}) under a multi-task learning setting, featuring multi-level information transitions and interactions \cite{li2023towards}. 


\subsection{Graph Neural Networks}
Graph neural networks are adept at transforming complex relationships into topological graph structures and then leveraging graph convolution operations to facilitate low or high-dimensional interactions between nodes and relationships, which has led to their successful application and impressive results in many NLP tasks \cite{zhou2020graph,wu2020comprehensive,zheng2022graph,zhang2019heterogeneous}. To mitigate the issues of excessive and imbalanced ICD codes, which present a severe long-tail distribution, and to fully exploit the hierarchical structure among the codes, graph neural networks have been extensively applied to automatic ICD coding tasks. Jeong et al. introduced a novel approach, the Joint Learning Framework Across Medical Coding Systems (JAMS). This framework utilized a modified version of the Graph Attention Network, termed the Medical Code Attention Network, to facilitate multi-task learning, enabling simultaneous learning from various coding systems through a shared encoder to acquire diverse representations \cite{jeong2024bridging}. Meanwhile, Wu et al. highlighted the primary challenges of automated ICD coding, including imbalanced label distribution, code hierarchy, and noisy texts. In response, they proposed a unique approach called the Hyperbolic Graph Convolutional Network \cite{chami2019hyperbolic} with Contrastive Learning \cite{you2020graph}. They employed a Hyperbolic Graph Convolutional Network to capture the hierarchical structure of ICD codes and implemented contrastive learning to automatically assign ICD codes, incorporating code features into a text encoder \cite{wu2024hyperbolic}. In a study by Fails et al., the researchers proposed data augmentation and synthesis techniques and also introduced an analysis technique for this setting inspired by confusion matrices \cite{falis2022horses}. Recognizing that existing methods could not model a personalized relation graph for each case, or discover implicit code relations, Luo et al. presented a novel contextualized code relation-enhanced ICD coding framework. This framework utilized a personalized flexible relation graph, enabling their model to learn implicit relations \cite{luo2024corelation}.

  
 \begin{table*}[h]
    \centering
    \caption{This is a small section of a complete prior knowledge table, listing the possible locations in medical records where clinically supporting evidence for three-digit code groups of interest may be found.}
     \renewcommand{\arraystretch}{1.2} % 可以调节, 1.2指高度是默认的1.2倍   
    \begin{tabular}{ccccc}
    \toprule[1.2pt]
    Code range and name\;   & \multicolumn{4}{c}{Possible locations for evidence appearance in Chinese EMRs}\\
    \hline
    \makecell[c]{A00-A09\\Intestinal infectious diseases\;}  & \makecell[c]{ Testing \\(and results)} & \makecell[c]{Symptom \\presentation} &\makecell[c]{History of \\exposure to endemic \\or epidemic areas}&\\
    \hline
    \makecell[c]{A15-A19\\Tuberculosis}  & \makecell[c]{Symptom \\presentation} & \makecell[c]{Examination\\ (and results)} & \makecell[c]{ Chest \\ examination results}& \,\makecell[c]{Vaccination  \\ and Infectious \\Disease History}     \\
    \hline
    \makecell[c]{A20-A28\\Certain zoonotic\\bacterial diseases}   & \makecell[c]{Testing \\(and results) }& \makecell[c]{Occupation and \\ working conditions}   && \\
    \hline
    \makecell[c]{A30-A49\\Other bacterial diseases}  &  \makecell[c]{Skin and \\mucous membrane \\examination results} & \makecell[c]{Testing \\(and results)} & \makecell[c]{Neurological reflex\\examination results}  & \\
    \hline
    \makecell[c]{ A50-A64\\Sexually transmitted\\infections } 
      & \makecell[c]{Skin and \\mucous membrane \\examination results}   &  \makecell[c]{External genital\\ examination results}  & Testing (and results) & \\
    \hline
    \makecell[c]{ A65-A69\\Other spirochetal diseases} & \makecell[c]{Skin and \\mucous membrane \\examination results}    & \makecell[c]{Testing \\(and results) }&  \makecell[c]{History of \\exposure to endemic \\or epidemic areas}& \\
    \hline
    \makecell[c]{A70-A74 \\Other diseases \\caused by chlamydia}& \makecell[c]{Chest \\examination results} & \makecell[c]{Head and facial \\examination results}  & Testing (and results) & \\
    \hline
    \makecell[c]{ A75-A79 \\Rickettsioses}  & Testing (and results)  &\,\makecell[c]{History of \\exposure to endemic \\or epidemic areas}& & \\
    \hline
    \makecell[c]{ ...}  & ... &...& ...& ...\\
    \bottomrule[1.2pt]
    \end{tabular}
    \label{tab:prior knowledge}
\end{table*}


\subsection{Pretrained Language Models}

Pretrained language models are trained on an auxiliary task, such as masked language modeling that predicts a word or sequence based on the surrounding context and gains improvement in many NLP tasks \cite{qiu2020pre}. Pretraining such auxiliary tasks benefits from large-scale training on unlabeled corpora that are readily available from the web or textbooks \cite{ji2021does}. For electronic health records that ICD coding is based on, language models that have been pre-trained in the medical field can effectively understand their specialized medical semantic information. As such, they are extensively employed in ICD coding tasks. A series of competitive frameworks for automatic ICD coding is based on the Bert method \cite{ji2021,zhang2020,pascual2021}, in which each ICD code is associated with a unique entity representation, and the automatic coding task is transformed into a multi-label classification across fine-tuning. Afkanpour et al. presented a simple and scalable method to process long electronic medical records (EMRs) with the existing transformer models such as BERT, showed that this method significantly improves the previous results reported for transformer models in ICD coding, and can outperform one of the prominent CNN-based methods \cite{afkanpour2022bert}. Gomes et al. employed a strong Transformer-based model as a text encoder and, to handle lengthy clinical narratives, explored either (a) adapting the base encoder model into a Longformer \cite{beltagy2020longformer}, or (b) dividing the text into chunks and processing each chunk independently \cite{gomes2024accurate}. Silva et al. proposed a method where Cosine text similarity is combined with a pretrained language model, PLM-ICD, to increase the number of probably useful suggestions of ICD-10 codes, based on the Medical Information Mart for Intensive Care (MIMIC)-IV dataset \cite{silva2024aiding}. 

Prompt-based fine-tuning is effective in few-shot tasks \cite{scao2021many,gao2020making}, even when the language model is relatively small \cite{schick2020s} because they introduce no new parameter during few-shot fine-tuning. Yang et al. proposed a Knowledge Enhanced PrompT (KEPT) framework that utilized prompt-tuning in the context of ICD coding tasks by adding a sequence of ICD code descriptions as prompts in addition to each clinical note as KEPT LM input \cite{yang2022knowledge}. Yang et al. transformed the multi-label classification task into an autoregressive generation task and developed a cross-attention reranker with prompts to incorporate previous state-of-the-art (SOTA) and the best few-shot coding predictions \cite{yang2023multi}. 
 
\subsection{Summary}

In summary, while automatic ICD coding has received significant attention, there has been limited validation of its effectiveness on Chinese EMR datasets. Consequently, a lack of research specifically focuses on the unique challenges and practical requirements of Chinese EMRs. Notably, existing methods have neglected to extract disease code-related information from Chinese EMRs and have overlooked the crucial multi-axial knowledge that underpins the ICD coding hierarchy. As a result, the outcomes of the Chinese ICD coding system suffer from insufficient coding accuracy and a lack of genuine interpretability, posing challenges to their effective implementation in real-world scenarios. Therefore, our work is motivated by these issues and proposes a novel framework that addresses the challenges of ICD coding in Chinese EMRs. Our framework considers the specific characteristics of Chinese EMRs, focusing on extracting multi-axial knowledge-related information. By doing so, we aim to improve coding accuracy and enhance interpretability, enabling effective system implementation in real-world scenarios.