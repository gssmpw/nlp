%File: anonymous-submission-latex-2025.tex
\documentclass[letterpaper]{article} % DO NOT CHANGE THIS
\usepackage[submission]{aaai25}  % DO NOT CHANGE THIS
\usepackage{times}  % DO NOT CHANGE THIS
\usepackage{helvet}  % DO NOT CHANGE THIS
\usepackage{courier}  % DO NOT CHANGE THIS
\usepackage[hyphens]{url}  % DO NOT CHANGE THIS
\usepackage{graphicx} % DO NOT CHANGE THIS
\urlstyle{rm} % DO NOT CHANGE THIS
\def\UrlFont{\rm}  % DO NOT CHANGE THIS
\usepackage{natbib}  % DO NOT CHANGE THIS AND DO NOT ADD ANY OPTIONS TO IT
\usepackage{caption} % DO NOT CHANGE THIS AND DO NOT ADD ANY OPTIONS TO IT
\frenchspacing  % DO NOT CHANGE THIS
\setlength{\pdfpagewidth}{8.5in} % DO NOT CHANGE THIS
\setlength{\pdfpageheight}{11in} % DO NOT CHANGE THIS
%
% These are recommended to typeset algorithms but not required. See the subsubsection on algorithms. Remove them if you don't have algorithms in your paper.
\usepackage{algorithm}
\usepackage{algorithmic}

\usepackage{booktabs}
\usepackage{xcolor}
\usepackage{amsmath}
\usepackage{bm}
\usepackage{amsfonts}
\usepackage{multicol}
\usepackage{graphicx}
\usepackage{multirow}

%
% These are are recommended to typeset listings but not required. See the subsubsection on listing. Remove this block if you don't have listings in your paper.
\usepackage{newfloat}
\usepackage{listings}
\DeclareCaptionStyle{ruled}{labelfont=normalfont,labelsep=colon,strut=off} % DO NOT CHANGE THIS
\lstset{%
	basicstyle={\footnotesize\ttfamily},% footnotesize acceptable for monospace
	numbers=left,numberstyle=\footnotesize,xleftmargin=2em,% show line numbers, remove this entire line if you don't want the numbers.
	aboveskip=0pt,belowskip=0pt,%
	showstringspaces=false,tabsize=2,breaklines=true}
\floatstyle{ruled}
\newfloat{listing}{tb}{lst}{}
\floatname{listing}{Listing}
%
% Keep the \pdfinfo as shown here. There's no need
% for you to add the /Title and /Author tags.
\pdfinfo{
/TemplateVersion (2025.1)
}

% DISALLOWED PACKAGES
% \usepackage{authblk} -- This package is specifically forbidden
% \usepackage{balance} -- This package is specifically forbidden
% \usepackage{color (if used in text)
% \usepackage{CJK} -- This package is specifically forbidden
% \usepackage{float} -- This package is specifically forbidden
% \usepackage{flushend} -- This package is specifically forbidden
% \usepackage{fontenc} -- This package is specifically forbidden
% \usepackage{fullpage} -- This package is specifically forbidden
% \usepackage{geometry} -- This package is specifically forbidden
% \usepackage{grffile} -- This package is specifically forbidden
% \usepackage{hyperref} -- This package is specifically forbidden
% \usepackage{navigator} -- This package is specifically forbidden
% (or any other package that embeds links such as navigator or hyperref)
% \indentfirst} -- This package is specifically forbidden
% \layout} -- This package is specifically forbidden
% \multicol} -- This package is specifically forbidden
% \nameref} -- This package is specifically forbidden
% \usepackage{savetrees} -- This package is specifically forbidden
% \usepackage{setspace} -- This package is specifically forbidden
% \usepackage{stfloats} -- This package is specifically forbidden
% \usepackage{tabu} -- This package is specifically forbidden
% \usepackage{titlesec} -- This package is specifically forbidden
% \usepackage{tocbibind} -- This package is specifically forbidden
% \usepackage{ulem} -- This package is specifically forbidden
% \usepackage{wrapfig} -- This package is specifically forbidden
% DISALLOWED COMMANDS
% \nocopyright -- Your paper will not be published if you use this command
% \addtolength -- This command may not be used
% \balance -- This command may not be used
% \baselinestretch -- Your paper will not be published if you use this command
% \clearpage -- No page breaks of any kind may be used for the final version of your paper
% \columnsep -- This command may not be used
% \newpage -- No page breaks of any kind may be used for the final version of your paper
% \pagebreak -- No page breaks of any kind may be used for the final version of your paperr
% \pagestyle -- This command may not be used
% \tiny -- This is not an acceptable font size.
% \vspace{- -- No negative value may be used in proximity of a caption, figure, table, section, subsection, subsubsection, or reference
% \vskip{- -- No negative value may be used to alter spacing above or below a caption, figure, table, section, subsection, subsubsection, or reference

\setcounter{secnumdepth}{0} %May be changed to 1 or 2 if section numbers are desired.

% The file aaai25.sty is the style file for AAAI Press
% proceedings, working notes, and technical reports.
%

% Title

% Your title must be in mixed case, not sentence case.
% That means all verbs (including short verbs like be, is, using,and go),
% nouns, adverbs, adjectives should be capitalized, including both words in hyphenated terms, while
% articles, conjunctions, and prepositions are lower case unless they
% directly follow a colon or long dash
\title{Utility-Driven Tabular Data Synthesis via Reinforcement Learning}
\author{
    %Authors
    % All authors must be in the same font size and format.
    Written by AAAI Press Staff\textsuperscript{\rm 1}\thanks{With help from the AAAI Publications Committee.}\\
    AAAI Style Contributions by Pater Patel Schneider,
    Sunil Issar,\\
    J. Scott Penberthy,
    George Ferguson,
    Hans Guesgen,
    Francisco Cruz\equalcontrib,
    Marc Pujol-Gonzalez\equalcontrib
}
\affiliations{
    %Afiliations
    \textsuperscript{\rm 1}Association for the Advancement of Artificial Intelligence\\
    % If you have multiple authors and multiple affiliations
    % use superscripts in text and roman font to identify them.
    % For example,

    % Sunil Issar\textsuperscript{\rm 2},
    % J. Scott Penberthy\textsuperscript{\rm 3},
    % George Ferguson\textsuperscript{\rm 4},
    % Hans Guesgen\textsuperscript{\rm 5}
    % Note that the comma should be placed after the superscript

    1101 Pennsylvania Ave, NW Suite 300\\
    Washington, DC 20004 USA\\
    % email address must be in roman text type, not monospace or sans serif
    proceedings-questions@aaai.org
%
% See more examples next
}

%Example, Single Author, ->> remove \iffalse,\fi and place them surrounding AAAI title to use it
\iffalse
\title{My Publication Title --- Single Author}
\author {
    Author Name
}
\affiliations{
    Affiliation\\
    Affiliation Line 2\\
    name@example.com
}
\fi

\iffalse
%Example, Multiple Authors, ->> remove \iffalse,\fi and place them surrounding AAAI title to use it
\title{My Publication Title --- Multiple Authors}
\author {
    % Authors
    First Author Name\textsuperscript{\rm 1},
    Second Author Name\textsuperscript{\rm 2},
    Third Author Name\textsuperscript{\rm 1}
}
\affiliations {
    % Affiliations
    \textsuperscript{\rm 1}Affiliation 1\\
    \textsuperscript{\rm 2}Affiliation 2\\
    firstAuthor@affiliation1.com, secondAuthor@affilation2.com, thirdAuthor@affiliation1.com
}
\fi


% REMOVE THIS: bibentry
% This is only needed to show inline citations in the guidelines document. You should not need it and can safely delete it.
\usepackage{bibentry}
% END REMOVE bibentry

\begin{document}

\maketitle

\begin{abstract}
Tree of Thoughts (ToT) enhances Large Language Model (LLM) reasoning by structuring problem-solving as a spanning tree. However, recent methods focus on search accuracy while overlooking computational efficiency. The challenges of accelerating the ToT lie in the frequent switching of reasoning focus, and the redundant exploration of suboptimal solutions. To alleviate this dilemma, we propose Dynamic Parallel Tree Search (DPTS), a novel parallelism framework that aims to dynamically optimize the reasoning path in inference. 
It includes the Parallelism Streamline in the generation phase to build up a flexible and adaptive parallelism with arbitrary paths by fine-grained cache management and alignment. 
Meanwhile, the Search and Transition Mechanism filters potential candidates to dynamically maintain the reasoning focus on more possible solutions and have less redundancy. Experiments on Qwen-2.5 and Llama-3 with Math500 and GSM8K datasets show that DPTS significantly improves efficiency by 2-4$\times$ on average while maintaining or even surpassing existing reasoning algorithms in accuracy, making ToT-based reasoning more scalable and computationally efficient. 
% Codes are provided in the submission.
% Tree of Thoughts~(ToT) enhances Large Language Model~(LLM) reasoning by structuring problem-solving as a spanning tree. However, recent methods focus on search accuracy while overlooked the computational efficiency. 
% The challenges of accelerating the ToT lies in the frequent switching of reasoning focus, and the redundant exploration on suboptimal solutions. 
% % its sequential data structure limits the utilization of GPU parallelism. In particular, existing algorithms typically cause redundant exploration and frequent switching issues, which becomes even severer when paralleled. 
% To alleviate this dilemma, we propose Dynamic Parallel Tree Search~(DPTS), a novel parallelism framework that aims to dynamically optimize reasoning path in inference. 
% % DPTS balances deep exploration and broad expansion through the search algorithm, and the bidirectional transition mechanism allows the tree to focus on high-confidence solutions. 
% % To this end, we design a flexible parallel framework to support arbitrary nodes with various path lengths for  simultaneous expanding. Meanwhile, the searching algorithm and bidirectional transition mechanism allows the trees to focus on high-confidence solutions with less redundant generation tokens. 
% It includes the Parallelism Streamline in generation phase to build up a finer-grained and flexible 
% % path arrangement, allowing a flexible 
% parallelism with arbitrary paths. Meanwhile, the Search and Transition Mechanism filters potential candidates 
% % with balanced exploitation and exploration during selection phase, which maintains 
% to maintain the reasoning focus on more possible solutions and have less redundancy. 
% Experiments on Qwen2.5 and Llama3 with Math500 and GSM8K datasets 
% % \ycj{xx} different models and datasets 
% show that DPTS significantly improves efficiency by $2\times$ on average while maintaining or even surpassing the existing reasoning algorithms in accuracy, making ToT-based reasoning more scalable and computationally efficient. 
% Our code can be found in the supplementary material.
% \ycj{Showcase your speed-up ratio}  % updated 
\end{abstract}

% Uncomment the following to link to your code, datasets, an extended version or similar.
%
% \begin{links}
%     \link{Code}{https://aaai.org/example/code}
%     \link{Datasets}{https://aaai.org/example/datasets}
%     \link{Extended version}{https://aaai.org/example/extended-version}
% \end{links}

\section{Introduction}\label{sec:intro}

Augmented and Virtual Reality (AR/VR) has made significant advancements in recent years in terms of quality and affordability \cite{meta_motiv1, meta_motiv2}, through the use of machine learning algorithms.
One crucial algorithm is depth estimation from stereo sensors, which plays a vital role in spatial computing, hand tracking \cite{depth_gesture_recog}, and %spatial and
passthrough rendering.
Conventional DNN based stereo depth algorithms use expensive hierarchical processing \cite{stereonet, mobilestereonet, hitnet}, 
% resulting in high storage and processing energy
\je{which are challenging to accelerate on power constrained platforms}.
Increased resolutions and frame rates in newer systems further increase these costs \cite{near_sensor_distributed}.
Consequently, accelerating these networks on AR/VR devices while meeting real-time latency requirements and operating within the energy budget of limited battery devices presents a challenge.

In this work, we propose \textit{\projname{}}, an AR/VR stereo depth system comprising a flexible architecture for processing dynamic Regions of Interest (ROIs), and a comprehensive mapping methodology to optimize ROI processing for energy efficiency while maintaining real-time performance. Our contributions are as follows:

\begin{itemize}[leftmargin=*]
    % \item Analysis of stereo depth compute across variable ROI sizes;
    \item \je{The \textit{SteROI-D Algorithm}, which leverages Region-of-Interest (ROI) Sparsity to reduce per-frame depth extraction cost, and interleaved object detection and tracking to reduce ROI detection cost;}
    % \item A flexible ROI-based stereo depth processing system and a comprehensive mapping methodology to achieve near-ideal average inference energy by balancing resource availability for the largest ROIs while maintaining high efficiency for average ROIs;
    % \item Special Compute Unit (SCU) and multipacket routing to enable real time, high framerate operation; and
    \item \je{Special Compute Units (SCUs) and NoC Multipackets, to address compute and communication challenges in accelerating stereo depth networks;}
    % \item An evaluation of this end-to-end system design across a range of algorithm components and candidate datasets.
    \item \je{\textit{Binned Mapping}, a method for split online-offline algorithm mapping to enable efficient processing for a continuous range of ROI sizes; and}
    \item \je{A design space exploration framework for jointly optimizing an accelerator's SRAM allocation with it's Binned Mapping.}
\end{itemize}

To our knowledge, this is the first study to achieve ROI-based stereo depth processing.
This is also the first work to address variable ROI processing through a mapping-system co-design approach via an efficient design space exploration.
While prior work has exploited ROIs for eye tracking on AR devices, it was limited to a static architecture \cite{eyecod}.
Furthermore, although prior work has also proposed lightweight stereo depth systems for AR devices, they have not leveraged ROI sparsity \cite{tiefenrausch}.

\section{Related Works}
\label{sec:related_works}
% Our 





% Introduce several speech enhancement and dereverberation methods based on DNN and Bayesian inference.

% VAE-NMF~\cite{baby2021speech}

% DVAE-VEM~\cite{bie2022unsupervised}

RVAE-EM~\cite{wang2024rvae}
The cubical increase in computation with input length addresses a major limitation observed in our previous publication RVAE-EM~\cite{wang2024rvae}.
% StoRM and its reference

% Introduce RIR and T60 estimation 

% BUDDy

% \cite{mateljan2003comparison}

% Fourier analyzer

% MLS-based system~\cite{rife1989transfer}

% swept sine system~\cite{farina2000simultaneous}



\section{Methodology}
\label{sec:method}

Solving the \gls{acr:mdp} as defined in Section \ref{sec:prob_form} presents several key challenges from a dynamic decision-making perspective. A primary challenge is the need for a model that describes the environment's transition dynamics, which is often complex and challenging to characterize. Another significant challenge is the intractable state space arising from the stochastic arrival of requests and gig workers, which renders exhaustive computation of optimal policies infeasible. Therefore, effective generalization across states is essential, as explicitly enumerating all possible states is unrealistic in many real-world scenarios.

Our proposed method offers several solutions to address these challenges. First, we utilize the \gls{acr:mnl} model to describe gig workers' decision-making process, capturing essential transition dynamics without requiring a fully known environment model. To manage the complexity of large state spaces, we leverage a post-decision state formulation, which eases computing optimal policies from a tractability perspective. This formulation also facilitates solving the inner optimization problem and deriving an exact form for optimal prices based on the post-decision state values. Our method is designed to ensure scalability, making it suitable for real-world applications where exhaustive enumeration of all possible states is infeasible. To this end, we incorporate value function approximators to generalize effectively across states, which enables us to estimate value functions without exhaustive state enumeration.  

This section details our approach. Section \ref{sec:bellman} introduces the Bellman equation for the \gls{acr:mdp} as defined in Section~\ref{sec:prob_form}. Section \ref{sec:mnl} discusses the \gls{acr:mnl} model for capturing essential transition dynamics, addresses the state space intractability by deriving the post-decision state formulation and presents an analytical solution to the optimization problem with optimal prices based on post-decision state values. Section \ref{sec:statistical_models} presents value function approximators and an algorithm for learning appropriate parameterizations. Section \ref{sec:training} details the statistical model used for gig worker utility estimation and presents the training procedure for the full algorithm. Finally, Section \ref{sec:meth_as} discusses methodological assumptions. 

\subsection{Bellman equation}
\label{sec:bellman}
In the following, we derive the Bellman equation for the \gls{acr:mdp} as defined in Section \ref{sec:prob_form}. To do so, we first elaborate on optimal value functions in general to elucidate the connection between pre- and post-decision states, which then allows us to derive the pre- and post-decision state value function accordingly. 

\noindent \textit{Optimal value functions:} The optimal value function of a pre-decision state $S\hspace{-0.1em}_t = (\mathcal{R}\hspace{0.05em}_{t},\mathcal{G}_t)$ or a post-decision state $\mathcal{R}\hspace{0.05em}_t^{p}$, denoted as $V_t(\mathcal{R}\hspace{0.05em}_{t},\mathcal{G}_t)$ and $V^{\mathrm{p}}_{t}(\mathcal{R}\hspace{0.05em}_{t}^{p})$, describes the maximum expected reward achievable from that state onward by following the optimal policy. In this context, the transition from a post-decision state to a successor pre-decision state exclusively involves the stochastic realization of the arrival processes of on-demand requests and gig workers. Therefore, a straightforward relationship between the two value functions is the following: 
\begin{align}
V^{\mathrm{p}}_{t}(\mathcal{R}\hspace{0.05em}_t^{p}) = \mathbb{E}_{\mathcal{R}^{\mathrm{new}},\mathcal{G}^{\mathrm{new}}}[V_{t+1}(\mathcal{R}\hspace{0.05em}_t^{p} \cup \mathcal{R}^{\mathrm{new}},\mathcal{G}^{\mathrm{new}})] \label{eq:value_function}
\end{align}
where the expectation over $\mathcal{R}^{\mathrm{new}},\mathcal{G}^{\mathrm{new}}$ indicates the dynamics of new request and new gig worker arrivals in the system correspondingly (see Figure~\ref{fig:state_trans}). Essentially, Equation \ref{eq:value_function} states that the value of a post-decision state $\mathcal{R}\hspace{0.05em}_{t}$ is equal to the expected value over all possible successor pre-decision states. 
%The expected immediate reward at time step $t$ when following the optimal policy. 

\noindent \textit{Bellman equation using the pre-decision state value function:}
The classic definition of the Bellman equation of the pre-decision state $S\hspace{-0.1em}_t = (\mathcal{R}\hspace{0.05em}_{t},\mathcal{G}_t)$ at time step $t = 0,1,...,T$ reads: 
\begin{align}
V_t(S\hspace{-0.1em}_t) & = \max_{\mathbf{c_t}} \{  R_t(\mathbf{c_t}) + \mathbb{E}_{S' \sim P(\cdot|S\hspace{-0.1em}_t,\mathbf{c_t})}[V_{t+1}(S')] \}
\end{align}
where $R_t(\mathbf{c_t})$ accounts for the expected immediate reward of being in state $S\hspace{-0.1em}_t = (\mathcal{R}\hspace{0.05em}_{t},\mathcal{G}_t)$ and following action $\mathbf{c_t}$, while the expectation $\mathbb{E}_{S' \sim P(\cdot|s,\mathbf{c_t})}[V_{t+1}(S')]$ accounts for the expected future reward that can be obtained from successor pre-decision states $S'$. In this standard definition, the value of each pre-decision state is decomposed using the value of the successor pre-decision state. Notably, the transitions from one pre-decision state to the next include two different sources of stochasticity. The first source of stochasticity results from the utility of the gig workers while the second source results from the stochastic nature of the request and gig worker arrivals. 
This formulation of the Bellman equation is highly intractable due to the complexities introduced by both sources of stochasticity. The primary challenge arises from the second source of stochasticity, as the arrival of new requests and gig workers leads to an explosion in the state transition space, making it impossible to model or track transitions to the next state accurately. In contrast, the first source of stochasticity, which depends solely on the gig workers' choice model, is less complex. To address this issue, we can alternatively formulate the Bellman equation using the post-decision state value function.

\noindent \textit{Bellman equation using the post-decision state value function:} Let us denote as $\mathcal{R}\hspace{0.05em}_t^{\mathrm{exp}}$ the set of requests in~$\mathcal{R}\hspace{0.05em}_{t}$ that expire, and let $\mathcal{R}\hspace{0.05em}_t^{'}$ correspond to the set of requests in $\mathcal{R}\hspace{0.05em}_t$ that do not expire after time step $t$, i.e., $\mathcal{R}\hspace{0.05em}_t^{'} = \mathcal{R}\hspace{0.05em}_{t} \setminus \mathcal{R}\hspace{0.05em}_t^{\mathrm{exp}}$. Furthermore, from here on, we assume for ease of notation that $\mathcal{R}\hspace{0.05em}_t^{'} \setminus \{\emptyset\}$ is equivalent to $\mathcal{R}\hspace{0.05em}_t^{'} $. Then, the Bellman equation of the pre-decision state $S\hspace{-0.1em}_t = (\mathcal{R}\hspace{0.05em}_{t},\mathcal{G}_t)$ at time step $t = 0,1,...,T$ is given by: 
\begin{align}
V_t(S\hspace{-0.1em}_t) & = \max_{\mathbf{c_t}} \big\{  R_t(\mathbf{c_t}) + \underbrace{(1 - \mathbbm{1}_{|\mathcal{G}_t|=1})  V^{\mathrm{p}}_{t}(\mathcal{R}\hspace{0.05em}_t^{'})\vphantom{\sum_{i \in \mathcal{R}\hspace{0.05em}_{t} \cup \{\emptyset\}}}}_{\text{(I)}}  + \underbrace{\mathbbm{1}_{|\mathcal{G}_t|=1} \! \sum_{i \in \mathcal{R}\hspace{0.05em}_{t} \cup \{\emptyset\}} \! P^i_{t}(\mathbf{c_t}) V^{\mathrm{p}}_{t}(\mathcal{R}\hspace{0.05em}_t^{'} \setminus \{i\})}_{\text{(II)}}  \big\}
\label{eq:bell_post}
\end{align}

\noindent where the term $R_t(\mathbf{c_t})$ accounts for the expected immediate reward of being in state $S\hspace{-0.1em}_t = (\mathcal{R}\hspace{0.05em}_{t},\mathcal{G}_t)$ and following action $\mathbf{c_t}$, while the remaining terms account for the expected future reward obtained from the successor post-decision state. Specifically, term (I) accounts for the expected future reward in the case where the pre-decision state does not contain any gig workers. Conversely, term (II) represents the expected future reward in the case where the pre-decision state contains a gig worker. In this case, the probabilities of request selection made by the gig worker determine the transition to post-decision states. Consequently, with a probability of \(P^i_{t}\), we transition to the post-decision state \(\mathcal{R}_{t}^{'} \setminus \{i\}\). 

Bellman Equation \ref{eq:bell_post} makes an explicit distinction between the two different sources of stochasticity in the \gls{acr:mdp}. The source of uncertainty which results from the gig workers' stochastic utility is reflected by the transition to all possible post-decision states, while the second source of uncertainty is embedded within the value function of the post-decision states. By separating the transition dynamics in this manner, the complexity of the Bellman equation is reduced, making the problem more tractable by directly depending only on the transition into the post-decision states.

\subsection{Representing gig worker's utility using the \gls{acr:mnl} model and deriving an analytical solution}
\label{sec:mnl}

To effectively utilize the tractability offered by the post-decision state formulation of the Bellman equation, it is essential to model the probability distribution for transitioning to each post-decision state. This necessitates a model that accurately describes gig worker behavior, and the \gls{acr:mnl} model is one of the most commonly used approaches in the literature on discrete choice behavior for both research and practical applications. Its popularity stems from the fact that it offers a closed-form expression for acceptance probabilities and its balance of simplicity and performance, which renders it both versatile and effective. Accordingly, we adopt the \gls{acr:mnl} model as the foundation for our analysis.

\noindent\textit{Multinomial Logit models:} In the \gls{acr:mnl} model, the random variables $\{e_{ij}\}_{i \in \mathcal{R}\hspace{0.05em}_{t}}$ of the utility function (see Equation \ref{eq:util}) are independent and identically distributed (i.i.d), following a Gumbel distribution. We utilize a special case of the MNL model which assumes a linear relationship between the offered compensation and Gumbel-distributed random variables with zero mean. Formally, this variation considers that the utility function of a gig worker $j$ for a request $i$ is:
\begin{equation} U_{ij} = u_{ij} + c_{i} + e_i \end{equation} where $u_{ij}$ is a value indicating the attractiveness of request $i$ to gig worker $j$, and whose value is determined by observable characteristics of the request, $c_i$ is the offered compensation to the gig worker for accepting the request, and $e_i$ are i.i.d. zero mean Gumbel variables with variance equal to $(\mu_j\pi)^2/6$ for some $\mu_j> 0$. Under this \gls{acr:mnl} model, when the request state is $\mathcal{R}\hspace{0.05em}_{t}$, the probability of a gig worker $j$ in time step $t$ accepting request $i$ for offered compensations $\mathbf{c_t} = (c^i_t)_{i \in \mathcal{R}\hspace{0.05em}_{t}}$ equals:
\begin{align} P^i_{t}(\mathbf{c_t}) = \frac{\exp((u_{ij} + c_t^i)/\mu_j)}{\sum_{l \in \mathcal{R}\hspace{0.05em}_{t}}(\exp((u_{lj} + c_t^j)/\mu_j) + \exp(u_0/\mu_j)},\label{eq:mnl1}\end{align} 
while the probability of gig worker $j$ not accepting any requests reads:
\begin{align} P^\emptyset_{t}(\mathbf{c_t}) = \frac{ \exp(u_0/\mu_j)}{\sum_{l \in \mathcal{R}\hspace{0.05em}_{t}}(\exp((u_{lj} + c_t^j)/\mu_j) + \exp(u_0/\mu_j)},\label{eq:mnl2}\end{align} 
where $u_0 \geq 0$ is a constant indicating the attractiveness of the all-reject alternative. 

In the simplest case, all gig workers share the same utility function. Formally, $u_{ij} = u_i$ for all requests $i \in \mathcal{R}\hspace{0.05em}_{t}$ and for all gig workers $j \in \mathcal{G}_t$, and $\mu_j = \mu$ for all gig workers $j \in \mathcal{G}_t$. However, to introduce variability in preferences among the population while maintaining model simplicity, we can assume that the population is divided into $D$ gig worker groups. This formulation corresponds to a Mixed Multinomial Logit (Mixed MNL) model, where within each group, individuals exhibit similar preferences. Formally, for each group $d \in [1,\dots,D]$, the utility function is $u_{ij} = u_i^d$  for all requests $i \in \mathcal{R}\hspace{0.05em}_{t}$ and all gig workers $j \in \mathcal{G}^d_t$, and $\mu_j = \mu_d$ for all gig workers $j \in \mathcal{G}^d_t$.

To solve Bellman equation \ref{eq:bell_post}, we adapt and extend the theory from \cite{dong2009dynamic}, which studies dynamic pricing in the context of inventory control of substitute products, to our problem setting. Specifically, we demonstrate that there exists an alternative formulation of the optimization problem defined by the post-decision state Bellman equation. This alternative formulation uses acceptance probabilities as decision variables and is concave. We then derive the optimal solution by solving the first-order condition.

To this end, we first reformulate the value function by applying Equation \eqref{eq:imm_rew} and representing $P^\emptyset_{t}(c_t^i)$ as $ 1 - \sum_{i \in \mathcal{R}\hspace{0.05em}_{t}}P^i_{t}(\mathbf{c_t})$ and derive the following lemma.
\begin{lemma}\label{lem:dynamic_programming}
The value of the pre-decision state $S\hspace{-0.1em}_t = (\mathcal{R}\hspace{0.05em}_{t},\mathcal{G}_t)$ is equal to:
$$ V_t(\mathcal{R}\hspace{0.05em}_{t},\mathcal{G}_t) = \max_{c_t} \{ \phi_t(\mathcal{R}\hspace{0.05em}_{t},\mathcal{G}_t,\mathbf{c_t}) \} + V^{\mathrm{p}}_{t}(\mathcal{R}\hspace{0.05em}_t^{'})  + \sum_{i \in \mathcal{R}\hspace{0.05em}_t^{\mathrm{exp}}}\! \beta_i $$
where 
$\phi_t(\mathcal{R}\hspace{0.05em}_{t},\mathcal{G}_t,\mathbf{c_t}) = \mathbbm{1}_{|\mathcal{G}_t|=1} \mathbb{E}_{i \sim P_t(\mathbf{c_t})} [r_i - c_t^i - \Delta^i_{V_{t}}(\mathcal{R}\hspace{0.05em}_t^{'}) -\beta_i\mathbbm{1}_{i \in \mathcal{R}\hspace{0.05em}_t^{\mathrm{exp}}}]$ and
$ \Delta^i_{V_{t}}(\mathcal{R}\hspace{0.05em}_t^{'}) = V^{\mathrm{p}}_{t}(\mathcal{R}\hspace{0.05em}_t^{'}) - V^{\mathrm{p}}_{t}(\mathcal{R}\hspace{0.05em}_t^{'} \setminus \{i\}).$

\end{lemma}
The proof of Lemma 1 can be found in Appendix \ref{sec:p_1}. 

As a result of this reformulation, we observe that finding the optimal prices $\mathbf{c_t}$ reduces to optimizing $\phi_t(\mathcal{R}\hspace{0.05em}_{t},\mathcal{G}_t,\mathbf{c_t})$. The function $\phi_t(\mathcal{R}\hspace{0.05em}_{t},\mathcal{G}_t,\mathbf{c_t})$ concerns only scenarios where a gig worker is available, as no decision is required otherwise. It accounts for the probability of each request $i \in \mathcal{R}\hspace{0.05em}_{t}$ being accepted given a compensation $c_t^i$, the corresponding reward and penalty, and an additional term $\Delta^i_{V_{t}}(\mathcal{R}\hspace{0.05em}_t^{'})$. The term $\Delta^i_{V_{t}}(\mathcal{R}\hspace{0.05em}_t^{'})$ reflects the difference in expected reward when request $i \in \mathcal{R}\hspace{0.05em}_{t}$ is accepted immediately versus when it remains unfulfilled, commonly referred to as the opportunity cost of request $i \in \mathcal{R}\hspace{0.05em}_{t}$. As demonstrated by \cite{hanson1996optimizing}, $\phi_t(\mathcal{R}\hspace{0.05em}_{t},\mathcal{G}_t,\mathbf{c_t})$ is not guaranteed to be concave in $\mathbf{c_t}$. 

We can address such optimization problems by expressing the compensation decision $c^t_i$ in terms of the acceptance probabilities. Using the \gls{acr:mnl} model to describe gig worker behavior, we combine Equations \eqref{eq:mnl1} and \eqref{eq:mnl2} to establish a bijection between the compensation \(c^t_i\) and the acceptance probabilities \(P_t^i\). Specifically, the compensation for each request can be expressed as a function of acceptance probabilities as follows:
\begin{align}
\frac{P_t^i(\mathbf{c_t})}{P_t^{\emptyset}(\mathbf{c_t})} = \exp((u_{ij} + c_t^i - u_0)/\mu_j) \Leftrightarrow c_t^i = -u_{ij} + u_0 + \mu_j \ln{P_t^i} - \mu_j \ln{P_t^{\emptyset}}.
\label{eq:mnl3}
\end{align}
Consequently, using the above expression we reformulate $\phi_t$ as a function of $P_t$ and establish the following result.
\begin{lemma}\label{lem:concavity}
$\phi_t(\mathcal{R}\hspace{0.05em}_{t},\mathcal{G}_t,P_t) = \mathbbm{1}_{|\mathcal{G}_t|=1}\mathbb{E}_{i \sim P_t}[r_i + u_{ij} - u_0 - \mu_j \ln{P_t^i} + \mu_j \ln{P_t^{\emptyset}}  - \Delta^i_{V_{t}}(\mathcal{R}\hspace{0.05em}_t^{'}) -\beta_i\mathbbm{1}_{i \in \mathcal{R}\hspace{0.05em}_t^{\mathrm{exp}}}]$ is concave in~$P_t$.
\end{lemma}
The proof of Lemma 2 can be found in Appendix \ref{sec:p_2}. 

By proving the concavity of $\phi_t$ as a function of $P_t$ we can solve the first order condition of $\phi_t$ and derive an optimal solution as a function of the acceptance probabilities $P_t$.
\begin{lemma}\label{lem:exact}
The optimal price $c^{i*}_t$ is given by:
$c^{i*}_t = r_i - \beta_i \mathbbm{1}_{i \in \mathcal{R}\hspace{0.05em}_t^{\mathrm{exp}}} 
 - \Delta^i_{V_{t}}(\mathcal{R}\hspace{0.05em}_t^{'}) - m_t$ where $m_t$ results from solving: 
 $(\frac{m_t}{\mu_j} - 1) \exp\{ \frac{m_t}{\mu_j} - 1\} = \sum_{i \in \mathcal{R}\hspace{0.05em}_{t}} \exp\{ \frac{1}{\mu_j} (r_i + u_{ij} - u_0 - \beta_i \mathbbm{1}_{i \in \mathcal{R}\hspace{0.05em}_t^{\mathrm{exp}}} - \Delta^i_{V_{t}}(\mathcal{R}\hspace{0.05em}_t^{'}) - \mu_j)\}$
\end{lemma}
The proof of Lemma \ref{lem:exact} can be found in Appendix \ref{sec:p_3}. 

Lemma \ref{lem:exact} states that the optimal price for a request~$i$ is equal to the reward of that request, reduced by the following terms: its penalty in the case that it expires within the current time step, the expected value discrepancy $\Delta^i_{V_{t}}(\mathcal{R}\hspace{0.05em}_t^{'})$ and the term $m_t$, which is a state-specific factor that adjusts pricing based on the overall system state rather than on individual requests. In essence, $m_t$ captures system-wide influences on the optimal pricing decision at any given time.

While we define feasible compensations as non-negative in our problem setting, the optimal compensation described in Lemma \ref{lem:exact} may result in negative values under certain conditions, e.g., due to specific worker-request dynamics or model assumptions. In practice, when implementing the optimal compensations, we set any negative values of $\mathbf{c_t} = (c^i_t)_{i \in \mathcal{R}\hspace{0.05em}_{t}}$ to zero to ensure all compensations remain feasible. Since gig workers would not accept requests with either negative or zero compensation, this adjustment aligns with realistic worker behavior and does not compromise the model's practical effectiveness. This adjustment allows for practical feasibility and ensures the model’s applicability in real-world scenarios, where negative compensations would be infeasible, while retaining most of the theoretical solution’s structure and insights.

\subsection{Value function approximation}
\label{sec:statistical_models}
Although our optimization problem has an analytical solution, derived in Section \ref{sec:mnl}, its dependence on the post-decision value function makes precise computation infeasible due to the vast and intractable state space. To overcome this, we employ statistical models to approximate the post-decision value function.

\noindent\underline{Statistical models for the post-decision value function:} A statistical model for the post-decision value function is defined as a function $\hat{V}$ parameterized by $\theta$ which receives as input a post-decision state $\mathcal{R}^{post}\hspace{0.05em}$ and predicts a value $\hat{v}$ for that state: $\hat{V}^\theta: \mathcal{R}^{post}\hspace{0.05em} \mapsto \hat{v} \in \mathbbm{R}$. The primary challenge in designing an effective statistical model to approximate the post-decision state value lies in the fact that the request state is represented as a set, containing a variable number of requests. Consequently, an effective statistical model for the post-decision value function approximation must be able to handle this dynamic structure. To account for such a variability, we utilize a neural network architecture that incorporates an attention mechanism.

\begin{figure}[b]
\centering
\resizebox{\textwidth}{!}{ % Use full width of the page
\begin{tikzpicture}[node distance=2cm]

    % Input node on the left, outside the enclosing box
    \node [block] (input) {$i \in \mathcal{R}^{post}\hspace{0.05em}$};

    % Horizontal structure of nodes inside the enclosing box
    \node [block, right=of input] (mlp_emb) {MLP$^{emb}$};
    \node [block, right=of mlp_emb] (mlp_w) {MLP$^W$};
    \node [block, right=of mlp_w] (cell) {Context \\[-0.3cm] $C = \sum_{i \in \mathcal{R}^{post}\hspace{0.05em}} \beta_{i} \cdot e_{i}$};

    % Enlarged enclosing box around MLP^emb, MLP^W, and Context nodes, including the extra arrow
    \node[fit=(mlp_emb) (mlp_w) (cell), draw, inner sep=10pt, rounded corners,  yshift=-0.18cm] (enclosing_box) {};

    % Arrows
    \draw [arrow] (input) -- (mlp_emb);
    \draw [arrow] (mlp_emb) -- node[above] {$e_i$} (mlp_w);
    \draw [arrow] (mlp_w) -- node[above] {$\beta_i$} (cell);
    
    % Direct arrow from MLP^emb to Context with centered label
    \draw [arrow] (mlp_emb) -- ++(0,-0.8cm) -| node[pos=0.12, above] {$e_i$} (cell);
\end{tikzpicture}
}
\caption{\textnormal{Attention mechanism}}
\label{fig:tikz_diagram}
\end{figure}

\noindent \textit{Attention mechanism:} The attention mechanism (see Figure \ref{fig:tikz_diagram}) calculates an embedding vector $e_i$ for each request $i \in \mathcal{R}^{post}\hspace{0.05em}$ using a multi-layer perceptron (MLP). Subsequently, $e_i$ is processed by a second MLP which performs the following calculation: $\beta_{i} = \sigma(w \cdot \tanh(W \cdot e_{i}))$, where $\sigma$ is the sigmoid function, $w$ is a trainable vector of weights, and $W$ is a trainable matrix of weights. Lastly, it calculates the context vector $C$ as $C = \sum_{i \in \mathcal{R}^{post}\hspace{0.05em}} \beta_{i} \cdot e_{i}$. The context vector is then given as an input to the rest of the neural network architecture. 

\noindent \textit{Neural network architecture:} The neural network for the post-decision value function approximation consists of an attention mechanism that comprises a feedforward layer with 32 units and a Swish activation \citep{ramachandran2017searching}, which provides the embedding vector $e_i$ for each request $i \in \mathcal{R}^{post}\hspace{0.05em}$. It then calculates the context vector using weights $w \in \mathbbm{R}^{64}$ and $W \in \mathbbm{R}^{64 \times 32}$. This context vector, along with other relevant state features (e.g., the number of requests, the most urgent deadline), constitutes the input to the subsequent layers of the network. These layers include two feedforward layers, each with 16 units and a Swish activation. Additional information about neural network training and hyperparameters can be found in Appendix \ref{sec:nn_app}. 

\noindent \underline{Algorithm for training the value function approximator:} To learn the parameters of the statistical model, we follow an approximate value iteration for the post-decision state value function, adapted from \cite{powell2021reinforcement}, to accommodate the continuous action space of our problem setting and the probabilistic transitions to post-decision states. Algorithm~1 shows a pseudocode of our learning procedure: initially, the algorithm receives estimates of the \gls{acr:mnl} parameters $\hat{\mathit{u}}$ and $\hat{\mu}$ (L.1). Then we randomly initialize a parameterization $\theta$ for the statistical model of the post-decision value function (L.2), and repeat the following procedure: first, we gather experiences by interacting with a simulated or real-world environment (L.3). During this time, we select the compensation by considering the estimated optimal price $\hat{c}_{t}$ using the latest estimate of the value function approximation, and we perturb the estimated optimal value using a Gaussian perturbation (L.4) to ensure exploration. After a sufficient amount of experiences is gathered, we update the parameterization $\theta$ (L.5-L.8). For updating the post-decision value function approximation around the observed post-decision state $\mathcal{R}\hspace{0.05em}_{t-1}^{post}$ of time step $t-1$, we use the (estimated) optimal value $v_t^*$ of the observed pre-decision state $S\hspace{-0.1em}_t$ of the following time step $t$.

\begin{algorithm}[b!]
\caption{Approximate value iteration using the post-decision value function}
\begin{algorithmic}[1]  % The [1] here enables line numbering
\REQUIRE MNL model utilities $\hat{\mathit{u}}$, and Gumbel parameter $\hat{\mu}$
\STATE Initialize a value function parameterization $\theta$ for $\hat{V}^\theta$ 
\FOR{episode = 1,\ldots,M}
    \STATE Step 1. Gather experiences using the e-greedy policy:
    \STATE \hspace{\algorithmicindent} $\begin{aligned}[t]
            \mathbf{\hat{c}}_{t} = (\max(0,r_i - \beta_i \mathbbm{1}_{i \in \mathcal{R}\hspace{0.05em}_t^{\mathrm{exp}}} - \Delta_{\hat{V}}^i(\mathcal{R}\hspace{0.05em}_t^{'}) - m_t + \varepsilon_{i}))_{i \in \mathcal{R}\hspace{0.05em}_{t}} 
            \end{aligned}$
            where $\varepsilon_{i} \sim \mathcal{N}(0,\delta)$  $\forall i \in \mathcal{R}\hspace{0.05em}_{t}$
    \STATE Step 2. Update $\theta$ using the target:
    \STATE \hspace{\algorithmicindent} $\begin{aligned}[t]
            \hat{V}^{\theta}(\mathcal{R}\hspace{0.05em}_{t-1}^{post}) \leftarrow v_t^*
            \end{aligned}$
    \STATE where $\mathcal{R}\hspace{0.05em}_{t-1}^{post}$ is the \underline{observed} post-decision state at time step $t-1$ and $v_t^*$ is the estimated optimal value of the successor pre-decision state $S\hspace{-0.1em}_t$:     
    \STATE \hspace{\algorithmicindent} $\begin{aligned}
            v_t^* = \max_{\mathbf{c_t}} \{ R_t(\mathbf{c_t}) +  \gamma \mathbbm{1}_{|\mathcal{G}_t|=1} \sum_{i \in \mathcal{R}\hspace{0.05em}_t\cup\{\emptyset\}}  P_t^i(\mathbf{c_t}) \cdot \hat{V}^{\theta}_t(\mathcal{R}\hspace{0.05em}_t^{'}\backslash\{i\}) \} + (1 - \mathbbm{1}_{|\mathcal{G}_t|=1}) \cdot \hat{V}^{\theta}_t(\mathcal{R}\hspace{0.05em}_t^{'}).
            \end{aligned}$
    \ENDFOR
\end{algorithmic}
\end{algorithm}

\subsection{Training procedure}
\label{sec:training}
\noindent \underline{Statistical model of the gig worker's utility:} In practice, the true \gls{acr:mnl} model parameters for the utilities $u_{ij}$ and the Gumbel parameter $\mu$ are not known. To estimate the parameters of the \gls{acr:mnl} model for each gig worker group, we use observed accept/reject gig worker decisions based on the characteristics of the on-demand request $\mathbf{x}_i$ and offered compensation $c_i$. This data comes either from interactions with the environment under any reasonable policy or from pre-existing historical data. For each gig worker group $d~\in~[1,\dots,D]$ we train a logistic regression model using the log-odds function: $(\mathbf{\hat{w}}_d^T\mathbf{x}_i + c_i)/\beta_d$ where $\mathbf{\hat{w}}_d$ and $\beta_d$ are trainable parameters. Finally, we estimate the utility $u_{ij}$ of a gig worker $j$ for request $i$ using the mapping $\hat{\mathit{u}}_j : \mathbf{x}_i \mapsto \mathbf{\hat{w}}_d^T \mathbf{x}_i$, so that $\hat{u}_{ij} = \hat{\mathit{u}}_j(\mathbf{x}_i)$, and $\beta_d$ by $\hat{\mu}_d = \beta_d$.

\noindent \underline{Training pipeline:} Our training procedure for each scenario involves the following steps: Initially, we gather experiences by interacting with training scenarios. The compensation policy employed sets the compensation for each request $i$  by randomly selecting a value between 40\%-85\% of the request's reward. From these experiences, we train the \gls{acr:mnl} estimator as previously defined, using stochastic gradient descent for optimization. Since the gig worker always chooses the offer that maximizes their utility, the data skews toward higher-compensation offers, especially when the sampling policy is non-optimal and tends to over-offer. Therefore, we use \(L2\) regularization on the weight $\beta_d$ to prevent the weight $\beta_d$ from increasing excessively. We then proceed to train the post-decision value function approximation as outlined in Algorithm 1. In order to mitigate the effect of the network weight initialization, we repeat this process with 5 different random seeds, resulting in 5 distinct models. We select the final model based on the one that demonstrates the best performance on the validation scenarios.

\subsection{Discussion}
\label{sec:meth_as}
Our proposed algorithmic paradigm adopts a model-free approach in many aspects, while it relies on the \gls{acr:mnl} model to capture gig worker behavior. As a result, it requires some knowledge of gig worker groups or at least observable characteristics of gig workers. Below, we discuss key considerations and limitations of our algorithm.

\noindent \textit{Modeling gig worker utility:} Modeling all stochastic aspects of the environment is impractical due to the complexity of real-world scenarios. Instead, selectively modeling key elements provides a practical balance between model-free and model-based approaches. For example, while the variability in gig worker arrivals and requests (e.g., weekday vs. weekend patterns) is too complex to be modeled precisely, focusing on gig worker decision-making is reasonable. In our approach, we use the \gls{acr:mnl} model to represent gig worker decisions. This model is widely used in research and practice, offering flexibility to capture diverse preferences across gig worker groups. However, it assumes a specific mathematical structure (e.g., independence of irrelevant alternatives), which may not fully reflect real-world decision-making. Despite this, its practicality and generalizability justify its use in our algorithm.

\noindent \textit{Knowledge of gig worker groups:} When the gig worker population displays heterogeneous preferences, i.e., when more that one gig worker group exists, the performance of our algorithm relies on having knowledge of the different groups. However, the availability of such information varies across platforms, influencing their ability to identify groups with similar preferences. Platforms seeking to identify potential groups can refer to existing studies, such as \cite{marcucci2017connected}, \cite{bathke2023occasional}, and \cite{miller2017crowdsourced}, which offer valuable insights into gig worker decision-making and subgroup behavior.


\section{Experiments}


\subsection{Experimental Settings}
\paragraph{Datasets} We trained SyncSpeech on datasets in both English and Mandarin, including the 585-hour LibriTTS \cite{libritts} dataset and 600 hours of internal Mandarin datasets. The internal Mandarin dataset was further expanded to approximately 2000 hours, employing techniques such as speed alteration and pitch shifting. The Montreal Forced Aligner (MFA) \cite{mfa}  aligned transcripts according to its phone set, after which the alignment was transformed into text BPE-level format. We evaluated SyncSpeech using three benchmarks: (1) LibriSpeech \textit{text-clean} \cite{librispeech}, a standard English TTS evaluation set; (2) SeedTTS \textit{test-zh} \cite{seedtts}, with 2,000 samples from the out-of-domain Mandarin DiDiSpeech dataset \cite{didispeech}; and (3) SeedTTS \textit{test-hard}, containing approximately 400 difficult cases to evaluate TTS model robustness with repeated text, tongue twisters, and other complex synthesis scenarios. 

\paragraph{Settings} 
We set the number of text tokens to look ahead $q=1$. The chunk size of speech decoder is 15. 
TMT has 16 layers, 16 attention heads, 1024-dimensional
embeddings, and 2048-dimensional feed-forward layers. 
SyncSpeech was trained on 4 NVIDIA A800 80G GPUs. 
The pre-training stage lasts for 70K steps, and the second stage lasts for 20K steps. 

\paragraph{Baseline Models}
This paper focuses on low-latency and efficient TTS in dual-stream scenarios. Under the same data scale, we reproduced the following baseline models for comparison: CosyVoice \cite{cosyvoice} and recently proposed CosyVoice2 \cite{cosyvoice2.0}. CosyVoice requires complete text input before speech generation. 
CosyVoice2 uses interleaved text-speech modeling to process streaming text input and simultaneously generate streaming speech. We trained CosyVoice, CosyVoice2, and SyncSpeech using the same speech tokenizer and text tokenizer, and employed the same open-source streaming speech decoder. We utilized the official code\footnote{https://github.com/FunAudioLLM/CosyVoice} to reproduce the model and adopted a Llama-style Transformer, matching the size of SyncSpeech, as the backbone of the text-to-speech model.  Additionally, we compared the open-sourced TTS model MaskGCT \cite{maskgct}, F5-TTS \cite{F5tts}, and VALL-E \cite{valle}, which were trained on large-scale data.
 More details about baseline models can be found in the Appendix \ref{baselines}.


\paragraph{Evaluation Metrics} For the three benchmarks, we evaluated
speech quality, latency, and  efficiency. 
For speech robustness, we chose Whisper-V3 and Paraformer as the ASR models for English and Mandarin, respectively, to transcribe the generated speech. Then, we calculated the WER compared to the original transcriptions to evaluate the spech robustness. We adopted the ERes2Net-based \cite{eres2net} speaker verification model\footnote{https://github.com/modelscope/3D-Speaker} to evaluate speaker similarity (SS). We selected 100 sentences from each system and invited 10 native listeners to conduct a subjective MOS evaluation for speech naturalness (MOS-N), scoring from 1 to 5. 
In terms of latency and efficiency, we compared the performance of various models on a single A800 GPU. 
Due to the off-the-shelf speech decoder, we evaluate the latency and efficiency of the text-to-token stage across all models, except for F5-TTS.
We calculated the time required for the number of speech tokens to reach the chunk size of the speech decoder as First-packet latency (FPL). There are two scenarios: one assumes the text is already available (FPL-A), while the other involves receiving output from the upstream LLM model (FPL-L), accounting for the time required for text generation.
 For the real-time factor (RTF), we measure the ratio of the total duration of generated speech to the total time taken by the model. More details about FPL and RTF can be found in the Appendix \ref{evaluation metrics}.

\begin{table*}[t]
\centering
\resizebox{0.99\textwidth}{!}{
\begin{tabular}{lccccccccc}
\toprule
\textbf{Model} & \textbf{\#Scenario}  & \textbf{\#Data(hrs)}   & \textbf{WER(\%)} $\downarrow$   & \textbf{SS(\%)} $\uparrow$ & \textbf{FPL-A(s)}$\downarrow$ & \textbf{FPL-L(s)} $\downarrow$ & \textbf{RTF(\%)} $\downarrow$ & \textbf{MOS-N} $\uparrow$ \\ \hline
\multicolumn{10}{c}{\textbf{LibriSpeech \textit{test-clean}}}   \\ \hline
\textbf{Ground Truth} &- &-  & 2.12   & 69.67 &- &- &-   &$\text{4.62}_{\pm 0.12}$       \\ \hdashline
\textbf{F5-TTS*} & Offline & 100K Multi.  & \textbf{2.51} & \textbf{73.10} &1.27 &1.98 &0.23  &-  \\
\textbf{MASK-GCT*} & Offline &100K Multi. &2.77 & 70.81 &2.15 &2.55 & 0.37 &- \\
\textbf{VALL-E*}  & Output Stream & 60K EN & 5.90 & 59.71 & 0.75 &1.47 &1.41 &- \\
\textbf{CosyVoice} & Output Stream & 585 EN  & 3.47  & \underline{63.52} &0.22 &0.94 &0.45   & $\text{4.39}_{\pm 0.12}$           \\ 
\textbf{CosyVoice2} & Dual-Stream & 585 EN   & \underline{3.00}      & 63.48 &0.22 &0.35 &0.45   &$\textbf{\text{4.48}}_{\pm 0.13}$           \\
\textbf{SyncSpeech} & Dual-Stream & 585 EN    & 3.07    & 63.47 & \textbf{0.06} &\textbf{0.11} &\textbf{0.07}   &$\textbf{\text{4.48}}_{\pm 0.14}$         \\ \hline
\multicolumn{10}{c}{\textbf{Seed \textit{test-zh}}}   \\ \hline
\textbf{Ground Truth} &-  &- & 1.26  & 75.15  &- &- &- & $\text{4.68}_{\pm 0.10}$      \\ \hdashline
\textbf{CosyVoice}  & Output Stream  &2K ZH      & 3.03    & 61.51  &0.22 &0.62 &0.43  & $\text{4.34}_{\pm 0.14}$            \\ 
\textbf{CosyVoice2} & Dual-Stream  &2K ZH & 3.31      & 61.89   &0.22 &0.35 &0.43 & $\text{4.37}_{\pm 0.13}$           \\
\textbf{SyncSpeech}& Dual-Stream  &2K ZH  & \textbf{2.38}    & \textbf{62.14}   & \textbf{0.04} & \textbf{0.09} &\textbf{0.05} &$\textbf{\text{4.45}}_{\pm 0.11}$           \\
\hline
\multicolumn{10}{c}{\textbf{Seed \textit{test-hard}}}   \\ \hline
\textbf{CosyVoice} & Output Stream  &2K ZH    & 26.26    & 66.71   &0.22 &1.22 &0.44 & $\text{3.84}_{\pm 0.15}$            \\ 
\textbf{CosyVoice2} & Dual-Stream  &2K ZH  & 21.61  & 67.13 &0.22 &0.35 &0.44      &$\text{3.86}_{\pm 0.14}$            \\
\textbf{SyncSpeech} & Dual-Stream &2K ZH  & \textbf{17.21}    & \textbf{67.21}  &\textbf{0.05} & \textbf{0.10} &\textbf{0.08} &$\text{3.86}_{\pm 0.11}$         \\
\bottomrule
\end{tabular}
}
\caption{The evaluation results of SyncSpeech and baseline models across the three benchmarks. * indicates the model trained on the large-scale dataset. Underline indicates the best performance in terms of WER and SS with the 585 hours training scale. \#Data refers to the used training dataset in hours.}
\label{table1}
\end{table*}


\subsection{Main Results}
The evaluation results for SyncSpeech and the baseline models are presented in Table \ref{table1}. 

\paragraph{Speech Robustness} 
We found that SyncSpeech exhibits different performance compared to the baselines across the three benchmarks. Specifically, on the LibriSpeech \textit{test-clean} benchmark, the performance of SyncSpeech was very close to that of CosyVoice2 based on the WER metric, with only a minor difference of 0.07\%. SyncSpeech achieved a lower WER score on the Seed \textit{test-zh} set compared to CosyVoice and CosyVoice2, with improvements of 0.65\% and 0.93\%, respectively.  A key difference between the English and Mandarin datasets is the higher compression rate of the LLM tokenizer for Mandarin. In English, one word typically equals one token, while in Mandarin, a common phrase often corresponds to a single token.
This means that, compared to the baseline model, SyncSpeech is better suited to the high compression rate tokenizer of the upstream large model. Furthermore, on the Seed \textit{test-hard} set, the robustness advantage of SyncSpeech was even more pronounced, with the improvements 9.05\% and 4.40\%, respectively. In handling complex text, the explicit duration modeling in SyncSpeech helped the model learn the alignment between text and speech.

\paragraph{Speaker Similarity} Due to the same speech decoder and the excellent voice disentanglement capability of the speech tokens, SyncSpeech, CosyVoice, and CosyVoice2 exhibited similar performance in terms of speaker similarity.
\paragraph{Speech Naturalness} The MOS-N scores for SyncSpeech and CosyVoice2 were quite similar on the LibriSpeech \textit{text-clean}, indicating that the naturalness of the generated speech was generally comparable. On the Seed \textit{test-zh} benchmark, SyncSpeech outperformed CosyVoice2 by 0.08.  In the Seed \textit{test-hard} benchmark, high WER and uncommon text led to unnatural prosody and generally low MOS-N scores in the generated speech.
\paragraph{Latency} SyncSpeech has made a breakthrough in terms of latency, as shown in Table \ref{table1}. Specifically, on the LibriSpeech \textit{test-clean} benchmark, SyncSpeech was approximately 4 times faster than traditional AR models and over 20 times faster than the SOTA offline models in terms of FPL-A. On the Seed \textit{test-zh} benchmark, SyncSpeech achieved speed improvements of over 5 times and 30 times, respectively. When receiving streaming text from the upstream large model (FPL-L), SyncSpeech can begin generating speech with just two text tokens. In contrast, CosyVoice2 requires five tokens, while CosyVoice and other baseline models need the entire text input. This highlights the distinct advantage of SyncSpeech in practical applications.

\paragraph{Efficiency} In terms of RTF, SyncSpeech is about 6.4 times faster on the LibriSpeech \textit{test-clean} benchmark and about 8.6 times faster on the Seed \textit{test-zh} benchmark compared to previous AR models. 
On the Seed \textit{test-hard} set, due to the increased number of text tokens caused by the uncommon text, the efficiency of SyncSpeech is slightly reduced. Theoretically, the time complexity of AR models is $O(T)$, while the time complexity of SyncSpeech is  $O(L)$, where  $T$ represents the number of speech tokens and 
$L$ denotes the number of text tokens, thereby significantly improving efficiency.

\section{Analysis}
\paragraph{Sampling Strategy} In the LibriSpeech validation set, we provided the ground-truth durations and applied greedy search along with different Top-k thresholds for duration prediction, as shown in Table \ref{table3}. We found that, in terms of speech robustness, both Top-k 3 and greedy search outperformed the use of ground-truth durations in terms of the WER metric. This is because the model struggled to effectively generalize to anomalies in the ground-truth durations. We employed
UTMOSv2\footnote{https://github.com/sarulab-speech/UTMOS22} as a surrogate objective metric of MOS-N. In terms of speech naturalness, the results of Top-k 3 sampling are slightly better than those with the given ground-truth durations.  Additionally, we applied different Top-k thresholds for speech token prediction. SyncSpeech exhibited superior performance during greedy search, which is different from the previous AR TTS models or offline models. This is because the speech tokens obtained through single-step decoding have the temporal dependency, which cannot be compensated by subsequent generation. 


\begin{table}[]
\centering
\resizebox{0.42\textwidth}{!}{
\begin{tabular}{lcc}

\toprule
\textbf{Sampling Strategy}       & \textbf{WER(\%)}$\downarrow$  & \textbf{UTMOSv2}$\uparrow$ \\ \hline
\multicolumn{3}{c}{Duration Prediction} \\
\hline
Ground Truth            & 2.59 & 3.45   \\ \hdashline 
Greedy Search   & 2.50 & 3.44   \\
Top-k 3         & \textbf{2.44} & \textbf{3.46}   \\
Top-k 5         & 2.93 & 3.44   \\
Top-k 10        & 2.76 & 3.41  \\ 
\hline
\multicolumn{3}{c}{Speech Token Prediction} \\
\hline
Greedy Search & \textbf{2.44} & \textbf{3.46}   \\
Top-k 3          & 3.82 & 3.43  \\
Top-k 5          & 4.23 & 3.43   \\ 
\bottomrule
\end{tabular}
}
\caption{Performance across various Top-k thresholds for duration prediction and speech token prediction on the LibriTTS validation set.}
\label{table3}
\end{table}

\paragraph{Number of Look-ahead Tokens}
We evaluated how varying the number of tokens to look ahead affects speech robustness and speech naturalness on two validation sets, with the results presented in Table \ref{table5}. We discovered that the optimal number of look-ahead text tokens varies across different languages in terms of WER performance.  This is influenced by the difference in the compression rate of text tokens and the contextual dependency in different languages. In terms of speech naturalness, when the look-ahead number $q$ is greater than $2$, the generated speech exhibits slightly more natural pauses and speed, but it results in increased latency.

\paragraph{Ablation Study}
We conducted an ablation study on the pre-training strategy by directly training the randomly initialized model in a manner consistent with the prediction process. The WER results on the two validation sets are shown in Table \ref{table6}. We found that pre-training significantly improved the speech robustness of the model, improving the WER metric by 1.17\% and 1.06\% on the two languages, respectively. This indicated that masked pre-training not only improved training efficiency but also enhanced the robustness of the synthesized speech. Additionally, a standard causal attention mask was applied to replace the designed attention mask, as shown in Table \ref{table6}. If the mask token sequence of the same text token cannot attend to each other during inference, the robustness of the generated speech significantly decreased. This further demonstrated the effectiveness of the designed attention mask.



\begin{table}[]
\centering
\resizebox{0.49\textwidth}{!}{
\begin{tabular}{lcccc}
\toprule
                    & \textbf{LH Num.}           & \textbf{WER(\%)}$\downarrow$  & \textbf{FPL-L(s)}$\downarrow$  & \textbf{UTMOS-v2}$\uparrow$  \\ \hline
\multirow{4}{*}{EN} & q=1    & \textbf{2.44}    & \textbf{0.11}    & 3.46     \\
                    & q=2    & 2.87    & 0.13   & 3.41       \\
                    & q=3    & 2.52   & 0.16    & \textbf{3.48}         \\ 
                    & q=4    & 2.52    & 0.19    &  \textbf{3.48}     \\          \hline
\multirow{4}{*}{ZH} & q=1   & 2.51 & \textbf{0.09} & -    \\
                    & q=2   & 2.49 & 0.12  & -   \\ 
                    & q=3   & \textbf{2.41} &0.14 & - \\
                    & q=4   & \textbf{2.41} &0.17 & - \\
\bottomrule
                    
\end{tabular}
}
\caption{Performance with different numbers of look-ahead text tokens across two validation sets.}
\label{table5}
\end{table}




\begin{table}[]
\centering
\resizebox{0.39\textwidth}{!}{
\begin{tabular}{lcc}

\toprule
      & \textbf{English}   &\textbf{Mandarin}   \\ \hline
SyncSpeech         & \textbf{2.44} & \textbf{2.41}   \\
w/o pretrain        &  3.61 & 3.47  \\
w/o designed Mask  & 8.19 & 7.97 \\ 
\bottomrule 
\end{tabular}
}
\caption{WER (\%) results of the ablation study across the two validation sets. }
\label{table6}
\end{table}




\section{Conclusion}


In this work, we introduced \ours, a pivot-based single model ensemble framework, to enhance translation in scenarios where parallel data are scarce.
By transferring knowledge from diverse pivot languages, we were able to obtain not only diverse but also high-quality candidates.
And the optimal path to generating the best candidate varies per sentence, our study underscores the significance of exploiting a spectrum of pivot languages.
Moreover, the single model generation process offers cost savings compared to multi-model ensemble approaches. 
Empirical results and qualitative analyses show that the proposed method can yield contextually suitable translations for the given source sentences by leveraging pivoted candidates.


\bibliography{aaai25}

\newpage
\appendix

\section*{Appendix}


% Appendix for AAAI-2025 submission
% Outline: 
% 1, additional results on augmented
% 3, baseline implementation
% 6, target correlation score?

\subsection{Datasets}

NPHA: \url{https://archive.ics.uci.edu/dataset/936/national+poll+on+healthy+aging+(npha)}

Obesity: \url{https://www.kaggle.com/code/mpwolke/obesity-levels-life-style}

Diabetes: \url{https://archive.ics.uci.edu/dataset/34/diabetes}

Churn Modeling: \url{https://www.kaggle.com/datasets/shrutimechlearn/churn-modelling}

Adult: \url{https://archive.ics.uci.edu/dataset/2/adult}

German Credit: \url{https://archive.ics.uci.edu/dataset/522/south+german+credit}


\subsection{Baseline Implementation}

\label{sec:baseline}
% Details on baseline implementations like hyperparameter
\textbf{SMOTE:} We used the implementation provided at https://github.com/yandex-research/tab-ddpm, without balancing for target class frequency. 

\textbf{CTGAN:} We use the official implementation at https://github.com/sdv-dev/CTGAN. We use embedding dimension =128, generator dimension=(256,256), discriminator dimension =(256,256), generator learning rate=0.0002, generator decay =0.000001, discriminator learning rate =0.0002, discriminator decay =0.000001, batch size=500, training epoch = 300, discriminator steps=1, pac size = 5.

\textbf{TabDDPM:} We used the official implementation at https://github.com/yandex-research/tab-ddpm. We used 2500 diffusion steps, 10000 training epochs, learning rate = 0.001, weight decay = 1e-05, batch size = 1024.

\textbf{AIM:} We use the code implementation at https://github.com/ryan112358/private-pgm, with default parameters: epsilon=3,delta=1e-9,max model size=80

\textbf{PATE-CTGAN:} We adapted the implementation posted at: https://github.com/opendp/smartnoise-sdk/blob/main/synth/snsynth, which combines the PATE~\cite{jordon2018pate} learning framework with CTGAN. We use epsilon = 3, 5 iterations for student and teacher network, and the same value for other parameters which are shared with CTGAN.

\textbf{GReaT:} We used the official implementation at \url{https://github.com/kathrinse/be_great/tree/main}. We used a batch size of 64 and save steps of 400000. We following the pre-training pipeline outlined in~\cite{zhao2023tabula}. During pre-training, we began with a randomized distilgpt2 model. For each pre-training dataset, we iteratively loaded the latest model, fitted the model on the new dataset, and saved the model to be used for the next iteration. We used 50 epochs for the health model and 200 epochs for the out-of-domain model. For finetuning, we fitted the pre-trained model on only one dataset using 200 epochs. The dataset used in finetuning was the dataset we wished to emulate during synthesis. For reference, we also fitted a newly randomized model on the data alone to serve as a base metric. During generation, we synthesized the same number of samples as the finetune dataset.

\subsection{Data Augmentation Utility}

\begin{table*}
\begin{tabular}{llllllll}
\toprule
Model & Churn & NPHA & Obesity & Adult & Diabetes & Credit & AvgRank \\
\midrule
Real Data & 0.79{(0.03)} & 0.52{(0.02)} & 0.96{(0.03)} & 0.86{(0.04)} & 0.78{(0.01)} & 0.76{(0.03)} & 2.83 \\
\hline
AIM & 0.80{(0.04)} & 0.53{(0.02)} & 0.95{(0.02)} & \textbf{0.87}{(0.03)} & 0.76{(0.04)} & 0.73{(0.02)} & 4.58 \\
SMOTE & 0.80{(0.02)} & 0.49{(0.04)} & \textbf{0.96}{(0.02)} & 0.85{(0.01)} & 0.77{(0.04)} & 0.76{(0.03)} & 4.33 \\
CTGAN & 0.77{(0.02)} & 0.51{(0.03)} & 0.96{(0.01)} & 0.86{(0.04)} & 0.76{(0.02)} & 0.74{(0.04)} & 5.92 \\
PATECTGAN & 0.74{(0.03)} & \textbf{0.54}{(0.01)} & 0.94{(0.02)} & 0.86{(0.03)} & 0.70{(0.01)} & 0.76{(0.04)} & 5.08 \\
TabDDPM & 0.79{(0.02)} & 0.51{(0.03)} & 0.95{(0.04)} & 0.86{(0.01)} & 0.77{(0.03)} & 0.75{(0.01)} & 5.42 \\
\hline
GReaT & 0.79{(0.01)} & 0.52{(0.04)} & 0.96{(0.02)} & 0.87{(0.04)} & 0.79{(0.02)} & 0.75{(0.04)} & 4.40 \\
TabSynRL & \textbf{0.80}{(0.04)} & 0.52{(0.01)} & 0.96{(0.03)} & 0.87{(0.04)} & \textbf{0.81}{(0.02)} & \textbf{0.77}{(0.01)} & 2.00 \\
\bottomrule
\end{tabular}
\caption{Average ROC AUC of classifiers trained on real data augmented with different synthetic data. The best performing score for each dataset is highlighted in bold.}
\label{tab:augmented}
\end{table*}

Table~\ref{tab:augmented} presents the average ROC AUC scores for classifiers trained on real data augmented with different synthetic data generators across six datasets, along with their average rank. Among all methods, TabSynRL achieved the highest utility, securing the top average rank across datasets and outperforming prior state-of-the-art (SOTA) models, including TabDDPM and CTGAN. SMOTE, while effective, ranked slightly lower than TabSynRL, highlighting the limitations of interpolation-based techniques when compared to advanced generative approaches. Notably, differential privacy-preserving models like AIM and PATECTGAN exhibited lower utility due to the inherent noise introduced for privacy protection. The real data model performed competitively, but the enhancement brought by TabSynRL illustrates the benefit of augmenting data with RL fine-tuned generators. Moreover, despite the focus on conditional generation, models like TabDDPM and GReaT did not demonstrate superior utility, underscoring the significance of learning \(P(y | \mathbf{X})\) rather than \(P(\mathbf{X} | y)\) for downstream ML applications.


\subsection{Similarity of Correlation}


\begin{table*}

\begin{tabular}{llllllll}
\toprule
Model & Churn & NPHA & Obesity & Adult & Diabetes & Credit & AvgRank \\
\midrule
Real Data & 0.97{(0.04)} & 0.91{(0.02)} & 0.92{(0.01)} & 0.98{(0.03)} & 0.91{(0.02)} & 0.89{(0.04)} & 1.25 \\
\hline
SMOTE & \textbf{0.96}{(0.03)} & \textbf{0.91}{(0.01)} & 0.87{(0.02)} & \textbf{0.96}{(0.02)} & 0.91{(0.03)} & \textbf{0.88}{(0.01)} & 2.25 \\
AIM & 0.88{(0.01)} & 0.87{(0.03)} & 0.65{(0.04)} & 0.80{(0.02)} & 0.80{(0.04)} & 0.68{(0.03)} & 5.67 \\
CTGAN & 0.89{(0.02)} & 0.86{(0.04)} & 0.70{(0.01)} & 0.81{(0.03)} & 0.81{(0.02)} & 0.60{(0.04)} & 5.50 \\
PATECTGAN & 0.59{(0.03)} & 0.84{(0.02)} & 0.35{(0.01)} & 0.30{(0.03)} & 0.65{(0.02)} & 0.22{(0.01)} & 7.50 \\
TabDDPM & 0.95{(0.01)} & 0.90{(0.04)} & \textbf{0.89}{(0.03)} & 0.92{(0.02)} & 0.90{(0.01)} & 0.76{(0.04)} & 3.17 \\
GReaT & 0.75{(0.03)} & 0.75{(0.02)} & 0.82{(0.01)} & 0.88{(0.04)} & \textbf{0.92}{(0.03)} & 0.70{(0.01)} & 4.83 \\
TabSynRL & 0.87{(0.02)} & 0.75{(0.04)} & 0.86{(0.03)} & 0.98{(0.04)} & 0.91{(0.01)} & 0.66{(0.02)} & 5.40 \\
\bottomrule
\end{tabular}
\caption{Pairwise column distribution similarity between real test set and synthetic data across models. Higher value indicates greater similarity. Real Data represents training sets for synthesizers.}
\label{tab:fidelityCorr}
\end{table*}

Table~\ref{tab:fidelityCorr} presents the pairwise column distribution similarity between real test sets and synthetic data generated by various models across six datasets, along with the average rank for each model. The Real Data benchmark, as expected, ranks highest with near-perfect similarity scores. SMOTE closely follows, demonstrating competitive performance with a high fidelity to real data distributions across datasets. TabDDPM, which also ranks well, shows strong alignment with real data, particularly in datasets like Obesity and Adult, reflecting its ability to maintain distributional integrity. AIM and CTGAN perform moderately, with their scores indicating some drop in fidelity, especially in complex datasets where capturing nuanced patterns is challenging. Models like PATECTGAN and GReaT rank lower, with PATECTGAN struggling significantly, likely due to the noise introduced by privacy-preserving mechanisms. Interestingly, TabSynRL, while shown superior performance in machine learning utility, did not consistent excel in preserving column correlation in real data compared to other baselines. This provides further evidence that TabSynRL`s performance gain is not simply due to improved statistical fidelity or improvement in generative modeling objective.


\subsection{Reproducibility Checklist}

This paper:

\begin{itemize}
    \item Includes a conceptual outline and/or pseudocode description of AI methods introduced \textbf{yes}
    \item Clearly delineates statements that are opinions, hypothesis, and speculation from objective facts and results \textbf{Yes}
    \item Provides well marked pedagogical references for less-familiare readers to gain background necessary to replicate the paper \textbf{Yes}
\end{itemize}

Does this paper make theoretical contributions? (\textbf{yes}/no)

Does this paper rely on one or more datasets? (\textbf{yes}/no)

If yes, please complete the list below.

\begin{itemize}
    \item A motivation is given for why the experiments are conducted on the selected datasets (\textbf{yes}/partial/no/NA)
    \item All novel datasets introduced in this paper are included in a data appendix. (yes/partial/no/\textbf{NA: no novel dataset used.})
    \item All novel datasets introduced in this paper will be made publicly available upon publication of the paper with a license that allows free usage for research purposes. (yes/partial/no/\textbf{NA: no novel dataset used.})
    \item All datasets drawn from the existing literature (potentially including authors’ own previously published work) are accompanied by appropriate citations. (\textbf{yes}/no/NA)
    \item All datasets drawn from the existing literature (potentially including authors’ own previously published work) are publicly available. (\textbf{yes}/partial/no/NA)
    \item All datasets that are not publicly available are described in detail, with explanation why publicly available alternatives are not scientifically satisficing. (yes/partial/no/\textbf{NA: all datasets are publically available})
\end{itemize}

 Does this paper include computational experiments? (\textbf{yes}/no)

If yes, please complete the list below.

\begin{itemize}
    \item Any code required for pre-processing data is included in the appendix. (\textbf{yes}/partial/no).
    \item All source code required for conducting and analyzing the experiments is included in a code appendix. (yes)
    \item All source code required for conducting and analyzing the experiments will be made publicly available upon publication of the paper with a license that allows free usage for research purposes. (yes)
    \item All source code implementing new methods has comments detailing the implementation, with references to the paper where each step comes from. (yes)
    \item If an algorithm depends on randomness, then the method used for setting seeds is described in a way sufficient to allow replication of results. (yes)
    \item This paper specifies the computing infrastructure used for running experiments (hardware and software), including GPU/CPU models; amount of memory; operating system; names and versions of relevant software libraries and frameworks. (yes)
    \item This paper formally describes evaluation metrics used and explains the motivation for choosing these metrics. (yes)
    \item This paper states the number of algorithm runs used to compute each reported result. (yes)
    \item Analysis of experiments goes beyond single-dimensional summaries of performance (e.g., average; median) to include measures of variation, confidence, or other distributional information. (yes)
    \item The significance of any improvement or decrease in performance is judged using appropriate statistical tests (e.g., Wilcoxon signed-rank). (yes)
    \item This paper lists all final (hyper-)parameters used for each model/algorithm in the paper’s experiments. (yes)
    \item This paper states the number and range of values tried per (hyper-)parameter during the development of the paper, along with the criterion used for selecting the final parameter setting. (yes)
\end{itemize}



\end{document}
