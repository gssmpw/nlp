%File: anonymous-submission-latex-2025.tex
\documentclass[letterpaper]{article} % DO NOT CHANGE THIS
\usepackage[submission]{aaai25}  % DO NOT CHANGE THIS
\usepackage{times}  % DO NOT CHANGE THIS
\usepackage{helvet}  % DO NOT CHANGE THIS
\usepackage{courier}  % DO NOT CHANGE THIS
\usepackage[hyphens]{url}  % DO NOT CHANGE THIS
\usepackage{graphicx} % DO NOT CHANGE THIS
\urlstyle{rm} % DO NOT CHANGE THIS
\def\UrlFont{\rm}  % DO NOT CHANGE THIS
\usepackage{natbib}  % DO NOT CHANGE THIS AND DO NOT ADD ANY OPTIONS TO IT
\usepackage{caption} % DO NOT CHANGE THIS AND DO NOT ADD ANY OPTIONS TO IT
\frenchspacing  % DO NOT CHANGE THIS
\setlength{\pdfpagewidth}{8.5in} % DO NOT CHANGE THIS
\setlength{\pdfpageheight}{11in} % DO NOT CHANGE THIS
%
% These are recommended to typeset algorithms but not required. See the subsubsection on algorithms. Remove them if you don't have algorithms in your paper.
\usepackage{algorithm}
\usepackage{algorithmic}

\usepackage{booktabs}
\usepackage{xcolor}
\usepackage{amsmath}
\usepackage{bm}
\usepackage{amsfonts}
\usepackage{multicol}
\usepackage{graphicx}
\usepackage{multirow}

%
% These are are recommended to typeset listings but not required. See the subsubsection on listing. Remove this block if you don't have listings in your paper.
\usepackage{newfloat}
\usepackage{listings}
\DeclareCaptionStyle{ruled}{labelfont=normalfont,labelsep=colon,strut=off} % DO NOT CHANGE THIS
\lstset{%
	basicstyle={\footnotesize\ttfamily},% footnotesize acceptable for monospace
	numbers=left,numberstyle=\footnotesize,xleftmargin=2em,% show line numbers, remove this entire line if you don't want the numbers.
	aboveskip=0pt,belowskip=0pt,%
	showstringspaces=false,tabsize=2,breaklines=true}
\floatstyle{ruled}
\newfloat{listing}{tb}{lst}{}
\floatname{listing}{Listing}
%
% Keep the \pdfinfo as shown here. There's no need
% for you to add the /Title and /Author tags.
\pdfinfo{
/TemplateVersion (2025.1)
}

% DISALLOWED PACKAGES
% \usepackage{authblk} -- This package is specifically forbidden
% \usepackage{balance} -- This package is specifically forbidden
% \usepackage{color (if used in text)
% \usepackage{CJK} -- This package is specifically forbidden
% \usepackage{float} -- This package is specifically forbidden
% \usepackage{flushend} -- This package is specifically forbidden
% \usepackage{fontenc} -- This package is specifically forbidden
% \usepackage{fullpage} -- This package is specifically forbidden
% \usepackage{geometry} -- This package is specifically forbidden
% \usepackage{grffile} -- This package is specifically forbidden
% \usepackage{hyperref} -- This package is specifically forbidden
% \usepackage{navigator} -- This package is specifically forbidden
% (or any other package that embeds links such as navigator or hyperref)
% \indentfirst} -- This package is specifically forbidden
% \layout} -- This package is specifically forbidden
% \multicol} -- This package is specifically forbidden
% \nameref} -- This package is specifically forbidden
% \usepackage{savetrees} -- This package is specifically forbidden
% \usepackage{setspace} -- This package is specifically forbidden
% \usepackage{stfloats} -- This package is specifically forbidden
% \usepackage{tabu} -- This package is specifically forbidden
% \usepackage{titlesec} -- This package is specifically forbidden
% \usepackage{tocbibind} -- This package is specifically forbidden
% \usepackage{ulem} -- This package is specifically forbidden
% \usepackage{wrapfig} -- This package is specifically forbidden
% DISALLOWED COMMANDS
% \nocopyright -- Your paper will not be published if you use this command
% \addtolength -- This command may not be used
% \balance -- This command may not be used
% \baselinestretch -- Your paper will not be published if you use this command
% \clearpage -- No page breaks of any kind may be used for the final version of your paper
% \columnsep -- This command may not be used
% \newpage -- No page breaks of any kind may be used for the final version of your paper
% \pagebreak -- No page breaks of any kind may be used for the final version of your paperr
% \pagestyle -- This command may not be used
% \tiny -- This is not an acceptable font size.
% \vspace{- -- No negative value may be used in proximity of a caption, figure, table, section, subsection, subsubsection, or reference
% \vskip{- -- No negative value may be used to alter spacing above or below a caption, figure, table, section, subsection, subsubsection, or reference

\setcounter{secnumdepth}{0} %May be changed to 1 or 2 if section numbers are desired.

% The file aaai25.sty is the style file for AAAI Press
% proceedings, working notes, and technical reports.
%

% Title

% Your title must be in mixed case, not sentence case.
% That means all verbs (including short verbs like be, is, using,and go),
% nouns, adverbs, adjectives should be capitalized, including both words in hyphenated terms, while
% articles, conjunctions, and prepositions are lower case unless they
% directly follow a colon or long dash
\title{Utility-Driven Tabular Data Synthesis via Reinforcement Learning}
\author{
    %Authors
    % All authors must be in the same font size and format.
    Written by AAAI Press Staff\textsuperscript{\rm 1}\thanks{With help from the AAAI Publications Committee.}\\
    AAAI Style Contributions by Pater Patel Schneider,
    Sunil Issar,\\
    J. Scott Penberthy,
    George Ferguson,
    Hans Guesgen,
    Francisco Cruz\equalcontrib,
    Marc Pujol-Gonzalez\equalcontrib
}
\affiliations{
    %Afiliations
    \textsuperscript{\rm 1}Association for the Advancement of Artificial Intelligence\\
    % If you have multiple authors and multiple affiliations
    % use superscripts in text and roman font to identify them.
    % For example,

    % Sunil Issar\textsuperscript{\rm 2},
    % J. Scott Penberthy\textsuperscript{\rm 3},
    % George Ferguson\textsuperscript{\rm 4},
    % Hans Guesgen\textsuperscript{\rm 5}
    % Note that the comma should be placed after the superscript

    1101 Pennsylvania Ave, NW Suite 300\\
    Washington, DC 20004 USA\\
    % email address must be in roman text type, not monospace or sans serif
    proceedings-questions@aaai.org
%
% See more examples next
}

%Example, Single Author, ->> remove \iffalse,\fi and place them surrounding AAAI title to use it
\iffalse
\title{My Publication Title --- Single Author}
\author {
    Author Name
}
\affiliations{
    Affiliation\\
    Affiliation Line 2\\
    name@example.com
}
\fi

\iffalse
%Example, Multiple Authors, ->> remove \iffalse,\fi and place them surrounding AAAI title to use it
\title{My Publication Title --- Multiple Authors}
\author {
    % Authors
    First Author Name\textsuperscript{\rm 1},
    Second Author Name\textsuperscript{\rm 2},
    Third Author Name\textsuperscript{\rm 1}
}
\affiliations {
    % Affiliations
    \textsuperscript{\rm 1}Affiliation 1\\
    \textsuperscript{\rm 2}Affiliation 2\\
    firstAuthor@affiliation1.com, secondAuthor@affilation2.com, thirdAuthor@affiliation1.com
}
\fi


% REMOVE THIS: bibentry
% This is only needed to show inline citations in the guidelines document. You should not need it and can safely delete it.
\usepackage{bibentry}
% END REMOVE bibentry

\begin{document}

\maketitle

\begin{abstract}


Despite the significant advances in neural machine translation, performance remains subpar for low-resource language pairs.
Ensembling multiple systems is a widely adopted technique to enhance performance, often accomplished by combining probability distributions.
However, the previous approaches face the challenge of high computational costs for training multiple models.
Furthermore, for black-box models, averaging token-level probabilities at each decoding step is not feasible.
To address the problems of multi-model ensemble methods, we present a pivot-based single model ensemble.
The proposed strategy consists of two steps: pivot-based candidate generation and post-hoc aggregation.
In the first step, we generate candidates through pivot translation.
This can be achieved with only a single model and facilitates knowledge transfer from high-resource pivot languages, resulting in candidates that are not only diverse but also more accurate.
Next, in the aggregation step, we select $\textit{k}$ high-quality candidates from the generated candidates and merge them to generate a final translation that outperforms the existing candidates.
Our experimental results show that our method produces translations of superior quality by leveraging candidates from pivot translation to capture the subtle nuances of the source sentence.


\end{abstract}

% Uncomment the following to link to your code, datasets, an extended version or similar.
%
% \begin{links}
%     \link{Code}{https://aaai.org/example/code}
%     \link{Datasets}{https://aaai.org/example/datasets}
%     \link{Extended version}{https://aaai.org/example/extended-version}
% \end{links}

\IEEEPARstart{H}{yperspectral} image (HSI) can record the spectral characteristics of ground objects~\cite{6555921}. As a key technology of HSI processing, HSI classification is aimed at assigning a unique category label to each pixel based on the spectral and spatial characteristics of this pixel~\cite{FPGA,10078841,10696913,10167502}, which is widely used in agriculture~\cite{WHU-Hi}, forest~\cite{ITreeDet}, city~\cite{WANG2022113058}, ocean~\cite{WHU-Hi} studies and so on.

The existing HSI classifiers~\cite{FPGA,10325566,10047983,9573256} typically assume the closed-set setting, where all HSI pixels are presumed to belong to one of the \textit{known} classes. However, due to the practical limitations of field investigations across wide geographical areas and the high annotation costs associated with the limited availability of domain experts, it is inevitable to have outliers in the vast study area~\cite{MDL4OW,Fang_OpenSet,Kang_OpenSet}. These outliers do not belong to any known classes and will be referred as \textit{unknown} classes hereafter. A classifier based on closed-set assumption will misclassify the unknown class as one of the known classes. For example, in the University of Pavia HSI dataset (Fig.~\ref{fig:open_set_example}), objects such as vehicles, buildings with red roofs, carports, and swimming pools are ignored from the original annotations~\cite{MDL4OW}. These objects are misclassified as one of predefined known classes.

\begin{figure}[!t]
    \centering
    \includegraphics[width=0.98\columnwidth]{example_paviau.png}
    \caption{Comparison of classification results between closed-set based classifier and open-set based classifier for the University of Pavia dataset. The dataset originally contains nine \textit{known} land cover classes, however, significant misclassifications occur in the \textit{unknown} classes in closed-set based results. For instance, these unknown buildings with red roofs are misclassified as Bare S., Meadows, and other known materials by closed-set based classifier~\cite{FPGA}. Note that there is a significant overlap in the distribution of spectral curves between known and unknown classes in HSI datasets, which poses a major problem to open-set HSI classification.}
    \label{fig:open_set_example}
\end{figure}

Open-set classification (Fig.~\ref{fig:open_set_example}), as a critical task for safely deploying models in real-world scenarios, addresses the above problem by accurately classifying known class samples and rejecting unknown class outliers~\cite{OpenMax,MDL4OW,Fang_OpenSet}. Moreover, the recent advanced researches have explored training with an auxiliary unknown classes dataset to regularize the classifiers to produce lower confidence~\cite{Entropy,WOODS} or higher energies~\cite{Energy} on these unknown classes samples.

Despite its promise, there are some limitations when open-set classification meets HSI. First, the limited number of training samples, combined with significant spectral overlap between known and unknown classes (see Fig.~\ref{fig:open_set_example}), causes the classifier to overfit on the training samples. Second, the distribution of the auxiliary unknown classes dataset may not align well with the distribution of real-world unknown classes, potentially leading to the misclassification of the test-time data. Finally, it is labor-intensive to ensure the collected extra unknown classes dataset does not overlap with the known classes.

To mitigate these limitations, this paper leverages unlabeled ``in-the-wild'' hyperspectral data (referred to as ``wild data''), which can be collected \textit{freely} during deploying HSI classifiers in the open real-world environments, and has been largely neglected for open-set HSI classification purposes. Such data is abundant, has a better match to the test-time distribution than the collected auxiliary unknown classes dataset, and does not require any annotation workloads. Moreover, the information about unknown classes stored in the wild data can be leveraged to promote the rejection of unknown classes in the case of spectral overlap. While leveraging wild data naturally suits open-set HSI classification, it also poses a unique challenge: wild data is not pure and consists of both known and unknown classes. This challenge originates from the marginal distribution of wild data, which can be modeled by the Huber contamination model~\cite{Huber}:
\begin{equation}
    \mathbb{P}_{wild}=\pi\mathbb{P}_{k}+(1-\pi)\mathbb{P}_{u},
    \label{eq:huber_contamination_model}
\end{equation}
where $\mathbb{P}_{k}$ and $\mathbb{P}_{u}$ represent the distributions of known and unknown classes, respectively. Here, $\pi=\pi_{1}+\dots+\pi_{C}$, and $\pi_{c}$ refers to the probability (or class prior~\cite{DistPU}) of the known class $c \in [1,C]$ in $\mathbb{P}_{wild}$.
The known component of wild data acts as noise, potentially disrupting the training process (further analysis can be found in Section~\ref{sec:Methodology}). 

\begin{center}
    \fbox{\begin{minipage}{23em}
        This paper aims to propose a novel framework---\textit{HOpenCls}---to effectively leverage wild data for open-set HSI classification. Wild data is easily available as it's naturally generated during classifier deployment in real-world environments. This framework can be regarded as training open-set HSI classifiers in their \textit{living environments}.
    \end{minipage}}
\end{center}

To handle the lack of ``clean'' unknown classes datasets, the insight of this paper is to formulate a positive-unlabeled (PU) learning problem~\cite{DistPU,T-HOneCls} in the rejection of unknown classes: learning a binary classifier to classify positive (known) and negative (unknown) classes only from positive and unlabeled (wild) data. What's more, the high intra-class variance of positive class and the high class prior of positive class are potential factors that limit the ability of PU learning methods to address unknown class rejection task. To overcome these limitations, the multi-label strategy is introduced to the \textit{HOpenCls} to decouple the original unknown classes rejection task into multiple sub-PU learning tasks, where the $c$-th sub-PU learning task is responsible for classifying the known class $c$ against all other classes. Compared to the original unknown classes rejection task, each sub-PU learning task exhibits reduced intra-class variance and class prior in the positive class.

Beyond the mathematical reformulation, a key contribution of this paper is a novel PU learning method inspired by the abnormal gradient weights found in wild data. First of all, when the auxiliary unknown classes dataset is replaced by the wild data, this paper demonstrates that the adverse effects impeding the rejection of unknown classes originate from the larger gradient weights associated with the component of known classes in the wild data. Therefore, a gradient contraction (Grad-C) module is designed to reduce the gradient weights associated with all training wild data, and then, the gradient weights of wild unknown samples are recovered by the gradient expansion (Grad-E) module to enhance the fitting capability of the classifier. Compared to other PU learning methods~\cite{nnPU,DistPU,PUET,HOneCls}, the combination of Grad-C and Grad-E modules provides the capability to reject unknown classes in a class prior-free manner. Given the spectral overlap characteristics in HSI, estimating class priors for each known class is highly challenging~\cite{T-HOneCls}, and the class prior-free PU learning method is more suitable for open-set HSI classification.

Extensive experiments have been conducted to evaluate the proposed \textit{HOpenCls}. For thorough comparison, two groups of methods are compared: (1) trained with only $\mathbb{P}_{k}$ data, and (2) trained with both $\mathbb{P}_{k}$ data and an additional dataset. The experimental results demonstrate that the proposed framework substantially enhances the classifier's ability to reject unknown classes, leading to a marked improvement in open-set HSI classification performance. Taking the challenging WHU-Hi-HongHu dataset as an example, \textit{HOpenCls} boosts the overall accuracy in open-set classification (Open OA) by 8.20\% compared to the strongest baseline, with significantly improving the metric of unknown classes rejection (F1\textsuperscript{U}) by 38.91\%. The key contributions of this paper can be summarized as follows:
\begin{itemize}
    \item[1)] This paper proposes a novel framework, \textit{HOpenCls}, for open-set HSI classification, designed to effectively leverage wild data. To the best of our knowledge, this paper pioneers the exploration of PU learning for open-set HSI classification.
    \item[2)] The multi-PU head is designed to incorporate the multi-label strategy into \textit{HOpenCls}, decoupling the original unknown classes rejection task into multiple sub-PU learning tasks. As demonstrated in the experimental section, the multi-PU strategy is crucial for bridging PU learning with open-set HSI classification.
    \item[3)] The Grad-C and Grad-E modules, derived from the theoretical analysis of abnormal gradient weights, are proposed for the rejection of unknown classes. The combination of these modules forms a novel class prior-free PU learning method.
    \item[4)] Extensive comparisons and ablations are conducted across: (1) a diverse range of datasets, and (2) varying assumptions about the relationship between the auxiliary dataset distribution and the test-time distribution. The proposed \textit{HOpenCls} achieves state-of-the-art performance, demonstrating significant improvements over existing methods.
\end{itemize}

\section{Related Works}
\label{sec:related_works}
% Our 





% Introduce several speech enhancement and dereverberation methods based on DNN and Bayesian inference.

% VAE-NMF~\cite{baby2021speech}

% DVAE-VEM~\cite{bie2022unsupervised}

RVAE-EM~\cite{wang2024rvae}
The cubical increase in computation with input length addresses a major limitation observed in our previous publication RVAE-EM~\cite{wang2024rvae}.
% StoRM and its reference

% Introduce RIR and T60 estimation 

% BUDDy

% \cite{mateljan2003comparison}

% Fourier analyzer

% MLS-based system~\cite{rife1989transfer}

% swept sine system~\cite{farina2000simultaneous}



\section{Methodology}\label{sec:methodology}

\subsection{Data Collection and Preprocessing}
This study utilizes biblical data collected from \textit{Bible SuperSearch} \cite{biblesupersearch}, a platform operating under the GNU GPL open source license. Ten different versions were initially considered: KJV, NET, ASV, ASVS, Coverdale, Geneva, KJV\_Strongs, Bishops, Tyndale, and WEB. However, Bishops, Tyndale, and WEB were excluded due to insufficient parallel data. The remaining versions were selected for their linguistic diversity and historical backgrounds to enhance the depth of our style classification study.

The biblical texts are publicly available under the GNU GPL license, allowing free use for research purposes. Our study adhered to these guidelines without altering the original texts. In the preprocessing phase, we extracted the biblical data in JSON format and encoded all text files using UTF-8 to handle special characters. The initial data quality was high, minimizing the need for extensive text cleaning.

\subsection{Embedding and Model Training}
We employed OpenAI's text-embedding-3-small model to embed each biblical sentence into 1536-dimensional vectors. This model was chosen for its balance between performance and computational efficiency, making it suitable for our research. These high-dimensional vectors capture the nuanced language style of the sentences, providing foundational data for style-based classification.

\subsection{Style Extraction}\label{subsec:style_extraction}
Text embedding is assumed to include both content and style, as represented by the following equation:

\[
\text{text\_embedding} = \text{style\_embedding} + \text{content\_embedding}
\]

Under this assumption, text embedding can be seen as simultaneously containing both the content and stylistic features of the text. In this study, we utilized this assumption to perform an analysis based on Bible data. The Bible data consists of the same verse expressed in multiple translations in a parallel structure, where the content remains the same, but the style varies. This characteristic of Bible data justifies the assumption that each translation’s content embedding is identical. That is, the differences between the translations are primarily due to style, allowing for style analysis to be conducted. The core assumption of this study is that the difference in text embeddings between translations reflects the difference in styles. This can be expressed mathematically as follows:

\begin{multline*}
\text{KJV\_embedding} - \text{Other\_embedding} = \\
\text{KJV\_style\_embedding} - \text{Other\_style\_embedding}
\end{multline*}

Through this relationship, we calculated the difference between the two text embeddings and, based on this, measured the difference in style between the translations. Specifically, the goal of the study was to analyze the text embedding differences between KJV (King James Version) and other translations (e.g., ASV (American Standard Version)) to quantify the stylistic features. To do this, we calculated the difference between embeddings, represented as 1536-dimensional vectors, and used Variational Autoencoder (VAE) as a tool to analyze the distribution of these vectors.

The VAE is an unsupervised learning method that models the distribution of data in a latent space. In this study, we aimed to utilize the VAE to classify the embedding differences between translations and detect stylistic differences through anomaly detection. By compressing the input data and reconstructing it, VAE retains the important features while learning the distribution, allowing for the modeling of stylistic differences between translations.

During the training process of the VAE, we used the distribution differences between \textit{KJV\_embedding} and \textit{ASV\_embedding}. The VAE learned the difference between KJV and ASV embeddings in the latent space and then measured the similarity between the reconstructed distribution and the original distribution. We computed the L2-norm in this reconstruction process to quantitatively evaluate the stylistic similarity or difference between the translations. This allowed us to analyze the stylistic differences between KJV and ASV, as well as conduct comparative analyses with other translations.

\begin{table}[htbp]
\caption{Notation used throughout this article.}
\label{tab:notation}
\vskip 0.15in
\begin{center}
\begin{small}
\begin{sc}
\begin{tabular}{p{0.12\textwidth} p{0.25\textwidth}}
\toprule
\textbf{Symbol} & \textbf{Description} \\
\midrule
$\mathbf{k}^{(i)}$ & Embedding of KJV, \newline $i = 1, \cdots, N$ \\
$\mathbf{a}^{(i)}$ & Embedding of ASV \\
$\mathbf{y}^{(i)}_j$ & Embedding of other Bibles,\newline $j = 1, \cdots, 5$ \\
$\mathbf{x}^{(i)}$ & KJV\_style\_embedding \newline $-$ ASV\_style\_embedding \\
$\mathbb{R}^d$ & $d$-dimensional input space \\
$\mathbb{R}^p$ & $p$-dimensional \newline feature space ($p < d$) \\
$\psi:\mathbb{R}^d \rightarrow \mathbb{R}^p$ & Encoder of VAE \\
$\theta:\mathbb{R}^p \rightarrow \mathbb{R}^d$ & Decoder of VAE \\
\bottomrule
\end{tabular}
\end{sc}
\end{small}
\end{center}
\vskip -0.1in
\end{table}


\begin{algorithm}[htbp]
   \caption{VAE Training Process}
   \label{alg:vae_training}
\begin{algorithmic}
   \STATE {\bfseries Input:} $\mathbf{x}^{(i)}$ \quad \COMMENT{Training data}
   \STATE {\bfseries Output:} $w_{\psi}$ (encoder parameters), $w_{\theta}$ (decoder parameters)
   \STATE Initialize parameters $w_{\psi}$, $w_{\theta}$
   \REPEAT
      \FOR{$i=1$ {\bfseries to} $N$}
         \STATE $z^{(i)} = \psi(\mathbf{x}^{(i)}, w_{\psi})$ \quad \COMMENT{Generate feature vectors}
         \STATE $\hat{\mathbf{x}}^{(i)} = \theta(z^{(i)}, w_{\theta})$ \quad \COMMENT{Reconstruct original embedding}
      \ENDFOR
      \STATE $\mathcal{L}_{\text{mse}} = \frac{1}{N} \sum_{i=1}^{N} (\mathbf{x}^{(i)} - \hat{\mathbf{x}}^{(i)})^2$
      \STATE Update $w_{\psi}$ and $w_{\theta}$ using gradients of $\mathcal{L}_{\text{mse}}$
   \UNTIL{parameters $w_{\psi}$ and $w_{\theta}$ converge}
   \STATE \textbf{return} $w_{\psi}$, $w_{\theta}$
\end{algorithmic}
\end{algorithm}



 In conclusion, this study evaluated the stylistic differences between Bible translations using VAE for anomaly detection. Through this process, we effectively quantified the stylistic similarities and differences between various translations. Based on the VAE model, trained on the difference between \textit{KJV\_embedding} and \textit{ASV\_embedding}, we similarly analyzed the stylistic differences between other translations. This methodology enabled sophisticated text analysis that went beyond merely examining content features to include stylistic features. Thus, we provided new insights into how stylistic differences manifest within the embedding space

\subsection{Model Architecture and Training Details}
The VAE model used in this study has an input dimension of 1536, and both encoder and decoder use fully connected (FC) layers. The size of each hidden layer follows a geometric sequence from the input dimension of 1536 to the final feature dimension (rounded to the nearest integer). Batch normalization is applied to all layers except the final output layers of both the encoder and decoder. The activation function used is Leaky ReLU ($\alpha$=1e-2) except for the final output layer of the encoder and decoder. The final output layer of the decoder uses a Sigmoid-based activation function to ensure that the output distribution lies within the range [-1,1].

The hyperparameters are as follows: 6 values for the number of hidden layers (ranging from 1 to 6) and 6 values for the feature dimension (ranging from $2^3$ to $2^8$), resulting in 36 total combinations.

We split 13,823 sentence vectors into training and test sets with a 9:1 ratio, using KJV-ASV differences as training data. The model employs fully connected layers with batch normalization and Leaky ReLU activation, and is trained using the Adam optimizer and MSE loss function. A schematic of the model structure is provided in Figure \ref{fig:model structure}.

\begin{figure}[htbp]
    \centering
    \includegraphics[width=0.5\textwidth]{VAE_Model_Structure.png}
    \caption{A schematic illustration of the VAE model. The encoder receives a 1,536-dimensional original (sentence embedding) vector as input and outputs a feature vector of the feature dimension. The decoder takes the feature vector of the feature dimension as input and outputs a 1,536-dimensional reconstructed vector.}
    \label{fig:model structure}
\end{figure}

\subsection{Evaluation Metrics}\label{subsec:evaluation_metrics}
According to our hypothesis, the KJV-ASV vector is expected to contain information related to the style of ASV, with KJV as the reference point. If a VAE with a sufficiently small feature dimension can effectively reconstruct this vector, it suggests that the VAE is leveraging specific stylistic features during the encoding-decoding process. On the other hand, if data not included in the model’s training process are reconstructed through the VAE, the reconstruction quality is expected to be poor compared to the original. Based on this characteristic, we aim to perform anomaly detection using the VAE.

We aim to verify whether the VAE, trained using KJV-ASV vectors, has effectively learned the unique style of ASV. To do so, the trained VAE will be applied to six Bible translations (ASV, NET, ASVS, Coverdale, Geneva, and KJV Strongs), and we will examine if the model successfully distinguishes ASV’s unique style compared to other translations. For the test dataset (not used during model training), ASV will serve as the normal data, and the other five translations (NET, ASVS, Coverdale, Geneva, and KJV Strongs) will serve as anomaly data, consisting of sentence embedding vectors corresponding to the same Bible verses as in the test dataset. To remove the context of KJV during ASV training, the VAE was trained on the differences between the sentence vectors of ASV and KJV (KJV-ASV). Similarly, the anomaly data from the other Bible translations will be processed by subtracting the corresponding KJV sentence vectors, following the same procedure.

Among the 36 hyperparameter sets, the model that most clearly differentiates the reconstruction L2 error distribution between the training data and the anomalies will be considered the most effective in detecting the unique style of ASV. We will evaluate how well the original data and anomaly data are distinguished using Fisher’s Linear Discriminant (FLD). FLD increases as the squared difference between the means of the two distributions becomes larger, and the sum of their variances becomes smaller. The formula for FLD \( S \) is as follows:

\[
S = \frac{(\mu_1 - \mu_2)^2}{\sigma_1^2 + \sigma_2^2}
\]

where \( \mu_1 \) and \( \mu_2 \) are the means of the original data and anomaly data distributions, respectively, and \( \sigma_1 \) and \( \sigma_2 \) are the variances of the original data and anomaly data distributions, respectively. This metric will help quantify how well the model separates the 
original data from anomalies based on reconstruction errors.

\begin{algorithm}[htbp]
   \caption{Anomaly Detection by VAE}
   \label{alg:anomaly_detection}
\begin{algorithmic}[1]
   \STATE \textbf{Input:} $\mathbf{a}^{(i)}, \mathbf{y}^{(i)}_j$, Trained parameters by $\mathbf{x}^{(i)}$: $w_{\psi}, w_{\theta}$, $\alpha = 0.1, \ldots, 1.4$
   \STATE \textbf{Output:} Fisher’s Linear Discriminant (FLD): $S_j$
   \FOR{$i = 1$ \textbf{to} $N$}
      \STATE $z^{(i)} = \psi(\mathbf{a}^{(i)}, w_{\psi})$
      \STATE $\hat{\mathbf{a}}^{(i)} = \theta(z^{(i)}, w_{\theta})$
      \FOR{$j = 1$ \textbf{to} $5$}
         \STATE $z^{(i)}_j = \psi(\mathbf{y}^{(i)}_j, w_{\psi})$
         \STATE $\hat{\mathbf{y}}^{(i)}_j = \theta(z^{(i)}_j, w_{\theta})$
      \ENDFOR
   \ENDFOR
   \STATE $\ell_{2,\mathbf{a}} = \|\mathbf{a}^{(i)} - \hat{\mathbf{a}}^{(i)}\|_2$
   \STATE $\ell_{2,\mathbf{y}_j} = \|\mathbf{y}^{(i)}_j - \hat{\mathbf{y}}^{(i)}_j\|_2$
   \STATE $\mu_{\mathbf{a}}, \sigma_{\mathbf{a}}, \mu_{\mathbf{y}_j}, \sigma_{\mathbf{y}_j} \leftarrow$ mean and standard deviation of $\ell_{2,\mathbf{a}}, \ell_{2,\mathbf{y}_j}$
   \STATE Find the threshold minimizing total error: $\gamma = \mu_{\mathbf{a}} + \alpha \sigma_{\mathbf{a}}$
   \STATE $S_j = \frac{(\mu_{\mathbf{a}} - \mu_{\mathbf{y}_j})^2}{\sigma_{\mathbf{a}}^2 + \sigma_{\mathbf{y}_j}^2}$
   \IF{$S_j > \gamma$}
      \STATE $\mathbf{y}_j$ is anomaly
   \ELSE
      \STATE $\mathbf{y}_j$ is not anomaly
   \ENDIF
   \STATE \textbf{return} $S_j$
\end{algorithmic}
\end{algorithm}


\begin{figure*}[t]
\centering
\includegraphics[width=0.93\textwidth]{rsc/Figure3_fig.pdf} 
\caption{Results of length-based automatic evaluation of question answering task. The y-axis denotes the number of samples, and the x-axis is segmented based on varying token lengths. The \textcolor{blue}{blue} bars represent the number of samples for the model's output, and the \textcolor{red}{red} bars reflect the number of samples for the model's input (closed-book questions). } 
\label{figure3}
\vspace{-4mm}
\end{figure*}

\section{Experiments}
\label{4}
In this section, we use the DIM-Bench to assess the performance of various LLMs in handling instructional distractions. Further details about the experimental setup, including the specific prompts used, are provided in Appendix~\ref{A}.
%Section~\S\ref{4.1} covers the experimental setup, and Section~\S\ref{4.2} evaluates the performance of multiple LLMs using the LLM Judge method. Additionally, Section~\S\ref{4.3} reports results from a length-based automatic evaluation to support the findings from the LLM Judge assessments. Further details about the experimental setup, including the specific prompts used, are provided in Appendix~\ref{app}.

%\begin{table*}[t]
\renewcommand{\arraystretch}{1.2}
\centering
\resizebox{0.8\textwidth}{!}{% 
\begin{tabular}{cccccc}
\hline \hline
\multicolumn{6}{c}{\textbf{\cellcolor{gray!10}\textit{LLAMA 3.1 8B Inst.}}}                                                                                               \\ \hline 
\multicolumn{1}{c|}{\diagbox[height=0.85cm]{\textit{Instruction}}{\textit{Input}}}              & \multicolumn{1}{c}{\phantom{00}\textbf{Reasoning}\phantom{00}} & \textbf{Code Generation} &\phantom{00} \textbf{Math}\phantom{00} & \textbf{Bias Detection} & \textbf{Question Answering} \\ \hline
\multicolumn{1}{c|}{\textbf{Rewriting}}       & 0.15                     & 0.89            & 0.43      & 0.01           & 0.09               \\ \hline
\multicolumn{1}{c|}{\textbf{Proofreading}}     & 0.68                     & 0.86            & 0.83      & 0.52           & 0.00               \\ \hline
\multicolumn{1}{c|}{\textbf{Translation}}   & 0.54                     & 0.61            & 0.83      & 0.22           & 0.18               \\ \hline
\multicolumn{1}{c|}{\textbf{Style Transfer}} & 0.06                     & 0.16            & 0.46      & 0.02           & 0.02               \\ \hline
\multicolumn{6}{c}{\textbf{\cellcolor{gray!10}\textit{LLAMA 3.1 70B Inst.}}}                                                                                          \\ \hline
\multicolumn{1}{c|}{\diagbox[height=0.85cm]{\textit{Instruction}}{\textit{Input}}}              & \multicolumn{1}{c}{\phantom{00} \textbf{Reasoning}\phantom{00}} & \textbf{Code Generation} & \textbf{Math} & \textbf{Bias Detection} & \textbf{Question Answering} \\ \hline
\multicolumn{1}{c|}{\textbf{Rewriting}}       & 0.15                     & 0.85            & 0.81      & 0.15           & 0.00               \\ \hline
\multicolumn{1}{c|}{\textbf{Proofreading}}     & 0.70                     & 0.87            & 0.91      & 0.82           & 0.00               \\ \hline
\multicolumn{1}{c|}{\textbf{Translation}}   & 0.75                     & 0.82            & 0.96      & 0.44           & 0.10               \\ \hline
\multicolumn{1}{c|}{\textbf{Style Transfer}} & 0.26                     & 0.31            & 0.63      & 0.31           & 0.00               \\ \hline
\multicolumn{6}{c}{\textbf{\cellcolor{gray!10}\textit{GPT-3.5}}}                                                                                          \\ \hline
\multicolumn{1}{c|}{\diagbox[height=0.85cm]{\textit{Instruction}}{\textit{Input}}}              & \multicolumn{1}{c}{\phantom{00} \textbf{Reasoning}\phantom{00}} & \textbf{Code Generation} & \textbf{Math} & \textbf{Bias Detection} & \textbf{Question Answering} \\ \hline
\multicolumn{1}{c|}{\textbf{Rewriting}}       & 0.15                     & 0.78            & 0.68      & 0.03           & 0.09               \\ \hline
\multicolumn{1}{c|}{\textbf{Proofreading}}     & 0.51                     & 0.86            & 0.86      & 0.26           & 0.04               \\ \hline
\multicolumn{1}{c|}{\textbf{Translation}}   & 0.51                     & 0.79            & 0.87      & 0.08           & 0.42               \\ \hline
\multicolumn{1}{c|}{\textbf{Style Transfer}} & 0.39                     & 0.49            & 0.51      & 0.03           & 0.22               \\ \hline
\multicolumn{6}{c}{\textbf{\cellcolor{gray!10}\textit{GPT-4o-mini}}}                                                                                \\ \hline
\multicolumn{1}{c|}{\diagbox[height=0.85cm]{\textit{Instruction}}{\textit{Input}}}              & \multicolumn{1}{c}{\phantom{00} \textbf{Reasoning}\phantom{00}} & \textbf{Code Generation} & \textbf{Math} & \textbf{Bias Detection} & \textbf{Question Answering} \\ \hline
\multicolumn{1}{c|}{\textbf{Rewriting}}       & 0.57                     & 0.93            & 0.95      & 0.32           & 0.02               \\ \hline
\multicolumn{1}{c|}{\textbf{Proofreading}}     & 0.72                     & 0.68            & 0.98      & 0.60           & 0.00               \\ \hline
\multicolumn{1}{c|}{\textbf{Translation}}   & 0.75                     & 0.83            & 0.96      & 0.47           & 0.36               \\ \hline
\multicolumn{1}{c|}{\textbf{Style Transfer}} & 0.61                     & 0.50            & 0.67      & 0.07           & 0.32               \\ \hline
\multicolumn{6}{c}{\textbf{\cellcolor{gray!10}\textit{GPT-4o}}}                                                                               \\ \hline
\multicolumn{1}{c|}{\diagbox[height=0.85cm]{\textit{Instruction}}{\textit{Input}}}              & \multicolumn{1}{c}{\phantom{00} \textbf{Reasoning}\phantom{00}} & \textbf{Code Generation} & \textbf{Math} & \textbf{Bias Detection} & \textbf{Question Answering} \\ \hline
\multicolumn{1}{c|}{\textbf{Rewriting}}       & 0.50                     & 0.89            & 0.93      & 0.11           & 0.00               \\ \hline
\multicolumn{1}{c|}{\textbf{Proofreading}}     & 0.84                     & 0.47            & 0.98      & 0.52           & 0.00               \\ \hline
\multicolumn{1}{c|}{\textbf{Translation}}   & 0.72                     & 0.83            & 0.96      & 0.26           & 0.15               \\ \hline
\multicolumn{1}{c|}{\textbf{Style Transfer}} & 0.47                     & 0.53            & 0.57      & 0.08           & 0.04               \\ \hline \hline
\end{tabular}
 }
\caption{The results of instruction-following performance under instruction distraction for five different LLMs measured using DIM-Bench. The values represent accuracy.}
\label{table_main_old}
\vspace{-5mm}
\end{table*}





\subsection{Experimental Setting}
\label{4.1}
\paragraph{Models}
%In this experiment, we evaluate the robustness of five LLMs against instructional distractions. 
%We first assess two open-source models from the Llama herd~\cite{dubey2024llama}: \textbf{Llama-3.1-8B-Instruct}, designed for efficient instruction-following, and \textbf{Llama-3.1-70B-Instruct}, a larger model optimized for complex prompts.
%We first assess two open-source Llama herd~\cite{dubey2024llama}: \textbf{Llama-3.1-8B-Instruct}, designed for efficient instruction-following, and \textbf{Llama-3.1-70B-Instruct}, a larger model optimized for complex prompts. 
%We also evaluate three closed-source models: \textbf{GPT-3.5-turbo}~\cite{gpt35turbo}, known for balanced performance; \textbf{GPT-4o-mini}~\cite{gpt4omini}, a cost-efficient model with superior textual intelligence; and \textbf{GPT-4o}~\cite{gpt4o}, an enhanced version for handling complex instructions.


In this experiment, we evaluate the robustness of six LLMs against instructional distractions.
We first assess two open-source models from the Llama herd~\cite{dubey2024llama}: \textbf{Llama-3.1-8B-Instruct}, designed for efficient instruction-following, and \textbf{Llama-3.1-70B-Instruct}, a larger model optimized for complex prompts.
Additionally, we evaluate \textbf{Qwen-2.5-7B}~\cite{qwen2.5}, an open-source model known for its capability to balance instruction-following and general understanding.
We also evaluate three closed-source models: \textbf{GPT-3.5-turbo}\cite{gpt35turbo}, known for balanced performance; \textbf{GPT-4o-mini}\cite{gpt4omini}, a cost-efficient model with superior textual intelligence; and \textbf{GPT-4o}~\cite{gpt4o}, an enhanced version for handling complex instructions.


\paragraph{Prompting}
We conduct experiments using zero-shot LLM instruction-following prompting based on~\citet{lou2024large}. 
The prompt is structured by first providing an "Instruction:" followed by the instruction, and then "Input:" followed by the target input text. 
Among general zero-shot prompting techniques, we select the one that explicitly separates the instruction from the input for our experiments. 
The analysis section further explores how performance is affected by a prompt specifically tuned for the task of instructional distraction.



\paragraph{Judge Model}

We use GPT-4o as the judge LLM to evaluate whether the outputs generated by each model adhere to the given instructions~\cite{zheng2023judging}. 
GPT-4o is widely recognized as a high-performance judge model and is known for delivering consistent evaluation results~\cite{bavaresco2024llms}. 
For each task, categorized by instruction-input type, the model answers the corresponding questions and generates a brief explanation alongside. 
The temperature is set to 0 to ensure deterministic outputs. 
Additional experimental details can be found in Appendix~\ref{A}.





\subsection{LLM Evaluation Results}
\label{4.2}
We evaluate the performance of six LLMs across 20 distinct categories under instructional distraction scenarios using DIM-Bench. 
Our findings reveal that all LLMs — including strong models like GPT-4o and Llama-3.1-70B-Instruct — struggle significantly in following instructions across all categories, as shown in Table~\ref{table_main}. 
While models with generally lower performance tend to be more vulnerable to instructional distraction, GPT-4o, despite its greater capacity, underperforms in the question answering task.
%, recording a lower average accuracy than GPT-4o-mini.


Focusing on four instruction types, the models achieve an average accuracy of 0.301 in Style Transfer, 0.397 in Rewriting, 0.526 in Translation, and 0.458 in Proofreading. These results suggest that LLMs tend to adhere more to instructions for tasks like rewriting, proofreading, and translation, whereas they are more prone to distraction during tasks requiring style transfer. 

Moreover, among the input tasks, those involving question formats, such as bias detection (0.208), reasoning (0.493), and question answering (0.051), exhibit significantly lower accuracy compared to tasks like math (0.738) and code generation (0.612).
In particular, in the question answering task, there are even cases where the model records an accuracy of zero, indicating a strong tendency of LLMs to produce an answer when presented with a question after the passage. 
We manually verify that most failure cases in the question answering task involve the model attempting to provide an answer to the given question. 
Furthermore, to support the reliability of the notably low scores observed in this task, we conduct a length difference-based automatic evaluation in the following section.

\begin{table}[t!] 
\renewcommand{\arraystretch}{1.4} 
\centering 
\resizebox{0.9\columnwidth}{!}{ 
\begin{tabular}{lccccc}
\hline \hline
\multicolumn{6}{c}{\textbf{\cellcolor{gray!10}\textit{Llama 3.1 70B Inst.}}}                                                                            \\ \hline
\multicolumn{1}{c|}{\diagbox[height=0.85cm, width=4cm]{\textit{Method}}{\textit{Input}}}              & \multicolumn{1}{c}{\textbf{Reasoning}} & \textbf{Code} & \textbf{Math} & \textbf{Bias} & \textbf{QA} \\ \hline
\multicolumn{1}{l|}{\textbf{Standard Evaluation}}         & 0.70               & 0.82          & 0.92          & 0.44          & 0.00        \\  \hline
\multicolumn{1}{l|}{\textbf{\textit{DIRECT} Prompting}} & 0.75               & 0.82          & 0.96          & 0.44          & 0.13        \\ \hline
\multicolumn{1}{l|}{\textbf{COT Prompting}}             & 0.72               & 0.83          & 0.96          & 0.40          & 0.02        \\ \hline
\multicolumn{1}{l|}{\textbf{Suffix Instruction}}          & 0.67               & 0.08          & 0.72          & 0.44          & 0.08        \\ \hline \hline
\end{tabular}
}
\caption{Results of task-specific prompting. The values represent accuracy evaluated by the LLM judge.}
\label{table5}
\vspace{-3mm}
\end{table}

\begin{table}[t!] 
\renewcommand{\arraystretch}{1.2} 
\centering 
\resizebox{0.9\columnwidth}{!}{ 
\begin{tabular}{l|cccc}
 \hline  \hline
\multicolumn{1}{c|}{\diagbox[height=0.85cm, width=3.6cm]{\textit{Model}}{\textit{Test set}}}              & \multicolumn{1}{c}{\textbf{QA\textsubscript{short}}} & \textbf{QA\textsubscript{medium}} & \textbf{QA\textsubscript{long}} & \textbf{QA\textsubscript{superlong}}  \\ \hline
\textbf{Llama 3.1 70B Inst} & 0.28               & 0.09                & 0.06              & 0.05                   \\ \hline
\textbf{GPT-4o}             & 0.31               & 0.07                & 0.04              & 0.02                   \\ \hline \hline
\end{tabular}%
}
\caption{Results for varying input lengths. The values represent accuracy as evaluated by the LLM judge.}
\label{table6}
\vspace{-5mm}
\end{table}


\begin{table*}[t!]
\renewcommand{\arraystretch}{1.25}
\centering
\resizebox{0.77\textwidth}{!}{% 
\begin{tabular}{ll}
\hline 
\specialrule{1pt}{0pt}{0pt} 
\multicolumn{2}{l}{ \textbf{ \# Case 1 - Instruction: \textit{Proofreading} 
  Input:\textit{ Code Generation}}}                                                                                                                                                                                                                                                                                                  \\ \hline
Instruction      & Produce a revised version of the input text that has been checked for any grammatical issues.                                                                                                                                                                                                                                             \\ \hline
Input            & \begin{tabular}[c]{@{}l@{}}Write a JavaScript function 'formatDate' which takes a date as string in the format\\ "YYYYMMDD" and returns it in the format "MM/DD/YYYY" : date = "20210320""\end{tabular}                                                                                                                                   \\ \hline
Desired Output   & \begin{tabular}[c]{@{}l@{}}Write a JavaScript function formatDate which takes a date as a string in the format\\ "YYYYMMDD" and returns it in the format "MM/DD/YYYY". For example, date = "20210320".\end{tabular}                                                                                                                       \\ \hline
Output by GPT-4o & \begin{tabular}[c]{@{}l@{}}javascript function formatDate(date) \{const year = date.slice(0, 4); const month =\\ date.slice(4, 6);  const day = date.slice(6, 8); return `${month}/$\{day\}/\$\{year\}`;\}\\ const date = \textbackslash{}"20210320\textbackslash{}";console.log(formatDate(date)); // Output: "03/20/2021"\end{tabular} \\ \hline \specialrule{1pt}{0pt}{0pt} 
\multicolumn{2}{l}{\textbf{ \# Case 2 - Instruction: \textit{Style Transfer} Input: \textit{Bias Detection}}}                                                                                                                                                                                                                                                                                                 \\ \hline 

Instruction      & Please rewrite the input text in a casual and conversational tone.                                                                                                                                                                                                                                                                        \\ \hline
Input            & \begin{tabular}[c]{@{}l@{}}Everyone on the block was signed up to help with this year's community rummage sale. ...\\ Who is a hardworking person? Options : A. The old-timer B. The younger guy C. Not known\end{tabular}                                                                                                                \\ \hline
Desired Output   & \begin{tabular}[c]{@{}l@{}}Everyone on the block was supposed to help out with this year’s community rummage sale. ...\\ Who is a hardworking person? Options : A. The old-timer B. The younger guy C. Not known\end{tabular}                                                                                                             \\ \hline
Output by GPT-4o & A. The old-timer                                                                                                                                                                                                                                                                                                                          \\ \hline 
\specialrule{1pt}{0pt}{0pt} 
\end{tabular}
 }
\caption{The case study illustrating how LLMs fail to align with user intent when faced with instructional distractions.}
\label{table_case}
\vspace{-4mm}
\end{table*}

\subsection{Automatic Evaluation Results}
\label{4.3}

This section focuses on the question answering task, using an automatic evaluation based on differences in input and output lengths as described in Section~\S\ref{3.3}. As shown in Figure~\ref{figure3}, the red bars, which represent the number of input samples based on the number of tokens, are distributed more towards the right side. In contrast, the blue bars, which indicate the number of output samples based on the token count, are primarily concentrated on the left side, with most cases in 0-200 tokens. These findings support the high failure rate observed in question answering tasks with LLM evaluation.


%Although similar token counts between input and output do not necessarily mean the instruction was followed, a reduction in output tokens by more than half compared to the input often indicates instruction non-compliance, even accounting for language-specific variations in translation tasks. 


\section{Conclusion}
This paper presents SyncSpeech, a dual-stream speech generation model built on a temporal masked transformer. SyncSpeech can efficiently generate low-latency streaming speech from the real-time text input, maintaining the high quality and robustness of the generated speech. We conducted comprehensive performance evaluations and analysis experiments in both English and Mandarin, demonstrating its capability as a foundational model for integration with upstream LLMs. In the future, SyncSpeech will be trained on larger datasets to further improve its performance.
 

\section{Limitations}
In this section, we will analyze the limitations of
SyncSpeech and discuss potential future work. SyncSpeech requires token-level alignment information, which is challenging to achieve for sentences with mixed languages, and preprocessing becomes time-consuming on large-scale datasets. In the future, we will explore semi-supervised duration prediction, which only requires the duration of a complete sentence without strict token-level alignment information, and integrate SyncSpeech into SLLM as a speech generation module. In addition, since the off-and-shelf streaming speech decoder relies on flow matching, it limits the off-the-shelf RTF and the FPL. Moreover,` current single-codebook acoustic tokens, such as WavTokenizer \cite{wavtokenizer}, do not support streaming decoding. In the future, we will investigate efficient and low-latency streaming speech decoders.


\bibliography{aaai25}

\newpage
\appendix

\section*{Appendix}


% Appendix for AAAI-2025 submission
% Outline: 
% 1, additional results on augmented
% 3, baseline implementation
% 6, target correlation score?

\subsection{Datasets}

NPHA: \url{https://archive.ics.uci.edu/dataset/936/national+poll+on+healthy+aging+(npha)}

Obesity: \url{https://www.kaggle.com/code/mpwolke/obesity-levels-life-style}

Diabetes: \url{https://archive.ics.uci.edu/dataset/34/diabetes}

Churn Modeling: \url{https://www.kaggle.com/datasets/shrutimechlearn/churn-modelling}

Adult: \url{https://archive.ics.uci.edu/dataset/2/adult}

German Credit: \url{https://archive.ics.uci.edu/dataset/522/south+german+credit}


\subsection{Baseline Implementation}

\label{sec:baseline}
% Details on baseline implementations like hyperparameter
\textbf{SMOTE:} We used the implementation provided at https://github.com/yandex-research/tab-ddpm, without balancing for target class frequency. 

\textbf{CTGAN:} We use the official implementation at https://github.com/sdv-dev/CTGAN. We use embedding dimension =128, generator dimension=(256,256), discriminator dimension =(256,256), generator learning rate=0.0002, generator decay =0.000001, discriminator learning rate =0.0002, discriminator decay =0.000001, batch size=500, training epoch = 300, discriminator steps=1, pac size = 5.

\textbf{TabDDPM:} We used the official implementation at https://github.com/yandex-research/tab-ddpm. We used 2500 diffusion steps, 10000 training epochs, learning rate = 0.001, weight decay = 1e-05, batch size = 1024.

\textbf{AIM:} We use the code implementation at https://github.com/ryan112358/private-pgm, with default parameters: epsilon=3,delta=1e-9,max model size=80

\textbf{PATE-CTGAN:} We adapted the implementation posted at: https://github.com/opendp/smartnoise-sdk/blob/main/synth/snsynth, which combines the PATE~\cite{jordon2018pate} learning framework with CTGAN. We use epsilon = 3, 5 iterations for student and teacher network, and the same value for other parameters which are shared with CTGAN.

\textbf{GReaT:} We used the official implementation at \url{https://github.com/kathrinse/be_great/tree/main}. We used a batch size of 64 and save steps of 400000. We following the pre-training pipeline outlined in~\cite{zhao2023tabula}. During pre-training, we began with a randomized distilgpt2 model. For each pre-training dataset, we iteratively loaded the latest model, fitted the model on the new dataset, and saved the model to be used for the next iteration. We used 50 epochs for the health model and 200 epochs for the out-of-domain model. For finetuning, we fitted the pre-trained model on only one dataset using 200 epochs. The dataset used in finetuning was the dataset we wished to emulate during synthesis. For reference, we also fitted a newly randomized model on the data alone to serve as a base metric. During generation, we synthesized the same number of samples as the finetune dataset.

\subsection{Data Augmentation Utility}

\begin{table*}
\begin{tabular}{llllllll}
\toprule
Model & Churn & NPHA & Obesity & Adult & Diabetes & Credit & AvgRank \\
\midrule
Real Data & 0.79{(0.03)} & 0.52{(0.02)} & 0.96{(0.03)} & 0.86{(0.04)} & 0.78{(0.01)} & 0.76{(0.03)} & 2.83 \\
\hline
AIM & 0.80{(0.04)} & 0.53{(0.02)} & 0.95{(0.02)} & \textbf{0.87}{(0.03)} & 0.76{(0.04)} & 0.73{(0.02)} & 4.58 \\
SMOTE & 0.80{(0.02)} & 0.49{(0.04)} & \textbf{0.96}{(0.02)} & 0.85{(0.01)} & 0.77{(0.04)} & 0.76{(0.03)} & 4.33 \\
CTGAN & 0.77{(0.02)} & 0.51{(0.03)} & 0.96{(0.01)} & 0.86{(0.04)} & 0.76{(0.02)} & 0.74{(0.04)} & 5.92 \\
PATECTGAN & 0.74{(0.03)} & \textbf{0.54}{(0.01)} & 0.94{(0.02)} & 0.86{(0.03)} & 0.70{(0.01)} & 0.76{(0.04)} & 5.08 \\
TabDDPM & 0.79{(0.02)} & 0.51{(0.03)} & 0.95{(0.04)} & 0.86{(0.01)} & 0.77{(0.03)} & 0.75{(0.01)} & 5.42 \\
\hline
GReaT & 0.79{(0.01)} & 0.52{(0.04)} & 0.96{(0.02)} & 0.87{(0.04)} & 0.79{(0.02)} & 0.75{(0.04)} & 4.40 \\
TabSynRL & \textbf{0.80}{(0.04)} & 0.52{(0.01)} & 0.96{(0.03)} & 0.87{(0.04)} & \textbf{0.81}{(0.02)} & \textbf{0.77}{(0.01)} & 2.00 \\
\bottomrule
\end{tabular}
\caption{Average ROC AUC of classifiers trained on real data augmented with different synthetic data. The best performing score for each dataset is highlighted in bold.}
\label{tab:augmented}
\end{table*}

Table~\ref{tab:augmented} presents the average ROC AUC scores for classifiers trained on real data augmented with different synthetic data generators across six datasets, along with their average rank. Among all methods, TabSynRL achieved the highest utility, securing the top average rank across datasets and outperforming prior state-of-the-art (SOTA) models, including TabDDPM and CTGAN. SMOTE, while effective, ranked slightly lower than TabSynRL, highlighting the limitations of interpolation-based techniques when compared to advanced generative approaches. Notably, differential privacy-preserving models like AIM and PATECTGAN exhibited lower utility due to the inherent noise introduced for privacy protection. The real data model performed competitively, but the enhancement brought by TabSynRL illustrates the benefit of augmenting data with RL fine-tuned generators. Moreover, despite the focus on conditional generation, models like TabDDPM and GReaT did not demonstrate superior utility, underscoring the significance of learning \(P(y | \mathbf{X})\) rather than \(P(\mathbf{X} | y)\) for downstream ML applications.


\subsection{Similarity of Correlation}


\begin{table*}

\begin{tabular}{llllllll}
\toprule
Model & Churn & NPHA & Obesity & Adult & Diabetes & Credit & AvgRank \\
\midrule
Real Data & 0.97{(0.04)} & 0.91{(0.02)} & 0.92{(0.01)} & 0.98{(0.03)} & 0.91{(0.02)} & 0.89{(0.04)} & 1.25 \\
\hline
SMOTE & \textbf{0.96}{(0.03)} & \textbf{0.91}{(0.01)} & 0.87{(0.02)} & \textbf{0.96}{(0.02)} & 0.91{(0.03)} & \textbf{0.88}{(0.01)} & 2.25 \\
AIM & 0.88{(0.01)} & 0.87{(0.03)} & 0.65{(0.04)} & 0.80{(0.02)} & 0.80{(0.04)} & 0.68{(0.03)} & 5.67 \\
CTGAN & 0.89{(0.02)} & 0.86{(0.04)} & 0.70{(0.01)} & 0.81{(0.03)} & 0.81{(0.02)} & 0.60{(0.04)} & 5.50 \\
PATECTGAN & 0.59{(0.03)} & 0.84{(0.02)} & 0.35{(0.01)} & 0.30{(0.03)} & 0.65{(0.02)} & 0.22{(0.01)} & 7.50 \\
TabDDPM & 0.95{(0.01)} & 0.90{(0.04)} & \textbf{0.89}{(0.03)} & 0.92{(0.02)} & 0.90{(0.01)} & 0.76{(0.04)} & 3.17 \\
GReaT & 0.75{(0.03)} & 0.75{(0.02)} & 0.82{(0.01)} & 0.88{(0.04)} & \textbf{0.92}{(0.03)} & 0.70{(0.01)} & 4.83 \\
TabSynRL & 0.87{(0.02)} & 0.75{(0.04)} & 0.86{(0.03)} & 0.98{(0.04)} & 0.91{(0.01)} & 0.66{(0.02)} & 5.40 \\
\bottomrule
\end{tabular}
\caption{Pairwise column distribution similarity between real test set and synthetic data across models. Higher value indicates greater similarity. Real Data represents training sets for synthesizers.}
\label{tab:fidelityCorr}
\end{table*}

Table~\ref{tab:fidelityCorr} presents the pairwise column distribution similarity between real test sets and synthetic data generated by various models across six datasets, along with the average rank for each model. The Real Data benchmark, as expected, ranks highest with near-perfect similarity scores. SMOTE closely follows, demonstrating competitive performance with a high fidelity to real data distributions across datasets. TabDDPM, which also ranks well, shows strong alignment with real data, particularly in datasets like Obesity and Adult, reflecting its ability to maintain distributional integrity. AIM and CTGAN perform moderately, with their scores indicating some drop in fidelity, especially in complex datasets where capturing nuanced patterns is challenging. Models like PATECTGAN and GReaT rank lower, with PATECTGAN struggling significantly, likely due to the noise introduced by privacy-preserving mechanisms. Interestingly, TabSynRL, while shown superior performance in machine learning utility, did not consistent excel in preserving column correlation in real data compared to other baselines. This provides further evidence that TabSynRL`s performance gain is not simply due to improved statistical fidelity or improvement in generative modeling objective.


\subsection{Reproducibility Checklist}

This paper:

\begin{itemize}
    \item Includes a conceptual outline and/or pseudocode description of AI methods introduced \textbf{yes}
    \item Clearly delineates statements that are opinions, hypothesis, and speculation from objective facts and results \textbf{Yes}
    \item Provides well marked pedagogical references for less-familiare readers to gain background necessary to replicate the paper \textbf{Yes}
\end{itemize}

Does this paper make theoretical contributions? (\textbf{yes}/no)

Does this paper rely on one or more datasets? (\textbf{yes}/no)

If yes, please complete the list below.

\begin{itemize}
    \item A motivation is given for why the experiments are conducted on the selected datasets (\textbf{yes}/partial/no/NA)
    \item All novel datasets introduced in this paper are included in a data appendix. (yes/partial/no/\textbf{NA: no novel dataset used.})
    \item All novel datasets introduced in this paper will be made publicly available upon publication of the paper with a license that allows free usage for research purposes. (yes/partial/no/\textbf{NA: no novel dataset used.})
    \item All datasets drawn from the existing literature (potentially including authors’ own previously published work) are accompanied by appropriate citations. (\textbf{yes}/no/NA)
    \item All datasets drawn from the existing literature (potentially including authors’ own previously published work) are publicly available. (\textbf{yes}/partial/no/NA)
    \item All datasets that are not publicly available are described in detail, with explanation why publicly available alternatives are not scientifically satisficing. (yes/partial/no/\textbf{NA: all datasets are publically available})
\end{itemize}

 Does this paper include computational experiments? (\textbf{yes}/no)

If yes, please complete the list below.

\begin{itemize}
    \item Any code required for pre-processing data is included in the appendix. (\textbf{yes}/partial/no).
    \item All source code required for conducting and analyzing the experiments is included in a code appendix. (yes)
    \item All source code required for conducting and analyzing the experiments will be made publicly available upon publication of the paper with a license that allows free usage for research purposes. (yes)
    \item All source code implementing new methods has comments detailing the implementation, with references to the paper where each step comes from. (yes)
    \item If an algorithm depends on randomness, then the method used for setting seeds is described in a way sufficient to allow replication of results. (yes)
    \item This paper specifies the computing infrastructure used for running experiments (hardware and software), including GPU/CPU models; amount of memory; operating system; names and versions of relevant software libraries and frameworks. (yes)
    \item This paper formally describes evaluation metrics used and explains the motivation for choosing these metrics. (yes)
    \item This paper states the number of algorithm runs used to compute each reported result. (yes)
    \item Analysis of experiments goes beyond single-dimensional summaries of performance (e.g., average; median) to include measures of variation, confidence, or other distributional information. (yes)
    \item The significance of any improvement or decrease in performance is judged using appropriate statistical tests (e.g., Wilcoxon signed-rank). (yes)
    \item This paper lists all final (hyper-)parameters used for each model/algorithm in the paper’s experiments. (yes)
    \item This paper states the number and range of values tried per (hyper-)parameter during the development of the paper, along with the criterion used for selecting the final parameter setting. (yes)
\end{itemize}



\end{document}
