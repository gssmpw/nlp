\section{Related Works}
\subsection{GNSS-Visual-Inertial Tight Integration}
To achieve a consistently accurate and reliable navigation solution in complex real-world scenarios, multi-sensor fusion is widely recognized as the most effective approach. The tightly coupled integration of GNSS, visual, and inertial data enables the joint optimization of estimated parameters by fusing raw measurements and leveraging all available error sources from each sensor.

In recent years, multiple GVINS frameworks have been developed and published. ____ represents the first attempt to integrate raw GNSS measurements of pseudorange and Doppler shift into an optimization-based visual-inertial SLAM system. It demonstrates superior performance compared to existing visual-inertial SLAM and GNSS Single-Point-Positioning (SPP), especially in GNSS-denied environments. GVINS ____ introduces a coarse-to-fine initialization technique that accurately establishes the real-time transformation between global measurements and local states, significantly reducing the GNSS state initialization time. Extensive evaluations validate its accuracy and robustness compared to other SOTA visual-inertial-odometry (VIO) systems. However, GVINS supports only the SPP algorithm for GNSS factors, limiting its ability to fully exploit the potential of GNSS positioning. To better utilize the advantages of INS, IC-GVINS ____ augments visual feature tracking and landmark triangulation by incorporating precise INS information. This integration significantly improves GVINS performance in high-dynamic conditions and complex environments. However, IC-GVINS only performs GNSS solution-level integration with raw inertial and visual measurements in a batch optimization framework. To fully harness the capabilities of GNSS, GICI-LIB ____ incorporates nearly all GNSS measurement error sources in its sensor fusion. Its RTK-RRR processing mode delivers superior pose estimation performance compared to other SOTA GNSS-Visual-Inertial navigation systems, particularly in challenging environments.

Despite these innovations, the primary focus of advancements has been on GNSS and INS. However, further improvement of GNSS or INS performance is increasingly challenging and presents a bottleneck. Meanwhile, visual sensors remain low-cost alternatives, but visual-based navigation still heavily relies on feature tracking and the minimization of reprojection errors. These areas require further attention to enhance overall navigation capabilities in GVINS. With the rapid evolution of graphical computing resources, more sophisticated visual representations and computer graphics techniques are being developed and can now be efficiently applied to address a variety of engineering challenges. These advancements open new opportunities to push the boundaries of visual navigation.      





\subsection{3DGS in SLAM}

Traditional SLAM systems rely on points ____, surfels ____ or voxel grids ____ as scene representations, enabling direct and fast computations. However, these methods often struggle to achieve high-fidelity mapping due to their fixed spatial resolution and the lack of correlation among the 3D primitives. Neural-based SLAM ____________ offers improved mapping quality but suffers from computationally expensive training processes, making it unsuitable for many real-time applications. In contrast, 3DGS has demonstrated superiority in efficient, real-time, and high-resolution image rendering. This is achieved by using expressive anisotropic 3D Gaussians as scene representations and leveraging tile-based rasterization techniques. As a result, 3DGS-based SLAM systems outperform traditional SLAM methods by enabling rapid photo-realistic rendering, and achieve superior performance and efficiency in both mapping and tracking. GS-SLAM ____ is the first to incorporate real-time 3DGS into the SLAM pipeline using RGB-D rendering. It integrates an adaptive expansion strategy to add or remove noisy 3D Gaussians, which enhances the mapping performance. Additionally, GS-SLAM introduces an effective coarse-to-fine method for selecting reliable 3D Gaussians, enhancing camera pose optimization. MonoGS ____ represents the first application of 3DGS in real-time monocular SLAM. It replaces the offline Structure-from-Motion (SfM) process used in the original 3DGS algorithm with direct optimization against 3D Gaussians, taking advantage of their wide convergence basin for robust camera tracking. MonoGS also introduces a geometric verification and regularization technique to address ambiguities in incremental 3D dense reconstruction. SplaTAM ____ proposes a single RGB-D camera SLAM framework that utilizes a silhouette mask to capture the presence of scene density. Both online tracking and mapping evaluations show that SplaTAM outperforms existing methods. Despite these advancements, 3DGS-based SLAM faces challenges, including error accumulation during pose tracking, scale ambiguity in monocular 3DGS SLAM, and degradation of pose tracking caused by inconsistent or erroneous 3D Gaussian maps. Our navigation system takes advantage of the tight integration of 3DGS with GNSS and INS to address these challenges.