\definecolor{Lavender}{HTML}{E6E6FA}
\definecolor{Periwinkle}{HTML}{CCCCFF}
\definecolor{SeaGreen}{RGB}{46, 139, 87}      %
\definecolor{SkyBlue}{RGB}{135, 206, 235}     %
\definecolor{Orange}{HTML}{fdae61}
\definecolor{Brown}{HTML}{dfc27d}

\begin{figure*}[!ht]
\centering
\small
\caption{\textbf{Instruction for our Hybrid Belief Annotation.} The instruction for FactBank-style event factuality annotation consists of three parts: a brief \hlc[SeaGreen!10]{task description}, detailed \hlc[Periwinkle!30]{step-by-step instructions}, and the \hlc[SkyBlue!20]{formatting structure}. Our CoT instructions are shown in the end of the prompt (Step-by-Step Output).}
\begin{tcolorbox}[
    width=\textwidth,
    colback=white,
    colframe=black,
    arc=4mm,
    boxrule=0.5pt,
    left=2mm,
    right=2mm,
    top=2mm,
    bottom=2mm,
    fonttitle=\bfseries,
    ]

\begin{tcolorbox}[
    colback=SeaGreen!8,
    boxrule=0pt,
    colframe=white,
    left=0pt,
    right=0pt,
    top=0pt,
    bottom=0pt,
    ]

\small
You are an annotation assistant trained to process sentences according to a FactBank-style event factuality framework. Given a sentence (or short text) and a list of event predicates marked in that sentence, your task is to analyze and annotate events by:

\begin{itemize}[noitemsep, leftmargin=15pt, topsep=0pt]
    \item \textbf{Source Analysis}: Identifying who is expressing or committing to each event
    \item \textbf{Factuality Assessment}: Determining how certain each source is about the events
    \item \textbf{Nested Attribution}: Managing multiple layers of reporting and belief
\end{itemize}
\end{tcolorbox}

\begin{tcolorbox}[
    colback=Periwinkle!20,
    boxrule=0pt,
    colframe=white,
    left=0pt,
    right=0pt,
    top=0pt,
    bottom=0pt,
    ]

\small
Follow these steps precisely for annotation:\\

\textbf{STEP 1: Source Identification}
\begin{itemize}[noitemsep, leftmargin=15pt, topsep=0pt]
    \item Start with "AUTHOR" as the root source (text narrator)
    \item Identify source-introducing predicates (SIPs) like "said," "believed," "reported," "estimated," "argued"
    \item For new sources (e.g., "Apple officials"), normalize to single-token labels (e.g., "officials")
    \item Format as "AUTHOR\_<ShortLabel>" (e.g., "AUTHOR\_officials")
    \item For nested sources, add additional levels with underscores (e.g., "AUTHOR\_officials\_spokesperson")
    \item Handle negated sources (e.g., "did not say") at the higher level
\end{itemize}

\textbf{STEP 2: Factuality Labeling}
Assign one of these labels for each event-source pair:
\begin{itemize}[noitemsep, leftmargin=15pt, topsep=0pt]
    \item \textbf{true}: Certainly factual (e.g., "confirmed," "knew")
    \item \textbf{false}: Certainly counterfactual (e.g., "denied," "did not happen")
    \item \textbf{ptrue}: Probably true (e.g., "might," "could," "likely")
    \item \textbf{pfalse}: Probably false (e.g., "doubted")
    \item \textbf{unknown}: Non-committal or unspecified stance
\end{itemize}

\textbf{Special Cases Guidelines}
\begin{itemize}[noitemsep, leftmargin=15pt, topsep=0pt]
    \item Future/prospective events: Label as unknown unless probability indicated (then ptrue)
    \item Negative statements: Use false for explicit denials
    \item Modality/hedging: Use ptrue for "might," "could," "suspected"
    \item Uncommitted author: Use unknown for purely reported events
\end{itemize}

\end{tcolorbox}

\begin{tcolorbox}[
    colback=SkyBlue!10,
    boxrule=0pt,
    colframe=white,
    left=0pt,
    right=0pt,
    top=0pt,
    bottom=0pt,
    ]

\small
Your annotation should be formatted as a JSON-style list of dictionaries:

\begin{verbatim}
[
  {
    "source": "<source_label>",  // e.g., "AUTHOR" or "AUTHOR_<source>"
    "event": "<event_token>",    // exact predicate from text
    "label": "<factuality_value>" // true/false/ptrue/pfalse/unknown
  },
  ...
]
\end{verbatim}

Step-by-Step Output Process:
\begin{itemize}[noitemsep, leftmargin=15pt, topsep=0pt]
    \item Walk through each event in the sentence
    \item Identify and explain all sources and their nesting
    \item Justify each factuality label from each source's viewpoint
    \item Produce the final JSON-style output
\end{itemize}

\end{tcolorbox}
\end{tcolorbox}
\label{fig:hybrid-prompt}
\end{figure*}
