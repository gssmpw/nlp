%%%% ijcai25.tex

\typeout{IJCAI--25 Instructions for Authors}

% These are the instructions for authors for IJCAI-25.

\documentclass{article}
\pdfpagewidth=8.5in
\pdfpageheight=11in

% The file ijcai25.sty is a copy from ijcai22.sty
% The file ijcai22.sty is NOT the same as previous years'
\usepackage{ijcai25}

% Use the postscript times font!
\usepackage{times}
\usepackage{soul}
\usepackage{url}
\usepackage[hidelinks]{hyperref}
\usepackage[utf8]{inputenc}
\usepackage[small]{caption}
\usepackage{graphicx}
\usepackage{amsmath}
\usepackage{amsthm}
\usepackage{booktabs}
\usepackage{algorithm}
\usepackage{algorithmic}
\usepackage[switch]{lineno}

\usepackage{caption}
\usepackage{subcaption}

% Comment out this line in the camera-ready submission
%\linenumbers

\urlstyle{same}

% the following package is optional:
%\usepackage{latexsym}

% See https://www.overleaf.com/learn/latex/theorems_and_proofs
% for a nice explanation of how to define new theorems, but keep
% in mind that the amsthm package is already included in this
% template and that you must *not* alter the styling.
\newtheorem{example}{Example}
\newtheorem{theorem}{Theorem}

% Following comment is from ijcai97-submit.tex:
% The preparation of these files was supported by Schlumberger Palo Alto
% Research, AT\&T Bell Laboratories, and Morgan Kaufmann Publishers.
% Shirley Jowell, of Morgan Kaufmann Publishers, and Peter F.
% Patel-Schneider, of AT\&T Bell Laboratories collaborated on their
% preparation.

% These instructions can be modified and used in other conferences as long
% as credit to the authors and supporting agencies is retained, this notice
% is not changed, and further modification or reuse is not restricted.
% Neither Shirley Jowell nor Peter F. Patel-Schneider can be listed as
% contacts for providing assistance without their prior permission.

% To use for other conferences, change references to files and the
% conference appropriate and use other authors, contacts, publishers, and
% organizations.
% Also change the deadline and address for returning papers and the length and
% page charge instructions.
% Put where the files are available in the appropriate places.


% PDF Info Is REQUIRED.

% Please leave this \pdfinfo block untouched both for the submission and
% Camera Ready Copy. Do not include Title and Author information in the pdfinfo section
\pdfinfo{
/TemplateVersion (IJCAI.2025.0)
}

\title{Trustworthy AI on Safety, Bias, and Privacy: A Survey}


% Single author syntax
% \author{
%     Author Name
%     \affiliations
%     Affiliation
%     \emails
%     email@example.com
% }

% Multiple author syntax (remove the single-author syntax above and the \iffalse ... \fi here)
%\iffalse
% \author{
% Xingli Fang\thanks{Equal contribution}$^1$
% \and
% Jianwei Li\thanksmark{*}$^1$\and
% Varun Mulchandani\thanks$^1$\And
% Jung-Eun Kim$^1$\\
% \affiliations
% $^1$Computer Science, North Carolina State University\\
% %$^2$Second Affiliation\\
% %$^3$Third Affiliation\\
% %$^4$Fourth Affiliation\\
% \emails
% \{first, second\}@example.com,
% %third@other.example.com,
% %fourth@example.com
% }

\author{
Xingli Fang \thanks{Equal contribution. In alphabetical order by last name.}$^1$
\and
Jianwei Li \footnotemark[1]$^1$\and
Varun Mulchandani \footnotemark[1]$^1$\And
Jung-Eun Kim \thanks{Correspondence.}$^1$\\
\affiliations
$^1$Computer Science, North Carolina State University\\
%$^2$Second Affiliation\\
%$^3$Third Affiliation\\
%$^4$Fourth Affiliation\\
\emails
\{xfang23, jli265, vmmulcha, jung-eun.kim\}@ncsu.edu
%third@other.example.com,
%fourth@example.com
}

%\fi

\begin{document}

\maketitle

\begin{abstract}
The capabilities of artificial intelligence systems have been advancing to a great extent, but these systems still struggle with failure modes, vulnerabilities, and biases. In this paper, we study the current state of the field, and present promising insights and perspectives regarding concerns that challenge the trustworthiness of AI models. In particular, this paper investigates the issues regarding three thrusts: safety, privacy, and bias, which hurt models' trustworthiness. For safety, we discuss safety alignment in the context of large language models, preventing them from generating toxic or harmful content. For bias, we focus on spurious biases that can mislead a network. Lastly, for privacy, we cover membership inference attacks in deep neural networks. The discussions addressed in this paper reflect our own experiments and observations. 
\end{abstract}


\vspace{-0.3cm}
%!TEX root = gcn.tex
\section{Introduction}
Graphs, representing structural data and topology, are widely used across various domains, such as social networks and merchandising transactions.
Graph convolutional networks (GCN)~\cite{iclr/KipfW17} have significantly enhanced model training on these interconnected nodes.
However, these graphs often contain sensitive information that should not be leaked to untrusted parties.
For example, companies may analyze sensitive demographic and behavioral data about users for applications ranging from targeted advertising to personalized medicine.
Given the data-centric nature and analytical power of GCN training, addressing these privacy concerns is imperative.

Secure multi-party computation (MPC)~\cite{crypto/ChaumDG87,crypto/ChenC06,eurocrypt/CiampiRSW22} is a critical tool for privacy-preserving machine learning, enabling mutually distrustful parties to collaboratively train models with privacy protection over inputs and (intermediate) computations.
While research advances (\eg,~\cite{ccs/RatheeRKCGRS20,uss/NgC21,sp21/TanKTW,uss/WatsonWP22,icml/Keller022,ccs/ABY318,folkerts2023redsec}) support secure training on convolutional neural networks (CNNs) efficiently, private GCN training with MPC over graphs remains challenging.

Graph convolutional layers in GCNs involve multiplications with a (normalized) adjacency matrix containing $\numedge$ non-zero values in a $\numnode \times \numnode$ matrix for a graph with $\numnode$ nodes and $\numedge$ edges.
The graphs are typically sparse but large.
One could use the standard Beaver-triple-based protocol to securely perform these sparse matrix multiplications by treating graph convolution as ordinary dense matrix multiplication.
However, this approach incurs $O(\numnode^2)$ communication and memory costs due to computations on irrelevant nodes.
%
Integrating existing cryptographic advances, the initial effort of SecGNN~\cite{tsc/WangZJ23,nips/RanXLWQW23} requires heavy communication or computational overhead.
Recently, CoGNN~\cite{ccs/ZouLSLXX24} optimizes the overhead in terms of  horizontal data partitioning, proposing a semi-honest secure framework.
Research for secure GCN over vertical data  remains nascent.

Current MPC studies, for GCN or not, have primarily targeted settings where participants own different data samples, \ie, horizontally partitioned data~\cite{ccs/ZouLSLXX24}.
MPC specialized for scenarios where parties hold different types of features~\cite{tkde/LiuKZPHYOZY24,icml/CastigliaZ0KBP23,nips/Wang0ZLWL23} is rare.
This paper studies $2$-party secure GCN training for these vertical partition cases, where one party holds private graph topology (\eg, edges) while the other owns private node features.
For instance, LinkedIn holds private social relationships between users, while banks own users' private bank statements.
Such real-world graph structures underpin the relevance of our focus.
To our knowledge, no prior work tackles secure GCN training in this context, which is crucial for cross-silo collaboration.


To realize secure GCN over vertically split data, we tailor MPC protocols for sparse graph convolution, which fundamentally involves sparse (adjacency) matrix multiplication.
Recent studies have begun exploring MPC protocols for sparse matrix multiplication (SMM).
ROOM~\cite{ccs/SchoppmannG0P19}, a seminal work on SMM, requires foreknowledge of sparsity types: whether the input matrices are row-sparse or column-sparse.
Unfortunately, GCN typically trains on graphs with arbitrary sparsity, where nodes have varying degrees and no specific sparsity constraints.
Moreover, the adjacency matrix in GCN often contains a self-loop operation represented by adding the identity matrix, which is neither row- nor column-sparse.
Araki~\etal~\cite{ccs/Araki0OPRT21} avoid this limitation in their scalable, secure graph analysis work, yet it does not cover vertical partition.

% and related primitives
To bridge this gap, we propose a secure sparse matrix multiplication protocol, \osmm, achieving \emph{accurate, efficient, and secure GCN training over vertical data} for the first time.

\subsection{New Techniques for Sparse Matrices}
The cost of evaluating a GCN layer is dominated by SMM in the form of $\adjmat\feamat$, where $\adjmat$ is a sparse adjacency matrix of a (directed) graph $\graph$ and $\feamat$ is a dense matrix of node features.
For unrelated nodes, which often constitute a substantial portion, the element-wise products $0\cdot x$ are always zero.
Our efficient MPC design 
avoids unnecessary secure computation over unrelated nodes by focusing on computing non-zero results while concealing the sparse topology.
We achieve this~by:
1) decomposing the sparse matrix $\adjmat$ into a product of matrices (\S\ref{sec::sgc}), including permutation and binary diagonal matrices, that can \emph{faithfully} represent the original graph topology;
2) devising specialized protocols (\S\ref{sec::smm_protocol}) for efficiently multiplying the structured matrices while hiding sparsity topology.


 
\subsubsection{Sparse Matrix Decomposition}
We decompose adjacency matrix $\adjmat$ of $\graph$ into two bipartite graphs: one represented by sparse matrix $\adjout$, linking the out-degree nodes to edges, the other 
by sparse matrix $\adjin$,
linking edges to in-degree nodes.

%\ie, we decompose $\adjmat$ into $\adjout \adjin$, where $\adjout$ and $\adjin$ are sparse matrices representing these connections.
%linking out-degree nodes to edges and edges to in-degree nodes of $\graph$, respectively.

We then permute the columns of $\adjout$ and the rows of $\adjin$ so that the permuted matrices $\adjout'$ and $\adjin'$ have non-zero positions with \emph{monotonically non-decreasing} row and column indices.
A permutation $\sigma$ is used to preserve the edge topology, leading to an initial decomposition of $\adjmat = \adjout'\sigma \adjin'$.
This is further refined into a sequence of \emph{linear transformations}, 
which can be efficiently computed by our MPC protocols for 
\emph{oblivious permutation}
%($\Pi_{\ssp}$) 
and \emph{oblivious selection-multiplication}.
% ($\Pi_\SM$)
\iffalse
Our approach leverages bipartite graph representation and the monotonicity of non-zero positions to decompose a general sparse matrix into linear transformations, enhancing the efficiency of our MPC protocols.
\fi
Our decomposition approach is not limited to GCNs but also general~SMM 
by 
%simply 
treating them 
as adjacency matrices.
%of a graph.
%Since any sparse matrix can be viewed 

%allowing the same technique to be applied.

 
\subsubsection{New Protocols for Linear Transformations}
\emph{Oblivious permutation} (OP) is a two-party protocol taking a private permutation $\sigma$ and a private vector $\xvec$ from the two parties, respectively, and generating a secret share $\l\sigma \xvec\r$ between them.
Our OP protocol employs correlated randomnesses generated in an input-independent offline phase to mask $\sigma$ and $\xvec$ for secure computations on intermediate results, requiring only $1$ round in the online phase (\cf, $\ge 2$ in previous works~\cite{ccs/AsharovHIKNPTT22, ccs/Araki0OPRT21}).

Another crucial two-party protocol in our work is \emph{oblivious selection-multiplication} (OSM).
It takes a private bit~$s$ from a party and secret share $\l x\r$ of an arithmetic number~$x$ owned by the two parties as input and generates secret share $\l sx\r$.
%between them.
%Like our OP protocol, o
Our $1$-round OSM protocol also uses pre-computed randomnesses to mask $s$ and $x$.
%for secure computations.
Compared to the Beaver-triple-based~\cite{crypto/Beaver91a} and oblivious-transfer (OT)-based approaches~\cite{pkc/Tzeng02}, our protocol saves ${\sim}50\%$ of online communication while having the same offline communication and round complexities.

By decomposing the sparse matrix into linear transformations and applying our specialized protocols, our \osmm protocol
%($\prosmm$) 
reduces the complexity of evaluating $\numnode \times \numnode$ sparse matrices with $\numedge$ non-zero values from $O(\numnode^2)$ to $O(\numedge)$.

%(\S\ref{sec::secgcn})
\subsection{\cgnn: Secure GCN made Efficient}
Supported by our new sparsity techniques, we build \cgnn, 
a two-party computation (2PC) framework for GCN inference and training over vertical
%ly split
data.
Our contributions include:

1) We are the first to explore sparsity over vertically split, secret-shared data in MPC, enabling decompositions of sparse matrices with arbitrary sparsity and isolating computations that can be performed in plaintext without sacrificing privacy.

2) We propose two efficient $2$PC primitives for OP and OSM, both optimally single-round.
Combined with our sparse matrix decomposition approach, our \osmm protocol ($\prosmm$) achieves constant-round communication costs of $O(\numedge)$, reducing memory requirements and avoiding out-of-memory errors for large matrices.
In practice, it saves $99\%+$ communication
%(Table~\ref{table:comm_smm}) 
and reduces ${\sim}72\%$ memory usage over large $(5000\times5000)$ matrices compared with using Beaver triples.
%(Table~\ref{table:mem_smm_sparse}) ${\sim}16\%$-

3) We build an end-to-end secure GCN framework for inference and training over vertically split data, maintaining accuracy on par with plaintext computations.
We will open-source our evaluation code for research and deployment.

To evaluate the performance of $\cgnn$, we conducted extensive experiments over three standard graph datasets (Cora~\cite{aim/SenNBGGE08}, Citeseer~\cite{dl/GilesBL98}, and Pubmed~\cite{ijcnlp/DernoncourtL17}),
reporting communication, memory usage, accuracy, and running time under varying network conditions, along with an ablation study with or without \osmm.
Below, we highlight our key achievements.

\textit{Communication (\S\ref{sec::comm_compare_gcn}).}
$\cgnn$ saves communication by $50$-$80\%$.
(\cf,~CoGNN~\cite{ccs/KotiKPG24}, OblivGNN~\cite{uss/XuL0AYY24}).

\textit{Memory usage (\S\ref{sec::smmmemory}).}
\cgnn alleviates out-of-memory problems of using %the standard 
Beaver-triples~\cite{crypto/Beaver91a} for large datasets.

\textit{Accuracy (\S\ref{sec::acc_compare_gcn}).}
$\cgnn$ achieves inference and training accuracy comparable to plaintext counterparts.
%training accuracy $\{76\%$, $65.1\%$, $75.2\%\}$ comparable to $\{75.7\%$, $65.4\%$, $74.5\%\}$ in plaintext.

{\textit{Computational efficiency (\S\ref{sec::time_net}).}} 
%If the network is worse in bandwidth and better in latency, $\cgnn$ shows more benefits.
$\cgnn$ is faster by $6$-$45\%$ in inference and $28$-$95\%$ in training across various networks and excels in narrow-bandwidth and low-latency~ones.

{\textit{Impact of \osmm (\S\ref{sec:ablation}).}}
Our \osmm protocol shows a $10$-$42\times$ speed-up for $5000\times 5000$ matrices and saves $10$-2$1\%$ memory for ``small'' datasets and up to $90\%$+ for larger ones.

%\clearpage
\section{Safety Alignment in LLMs}\label{sec:safety}

In Large Language Models, alignment aims to teach models human-desired behaviors and remove undesired behaviors. Safety alignment has often been treated as a subset of broader alignment challenges, with a primary focus on safety~\cite{li2024superficial}. In this context, the goal of safety alignment is preventing LLMs from generating toxic or harmful content and simultaneously considers security problems in adversarial scenarios, such as jailbreak attempts~\cite{qi2024ai}. 
% Since safety alignment has this specifically dedicated scope, unlike general alignment challenges, it has unique properties for the primary focus on safety.

\subsection{Why is Safety in LLMs Important?}

With the release of AI services such as ChatGPT, Claude, and Gemini, AI-powered applications have become increasingly integrated into our daily lives, spanning fields like healthcare, finance, education, transportation, and even military applications~\cite{openai2024chatgpt,team2023gemini}. However, this rapid adoption also raises serious concerns regarding AI misuse. For instance, in 2023, the first suspected case of AI-assisted suicide was reported~\cite{brusselstimes2023ai}, and AI-generated misinformation has been widely disseminated online, potentially manipulating public opinion~\cite{monteith2024artificial}. These incidents compel us to critically examine how to prevent AI from being misused in ways that harm society. 

This challenge has become even more pressing with the rise of open-sourced large language models such as Llama and Deepseek families~\cite{touvron2023llama,dubey2024llama,guo2025deepseek}, which allow individuals to control and finetune LLMs directly. As a result, the risks associated with AI misuse are expanding exponentially while government regulatory measures struggle to keep pace. Given these developments, AI safety has become a pressing need in the current research community.

\subsection{How to Implement Safety in LLMs?} %late 2022 to late 2023

% GPT-3 2020 powerful, 
% bias, safety ..... Discussione by news or researer, 
% OpenAI can't release it (GPT-3) to the public

% prevent direct attacks
Upon the release of GPT-3 in 2020, we witnessed LLM's remarkable language generation capabilities. However, concerns regarding bias, toxic content, and hallucinations also emerged, indicating that the model was still not ready for public release~\cite{brown2020language}. 
% However, even before LLMs were introduced to the public, concerns regarding bias, toxic content, and hallucinations also emerged, which highlighted the challenges.
Later, in 2021, Anthropic introduced the HHH principle—Helpful, Honest, and Harmless—as a key principle of a truly beneficial AI assistant~\cite{askell2021general}. This concept set a foundational standard for AI assistants, clearly indicating that safety is an important objective of LLMs.

The launch of ChatGPT in late 2022 marked a turning point, as it was the first large-scale exposure of LLMs to the public, allowing people to experience an AI assistant that felt genuinely helpful and capable of human-like reasoning. This breakthrough was largely built upon the techniques outlined in the InstructGPT, which introduced Supervised Fine-Tuning (SFT) and Reinforcement Learning with Human Feedback (RLHF) as key methods for aligning LLMs with human values~\cite{ouyang2022training}. Around the same time, Anthropic also released its own research on RLHF, demonstrating its effectiveness in guiding model behavior~\cite{bai2022training}. These efforts treated safety (or harmlessness) as a subset of preference optimization, laying the foundation for future safety alignment research. From a high-level perspective, subsequent alignment techniques can be approximately categorized into the following approaches:  

\paragraph{In-Context Learning}  
This category of methods does not rely on model retraining; instead, inspired by Chain-of-Thought (CoT) reasoning and GPT-3, researchers have designed either hard prompts or soft prompts to guide the model toward producing helpful and harmless outputs~\cite{wei2022chain,brown2020language}. Examples include the official system prompt in Llama2 and soft prompts optimized via P-tuning, which embed safety reasoning signals that may not be readable by humans~\cite{touvron2023llama,xie2023defending}. 
However, in-context learning has several limitations: \textbf{(1)} It requires careful manual design and optimization, making automation and scalability difficult. \textbf{(2)} It has limited generalization ability, struggling to handle long-tail scenarios where prompts may not be well-defined. \textbf{(3)} It is highly sensitive to prompt design; small variations in the prompt can lead to drastically different outputs, making the method vulnerable to jailbreak attacks. \textbf{(4)} Since in-context learning does not modify the model's weights, it cannot permanently alter its underlying behavior.  

\paragraph{Imitation Learning}  
This approach primarily relies on supervised fine-tuning (SFT) to train models using carefully curated aligned datasets~\cite{ouyang2022training}.~\cite{zhou2024lima} introduced the Superficial Alignment Hypothesis, suggesting that alignment may merely adjust the model’s output distribution to be more interaction-friendly rather than fundamentally changing its reasoning capabilities. They demonstrated that with only about 1k high-quality training examples, a model could achieve performance comparable to GPT-4. 
% This result suggests that large language models already acquire substantial human interaction knowledge during pretraining, and the alignment process might serve more to refine the style of responses rather than deeply alter the model’s reasoning mechanisms. 
However, this paper focused predominantly on helpfulness, and its training data contained only 13 safety-related samples, which led to poor overall safety performance.

\paragraph{Reinforcement Learning}  
Reinforcement Learning with Human Feedback (RLHF) improves model alignment by optimizing the policy (parameters) using reinforcement learning. The core idea is to train the model to generate better responses by maximizing a reward function, which is defined by a reward model trained on human feedback. The reward model assigns scores to different responses, providing a structured signal to guide the optimization process. To prevent the model from diverging too far from its original behavior, RLHF typically employs proximal policy optimization (PPO), which introduces a KL divergence constraint between the logits of the original and updated policies~\cite{christiano2017deep}. This ensures that while the model learns to produce more aligned outputs, it does not lose fluency or develop unintended artifacts. RLHF has significantly improved both helpfulness and harmlessness, establishing itself as a foundational technique in alignment research~\cite{ouyang2022training,bai2022training}. Despite its effectiveness, RLHF comes with significant challenges. The dependence on human feedback makes it highly labor-intensive, as continuous human involvement is required to annotate responses and update the reward model. To reduce reliance on human labor and improve scalability, researchers have proposed Reinforcement Learning with AI Feedback (RLAIF) as an alternative. For example, Anthropic incorporates an AI-generated feedback mechanism, where the model itself evaluates responses, reducing the need for human intervention while still refining behavior and mitigating harmful outputs~\cite{bai2022constitutional}. However, both RLHF and RLAF are resources intensive, as their training typically involves the following components: (1) a reference model, (2) a policy model (an adapted version of the base model being optimized), and (3) a reward model trained to assess response quality. In some implementations, additional models may even be required, further increasing the memory and computational burden~\cite{yao2023deepspeed}. 

\paragraph{Contrastive Learning}  
Researchers have explored contrastive learning as a more efficient alternative to reduce the complexity and resource demands of RL-based approaches. Direct Preference Optimization (DPO) introduced the key insight that large language models inherently act as implicit reward models~\cite{rafailov2024direct}. By leveraging the preference dataset, a model can directly learn preferences without requiring an explicit reinforcement learning loop and a reward model. A similar approach is proposed in~\cite{liu2023training}, where contrastive signals are embedded in datasets to steer models toward desired behaviors. This category of methods has notable advantages: (1) it significantly reduces memory costs since optimization requires loading at most two models at a time, and with logit caching, only one model may be sufficient; (2) it eliminates the need for an explicit reward model, simplifying the alignment process. However, these methods also come with challenges: the performance is highly dependent on the quality of the preference dataset.
% —if the dataset is insufficient or biased, the model may fail to learn appropriate alignment signals.

\paragraph{Conditional Learning}  
This approach is conceptually similar to in-context learning but differs in that it explicitly optimizes the model to recognize specific triggers, ensuring that desired behaviors are always generated when these triggers are present. The key idea is to induce the model to produce desired behavior rather than removing undesired outputs. However, this approach has significant vulnerabilities when confronted with jailbreak attacks and thus is rarely used as a standalone alignment technique~\cite{korbak2023pretraining}. 


\subsection{Safety in Existing LLMs is Still Brittle} % late 2023 to middle 2024

% fine-tuning attack
% Jailbreak attack
% Decoding exploitation attack
% Superficiality

Although various general alignment methods have been proposed, and safety alignment has been improved to some extent, treating safety merely as a subset of human preference overlooks its unique challenges. As a result, current alignment techniques remain vulnerable to adversarial attacks. In literature, adversarial attacks on LLMs can generally be classified into three types: 
% \textbf{(1) Direct Attacks}: The model is explicitly prompted with malicious requests to induce harmful outputs. While aligned models can resist many such attacks, failure cases persist, especially in previously unseen scenarios. 
\textbf{(1) Jailbreak Attacks}: Attackers exploit techniques such as role-playing or suffix injections to bypass safety guardrails and manipulate the model into generating harmful content. Studies show that these methods can effectively evade existing alignment mechanisms~\cite{zouuniversal}. This vulnerability extends beyond open-source models—even state-of-the-art systems like the GPT-4 series struggle to block harmful outputs in complex, nested scenarios consistently~\cite{li2023deepinception}. \textbf{(2) Finetuning Attacks}: Even unintentional finetuning can weaken a model’s safety mechanisms. A model trained with safety alignment may gradually lose its safeguards when adapted to downstream tasks via domain-specific finetuning, even if the dataset itself is benign. This phenomenon has been observed in both open-source and proprietary models~\cite{qi2023fine}. \textbf{(3) Decoding Attacks}: Safety-aligned models may still produce harmful content under certain decoding settings, such as modifications to Top-P, Top-K, or Temperature~\cite{huang2023catastrophic}. These variations may break built-in safeguards, leading to outputs that would otherwise be restricted under default configurations. These attack vectors underscore a critical issue: existing safety alignment methods lack robustness and, in many cases, remain highly brittle, especially in novel or adversarial conditions.  

\subsection{How to Implement Robust Safety in LLMs?} % middle 2024 to now

Recent studies have highlighted that existing alignment methods often achieve safety at a superficial level. \cite{wei2024assessing} identified safety-critical parameters in LLMs and found that removing them catastrophically degrades safety performance while leaving utility performance unaffected. However, their findings also revealed that merely retaining these safety-critical parameters does not preserve safety under finetuning attacks. In contrast, \cite{li2024superficial} demonstrated that the atomic functional unit for safety in LLMs resides at the \emph{neuron level} and successfully mitigated finetuning attacks by freezing updates to these safety-critical components. Their study further showed that aligned models remain vulnerable to finetuning attacks because key attributes, such as utility, can be achieved by repurposing neurons originally responsible for other functions, such as safety. Additionally, this research examined how alignment influences model behavior in safety-critical contexts and observed that, at its core, this effect could be framed as an implicit safety-related binary classification task. To resolve the superficiality issue above, they further propose that alignment should enable models to choose the correct safety-aware reasoning direction (either to refuse or fulfill) at each generation step, ensuring safety throughout the entire response. 
However, their work did not propose specific methods for implementing this deeper safety mechanism in practice.

~\cite{qi2024ai} have also examined the shallow alignment in existing LLMs and found that this issue often stems from alignment disproportionately affecting early-generated token distribution. This creates optimization shortcuts where models rely on superficial decision patterns, leading them toward local optima that fail to generalize to more complex safety challenges. To mitigate this, they introduced a data augmentation strategy designed to expose models to more nuanced scenarios where an initially harmful response later transitions into a safe refusal. Similarly,~\cite{yuan2024refuse} have adopted more aggressive data construction rules, aiming to add more variety of training examples. However, while these methods increase the diversity of training examples, they do not fundamentally address the root problem. All of these highlight a critical issue: Existing alignment techniques lack effective and robust mechanisms to handle complex and nuanced harmful reasoning patterns. In this context, this survey paper acknowledges the hypothesis from~\cite{li2024superficial}, and believes that a robust safety alignment should teach the model to select and maintain the correct safety reasoning direction throughout the entire text generation process.
\section{Spurious Biases and their Impact on Generalizability}\label{sec:bias}
\vspace{0.1in}
Deep neural networks tend to learn and rely on correlations between partly predictive spurious features that are causally unrelated to ground truth labels in the training data. For example, assume one wants to train a deep neural network to be able to correctly classify pictures of animals as Cows or Camels~\cite{Arjovsky2019}. Due to selection bias, most samples that have Cows in them are present in green backgrounds in the training set while most Camels are present in brown backgrounds. In such a setting, deep networks are shown to learn the correlation between the background color and the ground truth labels. Such correlations are referred to as spurious correlations. In practice, deep networks often prefer spurious correlations over correlations between fully predictive, general features (features of Cows or Camels) and ground truth labels. The learning of and reliance on these correlations is undesirable because these features may disappear or become correlated with a different label or task during testing, causing these networks to malfunction.

\vspace{0.1in}
\subsection{Why do Deep Neural Networks Learn and Rely on Spurious Correlations?} Deep networks learn and rely on spurious correlations due to a preference for simpler features over those that are more complex in nature~\cite{geirhos2020shortcut,Kirichenko2023ICLR}. \cite{Shah2020Neurips} show that such simplicity bias is extreme in practice. They consider a binary classification task, where every sample of each class contains two sets of features. One of these features is simpler than the other. They show that when a network is trained on this task, the network will fully ignore the more complex feature and rely only on the simpler feature when making predictions. In settings where the simpler feature does not exist in all samples, deep networks learn both sets of features but exhibit strong reliance on the simpler feature~\cite{Kirichenko2023ICLR}. All existing works that study spurious correlations generally assume the same set-up, where spurious features are only partly predictive of the task while general, invariant features exist in every sample within their respective class.

\vspace{0.1in}
\subsection{Mitigating Spurious Correlations: Existing Practice}

Existing solutions that enable a network to mitigate spurious correlations operate under the implicit assumption that a network trained using Empirical Risk Minimization (ERM)~\cite{Vapnik98} will learn and rely on spurious correlations due to a preference for simpler features. Based on this assumption, promising solutions generally fall into the following categories:


\paragraph{Altering the Training Distribution. } The degree to which a trained network relies on spurious correlations depends on various factors. Of these factors, the most extensively studied is the proportion of samples within the train set that contain the spurious feature. The greater the proportion of samples containing the spurious feature, the greater the reliance on spurious correlations. To reduce the proportion of samples containing spurious features, existing works aim to either up-weight samples that do not contain spurious features, down-weight samples that contain spurious features, or remove samples containing spurious features. Most works that attain state-of-the-art results on popular benchmarks rely on the availability of sample-environment membership information.~\cite{Liu2021ICML} up-weight samples that do not contain spurious features while~\cite{Yang2024AISTATS} down-weight samples containing spurious features in conjunction with a similar up-weighting step.~\cite{Kirichenko2023ICLR,Deng2023Neurips} simply balance the number of samples belonging to each environment when proposing mitigation strategies. However, they make use of the assumption that environments that are overrepresented force networks to rely on spurious correlations. Attaining such sample-environment information is expensive due to the need for human intervention and annotation. To overcome this problem, some works aim to infer such sample-wise environment labels.~\cite{Liu2021ICML} train a network with heavy regularization to identify samples with and without spurious features based on whether these samples were correctly classified during training.~\cite{Ahmed2021ICLR} aim to maximize the Invariant Risk Minimization penalty (IRM)~\cite{Arjovsky2019} during training to obtain environment-labels.~\cite{Zhang2022ICML} cluster a biased network's representations to obtain these labels.~\cite{pezeshki2024ICML} attain these labels by utilizing a twin-network setting where networks are encouraged to learn environmental cues, thereby aiding in sample-environment discovery.


\paragraph{Altering a Network's Learned Representations.   } These works either align the representation of samples within a class that contains spurious features and those that do not, or simply block parts of a network's representation that encodes spurious information.~\cite{Ahmed2021ICLR} aim to align the predicted distributions for samples belonging to the same class but different environments using a KL-divergence term in the optimization function.~\cite{Zhang2022ICML} make use of a contrastive loss function which brings representations of samples within the same class but different environments closer while distancing representations of samples belonging to the same environment but different classes.~\cite{gandelsman2024ICLR} identify the role of individual attention heads in CLIP-ViT and remove those heads associated with spurious cues.


\paragraph{Prioritizing Worst-Group Accuracy During Training.  } \cite{sagawa2020ICLR} optimize a network using an objective that minimizes the risk for the group of samples belonging to the environment with the maximum risk within a class.


\paragraph{Fine-tuning on an Unbiased Dataset. } \cite{Kirichenko2023ICLR} re-train the last layer of a trained (biased) network on a dataset where the proportion of samples containing the spurious feature is significantly lower than the original training set.~\cite{moayeri2023Neurips} follow similar retraining, where they fine-tune a trained network on a small dataset with minimal spurious features, where such a set is obtained using human supervision.


\begin{figure*}[t]
\centering     %%% not \center
\includegraphics[width=0.67\linewidth]{figs/R50-O-MTA.pdf}
\caption{Excluding only a handful of training samples with spurious features and hard core features mitigates spurious correlations. This is indicated by high Worst Group Accuracies (Female test samples with glasses.) Excluding up to 97\% of all training samples with spurious features and easy core features shows no improvements in worst group accuracy. This figure is excerpted from~\protect\cite{Mulchandani2025ICLR}.}
%\vspace{-0.3cm}
%\label{fig:Exclusion}
\label{fig:keyplayers}
\end{figure*}


\subsection{Limitations of Existing Techniques}

\paragraph{Heavy Dependence on Sample-Environment Membership Information.}

Promising solutions that overcome spurious correlations hinge on the availability or identifiability of sample-environment membership information. In other words, these solutions work with the assumption that it is possible to determine which groups of samples were drawn from which environments. Additionally, recent works that aim to infer this information are unable to attain competitive performances with techniques that directly use this information.

\paragraph{Assuming Over-Represented Environment Groups as Contributors to Learning of Spurious Correlations.}

All existing studies that aim to overcome spurious correlations work with the assumption that environments/groups that are overrepresented are the groups that contribute to the learning of spurious correlations. Reliance on this assumption makes it easy to identify which samples contain the spurious features causing problems, which allows for further representational alignment or changes to the training distribution.~\cite{Mulchandani2025ICLR} show that this assumption does not always hold in practice and that minority groups can contain spurious features that can mislead a network significantly.

\paragraph{Representational Collapse.}

Works by~\cite{Ahmed2021ICLR,Zhang2022ICML} align representations of samples belonging to different environments within the same class. While effective at overcoming spurious correlations, these techniques reduce overall testing accuracies due to the loss of representational richness.

\paragraph{Extensive Hyperparameter Tuning.}

Most works depend heavily on hyperparameter tuning, where they optimize for the best worst-group accuracy. Optimization is done with the help of a validation split that mimics the distribution of shifted testing environments. Such access to a validation split that mimics test-time distribution is unrealistic.~\cite{gulrajani2021ICLR} show that without access to such a validation set, standard Empirical Risk Minimization outperforms seemingly promising solutions.

\subsection{Creating Robust Solutions: Next Steps}

\paragraph{Moving Past Egalitarian Approaches.}

Most standard and state-of-the-art techniques assume an equal contribution to the learning and reliance of spurious correlations. In other words, every training sample belonging to the environment known to cause reliance on spurious correlations is treated the same way.~\cite{Mulchandani2025ICLR} show that samples within an environment contribute differently to learning of spurious correlations and show that these differences are extreme in practice. They train a network to learn gender classification, where a fraction of the male samples contain eyeglasses. In their work, the degree of spurious feature reliance is measured by observing the test accuracy of female samples containing eyeglasses (Worst-Group Accuracy). They observe that removing 97\% of easy-to-understand male samples with eyeglasses has almost no improvement on the testing accuracy of female samples with eyeglasses. However, removing 10\% of hard-to-understand male samples with eyeglasses doubles the testing accuracy of female samples with eyeglasses, as shown in Fig.~\ref{fig:keyplayers}. They show that such pruning has minimal impact on testing accuracy of the male class.


\paragraph{Overcoming Reliance on Unbiased Validation Sets.}

Access to unbiased, environment-balanced datasets for fine-tuning or environment-based hyperparameter tuning is unrealistic. The results presented in Fig.~\ref{fig:keyplayers} by pruning samples with hard-to-understand male features do not make use of any hyperparameter tuning.

% architecture; data;
% efficiency and bias

\subsection{Intersection of Spurious Correlations with Other Areas of Study}
% \subsubsection{Model Capacity and Spurious Correlations. } {Sagawa} show that increasing the number of parameters of a network increases the degree to which the network relies on spurious correlations.

\paragraph{Reasoning and Spurious Correlations. } Recent work has shown that deep neural networks have a tendency to rely on short-cut solutions or heuristics when learning to solve reasoning tasks, instead of robust rules that actually cover the solution to the problem~\cite{Zhang2023IJCAI,nikankin2025ICLR}. This makes it difficult for networks to generalize to different or more challenging domains. A good example of this is the length generalization problem, where a network is unable to solve simple arithmetic operations on numbers of length different from those observed during training, despite these operations requiring the same set of rules~\cite{Zhou2024ICLR,lee2024ICLR}.

\paragraph{Privacy and Spurious Correlations.   } \cite{yang2022} show that neural networks pick up on spurious features present in only a handful of training samples, which can lead to privacy leaks.

\section{Privacy}\label{sec:privacy}
In this section, we discuss the Membership Privacy Attack (MIA) in which an attacker tries to infer whether a sample belongs to the train set or not.
Common deep learning models are often vulnerable to such membership privacy attacks when they exhibit behavioral discrepancies between training and unseen data points. %These risks expose the training data information of the model. 
We discuss such privacy risks from two perspectives based on the current advancement. The first perspective is from the correlations between the capacity of the learning model and the complexity of training data points (and/or the set) regarding privacy. The other perspective is from privacy preservation and model generalizability.


\begin{figure}[t]
     \centering
     \begin{subfigure}[b]{0.21\textwidth}
         \centering
         \includegraphics[width=0.95\textwidth]{figs/tmg_attack_stat.pdf}
         \caption{Privacy attack success rate}
         \label{fig:cmp_half_full_tmg}
     \end{subfigure}
     \hspace{0.1in}
     \begin{subfigure}[b]{0.21\textwidth}
         \centering
         \includegraphics[width=0.95\textwidth]{figs/tmg_ce_stat.pdf}
         \caption{Cross-entropy distribution}
         \label{fig:cmp_half_full_tmg_ce}
     \end{subfigure}
      \caption{Per-sample attack success rates and loss distribution in the original trainset and the half (MobileNetV3-S, 40 runs, TinyImageNet). Test accuracies parenthesized in the legends.}
      \vspace{-0.1cm}
  \label{fig:cmp_data_tmg}
\end{figure}



% model capacity vs. data complexity
\subsection{Privacy Correlation on Model Capacity and Data Complexity}

The membership privacy risks of machine learning models are mainly caused by the model's memorization of the training data points. This means that over-memorization is one of the sources of privacy risks \cite{yeom2020overfitting}. \cite{carlini2022onion} claimed some data points must be more privacy-risky after the removal of original privacy-risky data points and retraining from scratch. This is mainly due to the relative changes between the model capacity and the data complexity.
\cite{tan2022parameters} found that excess model capacity (\textit{a.k.a.}, overparameterization) is another factor of privacy risks. Additionally, \cite{tan2023dimensionblessing} showed the larger-capacity model not only memorizes more on training data points than smaller networks but also memorizes faster (\emph{i.e.}, within fewer iterations). In fact, changing data complexity can also change the model's memorization behavior.
As shown in Fig.~\ref{fig:cmp_data_tmg}, we empirically find that increasing data capacity can prevent privacy leakage as utilizing the entire dataset shows much better privacy preservation than utilizing only the half, which implies that the model may have well-concealed privacy under proper data complexity. 
Since a lower-capacity model (considering the data complexity) can protect privacy better, the sparsity of the model can also be beneficial to privacy. \cite{kaya2020effectivenessregularizationmembership} showed that regularization can mitigate some privacy risks while data augmentation techniques also help with privacy. The role of data augmentation was further studied and it was pointed out that only specific data augmentation techniques have such ability to mitigate privacy risks \cite{kaya2021whendataaug,yu2021howdoesdataaug}. In addition, \cite{yuan2022miapruning} found that traditional model pruning techniques do not work as well as the layer-wise architectural changes of the model in terms of reducing model capacities for privacy.
Besides classification models, such privacy risk led by improper memorization also widely exists in models that are trained in various forms, e.g., regression learning \cite{tarun2023regression_ua} and self-supervised learning \cite{wang2024localizing}. 





% \begin{figure}
% \vspace{-0.4cm}
%   \begin{center}
%   \subfloat[Privacy attack success rate]{
%     \label{fig:cmp_half_full_tmg}
%     \includegraphics[width=0.45\textwidth]{figs/tmg_attack_stat.pdf}
%   }
%   \subfloat[Cross-entropy distribution]{
%     \label{fig:cmp_half_full_tmg_ce}
%     \includegraphics[width=0.45\textwidth]{figs/tmg_ce_stat.pdf}
%   }
%   \end{center}
%   \caption{Per-sample attack success rates and loss distribution in the original trainset and the half (MobileNetV3-S, 40 runs).}
%   \label{fig:cmp_data_tmg}
%   \vspace{-0.1cm}
% \end{figure}






\begin{figure}[t]
\begin{center}
\begin{subfigure}[b]{0.19\textwidth}
   \centering
   \includegraphics[width=\textwidth]{figs/d2b_mia_train.png}
   \caption{Member (train)}
   \label{fig:d2db_mia_train}
\end{subfigure}
\quad
\begin{subfigure}[b]{0.19\textwidth}
    \centering
   \includegraphics[width=\textwidth]{figs/d2b_mia_test.png}
   \caption{Non-Member (test)}
   \label{fig:d2db_mia_test}
\end{subfigure}
\\
\begin{subfigure}[b]{0.19\textwidth}
         \centering
   \includegraphics[width=\textwidth]{figs/d2o_d2b_train.png}
   \caption{Member (train)}
   \label{fig:d2o_d2db_train}
\end{subfigure}
\quad
\begin{subfigure}[b]{0.19\textwidth}
         \centering
   \includegraphics[width=\textwidth]{figs/d2o_d2b_test.png}
   \caption{Non-Member (test)}
   \label{fig:d2o_d2db_test}
\end{subfigure}
\end{center}
\caption{\textbf{[1st row]}: the distance to the decision boundary and MIAs accuracy; \textbf{[2nd row]}: the distance to the origin and the distance to the decision boundary. For a sample's distance to the decision boundary, we use the difference between 1st and 2nd maximum prediction probabilities. The results are obtained from dozens of independent experiments. The blue charts ((a) \& (c)) are from train set, and the green charts ((b) \& (d)) are from test set. (ResNet18, CIFAR-100). This figure is excerpted from \protect\cite{fang2024representation}.}
\vspace{-0.1cm}
\label{fig:d2o_d2db_mia}
\end{figure}




% privacy vs. generalizability
\subsection{Trade-Offs Between Privacy Preservation and Generalizability}

The behavioral inconsistency of deep learning models in training and testing time, i.e., bad generalizability, leads to privacy-leakage problems. The attacker can steal various information from highly valued samples used to train the model according to this inconsistency. 


To show the relationship between representation inconsistency and MIAs accuracy, we visualize the sample-level distribution of the training and testing sets. Fig.~\ref{fig:d2o_d2db_mia} displays the sample-level predictions of MIAs accuracy versus distance to the decision boundary, as well as the relationship between distance to the origin and distance to the decision boundary. The distance to the decision boundary and the distance to the origin are computed from the last and the penultimate layers, respectively. When trained with the standard cross-entropy loss, the model exhibits distinct prediction and attack distributions for members and non-members in both of the layers, indicating that there are multiple privacy-risky layers in the model due to disagreement of representation alignment.


Hence, a straightforward way to mitigate privacy vulnerability is to align the predictions (and representations) between training and testing sets. In the following paragraphs, the introduced approaches try to achieve this alignment goal from different aspects.
In this section, we categorize them into three categories: the model-level solutions, the external obfuscators, and the data-level solutions. The approaches are overviewed in Fig.~\ref{fig:overview_privacy_defence}.


\begin{figure}[t]
     \centering
         \includegraphics[width=0.92\linewidth]{figs/privacy_approaches.pdf}
         \vspace{-0.1cm}
      \caption{The overview of the privacy preservation approaches.}
      \vspace{-0.2cm}
  \label{fig:overview_privacy_defence}
\end{figure}


% model level
\paragraph{Model-Level Solutions}
The model-level solutions aim to develop a mechanism to make the prediction distributions aligned on the model's end. 
A classical model-level solution is differentially private stochastic gradient descent (\texttt{DP-SGD}) \cite{abadi2016dpsgd}. It adds the noise into the optimizer to prevent the model from taking the (undesirable) easiest way to fit on the training data points and also memorizing them.
\cite{nasr2018advreg} introduced an adversarial training framework (\texttt{AdvReg}) that mitigates membership inference attacks by aligning prediction distributions. It tries to develop a discriminator, similar to GAN, to identify the prediction inconsistency of the model while it makes the model try to deceive the discriminator to achieve prediction alignment.
\cite{chen2022relaxloss} (\texttt{RelaxLoss}) established a threshold to prevent improper fitting for the alignment between member and non-member distributions while preserving the model's generalizability through a technique similar to label smoothing.
\cite{tan2023ws} proposed weighted smoothing (\texttt{WS}) to mitigate memorization by adding normalized random noise to the weights.
\cite{liu2024ccloss} incorporated a concave term called Convex-Concave Loss (\texttt{CCLoss}) to lessen the convexity of loss functions, aiming to enhance privacy preservation.
Besides end-to-end solutions, there are also some studies exploring finer-grained solutions.
\cite{fang2024representation} introduced the Saturn Ring Classification Module (\texttt{SRCM}) to bound the representation magnitude to mitigate prediction disparity.
\cite{fang2024crl} tried to align representations in multiple layers by Center-based Relaxed Learning (\texttt{CRL}).
\cite{hu2024past} proposed Privacy-Aware Sparsity Tuning (\texttt{PAST}) to measure weight-level privacy sensitivity and deactivate privacy-risky weights via regularization. 



% Obfuscator
\paragraph{External Obfuscator}
The external obfuscator is a special kind of model-level solution. Instead of developing a privacy-safe model, it aims to build an obfuscator, which reproduces the prediction probabilities, to remedy the inconsistency in the prediction probabilities.
Similar to the idea of \texttt{DP-SGD}, \texttt{MemGuard} \cite{jia2019memguard} interferes with the prediction confidence distribution of the model by adding additional noise after the model has been trained. 
\cite{yang2023purifier} tried to develop a VAE-based external prediction obfuscator named \texttt{Purifier} to align the prediction probabilities' disparity. Different from \texttt{MemGuard}, it tries to reconstruct the prediction confidences to remove the prediction inconsistency instead of noise confusion.


% data level
\paragraph{Data-Level Solutions}
The data level approaches have two principles: training the privacy-safe model via \textbf{(i) privacy-safe data} or \textbf{(ii) privacy-safe labels}. It is straightforward that when all training data points are privacy-safe, there are no privacy-risky features included in the data, such as shortcut features \cite{geirhos2020shortcut}.


The most straightforward solution to produce privacy-safer data is \texttt{Data Augmentation}. There are some data augmentation techniques, such as random cropping and flipping, determined that are able to produce privacy-safer data \cite{kaya2021whendataaug,yu2021howdoesdataaug}. With augmented data, the model can usually achieve better privacy and generalizability. However, there are still no quantifiable metrics to measure how to further produce privacy-safe data through data augmentation. In other words, although the machine learning model can obtain privacy for free via data augmentation, it is unclear if the model achieves complete privacy safety yet.
\cite{stadler2022groundhogday} tried to analyze the effect of synthetic data on the model's privacy (\texttt{Data Synthesis}). 
Besides these two kinds of solutions, there is also an intuitive way to mitigate privacy risks. The first one is \texttt{Data Pruning}. With data pruning techniques, the model can use only a small amount of training data points to develop a well-generalized model. In other words, most membership privacy of the entire train set can still be protected. However, the privacy risk mitigation by data pruning is still not perfect, because some membership information of pruned data points could be leaked \cite{li2024datalineageinferenceuncovering}. The other one is \texttt{Data Distillation}. The data distillation aims to refine generalizability-critical features to produce some representative synthetic data. In this process, privacy-risky features can be removed \cite{dong2022privacy_dataset_condensation}. As this direction has not been extensively researched yet, it is foreseen that more studies will be contributed to this topic very soon.


Since the prediction disparity in training and testing is due to the improper fitting of the training data points, another way is to stop the model from further fitting into the data when it has learned enough information from the data. This means that if an ideal set of labels exists, the model can be trained perfectly privacy-safe on these labeled data. An intuitive idea is a distillation approach for membership privacy (\texttt{DMP}) \cite{shejwalkar2021dmp}. It trains a protected model via non-member data and produces labels from an unprotected model.
Another solution to better utilize limited data is self-ensemble architecture (\texttt{SELENA}) \cite{tang2022selena}. It developed an ensemble with an efficient sampling strategy to produce privacy-safe labels with better generalizability.


\section{Conclusion}

In this paper, we propose a sample weight averaging strategy to address variance inflation of previous independence-based sample reweighting algorithms. 
We prove its validity and benefits with theoretical analyses. 
Extensive experiments across synthetic and multiple real-world datasets demonstrate its superiority in mitigating variance inflation and improving covariate-shift generalization.  


%\newpage



% \section{Introduction}

% The {\it IJCAI--25 Proceedings} will be printed from electronic
% manuscripts submitted by the authors. These must be PDF ({\em Portable
%         Document Format}) files formatted for 8-1/2$''$ $\times$ 11$''$ paper.

% \subsection{Length of Papers}


% All paper {\em submissions} to the main track must have a maximum of seven pages, plus at most two for references / acknowledgements / contribution statement / ethics statement.

% The length rules may change for final camera-ready versions of accepted papers and
% differ between tracks. Some tracks may disallow any contents other than the references in the last two pages, whereas others allow for any content in all pages. Similarly, some tracks allow you to buy a few extra pages should you want to, whereas others don't.

% If your paper is accepted, please carefully read the notifications you receive, and check the proceedings submission information website\footnote{\url{https://proceedings.ijcai.org/info}} to know how many pages you can use for your final version. That website holds the most up-to-date information regarding paper length limits at all times.


% \subsection{Word Processing Software}

% As detailed below, IJCAI has prepared and made available a set of
% \LaTeX{} macros and a Microsoft Word template for use in formatting
% your paper. If you are using some other word processing software, please follow the format instructions given below and ensure that your final paper looks as much like this sample as possible.

% \section{Style and Format}

% \LaTeX{} and Word style files that implement these instructions
% can be retrieved electronically. (See Section~\ref{stylefiles} for
% instructions on how to obtain these files.)

% \subsection{Layout}

% Print manuscripts two columns to a page, in the manner in which these
% instructions are printed. The exact dimensions for pages are:
% \begin{itemize}
%     \item left and right margins: .75$''$
%     \item column width: 3.375$''$
%     \item gap between columns: .25$''$
%     \item top margin---first page: 1.375$''$
%     \item top margin---other pages: .75$''$
%     \item bottom margin: 1.25$''$
%     \item column height---first page: 6.625$''$
%     \item column height---other pages: 9$''$
% \end{itemize}

% All measurements assume an 8-1/2$''$ $\times$ 11$''$ page size. For
% A4-size paper, use the given top and left margins, column width,
% height, and gap, and modify the bottom and right margins as necessary.

% \subsection{Format of Electronic Manuscript}

% For the production of the electronic manuscript, you must use Adobe's
% {\em Portable Document Format} (PDF). A PDF file can be generated, for
% instance, on Unix systems using {\tt ps2pdf} or on Windows systems
% using Adobe's Distiller. There is also a website with free software
% and conversion services: \url{http://www.ps2pdf.com}. For reasons of
% uniformity, use of Adobe's {\em Times Roman} font is strongly suggested.
% In \LaTeX2e{} this is accomplished by writing
% \begin{quote}
%     \mbox{\tt $\backslash$usepackage\{times\}}
% \end{quote}
% in the preamble.\footnote{You may want to also use the package {\tt
%             latexsym}, which defines all symbols known from the old \LaTeX{}
%     version.}

% Additionally, it is of utmost importance to specify the {\bf
%         letter} format (corresponding to 8-1/2$''$ $\times$ 11$''$) when
% formatting the paper. When working with {\tt dvips}, for instance, one
% should specify {\tt -t letter}.

% \subsection{Papers Submitted for Review vs. Camera-ready Papers}
% In this document, we distinguish between papers submitted for review (henceforth, submissions) and camera-ready versions, i.e., accepted papers that will be included in the conference proceedings. The present document provides information to be used by both types of papers (submissions / camera-ready). There are relevant differences between the two versions. Find them next.

% \subsubsection{Anonymity}
% For the main track and some of the special tracks, submissions must be anonymous; for other special tracks they must be non-anonymous. The camera-ready versions for all tracks are non-anonymous. When preparing your submission, please check the track-specific instructions regarding anonymity.

% \subsubsection{Submissions}
% The following instructions apply to submissions:
% \begin{itemize}
% \item If your track requires submissions to be anonymous, they must be fully anonymized as discussed in the Modifications for Blind Review subsection below; in this case, Acknowledgements and Contribution Statement sections are not allowed.

% \item If your track requires non-anonymous submissions, you should provide all author information at the time of submission, just as for camera-ready papers (see below); Acknowledgements and Contribution Statement sections are allowed, but optional.

% \item Submissions must include line numbers to facilitate feedback in the review process . Enable line numbers by uncommenting the command {\tt \textbackslash{}linenumbers} in the preamble.

% \item The limit on the number of  content pages is \emph{strict}. All papers exceeding the limits will be desk rejected.
% \end{itemize}

% \subsubsection{Camera-Ready Papers}
% The following instructions apply to camera-ready papers:

% \begin{itemize}
% \item Authors and affiliations are mandatory. Explicit self-references are allowed. It is strictly forbidden to add authors not declared at submission time.

% \item Acknowledgements and Contribution Statement sections are allowed, but optional.

% \item Line numbering must be disabled. To achieve this, comment or disable {\tt \textbackslash{}linenumbers} in the preamble.

% \item For some of the tracks, you can exceed the page limit by purchasing extra pages.
% \end{itemize}

% \subsection{Title and Author Information}

% Center the title on the entire width of the page in a 14-point bold
% font. The title must be capitalized using Title Case. For non-anonymous papers, author names and affiliations should appear below the title. Center author name(s) in 12-point bold font. On the following line(s) place the affiliations.

% \subsubsection{Author Names}

% Each author name must be followed by:
% \begin{itemize}
%     \item A newline {\tt \textbackslash{}\textbackslash{}} command for the last author.
%     \item An {\tt \textbackslash{}And} command for the second to last author.
%     \item An {\tt \textbackslash{}and} command for the other authors.
% \end{itemize}

% \subsubsection{Affiliations}

% After all authors, start the affiliations section by using the {\tt \textbackslash{}affiliations} command.
% Each affiliation must be terminated by a newline {\tt \textbackslash{}\textbackslash{}} command. Make sure that you include the newline after the last affiliation, too.

% \subsubsection{Mapping Authors to Affiliations}

% If some scenarios, the affiliation of each author is clear without any further indication (\emph{e.g.}, all authors share the same affiliation, all authors have a single and different affiliation). In these situations you don't need to do anything special.

% In more complex scenarios you will have to clearly indicate the affiliation(s) for each author. This is done by using numeric math superscripts {\tt \$\{\^{}$i,j, \ldots$\}\$}. You must use numbers, not symbols, because those are reserved for footnotes in this section (should you need them). Check the authors definition in this example for reference.

% \subsubsection{Emails}

% This section is optional, and can be omitted entirely if you prefer. If you want to include e-mails, you should either include all authors' e-mails or just the contact author(s)' ones.

% Start the e-mails section with the {\tt \textbackslash{}emails} command. After that, write all emails you want to include separated by a comma and a space, following the order used for the authors (\emph{i.e.}, the first e-mail should correspond to the first author, the second e-mail to the second author and so on).

% You may ``contract" consecutive e-mails on the same domain as shown in this example (write the users' part within curly brackets, followed by the domain name). Only e-mails of the exact same domain may be contracted. For instance, you cannot contract ``person@example.com" and ``other@test.example.com" because the domains are different.


% \subsubsection{Modifications for Blind Review}
% When submitting to a track that requires anonymous submissions,
% in order to make blind reviewing possible, authors must omit their
% names, affiliations and e-mails. In place
% of names, affiliations and e-mails, you can optionally provide the submission number and/or
% a list of content areas. When referring to one's own work,
% use the third person rather than the
% first person. For example, say, ``Previously,
% Gottlob~\shortcite{gottlob:nonmon} has shown that\ldots'', rather
% than, ``In our previous work~\cite{gottlob:nonmon}, we have shown
% that\ldots'' Try to avoid including any information in the body of the
% paper or references that would identify the authors or their
% institutions, such as acknowledgements. Such information can be added post-acceptance to be included in the camera-ready
% version.
% Please also make sure that your paper metadata does not reveal
% the authors' identities.

% \subsection{Abstract}

% Place the abstract at the beginning of the first column 3$''$ from the
% top of the page, unless that does not leave enough room for the title
% and author information. Use a slightly smaller width than in the body
% of the paper. Head the abstract with ``Abstract'' centered above the
% body of the abstract in a 12-point bold font. The body of the abstract
% should be in the same font as the body of the paper.

% The abstract should be a concise, one-paragraph summary describing the
% general thesis and conclusion of your paper. A reader should be able
% to learn the purpose of the paper and the reason for its importance
% from the abstract. The abstract should be no more than 200 words long.

% \subsection{Text}

% The main body of the text immediately follows the abstract. Use
% 10-point type in a clear, readable font with 1-point leading (10 on
% 11).

% Indent when starting a new paragraph, except after major headings.

% \subsection{Headings and Sections}

% When necessary, headings should be used to separate major sections of
% your paper. (These instructions use many headings to demonstrate their
% appearance; your paper should have fewer headings.). All headings should be capitalized using Title Case.

% \subsubsection{Section Headings}

% Print section headings in 12-point bold type in the style shown in
% these instructions. Leave a blank space of approximately 10 points
% above and 4 points below section headings.  Number sections with
% Arabic numerals.

% \subsubsection{Subsection Headings}

% Print subsection headings in 11-point bold type. Leave a blank space
% of approximately 8 points above and 3 points below subsection
% headings. Number subsections with the section number and the
% subsection number (in Arabic numerals) separated by a
% period.

% \subsubsection{Subsubsection Headings}

% Print subsubsection headings in 10-point bold type. Leave a blank
% space of approximately 6 points above subsubsection headings. Do not
% number subsubsections.

% \paragraph{Titled paragraphs.} You should use titled paragraphs if and
% only if the title covers exactly one paragraph. Such paragraphs should be
% separated from the preceding content by at least 3pt, and no more than
% 6pt. The title should be in 10pt bold font and to end with a period.
% After that, a 1em horizontal space should follow the title before
% the paragraph's text.

% In \LaTeX{} titled paragraphs should be typeset using
% \begin{quote}
%     {\tt \textbackslash{}paragraph\{Title.\} text} .
% \end{quote}

% \subsection{Special Sections}

% \subsubsection{Appendices}
% You may move some of the contents of the paper into one or more appendices that appear after the main content, but before references. These appendices count towards the page limit and are distinct from the supplementary material that can be submitted separately through CMT. Such appendices are useful if you would like to include highly technical material (such as a lengthy calculation) that will disrupt the flow of the paper. They can be included both in papers submitted for review and in camera-ready versions; in the latter case, they will be included in the proceedings (whereas the supplementary materials will not be included in the proceedings).
% Appendices are optional. Appendices must appear after the main content.
% Appendix sections must use letters instead of Arabic numerals. In \LaTeX,  you can use the {\tt \textbackslash{}appendix} command to achieve this followed by  {\tt \textbackslash section\{Appendix\}} for your appendix sections.

% \subsubsection{Ethical Statement}

% Ethical Statement is optional. You may include an Ethical Statement to discuss  the ethical aspects and implications of your research. The section should be titled \emph{Ethical Statement} and be typeset like any regular section but without being numbered. This section may be placed on the References pages.

% Use
% \begin{quote}
%     {\tt \textbackslash{}section*\{Ethical Statement\}}
% \end{quote}

% \subsubsection{Acknowledgements}

% Acknowledgements are optional. In the camera-ready version you may include an unnumbered acknowledgments section, including acknowledgments of help from colleagues, financial support, and permission to publish. This is not allowed in the anonymous submission. If present, acknowledgements must be in a dedicated, unnumbered section appearing after all regular sections but before references.  This section may be placed on the References pages.

% Use
% \begin{quote}
%     {\tt \textbackslash{}section*\{Acknowledgements\}}
% \end{quote}
% to typeset the acknowledgements section in \LaTeX{}.


% \subsubsection{Contribution Statement}

% Contribution Statement is optional. In the camera-ready version you may include an unnumbered Contribution Statement section, explicitly describing the contribution of each of the co-authors to the paper. This is not allowed in the anonymous submission. If present, Contribution Statement must be in a dedicated, unnumbered section appearing after all regular sections but before references.  This section may be placed on the References pages.

% Use
% \begin{quote}
%     {\tt \textbackslash{}section*\{Contribution Statement\}}
% \end{quote}
% to typeset the Contribution Statement section in \LaTeX{}.

% \subsubsection{References}

% The references section is headed ``References'', printed in the same
% style as a section heading but without a number. A sample list of
% references is given at the end of these instructions. Use a consistent
% format for references. The reference list should not include publicly unavailable work.

% \subsubsection{Order of Sections}
% Sections should be arranged in the following order:
% \begin{enumerate}
%     \item Main content sections (numbered)
%     \item Appendices (optional, numbered using capital letters)
%     \item Ethical statement (optional, unnumbered)
%     \item Acknowledgements (optional, unnumbered)
%     \item Contribution statement (optional, unnumbered)
%     \item References (required, unnumbered)
% \end{enumerate}

% \subsection{Citations}

% Citations within the text should include the author's last name and
% the year of publication, for example~\cite{gottlob:nonmon}.  Append
% lowercase letters to the year in cases of ambiguity.  Treat multiple
% authors as in the following examples:~\cite{abelson-et-al:scheme}
% or~\cite{bgf:Lixto} (for more than two authors) and
% \cite{brachman-schmolze:kl-one} (for two authors).  If the author
% portion of a citation is obvious, omit it, e.g.,
% Nebel~\shortcite{nebel:jair-2000}.  Collapse multiple citations as
% follows:~\cite{gls:hypertrees,levesque:functional-foundations}.
% \nocite{abelson-et-al:scheme}
% \nocite{bgf:Lixto}
% \nocite{brachman-schmolze:kl-one}
% \nocite{gottlob:nonmon}
% \nocite{gls:hypertrees}
% \nocite{levesque:functional-foundations}
% \nocite{levesque:belief}
% \nocite{nebel:jair-2000}

% \subsection{Footnotes}

% Place footnotes at the bottom of the page in a 9-point font.  Refer to
% them with superscript numbers.\footnote{This is how your footnotes
%     should appear.} Separate them from the text by a short
% line.\footnote{Note the line separating these footnotes from the
%     text.} Avoid footnotes as much as possible; they interrupt the flow of
% the text.

% \section{Illustrations}

% Place all illustrations (figures, drawings, tables, and photographs)
% throughout the paper at the places where they are first discussed,
% rather than at the end of the paper.

% They should be floated to the top (preferred) or bottom of the page,
% unless they are an integral part
% of your narrative flow. When placed at the bottom or top of
% a page, illustrations may run across both columns, but not when they
% appear inline.

% Illustrations must be rendered electronically or scanned and placed
% directly in your document. They should be cropped outside \LaTeX{},
% otherwise portions of the image could reappear during the post-processing of your paper.
% When possible, generate your illustrations in a vector format.
% When using bitmaps, please use 300dpi resolution at least.
% All illustrations should be understandable when printed in black and
% white, albeit you can use colors to enhance them. Line weights should
% be 1/2-point or thicker. Avoid screens and superimposing type on
% patterns, as these effects may not reproduce well.

% Number illustrations sequentially. Use references of the following
% form: Figure 1, Table 2, etc. Place illustration numbers and captions
% under illustrations. Leave a margin of 1/4-inch around the area
% covered by the illustration and caption.  Use 9-point type for
% captions, labels, and other text in illustrations. Captions should always appear below the illustration.

% \section{Tables}

% Tables are treated as illustrations containing data. Therefore, they should also appear floated to the top (preferably) or bottom of the page, and with the captions below them.

% \begin{table}
%     \centering
%     \begin{tabular}{lll}
%         \hline
%         Scenario  & $\delta$ & Runtime \\
%         \hline
%         Paris     & 0.1s     & 13.65ms \\
%         Paris     & 0.2s     & 0.01ms  \\
%         New York  & 0.1s     & 92.50ms \\
%         Singapore & 0.1s     & 33.33ms \\
%         Singapore & 0.2s     & 23.01ms \\
%         \hline
%     \end{tabular}
%     \caption{Latex default table}
%     \label{tab:plain}
% \end{table}

% \begin{table}
%     \centering
%     \begin{tabular}{lrr}
%         \toprule
%         Scenario  & $\delta$ (s) & Runtime (ms) \\
%         \midrule
%         Paris     & 0.1          & 13.65        \\
%                   & 0.2          & 0.01         \\
%         New York  & 0.1          & 92.50        \\
%         Singapore & 0.1          & 33.33        \\
%                   & 0.2          & 23.01        \\
%         \bottomrule
%     \end{tabular}
%     \caption{Booktabs table}
%     \label{tab:booktabs}
% \end{table}

% If you are using \LaTeX, you should use the {\tt booktabs} package, because it produces tables that are better than the standard ones. Compare Tables~\ref{tab:plain} and~\ref{tab:booktabs}. The latter is clearly more readable for three reasons:

% \begin{enumerate}
%     \item The styling is better thanks to using the {\tt booktabs} rulers instead of the default ones.
%     \item Numeric columns are right-aligned, making it easier to compare the numbers. Make sure to also right-align the corresponding headers, and to use the same precision for all numbers.
%     \item We avoid unnecessary repetition, both between lines (no need to repeat the scenario name in this case) as well as in the content (units can be shown in the column header).
% \end{enumerate}

% \section{Formulas}

% IJCAI's two-column format makes it difficult to typeset long formulas. A usual temptation is to reduce the size of the formula by using the {\tt small} or {\tt tiny} sizes. This doesn't work correctly with the current \LaTeX{} versions, breaking the line spacing of the preceding paragraphs and title, as well as the equation number sizes. The following equation demonstrates the effects (notice that this entire paragraph looks badly formatted, and the line numbers no longer match the text):
% %
% \begin{tiny}
%     \begin{equation}
%         x = \prod_{i=1}^n \sum_{j=1}^n j_i + \prod_{i=1}^n \sum_{j=1}^n i_j + \prod_{i=1}^n \sum_{j=1}^n j_i + \prod_{i=1}^n \sum_{j=1}^n i_j + \prod_{i=1}^n \sum_{j=1}^n j_i
%     \end{equation}
% \end{tiny}%

% Reducing formula sizes this way is strictly forbidden. We {\bf strongly} recommend authors to split formulas in multiple lines when they don't fit in a single line. This is the easiest approach to typeset those formulas and provides the most readable output%
% %
% \begin{align}
%     x = & \prod_{i=1}^n \sum_{j=1}^n j_i + \prod_{i=1}^n \sum_{j=1}^n i_j + \prod_{i=1}^n \sum_{j=1}^n j_i + \prod_{i=1}^n \sum_{j=1}^n i_j + \nonumber \\
%     +   & \prod_{i=1}^n \sum_{j=1}^n j_i.
% \end{align}%

% If a line is just slightly longer than the column width, you may use the {\tt resizebox} environment on that equation. The result looks better and doesn't interfere with the paragraph's line spacing: %
% \begin{equation}
%     \resizebox{.91\linewidth}{!}{$
%             \displaystyle
%             x = \prod_{i=1}^n \sum_{j=1}^n j_i + \prod_{i=1}^n \sum_{j=1}^n i_j + \prod_{i=1}^n \sum_{j=1}^n j_i + \prod_{i=1}^n \sum_{j=1}^n i_j + \prod_{i=1}^n \sum_{j=1}^n j_i
%         $}.
% \end{equation}%

% This last solution may have to be adapted if you use different equation environments, but it can generally be made to work. Please notice that in any case:

% \begin{itemize}
%     \item Equation numbers must be in the same font and size as the main text (10pt).
%     \item Your formula's main symbols should not be smaller than {\small small} text (9pt).
% \end{itemize}

% For instance, the formula
% %
% \begin{equation}
%     \resizebox{.91\linewidth}{!}{$
%             \displaystyle
%             x = \prod_{i=1}^n \sum_{j=1}^n j_i + \prod_{i=1}^n \sum_{j=1}^n i_j + \prod_{i=1}^n \sum_{j=1}^n j_i + \prod_{i=1}^n \sum_{j=1}^n i_j + \prod_{i=1}^n \sum_{j=1}^n j_i + \prod_{i=1}^n \sum_{j=1}^n i_j
%         $}
% \end{equation}
% %
% would not be acceptable because the text is too small.

% \section{Examples, Definitions, Theorems and Similar}

% Examples, definitions, theorems, corollaries and similar must be written in their own paragraph. The paragraph must be separated by at least 2pt and no more than 5pt from the preceding and succeeding paragraphs. They must begin with the kind of item written in 10pt bold font followed by their number (e.g.: {\bf Theorem 1}),
% optionally followed by a title/summary between parentheses in non-bold font and ended with a period (in bold).
% After that the main body of the item follows, written in 10 pt italics font (see below for examples).

% In \LaTeX{} we strongly recommend that you define environments for your examples, definitions, propositions, lemmas, corollaries and similar. This can be done in your \LaTeX{} preamble using \texttt{\textbackslash{newtheorem}} -- see the source of this document for examples. Numbering for these items must be global, not per-section (e.g.: Theorem 1 instead of Theorem 6.1).

% \begin{example}[How to write an example]
%     Examples should be written using the example environment defined in this template.
% \end{example}

% \begin{theorem}
%     This is an example of an untitled theorem.
% \end{theorem}

% You may also include a title or description using these environments as shown in the following theorem.

% \begin{theorem}[A titled theorem]
%     This is an example of a titled theorem.
% \end{theorem}

% \section{Proofs}

% Proofs must be written in their own paragraph(s) separated by at least 2pt and no more than 5pt from the preceding and succeeding paragraphs. Proof paragraphs should start with the keyword ``Proof." in 10pt italics font. After that the proof follows in regular 10pt font. At the end of the proof, an unfilled square symbol (qed) marks the end of the proof.

% In \LaTeX{} proofs should be typeset using the \texttt{\textbackslash{proof}} environment.

% \begin{proof}
%     This paragraph is an example of how a proof looks like using the \texttt{\textbackslash{proof}} environment.
% \end{proof}


% \section{Algorithms and Listings}

% Algorithms and listings are a special kind of figures. Like all illustrations, they should appear floated to the top (preferably) or bottom of the page. However, their caption should appear in the header, left-justified and enclosed between horizontal lines, as shown in Algorithm~\ref{alg:algorithm}. The algorithm body should be terminated with another horizontal line. It is up to the authors to decide whether to show line numbers or not, how to format comments, etc.

% In \LaTeX{} algorithms may be typeset using the {\tt algorithm} and {\tt algorithmic} packages, but you can also use one of the many other packages for the task.

% \begin{algorithm}[tb]
%     \caption{Example algorithm}
%     \label{alg:algorithm}
%     \textbf{Input}: Your algorithm's input\\
%     \textbf{Parameter}: Optional list of parameters\\
%     \textbf{Output}: Your algorithm's output
%     \begin{algorithmic}[1] %[1] enables line numbers
%         \STATE Let $t=0$.
%         \WHILE{condition}
%         \STATE Do some action.
%         \IF {conditional}
%         \STATE Perform task A.
%         \ELSE
%         \STATE Perform task B.
%         \ENDIF
%         \ENDWHILE
%         \STATE \textbf{return} solution
%     \end{algorithmic}
% \end{algorithm}

% \section{\LaTeX{} and Word Style Files}\label{stylefiles}

% The \LaTeX{} and Word style files are available on the IJCAI--25
% website, \url{https://2025.ijcai.org/}.
% These style files implement the formatting instructions in this
% document.

% The \LaTeX{} files are {\tt ijcai25.sty} and {\tt ijcai25.tex}, and
% the Bib\TeX{} files are {\tt named.bst} and {\tt ijcai25.bib}. The
% \LaTeX{} style file is for version 2e of \LaTeX{}, and the Bib\TeX{}
% style file is for version 0.99c of Bib\TeX{} ({\em not} version
% 0.98i). .

% The Microsoft Word style file consists of a single file, {\tt
%         ijcai25.docx}. 
% %This template differs from the one used for IJCAI--23.

% These Microsoft Word and \LaTeX{} files contain the source of the
% present document and may serve as a formatting sample.

% Further information on using these styles for the preparation of
% papers for IJCAI--25 can be obtained by contacting {\tt
%         proceedings@ijcai.org}.

% \appendix

% \section*{Ethical Statement}

% There are no ethical issues.

% \section*{Acknowledgments}

% The preparation of these instructions and the \LaTeX{} and Bib\TeX{}
% files that implement them was supported by Schlumberger Palo Alto
% Research, AT\&T Bell Laboratories, and Morgan Kaufmann Publishers.
% Preparation of the Microsoft Word file was supported by IJCAI.  An
% early version of this document was created by Shirley Jowell and Peter
% F. Patel-Schneider.  It was subsequently modified by Jennifer
% Ballentine, Thomas Dean, Bernhard Nebel, Daniel Pagenstecher,
% Kurt Steinkraus, Toby Walsh, Carles Sierra, Marc Pujol-Gonzalez,
% Francisco Cruz-Mencia and Edith Elkind.


%% The file named.bst is a bibliography style file for BibTeX 0.99c
\bibliographystyle{named}
\bibliography{ijcai25}

\end{document}

