\section{Related Works}
\label{sec:related_works}
\paragraph{Next Day Wildfire Segmentation} To apply DNNs to fire spread prediction, recent research has framed the spread problem as a semantic segmentation one, and developed multiple datasets to support this framing. ____ created Next-Day Wildfire Spread, a dataset for mono-temporal fire spread prediction. They developed a custom convolutional autoencoder that takes as input various explanatory variables and outputs a binary mask indicating fire presence at each pixel. Similarly, ____ extended the dataset to include multi-temporal prediction, added more explanatory variables, and higher resolution fire masks. They achieved the best performance using a standard Unet model and a Unet model with temporal attention (UTAE) ____. Other datasets aim to predict beyond next-day and forecast fire behavior several days in advance, e.g., ____ developed SeasFire Cube and trained Unet++ models ____ for medium-term fire prediction, between 8 and 64 days. ____ improved upon the collected data cube and found that the LSTM and ConvLSTM models outperformed the Fire Weather Index (FWI). In FireSight, ____ collected a dataset using remote sensing data from 20 datasets, and trained a 3D UNet model to model short-term fire hazard, between 3 and 8 days.  

\paragraph{Other Approaches}  Aside from segmentation, other formulations have been developed for modeling fire spread using deep learning (DL). ____ used a CNN-based Reinforcement Learning model that predicts the best action as burn or no burn given current conditions. ____ developed a probabilistic cellular automata model to simulate wildfire spread. In ____, the authors developed Sim2Real-Fire, a synthetic, high-resolution dataset for fire spread forecast and backtracking and outperformed the considered baselines using a custom Transformer model. We refer the readers to ____ for a more comprehensive review of DL for wildfire prediction. 

\paragraph{Next Day Wildfire Prediction with Time-Series} In contrast to most existing work (e.g., ____), we focus on utilizing a time-series of features for next-day wildfire spread prediction, which has been cited as an important emerging area of research ____.  Historically, time-series modeling has been challenging due to the lack of appropriate public datasets to train and evaluate models for this task. Recently, ____, building upon the work of ____, developed the first multi-temporal dataset for time-series prediction.  Notably, they found that models using a time-series of input tend to outperform those using a single day, reinforcing the importance of this research direction. %We extend the existing time-series prediction research by investigating attention-based models: specifically the recent SwinUnet model ____, which has been found extremely effective in a variety of contexts for segmentation tasks.