
\section{Introduction}
%Introduction - describe roughly what we are doing, and why it is important
Wildfires are a global cause of concern that have severe human, economical, and environmental impacts, with the average annual economic burden from wildfires falling between \$71.1 billion and \$347.8 billion \cite{thomas2017costs}. Due to climate change, many regions are getting hotter and drier, which lengthens wildfire seasons and exacerbates their effect.  In order to better manage, mitigate, and prevent wildfires, accurately predicting their spread is essential. In this work, we focus on the problem of next-day wildfire spread prediction, where we are provided with current and/or historical information about a particular wildfire, and then tasked with predicting its spatial extent on the following day.  

\begin{figure}
    \centering
    \includegraphics[width=0.9\linewidth]{problem_illustration_v3.PNG}
    \caption{The wildfire prediction models take as input a geospatial raster of several variables: vegetation, topography, and weather features, alongside the current day fire mask. We consider two prediction scenarios: one in which the model receives input features from only one preceding day, denoted $t-1$, and another in which we provide the model five previous days of features as input. In either case, the model must predict a binary mask indicating the extent of the fire on day $t$}
    \label{fig:problem_setting}
\end{figure}

A variety of approaches have been investigated to solve this problem, such as those based upon machine learning models \cite{khanmohammadi2022prediction, bot2022systematic, chetehouna2015predicting}, or physics-based and observationally-informed models \cite{Finney_1998, alexander2006evaluating, finney2006overview}.  In this work, however, we focus on a promising emerging class of techniques that utilize high-capacity machine learning models -- namely deep neural networks (DNNs) -- to predict wildfire spread using high-dimensional explanatory input data.  These input data typically comprise a geospatial map of the current extent of the fire, as well as maps of explanatory features such as topography, climate, weather, and vegetation indices. Based upon these input data, the model is tasked with producing a geospatial map, or an image, reflecting the spatial extent of the fire on the following day, i.e. the spread of fire into regions that have not burned, and the extinction of fire in regions that have ceased burning. See \cref{fig:problem_setting} for an illustration.  

DNN-based models have achieved impressive prediction accuracy, and have garnered substantial attention in recent years for wildfire modeling, including specifically next-day wildfire prediction.  A variety of DNN-based models have been proposed to solve next-day prediction, including convolutional models \cite{liu2024fire, burge2023recurrent, marjani2024cnn}, attention-based models such as transformers \cite{shah2023wildfire}, and spatio-temporal models \cite{michail2024seasonal, bolt2022spatio}.  Most existing research has focused on next-day prediction where only explanatory data from the current day is provided (day $t$ only, in \cref{fig:problem_setting}).  However, recent research found that models utilizing a time-series of $T$ days of historical data can achieve greater prediction accuracy \cite{gerard2023wildfirespreadts}, suggesting this as an important new direction in next-day wildfire prediction.  

\paragraph{Contributions of this Work} In this work, we introduce a variety of modeling improvements to both the single-day ($T=1$) and time-series ($T>1$) input scenarios. Our results show that these improvements lead to substantial increases in accuracy over the existing state-of-the-art, as demonstrated on the WildfireSpreadTS (WSTS) benchmark \cite{gerard2023wildfirespreadts}, an extension of the well-studied Next Day Wildfire Spread benchmark \cite{huot2022next} to support time-series prediction. We chose WSTS because it is the only publicly available benchmark for time-series wildfire prediction, and employs a rigorous and realistic 12-fold leave-one-year-out cross-validation.  Our best model achieves a new SOTA on the WSTS benchmark by a large margin. Lastly, we introduce WSTS+, an extended benchmark for next-day wildfire spread, constructed by doubling the number of years of historical wildfire events in WSTS. The WSTS+ benchmark reveals an important emerging challenge in wildfire modeling: data heterogeneity.  We summarize our contributions as follows: 
 
\begin{itemize}
    \item \textit{SOTA Single-Day ($T=1$) Prediction.} We introduce a variety of improvements to modeling single-day wildfire prediction, leading to $37\%$ improvement in performance over existing models. 
    \item \textit{SOTA Time-Series ($T>1$) Prediction.}  We introduce a variety of modeling improvements, resulting in a $28 \%$ performance improvement in accuracy over existing models. We also find that our time-series models outperform single-day models, resulting in a new overall SOTA on wildfire prediction. 
    \item \textit{WSTS+: An expanded Public Benchmark for Next-Day Wildfire Modeling.} Our proposed benchmark includes twice the years of historical data as its predecessor, WSTS, resulting, to our knowledge, in the largest public benchmark for \textit{time-series} next-day wildfire spread prediction.  It also reveals an emerging challenge for the wildfire modeling community: data heterogeneity.   
\end{itemize}


The rest of the paper is structured as follows: we formulate our problem setting in \cref{sec:problem_setting}, \cref{sec:related_works} reviews related works,  \cref{sec:wsts} describes our adopted benchmark, we present some preliminaries in \cref{sec:preliminaries}, \cref{sec:experiments} details our experimental setting and results, \cref{sec:wsts+} introduces our expanded benchmark, and \cref{sec:conclusion} summarizes our findings.