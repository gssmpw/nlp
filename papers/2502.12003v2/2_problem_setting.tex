
\section{Problem Setting}
\label{sec:problem_setting}
In its general formulation, the goal of next-day wildfire spread prediction is to predict a wildfire's spatial extent on some $t^{th}$ day, denoted $y(t)$, given  explanatory data from one or more \textit{preceding} days, denoted $x(t)$.  We adopt the more specific settings of recent literature \cite{gerard2023wildfirespreadts,huot2022next}, as illustrated in \cref{fig:problem_setting}, which assume there are $T$ consecutive previous days of explanatory data, so that $x(t)=\{ \tilde{x}(t-i) \}_{i=1}^{T}$, and each $\tilde{x}(t)$ comprises a geospatial raster, so that $\tilde{x}(t) \in \mathbb{R}^{H \times W \times C}$, where $H,W$ correspond to spatial dimensions, and $C$ represents the number of explanatory variables, which may include previous fire masks (e.g., $y(t) \subset \tilde{x}(t)$).  The fire extent is encoded in a binary geospatial image, $y(t) \in  \{0,1\}^{H \times W}$, where a value of one indicates the presence of a fire. Our goal is then to use a dataset of historical wildfire data to infer parameters, $\theta$ of a predictive model of the form $y(t) = f_{\theta}(x(t))$.    