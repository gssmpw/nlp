\documentclass[]{fairmeta}

\usepackage{graphicx}
\usepackage{textcomp}
\usepackage{xcolor}
    
\usepackage{url}
\usepackage{algorithm}
\usepackage{algpseudocode}
\usepackage{subcaption}
\newcommand{\newmaterial}[1]{{#1}}

%%%%% NEW MATH DEFINITIONS %%%%%

\usepackage{amsmath,amsfonts,bm}
\usepackage{derivative}
% Mark sections of captions for referring to divisions of figures
\newcommand{\figleft}{{\em (Left)}}
\newcommand{\figcenter}{{\em (Center)}}
\newcommand{\figright}{{\em (Right)}}
\newcommand{\figtop}{{\em (Top)}}
\newcommand{\figbottom}{{\em (Bottom)}}
\newcommand{\captiona}{{\em (a)}}
\newcommand{\captionb}{{\em (b)}}
\newcommand{\captionc}{{\em (c)}}
\newcommand{\captiond}{{\em (d)}}

% Highlight a newly defined term
\newcommand{\newterm}[1]{{\bf #1}}

% Derivative d 
\newcommand{\deriv}{{\mathrm{d}}}

% Figure reference, lower-case.
\def\figref#1{figure~\ref{#1}}
% Figure reference, capital. For start of sentence
\def\Figref#1{Figure~\ref{#1}}
\def\twofigref#1#2{figures \ref{#1} and \ref{#2}}
\def\quadfigref#1#2#3#4{figures \ref{#1}, \ref{#2}, \ref{#3} and \ref{#4}}
% Section reference, lower-case.
\def\secref#1{section~\ref{#1}}
% Section reference, capital.
\def\Secref#1{Section~\ref{#1}}
% Reference to two sections.
\def\twosecrefs#1#2{sections \ref{#1} and \ref{#2}}
% Reference to three sections.
\def\secrefs#1#2#3{sections \ref{#1}, \ref{#2} and \ref{#3}}
% Reference to an equation, lower-case.
\def\eqref#1{equation~\ref{#1}}
% Reference to an equation, upper case
\def\Eqref#1{Equation~\ref{#1}}
% A raw reference to an equation---avoid using if possible
\def\plaineqref#1{\ref{#1}}
% Reference to a chapter, lower-case.
\def\chapref#1{chapter~\ref{#1}}
% Reference to an equation, upper case.
\def\Chapref#1{Chapter~\ref{#1}}
% Reference to a range of chapters
\def\rangechapref#1#2{chapters\ref{#1}--\ref{#2}}
% Reference to an algorithm, lower-case.
\def\algref#1{algorithm~\ref{#1}}
% Reference to an algorithm, upper case.
\def\Algref#1{Algorithm~\ref{#1}}
\def\twoalgref#1#2{algorithms \ref{#1} and \ref{#2}}
\def\Twoalgref#1#2{Algorithms \ref{#1} and \ref{#2}}
% Reference to a part, lower case
\def\partref#1{part~\ref{#1}}
% Reference to a part, upper case
\def\Partref#1{Part~\ref{#1}}
\def\twopartref#1#2{parts \ref{#1} and \ref{#2}}

\def\ceil#1{\lceil #1 \rceil}
\def\floor#1{\lfloor #1 \rfloor}
\def\1{\bm{1}}
\newcommand{\train}{\mathcal{D}}
\newcommand{\valid}{\mathcal{D_{\mathrm{valid}}}}
\newcommand{\test}{\mathcal{D_{\mathrm{test}}}}

\def\eps{{\epsilon}}


% Random variables
\def\reta{{\textnormal{$\eta$}}}
\def\ra{{\textnormal{a}}}
\def\rb{{\textnormal{b}}}
\def\rc{{\textnormal{c}}}
\def\rd{{\textnormal{d}}}
\def\re{{\textnormal{e}}}
\def\rf{{\textnormal{f}}}
\def\rg{{\textnormal{g}}}
\def\rh{{\textnormal{h}}}
\def\ri{{\textnormal{i}}}
\def\rj{{\textnormal{j}}}
\def\rk{{\textnormal{k}}}
\def\rl{{\textnormal{l}}}
% rm is already a command, just don't name any random variables m
\def\rn{{\textnormal{n}}}
\def\ro{{\textnormal{o}}}
\def\rp{{\textnormal{p}}}
\def\rq{{\textnormal{q}}}
\def\rr{{\textnormal{r}}}
\def\rs{{\textnormal{s}}}
\def\rt{{\textnormal{t}}}
\def\ru{{\textnormal{u}}}
\def\rv{{\textnormal{v}}}
\def\rw{{\textnormal{w}}}
\def\rx{{\textnormal{x}}}
\def\ry{{\textnormal{y}}}
\def\rz{{\textnormal{z}}}

% Random vectors
\def\rvepsilon{{\mathbf{\epsilon}}}
\def\rvphi{{\mathbf{\phi}}}
\def\rvtheta{{\mathbf{\theta}}}
\def\rva{{\mathbf{a}}}
\def\rvb{{\mathbf{b}}}
\def\rvc{{\mathbf{c}}}
\def\rvd{{\mathbf{d}}}
\def\rve{{\mathbf{e}}}
\def\rvf{{\mathbf{f}}}
\def\rvg{{\mathbf{g}}}
\def\rvh{{\mathbf{h}}}
\def\rvu{{\mathbf{i}}}
\def\rvj{{\mathbf{j}}}
\def\rvk{{\mathbf{k}}}
\def\rvl{{\mathbf{l}}}
\def\rvm{{\mathbf{m}}}
\def\rvn{{\mathbf{n}}}
\def\rvo{{\mathbf{o}}}
\def\rvp{{\mathbf{p}}}
\def\rvq{{\mathbf{q}}}
\def\rvr{{\mathbf{r}}}
\def\rvs{{\mathbf{s}}}
\def\rvt{{\mathbf{t}}}
\def\rvu{{\mathbf{u}}}
\def\rvv{{\mathbf{v}}}
\def\rvw{{\mathbf{w}}}
\def\rvx{{\mathbf{x}}}
\def\rvy{{\mathbf{y}}}
\def\rvz{{\mathbf{z}}}

% Elements of random vectors
\def\erva{{\textnormal{a}}}
\def\ervb{{\textnormal{b}}}
\def\ervc{{\textnormal{c}}}
\def\ervd{{\textnormal{d}}}
\def\erve{{\textnormal{e}}}
\def\ervf{{\textnormal{f}}}
\def\ervg{{\textnormal{g}}}
\def\ervh{{\textnormal{h}}}
\def\ervi{{\textnormal{i}}}
\def\ervj{{\textnormal{j}}}
\def\ervk{{\textnormal{k}}}
\def\ervl{{\textnormal{l}}}
\def\ervm{{\textnormal{m}}}
\def\ervn{{\textnormal{n}}}
\def\ervo{{\textnormal{o}}}
\def\ervp{{\textnormal{p}}}
\def\ervq{{\textnormal{q}}}
\def\ervr{{\textnormal{r}}}
\def\ervs{{\textnormal{s}}}
\def\ervt{{\textnormal{t}}}
\def\ervu{{\textnormal{u}}}
\def\ervv{{\textnormal{v}}}
\def\ervw{{\textnormal{w}}}
\def\ervx{{\textnormal{x}}}
\def\ervy{{\textnormal{y}}}
\def\ervz{{\textnormal{z}}}

% Random matrices
\def\rmA{{\mathbf{A}}}
\def\rmB{{\mathbf{B}}}
\def\rmC{{\mathbf{C}}}
\def\rmD{{\mathbf{D}}}
\def\rmE{{\mathbf{E}}}
\def\rmF{{\mathbf{F}}}
\def\rmG{{\mathbf{G}}}
\def\rmH{{\mathbf{H}}}
\def\rmI{{\mathbf{I}}}
\def\rmJ{{\mathbf{J}}}
\def\rmK{{\mathbf{K}}}
\def\rmL{{\mathbf{L}}}
\def\rmM{{\mathbf{M}}}
\def\rmN{{\mathbf{N}}}
\def\rmO{{\mathbf{O}}}
\def\rmP{{\mathbf{P}}}
\def\rmQ{{\mathbf{Q}}}
\def\rmR{{\mathbf{R}}}
\def\rmS{{\mathbf{S}}}
\def\rmT{{\mathbf{T}}}
\def\rmU{{\mathbf{U}}}
\def\rmV{{\mathbf{V}}}
\def\rmW{{\mathbf{W}}}
\def\rmX{{\mathbf{X}}}
\def\rmY{{\mathbf{Y}}}
\def\rmZ{{\mathbf{Z}}}

% Elements of random matrices
\def\ermA{{\textnormal{A}}}
\def\ermB{{\textnormal{B}}}
\def\ermC{{\textnormal{C}}}
\def\ermD{{\textnormal{D}}}
\def\ermE{{\textnormal{E}}}
\def\ermF{{\textnormal{F}}}
\def\ermG{{\textnormal{G}}}
\def\ermH{{\textnormal{H}}}
\def\ermI{{\textnormal{I}}}
\def\ermJ{{\textnormal{J}}}
\def\ermK{{\textnormal{K}}}
\def\ermL{{\textnormal{L}}}
\def\ermM{{\textnormal{M}}}
\def\ermN{{\textnormal{N}}}
\def\ermO{{\textnormal{O}}}
\def\ermP{{\textnormal{P}}}
\def\ermQ{{\textnormal{Q}}}
\def\ermR{{\textnormal{R}}}
\def\ermS{{\textnormal{S}}}
\def\ermT{{\textnormal{T}}}
\def\ermU{{\textnormal{U}}}
\def\ermV{{\textnormal{V}}}
\def\ermW{{\textnormal{W}}}
\def\ermX{{\textnormal{X}}}
\def\ermY{{\textnormal{Y}}}
\def\ermZ{{\textnormal{Z}}}

% Vectors
\def\vzero{{\bm{0}}}
\def\vone{{\bm{1}}}
\def\vmu{{\bm{\mu}}}
\def\vtheta{{\bm{\theta}}}
\def\vphi{{\bm{\phi}}}
\def\va{{\bm{a}}}
\def\vb{{\bm{b}}}
\def\vc{{\bm{c}}}
\def\vd{{\bm{d}}}
\def\ve{{\bm{e}}}
\def\vf{{\bm{f}}}
\def\vg{{\bm{g}}}
\def\vh{{\bm{h}}}
\def\vi{{\bm{i}}}
\def\vj{{\bm{j}}}
\def\vk{{\bm{k}}}
\def\vl{{\bm{l}}}
\def\vm{{\bm{m}}}
\def\vn{{\bm{n}}}
\def\vo{{\bm{o}}}
\def\vp{{\bm{p}}}
\def\vq{{\bm{q}}}
\def\vr{{\bm{r}}}
\def\vs{{\bm{s}}}
\def\vt{{\bm{t}}}
\def\vu{{\bm{u}}}
\def\vv{{\bm{v}}}
\def\vw{{\bm{w}}}
\def\vx{{\bm{x}}}
\def\vy{{\bm{y}}}
\def\vz{{\bm{z}}}

% Elements of vectors
\def\evalpha{{\alpha}}
\def\evbeta{{\beta}}
\def\evepsilon{{\epsilon}}
\def\evlambda{{\lambda}}
\def\evomega{{\omega}}
\def\evmu{{\mu}}
\def\evpsi{{\psi}}
\def\evsigma{{\sigma}}
\def\evtheta{{\theta}}
\def\eva{{a}}
\def\evb{{b}}
\def\evc{{c}}
\def\evd{{d}}
\def\eve{{e}}
\def\evf{{f}}
\def\evg{{g}}
\def\evh{{h}}
\def\evi{{i}}
\def\evj{{j}}
\def\evk{{k}}
\def\evl{{l}}
\def\evm{{m}}
\def\evn{{n}}
\def\evo{{o}}
\def\evp{{p}}
\def\evq{{q}}
\def\evr{{r}}
\def\evs{{s}}
\def\evt{{t}}
\def\evu{{u}}
\def\evv{{v}}
\def\evw{{w}}
\def\evx{{x}}
\def\evy{{y}}
\def\evz{{z}}

% Matrix
\def\mA{{\bm{A}}}
\def\mB{{\bm{B}}}
\def\mC{{\bm{C}}}
\def\mD{{\bm{D}}}
\def\mE{{\bm{E}}}
\def\mF{{\bm{F}}}
\def\mG{{\bm{G}}}
\def\mH{{\bm{H}}}
\def\mI{{\bm{I}}}
\def\mJ{{\bm{J}}}
\def\mK{{\bm{K}}}
\def\mL{{\bm{L}}}
\def\mM{{\bm{M}}}
\def\mN{{\bm{N}}}
\def\mO{{\bm{O}}}
\def\mP{{\bm{P}}}
\def\mQ{{\bm{Q}}}
\def\mR{{\bm{R}}}
\def\mS{{\bm{S}}}
\def\mT{{\bm{T}}}
\def\mU{{\bm{U}}}
\def\mV{{\bm{V}}}
\def\mW{{\bm{W}}}
\def\mX{{\bm{X}}}
\def\mY{{\bm{Y}}}
\def\mZ{{\bm{Z}}}
\def\mBeta{{\bm{\beta}}}
\def\mPhi{{\bm{\Phi}}}
\def\mLambda{{\bm{\Lambda}}}
\def\mSigma{{\bm{\Sigma}}}

% Tensor
\DeclareMathAlphabet{\mathsfit}{\encodingdefault}{\sfdefault}{m}{sl}
\SetMathAlphabet{\mathsfit}{bold}{\encodingdefault}{\sfdefault}{bx}{n}
\newcommand{\tens}[1]{\bm{\mathsfit{#1}}}
\def\tA{{\tens{A}}}
\def\tB{{\tens{B}}}
\def\tC{{\tens{C}}}
\def\tD{{\tens{D}}}
\def\tE{{\tens{E}}}
\def\tF{{\tens{F}}}
\def\tG{{\tens{G}}}
\def\tH{{\tens{H}}}
\def\tI{{\tens{I}}}
\def\tJ{{\tens{J}}}
\def\tK{{\tens{K}}}
\def\tL{{\tens{L}}}
\def\tM{{\tens{M}}}
\def\tN{{\tens{N}}}
\def\tO{{\tens{O}}}
\def\tP{{\tens{P}}}
\def\tQ{{\tens{Q}}}
\def\tR{{\tens{R}}}
\def\tS{{\tens{S}}}
\def\tT{{\tens{T}}}
\def\tU{{\tens{U}}}
\def\tV{{\tens{V}}}
\def\tW{{\tens{W}}}
\def\tX{{\tens{X}}}
\def\tY{{\tens{Y}}}
\def\tZ{{\tens{Z}}}


% Graph
\def\gA{{\mathcal{A}}}
\def\gB{{\mathcal{B}}}
\def\gC{{\mathcal{C}}}
\def\gD{{\mathcal{D}}}
\def\gE{{\mathcal{E}}}
\def\gF{{\mathcal{F}}}
\def\gG{{\mathcal{G}}}
\def\gH{{\mathcal{H}}}
\def\gI{{\mathcal{I}}}
\def\gJ{{\mathcal{J}}}
\def\gK{{\mathcal{K}}}
\def\gL{{\mathcal{L}}}
\def\gM{{\mathcal{M}}}
\def\gN{{\mathcal{N}}}
\def\gO{{\mathcal{O}}}
\def\gP{{\mathcal{P}}}
\def\gQ{{\mathcal{Q}}}
\def\gR{{\mathcal{R}}}
\def\gS{{\mathcal{S}}}
\def\gT{{\mathcal{T}}}
\def\gU{{\mathcal{U}}}
\def\gV{{\mathcal{V}}}
\def\gW{{\mathcal{W}}}
\def\gX{{\mathcal{X}}}
\def\gY{{\mathcal{Y}}}
\def\gZ{{\mathcal{Z}}}

% Sets
\def\sA{{\mathbb{A}}}
\def\sB{{\mathbb{B}}}
\def\sC{{\mathbb{C}}}
\def\sD{{\mathbb{D}}}
% Don't use a set called E, because this would be the same as our symbol
% for expectation.
\def\sF{{\mathbb{F}}}
\def\sG{{\mathbb{G}}}
\def\sH{{\mathbb{H}}}
\def\sI{{\mathbb{I}}}
\def\sJ{{\mathbb{J}}}
\def\sK{{\mathbb{K}}}
\def\sL{{\mathbb{L}}}
\def\sM{{\mathbb{M}}}
\def\sN{{\mathbb{N}}}
\def\sO{{\mathbb{O}}}
\def\sP{{\mathbb{P}}}
\def\sQ{{\mathbb{Q}}}
\def\sR{{\mathbb{R}}}
\def\sS{{\mathbb{S}}}
\def\sT{{\mathbb{T}}}
\def\sU{{\mathbb{U}}}
\def\sV{{\mathbb{V}}}
\def\sW{{\mathbb{W}}}
\def\sX{{\mathbb{X}}}
\def\sY{{\mathbb{Y}}}
\def\sZ{{\mathbb{Z}}}

% Entries of a matrix
\def\emLambda{{\Lambda}}
\def\emA{{A}}
\def\emB{{B}}
\def\emC{{C}}
\def\emD{{D}}
\def\emE{{E}}
\def\emF{{F}}
\def\emG{{G}}
\def\emH{{H}}
\def\emI{{I}}
\def\emJ{{J}}
\def\emK{{K}}
\def\emL{{L}}
\def\emM{{M}}
\def\emN{{N}}
\def\emO{{O}}
\def\emP{{P}}
\def\emQ{{Q}}
\def\emR{{R}}
\def\emS{{S}}
\def\emT{{T}}
\def\emU{{U}}
\def\emV{{V}}
\def\emW{{W}}
\def\emX{{X}}
\def\emY{{Y}}
\def\emZ{{Z}}
\def\emSigma{{\Sigma}}

% entries of a tensor
% Same font as tensor, without \bm wrapper
\newcommand{\etens}[1]{\mathsfit{#1}}
\def\etLambda{{\etens{\Lambda}}}
\def\etA{{\etens{A}}}
\def\etB{{\etens{B}}}
\def\etC{{\etens{C}}}
\def\etD{{\etens{D}}}
\def\etE{{\etens{E}}}
\def\etF{{\etens{F}}}
\def\etG{{\etens{G}}}
\def\etH{{\etens{H}}}
\def\etI{{\etens{I}}}
\def\etJ{{\etens{J}}}
\def\etK{{\etens{K}}}
\def\etL{{\etens{L}}}
\def\etM{{\etens{M}}}
\def\etN{{\etens{N}}}
\def\etO{{\etens{O}}}
\def\etP{{\etens{P}}}
\def\etQ{{\etens{Q}}}
\def\etR{{\etens{R}}}
\def\etS{{\etens{S}}}
\def\etT{{\etens{T}}}
\def\etU{{\etens{U}}}
\def\etV{{\etens{V}}}
\def\etW{{\etens{W}}}
\def\etX{{\etens{X}}}
\def\etY{{\etens{Y}}}
\def\etZ{{\etens{Z}}}

% The true underlying data generating distribution
\newcommand{\pdata}{p_{\rm{data}}}
\newcommand{\ptarget}{p_{\rm{target}}}
\newcommand{\pprior}{p_{\rm{prior}}}
\newcommand{\pbase}{p_{\rm{base}}}
\newcommand{\pref}{p_{\rm{ref}}}

% The empirical distribution defined by the training set
\newcommand{\ptrain}{\hat{p}_{\rm{data}}}
\newcommand{\Ptrain}{\hat{P}_{\rm{data}}}
% The model distribution
\newcommand{\pmodel}{p_{\rm{model}}}
\newcommand{\Pmodel}{P_{\rm{model}}}
\newcommand{\ptildemodel}{\tilde{p}_{\rm{model}}}
% Stochastic autoencoder distributions
\newcommand{\pencode}{p_{\rm{encoder}}}
\newcommand{\pdecode}{p_{\rm{decoder}}}
\newcommand{\precons}{p_{\rm{reconstruct}}}

\newcommand{\laplace}{\mathrm{Laplace}} % Laplace distribution

\newcommand{\E}{\mathbb{E}}
\newcommand{\Ls}{\mathcal{L}}
\newcommand{\R}{\mathbb{R}}
\newcommand{\emp}{\tilde{p}}
\newcommand{\lr}{\alpha}
\newcommand{\reg}{\lambda}
\newcommand{\rect}{\mathrm{rectifier}}
\newcommand{\softmax}{\mathrm{softmax}}
\newcommand{\sigmoid}{\sigma}
\newcommand{\softplus}{\zeta}
\newcommand{\KL}{D_{\mathrm{KL}}}
\newcommand{\Var}{\mathrm{Var}}
\newcommand{\standarderror}{\mathrm{SE}}
\newcommand{\Cov}{\mathrm{Cov}}
% Wolfram Mathworld says $L^2$ is for function spaces and $\ell^2$ is for vectors
% But then they seem to use $L^2$ for vectors throughout the site, and so does
% wikipedia.
\newcommand{\normlzero}{L^0}
\newcommand{\normlone}{L^1}
\newcommand{\normltwo}{L^2}
\newcommand{\normlp}{L^p}
\newcommand{\normmax}{L^\infty}

\newcommand{\parents}{Pa} % See usage in notation.tex. Chosen to match Daphne's book.

\DeclareMathOperator*{\argmax}{arg\,max}
\DeclareMathOperator*{\argmin}{arg\,min}

\DeclareMathOperator{\sign}{sign}
\DeclareMathOperator{\Tr}{Tr}
\let\ab\allowbreak


\title{Beyond Self-Consistency: Loss-Balanced Perturbation-Based Regularization Improves Industrial-Scale Ads Ranking}

\author[1]{Ilqar Ramazanli\textsuperscript{*}, Hamid Eghbalzadeh\textsuperscript{*}, Xiaoyi Liu, Yang Wang, Jiaxiang Fu, Kaushik Rangadurai, Sem Park, Bo Long, Xue Feng}

\affiliation[1]{AI at Meta}
\contribution[*]{Equal Contribution}

\abstract{Perturbation-based regularization techniques address many challenges in industrial-scale large models, particularly with sparse labels, and emphasize consistency and invariance for perturbation in model predictions.
One of the popular regularization techniques has been various forms of self-consistency, which involve making small modifications to input data while preserving contextual information and enforcing similar predictions through auxiliary loss functions.
In this work, we explore the first successful application of perturbation-based regularization algorithms in large-scale ads ranking models, and further propose a novel regularization algorithm, namely, Loss-Balanced Small Perturbation Regularization (LSPR) that can be used in potentially any deep learning model.
We have successfully demonstrate that both Self-Consistency Regularization approaches (SCR) and LSPR are scalable and can improve ads delivery systems. 
By conducting industrial-scale experiments, and numerical analysis, we additionally show that our proposed LSPR, performs consistently better compared to SCR, across various groups and signal availability setups.
Finally, we report a successful application of the proposed LSPR in a billion-scale industrial ranking system, which to the best of our knowledge, is the first of its kind, and it is specially designed to address the various scalability challenges (e.g, various surfaces, geological locations, clients and so on) as we will mention in this paper. }

\date{2024-12-12}
\correspondence{Xue Feng at \email{xfeng@meta.com}}


\begin{document}

\maketitle

\section{Introduction} 
\label{sec:intro}
\section{Introduction}


\begin{figure}[t]
\centering
\includegraphics[width=0.6\columnwidth]{figures/evaluation_desiderata_V5.pdf}
\vspace{-0.5cm}
\caption{\systemName is a platform for conducting realistic evaluations of code LLMs, collecting human preferences of coding models with real users, real tasks, and in realistic environments, aimed at addressing the limitations of existing evaluations.
}
\label{fig:motivation}
\end{figure}

\begin{figure*}[t]
\centering
\includegraphics[width=\textwidth]{figures/system_design_v2.png}
\caption{We introduce \systemName, a VSCode extension to collect human preferences of code directly in a developer's IDE. \systemName enables developers to use code completions from various models. The system comprises a) the interface in the user's IDE which presents paired completions to users (left), b) a sampling strategy that picks model pairs to reduce latency (right, top), and c) a prompting scheme that allows diverse LLMs to perform code completions with high fidelity.
Users can select between the top completion (green box) using \texttt{tab} or the bottom completion (blue box) using \texttt{shift+tab}.}
\label{fig:overview}
\end{figure*}

As model capabilities improve, large language models (LLMs) are increasingly integrated into user environments and workflows.
For example, software developers code with AI in integrated developer environments (IDEs)~\citep{peng2023impact}, doctors rely on notes generated through ambient listening~\citep{oberst2024science}, and lawyers consider case evidence identified by electronic discovery systems~\citep{yang2024beyond}.
Increasing deployment of models in productivity tools demands evaluation that more closely reflects real-world circumstances~\citep{hutchinson2022evaluation, saxon2024benchmarks, kapoor2024ai}.
While newer benchmarks and live platforms incorporate human feedback to capture real-world usage, they almost exclusively focus on evaluating LLMs in chat conversations~\citep{zheng2023judging,dubois2023alpacafarm,chiang2024chatbot, kirk2024the}.
Model evaluation must move beyond chat-based interactions and into specialized user environments.



 

In this work, we focus on evaluating LLM-based coding assistants. 
Despite the popularity of these tools---millions of developers use Github Copilot~\citep{Copilot}---existing
evaluations of the coding capabilities of new models exhibit multiple limitations (Figure~\ref{fig:motivation}, bottom).
Traditional ML benchmarks evaluate LLM capabilities by measuring how well a model can complete static, interview-style coding tasks~\citep{chen2021evaluating,austin2021program,jain2024livecodebench, white2024livebench} and lack \emph{real users}. 
User studies recruit real users to evaluate the effectiveness of LLMs as coding assistants, but are often limited to simple programming tasks as opposed to \emph{real tasks}~\citep{vaithilingam2022expectation,ross2023programmer, mozannar2024realhumaneval}.
Recent efforts to collect human feedback such as Chatbot Arena~\citep{chiang2024chatbot} are still removed from a \emph{realistic environment}, resulting in users and data that deviate from typical software development processes.
We introduce \systemName to address these limitations (Figure~\ref{fig:motivation}, top), and we describe our three main contributions below.


\textbf{We deploy \systemName in-the-wild to collect human preferences on code.} 
\systemName is a Visual Studio Code extension, collecting preferences directly in a developer's IDE within their actual workflow (Figure~\ref{fig:overview}).
\systemName provides developers with code completions, akin to the type of support provided by Github Copilot~\citep{Copilot}. 
Over the past 3 months, \systemName has served over~\completions suggestions from 10 state-of-the-art LLMs, 
gathering \sampleCount~votes from \userCount~users.
To collect user preferences,
\systemName presents a novel interface that shows users paired code completions from two different LLMs, which are determined based on a sampling strategy that aims to 
mitigate latency while preserving coverage across model comparisons.
Additionally, we devise a prompting scheme that allows a diverse set of models to perform code completions with high fidelity.
See Section~\ref{sec:system} and Section~\ref{sec:deployment} for details about system design and deployment respectively.



\textbf{We construct a leaderboard of user preferences and find notable differences from existing static benchmarks and human preference leaderboards.}
In general, we observe that smaller models seem to overperform in static benchmarks compared to our leaderboard, while performance among larger models is mixed (Section~\ref{sec:leaderboard_calculation}).
We attribute these differences to the fact that \systemName is exposed to users and tasks that differ drastically from code evaluations in the past. 
Our data spans 103 programming languages and 24 natural languages as well as a variety of real-world applications and code structures, while static benchmarks tend to focus on a specific programming and natural language and task (e.g. coding competition problems).
Additionally, while all of \systemName interactions contain code contexts and the majority involve infilling tasks, a much smaller fraction of Chatbot Arena's coding tasks contain code context, with infilling tasks appearing even more rarely. 
We analyze our data in depth in Section~\ref{subsec:comparison}.



\textbf{We derive new insights into user preferences of code by analyzing \systemName's diverse and distinct data distribution.}
We compare user preferences across different stratifications of input data (e.g., common versus rare languages) and observe which affect observed preferences most (Section~\ref{sec:analysis}).
For example, while user preferences stay relatively consistent across various programming languages, they differ drastically between different task categories (e.g. frontend/backend versus algorithm design).
We also observe variations in user preference due to different features related to code structure 
(e.g., context length and completion patterns).
We open-source \systemName and release a curated subset of code contexts.
Altogether, our results highlight the necessity of model evaluation in realistic and domain-specific settings.






\section{Related Work}
\label{sec:related_work}
\section{RELATED WORK}
\label{sec:relatedwork}
In this section, we describe the previous works related to our proposal, which are divided into two parts. In Section~\ref{sec:relatedwork_exoplanet}, we present a review of approaches based on machine learning techniques for the detection of planetary transit signals. Section~\ref{sec:relatedwork_attention} provides an account of the approaches based on attention mechanisms applied in Astronomy.\par

\subsection{Exoplanet detection}
\label{sec:relatedwork_exoplanet}
Machine learning methods have achieved great performance for the automatic selection of exoplanet transit signals. One of the earliest applications of machine learning is a model named Autovetter \citep{MCcauliff}, which is a random forest (RF) model based on characteristics derived from Kepler pipeline statistics to classify exoplanet and false positive signals. Then, other studies emerged that also used supervised learning. \cite{mislis2016sidra} also used a RF, but unlike the work by \citet{MCcauliff}, they used simulated light curves and a box least square \citep[BLS;][]{kovacs2002box}-based periodogram to search for transiting exoplanets. \citet{thompson2015machine} proposed a k-nearest neighbors model for Kepler data to determine if a given signal has similarity to known transits. Unsupervised learning techniques were also applied, such as self-organizing maps (SOM), proposed \citet{armstrong2016transit}; which implements an architecture to segment similar light curves. In the same way, \citet{armstrong2018automatic} developed a combination of supervised and unsupervised learning, including RF and SOM models. In general, these approaches require a previous phase of feature engineering for each light curve. \par

%DL is a modern data-driven technology that automatically extracts characteristics, and that has been successful in classification problems from a variety of application domains. The architecture relies on several layers of NNs of simple interconnected units and uses layers to build increasingly complex and useful features by means of linear and non-linear transformation. This family of models is capable of generating increasingly high-level representations \citep{lecun2015deep}.

The application of DL for exoplanetary signal detection has evolved rapidly in recent years and has become very popular in planetary science.  \citet{pearson2018} and \citet{zucker2018shallow} developed CNN-based algorithms that learn from synthetic data to search for exoplanets. Perhaps one of the most successful applications of the DL models in transit detection was that of \citet{Shallue_2018}; who, in collaboration with Google, proposed a CNN named AstroNet that recognizes exoplanet signals in real data from Kepler. AstroNet uses the training set of labelled TCEs from the Autovetter planet candidate catalog of Q1–Q17 data release 24 (DR24) of the Kepler mission \citep{catanzarite2015autovetter}. AstroNet analyses the data in two views: a ``global view'', and ``local view'' \citep{Shallue_2018}. \par


% The global view shows the characteristics of the light curve over an orbital period, and a local view shows the moment at occurring the transit in detail

%different = space-based

Based on AstroNet, researchers have modified the original AstroNet model to rank candidates from different surveys, specifically for Kepler and TESS missions. \citet{ansdell2018scientific} developed a CNN trained on Kepler data, and included for the first time the information on the centroids, showing that the model improves performance considerably. Then, \citet{osborn2020rapid} and \citet{yu2019identifying} also included the centroids information, but in addition, \citet{osborn2020rapid} included information of the stellar and transit parameters. Finally, \citet{rao2021nigraha} proposed a pipeline that includes a new ``half-phase'' view of the transit signal. This half-phase view represents a transit view with a different time and phase. The purpose of this view is to recover any possible secondary eclipse (the object hiding behind the disk of the primary star).


%last pipeline applies a procedure after the prediction of the model to obtain new candidates, this process is carried out through a series of steps that include the evaluation with Discovery and Validation of Exoplanets (DAVE) \citet{kostov2019discovery} that was adapted for the TESS telescope.\par
%



\subsection{Attention mechanisms in astronomy}
\label{sec:relatedwork_attention}
Despite the remarkable success of attention mechanisms in sequential data, few papers have exploited their advantages in astronomy. In particular, there are no models based on attention mechanisms for detecting planets. Below we present a summary of the main applications of this modeling approach to astronomy, based on two points of view; performance and interpretability of the model.\par
%Attention mechanisms have not yet been explored in all sub-areas of astronomy. However, recent works show a successful application of the mechanism.
%performance

The application of attention mechanisms has shown improvements in the performance of some regression and classification tasks compared to previous approaches. One of the first implementations of the attention mechanism was to find gravitational lenses proposed by \citet{thuruthipilly2021finding}. They designed 21 self-attention-based encoder models, where each model was trained separately with 18,000 simulated images, demonstrating that the model based on the Transformer has a better performance and uses fewer trainable parameters compared to CNN. A novel application was proposed by \citet{lin2021galaxy} for the morphological classification of galaxies, who used an architecture derived from the Transformer, named Vision Transformer (VIT) \citep{dosovitskiy2020image}. \citet{lin2021galaxy} demonstrated competitive results compared to CNNs. Another application with successful results was proposed by \citet{zerveas2021transformer}; which first proposed a transformer-based framework for learning unsupervised representations of multivariate time series. Their methodology takes advantage of unlabeled data to train an encoder and extract dense vector representations of time series. Subsequently, they evaluate the model for regression and classification tasks, demonstrating better performance than other state-of-the-art supervised methods, even with data sets with limited samples.

%interpretation
Regarding the interpretability of the model, a recent contribution that analyses the attention maps was presented by \citet{bowles20212}, which explored the use of group-equivariant self-attention for radio astronomy classification. Compared to other approaches, this model analysed the attention maps of the predictions and showed that the mechanism extracts the brightest spots and jets of the radio source more clearly. This indicates that attention maps for prediction interpretation could help experts see patterns that the human eye often misses. \par

In the field of variable stars, \citet{allam2021paying} employed the mechanism for classifying multivariate time series in variable stars. And additionally, \citet{allam2021paying} showed that the activation weights are accommodated according to the variation in brightness of the star, achieving a more interpretable model. And finally, related to the TESS telescope, \citet{morvan2022don} proposed a model that removes the noise from the light curves through the distribution of attention weights. \citet{morvan2022don} showed that the use of the attention mechanism is excellent for removing noise and outliers in time series datasets compared with other approaches. In addition, the use of attention maps allowed them to show the representations learned from the model. \par

Recent attention mechanism approaches in astronomy demonstrate comparable results with earlier approaches, such as CNNs. At the same time, they offer interpretability of their results, which allows a post-prediction analysis. \par




\section{Preliminaries}
\label{subsec:numerical_setup}
% !TEX root =  ../main.tex
\section{Background on causality and abstraction}\label{sec:preliminaries}

This section provides the notation and key concepts related to causal modeling and abstraction theory.

\spara{Notation.} The set of integers from $1$ to $n$ is $[n]$.
The vectors of zeros and ones of size $n$ are $\zeros_n$ and $\ones_n$.
The identity matrix of size $n \times n$ is $\identity_n$. The Frobenius norm is $\frob{\mathbf{A}}$.
The set of positive definite matrices over $\reall^{n\times n}$ is $\pd^n$. The Hadamard product is $\odot$.
Function composition is $\circ$.
The domain of a function is $\dom{\cdot}$ and its kernel $\ker$.
Let $\mathcal{M}(\mathcal{X}^n)$ be the set of Borel measures over $\mathcal{X}^n \subseteq \reall^n$. Given a measure $\mu^n \in \mathcal{M}(\mathcal{X}^n)$ and a measurable map $\varphi^{\V}$, $\mathcal{X}^n \ni \mathbf{x} \overset{\varphi^{\V}}{\longmapsto} \V^\top \mathbf{x} \in \mathcal{X}^m$, we denote by $\varphi^{\V}_{\#}(\mu^n) \coloneqq \mu^n(\varphi^{\V^{-1}}(\mathbf{x}))$ the pushforward measure $\mu^m \in \mathcal{M}(\mathcal{X}^m)$. 


We now present the standard definition of SCM.

\begin{definition}[SCM, \citealp{pearl2009causality}]\label{def:SCM}
A (Markovian) structural causal model (SCM) $\scm^n$ is a tuple $\langle \myendogenous, \myexogenous, \myfunctional, \zeta^\myexogenous \rangle$, where \emph{(i)} $\myendogenous = \{X_1, \ldots, X_n\}$ is a set of $n$ endogenous random variables; \emph{(ii)} $\myexogenous =\{Z_1,\ldots,Z_n\}$ is a set of $n$ exogenous variables; \emph{(iii)} $\myfunctional$ is a set of $n$ functional assignments such that $X_i=f_i(\parents_i, Z_i)$, $\forall \; i \in [n]$, with $ \parents_i \subseteq \myendogenous \setminus \{ X_i\}$; \emph{(iv)} $\zeta^\myexogenous$ is a product probability measure over independent exogenous variables $\zeta^\myexogenous=\prod_{i \in [n]} \zeta^i$, where $\zeta^i=P(Z_i)$. 
\end{definition}
A Markovian SCM induces a directed acyclic graph (DAG) $\mathcal{G}_{\scm^n}$ where the nodes represent the variables $\myendogenous$ and the edges are determined by the structural functions $\myfunctional$; $ \parents_i$ constitutes then the parent set for $X_i$. Furthermore, we can recursively rewrite the set of structural function $\myfunctional$ as a set of mixing functions $\mymixing$ dependent only on the exogenous variables (cf. \cref{app:CA}). A key feature for studying causality is the possibility of defining interventions on the model:
\begin{definition}[Hard intervention, \citealp{pearl2009causality}]\label{def:intervention}
Given SCM $\scm^n = \langle \myendogenous, \myexogenous, \myfunctional, \zeta^\myexogenous \rangle$, a (hard) intervention $\iota = \operatorname{do}(\myendogenous^{\iota} = \mathbf{x}^{\iota})$, $\myendogenous^{\iota}\subseteq \myendogenous$,
is an operator that generates a new post-intervention SCM $\scm^n_\iota = \langle \myendogenous, \myexogenous, \myfunctional_\iota, \zeta^\myexogenous \rangle$ by replacing each function $f_i$ for $X_i\in\myendogenous^{\iota}$ with the constant $x_i^\iota\in \mathbf{x}^\iota$. 
Graphically, an intervention mutilates $\mathcal{G}_{\mathsf{M}^n}$ by removing all the incoming edges of the variables in $\myendogenous^{\iota}$.
\end{definition}

Given multiple SCMs describing the same system at different levels of granularity, CA provides the definition of an $\alpha$-abstraction map to relate these SCMs:
\begin{definition}[$\abst$-abstraction, \citealp{rischel2020category}]\label{def:abstraction}
Given low-level $\mathsf{M}^\ell$ and high-level $\mathsf{M}^h$ SCMs, an $\abst$-abstraction is a triple $\abst = \langle \Rset, \amap, \alphamap{} \rangle$, where \emph{(i)} $\Rset \subseteq \datalow$ is a subset of relevant variables in $\mathsf{M}^\ell$; \emph{(ii)} $\amap: \Rset \rightarrow \datahigh$ is a surjective function between the relevant variables of $\mathsf{M}^\ell$ and the endogenous variables of $\mathsf{M}^h$; \emph{(iii)} $\alphamap{}: \dom{\Rset} \rightarrow \dom{\datahigh}$ is a modular function $\alphamap{} = \bigotimes_{i\in[n]} \alphamap{X^h_i}$ made up by surjective functions $\alphamap{X^h_i}: \dom{\amap^{-1}(X^h_i)} \rightarrow \dom{X^h_i}$ from the outcome of low-level variables $\amap^{-1}(X^h_i) \in \datalow$ onto outcomes of the high-level variables $X^h_i \in \datahigh$.
\end{definition}
Notice that an $\abst$-abstraction simultaneously maps variables via the function $\amap$ and values through the function $\alphamap{}$. The definition itself does not place any constraint on these functions, although a common requirement in the literature is for the abstraction to satisfy \emph{interventional consistency} \cite{rubenstein2017causal,rischel2020category,beckers2019abstracting}. An important class of such well-behaved abstractions is \emph{constructive linear abstraction}, for which the following properties hold. By constructivity, \emph{(i)} $\abst$ is interventionally consistent; \emph{(ii)} all low-level variables are relevant $\Rset=\datalow$; \emph{(iii)} in addition to the map $\alphamap{}$ between endogenous variables, there exists a map ${\alphamap{}}_U$ between exogenous variables satisfying interventional consistency \cite{beckers2019abstracting,schooltink2024aligning}. By linearity, $\alphamap{} = \V^\top \in \reall^{h \times \ell}$ \cite{massidda2024learningcausalabstractionslinear}. \cref{app:CA} provides formal definitions for interventional consistency, linear and constructive abstraction.

\section{Methodology}
\label{sec:modeling}
\section{Research Methodology}~\label{sec:Methodology}

In this section, we discuss the process of conducting our systematic review, e.g., our search strategy for data extraction of relevant studies, based on the guidelines of Kitchenham et al.~\cite{kitchenham2022segress} to conduct SLRs and Petersen et al.~\cite{PETERSEN20151} to conduct systematic mapping studies (SMSs) in Software Engineering. In this systematic review, we divide our work into a four-stage procedure, including planning, conducting, building a taxonomy, and reporting the review, illustrated in Fig.~\ref{fig:search}. The four stages are as follows: (1) the \emph{planning} stage involved identifying research questions (RQs) and specifying the detailed research plan for the study; (2) the \emph{conducting} stage involved analyzing and synthesizing the existing primary studies to answer the research questions; (3) the \emph{taxonomy} stage was introduced to optimize the data extraction results and consolidate a taxonomy schema for REDAST methodology; (4) the \emph{reporting} stage involved the reviewing, concluding and reporting the final result of our study.

\begin{figure}[!t]
    \centering
    \includegraphics[width=1\linewidth]{fig/methodology/searching-process.drawio.pdf}
    \caption{Systematic Literature Review Process}
    \label{fig:search}
\end{figure}

\subsection{Research Questions}
In this study, we developed five research questions (RQs) to identify the input and output, analyze technologies, evaluate metrics, identify challenges, and identify potential opportunities. 

\textbf{RQ1. What are the input configurations, formats, and notations used in the requirements in requirements-driven
automated software testing?} In requirements-driven testing, the input is some form of requirements specification -- which can vary significantly. RQ1 maps the input for REDAST and reports on the comparison among different formats for requirements specification.

\textbf{RQ2. What are the frameworks, tools, processing methods, and transformation techniques used in requirements-driven automated software testing studies?} RQ2 explores the technical solutions from requirements to generated artifacts, e.g., rule-based transformation applying natural language processing (NLP) pipelines and deep learning (DL) techniques, where we additionally discuss the potential intermediate representation and additional input for the transformation process.

\textbf{RQ3. What are the test formats and coverage criteria used in the requirements-driven automated software
testing process?} RQ3 focuses on identifying the formulation of generated artifacts (i.e., the final output). We map the adopted test formats and analyze their characteristics in the REDAST process.

\textbf{RQ4. How do existing studies evaluate the generated test artifacts in the requirements-driven automated software testing process?} RQ4 identifies the evaluation datasets, metrics, and case study methodologies in the selected papers. This aims to understand how researchers assess the effectiveness, accuracy, and practical applicability of the generated test artifacts.

\textbf{RQ5. What are the limitations and challenges of existing requirements-driven automated software testing methods in the current era?} RQ5 addresses the limitations and challenges of existing studies while exploring future directions in the current era of technology development. %It particularly highlights the potential benefits of advanced LLMs and examines their capacity to meet the high expectations placed on these cutting-edge language modeling technologies. %\textcolor{blue}{CA: Do we really need to focus on LLMs? TBD.} \textcolor{orange}{FW: About LLMs, I removed the direct emphase in RQ5 but kept the discussion in RQ5 and the solution section. I think that would be more appropriate.}

\subsection{Searching Strategy}

The overview of the search process is exhibited in Fig. \ref{fig:papers}, which includes all the details of our search steps.
\begin{table}[!ht]
\caption{List of Search Terms}
\label{table:search_term}
\begin{tabularx}{\textwidth}{lX}
\hline
\textbf{Terms Group} & \textbf{Terms} \\ \hline
Test Group & test* \\
Requirement Group & requirement* OR use case* OR user stor* OR specification* \\
Software Group & software* OR system* \\
Method Group & generat* OR deriv* OR map* OR creat* OR extract* OR design* OR priorit* OR construct* OR transform* \\ \hline
\end{tabularx}
\end{table}

\begin{figure}
    \centering
    \includegraphics[width=1\linewidth]{fig/methodology/search-papers.drawio.pdf}
    \caption{Study Search Process}
    \label{fig:papers}
\end{figure}

\subsubsection{Search String Formulation}
Our research questions (RQs) guided the identification of the main search terms. We designed our search string with generic keywords to avoid missing out on any related papers, where four groups of search terms are included, namely ``test group'', ``requirement group'', ``software group'', and ``method group''. In order to capture all the expressions of the search terms, we use wildcards to match the appendix of the word, e.g., ``test*'' can capture ``testing'', ``tests'' and so on. The search terms are listed in Table~\ref{table:search_term}, decided after iterative discussion and refinement among all the authors. As a result, we finally formed the search string as follows:


\hangindent=1.5em
 \textbf{ON ABSTRACT} ((``test*'') \textbf{AND} (``requirement*'' \textbf{OR} ``use case*'' \textbf{OR} ``user stor*'' \textbf{OR} ``specifications'') \textbf{AND} (``software*'' \textbf{OR} ``system*'') \textbf{AND} (``generat*'' \textbf{OR} ``deriv*'' \textbf{OR} ``map*'' \textbf{OR} ``creat*'' \textbf{OR} ``extract*'' \textbf{OR} ``design*'' \textbf{OR} ``priorit*'' \textbf{OR} ``construct*'' \textbf{OR} ``transform*''))

The search process was conducted in September 2024, and therefore, the search results reflect studies available up to that date. We conducted the search process on six online databases: IEEE Xplore, ACM Digital Library, Wiley, Scopus, Web of Science, and Science Direct. However, some databases were incompatible with our default search string in the following situations: (1) unsupported for searching within abstract, such as Scopus, and (2) limited search terms, such as ScienceDirect. Here, for (1) situation, we searched within the title, keyword, and abstract, and for (2) situation, we separately executed the search and removed the duplicate papers in the merging process. 

\subsubsection{Automated Searching and Duplicate Removal}
We used advanced search to execute our search string within our selected databases, following our designed selection criteria in Table \ref{table:selection}. The first search returned 27,333 papers. Specifically for the duplicate removal, we used a Python script to remove (1) overlapped search results among multiple databases and (2) conference or workshop papers, also found with the same title and authors in the other journals. After duplicate removal, we obtained 21,652 papers for further filtering.

\begin{table*}[]
\caption{Selection Criteria}
\label{table:selection}
\begin{tabularx}{\textwidth}{lX}
\hline
\textbf{Criterion ID} & \textbf{Criterion Description} \\ \hline
S01          & Papers written in English. \\
S02-1        & Papers in the subjects of "Computer Science" or "Software Engineering". \\
S02-2        & Papers published on software testing-related issues. \\
S03          & Papers published from 1991 to the present. \\ 
S04          & Papers with accessible full text. \\ \hline
\end{tabularx}
\end{table*}

\begin{table*}[]
\small
\caption{Inclusion and Exclusion Criteria}
\label{table:criteria}
\begin{tabularx}{\textwidth}{lX}
\hline
\textbf{ID}  & \textbf{Description} \\ \hline
\multicolumn{2}{l}{\textbf{Inclusion Criteria}} \\ \hline
I01 & Papers about requirements-driven automated system testing or acceptance testing generation, or studies that generate system-testing-related artifacts. \\
I02 & Peer-reviewed studies that have been used in academia with references from literature. \\ \hline
\multicolumn{2}{l}{\textbf{Exclusion Criteria}} \\ \hline
E01 & Studies that only support automated code generation, but not test-artifact generation. \\
E02 & Studies that do not use requirements-related information as an input. \\
E03 & Papers with fewer than 5 pages (1-4 pages). \\
E04 & Non-primary studies (secondary or tertiary studies). \\
E05 & Vision papers and grey literature (unpublished work), books (chapters), posters, discussions, opinions, keynotes, magazine articles, experience, and comparison papers. \\ \hline
\end{tabularx}
\end{table*}

\subsubsection{Filtering Process}

In this step, we filtered a total of 21,652 papers using the inclusion and exclusion criteria outlined in Table \ref{table:criteria}. This process was primarily carried out by the first and second authors. Our criteria are structured at different levels, facilitating a multi-step filtering process. This approach involves applying various criteria in three distinct phases. We employed a cross-verification method involving (1) the first and second authors and (2) the other authors. Initially, the filtering was conducted separately by the first and second authors. After cross-verifying their results, the results were then reviewed and discussed further by the other authors for final decision-making. We widely adopted this verification strategy within the filtering stages. During the filtering process, we managed our paper list using a BibTeX file and categorized the papers with color-coding through BibTeX management software\footnote{\url{https://bibdesk.sourceforge.io/}}, i.e., “red” for irrelevant papers, “yellow” for potentially relevant papers, and “blue” for relevant papers. This color-coding system facilitated the organization and review of papers according to their relevance.

The screening process is shown below,
\begin{itemize}
    \item \textbf{1st-round Filtering} was based on the title and abstract, using the criteria I01 and E01. At this stage, the number of papers was reduced from 21,652 to 9,071.
    \item \textbf{2nd-round Filtering}. We attempted to include requirements-related papers based on E02 on the title and abstract level, which resulted from 9,071 to 4,071 papers. We excluded all the papers that did not focus on requirements-related information as an input or only mentioned the term ``requirements'' but did not refer to the requirements specification.
    \item \textbf{3rd-round Filtering}. We selectively reviewed the content of papers identified as potentially relevant to requirements-driven automated test generation. This process resulted in 162 papers for further analysis.
\end{itemize}
Note that, especially for third-round filtering, we aimed to include as many relevant papers as possible, even borderline cases, according to our criteria. The results were then discussed iteratively among all the authors to reach a consensus.

\subsubsection{Snowballing}

Snowballing is necessary for identifying papers that may have been missed during the automated search. Following the guidelines by Wohlin~\cite{wohlin2014guidelines}, we conducted both forward and backward snowballing. As a result, we identified 24 additional papers through this process.

\subsubsection{Data Extraction}

Based on the formulated research questions (RQs), we designed 38 data extraction questions\footnote{\url{https://drive.google.com/file/d/1yjy-59Juu9L3WHaOPu-XQo-j-HHGTbx_/view?usp=sharing}} and created a Google Form to collect the required information from the relevant papers. The questions included 30 short-answer questions, six checkbox questions, and two selection questions. The data extraction was organized into five sections: (1) basic information: fundamental details such as title, author, venue, etc.; (2) open information: insights on motivation, limitations, challenges, etc.; (3) requirements: requirements format, notation, and related aspects; (4) methodology: details, including immediate representation and technique support; (5) test-related information: test format(s), coverage, and related elements. Similar to the filtering process, the first and second authors conducted the data extraction and then forwarded the results to the other authors to initiate the review meeting.

\subsubsection{Quality Assessment}

During the data extraction process, we encountered papers with insufficient information. To address this, we conducted a quality assessment in parallel to ensure the relevance of the papers to our objectives. This approach, also adopted in previous secondary studies~\cite{shamsujjoha2021developing, naveed2024model}, involved designing a set of assessment questions based on guidelines by Kitchenham et al.~\cite{kitchenham2022segress}. The quality assessment questions in our study are shown below:
\begin{itemize}
    \item \textbf{QA1}. Does this study clearly state \emph{how} requirements drive automated test generation?
    \item \textbf{QA2}. Does this study clearly state the \emph{aim} of REDAST?
    \item \textbf{QA3}. Does this study enable \emph{automation} in test generation?
    \item \textbf{QA4}. Does this study demonstrate the usability of the method from the perspective of methodology explanation, discussion, case examples, and experiments?
\end{itemize}
QA4 originates from an open perspective in the review process, where we focused on evaluation, discussion, and explanation. Our review also examined the study’s overall structure, including the methodology description, case studies, experiments, and analyses. The detailed results of the quality assessment are provided in the Appendix. Following this assessment, the final data extraction was based on 156 papers.

% \begin{table}[]
% \begin{tabular}{ll}
% \hline
% QA ID & QA Questions                                             \\ \hline
% Q01   & Does this study clearly state its aims?                  \\
% Q02   & Does this study clearly describe its methodology?        \\
% Q03   & Does this study involve automated test generation?       \\
% Q04   & Does this study include a promising evaluation?          \\
% Q05   & Does this study demonstrate the usability of the method? \\ \hline
% \end{tabular}%
% \caption{Questions for Quality Assessment}
% \label{table:qa}
% \end{table}

% automated quality assessment

% \textcolor{blue}{CA: Our search strategy focused on identifying requirements types first. We covered several sources, e.g., ~\cite{Pohl:11,wagner2019status} to identify different formats and notations of specifying requirements. However, this came out to be a long list, e.g., free-form NL requirements, semi-formal UML models, free-from textual use case models, UML class diagrams, UML activity diagrams, and so on. In this paper, we attempted to primarily focus on requirements-related aspects and not design-level information. Hence, we generalised our search string to include generic keywords, e.g., requirement*, use case*, and user stor*. We did so to avoid missing out on any papers, bringing too restrictive in our search strategy, and not creating a too-generic search string with all the aforementioned formats to avoid getting results beyond our review's scope.}


%% Use \subsection commands to start a subsection.



%\subsection{Study Selection}

% In this step, we further looked into the content of searched papers using our search strategy and applied our inclusion and exclusion criteria. Our filtering strategy aimed to pinpoint studies focused on requirements-driven system-level testing. Recognizing the presence of irrelevant papers in our search results, we established detailed selection criteria for preliminary inclusion and exclusion, as shown in Table \ref{table: criteria}. Specifically, we further developed the taxonomy schema to exclude two types of studies that did not meet the requirements for system-level testing: (1) studies supporting specification-driven test generation, such as UML-driven test generation, rather than requirements-driven testing, and (2) studies focusing on code-based test generation, such as requirement-driven code generation for unit testing.





\section{Analysis and Experimentation }
\label{sec:experiments}
\newmaterial{
In this section, we leverage a well-known theoretical framework proposed in~\cite{werfel2003learning} to demonstrate how LSPR results in a better alignment of weights in the model optimization to portray a clear picture of the construct of a optimization problem in ranking, and how LSPR affects it.}
We start by formalization of our framework, as well as the integration of Perturbation-Based Regularizations, namely SCR and LSPR. 
In Section~\ref{subsec:numerical_analysis} we analyze how LSPR compares to SCR via controlled experimentations and analysis on linear models, investigating the learning dynamics with these regularization applied.
Further, in Section~\ref{subsec:real_data}, we report our empirical results on an in-house dataset that was used to evaluate the methodologies applied here. 
We tracked model accuracy using Normalized Entropy (NE) in offline experiments \cite{he2014practical}.
In experiments with real data, each datapoint exhibits a substantial volume of features, comprising thousands of dense features and hundreds of sparse features and we employ the Adagrad optimizer for optimization.
Ranking has been done through multiple stages during learning, which are described below in more details.
In this section, we report performance improvements via the presented regularization techniques on a multi-stage ranking system with 3 stages of retrieval, early stage ranking, and final-stage ranker.


\subsection{Numerical Analysis}
\label{subsec:numerical_analysis}

In this section, we provide a numerical analysis for the linear models trained with SGD 1) with Self-consistency Regularization (SCR), and 2) with Loss-Balanced Small Perturbation Regularization (LSPR).
We analyse the gradient update directions and the alignment with the optimal weight (See Section.~\ref{subsec:numerical_setup}) by calculating the cosine similarity in the model's weight space, comparing weights of different iterations to the optimal weight.
Our numerical analysis (see Figure~\ref{fig:numerical_analysis}) shows that: 
\begin{enumerate}
    \item compared to SCR, LSPR finds a better alignment with the optimal weight, while converging faster and achieving a lower error in the weight space. 
    \item we also show that balancing both amount of noise $\omega$ and loss $\lambda$ is crucial to the success of LSPR and SCR. As we show, smaller values for these weights are recommended for better convergence and performance.
\end{enumerate} 


\subsubsection{Setup}
The goal in this section is to analyse how different perturbation-based regularizations, namely SCR and LSPR, impact learning and performance.
To this end, we simplify both LSPR and SCR frameworks to their core, and furthermore using linear models study their effects in learning dynamics and performance. We use 2-layer linear models which strike a good balance between model expressiveness and simplicity~\cite{werfel2003learning}. To this end, we define a ground-truth function with the weight $\mW^*$ that maps input data to their labels as follows:
\begin{align}
    \vy={\mW}^*\vx
\end{align}
where $\vx$ denotes an input feature and $\vy$ denotes the ground-truth output, and ${\mW}^*$ is a $L_y\times L_x$ matrix where $L_y$ and $L_x$  are input and output dimensionalities.


We now define the following linear model that we use to learn the input-output relationship by:
\begin{align}
    \vy={\mW_2}{\mW_1}\vx
\end{align}
where $\vx$ denotes an input feature and $\vy$ denotes the ground-truth output,  ${\mW_1}$ is a matrix of size $L_h\times L_x$ and  and $\mW_2$ is a matrix of size $L_y\times L_h$ and $L_h$ is the dimensionality of the intermediate representations.
To simulate the effect of regularization, we use Stochastic Gradient Descent (SGD) with an MSE error as follows:
\begin{align}
    \mathcal{L}(\vx,\vy)=\frac{1}{2}||\vy-\hat{\vy}||^2,
\end{align}
and $\hat{\vy}$ is the output of the linear model.
In order to study the learning dynamics, we denote the  weight error as:

\begin{align}
    \mE=\mW_2\mW_1-\mW^*
\end{align}
and further introduce:
\begin{align}
    \epsilon=\frac{1}{L_x L_y}\trace[\mE^T\mE], \gamma=\frac{{\mW_2\mW_1 \cdot \mW^*}}{{\|\mW_2\mW_1\| \|\mW^*\|}}
\end{align}
where $\epsilon$ represents the error in the weight space to the optimal weight, while $\gamma$ demonstrates weight alignment with the optimal weight $\mW^*$.
The SGD weight updates are as follows:
\begin{align}
    \delta\mW_1^{SGD}&=-\eta\frac{\partial \mathcal{L}(\vx,\vy)}{\partial \mW_1} \\
    &=-\eta\left(\mW_2^T(\mW_2\mW_1\vx-\vy)\right) \otimes \vx \\
    \delta\mW_2^{SGD}&=-\eta\frac{\partial \mathcal{L}(\vx,\vy)}{\partial \mW_2} \\
    &=-\eta(\mW_2\mW_1\vx-\vy) \otimes \mW_1\vx 
\end{align}
with $\otimes$ denoting the outer product.

In order to simulate LSPR, we sample noise $\vz\sim\mathcal{N}(0, I)$ and additionally add $\mathcal{L}(\vx+\omega\vz,\vy)$ where $\omega$ is a small weight for the perturbation $\vz$.
The final loss will be a balance of $\mathcal{L}(\vx,\vy)+\lambda\mathcal{L}(\omega\vz+\vx,\vy)$.
The LSPR weight updates are then defined as:
\begin{align}
\label{eq:lspr_updates}
    \delta\mW_1^{LSPR}&=-\eta\left(\mW_2^T(\mW_2\mW_1\vx-\vy)\right) \otimes \vx \nonumber\\
&-\lambda\eta\left(\mW_2^T(\mW_2\mW_1(\omega\sigma\vz+\vx)-\vy)\right) \otimes (\omega\sigma\vz+\vx)\\  
\delta\mW_2^{LSPR}&=-\eta(\mW_2\mW_1\vx-\vy) \otimes \mW_1\vx \nonumber\\
&-\lambda\eta(\mW_2\mW_1(\omega\sigma\vz+\vx)-\vy) \otimes \mW_1(\omega\sigma\vz+\vx)
\end{align}

To analyse the SCR method, we rely on the additional learning signal that pushes the output of a model on clean and noisy inputs closer together, namely, $\mathcal{L}(\vx+\omega\vz,\hat{\vy})$ where $\hat{\vy}=\mW_2\mW_1\vx$.
Consequently, the SCP weight updates are as follows:
\begin{align}
\label{eq:scr_updates}
    \delta\mW_1^{SCR}&=-\eta\left(\mW_2^T(\mW_2\mW_1\vx-\vy)\right) \otimes \vx \nonumber\\
&-\lambda\eta\left(\mW_2^T(\mW_2\mW_1(\omega\sigma\vz+\vx)-\mW_2\mW_1\vx)\right) \nonumber\\
&\otimes (\omega\sigma\vz+\vx)\\  
\delta\mW_2^{SCR}&=-\eta(\mW_2\mW_1\vx-\vy) \otimes \mW_1\vx \nonumber\\
&-\lambda\eta(\mW_2\mW_1(\omega\sigma\vz+\vx)-\mW_2\mW_1\vx) \nonumber\\
&\otimes \mW_1(\omega\sigma\vz+\vx)
\end{align}

We use the following parameters for our analysis: $\omega=\{0.1,0.9\}, \lambda=\{0.001,1\}, \eta=1.4, L_x=100,L_h=10^4,L_y=10$, and we perform the weight updates for the number of $100k$ times, and for every update we sample new data from the denoted distributions.

\subsubsection{Results}
As can be seen in Figure~\ref{fig:numerical_analysis}, we can observe that for small perturbation weights $\omega$ and loss weights $\lambda$, LSPR tends to better find the optimal weight as can be seen by looking at the two presented plots.  


\begin{figure*}[t]
    \centering
    \begin{subfigure}{0.45\textwidth}
        \centering
        \includegraphics[width=\textwidth]{used_figures/alignment_with_wstart.png}
        \caption{Alignment with the optimal weight $\mW^*$.}
        
        \label{fig:first_fig}
    \end{subfigure}
    \hfill
    \begin{subfigure}{0.45\textwidth}
        \centering
        \includegraphics[width=\textwidth]{used_figures/weight_space_error.png}
        \caption{Error in the weight space.}
        \label{fig:second_fig}
    \end{subfigure}
    \caption{Numerical analysis comparing LSPR and SCR.
    $\omega$ denotes noise sample weight and $\lambda$ depicts loss weight.} 
\label{fig:numerical_analysis}
\end{figure*}

\subsection{Experiments on Real Data}
\label{subsec:real_data}
\subsubsection{Self-Consistency Regularization (SCR)}

We experimented with perturbation-based consistency regularization on different stages of various prediction problems. We present these results in Table~\ref{tab:rel_ne}.
We observed a relative NE gain of approximately 0.1\%-0.3\%, depending on the prediction model tested in various offline experiments.
We will first present the results for the Retrieval stage from the offline experiments, followed by the experimental results for the Early and Final Stage ranker.

\noindent\textbf{Offline Retrieval Stage:} We have experimented with consistency regularization in two different models that predict conversion rate and click-through rate, respectively. 
We obtained the best results when regularizing both logit and representative of embedding with 0.15\%-0.2\% relative NE improvements.

\noindent\textbf{Offline Early Stage Ranking:} models are generally simpler ranking models compared to final stage models. Therefore, we applied regularization to the entire object and user embedding, resulting in a 0.3\% relative NE gain in various offline experiments.


\noindent\textbf{Offline Final Ranking Stage:} 
Models in this stage are generally much larger and complex compared to previous stages, as we are looking for more precise ranking of ads. 
We obtained the best results by regularizing both the logit and output of the embedding together.
Results from several experiments suggest an average 0.1\% relative NE gain, which has been further validated with online testing.


\begin{table}
  \caption{Relative NE gains for SCR across various stages.}
  \label{tab:rel_ne}
    \centering
  \begin{tabular}{cccc}
  \toprule
    Model & 33\% of data &  66\% of data & 100 \% data\\
    \midrule
    \texttt{Baseline} & 0 \% & 0 \% &  0 \% \\
    \textbf{Retrieval} & \bf{0.14 \%} & \bf{0.19} &  \bf{0.14 \%}\\
    \textbf{Early Stage} & \bf{0.28 \%} & \bf{0.3 \%} &  \bf{0.35 \%}\\
    \textbf{Final-stage Ranker} & \bf{0.1 \%} & \bf{0.08 \%} &  \bf{0.07 \%}\\
    \bottomrule
  \end{tabular}
\end{table}

\subsubsection{Loss-Balanced Small Perturbation Regularization (LSPR)}
We have explored LSPR primarily in the offline Final Ranker Stage, under various signal availability setups. We have observed that the technique has performed promisingly in various setups, ultimately leading to improved performance in each of these environments. These results are depicted in Table~\ref{tab:rel_ne2}.



\noindent\textbf{Offline Final Ranking Stage:} testing is very similar to Consistency Regularization testing; however, in the former, we perturbed the entire batch each time, leading to doubled batch size. 
In contrast, for Consistency Regularization, we only perturbed a small fraction of points in each batch.\\

\begin{table}
  \caption{Relative NE gains comparing SCR and LSPR on Final-stage Ranker.} 
  \label{tab:rel_ne2}
  \centering
  \begin{tabular}{cccc}
    \toprule
    Model & 33\% of data &  66\% of data & 100 \% data\\
    \midrule
   \texttt{Baseline} & 0 \% & 0 \% &  0 \% \\
    \textbf{SCR} & \bf{0.1\ \% } & \bf{0.08 \%} &  \bf{0.07 \%}\\
    \textbf{LSPR} & \bf{0.13 \% } & \bf{0.11 \%} &  \bf{0.1 \%}\\
    \bottomrule
  \end{tabular}
\end{table}


\subsubsection{Online Experiments}

We additionally have conducted online experimentation for a prediction model after testing it in offline setup.
The online experimentation is different than offline one in the nature that, it runs in continuous training and inferring routine, compared to full training and inferring mode.
% 
These online experiments on various data from different parts of the data stream, using both noisy and clean labels, have demonstrated a similar trend to the offline experiments we reported in the previous sections.
\newmaterial{
Our results indicate that LSPR has achieved a 0.1\% to 0.2\% relative improvement in online top-line metrics, consistently across multiple launches. Note that the magnitude of the impact is significant at the level of a billion-scale industrial production ads ranking system, which serves billions of users across various surfaces , across global geological locations, and across various clients. 
}

\subsection{Baselines}

\newmaterial{
The experiment comparisons in this manuscript are all compared against the latest production models in a multi-billion-scale industrial ads ranking system, prior to the adoption of LSPR. 
Our criteria for selecting baselines was to identify models that 1) have been proven to operate effectively at the industry scale; 2) represent the  state-of-the-art ads ranking product models  in the industry.
We consider these production-level recommendation models to be among the state-of-the-art baselines that meet the above criterion. 
}



\section{Conclusion and Future Work}
\label{sec:conclusion_related}

Our study has explored the application of perturbation based regularization algorithms in an Industrial-Scale Recommendation Systems. 
To this end, we have made significant contributions: firstly, to the best of our knowledge, we showed for the first time that Perturbation Based Regularization techniques can bring meaningful improvements to Industrial-Scale Recommendation Systems.
Secondly, we introduced a novel regularization technique - LSPR,a general method that is applicable in many Deep Learning setups.
\newmaterial{In summary, LSPR has been launched to major industrial-scale ads recommendation models across different ranking stages and traffic. This indicates that it can be generalized to diverse user demographics and content types, considering the scale and reach of the deployed ads platform.}
Our future research endeavors are poised to focus on other variations of the use of unlabeled data, tailored for Large Scale Recommendation Systems, pushing on both theoretical understanding, as well as industrial-scalability.
These next steps represent our commitment to pushing the boundaries of recommendation systems, with a keen focus on understanding and optimizing ad recommendations to better serve both users and businesses.



\clearpage
\newpage
\bibliographystyle{assets/plainnat}
\bibliography{refs}


\end{document}