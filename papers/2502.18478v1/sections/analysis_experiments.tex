\newmaterial{
In this section, we leverage a well-known theoretical framework proposed in~\cite{werfel2003learning} to demonstrate how LSPR results in a better alignment of weights in the model optimization to portray a clear picture of the construct of a optimization problem in ranking, and how LSPR affects it.}
We start by formalization of our framework, as well as the integration of Perturbation-Based Regularizations, namely SCR and LSPR. 
In Section~\ref{subsec:numerical_analysis} we analyze how LSPR compares to SCR via controlled experimentations and analysis on linear models, investigating the learning dynamics with these regularization applied.
Further, in Section~\ref{subsec:real_data}, we report our empirical results on an in-house dataset that was used to evaluate the methodologies applied here. 
We tracked model accuracy using Normalized Entropy (NE) in offline experiments \cite{he2014practical}.
In experiments with real data, each datapoint exhibits a substantial volume of features, comprising thousands of dense features and hundreds of sparse features and we employ the Adagrad optimizer for optimization.
Ranking has been done through multiple stages during learning, which are described below in more details.
In this section, we report performance improvements via the presented regularization techniques on a multi-stage ranking system with 3 stages of retrieval, early stage ranking, and final-stage ranker.


\subsection{Numerical Analysis}
\label{subsec:numerical_analysis}

In this section, we provide a numerical analysis for the linear models trained with SGD 1) with Self-consistency Regularization (SCR), and 2) with Loss-Balanced Small Perturbation Regularization (LSPR).
We analyse the gradient update directions and the alignment with the optimal weight (See Section.~\ref{subsec:numerical_setup}) by calculating the cosine similarity in the model's weight space, comparing weights of different iterations to the optimal weight.
Our numerical analysis (see Figure~\ref{fig:numerical_analysis}) shows that: 
\begin{enumerate}
    \item compared to SCR, LSPR finds a better alignment with the optimal weight, while converging faster and achieving a lower error in the weight space. 
    \item we also show that balancing both amount of noise $\omega$ and loss $\lambda$ is crucial to the success of LSPR and SCR. As we show, smaller values for these weights are recommended for better convergence and performance.
\end{enumerate} 


\subsubsection{Setup}
The goal in this section is to analyse how different perturbation-based regularizations, namely SCR and LSPR, impact learning and performance.
To this end, we simplify both LSPR and SCR frameworks to their core, and furthermore using linear models study their effects in learning dynamics and performance. We use 2-layer linear models which strike a good balance between model expressiveness and simplicity~\cite{werfel2003learning}. To this end, we define a ground-truth function with the weight $\mW^*$ that maps input data to their labels as follows:
\begin{align}
    \vy={\mW}^*\vx
\end{align}
where $\vx$ denotes an input feature and $\vy$ denotes the ground-truth output, and ${\mW}^*$ is a $L_y\times L_x$ matrix where $L_y$ and $L_x$  are input and output dimensionalities.


We now define the following linear model that we use to learn the input-output relationship by:
\begin{align}
    \vy={\mW_2}{\mW_1}\vx
\end{align}
where $\vx$ denotes an input feature and $\vy$ denotes the ground-truth output,  ${\mW_1}$ is a matrix of size $L_h\times L_x$ and  and $\mW_2$ is a matrix of size $L_y\times L_h$ and $L_h$ is the dimensionality of the intermediate representations.
To simulate the effect of regularization, we use Stochastic Gradient Descent (SGD) with an MSE error as follows:
\begin{align}
    \mathcal{L}(\vx,\vy)=\frac{1}{2}||\vy-\hat{\vy}||^2,
\end{align}
and $\hat{\vy}$ is the output of the linear model.
In order to study the learning dynamics, we denote the  weight error as:

\begin{align}
    \mE=\mW_2\mW_1-\mW^*
\end{align}
and further introduce:
\begin{align}
    \epsilon=\frac{1}{L_x L_y}\trace[\mE^T\mE], \gamma=\frac{{\mW_2\mW_1 \cdot \mW^*}}{{\|\mW_2\mW_1\| \|\mW^*\|}}
\end{align}
where $\epsilon$ represents the error in the weight space to the optimal weight, while $\gamma$ demonstrates weight alignment with the optimal weight $\mW^*$.
The SGD weight updates are as follows:
\begin{align}
    \delta\mW_1^{SGD}&=-\eta\frac{\partial \mathcal{L}(\vx,\vy)}{\partial \mW_1} \\
    &=-\eta\left(\mW_2^T(\mW_2\mW_1\vx-\vy)\right) \otimes \vx \\
    \delta\mW_2^{SGD}&=-\eta\frac{\partial \mathcal{L}(\vx,\vy)}{\partial \mW_2} \\
    &=-\eta(\mW_2\mW_1\vx-\vy) \otimes \mW_1\vx 
\end{align}
with $\otimes$ denoting the outer product.

In order to simulate LSPR, we sample noise $\vz\sim\mathcal{N}(0, I)$ and additionally add $\mathcal{L}(\vx+\omega\vz,\vy)$ where $\omega$ is a small weight for the perturbation $\vz$.
The final loss will be a balance of $\mathcal{L}(\vx,\vy)+\lambda\mathcal{L}(\omega\vz+\vx,\vy)$.
The LSPR weight updates are then defined as:
\begin{align}
\label{eq:lspr_updates}
    \delta\mW_1^{LSPR}&=-\eta\left(\mW_2^T(\mW_2\mW_1\vx-\vy)\right) \otimes \vx \nonumber\\
&-\lambda\eta\left(\mW_2^T(\mW_2\mW_1(\omega\sigma\vz+\vx)-\vy)\right) \otimes (\omega\sigma\vz+\vx)\\  
\delta\mW_2^{LSPR}&=-\eta(\mW_2\mW_1\vx-\vy) \otimes \mW_1\vx \nonumber\\
&-\lambda\eta(\mW_2\mW_1(\omega\sigma\vz+\vx)-\vy) \otimes \mW_1(\omega\sigma\vz+\vx)
\end{align}

To analyse the SCR method, we rely on the additional learning signal that pushes the output of a model on clean and noisy inputs closer together, namely, $\mathcal{L}(\vx+\omega\vz,\hat{\vy})$ where $\hat{\vy}=\mW_2\mW_1\vx$.
Consequently, the SCP weight updates are as follows:
\begin{align}
\label{eq:scr_updates}
    \delta\mW_1^{SCR}&=-\eta\left(\mW_2^T(\mW_2\mW_1\vx-\vy)\right) \otimes \vx \nonumber\\
&-\lambda\eta\left(\mW_2^T(\mW_2\mW_1(\omega\sigma\vz+\vx)-\mW_2\mW_1\vx)\right) \nonumber\\
&\otimes (\omega\sigma\vz+\vx)\\  
\delta\mW_2^{SCR}&=-\eta(\mW_2\mW_1\vx-\vy) \otimes \mW_1\vx \nonumber\\
&-\lambda\eta(\mW_2\mW_1(\omega\sigma\vz+\vx)-\mW_2\mW_1\vx) \nonumber\\
&\otimes \mW_1(\omega\sigma\vz+\vx)
\end{align}

We use the following parameters for our analysis: $\omega=\{0.1,0.9\}, \lambda=\{0.001,1\}, \eta=1.4, L_x=100,L_h=10^4,L_y=10$, and we perform the weight updates for the number of $100k$ times, and for every update we sample new data from the denoted distributions.

\subsubsection{Results}
As can be seen in Figure~\ref{fig:numerical_analysis}, we can observe that for small perturbation weights $\omega$ and loss weights $\lambda$, LSPR tends to better find the optimal weight as can be seen by looking at the two presented plots.  


\begin{figure*}[t]
    \centering
    \begin{subfigure}{0.45\textwidth}
        \centering
        \includegraphics[width=\textwidth]{used_figures/alignment_with_wstart.png}
        \caption{Alignment with the optimal weight $\mW^*$.}
        
        \label{fig:first_fig}
    \end{subfigure}
    \hfill
    \begin{subfigure}{0.45\textwidth}
        \centering
        \includegraphics[width=\textwidth]{used_figures/weight_space_error.png}
        \caption{Error in the weight space.}
        \label{fig:second_fig}
    \end{subfigure}
    \caption{Numerical analysis comparing LSPR and SCR.
    $\omega$ denotes noise sample weight and $\lambda$ depicts loss weight.} 
\label{fig:numerical_analysis}
\end{figure*}

\subsection{Experiments on Real Data}
\label{subsec:real_data}
\subsubsection{Self-Consistency Regularization (SCR)}

We experimented with perturbation-based consistency regularization on different stages of various prediction problems. We present these results in Table~\ref{tab:rel_ne}.
We observed a relative NE gain of approximately 0.1\%-0.3\%, depending on the prediction model tested in various offline experiments.
We will first present the results for the Retrieval stage from the offline experiments, followed by the experimental results for the Early and Final Stage ranker.

\noindent\textbf{Offline Retrieval Stage:} We have experimented with consistency regularization in two different models that predict conversion rate and click-through rate, respectively. 
We obtained the best results when regularizing both logit and representative of embedding with 0.15\%-0.2\% relative NE improvements.

\noindent\textbf{Offline Early Stage Ranking:} models are generally simpler ranking models compared to final stage models. Therefore, we applied regularization to the entire object and user embedding, resulting in a 0.3\% relative NE gain in various offline experiments.


\noindent\textbf{Offline Final Ranking Stage:} 
Models in this stage are generally much larger and complex compared to previous stages, as we are looking for more precise ranking of ads. 
We obtained the best results by regularizing both the logit and output of the embedding together.
Results from several experiments suggest an average 0.1\% relative NE gain, which has been further validated with online testing.


\begin{table}
  \caption{Relative NE gains for SCR across various stages.}
  \label{tab:rel_ne}
    \centering
  \begin{tabular}{cccc}
  \toprule
    Model & 33\% of data &  66\% of data & 100 \% data\\
    \midrule
    \texttt{Baseline} & 0 \% & 0 \% &  0 \% \\
    \textbf{Retrieval} & \bf{0.14 \%} & \bf{0.19} &  \bf{0.14 \%}\\
    \textbf{Early Stage} & \bf{0.28 \%} & \bf{0.3 \%} &  \bf{0.35 \%}\\
    \textbf{Final-stage Ranker} & \bf{0.1 \%} & \bf{0.08 \%} &  \bf{0.07 \%}\\
    \bottomrule
  \end{tabular}
\end{table}

\subsubsection{Loss-Balanced Small Perturbation Regularization (LSPR)}
We have explored LSPR primarily in the offline Final Ranker Stage, under various signal availability setups. We have observed that the technique has performed promisingly in various setups, ultimately leading to improved performance in each of these environments. These results are depicted in Table~\ref{tab:rel_ne2}.



\noindent\textbf{Offline Final Ranking Stage:} testing is very similar to Consistency Regularization testing; however, in the former, we perturbed the entire batch each time, leading to doubled batch size. 
In contrast, for Consistency Regularization, we only perturbed a small fraction of points in each batch.\\

\begin{table}
  \caption{Relative NE gains comparing SCR and LSPR on Final-stage Ranker.} 
  \label{tab:rel_ne2}
  \centering
  \begin{tabular}{cccc}
    \toprule
    Model & 33\% of data &  66\% of data & 100 \% data\\
    \midrule
   \texttt{Baseline} & 0 \% & 0 \% &  0 \% \\
    \textbf{SCR} & \bf{0.1\ \% } & \bf{0.08 \%} &  \bf{0.07 \%}\\
    \textbf{LSPR} & \bf{0.13 \% } & \bf{0.11 \%} &  \bf{0.1 \%}\\
    \bottomrule
  \end{tabular}
\end{table}


\subsubsection{Online Experiments}

We additionally have conducted online experimentation for a prediction model after testing it in offline setup.
The online experimentation is different than offline one in the nature that, it runs in continuous training and inferring routine, compared to full training and inferring mode.
% 
These online experiments on various data from different parts of the data stream, using both noisy and clean labels, have demonstrated a similar trend to the offline experiments we reported in the previous sections.
\newmaterial{
Our results indicate that LSPR has achieved a 0.1\% to 0.2\% relative improvement in online top-line metrics, consistently across multiple launches. Note that the magnitude of the impact is significant at the level of a billion-scale industrial production ads ranking system, which serves billions of users across various surfaces , across global geological locations, and across various clients. 
}

\subsection{Baselines}

\newmaterial{
The experiment comparisons in this manuscript are all compared against the latest production models in a multi-billion-scale industrial ads ranking system, prior to the adoption of LSPR. 
Our criteria for selecting baselines was to identify models that 1) have been proven to operate effectively at the industry scale; 2) represent the  state-of-the-art ads ranking product models  in the industry.
We consider these production-level recommendation models to be among the state-of-the-art baselines that meet the above criterion. 
}
