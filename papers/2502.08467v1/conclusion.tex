\section{Conclusion}%
Our analysis sheds light on a web security problem that has stayed in the dark, so far.
While detecting and preventing XSS vulnerabilities in front end code has been a prime topic of research, little attention has been paid to analyzing these issues in backends, largely due to a lack of appropriate tools for blind security testing.
Our approach to synthesizing polyglots fills this gap and provides the basis for the first large-scale study of XSS vulnerabilities in backend code in the Web.

The generation of polyglots, however, is not only a tool for research.
Our synthesis approach is flexible.
It can be narrowed down to specific vulnerabilities as well as expanded by supplementing additional test cases.
As a result, it provides a new and versatile instrument for web security that cannot only help investigate public-to-internal flows in backends but ultimately serve as a ``Swiss-army-knife'' for general vulnerability assessment of web applications.
