

Electroencephalogram (EEG) provides a non-invasive, highly accessible, and cost-effective solution for Alzheimer’s Disease (AD) detection. However, existing methods, whether based on manual feature extraction or deep learning, face two major challenges: the lack of large-scale datasets for robust feature learning and evaluation, and poor detection performance due to inter-subject variations. To address these challenges, we curate an EEG-AD corpus containing 813 subjects, which forms the world’s largest EEG-AD dataset to the best of our knowledge. Using this unique dataset, we propose \name, the first large foundation model for EEG-based AD detection. Our method encompasses an entire pipeline, from data selection and preprocessing to self-supervised contrastive pretraining, fine-tuning, and key setups such as subject-independent evaluation and majority voting for subject-level detection. We pre-train the model on 11 EEG datasets and unified fine-tune it on 5 AD datasets. Our self-supervised pre-training design includes sample-level and subject-level contrasting to extract useful general EEG features. Fine-tuning is performed on 5 channel-aligned datasets together. The backbone encoder incorporates temporal and channel embeddings to capture features across both temporal and spatial dimensions. Our method demonstrates outstanding AD detection performance, achieving up to a 9.86\% increase in F1 score at the sample-level and up to a 9.31\%  at the subject-level compared to state-of-the-art methods. The results of our model strongly confirm the effectiveness of contrastive pre-training and channel-aligned unified fine-tuning for addressing inter-subject variation. The source code is at \url{https://github.com/DL4mHealth/LEAD}.