



Alzheimer's disease (AD) is the most common neurodegenerative disorder in the elderly, affecting 10–30\% of individuals over the age of 65, with an annual incidence rate of 1–3\%~\cite{breijyeh2020comprehensive,masters2015alzheimer}. AD results from the failure to clear amyloid-$\beta$ peptide from the brain, leading to the progressive decline of cognitive functions. While there is currently no cure for AD, early intervention and treatment can slow the progression of symptoms, thereby improving patients' quality of life\cite{nelson2015slowing,chu2012alzheimer}. Existing detection tools such as Magnetic Resonance Imaging (MRI) and Positron Emission Tomography (PET) neuroimaging are costly and require specialized clinical expertise, often resulting in a detection only after significant symptoms have manifested. Recently, there has been growing interest in using non-invasive techniques, such as Electroencephalogram (EEG), to identify biomarkers for AD. EEG offers real-time brain activity data and is more cost-effective than traditional methods, making it a promising tool for early detection and continuous monitoring of disease progression~\cite{ieracitano2019convolutional}.



Currently, there are two main research directions for EEG-based Alzheimer's Disease (AD) detection. The first focuses on manually extracting feature biomarkers from the data, such as statistical features (e.g., Mean, Standard Deviation)\cite{tzimourta2019eeg, tzimourta2019analysis}, spectral features (e.g., Phase Shift, Phase Coherence)\cite{wang2017enhanced, cassani2014effects}, power features (e.g., Power Spectrum Density, Relative Band Power)\cite{fahimi2017index, schmidt2013index}, and complexity features (e.g., Shannon entropy, Tsallis Entropy)\cite{garn2015quantitative, azami2019multiscale}. Among these features, brain slowing in specific frequency bands is most commonly observed in existing research~\cite{abasolo2005analysis, fahimi2017index}. The second direction involves using deep learning methods for automatic feature extraction. Models such as convolutional neural networks~\cite{li2022predictive, cura2022deep}, graph neural networks~\cite{shan2022spatial, klepl2023adaptive}, and transformers~\cite{wang2024adformer} have been employed for representation learning. Some research also explores combining manual feature extraction with deep learning, such as extracting relative band powers and spectral coherence connectivity across different frequency bands and training convolutional networks on these extracted features~\cite{miltiadous2023dice}.


\begin{figure}
    \centering
    \includegraphics[width=1.0\linewidth]{figures/methods/lead_pipeline.pdf}
    \caption{\textbf{Pipeline of LEAD method.} 
    }
    \label{fig:lead_pipeline}
    \vspace{-5mm}
\end{figure}



However, the detection of Alzheimer's Disease (AD) using EEG remains an open challenge, facing difficulties from both application and theoretical perspectives. From an application perspective, large, high-quality datasets are scarce. The expense and complexity of collecting EEG-AD data lead to most studies involving a limited number of subjects, often no more than 50, and typically generate only thousands of 1-second samples if segmentation is applied~\cite{aviles2024machine}. Such "reinventing the wheel" with self-collected small datasets causes a significant waste of resources, considering the expense of collecting EEG data from AD patients. Furthermore, the relatively small datasets used in most existing research make it difficult to demonstrate the robustness of models, limiting the generalizability of findings. From a theory perspective, the subject-independent classification in EEG-AD detection is particularly challenging due to the inter-subject variance caused by subject features interference~\cite{wang2024medformer}. While subjects diagnosed with AD typically should exhibit consistent patterns related to the disease, subject features such as age, gender, or other personal factors may obscure these patterns. As a result, models may overfit to these subject features rather than capturing the AD patterns~\cite{wang2024evaluate}. This challenge is further complicated by the difficulty of interpreting EEG signals, even for experts. Unlike other domains, such as computer vision, where we can manually "remove" background features (e.g., in images), it is impossible to easily "remove" subject features from EEG data. This inherent difficulty hinders the development of methods that can effectively generalize to unseen subjects.



To address the challenges mentioned above, we propose \name, the world’s first \textbf{L}arge foundational\footnote{In this paper, we focus on downstream tasks for EEG-Based AD detection, but our pre-trained model on different neurological disease datasets can easily extend to other brain disease detection.} model for \textbf{E}EG-based \textbf{A}lzheimer’s \textbf{D}isease (AD) detection. We curate 9 EEG datasets for AD detection, both public and private, totaling 813 subjects(330 from public datasets and 483 from private sources), aiming to provide a comprehensive resource for training and evaluating detection models. \textbf{While this corpus may be relatively small compared to datasets in domains such as computer vision and natural language processing, it remains the largest EEG-based AD detection corpus to the best of our knowledge}. Our approach includes a full detection pipeline, including dataset selection, data preprocessing(e.g., channel and frequency alignment), self-supervised contrastive pre-training, and unified fine-tuning. We also introduce essential setups like subject-independent evaluation and majority voting for subject-level detection. The channel alignment in data preprocessing aligns all datasets into 19 standard channels, allowing us to train on different datasets. We pre-train our model on 11 datasets, which consist of 4 AD datasets and 7 additional datasets of other neurological diseases and healthy controls, including datasets for conditions like epilepsy and Parkinson's disease. This results in 2,354 subjects and 1,165,361 1-second, 128Hz samples. Our self-supervised learning design includes both sample-level and subject-level contrastive learning tasks. These tasks aim to do sample and subject discrimination, allowing the model to learn diverse EEG features that help minimize the interference of subject features in downstream tasks. We perform unified fine-tuning of the model in one run on 5 AD datasets to classify AD patients and healthy subjects, totaling 615 subjects and 223,039 1-second, 128Hz samples. We use the backbone that embeds cross-channel patches and the entire channel in parallel, capturing temporal and spatial features.

The final subject-level classification results for the 5 AD datasets—ADFTD, BrainLat, CNBPM, Cognision-ERP, and Cognision-rsEEG—are 91.34\%, 89.98\%, 100.00\%, 84.42\%, and 91.86\%, respectively. We compare \name with state-of-the-art (SOTA) methods, including fully supervised, self-supervised, and EEG foundational model methods. Our results demonstrate significant improvements, with up to a 9.86\% increase in F1 score at the sample level and up to a 9.31\% improvement at the subject level, compared to SOTA methods. We also conduct a detailed ablation study to evaluate the impact of pre-training modules, the benefit of AD and non-AD datasets, and various training setups. Additionally, we provide supplementary studies on brain interpretability, including channel importance and frequency band analysis. \textbf{Related works} for EEG-based AD detection and self-supervised learning in EEG are in Appendix~\ref{sec:related_work}.




We summarize our main contributions here: 
\vspace{-10mm}
\begin{itemize}[noitemsep, nolistsep]
    \item We present \name, the world’s first large foundational model for EEG-based AD detection, including a comprehensive method pipeline.
    \item We construct the world’s largest EEG-based AD detection corpus, consisting of 9 datasets with 813 subjects.
    \item Our strong performance validates the effectiveness of subject-level contrastive pre-training and unified fine-tuning for EEG-based AD detection.
    \item We release our code and model checkpoints to break the isolation in the EEG-based AD detection domain and facilitate future research. 
\end{itemize}


