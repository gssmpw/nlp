\section{Annotator Details}
\subsection{Guidelines for Annotators}
The annotation process involves generating a travel plan that is both feasible and, if possible, optimal. The annotated plan must be based on reference information while considering constraints such as local preferences (e.g., cuisine type, attraction category) and traveler personas (e.g., laidback, economical). Additionally, common sense should be applied when selecting points of interest, and deviations from suggested durations or costs must be justified. A detailed breakdown of these annotation guidelines, including priority handling, public transit considerations, and documentation requirements, is provided in Table \ref{tab:annotation_guidelines}.  

\begin{figure}[h]
    \centering
    \includegraphics[width=0.8\linewidth]{Images/age_dist.png}
    \caption{Age Distribution of our graduate student annotators.}
    \label{fig:age_distribution}
\end{figure}


\subsection{Annotator Demographics}

The annotator demographics, as illustrated by the figures, show a diverse range of experience levels and backgrounds. The years of English education vary significantly, with a distribution indicating that most annotators have between 12 to 20 years of formal English instruction (Figure \ref{fig:eng_edu_distribution}). This suggests a predominantly well-educated group with proficiency in the language. Additionally, the age distribution skews toward early-to-mid adulthood, with a concentration around 22 to 25 years old (Figure \ref{fig:age_distribution}), implying that most annotators are young graduate students. The gender distribution reflects participation from a diverse range of genders among the 25 graduate students. (Figure \ref{fig:gender_distribution}). 

\begin{figure}[h]
    \centering
    \includegraphics[width=0.8\linewidth]{Images/gender_dist.png}
    \caption{Gender Distribution of our graduate student annotators.}
    \label{fig:gender_distribution}
\end{figure}

\begin{figure}[h]
    \centering
    \includegraphics[width=0.8\linewidth]{Images/eng_edu_dist.png}
    \caption{Years of Formal English Education Statistics of our graduate student annotators.}
    \label{fig:eng_edu_distribution}
\end{figure}

The combination of these factors suggests that the dataset is annotated by individuals with substantial English proficiency and cognitive maturity, contributing to reliable and contextually aware annotations.


\FloatBarrier

\begin{table*}
    \centering
    \renewcommand{\arraystretch}{1.2}
    \begin{tabular}{|c|p{13cm}|}
        \hline
        \textbf{\#} & \textbf{Annotation Guideline} \\
        \hline
        1 & The goal of annotation is to generate a feasible travel plan that meets the query requirements. If multiple valid plans exist, selecting the most optimal plan is encouraged. \\
        \hline
        2 & Every element in the annotated travel plan must be grounded in reference data relevant to the given query. No fabricated or unverifiable information should be included. \\
        \hline
        3 & Common sense should be maintained when selecting travel plans. Refer Table \ref{tab:full_const_detail}. \\
        \hline
        4 & Any local constraints specified in the query must be respected. These may include preferences for cuisine types (e.g., Indian, Mediterranean) or attraction categories (e.g., Sights \& Landmarks, Zoos \& Aquariums). \\
        \hline
        5 & The travel plan should align with the traveler’s persona. For instance, a laidback traveler would prefer a schedule with 1-2 attractions per day, even if more options exist. An economical traveler would favor budget-friendly choices over expensive alternatives. \\
        \hline
        6 & If there is a conflict between local constraints and traveler persona preferences, the local constraints must be prioritized. If no valid plan can be formed while satisfying local constraints, a justification must be provided in Remarks. \\
        \hline
        7 & When selecting Points of Interest (PoIs), priority should be given to those with a public transit stop within 5km. If choosing between a PoI that meets local constraints but lacks transit access and one that is transit-friendly but does not meet constraints, the former should be preferred—unless a better alternative exists. \\
        \hline
        8 & The values for visit duration at attractions and average cost at restaurants should be treated as reference values rather than strict limits. Annotators are allowed to adjust these values if necessary but must document any deviations in the Remarks section. \\
        \hline
        9 & Annotators should use their best judgment to ensure that the generated plans are practical and reasonable. Any significant decision-making considerations should be explicitly noted in the Remarks section. \\
        \hline
    \end{tabular}
    \caption{Guidelines for Annotation of Travel Plans and Remarks}
    \label{tab:annotation_guidelines}
\end{table*}
