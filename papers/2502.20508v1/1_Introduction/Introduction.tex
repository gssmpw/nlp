\section{Introduction}

\begin{figure*}[t]
    \centering
    \includegraphics[width=\textwidth]{Images/overview_6-cropped.pdf}  % Change filename accordingly
    \caption{\tripcraft \;overview: Continuous evaluation metrics along-with fine-grained itinerary details.}
    \label{fig:main_framework}
\end{figure*}


Large Language Models (LLMs) have demonstrated remarkable capabilities in structured reasoning and decision-making \citep{wei2022chain, yao2023reactsynergizingreasoningacting}, making them promising candidates for automated planning tasks. Recently, LLMs have been explored to generate travel itineraries, where the goal is to create coherent, personalized, and logistically sound travel plans. However, existing benchmarks such as \textit{TravelPlanner} \citep{xie2024travelplanner} and \textit{TravelPlanner+} \cite{singh-etal-2024-personal} suffer from several limitations that restrict their real-world applicability. First, these datasets rely heavily on semi-synthetic data, leading to spatial inconsistencies. For instance, TravelPlanner assigns accommodations in New York across 312 different cities, causing unrealistic itinerary generation. Second, key real-world travel constraints, such as public transit schedules, event availability, attraction categories, and user preferences—are either missing or insufficiently modeled~\cite{xie2024travelplanner, chen2024travelagentaiassistantpersonalized, singh-etal-2024-personal}. While TravelPlanner+ incorporates user personas, it lacks detailed modeling of persona-driven attraction selection, travel pace, and scheduling preferences, limiting its ability to produce  customized plans. %As a result, current benchmarks fail to produce highly detailed, contextually coherent travel plans.

%These datasets often rely on outdated sources, introduce inconsistencies in spatial data, and lack the granularity needed for high-quality travel planning. For instance, a major limitation of \textit{TravelPlanner} is its reliance on semi-synthetic data, leading to spatial inconsistencies such as assigning accommodations in New York to 312 different cities. Furthermore, prior datasets \cite{xie2024travelplanner, chen2024travelagentaiassistantpersonalized, singh-etal-2024-personal} also neglect crucial factors like public transit, event schedules, and attraction categories—key elements for realistic itinerary generation.

% One critical shortcoming of \textit{TravelPlanner} is its heavy reliance on semi-synthetic data. For instance, it %randomly 
% assigns accommodations in New York to 312 different cities, %discarding
% without taking into account the latitude and longitude information necessary for spatial consistency. Furthermore, previous datasets \cite{xie2024travelplanner, chen2024travelagentaiassistantpersonalized, singh-etal-2024-personal} primarily overlook key aspects such as public transit connectivity, event schedules, and attraction subcategories, which are essential for a comprehensive and realistic travel itinerary.

% To address these challenges, we introduce \tripcraft, a novel travel planning benchmark constructed entirely from real-world web-scraped data. Unlike prior works, \tripcraft \; maintains spatial consistency, incorporates public transit stop information, and includes diverse event types (e.g., concerts, art exhibitions, and theater performances), making itineraries more holistic and personalized. Our dataset also enhances personalization by integrating user personas, similar to \textit{TravelPlanner+}, while improving temporal coherence by modeling time spent at each point of interest, inspired by what has been done in \textit{TravelAgent} \cite{chen2024travelagentaiassistantpersonalized}. Our approach enhances planning granularity by incorporating attraction categories, event participation, and transit logistics, going beyond existing methods.

% Our dataset comprises \textbf{1000 detailed travel queries} spanning 140 cities of the United States of America. Each query corresponds to either a 3-day, 5-day, or 7-day travel plan, with gold-standard itineraries annotated by 20 human annotators through multiple rounds of refinement. To further enhance interpretability, annotators provide detailed remarks explaining their decision-making process, recognizing that multiple valid itineraries can exist for a single query.


To address these limitations, we introduce \tripcraft, a benchmark for spatio-temporally fine-grained travel planning. Unlike prior datasets, \tripcraft \;is constructed entirely from real-world data sources, ensuring geographic consistency, valid transit connectivity, and contextually accurate event and attraction information. It integrates public transit stops and schedules, enabling realistic transit-aware itinerary generation, and incorporates diverse attraction and event categories, such as concerts, museums, and sports, to enhance itinerary richness. Additionally, \tripcraft{} models user personas, capturing travel styles, budget preferences, and location affinities to enable a more personalized and contextually relevant trip planning experience.
%To address these issues, we introduce \tripcraft, a travel planning benchmark constructed entirely from real-world scraped data. Unlike previous works, \tripcraft \;ensures spatial consistency, integrates public transit stops, incorporates diverse events (e.g., concerts, art exhibitions, theater performances) and attraction categories, enhancing personalization and granularity. Our dataset also enhances personalization by integrating user personas, similar to \textit{TravelPlanner+}, while improving temporal coherence by modeling time spent at each point of interest, inspired by what has been done in \textit{TravelAgent} \cite{chen2024travelagentaiassistantpersonalized}.
Our dataset consists of \textbf{1000 travel queries} spanning 140 U.S. cities, covering 3-day, 5-day, and 7-day itineraries. Gold-standard plans are annotated by 25 human annotators through multiple refinement rounds, with detailed remarks to enhance interpretability, acknowledging multiple valid itineraries per query.


% Note that, evaluating LLM-generated plans is inherently challenging, as existing constraint-checking approaches rely on discrete binary evaluations, such as determining if a constraint is satisfied or not \cite{xie2024travelplanner} or whether one plan is preferable to another \cite{singh-etal-2024-personal}. However, these methods fail to capture nuances in temporal and spatial coherence. To bridge this gap, we introduce five novel continuous evaluation metrics—Temporal Meal Score, Temporal Attraction Score, Persona Score, Spatial Score, and Ordering Score—that assess itinerary quality along multiple dimensions. The temporal meal score measures the alignment of restaurant visit times with an ideal distribution, modeled using a bivariate Gaussian distribution \cite{10.1007/978-3-642-41398-8_15, ansari2024chronos} based on annotated data. The temporal attraction score assesses the feasibility of attraction visits using a joint mixed random distribution that combines continuous duration variables with discrete counts of visited attractions. The spatial score quantifies the logical proximity of transit stops to points of interest (PoIs) within the itinerary, ensuring feasible and efficient travel between locations. The ordering score evaluates whether the sequence of visits follows a reasonable and effective order, minimizing unnecessary detours. Finally, the persona alignment score determines how well the generated plan reflects the traveler’s stated preferences, ensuring a tailored experience.


Evaluating LLM-generated travel plans is challenging, as existing methods rely on binary checks—whether constraints are met or if one plan is better than the other—without capturing finer details. \textit{But what truly makes a travel plan effective? Is it ensuring that meal schedules align with natural times? Does each attraction visit allow enough time for meaningful exploration, or is it unrealistically short or excessively long? Is travel between locations efficient, minimizing unnecessary detours and delays? Does the itinerary follow a logical sequence, or does it jump between distant places without consideration for practicality? Most importantly, does the plan align with personal preferences—catering to adventure seekers, cultural enthusiasts, or those looking for relaxation?}
%Evaluating LLM-generated travel plans is challenging, as existing methods rely on binary checks—whether constraints are met or if one plan is better—without capturing finer details. \textit{But what truly makes a travel plan effective? Is it ensuring that meal schedules align with natural times? That attraction visits are feasible and suited to the traveler’s pace? Is it about making travel efficient, with a logical sequence of visits? Most importantly, does the plan align with personal preferences—catering to adventure seekers, cultural enthusiasts, or those looking for relaxation?}
To systematically assess these aspects, we propose five novel evaluation metrics: \textit{Temporal Meal Score}, ensuring natural meal scheduling; \textit{Temporal Attraction Score}, evaluating visit durations; \textit{Spatial Score}, measuring travel efficiency; \textit{Ordering Score}, assessing logical sequencing; and \textit{Persona Score}, capturing alignment with user preferences.
%To address these gaps, we introduce five novel evaluation metrics: Temporal Meal Score, Temporal Attraction Score, Spatial Score, Ordering Score, and Persona Score. The Temporal Meal Score ensures restaurant visits align with natural meal times. The Temporal Attraction Score checks if each visit has sufficient time and if the daily attraction count matches the traveler’s style. The Spatial Score evaluates efficient travel between locations, while the Ordering Score ensures a logical visit sequence. Lastly, the Persona Score measures alignment with the traveler’s preferences. 
These metrics move beyond binary validation \cite{xie2024travelplanner, singh-etal-2024-personal}, providing a continuous and explainable framework for assessing the quality of itineraries. Fig.~\ref{fig:main_framework} shows an overview of the proposed \tripcraft\; benchmark.
In a nutshell, our key contributions are three-fold:
\begin{enumerate}
    \item \textbf{Fine-grained travel planning dataset: } \tripcraft \; leverages real-world data to eliminate geographic inconsistencies, ensuring coherent plans. Our dataset incorporates detailed attraction types, persona-based preferences, event categories, and public transit information, making it significantly richer than prior benchmarks.
    
    \item \textbf{Highly detailed itinerary generation: }Our approach generates travel itineraries with enhanced spatio-temporal granularity, producing a comprehensive and contextually coherent sequence of Points of Interest (PoIs) that provides a holistic representation of the journey. \item \textbf{ Novel continuous evaluation metrics: } To the best of our knowledge, this is the first ever attempt to introduce continuous evaluation metrics for LLM-generated travel plans, moving beyond binary constraints to assess itinerary quality with greater nuance.
\end{enumerate}

%\end{itemize}

% \begin{itemize}
%     \item \textbf{Fine-grained travel planning dataset:} \textit{TripCraft} leverages real-world data to eliminate geographic inconsistencies, ensuring coherent plans. Our dataset incorporates detailed attraction types, persona-based preferences, event categories, and public transit information, making it significantly richer than prior benchmarks.
    
%     \item \textbf{More detailed itinerary generation:} Our approach generates travel itineraries with enhanced spatio-temporal granularity, producing a comprehensive and contextually coherent sequence of Points of Interest (PoIs) that provides a holistic representation of the journey.
    
%     \item \textbf{Novel continuous evaluation metrics:} To the best of our knowledge, this is the first attempt to introduce continuous evaluation metrics for LLM-generated travel plans, moving beyond binary constraints to assess itinerary quality with greater nuance. Notably, our parameter-informed setting significantly enhances meal scheduling, boosting the temporal meal score from 61\% to 80\% in the 7-day scenario—a 19\% relative gain over the baseline.
% \end{itemize}


By addressing fundamental limitations in prior datasets and evaluation strategies, \tripcraft \; sets a new standard for benchmarking LLMs in travel planning, enabling more detailed, explainable, and personalized itinerary generation.

