\section{Data Sourcing Details}
Our dataset is constructed using current data sources to ensure spatio-temporal consistency and personalization. Below, we detail the sourcing methodology and heuristics for each component:

\subsection{Restaurants}  
We extracted restaurant details using \textbf{TripAdvisor’s Apify scraper}\footnote{\url{https://console.apify.com/actors/dbEyMBriog95Fv8CW/input}}, which provided all necessary attributes except precise pricing. TripAdvisor denotes cost using dollar symbols (\$–\$\$\$) instead of exact values. To estimate absolute prices, we leveraged city-specific restaurant price indices from \textbf{Numbeo}\footnote{\url{https://www.numbeo.com/cost-of-living/}}, scaling them according to the number of dollar symbols in each price rating.  

\subsection{Attractions}  
Attraction details, including subcategories, were sourced from \textbf{TripAdvisor’s Apify scraper}\footnote{\url{https://console.apify.com/actors/dbEyMBriog95Fv8CW/input}}. Since a majority of attractions lacked predefined visit durations, we consulted domain experts to establish category-wise average durations for each attraction type. Finally, each attraction’s duration was assigned as the mean of the categories it belonged to, ensuring a realistic time allocation (Table \ref{tab:subcategory_duration}).

\subsection{Flights}  
We adopted the \cite{xie2024travelplanner} flight database but adjusted all dates to November 2024 to maximize temporal alignment with event data. This adjustment ensures that LLM-generated itineraries incorporate relevant event-based recommendations.  

\subsection{Distance Matrices}  
All pairwise distances were computed using \textbf{OpenStreetMap’s OSRM API}\footnote{\url{http://project-osrm.org/}}, ensuring accurate and real-time routing information.  

\subsection{Accommodations}  
We scraped accommodation listings from Airbnb using \textbf{Apify’s Airbnb scraper}\footnote{\url{https://apify.com/dtrungtin/airbnb-scraper}}. Since minimum stay requirements were not available in the extracted data, we excluded this attribute from our dataset.  

\subsection{Events}  
Event data was collected using \textbf{Ticketmaster’s Apify scraper}\footnote{\url{https://console.apify.com/actors/Hi7bNMx0vqaqvdfZQ}}, covering a diverse range of concerts, sports, theater, and other entertainment events.  

\subsection{Public Transit}  
We sourced transit schedules from the \textbf{General Transit Feed Specification (GTFS)}\footnote{\url{https://gtfs.org/}} for 140 cities. For each Point of Interest (PoI)—including accommodations, restaurants, and attractions—we determined the nearest public transit stop using geodesic distance (computed via \textbf{Geopy}). This enables LLMs to incorporate realistic public transit connectivity when generating travel itineraries. 

\begin{table}[h]
    \centering
    \renewcommand{\arraystretch}{0.9}
    \setlength{\tabcolsep}{8pt} % Adjust column spacing
    \begin{tabular}{l c}
        \toprule
        \textbf{Category} & \textbf{Duration (hrs)} \\
        \midrule
        Boat Tours \& Water Sports & 3.5 \\
        Casinos \& Gambling & 2.5 \\
        Classes \& Workshops & 1.5 \\
        Concerts \& Shows & 2.5 \\
        Food \& Drink & 2.5 \\
        Fun \& Games & 1.5 \\
        Museums & 3.0 \\
        Nature \& Parks & 4.5 \\
        Nightlife & 2.5 \\
        Outdoor Activities & 4.0 \\
        Shopping & 1.5 \\
        Sights \& Landmarks & 3.0 \\
        Spas \& Wellness & 2.0 \\
        Water \& Amusement Parks & 5.0 \\
        Zoos \& Aquariums & 2.5 \\
        \bottomrule
    \end{tabular}
    \caption{Attraction visiting duration (hrs) for each category. Note that an attraction can belong to one or more than one categories.}
    \label{tab:subcategory_duration}
\end{table}



\begin{table*}[h]
    \centering
    \renewcommand{\arraystretch}{1.05}
    \begin{tabular}{|p{4cm}|p{11cm}|}
        \hline
        \textbf{Constraint} & \textbf{Description} \\
        \hline
        \multicolumn{2}{|c|}{\cellcolor{gray!25} \textit{Environment Constraint}} \\
        \hline \rule{0pt}{2.5ex}Unavailable Transportation & There is no available flight or driving information between the two cities.  \\
        % \hline
        Unavailable Attractions & There is no available attraction information in the queried city. \\
        \hline
        \multicolumn{2}{|c|}{\cellcolor{gray!25} \textit{Commonsense Constraint}} \\
        \hline \rule{0pt}{2.5ex}Within Sandbox & All information in the plan must be within the closed sandbox; otherwise, it will be considered a hallucination. \\
        Complete Information & No key information should be left out of the plan, such as the lack of accommodation during travel. \\
        Sufficient Meal Gaps (+) & Meal timings must have a minimum
        gap of four hours between breakfast, lunch, and
        dinner to maintain a natural schedule. \\
        Valid PoI list (+) & The
        point-of-interest (PoI) list must follow strict validity rules: each day’s itinerary must begin and end at the designated accommodation, except on the final day when the traveler departs. The list should be limited to accommodations, attractions, and restaurants, ensuring adequate time gaps between flight arrivals and accommodation check-ins, as well as between accommodation check-outs and departures. \\
        Diverse Events (+) & Event choices should not be repeated throughout the trip. \\
        Within Current City & All scheduled activities for the day must be located within that day’s city(ies). \\
        Reasonable City Route & Changes in cities during the trip must be reasonable. \\
        Diverse Restaurants & Restaurant choices should not be repeated throughout the trip. \\
        Diverse Attractions & Attraction choices should not be repeated throughout the trip. \\
        Non-conf. Transportation & Transportation choices within the trip must be reasonable. For example, having both “self-driving” and “flight” would be considered a conflict. \\
        \hline
        \multicolumn{2}{|c|}{\cellcolor{gray!25} \textit{Hard Constraint}} \\
        \hline \rule{0pt}{2.5ex}Budget & The total budget of the trip. \\
        Room Rule & Room rules include “No parties”, “No smoking”, “No children under 10”, “No pets”, and “No visitors”. \\
        Room Type & Room types include “Entire Room”, “Private Room”, “Shared Room”, and “No Shared Room”. \\
        Cuisine & Cuisines include “Chinese”, “American”, “Italian”, “Mexican”, “Indian”, “Mediterranean”, and “French”. \\
        Transportation & Transportation options include “No flight” and “No self-driving”. \\
        Event Types (+) & Event Types include four distinct categories—Sports, Arts \& Theatre, Music, and Film. \\
        Attraction Types (+) &  Each attraction belongs to one or more of 15 predefined categories, ensuring a well-distributed selection of activities. \\
        \hline
        \multicolumn{2}{|c|}{\cellcolor{gray!25} \textit{Persona Components}} \\
        \hline \rule{0pt}{2.5ex}Traveler Type (+) & Defines how a traveler approaches their journey—whether they seek relaxation in cozy spots or adrenaline-pumping adventures. \\
        Purpose of Travel (+) &  Captures the main motivation behind the trip, whether it’s to unwind, seek thrills, explore cultures, or connect with nature. \\
        Spending Preference (+) &  Reflects the traveler’s budget and style, from luxurious indulgence to cost-conscious experiences. \\
        Location Preference (+) &  Highlights preferred environments, such as beaches, mountains, cities, or wildlife-rich forests. \\
        \hline
    \end{tabular}
    \caption{\textit{Comprehensive Constraint and Persona Description. (+) denotes the ones we have added.} }
    \label{tab:full_const_detail}
\end{table*}

\begin{table*}[h]
    \centering
    \renewcommand{\arraystretch}{1}
    \setlength{\tabcolsep}{2pt} % Adjust column spacing
    \begin{tabularx}{\columnwidth}{l *{3}{>{\centering\arraybackslash}X}}
        \toprule
        \textbf{Parameter} & \textbf{3-day} & \textbf{5-day} & \textbf{7-day} \\
        \midrule
        \multicolumn{4}{c}{\textbf{Restaurant Parameters}} \\
        \midrule
        \textbf{Breakfast} & & & \\
        Mean Time & 9.63 & 9.80 & 9.84 \\ 
        Mean Duration (hrs) & 0.90 & 1.08 & 0.85 \\ 
        Std. Time & 1.08 & 1.08 & 1.34 \\ 
        Std. Duration (hrs) & 0.24 & 1.43 & 0.23 \\ 
        Beta & 0.21 & 0.63 & 0.03 \\ 
        \midrule
        \textbf{Lunch} & & & \\
        Mean Time & 14.30 & 14.46 & 14.44 \\ 
        Mean Duration (hrs) & 1.11 & 1.10 & 0.99 \\ 
        Std. Time & 1.03 & 1.07 & 1.07 \\ 
        Std. Duration (hrs) & 0.36 & 0.35 & 0.26 \\ 
        Beta & 0.10 & 0.04 & 0.30 \\ 
        \midrule
        \textbf{Dinner} & & & \\
        Mean Time & 20.75 & 20.67 & 20.42 \\ 
        Mean Duration (hrs) & 1.19 & 1.32 & 1.15 \\ 
        Std. Time & 1.25 & 1.37 & 1.66 \\ 
        Std. Duration (hrs) & 0.43 & 0.91 & 1.15 \\ 
        Beta & -0.20 & -0.18 & -0.07 \\ 
        \midrule
        \multicolumn{4}{c}{\textbf{Attraction Parameters}} \\
        \midrule
        $\lambda_{laidback}$ & 1.10 & 1.26 & 1.11 \\ 
        $\lambda_{adventurous}$ & 2.01 & 1.61 & 1.82 \\ 
        $\sigma_d$ (hrs) & 1.11 & 1.07 & 0.90 \\ 
        $n^{max}$ & 5 & 4 & 4 \\ 
        $n^{min}$ & 0 & 0 & 0 \\ 
        $k$ (hrs) & 0.28 & 0.28 & 0.28 \\ 
        \bottomrule
    \end{tabularx}
    \caption{A comprehensive list of parameter details for 3-day, 5-day, and 7-day scenarios as calculated from the annotation distribution statistics.}
    \label{tab:parameter_details}
\end{table*}



