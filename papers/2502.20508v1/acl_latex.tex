% This must be in the first 5 lines to tell arXiv to use pdfLaTeX, which is strongly recommended.
\pdfoutput=1
% In particular, the hyperref package requires pdfLaTeX in order to break URLs across lines.

\documentclass[11pt]{article}

% Change "review" to "final" to generate the final (sometimes called camera-ready) version.
% Change to "preprint" to generate a non-anonymous version with page numbers.
% \usepackage[review]{acl}
\usepackage[preprint]{acl}

% Standard package includes
\usepackage{times}
\usepackage{latexsym}

% For proper rendering and hyphenation of words containing Latin characters (including in bib files)
\usepackage[T1]{fontenc}
% For Vietnamese characters
% \usepackage[T5]{fontenc}
% See https://www.latex-project.org/help/documentation/encguide.pdf for other character sets

% This assumes your files are encoded as UTF8
\usepackage[utf8]{inputenc}

% This is not strictly necessary, and may be commented out,
% but it will improve the layout of the manuscript,
% and will typically save some space.
\usepackage{microtype}

% This is also not strictly necessary, and may be commented out.
% However, it will improve the aesthetics of text in
% the typewriter font.
\usepackage{inconsolata}

%Including images in your LaTeX document requires adding
%additional package(s)
\usepackage{graphicx}
\usepackage{amsmath}
\usepackage{amsfonts}
\usepackage{bm}
% \usepackage[table]{xcolor}
\usepackage{array}
\usepackage{booktabs}
\usepackage{multirow}
\usepackage{tabularx}
\usepackage{ulem}
\usepackage{float}

% For creating boxes
\usepackage[most]{tcolorbox}
%\usepackage{txfonts}
% For Appendix tables
\usepackage{longtable}
\usepackage{colortbl}
\usepackage{arydshln}
\usepackage{makecell}
\usepackage{placeins}
% If the title and author information does not fit in the area allocated, uncomment the following
%
%\setlength\titlebox{<dim>}
%
% and set <dim> to something 5cm or larger.
%\newcommand{\tripcraft}%{\texttt{TripCraft}}
\newcommand{\tripcraft}
{\textit{TripCraft}}
\definecolor{lightgray}{gray}{0.9}
\definecolor{highlightyellow}{rgb}{1,1,0.6}

% \title{TripCraft: Towards Spatio-Temporally Fine Grained Travel Planning using Large Language Models}

\title{\tripcraft: A Benchmark for Spatio-Temporally Fine Grained Travel Planning}


% \title{TripCraft: A Benchmark for spatio-temporally fine grained travel planning using GPT-4o}

% Author information can be set in various styles:
% For several authors from the same institution:
% \author{Author 1 \and ... \and Author n \\
%         Address line \\ ... \\ Address line}
% if the names do not fit well on one line use
%         Author 1 \\ {\bf Author 2} \\ ... \\ {\bf Author n} \\
% For authors from different institutions:
% \author{Author 1 \\ Address line \\  ... \\ Address line
%         \And  ... \And
%         Author n \\ Address line \\ ... \\ Address line}
% To start a separate ``row'' of authors use \AND, as in
% \author{Author 1 \\ Address line \\  ... \\ Address line
%         \AND
%         Author 2 \\ Address line \\ ... \\ Address line \And
%         Author 3 \\ Address line \\ ... \\ Address line}

% \author{First Author \\
%   Affiliation / Address line 1 \\
%   Affiliation / Address line 2 \\
%   Affiliation / Address line 3 \\
%   \texttt{email@domain} \\\And
%   Second Author \\
%   Affiliation / Address line 1 \\
%   Affiliation / Address line 2 \\
%   Affiliation / Address line 3 \\
%   \texttt{email@domain} \\}

% \author{
%  \textbf{Soumyabrata Chaudhuri\textsuperscript{1}},
%  \textbf{Pranav Ravindra Purkar\textsuperscript{1}},
%  \textbf{Ritwik Raghav\textsuperscript{1}},
%  \textbf{Shubhojit Mallick\textsuperscript{2}},
% \\
%  \textbf{Manish Gupta\textsuperscript{2}},
%  \textbf{Abhik Jana\textsuperscript{1}},
%  \textbf{Shreya Ghosh\textsuperscript{1}},
% %  \textbf{Eighth Author \textsuperscript{1,2,3,4}},
% % \\
% %  \textbf{Ninth Author\textsuperscript{1}},
% %  \textbf{Tenth Author\textsuperscript{1}},
% %  \textbf{Eleventh E. Author\textsuperscript{1,2,3,4,5}},
% %  \textbf{Twelfth Author\textsuperscript{1}},
% % \\
% %  \textbf{Thirteenth Author\textsuperscript{3}},
% %  \textbf{Fourteenth F. Author\textsuperscript{2,4}},
% %  \textbf{Fifteenth Author\textsuperscript{1}},
% %  \textbf{Sixteenth Author\textsuperscript{1}},
% % \\
% %  \textbf{Seventeenth S. Author\textsuperscript{4,5}},
% %  \textbf{Eighteenth Author\textsuperscript{3,4}},
% %  \textbf{Nineteenth N. Author\textsuperscript{2,5}},
% %  \textbf{Twentieth Author\textsuperscript{1}}
% \\
% \\
%  \textsuperscript{1}IIT Bhubaneswar,
%  \textsuperscript{2}Microsoft, India
% \\
%  \small{
%    \textbf{Correspondence:} \href{mailto:email@domain}{email@domain}
%  }
% }
\author{
    \begin{tabular}{c}
        \textbf{Soumyabrata Chaudhuri\textsuperscript{1}},  
        \textbf{Pranav Purkar\textsuperscript{1}},  
        \textbf{Ritwik Raghav\textsuperscript{1}},  
        \textbf{Shubhojit Mallick\textsuperscript{2}} \\  
        \textbf{Manish Gupta\textsuperscript{2}},  
        \textbf{Abhik Jana\textsuperscript{1}},  
        \textbf{Shreya Ghosh\textsuperscript{1}}  
    \end{tabular}  
    \\
    \textsuperscript{1}School of Electrical and Computer Sciences, IIT Bhubaneswar, India \\  
    \textsuperscript{2}Microsoft, India \\  
    \begin{tabular}{c}  
        \texttt{chaudhurisoumyabrata@gmail.com} \\  
        \texttt{\{23cs06011,a23cs09001,abhikjana,shreya\}}@iitbbs.ac.in \\  
        \texttt{\{shubhojit.mallick,gmanish\}}@microsoft.com  
    \end{tabular}  
}


\begin{document}
\maketitle
\begin{abstract}
Recent advancements in probing Large Language Models (LLMs) have explored their latent potential as personalized travel planning agents, yet existing benchmarks remain limited in real-world applicability. Existing datasets, such as TravelPlanner and TravelPlanner+, suffer from semi-synthetic data reliance, spatial inconsistencies, and a lack of key travel constraints, making them inadequate for practical itinerary generation.
%rely on semi-synthetic data as well as ignoring several key components of travel planning, limiting their real-world applicability. 
To address these gaps, we introduce \textbf{\tripcraft},  
a spatio-temporally coherent travel planning  dataset that integrates real-world constraints, including public transit schedules, event availability, diverse attraction categories, and user personas for enhanced personalization.
%incorporating real-world constraints, including public transit schedules, public events, varied attraction categories, and user personas for enhanced personalization. Our dataset enables more detailed trip itinerary generation (including duration spent at each point of interest based on users' persona, transit between two points of interest, etc.) while ensuring spatio-temporal consistency. 
%TripCraft enables the generation of more structured and contextually rich itineraries by ensuring spatial and temporal alignment, and persona-aware activity selection.
%Further, we propose novel evaluation metrics (temporal meal score, attraction score, spatial score, ordering score, and persona score) to assess LLM-generated plans across temporal, spatial, sequential, and personal dimensions, overcoming the limitations of existing commonsense and hard constraint metrics.
To evaluate LLM-generated plans beyond existing binary validation methods, we propose five continuous evaluation metrics, namely Temporal Meal Score, Temporal Attraction Score, Spatial Score, Ordering Score, and Persona Score—which assess itinerary quality across multiple dimensions.
Our parameter-informed setting significantly enhances meal scheduling, improving the Temporal Meal Score from 61\% to 80\% in a 7-day scenario.
%Interestingly, our parameter-informed setting significantly enhances meal scheduling, improving performance from \textit{61\% to 80\% in the 7-day scenario- as quantified by a 19\% gain in our temporal meal score}. 
\tripcraft \footnote{Dataset and Codebase will be made publicly available upon acceptance.} establishes a new benchmark for LLM-driven personalized travel planning, offering a more realistic, constraint-aware framework for itinerary generation.
\end{abstract}

% 
% 
The widespread integration of communication networks and smart devices in modern control systems has increased the vulnerability of industrial systems to online cyber-attacks, e.g., Industroyer, Blackenergy, etc \citep{osti_1505628}.
% Modern control systems have seen a large push to include communication networks and smart devices to increase performance, made possible by improvements in communication device cost and energy consumption. This trend has been coupled with the usage of open-standard communication protocols among industrial control systems, making them vulnerable to online cyber-attacks such as Industroyer, Blackenergy, etc \citep{osti_1505628}. 
To counter this, methods have been developed to improve security by achieving attack detection, mitigation, and monitoring, among others \citep{sandberg2022secure}. This paper focuses on active attack diagnosis to mitigate stealthy attacks. 
%
%\subsection{Literature review}

Active diagnosis techniques rely on the inclusion of additional moduli to control systems
% inclusion within the control system of additional moduli 
to alter the behavior of the system compared to information known by the attacker. 
For instance, the concept of additive watermarking was introduced in \cite{mo2015physical}, where noise signals of known mean and variance are added at the plant and compensated for it at the controller. 
This compensation, however, is not exact, causing some performance degradation. Thus, trade-offs between performance and detectability  are necessary \citep{zhu2023detection}.
% A later work \citep{zhu2023detection} designs the watermark signal by trading performance for detection. Thus, although additive watermarking serves as a good detection scheme, they endure performance losses even in the nominal case. 

In encrypted control \citep{darup2021encrypted}, the sensor data is encrypted, sent to the controller, and then operated on directly. Encrypted input signals are sent back to the plant for decryption. Although encryption is widespread in IT security, in control systems it presents some concerns, such as the introduction of time delays \citep{stabile2024verifiable}, while it may present inherent weaknesses \citep{alisic2023model}.
% they are not preferred as they introduce time delays \citep{stabile2024verifiable} which can cause instability, and some encryption schemes can be very weak  \citep{alisic2023model}. 

In moving target defense \citep{griffioen2020moving}, the plant is augmented with fictitious dynamics, known to the controller. The plant output is transmitted to the controller along with the fictitious states over a network under attack. 
The additional measurements then aide in the detection of attacks. 
This comes at the cost of higher communication bandwidth needs, which increases rapidly with the dimension of the augmented systems.
% Since the dynamics of the fictitious dynamics are exactly known to the controller, the attack is detected easily. However, when the scale of the system increases, the communication bandwidth used by moving the target defense approach increases rapidly. 

Other recently proposed works include two-way coding \citep{fang2019two}, a weak encryuption technique, and dynamic masking \citep{abdalmoaty2023privacy}, which enhances privacy as well as security, have been shown to be effective against zero-dynamics attacks.
% Two-way coding \citep{fang2019two} and dynamic masking \citep{abdalmoaty2023privacy} are other recently proposed approaches. Two-way coding is another form of weak encryption technique whilst dynamic masking proposes an architecture that enhances both privacy and security. These schemes are shown to be effective against zero dynamics attacks but remain to be studied for other classes of attacks. 
% Recent extensions include \citep{mukherjee2021secure,ramos2024privacy}.
% Some other works which are related are \citep{mukherjee2021secure}, an extension of \cite{fang2019two}. The work \citep{ramos2024privacy} is an extension of moving target defense for multi-agent systems. 
Furthermore, filtering techniques for attack detection are proposed by \cite{murguia2020security,hashemi2022codesign,escudero2023safety}, while not focusing on stealthy attacks.
% The works \citep{murguia2020security,hashemi2022codesign,escudero2023safety} develop filtering techniques to guarantee safety, without being focused on stealthy covert attacks.

Multiplicative watermarking (mWM) has been proposed by the authors as a diagnosis technique \citep{ferrari2020switching}. mWM consists of a pair of filters on each communication channel between the plant and its controller; the scheme is affine to weak encryption, whereby ``encoding'' and ``decoding'' are done by changing signals' dynamic characteristics through inverse pairs of filters. This enables original signals to be recovered exactly, and thus does not lead to performance degradation.
% A multiplicative watermark is an affine to a weak encryption technique, through which the signal is ``encoded'' by a filter, changing its dynamic behavior. The use of inverse pairs means that the original signal can be recovered, through ``decoding'' via an inverse filter. As such, differently to techniques based on additive watermarking, no performance is lost due to the injection of noise, and there are no bandwidth limitations.

%\subsection{Contributions}
One of the critical features of multiplicative watermarking is that to detect stealthy attacks, the mWM filter parameters must be switched over time. In this paper, an algorithm to optimally design the mWM parameters after a switching event is presented, enhancing detection performance, without changing the switching time.
% This is done without changing the switching time, which is taken as given.

\textcolor{black}{
To formalize the filter design problem, we suppose the defender is interested in optimal performance against adversaries injecting covert attacks with matched system parameters \citep{smith2015covert}, including the mWM parameters prior to the switch. This scenario represents a worst case where malicious agents can take full control of the system while remaining undetected.
Thus, the attack strategy is explicitly included within the formulation of the closed-loop system, and the mWM filters are chosen by solving an optimization problem minimizing the attack-energy-constrained output-to-output gain (AEC-OOG) \citep{anand2023risk}, a variation of the output-to-output gain proposed in  \cite{teixeira2015strategic}.
}
The main contributions of this paper are:
% We consider an adversary injecting a covert attack with matched system parameters \citep{smith2015covert}, i.e., an attacker with full knowledge of the control system parameters, including those of the mWM filters before the switch. This scenario is taken as a worst case, as it has been shown that this class of attacks can be made stealthy. To quantitatively define a cost, the output-to-output gain (OOG) \citep{teixeira2015strategic} is leveraged,
% a metric introduced to evaluate the impact of an additive attack in a control system. %Specifically, OOG evaluates the worst-case performance loss that an attacker injecting an undetectable attack can obtain. 
% Here, the maximum performance loss caused by a stealthy adversary with limited energy is taken, the attack-energy-constrained OOG (AEC-OOG) \citep{anand2023risk}. The main contributions of this paper are:
\begin{enumerate}
%[label=\alph*.]
\item The problem of optimally designing the switching mWM filters is formulated as an optimization problem, with the AEC-OOG is taken as the objective;%where the AEC-OOG is taken as the impact metric; 
\item The worst-case scenario of a covert attack with exact knowledge of plant and mWM filter parameters is embedded within the design problem;
% The optimization problem is defined to incorporate the worst-case scenario of a covert attack with exact knowledge of plant and mWM filter parameters;
\item The feasibility of the optimization problem is shown to be dependent only on stability conditions; 
\item A solution scheme is proposed to promote randomization of the mWM filter parameters such that an eavesdropping adversary cannot remain stealthy.
\end{enumerate} 

This builds on the results of \cite{ferrari2020switching}, where the focus was on the design of the switching protocols, rather than the parameters themselves.
Compared to previous work \citep{gallo2021design}, this paper introduces an optimization problem which is always feasible (thanks to the use of AEC-OOG in the objective), while also considering a more sophisticated class of covert attacks, where the presence of watermark is known to the adversary. 
Moreover, this paper poses a different objective than \citep{zhang2023hybrid}; indeed, while \citep{zhang2023hybrid} provided a design strategy to ensure certain privacy properties, in this paper we address the problem of optimal parameter design following a switching event.


%\subsection{Organization}
The rest of the paper is organized as follows. 
After formulating the problem in Section~\ref{sec:PF}, we propose our design algorithm in Section~\ref{sec:main}, and analyze its properties. It is then evaluated through a numerical example in Section~\ref{sec:NE}, and concluding remarks are given Section~\ref{sec:Con}.
% We provide the problem background in Section~\ref{sec:PF}. We formulate the design problem in Section~\ref{sec:main}, together with an analysis of its properties. The proposed algorithm is evaluated through a numerical example in Section \ref{sec:NE}. Concluding remarks are offered in Section \ref{sec:Con}.

\section{Related Work}
\subsection{Multimodal Large Language Models}
% Building on the success of large language models (LLMs) \citep{yao2024tree, glm2024chatglm, achiam2023gpt, touvron2023llama, brown2020language}, multimodal large language models (MLLMs) \citep{liu2024improved, li2023blip, zhu2023minigpt, wang2023cogvlm, liu2024visual} extend these capabilities by integrating vision and text processing, achieving remarkable performance in tasks involving images, videos, and multimodal reasoning. However, handling visual data poses computational challenges due to the redundancy and low information density of high-resolution tokens \citep{liang2022evit} and the quadratic scaling of attention mechanisms \citep{vaswani2017attention}.
% For instance, models like LLaVA \citep{liu2023improvedllava} and mini-Gemini-HD \citep{li2024mini} encode high-resolution images into thousands of tokens, while video-based models such as VideoLLaVA \citep{lin2023video} and VideoPoet \citep{kondratyuk2023videopoet} allocate even more tokens to process multiple frames. These challenges highlight the need for more efficient token representations and longer context lengths to enable scalability. Recent advancements, such as Gemini \citep{geminiteam2023gemini} and LWM \citep{liu2024world}, have focused on addressing these issues by optimizing token efficiency and extending the context length, paving the way for more scalable and effective MLLMs.

The remarkable success of large language models (LLMs) \citep{radford2019language, brown2020language} has spurred a growing trend of extending their advanced reasoning capabilities to multi-modal tasks, leading to the development of vision-language models (VLMs) \citep{huang2023languageneedaligningperception, driess2023palmeembodiedmultimodallanguage, liu2024visual, Qwen-VL}. These VLMs typically consist of a visual encoder \citep{radford2021learning} that serializes input image representations and an LLM responsible for text generation. To enable the LLM to process visual inputs, an alignment module is employed to bridge the gap between visual and textual modalities. This module can take various forms, such as a simple linear layer, an MLP projector, or a more complex query-based network. While this integration allows the LLM to gain visual perception, it also introduces significant computational challenges due to the long sequences of visual tokens.

Moreover, existing VLMs often exhibit limitations, such as visual shortcomings or hallucinations, which hinder their performance. Efforts to enhance VLM capabilities by increasing input image resolution have further exacerbated computational demands. For instance, encoding higher-resolution images results in a substantial increase in the number of visual tokens. A model like LLaVA-1.5 \citep{liu2024improved} generates 576 visual tokens for a single image, while its successor, LLaVA-NeXT \citep{liu2024llavanext}, produces up to 2880 tokens at double the resolution, far exceeding the length of typical textual prompts.
Optimizing the inference efficiency of VLMs is thus a critical task to facilitate their deployment in real-world scenarios with limited computational resources.

\subsection{Visual Token Compression}
% Visual tokens often exceed text tokens by tens to hundreds of times, with visual signals being more spatially redundant compared to information dense text \citep{marr2010vision}.
% Various methods have been proposed to address this issue. For instance, LLaMA-VID \citep{li2023llama} uses a Q-Former with context tokens, and DeCo \citep{yao2024deco} applies adaptive pooling to downsample visual tokens at the patch level.
% However, these approaches require modifying model components and additional training, increasing computational and training costs.
% ToMe~\citep{bolya2022tome} reduces tokens without training by adding a token merge module to ViTs, but this disrupts early cross-modal interactions in language models~\citep{xing2024PyramidDrop}. FastV~\citep{chen2024image} selects important visual tokens using attention scores, while SparseVLM~\citep{zhang2024sparsevlm} incorporates text guidance via cross-modal attention.
% However, these methods forgo flash-attention~\citep{dao2022flashattention, dao2023flashattention2} and primarily focus on token importance, overlooking the impact of token duplication.
% In our work, we preserve hardware acceleration compatibility, including flash attention, while considering both token importance and duplication for token reduction.

Visual tokens are often significantly more numerous than text tokens, with higher spatial redundancy and lower information density. To address this issue, various methods have been proposed for reducing visual token counts in vision language models. For instance, some approaches modify model components, such as using context tokens in Q-Former \citep{li2023llama} or applying adaptive pooling at the patch level, but these typically require additional training and increase computational costs. Other techniques, like Token Merging (ToMe) \citep{bolya2022tome} and FastV \citep{chen2024image}, focus on reducing tokens without retraining by merging tokens or selecting important ones based on attention scores. SparseVLM \cite{zhang2024sparsevlm} incorporates text guidance through cross-modal attention to refine token selection. However, these methods often overlook hardware acceleration compatibility and fail to account for token duplication alongside token importance. Furthermore, while token pruning has been extensively explored in natural language processing and computer vision to improve inference efficiency, its application to VLMs remains under-explored. Existing pruning strategies, such as those in FastV and SparseVLM, rely on text-visual attention within large language models (LLMs) to evaluate token importance, which may not align well with actual visual token relevance.



\renewcommand{\arraystretch}{1.3} % Adjust row height
\setlength{\tabcolsep}{6pt} % Adjust column spacin

% \section{\tripcraft \;}
\section{TripCraft}

% The first line of the file must be
% \begin{quote}
% \begin{verbatim}
% \documentclass[11pt]{article}
% \end{verbatim}
% \end{quote}

% To load the style file in the review version:
% \begin{quote}
% \begin{verbatim}
% \usepackage[review]{acl}
% \end{verbatim}
% \end{quote}
% For the final version, omit the \verb|review| option:
% \begin{quote}
% \begin{verbatim}
% \usepackage{acl}
% \end{verbatim}
% \end{quote}

% To use Times Roman, put the following in the preamble:
% \begin{quote}
% \begin{verbatim}
% \usepackage{times}
% \end{verbatim}
% \end{quote}
% (Alternatives like txfonts or newtx are also acceptable.)

% Please see the \LaTeX{} source of this document for comments on other packages that may be useful.

% Set the title and author using \verb|\title| and \verb|\author|. Within the author list, format multiple authors using \verb|\and| and \verb|\And| and \verb|\AND|; please see the \LaTeX{} source for examples.

% By default, the box containing the title and author names is set to the minimum of 5 cm. If you need more space, include the following in the preamble:
% \begin{quote}
% \begin{verbatim}
% \setlength\titlebox{<dim>}
% \end{verbatim}
% \end{quote}
% where \verb|<dim>| is replaced with a length. Do not set this length smaller than 5 cm.







\subsection{Overview}
% We introduce \tripcraft, a benchmark designed to evaluate the capability of language agents in complex, constraint-aware planning. Grounded in the real-world task of travel itinerary generation, \tripcraft \;assesses whether agents can construct flexible travel plans while adhering to diverse constraints, including user preferences, commonsense considerations, and persona-driven requirements. The benchmark further evaluates temporal, spatial, sequential, and personal consistency in generated plans.

% \tripcraft \;comprises $1000$ diverse travel queries, categorized into three groups based on trip duration: 3-day, 5-day, and 7-day itineraries with three different difficulty levels based on the availability of data (distribution given in Table \ref{tab:tripcraft_dist}). Each query is accompanied by a human-annotated reference plan and a rationale explaining the reasoning behind the chosen itinerary. \tripcraft \;offers a structured evaluation framework with well-defined constraints and human-curated plans, making it a robust benchmark for measuring the reasoning and planning abilities of language agents.




We introduce \tripcraft, a benchmark for evaluating language agents in complex, constraint-aware planning, specifically in travel itinerary generation. \tripcraft \;assesses agents' ability to construct flexible yet constrained plans by incorporating user preferences, commonsense considerations, and persona-driven requirements while ensuring temporal, spatial, sequential, and personal consistency. The benchmark comprises 1000 diverse travel queries, categorized into three trip durations (3-day, 5-day, and 7-day) and three difficulty levels based on data availability (distribution in Table \ref{tab:tripcraft_dist}). Each query includes a human-annotated reference plan with a rationale explaining the itinerary's reasoning. By providing a structured evaluation framework with well-defined constraints and human-curated plans, \tripcraft \;serves as a rigorous benchmark for measuring the reasoning and planning capabilities of language agents.


\begin{table}[h]
    \centering
    \renewcommand{\arraystretch}{0.8}
    \begin{tabular}{lcccc}
        \toprule
        \textbf{Days} & \textbf{Easy} & \textbf{Medium} & \textbf{Hard} & \textbf{Total}\\
        \midrule
        3-day & 130 & 106 & 108 & 344\\
        5-day & 87 & 164 & 73 & 324\\
        7-day & 99 & 178 & 55 & 332\\
        \bottomrule
    \end{tabular}
    \caption{\tripcraft \; dataset distribution}
    \label{tab:tripcraft_dist}
\end{table}


\begin{table*}[!t]
    \centering
    \rowcolors{3}{gray!15}{white} % Alternating row colors starting from the third row
    \begin{tabular}{>{\bfseries}l p{11cm}}
        \hline
        \multicolumn{2}{c}{\cellcolor{gray!25} \textbf{Commonsense Constraints}} \\
        \hline
        Sufficient Meal Gaps & Defines a minimum gap of four hours between the meals. \\
        Valid PoI list & Defines validity rules for the Point of Interest list. \\
        Diverse Events & Event choices should not be repeated throughout the trip. \\
        \hline
        \multicolumn{2}{c}{\cellcolor{gray!25} \textbf{Hard Constraints}} \\
        \hline
        Event Types & Event Types include four distinct categories—Sports, Arts \& Theatre, Music, and Film. \\
        Attraction Types &  Each attraction belongs to one or more of 15 predefined categories, ensuring a well-distributed selection of activities. \\
        \multicolumn{2}{c}{\cellcolor{gray!25} \textbf{Persona Components}} \\
        \hline
        Traveler Type & Defines how a traveler approaches their journey—whether they seek relaxation in cozy spots or adrenaline-pumping adventures. \\
        Purpose of Travel & Captures trip motivation. Examples: to unwind, explore cultures etc.\\
        Spending Preference &  Reflects the traveler’s budget and style, from luxurious indulgence to cost-conscious experiences. \\
        Location Preference &  Highlights preferred environments, such as beaches, mountains, cities, or wildlife-rich forests. \\
        \hline
    \end{tabular}
    \caption{Addition of constraints and persona details based on availability and limitations of scraped data has been given in the above table. The full list of constraints used in \tripcraft\;has been given in Table 9 of Appendix.}
    \label{tab:constraints}
\end{table*}
\subsection{Constraint and Persona details}
\label{subsec:const_and_pers}

\tripcraft \;integrates numerous constraints and persona components to enhance the evaluation of language agents in constraint-aware itinerary generation, as shown in Table~\ref{tab:constraints}. These improvements ensure that generated travel plans are realistic, well-structured, and aligned with user preferences.

\paragraph{Commonsense Constraints.} In addition to the commonsense constraints in TravelPlanner, we introduce refinements to improve itinerary realism. First, the same event should not be repeated multiple times across a trip, ensuring diversity in experiences. Second, meal timings must have a minimum gap of four hours between breakfast, lunch, and dinner to maintain a natural schedule. Third, the point-of-interest (PoI) list must follow strict validity rules: each day's itinerary must begin and end at the designated accommodation, except on the final day when the traveler departs. The list should be an ordered sequence of accommodations, attractions, and restaurants, ensuring adequate time gaps between flight arrivals and accommodation check-ins, as well as between accommodation check-outs and departures. %We also remove the minimum night constraint due to data limitations. 
%Additionally, all existing constraints were heavily modified to align with this new spatio-temporal setting.

\begin{table}[!b]
    \centering
    \renewcommand{\arraystretch}{0.8}
    \setlength{\tabcolsep}{8pt} % Adjust column spacing
    \begin{tabular}{l c}
        \toprule
        \textbf{Category} & \textbf{Duration (hrs)} \\
        \midrule
        Boat Tours \& Water Sports & 3.5 \\
        Casinos \& Gambling & 2.5 \\
        Museums & 3.0 \\
        Nature \& Parks & 4.5 \\
        Nightlife & 2.5 \\
        Sights \& Landmarks & 3.0 \\
        %Water \& Amusement Parks & 5.0 \\
        %Zoos \& Aquariums & 2.5 \\
        \bottomrule
    \end{tabular}
    \caption{Attraction visiting duration (hrs) for a subset of categories (due to page limit). The complete list of categories has been given in Table 8 of Appendix.}
    \label{tab:subcategory_dur_sample}
\end{table}

\paragraph{Hard Constraints.} We introduce two new hard constraints to enhance itinerary structuring. First, each attraction belongs to one or more of 15 diverse categories (see Table \ref{tab:subcategory_dur_sample}), ensuring a well-distributed selection of activities. Second, events are categorized into four distinct types — Sports, Arts \& Theatre, Music, and Film — allowing for a more structured and personalized planning process.
\paragraph{Persona Information.}  
Each query in \tripcraft \; is accompanied by a persona profile that influences travel planning, consisting of traveler type, purpose of travel, spending preference, and location preferences. Traveler type distinguishes between laid-back travelers, who prefer relaxation and scenic spots, and adventure seekers, who prioritize extreme activities like paragliding and bungee jumping. Purpose of travel refines traveler intent into categories such as relaxation, adventure, cultural exploration, and nature-focused experiences. Spending preference differentiates between luxury and budget-conscious travelers, shaping accommodation and activity choices. Location preferences specify favored destinations, such as beaches, mountains, cities, or wildlife/forest regions.

These persona aspects were carefully selected in consultation with domain experts as the most influential factors in determining a fulfilling travel experience. Unlike the work of \citet{singh-etal-2024-personal}, which includes demographic attributes such as age, gender, and education, \tripcraft \; deliberately omits such details to minimize potential biases in language model behavior.
\begin{table}[h]
    \centering
    \renewcommand{\arraystretch}{0.8}
    \begin{tabular}{lp{3cm}}
        \toprule
        \textbf{Database} & \textbf{Data Entries (\#)} \\
        \midrule
        City Set & 140 \\
        Flights & 3,446,829 \\
        Restaurants & 3,892 \\
        Attractions & 5,043 \\
        Accommodations & 2,800 \\
        Events & 21,980 \\
        Nearest Transit Stop & 8,723 \\
        Distance Matrix & 19,460 \\
        \bottomrule
    \end{tabular}
    \caption{Data entries in the database have been scraped to include the most recent data.}
    \label{tab:data_entries}
\end{table}






\subsection{Construction pipeline}
The benchmark construction (Figure \ref{fig:pipeline}) consists of three key steps as follows. %: 1) Data Collection and Cleaning, 2) Query and Persona Construction, and 3) Annotation and Refinement. The whole process has been illustrated with clarity in Figure \ref{fig:pipeline}.

\begin{itemize}  
    \item \textbf{Data Collection and Cleaning:} 
    % The databases are gathered through web scraping and open-source tools such as OpenStreetMap\footnote{\url{https://www.openstreetmap.org/}} (detailed in Appendix and Table \ref{tab:data_entries}). Due to the extensive size of the flights database, we adopt the dataset from the work of \citet{xie2024travelplanner}, adjusting dates to maximize overlap with event timelines, ensuring that a significant portion of queries could incorporate relevant events. Unlike earlier datasets with outdated information, ours is more recent, though some entries lack complete details. These were either carefully removed or filled with default values as necessary. Additionally, since GTFS public transit\footnote{\url{https://gtfs.org/}} data covers only 140 cities, all databases are filtered to include information exclusively for these cities.
    The databases are sourced via web scraping and open-source tools like OSM\footnote{\url{https://www.openstreetmap.org/}} (details in Appendix and Table \ref{tab:data_entries}). Given the extensive size of the flights database, we adopt the dataset from \citet{xie2024travelplanner}, adjusting dates to align with event timelines, ensuring queries incorporate relevant events while maintaining computational feasibility. Unlike prior datasets with outdated information, ours is more recent; however, missing or incomplete entries were either carefully removed or filled with default values to ensure consistency. Since GTFS public transit\footnote{\url{https://gtfs.org/}} data covers only 140 cities, all databases are filtered accordingly.


    \item \textbf{Persona and Query Construction:} 
    % Following \citet{xie2024travelplanner}, queries are constructed using randomly selected fundamental elements such as departure city, destination, and date range. The trip duration dictates the number of cities included: 3-day plans focuses on a single city, while 5-day and 7-day plans involve one randomly chosen state, with visits to 2 and 3 cities, respectively. This ensures that agents had to reason about multi-city itineraries while considering inter-city connectivity. To enhance complexity and realism, diverse user needs are introduced as hard constraints, and relevant persona profiles are assigned by selecting values from the four persona components. These structured inputs are then stitched together using GPT-4o\footnote{\url{https://openai.com/index/gpt-4o-system-card/}} in a few-shot setting to generate high-quality queries.
    Following \citet{xie2024travelplanner}, queries are generated by randomly selecting key elements—departure city, destination, and date range etc. Trip duration determines city coverage: 3-day plans focus on one city, while 5-day and 7-day plans span one state with visits to 2 and 3 cities, respectively, requiring agents to reason about multi-city itineraries and inter-city connectivity. To enhance complexity, hard constraints and persona profiles are incorporated. These structured inputs are then composed using GPT-4o\footnote{\url{https://openai.com/index/gpt-4o-system-card/}} in a few-shot setting to generate high-quality queries.

    \item \textbf{Annotation and Refinement:} 
    % A team of 25 graduate students are employed to carefully annotate the plans for synthesized queries\footnote{This task was completed during their internship at our NLP lab.}. Since each query can have multiple valid plans, annotators provide justifications for their choices, improving plan explainability. The annotation process underwent several iterative refinement rounds, incorporating feedback from the authors (verifiers) at each step to ensure nuanced interpretations of persona constraints. As a final validation, the verifiers manually review all query-plan pairs, employing evaluation scripts alongside manual checks to ensure feasibility and optimality. The annotation process in \tripcraft \; is significantly more challenging and time-consuming (\textasciitilde 30 minutes per annotation) than prior datasets due to the added temporal and spatial dimensions in the Point of Interest List which requires additional scrutiny.
    A team of 25 graduate students annotated plans for the queries\footnote{Interns at our NLP lab.}, providing justifications to enhance explainability. The process involved iterative refinements, integrating expert feedback to ensure nuanced interpretations of persona constraints. Domain Experts conducted a final manual review of all query-plan pairs, combining evaluation scripts with manual checks for feasibility and optimality. Annotation in \tripcraft \; is notably more demanding (\textasciitilde 30 minutes per instance) than prior datasets due to the added temporal and spatial complexities in the PoI list, requiring heightened scrutiny.
\end{itemize}


\begin{figure}[!t]
    \raggedleft
    \includegraphics[width=0.45\textwidth]{Images/pipeline_2-cropped.pdf}  % Change filename accordingly
    \caption{Visualization of the Construction Pipeline.}
    \label{fig:pipeline}
\end{figure}

\subsection{Evaluation Metrics}
\label{subsec:eva_metrics}
% The hard and commonsense constraints outlined in subsection \ref{subsec:const_and_pers} are employed to assess feasibility. In addition to these, continuous evaluation metrics are introduced to address the limitations of discrete binary checks, providing a more nuanced understanding of the temporal, spatial, sequential, and personal aspects of the plans. Our proposed metrics provide a comprehensive evaluation of a given plan by assessing its quality across temporal, spatial, sequential, and persona-specific dimensions. The benefit of using our metrics has been further elaborated on in subsection \ref{subsec:analysis}. The parameters of these metrics have been determined from our annotation distribution (Table \ref{tab:parameter_details_sample}). 
% In general, given annotations for any dataset, the parameters of these metrics can always be derived from the underlying distribution of the corresponding variables in the annotation.
Plan feasibility is assessed using the hard and commonsense constraints from subsection \ref{subsec:const_and_pers}. Furthermore, we introduce continuous evaluation metrics that capture temporal, spatial, sequential, and persona-specific nuances of a travel plan, addressing the limitations of discrete binary checks (the advantages of these metrics are detailed in subsection \ref{subsec:analysis}). The parameters of these metrics have been determined from our annotation distribution (Table \ref{tab:parameter_details_sample}).  More generally, given annotations for any dataset, metric parameters can always be inferred from the distribution of corresponding variables in the annotation.


\begin{table}[!t]
    \centering
    \renewcommand{\arraystretch}{0.9}
    \setlength{\tabcolsep}{2pt} % Adjust column spacing
    \begin{tabularx}{\columnwidth}{l *{3}{>{\centering\arraybackslash}X}}
        \toprule
        \textbf{Parameter} & \textbf{3-day} & \textbf{5-day} & \textbf{7-day} \\
        \midrule
        \multicolumn{4}{c}{\textbf{Restaurant Parameters}} \\
        \midrule
        \textbf{Breakfast} & & & \\
        %Mean Time & 9.63 & 9.80 & 9.84 \\ 
        Mean Duration (hrs) & 0.90 & 1.08 & 0.85 \\ 
        %Std. Time & 1.08 & 1.08 & 1.34 \\ 
        Std. Duration (hrs) & 0.24 & 1.43 & 0.23 \\ 
        %Beta & 0.21 & 0.63 & 0.03 \\ 
        \midrule
        \textbf{Lunch} & & & \\
        %Mean Time & 14.30 & 14.46 & 14.44 \\ 
        Mean Duration (hrs) & 1.11 & 1.10 & 0.99 \\ 
        %Std. Time & 1.03 & 1.07 & 1.07 \\ 
        Std. Duration (hrs) & 0.36 & 0.35 & 0.26 \\ 
        %Beta & 0.10 & 0.04 & 0.30 \\ 
        \midrule
        \textbf{Dinner} & & & \\
        %Mean Time & 20.75 & 20.67 & 20.42 \\ 
        Mean Duration (hrs) & 1.19 & 1.32 & 1.15 \\ 
        %Std. Time & 1.25 & 1.37 & 1.66 \\ 
        Std. Duration (hrs) & 0.43 & 0.91 & 1.15 \\ 
        %Beta & -0.20 & -0.18 & -0.07 \\ 
        \midrule
        \multicolumn{4}{c}{\textbf{Attraction Parameters}} \\
        \midrule
        $\lambda_{laidback}$ & 1.10 & 1.26 & 1.11 \\ 
        $\lambda_{adventurous}$ & 2.01 & 1.61 & 1.82 \\ 
        $\sigma_d$ (hrs) & 1.11 & 1.07 & 0.90 \\ 
        $n^{max}$ & 5 & 4 & 4 \\ 
        $n^{min}$ & 0 & 0 & 0 \\ 
        $k$ (hrs) & 0.28 & 0.28 & 0.28 \\ 
        \bottomrule
    \end{tabularx}
    \caption{Day-wise parameter details calculated from our annotation distribution. For a complete list of parameters refer to the Appendix.}
    \label{tab:parameter_details_sample}
\end{table}


\noindent\textbf{Temporal Meal Score: }
% For each meal type (breakfast, lunch, or dinner), it is computed based on a bivariate normal distribution. The use of a Gaussian distribution is motivated by prior work that has successfully employed Gaussian models for time series data \cite{ansari2024chronos,10.1007/978-3-642-41398-8_15}. Additionally, the Gaussian distribution facilitates smooth penalization of incorrect time intervals, ensuring a continuous and coherent representation of temporal dependencies. Formally, for a particular meal interval $a$, its temporal score $T^a_{meal}$ is calculated as follows:
For each meal type (breakfast, lunch, or dinner), the temporal meal score $T^a_{meal}$ is modeled using a bivariate normal distribution, motivated by prior work leveraging Gaussian models for time series data \cite{ansari2024chronos,10.1007/978-3-642-41398-8_15}. This choice enables smooth penalization of deviations, ensuring a continuous representation of temporal dependencies. Formally, $T^a_{meal}$ is computed as:
\begin{equation}
\small
    T^a_{meal} = \frac{1}{(2\pi)\sqrt{\bm{\Sigma}}} \exp \left( -\frac{1}{2} (\bm{x} - \bm{\mu})^T \bm{\Sigma}^{-1} (\bm{x} - \bm{\mu}) \right)
\end{equation}
% \begin{equation}
% \begin{split}
%     T^a_{meal} &= \frac{1}{(2\pi)\sqrt{\bm{\Sigma}}} \exp \Bigg( -\frac{1}{2} (\bm{x} - \bm{\mu})^T \\
%     &\quad \bm{\Sigma}^{-1} (\bm{x} - \bm{\mu}) \Bigg)
% \end{split}
% \end{equation}


\noindent where $ \bm{x} = [t_m, d_m] $ represents the midpoint time $ t_m $ and duration $ d_m $ of the meal, and $ \bm{\Sigma} $ is the determinant of the covariance matrix with the correlation coefficient \( \beta \). $ T^a_{meal} $ is normalized to obtain a score between 0 and 1 for each meal. And, these values are averaged over all the meals in a plan:
\begin{equation}
\small
    \bar{T}_{meal} = \frac{1}{M} \sum_{a=1}^{M} T_{meal}^{(a)}
\end{equation}
Here, $M$ is the total number of meals in the plan, and $T^a_{meal}$ represents the temporal score for meal interval $a$.

\noindent\textbf{Temporal Attraction Score:} Evaluates whether the visit duration of an attraction aligns with an expected duration for that type of attraction while also considering the total number of attractions visited in a day. It is modeled as the probability density function of a joint mixed random distribution combining a Gaussian for duration ($X$) and a Poisson for the number of attractions ($N$):
\begin{equation}
\small
    f_{X, N}(d_i, n) = f_{X \mid N}(d_i \mid n) \cdot P(N = n) 
\end{equation}
Averaging over all attractions, $n$, in a plan, we have:
\begin{equation}
\small
    \bar T_{attrac} = \frac{1}{n} \sum_{i=1}^{n} \exp \left( -\frac{(d_i - \mu^i_d)^2}{2\sigma_d^2} \right) \cdot \frac{\lambda^n e^{-\lambda}}{n!}
\end{equation}

\noindent Here, \( d_i \) represents the actual duration spent at attraction \( i \), while \( \mu^i_d \) denotes the expected visit duration for $i$, which is dynamically adjusted based on the number of attractions visited and attraction type. \( \sigma_d \) captures variability in visit durations. To incorporate the influence of the total number of attractions, we use a Poisson distribution \( P \), where n represents the number of attractions visited whose \( \lambda \) parameter represents the expected number of attractions for a given persona (e.g., adventurous or laidback). The expected visit duration \( \mu^i_d \) is adjusted dynamically:
\begin{equation}
\small
    \mu^i_d =
    \begin{cases}
        \mu_d^{\text{type}} - k (n - n^{\text{min}}), & \text{if adventurous} \\
        \mu_d^{\text{type}} + k (n^{\text{max}}- n), & \text{if laidback}
    \end{cases}
\end{equation}
where \( \mu_d^{\text{type}} \) represents the expected duration for the category of attraction $i$, and \( k \) denotes the rate at which the average visit duration changes per unit increase in the number of attractions. Intuitively, for a particular type of attraction, a laid-back traveler generally spends more time compared to an adventure seeker. Moreover, irrespective of the traveler's persona, the average duration spent at each attraction decreases as the total number of attractions visited in a day increases.


\begin{table*}[h]
    \centering
    \renewcommand{\arraystretch}{0.7}  % Reduce row height
    % \setlength{\tabcolsep}{4pt}
    \begin{tabular}{lcccccc}
        \toprule
        \textbf{Setting} & \textbf{Category} & \textbf{$\bar T_{meal}$} & \textbf{$\bar T_{attrac}$} & \textbf{$\bar S_{spatial}$} & \textbf{$\bar S_{persona}$} & \textbf{$\bar S_{ord}$} \\
        \midrule
        \multirow{3}{*}{\textit{w/o Parameter Info}} 
        & 3-day  & 0.56 & 0.0169 & 0.80 & 0.46 & 0.70 \\
        & 5-day  & 0.69 & 0.0057 & 0.86 & 0.49 & 0.91 \\
        & 7-day  & 0.61 & \textbf{0.0122} & 0.85 & 0.50 & 0.96 \\
        \midrule
        \multirow{3}{*}{\textit{w/ Parameter Info}} 
        & 3-day  & \textbf{0.70} & \textbf{0.0171} & \textbf{0.83} & \textbf{0.48} & \textbf{0.74} \\
        & 5-day  & \textbf{0.75} & \textbf{0.0078} & \textbf{0.87} & \textbf{0.50} & \textbf{0.93} \\
        & 7-day  & \textbf{0.80} & 0.0057 & \textbf{0.86} & \textbf{0.51} & \textbf{0.97} \\
        \bottomrule
    \end{tabular}
    \caption{Results of our proposed evaluation metrics for the two settings. For each category, the \textbf{bolded} values represent the better ones between the two settings.}
    \label{tab:new_results}
\end{table*}

\begin{table*}[h]
    \centering
    \renewcommand{\arraystretch}{0.7}  % Reduce row height
    % \setlength{\tabcolsep}{4pt}  % Reduce column spacing
    \begin{tabular}{p{3cm} p{1.5cm} ccccccc}
        \toprule
       % \multirow{2}{*}{\textbf{Setting}} & \multirow{2}{*}{\textbf{Category}} & \textbf{Delivery} & \multicolumn{2}{c}{\textbf{CPR}} & \multicolumn{2}{c}{\textbf{HCPR}} & \multirow{2}{*}{\textbf{Final Pass Rate}} \\
       % \cmidrule(lr){4-5} \cmidrule(lr){6-7} &
       %  & \textbf{Rate} & \textbf{Micro} & \textbf{Macro} & \textbf{Micro} & \textbf{Macro} & \\
       %  \midrule
\multirow{2}{*}{\textbf{Setting}} & \multirow{2}{*}{\textbf{Category}} & Delivery & \multicolumn{2}{c}{CPR} & \multicolumn{2}{c}{HCPR} & \multirow{2}{*}{Final Pass Rate} \\
\cmidrule(lr){4-5} \cmidrule(lr){6-7} &
& Rate & Micro & Macro & Micro& Macro & \\
\midrule
        \multirow{3}{*}{\textit{w/o Parameter Info}} 
        & 3-day  & 92.60 & 77.61 & \textbf{6.08} & \textbf{27.63} & \textbf{26.08} & \textbf{3.47} \\
        & 5-day  & \textbf{98.69} & \textbf{66.99} & \textbf{2.17} & 4.21 & 3.04 & \textbf{1.74} \\
        & 7-day  & \textbf{96.03} & \textbf{70.35} & \underline{0.00} & 0.62 & \underline{0.00} & \underline{0.00} \\
        \midrule
        \multirow{3}{*}{\textit{w/ Parameter Info}} 
        & 3-day  & \textbf{96.08} & \textbf{80.08} & 5.21 & 26.10 & 25.21 & 1.74 \\
        & 5-day  & 91.30 & 61.69 & 0.87 & \textbf{8.60} & \textbf{5.21} & 0.43 \\
        & 7-day  & 92.51 & 66.43 & \underline{0.00} & \textbf{0.83} & \underline{0.00} & \underline{0.00} \\
        \bottomrule
    \end{tabular}
    \caption{Results of existing evaluation metrics for the two settings. CPR and HCPR stand for Commonsense Pass Rate and Hard Constraint Pass Rate respectively. For each category, the \textbf{bolded} values represent the better ones between the two settings. Moreover, equal values have been \underline{underlined}. }
    \label{tab:old_results}
\end{table*}

\noindent\textbf{Spatial Score:} Computed based on the distance \( d \) of a point of interest from the nearest transit station:
\begin{equation}
\small
S_s(d) =
\begin{cases}
    1 - 0.5 \left( \frac{d}{d_0} \right), & \text{if } d \leq d_0 \\
    0.5 \exp \left( -\lambda (d - d_0) \right), & \text{if } d > d_0
\end{cases}
\end{equation}
\normalsize
where $d_0$ (= 5 km) is a threshold distance and $\lambda$ (= 0.0002) is the decay rate for larger distances. The final spatial score for a plan is the average of all individual PoI scores:
\begin{equation}
\small
\bar{S}_{spatial} = \frac{1}{N} \sum_{i=1}^{N} S_s(d_i), \quad N \text{ is the no. of PoIs visited.}
\end{equation}
\normalsize
% where \( N \) is the number of PoIs visited in all the days of the trip.
\noindent\textbf{Persona Score: }
% Quantifies the alignment between a traveler's persona components and the Points of Interest (PoIs) they visit. Building on prior work in PoI conflation \cite{sun2023conflating}, which explored methods based on PoI types and names, we opt for PoI names due to their richer semantic information—often including key adjectives like \textit{relaxing} or \textit{luxury}—compared to the typically concise PoI types. Accordingly, we compute the persona score as the average cosine similarity between BERT \cite{devlin-etal-2019-bert} embeddings of individual persona components and PoI names.
Quantifies the alignment between a traveler's persona and visited Points of Interest (PoIs). Prior work on PoI conflation \cite{sun2023conflating} explored type- and name-based methods; we prioritize PoI names as they offer richer semantics, often embedding key descriptors like \textit{relaxing} or \textit{luxury}. The persona score is computed as the average cosine similarity between BERT \cite{devlin-etal-2019-bert} embeddings of persona components and PoI names:
\begin{equation}
\small
\bar S_{persona} = \frac{1}{M \cdot N} \sum_{j=1}^{M} \sum_{i=1}^{N} \frac{\bm{p}_j \cdot \bm{q}_i}{\|\bm{p}_j\| \|\bm{q}_i\|}
\end{equation}
\normalsize
where \( \bm{p}_j \) represents the BERT embedding of the \( j \)-th persona component, \( \bm{q}_i \) is the BERT embedding of the \( i \)-th PoI name, \( M \) is the total number of persona components, and \( N \) is the total number of PoIs in the travel plan.

\noindent\textbf{Ordering Score: } Measures the sequential alignment of the generated PoI list with the annotated PoI list for a particular day:
\begin{equation}
\small
S_{\text{ord}} = 1 - \frac{\text{ED}(\mathcal{G}, \mathcal{A})}{\max(|\mathcal{G}|, |\mathcal{A}|)}
\end{equation}
\normalsize
where \( \mathcal{G} \) and \( \mathcal{A} \) represent the generated and annotated sequences of points of interest, respectively, and \( \text{ED}(\mathcal{G}, \mathcal{A}) \) denotes the Edit Distance \cite{Levenshtein1965BinaryCC} between them. This is averaged over all the days of the journey to get the ordering score for a plan, which is denoted by $\bar S_{ord}$.


\begin{figure*}[t]
    \centering
    \includegraphics[width=0.8\textwidth]{Images/example_v2-cropped.pdf}  % Change filename accordingly
    \caption{Existing constraint-based metrics rate these plans equally, but, our metrics provide a continuous score, capturing temporal, spatial, sequential, and personal differences.}
    \label{fig:potential_proposed_metric}
\end{figure*}

% \subsection{Methodology}



% \section{Methodology}
In this section, we outline the key research questions driving this study, followed by a detailed description of the methodology used to design and conduct the survey.
\subsection{Research Questions}
\begin{enumerate}
    \item[\textbf{RQ1:}] How do developers allocate their time during a typical workweek, and how does this compare to their perception of an \textbf{ideal workweek?}
    \item[\textbf{RQ2:}] How are developer's satisfaction and productivity affected by \textbf{deviations} from their ideal workweek?
     \item[\textbf{RQ3:}] For which tasks do developers prefer using \textbf{AI tools}, and how does the frequency of AI tool usage \textbf{influence} their satisfaction and productivity?
\end{enumerate}

\subsection{Survey Design}
% Describe how the survey was conducted, survey structure, sample size, which activities were selected and how, incentives, etc. 

To gain insights into the types of activities developers engage in during a typical work week, we conducted a series of exploratory interviews with 12 randomly selected participants. These semi-structured interviews provided a qualitative foundation, allowing us to iteratively develop a comprehensive list of higher-level activities that reflect both ideal and actual workweek allocations. The findings from these interviews were instrumental in refining our survey questions and design.

% - When was it distributed
% - How many people were invited
% - how was the survey advertised
% - incentive provided to participants
% - how many responses received (with response rates)
% - Board of ethics description \& instruments
% - Describe the main questions asked in the survey

The survey was distributed in \textcolor{blue}{May 2024} to software engineers working in Microsoft teams across India and the United States. A total of 6000 developers were invited to participate via email. Framed as a study aimed at boosting developer productivity by understanding how they allocate their time in a workday, the survey received 510 complete responses (responses rate of 8.5\%). After finishing the survey, the participants could enter a sweepstake to win one out of ten \$50 Amazon.com Gift Cards.
\textcolor{blue}{description of ethics}.

The main questions in the survey were as follows:
\begin{enumerate}
    \item Their roles and years of experience in the industry/team
    \item The hours spent on various activities in their typical workweek
    \item Ideally, the percentage of time they would want to allocate to each activity in a workweek
    \item How productive and satisfied were they by their past workweek
    \item Activities they find most cognitively challenging
    \item How often do they use AI tools to assist in their daily activities
    \item Two open-ended questions about the activities they would want to automate using AI tools, and advice for new hires to boost their productivity and satisfaction levels 
\end{enumerate}



\subsection{Data Analysis \& Exploration}
% Here, we could start with discussing the survey group:
% - demographic observations
% - distribution of participants (based on the years experience in the industry/team), 

From the exploratory interviews, we identified sixteen key activities, which were subsequently used to quantify the developers' time allocation across their work week. 

\subsection{Limitations}


\newpage
\section{Experiment}\label{sec-experiment}
\subsection{Experimental Setup}
We briefly introduce experimental settings to verify our proposed MoR, including Datasets \& Baselines, Implementation Details, and Evaluation Metrics. More details are in Appendix~\ref{app-expr-setting}.

\textbf{Datasets \& Baselines:} We use three TG-KBs from STaRK~\cite{wu2024stark} covering three knowledge domains, including E-commerce Products (Amazon), Academic Papers (MAG), and Biomedicine (Prime). We compare our MoR with baselines established by~\citet{wu2024stark} and categorize them into textual/structural/hybrid-based ones. More recent state-of-the-art hybird retrieval approaches fro TG-KBs such as KAR~\cite{xia2024knowledge} and MFAR$^{*}$~\cite{li2024multi} are also compared.


\textbf{Implementation Details:} 
To enhance the planning capability of our planning module, we fine-tune the Llama 3.2 (3B) on 1000 sampled queries with their corresponding ground-truth planning graphs, serving as the textual graph generator. In the absence of ground-truths, we synthesize them using LLMs. For the Prime dataset, we empirically find that directly prompting LLMs can hardly generate accurate planning graphs due to the lack of biomedical domain knowledge~\cite{Shen2024TagLLMRG}. Therefore, we adopt an alternative approach. First, we instruct LLMs to extract triplets from each query and then construct the planning graphs by merging triplets with shared entities. 
During mixed traversal, textual matching can be implemented using any lexical or semantic methods. For this study, we employ BM25 for Amazon and MAG and fine-tune a contriever to complement the biomedical knowledge for Prime.
To initialize the structural traversal, we employ textual matching to locate the top 5 nodes that are most relevant to the query as seeds. Additionally, at each layer, we incorporate the top 10 nodes retrieved via textual matching and append them to the current candidate set for the next round of traversal. Notably, due to the uncertainty of LLMs, the generated planning graphs can be invalid. In this case, we will directly conduct textual matching to retrieve candidates. For our ablations without reranker, we employ Ada-002~\cite{wu2024stark} with cosine similarity as the scorer to rank candidates for evaluating performance.

\textbf{Evaluation Metrics:}
We follow~\citet{wu2024stark} for evaluation by reporting Hit@1 (H@1), Hit@5 (H@5), Recall@20 (R@20), and mean reciprocal rank MRR to evaluate in the full spectrum. 


 

\newpage
\subsection{Overall Retrieval Performance}
We compare MoR with other baselines on three TG-KBs in Table~\ref{tab-merged}. Generally, hybrid methods, AvaTAR, KAR, MFAR$^{*}$, and our MoR, achieve better performance than purely textual or structural methods owing to their ability to integrate both structural and textual knowledge. 
Among all baselines, our proposed MoR achieves the overall best performance with a substantial margin on average, with the first ranking on MAG and the second ranking on Amazon/Prime datasets. This demonstrates the effectiveness of our proposed mixture of structural and textual knowledge retrieval. 
Textual retrieval performs better on Amazon than on MAG, suggesting that Amazon queries rely more on textual knowledge. In contrast, its weaker performance on MAG is due to MAG's lower textual richness and stronger structural signals. This disparity aligns with the distribution analysis presented by~\citet{wu2024stark} and supports our hypothesis that queries in different TG-KB datasets require varying desires for textual and structural knowledge. Meanwhile, structural retrieval methods such as conventional knowledge graph-based ones perform poorly because they are designed for graphs with minimal textual information compared to TG-KBs.
Different from Amazon and MAG, all existing methods without supervised tuning (e.g., Ada-002) exhibit significantly lower performance on Prime. This is due to the extreme domain expertise required in biology, where word-count-based, pre-trained textual similarity-based, and even more powerful LLMs are all poorly applicable here. Through fine-tuning, MFAR$^{*}$ and our proposed MoR generally achieve better performance, demonstrating the necessity of domain-specific knowledge for answering queries in knowledge-intensive domains. 




\newpage
\subsection{Ablation Study}
After verifying the superiority of MoR, we conduct ablation studies to assess its different components, including module and feature ablation.

\subsubsection{Module Ablation}


To assess the contribution of each module in MoR, namely, Text Matching-based Retrieval, Neighborhood-Fetching-based Structural Retrieval, and Reranker, we conduct a series of ablation experiments. First, we remove the Reranker, resulting in the variant MoR$_{\text{w/o R}}$. On top of that, we further separately eliminate Text Retrieval and Structural Retrieval, yielding MoR$_{\text{w/o RT}}$ and MoR$_{\text{w/o RS}}$, respectively.
As shown in Table~\ref{tab-merged}, the complete MoR framework consistently achieves the highest performance across all datasets, demonstrating the synergistic effect of the Textual Retriever, Structural Retriever, and Reranker.
After removing Reranker, MoR$_{\text{w/o R}}$ exhibits a consistent performance drop across all datasets and evaluation metrics. This underscores the importance of the Reranker in refining retrieval by filtering noisy candidates from the intermediate reasoning stage. 
Eliminating Text Retrieval, i.e., MoR$_{\text{w/o RT}}$, leads to a notable performance drop on Amazon but an unexpected improvement on MAG. This suggests that while textual knowledge benefits Amazon, it introduces misleading hard negatives that compromise the ranking method (e.g., Ada-002) for MAG. Conversely, removing Structural Retrieval, MoR$_{\text{w/o RS}}$, results in a slight performance decrease further on MAG, reinforcing the importance of structural knowledge in MAG-related queries.
%
These results underscore the Reranker's crucial role in adaptively harmonizing, balancing, and selecting knowledge from both structural and textual retrieval experts.






\begin{table}[t!]
\small
\setlength\tabcolsep{4.5pt}
\centering
\begin{tabular}{l|ccc|cccc}
\toprule
\textbf{Dataset} &\textbf{TF} & \textbf{SF} & \textbf{TI} & \textbf{H@1} & \textbf{H@5} & \textbf{R@20} & \textbf{MRR} \\ \midrule
\multirow{7}{*}{\textbf{MAG}} 
& \cmark & \xmark & \xmark & 48.96 & 73.02 & 72.44 & 59.79 \\
&      \xmark            & \cmark       &         \xmark         & 18.79 & 41.91 & 52.85 & 29.84 \\
&        \xmark          &         \xmark         & \cmark       & 18.16 & 41.53 & 52.78 & 29.31 \\
\cline{2-8}
& \cmark       & \cmark       &    \xmark              & 58.04 & 77.14 & 74.42 & 66.75 \\
& \cmark       &        \xmark          & \cmark       & \underline{58.16} & \underline{77.59} & \underline{74.96} & \underline{66.85} \\
&          \xmark        & \cmark       & \cmark       & 17.93 & 38.01 & 46.79 & 27.48 \\
\cline{2-8}
& \cmark       & \cmark       & \cmark       & \textbf{58.19} & \textbf{78.34} & \textbf{75.01} & \textbf{67.14} \\ \midrule
\multirow{7}{*}{\textbf{Amazon}}    
& \cmark       &      \xmark            &       \xmark           & \underline{51.21} & \underline{74.05} & \underline{59.79} & \underline{61.27} \\
&        \xmark          & \cmark       &      \xmark            & 8.09  & 24.48 & 25.62 & 16.94 \\
&         \xmark         &      \xmark            & \cmark       & 5.84  & 16.62 & 12.94 & 11.57 \\
\cline{2-8}
& \cmark       & \cmark       &      \xmark            & 50.91 & 73.38 & 59.58 & 61.15 \\
& \cmark       &         \xmark         & \cmark       & 51.09 & 73.56 & 59.61 & 61.14 \\
&            \xmark      & \cmark       & \cmark       & 8.09  & 24.48 & 25.62 & 16.94 \\
\cline{2-8}
& \cmark       & \cmark       & \cmark       & \textbf{52.19} & \textbf{74.65} & \textbf{59.92} & \textbf{62.24} \\ \bottomrule
\end{tabular}
\caption{Ablation study investigating the importance of three features, Textual Fingerprint (\textbf{TF}), Structural Fingerprint (\textbf{SF}), and Traversal Identifier (\textbf{TI}), of the traversal trajectories used in our Structure-aware Reranker.}
\label{tab-feature-ablation}
\vspace{-2ex}
\end{table}



\subsubsection{Feature Ablation}
The above ablation study highlights the crucial role of Structure-aware Reranker in adaptively integrating structural and textual knowledge. To further analyze the contributions of its three key features, \textbf{Textual Fingerprint (TF)}, \textbf{Structural Fingerprint (SF)}, and \textbf{Traversal Identifier (TI)} defined in Section~\ref{sec-organizing}, we conduct a feature ablation analysis and report retrieval performance across different feature configurations in Table~\ref{tab-feature-ablation}.
%Overall and individual performance
Overall, using three features together yields the best performance on both MAG and Amazon, highlighting their synergistic effect. Individually, TF contributes the most and outperforms SF and TI on both datasets. 
The reason is that based on the definition in Section~\ref{sec-organizing}, TF directly captures the relevance between the query and the retrieved nodes along the trajectory, whereas SF and TI primarily characterize the structural patterns and retrieval types, serving more as complementary factors. Therefore, equipping TF with these complementary factors (i.e., SF or TI) yields around 10\% additional gains on MAG. This is because SF and TI help the reranker selectively emphasize the relevance scores given by TF for certain nodes along the path. However, this boost is not observed on Amazon. We hypothesize that the textual knowledge needed there is predominantly derived from the final node on each path, making the structural cues provided by SF and TI less beneficial and even prone to overfitting. A deeper analysis to further justify this hypothesis is in Section~\ref{sec-further}. Overall, these findings underscore the varying importance of structural features in ranking across datasets.



\begin{table}[t!]
\small
\setlength\tabcolsep{4pt}
\centering
\begin{tabular}{l|ccc|ccc}
\toprule
\multirow{2}{*}{\textbf{Feature}} & \multicolumn{3}{c|}{\textbf{MAG}} & \multicolumn{3}{c}{\textbf{Amazon}} \\

 & H@1 & R@20 & MRR & H@1 & R@20 & MRR \\
\midrule
Last Node & 49.91 & 73.49 & 59.92 & 50.36 & 59.62 & 61.05   \\
Full Path & \textbf{58.19} & \textbf{75.01} & \textbf{67.14} & \textbf{52.19} & \textbf{59.92} & \textbf{62.24}   \\
\bottomrule
\end{tabular}
\caption{Comparing reranking performance using last node in the retrieved trajectory and the whole trajectory.}
\label{tab-Reranker-ablation}
\vspace{-2ex}
\end{table}

\begin{figure}[t!]
    \centering
    \includegraphics[width=0.49\textwidth, height = 0.22\textwidth]{figures/query-pattern-20250215.png}
    \vspace{-4.5ex}
    \caption{Imbalance number of queries and performance of different retrievers across different logical structures.}
    \label{fig-analysis}
    \vspace{-3ex}
\end{figure}





\subsection{Further Analysis}\label{sec-further}
This section understands MoR’s behavior by examining three questions, each of which enriches our insight into MoR’s functionality and offers novel perspectives inspiring future query retrieval research.

\textbf{Do structure signals affect reranking?}
To assess the impact of trajectory information on the Reranker's decision-making, we introduce a node-based Reranker that constructs trajectory features using only TF/SF/TI of the last node. In Table~\ref{tab-Reranker-ablation}, the path-based Reranker outperforms the node-based variant, especially on MAG. This highlights the critical role of trajectory features/structural knowledge in reranking. The minor performance boost on Amazon after switching to the full path trajectory indicates its textual knowledge preference over the last node rather than the whole trajectory.


\textbf{How does MoR perform on different logical structures?}
Figure~\ref{fig-analysis} shows the average performance of MoR on each query group categorized by their logical structures, where "Others" refer to queries with undefined logical structures in~\citet{wu2024stark} MoR consistently outperforms structural and textual retrievers across different logical structures. Among all queries, MoR performs the worst on "P → P" queries due to the ambiguity, although well-known, uniquely caused by repeated product entities from multi-step traversal.
The average-performing ``Others" group underscores the utility of diverse planning strategies for the same query.
Lastly, the skewed query distribution and retrieval performance across planning patterns reflect the varying nature of real-world planning needs. We hope these insights inspire research on data-centric reasoning designs and error control of planning.


\begin{figure}[t!]
    \centering
    \includegraphics[width=0.5\textwidth]{figures/heatmap-20250215.pdf}
    \vspace{-3ex}
    \caption{Saliency map visualization of query attention over three entities along the retrieved paths}
    \label{fig-map}
    \vspace{-2ex}
\end{figure}

\textbf{Does MoR indeed adaptively leverage the trajectory knowledge?} To understand how our proposed reranker prioritizes candidates in the Top-K results, we visualize the saliency map by computing the gradient of ranking scores with respect to the textual fingerprint (TF) of three nodes along the traversed path, which quantifies their importance for answering a given query. Figure~\ref{fig-map} illustrates this by analyzing trajectories for 100 ground-truth candidates across 100 queries on the Amazon and MAG datasets. Each dimension corresponds to a traversed node, with the final one representing the candidate itself. 
While the saliency score is concentrated in the last dimension for Amazon, 
MAG exhibits a more evenly distributed saliency pattern, where multiple nodes along the path contribute significantly to ranking score computation. This suggests that structural knowledge is more critical for answering queries in MAG, aligning with the previously observed lower performance of purely textual retrieval on MAG in Table~\ref{tab-merged}. Further case studies explain why the reranker attends different nodes for different queries. In Figure~\ref{fig-map}(a), the reranker favors the last two dimensions as the rich textual restriction (i.e., "Northwest Company..." and "NFL Seattle...") aids in identifying the correct node at the corresponding reasoning step, as discussed in Section~\ref{sec-reasoning}. The correct nodes, having higher similarity scores with the query, help guide the retrieval process toward the ground truth.
Conversely, in Figure~\ref{fig-map}(b),
since the first node ("University of Lausanne") helps narrow the search space and the last node ("frameless...") further filter candidates, both nodes have high saliency scores. Overall, our findings demonstrate that the reranker dynamically adapts its reliance on structural and textual knowledge depending on the dataset and query. 


% 
% \begin{figure*}[htpb!]
% \label{}
% \centering

%     {{\label{ROCIowaCedar} \includegraphics[width=\textwidth/3]{figures/IowaCedar_roc.png}}}%
%     \qquad
%     {{\label{ROCIowaDesMoines} \includegraphics[width=\textwidth/3]{figures/IowaDesMoines_roc.png} }%
%   \captionsetup{justification=centering}
%   \caption{\Acf{ROC} curves for \acf{RW} Iowa (CR) and  \acf{RW} Iowa (DM) dataset. Dummy model here represents a model whose output is solely a ``no Flood'' for all pixels.}
%   \label{fig:RW_ROC_Curves}%
% \end{figure*}



\section{Results and Discussions}
\label{sec:Results}

In this section, we aim to answer three main questions. First, we want to validate our hypothesis that \ac{SYN} data is a viable proxy for \ac{RW} data when training ML models for downscaling. Secondly, we seek to assess how much more skillful ML-based downscaling is compared to classical, non-data-driven techniques, such as our baseline methods, \textit{i.e.}, thresholded bicubic and Lanczos interpolation. Finally, we would like to appraise the extent to which data-driven models like ours are transferable (in terms of usefulness) to other regions without major performance degradations.  
To assess the quality of the models, we conduct a multiple comparison test --namely the Holm-Bonferroni procedure \cite{HolmBonferroni1979} -- that is designed to control the \ac{FWER}. We notice that, with a \ac{FWER} of $10^{-3}$, all the differences in model performance are significant. The only exception to this trend was observed in \ac{RW}-GH for whom the pairwise differences between \ac{RCAN} and \ac{ESRT}, Lanczos and Bicubic were not significant with the aforementioned \ac{FWER}. 

%Finally, we aim to find out the factors influencing the transferability of our models from one region to another.

\subsection{Potential of using SYN Data for RW downscaling}

In order to evaluate the utility of synthetic data for training, we compare performances of our candidate models on both \ac{SYN} and \ac{RW} Iowa data whose results are presented in Table \ref{tab:IowaResults}. We notice that 
\textbf{(i)} For the Iowa datasets, there is a drop in performance of all the models when going from \ac{SYN} to \ac{RW} datasets, 
\textbf{(ii)} for the \ac{RW}-IA (CR) as well as \ac{RW}-IA (DM) datasets, both bicubic and Lanczos interpolation have accuracies and MCC up to 70.89\% and 0.42 respectively while the deep learning models have accuracies and MCC up to 73.34\% and 0.46 respectively, 
\textbf{(iii)} There is a roughly 6\% accuracy improvement for the \ac{SYN} data for the deep learning models compared to the bicubic and lanczos models and this improvement drops to about 3\% for \ac{RW} data,  
\textbf{(iv)} the performance of all the models remain consistent across both \ac{RW}-IA datasets and \textbf{(v)} in \figref{fig:RW_ROC_Curves}, we observe that there is a high degree of overlap among the \ac{ROC} curves for the data-driven models.

From (i) and (iv) we can conclude that \ac{SYN} data is more intricate than \ac{RW} data. This implies that the benefits yielded by training with \ac{SYN} dataset, while significant, is not as prominent in the \ac{RW} Iowa datasets. 
% This may be due to sensor noise prevalent in the \ac{RW} Landsat-8 data that can be harder to reproduce in the synthetically generated examples. 
(i), (iii) and (v) implies that while \ac{SYN} data is not an exact replacement for \ac{RW} data, it provides a rather significant edge, which is all the more important when there is insufficient \ac{RW} for training. From (ii) we can conclude that the three proposed data driven models outperform classical super-resolution techniques such as bicubic and lanczos, conclusion supported by the \ac{ROC} curves in Figure \ref{fig:RW_ROC_Curves} for whom the data-driven models, in general, lie above the non-data-driven alternatives. Observation (iv) shows that  for the climatically similar \ac{RW}-Iowa(CR) and \ac{RW}-Iowa(DM) regions, training on \ac{SYN} Iowa data does indeed provide an edge. 

% have similar climate. 

\begin{figure*}[t!]
    \centering
    \begin{subfigure}[t]{0.5\textwidth}
        \centering
        \includegraphics[width=\textwidth/2]{figures/IowaCedar_roc.png}
        \caption{}
    \end{subfigure}%
    ~ 
    \begin{subfigure}[t]{0.5\textwidth}
        \centering
        \includegraphics[width=\textwidth/2]{figures/IowaDesMoines_roc.png}
        \caption{}
    \end{subfigure}
    \vspace*{0.5cm}
    \caption{    \label{fig:RW_ROC_Curves} \Acf{ROC} curves for (a) RW-IA (CR) and (b) RW-IA (DM) dataset. Na\"ive model here represents a model whose output is solely a ``no Flood'' for all pixels. Star here represents the pixel-wise classifier with a threshold of 0.5.}
\end{figure*}


\subsection{Effectiveness of data-driven approaches}

In order to evaluate the effectiveness of ML models in the downscaling task, we compare performances of our candidate models to Lanczos and bicubic interpolation methods by looking at figures of some sample predictions from Iowa (Figure \ref{fig:RWIowaDesMoines}), performance comparison in the region of Iowa in Table \ref{tab:IowaResults} and the ROC curves in Figure \ref{fig:RW_ROC_Curves} for \ac{RW} data. We notice that 
\textbf{(vi)} For RW-IA (DM) samples, the deep learning models maintain a higher degree of spatial continuity in the predicted \ac{FIM}, 
\textbf{(vii)} We observe that  bicubic and Lanczos interpolation produces over-smoothed \ac{FIM} reconstructions, while the plain \ac{RDN}, \ac{RCAN} and \ac{ESRT} models are more detail-inclusive. Similar conclusions can be drawn upon inspecting the \ac{ROC} curves in Figure \ref{fig:RW_ROC_Curves} and 
\textbf{(viii)} For RW-IA (CR), the ML models show a performance improvement of 3.06\% when comparing the best ML model and non-data-driven method and, while for RW-IA (DM) there is a performance improvement of 2.45\%.


Figures \ref{fig:EUSamples} and \ref{fig:RWIowaDesMoines} show the spatial disparity among the models whose details are often obscured in aggregated metrics such as accuracy. (vi) This implies that these data-driven models are better are recognizing an underlying stream network geometry than the classical methods. However, when it comes to narrow river streams, all the models struggle capturing the nuances of the \ac{FIM} resultant from localized high elevation features such as small islands within rivers or man-made structures. (vii) shows a clear advantage of our data-driven approaches over the non-data-driven alternatives. (viii) indicates the benefits of the data-driven models when evaluated over Iowa. 



\subsection{Applicability of our models to external regions}

To evaluate how transferable our models are, we draw conclusions from figures of the sample predictions from Western Europe (Figure \ref{fig:EUSamples}) and Ghana (Figure \ref{fig:GhanaSamples}) as well as the performance comparison in Table \ref{tab:ExternalResults}. We notice that 
\textbf{(ix)} for Ghana all of the models fail to adequately inundate the pixels over separated areas on account of several disconnected regions of inundation in the chosen area,
\textbf{(x)} the ML models outperform non-data driven methods for RW-EU, 
\textbf{(xi)} for the RW-EU dataset, there is an improvement of 4.89\% when comparing the accuracy of the best data- and non-data-driven methods, 
\textbf{(xii)} For RW-RR and RW-GH, there is marginal improvement (up to 0.77\% in accuracy) of the ML methods over the non-data driven methods and 
\textbf{(xiii)} For RW-EU, we notice that the ML models produce more connected streams over the non-data-driven models. 

(x) and (xi) implies that the models are transferable when considering hydroclimaticalogically similar regions since Iowa and the Meuse river in Europe lie within mid temperate zones. Similar to the observation (vi) for RW-IA (DM), (xiii) implies that the benefits of the ML model in identifying underlying network streams is also transferable to hydroclimatologically similar regions. In contrast, (xii) and (ix) both imply that the trained ML models struggle to generalize to RW-RR \& RW-GH. We speculate that this may be due to the significant differences in geography and climate when compared to Iowa. 

% More specifically, we notice that Ghana has a lot of disconnected regions when compared to Iowa and Western Europe, possibly indicating a geomorphological dissimilarity. Additionally, in the case of Red River and Ghana, we also speculate that they include drivers to flood inundation that are different from Iowa and Western Europe, which lie within mild temperate zones. Ghana on the other hand has a tropical (dry and hot) climate.

Our study directly implies that good quality synthetic data can be useful surrogates for downscaling low-resolution \acp{WFM} to high-resolution \acp{FIM} in regions, where such data are hard to come by, even when downscaling by a factor of 10. We noticed that such models were readily transferable to climatically similar regions as the region of training. However, Such derived ML models did not feature significantly different transferability when evaluated over hydroclimatologically dissimilar regions, which we attribute to different flood inundation characteristics, primarily at finer scales. A possible avenue to circumvent such issues is to explore additional training approaches that fall under the general area of domain adaptation. Nevertheless, data-driven models are still advantageous (and, hence, preferable) over non-data-driven alternatives in transfer scenarios like the one we considered here. 


%%%%%%%%%%%%%%%%%%%%%%%%%%%%%%% unused text %%%%%%%%%%%%%%%%%%%%%%%%%%%%%%%%%%%%%%%



% \tabref{tab:AccuracyResults} depicts test accuracies obtained by our models on both \ac{SYN} and \ac{RW} data. For Iowan floods, a comparison of \ac{SYN} and \ac{RW} results shows \textbf{(i)} bicubic and Lanczos interpolations remarkably gaining about $3\%$ in accuracy, as well as \textbf{(ii)} \ac{RDN} and \ac{RCAN} remaining relatively stable, while \textbf{(iii)} topography-aware models loosing $2.7\%$ in performance. From (i) one can conclude that \ac{SYN} data are morphologically slightly more intricate than \ac{RW} data. Also, (i) and (ii) likely imply that \ac{SYN} data, excluding topography, can serve as satisfactory surrogates of \ac{RW} data. However, as implied by (iii), our topography-dependent models seems to be particularly sensitive to distributional shifts of their combined inputs (\acp{WFM} and topographic features). More specifically, the topography-informed models' performance edge, while still statistically significant, is extremely marginal, even when compared to our non-data-driven approaches. Next, when comparing results between the cases of Iowan and Ghanaian \ac{RW} data, one observes that \textbf{(iv)} the accuracy of bicubic and Lanczos interpolations drops by almost $5\%$ due to over-smoothing. This may imply that Ghanaian \acp{FIM} bare a more complex morphology, when compared to Iowan \acp{FIM}. Also, \textbf{(v)} our topography-agnostic, data-driven models' performance degrades more gracefully (by about $2\%$), while \textbf{(vi)} our topography-aware models perform, virtually, as bad as our non-data-driven approaches. Hence, the differences in the data populations of the two regions we considered are significant enough to render our topography-dependent models noncompetitive. 




Software development is increasingly conceived as a collaboration activity between developers and AIs. Indeed, IDEs already implement features to enable interactive development, with AI suggesting implementations that are reused by developers.

Although multiple studies show this interaction can be successful, there is still limited understanding of how the models must be configured and used in the context of code generation tasks. This study addresses this gap, systematically investigating the impact of several key parameters, including the repeated submission of a prompt to accommodate for the non-deterministic nature of the models.

Our study reveals several key findings about the usage of ChatGPT. In particular, we discovered how creativity, although up to a limited extent, is useful to increase the range of methods whose code can be generated correctly. A major role is played by parameter top-p, which is commonly underrated, and instead has a major impact on the correctness of the results, with lower values producing better results. Finally, prompts should be submitted multiple times, with $5$ repetitions combined with a temperature of $1.2$ resulting in an effective configuration in our experiments.  

Future work concerns two main research directions. One is about replicating this experiment with other AI assistants, to validate our findings in multiple contexts. The second research direction concerns finding strategies to deal with the need to submit the same prompt multiple times to obtain a useful result, and thus developing approaches able to select or merge multiple responses automatically. 

\section*{Limitations}
While \tripcraft \;significantly enhances the realism and coherence of travel planning datasets, certain limitations remain. One key constraint is the exclusion of events from the core Point of Interest (PoI) list due to the lack of structured temporal information. If such data becomes available, it can be incorporated into our dataset, enabling LLMs to explicitly factor in event timings, ultimately leading to more mature and contextually aware itineraries.

Our dataset is currently designed for U.S. cities, but the construction pipeline can be extended to other geographical regions if the necessary data is available. Expanding to a global scale would require accounting for region-specific differences in travel preferences, transportation infrastructure, and cultural factors, which remain open challenges for future research.

Additionally, our primary objective is not to propose a novel travel planning methodology but to establish a robust dataset, enhanced itinerary generation, and fine-grained evaluation metrics. While \tripcraft \;provides a strong benchmark for assessing LLM-driven travel planning, future research may explore diverse methodologies on this dataset, further advancing personalized AI-driven itinerary synthesis.


\section*{Ethics Statement}
Our study utilizes publicly available web data, which we have carefully scraped to construct our databases while ensuring compliance with relevant terms of use and ethical considerations. To safeguard privacy, we have fully anonymized sensitive personal details. However, with annotators' consent, aggregate demographic statistics are provided in the Appendix. We are committed to transparency and reproducibility in research; therefore, upon acceptance, we will release both the datasets and code to facilitate further exploration by the scientific community.

% \section*{Acknowledgments}



% Bibliography entries for the entire Anthology, followed by custom entries
%\bibliography{anthology,custom}
% Custom bibliography entries only
\bibliography{custom}

\clearpage
\appendix
% \onecolumn

% \section{Appendices}

\noindent{\LARGE \textbf{Appendices}}  % Large, bold title without section numbering

\vspace{0.5cm}
% \label{sec:appendix}

This supplementary material presents additional details on the following aspects:  
\begin{itemize}
    \setlength{\itemsep}{2pt}
    \item \textbf{Appendix A:} Data Sourcing Details
    \item \textbf{Appendix B:} Prompt and Annotation Details  
    \item \textbf{Appendix C:} Case Studies
    \item \textbf{Appendix D:} Annotator Details 
\end{itemize}


\section{Related Work}
\label{appendix:A}
\textbf{Learning with Noisy Labels (LNL)} has been an active research area in recent years \cite{li2024transferring, wang2024tackling, xia2023regularly, xia2022extended, 9784878, 10689264, baek2024sam, englesson2024robust}, with numerous methods proposed to mitigate the impact of label noise on deep neural networks (DNNs). Formally, let \( X \) denote the input space, and let \( Y = \{1, 2, \dots, C\} \) be the set of clean labels. Consider the clean data distribution \( P(X, Y) \), from which clean samples \( (\mathbf{x}, y) \) are drawn. In practice, we often have access only to a training dataset with potentially noisy labels:
\begin{equation}
\tilde{D} = \{(\mathbf{x}_i, \tilde{y}_i)\}_{i=1}^n,
\end{equation}
where \( \mathbf{x}_i \in X \) and \( \tilde{y}_i \in Y \) are observed noisy labels. The aim is to learn a robust classifier \( f: X \rightarrow Y \) parameterized by \( \theta \), which performs well on clean test data drawn from the distribution \( P(X, Y) \).
The noise process is typically modeled using a noise transition matrix \( T \in \mathbb{R}^{C \times C} \), defined as:
\begin{equation}
T_{ij} = P(\tilde{y} = j \mid y = i), \quad \text{for } i, j \in \{1, 2, \dots, C\},
\end{equation}
which represents the probability that a clean label \( y = i \) is flipped to a noisy label \( \tilde{y} = j \). The relationship between the clean and noisy label distributions can be expressed as:
\begin{equation}
P(\tilde{y} \mid \mathbf{x}) = \sum_{k=1}^C T_{k\tilde{y}} \, P(y = k \mid \mathbf{x}).
\end{equation}
In the context of deep learning, the classifier \( f_\theta(\mathbf{x}) \) is often trained by minimizing the empirical risk over the noisy dataset:
\begin{equation}
\min_{\theta} \frac{1}{n} \sum_{i=1}^n \ell\left(f_\theta(\mathbf{x}_i), \tilde{y}_i\right),
\end{equation}
where \( \ell(\cdot, \cdot) \) is a loss function, such as the cross-entropy loss:
\begin{equation}
\ell\left(f_\theta(\mathbf{x}), \tilde{y}\right) = -\log \left( f_\theta^{(\tilde{y})}(\mathbf{x}) \right),
\end{equation}
and \( f_\theta^{(\tilde{y})}(\mathbf{x}) \) denotes the predicted probability for class \( \tilde{y} \). However, due to label noise, directly minimizing this loss can lead to the model overfitting to noisy labels, degrading its performance on clean data. To address this issue, various strategies have been proposed.
In the following discussion, we focus on heuristic methods, specifically sample selection techniques, which do not rely on the explicit estimation of \( T \) but instead incorporate strategies to mitigate the impact of noisy labels.

\textbf{Sample selection strategies.}
Sample selection has been widely used in learning with noisy labels to improve the robustness of model training by prioritizing confident samples. An in-depth understanding of deep learning models, particularly their learning dynamics, has facilitated research in this area. Extensive studies on the \textit{learning dynamics} of DNNs have revealed that difficult clean examples are typically learned in the later stages of training \cite{arpit2017closer, toneva2018empirical, lin2024on}. 

In general, sample selection methods assign a statistical characteristic to each sample and select a subset of samples that fall below a certain threshold \cite{han2018co}. The selection indicator function \( s_i \) is defined as:
\begin{equation}
s_i = \begin{cases}
1, & \text{if } \ell\left(f_\theta(\mathbf{x}_i), \tilde{y}_i\right) \leq \tau, \\
0, & \text{otherwise},
\label{eq11}
\end{cases}
\end{equation}
where \( \tau \) is a dynamically adjusted threshold. The training objective becomes:
\begin{equation}
\min_{\theta} \frac{1}{\sum_{i=1}^n s_i} \sum_{i=1}^n s_i \ell\left(f_\theta(\mathbf{x}_i), \tilde{y}_i\right).
\end{equation}
A common approach is the small-loss trick, by focusing on low-loss samples, the model is less influenced by potentially mislabeled data. Methods like Co-teaching \cite{han2018co}, Co-teaching+ \cite{yu2019does}, JoCoR \cite{wei2020combating}, and Co-learning \cite{tan2021co} utilize two networks trained in parallel that teach each other using reliable samples. SELF \cite{nguyen2019self} identifies clean samples by checking the consistency between network predictions and given labels, while DivideMix \cite{li2020dividemix} employs a two-component mixture model to separate the training data into clean and noisy groups.
Moreover, ELR \cite{liu2020early} avoid overfitting to noisy labels by relying on early-learning.


\textbf{Learning dynamics reaearch for sample selection.}
The intriguing generalization ability of modern DNNs has motivated extensive studies on their learning dynamics, which in turn has inspired a series of sample selection criteria using in Eq.(\ref{eq11}) based on these dynamics. Studies have revealed that hard and mislabeled examples are typically learned during the later stages of training \cite{arpit2017closer, toneva2018empirical, song2019does, song2021robust, maini2022characterizing, bai2021understanding, lin2024on}. This empirical observation has led to the development of various training-time metrics to quantify the ``hardness'' of examples, such as forgetting events \cite{toneva2018empirical}, example consistency \cite{pleiss2020identifying}, and learning speed \cite{maini2022characterizing, jiang2021characterizing}. These metrics have inspired LNL approaches that leverage learning dynamics to select clean samples.
Methods like Self-Filtering \cite{wei2022self}, FSLT \& SSFT \cite{maini2022characterizing}, SELFIE \cite{song2019selfie}, and RoCL \cite{zhou2021robust} adopt criteria to identify clean samples based on their learning dynamics. The success of learning dynamics-based sample selection criteria in identifying high-confidence clean samples has driven researchers to further refine these strategies.
By identifying a larger subset of clean samples for model training, the generalization performance of the trained model can be improved. \cite{xia2021sample} discovered that using loss alone to select CHEs is suboptimal. RLM \cite{li2024regroup} obtain robust loss estimation for noisy samples.

An advanced paradigm for sample selection involves a positive feedback loop: iteratively optimizing the classifier and updating the training set. Under this loop, the model's performance gradually improves, leading to better sample selection capabilities and, consequently, an enhanced ability to select clean hard examples. Me-Momentum \cite{bai2021me} and Late Stopping \cite{yuan2023late} employ similar positive feedback loops to iteratively update the model parameters and the training set, gradually improving the model's performance on noisy data.

\textbf{Hard label noise.}
Various forms of \textit{hard label noise} have been studied, including asymmetric noise~\cite{scott2013classification}, instance-dependent noise~\cite{xia2020part}, natural noise~\cite{wei2021learning}, adversarially crafted labels~\cite{zhang2024badlabel}, open-set noise~\cite{wei2021open}, and subclass-dominant noise~\cite{bai2023subclassdominant}. These noise types are designed from the perspective of the labels, aiming to simulate challenging real-world scenarios or malicious attacks. 
Recent work has also explored the impact of label noise in specific data distributions. For instance, H2E \cite{yi2022identifying} and TABASCO \cite{lu2023label} focus on the challenges posed by label noise in long-tailed distributions, where minority classes are more susceptible to mislabeling. NoiseCluster \cite{bai2023subclassdominant} introduces the concept of subclass-dominant label noise, where mislabeled examples dominate at least one subclass, leading to suboptimal classifier performance.

\textbf{Our contributions.}
In contrast to prior studies that mainly focus on different types of label noise or sample selection based on learning dynamics, our work offers a fresh perspective by re-examining sample selection methods that rely on a model's early learning stages. We demonstrate that some samples hidden among those considered ``confident'' are, in fact, the most harmful when mislabeled. Specifically, we systematically investigate the detrimental impact of \textit{Mislabeled Easy Examples (MEEs)}—mislabeled samples that are correctly predicted by the model early in the training process.
This insight challenges the conventional assumptions of existing methods, which often prioritize samples learned early in training as being clean. Our findings highlight the need for a more cautious approach when selecting samples based on early learning confidence. By adopting a refined sample selection criterion that accounts for the potential harm of MEEs, we can seamlessly integrate this approach with existing sample selection method \cite{yuan2023late} to further boost it performance.

\clearpage

\section{Detailed Settings}
\label{appendix:C}

\subsection{Datasets}

\emph{CIFAR-10} and \emph{CIFAR-100} \cite{krizhevsky2009learning} are standard image classification datasets consisting of $32 \times 32$ color images. Both datasets were divided into 50,000 training images and 10,000 test images.
\emph{CIFAR-N} \cite{wei2021learning} is a version of CIFAR-10 and CIFAR-100 with real-world noisy labels collected from Amazon Mechanical Turk. These datasets simulate real-world scenarios where labels may be noisy due to human error.
We used a consistent 90\%-10\% data splits for training and validation across runs in all competitors.


\emph{WebVision} \cite{li2017webvision} is a large-scale dataset containing over 2.4 million web images crawled from the internet. It covers the same 1,000 classes as the \emph{ILSVRC12} \emph{ImageNet-1K} dataset \cite{deng2009imagenet} but includes noisy labels due to the automatic collection process.
\emph{ILSVRC12} \emph{ImageNet-1K} \cite{deng2009imagenet} is a large-scale dataset of natural images with 1,000 classes. We used it to assess the scalability of our method on real-world data with synthetic noise.

\subsection{Noise Settings}

In preliminary presentation of our proposed method's effectiveness (Table \ref{tab1}), we tested four types of synthetic label noise. For \emph{Symmetric Noise}, each label has a fixed probability $r$ of being uniformly flipped to any other class.
\emph{Asymmetric Noise} flips labels to similar but incorrect classes, mimicking mistakes that might occur in real-world classification tasks. 
\emph{Pairflip Noise} involves flipping labels to a specific incorrect class in a pairwise manner.
\emph{Instance-Dependent Noise} \cite{xia2020part} is a more challenging setting where the probability of label corruption depends on the instance features. It reflects more realistic scenarios where difficult or ambiguous examples are more likely to be mislabeled.

Following prior practices \cite{bai2021me, yuan2023late}, we primarily focused on \emph{Symmetric} and \emph{Instance-Dependent} noise types in our baseline comparisons (Table \ref{tab2} and \ref{tab3}), as they are the most common and challenging synthetic noise settings used to evaluate robustness methods. We experimented with noise rates of 20\% and 40\% to assess our method's performance under varying noise intensities. 
For the \emph{CIFAR-N} task, we utilized the provided noisy labels.

\subsection{Model Architectures}

We employed variants of the ResNet architecture \cite{he2016deep} in all our experiments, training each model from scratch. Specifically, we used \emph{ResNet-18} for \emph{CIFAR-10}, \emph{ResNet-34} for \emph{CIFAR-100}, and \emph{ResNet-50} for \emph{WebVision} and \emph{ImageNet-1K} datasets. This selection aligns with previous works and provides appropriate model capacity relative to each dataset.

\subsection{Training Procedures and Hyperparameters}

Training was performed using \emph{Stochastic Gradient Descent (SGD)} with a momentum of 0.9 and a weight decay of $5 \times 10^{-4}$. The initial learning rate was set to 0.1 and decayed using a cosine annealing schedule without restarts, decreasing to $1 \times 10^{-5}$ over the course of training. 
The number of training epochs was set to 300 for \emph{CIFAR}, 200 for \emph{WebVision}, and 150 for full \emph{ImageNet-1K} experiments. Batch sizes were set to 32 for \emph{CIFAR} datasets and \emph{WebVision}, and 256 for \emph{ImageNet-1K}.

To enhance the robustness of our sample selection model, we also incorporated certain strategies from prior works \cite{linlearning, li2024regroup}, training two networks and each network learn from the other's soft predictions and utilizing exponential moving averages to stabilize training. Weak data augmentation techniques were applied during training to improve generalization. These included random cropping with a padding of 4 pixels, random horizontal flipping, and normalization using the dataset-specific mean and standard deviation. 


\subsection{Sample Selection Mechanism}

Building upon the \emph{Late Stopping} strategy \cite{yuan2023late}, we iteratively select a confident subset $\mathcal{D}^s$ of training samples, progressively reducing mislabeled data and enhancing the model's focus on clean samples.
We identify early-learned samples based on their \emph{learning times}. For each sample $(\mathbf{x}_i, y_i)$, we define its learning time $L_i$ as the earliest epoch when the model's prediction stabilizes:
\begin{equation}
L_i = \min \left\{ T_i \ \big| \ \hat{y}_i^{T_i-2} = \hat{y}_i^{T_i-1} = \hat{y}_i^{T_i} = y_i \right\},
\label{eq16}
\end{equation}
where $\hat{y}_i^t$ denotes the model's predicted label at epoch $t$.

To further address the issue of \emph{Mislabeled Easy Examples} (MEEs), we introduce an \emph{Early Cutting} step in the training loop. We first select candidates using an \emph{Early Cutting Rate} $\gamma$ of 1.5, which corresponds to selecting the earliest $\approx \frac{2}{3}$ of samples learned. Within these candidates, we remove samples that meet all three of the following criteria (detailed in Section \ref{sec3}). First, we consider samples with high loss, specifically those within the top 10\% of loss values $L_i$. Second, we look at samples with high prediction confidence, namely those within the top 20\% of confidence scores $c_i$. Third, we identify samples with low gradient norms, that is, those within the bottom 20\% of gradient norms $g_i$. By removing samples that satisfy all three conditions, we aim to eliminate MEEs that the model has confidently mislearned early on.

The refined subset $\mathcal{D}'^s$ is then used for subsequent training. We repeat the sample selection process for a total of $I_{\text{rate}}$ rounds (set to 3), progressively improving data quality and model performance. The proportion of $\mathcal{D}^s$ retained in each round is calculated to achieve an overall retention rate equal to the complement of the noise rate after $I_{\text{rate}}$ rounds. For example, with a noise rate of 40\% (aiming to retain 60\% of the data), the retention rate per round is $(60\%)^{1/3} \approx 84\%$.

\subsection{Algorithm}
\label{appendix:B}

\begin{algorithm}[h]
\caption{Iterative Sample Selection with Early Cutting}\label{alg:proposed_method}
\begin{algorithmic}[1]
\REQUIRE Training data $\mathcal{D}^0$; Number of iterations $I_{\text{rate}}$; Early cutting rate $\gamma$
\ENSURE Trained model parameters $\theta$
\FOR{$i = 1$ to $I_{\text{rate}}$}
    \STATE \emph{1. Base Sample Selection:}
    \FORALL{samples $(\mathbf{x}_j, y_j) \in \mathcal{D}^{i-1}$}
        \STATE Compute learning time $L_j$ using Eq.~(\ref{eq16})
        \STATE Select subset $\mathcal{D}^s$ using learning time $L_j$
    \ENDFOR
    \STATE \emph{2. Early Cutting:}
    \FORALL{samples $(\mathbf{x}_j, y_j) \in \mathcal{D}^s$}
        \STATE Select subset $\mathcal{D'}^s$ with earliest $\frac{1}{\gamma}$ proportion of samples in $\mathcal{D}^s$ based on $L_j$
        \STATE Compute confidence score $c_j$ and gradient norm $g_j$
        \STATE Remove samples from $\mathcal{D'}^s$ satisfying all the following conditions:
        \STATE \quad (a) Large loss; (b) High confidence; (c) Low gradient norm.
        \STATE Update training set $\mathcal{D}^{i}$
    \ENDFOR
\ENDFOR
\STATE \textbf{Final Training Phase:} Train model on refined $\mathcal{D}^{I_{\text{rate}}}$ until convergence
\RETURN Trained model parameters $\theta^*$
\end{algorithmic}
\end{algorithm}

\subsection{Training Time}
\label{appb7}
\begin{table*}[h]
\renewcommand{\arraystretch}{1.0}
\centering
	\caption{A comparison of training hours for ResNet-34 on CIFAR-100 (Tested on servers with a single RTX 4090 GPU)}
	\label{tab7}
\resizebox{1.0\textwidth}{!}{
\setlength{\tabcolsep}{2mm}{
\begin{tabular}{ccccccc}
\toprule
 Me-Momentum \cite{bai2021me} & Late Stopping \cite{yuan2023late} & RLM \cite{li2024regroup}  & CSGN \cite{linlearning} & Early Cutting (Ours)\\
\midrule
 $15$ hours & $17$ hours & $4$ hours & $9$ hours & $15$ hours  \\
\bottomrule  
\end{tabular}
}
}
\end{table*}

\subsection{Baselines and Competitors}

We re-implemented these methods under the same experimental settings as our proposed method. When re-implementing CSGN \cite{linlearning}, we used the \texttt{AdamW} optimizer and a stepped decay learning rate schedule, as specified in the original code. Notably, CSGN \cite{linlearning} cannot handle tasks with too many classes such as ImageNet-1k well.

\clearpage



\onecolumn
\section{Links to open-source project}
\label{appendix:B}

We summarize the datasets, methods, and models used in this work, as shown in Table~\ref{table:7}, Table~\ref{table:8}, and Table~\ref{table:9}, respectively.

\begin{table*}[h]
\fontsize{7.6}{9.}\selectfont
\caption{Links to datasets.}
\label{table:7}
\centering
  \begin{tabular}{l | l }
    \toprule
    \toprule
    \multirow{1}*{Dataset} & Link\\
    \midrule
    AdvBench & \url{https://github.com/llm-attacks/llm-attacks/tree/main/data/advbench} \\
    MaliciousInstruct & \url{https://github.com/Princeton-SysML/Jailbreak_LLM/blob/main/data} \\
    \bottomrule
   \bottomrule
  \end{tabular}
% \vspace{-1.5em}
\end{table*}

\begin{table*}[h]
\fontsize{7.6}{9.}\selectfont
\caption{Links to baseline methods.}
\label{table:8}
\centering
\begin{tabular}{p{1.8cm} | p{10.5cm}}
\toprule
\toprule
\multirow{1}*{Method} & Link\\
\midrule
GCG & \url{https://github.com/llm-attacks/llm-attacks/tree/main/llm_attacks} \\
PAIR & \url{https://github.com/patrickrchao/JailbreakingLLMs} \\
Perplexity Filter & \url{https://huggingface.co/docs/transformers/en/perplexity} \\
Instruction Filter & \url{https://huggingface.co/meta-llama/Llama-Guard-3-8B} \\
SmoothLLM & \url{https://github.com/arobey1/smooth-llm} \\
APAs & \url{https://huggingface.co/datasets/CHATS-Lab/Persuasive-Jailbreaker-Data/blob/main/adv_bench_sub_gpt4.jsonl} \\
\bottomrule
\bottomrule
\end{tabular}
% \vspace{-1.5em}
\end{table*}

\begin{table*}[h]
\fontsize{7.6}{9.}\selectfont
\caption{Links to large language models.}
\label{table:9}
\centering
  \begin{tabular}{l | l }
    \toprule
    \toprule
    \multirow{1}*{Model} & Link\\
    \midrule
    Bert-Base & \url{https://huggingface.co/google-bert/bert-base-cased} \\    
    Bert-Large & \url{https://huggingface.co/google-bert/bert-large-uncased} \\
    Llama-2-7B & \url{https://huggingface.co/meta-llama/Llama-2-7b-hf} \\
    Llama-2-7B-Chat & \url{https://huggingface.co/meta-llama/Llama-2-7b-chat-hf} \\
    Llama-2-13B-Chat & \url{https://huggingface.co/meta-llama/Llama-2-13b-chat-hf} \\
    Llama-3.1-8B-Instruct & \url{https://huggingface.co/meta-llama/Meta-Llama-3.1-8B-Instruct} \\
    Vicuna-13B-V1.5 & \url{https://huggingface.co/lmsys/vicuna-13b-v1.5} \\
    Mixtral-7B-Instruct & \url{https://huggingface.co/mistralai/Mistral-7B-Instruct-v0.2} \\
    GPT-2-Large & \url{https://huggingface.co/openai-community/gpt2-large} \\
    GPT-3.5 & \url{https://platform.openai.com} \\
    GPT-4-0613 & \url{https://platform.openai.com} \\
    Claude-3.5-Sonnet & \url{https://console.anthropic.com} \\
    Gemini-1.5-Flash  & \url{https://aistudio.google.com} \\
    \bottomrule
   \bottomrule
  \end{tabular}
\vspace{-1.2em}
\end{table*}






\newpage
\section{Case Studies}
We showcase examples in this appendix that highlight the challenges within our dataset, underscoring its value in advancing LLMs' travel planning capabilities.

\definecolor{lightyellow}{RGB}{255, 249, 196} % Light yellow for background
\definecolor{fluorescentyellow}{RGB}{255, 235, 59}

\begin{longtable}{|>{\columncolor{lightyellow}}p{0.8\textwidth}|}
    \hline
    \textbf{PoI List Time Exceeds the Departure Limit} \\ \hline
    %\endfirsthead
%
    %\hline
    %\textbf{POI List Time Exceeds the Departure Limit Continued} \\ \hline
    %\endhead

    %Table content starts here
    \makecell[l]{\textbf{Query:} \\ Plan a 3-day trip for 1 person from Denver to Santa Fe from November 1st to \\ November 3rd, 2024, with a budget of \$1,200.} \\ \hdashline
    \makecell[l]{\textbf{Plan:} \\ \{ ...\\       "days": 3,\\       "current\_city": "from Santa Fe to Denver", \\       "transportation": "Flight Number: F3932864, from Santa Fe to Denver, \\ \fcolorbox{fluorescentyellow}{fluorescentyellow}{Departure Time: 13:05}, Arrival Time: 14:16", \\       … \\       "point\_of\_interest\_list": "Cozy cottage in central Santa Fe, stay from 07:00 to \\ 08:30, nearest transit: Cerrillos @ 5th OB, 46.71m away; Tia Sophia's, visit from \\ 09:00 to 09:50, nearest transit: Sandoval @ San Francisco OB, 104.85m away; \\ Museum of International Folk Art, visit from 10:30 to 12:30, nearest transit: Cam. \\ Lejo @ Museum of Int'l. Folk Art, 73.48m away; \fcolorbox{fluorescentyellow}{fluorescentyellow}{La Plazuela, visit from 13:00 to} \\ \fcolorbox{fluorescentyellow}{fluorescentyellow}{14:00}, nearest transit: Cathedral @ Water, 130.45m away." \\     \}} \\ \hdashline
    \makecell[l]{\textbf{Analysis:} \\ In this case, the plan recommends visiting an attraction after the departure time of \\ the flight.} \\ \hline
\end{longtable}

% \begin{figure*}
%     \centering
%     \includegraphics[scale=1]{Images/C_1.png}  % Change filename accordingly
%     \caption{Your caption describing the image.}
%     \label{fig:your_label}
% \end{figure*}
\vspace{20pt}

\begin{longtable}{|>{\columncolor{lightyellow}}p{0.8\textwidth}|}
    \hline
    \textbf{Timestamp Sequence Inconsistency} \\ \hline
    
    %Table content starts here
    \makecell[l]{\textbf{Query:} \\ Plan a 3-day trip for 3 people from Charleston to Washington from November 24th \\ to November 26th, 2024, with a budget of \$3,750. Include visits to sights and \\ landmarks.} \\ \hdashline
    \makecell[l]{\textbf{Plan:} \\ \{ \\       "days": 1, \\ "current\_city": "from Charleston to Washington",\\ … \\ "point\_of\_interest\_list": "\fcolorbox{fluorescentyellow}{fluorescentyellow}{King Bed | Free Parking | DCA Airport, stay from 09:30} \\ \fcolorbox{fluorescentyellow}{fluorescentyellow}{to 08:30}, nearest transit: ENTRANCE B - FORT MYER DR., 360.84m away; \\ Clyde's of Georgetown, visit from 09:40 to 10:30, nearest transit: ROSSLYN, \\ 1120.37m away; White House, visit from 11:00 to 14:00, nearest transit: \\ MCPHERSON SQ METRORAIL STATION - SW CORNER VERMONT AVE. \& \\ I ST., 409.99m away; Joe's Seafood Prime Steak \& Stone Crab, visit from 14:20 \\ to 15:20, nearest transit: MCPHERSON SQ, 150.59m away; The Capital Grille, \\ visit from 20:45 to 22:00, nearest transit: ARCHIVES, 121.81m away; King Bed | \\ Free Parking | DCA Airport, stay from 22:30 to 08:30, nearest transit: ENTRANCE \\ B - FORT MYER DR., 360.84m away." \\ ... \}} \\ \hdashline
    \makecell[l]{\textbf{Analysis:} \\ In this case, the plan mis-structures the time sequence when mentioning the time \\ duration to be spent at the accommodation.} \\ \hline
\end{longtable}

% \begin{figure*}
%     \centering
%     \includegraphics[scale=1]{Images/C_2.png}  % Change filename accordingly
%     \caption{Your caption describing the image.}
%     \label{fig:your_label}
% \end{figure*}
% \newpage
\vspace{20pt}


\begin{longtable}{|>{\columncolor{lightyellow}}p{0.8\textwidth}|}
    \hline
    \textbf{Meal Timing Misalignment} \\ \hline
    
    %Table content starts here
    \makecell[l]{\textbf{Query:} \\ Plan a 3-day trip for 1 person from Birmingham to Detroit from November 19th \\ to November 21st, 2024, with a budget of \$1,500.} \\ \hdashline
    \makecell[l]{\textbf{Plan:} \\ \{ ...\\       "days": 3,\\ … \\ "dinner": "Vertical Detroit, Detroit", \\ … \\ "point\_of\_interest\_list": "Modern Boutique Condo - "Au coeur de Detroit", stay \\ from 08:30 to 09:30, nearest transit: Sproat St - Southbound, 302.98m away; \\ Honest John's Bar \& Grill - Selden St., visit from 09:40 to 10:30, nearest transit: \\ Mack Ave - Southbound, 509.79m away; The Ford Piquette Avenue Plant, visit \\ from 11:00 to 14:00, nearest transit: Amsterdam St - Southbound, 528.25m away; \\ Selden Standard, visit from 14:30 to 15:30, nearest transit: Canfield St \\ - Southbound, 586.42m away; \fcolorbox{fluorescentyellow}{fluorescentyellow}{Vertical Detroit, visit from 16:00 to 17:15}, nearest \\ transit: Grand Circus - Northbound, 257.87m away." \\ \}} \\ \hdashline
    \makecell[l]{\textbf{Analysis:} \\ In this case, the plan schedules dinner from 16:00 to 17:15, which is unrealistic and \\ impractical.} \\ \hline
\end{longtable}

% \begin{figure*}
%     \centering
%     \includegraphics[scale=1]{Images/C_3.png}  % Change filename accordingly
%     \caption{Your caption describing the image.}
%     \label{fig:your_label}
% \end{figure*}

\vspace{20pt}

\begin{longtable}{|>{\columncolor{lightyellow}}p{0.8\textwidth}|}
    \hline
    \textbf{Consistent PoI Time Duration Adherence} \\ \hline
    
    %Table content starts here
    \makecell[l]{\textbf{Query:} \\ Plan a 3-day trip for 1 person from Las Vegas to Fort Wayne from November 17th \\ to November 19th, 2024, with a budget of \$2,100.} \\ \hdashline
    \makecell[l]{\textbf{Plan:} \\ \{ \\       "days": 1, \\ "current\_city": "from Las Vegas to Fort Wayne", \\ "transportation": "Flight Number: F0373120, from Las Vegas to Fort Wayne, \\ Departure Time: 23:24, \fcolorbox{fluorescentyellow}{fluorescentyellow}{Arrival Time: 06:06}", \\ "breakfast": "-", \\ "attraction": "-",} \\ 
    \makecell[l]{"lunch": "-", \\ "dinner": "-", \\ "accommodation": "Sunny Victorian 1-Bedroom apartment near Downtown, \\ Fort Wayne", \\ "event": "-", \\ "point\_of\_interest\_list": "Sunny Victorian 1-Bedroom apartment near Downtown, \\ \fcolorbox{fluorescentyellow}{fluorescentyellow}{stay from 06:30 to 07:00}, nearest transit: Calhoun St. at Taber St. (SB), 76.02m \\ away." \\ \}, \\ \{ \\ "days": 2, \\ "current\_city": "Fort Wayne", \\ "transportation": "-", \\ "breakfast": "Sara's Family Restaurant, Fort Wayne", \\ "attraction": "Fort Wayne Museum of Art, Fort Wayne; Historic Fort Wayne, \\ Fort Wayne", \\ "lunch": "Fort Wayne's Famous Coney Island, Fort Wayne", \\ "dinner": "The Original Oley's Pizza, Fort Wayne", \\ "accommodation": "Sunny Victorian 1-Bedroom apartment near Downtown, \\ Fort Wayne", \\ "event": "-", \\ "point\_of\_interest\_list": "Sunny Victorian 1-Bedroom apartment near Downtown, \\ \fcolorbox{fluorescentyellow}{fluorescentyellow}{stay from 07:00 to 08:30}, nearest transit: Calhoun St. at Taber St. (SB), 76.02m \\ away; Sara's Family Restaurant, \fcolorbox{fluorescentyellow}{fluorescentyellow}{visit from 09:00 to 09:50}, nearest transit: Kroger, \\ 210.11m away; Fort Wayne Museum of Art, \fcolorbox{fluorescentyellow}{fluorescentyellow}{visit from 10:30 to 13:30}, nearest \\ transit: Main St. just past bus hut west of Lafayette St. (WB), 34.80m away; \\ Fort Wayne's Famous Coney Island, \fcolorbox{fluorescentyellow}{fluorescentyellow}{visit from 14:00 to 15:00}, nearest transit: W. \\ Main St. at Harrison St. (WB), 31.97m away; Historic Fort Wayne, \fcolorbox{fluorescentyellow}{fluorescentyellow}{visit from} \\ \fcolorbox{fluorescentyellow}{fluorescentyellow}{15:30 to 18:30}, nearest transit: Spy Run Ave. at Baltes Ave. (NB), 92.88m away; \\ The Original Oley's Pizza, \fcolorbox{fluorescentyellow}{fluorescentyellow}{visit from 19:00 to 20:15}, nearest transit: Liberty Mills \\ Apts., 6101 Cornwallis Dr., 2376.66m away; Sunny Victorian 1-Bedroom \\ apartment near Downtown, \fcolorbox{fluorescentyellow}{fluorescentyellow}{stay from 21:00 to 07:00}, nearest transit: Calhoun St. \\ at Taber St. (SB), 76.02m away." \\ \}, \\ \{ \\ "days": 3, \\ "current\_city": "from Fort Wayne to Las Vegas", \\ "transportation": "Flight Number: F2558843, from Fort Wayne to Las Vegas, \\ \fcolorbox{fluorescentyellow}{fluorescentyellow}{Departure Time: 13:16}, Arrival Time: 14:01", \\ "breakfast": "Cosmos House Of Pancakes, Fort Wayne", \\ "attraction": "Fort Wayne Children's Zoo, Fort Wayne", \\ "lunch": "-", \\ "dinner": "-", \\ "accommodation": "-", \\ "event": "-", \\ "point\_of\_interest\_list": "Sunny Victorian 1-Bedroom apartment near Downtown, \\ \fcolorbox{fluorescentyellow}{fluorescentyellow}{stay from 07:00 to 08:30}, nearest transit: Calhoun St. at Taber St. (SB), 76.02m \\ away; Cosmos House Of Pancakes, \fcolorbox{fluorescentyellow}{fluorescentyellow}{visit from 09:00 to 09:50}, nearest transit: \\ Canterbury Marketplace, 18.19m away; Fort Wayne Children's Zoo, \fcolorbox{fluorescentyellow}{fluorescentyellow}{visit from} \\ \fcolorbox{fluorescentyellow}{fluorescentyellow}{10:30 to 12:30}, nearest transit: Franke Park, 101.97m away." \\ \}} \\ \hdashline
    \makecell[l]{\textbf{Analysis:} \\ In this case, the plan accurately follows predefined time constraints for Points of \\ Interest (PoIs), ensuring structured scheduling.} \\ \hline
\end{longtable}

% \begin{figure*}
%     \centering
%     \includegraphics[scale=0.5]{Images/C_4.png}  % Change filename accordingly
%     \caption{Your caption describing the image.}
%     \label{fig:your_label}
% \end{figure*}

% \newpage
\vspace{20pt}


\begin{longtable}{|>{\columncolor{lightyellow}}p{0.8\textwidth}|}
    \hline
    \textbf{Inconsistent Itinerary with Adventure Seeker Profile} \\ \hline
    
    %Table content starts here
    \makecell[l]{\textbf{Query:} \\ Plan a 3-day trip for 2 people from Miami to Baltimore from November 3rd \\ to November 5th, 2024, with a budget of \$1,800. The itinerary should include visits \\ to zoos and aquariums, as well as fun and games attractions. Additionally, \\ incorporate arts and theatre, and film events. Accommodations should allow pets.} \\ \hdashline
    \makecell[l]{\textbf{Plan:} \\ "persona": \fcolorbox{fluorescentyellow}{fluorescentyellow}{"Traveler Type: Adventure Seeker}; Purpose of Travel: Adventure; \\ Spending Preference: Luxury Traveler; Location Preference: Mountains", \\ "plan": [ \\ \{ \\ "days": 1, \\ … \\ \fcolorbox{fluorescentyellow}{fluorescentyellow}{"attraction": "National Aquarium, Baltimore"}, \\ "lunch": "Miss Shirley's Cafe, Inner Harbor", \\ "dinner": "Restaurante Tio Pepe", \\ "accommodation": "Charming Baltimore Home w/Parking - Dog Friendly!", \\ … \\ \} \\ ]} \\ \hdashline
    \makecell[l]{\textbf{Analysis:} \\ In this case, the plan fails to align with the Adventure Seeker persona type, which \\ expects more than 2 attractions per day.} \\ \hline
\end{longtable}

% \begin{figure*}
%     \centering
%     \includegraphics[scale=1]{Images/C_5.png}  % Change filename accordingly
%     \caption{Your caption describing the image.}
%     \label{fig:your_label}
% \end{figure*}

\vspace{20pt}

\begin{longtable}{|>{\columncolor{lightyellow}}p{0.8\textwidth}|}
    \hline
    \textbf{Constraint-Compliant Itinerary Generation} \\ \hline
    
    %Table content starts here
    \makecell[l]{\textbf{Query:} \\ Plan a 3-day trip for 2 people from Portland to Baltimore from November 5th to \\ November 7th, 2024, with a budget of \$1,500. The trip should include visits to \\ Boat Tours \& Water Sports and Sights \& Landmarks. Note that children under 10 \\ are allowed, and no self-driving transportation should be used.} \\ \hdashline
    \makecell[l]{\textbf{Plan:} \\ "persona": "Traveler Type: Laidback Traveler; Purpose of Travel: Relaxation; \\ Spending Preference: Economical Traveler; Location Preference: Mountains", \\ "plan": [ \\ \{ \\ "days": 1, \\ "current\_city": "from Portland to Baltimore", \\ "transportation": "Flight Number: F0060549, from Portland to Baltimore,} \\ 
    \makecell[l]{Departure Time: 05:42, Arrival Time: 07:07", \\ "breakfast": "Miss Shirley's Cafe, Inner Harbor, Baltimore", \\ "attraction": "Top of the World Observation Level, Baltimore", \\ "lunch": "Phillips Seafood, Baltimore", \\ "dinner": "The Capital Grille, Baltimore", \\ "accommodation": "Private Queen Suite (B) Fells and Hopkins Medical, \\ Baltimore", \\ "event": "-", \\ "point\_of\_interest\_list": "Private Queen Suite (B) Fells and Hopkins Medical, stay \\ from 07:30 to 08:30, nearest transit: JOHNS HOPKINS STATION (METRO), \\ 821.12m away; Miss Shirley's Cafe, Inner Harbor, visit from 09:00 to 09:50, \\ nearest transit: SHOT TOWER METRO West Entrance, 285.29m away; Top of the \\ World Observation Level, visit from 10:30 to 13:30, nearest transit: SHOT TOWER \\ METRO West Entrance, 453.53m away; Phillips Seafood, visit from 14:00 to \\ 15:00, nearest transit: SHOT TOWER METRO West Entrance, 344.26m away; \\ The Capital Grille, visit from 19:00 to 20:15, nearest transit: SHOT TOWER \\ METRO West Entrance, 293.21m away; Private Queen Suite (B) Fells and Hopkins \\ Medical, stay from 21:00 to 07:00, nearest transit: JOHNS HOPKINS STATION \\ (METRO), 821.12m away." \\ \}, \\ \{ \\ "days": 2, \\ "current\_city": "Baltimore", \\ "transportation": "-", \\ "breakfast": "Blue Moon Cafe, Baltimore", \\ "attraction": "Federal Hill Park, Baltimore", \\ "lunch": "La Tavola, Baltimore", \\ "dinner": "Restaurante Tio Pepe, Baltimore", \\ "accommodation": "Private Queen Suite (B) Fells and Hopkins Medical, \\ Baltimore", \\ "event": "-", \\ "point\_of\_interest\_list": "Private Queen Suite (B) Fells and Hopkins Medical, stay \\ from 07:00 to 08:30, nearest transit: JOHNS HOPKINS STATION (METRO), \\ 821.12m away; Blue Moon Cafe, visit from 09:00 to 09:50, nearest transit: SHOT \\ TOWER STATION (METRO) sb, 1161.67m away; Federal Hill Park, visit from \\ 10:30 to 14:00, nearest transit: SHOT TOWER METRO West Entrance, 1089.59m \\ away; La Tavola, visit from 14:30 to 15:30, nearest transit: SHOT TOWER \\ STATION (METRO) sb, 472.69m away; Restaurante Tio Pepe, visit from 19:00 to \\ 20:15, nearest transit: LEXINGTON MARKET METRO North Entrance, 554.30m \\ away; Private Queen Suite (B) Fells and Hopkins Medical, stay from 21:00 to \\ 07:00, nearest transit: JOHNS HOPKINS STATION (METRO), 821.12m away." \\ \}, \\ \{ \\ "days": 3, \\ "current\_city": "from Baltimore to Portland", \\ "transportation": "Flight Number: F0683635, from Baltimore to Portland, \\ Departure Time: 21:41, Arrival Time: 23:01", \\ "breakfast": "Miss Shirley's Cafe, Inner Harbor, Baltimore", \\ "attraction": "Edgar Allan Poe's Grave Site and Memorial, Baltimore",} \\ 
    \makecell[l]{"lunch": "Dalesio's of Little Italy Restaurant, Baltimore", \\ "dinner": "Thames Street Oyster House, Baltimore", \\ "accommodation": "-", \\ "event": "-", \\ "point\_of\_interest\_list": "Private Queen Suite (B) Fells and Hopkins Medical, stay \\ from 07:00 to 08:30, nearest transit: JOHNS HOPKINS STATION (METRO), \\ 821.12m away; Miss Shirley's Cafe, Inner Harbor, visit from 09:00 to 09:50, \\ nearest transit: SHOT TOWER METRO West Entrance, 285.29m away; Edgar \\ Allan Poe's Grave Site and Memorial, visit from 10:30 to 13:30, nearest transit: \\ LEXINGTON MARKET (METRO) sb, 246.16m away; Dalesio's of Little Italy \\ Restaurant, visit from 14:00 to 15:00, nearest transit: SHOT TOWER STATION \\ (METRO) sb, 597.76m away; Thames Street Oyster House, visit from 19:00 to \\ 20:15, nearest transit: SHOT TOWER STATION (METRO) sb, 1428.09m away." \\ \} \\ ]} \\ \hdashline
    \makecell[l]{\textbf{Analysis:} \\ In this case, the plan successfully adheres to persona-based preferences and local \\ constraints, ensuring alignment with user-defined travel expectations.} \\ \hline
\end{longtable}

% \begin{figure*}
%     \centering
%     \includegraphics[scale=0.45]{Images/C_6.png}  % Change filename accordingly
%     \caption{Your caption describing the image. Continued}
%     \label{fig:your_label}
% \end{figure*}

\vspace{20pt}

\begin{longtable}{|>{\columncolor{lightyellow}}p{0.8\textwidth}|}
    \hline
    \textbf{Unrealistic Transit Proximity Issue} \\ \hline
    
    %Table content starts here
    \makecell[l]{\textbf{Query:} \\ Plan a 3-day trip for 1 person from Tampa to Bloomington from November 5th to \\ November 7th, 2024, with a budget of \$1,650.} \\ \hdashline
    \makecell[l]{\textbf{Plan:} \\ \{ \\ "days": 1, \\ "current\_city": "from Tampa to Bloomington", \\ … \\ "accommodation": "Arcade House - Hot Tub, Games, Grill, FirePit, FUN, \\ Bloomington", \\ "event": "-", \\ "point\_of\_interest\_list": "Arcade House - Hot Tub, Games, Grill, FirePit, FUN, \\ stay from 07:00 to 08:30, \fcolorbox{fluorescentyellow}{fluorescentyellow}{nearest transit: Prow Rd @ Meadows Hospital,} \\ \fcolorbox{fluorescentyellow}{fluorescentyellow}{251003.54m away}; Miller Park Zoo, visit from 09:00 to 12:30, \fcolorbox{fluorescentyellow}{fluorescentyellow}{nearest transit:} \\ \fcolorbox{fluorescentyellow}{fluorescentyellow}{Belle \& Park Square, 252594.59m away}; Baxter's American Grill, visit from \\ 13:00 to 14:00, \fcolorbox{fluorescentyellow}{fluorescentyellow}{nearest transit: Prow Rd @ Meadows Hospital, 247021.85m away;} \\ Janko's Little Zagreb, visit from 19:00 to 20:15, nearest transit: Kirkwood Ave @ \\ B-Line Trail IB, 98.16m away; Arcade House - Hot Tub, Games, Grill, FirePit, \\ FUN, stay from 21:00 to 07:00, \fcolorbox{fluorescentyellow}{fluorescentyellow}{nearest transit: Prow Rd @ Meadows Hospital,} \\ \fcolorbox{fluorescentyellow}{fluorescentyellow}{251003.54m away}." \\ \}} \\ \hdashline
    \makecell[l]{\textbf{Analysis:} \\ In this case, the plan selects PoIs whose nearest transit stop is hundreds of\\ kilometers away from accommodations and attractions.} \\ \hline
\end{longtable}

% \begin{figure*}
%     \centering
%     \includegraphics[scale=1]{Images/C_7.png}  % Change filename accordingly
%     \caption{Your caption describing the image.}
%     \label{fig:your_label}
% \end{figure*}

\newpage
\twocolumn
\section{Source Model, Hyperparameters and Template Selection}
\label{appendix:D}

\textbf{Sources Model.}\hspace*{2mm}As shown in Table~\ref{table:10}, we can observe a minor decline in both ASR and AHS when using Bert-Base and Llama-2 as the source models.
For Bert-Base, this performance reduction can be simply attributed to the decrease in the quantity of model parameters.
For heavyweight Llama-2, this performance loss stems from the fact that the architecture of MLMs is more suitable for token-level intent perception.
We also extend our evaluation to include Vicuna-13B-V1.5, aligning with the original settings recommended in the GCG and PAIR.
As shown in Table~\ref{table:11}, our PiF method can effectively adapt to various source LLMs, consistently achieving superior performance with an ASR of nearly 100\% and an AHS of around 4.0.
More importantly, the above results effectively demonstrate that PiF consistently achieves excellent results across MLMs and CLMs.

\begin{table*}[h]
\setlength{\tabcolsep}{15pt}
\caption{{Comparison of different PiF's sources model on AdvBench.}}
\label{table:10}
\centering
  \begin{tabular}{l | c c}
    \toprule
    \toprule
    \multirow{1}*{Source Model} & ASR ($\uparrow$) & AHS ($\uparrow$) \\
    \midrule
     Bert-Large & \textbf{100.0} & \textbf{3.87} \\ 
     Bert-Base & 99.8 & 3.75 \\        
     Llama-2-7B-Chat & 97.7 & 3.40 \\ 
    \bottomrule
   \bottomrule
  \end{tabular}
\end{table*}


\begin{table*}[h]
\setlength{\tabcolsep}{8pt} % Adjust column separation
\caption{{Compare the result of jailbreaking attacks targeting Llama-2-7B-Chat on MaliciousInstruct.}}
\label{table:11}
\centering
  \begin{tabular}{l | c | c | c  c}
    \toprule
    \toprule
    {Method} & {GCG} & {PAIR} & \multicolumn{2}{c}{PiF} \\
    \midrule
    {Sources Model} & {Vicuna-13B-V1.5} & {Vicuna-13B-V1.5} & {Vicuna-13B-V1.5} & {Bert-Large} \\ 
    {ASR ($\uparrow$)} & {0.0}  & {84.0} & \cellcolor{SelfColor!15}{97.0} & \cellcolor{SelfColor!15}\textbf{{100.0}} \\ 
    {AHS ($\uparrow$)} & {1.0} & {2.36}  & \cellcolor{SelfColor!15}{3.67} & \cellcolor{SelfColor!15}\textbf{{4.65}} \\
    \bottomrule
    \bottomrule
  \end{tabular}
\end{table*}

\begin{figure*}[b]
\begin{center}
    \begin{subfigure}
    {
            \includegraphics[width=0.33\columnwidth]{image/Result_1.pdf}\hspace{-0.65em}
            \includegraphics[width=0.33\columnwidth]{image/Result_2.pdf}\hspace{-0.65em}
            \includegraphics[width=0.33\columnwidth]{image/Result_3.pdf}
    }
    \end{subfigure}
    \\        
    \vspace{-0.6em}
    \begin{subfigure}
    {
            \includegraphics[width=0.33\columnwidth]{image/Result_5.pdf}\hspace{-0.65em}
            \includegraphics[width=0.33\columnwidth]{image/Result_4.pdf}\hspace{-0.65em}
            \includegraphics[width=0.33\columnwidth]{image/Result_6.pdf}
    }
    \end{subfigure}
\vspace{-1.2em}
\caption{The impact of hyperparameters $\mathcal{N}$, $\mathcal{M}$, $\mathcal{K}$ (top row), and $\tau$, ${T}$, $\Theta$ (bottom row) on AdvBench.
}
\label{fig:5}
\end{center}
\end{figure*}

\textbf{Hyperparameters.}\hspace*{2mm}For variable control, we tune one hyperparameter at a time while keeping the others fixed.
As depicted in Figure~\ref{fig:5} (top row), we can observe that changing the values of $\mathcal{N}$, $\mathcal{M}$, and $\mathcal{K}$ leads to a slight reduction in both ASR and AHS.
This is because a search space that is too small will limit PiF's ability to identify vulnerabilities, whereas an excessively large search space will result in a scattered focus.
From Figure~\ref{fig:5} (bottom row), it is evident that both sharp and flat sampling distributions hinder effectiveness, while a moderate $\tau$ promotes the generation of stable jailbreaking attacks.
Meanwhile, we can observe that setting inadequate iterations $T$ or overly strict thresholds $\Theta$ leads to a substantial decrease in performance.
However, redundant iterations will result in unnecessary computational overhead, while looser thresholds may increase the attack's perplexity.
In light of the above trade-off, we set appropriate hyperparameters to achieve the optimal balance.

\textbf{Evaluation Template.}\hspace*{2mm}
From Table~\ref{table:12}, it is evident that our method achieves similar ASR across different evaluation templates, but the AHS tends to decrease as the template length increases.
Importantly, templates that contain tokens after the \texttt{[MASK]} position are unsuitable for the CLMs' architecture.

\begin{table*}[h]
\setlength{\tabcolsep}{8pt} % Adjust column separation
\caption{{Comparison of different PiF's evaluation templates on AdvBench.}}
\label{table:12}
\centering
  \begin{tabular}{l | c c}
    \toprule
    \toprule
    \multirow{1}*{Template} & ASR ($\uparrow$) & AHS ($\uparrow$) \\
    \midrule
     \texttt{This intent is [MASK]} & \textbf{100.0} & \textbf{3.87} \\ 
     \texttt{This is a [MASK] intent} (designed for MLMs) & 99.8 & 3.85 \\        
     \texttt{The intent of this sentence is [MASK]} & 99.8  & 3.73 \\ 
    \bottomrule
   \bottomrule
  \end{tabular}
\end{table*}



\end{document}
