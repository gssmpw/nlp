% This must be in the first 5 lines to tell arXiv to use pdfLaTeX, which is strongly recommended.
\pdfoutput=1
% In particular, the hyperref package requires pdfLaTeX in order to break URLs across lines.

\documentclass[11pt]{article}

% Change "review" to "final" to generate the final (sometimes called camera-ready) version.
% Change to "preprint" to generate a non-anonymous version with page numbers.
% \usepackage[review]{acl}
\usepackage[preprint]{acl}

% Standard package includes
\usepackage{times}
\usepackage{latexsym}

% For proper rendering and hyphenation of words containing Latin characters (including in bib files)
\usepackage[T1]{fontenc}
% For Vietnamese characters
% \usepackage[T5]{fontenc}
% See https://www.latex-project.org/help/documentation/encguide.pdf for other character sets

% This assumes your files are encoded as UTF8
\usepackage[utf8]{inputenc}

% This is not strictly necessary, and may be commented out,
% but it will improve the layout of the manuscript,
% and will typically save some space.
\usepackage{microtype}

% This is also not strictly necessary, and may be commented out.
% However, it will improve the aesthetics of text in
% the typewriter font.
\usepackage{inconsolata}

%Including images in your LaTeX document requires adding
%additional package(s)
\usepackage{graphicx}
\usepackage{amsmath}
\usepackage{amsfonts}
\usepackage{bm}
% \usepackage[table]{xcolor}
\usepackage{array}
\usepackage{booktabs}
\usepackage{multirow}
\usepackage{tabularx}
\usepackage{ulem}
\usepackage{float}

% For creating boxes
\usepackage[most]{tcolorbox}
%\usepackage{txfonts}
% For Appendix tables
\usepackage{longtable}
\usepackage{colortbl}
\usepackage{arydshln}
\usepackage{makecell}
\usepackage{placeins}
% If the title and author information does not fit in the area allocated, uncomment the following
%
%\setlength\titlebox{<dim>}
%
% and set <dim> to something 5cm or larger.
%\newcommand{\tripcraft}%{\texttt{TripCraft}}
\newcommand{\tripcraft}
{\textit{TripCraft}}
\definecolor{lightgray}{gray}{0.9}
\definecolor{highlightyellow}{rgb}{1,1,0.6}

% \title{TripCraft: Towards Spatio-Temporally Fine Grained Travel Planning using Large Language Models}

\title{\tripcraft: A Benchmark for Spatio-Temporally Fine Grained Travel Planning}


% \title{TripCraft: A Benchmark for spatio-temporally fine grained travel planning using GPT-4o}

% Author information can be set in various styles:
% For several authors from the same institution:
% \author{Author 1 \and ... \and Author n \\
%         Address line \\ ... \\ Address line}
% if the names do not fit well on one line use
%         Author 1 \\ {\bf Author 2} \\ ... \\ {\bf Author n} \\
% For authors from different institutions:
% \author{Author 1 \\ Address line \\  ... \\ Address line
%         \And  ... \And
%         Author n \\ Address line \\ ... \\ Address line}
% To start a separate ``row'' of authors use \AND, as in
% \author{Author 1 \\ Address line \\  ... \\ Address line
%         \AND
%         Author 2 \\ Address line \\ ... \\ Address line \And
%         Author 3 \\ Address line \\ ... \\ Address line}

% \author{First Author \\
%   Affiliation / Address line 1 \\
%   Affiliation / Address line 2 \\
%   Affiliation / Address line 3 \\
%   \texttt{email@domain} \\\And
%   Second Author \\
%   Affiliation / Address line 1 \\
%   Affiliation / Address line 2 \\
%   Affiliation / Address line 3 \\
%   \texttt{email@domain} \\}

% \author{
%  \textbf{Soumyabrata Chaudhuri\textsuperscript{1}},
%  \textbf{Pranav Ravindra Purkar\textsuperscript{1}},
%  \textbf{Ritwik Raghav\textsuperscript{1}},
%  \textbf{Shubhojit Mallick\textsuperscript{2}},
% \\
%  \textbf{Manish Gupta\textsuperscript{2}},
%  \textbf{Abhik Jana\textsuperscript{1}},
%  \textbf{Shreya Ghosh\textsuperscript{1}},
% %  \textbf{Eighth Author \textsuperscript{1,2,3,4}},
% % \\
% %  \textbf{Ninth Author\textsuperscript{1}},
% %  \textbf{Tenth Author\textsuperscript{1}},
% %  \textbf{Eleventh E. Author\textsuperscript{1,2,3,4,5}},
% %  \textbf{Twelfth Author\textsuperscript{1}},
% % \\
% %  \textbf{Thirteenth Author\textsuperscript{3}},
% %  \textbf{Fourteenth F. Author\textsuperscript{2,4}},
% %  \textbf{Fifteenth Author\textsuperscript{1}},
% %  \textbf{Sixteenth Author\textsuperscript{1}},
% % \\
% %  \textbf{Seventeenth S. Author\textsuperscript{4,5}},
% %  \textbf{Eighteenth Author\textsuperscript{3,4}},
% %  \textbf{Nineteenth N. Author\textsuperscript{2,5}},
% %  \textbf{Twentieth Author\textsuperscript{1}}
% \\
% \\
%  \textsuperscript{1}IIT Bhubaneswar,
%  \textsuperscript{2}Microsoft, India
% \\
%  \small{
%    \textbf{Correspondence:} \href{mailto:email@domain}{email@domain}
%  }
% }
\author{
    \begin{tabular}{c}
        \textbf{Soumyabrata Chaudhuri\textsuperscript{1}},  
        \textbf{Pranav Purkar\textsuperscript{1}},  
        \textbf{Ritwik Raghav\textsuperscript{1}},  
        \textbf{Shubhojit Mallick\textsuperscript{2}} \\  
        \textbf{Manish Gupta\textsuperscript{2}},  
        \textbf{Abhik Jana\textsuperscript{1}},  
        \textbf{Shreya Ghosh\textsuperscript{1}}  
    \end{tabular}  
    \\
    \textsuperscript{1}School of Electrical and Computer Sciences, IIT Bhubaneswar, India \\  
    \textsuperscript{2}Microsoft, India \\  
    \begin{tabular}{c}  
        \texttt{chaudhurisoumyabrata@gmail.com} \\  
        \texttt{\{23cs06011,a23cs09001,abhikjana,shreya\}}@iitbbs.ac.in \\  
        \texttt{\{shubhojit.mallick,gmanish\}}@microsoft.com  
    \end{tabular}  
}


\begin{document}
\maketitle
\begin{abstract}
Recent advancements in probing Large Language Models (LLMs) have explored their latent potential as personalized travel planning agents, yet existing benchmarks remain limited in real-world applicability. Existing datasets, such as TravelPlanner and TravelPlanner+, suffer from semi-synthetic data reliance, spatial inconsistencies, and a lack of key travel constraints, making them inadequate for practical itinerary generation.
%rely on semi-synthetic data as well as ignoring several key components of travel planning, limiting their real-world applicability. 
To address these gaps, we introduce \textbf{\tripcraft},  
a spatio-temporally coherent travel planning  dataset that integrates real-world constraints, including public transit schedules, event availability, diverse attraction categories, and user personas for enhanced personalization.
%incorporating real-world constraints, including public transit schedules, public events, varied attraction categories, and user personas for enhanced personalization. Our dataset enables more detailed trip itinerary generation (including duration spent at each point of interest based on users' persona, transit between two points of interest, etc.) while ensuring spatio-temporal consistency. 
%TripCraft enables the generation of more structured and contextually rich itineraries by ensuring spatial and temporal alignment, and persona-aware activity selection.
%Further, we propose novel evaluation metrics (temporal meal score, attraction score, spatial score, ordering score, and persona score) to assess LLM-generated plans across temporal, spatial, sequential, and personal dimensions, overcoming the limitations of existing commonsense and hard constraint metrics.
To evaluate LLM-generated plans beyond existing binary validation methods, we propose five continuous evaluation metrics, namely Temporal Meal Score, Temporal Attraction Score, Spatial Score, Ordering Score, and Persona Score—which assess itinerary quality across multiple dimensions.
Our parameter-informed setting significantly enhances meal scheduling, improving the Temporal Meal Score from 61\% to 80\% in a 7-day scenario.
%Interestingly, our parameter-informed setting significantly enhances meal scheduling, improving performance from \textit{61\% to 80\% in the 7-day scenario- as quantified by a 19\% gain in our temporal meal score}. 
\tripcraft \footnote{Dataset and Codebase will be made publicly available upon acceptance.} establishes a new benchmark for LLM-driven personalized travel planning, offering a more realistic, constraint-aware framework for itinerary generation.
\end{abstract}

\section{Introduction}

Large language models (LLMs) have achieved remarkable success in automated math problem solving, particularly through code-generation capabilities integrated with proof assistants~\citep{lean,isabelle,POT,autoformalization,MATH}. Although LLMs excel at generating solution steps and correct answers in algebra and calculus~\citep{math_solving}, their unimodal nature limits performance in plane geometry, where solution depends on both diagram and text~\citep{math_solving}. 

Specialized vision-language models (VLMs) have accordingly been developed for plane geometry problem solving (PGPS)~\citep{geoqa,unigeo,intergps,pgps,GOLD,LANS,geox}. Yet, it remains unclear whether these models genuinely leverage diagrams or rely almost exclusively on textual features. This ambiguity arises because existing PGPS datasets typically embed sufficient geometric details within problem statements, potentially making the vision encoder unnecessary~\citep{GOLD}. \cref{fig:pgps_examples} illustrates example questions from GeoQA and PGPS9K, where solutions can be derived without referencing the diagrams.

\begin{figure}
    \centering
    \begin{subfigure}[t]{.49\linewidth}
        \centering
        \includegraphics[width=\linewidth]{latex/figures/images/geoqa_example.pdf}
        \caption{GeoQA}
        \label{fig:geoqa_example}
    \end{subfigure}
    \begin{subfigure}[t]{.48\linewidth}
        \centering
        \includegraphics[width=\linewidth]{latex/figures/images/pgps_example.pdf}
        \caption{PGPS9K}
        \label{fig:pgps9k_example}
    \end{subfigure}
    \caption{
    Examples of diagram-caption pairs and their solution steps written in formal languages from GeoQA and PGPS9k datasets. In the problem description, the visual geometric premises and numerical variables are highlighted in green and red, respectively. A significant difference in the style of the diagram and formal language can be observable. %, along with the differences in formal languages supported by the corresponding datasets.
    \label{fig:pgps_examples}
    }
\end{figure}



We propose a new benchmark created via a synthetic data engine, which systematically evaluates the ability of VLM vision encoders to recognize geometric premises. Our empirical findings reveal that previously suggested self-supervised learning (SSL) approaches, e.g., vector quantized variataional auto-encoder (VQ-VAE)~\citep{unimath} and masked auto-encoder (MAE)~\citep{scagps,geox}, and widely adopted encoders, e.g., OpenCLIP~\citep{clip} and DinoV2~\citep{dinov2}, struggle to detect geometric features such as perpendicularity and degrees. 

To this end, we propose \geoclip{}, a model pre-trained on a large corpus of synthetic diagram–caption pairs. By varying diagram styles (e.g., color, font size, resolution, line width), \geoclip{} learns robust geometric representations and outperforms prior SSL-based methods on our benchmark. Building on \geoclip{}, we introduce a few-shot domain adaptation technique that efficiently transfers the recognition ability to real-world diagrams. We further combine this domain-adapted GeoCLIP with an LLM, forming a domain-agnostic VLM for solving PGPS tasks in MathVerse~\citep{mathverse}. 
%To accommodate diverse diagram styles and solution formats, we unify the solution program languages across multiple PGPS datasets, ensuring comprehensive evaluation. 

In our experiments on MathVerse~\citep{mathverse}, which encompasses diverse plane geometry tasks and diagram styles, our VLM with a domain-adapted \geoclip{} consistently outperforms both task-specific PGPS models and generalist VLMs. 
% In particular, it achieves higher accuracy on tasks requiring geometric-feature recognition, even when critical numerical measurements are moved from text to diagrams. 
Ablation studies confirm the effectiveness of our domain adaptation strategy, showing improvements in optical character recognition (OCR)-based tasks and robust diagram embeddings across different styles. 
% By unifying the solution program languages of existing datasets and incorporating OCR capability, we enable a single VLM, named \geovlm{}, to handle a broad class of plane geometry problems.

% Contributions
We summarize the contributions as follows:
We propose a novel benchmark for systematically assessing how well vision encoders recognize geometric premises in plane geometry diagrams~(\cref{sec:visual_feature}); We introduce \geoclip{}, a vision encoder capable of accurately detecting visual geometric premises~(\cref{sec:geoclip}), and a few-shot domain adaptation technique that efficiently transfers this capability across different diagram styles (\cref{sec:domain_adaptation});
We show that our VLM, incorporating domain-adapted GeoCLIP, surpasses existing specialized PGPS VLMs and generalist VLMs on the MathVerse benchmark~(\cref{sec:experiments}) and effectively interprets diverse diagram styles~(\cref{sec:abl}).

\iffalse
\begin{itemize}
    \item We propose a novel benchmark for systematically assessing how well vision encoders recognize geometric premises, e.g., perpendicularity and angle measures, in plane geometry diagrams.
	\item We introduce \geoclip{}, a vision encoder capable of accurately detecting visual geometric premises, and a few-shot domain adaptation technique that efficiently transfers this capability across different diagram styles.
	\item We show that our final VLM, incorporating GeoCLIP-DA, effectively interprets diverse diagram styles and achieves state-of-the-art performance on the MathVerse benchmark, surpassing existing specialized PGPS models and generalist VLM models.
\end{itemize}
\fi

\iffalse

Large language models (LLMs) have made significant strides in automated math word problem solving. In particular, their code-generation capabilities combined with proof assistants~\citep{lean,isabelle} help minimize computational errors~\citep{POT}, improve solution precision~\citep{autoformalization}, and offer rigorous feedback and evaluation~\citep{MATH}. Although LLMs excel in generating solution steps and correct answers for algebra and calculus~\citep{math_solving}, their uni-modal nature limits performance in domains like plane geometry, where both diagrams and text are vital.

Plane geometry problem solving (PGPS) tasks typically include diagrams and textual descriptions, requiring solvers to interpret premises from both sources. To facilitate automated solutions for these problems, several studies have introduced formal languages tailored for plane geometry to represent solution steps as a program with training datasets composed of diagrams, textual descriptions, and solution programs~\citep{geoqa,unigeo,intergps,pgps}. Building on these datasets, a number of PGPS specialized vision-language models (VLMs) have been developed so far~\citep{GOLD, LANS, geox}.

Most existing VLMs, however, fail to use diagrams when solving geometry problems. Well-known PGPS datasets such as GeoQA~\citep{geoqa}, UniGeo~\citep{unigeo}, and PGPS9K~\citep{pgps}, can be solved without accessing diagrams, as their problem descriptions often contain all geometric information. \cref{fig:pgps_examples} shows an example from GeoQA and PGPS9K datasets, where one can deduce the solution steps without knowing the diagrams. 
As a result, models trained on these datasets rely almost exclusively on textual information, leaving the vision encoder under-utilized~\citep{GOLD}. 
Consequently, the VLMs trained on these datasets cannot solve the plane geometry problem when necessary geometric properties or relations are excluded from the problem statement.

Some studies seek to enhance the recognition of geometric premises from a diagram by directly predicting the premises from the diagram~\citep{GOLD, intergps} or as an auxiliary task for vision encoders~\citep{geoqa,geoqa-plus}. However, these approaches remain highly domain-specific because the labels for training are difficult to obtain, thus limiting generalization across different domains. While self-supervised learning (SSL) methods that depend exclusively on geometric diagrams, e.g., vector quantized variational auto-encoder (VQ-VAE)~\citep{unimath} and masked auto-encoder (MAE)~\citep{scagps,geox}, have also been explored, the effectiveness of the SSL approaches on recognizing geometric features has not been thoroughly investigated.

We introduce a benchmark constructed with a synthetic data engine to evaluate the effectiveness of SSL approaches in recognizing geometric premises from diagrams. Our empirical results with the proposed benchmark show that the vision encoders trained with SSL methods fail to capture visual \geofeat{}s such as perpendicularity between two lines and angle measure.
Furthermore, we find that the pre-trained vision encoders often used in general-purpose VLMs, e.g., OpenCLIP~\citep{clip} and DinoV2~\citep{dinov2}, fail to recognize geometric premises from diagrams.

To improve the vision encoder for PGPS, we propose \geoclip{}, a model trained with a massive amount of diagram-caption pairs.
Since the amount of diagram-caption pairs in existing benchmarks is often limited, we develop a plane diagram generator that can randomly sample plane geometry problems with the help of existing proof assistant~\citep{alphageometry}.
To make \geoclip{} robust against different styles, we vary the visual properties of diagrams, such as color, font size, resolution, and line width.
We show that \geoclip{} performs better than the other SSL approaches and commonly used vision encoders on the newly proposed benchmark.

Another major challenge in PGPS is developing a domain-agnostic VLM capable of handling multiple PGPS benchmarks. As shown in \cref{fig:pgps_examples}, the main difficulties arise from variations in diagram styles. 
To address the issue, we propose a few-shot domain adaptation technique for \geoclip{} which transfers its visual \geofeat{} perception from the synthetic diagrams to the real-world diagrams efficiently. 

We study the efficacy of the domain adapted \geoclip{} on PGPS when equipped with the language model. To be specific, we compare the VLM with the previous PGPS models on MathVerse~\citep{mathverse}, which is designed to evaluate both the PGPS and visual \geofeat{} perception performance on various domains.
While previous PGPS models are inapplicable to certain types of MathVerse problems, we modify the prediction target and unify the solution program languages of the existing PGPS training data to make our VLM applicable to all types of MathVerse problems.
Results on MathVerse demonstrate that our VLM more effectively integrates diagrammatic information and remains robust under conditions of various diagram styles.

\begin{itemize}
    \item We propose a benchmark to measure the visual \geofeat{} recognition performance of different vision encoders.
    % \item \sh{We introduce geometric CLIP (\geoclip{} and train the VLM equipped with \geoclip{} to predict both solution steps and the numerical measurements of the problem.}
    \item We introduce \geoclip{}, a vision encoder which can accurately recognize visual \geofeat{}s and a few-shot domain adaptation technique which can transfer such ability to different domains efficiently. 
    % \item \sh{We develop our final PGPS model, \geovlm{}, by adapting \geoclip{} to different domains and training with unified languages of solution program data.}
    % We develop a domain-agnostic VLM, namely \geovlm{}, by applying a simple yet effective domain adaptation method to \geoclip{} and training on the refined training data.
    \item We demonstrate our VLM equipped with GeoCLIP-DA effectively interprets diverse diagram styles, achieving superior performance on MathVerse compared to the existing PGPS models.
\end{itemize}

\fi 


\section{Related Works}
\label{sec:rw}

%-------------------------------------------------------------------------
\noindent \textbf{Vision-Language Model.}
In recent years, vision-language models, as a novel tool capable of processing both visual and linguistic modalities, have garnered widespread attention. These models, such as CLIP~\cite{clip}, ALIGN~\cite{ALIGN}, BLIP~\cite{BLIP}, FILIP~\cite{filip}, etc., leverage self-supervised training on image-text pairs to establish connections between vision and text, enabling the models to comprehend image semantics and their corresponding textual descriptions. This powerful understanding allows vision-language models (e.g., CLIP) to exhibit remarkable generalization capabilities across various downstream tasks~\cite{downsteam1,downsteam2,downsteam3,h2b}. To further enhance the transferability of vision-language models to downstream tasks, prompt tuning and adapter methods have been applied. However, methods based on prompt tuning (such as CoOp~\cite{coop}, CoCoOp~\cite{cocoop}, Maple~\cite{maple}) and adapter-based methods (such as Tip-Adapter~\cite{tip}, CLIP-Adapter~\cite{clip_adapter}) often require large amounts of training data when transferring to downstream tasks, which conflicts with the need for rapid adaptation in real-world applications. Therefore, this paper focuses on test-time adaptation~\cite{tpt}, a method that enables transfer to downstream tasks without relying on training data.

%-------------------------------------------------------------------------
\noindent \textbf{Test-Time Adaptation.}
Test-time adaptation~(TTA) refers to the process by which a model quickly adapts to test data that exhibits distributional shifts~\cite{tta1,memo,ptta,domainadaptor,dota}. Specifically, it requires the model to handle these shifts in downstream tasks without access to training data. TPT~\cite{tpt} optimizes adaptive text prompts using the principle of entropy minimization, ensuring that the model produces consistent predictions for different augmentations of test images generated by AugMix~\cite{augmix}. DiffTPT~\cite{difftpt} builds on TPT by introducing the Stable Diffusion Model~\cite{stable} to create more diverse augmentations and filters these views based on their cosine similarity to the original image. However, both TPT and DiffTPT still rely on backpropagation to optimize text prompts, which limits their ability to meet the need for fast adaptation during test-time. TDA~\cite{tda}, on the other hand, introduces a cache model like Tip-Adapter~\cite{tip} that stores representative test samples. By comparing incoming test samples with those in the cache, TDA refines the model’s predictions without the need for backpropagation, allowing for test-time enhancement. Although TDA has made significant improvements in the TTA task, it still does not fundamentally address the impact of test data distribution shifts on the model and remains within the scope of CLIP's original feature space. We believe that in TTA tasks, instead of making decisions in the original space, it would be more effective to map the features to a different spherical space to achieve a better decision boundary.

%-------------------------------------------------------------------------
\noindent \textbf{Statistical Learning.}
Statistical learning techniques play an important role in dimensionality reduction and feature extraction. Support Vector Machines~(SVM)~\cite{svm} are primarily used for classification tasks but have been adapted for space mapping through their ability to create hyperplanes that separate data in high-dimensional spaces. The kernel trick enables SVM to operate in transformed feature spaces, effectively mapping non-linearly separable data. PCA~\cite{pca} is a linear transformation method that maps high-dimensional data to a new lower-dimensional space through a linear transformation, while preserving as much important information from the original data as possible.

\renewcommand{\arraystretch}{1.3} % Adjust row height
\setlength{\tabcolsep}{6pt} % Adjust column spacin

% \section{\tripcraft \;}
\section{TripCraft}

% The first line of the file must be
% \begin{quote}
% \begin{verbatim}
% \documentclass[11pt]{article}
% \end{verbatim}
% \end{quote}

% To load the style file in the review version:
% \begin{quote}
% \begin{verbatim}
% \usepackage[review]{acl}
% \end{verbatim}
% \end{quote}
% For the final version, omit the \verb|review| option:
% \begin{quote}
% \begin{verbatim}
% \usepackage{acl}
% \end{verbatim}
% \end{quote}

% To use Times Roman, put the following in the preamble:
% \begin{quote}
% \begin{verbatim}
% \usepackage{times}
% \end{verbatim}
% \end{quote}
% (Alternatives like txfonts or newtx are also acceptable.)

% Please see the \LaTeX{} source of this document for comments on other packages that may be useful.

% Set the title and author using \verb|\title| and \verb|\author|. Within the author list, format multiple authors using \verb|\and| and \verb|\And| and \verb|\AND|; please see the \LaTeX{} source for examples.

% By default, the box containing the title and author names is set to the minimum of 5 cm. If you need more space, include the following in the preamble:
% \begin{quote}
% \begin{verbatim}
% \setlength\titlebox{<dim>}
% \end{verbatim}
% \end{quote}
% where \verb|<dim>| is replaced with a length. Do not set this length smaller than 5 cm.







\subsection{Overview}
% We introduce \tripcraft, a benchmark designed to evaluate the capability of language agents in complex, constraint-aware planning. Grounded in the real-world task of travel itinerary generation, \tripcraft \;assesses whether agents can construct flexible travel plans while adhering to diverse constraints, including user preferences, commonsense considerations, and persona-driven requirements. The benchmark further evaluates temporal, spatial, sequential, and personal consistency in generated plans.

% \tripcraft \;comprises $1000$ diverse travel queries, categorized into three groups based on trip duration: 3-day, 5-day, and 7-day itineraries with three different difficulty levels based on the availability of data (distribution given in Table \ref{tab:tripcraft_dist}). Each query is accompanied by a human-annotated reference plan and a rationale explaining the reasoning behind the chosen itinerary. \tripcraft \;offers a structured evaluation framework with well-defined constraints and human-curated plans, making it a robust benchmark for measuring the reasoning and planning abilities of language agents.




We introduce \tripcraft, a benchmark for evaluating language agents in complex, constraint-aware planning, specifically in travel itinerary generation. \tripcraft \;assesses agents' ability to construct flexible yet constrained plans by incorporating user preferences, commonsense considerations, and persona-driven requirements while ensuring temporal, spatial, sequential, and personal consistency. The benchmark comprises 1000 diverse travel queries, categorized into three trip durations (3-day, 5-day, and 7-day) and three difficulty levels based on data availability (distribution in Table \ref{tab:tripcraft_dist}). Each query includes a human-annotated reference plan with a rationale explaining the itinerary's reasoning. By providing a structured evaluation framework with well-defined constraints and human-curated plans, \tripcraft \;serves as a rigorous benchmark for measuring the reasoning and planning capabilities of language agents.


\begin{table}[h]
    \centering
    \renewcommand{\arraystretch}{0.8}
    \begin{tabular}{lcccc}
        \toprule
        \textbf{Days} & \textbf{Easy} & \textbf{Medium} & \textbf{Hard} & \textbf{Total}\\
        \midrule
        3-day & 130 & 106 & 108 & 344\\
        5-day & 87 & 164 & 73 & 324\\
        7-day & 99 & 178 & 55 & 332\\
        \bottomrule
    \end{tabular}
    \caption{\tripcraft \; dataset distribution}
    \label{tab:tripcraft_dist}
\end{table}


\begin{table*}[!t]
    \centering
    \rowcolors{3}{gray!15}{white} % Alternating row colors starting from the third row
    \begin{tabular}{>{\bfseries}l p{11cm}}
        \hline
        \multicolumn{2}{c}{\cellcolor{gray!25} \textbf{Commonsense Constraints}} \\
        \hline
        Sufficient Meal Gaps & Defines a minimum gap of four hours between the meals. \\
        Valid PoI list & Defines validity rules for the Point of Interest list. \\
        Diverse Events & Event choices should not be repeated throughout the trip. \\
        \hline
        \multicolumn{2}{c}{\cellcolor{gray!25} \textbf{Hard Constraints}} \\
        \hline
        Event Types & Event Types include four distinct categories—Sports, Arts \& Theatre, Music, and Film. \\
        Attraction Types &  Each attraction belongs to one or more of 15 predefined categories, ensuring a well-distributed selection of activities. \\
        \multicolumn{2}{c}{\cellcolor{gray!25} \textbf{Persona Components}} \\
        \hline
        Traveler Type & Defines how a traveler approaches their journey—whether they seek relaxation in cozy spots or adrenaline-pumping adventures. \\
        Purpose of Travel & Captures trip motivation. Examples: to unwind, explore cultures etc.\\
        Spending Preference &  Reflects the traveler’s budget and style, from luxurious indulgence to cost-conscious experiences. \\
        Location Preference &  Highlights preferred environments, such as beaches, mountains, cities, or wildlife-rich forests. \\
        \hline
    \end{tabular}
    \caption{Addition of constraints and persona details based on availability and limitations of scraped data has been given in the above table. The full list of constraints used in \tripcraft\;has been given in Table 9 of Appendix.}
    \label{tab:constraints}
\end{table*}
\subsection{Constraint and Persona details}
\label{subsec:const_and_pers}

\tripcraft \;integrates numerous constraints and persona components to enhance the evaluation of language agents in constraint-aware itinerary generation, as shown in Table~\ref{tab:constraints}. These improvements ensure that generated travel plans are realistic, well-structured, and aligned with user preferences.

\paragraph{Commonsense Constraints.} In addition to the commonsense constraints in TravelPlanner, we introduce refinements to improve itinerary realism. First, the same event should not be repeated multiple times across a trip, ensuring diversity in experiences. Second, meal timings must have a minimum gap of four hours between breakfast, lunch, and dinner to maintain a natural schedule. Third, the point-of-interest (PoI) list must follow strict validity rules: each day's itinerary must begin and end at the designated accommodation, except on the final day when the traveler departs. The list should be an ordered sequence of accommodations, attractions, and restaurants, ensuring adequate time gaps between flight arrivals and accommodation check-ins, as well as between accommodation check-outs and departures. %We also remove the minimum night constraint due to data limitations. 
%Additionally, all existing constraints were heavily modified to align with this new spatio-temporal setting.

\begin{table}[!b]
    \centering
    \renewcommand{\arraystretch}{0.8}
    \setlength{\tabcolsep}{8pt} % Adjust column spacing
    \begin{tabular}{l c}
        \toprule
        \textbf{Category} & \textbf{Duration (hrs)} \\
        \midrule
        Boat Tours \& Water Sports & 3.5 \\
        Casinos \& Gambling & 2.5 \\
        Museums & 3.0 \\
        Nature \& Parks & 4.5 \\
        Nightlife & 2.5 \\
        Sights \& Landmarks & 3.0 \\
        %Water \& Amusement Parks & 5.0 \\
        %Zoos \& Aquariums & 2.5 \\
        \bottomrule
    \end{tabular}
    \caption{Attraction visiting duration (hrs) for a subset of categories (due to page limit). The complete list of categories has been given in Table 8 of Appendix.}
    \label{tab:subcategory_dur_sample}
\end{table}

\paragraph{Hard Constraints.} We introduce two new hard constraints to enhance itinerary structuring. First, each attraction belongs to one or more of 15 diverse categories (see Table \ref{tab:subcategory_dur_sample}), ensuring a well-distributed selection of activities. Second, events are categorized into four distinct types — Sports, Arts \& Theatre, Music, and Film — allowing for a more structured and personalized planning process.
\paragraph{Persona Information.}  
Each query in \tripcraft \; is accompanied by a persona profile that influences travel planning, consisting of traveler type, purpose of travel, spending preference, and location preferences. Traveler type distinguishes between laid-back travelers, who prefer relaxation and scenic spots, and adventure seekers, who prioritize extreme activities like paragliding and bungee jumping. Purpose of travel refines traveler intent into categories such as relaxation, adventure, cultural exploration, and nature-focused experiences. Spending preference differentiates between luxury and budget-conscious travelers, shaping accommodation and activity choices. Location preferences specify favored destinations, such as beaches, mountains, cities, or wildlife/forest regions.

These persona aspects were carefully selected in consultation with domain experts as the most influential factors in determining a fulfilling travel experience. Unlike the work of \citet{singh-etal-2024-personal}, which includes demographic attributes such as age, gender, and education, \tripcraft \; deliberately omits such details to minimize potential biases in language model behavior.
\begin{table}[h]
    \centering
    \renewcommand{\arraystretch}{0.8}
    \begin{tabular}{lp{3cm}}
        \toprule
        \textbf{Database} & \textbf{Data Entries (\#)} \\
        \midrule
        City Set & 140 \\
        Flights & 3,446,829 \\
        Restaurants & 3,892 \\
        Attractions & 5,043 \\
        Accommodations & 2,800 \\
        Events & 21,980 \\
        Nearest Transit Stop & 8,723 \\
        Distance Matrix & 19,460 \\
        \bottomrule
    \end{tabular}
    \caption{Data entries in the database have been scraped to include the most recent data.}
    \label{tab:data_entries}
\end{table}






\subsection{Construction pipeline}
The benchmark construction (Figure \ref{fig:pipeline}) consists of three key steps as follows. %: 1) Data Collection and Cleaning, 2) Query and Persona Construction, and 3) Annotation and Refinement. The whole process has been illustrated with clarity in Figure \ref{fig:pipeline}.

\begin{itemize}  
    \item \textbf{Data Collection and Cleaning:} 
    % The databases are gathered through web scraping and open-source tools such as OpenStreetMap\footnote{\url{https://www.openstreetmap.org/}} (detailed in Appendix and Table \ref{tab:data_entries}). Due to the extensive size of the flights database, we adopt the dataset from the work of \citet{xie2024travelplanner}, adjusting dates to maximize overlap with event timelines, ensuring that a significant portion of queries could incorporate relevant events. Unlike earlier datasets with outdated information, ours is more recent, though some entries lack complete details. These were either carefully removed or filled with default values as necessary. Additionally, since GTFS public transit\footnote{\url{https://gtfs.org/}} data covers only 140 cities, all databases are filtered to include information exclusively for these cities.
    The databases are sourced via web scraping and open-source tools like OSM\footnote{\url{https://www.openstreetmap.org/}} (details in Appendix and Table \ref{tab:data_entries}). Given the extensive size of the flights database, we adopt the dataset from \citet{xie2024travelplanner}, adjusting dates to align with event timelines, ensuring queries incorporate relevant events while maintaining computational feasibility. Unlike prior datasets with outdated information, ours is more recent; however, missing or incomplete entries were either carefully removed or filled with default values to ensure consistency. Since GTFS public transit\footnote{\url{https://gtfs.org/}} data covers only 140 cities, all databases are filtered accordingly.


    \item \textbf{Persona and Query Construction:} 
    % Following \citet{xie2024travelplanner}, queries are constructed using randomly selected fundamental elements such as departure city, destination, and date range. The trip duration dictates the number of cities included: 3-day plans focuses on a single city, while 5-day and 7-day plans involve one randomly chosen state, with visits to 2 and 3 cities, respectively. This ensures that agents had to reason about multi-city itineraries while considering inter-city connectivity. To enhance complexity and realism, diverse user needs are introduced as hard constraints, and relevant persona profiles are assigned by selecting values from the four persona components. These structured inputs are then stitched together using GPT-4o\footnote{\url{https://openai.com/index/gpt-4o-system-card/}} in a few-shot setting to generate high-quality queries.
    Following \citet{xie2024travelplanner}, queries are generated by randomly selecting key elements—departure city, destination, and date range etc. Trip duration determines city coverage: 3-day plans focus on one city, while 5-day and 7-day plans span one state with visits to 2 and 3 cities, respectively, requiring agents to reason about multi-city itineraries and inter-city connectivity. To enhance complexity, hard constraints and persona profiles are incorporated. These structured inputs are then composed using GPT-4o\footnote{\url{https://openai.com/index/gpt-4o-system-card/}} in a few-shot setting to generate high-quality queries.

    \item \textbf{Annotation and Refinement:} 
    % A team of 25 graduate students are employed to carefully annotate the plans for synthesized queries\footnote{This task was completed during their internship at our NLP lab.}. Since each query can have multiple valid plans, annotators provide justifications for their choices, improving plan explainability. The annotation process underwent several iterative refinement rounds, incorporating feedback from the authors (verifiers) at each step to ensure nuanced interpretations of persona constraints. As a final validation, the verifiers manually review all query-plan pairs, employing evaluation scripts alongside manual checks to ensure feasibility and optimality. The annotation process in \tripcraft \; is significantly more challenging and time-consuming (\textasciitilde 30 minutes per annotation) than prior datasets due to the added temporal and spatial dimensions in the Point of Interest List which requires additional scrutiny.
    A team of 25 graduate students annotated plans for the queries\footnote{Interns at our NLP lab.}, providing justifications to enhance explainability. The process involved iterative refinements, integrating expert feedback to ensure nuanced interpretations of persona constraints. Domain Experts conducted a final manual review of all query-plan pairs, combining evaluation scripts with manual checks for feasibility and optimality. Annotation in \tripcraft \; is notably more demanding (\textasciitilde 30 minutes per instance) than prior datasets due to the added temporal and spatial complexities in the PoI list, requiring heightened scrutiny.
\end{itemize}


\begin{figure}[!t]
    \raggedleft
    \includegraphics[width=0.45\textwidth]{Images/pipeline_2-cropped.pdf}  % Change filename accordingly
    \caption{Visualization of the Construction Pipeline.}
    \label{fig:pipeline}
\end{figure}

\subsection{Evaluation Metrics}
\label{subsec:eva_metrics}
% The hard and commonsense constraints outlined in subsection \ref{subsec:const_and_pers} are employed to assess feasibility. In addition to these, continuous evaluation metrics are introduced to address the limitations of discrete binary checks, providing a more nuanced understanding of the temporal, spatial, sequential, and personal aspects of the plans. Our proposed metrics provide a comprehensive evaluation of a given plan by assessing its quality across temporal, spatial, sequential, and persona-specific dimensions. The benefit of using our metrics has been further elaborated on in subsection \ref{subsec:analysis}. The parameters of these metrics have been determined from our annotation distribution (Table \ref{tab:parameter_details_sample}). 
% In general, given annotations for any dataset, the parameters of these metrics can always be derived from the underlying distribution of the corresponding variables in the annotation.
Plan feasibility is assessed using the hard and commonsense constraints from subsection \ref{subsec:const_and_pers}. Furthermore, we introduce continuous evaluation metrics that capture temporal, spatial, sequential, and persona-specific nuances of a travel plan, addressing the limitations of discrete binary checks (the advantages of these metrics are detailed in subsection \ref{subsec:analysis}). The parameters of these metrics have been determined from our annotation distribution (Table \ref{tab:parameter_details_sample}).  More generally, given annotations for any dataset, metric parameters can always be inferred from the distribution of corresponding variables in the annotation.


\begin{table}[!t]
    \centering
    \renewcommand{\arraystretch}{0.9}
    \setlength{\tabcolsep}{2pt} % Adjust column spacing
    \begin{tabularx}{\columnwidth}{l *{3}{>{\centering\arraybackslash}X}}
        \toprule
        \textbf{Parameter} & \textbf{3-day} & \textbf{5-day} & \textbf{7-day} \\
        \midrule
        \multicolumn{4}{c}{\textbf{Restaurant Parameters}} \\
        \midrule
        \textbf{Breakfast} & & & \\
        %Mean Time & 9.63 & 9.80 & 9.84 \\ 
        Mean Duration (hrs) & 0.90 & 1.08 & 0.85 \\ 
        %Std. Time & 1.08 & 1.08 & 1.34 \\ 
        Std. Duration (hrs) & 0.24 & 1.43 & 0.23 \\ 
        %Beta & 0.21 & 0.63 & 0.03 \\ 
        \midrule
        \textbf{Lunch} & & & \\
        %Mean Time & 14.30 & 14.46 & 14.44 \\ 
        Mean Duration (hrs) & 1.11 & 1.10 & 0.99 \\ 
        %Std. Time & 1.03 & 1.07 & 1.07 \\ 
        Std. Duration (hrs) & 0.36 & 0.35 & 0.26 \\ 
        %Beta & 0.10 & 0.04 & 0.30 \\ 
        \midrule
        \textbf{Dinner} & & & \\
        %Mean Time & 20.75 & 20.67 & 20.42 \\ 
        Mean Duration (hrs) & 1.19 & 1.32 & 1.15 \\ 
        %Std. Time & 1.25 & 1.37 & 1.66 \\ 
        Std. Duration (hrs) & 0.43 & 0.91 & 1.15 \\ 
        %Beta & -0.20 & -0.18 & -0.07 \\ 
        \midrule
        \multicolumn{4}{c}{\textbf{Attraction Parameters}} \\
        \midrule
        $\lambda_{laidback}$ & 1.10 & 1.26 & 1.11 \\ 
        $\lambda_{adventurous}$ & 2.01 & 1.61 & 1.82 \\ 
        $\sigma_d$ (hrs) & 1.11 & 1.07 & 0.90 \\ 
        $n^{max}$ & 5 & 4 & 4 \\ 
        $n^{min}$ & 0 & 0 & 0 \\ 
        $k$ (hrs) & 0.28 & 0.28 & 0.28 \\ 
        \bottomrule
    \end{tabularx}
    \caption{Day-wise parameter details calculated from our annotation distribution. For a complete list of parameters refer to the Appendix.}
    \label{tab:parameter_details_sample}
\end{table}


\noindent\textbf{Temporal Meal Score: }
% For each meal type (breakfast, lunch, or dinner), it is computed based on a bivariate normal distribution. The use of a Gaussian distribution is motivated by prior work that has successfully employed Gaussian models for time series data \cite{ansari2024chronos,10.1007/978-3-642-41398-8_15}. Additionally, the Gaussian distribution facilitates smooth penalization of incorrect time intervals, ensuring a continuous and coherent representation of temporal dependencies. Formally, for a particular meal interval $a$, its temporal score $T^a_{meal}$ is calculated as follows:
For each meal type (breakfast, lunch, or dinner), the temporal meal score $T^a_{meal}$ is modeled using a bivariate normal distribution, motivated by prior work leveraging Gaussian models for time series data \cite{ansari2024chronos,10.1007/978-3-642-41398-8_15}. This choice enables smooth penalization of deviations, ensuring a continuous representation of temporal dependencies. Formally, $T^a_{meal}$ is computed as:
\begin{equation}
\small
    T^a_{meal} = \frac{1}{(2\pi)\sqrt{\bm{\Sigma}}} \exp \left( -\frac{1}{2} (\bm{x} - \bm{\mu})^T \bm{\Sigma}^{-1} (\bm{x} - \bm{\mu}) \right)
\end{equation}
% \begin{equation}
% \begin{split}
%     T^a_{meal} &= \frac{1}{(2\pi)\sqrt{\bm{\Sigma}}} \exp \Bigg( -\frac{1}{2} (\bm{x} - \bm{\mu})^T \\
%     &\quad \bm{\Sigma}^{-1} (\bm{x} - \bm{\mu}) \Bigg)
% \end{split}
% \end{equation}


\noindent where $ \bm{x} = [t_m, d_m] $ represents the midpoint time $ t_m $ and duration $ d_m $ of the meal, and $ \bm{\Sigma} $ is the determinant of the covariance matrix with the correlation coefficient \( \beta \). $ T^a_{meal} $ is normalized to obtain a score between 0 and 1 for each meal. And, these values are averaged over all the meals in a plan:
\begin{equation}
\small
    \bar{T}_{meal} = \frac{1}{M} \sum_{a=1}^{M} T_{meal}^{(a)}
\end{equation}
Here, $M$ is the total number of meals in the plan, and $T^a_{meal}$ represents the temporal score for meal interval $a$.

\noindent\textbf{Temporal Attraction Score:} Evaluates whether the visit duration of an attraction aligns with an expected duration for that type of attraction while also considering the total number of attractions visited in a day. It is modeled as the probability density function of a joint mixed random distribution combining a Gaussian for duration ($X$) and a Poisson for the number of attractions ($N$):
\begin{equation}
\small
    f_{X, N}(d_i, n) = f_{X \mid N}(d_i \mid n) \cdot P(N = n) 
\end{equation}
Averaging over all attractions, $n$, in a plan, we have:
\begin{equation}
\small
    \bar T_{attrac} = \frac{1}{n} \sum_{i=1}^{n} \exp \left( -\frac{(d_i - \mu^i_d)^2}{2\sigma_d^2} \right) \cdot \frac{\lambda^n e^{-\lambda}}{n!}
\end{equation}

\noindent Here, \( d_i \) represents the actual duration spent at attraction \( i \), while \( \mu^i_d \) denotes the expected visit duration for $i$, which is dynamically adjusted based on the number of attractions visited and attraction type. \( \sigma_d \) captures variability in visit durations. To incorporate the influence of the total number of attractions, we use a Poisson distribution \( P \), where n represents the number of attractions visited whose \( \lambda \) parameter represents the expected number of attractions for a given persona (e.g., adventurous or laidback). The expected visit duration \( \mu^i_d \) is adjusted dynamically:
\begin{equation}
\small
    \mu^i_d =
    \begin{cases}
        \mu_d^{\text{type}} - k (n - n^{\text{min}}), & \text{if adventurous} \\
        \mu_d^{\text{type}} + k (n^{\text{max}}- n), & \text{if laidback}
    \end{cases}
\end{equation}
where \( \mu_d^{\text{type}} \) represents the expected duration for the category of attraction $i$, and \( k \) denotes the rate at which the average visit duration changes per unit increase in the number of attractions. Intuitively, for a particular type of attraction, a laid-back traveler generally spends more time compared to an adventure seeker. Moreover, irrespective of the traveler's persona, the average duration spent at each attraction decreases as the total number of attractions visited in a day increases.


\begin{table*}[h]
    \centering
    \renewcommand{\arraystretch}{0.7}  % Reduce row height
    % \setlength{\tabcolsep}{4pt}
    \begin{tabular}{lcccccc}
        \toprule
        \textbf{Setting} & \textbf{Category} & \textbf{$\bar T_{meal}$} & \textbf{$\bar T_{attrac}$} & \textbf{$\bar S_{spatial}$} & \textbf{$\bar S_{persona}$} & \textbf{$\bar S_{ord}$} \\
        \midrule
        \multirow{3}{*}{\textit{w/o Parameter Info}} 
        & 3-day  & 0.56 & 0.0169 & 0.80 & 0.46 & 0.70 \\
        & 5-day  & 0.69 & 0.0057 & 0.86 & 0.49 & 0.91 \\
        & 7-day  & 0.61 & \textbf{0.0122} & 0.85 & 0.50 & 0.96 \\
        \midrule
        \multirow{3}{*}{\textit{w/ Parameter Info}} 
        & 3-day  & \textbf{0.70} & \textbf{0.0171} & \textbf{0.83} & \textbf{0.48} & \textbf{0.74} \\
        & 5-day  & \textbf{0.75} & \textbf{0.0078} & \textbf{0.87} & \textbf{0.50} & \textbf{0.93} \\
        & 7-day  & \textbf{0.80} & 0.0057 & \textbf{0.86} & \textbf{0.51} & \textbf{0.97} \\
        \bottomrule
    \end{tabular}
    \caption{Results of our proposed evaluation metrics for the two settings. For each category, the \textbf{bolded} values represent the better ones between the two settings.}
    \label{tab:new_results}
\end{table*}

\begin{table*}[h]
    \centering
    \renewcommand{\arraystretch}{0.7}  % Reduce row height
    % \setlength{\tabcolsep}{4pt}  % Reduce column spacing
    \begin{tabular}{p{3cm} p{1.5cm} ccccccc}
        \toprule
       % \multirow{2}{*}{\textbf{Setting}} & \multirow{2}{*}{\textbf{Category}} & \textbf{Delivery} & \multicolumn{2}{c}{\textbf{CPR}} & \multicolumn{2}{c}{\textbf{HCPR}} & \multirow{2}{*}{\textbf{Final Pass Rate}} \\
       % \cmidrule(lr){4-5} \cmidrule(lr){6-7} &
       %  & \textbf{Rate} & \textbf{Micro} & \textbf{Macro} & \textbf{Micro} & \textbf{Macro} & \\
       %  \midrule
\multirow{2}{*}{\textbf{Setting}} & \multirow{2}{*}{\textbf{Category}} & Delivery & \multicolumn{2}{c}{CPR} & \multicolumn{2}{c}{HCPR} & \multirow{2}{*}{Final Pass Rate} \\
\cmidrule(lr){4-5} \cmidrule(lr){6-7} &
& Rate & Micro & Macro & Micro& Macro & \\
\midrule
        \multirow{3}{*}{\textit{w/o Parameter Info}} 
        & 3-day  & 92.60 & 77.61 & \textbf{6.08} & \textbf{27.63} & \textbf{26.08} & \textbf{3.47} \\
        & 5-day  & \textbf{98.69} & \textbf{66.99} & \textbf{2.17} & 4.21 & 3.04 & \textbf{1.74} \\
        & 7-day  & \textbf{96.03} & \textbf{70.35} & \underline{0.00} & 0.62 & \underline{0.00} & \underline{0.00} \\
        \midrule
        \multirow{3}{*}{\textit{w/ Parameter Info}} 
        & 3-day  & \textbf{96.08} & \textbf{80.08} & 5.21 & 26.10 & 25.21 & 1.74 \\
        & 5-day  & 91.30 & 61.69 & 0.87 & \textbf{8.60} & \textbf{5.21} & 0.43 \\
        & 7-day  & 92.51 & 66.43 & \underline{0.00} & \textbf{0.83} & \underline{0.00} & \underline{0.00} \\
        \bottomrule
    \end{tabular}
    \caption{Results of existing evaluation metrics for the two settings. CPR and HCPR stand for Commonsense Pass Rate and Hard Constraint Pass Rate respectively. For each category, the \textbf{bolded} values represent the better ones between the two settings. Moreover, equal values have been \underline{underlined}. }
    \label{tab:old_results}
\end{table*}

\noindent\textbf{Spatial Score:} Computed based on the distance \( d \) of a point of interest from the nearest transit station:
\begin{equation}
\small
S_s(d) =
\begin{cases}
    1 - 0.5 \left( \frac{d}{d_0} \right), & \text{if } d \leq d_0 \\
    0.5 \exp \left( -\lambda (d - d_0) \right), & \text{if } d > d_0
\end{cases}
\end{equation}
\normalsize
where $d_0$ (= 5 km) is a threshold distance and $\lambda$ (= 0.0002) is the decay rate for larger distances. The final spatial score for a plan is the average of all individual PoI scores:
\begin{equation}
\small
\bar{S}_{spatial} = \frac{1}{N} \sum_{i=1}^{N} S_s(d_i), \quad N \text{ is the no. of PoIs visited.}
\end{equation}
\normalsize
% where \( N \) is the number of PoIs visited in all the days of the trip.
\noindent\textbf{Persona Score: }
% Quantifies the alignment between a traveler's persona components and the Points of Interest (PoIs) they visit. Building on prior work in PoI conflation \cite{sun2023conflating}, which explored methods based on PoI types and names, we opt for PoI names due to their richer semantic information—often including key adjectives like \textit{relaxing} or \textit{luxury}—compared to the typically concise PoI types. Accordingly, we compute the persona score as the average cosine similarity between BERT \cite{devlin-etal-2019-bert} embeddings of individual persona components and PoI names.
Quantifies the alignment between a traveler's persona and visited Points of Interest (PoIs). Prior work on PoI conflation \cite{sun2023conflating} explored type- and name-based methods; we prioritize PoI names as they offer richer semantics, often embedding key descriptors like \textit{relaxing} or \textit{luxury}. The persona score is computed as the average cosine similarity between BERT \cite{devlin-etal-2019-bert} embeddings of persona components and PoI names:
\begin{equation}
\small
\bar S_{persona} = \frac{1}{M \cdot N} \sum_{j=1}^{M} \sum_{i=1}^{N} \frac{\bm{p}_j \cdot \bm{q}_i}{\|\bm{p}_j\| \|\bm{q}_i\|}
\end{equation}
\normalsize
where \( \bm{p}_j \) represents the BERT embedding of the \( j \)-th persona component, \( \bm{q}_i \) is the BERT embedding of the \( i \)-th PoI name, \( M \) is the total number of persona components, and \( N \) is the total number of PoIs in the travel plan.

\noindent\textbf{Ordering Score: } Measures the sequential alignment of the generated PoI list with the annotated PoI list for a particular day:
\begin{equation}
\small
S_{\text{ord}} = 1 - \frac{\text{ED}(\mathcal{G}, \mathcal{A})}{\max(|\mathcal{G}|, |\mathcal{A}|)}
\end{equation}
\normalsize
where \( \mathcal{G} \) and \( \mathcal{A} \) represent the generated and annotated sequences of points of interest, respectively, and \( \text{ED}(\mathcal{G}, \mathcal{A}) \) denotes the Edit Distance \cite{Levenshtein1965BinaryCC} between them. This is averaged over all the days of the journey to get the ordering score for a plan, which is denoted by $\bar S_{ord}$.


\begin{figure*}[t]
    \centering
    \includegraphics[width=0.8\textwidth]{Images/example_v2-cropped.pdf}  % Change filename accordingly
    \caption{Existing constraint-based metrics rate these plans equally, but, our metrics provide a continuous score, capturing temporal, spatial, sequential, and personal differences.}
    \label{fig:potential_proposed_metric}
\end{figure*}

% \subsection{Methodology}



% \section{Methodology}
\subsection{Preliminary}
\label{sec:preliminary}
\mypara{Architecture of MLLM.}
% The MLLM architectures generally consist of three components: a visual encoder, a modality projector, and a LLM. The visual encoder, typically a pre-trained image encoder like CLIP's vision model, converts input images into visual tokens. The projector module aligns these visual tokens with the LLM's word embedding space, enabling the LLM to process visual data effectively. The LLM then integrates the aligned visual and textual information to generate responses.
The architecture of Multimodal Large Language Models (MLLMs) typically comprises three core components: a visual encoder, a modality projector, and a language model (LLM). Given an image $I$, the visual encoder and a subsequent learnable MLP are used to encode $I$ into a set of visual tokens $e_v$. These visual tokens $e_v$ are then concatenated with text tokens $e_t$ encoded from text prompt $p_t$, forming the input for the LLM. The LLM decodes the output tokens $y$ sequentially, which can be formulated as:
\begin{equation}
\label{eq1}
    y_i = f(I, p_t, y_0, y_1, \cdots, y_{i-1}).
\end{equation}

\mypara{Computational Complexity.}  
To evaluate the computational complexity of MLLMs, it is essential to analyze their core components, including the self-attention mechanism and the feed-forward network (FFN). The total floating-point operations (FLOPs) required can be expressed as:  
\begin{equation}
\text{Total FLOPs} = T \times (4nd^2 + 2n^2d + 2ndm),
\end{equation}  
where $T$ denotes the number of transformer layers, $n$ is the sequence length, $d$ represents the hidden dimension size, and $m$ is the intermediate size of the FFN.  
This equation highlights the significant impact of sequence length $n$ on computational complexity. In typical MLLM tasks, the sequence length is defined as: 
\begin{equation}
    n = n_S + n_I + n_Q, 
\end{equation}
where $n_I$, the tokenized image representation, often dominates, sometimes exceeding other components by an order of magnitude or more.  
As a result, minimizing $n_I$ becomes a critical strategy for enhancing the efficiency of MLLMs.

\subsection{Beyond Token Importance: Questioning the Status Quo}
Given the computational burden associated with the length of visual tokens in MLLMs, numerous studies have embraced a paradigm that utilizes attention scores to evaluate the significance of visual tokens, thereby facilitating token reduction.
Specifically, in transformer-based MLLMs, each layer performs attention computation as illustrated below:
\begin{equation}
   \text{Attention}(\mathbf{Q}, \mathbf{K}, \mathbf{V}) = \text{softmax}\left(\frac{\mathbf{Q} \cdot \mathbf{K}^\mathbf{T}}{\sqrt{d_k}}\right)\cdot \mathbf{V},
\end{equation}
where $d_k$ is the dimension of $\mathbf{K}$. The result of $\text{Softmax}(\mathbf{Q}\cdot \mathbf{K}^\mathbf{T}/\sqrt{d_k})$ is a square matrix known as the attention map.
Existing methods extract the corresponding attention maps from one or multiple layers and compute the average attention score for each visual token based on these attention maps:
\begin{equation}
    \phi_{\text{attn}}(x_i) = \frac{1}{N} \sum_{j=1}^{N} \text{Attention}(x_i, x_j),
\end{equation}
where $\text{Attention}(x_i, x_j)$ denotes the attention score between token $x_i$ and token $x_j$, $\phi_{\text{attn}}(x_i)$ is regarded as the importance score of the token $x_i$, $N$ represents the number of visual tokens.
Finally, based on the importance score of each token and the predefined reduction ratio, the most significant tokens are selectively retained:
\begin{equation}
    \mathcal{R} = \{ x_i \mid (\phi_{\text{attn}}(x_i) \geq \tau) \},
\end{equation}
where $\mathcal{R}$ represents the set of retained tokens, and $\tau$ is a threshold determined by the predefined reduction ratio.

\noindent{\textbf{Problems:}} Although this paradigm has demonstrated initial success in enhancing the efficiency of MLLMs, it is accompanied by several inherent limitations that are challenging to overcome.

First, when it comes to leveraging attention scores to derive token importance, it inherently lacks full compatibility with Flash Attention, resulting in limited hardware acceleration affinity and diminished acceleration benefits.

Second, does the paradigm of using attention scores to evaluate token importance truly ensure the effective retention of crucial visual tokens? Our empirical investigations reveal that it is not the optimal approach.

% As illustrated in Figure~\ref{fig:random_vs_others}, performance evaluations on certain benchmarks show that methods meticulously designed based on this paradigm sometimes underperform compared to randomly retaining the same number of visual tokens.
Performance evaluations on certain benchmarks, as illustrated in Figure~\ref{fig:random_vs_others}, demonstrate that methods meticulously designed based on this paradigm sometimes underperform compared to randomly retaining the same number of visual tokens.

% As depicted in Figure~\ref{fig:teaser_curry}, which visualizes the results of token reduction, the selection of visual tokens based on attention scores exhibits a noticeable bias, favoring tokens located in the lower-right region of the image—those positioned later in the visual token sequence. However, it is evident that the lower-right region is not always the most significant in every image.
% Furthermore, in Figure~\ref{fig:teaser_curry}, we present the outputs of the original LLaVA-1.5-7B, FastV, and our proposed \algname. Notably, FastV introduces more hallucinations compared to the vanilla model, while \algname demonstrates a noticeable trend of reducing hallucinations.
% We suppose that this phenomenon arises because the important-based method, which relies on attention scores, tends to retain visual tokens that are concentrated in specific regions of the image due to the inherent bias in attention scores. As a result, relying on only a portion of the image often leads to outputs that are inconsistent with the overall image content. In contrast, \algname primarily removes highly duplication tokens and retains tokens that are more evenly distributed across the entire image, enabling it to make more accurate and consistent judgments.
%--------------- shorter version ---------------------
Figure~\ref{fig:teaser_curry} visualizes the results of token reduction, revealing that selecting visual tokens based on attention scores introduces a noticeable bias toward tokens in the lower-right region of the image—those appearing later in the visual token sequence. However, this region is not always the most significant in every image. Additionally, we present the outputs of the original LLaVA-1.5-7B, FastV, and our proposed \algname. Notably, FastV generates more hallucinations compared to the vanilla model, while \algname effectively reduces them. 
We attribute this to the inherent bias of attention-based methods, which tend to retain tokens concentrated in specific regions, often neglecting the broader context of the image. In contrast, \algname removes highly duplication tokens and preserves a more balanced distribution across the image, enabling more accurate and consistent outputs.

\subsection{Token Duplication: Rethinking Reduction}
Given the numerous drawbacks associated with the paradigm of using attention scores to evaluate token importance for token reduction, \textit{what additional factors should we consider beyond token importance in the process of token reduction?}
Inspired by the intuitive ideas mentioned in \secref{sec:introduction} and the phenomenon of tokens in transformers tending toward uniformity (i.e., over-smoothing)~\citep{nguyen2023mitigating, gong2021vision}, we propose that token duplication should be a critical focus.

Due to the prohibitively high computational cost of directly measuring duplication among all tokens, we adopt a paradigm that involves selecting a minimal number of pivot tokens. 
\begin{equation}
    \mathcal{P} = \{p_1, p_2, \dots, p_k\}, \quad k \ll n,
\end{equation}
where $p_i$ denotes pivot token, $\mathcal{P}$ represents the set of pivot tokens and $n$ means the length of tokens.

Subsequently, we compute the cosine similarity between these pivot tokens and the remaining visual tokens:
\begin{equation}
    dup (p_i, x_j) = \frac{p_i \cdot x_j}{\|p_i\| \cdot \|x_j\|}, \quad p_i \in \mathcal{P}, \, x_j \in \mathcal{X},
\end{equation}
where $dup (p_i, x_j)$ represents the token duplication score between $i$-th pivot token $p_i$ and $j$-th visual token $x_j$,
ultimately retaining those tokens that exhibit the lowest duplication with the pivot tokens.
\begin{equation}
    \mathcal{R} = \{ x_j \mid \min_{p_i \in \mathcal{P}} dup (p_i, x_j) \leq \epsilon \}.
\end{equation}
Here, $\mathcal{R}$ denotes the set of retained tokens, and $\epsilon$ is a threshold determined by the reduction ratio.

Our method is orthogonal to the paradigm of using attention scores to measure token importance, meaning it is compatible with existing approaches. Specifically, we can leverage attention scores to select pivot tokens, and subsequently incorporate token duplication into the process.

However, this approach still does not fully achieve compatibility with Flash Attention. To this end, we explored alternative strategies for selecting pivot tokens, such as using K-norm, V-norm\footnote{Here, the K-norm and V-norm refer to the L1-norm of K matrix and V matrix in attention computing, respectively.}, or even random selection. Surprisingly, we found that all these methods achieve competitive performance across multiple benchmarks. This indicates that our token reduction paradigm based on token duplication is not highly sensitive to the choice of pivot tokens. Furthermore, it suggests that removing duplicate tokens may be more critical than identifying ``important tokens'', highlighting token duplication as a potentially more significant factor to consider in token reduction.
The selection of pivot tokens is discussed in greater detail in \secref{pivot_token_selection}.
% 加个总结


\subsection{Experimental Setup}
\label{section:experimental_setup}
\textbf{Datasets:} Table~\ref{tab:datasets} provides a detailed breakdown of the SOTA intrusion datasets utilized in our study. 
%For each dataset we follow the data preparation steps outlined in section~\ref{section:data_preparation}. 
% \sean{is this section necessary with reduced page limit?}
% \begin{enumerate}
%     \item X-IIoTID \cite{al2021x}: The dataset consists of 59 features which are collected with the independence of devices and connectivity, generating a holistic intrusion data set to represent the heterogeneity of IIoT systems. It includes novel IIoT connectivity protocols, activities of various devices, and attack scenarios.  
%     \item WUSTL-IIoT \cite{zolanvari2021wustl}: WUSTL-IIoT aims to emulate real-world industrial systems. The dataset is deliberately unbalanced to imitate real-world industrial control systems, consisting of 41 features and 1,194,464 observations.
%     \item CICIDS2017 \cite{Sharafaldin2018TowardGA} The CICIDS2017 dataset includes a comprehensive collection of benign and malicious network traffic. It contains 80 features and represents a broad range of attacks, such as DoS, DDoS, Brute Force, XSS, and SQL Injection, across more than 2.8 million network flows. The dataset is widely used in evaluating intrusion detection systems.
%     \item UNSW-NB15 \cite{moustafa2015unsw, moustafa2016evaluation, moustafa2017novel, moustafa2017big, sarhan2020netflow} UNSW-NB15 is a comprehensive network intrusion dataset created by the University of New South Wales. It contains 49 features representing normal and malicious activities generated using IXIA's network traffic generator, covering a variety of contemporary attack types. 
% \end{enumerate}
For IIoT intrusion, we use IIoT datasets X-IIoTID \cite{al2021x} and WUSTL-IIoT \cite{zolanvari2021wustl}. We also include commonly used network intrusion datasets CICIDS2017 \cite{Sharafaldin2018TowardGA} and UNSW-NB15 \cite{moustafa2015unsw}. For X-IIoTID \cite{al2021x}, CICIDS2017 \cite{Sharafaldin2018TowardGA}, and UNSW-NB15 \cite{moustafa2015unsw}, we split the data across five experiences such that each experience contains two to four attacks. For WUSTL-IIoT \cite{zolanvari2021wustl}, we split the data across four experiences such that each experience contains one attack. We perform this data split to simulate an evolving data stream with emerging cyber attacks over time where each experience contains different attacks. 


%%%%%%%%%%%%%%%%%%%%%%%%%%%%%%%%%%%%%%%%%%%%%%%%%%%%%%%%%%%%%%%%%%%%%%%%%%%
\begin{table}[h]
    \caption{Selected Intrusion Datasets}
    \centering
    \label{tab:datasets}
    \resizebox{.99\columnwidth}{!}{
    \begin{tabular}{c|c|c|c|c}
    \hline
    Dataset    & Size      & Normal Data & Attack Data & Attack Types \\ 
    \hline
    X-IIoTID \cite{al2021x}   & 820,502   & 421,417     & 399,417     & 18           \\
    \hline
    WUSTL-IIoT \cite{zolanvari2021wustl} & 1,194,464 & 1,107,448   & 87,016      & 4       \\
    \hline
    CICIDS2017 \cite{Sharafaldin2018TowardGA} & 2,830,743 & 2,273,097 & 557,646 & 15 \\
    \hline
    UNSW-NB15 \cite{moustafa2015unsw}
 & 257,673 & 164,673 & 93,000 & 10 \\
    \hline
    \end{tabular}}
\end{table}
%%%%%%%%%%%%%%%%%%%%%%%%%%%%%%%%%%%%%%%%%%%%%%%%%%%%%%%%%%%%%%%%%%%%%%%%%%%

\textbf{Baselines:} %Due to the novelty of this problem formulation, there are no directly comparable methods. However, the most similar widely studied problem would be unsupervised continual learning (UCL). Therefore, 
We evaluate our algorithm against two SOTA unsupervised continual learning (UCL) algorithms: the Autonomous Deep Clustering Network (\textbf{ADCN}) \cite{ashfahani2023unsupervised}, and an autoencoder paired with K-Means clustering. The autoencoder K-Means model is combined with Learning without Forgetting \cite{lwf2019Li} continual learning loss; we refer to this model as \textbf{LwF}. Note that both \textbf{ADCN} and \textbf{LwF} require a small amount of labeled normal and attack data to perform classification. We also compare our approach against SOTA ND methods: local outlier factor (\textbf{LOF})\cite{Faber_2024}, one-class support vector machine (\textbf{OC-SVM})\cite{Faber_2024}, principal component analysis (\textbf{PCA})\cite{rios2022incdfm}, and Deep Isolation Forest (\textbf{DIF}) \cite{xu2023deep}. 
%We train the ND algorithms on the clean subset of normal data, $N_c$, and evaluate their performance on the remainder of the dataset. 
Since these ND models cannot be retrained on unlabeled contaminated data, continual learning is not feasible for these methods.

%an autoencoder with K-Means clustering paired with SOTA Learning without Forgetting (LwF) continual loss (LwF) \cite{lwf2019Li}.
%Notably, many SOTA UCL algorithms rely on image-specific contrastive pairs, which is not directly applicable to intrusion detection \cite{madaan2022representational, yu2023scale, fini2022self, liu2024unsupervised}.

%%%%%%%%%%%%%%%%%%%%%%%%%%%%%%%%%%%%%%%%%%%%%%%%%%%%%%%
\begin{figure*}
    \centering
    \includegraphics[width=.95\linewidth]{figures/cl_experiments.pdf}
    \caption{Continual learning metric results of ADCN\cite{ashfahani2023unsupervised}, LwF\cite{lwf2019Li}, and \Design{}}
    \label{fig:continual_methods_results}
\end{figure*}
%%%%%%%%%%%%%%%%%%%%%%%%%%%%%%%%%%%%%%%%%%%%%%%%%%%%%%%

\textbf{Evaluation Metrics:} To evaluate the model performance, we report $F_{1}$ score. Since there is a class imbalance within these datasets, to simulate real world IDS, $F_{1}$ score gives an accurate idea on attack detection. For the continual learning methods, we evaluate their performance at the end of each training experience on all experience test sets. This generates a matrix of $F_{1}$ score results $R_{ij}$ such that $i$ is the current training experience, and $j$ is the testing experience. To summarize this matrix of results, we report widely used CL metrics \cite{diaz2018don}: average $F_{1}$ score on current experience (AVG), forward transfer (FwdTrans), and backward transfer (BwdTrans). For a matrix $R_{ij}$ with $m$ total experiences, our metrics are formulated as follows: $\text{AVG}_{F_1} = \frac{\sum_{i = j} R_{ij}}{m}$; $\text{FwdTrans}_{F_1} = \frac{\sum_{j>i} R_{ij}}{\frac{m * (m-1)}{2}}$; $\text{BwdTrans}_{F_1} = \frac{\sum_{i}^m R_{mi} - R_{ii}}{\frac{m * (m-1)}{2}}$.
AVG is the average performance on the current test experience at every point of training. FwdTrans is the average performance on ``future'' experiences, which simulates performance on zero-day attacks. Finally, BwdTrans is the average change in performance of ``past'' test experiences at a ``future'' point of training. A negative BwdTrans indicates catastrophic forgetting, whereas a positive BwdTrans  indicates the model actually improved performance on past experiences after learning a future experience. Overall, AVG measures seen attacks, FwdTrans measures zero-day attacks, and BwdTrans measures forgetting. For all metrics, a higher positive result indicates a better performance. 

We also report the threshold-free metric Precision-Recall Area Under the Curve (PR-AUC) \cite{praucDavid06}. Since \Design{} requires selecting a threshold, PR-AUC allows us to assess model performance independently of the threshold. We choose PR-AUC over Receiver Operating Characteristic Area Under the Curve (ROC-AUC) because ROC-AUC can give misleadingly high results in the presence of class imbalance \cite{praucDavid06}.

\textbf{Hyperparameters:} %For $L_{CND}$ hyperparameters are the number of K-Means clusters $K$, the reconstruction loss strength $\lambda_R$,  the continual learning loss strength $\lambda_{CL}$, and the cluster separation loss margin $m$. 
We utilize \textit{elbow method} \cite{han2011data} for determining the number of clusters $K$. 
%It tests a range of $K$ values and then selects the value   where there is a significant change in slope, called the elbow point. 
%This resulted in $K$ values between 100-500. 
We set $\lambda_R$ and $\lambda_{CL}$ to 0.1, and for $m$ we use 2 after careful experimentation. For the AE modules of \Design{}, we use 4-layer MLP with 256 neurons in the hidden layers. We train it using Adam optimizer \cite{kingma2017adammethods} with a learning rate of 0.001. For PCA, we use the explained variance method and set it to 95\% \cite{rios2022incdfm}.

\textbf{Hardware:} We run our experiments on NVIDIA GeForce RTX 3090 GPU, with a AMD EPYC 7343 16-Core processor.

\subsection{Results}

\textbf{Continual Learning Comparison:} Fig.~\ref{fig:continual_methods_results} presents the results of our approach \Design{} compared with ADCN\cite{ashfahani2023unsupervised} and LwF\cite{lwf2019Li}. \Design{} shows the best performance on both seen (AVG) and unseen (FwdTrans) attacks across all datasets. \Design{} also has the highest BwdTrans on all except one dataset (UNSW-NB15). The average BwdTrans of \Design{} (0.87\%) is higher than the average BwdTrans of both ADCN (-0.06\%) and LwF (0.09\%). Notably, the BwdTrans of \Design{} is positive for three datasets. Indicating past experiences actually improve after training on future experiences for these datasets. Given the high FwdTrans as well, our approach finds features that generalize well to future experiences. 

Table~\ref{tab:improvement} shows the improvement of \Design{} over the UCL baselines on all datasets. Bold and underlined cases indicate the best and the second best improvements with respect to each metric, respectively. These improvements were calculated by comparing the performance of \Design{} to the baselines, where the improvement values represent the proportional increase over the baseline performance. We do not include BwdTrans because a proportional increase does not make sense for a metric that can be negative. \Design{} has up to $4.50\times$ and $6.1\times$ AVG improvement on ADCN and LwF, respectively. In addition, \Design{} has up to $6.47\times$ and $3.47\times$ FwdTrans improvement on ADCN and LwF. Averaged across all datasets, \Design{} shows a $1.88\times$ and $1.78\times$ improvement on AVG, and a $2.63\times$ and $1.60\times$ improvement on FwdTrans, compared to ADCN and LwF, respectively. %These results underscore the benefit of our continual novelty detection method \Design{}. The notably high FwdTrans score emphasizes how novelty detection can be used to identify unseen anomalous data, thereby significantly enhancing performance on zero-day attacks.

Overall, these results highlight the benefit of continual ND over UCL methods for IDS. \Design{}, with its PCA-based novelty detector, excels by effectively harnessing the normal data to identify attacks. A key strength of our approach lies in the assumption that normal data forms a distinct class, while everything else is treated as anomalous. This assumption is particularly well-suited to IDS. In contrast, methods like ADCN and LwF do not make this distinction where they handle both normal and attack data similarly, limiting their ability to fully exploit the inherent structure of the data. 



% %%%%%%%%%%%%%%%%%%%%%%%%%%%%%%%%%%%%%%%%%%%%%%%%%%%%%%%
% \begin{table}[]
% \centering
% \caption{\Design{} Percentage Improvement over UCL Baselines on AVG and FwdTrans}
% \label{tab:improvement}
% \begin{tabular}{|c|c|c|c|}
% \hline
% Baseline      & Dataset    & AVG  & FwdTrans  \\ \hline
% ADCN\cite{ashfahani2023unsupervised}          & X-IIoTID   & 101.88\%        & 400.35\%        \\ \cline{2-4} 
%               & WUSTL-IIoT & 349.86\%        & 546.68\%        \\ \cline{2-4} 
%               & CICIDS2017 & 37.19\%         & 73.46\%         \\ \cline{2-4} 
%               & UNSW-NB15  & 29.25\%         & 43.90\%         \\ \hline
% LwF\cite{lwf2019Li} & X-IIoTID   & 46.43\%         & 35.39\%         \\ \cline{2-4} 
%               & WUSTL-IIoT & 510.92\%        & 246.81\%        \\ \cline{2-4} 
%               & CICIDS2017 & 92.72\%         & 163.81\%        \\ \cline{2-4} 
%               & UNSW-NB15  & 11.07\%         & 2.20\%          \\ \hline
% \end{tabular}
% \end{table}
% %%%%%%%%%%%%%%%%%%%%%%%%%%%%%%%%%%%%%%%%%%%%%%%%%%%%%%%

%%%%%%%%%%%%%%%%%%%%%%%%%%%%%%%%%%%%%%%%%%%%%%%%%%%%%%%
\begin{table}[]
\centering
\caption{\Design{} Improvement over UCL Baselines}
\label{tab:improvement}
\scalebox{1}{
\begin{tabular}{|c|c|c|c|}
\hline
Baseline      & Dataset    & AVG  & FwdTrans  \\ \hline
ADCN\cite{ashfahani2023unsupervised}  & X-IIoTID   & $\underline{2.02\times}$  & $\underline{5.00\times}$   \\ \cline{2-4} 
                                      & WUSTL-IIoT & $\mathbf{4.50\times}$  & $\mathbf{6.47\times}$   \\ \cline{2-4} 
                                      & CICIDS2017 & $1.37\times$  & $1.73\times$   \\ \cline{2-4} 
                                      & UNSW-NB15  & $1.29\times$  & $1.44\times$   \\ \hline
LwF\cite{lwf2019Li}                   & X-IIoTID   & $1.46\times$  & $1.35\times$   \\ \cline{2-4} 
                                      & WUSTL-IIoT & $\mathbf{6.11\times}$  & $\mathbf{3.47\times}$   \\ \cline{2-4} 
                                      & CICIDS2017 & $\underline{1.93\times}$  & $\underline{2.64\times}$   \\ \cline{2-4} 
                                      & UNSW-NB15  & $1.11\times$  & $1.02\times$   \\ \hline
\end{tabular}}
\end{table}

%%%%%%%%%%%%%%%%%%%%%%%%%%%%%%%%%%%%%%%%%%%%%%%%%%%%%%%

%Figure~\ref{fig:XIIoT_graph} shows the $F_{1}$ score of ADCN and \Design{} for each experience on both datasets. Similarly, we use green and red colors for \Design{} and ADCN respectively. Notably for \Design{}, the $F_{1}$ score of each experience has little change over training time. This highlights the strength of novelty detection for IDSs, as even before seeing attacks \Design{} has good performance. On the other hand, ADCN test experiences do not improve until the associated training experience, meaning ADCN does not have an ability to generalize to future attacks. ADCN utilizes a subset of labeled data to assign labels to clusters. This subset of labeled might be causing ADCN to overfit to the attacks within the current experience, therefore leading ADCN to not generalize well. We can also clearly see that our approach is consistently better (higher $F_{1}$ score) than the state-of-the-art ADCN. 

% %%%%%%%%%%%%%%%%%%%%%%%%%%%%%%%%%%%%%%%%%%%%%%%%%%%%%%%
% \begin{figure*}[t]
%     \centering
%     \begin{subfigure}[t]{\linewidth}
%         \centering
%         \includegraphics[width=\linewidth]{figures/X-IIoTID-experiences.pdf}
%         \caption{X-IIoTID}
%         \label{fig:ADCN_XIIoT_results}
%     \end{subfigure}
%     \begin{subfigure}[t]{\linewidth}
%         \centering
%         \includegraphics[width=\linewidth]{figures/WUSTL-IIoT-experiences.pdf}
%         \caption{WUSTL-IIoT}
%         \label{fig:WUSTL-}
%     \end{subfigure}
%     \caption{$F_1$ Score of ADCN and \Design{} of each test experience over training experiences.}
%     \label{fig:XIIoT_graph}
% \end{figure*}
% %%%%%%%%%%%%%%%%%%%%%%%%%%%%%%%%%%%%%%%%%%%%%%%%%%%%%%%

\textbf{Novelty Detectors Comparison:} Fig.~\ref{fig:novelty_methods_results} compares LOF\cite{Faber_2024}, OC-SVM\cite{Faber_2024}, PCA\cite{rios2022incdfm}, and DIF \cite{xu2023deep} with \Design{} on all datasets. The average $F_{1}$ score of the novelty detection methods are compared to the AVG of \Design{}.  It can be seen \Design{} outperforms all other methods across all datasets. The two best performing methods are DIF and PCA. The average $F_{1}$ score improvement across all datasets of \Design{} is $1.16\times$ and $1.08\times$ over DIF and PCA, respectively. These results highlight the critical role of leveraging information from unsupervised data streams. Unlike these ND algorithms, \Design{} is capable of continuously learning from this unsupervised data, enabling it to enhance PCA reconstruction over time. By integrating evolving data patterns, \Design{} not only adapts to new anomalies but also improves its overall detection accuracy, demonstrating a clear advantage in dynamic environments.

%Given that \Design{} employs PCA detection, this indicates that the CFE effectively extracts useful features from the unlabeled training experiences. T

%%%%%%%%%%%%%%%%%%%%%%%%%%%%%%%%%%%%%%%%%%%%%%%%%%%%%%%   
\begin{figure}
    \centering
    \includegraphics[width=0.9\linewidth]{figures/novelty_detectors_experiments.pdf}
    \caption{Average $F_1$ score on all experiences of \Design{} and novelty detection methods: LOF, OC-SVM, PCA, DIF}
    \label{fig:novelty_methods_results}
\end{figure}
%%%%%%%%%%%%%%%%%%%%%%%%%%%%%%%%%%%%%%%%%%%%%%%%%%%%%%%
%%%%%%%%%%%%%%%%%%%%%%%%%%%%%%%%%%%%%%%%%%%%%%%%%%%%%%% 
\begin{figure}
    \centering
    \includegraphics[width=0.86\linewidth]{figures/novelty_detectors_pr_auc.pdf}
    \caption{Thresholding Free Evaluation of \Design{}}
    \label{fig:thresholding_free}
\end{figure}

%%%%%%%%%%%%%%%%%%%%%%%%%%%%%%%%%%%%%%%%%%%%%%%%%%%%%%%

\textbf{Pre-threshold Evaluation:} While thresholding plays a crucial role in attack decision-making, evaluating model prediction performance before applying threshold is also important. The UCL algorithms (ADCN\cite{ashfahani2023unsupervised} and LwF\cite{lwf2019Li}) do not output anomaly scores because they select classes based on the closest labeled cluster. Therefore we compare against the two best ND methods: DIF\cite{xu2023deep} and PCA\cite{rios2022incdfm}. Fig.~\ref{fig:thresholding_free} presents the PR-AUC values of DIF, PCA, and \Design{}. It can be seen that \Design{} provides the best threshold free results, which aligns with the threshold-based results presented earlier. The strong performance of \Design{} in both pre-threshold and threshold-based evaluations demonstrates that the model is robust regardless of the decision threshold. 

\subsection{Ablation Study}

To demonstrate the impact of our loss function components, we perform an ablation study. Table~\ref{tab:ablation_loss} shows the results of \Design{} with each loss function removed to demonstrate their individual effectiveness. Bold and underlined cases indicate the best and the second best performances with respect to each metric, respectively. \Design{} without reconstruction loss ($L_R$) and \Design{} without cluster separation loss ($L_{CS}$) performs worse in all categories. \Design{} without both $L_R$ and continual learning loss ($L_{CL}$) actually performs better AVG but has worse BwdTrans and FwdTrans. AVG does not account for past experiences, so the significantly negative BwdTrans indicates \Design{} w/o $L_R$ and $L_{CL}$ forgets, and therefore would perform worse on those experiences in the future. This would make sense as a regularization loss to improve continual learning would slightly decrease performance in non-continual scenario. Overall \Design{} has the best results when taking every metric category into account. Notably the low BwdTrans and FwdTrans of \Design{} (w/o $L_R$) showcases how the reconstruction loss helps \Design{} generalize better to unseen and past data. This highlights the power of $L_R$ to provide good features for continual learning. 

%%%%%%%%%%%%%%%%%%%%%%%%%%%%%%%%%%%%%%%%%%%%%%%%%%%%%%%%%%%%%%%%%%%%%
\begin{table}[]
\caption{Ablation Study of \Design{} Loss Functions}
\label{tab:ablation_loss}
\centering
\begin{tabular}{|c|c|c|c|}
\hline
Strategy                         & AVG              & BwdTrans        & FwdTrans         \\ \hline
CND-IDS                          &\underline{76.92\%}    & \textbf{0.87\%} & \textbf{73.70\%} \\ \hline
CND-IDS (w/o $L_{CS}$)           & 66.23\%          & \underline{0.09\%}    & 70.26\%          \\ \hline
CND-IDS (w/o $L_R$)              & 72.86\%          & -5.44\%         & 67.82\%          \\ \hline
CND-IDS (w/o $L_R$ and $L_{CL}$) & \textbf{79.92\%} & -11.26\%        & \underline{71.01\%}    \\ \hline
\end{tabular}
\end{table}
%%%%%%%%%%%%%%%%%%%%%%%%%%%%%%%%%%%%%%%%%%%%%%%%%%%%%%%%%%%%%%%%%%%%%%%

\subsection{Overhead Analysis}
%%%%%%%%%%%%%%%%%%%%%%%%%%%%%%%%%%%%%%%%%%%%%%%%%%%%%%%%%%%
% \begin{table}[]
% \centering
% \caption{Average training time and inference time per sample across all datasets in milliseconds}
% \label{tab:overhead}
% \begin{tabular}{|c|c|c|}
% \hline
% Strategy               & Inference Time(ms) \\ \hline
% \Design{}                   & 0.0019             \\ \hline
% ADCN\cite{ashfahani2023unsupervised}    & 0.4061             \\ \hline
% LwF\cite{lwf2019Li}           & 0.0677             \\ \hline
% DIF\cite{xu2023deep}         & 1.0535             \\ \hline
% PCA\cite{rios2022incdfm}       & 0.0018             \\ \hline
% \end{tabular}
% \end{table}
%%%%%%%%%%%%%%%%%%%%%%%%%%%%%%%%%%%%%%%%%%%%%%%%%%%%%%%%%%%%%
\begin{table}[]
\centering

\caption{Average inference time (in ms) per test sample}
\label{tab:overhead}
\scalebox{0.95}{
\begin{tabular}{|c|c|c|c|c|c|}
\hline
Strategy           & \Design{} & ADCN   & LwF    & DIF    & PCA    \\ \hline
Inference Time (ms) & \underline{0.0019}                     & 0.4061 & 0.0677 & 1.0535 & \textbf{0.0018} \\ \hline
\end{tabular}}
\end{table}
%%%%%%%%%%%%%%%%%%%%%%%%%%%%%%%%%%%%%%%%%%%%%%%%%%%%%%%%
Table~\ref{tab:overhead} evaluates the inference overhead of \Design{} compared to ADCN \cite{ashfahani2023unsupervised}, LwF \cite{lwf2019Li}, DIF \cite{xu2023deep}, and PCA \cite{rios2022incdfm}. %, excluding OC-SVM \cite{Faber_2024} and LOF \cite{Faber_2024} due to poor performance. 
\Design{} offers the fastest inference time among continual learning methods. Out of novelty detection methods, \Design{} is second only to PCA. We attribute the efficiency of \Design{} to avoiding the clustering classification used by LwF and ADCN. %\Design{} instead uses PCA reconstruction, which is much quicker than comparing data points to clusters. In addition, 
The difference between \Design{} and PCA is minimal, only 0.0001 milliseconds slower, due to the additional but lightweight step of encoding the data. Considering that the average median flow duration across datasets is 27.77 milliseconds, the overhead introduced by \Design{} is negligible in the context of real-time traffic flow.

%In this section we analyze the inference overhead of \Design{} compared to ADCN\cite{ashfahani2023unsupervised}, LwF\cite{lwf2019Li}, DIF\cite{xu2023deep}, and PCA\cite{rios2022incdfm}. We do not include OC-SVM\cite{Faber_2024} and LOF \cite{Faber_2024} due to weak performance. Table~\ref{tab:overhead} shows the average inference time in milliseconds per sample across all datasets. \Design{} has the best inference time besides PCA. We attribute this good inference time to \Design{} not using clustering classification like LwF and ADCN. Evidently, PCA reconstruction utilized by \Design{} is more time efficient than having to compare a data point to all saved clusters. Compared to pure PCA reconstruction, \Design{} is only 0.0001 ms slower. This small increase in inference time is due to the only added computation at inference is encoding the data with the encoder, which is simply a 4 layer MLP. Across all datasets, the average median travel flow duration is 27.77 ms, and the dataset with the quickest median travel flow is UNSW with 4.29 ms. Therefore the overhead introduced by \Design{} is irrelevant compared to the speed of the traffic flow. 

%\label{section:ablation_study}
%To assess the impact of our design choices, we perform an ablation study. Our goal is to analyze (i) threshold function evaluation, and (ii) novelty detection algorithm selection. 

 

%\textbf{Threshold Function Evaluation:} AE, PCA, and \Design{} all require a threshold to classify an anomaly based on the anomaly score. In all previously reported results, we select a widely used threshold that maximizes the $F_{1}$ score on the test set, i.e., Best-F. %This is not realistic but was used to compare the effectiveness of these methods. In this section 
%Here, we analyze three different threshold methods, which we denote: Best-F \cite{su2019robust}, Top-k \cite{zong2018deep}, and validation percentile (ValPer). Best-F uses the threshold that maximizes the $F_{1}$ score on test set. Top-k utilizes the contamination ratio $r$ of the test set, such that $r$ is the percentage of anomalies within the test set. Top-k selects a threshold so that the percentile of data within the test set classified as anomalies is equal to $r$. ValPer utilizes a validation set of normal data, and selects a threshold such that 99.7\% (3 standard deviations) of the normal data is within this threshold. 
%ValPer is the most realistic method as it does not rely on any information from the test set. 
%A breakdown of the $F_{1}$ score results for the different threshold methods is show in Table~\ref{tab:thresholding_results} where the best within each category is bolded. Overall Best-F performs significantly better than the other threshold methods, which is obvious as Best-F is an upper-bound for threshold selection. However the significant gap highlights the importance of threshold selection. Most importantly, \Design{} still performs better than PCA and AE through all threshold methods. 

%%%%%%%%%%%%%%%%%%%%%%%%%%%%%%%%%%%%%%%%%%%%%%%%%%%%%%%
%\begin{table}[]
%    \centering
%    \caption{Threshold Function Evaluation}
%    \resizebox{.97\columnwidth}{!}{
%    \begin{tabular}{c|c|c|c|c}
%        \hline
%         Dataset & Stategy & Best-F & Top-k & ValPer\\
%         \hline
%         & PCA  & 70.9 & 4.03 & 3.56 \\
%         \cline{2-5}
%         X-IIoTID & AE  & 75.6 & 4.03 & 29.4 \\
%         \cline{2-5}
%         & \Design{} & \textbf{78.8} & \textbf{5.63} &  %\textbf{52.9} \\	
%         \hline
%        & PCA  & 85.6 &19.9 & 52.8\\
%         \cline{2-5}
%         WUSTL-IIoT & AE  & 79.6 &19.7 & 37.8\\
%         \cline{2-5}
%         & \Design{} & \textbf{88.2} & \textbf{21.1} & \textbf{55.6}\\	
%         \hline
%    \end{tabular}}
%    \label{tab:thresholding_results}
%\end{table}
%%%%%%%%%%%%%%%%%%%%%%%%%%%%%%%%%%%%%%%%%%%%%%%%%%%%%%%

% %%%%%%%%%%%%%%%%%%%%%%%%%%%%%%%%%%%%%%%%%%%%%%%%%%%%%%%
% \begin{figure}
%     \centering
%     \includegraphics[width=0.95\linewidth]{figures/novelty_ablation.pdf}
%     \caption{Comparison of \Design{} with PCA and AE novelty detection models}
%     \label{fig:novelty_ablation_results}
% \end{figure}
% %%%%%%%%%%%%%%%%%%%%%%%%%%%%%%%%%%%%%%%%%%%%%%%%%%%%%%%

% \textbf{Novelty Detection Algorithm Selection:} For \Design{}, we select PCA as the novelty detection algorithm. As shown in Figure~\ref{fig:novelty_methods_results}, both PCA and AE perform well for detecting intrusions. Therefore, we test both AE and PCA as the novelty detection methods for \Design{}. Figure~\ref{fig:novelty_ablation_results} illustrates the AVG performance of \Design{} with AE and PCA as the novelty detection models. It is evident that PCA outperforms AE, justifying our selection of this algorithm for novelty detection. This could be because the CFE utilizes SAEs, which generate features based on the same reconstruction loss used by AE to classify anomalies. It may be beneficial to use PCA as it deconstructs the input in a different manner, thereby identifying different features and functioning better in conjunction with the SAE-based CFE.


% \section{Results}
\label{sec:Results}

In this section, we present various analysis results that demonstrate the adoption of code obfuscation in Google Play.

\subsection{Overall Obfuscation Trends} 
\label{sec:obstrend}

\subsubsection{Presence of obfuscation} Out of the 548,967 Google Play Store APKs analyzed, we identified 308,782 obfuscated apps, representing approximately 56.25\% of the total. In Figure~\ref{fig:obfuscated_percentage}, we show the year-wise percentage of obfuscated apps for 2016-2023. There is an overall obfuscation increase of 13\% between 2016 and 2023, and as can be seen, the percentage of obfuscated apps has been increasing in the last few years, barring 2019 and 2020. As explained in Section~\ref{subsec:dataset}, 2019 and 2020 contain apps that are more likely to be abandoned by developers, and as such, they may not use advanced development practices.

\begin{figure}[h!]
\centering
    \includegraphics[width=\linewidth]{Figures/Only_obfuscation_trendV2.pdf}
    \caption{Percentage of obfuscated apps by year} \vspace{-4mm}
    \label{fig:obfuscated_percentage}
\end{figure}


From 2016 to 2018, the obfuscation levels were relatively stable at around 50-55\%, while from 2021 to 2023, there was a marked rise, reaching approximately 66\% in 2023. This indicates a growing focus on app protection measures among developers, likely driven by heightened security and IP concerns and the availability of advanced obfuscation tools.


\subsubsection{Obfuscation tools} Among the obfuscated APKs, our tool detector identified that 40.92\% of the apps use Proguard, 36.64\% use Allatori, 1.01\% use DashO, and 21.43\% use other (i.e., unknown) tools. We show the yearly trends in Figure~\ref{fig:ofbuscated_tool}. Note that we omit results in 2019 and 2020 ({\bf cf.} Section~\ref{subsec:dataset}).

ProGuard and Allatori are the most consistently used obfuscation tools, with ProGuard showing a slight overall increase in popularity and Allatori demonstrating variability. This inclination could be attributed to ProGuard being the default obfuscator integrated into Android Studio, a widely used development environment for Android applications. Notably, ProGuard usage increased by 13\% from 2018 to 2021, likely due to the introduction of R8 in April 2019~\cite{release_note_android}, which further simplified ProGuard integration with Android apps.

\begin{figure}[h]
\centering
    \includegraphics[width=\linewidth]{Figures/Initial_Tool_Trend_2019_dropV2.pdf} 
    \caption{Yearly obfuscation tool usage}
    \label{fig:ofbuscated_tool}
\end{figure}


DashO consistently remains low in usage, likely due to its high cost. The use of other obfuscation tools decreased until 2018 but has shown a resurgence from 2021 to 2023. This suggests that developers might be using other or custom tools, or our detector might be predicting some apps obfuscated with Proguard or Allatori as `other.' To investigate, we manually checked a sample of apps from the `other' category and confirmed they are indeed obfuscated. However, we could not determine which obfuscation tools the developers used. We discuss this potential limitation further in Section~\ref{sec:limitations}.


\subsubsection{Obfuscation techniques} We show the year-wise breakdown of obfuscation technique usage in Figure~\ref{fig:obfuscated_tech}. Among the various obfuscation techniques, Identifier Renaming emerged as the most prevalent, with 99.62\% of obfuscated apps using it alone or in combination with other methods (Categories of Only IR, IR and CF, IR and SE, or All three). Furthermore, 81.04\% of obfuscated apps used Control Flow Modification, and 62.76\% used String Encryption. The pervasive use of Identifier Renaming (IR) can be attributed to the fact that all obfuscation tools support it ({\bf cf.} Table~\ref{tab:ob_tool_cap}). Similarly, lower adoption of Control Flow Modification and String Encryption can be attributed to Proguard not supporting it. 

\begin{figure}[h]
\centering
    \includegraphics[width=\linewidth]{Figures/Initial_Tech_Trend_2019_dropV2.pdf} 
    \caption{Yearly obfuscation technique usage}
    \label{fig:obfuscated_tech}
\end{figure}



Next, we investigate the adoption of obfuscation on Google Play Store from various perspectives. Same as earlier, due to the smaller dataset size and possible bias ({\bf cf.} Section~\ref{subsec:dataset}), we exclude the APKs from 2019 and 2020 from this analyses.


\subsection{App Genre}
\label{sec:app_genre}

First, we investigate whether the obfuscation practices vary according to the App genre. Initially, we analysed all the APKs together before separating them into two snapshots.


\begin{figure*}[h]
    \centering
    \includegraphics[width=\linewidth]{Figures/AppGenreObfuscationV3.pdf}
    \caption{Obfuscated app percentage by genre (overall)}
    \label{fig:app_genre_overall}
\end{figure*}

Figure~\ref{fig:app_genre_overall} shows the genre-wise obfuscated app percentage. We note that 19 genres have more than 60\% of the apps obfuscated, and almost all the genres have more than 40\% obfuscation percentage. \textit{Casino} genre has the highest obfuscation percentage rate at 80\%, and overall, game genres tend to be more obfuscated than the other genres. The higher obfuscation usage in casino apps is logical due to their nature. These apps often simulate or involve gambling activities and handle monetary transactions and sensitive data related to in-game purchases, making them attractive targets for reverse engineering and hacking. This necessitates robust security measures to prevent fraud and protect user data. 


\begin{figure}[h]
    \centering
    \includegraphics[width=\linewidth]{Figures/AppGenre2018_2023ComparisonV3.pdf}
    \caption{Percentage of obfuscated apps by genre (2018-2023)}
    \label{fig:app_genre_comparison}
\end{figure}



\subsubsection{Genre-wise obfuscation trends in the two snapshots} To investigate the adoption of obfuscation over time, we study the two snapshots of Google Play separately, i.e., APKs from 2016-2018 as one group and APKs from 2021-2023 as another. 

Figure~\ref{fig:app_genre_comparison} illustrates the change in obfuscation levels by app genre between 2016-2018 to 2021-2023. Notably, app categories such as Education, Weather, and Parenting, which had obfuscation levels below the 2018 average, have increased to above the 2023 average by 2023. One possible reason for this in Education and Parenting apps can be the increase in online education activities during and after COVID-19 and the developers identifying the need for app hardening.

There are some genres, such as Casino and Action, for which the percentage of obfuscated apps didn't change across the two snapshots (i.e., purple and orange circles are close together in Figure~\ref{fig:app_genre_comparison}). This is because these genres are highly obfuscated from the beginning. Finally, the four genres, including Simulation and Role Playing, have a lower percentage of obfuscated apps in the 2021-2023 dataset. Our manual analysis didn't result in a conclusion as to why.


\begin{figure}[!h]
    \centering
    \includegraphics[width=\linewidth]{Figures/AppGenreTechAllV5.pdf}
    \caption{Obfuscation technique usage by genre (overall)}
    \label{fig:app_genre_all_tech}
\end{figure}


\subsubsection{Obfuscation techniques in different app genres} In Figure~\ref{fig:app_genre_all_tech}, we show the prevalence of key obfuscation techniques among various genres. As expected, almost all obfuscated apps in all genres used  Identifier Renaming. Also, it can be noted that genres with more obfuscated app percentages tend to use all three obfuscation techniques. Notably, more than 85\% of \textit{Casino} genre apps employ multiple obfuscation techniques

\subsubsection{Obfuscation tool usage in different app genres} We also investigated whether specific obfuscation tools are favoured by developers in different genres. However, apart from the expected observation that  ProGuard and Allatori being the most used tools, we didn't find any other interesting patterns. Therefore, we haven't included those measurement results.

\subsection{App Developers}
Next, we investigate individual developer-wise code obfuscation practices. From the pool of analyzed APKs, we identified the number of apps associated with each developer. Subsequently, we sorted the developers according to the number of apps they had created and selected the top 100 developers with the highest number of APKs for the 2016-2018 and 2021-2023 datasets. For the 2018 snapshot, we had 8,349 apps among the top 100 developers, while for the 2023 snapshot, we had 11,338 apps among the top 100 developers.

We then proceeded to detect whether or not these developers obfuscate their apps and, if so, what kind of tools and techniques they use. We present our results in five levels; developer obfuscating over 80\% of their apps, 60\%--80\% of apps, 40\%--60\% of apps, less than 40\%, and no obfuscation.

Figure~\ref{fig:developer_trend_my_apps_all} compares the two datasets in terms of developer obfuscation adoption. It shows that more developers have moved to obfuscate more than 80\% of their apps in the 2021-2023 dataset (76\%) compared to the 2016-2018 dataset (48\%).

We also found that among developers who obfuscate more than 80\% of their apps, 73\% in 2018 and 93\% in 2023 used the same obfuscation tool. Additionally, these top developers employ Control Flow Modification (CF) and String Encryption (SE) above the average values discussed in Section~\ref{sec:obstrend}. Specifically, in 2018, top developers used CF in 81.3\% of cases and SE in 66.7\%, while in 2023, these figures increased to 88.2\% and 78.9\%. This results in two insights: 1) Most top developers obfuscate all their apps with advanced techniques, possibly due to concerns about IP and security, and 2) Developers stick to a single tool, possibly due to specialized knowledge or because they bought a commercial licence.

\begin{figure}[]
    \centering
    \includegraphics[width=\linewidth]{Figures/Developer_Analysed_Comparison.pdf}
    \caption{Obfuscation usage (Top-100 developers)}
    \label{fig:developer_trend_my_apps_all}
\end{figure}


Finally, we investigate the obfuscation practices of developers with only one app in Table~\ref{tab:my-table}. According to the table, from those developers, 45.5\% of them obfuscated their apps in the 2016-2018 dataset and 57.2\% obfuscated their apps in the 2021-2023 dataset, showing a clear increase. However, these percentages are approximately 10\% lower than the average obfuscation rate in both cohorts discussed in Section~\ref{sec:obstrend}. This indicates that single-app developers may be less aware or concerned about code protection.


\begin{table}[]
\caption{Developers with only one app}
\label{tab:my-table}
\resizebox{\columnwidth}{!}{%
\begin{tabular}{cccccc}
\hline
\textbf{Year} & \textbf{\begin{tabular}[c]{@{}c@{}}Non\\ Obfuscated\end{tabular}} & \multicolumn{4}{c}{\textbf{Obfuscated}} \\ \hline
\multirow{3}{*}{\textbf{\begin{tabular}[c]{@{}c@{}}2018 \\ Snapshot\end{tabular}}} & \multirow{3}{*}{\begin{tabular}[c]{@{}c@{}}26,581 \\ (54.5\%)\end{tabular}} & \multicolumn{4}{c}{\begin{tabular}[c]{@{}c@{}}22,214 (45.5\%)\end{tabular}} \\ \cline{3-6} 
 &  & \textbf{ProGuard} & \textbf{Allatori} & \textbf{DashO} & \textbf{Other} \\ \cline{3-6} 
 &  & 6,131 & 8,050 & 658 & 7,375 \\ \hline
\multirow{3}{*}{\textbf{\begin{tabular}[c]{@{}c@{}}2023 \\ Snapshot\end{tabular}}} & \multirow{3}{*}{\begin{tabular}[c]{@{}c@{}}19,510 \\ (42.8\%)\end{tabular}} & \multicolumn{4}{c}{\begin{tabular}[c]{@{}c@{}}26,084 (57.2\%)\end{tabular}} \\ \cline{3-6} 
 &  & \textbf{ProGuard} & \textbf{Allatori} & \textbf{DashO} & \textbf{Other} \\ \cline{3-6} 
 &  & 12,697 & 9,672 & 234 & 3,581 \\ \hline
\end{tabular}%
}
\end{table}

\subsection{Top-k Apps}

Next, we investigate the obfuscation practices of top apps in Google Play Store. First, we rank the apps using the same criterion used by our previous work~\cite{rajasegaran2019multi, karunanayake2020multi, seneviratne2015early}. That is, we sort the apps in descending order of number of downloads, average rating, and rating count, with the intuition that top apps have high download numbers and high ratings, even when reviewed by a large number of users. Then, we investigated the percentage of obfuscated apps and obfuscation tools and technique usage as summarized in Table~\ref{tab:top_k_apps_2018_2023}.

When considering the highly ranked applications (i.e., top-1,000), the obfuscation percentage is notably higher, at around 93\%, in both datasets, which is significantly higher than the average percentage of obfuscation we observed in Section~\ref{sec:obstrend}. Top-ranked apps, likely due to their higher visibility and potential revenue, invest more in obfuscation to safeguard their intellectual property and enhance security. 

The obfuscation percentage decreases when going from the top 1,000 apps to the top 30,000 apps. Nonetheless, the obfuscation percentage in both datasets remains around similar values until the top 30,000 (e.g., $\sim$74\% for top-30,000). This indicates that the major increase in obfuscation in the 2021-2023 dataset comes from apps beyond the top 30,000.

When observing the tools used, the usage of ProGuard increases as we move from top to lower-ranked apps in both datasets. This may be because ProGuard is free and the default in Android Studio, while commercial tools like Allatori and DashO are expensive. There is a notable increase in the use of Allatori among the top apps in the 2021-2023 dataset. Regarding obfuscation techniques, the top 1,000 apps utilize all three techniques more frequently than lower-ranked apps in both snapshots. This indicates that the top 1,000 apps are more heavily protected compared to lower-ranked ones.

\begin{table*}[]
\caption{Summary of analysis results for Top-k apps in 2018 and 2023}
\label{tab:top_k_apps_2018_2023}
\resizebox{\textwidth}{!}{%
\begin{tabular}{lccccccccc}
\hline
\multicolumn{1}{c}{\begin{tabular}[c]{@{}c@{}}Top k apps - \\ Year\end{tabular}} & \begin{tabular}[c]{@{}c@{}}Total \\ Apps\end{tabular} & \begin{tabular}[c]{@{}c@{}}Obfuscation\\ Percentage\end{tabular} & \begin{tabular}[c]{@{}c@{}}ProGuard\\ Percentage\end{tabular} & \begin{tabular}[c]{@{}c@{}}Allatori\\ Percentage\end{tabular} & \begin{tabular}[c]{@{}c@{}}DashO\\ Percentage\end{tabular} & \begin{tabular}[c]{@{}c@{}}Other\\ Percentage\end{tabular} & \begin{tabular}[c]{@{}c@{}}IR\\ Percentage\end{tabular} & \begin{tabular}[c]{@{}c@{}}CF\\ Percentage\end{tabular} & \begin{tabular}[c]{@{}c@{}}SE\\ Percentage\end{tabular} \\ \hline
1k (2018) & 1,000 & 93.40 & 29.98 & 28.48 & 0.64 & 40.90 & 99.90 & 88.76 & 65.42 \\
10k (2018) & 10,000 & 85.19 & 25.55 & 35.32 & 0.47 & 38.65 & 99.90 & 88.76 & 71.91 \\
20k (2018) & 20,000 & 78.42 & 26.31 & 36.76 & 0.57 & 36.36 & 99.87 & 87.37 & 71.49 \\
30k (2018) & 30,000 & 74.40 & 27.30 & 37.71 & 0.64 & 34.36 & 99.82 & 86.75 & 71.11 \\
30k+ (2018) & 314,568 & 53.36 & 36.72 & 34.70 & 1.33 & 27.24 & 99.34 & 83.54 & 63.11 \\ \hline
1k (2023) & 1,000 & 92.50 & 24.00 & 51.89 & 1.95 & 22.16 & 100.0 & 92.54 & 83.68 \\
10k (2023) & 10,000 & 81.88 & 26.03 & 56.20 & 1.03 & 16.74 & 99.89 & 89.40 & 82.01 \\
20k (2023) & 20,000 & 76.62 & 30.48 & 52.92 & 0.96 & 15.64 & 99.93 & 85.80 & 78.01 \\
30k (2023) & 30,000 & 73.72 & 33.87 & 50.34 & 0.89 & 14.90 & 99.95 & 83.31 & 75.34 \\
30k+ (2023) & 206,216 & 61.90 & 46.56 & 38.21 & 0.64 & 14.59 & 99.97 & 77.51 & 62.50 \\ \hline
\end{tabular}%
}
\end{table*}


We present RiskHarvester, a risk-based tool to compute a security risk score based on the value of the asset and ease of attack on a database. We calculated the value of asset by identifying the sensitive data categories present in a database from the database keywords. We utilized data flow analysis, SQL, and Object Relational Mapper (ORM) parsing to identify the database keywords. To calculate the ease of attack, we utilized passive network analysis to retrieve the database host information. To evaluate RiskHarvester, we curated RiskBench, a benchmark of 1,791 database secret-asset pairs with sensitive data categories and host information manually retrieved from 188 GitHub repositories. RiskHarvester demonstrates precision of (95\%) and recall (90\%) in detecting database keywords for the value of asset and precision of (96\%) and recall (94\%) in detecting valid hosts for ease of attack. Finally, we conducted an online survey to understand whether developers prioritize secret removal based on security risk score. We found that 86\% of the developers prioritized the secrets for removal with descending security risk scores.

\section*{Limitations}
While \tripcraft \;significantly enhances the realism and coherence of travel planning datasets, certain limitations remain. One key constraint is the exclusion of events from the core Point of Interest (PoI) list due to the lack of structured temporal information. If such data becomes available, it can be incorporated into our dataset, enabling LLMs to explicitly factor in event timings, ultimately leading to more mature and contextually aware itineraries.

Our dataset is currently designed for U.S. cities, but the construction pipeline can be extended to other geographical regions if the necessary data is available. Expanding to a global scale would require accounting for region-specific differences in travel preferences, transportation infrastructure, and cultural factors, which remain open challenges for future research.

Additionally, our primary objective is not to propose a novel travel planning methodology but to establish a robust dataset, enhanced itinerary generation, and fine-grained evaluation metrics. While \tripcraft \;provides a strong benchmark for assessing LLM-driven travel planning, future research may explore diverse methodologies on this dataset, further advancing personalized AI-driven itinerary synthesis.


\section*{Ethics Statement}
Our study utilizes publicly available web data, which we have carefully scraped to construct our databases while ensuring compliance with relevant terms of use and ethical considerations. To safeguard privacy, we have fully anonymized sensitive personal details. However, with annotators' consent, aggregate demographic statistics are provided in the Appendix. We are committed to transparency and reproducibility in research; therefore, upon acceptance, we will release both the datasets and code to facilitate further exploration by the scientific community.

% \section*{Acknowledgments}



% Bibliography entries for the entire Anthology, followed by custom entries
%\bibliography{anthology,custom}
% Custom bibliography entries only
\bibliography{custom}

\clearpage
\appendix
% \onecolumn

% \section{Appendices}

\noindent{\LARGE \textbf{Appendices}}  % Large, bold title without section numbering

\vspace{0.5cm}
% \label{sec:appendix}

This supplementary material presents additional details on the following aspects:  
\begin{itemize}
    \setlength{\itemsep}{2pt}
    \item \textbf{Appendix A:} Data Sourcing Details
    \item \textbf{Appendix B:} Prompt and Annotation Details  
    \item \textbf{Appendix C:} Case Studies
    \item \textbf{Appendix D:} Annotator Details 
\end{itemize}


\section{Data Sourcing Details}
Our dataset is constructed using current data sources to ensure spatio-temporal consistency and personalization. Below, we detail the sourcing methodology and heuristics for each component:

\subsection{Restaurants}  
We extracted restaurant details using \textbf{TripAdvisor’s Apify scraper}\footnote{\url{https://console.apify.com/actors/dbEyMBriog95Fv8CW/input}}, which provided all necessary attributes except precise pricing. TripAdvisor denotes cost using dollar symbols (\$–\$\$\$) instead of exact values. To estimate absolute prices, we leveraged city-specific restaurant price indices from \textbf{Numbeo}\footnote{\url{https://www.numbeo.com/cost-of-living/}}, scaling them according to the number of dollar symbols in each price rating.  

\subsection{Attractions}  
Attraction details, including subcategories, were sourced from \textbf{TripAdvisor’s Apify scraper}\footnote{\url{https://console.apify.com/actors/dbEyMBriog95Fv8CW/input}}. Since a majority of attractions lacked predefined visit durations, we consulted domain experts to establish category-wise average durations for each attraction type. Finally, each attraction’s duration was assigned as the mean of the categories it belonged to, ensuring a realistic time allocation (Table \ref{tab:subcategory_duration}).

\subsection{Flights}  
We adopted the \cite{xie2024travelplanner} flight database but adjusted all dates to November 2024 to maximize temporal alignment with event data. This adjustment ensures that LLM-generated itineraries incorporate relevant event-based recommendations.  

\subsection{Distance Matrices}  
All pairwise distances were computed using \textbf{OpenStreetMap’s OSRM API}\footnote{\url{http://project-osrm.org/}}, ensuring accurate and real-time routing information.  

\subsection{Accommodations}  
We scraped accommodation listings from Airbnb using \textbf{Apify’s Airbnb scraper}\footnote{\url{https://apify.com/dtrungtin/airbnb-scraper}}. Since minimum stay requirements were not available in the extracted data, we excluded this attribute from our dataset.  

\subsection{Events}  
Event data was collected using \textbf{Ticketmaster’s Apify scraper}\footnote{\url{https://console.apify.com/actors/Hi7bNMx0vqaqvdfZQ}}, covering a diverse range of concerts, sports, theater, and other entertainment events.  

\subsection{Public Transit}  
We sourced transit schedules from the \textbf{General Transit Feed Specification (GTFS)}\footnote{\url{https://gtfs.org/}} for 140 cities. For each Point of Interest (PoI)—including accommodations, restaurants, and attractions—we determined the nearest public transit stop using geodesic distance (computed via \textbf{Geopy}). This enables LLMs to incorporate realistic public transit connectivity when generating travel itineraries. 

\begin{table}[h]
    \centering
    \renewcommand{\arraystretch}{0.9}
    \setlength{\tabcolsep}{8pt} % Adjust column spacing
    \begin{tabular}{l c}
        \toprule
        \textbf{Category} & \textbf{Duration (hrs)} \\
        \midrule
        Boat Tours \& Water Sports & 3.5 \\
        Casinos \& Gambling & 2.5 \\
        Classes \& Workshops & 1.5 \\
        Concerts \& Shows & 2.5 \\
        Food \& Drink & 2.5 \\
        Fun \& Games & 1.5 \\
        Museums & 3.0 \\
        Nature \& Parks & 4.5 \\
        Nightlife & 2.5 \\
        Outdoor Activities & 4.0 \\
        Shopping & 1.5 \\
        Sights \& Landmarks & 3.0 \\
        Spas \& Wellness & 2.0 \\
        Water \& Amusement Parks & 5.0 \\
        Zoos \& Aquariums & 2.5 \\
        \bottomrule
    \end{tabular}
    \caption{Attraction visiting duration (hrs) for each category. Note that an attraction can belong to one or more than one categories.}
    \label{tab:subcategory_duration}
\end{table}



\begin{table*}[h]
    \centering
    \renewcommand{\arraystretch}{1.05}
    \begin{tabular}{|p{4cm}|p{11cm}|}
        \hline
        \textbf{Constraint} & \textbf{Description} \\
        \hline
        \multicolumn{2}{|c|}{\cellcolor{gray!25} \textit{Environment Constraint}} \\
        \hline \rule{0pt}{2.5ex}Unavailable Transportation & There is no available flight or driving information between the two cities.  \\
        % \hline
        Unavailable Attractions & There is no available attraction information in the queried city. \\
        \hline
        \multicolumn{2}{|c|}{\cellcolor{gray!25} \textit{Commonsense Constraint}} \\
        \hline \rule{0pt}{2.5ex}Within Sandbox & All information in the plan must be within the closed sandbox; otherwise, it will be considered a hallucination. \\
        Complete Information & No key information should be left out of the plan, such as the lack of accommodation during travel. \\
        Sufficient Meal Gaps (+) & Meal timings must have a minimum
        gap of four hours between breakfast, lunch, and
        dinner to maintain a natural schedule. \\
        Valid PoI list (+) & The
        point-of-interest (PoI) list must follow strict validity rules: each day’s itinerary must begin and end at the designated accommodation, except on the final day when the traveler departs. The list should be limited to accommodations, attractions, and restaurants, ensuring adequate time gaps between flight arrivals and accommodation check-ins, as well as between accommodation check-outs and departures. \\
        Diverse Events (+) & Event choices should not be repeated throughout the trip. \\
        Within Current City & All scheduled activities for the day must be located within that day’s city(ies). \\
        Reasonable City Route & Changes in cities during the trip must be reasonable. \\
        Diverse Restaurants & Restaurant choices should not be repeated throughout the trip. \\
        Diverse Attractions & Attraction choices should not be repeated throughout the trip. \\
        Non-conf. Transportation & Transportation choices within the trip must be reasonable. For example, having both “self-driving” and “flight” would be considered a conflict. \\
        \hline
        \multicolumn{2}{|c|}{\cellcolor{gray!25} \textit{Hard Constraint}} \\
        \hline \rule{0pt}{2.5ex}Budget & The total budget of the trip. \\
        Room Rule & Room rules include “No parties”, “No smoking”, “No children under 10”, “No pets”, and “No visitors”. \\
        Room Type & Room types include “Entire Room”, “Private Room”, “Shared Room”, and “No Shared Room”. \\
        Cuisine & Cuisines include “Chinese”, “American”, “Italian”, “Mexican”, “Indian”, “Mediterranean”, and “French”. \\
        Transportation & Transportation options include “No flight” and “No self-driving”. \\
        Event Types (+) & Event Types include four distinct categories—Sports, Arts \& Theatre, Music, and Film. \\
        Attraction Types (+) &  Each attraction belongs to one or more of 15 predefined categories, ensuring a well-distributed selection of activities. \\
        \hline
        \multicolumn{2}{|c|}{\cellcolor{gray!25} \textit{Persona Components}} \\
        \hline \rule{0pt}{2.5ex}Traveler Type (+) & Defines how a traveler approaches their journey—whether they seek relaxation in cozy spots or adrenaline-pumping adventures. \\
        Purpose of Travel (+) &  Captures the main motivation behind the trip, whether it’s to unwind, seek thrills, explore cultures, or connect with nature. \\
        Spending Preference (+) &  Reflects the traveler’s budget and style, from luxurious indulgence to cost-conscious experiences. \\
        Location Preference (+) &  Highlights preferred environments, such as beaches, mountains, cities, or wildlife-rich forests. \\
        \hline
    \end{tabular}
    \caption{\textit{Comprehensive Constraint and Persona Description. (+) denotes the ones we have added.} }
    \label{tab:full_const_detail}
\end{table*}

\begin{table*}[h]
    \centering
    \renewcommand{\arraystretch}{1}
    \setlength{\tabcolsep}{2pt} % Adjust column spacing
    \begin{tabularx}{\columnwidth}{l *{3}{>{\centering\arraybackslash}X}}
        \toprule
        \textbf{Parameter} & \textbf{3-day} & \textbf{5-day} & \textbf{7-day} \\
        \midrule
        \multicolumn{4}{c}{\textbf{Restaurant Parameters}} \\
        \midrule
        \textbf{Breakfast} & & & \\
        Mean Time & 9.63 & 9.80 & 9.84 \\ 
        Mean Duration (hrs) & 0.90 & 1.08 & 0.85 \\ 
        Std. Time & 1.08 & 1.08 & 1.34 \\ 
        Std. Duration (hrs) & 0.24 & 1.43 & 0.23 \\ 
        Beta & 0.21 & 0.63 & 0.03 \\ 
        \midrule
        \textbf{Lunch} & & & \\
        Mean Time & 14.30 & 14.46 & 14.44 \\ 
        Mean Duration (hrs) & 1.11 & 1.10 & 0.99 \\ 
        Std. Time & 1.03 & 1.07 & 1.07 \\ 
        Std. Duration (hrs) & 0.36 & 0.35 & 0.26 \\ 
        Beta & 0.10 & 0.04 & 0.30 \\ 
        \midrule
        \textbf{Dinner} & & & \\
        Mean Time & 20.75 & 20.67 & 20.42 \\ 
        Mean Duration (hrs) & 1.19 & 1.32 & 1.15 \\ 
        Std. Time & 1.25 & 1.37 & 1.66 \\ 
        Std. Duration (hrs) & 0.43 & 0.91 & 1.15 \\ 
        Beta & -0.20 & -0.18 & -0.07 \\ 
        \midrule
        \multicolumn{4}{c}{\textbf{Attraction Parameters}} \\
        \midrule
        $\lambda_{laidback}$ & 1.10 & 1.26 & 1.11 \\ 
        $\lambda_{adventurous}$ & 2.01 & 1.61 & 1.82 \\ 
        $\sigma_d$ (hrs) & 1.11 & 1.07 & 0.90 \\ 
        $n^{max}$ & 5 & 4 & 4 \\ 
        $n^{min}$ & 0 & 0 & 0 \\ 
        $k$ (hrs) & 0.28 & 0.28 & 0.28 \\ 
        \bottomrule
    \end{tabularx}
    \caption{A comprehensive list of parameter details for 3-day, 5-day, and 7-day scenarios as calculated from the annotation distribution statistics.}
    \label{tab:parameter_details}
\end{table*}





\onecolumn
\section{Links to open-source project}
\label{appendix:B}

We summarize the datasets, methods, and models used in this work, as shown in Table~\ref{table:7}, Table~\ref{table:8}, and Table~\ref{table:9}, respectively.

\begin{table*}[h]
\fontsize{7.6}{9.}\selectfont
\caption{Links to datasets.}
\label{table:7}
\centering
  \begin{tabular}{l | l }
    \toprule
    \toprule
    \multirow{1}*{Dataset} & Link\\
    \midrule
    AdvBench & \url{https://github.com/llm-attacks/llm-attacks/tree/main/data/advbench} \\
    MaliciousInstruct & \url{https://github.com/Princeton-SysML/Jailbreak_LLM/blob/main/data} \\
    \bottomrule
   \bottomrule
  \end{tabular}
% \vspace{-1.5em}
\end{table*}

\begin{table*}[h]
\fontsize{7.6}{9.}\selectfont
\caption{Links to baseline methods.}
\label{table:8}
\centering
\begin{tabular}{p{1.8cm} | p{10.5cm}}
\toprule
\toprule
\multirow{1}*{Method} & Link\\
\midrule
GCG & \url{https://github.com/llm-attacks/llm-attacks/tree/main/llm_attacks} \\
PAIR & \url{https://github.com/patrickrchao/JailbreakingLLMs} \\
Perplexity Filter & \url{https://huggingface.co/docs/transformers/en/perplexity} \\
Instruction Filter & \url{https://huggingface.co/meta-llama/Llama-Guard-3-8B} \\
SmoothLLM & \url{https://github.com/arobey1/smooth-llm} \\
APAs & \url{https://huggingface.co/datasets/CHATS-Lab/Persuasive-Jailbreaker-Data/blob/main/adv_bench_sub_gpt4.jsonl} \\
\bottomrule
\bottomrule
\end{tabular}
% \vspace{-1.5em}
\end{table*}

\begin{table*}[h]
\fontsize{7.6}{9.}\selectfont
\caption{Links to large language models.}
\label{table:9}
\centering
  \begin{tabular}{l | l }
    \toprule
    \toprule
    \multirow{1}*{Model} & Link\\
    \midrule
    Bert-Base & \url{https://huggingface.co/google-bert/bert-base-cased} \\    
    Bert-Large & \url{https://huggingface.co/google-bert/bert-large-uncased} \\
    Llama-2-7B & \url{https://huggingface.co/meta-llama/Llama-2-7b-hf} \\
    Llama-2-7B-Chat & \url{https://huggingface.co/meta-llama/Llama-2-7b-chat-hf} \\
    Llama-2-13B-Chat & \url{https://huggingface.co/meta-llama/Llama-2-13b-chat-hf} \\
    Llama-3.1-8B-Instruct & \url{https://huggingface.co/meta-llama/Meta-Llama-3.1-8B-Instruct} \\
    Vicuna-13B-V1.5 & \url{https://huggingface.co/lmsys/vicuna-13b-v1.5} \\
    Mixtral-7B-Instruct & \url{https://huggingface.co/mistralai/Mistral-7B-Instruct-v0.2} \\
    GPT-2-Large & \url{https://huggingface.co/openai-community/gpt2-large} \\
    GPT-3.5 & \url{https://platform.openai.com} \\
    GPT-4-0613 & \url{https://platform.openai.com} \\
    Claude-3.5-Sonnet & \url{https://console.anthropic.com} \\
    Gemini-1.5-Flash  & \url{https://aistudio.google.com} \\
    \bottomrule
   \bottomrule
  \end{tabular}
\vspace{-1.2em}
\end{table*}






\newpage
\section{Case Studies}
We showcase examples in this appendix that highlight the challenges within our dataset, underscoring its value in advancing LLMs' travel planning capabilities.

\definecolor{lightyellow}{RGB}{255, 249, 196} % Light yellow for background
\definecolor{fluorescentyellow}{RGB}{255, 235, 59}

\begin{longtable}{|>{\columncolor{lightyellow}}p{0.8\textwidth}|}
    \hline
    \textbf{PoI List Time Exceeds the Departure Limit} \\ \hline
    %\endfirsthead
%
    %\hline
    %\textbf{POI List Time Exceeds the Departure Limit Continued} \\ \hline
    %\endhead

    %Table content starts here
    \makecell[l]{\textbf{Query:} \\ Plan a 3-day trip for 1 person from Denver to Santa Fe from November 1st to \\ November 3rd, 2024, with a budget of \$1,200.} \\ \hdashline
    \makecell[l]{\textbf{Plan:} \\ \{ ...\\       "days": 3,\\       "current\_city": "from Santa Fe to Denver", \\       "transportation": "Flight Number: F3932864, from Santa Fe to Denver, \\ \fcolorbox{fluorescentyellow}{fluorescentyellow}{Departure Time: 13:05}, Arrival Time: 14:16", \\       … \\       "point\_of\_interest\_list": "Cozy cottage in central Santa Fe, stay from 07:00 to \\ 08:30, nearest transit: Cerrillos @ 5th OB, 46.71m away; Tia Sophia's, visit from \\ 09:00 to 09:50, nearest transit: Sandoval @ San Francisco OB, 104.85m away; \\ Museum of International Folk Art, visit from 10:30 to 12:30, nearest transit: Cam. \\ Lejo @ Museum of Int'l. Folk Art, 73.48m away; \fcolorbox{fluorescentyellow}{fluorescentyellow}{La Plazuela, visit from 13:00 to} \\ \fcolorbox{fluorescentyellow}{fluorescentyellow}{14:00}, nearest transit: Cathedral @ Water, 130.45m away." \\     \}} \\ \hdashline
    \makecell[l]{\textbf{Analysis:} \\ In this case, the plan recommends visiting an attraction after the departure time of \\ the flight.} \\ \hline
\end{longtable}

% \begin{figure*}
%     \centering
%     \includegraphics[scale=1]{Images/C_1.png}  % Change filename accordingly
%     \caption{Your caption describing the image.}
%     \label{fig:your_label}
% \end{figure*}
\vspace{20pt}

\begin{longtable}{|>{\columncolor{lightyellow}}p{0.8\textwidth}|}
    \hline
    \textbf{Timestamp Sequence Inconsistency} \\ \hline
    
    %Table content starts here
    \makecell[l]{\textbf{Query:} \\ Plan a 3-day trip for 3 people from Charleston to Washington from November 24th \\ to November 26th, 2024, with a budget of \$3,750. Include visits to sights and \\ landmarks.} \\ \hdashline
    \makecell[l]{\textbf{Plan:} \\ \{ \\       "days": 1, \\ "current\_city": "from Charleston to Washington",\\ … \\ "point\_of\_interest\_list": "\fcolorbox{fluorescentyellow}{fluorescentyellow}{King Bed | Free Parking | DCA Airport, stay from 09:30} \\ \fcolorbox{fluorescentyellow}{fluorescentyellow}{to 08:30}, nearest transit: ENTRANCE B - FORT MYER DR., 360.84m away; \\ Clyde's of Georgetown, visit from 09:40 to 10:30, nearest transit: ROSSLYN, \\ 1120.37m away; White House, visit from 11:00 to 14:00, nearest transit: \\ MCPHERSON SQ METRORAIL STATION - SW CORNER VERMONT AVE. \& \\ I ST., 409.99m away; Joe's Seafood Prime Steak \& Stone Crab, visit from 14:20 \\ to 15:20, nearest transit: MCPHERSON SQ, 150.59m away; The Capital Grille, \\ visit from 20:45 to 22:00, nearest transit: ARCHIVES, 121.81m away; King Bed | \\ Free Parking | DCA Airport, stay from 22:30 to 08:30, nearest transit: ENTRANCE \\ B - FORT MYER DR., 360.84m away." \\ ... \}} \\ \hdashline
    \makecell[l]{\textbf{Analysis:} \\ In this case, the plan mis-structures the time sequence when mentioning the time \\ duration to be spent at the accommodation.} \\ \hline
\end{longtable}

% \begin{figure*}
%     \centering
%     \includegraphics[scale=1]{Images/C_2.png}  % Change filename accordingly
%     \caption{Your caption describing the image.}
%     \label{fig:your_label}
% \end{figure*}
% \newpage
\vspace{20pt}


\begin{longtable}{|>{\columncolor{lightyellow}}p{0.8\textwidth}|}
    \hline
    \textbf{Meal Timing Misalignment} \\ \hline
    
    %Table content starts here
    \makecell[l]{\textbf{Query:} \\ Plan a 3-day trip for 1 person from Birmingham to Detroit from November 19th \\ to November 21st, 2024, with a budget of \$1,500.} \\ \hdashline
    \makecell[l]{\textbf{Plan:} \\ \{ ...\\       "days": 3,\\ … \\ "dinner": "Vertical Detroit, Detroit", \\ … \\ "point\_of\_interest\_list": "Modern Boutique Condo - "Au coeur de Detroit", stay \\ from 08:30 to 09:30, nearest transit: Sproat St - Southbound, 302.98m away; \\ Honest John's Bar \& Grill - Selden St., visit from 09:40 to 10:30, nearest transit: \\ Mack Ave - Southbound, 509.79m away; The Ford Piquette Avenue Plant, visit \\ from 11:00 to 14:00, nearest transit: Amsterdam St - Southbound, 528.25m away; \\ Selden Standard, visit from 14:30 to 15:30, nearest transit: Canfield St \\ - Southbound, 586.42m away; \fcolorbox{fluorescentyellow}{fluorescentyellow}{Vertical Detroit, visit from 16:00 to 17:15}, nearest \\ transit: Grand Circus - Northbound, 257.87m away." \\ \}} \\ \hdashline
    \makecell[l]{\textbf{Analysis:} \\ In this case, the plan schedules dinner from 16:00 to 17:15, which is unrealistic and \\ impractical.} \\ \hline
\end{longtable}

% \begin{figure*}
%     \centering
%     \includegraphics[scale=1]{Images/C_3.png}  % Change filename accordingly
%     \caption{Your caption describing the image.}
%     \label{fig:your_label}
% \end{figure*}

\vspace{20pt}

\begin{longtable}{|>{\columncolor{lightyellow}}p{0.8\textwidth}|}
    \hline
    \textbf{Consistent PoI Time Duration Adherence} \\ \hline
    
    %Table content starts here
    \makecell[l]{\textbf{Query:} \\ Plan a 3-day trip for 1 person from Las Vegas to Fort Wayne from November 17th \\ to November 19th, 2024, with a budget of \$2,100.} \\ \hdashline
    \makecell[l]{\textbf{Plan:} \\ \{ \\       "days": 1, \\ "current\_city": "from Las Vegas to Fort Wayne", \\ "transportation": "Flight Number: F0373120, from Las Vegas to Fort Wayne, \\ Departure Time: 23:24, \fcolorbox{fluorescentyellow}{fluorescentyellow}{Arrival Time: 06:06}", \\ "breakfast": "-", \\ "attraction": "-",} \\ 
    \makecell[l]{"lunch": "-", \\ "dinner": "-", \\ "accommodation": "Sunny Victorian 1-Bedroom apartment near Downtown, \\ Fort Wayne", \\ "event": "-", \\ "point\_of\_interest\_list": "Sunny Victorian 1-Bedroom apartment near Downtown, \\ \fcolorbox{fluorescentyellow}{fluorescentyellow}{stay from 06:30 to 07:00}, nearest transit: Calhoun St. at Taber St. (SB), 76.02m \\ away." \\ \}, \\ \{ \\ "days": 2, \\ "current\_city": "Fort Wayne", \\ "transportation": "-", \\ "breakfast": "Sara's Family Restaurant, Fort Wayne", \\ "attraction": "Fort Wayne Museum of Art, Fort Wayne; Historic Fort Wayne, \\ Fort Wayne", \\ "lunch": "Fort Wayne's Famous Coney Island, Fort Wayne", \\ "dinner": "The Original Oley's Pizza, Fort Wayne", \\ "accommodation": "Sunny Victorian 1-Bedroom apartment near Downtown, \\ Fort Wayne", \\ "event": "-", \\ "point\_of\_interest\_list": "Sunny Victorian 1-Bedroom apartment near Downtown, \\ \fcolorbox{fluorescentyellow}{fluorescentyellow}{stay from 07:00 to 08:30}, nearest transit: Calhoun St. at Taber St. (SB), 76.02m \\ away; Sara's Family Restaurant, \fcolorbox{fluorescentyellow}{fluorescentyellow}{visit from 09:00 to 09:50}, nearest transit: Kroger, \\ 210.11m away; Fort Wayne Museum of Art, \fcolorbox{fluorescentyellow}{fluorescentyellow}{visit from 10:30 to 13:30}, nearest \\ transit: Main St. just past bus hut west of Lafayette St. (WB), 34.80m away; \\ Fort Wayne's Famous Coney Island, \fcolorbox{fluorescentyellow}{fluorescentyellow}{visit from 14:00 to 15:00}, nearest transit: W. \\ Main St. at Harrison St. (WB), 31.97m away; Historic Fort Wayne, \fcolorbox{fluorescentyellow}{fluorescentyellow}{visit from} \\ \fcolorbox{fluorescentyellow}{fluorescentyellow}{15:30 to 18:30}, nearest transit: Spy Run Ave. at Baltes Ave. (NB), 92.88m away; \\ The Original Oley's Pizza, \fcolorbox{fluorescentyellow}{fluorescentyellow}{visit from 19:00 to 20:15}, nearest transit: Liberty Mills \\ Apts., 6101 Cornwallis Dr., 2376.66m away; Sunny Victorian 1-Bedroom \\ apartment near Downtown, \fcolorbox{fluorescentyellow}{fluorescentyellow}{stay from 21:00 to 07:00}, nearest transit: Calhoun St. \\ at Taber St. (SB), 76.02m away." \\ \}, \\ \{ \\ "days": 3, \\ "current\_city": "from Fort Wayne to Las Vegas", \\ "transportation": "Flight Number: F2558843, from Fort Wayne to Las Vegas, \\ \fcolorbox{fluorescentyellow}{fluorescentyellow}{Departure Time: 13:16}, Arrival Time: 14:01", \\ "breakfast": "Cosmos House Of Pancakes, Fort Wayne", \\ "attraction": "Fort Wayne Children's Zoo, Fort Wayne", \\ "lunch": "-", \\ "dinner": "-", \\ "accommodation": "-", \\ "event": "-", \\ "point\_of\_interest\_list": "Sunny Victorian 1-Bedroom apartment near Downtown, \\ \fcolorbox{fluorescentyellow}{fluorescentyellow}{stay from 07:00 to 08:30}, nearest transit: Calhoun St. at Taber St. (SB), 76.02m \\ away; Cosmos House Of Pancakes, \fcolorbox{fluorescentyellow}{fluorescentyellow}{visit from 09:00 to 09:50}, nearest transit: \\ Canterbury Marketplace, 18.19m away; Fort Wayne Children's Zoo, \fcolorbox{fluorescentyellow}{fluorescentyellow}{visit from} \\ \fcolorbox{fluorescentyellow}{fluorescentyellow}{10:30 to 12:30}, nearest transit: Franke Park, 101.97m away." \\ \}} \\ \hdashline
    \makecell[l]{\textbf{Analysis:} \\ In this case, the plan accurately follows predefined time constraints for Points of \\ Interest (PoIs), ensuring structured scheduling.} \\ \hline
\end{longtable}

% \begin{figure*}
%     \centering
%     \includegraphics[scale=0.5]{Images/C_4.png}  % Change filename accordingly
%     \caption{Your caption describing the image.}
%     \label{fig:your_label}
% \end{figure*}

% \newpage
\vspace{20pt}


\begin{longtable}{|>{\columncolor{lightyellow}}p{0.8\textwidth}|}
    \hline
    \textbf{Inconsistent Itinerary with Adventure Seeker Profile} \\ \hline
    
    %Table content starts here
    \makecell[l]{\textbf{Query:} \\ Plan a 3-day trip for 2 people from Miami to Baltimore from November 3rd \\ to November 5th, 2024, with a budget of \$1,800. The itinerary should include visits \\ to zoos and aquariums, as well as fun and games attractions. Additionally, \\ incorporate arts and theatre, and film events. Accommodations should allow pets.} \\ \hdashline
    \makecell[l]{\textbf{Plan:} \\ "persona": \fcolorbox{fluorescentyellow}{fluorescentyellow}{"Traveler Type: Adventure Seeker}; Purpose of Travel: Adventure; \\ Spending Preference: Luxury Traveler; Location Preference: Mountains", \\ "plan": [ \\ \{ \\ "days": 1, \\ … \\ \fcolorbox{fluorescentyellow}{fluorescentyellow}{"attraction": "National Aquarium, Baltimore"}, \\ "lunch": "Miss Shirley's Cafe, Inner Harbor", \\ "dinner": "Restaurante Tio Pepe", \\ "accommodation": "Charming Baltimore Home w/Parking - Dog Friendly!", \\ … \\ \} \\ ]} \\ \hdashline
    \makecell[l]{\textbf{Analysis:} \\ In this case, the plan fails to align with the Adventure Seeker persona type, which \\ expects more than 2 attractions per day.} \\ \hline
\end{longtable}

% \begin{figure*}
%     \centering
%     \includegraphics[scale=1]{Images/C_5.png}  % Change filename accordingly
%     \caption{Your caption describing the image.}
%     \label{fig:your_label}
% \end{figure*}

\vspace{20pt}

\begin{longtable}{|>{\columncolor{lightyellow}}p{0.8\textwidth}|}
    \hline
    \textbf{Constraint-Compliant Itinerary Generation} \\ \hline
    
    %Table content starts here
    \makecell[l]{\textbf{Query:} \\ Plan a 3-day trip for 2 people from Portland to Baltimore from November 5th to \\ November 7th, 2024, with a budget of \$1,500. The trip should include visits to \\ Boat Tours \& Water Sports and Sights \& Landmarks. Note that children under 10 \\ are allowed, and no self-driving transportation should be used.} \\ \hdashline
    \makecell[l]{\textbf{Plan:} \\ "persona": "Traveler Type: Laidback Traveler; Purpose of Travel: Relaxation; \\ Spending Preference: Economical Traveler; Location Preference: Mountains", \\ "plan": [ \\ \{ \\ "days": 1, \\ "current\_city": "from Portland to Baltimore", \\ "transportation": "Flight Number: F0060549, from Portland to Baltimore,} \\ 
    \makecell[l]{Departure Time: 05:42, Arrival Time: 07:07", \\ "breakfast": "Miss Shirley's Cafe, Inner Harbor, Baltimore", \\ "attraction": "Top of the World Observation Level, Baltimore", \\ "lunch": "Phillips Seafood, Baltimore", \\ "dinner": "The Capital Grille, Baltimore", \\ "accommodation": "Private Queen Suite (B) Fells and Hopkins Medical, \\ Baltimore", \\ "event": "-", \\ "point\_of\_interest\_list": "Private Queen Suite (B) Fells and Hopkins Medical, stay \\ from 07:30 to 08:30, nearest transit: JOHNS HOPKINS STATION (METRO), \\ 821.12m away; Miss Shirley's Cafe, Inner Harbor, visit from 09:00 to 09:50, \\ nearest transit: SHOT TOWER METRO West Entrance, 285.29m away; Top of the \\ World Observation Level, visit from 10:30 to 13:30, nearest transit: SHOT TOWER \\ METRO West Entrance, 453.53m away; Phillips Seafood, visit from 14:00 to \\ 15:00, nearest transit: SHOT TOWER METRO West Entrance, 344.26m away; \\ The Capital Grille, visit from 19:00 to 20:15, nearest transit: SHOT TOWER \\ METRO West Entrance, 293.21m away; Private Queen Suite (B) Fells and Hopkins \\ Medical, stay from 21:00 to 07:00, nearest transit: JOHNS HOPKINS STATION \\ (METRO), 821.12m away." \\ \}, \\ \{ \\ "days": 2, \\ "current\_city": "Baltimore", \\ "transportation": "-", \\ "breakfast": "Blue Moon Cafe, Baltimore", \\ "attraction": "Federal Hill Park, Baltimore", \\ "lunch": "La Tavola, Baltimore", \\ "dinner": "Restaurante Tio Pepe, Baltimore", \\ "accommodation": "Private Queen Suite (B) Fells and Hopkins Medical, \\ Baltimore", \\ "event": "-", \\ "point\_of\_interest\_list": "Private Queen Suite (B) Fells and Hopkins Medical, stay \\ from 07:00 to 08:30, nearest transit: JOHNS HOPKINS STATION (METRO), \\ 821.12m away; Blue Moon Cafe, visit from 09:00 to 09:50, nearest transit: SHOT \\ TOWER STATION (METRO) sb, 1161.67m away; Federal Hill Park, visit from \\ 10:30 to 14:00, nearest transit: SHOT TOWER METRO West Entrance, 1089.59m \\ away; La Tavola, visit from 14:30 to 15:30, nearest transit: SHOT TOWER \\ STATION (METRO) sb, 472.69m away; Restaurante Tio Pepe, visit from 19:00 to \\ 20:15, nearest transit: LEXINGTON MARKET METRO North Entrance, 554.30m \\ away; Private Queen Suite (B) Fells and Hopkins Medical, stay from 21:00 to \\ 07:00, nearest transit: JOHNS HOPKINS STATION (METRO), 821.12m away." \\ \}, \\ \{ \\ "days": 3, \\ "current\_city": "from Baltimore to Portland", \\ "transportation": "Flight Number: F0683635, from Baltimore to Portland, \\ Departure Time: 21:41, Arrival Time: 23:01", \\ "breakfast": "Miss Shirley's Cafe, Inner Harbor, Baltimore", \\ "attraction": "Edgar Allan Poe's Grave Site and Memorial, Baltimore",} \\ 
    \makecell[l]{"lunch": "Dalesio's of Little Italy Restaurant, Baltimore", \\ "dinner": "Thames Street Oyster House, Baltimore", \\ "accommodation": "-", \\ "event": "-", \\ "point\_of\_interest\_list": "Private Queen Suite (B) Fells and Hopkins Medical, stay \\ from 07:00 to 08:30, nearest transit: JOHNS HOPKINS STATION (METRO), \\ 821.12m away; Miss Shirley's Cafe, Inner Harbor, visit from 09:00 to 09:50, \\ nearest transit: SHOT TOWER METRO West Entrance, 285.29m away; Edgar \\ Allan Poe's Grave Site and Memorial, visit from 10:30 to 13:30, nearest transit: \\ LEXINGTON MARKET (METRO) sb, 246.16m away; Dalesio's of Little Italy \\ Restaurant, visit from 14:00 to 15:00, nearest transit: SHOT TOWER STATION \\ (METRO) sb, 597.76m away; Thames Street Oyster House, visit from 19:00 to \\ 20:15, nearest transit: SHOT TOWER STATION (METRO) sb, 1428.09m away." \\ \} \\ ]} \\ \hdashline
    \makecell[l]{\textbf{Analysis:} \\ In this case, the plan successfully adheres to persona-based preferences and local \\ constraints, ensuring alignment with user-defined travel expectations.} \\ \hline
\end{longtable}

% \begin{figure*}
%     \centering
%     \includegraphics[scale=0.45]{Images/C_6.png}  % Change filename accordingly
%     \caption{Your caption describing the image. Continued}
%     \label{fig:your_label}
% \end{figure*}

\vspace{20pt}

\begin{longtable}{|>{\columncolor{lightyellow}}p{0.8\textwidth}|}
    \hline
    \textbf{Unrealistic Transit Proximity Issue} \\ \hline
    
    %Table content starts here
    \makecell[l]{\textbf{Query:} \\ Plan a 3-day trip for 1 person from Tampa to Bloomington from November 5th to \\ November 7th, 2024, with a budget of \$1,650.} \\ \hdashline
    \makecell[l]{\textbf{Plan:} \\ \{ \\ "days": 1, \\ "current\_city": "from Tampa to Bloomington", \\ … \\ "accommodation": "Arcade House - Hot Tub, Games, Grill, FirePit, FUN, \\ Bloomington", \\ "event": "-", \\ "point\_of\_interest\_list": "Arcade House - Hot Tub, Games, Grill, FirePit, FUN, \\ stay from 07:00 to 08:30, \fcolorbox{fluorescentyellow}{fluorescentyellow}{nearest transit: Prow Rd @ Meadows Hospital,} \\ \fcolorbox{fluorescentyellow}{fluorescentyellow}{251003.54m away}; Miller Park Zoo, visit from 09:00 to 12:30, \fcolorbox{fluorescentyellow}{fluorescentyellow}{nearest transit:} \\ \fcolorbox{fluorescentyellow}{fluorescentyellow}{Belle \& Park Square, 252594.59m away}; Baxter's American Grill, visit from \\ 13:00 to 14:00, \fcolorbox{fluorescentyellow}{fluorescentyellow}{nearest transit: Prow Rd @ Meadows Hospital, 247021.85m away;} \\ Janko's Little Zagreb, visit from 19:00 to 20:15, nearest transit: Kirkwood Ave @ \\ B-Line Trail IB, 98.16m away; Arcade House - Hot Tub, Games, Grill, FirePit, \\ FUN, stay from 21:00 to 07:00, \fcolorbox{fluorescentyellow}{fluorescentyellow}{nearest transit: Prow Rd @ Meadows Hospital,} \\ \fcolorbox{fluorescentyellow}{fluorescentyellow}{251003.54m away}." \\ \}} \\ \hdashline
    \makecell[l]{\textbf{Analysis:} \\ In this case, the plan selects PoIs whose nearest transit stop is hundreds of\\ kilometers away from accommodations and attractions.} \\ \hline
\end{longtable}

% \begin{figure*}
%     \centering
%     \includegraphics[scale=1]{Images/C_7.png}  % Change filename accordingly
%     \caption{Your caption describing the image.}
%     \label{fig:your_label}
% \end{figure*}

\newpage
\twocolumn
\section{Annotator Details}
\subsection{Guidelines for Annotators}
The annotation process involves generating a travel plan that is both feasible and, if possible, optimal. The annotated plan must be based on reference information while considering constraints such as local preferences (e.g., cuisine type, attraction category) and traveler personas (e.g., laidback, economical). Additionally, common sense should be applied when selecting points of interest, and deviations from suggested durations or costs must be justified. A detailed breakdown of these annotation guidelines, including priority handling, public transit considerations, and documentation requirements, is provided in Table \ref{tab:annotation_guidelines}.  

\begin{figure}[h]
    \centering
    \includegraphics[width=0.8\linewidth]{Images/age_dist.png}
    \caption{Age Distribution of our graduate student annotators.}
    \label{fig:age_distribution}
\end{figure}


\subsection{Annotator Demographics}

The annotator demographics, as illustrated by the figures, show a diverse range of experience levels and backgrounds. The years of English education vary significantly, with a distribution indicating that most annotators have between 12 to 20 years of formal English instruction (Figure \ref{fig:eng_edu_distribution}). This suggests a predominantly well-educated group with proficiency in the language. Additionally, the age distribution skews toward early-to-mid adulthood, with a concentration around 22 to 25 years old (Figure \ref{fig:age_distribution}), implying that most annotators are young graduate students. The gender distribution reflects participation from a diverse range of genders among the 25 graduate students. (Figure \ref{fig:gender_distribution}). 

\begin{figure}[h]
    \centering
    \includegraphics[width=0.8\linewidth]{Images/gender_dist.png}
    \caption{Gender Distribution of our graduate student annotators.}
    \label{fig:gender_distribution}
\end{figure}

\begin{figure}[h]
    \centering
    \includegraphics[width=0.8\linewidth]{Images/eng_edu_dist.png}
    \caption{Years of Formal English Education Statistics of our graduate student annotators.}
    \label{fig:eng_edu_distribution}
\end{figure}

The combination of these factors suggests that the dataset is annotated by individuals with substantial English proficiency and cognitive maturity, contributing to reliable and contextually aware annotations.


\FloatBarrier

\begin{table*}
    \centering
    \renewcommand{\arraystretch}{1.2}
    \begin{tabular}{|c|p{13cm}|}
        \hline
        \textbf{\#} & \textbf{Annotation Guideline} \\
        \hline
        1 & The goal of annotation is to generate a feasible travel plan that meets the query requirements. If multiple valid plans exist, selecting the most optimal plan is encouraged. \\
        \hline
        2 & Every element in the annotated travel plan must be grounded in reference data relevant to the given query. No fabricated or unverifiable information should be included. \\
        \hline
        3 & Common sense should be maintained when selecting travel plans. Refer Table \ref{tab:full_const_detail}. \\
        \hline
        4 & Any local constraints specified in the query must be respected. These may include preferences for cuisine types (e.g., Indian, Mediterranean) or attraction categories (e.g., Sights \& Landmarks, Zoos \& Aquariums). \\
        \hline
        5 & The travel plan should align with the traveler’s persona. For instance, a laidback traveler would prefer a schedule with 1-2 attractions per day, even if more options exist. An economical traveler would favor budget-friendly choices over expensive alternatives. \\
        \hline
        6 & If there is a conflict between local constraints and traveler persona preferences, the local constraints must be prioritized. If no valid plan can be formed while satisfying local constraints, a justification must be provided in Remarks. \\
        \hline
        7 & When selecting Points of Interest (PoIs), priority should be given to those with a public transit stop within 5km. If choosing between a PoI that meets local constraints but lacks transit access and one that is transit-friendly but does not meet constraints, the former should be preferred—unless a better alternative exists. \\
        \hline
        8 & The values for visit duration at attractions and average cost at restaurants should be treated as reference values rather than strict limits. Annotators are allowed to adjust these values if necessary but must document any deviations in the Remarks section. \\
        \hline
        9 & Annotators should use their best judgment to ensure that the generated plans are practical and reasonable. Any significant decision-making considerations should be explicitly noted in the Remarks section. \\
        \hline
    \end{tabular}
    \caption{Guidelines for Annotation of Travel Plans and Remarks}
    \label{tab:annotation_guidelines}
\end{table*}




\end{document}
