\section{Experiments and Results}

% Unicode cannot be used in Bib\TeX{} entries, and some ways of typing special characters can disrupt Bib\TeX's alphabetization. The recommended way of typing special characters is shown in Table~\ref{tab:accents}.

% Please ensure that Bib\TeX{} records contain DOIs or URLs when possible, and for all the ACL materials that you reference.
% Use the \verb|doi| field for DOIs and the \verb|url| field for URLs.
% If a Bib\TeX{} entry has a URL or DOI field, the paper title in the references section will appear as a hyperlink to the paper, using the hyperref \LaTeX{} package.

% \begin{table}[h]
%     \centering
%     \caption{Evaluation Results: New metrics}
%     \label{tab:results}
%     \begin{tabular}{lccc}
%         \toprule
%         \textbf{Metric} & \textbf{3-day} & \textbf{5-day} & \textbf{7-day} \\
%         \midrule
%         \multicolumn{4}{l}{\textit{Direct Sole Planning (w/o Parameter Info)}} \\
%         \midrule
%         Temporal Meal Score & X.XX & X.XX & X.XX \\
%         Temporal Attraction Score & X.XX & X.XX & X.XX \\
%         Spatial Score & X.XX & X.XX & X.XX \\
%         Persona Score & X.XX & X.XX & X.XX \\
%         Ordering Score & X.XX & X.XX & X.XX \\
%         \midrule
%         \multicolumn{4}{l}{\textit{Direct Sole Planning (w/ Parameter Info)}} \\
%         \midrule
%         Temporal Meal Score & X.XX & X.XX & X.XX \\
%         Temporal Attraction Score & X.XX & X.XX & X.XX \\
%         Spatial Score & X.XX & X.XX & X.XX \\
%         Persona Score & X.XX & X.XX & X.XX \\
%         Ordering Score & X.XX & X.XX & X.XX \\
%         \bottomrule
%     \end{tabular}
% \end{table}




\subsection{Experimental Settings}
% We follow the direct sole planning strategy as proposed by \citet{xie2024travelplanner} and employed by \citet{singh-etal-2024-personal}. In addition to this, we modify the prompt by incorporating details about events and the list of points of interest (PoIs) while also refining the one-shot example provided to better suit our specific requirements. These modifications ensure that the generated plans align more closely with our evaluation criteria and domain-specific constraints.

% Beyond this, we introduce a novel setting, direct sole planning (with parameter information), which extends the prompt template by including natural language descriptions of the parameterized distributions used to model the metrics outlined in subsection \ref{subsec:eva_metrics}. The distinction between these settings is illustrated in Appendix B and Figure \ref{fig:potential_proposed_metric}.  

% The motivation behind incorporating parameter information in the prompt is to encourage the LLM to adhere to ideal timings and contextual constraints that may not align with commonsense reasoning, which is inherently subjective. For instance, while it may be commonsense that breakfast occurs in the morning, such a statement remains vague. Given that our evaluation framework penalizes or rewards plans based on continuous probability distributions, omitting precise timing information would introduce unfairness in assessment. By explicitly modeling such constraints, we aim to optimize travel planning specifically for LLMs.  
 
% We conduct comprehensive experiments across three dataset categories — 3-day, 5-day, and 7-day travel plans — using GPT-4o. Our evaluation compares both settings against proposed and existing metrics. For fairness, we report our proposed metrics by averaging results over the set of travel plans that got delivered in both settings (i.e., the intersection of their delivered plans). The results and challenges of our dataset are summarized in Table \ref{tab:new_results} and Table \ref{tab:old_results}, with key insights discussed in the following subsection.

We adopt the direct sole planning strategy \cite{xie2024travelplanner, singh-etal-2024-personal}, modifying the prompt to include event details, PoI lists, and a refined one-shot example tailored to our constraints. These enhancements improve alignment with the domain-specific requirements. Additionally, we introduce a novel setting, direct sole planning (with parameter information), which extends the prompt with natural language descriptions of the parameterized distributions modeling the metrics in subsection \ref{subsec:eva_metrics}. Appendix B and Figure \ref{fig:potential_proposed_metric} illustrate the distinction between these settings.

\noindent Incorporating parameter information encourages LLMs to adhere to precise timings and contextual constraints beyond subjective commonsense reasoning. For example, while breakfast is typically in the morning, its exact timing is ambiguous. Since our framework evaluates plans using continuous probability distributions, explicit constraints ensure fairness and optimization for LLM-based planning.

\noindent We evaluate GPT-4o across 3-day, 5-day, and 7-day travel plans, comparing both settings using proposed and existing metrics\footnote{Since we are not training/ fine-tuning LLMs, we directly report results without making train-val splits.}. For fairness, we report our proposed metrics by averaging results over the set of travel plans that got delivered in both settings (i.e., the intersection of their delivered plans). Results are summarized in Tables \ref{tab:new_results} and \ref{tab:old_results}.%, with key insights in the next subsection.


% \subsection{Benefits and Challenges}
% \label{subsec:analysis}
% In this subsection, we have discussed the advantages of our proposed metrics and highlighted the challenges posed by our dataset. Notably, the agent often struggles to accurately capture constraints and persona preferences, underscoring the value of evaluating future methods on our dataset and reinforcing the significance of our contribution.

% \textbf{Benefit 1. Trade-off between objective metrics and constraint adherence:} From Table \ref{tab:new_results}, we can observe that the parameter information incorporated setting generally enhances performance across our objective metrics, except for the temporal attraction score in the 7-day plan. However, this improvement often comes at the cost of violating constraints, as can be seen by checking the previous metrics in Table \ref{tab:old_results}.

% \noindent\textbf{Benefit 2. Potential of objective criteria:} In Table \ref{tab:old_results}, the 7-day scenario scores zero for common sense macro pass rate, hard constraint macro pass rate, and final pass rate. In such cases, our proposed objective metrics become crucial for differentiating travel plans based on aspects beyond binary pass/fail outcomes. As illustrated in Figure \ref{fig:potential_proposed_metric}, these metrics capture nuanced variations that would otherwise be overlooked, highlighting the trade-off between optimizing for objectives and maintaining constraint adherence.

% \noindent\textbf{Challenge 3 (a). PoI List Time Exceeds the Departure Limit:} The itinerary lacks temporal consistency constraints, causing PoI visits to extend beyond scheduled departure times. This issue arises due to the absence of cross-validation between PoI schedules and transit time requirements, leading to unrealistic plans. \noindent\textbf{3 (b). Timestamp Sequence Inconsistency:} The itinerary exhibits chronological misalignment, where time sequences for activities are incorrectly structured. This issue arises due to improper temporal ordering during itinerary generation, leading to unrealistic scheduling. \noindent\textbf{3 (c). Meal Timing Misalignment:} The itinerary assigns inconsistent meal timings, scheduling dinner significantly earlier than typical dining hours. This issue arises from the lack of meal-time contextual constraints, leading to unrealistic scheduling. Despite specifying meal categories in the prompt, the model fails to interpret contextual meal timings accurately, resulting in impractical itinerary planning. \noindent\textbf{3 (d). Consistent PoI Time Duration Adherence: }
% The itinerary accurately follows predefined time constraints for Points of Interest (PoIs), ensuring structured scheduling. The model correctly interprets duration constraints for meals, maintaining temporal coherence across the plan. This adherence to specified time allocations enhances plan feasibility, optimizing user experience and itinerary practicality.

% \noindent\textbf{Challenge 4 (a). Inconsistent Itinerary with Adventure Seeker Profile:} The generated itinerary fails to align with the Adventure Seeker persona type, which expects more than 2 attractions per day. The model lacks profile-based dynamic itinerary adaptation, resulting in insufficient activity density. Despite explicit prompt instructions, it does not adjust the plan to match the traveler's adventure-oriented preferences.\noindent\textbf{4 (b). Constraint-Compliant Itinerary Generation:} The itinerary generation process successfully adheres to persona-based preferences and local constraints, ensuring alignment with user-defined travel expectations. The model accurately selects PoIs based on specified categories, demonstrating effective constraint handling. This reflects the system’s capability to maintain semantic consistency while dynamically generating personalized travel plans.

% \noindent\textbf{Challenge 5. Unrealistic Transit Proximity Issue.}
% The itinerary assigns impractical transit accessibility, positioning the nearest transit stop hundreds of kilometers away from accommodations and attractions. This discrepancy results in an unrealistic and infeasible travel plan, failing to ensure practical transit connectivity within reasonable proximity.

% Challenges 3 to 5 are presented as case studies in Appendix C.


%\subsection{Benefits and Challenges}
\subsection{Discussions}
\label{subsec:analysis}
This section discusses the observations from the experiments, the advantages of our proposed evaluation metrics, and highlights key challenges posed by our dataset. 
%Our results indicate that while parameter-informed planning improves quantitative itinerary quality, models still struggle with constraint adherence and persona alignment, underscoring the need for more sophisticated reasoning in travel itinerary generation. These findings reinforce the significance of TripCraft as a benchmark for evaluating and advancing LLM-driven travel planning.

\noindent \uline{Observation 1.} \textbf{Trade-off between objective metrics and constraint adherence:} 
Table \ref{tab:new_results} shows that incorporating parameter information significantly improves objective metrics, particularly in meal scheduling and spatial efficiency. However, this improvement comes at the cost of increased constraint violations, as evidenced by lower commonsense pass rates in Table \ref{tab:old_results}. This trade-off suggests that while parameter-informed guidance aids LLMs in structuring itineraries, it may also introduce rigid assumptions that conflict with real-world constraints, requiring future methods to balance optimization and feasibility effectively.

\noindent \uline{Observation 2.} \textbf{Potential of objective criteria:}
Traditional percentage-based evaluations fail to distinguish between travel plans of equal feasibility under given constraints. In Table \ref{tab:old_results}, the 7-day travel plans receive a zero macro pass rate for commonsense and hard constraints, yet our proposed metrics differentiate plans based on aspects beyond binary pass/fail outcomes. As illustrated in Figure \ref{fig:potential_proposed_metric}, our proposed continuous scoring metric captures the difference, such as partial alignment with temporal constraints and near-optimal spatial organization, providing a more interpretable assessment of LLM-generated plans.

\noindent \uline{Observation 3.} \textbf{Limitations in LLM-Generated Itineraries:} Despite structured prompts, the LLM agent exhibits temporal inconsistencies, where PoI visits extend beyond departure times, meal schedules fail to align with natural dining hours, and activity timestamps are misordered, disrupting itinerary flow (Refer to Appendix C). Additionally, agents struggle to adapt activity density to different personas, often under-scheduling for adventure-seeking travelers while generating excessive transit between locations for relaxed travelers. Another limitation is spatial reasoning, where transit assignments are often impractical, with accommodations and attractions placed far from available transit stops, leading to infeasible travel plans. These issues suggest that while LLMs can incorporate structured constraints, they lack robust reasoning over real-world travel logistics, user preferences, and sequential dependencies, making \textit{TripCraft} a crucial benchmark for future advancements.
%This subsection discusses the advantages of our proposed metrics and highlights the challenges posed by our dataset. Notably, the agent often struggles to accurately capture constraints and persona preferences, underscoring the value of evaluating future methods on our dataset and reinforcing the significance of our contribution.

%\noindent \uline{Benefit 1.} \textbf{Trade-off between objective metrics and constraint adherence:}  
%Table \ref{tab:new_results} shows that incorporating parameter information improves objective metrics, except for the 7-day temporal attraction score. However, this often leads to constraint violations, as seen in Table \ref{tab:old_results}.  

%\noindent \uline{Benefit 2.} \textbf{Potential of objective criteria:}  
%In Table \ref{tab:old_results}, the 7-day plan scores zero for common sense macro pass rate, hard constraint macro pass rate, and final pass rate. In such cases, our objective metrics differentiate plans beyond binary pass/fail outcomes. Figure \ref{fig:potential_proposed_metric} illustrates their ability to capture nuances otherwise overlooked.  

%\noindent \uline{Challenge 3.} \textbf{(a) PoI List Time Exceeds Departure Limit:}  
%Temporal inconsistency causes PoI visits to exceed scheduled departure times due to missing coordination between PoI schedules and transit times. \textbf{(b) Timestamp Sequence Inconsistency:}  
%Chronological misalignment occurs due to improper temporal ordering during itinerary generation, leading to unrealistic scheduling.  
%\textbf{(c) Meal Timing Misalignment:}  
%Inconsistent meal timings, such as unusually early dinners, arise from missing contextual constraints, leading the model to misinterpret scheduling despite prompt specifications.
%\textbf{(d) Consistent PoI Time Duration Adherence:}  
%The model correctly adheres to predefined PoI time constraints, ensuring structured scheduling and maintaining temporal coherence.  

%\noindent \uline{Challenge 4.} \textbf{(a) Inconsistent Itinerary with Adventure Seeker Profile:}  
%The itinerary lacks sufficient activity density for the Adventure Seeker persona, failing to exceed two attractions per day despite explicit prompts.  
%\textbf{(b) Constraint-Compliant Itinerary Generation:}  
%The model successfully aligns itineraries with persona preferences and local constraints, ensuring category-based PoI selection and maintaining semantic consistency.  

%\noindent \uline{Challenge 5.} \textbf{Unrealistic Transit Proximity Issue:}  
%The itinerary assigns impractical transit access, placing the nearest stop hundreds of kilometers from accommodations and attractions, leading to infeasible travel plans.  
%Challenges 3 to 5 are presented as case studies in Appendix C.  
