\section{Related Work}
\label{sec:related-work}
\paragraph{Perspectives in linguistics.}CS naturally occurs in communities where two or more languages are in contact, making it a subject of great interest to fields like sociolinguistics and psycholinguistics. From a social perspective, it can be affected by the attitudes of the speakers towards the languages and the CS phenomenon itself. In this respect, it is related and associated with notions of prestige and identity \citep{2025-heredia-actitudes}. For example, in bilingual communities where a language is minoritized, CS can be regarded as an intrusion of the majority language \citep{dewaele-attitudes}. However, for migrant communities, it may be a way to preserve their mother tongue and as an ``emblem of ethnic identity'' \citep{poplack-sometimes}. Once again, its importance in different social contexts highlights the need to consider CS in NLP research, as it plays a crucial role in linguistic interactions and, consequently, the development of language technologies.
\paragraph{CS in NLP.} The processing and understanding of code-switched text can be crucial in the processing of social media data \citep{bali-etal-2014-borrowing-social-media}, and for speech applications, such as speech recognition or speech synthesis \citep{krishna-2017-synthesis}. In fact, non-monolingual speakers have shown preference for chatbots that use CS \citep{10.1145/3392846}. Different approaches may include normalization \citep{parikh-solorio-2021-normalization}, machine translation \citep{xu-yvon-2021-traducir} or modeling code-switched text \citep{gonen-goldberg-2019-language-modeling}. The survey by \citet{winata-etal-2023-decades} covers trends and advances in NLP for code-switched text, including main fields of interest and future research lines. \citet{dogruoz-etal-2021-survey} explain advances in applications of language technologies for code-switched text from a linguistic and social perspective. 

\paragraph{Datasets \& benchmarks for CS.} The majority of code-switched data is obtained from social media, and other popular data sources include recordings and transcriptions \citep{winata-etal-2023-decades}. There have been several shared tasks that deal with CS, for the tasks of Language Identification \citep{solorio-etal-2014-first-shared-task,molina-etal-2016-overview-second-shared-task} and Sentiment Analysis \citep{patwa-etal-2020-semeval}. Two popular benchmarks have been created to answer the demand for evaluation of CS that covers different language pairs and tasks: LINCE \citep{aguilar-etal-2020-lince}, which covers traditional tasks such as Part Of Speech tagging (POS) or Sentiment Analysis (SA); and GLUECoS \citep{khanuja-etal-2020-gluecos}, which incorporates NLU tasks for the Hindi-English pair. As of today, GLUECoS cannot be used without access to the X API.

\paragraph{CS generation.} CS generation has seldom been tackled in previous research. Approaches include using linguistically informed techniques that aim to find out plausible switching points \citep{pratapa-etal-2018-language-modeling-synthetic,gupta-etal-2020-semi,gregorius-okadome-2022-generating-dependency}, data augmentation \citep{tarunesh-etal-2021-machine-translation} and, more recently, prompting LLMs for CS generation \citep{yong-etal-2023-prompting}. To the best of our knowledge, there is no previous research on CS generation with natural CS as a starting point.