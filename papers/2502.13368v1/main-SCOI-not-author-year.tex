% autosam.tex
% Annotated sample file for the preparation of LaTeX files
% for the final versions of papers submitted to or accepted for
% publication in AUTOMATICA.

% See also the Information for Authors.

% Make sure that the zip file that you send contains all the
% files, including the files for the figures and the bib file.

% Output produced with the elsart style file does not imitate the
% AUTOMATICA style. The style file is generic for all Elsevier
% journals and the output is laid out for easy copy editing. The
% final document is produced from the source file in the
% AUTOMATICA style at Elsevier.

% You may use the style file autart.cls to obtain a two-column
% document (see below) that more or less imitates the printed
% Automatica style. This may helpful to improve the formatting
% of the equations, tables and figures, and also serves to check
% whether the paper satisfies the length requirements.

% Please note: Authors must not create their own macros.

% For further information regarding the preparation of LaTeX files
% for Elsevier, please refer to the "Full Instructions to Authors"
% from Elsevier's anonymous ftp server on ftp.elsevier.nl in the
% directory pub/styles, or from the internet (CTAN sites) on
% ftp.shsu.edu, ftp.dante.de and ftp.tex.ac.uk in the directory
% tex-archive/macros/latex/contrib/supported/elsevier.


%\documentclass{elsart}               % The use of LaTeX2e is preferred.

%\documentclass[twocolumn]{autart}    % Enable this line and disable the
% preceding line to obtain a two-column
% document whose style resembles the
% printed Automatica style.
\documentclass[compress,sort,twocolumn]{autart-small-skip}
%\documentclass[compress,sort,twocolumn]{autart}
%\usepackage[round]{natbib}
%\bibliographystyle{abbrvnat}

%\journal{Elsevier}
\usepackage{graphicx}          % Include this line if your
% document contains figures,
%\usepackage[dvips]{epsfig}    % or this line, depending on which
% you prefer.
\usepackage{subfigure}
\usepackage{fancyhdr}
\usepackage{amsmath}
\usepackage{multirow}
\usepackage{amssymb }
\usepackage{color}
\usepackage{graphics} % for pdf,  bitmapped graphics files
\usepackage{graphicx}
%%%%%%%%%%%%%%%%%%%%%%%
%     \usepackage{geometry}
\usepackage{amsmath}
\newtheorem{theorem}{\textbf{Theorem}}
\newtheorem{lemma}{\textbf{Lemma}}
\newtheorem{example}{\textbf{Example}}
%\newtheorem{proof}{\textbf{Proof}}
\newtheorem{corollary}{\textbf{Corollary}}
\newtheorem{remark}{\textbf{Remark}}
\newtheorem{definition}{\textbf{Definition}}
\newtheorem{problem}{\textbf{Problem}}
\newtheorem{proposition}{\textbf{Proposition}}
\newtheorem{assumption}{\textbf{Assumption}}
\usepackage{url}
%   \usepackage{algpseudocode}
%\newtheorem{proof}{\}
\newenvironment{proof}{{{\bf Proof:}}}{\hfill $\square$\par}

\usepackage{multirow} %multirow for format of table
\usepackage{xcolor}
\usepackage{multirow}
\usepackage{setspace}
%\usepackage{geometry}
\usepackage{url}
\usepackage{subfigure}
\usepackage{fancyhdr}
\usepackage{amsmath}
\usepackage{multirow}
\usepackage{amssymb }
\usepackage{color}
\usepackage{graphics} % for pdf,  bitmapped graphics files
\usepackage{graphicx}
%                               	\usepackage{algorithm} %format of the algorithm
%                               	\usepackage{algorithmic} %format of the algorithm
%\usepackage{subcaption}
\usepackage{subeqnarray}
\usepackage{cases}
\usepackage{threeparttable}
\usepackage{algorithm}
\usepackage{algpseudocode}
%     	\usepackage[paper=letterpaper,top=0.50in,bottom=0.60in,right=0.50in,left=0.50in]{geometry}
\renewcommand{\baselinestretch}{0.88} \normalsize
\begin{document}
	
	\begin{frontmatter}
		%\runtitle{Insert a suggested running title}  % Running title for regular
		% papers but only if the title
		% is over 5 words. Running title
		% is not shown in output.
		
		\title{{\large{A Note on Structural Controllability and Observability Indices}}} % Title, preferably not more
		% than 10 words.
		
		\thanks[footnoteinfo]{This paper was not presented at any IFAC
			meeting. This work was supported in part by the
			National Natural Science Foundation of China under Grant 62373059.}
		
		{\small{	\author[bit]{Yuan Zhang}\ead{zhangyuan14@bit.edu.cn},    % Add the
				\author[bit]{Ranbo Cheng}\ead{chengranbo123@163.com},               % e-mail address
				\author[bit]{Ziyuan Luo}\ead{ziyuan.luo@bit.edu.cn},
				\author[bit]{Yuanqing Xia}\ead{xia\_yuanqing@bit.edu.cn} }} % (ead) as shown
		
		\address[bit]{School of Automation, Beijing Institute of Technology, Beijing, China}  % Please supply
		% here.
		
		
		\begin{keyword}                           % Five to ten keywords,
			Structural controllability index, graph-theoretic characterizations, cactus, dynamic graph, gammoid              % chosen from the IFAC
		\end{keyword}                             % keyword list or with the
		% help of the Automatica
		% keyword wizard
		
		
		% Abstract of not more than 200 words.  irrespective of the symmetric weight constraint
		%In this note, structural controllability index of structured systems is investigated. A new graph-theoretic characterization is given when the systems are not necessarily structurally controllable, which not only covers the existing one for structurally controllable systems, but also fixes a flaw in its proof. It is further demonstrated that several existing algorithms, claimed to be right, fail to compute the exact structural controllability indices. Therefore, whether structural controllability indices can be computed in polynomial time is still open. An efficiently computable tight lower bound is finally given.
		%\begin{abstract} % In this note, we investigate the structural controllability index of structured systems. We provide a new graph-theoretic characterization applicable to systems that are not necessarily structurally controllable. This characterization not only encompasses existing results for structurally controllable systems but also corrects a flaw in the previous proof. Additionally, we demonstrate that several existing algorithms, previously asserted to be correct, fail to compute the exact structural controllability indices. Consequently, whether structural controllability indices can be computed in polynomial time remains open.  Given this, we present an efficiently computable tight lower bound and a greedy graph-partitioning based upper bound for structural controllability indices. %Finally, we present an efficiently computable tight lower bound for structural controllability indices.
		%	\\
		\begin{abstract} In this note, we investigate the structural controllability and observability indices of structured systems. We provide counter-examples showing that an existing graph-theoretic characterization for the structural controllability index (SCOI) may not hold, even for systems with self-loop at every state node. We further demonstrate that this characterization actually provides upper bounds, and extend them to new graph-theoretic characterizations applicable to systems that are not necessarily structurally controllable. Additionally, we reveal that an existing method may fail to obtain the exact SCOI. Consequently, complete graph-theoretic characterizations and polynomial-time computation of SCOI remain open.  Given this, we present an efficiently computable tight lower bound, whose tightness is validated by numerical simulations. All these results apply to the structural observability index by the duality between controllability and observability.
		\end{abstract}
	\end{frontmatter}
	% and a greedy graph-partitioning based upper bound for the structural controllability index
	{\small{
			\section{Introduction} \label{intro-sec}
			%Controllability and observability are two fundamental concepts in control theory since \cite{kalman1963controllability}.
			%However, in many settings, it is often not enough to know the controllability and observability of a system, but the information on the minimum time required to reach a desired state and the minimum length of outputs required to uniquely infer the initial states is also desirable \cite{pequito2017trade,dey2021complexity}. The controllability and observability indices capture such quantities.
			Controllability and observability are two fundamental concepts in control theory \cite{chen1984linear}. However, in many settings, it is often not sufficient to merely know a system's controllability and observability. It is also desirable to acquire the time required to reach a desired state and the length of outputs needed to uniquely infer the initial states \cite{pequito2017trade,dey2021complexity}. The controllability and observability indices capture these quantities.
			%However, in many settings, it is not only important to know whether a system is controllable (observable), but also desirable to acquire the minimum time required to reach a desired state (the minimum length of outputs required to uniquely infer the initial states) \cite{pequito2017trade,dey2021complexity}. The controllability and observability indices capture such quantities.
			
			To be specific, consider a linear time-invariant system
			\begin{equation}\label{plant}
			x(t+1)=\tilde Ax(t)+\tilde Bu(t), y(t)=\tilde Cx(t),
			\end{equation}
			where $t\in {\mathbb N}$, $x(t)\in {\mathbb R}^n$, $u(t)\in {\mathbb R}^m$, $y(t)\in {\mathbb R}^p$ are state, input, and output vectors at time $t$, respectively. Accordingly, $\tilde A\in {\mathbb R}^{n\times n}$, $\tilde B\in {\mathbb R}^{n\times m}$, and $\tilde C\in {\mathbb R}^{p\times n}$. The $k$-step controllability matrix of $(\tilde A,\tilde B)$ is given by ${\mathcal C}_{k}(\tilde A,\tilde B)=[\tilde B,\tilde A\tilde B,...,\tilde A^{k-1}\tilde B]$. The controllability index of system (\ref{plant}) (or the pair $(\tilde A,\tilde B)$) is defined as \cite[Chap 6.2.1]{chen1984linear}
			$${\mu(\tilde A,\tilde B)}\doteq \min\{k\in [n]: {\rm rank}\,{\mathcal C}_{k}(\tilde A,\tilde B)\!=\!{\rm rank}\,{\mathcal C}_{k+1}(\tilde A,\tilde B)\},$$where the notation $[n]\doteq \{1,...,n\}$ for any $n\ge 1$. It is worth mentioning that the above definition does not require system (\ref{plant})  to be controllable (if $(\tilde A,\tilde B)$ is not controllable, $\mu(\tilde A,\tilde B)$ is actually the controllability index of its controllable subsystem \cite{chen1984linear}). When $(\tilde A,\tilde B)$ is controllable, ${\mu}(\tilde A,\tilde B)$ is the minimum number of time steps required to steer the system from an initial state $x_0$ to any final state $x_f\in {\mathbb R}^n$. The controllability index also dictates the minimum degree required to achieve pole placement and model matching \cite[Chap 9]{chen1984linear}. By the duality between controllability and observability, the observability index of system (\ref{plant}) (or the pair $(\tilde A,\tilde C)$), given by $\omega(\tilde A,\tilde C)$,  is defined as $\omega(\tilde A,\tilde C)=\mu(\tilde A^{\intercal},\tilde C^{\intercal})$. When system (\ref{plant}) is observable, $\omega(\tilde A,\tilde C)$ is the minimum length of outputs required to uniquely reconstruct the initial states \cite[Chap 6.3.1]{chen1984linear}.
			In the behavior system theory, $\omega(\tilde A,\tilde C)$, also termed {\emph{lag}} of $(\tilde A,\tilde C)$, plays an important role in the system input/output/state description \cite{willems1986time,camlibel2024shortest}.
			
%			 Similarly, define ${\mathcal O}_k(\tilde A,\tilde C)\doteq[\tilde C^{\intercal},\tilde A^{\intercal}\tilde C^{\intercal},...,\tilde {A^{\intercal}}^{k-1}\tilde C^{\intercal}]^{\intercal}$. The observability index of system (\ref{plant}) (or the pair $(\tilde A,\tilde C)$) is defined as
%			\begin{equation}\label{observability-index}
%			{\omega(\tilde A,\tilde C)}\doteq \min\{k\in [n]: {\rm rank}\,{\mathcal O}_{k}(\tilde A,\tilde C)\!=\!{\rm rank}\,{\mathcal O}_{k+1}(\tilde A,\tilde C)\}.
%			\end{equation}When system (\ref{plant}) is observable, $\omega(\tilde A,\tilde C)$ is the minimum length of outputs required to uniquely reconstruct the initial states \cite[Chap 6.3.1]{chen1984linear}.
%			In the behavior system theory, $\omega(\tilde A,\tilde C)$, also termed {\emph{lag}} of $(\tilde A,\tilde C)$, plays an important role in the system input/output/state description \cite{willems1986time,camlibel2024shortest}. By the duality between controllability and observability, $\omega(\tilde A,\tilde C)=\mu(\tilde A^{\intercal},\tilde C^{\intercal})$. Hence, in the rest of the paper, we will mainly focus on characterizing the controllability indices.
%			
			%characterizing the minimal data required for data-driven analysis and control
			
			In practice, the exact values of $(\tilde A,\tilde B,\tilde C)$ may be hard to know due to parameter uncertainties or modeling errors. Instead, their zero-nonzero patterns are often easier to obtain \cite{Ramos2022AnOO,zhang2024reachability}. In this case, the structured system theory provides an alternative framework for system analysis based on the combinatorial properties of system structure \cite{Ramos2022AnOO}. The structural counterpart of controllability and observability indices, namely, structural controllability and observability indices,  introduced in \cite{mortazavian1982k}, captures the genericity embedded in the original concepts. That is, almost all realizations of a structured system have the same value of the controllability/observability index that is sorely determined by the system zero-nonzero structure (see Definition \ref{def-structural-index}). Various graph-theoretic characterizations for structural controllability and observability indices have been proposed \cite{sueur1997controllability,pequito2017trade}. In particular, \cite{sueur1997controllability} presented an algorithm for computing the structural controllability index (SCOI). Some graph-theoretic criteria were given in \cite{sundaram2012structural} and \cite{pequito2017trade}. Heavily based on a characterization in the latter reference,  \cite{pequito2017trade} and \cite{dey2021complexity} studied the minimal actuator and sensor placement problems for bounding the controllability and observability indices of the resulting systems.
			
			In this note, we reveal via counter-examples, that a fundamental characterization for the SCOI given in \cite{pequito2017trade} may not hold, even for systems with self-loop at each state node. We  show that this characterization actually provides upper bounds for SCOI. We further propose a new graph-theoretic characterization, providing upper bounds for SCOI, which is applicable to structurally uncontrollable systems. We also demonstrate that the algorithm given in \cite{sueur1997controllability} may fail to compute the exact SCOI. These results imply that complete graph-theoretic characterizations and polynomial-time computation of SCOI are still open.  Given this, we present an efficiently computed tight lower bound for SCOI, based on the {\emph{dynamic graph} and its {\emph{gammoid}} structure.
			
			We highlight that our established upper bound and lower one apply to systems that are not necessarily structurally controllable. This is desirable when the
			controllability (observability) index of uncontrollable (unobservable) systems is involved. Such scenarios include, characterizing the data length required in the data-driven attack detection \cite{krishnan2020data}, describing the shortest experiment for linear system identification \cite{camlibel2024shortest}, and the reduced-order functional observer design \cite{zhang2024functional,on2025Mohamed}, where the observability index of unobservable systems plays an important role.
			
			%The controllability/observability indices are studied with the prerequisite that the corresponding system is controllable/observable in the majority of literature. Bellow, we provide several scenarios where controllability/observability indices of uncontrollable/unobservable systems are desirable.
			
			%			{\bf Scenario 1: Data length in data-driven attack detection.}  The observability index of (\ref{plant}) is used in \cite{krishnan2020data} to characterize the minimum horizon length of outputs $y(t)$ to make the proposed data-driven attack detection algorithm valid (see Theorem 3 of \cite{krishnan2020data}), which does not require $(\tilde A,\tilde C)$ to be observable.
			%
			%{\bf Scenario 2: Shortest experiment for linear system identification. } The observability index of (\ref{plant}) plays an important role in characterizing the minimal data length of any experiment that is informative for system identification; see \cite{camlibel2024shortest}. %also known as {\emph{lag}} \cite{willems1986time},
			%			
			%			{\bf Scenario 3: Reduced-order functional observer design.} The observability index of (\ref{plant}) is directly linked to the reduced-order functional observer design \cite[Sec 3]{trinh2011functional}. It has been shown in \citet[Theo 4]{asymptotic2022Mohamed} that if a system is functionally observable with respective to a linear function of states $z(t)=Fx(t)$ (see \cite{fernando2010functional} for the definition of functional observability, which does not require the observability of $(\tilde A,\tilde C)$) and $(\tilde A,\tilde C)$ has an observability index $\omega(\tilde A,\tilde C)$, then there exists a functional observer of order $r\times (\omega(\tilde A,\tilde C)-1)$ that can estimate the function of states $z(t)=Fx(t)$ asymptotically.
			
			
			\section{Preliminaries}
		 	A directed graph (digraph) is denoted by ${\mathcal G}=(V,E)$, where $V$ is the vertex (or node) set and $E\subseteq V\times V$ is the edge set. A subgraph ${\mathcal G}_s=(V_s,E_s)$ of ${\mathcal G}$ is a graph such that $V_s\subseteq V$ and $E_s\subseteq E$. We say ${\mathcal G}_s$ spans ${\mathcal G}$ if $V_s=V$. A path is a sequence of edges $(v_1,v_2)$, $(v_2,v_3)$,...,$(v_{k-1},v_k)$ with $(v_j,v_{j+1})\in E$, $j=1,...,k-1$. A cycle is a path whose start vertex and end vertex coincide, and no other vertices appear more than once in this path. A cycle is a {\emph{self-loop}} if it contains only one edge. A tree is a digraph with no cycles and every vertex, except one, which is called the root, has in-degree exactly being $1$. A forest is a collection of disjoint\footnote{In this paper, ``disjoint'' means ``vertex-disjoint''.} trees.

A structured matrix is a matrix whose entries are either fixed zero or free parameters that can take arbitrary real values independently of other entries. The latter class of entries is called nonzero entries. Assigning values to the nonzero entries of a structured matrix yields a realization. Throughout this paper, let $A$ and $B$ be structured matrices, with dimension of $n\times n$ and of $n\times m$ respectively, such that $(\tilde A, \tilde B)$ in (\ref{plant}) is a realization of $(A,B)$. We say $(A,B)$ is structurally controllable, if there exists a controllable realization for it \cite{Ramos2022AnOO}. The system digraph ${\mathcal G}(A, B)$ of $(A,B)$ is defined as ${\mathcal G}(A, B)=(X\cup U, E_{UX}\cup E_{XX})$, where the state nodes $X\!=\!\{x_1,...,x_n\}$, input nodes $U=\{u_1,...,u_m\}$, edges $E_{XX}=\{(x_i,x_j): A_{ji}\ne 0\}$ and $E_{UX}=\{(u_i,x_j):B_{ji}\ne 0\}$.  Let ${\mathcal G}(A)\doteq (X,E_{XX})$. A {\emph{stem}} is a path starting from an input node and ending at a state node without repeated nodes in this path. 


Let the entries $P_{ij}$ of a matrix $P$ be polynomials in $d$ free parameters (for example, $P$ is the product of several structured matrices). Its {\emph{generic rank}}, given by ${\rm gk}\, P$, is the maximum rank this matrix can achieve as a function of the $d$ free parameters in $P$. Here, the generic rank also equals the rank that this matrix can achieve for almost all choices of parameter values (i.e., all except for some proper variety) in the parameter space ${\mathbb R}^d$ \cite[page 38]{Murota_Book}.
			
			\begin{definition}\label{def-structural-index}
				The {\emph{structural controllability index}} of $(A, B)$, given by $\mu(A, B)$, is
				\begin{equation}\label{controllability-index}
				{\mu(A,B)}\doteq \min\{k\in [n]: {\rm gk}\,{\mathcal C}_{k}(A,B)\!=\!{\rm gk}\,{\mathcal C}_{k+1}(A,B)\}.
				\end{equation}
			\end{definition}
			
			
			\begin{lemma} \label{lemma-genericity}
				The {\emph{SCOI}} of $(A, B)$ is the controllability index of almost all realizations of $(A, B)$.
				%Given $(A, B)$, almost all of its realizations have the same controllability indices.
			\end{lemma}
			
			\begin{proof} In what follows, for a structured matrix $M$, let $n_M$ be the number of free parameters in $M$. For a set ${\mathbb V}\subseteq {\mathbb R}^{n_M}$, by $\tilde M\in {\mathbb V}$ we mean $\tilde M$ is a realization of $M$ obtained by assigning values from ${\mathbb V}$ to the free parameters of $M$.
				Suppose $\mu(A,B)=h$ and ${\rm gk}{\mathcal C}_{h}(A,B)={\rm gk}{\mathcal C}_{h+1}(A,B)=q\le n$. By Definition \ref{def-structural-index}, it holds
				${\rm gk}\,{\mathcal C}_{h-1}(A,B)<q$.
 From the definition of generic rank, there exist proper varieties ${\mathbb V}_1$, ${\mathbb V}_2$, and ${\mathbb V}_3$ of ${\mathbb R}^{n_A}\times {\mathbb R}^{n_B}$, such that ${\rm rank}\,{\mathcal C}_{h}(\tilde A, \tilde B)=q, \forall (\tilde A, \tilde B)\in {\mathbb R}^{n_A}\times {\mathbb R}^{n_B}\backslash {\mathbb V}_1$, ${\rm rank}\,{\mathcal C}_{h+1}(\tilde A, \tilde B)=q, \forall (\tilde A, \tilde B)\in {\mathbb R}^{n_A}\times {\mathbb R}^{n_B}\backslash {\mathbb V}_2$, and $ {\rm rank}\,{\mathcal C}_{h-1}(\tilde A, \tilde B)<q, \forall (\tilde A, \tilde B)\in {\mathbb R}^{n_A}\times {\mathbb R}^{n_B}\backslash {\mathbb V}_3$.
			

%	\[\begin{array}{c}{\rm rank}\,{\mathcal C}_{h}(\tilde A, \tilde B)=q, \forall (\tilde A, \tilde B)\in {\mathbb R}^{n_A}\times {\mathbb R}^{n_B}\backslash {\mathbb V}_1,\\
%				{\rm rank}\,{\mathcal C}_{h+1}(\tilde A, \tilde B)=q, \forall (\tilde A, \tilde B)\in {\mathbb R}^{n_A}\times {\mathbb R}^{n_B}\backslash {\mathbb V}_2,\\
%				{\rm rank}\,{\mathcal C}_{h-1}(\tilde A, \tilde B)<q, \forall (\tilde A, \tilde B)\in {\mathbb R}^{n_A}\times {\mathbb R}^{n_B}\backslash {\mathbb V}_3,\end{array}.\]
				It then follows that $\forall (\tilde A, \tilde B)\in {\mathbb R}^{n_A}\times {\mathbb R}^{n_B}\backslash ({\mathbb V}_1\cup {\mathbb V}_2\cup {\mathbb V}_3)$,
				${\rm rank}\,{\mathcal C}_{h-1}(\tilde A, \tilde B)<{\rm rank}\,{\mathcal C}_{h}(\tilde A, \tilde B)={\rm rank}\,{\mathcal C}_{h+1}(\tilde A, \tilde B)$. This implies $\mu(\tilde A, \tilde B)=h$, for almost all realizations $(\tilde A, \tilde B)$ of $(A,B)$, noting that ${\mathbb V}_1\cup {\mathbb V}_2\cup {\mathbb V}_3$ is a proper variety.
			\end{proof}
			
					
			Associated with ${\mathcal C}_l(A,B)$, $l\in [n]$, the dynamic graph ${\mathcal D}_l(A,B)$ is defined on the vertex set $V_A\cup V_B$ with $V_A=\{x_i^k: i=1,\cdots, n, k=1,\cdots,l\}$ and $V_B=\{u_{i}^k:i=1,\cdots, m, k=1,\cdots, l\}$, and the edge set $\{(x_j^{{k}+1},x_i^{{k}}): A_{ij}\ne 0,{k}=1,...,l-1\}\cup \{(u^{{k}}_j,x^{{k}}_i): B_{ij}\ne 0:, {k}=1,...,l\}$. Let $V_{A}^{(k)}=\{x^k_i:i=1,...,n\}$ for $k=1,...,l$ and $V_B^{(k)}=\{u^k_i:i=1,...,m\}$ for $k=1,...,l$. {Here, $x_i^k$ ($u_i^{k}$, respectively) is the $k$th copy of the state node $x_i$ (input node $u_i$), $k=1,...,l$.} Let $V_X\doteq \{x_1^1,...,x_n^1\}$.  Upon letting $A=[a_{ij}],B=[b_{ij}]$,  the weight of an edge in ${\mathcal D}_l(A,B)$ is set as $w(e)=a_{ij}$ for $e=(x_j^{{k}+1},x_i^{{k}})$, and $w(e)=b_{ij}$ for $e=(u^{{k}}_j,x^{{k}}_i)$.
			{Fig. \ref{counter-example-fig}(d) illustrates a dynamic graph ${\mathcal D}_5(A,B)$ with the digraph ${\mathcal G}(A,B)$ given in Fig. \ref{counter-example-fig}(a).}
			
			
			A collection $L=(p_1,...,p_k)$ of vertex-disjoint paths in ${\mathcal D}_l(A,B)$ is called a {\emph{linking}}, and the size of a linking is the number $k$ of paths it contains.
			Let ${\rm tail}(L)$ and ${\rm head}(L)$ be respectively the set of start vertices and the set of end vertices of paths in $L$.
			A linking $L$ is called a $(S,T)$ linking, if $S={\rm tail}(L)$, and $T={\rm head}(L)$. For a linking $L=(p_1,...,p_k)$, its weight $w(L)$ is defined as the product of the weights of all individual edges of the paths $p_i$ in $L$, $i=1,...,k$. Let $\prec$ be a fixed order of vertices in ${\mathcal D}_l(A,B)$. Consider the linkings from $V_B$ to $V_X$ in ${\mathcal D}_l(A,B)$,  which are $(J,I)$ linkings with $J\subseteq V_B$ and $I\subseteq V_X$. For such a linking $L=(p_1,...,p_k)$, suppose ${\rm tail}(L)=\{s_1,...,s_k\}$ and ${\rm head}(L)=\{t_1,...,t_k\}$ such that $s_1\prec\cdots\prec s_k$ and
			$t_1\prec\cdots\prec t_k$. Moreover, suppose that $s_{\pi(i)}$ and $t_i$ are respectively the start vertex and end vertex of $p_i$, $\pi(i)\in \{1,...,k\}$, $i=1,...,k$. Then, $\pi=(\pi(1),...,\pi(k))$ is a permutation of $(1,...,k)$. The sign of $L$ is defined as ${\rm sign}(L)\doteq {\rm sign}(\pi)$, where ${\rm sign}(\pi)$ is the sign of the permutation $\pi$. %Recall that ${\rm sign}(\pi)=(-1)^q$, where $q$ is
			%the number of transpositions (swaps of two positions) required to transform $\pi$ into
		%	an ordered sequence. 
			
			
		For a fixed $l\in [n]$, a linking from $V_B$ to $V_X$  of the maximum possible size in ${\mathcal D}_l(A,B)$ is called a maximum linking, whose size is denoted by $d({\mathcal D}_l(A,B))$. Observe that there are one-one correspondences between rows (columns) of ${\mathcal C}_l(A,B)$ and the set $V_X$ ($V_B$).  {The lemma below relates the determinant of a square submatrix of ${\mathcal C}_l(A,B)$ to linkings in ${\mathcal D}_l(A,B)$.
				
				\begin{lemma}[\cite{murota1990note}] \label{linking-lemma} Let ${\mathcal C}_l(I,J)$ be the submatrix of ${\mathcal C}_l(A,B)$ consisting of rows corresponding to $I$ and columns corresponding to $J$, $I\subseteq V_X$, $J\subseteq V_B$. Then,
					\begin{equation} \label{det}
						\det {\mathcal C}_l(I,J)=\sum\nolimits_{L}{\rm sign}(L)w(L),
					\end{equation}
					where the summation is taken over all $(J,I)$ linkings $L$ from $V_B$ to $V_X$ of ${\mathcal D}_l(A,B)$.
				\end{lemma}
			%	Observe that there are one-one correspondences between rows of ${\mathcal C}_l(A,B)$ and the set $V_X$, and the columns of ${\mathcal C}_l(A,B)$ and the set $V_B$. %, and $J$ and $I$ are regarded as subsets of $V_{B}$ and $V_X$, respectively
			
				\section{Existing results revisited and upper bound}
			In this section, we demonstrate that several existing characterizations for SCOI may not hold. We then develop a new graph-theoretic characterization applicable to systems that are not necessarily structurally controllable, providing upper bounds for SCOI. 
			
\begin{definition}[\cite{pequito2017trade}] 
	A cactus of ${\mathcal G}(A,B)$ is defined recursively as follows: a stem is a cactus; a cactus connected by an {\emph{edge}} (not belonging to the former cactus) to a disjoint cycle is also a cactus.	A disjoint union of cacti is called a {\emph{spanning cactus family}} of ${\mathcal G}(A,B)$, if the set of state nodes in these cacti is the whole state set $X$, and is called an {\emph{optimal spanning cactus family}} if the number of state nodes in its {\emph{largest}} cactus\footnote{Here, the size of a graph is the number of nodes it contains.} is minimal over all possible spanning cactus families of ${\mathcal G}(A,B)$.
\end{definition}			
%\footnote{Notice that only if $(A,B)$ is structurally controllable can ${\mathcal G}(A,B)$ have a spanning cactus family.}
			
		
			
			
			\begin{lemma}[Theo 2, Coro 1 of \cite{pequito2017trade}] \label{known-result}
			A structurally controllable pair $(A,B)$\footnote{$(A,B)$ is structurally controllable if and only if there is a spanning cactus family in ${\mathcal G}(A,B)$ \cite{generic}. } has an SCOI $k$, if and only if the largest cactus in an optimal spanning cactus family of ${\mathcal G}(A,B)$ contains $k$ state nodes.  Moreover, suppose every state node has a self-loop in ${\mathcal G}(A,B)$. Let $\{{\mathcal H}_i\}_{i\in {\mathcal K}}$ be the collection of all forests rooted at $U$ spanning ${\mathcal G}(A,B)$, where ${\mathcal K}$ contains the indices of such forests. Let ${\mathcal H}_i=\{{\mathcal T}^i_1,...,{\mathcal T}^i_{l_i}\}$ consist of $l_i$ directed trees. Denote $|{\mathcal T}_j^i|_s$ as the number of state vertices in ${\mathcal T}_j^i$.  Then,
			$\mu(A,B)=\min_{i\in {\mathcal K}} \max_{{\mathcal T}^i_j\in {\mathcal H}_i} |{\mathcal T}^i_j|_s.$
			\end{lemma}
		%\footnote{The text of this lemma is a refined version of \cite[Theo 2]{pequito2017trade} based on its proof.}
			%		${\mathcal G}(A,B)$ is spanned by a disjoint union of cacti, where every cactus contains at most $k$ state nodes.\footnote{From the proof of \cite[Theo 2]{pequito2017trade}, consider a disjoint union of cacti ${\mathcal C}^*$ that spans ${\mathcal G}(A,B)$, with the property that the number of state nodes in the largest cactus is minimal over all possible disjoint unions of cacti of ${\mathcal C}^*$  }
			%			\begin{theorem}
			%				A pair $(A,B)$ has a structural controllability index at most $k$ (resp., has a structural controllabilit index $k$), if and only if the ${\mathcal G}(A,B)$ is spanned by a disjoint union of maximum cactus configurations, in which every cactus configuration covers at most $k$ state vertices (resp., in which the maximum number of state vertices covered by a cactus configuration is exactly $k$).
			%			\end{theorem}
			
			As mentioned in Section \ref{intro-sec}, the NP-hardness results of the actuator and sensor placement problems addressed in \cite{pequito2017trade} and \cite{dey2021complexity}, as well as the approximation algorithms therein, heavily rely on Lemma \ref{known-result}.  Unfortunately, the following counter-example indicates that Lemma \ref{known-result} may fail to capture the exact SCOI.
			
			
				\begin{figure}
				\centering
				% Requires \usepackage{graphicx}
				\includegraphics[width=2.8in]{example.eps}\\
				\caption{A counter-example for Lemma \ref{known-result}.} \label{counter-example-fig}
			\end{figure}
			
			\begin{example}[Counter-example for Lemma \ref{known-result}] \label{counter-example}
			Consider a pair $(A,B)$ whose ${\mathcal G}(A,B)$ is given in Fig. \ref{counter-example-fig}(a). There are in total two different spanning cactus families in ${\mathcal G}(A,B)$, given respectively in Fig. \ref{counter-example-fig}(b) and \ref{counter-example-fig}(c). In both cases, the largest cactus contains $6$ state nodes. Hence, Lemma \ref{known-result} indicates $\mu(A,B)=6$. Next, consider the dynamic graph ${\mathcal D}_5(A,B)$, given by Fig. \ref{counter-example-fig}(d). From it, there is a linking of size $10$ from $V_B$ to $V_X$, highlighted in bold, which is the unique linking of size $10$ from $V_B$ to $V_X$ in ${\mathcal D}_5(A,B)$. By Lemma \ref{linking-lemma}, ${\rm gk}\,{\mathcal C}_5(A,B)=10$. Moreover, since ${\mathcal C}_4(A,B)$ has only $8$ columns, apparently ${\rm gk}\,{\mathcal C}_4(A,B)<10$. As a result, $\mu(A,B)=5$, which is smaller than the one obtained via Lemma \ref{known-result}. Next, consider a pair $(A',B)$, which is obtained from $(A,B)$ by adding nonzero entries to all zero diagonal entries of $A$ (leading to that every state node of ${\mathcal G}(A',B)$ has a self-loop). It turns out that ${\mathcal G}(A',B)$ can be spanned by two different forests rooted at $\{u_1,u_2\}$, which are similar to Fig. \ref{counter-example-fig}(b) and Fig. \ref{counter-example-fig}(c) except that no self-loop is included. Since the maximum directed tree in each forest contains $6$ state nodes, Lemma \ref{known-result} predicts ${\mu}(A',B)=6$. However, it is obvious that ${\rm gk}\,{\mathcal C}_5(A',B)\ge {\rm gk}\,{\mathcal C}_5(A,B)=10$, and meanwhile, ${\rm gk}\,{\mathcal C}_4(A',B)<10$. As a result, ${\mu}(A',B)=5$, again smaller than the one predicted by Lemma \ref{known-result}. % {\hfill $\square$\par}
			\end{example}

We now propose a new graph-theoretic characterization for SCOI applicable to systems that are not necessarily structurally controllable. As a by-product, we reveal that Lemma \ref{known-result} actually provides upper bounds for SCOI.
			
			\begin{definition}
				A cactus structure of ${\mathcal G}(A,B)$ is defined recursively as follows: a stem is a cactus structure; a cactus structure connected by a {\emph{path}} (without sharing edges with the former cactus structure) to a disjoint cycle is also a cactus structure. The set of state nodes {\emph{essentially covered}} by a cactus structure is the set of state nodes in the disjoint stem and cycles of this cactus structure.  A disjoint union of cactus structures is called a cactus structure family. A cactus structure family of ${\mathcal G}(A,B)$ is maximum if the sum of numbers of state nodes essentially covered by its cactus structures is maximum (this maximum number is denoted by  $|{\mathcal G}(A,B)|_{es}$) over all possible cactus structure families. %, this number is denoted by  $|{\mathcal G}(A,B)|_{es}$ as the number of state nodes essentially covered by the cactus structures in a maximum cactus structure family.
			\end{definition}

From the above definition, a cactus is a cacuts structure, while the reverse is not true. Every state node in a cactus is {\emph{essentially covered}} by the cactus structure corresponding it. By contrast, state nodes that are not in the disjoint cycles or stem of a cactus structure are not essentially covered by this cactus structure (cf. Fig. \ref{uncontrollable-example}(b)). %Notice that $|{\mathcal G}(A,B)|_{es}$ equals the maximum number of input-reachable state nodes that are covered by disjoint stems and cycles. The computation of the latter number can be reduced to a weighted maximum matching problem \cite[Theo 6]{murota1990note}, hence polynomially solvable. % Hence, $|{\mathcal G}(A,B)|_{es}$ can be determined in polynomial time.
			
%			\begin{definition}
%				Given $(A,B)$ (not necessarily structurally controllable), a disjoint union of cactus structures is maximum, if the number of
%			\end{definition}

					\begin{theorem} \label{main-theo}
					Given $(A,B)$ (not necessarily structurally controllable), if there is a maximum cactus structure family of ${\mathcal G}(A,B)$ such that each of its cactus structures essentially covers at most $k$ state nodes, then $\mu(A,B)\le k$. Moreover, for a single-input system $(A,B)$, where $B$ is of size $n\times 1$, $\mu(A,B)=|{\mathcal G}(A,B)|_{es}$. 					
					\end{theorem}


The proof relies on the following characterization for the generic dimension of the controllable subspace.
\begin{lemma}[\cite{poljak1990generic}] \label{generic-dimension}
	Given a pair $(A,B)$, the generic dimension of its controllable subspace equals the state nodes essentially covered by a maximum cactus structure family of ${\mathcal G}(A,B)$, and also equals the maximum size of a linking from $V_B$ to $V_X$ in ${\mathcal D}_n(A,B)$, i.e.,
	$${\rm gk}\,{\mathcal C}_n(A,B)=|{\mathcal G}(A,B)|_{es}=d({\mathcal D}_n(A,B)).$$
\end{lemma}
%, and also equals the maximum size of a linking from $V_B$ to $V_X$ in ${\mathcal D}_n(A,B)$

%\begin{lemma}[Theo 4, \cite{poljak1992gap}, \citeyear{poljak1992gap}] \label{single-input-dimension}
%For a single-input system $(A,B)$, where $B$ is of size $n\times 1$, ${\rm gk}\,{\mathcal C}_k(A,B)$ equals the maximum size of a linking from $V_B$ to $V_X$ in ${\mathcal D}_k(A,B)$ for every $k\ge 1$.
%\end{lemma}

{\bf Proof of Theorem \ref{main-theo}:}
Suppose ${\mathcal G}(A,B)$ contains a maximum cactus structure family ${\mathcal S}=\{{\mathcal S}_i\}_{i=1}^p$ that essentially covers $h\,(\doteq|{\mathcal G}(A,B)|_{es})$ state nodes,
where every
cactus structure ${\mathcal S}_i$ essentially covers at most $k$ state nodes. Then, we can remove the edges of ${\mathcal G}(A, B)$ that do not belong to any of the cactus structures in ${\mathcal S}$ (equivalently, set to zero the free parameters corresponding to these edges). Denote the resulting system by $(A',B')$. Therefore, we obtain $p$ disjoint single-input systems $(A'_i, B'_i)$, where $M'_i$ is the submatrix
of $M'$ with the columns and rows associated with the nodes in ${\mathcal S}_i$, $M=A$ or $B$. It is apparent that ${\mathcal S}_i$ is a maximum cactus structure in ${\mathcal G}(A'_i, B'_i)$, as otherwise ${\mathcal S}=\{{\mathcal S}_i\}_{i=1}^p$ cannot be a maximum cactus structure family in  ${\mathcal G}(A,B)$. Let $h_i$ and $n_i$ be respectively the number of state nodes essentially covered by ${\mathcal S}_i$ and the number of total state nodes in ${\mathcal S}_i$, which implies $h_i\le k, \forall i\in [p]$ and $\sum_{i=1}^ph_i=h$. Then, Lemma \ref{generic-dimension} yields that the generic dimension of the controllable subspace of $(A'_i, B'_i)$, i.e., ${\rm gk}\,{\mathcal C}_{n_i}(A'_i,B'_i)$, equals $h_i$. If ${\rm gk}\,{\mathcal C}_{h_i}(A'_i, B'_i)<h_i$, since $B'_i$ is of size $n_i\times 1$, there must exists some $\underline{h_i}\in [h_i-1]$ such that ${\rm gk}\,{\mathcal C}_{\underline{h_i}}(A'_i, B'_i)={\rm gk}\,{\mathcal C}_{\underline{h_i}+1}(A'_i, B'_i)<h_i$. This further leads to ${\rm gk}\,{\mathcal C}_{\underline{h_i}+l}(A'_i, B'_i)={\rm gk}\,{\mathcal C}_{\underline{h_i}}(A'_i, B'_i)<h_i$ for any $l\ge 0$, causing a contradiction. Therefore, it holds  ${\rm gk}\,{\mathcal C}_{h_i}(A'_i, B'_i)=h_i$, $\forall i\in [p]$. Since $h_i\le k$, $\forall i\in [p]$, we have ${\rm gk}\,{\mathcal C}_{k}(A',B')=\sum_{i=1}^p {\rm gk}\,{\mathcal C}_{k}(A'_i,B'_i)=\sum_{i=1}^ph_i=h$. From Lemma \ref{generic-dimension}, ${\rm gk}\,{\mathcal C}_{k}(A,B)\ge {\rm gk}\,{\mathcal C}_{k}(A',B')=h={\rm gk}\,{\mathcal C}_{n}(A,B)$. By definition, $\mu(A,B)\le k$.  When $B$ is of size $n\times 1$, the above analysis yields ${\rm gk}\, {\mathcal C}_{h}(A, B)=h={\rm gk}\, {\mathcal C}_{n}(A, B)$. Since ${\rm gk}\, {\mathcal C}_{h-1}(A, B)<h$, Definition~\ref{def-structural-index} yields $\mu(A,B)=h$.
{\hfill $\square$\par}
		%{\hfill $\square$\par}		
%			\begin{theorem}
%				A pair $(A,B)$ has a structural controllability index $k$, if and only if among all maximum disjoint unions of cactus configurations of ${\mathcal G}(A,B)$, the minimum value of the maximum number of state vertices covered by a cactus configuration is exactly $k$.
%				
%				Let $\{{\mathcal H}_i\}_{i\in {\mathcal K}}$ be the collection of all maximum disjoint unions of cactus configurations of ${\mathcal G}(A,B)$, where ${
%					\mathcal K}$ contains the indices of members in this collection. Let ${\mathcal H}_i=\{{\mathcal C}_1,...,{\mathcal C}_{l_i}\}$ consist of $l_i$ cactus configurations, with ${\mathcal C}_i$ being the $i$th one. Denote $|{\mathcal C}_i|_s$ as the number of state vertices covered by the cactus configuration ${\mathcal C}_i$.  Then,
%				$$\mu(A,B)=\min_{i\in {\mathcal K}} \max_{{\mathcal C}_j\in {\mathcal H}_i} |{\mathcal C}_j|_s $$
%			\end{theorem}
			\begin{example}\label{example-2} Consider a pair $(A,B)$ with ${\mathcal G}(A,B)$ given in Fig. \ref{uncontrollable-example}(a). Since there is no spanning cactus family in ${\mathcal G}(A,B)$, $(A,B)$ is structurally uncontrollable. Figs. \ref{uncontrollable-example}(b) and (c) provide two different maximum cactus structure families in ${\mathcal G}(A,B)$. In Fig. \ref{uncontrollable-example}(b), each of the cactus structures essentially covers $4$ state nodes, while in (c), one of the cactus structure essentially covers $3$ state nodes and the other $5$. This yields $\mu(A,B)\le 4$. With the fact ${\rm gk}\,{\mathcal C}_3(A,B)\le 6$ (as ${\mathcal C}_3(A,B)$ has $6$ columns), we reach $\mu(A,B)=4$.
\end{example}
			
			Notice that if $(A,B)$ is structurally controllable, a maximum cactus structure family reduces to a spanning cactus family. With Theorem \ref{main-theo}, we can correct Lemma \ref{known-result} as follows. %Put differently, Lemma \ref{known-result} actually provides an upper bound for SCOI.
			\begin{corollary}[Revised Lemma \ref{known-result}]
				A structurally controllable pair $(A,B)$ has an SCOI upper bounded by $k$, if ${\mathcal G}(A,B)$ contains a spanning cactus family, in which every cactus contains at most $k$ state nodes.
			\end{corollary}


				\begin{figure}
				\centering
				% Requires \usepackage{graphicx}
				\includegraphics[width=2.8in]{example2.eps}\\
				\caption{The system in Example \ref{example-2}. The filled node in (b) is not {\emph{essentially covered}} by the respective cactus structure.} \label{uncontrollable-example}
			\end{figure}
%[Structurally uncontrollable systems]




%\begin{remark} Theorem \ref{main-theo} indicates that any maximum cactus structure family of ${\mathcal G}(A,B)$ can provide an upper bound for $\mu(A,B)$.
% Notice that finding a collection of disjoint cycles and stems that contain the maximum number of state nodes in ${\mathcal G}(A,B)$ (called a maximum cactus configuration) can be reduced to a weighted bipartite graph problem \cite[Theo 6]{murota1990note}. Starting from a maximum cactus configuration, we can connect each cycle therein to a unique stem via some path in ${\mathcal G}(A,B)$, forming a collection of disjoint cactus structures so that the maximum number of state nodes essentially covered by a cactus structure is as small as possible.  This can be done, perhaps via certain heuristic graph-partitioning methods \cite{battiti1999greedy}. For various maximum cactus configurations of ${\mathcal G}(A,B)$, we may get different upper bounds for ${\mathcal G}(A,B)$, among which the minimum one can be taken as an upper bound of $\mu(A,B)$.



\begin{remark} Theorem \ref{main-theo} indicates that any maximum cactus structure family of ${\mathcal G}(A,B)$ provides an upper bound for $\mu(A,B)$. Identifying a collection of disjoint cycles and stems containing the maximum number of state nodes in ${\mathcal G}(A,B)$ (referred to as a maximum cycle-stem structure) can be reduced to a weighted bipartite graph problem \cite[Theo 6]{murota1990note}. Hence, $|{\mathcal G}(A,B)|_{es}$ can be determined in polynomial time. Starting from a maximum cycle-stem structure, each cycle can be connected to a unique stem via a path in ${\mathcal G}(A,B)$, forming a collection of disjoint cactus structures to balance the number of state nodes essentially covered by each cactus structure. This can potentially be achieved through heuristic graph-partitioning methods \cite{battiti1999greedy}. In this way, for different maximum cycle-stem structures of ${\mathcal G}(A,B)$, varying upper bounds for $\mu(A,B)$ may be obtained, with the minimum among them serving as the tightest upper bound.
  % naturally induces a method to compute an upper bound for $\mu(A,B)$. That is,
 % construct a maximum cactus structure family by
\end{remark}


%\begin{remark} \label{second-wrong-result}
The authors of \cite{sueur1997controllability}  proposed another method to obtain $\mu(A,B)$. Their basic idea is to calculate ${\rm gk}\,{\mathcal C}_k(A,B)$ for every $k\in [n]$. For each fixed $k\in [n]$, they construct a digraph ${\mathcal G}_k(A,B)$ from ${\mathcal G}(A,B)$, consisting of all paths starting from $U$ whose lengths are at most $k$, resulting in ${\mathcal G}_k(A,B)$ being the union of these paths without multiple edges between any two vertices. They claim that ${\rm gk}\,{\mathcal C}_k(A,B)$ equals $|{\mathcal G}_k(A,B)|_{es}$. However, this assertion is not necessarily true.  A counter-example is again the system $(A,B)$ associated with Fig. \ref{counter-example-fig}(a). It can be verified that ${\mathcal G}_4(A,B)$ coincides with ${\mathcal G}(A,B)$ in Fig. \ref{counter-example-fig}(a). %Since $(A,B)$ is structurally controllable, $|{\mathcal G}(A,B)|_{es}=10$. 
This indicates $\mu(A,B)\le 4$, different from $\mu(A,B)=5$ in Example \ref{counter-example}. As such, it seems safe to conclude that complete graph-theoretic characterizations and polynomial-time computation of the SCOI are still open. %Indeed, \cite{poljak1992gap} has shown that computing ${\mathcal G}_k(A,B)$ for a given $k$ with $2<k<n$ is generically hard, except in some special cases.
%\end{remark}





%The above results indicate that whether $\mu(A,B)$ can be determined in polynomial time by deterministic algorithms is still open. %Therefore, the NP-completeness of determining the minimum number of dedicated inputs to guarantee an upper bound on the SCOI, claimed in \cite{dey2021complexity}, is questionable, since given an instance of $B$ and a bound $k$, there is no known algorithm that can determine whether $\mu(A,B)\le k$ in polynomial time.
%			(by the duality between controllability and observability)
%			\begin{corollary}
%				A structurally controllable pair $(A,B)$ has a structural controllability index at most $k$, if and only if ${\mathcal G}(A,B)$ is spanned by a disjoint union of cacti,  in which every cactus contains at most $k$ state vertices.
%			\end{corollary}
			

			
			
			\section{Tight lower bound}
		%	Examples demonstrating the failure of existing algorithms.
			
		

In this section, we provide a tight lower bound for $\mu(A,B)$. Here, `tight' means this bound is exact in most cases, which shall be explained later.


From Lemma \ref{linking-lemma}, if $\mu(A,B)=h$, then $d({\mathcal D}_h(A,B))\ge {\rm gk}\, {\mathcal C}_n(A,B)$. By Lemma \ref{generic-dimension}, the following number $\mu_{\rm {low}}$ is a lower bound of $\mu(A,B)$:
\begin{equation} \label{lower-bound}
\mu_{\rm {low}}\doteq \min\{k\in [n]: d({\mathcal D}_k(A,B))=d({\mathcal D}_n(A,B))\}.
\end{equation}

%A simple idea to obtain $\mu_{\rm {low}}$ is computing $d({\mathcal D}_{k}(A,B))$ for $k=1,2,...,$ until some $h$ such that the condition $d({\mathcal D}_{h}(A,B))=d({\mathcal D}_{n}(A,B))$ first holds. Then, $\mu_{\rm {low}}=h$. However, in the worst case, this approach needs to compute $d({\mathcal D}_{k}(A,B))$ for $n$ different $k$ (a bisection search on $n$ could reduce this number to $\log_2 n$). Below, we show that $\mu_{\rm {low}}$ can be obtained via solving a single minimum cost maximum flow problem.

A straightforward approach to obtain \(\mu_{\rm {low}}\) involves computing \(d({\mathcal D}_{k}(A,B))\) for \(k = 1, 2, \dots\) until we find the first \(h\) such that \(d({\mathcal D}_{h}(A,B)) = d({\mathcal D}_{n}(A,B))\). At this point, \(\mu_{\rm {low}} = h\). However, in the worst-case scenario, this method requires calculating \(d({\mathcal D}_{k}(A,B))\) for \(n\) different values of \(k\), though a bisection search could reduce this number to \(\log_2 n\). In the following, we demonstrate that \(\mu_{\rm {low}}\) can be determined by solving a single minimum cost maximum flow problem.


%(a bisection method can reduce this number to $\log_2n$)

Given a directed graph (flow network) ${\mathcal G} = (V, E)$ with a source node $s\in V$ and a sink node $t\in V$, each edge $e = (u, v)$ is assigned a capacity $c(u, v)$ (or $c(e)$) and a cost $w(u, v)$ (or $w(e)$). A flow $f: E \to {\mathbb R}_{\ge 0}$ over ${\mathcal G}$ is a function that assigns a non-negative value $f(e)$ to each edge $e \in E$, such that the flow $f(e)$ on each edge $e$ does not exceed the edge capacity $c(e)$, i.e., $f(e)\le c(e)$. Additionally, the sum of flows into any node equals the sum of flows out of that node, except at the source $s$ and the sink $t$, i.e., for every $v\in V\backslash \{s,t\}$, $\sum_{(u,v)\in E}f(u,v)=\sum_{(v,w)\in E}f(v,w)$. The value of a flow on ${\mathcal G}$ is the total flow from the source to the sink, and the cost of a flow $f$ is defined as $w(f) = \sum_{e \in E} f(e)w(e)$. An integral flow is one where $f(e)$ is an integer for every $e \in E$.  A maximum flow of the network ${\mathcal G}$ is the highest possible flow on ${\mathcal G}$. A minimum cost maximum flow (MCMF) is a maximum flow that incurs the lowest possible cost.	According to the integral flow theorem \cite{Ahuja1993NetworkFT}, if each edge has integral capacity, then there exists an MCMF that is integral. We say an edge $e$ is {\emph{occupied}} in a flow $f$ if $f(e)>0$.

	
For the dynamic graph ${\mathcal D}_n(A,B)$, construct a flow network ${\mathcal F}(A,B)$ as follows.
Duplicate each vertex $v$ of ${\mathcal D}_n(A,B)$ with two vertices $v^i,v^o$ and an edge $(v^i,v^o)$ from $v^i$ to $v^o$. For each edge $(v,w)$ of ${\mathcal D}_n(A,B)$, replace it with an edge $(v^o,w^i)$, and add to the resultant graph a source $s$, a sink $t$, and the incident edges $\{(s,u_j^{ki}): j=1,...,m, k=1,...,n\}\cup \{(x_{j}^{1o},t):j=1,...,n\}$, which generates the flow network ${\mathcal F}(A,B)$. The edge capacity is set as $c(e)=1$ for each edge $e$ of ${\mathcal F}(A,B)$. %The edge cost is set as $w(e)=n^{k-1}$ if $e=(u_j^{ki},u_j^{ko})$ for all $j\in [m], k\in [n]$, and $w(e)=0$ otherwise.
The edge cost is set as
\begin{equation}\label{cost-func}w(e) =
\begin{cases}
	k, & \text{if } e=(u_j^{ki},u_j^{ko}), \forall j\in [m], k\in [n], \\
	0, & \text{else.}
\end{cases}\end{equation}
Here, the superscript $k$ in $u_j^{ki}$ and $u_j^{ko}$ is called the layer index.
Given a flow $f$ on ${\mathcal F}(A,B)$, let
 $$\gamma(f)= \max \left\{k\in [n]:f(e)>0, e=(u_j^{ki},u_j^{ko}), j\in [m]\right\},$$i.e., $\gamma(f)$ takes the highest layer index $k$ corresponding to an occupied edge $(u_j^{ki},u_j^{ko})$ in $f$. As shown below, the cost function (\ref{cost-func}) penalizes edges $(u_j^{ki},u_j^{ko})$ w.r.t. $k$, so that in an integral MCMF $f^*$ on ${\mathcal F}(A,B)$, $\gamma(f^*)$ is minimized over all integral maximum flows.% of ${\mathcal F}(A,B)$.
%As shown subsequently, the cost function (\ref{cost-func}) penalizes edges $(u_j^{ki},u_j^{ko})$ with respect to $k$, so that in an integral MCMF $f$ on ${\mathcal F}(A,B)$, $\gamma(f)$ is minimized over all possible integral maximum flows of ${\mathcal F}(A,B)$.
%As shown subsequently, the above function penalizes edges $(u_j^{ki},u_j^{ko})$, so that in an MCMF on ${\mathcal F}(A,B)$, the largest $k$ (corresponding to $u_j^k$ in the $k$th layer of ${\mathcal D}_n(A,B)$), for which there is an edge $(u_j^{ki},u_j^{ko})$ with a nonzero flow, is minimized. % as small as possible.
%The MCMF problem associated with the flow network ${\mathcal F}(A,B)$ is denoted by MCMF$({\mathcal F}(A,B))$.
%Put differently, edges $(u_j^{ki},u_j^{ko})$ corresponding to the input edges in the $k$th layer of ${\mathcal D}_n(A,B)$ have cost $2^{k-1}$, and

% in which input edges in the $i$th layer ($i=1,...,n$) are assigned with weight $2^{i-1}$. Then, a tight lower bound for $\mu(A,B)$ is the largest layer index of an optimal solution to the aforementioned minimum cost maximum flow problem.



\begin{theorem} \label{tight-lower-theorem}
	Let $f^*$ be an integral MCMF of ${\mathcal F}(A,B)$. Then, $\mu_{\rm low}=\gamma(f^*)$. Moreover, $\mu_{\rm low}$ can be computed in $\tilde O(n^5)$ time.\footnote{The notation $\tilde O(\cdot)$ denotes a function where the logarithmic factors in $O(\cdot)$ are hidden, i.e., $\tilde O(n^c)=O(n^c\log^p n)$ ($c\ge 0$) for some constant $p$.}
	% $$\mu_{\rm low}= \max \left\{k\in [n]:f^*(e)>0, e=(u_j^{ki},u_j^{ko}), j\in [m]\right\}.$$
\end{theorem}

The proof relies on the {\emph{gammoid}} structure of linkings.

\begin{definition}(\cite[Chap 2-3-2]{Murota_Book})\label{def_matroid}
A matroid is a pair $(V,  {\mathcal I})$ of a finite set $V$ and a collection ${\mathcal I}$ of
subsets of $V$ such that (a1) $\emptyset \in {\mathcal I}$, (a2) $I\subseteq J$ and $J\in {\mathcal I}$ implies $I\in {\mathcal I}$, and (a3) if $I,J\in {\mathcal I}$, $|I|<|J|$, then there exists some $v\in J\backslash I$ such that $I\cup \{v\}\in {\mathcal I}$. The set $V$ is called the ground set and $I\in {\mathcal I}$ an independent set.
\end{definition}

\begin{lemma}(\cite[Example 2.3.6]{Murota_Book}) \label{gammoid}
Let ${\mathcal G}=(V,E;S,T)$ be a digraph with vertex set $V$ and edge set $E$, where $S\subseteq V$ and $T\subseteq V$ are two disjoint vertex sets. Let a collection  ${\mathcal I}$ of subsets of $S$ be such that a subset $I\subseteq S$ belongs to ${\mathcal I}$ (i.e., $I$ is an independent set) if there is a set of vertex-disjoint paths whose start vertices are exactly $I$ and whose end vertices all belong to $T$. Then, $(S,{\mathcal I})$ defines a matroid on $S$, which is called a {\emph{gammoid}}.
\end{lemma}

{\bf Proof of Theorem \ref{tight-lower-theorem}}: 	By the construction of ${\mathcal F}(A,B)$, the value of the flow $f^*$ equals the maximum size of a linking from $V_B$ to $V_X$ in ${\mathcal D}_n(A,B)$, i.e., $d({\mathcal D}_n(A,B))$. Moreover, there is a one-one correspondence between an integral maximum flow on ${\mathcal F}(A,B)$ (or the set of occupied edges in this flow) and a maximum linking of ${\mathcal D}_n(A,B)$.  Suppose $f'$ is an integral maximum flow such that $\gamma(f')>\gamma(f^*)$. For the sake of contradiction, assume that $w(f')\le w(f^*)$. Consider the vertex sets $V_B^{i}\doteq \{u_j^{ki}:j\in [m],k\in [n]\}$ and $V_X^o\doteq \{x_j^{1o}:j\in [n]\}$, which are copies of $V_B$ and $V_X$, respectively. For an integral maximum flow $f$ on ${\mathcal F}(A,B)$, define $V_B^{i}(f)\doteq \{u_j^{ki}: f(e)>0, e=(u_j^{ki},u_j^{ko}), j\in [m], k\in [n]\}$, i.e., $V_B^i(f)$ is the set of $u_j^{ki}\in V_B^i$ such that $(u_j^{ki},u_j^{ko})$ is occupied in $f$. Let $V_{f'}^1= \{u_j^{ki}: k>\gamma(f^*),j\in [m]\}$ and $V_{f'}^2=V_B^i(f')\backslash V_{f'}^1$. From Lemma \ref{gammoid}, upon letting ${\mathcal I}$ be the collection of subsets $I$ of $V_B^i$ such that there is a set of vertex-disjoint paths whose start vertices are exactly $I$ and whose end vertices all belong to $V_X^o$, $(V_B^i, {\mathcal I})$ forms a gammoid (thus a matroid). By definition, $V_{f'}^2, V_B^i(f^*)\in {\mathcal I}$, and $|V_{f'}^2|<|V_B^i(f^*)|=|V_B^i(f')|$. By repeatedly using property (a3) in Definition \ref{def_matroid} on the matroid $(V_B^i, {\mathcal I})$, we reach that there is a subset $V_{f^*}\subseteq V_B^i(f^*)\backslash V_{f'}^2$, such that $V_{f^*}\cup V_{f'}^2\in {\mathcal I}$ and $|V_{f^*}\cup V_{f'}^2|=|V_B^i(f^*)|$, that is, there is a set of $|V_B^i(f^*)|$ vertex-disjoint paths whose start vertices are exactly $V_{f^*}\cup V_{f'}^2\in {\mathcal I}$ and end vertices belong to $V_X^o$. Denote by $\tilde f$ the integral maximum flow on ${\mathcal F}(A,B)$ in which edges of these paths are occupied. Notice that each vertex in $V_{f^*}$ has a smaller layer index than that of any vertex in $V_{f'}^1$ and $|V_{f^*}|=|V_{f'}^1|$. From the cost function (\ref{cost-func}), for a subset $V_s\subseteq V_B^i$ upon letting $w(V_s)=\sum_{u_j^{ki}\in V_s} k$, we have $w(V_{f^*})< w(V_{f'}^1)$.
The relation $w(f')=w(V_{f'}^1)+w(V_{f'}^2)\le w(f^*)$ then yields
$w(\tilde f)= w(V_{f^*})+w(V_{f'}^2)<w(V_{f'}^1)+w(V_{f'}^2)=w(f')\le  w(f^*),$
contradicting the assumption that $f^*$ is an MCMF. Therefore, $f^*$ must achieve the minimum $\gamma(f)$ among all integral maximum flows $f$ on ${\mathcal F}(A,B)$, thus equaling $\mu_{\rm low}$.

By \cite{lee2014path}, finding an MCMF on a flow network ${\mathcal G}=(V,E)$ can be done in time $\tilde O(|V||E|\log(U))$, where $U$ denotes the maximum absolute value of capacities and costs. Notice that ${\mathcal F}(A,B)$ has $2n(n+m)+2 \to O(n^2)$ vertices and $O(n|E_{XX}\cup E_{UX}|)$ edges. The complexity of obtaining $f^*$ (and $\mu_{\rm low}$) is therefore $\tilde O(n^3|E_{XX}\cup E_{UX}|\log n)$, at most $\tilde O(n^5)$.   {\hfill $\square$\par}

%  As $w(f')\le w(f)$ and $\gamma(f')>\gamma(f^*)$, from the cost function in  (\ref{cost-func}),

%Then, the cost of $f'$ satisfies $w(f')\ge n^{\gamma(f')-1}> d({\mathcal D}_n(A,B)) n^{\gamma(f^*)-1}\ge w(f^*)$, where $d({\mathcal D}_n(A,B))\le n$ has been used.   Hence, $f^*$ achieves the minimum $\gamma(f)$ among all integral maximum flows $f$ on ${\mathcal F}(A,B)$, which equals $\mu_{\rm low}$.
%\begin{remark}
%Notice that the input length (i.e., the number of bits needed to represent the number) of the cost $\{w(e)\}_{e\in E}$, is $O(\log_2n^{n-1})\to \tilde O(n)$, polynomially bounded by $n$. Moreover, due to the logarithmic dependence on $w(e)$, computing $\mu_{\rm low}$ based on the MCMF incurs polynomial time in $n$. It is worth mentioning that, a bisection method on $n$, which computes $O(\log_2 n)$ times of $d({\mathcal D}_k(A,B))$ ($k\in [n]$), may achieve a better time complexity $\tilde O(n^5)$ than the MCMF based method. Nevertheless, the theoretical significance of Theorem \ref{tight-lower} lies in the reduction of computing $\mu_{\rm low}$ to a {\emph{single}} MCMF problem, without solving multiple maximum flow problems. We anticipate that, by delving into the structure of the dynamic graphs,  the exponential (w.r.t. $k$) cost function in (\ref{cost-func})  can be replaced with a polynomial cost function, which may further reduce the time complicity of the MCMF based method.
%\end{remark}

%\begin{remark}
%It is worth mentioning that, a bisection method on $n$, which computes $O(\log_2 n)$ times of $d({\mathcal D}_k(A,B))$ ($k\in [n]$), may achieve a similar time complexity $\tilde O(n^5)$ to the MCMF based method. Nevertheless, the theoretical significance of Theorem \ref{tight-lower} lies in the reduction of computing $\mu_{\rm low}$ to a {\emph{single}} MCMF problem, without solving multiple maximum flow problems. %We anticipate that, by delving into the structure of the dynamic graphs,  the exponential (w.r.t. $k$) cost function in (\ref{cost-func})  can be replaced with a polynomial cost function, which may further reduce the time complicity of the MCMF based method.
%\end{remark}
%The bisection method has a time complexity of $\tilde O(n^5)$
%{\bf Computational complexity}: Moreover, for a wide class of systems, for
%example, when where c1 is a positive constant,
%independent of n, and similarly for a, this algorithm runs in
%polynomial time, due to the logarithmic dependence on E and
%a, respectively.

%By \cite{lee2014path}, finding an MCMF on a flow network $G=(V,E)$ can be done in time $\tilde O(|V||E|\log(U))$, where $U$ denotes the maximum absolute value of capacities and costs. The computational complexity of obtaining the lower bound is $\tilde O(n^2n|E_{XX}\cup E_{UX}|\log n^n)$, at most $\tilde O(n^6)$. The bisection method has a time complexity of $\tilde O(n^5)$.


				\begin{figure}
				\centering
				% Requires \usepackage{graphicx}
				\includegraphics[width=3.3in]{controllability_index2.eps}\\
				\caption{Tightness of the lower bound via simulations} \label{simulation-result}
			\end{figure}

%\begin{remark}[Tightness of $\mu_{\rm low}$] From (\ref{linking-lemma}), it is evident that $\mu_{\rm low}$ may fail to coincide with $\mu(A,B)$, only when for every maximum linking $L$ of ${\mathcal D}_{\mu_{\rm low}}(A,B)$, there is another linking $L'$ such that the collection of edges in $L'$ is the same as that of $L$, yielding $w(L) = w(L')$, meanwhile ${\rm sign}(L) = -{\rm sign}(L')$, making ${\rm sign}(L)w(L)$ and ${\rm sign}(L')w(L')$ cancel out in (\ref{det}). However, as shown in \cite{poljak1992gap}, such a situation is rare and requires a carefully designed
%instance. Therefore, for practical systems, Theorem \ref{tight-lower-theorem} is highly
%likely to provide the exact value of $\mu(A,B)$. In this sense, it seems safe to call $\mu_{\rm low}$ a tight lower bound of $\mu(A,B)$.
%\end{remark}

From (\ref{linking-lemma}), it is evident that $\mu_{\rm low}$ may differ from $\mu(A,B)$ only in cases where, for every maximum linking \(L\) of \({\mathcal D}_{\mu_{\rm low}}(A,B)\), there exists another linking \(L'\) with the same set of edges as \(L\), resulting in \(w(L) = w(L')\), but with opposite signs, i.e., \({\rm sign}(L) = -{\rm sign}(L')\). This causes \({\rm sign}(L)w(L)\) and \({\rm sign}(L')w(L')\) to cancel each other out in (\ref{det}). However, as demonstrated in \cite{poljak1992gap}, such occurrences are rare and typically require a carefully crafted instance. Thus, for practical systems, Theorem \ref{tight-lower-theorem} is highly likely to yield the exact value of \(\mu(A,B)\). In this context, it is reasonable to consider $\mu_{\rm low}$ a tight lower bound of \(\mu(A,B)\).

%We provide numerical results to validate the above assertion. To obtain the matrices $A$, we generate directed Erdos-Renyi (ER) random
%graphs for ${\mathcal G}(A)$, where each (directed) link (including self-loop) is present with a probability
%of $\log n/n$. We set $B$ so that each column of $B$ contains exactly one nonzero entry, and the corresponding rows of these nonzero entries are randomly selected without repetition. For each $(A,B)$, we use the MCMF based approach to obtain $\mu_{\rm low}$. The exact SCOI $\mu(A,B)$ is evaluated by assigning independent random values from $[0,1]$ to the nonzero entries of $(A,B)$ and then computing the controllability index of the resulting numerical systems, which is denoted by $\mu_{\rm num}$. Due to the genericity of SCOI (cf. Lemma \ref{lemma-genericity}), as demonstrated in \cite{sundaram2012structural}, this can provide the exact $\mu(A,B)$ with probability one.
%For each $n$ ranging from $[5,50]$, we generate $50$ ER random graphs. For each graph, we generate $4$ input matrices $B$ with $m=2,3,4,5$ columns in a manner described above.  We collect the averaged value of $\mu_{\rm low}$ and $\mu_{\rm num}$ for each $n$ at Fig. \ref{simulation-result}.
%
%As shown in Fig. \ref{simulation-result}, for almost every pair $(n,m)$, $\mu_{\rm num}$ and $\mu_{\rm low}$ coincide. This verifies the tightness of $\mu_{\rm low}$.
%It is notable that for $n\ge 25$ and $m=2$, some nonsensical values for $\mu_{\rm num}$ may appear when computing ${\rm rank}{\mathcal C}_k(\tilde A,\tilde B)$ for a numerical $(\tilde A,\tilde B)$, perhaps due to the ill-conditioning of the involved controllability matrices. Additionally, even for the same structured $(A,B)$, different random realizations may procedure dramatically different $\mu_{\rm low}$, due to the large condition number of the controllability matrices and the sensitivity of the Matlab rank function to rounding errors. This may cause the minor inconsistence between $\mu_{\rm low}$ and $\mu_{\rm num}$ when $n\ge 25$ and $m=2$. On the other hand, this also highlights the advantage of the graph-theoretic method in Theorem \ref{tight-lower-theorem}, which gets rid of rounding errors and computational instability often encountered in numerical computations.


We present numerical results to support the above claims. The matrices $A$ were generated using directed Erdos-Renyi (ER) random graphs for the graph ${\mathcal G}(A)$, with each (directed) link, including self-loops, having a probability of $\log n/n$ of being present. The matrix $B$ was constructed such that each column contains exactly one nonzero entry, with the corresponding rows randomly selected without repetition. For each pair $(A,B)$, we used the MCMF-based approach to calculate $\mu_{\rm low}$. To determine the exact SCOI $\mu(A,B)$, we assigned independent random values from $[0,1]$ to the nonzero entries of $(A,B)$ and computed the controllability index for the resulting numerical systems, denoted as $\mu_{\rm num}$. Due to the generic nature of SCOI (cf. Lemma \ref{lemma-genericity}) and as shown in \cite{sundaram2012structural}, this method provides the exact value of $\mu(A,B)$ with probability one.
We generated $50$ ER random graphs for each value of $n$ in the range [5,50]. For each graph, we created four input matrices $B$ with $m = 2, 3, 4, 5$ columns as described above. The average values of $\mu_{\rm low}$ and $\mu_{\rm num}$ for each $n$ are plotted in Fig. \ref{simulation-result}.

As illustrated in Fig. \ref{simulation-result}, $\mu_{\rm num}$ and $\mu_{\rm low}$ nearly coincide for almost all pairs $(n,m)$, confirming the tightness of $\mu_{\rm low}$. However, for $n \ge 25$ and $m = 2$, we found some anomalous values of $\mu_{\rm num}$ may arise when computing ${\rm rank}\,{\mathcal C}_k(\tilde A,\tilde B)$ for a numerical $(\tilde A,\tilde B)$, likely due to the ill-conditioning of the controllability matrices and the sensitivity of the Matlab rank function to rounding errors. This may explain the minor inconsistencies between $\mu_{\rm low}$ and $\mu_{\rm num}$ when $n \ge 25$ and $m = 2$. This also underscores the advantage of the graph-theoretic method in Theorem \ref{tight-lower-theorem}, which avoids the rounding errors and computational instability often encountered in numerical computations.

%As illustrated in Fig. \ref{simulation-result}, $\mu_{\rm num}$ and $\mu_{\rm low}$ nearly coincide for almost all pairs $(n,m)$, confirming the tightness of $\mu_{\rm low}$. However, for $n \ge 25$ and $m = 2$, some anomalous values of $\mu_{\rm num}$ may arise when computing ${\rm rank}{\mathcal C}_k(\tilde A,\tilde B)$ for a numerical $(\tilde A,\tilde B)$, likely due to the ill-conditioning of the controllability matrices. Additionally, even with the same structured $(A,B)$, different random realizations may yield significantly different $\mu_{\rm low}$ due to the high condition number of the controllability matrices and the sensitivity of the Matlab rank function to rounding errors. This may explain the minor inconsistencies between $\mu_{\rm low}$ and $\mu_{\rm num}$ when $n \ge 25$ and $m = 2$. This also underscores the advantage of the graph-theoretic method in Theorem \ref{tight-lower-theorem}, which avoids the rounding errors and computational instability often encountered in numerical computations.

\section{Conclusions}
In this note, we show that an existing graph-theoretic characterization for the SCOI may not always hold and provide only upper bounds. We further extend it to systems that are not necessarily structurally controllable and reveal the limitations of an existing method in obtaining the exact index. We also provide an efficiently computable tight lower bound  that can serve as a useful tool, based on the dynamic graph and gammoid structure. Our results reveal that complete graph-theoretic characterizations and polynomial-time computation of the SCOI are still open. All these results apply to the structural observability index by duality.% between controllability and observability.

%\begin{corollary}
%If ${\mathcal G}(A,B)$ is acyclic (i.e., not containing any cycles, including self-loops) or $B$ is of size $n\times 1$, then $\mu(A,B)=\mu_{\rm low}$.
%\end{corollary}
%
%\begin{proof}
%???
%When $(A,B)$ is single-input, \cite[Theo 4]{poljak1992gap} proves that ${\rm gk}\, {\mathcal C}_k(A,B)=d({\mathcal D}_k(A,B))$, for any $k\ge 1$. The result then follows.
%\end{proof}
			
			
%Discussions on upper bounds for $\mu(A,B)$:
%			
%			\begin{itemize}
%				\item Find a maximum cactus configuration of ${\mathcal G}(A,B)$;
%				\item Using greedy algorithms from graph partitioning to assign cycles to stems.
%			\end{itemize}
%			
%			Any returned maximum number of state vertices covered by a cactus configuration gives an upper bound for $\mu(A,B)$.
%			
%			{\bf Greedy graph-partitioning based upper bound:}
%			\begin{itemize}
%				\item Initialization:
%				\begin{itemize}
%					\item Start with each subgraph initialized to contain its known a priori vertices.
%					\item Create a list to keep track of the sizes of each subgraph.
%					\item Maintain an adjacency list for the graph to track the connections between vertices.
%				\end{itemize}
%				\item Assignment:
%				\begin{itemize}
%					\item For each unassigned vertex, use a priority queue to determine the subgraph with the smallest size.
%					\item Assign the vertex to this subgraph, including any connected vertices and edges.
%					\item Update the size of the subgraph accordingly.
%					\item Mark the assigned vertices and their neighbors as assigned.
%				\end{itemize}
%			\end{itemize}

%\section*{Appendix: Proof of Theorem \ref{main-theo}}


%{\bf Proof of Theorem \ref{main-theo}:}
%Suppose ${\mathcal G}(A,B)$ contains a maximum cactus structure family ${\mathcal S}=\{{\mathcal S}_i\}_{i=1}^p$ that essentially covers $h$ state nodes,
%where every
%cactus structure ${\mathcal S}_i$ essentially covers at most $k$ state nodes. Then, we can remove the edges of ${\mathcal G}(A, B)$ that do not belong to any of the cactus structures in ${\mathcal S}$ (equivalently, set to zero the free parameters corresponding to these edges). Denote the resulting system by $(A',B')$. Therefore, we obtain $p$ disjoint single-input systems $(A'_i, B'_i)$, where $M'_i$ is the submatrix
%of $M'$ with the columns and rows associated with the nodes in ${\mathcal S}_i$, $M=A$ or $B$. It is apparent that ${\mathcal S}_i$ is a maximum cactus structure in ${\mathcal G}(A'_i, B'_i)$, as otherwise ${\mathcal S}=\{{\mathcal S}_i\}_{i=1}^p$ cannot be a maximum cactus structure family in  ${\mathcal G}(A,B)$. Let $h_i$ and $n_i$ be respectively the number of state nodes essentially covered by ${\mathcal S}_i$ and the number of total state nodes in ${\mathcal S}_i$, which implies $h_i\le k, \forall i\in [p]$ and $\sum_{i=1}^ph_i=h$. Then, Lemma \ref{generic-dimension} yields that the generic dimension ${\rm gk}\,{\mathcal C}_{n_i}(A'_i,B'_i)$ of the controllable subspace of $(A'_i, B'_i)$, as well as the maximum size of a linking from $V_B$ to $V_A$ in ${\mathcal D}_{n_i}(A'_i,B'_i)$, equals $h_i$.
%Subsequently, invoking Lemma \ref{single-input-dimension}, it
%follows that ${\rm gk}\,{\mathcal C}_{h_i}(A'_i, B'_i)=h_i$. Since $h_i\le k$, $\forall i\in [p]$, we have ${\rm gk}\,{\mathcal C}_{k}(A',B')=\sum_{i=1}^p {\rm gk}\,{\mathcal C}_{k}(A'_i,B'_i)=h$. That is, ${\rm gk}\,{\mathcal C}_{k}(A,B)\ge ={\rm gk}\,{\mathcal C}_{k}(A',B')={\rm gk}\,{\mathcal C}_{n}(A,B)$. By definition, $\mu(A,B)\le k$. {\hfill $\square$\par}
			
			%\bibliographystyle{elsarticle-harv}
   \bibliographystyle{elsarticle-num}
			%\bibliographystyle{ieeetr}
			%\bibliographystyle{model1-num-names}
			{\small
				\bibliography{yuanz3}
			}
	}}
\end{document}
%{\small{
%			\begin{thebibliography}{27}
%			\providecommand{\natexlab}[1]{#1}
%			\providecommand{\url}[1]{\texttt{#1}}
%			\expandafter\ifx\csname urlstyle\endcsname\relax
%		\providecommand{\doi}[1]{doi: #1}\else
%			\providecommand{\doi}{doi: \begingroup \urlstyle{rm}\Url}\fi

%			\bibitem[Ahuja et~al.(1993)Ahuja, Magnanti, and Orlin]{Ahuja1993NetworkFT}
%			Ravindra~K. Ahuja, Thomas~L. Magnanti, and James~B. Orlin.
%			\newblock \emph{Network Flows: Theory, Algorithms, and Applications}.
%			\newblock Prentice Hall, Upper Saddle River, NJ, 1993.
%			
%			\bibitem[Belabbas and Kirkoryan(2022)]{belabbas2022stable}
%			Mohamed-Ali Belabbas and Artur Kirkoryan.
%			\newblock On stable systems with random structure.
%			\newblock \emph{SIAM Journal on Control and Optimization}, 60\penalty0
%			(\ref{plant}):\penalty0 458--478, 2022.

%				\bibitem[Chen(1984)]{chen1984linear}
%				Chi-Tsong Chen.
%				\newblock \emph{Linear System Theory and Design}.
%				\newblock Saunders College Publishing, 1984.
%				
%				\bibitem[Czeizler et~al.(2018)Czeizler, Wu, Gratie, Kanhaiya, and
%				Petre]{czeizler2018structural}
%				Eugen Czeizler, Kai-Chiu Wu, Cristian Gratie, Krishna Kanhaiya, and Ion Petre.
%				\newblock Structural target controllability of linear networks.
%				\newblock \emph{IEEE/ACM Transactions on Computational Biology and
%					Bioinformatics}, 15\penalty0 (4):\penalty0 1217--1228, 2018.
%				
%				\bibitem[Dion et~al.(2003)Dion, Commault, and Van~DerWoude]{generic}
%				J~M Dion, C~Commault, and J~Van~DerWoude.
%				\newblock Generic properties and control of linear structured systems: {a}
%				survey.
%				\newblock \emph{Automatica}, 39:\penalty0 1125--1144, 2003.
%				
%				\bibitem[Fernando et~al.(2010)Fernando, Trinh, and
%				Jennings]{fernando2010functional}
%Tyrone~Lucius Fernando, Hieu~Minh Trinh, and Les Jennings.
%\newblock Functional observability and the design of minimum order linear
%functional observers.
%\newblock \emph{IEEE Transactions on Automatic Control}, 55\penalty0
%(5):\penalty0 1268--1273, 2010.

%\bibitem[Gao et~al.(2014)Gao, Liu, D'souza, and Barab{\'a}si]{gao2014target}
%Jianxi Gao, Yang-Yu Liu, Raissa~M D'souza, and Albert-L{\'a}szl{\'o}
%Barab{\'a}si.
%\newblock Target control of complex networks.
%\newblock \emph{Nature communications}, 5\penalty0 (\ref{plant}):\penalty0 1--8, 2014.

%\bibitem[Hosoe(1980)]{hosoe1980determination}
%Shigeyuki Hosoe.
%\newblock Determination of generic dimensions of controllable subspaces and its
%application.
%			\newblock \emph{IEEE Transactions on Automatic Control}, 25\penalty0
%(6):\penalty0 1192--1196, 1980.

%\bibitem[Jennings et~al.(2011)Jennings, Fernando, and
%Trinh]{jennings2011existence}
%Les~S Jennings, Tyrone~Lucius Fernando, and Hieu~Minh Trinh.
%\newblock Existence conditions for functional observability from an eigenspace
%perspective.
%\newblock \emph{IEEE Transactions on Automatic Control}, 56\penalty0
%(12):\penalty0 2957--2961, 2011.

%\bibitem[Li et~al.(2019)Li, Chen, Pequito, Pappas, and
%Preciado]{li2019resilient}
%Jingqi Li, Ximing Chen, S{\'e}rgio Pequito, George~J Pappas, and Victor~M
%Preciado.
%\newblock Resilient structural stabilizability of undirected networks.
%\newblock In \emph{2019 American Control Conference (ACC)}, pages 5173--5178.
%IEEE, 2019.

%\bibitem[Li et~al.(2021)Li, Chen, Pequito, Pappas, and
%Preciado]{li2020structural}
%Jingqi Li, Ximing Chen, S{\'e}rgio Pequito, George~J Pappas, and Victor~M
%Preciado.
%\newblock On the structural target controllability of undirected networks.
%\newblock \emph{IEEE Transactions on Automatic Control}, 66\penalty0
%(10):\penalty0 4836--4843, 2021.

%\bibitem[Lin(1974)]{C.T.1974Structural}
%C~T Lin.
%\newblock Structural controllability.
%\newblock \emph{IEEE Transactions on Automatic Control}, 48\penalty0
%(3):\penalty0 201--208, 1974.

%\bibitem[Liu et~al.(2011)Liu, Slotine, and Barabasi]{Y.Y.2011Controllability}
%Y~Y Liu, J~J Slotine, and A~L Barabasi.
%\newblock Controllability of complex networks.
%\newblock \emph{Nature}, 48\penalty0 (7346):\penalty0 167--173, 2011.

%\bibitem[Menara et~al.(2019)Menara, Bassett, and
%Pasqualetti]{menara2018structural}
%Tommaso Menara, Danielle~S Bassett, and Fabio Pasqualetti.
%%%\newblock Structural controllability of symmetric networks.
%%\newblock \emph{IEEE Transactions on Automatic Control}, 64\penalty0
%%(9):\penalty0 3740--3747, 2019.

%\bibitem[Montanari et~al.(2022)Montanari, Duan, Aguirre, and
%Motter]{montanari2022functional}
%Arthur~N Montanari, Chao Duan, Luis~A Aguirre, and Adilson~E Motter.
%\newblock Functional observability and target state estimation in large-scale
%networks.
%\newblock \emph{Proceedings of the National Academy of Sciences (PNAS)},
%119\penalty0 (\ref{plant}):\penalty0 e2113750119, 2022.

%\bibitem[Mortazavian(1982)]{mortazavian1982k}
%H~Mortazavian.
%\newblock On k-controllability and k-observability of linear systems.
%\newblock In \emph{Analysis and Optimization of Systems: Proceedings of the
%		Fifth International Conference on Analysis and Optimization of Systems},
%pages 600--612. Springer, 1982.

%\bibitem[Mousavi et~al.(2017)Mousavi, Haeri, and
%Mesbahi]{mousavi2017structural}
%Shima~Sadat Mousavi, Mohammad Haeri, and Mehran %%%Mesbahi.
%\newblock On the structural and strong structural %controllability of undirected
%networks.
%\newblock \emph{IEEE Transactions on Automatic %%Control}, 63\penalty0
%(7):\penalty0 2234--2241, 2017.

%\bibitem[Murota(2009)]{Murota_Book}
%K~Murota.
%\newblock \emph{Matrices and Matroids for Systems %Analysis}.
%%\newblock Springer Science Business Media, 2009.

%\bibitem[Murota and Poljak(1990)]{murota1990note}
%Kazuo Murota and Svatopluk Poljak.
%\newblock Note on a graph-theoretic criterion for %structural output
%controllability.
%\newblock \emph{IEEE Transactions on Automatic %Control}, 35\penalty0
%(8):\penalty0 939--942, 1990.

%\bibitem[Pequito et~al.(2016)Pequito, Kar, and Aguiar]%{S.Pe2016A}
%S~Pequito, S~Kar, and A~P Aguiar.
%\newblock A framework for structural input/output and %control configuration
%selection in large-scale systems.
%	\newblock \emph{IEEE Transactions on Automatic %Control}, 48\penalty0
%(2):\penalty0 303--318, 2016.

%\bibitem[Poljak(1989)]{poljak1989maximum}
%Svatopluk Poljak.
%\newblock Maximum rank of powers of a matrix of a %given pattern.
%%%\newblock \emph{Proceedings of the American %Mathematical Society}, 106\penalty0
%%(4):\penalty0 1137--1144, 1989.
%	
%\bibitem[Poljak(1990)]{poljak1990generic}
%Svatopluk Poljak.
%\newblock On the generic dimension of controllable %subspaces.
%%\newblock \emph{IEEE Transactions on Automatic %Control}, 35\penalty0
%(3):\penalty0 367--369, 1990.

%\bibitem[Poljak(1992)]{poljak1992gap}
%Svatopluk Poljak.
%\newblock On the gap between the structural %controllability of time-varying and
%time-invariant systems.
%\newblock \emph{IEEE transactions on automatic control}, 37\penalty0
%(12):\penalty0 1961--1965, 1992.

%\bibitem[Ramos et~al.(2022)Ramos, Aguiar, and Pequito]{Ramos2022AnOO}
%Guilherme Ramos, Antonio~Pedro Aguiar, and %S{\'e}rgio~Daniel Pequito.
%%\newblock An overview of structural systems theory.
%\newblock \emph{Automatica}, 140:\penalty0 110229, %2022.

%\bibitem[Reinschke(1988)]{reinschke1988multivariable}
%Kurt~J Reinschke.
%\newblock \emph{Multivariable Control: A Graph-%theoretic Approach}.
%\newblock Springer, 1988.

%\bibitem[Zhang et~al.(2023)Zhang, Fernando, and %Darouach]{zhang2023functional}
%Yuan Zhang, Tyrone Fernando, and Mohanmed Darouach.
%\newblock Functional observability, structural %functional observability and
%optimal sensor placement.
%\newblock \emph{arXiv preprint arXiv:2307.08923}, %2023.

%\bibitem[Zhang et~al.(2024)Zhang, Fernando, and %Darouach]{zhang2023generic}
%Yuan Zhang, Tyrone Fernando, and Mohanmed Darouach.
%\newblock Generic diagonalizability, structural %functional observability and
%output controllability.
%\newblock \emph{under view}, 2024.
%
%\end{thebibliography}
%			
%		}}
%		\fi
%\appendix
%\section{A summary of Latin grammar}    % Each appendix must have a short title.
%\section{Some Latin vocabulary}         % Sections and subsections are supported
%                                        % in the appendices.

