	In this section, we evaluate our proposed Algorithm~\ref{alg:BDOMA} and Algorithm~\ref{alg:BDOEA} in simulated multi-target tracking tasks~\citep{corah2021scalable,xu2023online} with multiple agents.
		
		\textbf{Experiment Setup.} We consider a 2D scenario where $20$ agents are pursuing $30$ moving targets with $T=2500$ iterations over $50$ seconds. At every iteration, each agent selects its direction of movement from ``up", ``down", ``left", ``right", or ``diagonally". Concurrently, agents also adjust their speeds from a set of $5$, $10$, or $15$ units/s. As for targets, we categorize them into three distinct types: the unpredictable `Random', the structured `Polyline', and the challenging `Adversarial'. Specifically, at each iteration, a `Random' target randomly changes its movement angle $\theta$ from $[0,2\pi]$ and moves at a random speed between 5 and 10 units/s. A `Polyline' target generally maintains its trajectory and only
	behaves like the `Random' target at $\{0, \frac{T}{k}, \frac{2T}{k},\dots, \frac{(k-1)T}{k}\}$-th iteration where $T$ is the predefined total iterations and $k$ is a random number from $\{1,2,4\}$. As for the `Adversarial' target, it acts like a `Random' target when all agents are beyond 20 units. However, if any agent is within a 20-unit range, the `Adversarial' target escapes at a speed of 15 units/s for one second, pointing to the direction that maximizes the average distance from all agents. In addition, we initialize the starting positions of all agents and targets randomly within 20-unit radius circle centered at the origin. 
	
	
	\textbf{Objective Function.} We begin by defining the ground set $\V=\{(\theta,s,i):s\in\{5,10,15\}\text{units/s},i\in[20],\theta\in\{\frac{\pi}{4},\frac{\pi}{2},\frac{3\pi}{4},\pi,\dots,2\pi\}\}$ where $\theta,s,i$ represent the movement angle, speed and the identifier of agents, respectively. Moreover, the symbol  $o_{t}(j)$ denotes the 2D location of target $j\in[30]$ at time $t\in[T]$ and $o^{a}_{t}(\theta,s,i)$ stands for the new position of agent $i$ after moving from its location at time $t-1$  in the direction of $\theta$ at a speed of $s$. In order to keep up with all targets, we naturally consider the following submodular objective function for each time $t$: $f_{t}(\mathcal{A})=\sum_{j=1}^{30}\max_{(\theta,s,i)\in\mathcal{A}}\frac{1}{d(o^{a}_{t}(\theta,s,i),o_{t}(j))}$ where $d(\cdot,\cdot)$ is the distance between two locations and $\mathcal{A}\subseteq\V$.
	
	\textbf{Analysis.} In Figure \ref{graph:total}, we assess our proposed \textbf{MA-OSMA} and \textbf{MA-OSEA} against  OSG~\citep{xu2023online} across scenarios with different proportions of `Random', `Polyline', and `Adversarial' targets. The ratios are setting as `R':`A':`P'=$8$:$1$:$1$ in Figure~\ref{graph1}-\ref{graph3}, $6$:$3$:$1$ in Figure~\ref{graph12}-\ref{graph32} and $4$:$5$:$1$ in Figure~\ref{graph13}-\ref{graph33}. The suffixes in \textbf{MA-OSMA} and \textbf{MA-OSEA} represent two different choices for communication graphs: `c' for a complete graph and `r' for an Erdos-Renyi random graph with average degree $4$. From  Figure~\ref{graph1},\ref{graph12} and \ref{graph13}, we can find that the running average utility $\frac{\sum_{\tau=1}^{t}f_{\tau}(\cup_{i\in\N}\{a_{\tau,i}\})}{t}$ of our proposed \textbf{MA-OSMA} and \textbf{MA-OSEA} significantly outperforms the OSG algorithm, which is in accord with our theoretical analysis. Similarly, the average number of targets within $5$ units for \textbf{MA-OSMA} and \textbf{MA-OSEA} greatly exceeds that of the OSG, as depicted in Figure~\ref{graph2},\ref{graph22} and \ref{graph23}. Note that, due to `Adversarial' targets, all curves for the average number exhibit a downward trend. Furthermore, our proposed \textbf{MA-OSMA} and \textbf{MA-OSEA} also can effectively reduce the average distance as shown in Figure \ref{graph3}, \ref{graph32}, and \ref{graph33}. 
 Note that the algorithms over random graph  perform comparably to those on complete graph in all figures, which, to some extent, reflects the communication efficiency of our proposed algorithms.
		\begin{figure*}[t]
		\vspace{-1.0em}
		\centering
		\subfigure[Average Utility\label{graph1}]{\includegraphics[scale=0.175]{ICLR/Figure/Utility_Coordination-1-8-1.pdf}}
		\subfigure[Average Number \label{graph2}]{\includegraphics[scale=0.175]{ICLR/Figure/Average_Number-1-8-1.pdf}}
		\subfigure[Average Distance\label{graph3}]{\includegraphics[scale=0.175]{ICLR/Figure/Top_Average_Distance-1-8-1.pdf }}
		
		\subfigure[Average Utility\label{graph12}]{\includegraphics[scale=0.175]{ICLR/Figure/Utility_Coordination-1-6-3.pdf}}
		\subfigure[Average Number \label{graph22}]{\includegraphics[scale=0.175]{ICLR/Figure/Average_Number-1-6-3.pdf}}
		\subfigure[Average Distance\label{graph32}]{\includegraphics[scale=0.175]{ICLR/Figure/Top_Average_Distance-1-6-3.pdf }}
		
		\subfigure[Average Utility\label{graph13}]{\includegraphics[scale=0.175]{ICLR/Figure/Utility_Coordination-1-4-5.pdf}}
		\subfigure[Average Number \label{graph23}]{\includegraphics[scale=0.175]{ICLR/Figure/Average_Number-1-4-5.pdf}}
		\subfigure[Average Distance\label{graph33}]{\includegraphics[scale=0.175]{ICLR/Figure/Top_Average_Distance-1-4-5.pdf }}
		\vspace{-0.5em}
		\caption{Comparison of average cumulative utility, average number of targets within 5 units, average distance of Top-5 nearest targets of \textbf{MA-OSMA-c},\textbf{MA-OSMA-r},\textbf{MA-OSEA-c} and \textbf{MA-OSEA-c}  with OSG on different multi-target tracking scenarios(averaged over 5 runs).}\label{graph:total}
  \vspace{-1.5em}
	\end{figure*}