	%Submodular functions constitute a wide-ranging category of set functions characterized by the law of diminishing returns, which unify and generalize diverse problems in combinatorial optimization, encompassing the Max-Cover, Max-Cut, and Facility Location. Generally speaking, a set function $f:2^{V}\rightarrow\R_{+}$ is said to be submodular if $f(S\cup\{e\})-f(S)\ge f(T\cup\{e\})-f(T)$ for any $S\subseteq T\subseteq \V$ and $e\in \V\setminus T$, where $\V$ is a finite  ground set. Intuitively, this means that the marginal gain of adding an element to $f(S)$  will decrease as the size of set $S$ increases.  Additionally, maximization of submodular functions has recently found numerous applications in machine learning, operations research and economics, including data summarization~\citep{lin2010multi,lin2011class,wei2013using,wei2015submodularity,mirzasoleiman2016distributed}, dictionary learning~\citep{das2018approximate}, product recommendation~\citep{kempe2003maximizing,el2009turning,mirzasoleiman2016fast}, federated learning~\citep{balakrishnan2022diverse,FedSub} and in-context learning~\citep{kumari2024end}.Beyond the abundance of applications already mentioned, 
 Recent years have witnessed an upsurge in research focused on leveraging submodular functions to coordinate the actions of multiple agents in accomplishing tasks that are spatially distributed. A compelling example is the dynamic deployment of mobile sensors, particularly unmanned aerial vehicles (UAVs), for multi-target tracking~\citep{zhou2018resilient,corah2021scalable} as depicted in Figure~\ref{figue_intro_target}. In this scenario, at each critical moment of decision, every mobile sensor needs to determine its trajectory and velocity through interactions with others to effectively track all moving points of interest. The primary challenges of this tracking challenge lie in the unpredictability of the targets' movements and the limited sensing capabilities of agents. To address these issues, various modeling techniques have been developed, including one based on dynamically maximizing a sequence of submodular functions that capture the spatial relationship between sensors and moving targets~\citep{xu2023online,rezazadeh2023distributed,robey2021optimal}. As a result, the problem of target tracking can be cast into a specific instance of multi-agent online submodular maximization(MA-OSM) problem. Besides target tracking, the MA-OSM problem also offers a versatile framework for a variety of complex tasks such as area monitoring~\citep{schlotfeldt2021resilient,li2023submodularity}, environmental mapping~\citep{atanasov2015decentralized,liu2021distributed}, data summarization~\citep{mirzasoleiman2016fast,mirzasoleiman2016distributed} and task assignment~\citep{qu2019distributed}.  Motivated by these practical use cases, this paper delves into the multi-agent online submodular maximization problem.
%The primary challenges of this tracking task lie  in the unpredictability of the targets' movement and the limited sensing capabilities of the agents.	

	To tackle the aforementioned MA-OSM problem, \citet{xu2023online} have recently proposed an \emph{online sequential greedy} (OSG) algorithm, building upon the foundations of the classical greedy method ~\citep{fisher1978analysis}. Nevertheless, this online algorithm suffers from two notable limitations: \textbf{i) Sub-optimal Approximation:} In contrast with the tight $(\frac{1-e^{-c}}{c})$-approximation ratio~\citep{vondrak2010submodularity,bian2017guarantees}, OSG only can guarantee a sub-optimal $(\frac{1}{1+c})$-approximation where $c\in[0,1]$ is the joint curvature of submodular objectives; \textbf{ii) Requirement of a Fully Connected Communication Network:} OSG begins by assigning a unique order to each agent and then  requires every agent to have full access to the decisions made by all predecessors, which leads to a \emph{complete} directed acyclic communication graph (Refer to the left side of Figure~\ref{figue_intro_target}). As the number of agents grows, the communication overheads associated with this operation may become prohibitively high. Furthermore, \citet{grimsman2018impact} have pointed out that the approximation guarantee of OSG continuously degrades as the communication graph becomes less dense. This highlights the necessity of a \emph{complete} communication graph for maintaining the effectiveness of OSG. Given these disadvantages of OSG algorithm, the objective of this paper is to address the following  question:	\vspace{-0.6em}
	\begin{center}
		\emph{Is it possible to devise an online algorithm with tight $(\frac{1-e^{-c}}{c})$-approximation for \emph{MA-OSM} problem over a connected and sparse communication network?}
\end{center}
		\begin{figure}
		\vspace{-2.6em}
		\centering
		\includegraphics[scale=0.62]{ICLR/Figure/target_tracking1.drawio.pdf}
		\caption{Left: Multi-target tracking with $4$ mobile sensors over a \emph{complete} directed acyclic communication network. Right: Multi-target tracking with $4$ sensors over a \emph{connected} undirected graph.}\label{figue_intro_target}
  	\vspace{-1.0em}
	\end{figure}
	 In this paper, we provide an affirmative answer to this question by presenting two online algorithms, i.e., \textbf{MA-OSMA} and \textbf{MA-OSEA}, both of which not only can achieve the optimal $(\frac{1-e^{-c}}{c})$-approximation guarantee but also reduce the strict requirement for a \emph{complete} communication graph. 
  
  Specifically, our proposed algorithms incorporate three key innovations. First, we utilize the multi-linear extension to convert the discrete submodular maximization into a continuous optimization problem, which enables us to reduce the rigid requirement for a complete communication graph via the well-established consensus techniques in the field of decentralized optimization. Second, we develop a surrogate function for the multi-linear extension of submodular functions with curvature $c$, which empowers us to move beyond the sub-optimal $(\frac{1}{1+c})$-approximation stationary points. Last but not least, for each agent, we implement a distinct strategy to update the selected probabilities associated with its own actions and those of other agents, which only requires agents to assess the marginal gains of actions within their own action sets, thereby reducing the practical requirement on the observational scope of each agent. To summarize, we make the following contributions.
  \vspace{-0.2em}
	\begin{itemize}[leftmargin=*]
	\item We construct a surrogate function for the multi-linear extension of submodular functions with curvature $c\in[0,1]$. The stationary points of this surrogate can guarantee a tight $(\frac{1-e^{-c}}{c})$-approximation to the maximum value of the multi-linear extension, significantly outperforming the $(\frac{1}{1+c})$-approximation provided by the stationary points of the original multi-linear extension itself.
\item We propose a new algorithm \textbf{MA-OSMA}, which seamlessly integrates consensus techniques, lossless rounding and the surrogate function previously discussed. Moreover, we prove that \textbf{MA-OSMA} enjoys a regret bound of $O\Big(\sqrt{\frac{C_{T}T}{1-\beta}}\Big)$ against a  $(\frac{1-e^{-c}}{c})$-approximation to the best comparator in hindsight, where $C_{T}$ is the deviation of maximizer sequence and $\beta$ is the spectral gap of the communication network. Subsequently, we present a \emph{projection-free} variant of \textbf{MA-OSMA}, titled \textbf{MA-OSEA}, which effectively utilizes the KL divergence by mixing a uniform distribution. We also prove that \textbf{MA-OSEA} can attain a $(\frac{1-e^{-c}}{c})$-regret bound of $\widetilde{O}\Big(\sqrt{\frac{C_{T}T}{1-\beta}}\Big)$. A detailed comparison of our \textbf{MA-OSMA} and \textbf{MA-OSEA} with existing studies is presented in Table~\ref{tab:result}.
	\item 
 We conduct a simulation-based evaluation of our proposed algorithms within a multi-target tracking scenario.
 Our experiments demonstrate the effectiveness of our \textbf{MA-OSMA} and \textbf{MA-OSEA}.
	\end{itemize}
	 \begin{table}[t]
\renewcommand\arraystretch{1.35}
	\centering	\vspace{-0.9em}
	\resizebox{0.9\textwidth}{!}{
		\setlength{\tabcolsep}{1.0mm}{
			\begin{tabular}{cccccc}
				\toprule[1.0pt]
				Method&Approx.Ratio&Graph($G$)&Regret& Projection-free?&Reference \\
				\hline
				OSG &$\Big(\frac{1}{1+c}\Big)$ &\textbf{complete} &$\widetilde{O}\Big(\sqrt{C_{T}T}\Big)$&\ding{52}&\citet{xu2023online}\\
				OSG &$\Big(\frac{1}{1+\alpha(G)}\Big)$ &connected &$\widetilde{O}\Big(\sqrt{C_{T}T}\Big)$&\ding{52}&\citet{grimsman2018impact,xu2023online}\\
				\rowcolor{cyan!18}
				\textbf{MA-OSMA}&$\Big(\frac{1-e^{-c}}{c}\Big)$ &connected &$O\Big(\sqrt{\frac{C_{T}T}{1-\beta}}\Big)$ &\ding{56}&Theorem~\ref{thm:final_one} \& Remark~\ref{Remark:final}\\
				\rowcolor{cyan!18}
				\textbf{MA-OSEA}&$\Big(\frac{1-e^{-c}}{c}\Big)$ &connected &$\widetilde{O}\Big(\sqrt{\frac{C_{T}T}{1-\beta}}\Big)$ &\ding{52}&Theorem~\ref{thm:final_one1} \& Remark~\ref{Remark:final1}\\
				\midrule[1.0pt]
			\end{tabular}
	}}\caption{Comparison with prior works. $T$ is the horizon length, $c\in[0,1]$ is the joint curvature of submodular objectives, $C_{T}$ is the deviation of maximizer sequence, $\beta$ is the second largest magnitude of the eigenvalues of the weight matrix, $\alpha(G)$ is the number of nodes in the largest independent set in communication graph $G$ where $\alpha(G)\ge1$ and $\widetilde{O}(\cdot)$ hides $\log(T)$ term.}\label{tab:result}\vspace{-0.5em}
\end{table}
\textbf{Related Work.} Due to space limits, we only focus on the most relevant studies. A more comprehensive discussion is provided in Appendix~\ref{Appendix:related_work}. Multi-agent submodular maximization (MA-SM) problem involves coordinating multiple agents to collaboratively maximize a submodular utility function, with numerous applications in sensor coverage~\citep{krause2008near,prajapat2022near} and multi-robot planning~\citep{singh2009efficient,zhou2022risk}. A commonly used solution for MA-SM problem heavily depends on the distributed implementation of the classic \emph{sequential greedy} method~\citep{fisher1978analysis}, which can ensure a $(\frac{1}{1+c})$-approximation~\citep{conforti1984submodular}. However, this distributed algorithm requires each agent to have full access to the decisions of all previous agents, thereby forming a \emph{complete} directed communication graph. Subsequently, several studies~\citep{grimsman2018impact,gharesifard2017distributed,marden2016role} have investigated how the topology of the communication network affects the performance of the distributed greedy method. Particularly, \citet{grimsman2018impact} pointed out that the worst-case performance of the distributed greedy algorithm will deteriorate in proportion to the size of the largest independent group of agents in the communication graph. Given that the majority of applications occur in time-varying environments, \cite{xu2023online} proposed the \emph{online sequence greedy}(OSG) algorithm for online MA-SM problem, which also ensures a sub-optimal $(\frac{1}{1+c})$-approximation over a \emph{complete} communication graph.