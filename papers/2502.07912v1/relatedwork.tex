\section{Related work}
\subsection{Retrieval-Augmented Generation (RAG)}

RAG can significantly improve the model performance by leveraging additional knowledge and has been widely applied in various tasks, such as question \& answering (Q\&A)~\citep{lewis2020retrieval, mao2020generation}, machine
translation~\citep{gu2018search}, and summarization~\citep{liu2020retrieval, parvez2021retrieval}. With the emergence of LLMs such as LLaMA and ChatGPT, the integration of RAG with LLMs has gained significant popularity and led to significant advancements in multiple tasks~\citep{liu2023reta, kim2023tree, sharma2024retrieval, feng2024retrieval}.


RAG is also widely applied in research within the legal domain, such as legal Q\&A~\citep{cui2023chatlaw, louis2024interpretable, wiratunga2024cbr}, legal judgment prediction~\citep{wu2023precedent}, legal text evaluation~\citep{ryu2023retrieval}, and terminology drafting for legislative documents~\citep{chouhan2024lexdrafter}. 

However, most prior research primarily concentrates on improving the performance of retrieval models from a semantic perspective. While semantic information is undoubtedly important, the significance of logical structure is particularly prominent in dealing with legal questions. Legal reasoning often relies on a well-defined logical flow. To address this challenge, our study emphasizes the integration of both semantic information and logical structure in retrieval processing.


\subsection{Question \& Answering (Q\&A)}
Q\&A is an active research area in NLP that aims to develop systems capable of providing accurate and relevant answers to questions posed in natural language by users based on large knowledge sources~\citep{Rogers2023}. Current Q\&A studies mainly focus on 1) knowledge retrieval which aims to develop effective and efficient methods to retrieve relevant information from large knowledge bases or corpora~\citep{Vladimir2020}, 2) reading comprehension which aims to build models that can comprehend passages to identify answer-relevant information~\citep{Baradaran2022}, 3) multi-hop reasoning, which aims to perform multi-step reasoning by combining information from multiple sources~\citep{wang-etal-2022-new}, and 4) explainable Q\&A which aims to generate human-understandable explanations or rationales to support their answers~\citep{Veronica2020}.


\subsection{AI Applications in Law}
The legal domain has seen increasing interest in applying AI and machine learning techniques to assist with various tasks in Law. One active area of research is using NLP for legal document analysis and information extraction~\citep{Zhong2020}.~\citet{mistica2020} created a schema based on related information that legal professionals seek within judgements and performed classification based on it.~\citet{Sun-xu-2023} proposed a model-agnostic causal learning framework to for legal case matching. There is also work on using AI for legal judgment prediction, as in~\citet{Liu-Zhang-2023} who develop a neural framework to predict judgments from fact descriptions.

Another emerging application is using AI for legal QA, legal reasoning, and argument mining from texts.~\citet{chen2023equals} proposed a well-annotated real-world dataset for legal QA.~\citet{Mumford2023} establihsed a new dataset and explored neural methods to capture patterns of reasoning over legal texts.~\citet{Zhang-Nulty-2023} investigated extracting argumentative components like claims and evidence from legal cases. Some researchers are also exploring constitutionality analysis, with~\citet{Sert2022} proposing an AI system to predict decisions of the Turkish constitutional court. While promising, these AI-based legal methods still face challenges around interpretability, generalization, and capturing the nuanced reasoning required in law.