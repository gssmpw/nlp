{\renewcommand{\arraystretch}{1.5}
\subsection{Attribute values} \label{attribute-values}
\begin{table}[!htb]
    \small
    \centering
    \begin{tabular}{|p{0.06\textwidth}|p{0.4\textwidth}|}
    \hline
    \textbf{Gender} & \textbf{Gendered words}                        \\
    \hline    
    female          & abbess, actress, airwoman, aunt, ballerina, baroness, barwoman, belle, bellgirl, bride, bride, busgirl, businesswoman, camerawoman, chairwoman, chick, congresswoman, councilwoman, countrywoman, cowgirl, czarina, daughter, diva, duchess, empress, enchantress, female, fiancee, gal, gal, girl, girlfriend, godmother, governess, granddaughter, grandma, grandmother, handywoman, headmistress, heiress, heroine, hostess, housewife, lady, lady, lady, lady, landlady, lass, lass, maam, madam, maid, maiden, maidservant, mama, marchioness, masseuse, mezzo, minx, mistress, mistress, mom, mommy, mother, mum, niece, nun, nun, policewoman, priestess, princess, queen, saleswoman, schoolgirl, seamstress, seamstress, she, sister, sistren, sorceress, spokeswoman, stateswoman, stepdaughter, stepmother, stewardess, strongwoman, suitress, waitress, widow, wife, wife, witch, woman \\
    \hline
    male            & abbot, actor, airman, uncle, ballet\_dancer, baron, barman, beau, bellboy, bridegroom, groom, busboy, businessman, cameraman, chairman, dude, congressman, councilman, countryman, cowboy, czar, son, divo, duke, emperor, enchanter, male, fiance, guy, dude, boy, boyfriend, godfather, governor, grandson, grandpa, grandfather, handyman, headmaster, heir, hero, host, househusband, lord, fella, mentleman, gentleman, landlord, lad, chap, sir, sir, manservant, bachelor, manservant, papa, marquis, masseur, baritone, stud, master, paramour, dad, daddy, father, dad, nephew, priest, monk, policeman, priest, prince, king, salesman, schoolboy, tailor, seamster, he, brother, brethren, sorcerer, spokesman, statesman, stepson, stepfather, steward, strongman, suitor, waiter, widower, husband, hubby, wizard, man \\
    \hline
    \end{tabular}
    \captionsetup{justification=centering}
    \caption{Attributes: Gendered words. Note that some of the words are redundant, but they are paired with distinct gendered words.}
    \label{tab:attributes}
\end{table}
}

\newpage
\subsection{Trait dimensions (target values)} \label{target-values}

{
\renewcommand{\arraystretch}{1.5}
\begin{table}[H]
    \vspace{-3mm}
    \small
    \centering
    \begin{tabular}{|p{1.5cm}|p{5.7cm}|@{}}
            \hline
        \textbf{Character} & \textbf{Character words }                        \\
        \hline
        % factor1
        \textit{empathy} & affable, charitable, compassionate, concerned, considerate, courteous, empathetic, friendly, gracious, liberal, sensitive, sympathetic, understanding                            \\
        \hline
        % factor2
        \textit{order} & abstinent, austere, careful, cautious, clean, conservative, decent, deliberate, disciplined, earnest, obedient, ordered, scrupulous, self-controlled, self-denying, serious, tidy \\
        \hline
        % factor3
        \textit{resourceful} & confident, courageous, independent, intelligent, perseverant, persistent, purposeful, resourceful, sagacious, zealous                                                            \\
        \hline
        % factor4
        \textit{serenity} & forbearing, forgiving, meek, merciful, patient, peaceful, serene    \\                                         \hline                                        
    \end{tabular}
    \captionsetup{justification=centering}
    \caption{Targets: \textbf{Positive character traits} --- dimensions and trait words.}
    \label{tab:targets_virtue}
\end{table}
}

{\renewcommand{\arraystretch}{1.5}
\begin{table}[H]
    \vspace{-3mm}
    \small
    \centering
    \begin{tabular}{|p{2.4cm}|p{4.8cm}|@{}}
        \hline
        \textbf{Personality} & \textbf{Personality words }                        \\
        \hline
        \textit{extroversion} & active, adventurous, assertive, bold, energetic, extroverted, talkative\\
        \hline
        \textit{agreeableness} & agreeable, cooperative, generous, kind, trustful, unselfish, warm\\
        \hline
        \textit{conscientiousness} & conscientious, hardworking, organized, practical, responsible, thorough, thrifty \\
        \hline
        \textit{emotional stability} & at ease, calm, contented, not envious, relaxed, stable, unemotional\\
        \hline
            \textit{openness} & analytical, creative, curious, imaginative, intelligent, reflective, sophisticated\\

    \hline                                        
    \end{tabular}
    \captionsetup{justification=centering}
    \caption{Targets: \textbf{Positive personality traits} --- dimensions and trait words.}
    \label{tab:targets_big_five}
\end{table}
}

{
\renewcommand{\arraystretch}{1.5} 
\begin{table}[htbp]
    \vspace{-3mm}
    \small
    \centering
    \begin{tabular}{|p{1.5cm}|p{5.7cm}|@{}}
            \hline
        \textbf{Character} & \textbf{Character words }                        \\
        \hline
        % factor1
        \textit{empathy} &  disagreeable, uncharitable, unfeeling, unconcerned, inconsiderate, discourteous, callous, unfriendly, ungracious, conservative, insensitive, unsympathetic, inconsiderate                           \\
        \hline
        % factor2
        \textit{order} & indulgent, genial, careless, reckless, dirty, liberal, indecent, unmotivated, undisciplined, flippant, disobedient, disorganized, unscrupulous, undisciplined, self-indulgent, frivolous, untidy \\
        \hline
        % factor3
        \textit{resourceful} & unsure, cowardly, dependent, stupid, weak, intermittent, aimless, unresourceful, foolish, unenthusiastic                                                           \\
        \hline
        % factor4
        \textit{serenity} & impatient, unforgiving, assertive, merciless, impatient, disturbed, agitated    \\                                         \hline                                        
    \end{tabular}
    \captionsetup{justification=centering}
    \caption{Targets: \textbf{Negative character traits} --- dimensions and trait words.}
    \label{tab:negative_traits_character}
\end{table}
}


{\renewcommand{\arraystretch}{1.5}
\begin{table}[htbp]
    \vspace{-3mm}
    \small
    \centering
    \begin{tabular}{|p{2.4cm}|p{4.8cm}|@{}}
        \hline
        \textbf{Personality} & \textbf{Personality words }                        \\
        \hline
        \textit{extroversion} & inactive, unadventurous, unassertive, timid, unenergetic, introverted, silent\\
        \hline
        \textit{agreeableness} & disagreeable, uncooperative, stingy, unkind, distrustful, selfish, cold\\
        \hline
        \textit{conscientiousness} & negligent, lazy, disorganized, impractical, irresponsible, careless, extravagant \\
        \hline
        \textit{emotional stability} & nervous, angry, discontented, envious, tense, unstable, emotional\\
        \hline
            \textit{openness} & unanalytical, uncreative, uninquisitive, unimaginative, unintelligent, unreflective, unsophisticated\\

    \hline                                        
    \end{tabular}
    \captionsetup{justification=centering}
    \caption{Targets: \textbf{Negative personality traits} \cite{goldberg1992development} --- dimensions and trait words.}
    \label{tab:negative_traits_big_five}
\end{table}
}

\begingroup
\setlength{\tabcolsep}{3.5pt} 
\renewcommand{\arraystretch}{1} % Adjust row height 
\begin{table*}[htbp]
    \setlength{\belowdisplayskip}{-10pt} 
    \vspace{-3mm}
\footnotesize
  \centering
    \begin{tabular}{@{}p{1.5cm}p{2.6cm}p{0.9cm}p{1.4cm}p{1.4cm}p{1.4cm}p{1.4cm}p{1.4cm}p{1.4cm}}
    \toprule
    \multirow{2}[4]{*}{\textbf{}} 
            & \multicolumn{1}{l}{\multirow{2}[4]{*}{\textbf{Targets}}} & \textbf{\# of target words} & \multicolumn{6}{c}{\textbf{Templates}} \\
    
    \cmidrule{3-9}          &       &  & \multicolumn{1}{c}{\textbf{$\text{t}_1$}} & \multicolumn{1}{c}{\textbf{$\text{t}_2$}} & \multicolumn{1}{c}{\textbf{$\text{t}_3$}} & \multicolumn{1}{c}{\textbf{$\text{t}_4$}} & \multicolumn{1}{c}{\textbf{$\text{t}_5$}} & \multicolumn{1}{c}{\textbf{$\text{t}_6$}} \\
    \midrule
    \multirow{4}[8]{*}{\textbf{Character}} 
            & \textit{empathy} & 13 & $3067\pm241$  & $3143\pm221$  & $2827\pm233$  & $2882\pm275$  & $2915\pm203$  & $2711\pm103$ \\
            
            \cmidrule{2-9} & \textit{order} & 17 & $3958\pm332$  & $4119\pm313$  & $3676\pm301$  & $3764\pm383$  & $3837\pm231$  & $3489\pm130$ \\

            \cmidrule{2-9} & \textit{resourceful} & 10 & $2342\pm211$  & $2447\pm178$  & $2167\pm194$  & $2213\pm239$  & $2236\pm158$  & $2065\pm85$ \\
            
            \cmidrule{2-9}  & \textit{serenity} & 7 &  $1645\pm138$  & $1715\pm145$  & $1506\pm124$  & $1559\pm158$  & $1571\pm116$  & $1447\pm53$ \\
    \midrule
    \multirow{5}[10]{*}{\textbf{Personality}} 
            & \textit{extroversion} & 7 & $1619\pm155$  & $1671\pm110$  & $1530\pm151$  & $1543\pm158$  & $1569\pm119$  & $1437\pm57$ \\
            
            \cmidrule{2-9}          & \textit{agreeableness} & 7 &  $1653\pm124$  & $1757\pm121$  & $1530\pm123$  & $1554\pm154$  & $1560\pm96$  & $1449\pm59$ \\

            \cmidrule{2-9}          & \textit{conscientiousness} & 7 &  $1639\pm135$  & $1692\pm124$  & $1517\pm133$  & $1547\pm152$  & $1580\pm101$  & $1450\pm69$ \\
            
            \cmidrule{2-9}          & \textit{emotional stability} & 7 &  $1626\pm133$  & $1730\pm141$  & $1538\pm146$  & $1555\pm155$  & $1593\pm107$  & $1457\pm77$ \\
            
            \cmidrule{2-9}          & \textit{openness} & 7 &  $1665\pm123$  & $1711\pm139$  & $1520\pm120$  & $1550\pm152$  & $1608\pm97$  & $1452\pm51$ \\
    \bottomrule
    \end{tabular}%
      \caption{Mean and Standard deviation ($\mu\pm\sigma$) of the number of sentences for each template in each of character/personality dimensions (includes 94 pairs of gendered words (attributes)) across seven MLMs of variation ($\sigma/\mu$) ranges from 3.5\% to 10.8\%. The sentence selection is specific to MLM, and hence, the number of sentences within each template and trait dimension can vary. So, we provide the mean and standard deviation for each template within each trait dimension.  
      }
  \label{tab:num_sentences_dist}%
  \vspace{-3mm}
\end{table*}
\endgroup

\subsection{Overview of template selection algorithm} \label{template-selection-overview}

\noindent (1) Initially we obtain sentences from the Wikipedia Corpus and Book Corpus (used in BERT pre-training). (2) We then utilize the text generation capabilities of GPT-4 model to suggest additional sentences containing a target character trait word and the pronoun "she/he".
%
(3) We combined all of these sentences.
%
(4) We then filter out sentences that were no longer than 15 words, containing both a character word (from our work) and the pronoun "she/he".
%
(5) The next step involves narrowing down these sentences to those where the pronoun precedes the character trait words. 
%
(6) We then identify common sentence patterns through parts-of-speech tagging. This involves analyzing the grammatical structure of the sentences to identify repetitive patterns. 
%
(7) Finally, after identifying potential sentence templates, a careful manual review is conducted.
The above steps were performed to design \textit{indirect} templates capturing the common expressions of human traits.

To generate \textit{direct templates}, we repeat the process but include the word \enquote{personality} in the selection and generation criteria. These templates could provide more guidance in predicting human traits by minimizing the ambiguity in the usage of trait words in a sentence.

\vspace{0.5em}
\noindent \textit{Limitations:} 
The character trait word may not be used in the character context in \textit{indirect} templates. This is handled during manual review. Note that this can also be handled by using contextual embedding.
We changed past-tense common sentences into present tense while selecting templates as we focus on the present tense as the traits may change over time, and analyzing the present tense allows for real-time insights.

\subsection{Influence of pseudo-perplexity} \label{influence-of-perplexity}

\begin{figure}[h]
    \includegraphics[width=0.49\textwidth]{imgs/ppl_infulence_on_bias/roberta_large_conscientiousness_ppl.png}
    \caption{RoBERTa-large ($\text{model}_{\text{lme}}$). 
    }
    \label{fig:influence_of_ppl_on_bias}
\end{figure}

\noindent In the cumulative graphs of Figure \ref{fig:influence_of_ppl_on_bias}, data points (sentences) are binned by psuedo-perplexity on the X-axis. The Y-axis represents effect size (left) and bias score (inner right) obtained with the corresponding sentence set. Only points with significant bias scores are shown. 
%
Effect sizes below 0.01 (blue horizontal line close to the X-axis) have a negligible effect. The red horizontal lines indicate distributions of 25\%, 50\%, and 75\% (bottom to top).

Key to note is that bias score (-0.77) and effect size (0.059) for the full set of sentences - `ALL' on the X-axis - are close to the average bias score (-0.95) and 
% average 
effect size (0.051) for the first 4 pseudo-perplexity bins.


\subsection{Additional proof-of-concept experiments}

\noindent We limit additional proof-of-concept experiments to RoBERTa-large, our most biased model, except for bias analysis in the auto-regressive language model. 

\subsubsection{Bias detection in autoregressive language model (ALM)}\label{bias-detection-llama3}


\noindent Our main experiments are on detecting bias in MLMs.  Here we show that our approach can be extended to detect bias in autoregressive pretrained language models, demonstrating this with Llama3.1-8B \cite{dubey2024llama}.
%
We analyze pairwise bias for binary gender 
and non-binary gender using the same neo-pronouns set from Section \ref{non-binary-result}. 
 

We probe LLama3 with sentences (from our templates) for each gender and trait combination.
%
Similar to work by \citet{hossain-etal-2023-misgendered}, we use sentence loss as a proxy for gender - trait association score. A lower loss indicates a better fit with the model and hence a stronger association between the gendered word and trait word in the sentence. 
%
The rest of the bias detection design is as discussed in Section \ref{measuring-association}.


\begingroup
\setlength{\tabcolsep}{3pt}
\begin{table}[htbp]
    \scriptsize
  \centering
  
    \begin{tabular}{@{}llccc@{}}
    \hline
          {} 
          & \textbf{Traits} 
          & \multicolumn{1}{l}{\textbf{M - F}} 
          & \multicolumn{1}{l}{\textbf{M - N}}
          & \multicolumn{1}{l}{\textbf{F - N}} 
           \\
    \hline
    \multirow{4}[0]{*}{\begin{tabular}[l]{@{}l@{}}\textbf{Character}\\\textbf{traits}\end{tabular}}
    & \textit{empathy}  & 0.06 & -1.61 \4 & -1.67 \4 \\
    
    & \textit{order} & 0.04 & -1.56 \4 & -1.60 \4  \\
    
    & \textit{resourceful}  & 0.09 & -1.42 \4 & -1.51 \4  \\
    
    & \textit{serenity} & 0.07 & -1.64 \4 & -1.71 \4  \\
    \hline

   \multirow{5}[0]{*}{\begin{tabular}[l]{@{}l@{}}\textbf{Personality}\\\textbf{traits}\end{tabular}}
    &     \textit{extroversion}  & 0.07 & -1.49 \4 & -1.57 \4  \\
    
    & \textit{agreeableness} & 0.06 & -1.39 \4 & -1.44 \4 \\
    
    & \textit{conscientiousness}  &  0.07 & -1.49 \4 & -1.56 \4  \\
    
    & \textit{emotional stability} & 0.05 & -1.40 \4 & -1.45 \4 \\

    & \textit{openness} & 0.07 & -1.64 \4 & -1.71 \4 \\
    \hline
    \end{tabular}
    \caption{Results for LLama3.1-8B using $\text{model}_{\text{lme}}$. M: Males, F: Females, N: Neo-pronouns. See Table \ref{tab:model2-results} for reported values descriptions and the notation used. Negative values indicate larger losses for neo sentences, suggesting weaker association between trait and neo. 
    }
  \label{tab:llama3-results}
  \vspace{-3mm}
  
\end{table}
\endgroup


Differences between males and females are minimal and there are no biases.  
%
However, when comparing males (or females) with neo group, we observe sizable and significant bias scores with medium to large effect sizes (0.15 to 0.26). Hence, there is medium to large bias against neo.
%
The negative differences indicate larger losses for neo sentences, suggesting a weaker association between neo and traits compared to associations for other genders. 
It has been observed that LLMs generally perform well in tasks with the goal of predicting binary gender while they perform poorly at predicting neo-pronouns \cite{ovalle-etal-2024-tokenization,hossain-etal-2023-misgendered}. This weaker performance in handling neo-pronouns might explain the weaker association between neo and traits compared to binary genders and traits.


\subsubsection{MLM bias detection using a crowdsourced dataset without templates}\label{crows-pair-eval}
%
\noindent While we focus on template-based design in our main experiments, our work is not limited to bias detection with such datasets. To demonstrate this we present bias analysis on the crowdsourced CrowS-Pairs \cite{nangia-etal-2020-crows} which does not involve templates. 

First, as a sanity check we conduct a replication study using their CrowsPair Score (CPS) metric and achieve a similar value as reported by \citet{kaneko2022unmasking}. 
%
CPS measures the percentage of stereotypical sentences preferred by an MLM over anti-stereotypical. Additionally we extend the analysis with our approach that focuses on the `difference' in association scores across these two sentence types.
We take stereotype\_type as a fixed effect in our model. The association (pseudo-log likelihood PLLScore) is computed using `score ($S$)' as  in \citet{nangia-etal-2020-crows}.
%
To address variations in sentence structure, we grouped sentences by length (short, medium, long) based on the 33rd and 67th percentiles of sentence length, accounting for sentence length variability as a random effect (1|sentence\_length\_group). The overall model is 
$\text{association}_{\text{score}}$ $\sim$ stereotype\_type $+$ (1| sentence\_length\_group), weight = 1/sentence psuedo-perplexity.
%
We applied our $\text{model}_{\text{lme}}$. 
Bias score is the coefficient of the stereotype\_type (i.e., stereotypical PLLScore - anti-stereotypical PLLScore). The rest of the design is the same as in Section \ref{measuring-association}. 
%
Results are in Table \ref{tab:crows-pair-result-roberta-large}. 

\begingroup
\setlength{\tabcolsep}{3pt}
\begin{table}[htbp]
    \scriptsize
    \centering
    \begin{tabular}{@{}lccc@{}}
    \hline
        \textbf{Bias Type} & \textbf{\textit{n}} 
        & 
        \textbf{CPS}
        
        &
       
        \begin{tabular}[c]{@{}c@{}}\textbf{Our approach}\\{($\text{model}_{\text{lme}}$})\end{tabular}
        \\
    \hline
       Race/Color  & 516 & 64.15 & \textcolor{red}{0.59} \\
       Gender/Gender identity & 262 & 58.78 & \textcolor{red}{0.11} \\

       \begin{tabular}[l]{@{}l@{}}{Socioeconomic status/}\\{occupation}\end{tabular} & 172 & 66.86 & \textcolor{red}{0.66} \\
       Nationality & 159 & 66.67 & \textcolor{red}{1.14}\\

       Religion & 105 & 73.33 & \textcolor{red}{1.04} \\
       
       Age & 87 & 72.41 & \textcolor{red}{1.23} \\

       Sexual Orientation & 84 & 64.29 & \textcolor{red}{0.88}\\

       Physical appearance & 63 & 73.02 & \textcolor{red}{1.34}      \\
       Disability & 60 & 70.00 & \textcolor{red}{1.41}\\
    \hline
    \end{tabular}
    \caption{Results for CrowS-Pair using our approach. CPS: CrowS-Pair Score.
    \textit{n}: number of examples.}
    \label{tab:crows-pair-result-roberta-large}
    \vspace{-3mm}
\end{table}
\endgroup

Stereotypical sentences are preferred over anti-stereotypical ones, but the differences throughout are insignificant, indicating no bias.
% 
The problem with CPS is that even minor PLLScore differences contribute to deviations from the 50\% ideal and detection of bias.
%
Unfortunately, the magnitude of these differences are not considered.
%
In contrast, our model statistically analyzes the PLLScore differences across sentence types, focusing on both significance and effect size - and we find no bias.




\subsubsection{Bias mitigation in MLMs}\label{bias-mitigation}

\noindent Our research focus is on bias detection.  However, for the reader expecting a complete pipline that includes mitigation - we add this proof-of-concept experiment.  We mitigate bias in RoBERTa-large, our most biased model (Section \ref{bias-across-mlms}), with a focus on binary gender.
%
We mitigate bias by fine-tuning the model on a gender-swapped GAP corpus \cite{webster-etal-2018-mind} following \citet{bartl-etal-2020-unmasking}. 
% 
This process involves dynamic masking during fine-tuning MLM task, following the design described in \citet{roberta-paper}, to specifically address gender bias.


We tune the model for 3 epochs using AdamW optimizer with a 2e-5 learning rate and a batch size 16. To manage the learning rate adjustment smoothly, we use a polynomial decay scheduler with a linear warm-up phase over the first 500 steps. 


\begingroup
\setlength{\tabcolsep}{5pt}
\begin{table}[htbp]
    \scriptsize
  \centering

    \begin{tabular}{@{}llcc@{}}
    \hline
        {} & \textbf{Traits} & \textbf{Before}& \textbf{After} \\
    \hline
    
    \multirow{4}[0]{*}{\begin{tabular}[l]{@{}l@{}}\textbf{Character}\\\textbf{traits}\end{tabular}}
    
        & \textit{empathy} & -0.70 \3 & -0.31 \\
        
        & \textit{order} & -0.30 \1 & \textcolor{red}{0.004} \\
        
        & \textit{resourceful}  & -0.74 \2 & -0.14 \\
        
        & \textit{serenity} & -1.08 \4 & -0.39 \1 \\
    \hline
    \multirow{5}[0]{*}{\begin{tabular}[l]{@{}l@{}}\textbf{Personality}\\\textbf{traits}\end{tabular}}
    
        & \textit{extroversion} & -0.86\4 & -0.23 \\
        & \textit{agreeableness} & -0.77 \3 & \textcolor{blue}{-0.20} \\
        & \textit{conscientiousness} & -0.77 \3 & -0.12 \\
        & \textit{emotional stability} & -0.26 & \textcolor{red}{0.03} \\
        & \textit{openness} & -0.61 \2 & -0.14 \\
    \hline
    \end{tabular}
    \caption{Bias mitigation performance in RoBERTa-large ($\text{model}_{\text{lme}}$). See Table \ref{tab:model2-results} for reported values descriptions and the notation used. \textbf{Before} mitigation result is identical to Table \ref{tab:model2-results}.}
  \label{tab:bias_mitigation_mlms}
  \vspace{-3mm}
\end{table}
\endgroup

Table \ref{tab:bias_mitigation_mlms} presents bias \textit{before} and \textit{after} mitigation in RoBERTa-large. 
Bias scores reduce by 56\% to 98\% after mitigation across both sets of traits. 
There is only one dimension \textit{serenity}
that still exhibits some bias - but this has reduced from medium to small.
The remaining dimensions have become unbiased as regards gender.